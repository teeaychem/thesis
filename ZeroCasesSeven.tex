\documentclass[10pt]{article}
% \usepackage[margin=1in]{geometry}
% \newcommand\hmmax{0}
% \newcommand\bmmax{0}
% % % Fonts% %
% \usepackage{luatexja}

\usepackage[T1]{fontenc}
   % \usepackage{textcomp}
   % \usepackage{newtxtext}
   % \renewcommand\rmdefault{Pym} %\usepackage{mathptmx} %\usepackage{times}
\usepackage[complete, subscriptcorrection, slantedGreek, mtpfrak, mtpbb, mtpcal]{mtpro2}
   \usepackage{bm}% Access to bold math symbols
   % \usepackage[onlytext]{MinionPro}
   \usepackage[no-math]{fontspec}
   \defaultfontfeatures{Ligatures=TeX,Numbers={Proportional}}
   \newfontfeature{Microtype}{protrusion=default;expansion=default;}
   \setmainfont[Ligatures=TeX,BoldFont={*-Semibold}]{Source Serif Pro}
   \setsansfont[Microtype,Scale=MatchLowercase,Ligatures=TeX,BoldFont={*-Semibold}]{Source Sans Pro}
   \setmonofont[Scale=0.8]{Atlas Typewriter}
   % \usepackage{selnolig}% For suppressing certain typographic ligatures automatically
% % % % % % %
\usepackage{amsthm}         % (in part) For the defined environments
\usepackage{mathtools}      % Improves  on amsmaths/mtpro2
\usepackage{xfrac}

% % % The bibliography % % %
\usepackage[backend=biber,
  style=authoryear-comp,
  bibstyle=authoryear,
  citestyle=authoryear-comp,
  uniquename=false,
  % allinit,
  % giveninits=true,
  backref=false,
  hyperref=true,
  url=false,
  isbn=false,
  useprefix=true,
  ]{biblatex}
\DeclareFieldFormat{postnote}{#1}
\DeclareFieldFormat{multipostnote}{#1}
% \setlength\bibitemsep{1.5\itemsep}
\newcommand{\noopsort}[1]{}
\addbibresource{Thesis.bib}

% % % % % % % % % % % % % % %

\usepackage[inline]{enumitem}
\setlist[enumerate]{noitemsep}
\setlist[description]{style=unboxed,leftmargin=\parindent,labelindent=\parindent,font=\normalfont\space}
\setlist[itemize]{noitemsep}

% % % Misc packages % % %
\usepackage{setspace}
% \usepackage{refcheck} % Can be used for checking references
% \usepackage{lineno}   % For line numbers
% \usepackage{hyphenat} % For \hyp{} hyphenation command, and general hyphenation stuff

% % % % % % % % % % % % %

% % % Red Math % % %
\usepackage[usenames, dvipsnames]{xcolor}
% \usepackage{everysel}
% \EverySelectfont{\color{black}}
% \everymath{\color{red}}
% \everydisplay{\color{black}}
\definecolor{fuchsia}{HTML}{FE4164}%Neon Fuchsia %{F535AA}%Neon Pink
% % % % % % % % % %

\usepackage[export]{adjustbox}
\usepackage{subcaption}

% \usepackage{pifont}
% \newcommand{\hand}{\ding{43}}
\usepackage{array}


\usepackage{multirow}
% \usepackage{adjustbox}

\usepackage{titlesec}

\usepackage{multicol}

\setcounter{secnumdepth}{4}
\setcounter{tocdepth}{4}

\usepackage{tikz}
\usetikzlibrary{bending,arrows,positioning,calc}
\usetikzlibrary{arrows.meta}
\usepackage{tikz-qtree} %for simple tree syntax
% \usepgflibrary{arrows} %for arrow endings
% \usetikzlibrary{positioning,shapes.multipart} %for structured nodes
\usetikzlibrary{tikzmark}
\usetikzlibrary{patterns}


\usepackage{graphicx} % for images (png/jpeg etc.)
\usepackage{caption} % for \caption* command


\usepackage{tabularx}

\usepackage{bussalt}

\usepackage{Oblique} % Custom package for oblique commands
\usepackage{CustomTheorems}
\usepackage{FuturePromisedEvents}

\usepackage{svg}
\usepackage[off]{svg-extract}
\svgsetup{clean=true}

\usepackage{dashrule}

\newcommand{\hozline}[0]{%
  \noindent\hdashrule[0.5ex][c]{\textwidth}{.1pt}{}
  %\vspace{-10pt}
  % \noindent\rule{\textwidth}{.1pt}
}

\newcommand{\hozlinedash}[0]{%
  \noindent\hdashrule[0.5ex][c]{\textwidth}{.1pt}{2.5pt}
  %\vspace{-10pt}
}

\usepackage{contour}
% \usepackage{pdfrender}

\usepackage{extarrows}

% % % My commands % % %

% % % % % % % % % % % %

\usepackage{xskak}


\usepackage[hidelinks,breaklinks]{hyperref}

\title{\dots and you don't need me to tell you}
\author{Ben Sparkes}
% \date{ }


\begin{document}

\tableofcontents

\hozlinedash

\begin{itemize}
\item I may want to distinguish talk of `holding on the basis of' from `an instance of the basing relation', as the kind of relation I have in mind is likely ruled out by many accounts of the basing relation.
  E.g.
  \begin{itemize}
  \item Dispositionalism
  \item Representionalism
  \end{itemize}
\end{itemize}

\begin{itemize}
\item significant interest:
  \begin{enumerate}
  \item Involuntarism
  \item Weakness of will
  \end{enumerate}
\end{itemize}

\begin{note}
  The point about involuntarism can be held independently from the broader point that I want to make.
  For, this is about restricting the choices of an agent, rather than providing the agent with choices, roughly stated.
  Hence, lack of ability make block, but it does not follow from this that instance of abilities opens.
\end{note}

\begin{itemize}
\item Observation about the possibility of rewriting support for an attitude may also be motivated by remembering.
\item E.g.\ ``Do you remember \(\phi\)''.
  In some cases there is additional information, but in other cases this is nothing more than an issue of attribution.
  The pragmatics are complex, however.
  If you do remember \(\phi\), then you also remember \(\psi\), hence if you answer `yes' then I don't need to tell you about \(\psi\), etc.
\end{itemize}

\newpage



\begin{note}
  Motivation by considering the enkratic principle.
  \begin{quote}
    Enkratic Condition. Necessarily, if you are rational, you intend to F whenever you believe your reasons require you to F.
  \end{quote}
  Issue here is what `your reasons' are.
  In particular, link between this condition and responding to reasons, as outlined by \citeauthor{Broome:2013aa}.
  \begin{itemize}
  \item Focusing on this condition allows me to avoid some of the bigger issues concerning rationality.
  \item While, at the same time, something like this is widely adopted.
  \item The assumption to challenge is that these are `direct', either in representationalist or dispositionalist sense.
  \end{itemize}
\end{note}

{
  \color{red}
  \begin{note}[Main claim]
    The topic of interest is an agent's ability to reason to a proposition.

    What is interesting here is that an agent can be informed of their ability, independently of executing the ability.
  \end{note}
  
 


    
    
    In this way, an agent may hold an attitude without representing or responding to reasons which \emph{directly} support the attitude.
    

    \begin{itemize}
    \item Primary support is the agent's ability to reason.
    \item Secondary support is that the agent has information about their ability to reason.
    \end{itemize}

}



\textbf{Reasons that the agent forms an attitude in response to are not necessarily those reasons which the agent responds to when forming the attitude}, or so I will argue.

\textbf{Reasons that an agent takes to support an attitude to are not necessarily those reasons which the agent responds to when forming the attitude}.
This isn't too controversial.
For, simple cases of rewriting.


\textbf{It is possible for an agent to take reasons to support a held attitude without (directly) responding to those reasons.}

The purpose of this paper is to examine this proposition.

The main motivation here is the common use of the idea of responding to reasons.
In particular, there's \citeauthor{Lord:2018aa} and others who hold the identity these, or \citeauthor{Broome:2013aa} with enkrasia, \citeauthor{Hieronymi:2018aa} with considerations, and others.
Representationalism and dispositionalismn, roughly.
If the proposition is true, then there's pressure.

On the one hand, understanding of phenomena.
On the other, ways in which to understand reasons/the relationship that agent's have with reasons.


\begin{note}[Functional role/function]
  The function of the attitude is (in part) determined by the reasons for which it is held. Individuate attitudes by their functional role, but an attitudes functional role does not (completely) determine its function.
\end{note}

\begin{note}[Quick argument]
  Quick argument is that reasons require direct response of a kind.
  Though I think I need to say more about direct response for this to be of interest.
  \begin{itemize}
  \item Agent to hold \(A\) on the basis of \(R\) requires \(R\) to do some explanatory work.
  \item If agent does not directly respond to \(R\) then \(R\) does not do any explanatory work.
  \item Therefore, reason only if directly respond.
  \end{itemize}
  The basic idea is that without direct response there's no difference between the reason obtaining and the reason not obtaining, from the agent's point of view.
  In this sense, the argument is similar to those that could be made against certain forms of responding directly to reasons, etc.

  The response to this is that the second premise is false.
  But if this is the case then superveniece is false.
  So, I should use superveniece in the argument!

  And, the `insight' is that this is replaced by supervenience on mind + ability, roughly stated.
  The difficulty with ability is that this is not something that depends on the agent's mind, so to speak, because whether the agent is able to do something depends on how the world develops.

  Right, although the argument here is similar to the one used against Lord and co.\ it is not clear that it generalises.
  This is because I depend on the role of ability, and without this there is no argument.

  The argument against this, the one that suggests this is a framework issue, is that one may consider an agent's perception of their abilities to be sufficient.
\end{note}



\newpage

\maketitle

\textbf{How an agent's (recognised) ability to establish an propositional attitude by reasoning relates to the reasons for which the agent holds the attitude.}
Roughly stated, the claim is that an agent may hold (or maintain) an attitude on the basis of their ability to establish the propositional attitude by reasoning.

Consider the following claim:
\begin{enumerate}
\item\label{chess:claim:1} You are able to reason from the rules of chess and the game state (described in figure~\ref{fig:chess:board}) to the proposition that White cannot prevent Black from occupying c4 on their second move.
\end{enumerate}
I assume claim~\ref{chess:claim:1} is true, but not immediately obvious.

For, in order to show that White cannot prevent Black from occupying c4 on their second move, you need to consider the moves that would be possible for Black on their second turn given the move that White made on their first turn in response to the move that Black made on their first turn.

\begin{figure}[h]
  \centering
  \mbox{ }
  \hfill
  \begin{subfigure}{.4\textwidth}
    \begin{adjustbox}{minipage=\linewidth,scale=0.7}
      \centering
      \newchessgame[
      setwhite={ka5,pa3,pb4,pc4,pe5,pf6,bg5,bh5}, %{rc1,kh1,pa2,pb2,ph2,pf6,pg6,nc7,qf7},
      addblack={pa6,pb7,pc6,pe6,pf7,kc7,nd7,nd4}, %{rg2,pb5,pe5,qd6,pa7,pb7,ra8,bc8,kd8,bf8},
      ]%
      \setchessboard{showmover=false}%
      \chessboard
    \end{adjustbox}
    \caption{Starting board}
    \label{fig:chess:board}
  \end{subfigure}
  \mbox{ }
  \hfill
  \mbox{ }
  \begin{subfigure}{.4\textwidth}
    \begin{adjustbox}{minipage=\linewidth,scale=0.7}
      \centering
      \newchessgame[
      setwhite={ka5,pa3,pb4,pc4,pe5,pf6,bg5,bh5}, %{rc1,kh1,pa2,pb2,ph2,pf6,pg6,nc7,qf7},
      addblack={pa6,pb7,pc6,pe6,pf7,kc7,nd7,nd4}, %{rg2,pb5,pe5,qd6,pa7,pb7,ra8,bc8,kd8,bf8},
      ]%
      \setchessboard{showmover=false}%
      \chessboard[
      arrow=latex,
      linewidth=1pt,
      shortenstart=.8ex,
      shortenend=.5ex,
      pgfstyle=straightmove,
      strokeopacity=0.4,
      fillopacity=0.4,
      color=black, markmoves={b7-b6,c6-c5,d4-c2,d4-b5,d4-f5,d4-e2,d4-f3,d4-b3,d7-c5,d7-b6,d7-b8,d7-f8,d7-f6,d7-e5,d7-e5,c7-c8,c7-b8,c7-d8,c7-b6,c7-d6}%{f7-g8,f7-e6,f7-d5,f7-c4,f7-b3,f7-e8,c7-d5,c7-b5,c7-a8,c7-e8,g6-g7,a2-a3,b2-b3,c1-a1,c1-b1,c1-d1,c1-e1,c1-f1,c1-g1,h2-h3,h1-g1,c1-c2,c1-c3,c1-c4,c1-c5,c1-c6}
      ]
    \end{adjustbox}
    \caption{Available moves}
    \label{fig:chess:move}
  \end{subfigure}
  \hfill
  \mbox{ }
  \caption{Black to checkmate in four moves.\protect\footnotemark}
  \label{fig:chess}
\end{figure}
\footnotetext{
  Puzzle 150 of \citeauthor{Emms:2000aa} (\citeyear[33]{Emms:2000aa}).
  \citeauthor{Emms:2000aa} provides the following solution:
  \begin{quote}
    \variation{1... Nb6!}
    (threatening \variation{2... Nb3\#})
    \variation{2. b5}
    (or \variation{2. Bd1 Nxc4+} \variation{3. Ka4 b5\#})
    \variation{2... c5!}
    \variation{3. bxa6 Nxc4+}
    \variation{4. Ka4 b5\#}
    \textbf{(0-1)}\nolinebreak
    \mbox{}
    \hfill
    (\citeyear[46]{Emms:2000aa})
  \end{quote}
}

To illustrate, Black may move the pawn from b7 to b5 in their first turn, and so be in a position to capture White's pawn on c4 in their second turn.
However, White may then prevent Black from occupying c4 on their next turn by using their pawn on c4 to capture Black's pawn on b5.

Still, there are only a handful of alternative moves and countermoves to consider, and so I not only assume that claim~\ref{chess:claim:1} is true, but I also assume that you also hold claim~\ref{chess:claim:1} to be true.
Specifically, I also assume that you hold claim~\ref{chess:claim:1} to be true at some interval between the point at which I claimed~\ref{chess:claim:1} is true and the present time or the point at which you demonstrated claim~\ref{chess:claim:1} to be true.\nolinebreak
\footnote{
  One may substitute `a3', `a5', `g6', or `c5' (non-exhaustive) for `c4' in claim~\ref{chess:claim:1} if a fresh claim is desired.
}
For ease of exposition, I will adopt the perspective of some point in that interval.
Therefore:
\begin{enumerate}[resume]
\item\label{chess:claim:2} White cannot prevent Black from occupying c4 on their second move.
% \item\label{chess:claim:3} You hold claim~\ref{chess:claim:2} to be true.
\item\label{chess:claim:4} you have not reasoned from the rules of chess and the game state (described in figure~\ref{fig:chess:board}) to \ref{chess:claim:2}.
\end{enumerate}

Claim~\ref{chess:claim:4} is true by assumption.

Further, claim~\ref{chess:claim:4} ensures that, as you have not reasoned from the rules of chess and the game state to \ref{chess:claim:2}, you do not hold claim~\ref{chess:claim:2} to be true due to a particular strategy that Black can enact.
Additionally, an account of why you hold claim \ref{chess:claim:2} is true does not follow from understanding the rules of chess and the game state alone.
For, an agent may understand the rules of chess and the game state while lacking the ability to demonstrate that claim~\ref{chess:claim:2} is true --- understanding the rules of chess and the game board is distinct from the ability to construct strategies.\nolinebreak
\footnote{
  Here I am relying on a principle \emph{like}:
  \begin{itemize}
  \item An agent's grasp of \(\Sigma\) can provide an account of why they hold \(\phi\) is true only if the agent has the ability to demonstrate that \(\phi\) follows from \(\Sigma\).
  \end{itemize}
  {
    \color{red}
    Whether or not this claim is true depends in part of how ability is understood\dots
  }

  Perhaps it is the case that claim~\ref{chess:claim:2} is sufficiently obvious to guarantee that the agent has the ability.
  Still, the reasoning involved differs only in complexity from the reasoning involved in showing that White cannot prevent Black from checkmating in four moves, and I doubt that this claim is sufficiently obvious.
}

Claim~\ref{chess:claim:2} follows from claim~\ref{chess:claim:1}.
For, if you are able to reason from the rules of chess and the game state to the proposition that White cannot prevent Black from occupying c4 on their second move, then \ref{chess:claim:2} must be true --- White cannot prevent Black from occupying c4 on their second move.
If White could prevent Black from occupying c4 on their second move, then you would not be able to reason from the rules of chess and the game state to an incompatible proposition.

My claim is important, given claim~\ref{chess:claim:4}.
However, the are (at least) two ways in which my claim relates to why you hold claim \ref{chess:claim:2} to be true.

First is testimony as a reason, roughly.
Extend the line of reasoning above.
You have my testimony that claim~\ref{chess:claim:1} is true, and claim~\ref{chess:claim:2} must be true if claim~\ref{chess:claim:1} is true, as noted.

This line of reasoning is sound.
Something puzzling.
Indirect appeal to your ability to reason.
I claim that you have an ability --- specifically the ability to reason from the rules of chess and the game board to the proposition that White cannot prevent Black from occupying c4 on their second move.
If you were to do the reasoning I claim you are able to do, then you would have a strategy for Black.

My claim entails that claim~\ref{chess:claim:2} is true.
However, my claim also entails that you do not require my claim in order to hold that claim~\ref{chess:claim:2} is true.

Instead of taking claim~\ref{chess:claim:1} to entail the truth of claim~\ref{chess:claim:2}, you may take claim~\ref{chess:claim:1} as a guarantee that you are able to provide a reason supporting the proposition that White cannot prevent Black from occupying c4 on their second move.

A sort of inferential license.


Second, testimony as guaranteeing reasoning.




\newpage
% \xskakset{defaultmoveid=1b}



\begin{itemize}
\item Testimony.
\item Overwrite with reasoning.
\item No need for testimony.
\end{itemize}

\begin{itemize}
\item Testimony.
\item Overwrite with ability to reason.
\item No need for testimony.
\end{itemize}


\begin{note}
  Some revision is needed.
  For, the basing relation is plausibly something imposed from a theoretical standpoint.
  And, I do not hold that the indirect reason is explanatory, as would be implied if this were part of the basing relation.
  Rather, the claim is that the agent holding \(\phi\) on the basis of \(\Sigma\) is important.
  However, this is also difficult, as it is not clear whether this is really what is important, or whether it is the premises from which the agent infer that \(\phi\) holds on the basis of \(\Sigma\) which is important.
  Still, whether one looking at the relation or the ability, an unrepresented reason is part of the puzzle.
  Either because it is part of a relation of support that the agent appeals to, or because it is involved in supporting reasons which stem from the agent's ability.
\end{note}

\begin{note}
  Still, at least in the chess scenario, what the agent doesn't have access to is the particular relation.
  The issue is that the agent hasn't worked out how the attitude is supported by the reasons --- \emph{not} that the agent is unaware of the reasons.
  This is an important distinction between the chess scenario and the shopping scenario.
  The point can be argued.
  For, one may hold that the move doesn't follow from the board and the rules, as one needs to work through the details.
  And, it is then the case that whatever this reason is, the agent puts \emph{that} in the relation of support.
  I kind of doubt that this argument would be successful, but it also doesn't matter.
  For, I'm not going to rely on the accessibility of the `antecedent' reason in any case.

  Hence, the setup in the simple case is about the relationship between `recognised' reasons and something which follows, but without knowing how it follows.
  The more complex cases then drop the `recognised' reasons, and it is simply about something following.
\end{note}

\begin{note}[Important distinction --- basis vs.\ reasons]
  The fine distinction is between an agent responding to reasons and an agent's action being supported by reasons.
  {
    \color{red} \emph{This!!!}
  }
  It is important to distinguish between the state of the board and the rules of chess, which are sufficient to derive the particular reasons, from this specific reason for which the move is possible.
  That it is a knight, knights can move, etc.

  This is perhaps a fine distinction in the chess scenario.
  However, consider a simple propositional deduction.
  The key idea is that it is possible for someone to understand the rules and the board, but fail to notice that some move is possible.
  Right, this works a little better when I string together a couple of moves, as then one needs to consider what moves White is able to make and so on.
  Hence, I do not think that it is \emph{too} fine of a distinction.

  Testimony is then understood to guarantee the existence of these reasons, while the agents ability is what allows the agent to make use of these reasons.

  It is in this sense that the reasons are `independent' of the agent.
  And, this is important in supporting the generalisation to supermarket cases.
  This isn't quite true.
  For, the supermarket can be motivated by focusing on reasoning, with the difference being that the agent is unaware of what they are reasoning from.
  Yet, it remains the case that the agent has the ability to do the reasoning, and this is what matters for my purposes.
  
\end{note}

\begin{note}[More on basis vs.\ reasons]
  The main claim is that the agent does not represent particular reasons for the possibility of a move.

  A possible resource to draw on here is Carrol.
  The distinction between mention and use, etc.\ is recognised and potentially helpful.
  The problem here is that the use/mention distinction isn't quite right, because the difference is in types of use, very roughly.
  In one case, an inference follows some structure, while in the other case some structure is cited in support of an inference.
\end{note}

\begin{note}[Does basis vs.\ reasons help with supervenience]
  Does basis vs.\ reasons help with supervenience?

  In a sense, this supports a kind of externalism.
  A difficulty here is the mention/use issue.
  For, it is tempting to say that the agent would be responding to specific reasons.
  However, this is somewhat different from the agent doing the reasoning.
  So, it's not super clear that the agent is appealing to those reasons that would be generated in the process of reasoning.

  More verbosely:
  There are additional, specific, reasons that demonstrate the possibility of a particular move.
  However, this is the result of some reasoning rather than an input to some reasoning.

  The issue is whether the agent is appealing to the \emph{existence of reasons} or the \emph{possibility of reasoning}.
  These are, arguably, two different things.
  I think the supermarket case incorporates both, but (at least in the basic cases) it is the possibility of reasoning (which would generate reasons) that is of interest.

  At the very least this distinction permits flexibility on what reasons are.
  For, I don't need to be committed to any kind of externalism about reasons.
\end{note}


\hozlinedash

\begin{itemize}
\item This is the distinction that I am interested in.
\item Main claim is that this makes a difference.
\item And, as a result, there is something to be said about the relationship between abilities and reasons.
\item Primary content of this paper is remarks on abilities and reasons.
\item Primary contribution is a pair of conditionals relating ability and indirectly responding.
\end{itemize}


\hozlinedash

\textbf{Reasons that the agent forms an attitude in response to are not necessarily those reasons which the agent responds to when forming the attitude.}

{
  \color{red}
  This statement isn't quite right.
  Or, at the very least is a little misleading, because it's not quite the case that the agent rewrites which reasons they hold an attitude in response to.
  Rather, it's how a particular reason is put to use.
  Whether this is as an input to further reasoning, or whether this informs the agent of reasoning that they are able to do.
  So, {\color{blue} this introduction needs a rewrite.}
}


(This is different to the relationship that \citeauthor{Neta:2019aa} notes between reasons why and reasons for which.
In short, reasons why are not always reasons for which, because there are explanatory relations which are in a sense `structural' and therefore are not seen as reasons from the agent's point of view.
So, I'm colourblind, but that is not a reason for which I believe the two different colours are the same colour, though it explains why.
Rather, the reasons for which is that both objects appear the same colour to me.)


\newpage


Difference between:
\begin{enumerate}
\item\label{bhess:r} That \emph{it is possible for Black to move their knight from d7 to b6} holds on the basis of the arrangement of the board and the rules of chess.
\item\label{chess:t} That \emph{it is possible for Black to move their knight from d7 to b6 follows from the arrangement of the board and the rules of chess} holds on the basis of my testimony.
\end{enumerate}


\subsection{Framing}
\label{sec:framing}

So, rewrite the above to avoid talk about the basing relation.
Then, I can talk about the basing relation as something that we project onto the agent in order to explain their reasons for acting.
For, we aren't really too interested in how the agent represents their reasons, at least on a simple understanding of the basing relation.
This usage kind of follows \textcite[e.g.\ 197]{Neta:2019aa}, where a basing relation can fail to obtain without the agent recognising this to be the case.

\begin{itemize}
\item Agent acts on the basis of the reason \emph{R}
\item Agent acts on the basis of their ability to respond to reason \emph{R} and the information that `\emph{R} obtains'.
\end{itemize}

The key distinction here is that the latter is represented, whereas the former is not.

\subsubsection{Objection}

The objection brewing here is that the agent representing themselves the structure of their reasons in a particular way doesn't necessarily demonstrate anything of interest.

The issue is whether I'm interested in this basing relation, or in ability.


\section{Supervenience}
\label{sec:supervenience}

\begin{note}
  The broad issue here is that one doesn't want to tie rationality to objective reasons.
  For, in many cases an agent may be unaware of what objective reasons there are, and yet this does not seem to entail that the agent is not rational.
  So, e.g.\ \citeauthor{Lord:2018aa} and \citeauthor{Kiesewetter:2017aa} restrict which objective reasons are of interest.
  These are those reasons which are recognised.

  So, subjectivism is a worry here.
  For, it's now the case that the agent may not be able to determine what they ought to do.
  Or something, though this objection is odd.

  Well, a better way to put the issue here is that we're interested in reasons, and how these relate to the agent doing stuff.
  Therefore, there's something normative going on.
  Yet, if we're interested in objective reasons, then it is unclear how one is going to obtain any applicable normative concept, as the agent doesn't have the information required to determine the normative state that obtains.
  Therefore, there must e some derived subjective normative state.
  But, then this simply appears to be the standard subjective normative state.
\end{note}

The objection here is that the reasons an agent has supervene on the agent's mind.

This is likely claimed in the literature somewhere, but if not this can be indirectly motivated by \citeauthor{Broome:2013aa}
's claim that rationality supervenes on the mind (\citeyear[151, etc.]{Broome:2013aa}).

However, there are a few ways the indirect link can be made.
\begin{itemize}
\item Rationality involves responding to reasons, etc.\
\item Rationality is about structure, and whether or not structure is supported depends on ability.
\end{itemize}

\begin{note}
  The link to rationality is not so smooth.
  For, one may argue that rationality should be divorced from responding to reasons.
  Instead, it is premised on agent's attitude toward what reasons they have.
  If this is the case, then the two positions are compatible.
  However, whether an agent is rational or not will then not depend on what reasons the agent has.
  For example, \citeauthor{Broome:2013aa}'s account of enkrasia is premised on the agent's belief about what reasons they have.
\end{note}

To demonstrate the problem, consider two contrasting situations.
In one situation the agent has the ability to respond to some reason, and in the other the agent does not.
The difference is generated by the presence or absence of some information within the agent's grasp.
For example, whether or not there is supporting evidence.

A cleaner example with respect to my view is that an agent is able to form a dependency relation of reasons that they do not have direct access to.
And, in these cases the agent is rational if those (independent) reasons exist, and is not rational otherwise.
However, if this is the case then the agent's rationality does not supervene on their mind, for the existence of the reasons in (by definition) independent of the agent's mind.
Hence, the supervenience claim must be denied.

\begin{note}
  This doesn't seem to follow.
  For, in order for superveniece to fail it must matter whether or not the agent actually has the ability.
  Yet, this does not immediately follow.
  For, one may argue that whether or not the agent has the ability is akin to whether or not the glass of gin really is gin.
  The agent may be rational in forming/maintaining the dependency independently of whether or not the dependency actually holds.
  So, there's a difference between the two kinds of supervenience.
  There's rationality on the one hand.
  And there's reasons on the other.
  Rationality is fine.
  Reasons is not.
  Reasons is not, because whether or not the agent actually has a reason goes beyond the agent's mind, so to speak.
  Hence, there must be some kind of factivity associated with reasons.
  \emph{However}, this does not require one to go all in.
  There is the possibility of a restricted kind of factivity, concerning only those reasons that the agent has access to, and even this may be restricted to counterfactual appearances.

  The small upshot here is that this doesn't require much in terms of general commitments.

  There are also other claims about supervenience and reasons to distinguish between.
  For example, it may still be up to the agent as to what counts as a reason, etc.
  This is potentially clear in the shopping case, where the dependent reason is something like a desire.
  It may be claimed here that the desirability is determined by the agent's overall mental state, and thus a kind of supervenience holds.
\end{note}

\begin{note}[The reduction claim]
  The reduction claim is that there's some way to reduce the basic reading, where the reason is `independent' of the agent to some other claim which is `dependent'.
  This may be better put in terms of representation or disposition, etc.

  It is perhaps important to note that this is not simply an issue of factivity.
  If factivity is a requirement, then the `independent' reason must exist.
  However, the issue is with whether the `independent' reason really is a reason.
  One may argue that it is only the ability that is a reason, and then argue as to whether this needs to be factive, etc.
  \end{note}

Object to this by running standard BIV type scenarios.
And, deny the move by \cite{Kiesewetter:2017aa} relating to backup reasons/appearances.
For, independently of the problems this faces, the appearance of an ability is certainly not the same thing as an ability.
\citeauthor{Kiesewetter:2017aa} has two arguments:
\begin{enumerate}
\item Externalism (7.4)
  \begin{itemize}
  \item The problem here is that this is by the letter rather than by the spirit.
  \end{itemize}
\item Backup (7.5)
  \begin{itemize}
  \item Appearances provide reasons, roughly.
  \item However, this doesn't sit well.
  \end{itemize}
\end{enumerate}

My intuition is to claim that the agent's rationality does differ, but the agent's reason to think they are rational does not.


\subsection{The Williamson idea}
\label{sec:williamson-idea}

The Williamson idea is that it's reasonable to hold that appearance is factive.
Hence, \citeauthor[Ch.\ 7]{Lord:2018aa} and \citeauthor{Kiesewetter:2017aa} are able to argue that appearances provide reasons.
Not necessarily the same (kind of) reasons, but sufficient reasons.

(
As an aside, this is somewhat interesting because it provides an account of why appearances are reason-giving.
One could argue that this is independently true, but the interesting point is that objective reasons are non-redundant even if they are reason giving.
For, if there were no objective reasons then there would not be an entailment from appearance to reason.
)

The strategy doesn't apply to the cases that I am interested in.
For, this would require me that the agent's confidence in their ability provides a reason.
However, the natural understanding of the scenarios is that the agent relies on their ability to guarantee the existence of a reason, rather than the ability \emph{itself} being a reason.

\subsection{Difference in the kind of case}
\label{sec:difference-kind-case}

There is a difference in the kind of case.
For, in the demon style cases it is not particularly clear what it is that the agent could do differently.
This is something like an ought implies can principle.
The idea, roughly, would seem to be that if the expected reasons do not obtain in the bad case, then some other reasons do obtain.
Yet, as the agent is in the bad case, then agent is unable to access whatever it is those reasons promote.
Hence, as the agent is unable to respond to those reasons, then those reasons can not be normative.
This \dots is interesting.
One then also needs the claim that ether set of reasons is normative, and then the conclusion follows by disjunctive elimination.
I.e.\ something determines what the agent is to do.
I mean, the easy version of this is performing the same argument for anything that the agent does not recognise.
One can always construct a different case, and therefore these can't do the work.

\begin{itemize}
\item The agent is able to respond to reasons.
\item For any reason not recognised by the agent, it is possible for the reason to not obtain and the agent to remain the same.
\item If the reason does not obtain then it is not possible for the agent to respond to the reason.
\item It is not possible to determine whether or not the reasons not recognised obtain.
\item So, it is not possible for any reason not recognised to be normative.
\end{itemize}

It's something more like transparency.

\begin{itemize}
\item Reason only if the agent can determine whether or not it obtains.
\item Agent cannot determine whether or not factivity obtains.
\item Factivity cannot be a reason.
\end{itemize}

This sort of argument generalises to ability, in a sense.

This strays a little from the core intuition, which was that the relevant issues in the demon case were outside the control of the agent, whereas in the ability case the agent has the option of relying on their ability.

Hence, there is an argument that the idea of motivating internalism by demon cases isn't so clear.
For one may think that demon cases remain an issue on this picture.
For, there's nothing about the claim that ability can matter which requires a difference in rationality between the demon cases.

This is probably a useful observation to make.
If the motivation for internalism comes from demon cases, then it's not a good argument for internalism.
For, this is a non-internalist position which does not require there to be a difference in the demon cases.

On the other hand, one has a way to generate demon-like cases, so an argument agains texternalism will go through.
However, this argument cannot be premised on intuitions regarding demon cases.
That is the single upshot of this observation.

Yet, it seems quite weak.
For, the intuition is that two situations that are indistinguishable from the point of view of the agent are indistinguishable from the perspective of rationality.
And, this is going to occur in cases of ability.
If the demon cases are illustrating this distinction, then there's nothing to really be said.

Right, it's hard to see how this can really be useful.
For, the issue is whether all reasons are represented.
I can agree that all represented reasons are internal in the relevant sense.
So, I can agree that the demon case doesn't make a difference to represented reasons.
In this sense the `looking the same' is the relevant intuition.
Where I depart is in whether representation is all that matters.

So, the only thing of use is the idea that it \emph{might} be the case that one has a representationalist presupposition in the demon cases.
This is how I can claim compatibility with the demon cases, yet still push for something of an externalist flavour.

\hozlinedash


\section*{Outline}
\label{sec:outline}


\begin{itemize}
\item Introduce something on a theoretical basis.
  \begin{itemize}
  \item The observation here is that we can process information that `sets up' anticipation.
  \item However, this isn't something that's explicitly stated, in part because it's relatively complex.
  \end{itemize}
\item This gives us something that could be used to understand agency.
  \begin{itemize}
  \item And this has a number of benefits, e.g.\ preserving various simple relations in the case of desire.
  \end{itemize}
\item As an attitude, there are (at least) two questions of interest
  \begin{enumerate}
  \item When is the agent permitted to hold this attitude
  \item What is permitted given on this attitude.
  \end{enumerate}
\item The second question is difficult.
  This can be seen by thinking about belief.
  Here, the Lewis example is useful.
  Lewis may have the belief that agent is guilty, have the evidence and so on.
  However, whether this allows Lewis to make an arrest is not clear.
  So, shouldn't expect anticipation to be any simpler.
\item The first question is also difficult.
  Goal here is to provide necessary and sufficient conditions.
  Ideally we would have something extensionally equivalent.
  Ability, and how this is understood.
  Looking for something explanatory.
  Trade off between complexity and understanding.
  So, look to squeeze.
  Get a necessary and a sufficient condition, separate.
  Also a flexible tool.
\item So at the end of the paper I will have explained this kind of attitude, and outlined a key part of understanding the permissibility of the attitude.
\item Big thing is ability, and async.
\end{itemize}

\begin{itemize}
\item Understanding of support gives rise to phenomena.
\item Can understand how this can occur, even if there's no simple expression.
\item Explain how an essential part works, and a couple of simple results.
\item Upshot is some progress on ability and rationality.
\end{itemize}

{
  \color{red}
  Have the basic phenomena.
  Anticipation and ability, etc.
  Interest is in interaction.
  Conditionals of the form:
  \begin{itemize}
  \item Ability -> Permitted to anticipate
  \item Permitted to anticipate -> Ability
  \end{itemize}
  Necessary and sufficient conditions.

  These are, roughly, `reason why' conditionals.
  The ability is there to explain why the agent is permitted to anticipate something.
  However, partial explanations.

  This is due to there being various things to say about ability.
  The main problem here is that ability is a modal, so it's instances are going to depend on how the modal is understood.
  Unless the structural features do enough, then we need \emph{at least} need to supplement the account with information about the relevant alternatives.
  
  Further, it's not clear that this could provide a full explanation.
  Non0deal agents, and something tied so strongly to action.
  Pragmatic encroachment, etc.\ in knowledge, similar worries here.

  {
    \color{red}
    Goal of the paper is to outline structural features of the relation, and identify some simple instances of the conditional.
    
  }

  {
    \color{blue}
    One way of understanding things is that I am informed that I have propositional support (abstracting from evidential justification), and I am informed that I can also provide doxastic support.

    The main idea is that the agent is okay to anticipate because this is a truth oriented attitude.
    And, here the ability is important, because the availability of the truth of \(\phi\) holding on the basis of \(\Sigma\) is given by the agent's ability.

    If agent simply holds that \(\phi\) on the basis of \(\Sigma\), then the agent is not responding to reasons, etc.\
    Assume that responding to reasons is important.
    Instead, something like a safety condition.
    It is unlikely that the relevant reasons do not hold.
    So, the agent is responding in accordance with reasons, roughly.
  }

  {
    \color{red}
    So, the problem is arguing for the conditionals.
    What I want is an assumption that I can make use of.
    This is an assumption about permissible action.

    Reasons.
    Some accessible, other inaccessible.
    Recognised and unrecognised.
    Assumption is that an action is permissible if it is supported by the available reasons.
    Different from being supported by recognised reasons.

    Well, the main issue is with the attitude.
    So I'm wondering when it's okay to base my attitude toward \(\phi\) on \(\Sigma\), without having demonstrated that \(\phi\) follows from \(\Sigma\).

    So, I'm \emph{not} making a general claim about rationality.
    Rather, I'm interested in a fairly straightforward attitude, that is of interest for a number of reasons.
    The attitude is, roughly, about relations between mental states.
    Holding \(\phi\) on the basis of \(\Sigma\) (without having demonstrated).
    How the attitude is put to use is something else (can see in the Lewis scenario that one can toy with factors independent of the attitude to change whether or not it seems rational).
    First step is figuring out the attitude.
    So, looking at ability to reason.

    The argument is that this is an attitude of interest, and that there are these links to ability.
    This doesn't tell us that the agent is going to be able to do anything with the attitude.
    So, then, attitude with conjecture that it is rational for this attitude to be the basis of action in certain cases.
    Perhaps this is only due to the limited resources of the agent.
    Or perhaps the fact that an ideal agent will always have the results clouds more general observation.
    
  }


  
  Often the case that necessary and sufficient conditions provide an analysis.
  E.g.\ in the case of knowledge.
  Reduction to other concepts.

  Not the case here.
  On the one hand, something of a category mistake.
  

  Simple to see in the existential case.
  Anticipate, then it must be possible, but the contraposition doesn't hold.
}


\begin{itemize}
\item Includes a fairly straightforward account of non-voluntarism, if a link between anticipation and belief is assumed.
\end{itemize}


\begin{itemize}
\item There are ways to complicate the basic puzzle.
\item The hypothesis here is that these are explained by the link between anticipation and action.
\item I assume a simple picture, where the agent is permitted to act on an anticipated proposition, but this isn't always going to be the case.
\item When this fails, it is due to limitations placed on what an agent can do with an anticipated proposition, rather than limitations on anticipation.
\end{itemize}

{
  \color{red}
  The `you don't need me to tell you' is an assertion.
  Also works as a question.
  Might even work better as a question, as the intuitive response is `no'.
  
}





\section{Simple example}
\label{sec:simple-example}

{
  \color{red}
  A better example may be chess outlining moves.
  This explains the basic problem, but the arithmetic case is surely easier to work with.
}

\begin{itemize}
\item \(22^{2} = 484\) but you don't need me to tell you.
  A basic grasp of arithmetic allows you to calculate that twenty-two squared is four hundred and eighty-four.
\item \(22^{2} = 484\) follows from simple arithmetic, but you don’t need me to tell you.
\item Split:
  \begin{enumerate}
  \item\label{se:1} \(22^{2} = 484\) follows from arithmetic, and
  \item\label{se:2} I don't need to tell you \ref{se:1}.
  \end{enumerate}
  \ref{se:1} states a relation between the equality and arithmetic.
\item Interest is in
  \begin{enumerate}[label=\ref{se:1}\alph*.]
  \item\label{se:1:t} Agent holding that \emph{\(22^{2} = 484\) follows from arithmetic} on the basis of your testimony.
  \item\label{se:1:d} Agent holding that \(22^{2} = 484\) on the basis of arithmetic.
  \end{enumerate}
  \footnote{
    \color{red}
    \begin{enumerate}
    \item Agent holds that [\(\phi\) holds on the basis of \(\Sigma\)] on the basis of testimony.
    \item Agent holds that \(\phi\) on the basis of \(\Sigma\)
    \end{enumerate}

    \begin{itemize}
    \item Put another way, in the second case the agent takes \(\Sigma\) to be sufficient for them holding \(\phi\).
    \item In the first case, the agent needs \emph{additional} support.
    \end{itemize}
  }
\item In both \ref{se:1:t} and \ref{se:1:d} proposition and support.
  In both cases agent holds that \(22^{2} = 484\) follows from arithmetic.
\item However, in \ref{se:1:t} appeals to additional supporting information, testimony.
  For example, agent does not have the ability to demonstrate that \(22^{2} = 484\) follows from arithmetic, and so they appeal to testimony.
\item In \ref{se:1:d} agent holds that \(22^{2} = 484\) follows from arithmetic.
  For example, agent has calculated the result of \(22^{2}\) to be \(484\) by using arithmetic.
\item If agent has calculated, then they do not need to make an additional appeal to testimony.
  Agent witnesses the relation between the sum and arithmetic by performing the calculation.
  In this respect, the agent does not need to rely on an additional source of information.
  The sum follows by reasoning.\nolinebreak
  \footnote{
    Simple propositional logic analogue.

    Any axiomatization of proposition logic.
    \[
      ((P \rightarrow Q) \rightarrow R) \rightarrow ((R \rightarrow P) \rightarrow (S \rightarrow P))
    \]
    \cite{lukasiewicz:1948aa}
  }

  {
    \color{red}
    If one thinks of reasoning as a rule governed activity, may say that the agent possesses the rules or something like this.
  }

  In both cases the relation can be used.
  Is it the case that \(22^{2} = 484\)?
  Yes, on the basis of testimony of a fellow agent.
  Yes, on the basis of my calculating that \(22^{2} = 484\).

  This iterates.
  So now on the basis of me testimony of the testimony of your testimony, etc.

  This is not a use/mention distinction. (Perhaps an additional example in a footnote)\nolinebreak
  \footnote{
    Tortoise and the hare.
    See \cite{Simchen:2001aa}, perhaps.
  }
  Rather, distinction between conditions of use.
  One's own reasoning and a distinct source of information (another agent).
  Structural and licensed\nolinebreak
  \footnote{
    Maybe `accredited', `signed', or `attributed' relation?
  }
  relations.\nolinebreak
  \footnote{
    This is local and targeted.
    Agent may need to appeal to their ability, and their understanding of arithmetic may be based on testimony.

    Agent needs to appeal to their ability to do (simple) arithmetic, but this is different from appealing to a different source of information.
  }\nolinebreak
  \footnote{
    `Licensed' in the sense that one has been granted the relation by an authority, in a sense.
  }

  No trouble in understanding the conditions of use as outsourced (testimony).
  However, let us turn to the second part of the assertion.

  \begin{itemize}
  \item You don't need to tell me that \(22^{2} = 484\) follows from arithmetic.
  \end{itemize}


  This is not the only reason that you don't need me to tell you that \(22^{2}\) is \(484\).
  Someone else may tell you, you may have access to a calculator, perhaps you purchased twenty-two items priced twenty-two dollars and inferred the equality after looking at the total on your receipt, and so on.

  However, salient disambiguation is that I have the ability to calculate the result of \(22^{2}\).

\item You don't need me to tell you what \(22^{2}\) is.

  Metaphorically, I don't need a license.
  I do not need to appeal to testimony to support holding \(22^{2} = 484\) on the basis of arithmetic.
\end{itemize}

\begin{itemize}
\item As things stand, this isn't too difficult.
\item I am informed that I have the ability.
  However, this comes from testimony.
  So it doesn't seem as though I have a structural relation, because I rely on the testimony to be confident that I have the ability.
\item As an example, consider a theorem.
  No idea whether I can prove the theorem.
  You tell me I can, and looking at the theorem I have no reason to think this is the case other than your testimony.
\item The final part of the puzzle is that I am independently confident that I can calculate \(22^{2}\).
  This I am confident of independent of your testimony.
\item Therefore, you've told me the result of exercising my ability to calculate.
  The slip of information explains why, but it is arithmetic for which.
\end{itemize}

\newpage


\section{Introduction}
\label{sec:introduction}

Simple example: two people of different heights.
Shorter person reaches up and touches the ceiling, taller person learns that they can touch the ceiling.
Straightforward deduction, barring something strange.

Calculation example.
Touching the ceiling wasn't shared, the answer here is shared.
On the one had testimony, as this is new information, etc.\
On the other hand, it is the same as the ceiling, and it's the proposition that's different.

Here, set up as a straightforward case of `testimony'.
Shay touches the ceiling.
Learn this via observation, and also that Taylor can touch the ceiling.

Here, Taylor is confident they can touch the ceiling because they have been informed that Shay can touch the ceiling.
However, the support for Taylor being able to touch the ceiling is independent.
Taylor's ability to touch the ceiling is independent from Shay's ability to touch the ceiling.
For, if we keep Taylor and the ceiling fixed, then Shay's height can vary without affecting whether Taylor can touch the ceiling.
Shay's action allowed Taylor to do a simple piece of reasoning.
The difference is between \emph{how} Taylor came to be confident and \emph{why} Taylor is confident.

The argument here is that we're talking about ability.
Hence, we're talking about what Taylor can do.
Shay's reach doesn't need to factor into this.
That is, the ability is based on it being possible for Taylor to reach up and touch the ceiling.
It would be true regardless of Shay's height.
How things are set up all for Taylor to construct a witnessing event.

Suppose Taylor touches the ceiling.
Taylor constructs a witnessing event.
Then the how and why is the same.

Suppose Shay states `Taylor can touch the ceiling'.
Taylor doesn't have access to an independent verification, so the how and why is testimony.
It may be the case that the ceiling can be raised and lowered.
So, there's this additional variable that needs to be added to make the `can' statement true, but Taylor doesn't have access to this.
Need something that goes beyond what Taylor is aware of to construct a witnessing event.
So how and why is the same.


`X Calculated'

`Y Can calculate'

The first statement allows a move from: `Can calculate' to `can calculate that the solution is \dots'

The question is how and why.
The statement involving the solution would not be made without appealing to the statement made by the companion.
However, the agent has all the information, and is able to do simple addition and multiplication.
Hence, the agent is able to witness the calculation independently.
There is nothing hidden.\nolinebreak
\footnote{
  This is not the same as typing in some integration into a computer program, where it may be solved using some techniques that the agent has not mastered.
}
So, the reason \emph{why} the statement is true does not reference the companion.
Yet, without the companion the agent would not be aware that the statement is true prior to doing the calculation.

The basic problem here is that if the statement is true then it is true independently of the companion.
Yet, recognition that the statement is true depends on the companion.\nolinebreak
\footnote{
  This is different from hidden information, so I should probably rework the ceiling example.
  Or, perhaps the observation is that one can avoid this problem by making it clear that testimony is part of why due to hidden information.
}

{
  \color{red}
  Something is missing here.
  It's typically the case that \(\phi\) is independent of being informed that \(\phi\).
  However, being informed that \(\phi\) does not provide support for \(\phi\), independent of being informed.
  So, the difference to more standard cases is that with ability the agent is aware that they can support \(\phi\) independently of being informed.
  However, that they are able to support independently still relies on the agent being informed that they have the ability.

  \begin{itemize}
  \item \(\phi\) is true but you do not know it.
  \item \(\phi\) is true and you needed me to tell you.
  \item \(\phi\) is true but you did not need me to tell you.
  \end{itemize}
  The second is a variation on the first.
}

{
  \color{red}
  Could possibly include the diagram here.
}


{
  \color{red}
  Interested in attitude.
  Kind of like belief, and if we take the standard belief/desire dichotomy there this is a kind of belief.
  Still, preference is to sketch out an attitude and then see whether it reduces to belief, or to any other recognised attitude.
  \emph{Anticipate}, in the sense the agent anticipates support.
  This is the main feature of the attitude.

  Restructure the above scenarios.
}

Two questions.
\begin{enumerate}
\item Why anticipation?
\item When is anticipation permissible?
\end{enumerate}

{
  \color{red}
  The permissibility stuff is relatively weak.
  I have two broad conditionals, that do some approximation.
  But these don't amount to necessary and sufficient conditions.
  Sketch out some of the relation between ability and permissibility.
  Template for understanding.
  As I take the primary objection to be the way in which this is not something that can be captured by thinking about ideal rationality.

  Involuntarism.
}

Question is whether this relation is mirrored in the use of ability.
Whether in performing the act one relies on the ability, while the licence to perform the act rests on something else.


\section{Literature}
\label{sec:literature}

\begin{itemize}
\item \textcite{Worsnip:2018aa}
\end{itemize}

The type of scenario I have outlined can be seen as a simple variation on a type of scenario consider by \textcite{Worsnip:2018aa} (among others).

The main difference is that in the case of \citeauthor{Worsnip:2018aa} there is testimony that \emph{indirectly} `destroys' the possibility of a structural relation.
Whereas in these kinds of cases there is testimony that \emph{indirectly} `creates' the possibility of a structural relation.

A cleaner parallel is with \cite{Christensen:2007aa} and the drug example.
This sort of issue is picked up on by \citeauthor{Neta:2019aa} (\citeyear[189]{Neta:2019aa}).

\begin{itemize}
\item In these scenarios an agent receives testimony that some reasons do not form a basis for holding a certain proposition to be true, but doesn't have access to `why'.
\item Contrast: testimony that some reasons do form a basis for holding a certain proposition to be true, but doesn't have access to `why'.
\end{itemize}

These arguments cause trouble for harmony between evidence and coherence.
This isn't something I'm going to explore.
However, if you are persuaded by coherence, then one way of understand the scenario I'm interested in is the recognition of coherence amongst attitudes.


\section{Objection}
\label{sec:objection}

It seems as though \citeauthor{Neta:2019aa}'s claim that reasons for which are always reasons why presents a difficulty for my account of what's going on.
For, it does not seem as though any of the indirect reasons can be reasons why.
A quick way to argue for this is to note that the reasons need not exist, and the story would continue in the same way, with the only difference being that my testimony is not truthful.

The core of the problem, then, is the claim that something can be a reason for which only if it can also be a reason why.
I sort of deny this.

The basic objection is that there is some rewriting.
Start with testimony, then rewrite so that testimony is avoided.

So, in a sense the reason that you came to hold the attitude is testimony.
Yet, this isn't the reason for which you now hold the attitude.

The issue here is whether some kind of causation is involved.
For, in the cases I consider it is for sure the case that no causation is involved.

A potential worry, then, is the divergence between causality and the agent's representation.
However, in these types of cases, the agent's representation is likely part of the causal explanation.
In that, it not only matters that the agent has an attitude, but it may also matter why the agent takes that attitude to be supported.

\citeauthor{Neta:2019aa}'s hybrid representational-dispositionalist view may be useful to consider here.

However, it's not super obvious that there is a conflict, in a sense.
Because, the `de dicto' existence is still represented.

Hm, the setup is:
\begin{itemize}
\item the agent wants star fruit
\end{itemize}
The question is:
\begin{itemize}
\item whether this is part of the explanation of why the agent purchases carambola.
\end{itemize}
The problem is:
\begin{itemize}
\item the agent does not represent wanting star fruit when purchasing carambola.
\end{itemize}

So, the argument against explanation is that the initial want is not needed when the agent performs the action.
It is clear that the \emph{specific} reason doesn't do any work (as this can be toggled without affecting anything else).
Therefore, one can look to the surrounding reasons.

So, either we have an instance of the relation that is not explanatory, or we don't have an instance of the relation.

{\color{red}
  The problem here isn't the possibility of the reason not existing, though this illustrate the worry.
  Rather, it is with the existence of the reason doing any explanatory work.
}
The problem is often raised with reasons first.
Roughly, the appearance of the reason is sufficient.

Still, this only goes so far.
For, it remains the case that the `supporting' reasons, so to speak, do the explanation.


\section{Why anticipation?}
\label{sec:why-anticipation}

\begin{itemize}
\item The phenomena is puzzling/interesting. (Whether possible\dots)
\item Make use of this kind of stuff
  \begin{itemize}
  \item Shopping lists
  \item Morse, etc.
  \item Bounded agency.
    \begin{itemize}
    \item Understanding how the structure differs from ideal agents, though shares certain features.
    \end{itemize}
  \end{itemize}
  There may be other ways to understand what's going on here.
  However, this introduces complications.
  For, one needs to expand the basic understanding, and this may overgenerate.
  For example, with desire and testimony, it's kind of complex.
  If testimony alone, then other people, if only the agent, then what makes the agent's testimony unique?
  Avoid these problems by anticipating the relevant reasoning.

  Similarly, in cases of testimony there's a way to understand why the responsibility falls on the agent if they have the ability.
  \begin{scenario}
    Something of a test.
    I've got to come up with some answers, and then the file will tell me if I'm right.
    Not allowed to learn based on file.
    Have password.
    If the source is testimony, then I haven't learnt from the file.
    But, it seems that the file does the work.
    Confident that \(\phi\) on the basis of the file, because I can cut out the middleman.
    So the issue is whether this is permissible.
    Unclear.
    Difficulty is that there are typically many other factors.

    Another way to look at this is to ask whether I am the middleman can tell you.
    It's certainly not clear that learning by accident would be okay, though likewise it's not clear where the fault is.
    But this isn't unique to these scenarios.

    In footnote:
    Friends, playground has age limit, one is over, gate is open for repairs, owner only verifies one.
    Better with going into red barns, etc.
  \end{scenario}
\end{itemize}

{
  \color{red}
  Can argue that this reduces to testimony, of a kind.
  But there are cases where this would be unsatisfactory.
  \begin{itemize}
  \item Shopping list
  \item Morse and Lewis
  \end{itemize}
  Furthermore, if we are careful with constructing these scenarios, there are cases where testimony doesn't really work.
  \begin{itemize}
  \item Logic students.
  \item Wilson in a footnote.
  \end{itemize}
}

\begin{itemize}
\item Lord and responding to reasons.
\item Shopping lists and desires.
\item Epistemic cases where information of ability seems irrelevant.
\end{itemize}



\section{Can}
\label{sec:can}

\begin{itemize}
\item Interest is in the agent's ability to reason.
\item Conceptual, rather than semantic, so while we use `can' for simplicity, this isn't an account of `can'.
\item Modal, so restricted possibility of a kind.
  Link to ability modals.
  Do not claim that the concept is the same as ability modals --- due to (lack of) focus on action.
\item Target is specific ability, but motivation from general considerations.
\item The approach avoids being ad-hoc/overfitting, and allows us to leverage general observations about ability.
  \begin{itemize}
  \item It's not immediate that general observations hold for specific cases, so this is something of a balancing act.
    That is, observe for some ability, show the same holds for reasoning, etc.\
  \end{itemize}
\end{itemize}

\begin{scenario}[Crossword puzzle (static)]
  \begin{itemize}
  \item Crossword puzzle.
  \item Can\(_{\forall}\), because the agent has some skill, knows the vocab, will correct mistakes, etc.
  \item Can\(_{\exists}\), because the agent has some skill, but they might get stuck with a wrong entry and be convinced that they don't know the vocab for other entries.
  \item Not can\(_{\forall}\), because ``same as above''
  \item Not can \(_{\exists}\), because the agent doesn't have the vocabulary.
  \end{itemize}
  \begin{itemize}
  \item Can\(_{\forall}\) implies can\(_{\exists}\), not conversely.
  \item The difference is in how the agent responds to the development of the solution.
  \end{itemize}
\end{scenario}

\begin{itemize}
\item The scenario applies to Lewis.
\item Mistaken inference on some piece of information, wouldn't happen if ordered differently.
\item Lewis is confident that they can\(_{\forall}\) reason, by stipulation.
\end{itemize}

In the static scenario, it is only the agent's own actions that develop the scenario.
Static, because the confounder doesn't change.
Dynamic, where the confounder does change.

\begin{scenario}[Dynamic]
  \begin{itemize}
  \item Tennis players.
  \item Vary the skill of player \(B\).
  \end{itemize}
\end{scenario}

\begin{itemize}
\item No straightforward parallel in the Lewis example.
\item However, before the investigation starts, the information that Lewis will work through can be considered dynamic.
\item Hence, same idea can apply.
\end{itemize}

Difference is in response to developing events.
Modal informs both the actions that the agent can take, and also the events considered.

Clearer in the dynamic case, where we do not consider the opposition playing a stellar game.\nolinebreak
\footnote{
  Here we get something like a (reverse) sobel sequence.
}
Failure and existential.

This is a general observation.
Motivates an existential analysis.



Parallel the difference between a strategy and a winning strategy, to some extent, though we do not have the additional background assumptions of game/decision theory to state the difference in these terms.

Note that the difference between the readings is in how the outcome is established.
Or, better put, how the agent responds to developments.

On the \(\forall\) reading, we don't assume that there's always a success, as we limit the kind of alternatives available.

\begin{scenario}[Dart]
  Etc.\
\end{scenario}

Can hit the board, but of course there are various ways to ensure that this doesn't happen.
Tennis works, and arguably so does the cipher.
We have something like a sobel sequence here.

This suggests that we're not dealing with a propositional modal.

On the one hand, a bare existential.

Choosing a single act doesn't allow for response.





\begin{itemize}
\item Modal, so restricted possibility.
\end{itemize}

\begin{itemize}
\item Observations
  \begin{enumerate}
  \item (At least) strong and weak readings.
    \begin{itemize}
    \item Racing case, Lewis discarding evidence, tennis.
    \item Variation on \citeauthor{Schwarz:2020aa} where there's a changing code.
      Analogy here to beating someone at chess.
      Can win, meaning that there's a chance I can play the strategy the person is not able to respond to, or can win in that I have a winning strategy, in effect.
    \end{itemize}
  \item Ability/action is implicit.
  \item Fixed ability.
  \item Ignore opportunity.
  \end{enumerate}
\end{itemize}

Two readings.

`Strong' and `weak'.

Pair of conditionals.

With reasoning, the weak instance is to rule out some stuff.
However, there are some interesting cases where the weak reading is true.
For example, Lewis discards evidence.
Too bad for Lewis, but plausible!

The primary intuition behind `can \(\phi\)' is that something happens and \(\phi\) comes about.




The simple proposal faces some immediate difficulties.
For, chance.
E.g.\ the dartboard example.
Throw, hit any area of the dartboard.
One reading where `Can hit area' is true, and another on which it is false.


Propositional restrictor analysis.

\begin{itemize}
\item Simple modal is a problem because the appropriate reading is an existential.
\item So, with strong and weak there's no straightforward way to capture the difference.
\end{itemize}

Observation is that we start with an existential, for which there is general agreement.
However, if this is the case then there's no simple dual.

One can play around with the what the existential captures.
However, this prevents manipulating the accessibility relation.
(This is \citeauthor{Mandelkern:2017aa}'s act conditional analysis, which places the existential on the act, and then has universal over the outcomes.)

It's also not easy to paraphrase.
For, although one could say that the weaker version of can is the absence of impossibility, this isn't quite right.
The problem is that the agent's ability restricts the relevant worlds.
With throwing a dart, there may not be much of a difference.
But, with reasoning there are clearer limits.
In a sense, it's not impossible for me calculate something complex.
However, if we fix my reasoning then it is.
Some clear parallels to counterfactuals here, in that some things are possible, but not counterfactually possible.

Using counterfactuals is a nice proxy for the different readings, but the trouble here is that we don't want all the things that are unique to counterfactuals.
And, triviality, or, I mean, we sort of have a way with counterfactuals if we already have an analysis of what the agent can do, because we need to specify something in the counterfactual antecedent.

Trouble is that this limits different readings of `can' to different modals, different ordering sources, or different propositions.

Taking things apart, the agent's reasoning provides the relevant restriction, so both readings share the same ordering source.
Further, we are interested in the same proposition, and it seems the same modal, as we're talking about ability.

The claim about the same modal here is delicate.
The idea is that we've fixed our attention on an ability of the agent.
This is what determines the relevant restriction.
Different modals have different restrictions.

A weaker argument is that there is an entailment between the two.





Assume that events are not dense, otherwise the existential version becomes trivial.
Though definitions could be rewritten.

If one considers finite sequences, then there must be a \(\phi\) point.
Proof is to suppose that there are no \(\phi\) points, then this holds for terminal nodes, the predicate is false at these, and so by induction from terminal to the root the predicate is false.

For the case of the primes, etc.\ the idea would be to add a recursive predicate.
In a finite amount of time the agent will produce the \(n\)th prime, and then go on to produce the \(n+1\)th prime, and so on.
Then, similar argument to show that this produces all of the primes.
Here, the match between natural language and logic form isn't particularly tight.
Still, these seem like fairly intuitive truth conditions.

\newpage

\section*{Quotes}
\label{sec:quotes}

\begin{itemize}
\item If your IQ is higher than 85, you don’t need me to tell you that this line of reasoning depends on one very big and very false assumption\dots
  \begin{itemize}
  \item \url{http://racingweight.com/blog/tag/standard-american-diet/}
  \end{itemize}
\item You don’t need me to tell you that unimaginable volumes of data are created every second. Data is the cornerstone of business in the digital age.
  \begin{itemize}
  \item \url{https://www.share.org/p/bl/et/blogid=17&blogaid=693}
  \end{itemize}
\item You don’t need me to tell you that it’s never a good idea for an owner to become mixed up in the personnel side of NFL business.
  \begin{itemize}
  \item \url{https://www.nbcdfw.com/news/sports/awaiting-the-day-jerry-jones-becomes-al-davis/1884674/}
  \end{itemize}
\item \dots However, you don't need me to tell you that simply claiming your own ignorance and hanging that other person out to dry isn't necessarily an effective communication strategy.
  \begin{itemize}
  \item \url{https://www.inc.com/kat-boogaard/4-phrases-that-are-better-than-i-dont-know.html}
  \end{itemize}
\end{itemize}

\section{Davidson}
\label{sec:davidson}

\citeauthor{Neta:2019aa} talks about Davidson's distinction between \emph{reasons that one has to act} and \emph{reasons for which one acts}, where the latter are the reasons which \emph{cause} and agent's action.

A concern is that in my scenarios causality and support come apart.
\citeauthor{Neta:2019aa} reconstructs this in terms of explanatory reasons in response to Anscombian `Why' questions.


\newpage

\printbibliography


\newpage

\section*{Old notes}
\label{sec:old-notes}

{
  \color{red}
  The idea is that there's no way to avoid the fact that access is grated by the companion.

  Maybe an observation to make here is that there are two why questions.
  So why is the statement true, and why is the agent confident that the statement is true.
  Then the issue is whether \dots but this would be hard, as I think I need confidence to support the conditionals that I have in mind.
  Because, it doesn't seem clear that the ability allows the agent to speculate independently of whether they recognise that they have the ability.
  Though this is tricky.
  For, there's a difference between being able to demonstrate that one is rational (so to speak) and being rational.
  So, I make the correct move in a game of chess, and one considers whether this is criticisable.
  Well, it was the right move, but I cannot demonstrate that it was not the wrong move.
  Hence, from my standpoint rationality is not guaranteed, though it was in fact rational.
  Even so, the testimony of the agent is still going to do some work is licensing the act.
  Yet, in acting one appeals only to the ability?

  So, the question is how this relates to acting.
  Whether there's a difference between support licensing the act, and between support for the act.
  Argument is that there is a difference.
}

\end{document}
