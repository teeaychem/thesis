\documentclass[10pt]{article}
% \usepackage[margin=1in]{geometry}
% \newcommand\hmmax{0}
% \newcommand\bmmax{0}

\usepackage{luatexbase} % While TeXLive is broken.

% % % Fonts% %
\usepackage[T1]{fontenc}
   % \usepackage{textcomp}
   % \usepackage{newtxtext}
   % \renewcommand\rmdefault{Pym} %\usepackage{mathptmx} %\usepackage{times}
\usepackage[complete, subscriptcorrection, slantedGreek, mtpfrak, mtpbbi, mtpcal]{mtpro2}
   \usepackage{bm}% Access to bold math symbols
   % \usepackage[onlytext]{MinionPro}
   \usepackage[no-math]{fontspec}
   \defaultfontfeatures{Ligatures=TeX,Numbers={Proportional}}
   \newfontfeature{Microtype}{protrusion=default;expansion=default;}
   \setmainfont[Ligatures=TeX]{TimesNRMTPro}
   \setsansfont[Microtype,Scale=MatchLowercase,Ligatures=TeX,BoldFont={* Semibold}]{Myriad Pro}
   \setmonofont[Scale=0.8]{Atlas Typewriter}
   % \usepackage{selnolig}% For suppressing certain typographic ligatures automatically
   \usepackage{microtype}
% % % % % % %
\usepackage{amsthm}         % (in part) For the defined environments
\usepackage{mathtools}      % Improves  on amsmaths/mtpro2
\usepackage{amsthm}         % (in part) For the defined environments
\usepackage{mathtools}      % Improves on amsmaths/mtpro2

% % % The bibliography % % %
\usepackage[backend=biber,
  style=authoryear-comp,
  bibstyle=authoryear,
  citestyle=authoryear-comp,
  uniquename=allinit,
  % giveninits=true,
  backref=false,
  hyperref=true,
  url=false,
  isbn=false,
]{biblatex}
\DeclareFieldFormat{postnote}{#1}
\DeclareFieldFormat{multipostnote}{#1}
% \setlength\bibitemsep{1.5\itemsep}
\addbibresource{Thesis.bib}

% % % % % % % % % % % % % % %

\usepackage[inline]{enumitem}
\setlist[itemize]{noitemsep}
\setlist[description]{noitemsep,style=unboxed,leftmargin=.5cm,font=\normalfont\space}
\setlist[enumerate]{noitemsep}

% % % The following section relates to theorems, etc. % % %
\usepackage{thmtools}

\declaretheoremstyle[
spaceabove=6pt, spacebelow=6pt,
headfont=\normalfont\bfseries,
notefont=\mdseries, notebraces={(}{)},
bodyfont=\normalfont,
% postheadspace=1em,
% qed=\qedsymbol
]{defstyle}

\declaretheoremstyle[
spaceabove=6pt, spacebelow=6pt,
headfont=\normalfont\bfseries,
notefont=\normalfont\bfseries, notebraces={}{},
bodyfont=\normalfont,
% postheadspace=1em,
% qed=\qedsymbol
]{defsstyle}


\declaretheoremstyle[
spaceabove=6pt, spacebelow=6pt,
headfont=\normalfont\bfseries,
notefont=\normalfont\bfseries, notebraces={}{},
bodyfont=\normalfont\color{red},
% postheadspace=1em,
qed=\qedsymbol
]{notestyle}

\declaretheorem[name=Theorem,numberwithin=section]{theorem}
\declaretheorem[sibling=theorem,style=remark]{remark}
\declaretheorem[sibling=theorem,name=Corollary]{corollary}
\declaretheorem[sibling=theorem,name=Lemma]{lemma}
\declaretheorem[sibling=theorem,name=Fact]{fact}
\declaretheorem[sibling=theorem,name=Proposition]{proposition}
\declaretheorem[sibling=theorem,name=Definition,style=defstyle]{definition}
\declaretheorem[name=Definitions,numbered=no,style=defsstyle]{definitions}
\declaretheorem[sibling=theorem,name=Example,style=defstyle]{example}
\declaretheorem[name=Note,style=notestyle]{note}
\declaretheorem[name=Ramble,style=notestyle]{ramble}
\declaretheorem[name=Scenario,style=defstyle]{scenario}
% % % % % % % % % % % % % % % % % % % % % % % % % % % % % %

% % % Misc packages % % %
\usepackage{setspace}
% \usepackage{refcheck} % Can be used for checking references
% \usepackage{lineno}   % For line numbers
% \usepackage{hyphenat} % For \hyp{} hyphenation command, and general hyphenation stuff

% % % % % % % % % % % % %

% % % Red Math % % %
    \usepackage[usenames, dvipsnames]{xcolor}
    % \usepackage{everysel}
    % \EverySelectfont{\color{black}}
    % \everymath{\color{red}}
    % \everydisplay{\color{black}}
% % % % % % % % % %

\usepackage{pifont}
\newcommand{\hand}{\ding{43}}
\usepackage{array}
\usepackage{epigraph}

\usepackage{titlesec}
\usepackage[hidelinks,breaklinks]{hyperref}



\titleclass{\subsubsubsection}{straight}[\subsection]

\newcounter{subsubsubsection}[subsubsection]
\renewcommand\thesubsubsubsection{\thesubsubsection.\arabic{subsubsubsection}}
\renewcommand\theparagraph{\thesubsubsubsection.\arabic{paragraph}} % optional; useful if paragraphs are to be numbered

\titleformat{\subsubsubsection}
  {\normalfont\normalsize\bfseries}{\thesubsubsubsection}{1em}{}
\titlespacing*{\subsubsubsection}
{0pt}{3.25ex plus 1ex minus .2ex}{1.5ex plus .2ex}

\makeatletter
\renewcommand\paragraph{\@startsection{paragraph}{5}{\z@}%
  {3.25ex \@plus1ex \@minus.2ex}%
  {-1em}%
  {\normalfont\normalsize\bfseries}}
\renewcommand\subparagraph{\@startsection{subparagraph}{6}{\parindent}%
  {3.25ex \@plus1ex \@minus .2ex}%
  {-1em}%
  {\normalfont\normalsize\bfseries}}
\def\toclevel@subsubsubsection{4}
\def\toclevel@paragraph{5}
\def\toclevel@paragraph{6}
\def\l@subsubsubsection{\@dottedtocline{4}{7em}{4em}}
\def\l@paragraph{\@dottedtocline{5}{10em}{5em}}
\def\l@subparagraph{\@dottedtocline{6}{14em}{6em}}
\makeatother



\setcounter{secnumdepth}{4}
\setcounter{tocdepth}{4}

% \titleclass{\todopar}{straight}[\section]
% \newcounter{todopar}
% \renewcommand{\thetodopar}{\Alph{todopar}.}
% \titleformat{\todopar}[runin]{\normalfont\normalsize\bfseries\color{WildStrawberry}}{\thesection.\thetodopar}{\wordsep}{}
% \titlespacing*{\todopar} {\parindent}{3.25ex plus 1ex minus .2ex}{1em}

\title{Second Year Paper: Practical Reasoning}
\author{Benjamin Sparkes}
% \date{ }

\begin{document}


\citeauthor{Lewis:1999ab} holds that folk psychology is an extensive, but tacit, shared understanding of how we work mentally which holds that `a system of beliefs and desires tends to cause behaviour that serves the subject's desires according to [their] beliefs'. (\citeyear[320]{Lewis:1999ab})
In this respect folk psychology is a power instrument of prediction and explanation, and we can tell which predictions and explanations conform to its principles, even if we cannot systematically present those principles.
Mental states are characterised by their causal role in a system, and the particular content that these mental states have when the system is applied to an agent explains and predicts the particular behaviour of the agent.

\citeauthor{Lewis:1999ab} advances two important arguments about folk psychology.
First, given that folk psychology identifies mental states by their causal role, folk psychology does not carry information about what (if anything) instantiates the mental.
Second, even though folk psychology does not carry information about what instantiates causal roles (and hence is agnostic about how content works), it still places constraints on what the content of mental states can be, defined by the typical causal role of that content.

The primary upshot of the above understanding of folk psychology is that we have a shared understanding of how we work mentally which doesn't carry substantive metaphysically commitments about what the mental is, but is secured by its explanatory and predictive power.
This, in turn, allows us to construct a simple theory of the mental via conceptual analysis which can aid a metaphysical account of what there is.\nolinebreak
\footnote{\citeauthor{Lewis:1999ab}'s broader argument in \citetitle{Lewis:1999ab} is that mental states are contingently identical to physical states, hence the focus on the relationship between folk psychology and metaphysics.}

\citeauthor{Lewis:1999ab}'s argument that folk psychology does not carry information about what (if anything) instantiates the mental is a consequence of identifying mental states with causal roles in a certain kind of system.
For something to occupy a certain role is for that thing to function as part of a system, and so long as that system does not make reference to how it is instantiated, it will not carry information about what occupies any of the roles it specifies.
That folk psychology does not carry information to how it is instantiated can be seen as assumed by \citeauthor{Lewis:1999ab}'s characterisation of folk psychology as a \emph{theory}, if one takes theories to (roughly) be symbolic systems with rules for manipulation which seems implied by \citeauthor{Lewis:1999ab}, with a fixed interpretation with which the meaning of the symbolic expressions can be reconstructed (\citeyear[cf.][298]{Lewis:1999ab} and \citeauthor{Lewis:1999ab}'s discussion of Ramsification).
However, \citeauthor{Lewis:1999ab} also provides a direct argument by arguing that our folk-psychological conceptions of mental states cannot discriminate between what actually instantiates the causal role and some other unactualized alternative.
If folk psychology carried information about what instantiated causal roles we would be able to discriminate between distinct instantiations of the same role, but intuitively we cannot.
For example, consider whether pain could be instantiated in multiple ways; intuitively we reference only a certain experience which typically leads to certain patterns of behaviour and do not make reference to anything more.
Hence, as folk psychology concerns shared understanding, this intuition is evidence that folk psychology does not carry information about what (if anything) instantiates the mental.
(\citeyear[304]{Lewis:1999ab})

\citeauthor{Lewis:1999ab}'s argument that folk psychology places constraints on what the content of mental states can be denies that the content of mental states is given only by wide content; content which depends upon which external things you are suitably connected to be relations of acquaintance (\citeyear[312]{Lewis:1999ab}).
\citeauthor{Lewis:1999ab} argues that wide content gives a suitable account of what is believed, but not \emph{how} it is believed.
Narrow content, but contrast, is determined by the typical causal roles of mental states and gives information about how your behaviour is caused.
It is independent of what one is acquainted with.
For example, you and I may share the narrow content given by 'I want to go home', even though I (the agent) am neither acquainted with you nor your home, given that we both exhibit similar behaviour with respect to the contexts in which we act.
So, given that folk psychology posits a shared system of mental states which cause behaviour, and our best sense of what causes our behaviour appears to be accounted for by shared beliefs and desires which are acquainted with different external things, folk psychology posits the same (narrow) content to our mental states.

In summary, as folk psychology specifies both the function role of mental states and the content of mental states without reference to what instantiates these roles and their content, and so folk theory does not carry substantive metaphysical implications.

\vfill

\printbibliography

\end{document}


% % First Assignment

\paragraph{Writing Task}\mbox{ }

\citeauthor{Bratman:2007ab}'s Two Glasses of Wine scenario highlights a problem which arises from the combination of desire for general patterns of behaviour where on any given occasion the satisfaction of one desire leads to the frustration of the other deisre.
Below we present a simplified version of \citeauthor{Bratman:2007ab}'s scenario.
In the background we assume a principle which links desire to action by stating that if you have a desire and are able to perform an action to satisfy that desire, you will perform that action.

\begin{enumerate}
\item\label{w:gP} You have a desire for a general pattern of behaviour in which you have a pleasant dinner.
\item\label{w:gW} You have a desire for a general pattern of behaviour in which you engage in productive after dinner work.
\item\label{w:glass} If you have a glass of wine with your dinner, you will satisfy your desire for a pleasant dinner.
\item\label{w:condBad} If you have a second glass of wine, you will frustrate your desire to engage in productive after dinner work.
\item\label{w:oneGlass} Given~\ref{w:gP} and~\ref{w:glass}, on any given occasion you desire to have a glass of wine with each dinner.
\item\label{w:good} By~\ref{w:gW} and~\ref{w:condBad}, on any given occasion you desire not to have a second glass of wine with dinner.
\item\label{w:glassCond} On any given occasion, if you have a glass of wine with your dinner, you will desire a second glass of wine.
% \item\label{w:bad} By~\ref{w:condBad}, if you have a second glass of wine you will frustrate your desire to engage in productive after dinner work.
\item\label{w:rTwo} On any given occasion, if you have a glass of wine with your dinner, then by \ref{w:glassCond} and the background assumption you will have a second glass of wine with your dinner.
\item\label{w:conflct} By~\ref{w:oneGlass} and~\ref{w:rTwo}, on any given occasion if you satisfy your desire to have a pleasant dinner you will frustrate your desire not to have a second glass of wine with dinner (\ref{w:good}).
\item\label{w:notGood} Given that \ref{w:conflct} holds for any given occasion, you cannot satisfy your general desires given in~\ref{w:gP} and~\ref{w:gW}
\end{enumerate}

Steps \ref{w:glass} and \ref{w:condBad} link your behaviour on a given occasion to the satisfaction and frustration of your desires for general patterns of behavrious (premises \ref{w:gP} and~\ref{w:gW}, respectively).
Steps~\ref{w:oneGlass} and~\ref{w:good} relate your general desires to desires regarding particular occasions.
Premise \ref{w:glassCond} is the crucial premise to generating the problem, it is a premise about your desire on particular occasions, and does not directly conflict with your desires for general patterns of behaviour.
Step \ref{w:rTwo} uses step \ref{w:oneGlass} and the background assumption to state your behaviour on any given occasion given the assumption that you satisfy your desire for a glass of wine.
Step \ref{w:conflct} observes that your desires regarding particular occasions cannot be jointly satisfied, and step \ref{w:notGood} generalises this failure of satisfaction to a problem for your general pattern of behaviour.
The problem, then, is that two distinct desires for general patterns of behaviour may be conflict free when considered independently of individual instances of behaviour, but conflict on every instance.


\newpage
\paragraph{Original scenario}

\begin{scenario}[Wine]\label{sc:wine}
  Suppose you value both pleasant dinners and productive work after dinner.
  One pleasant aspect of dinner is a glass of wine.
  Indeed, two glasses would make the dinner even more pleasant.
  The problem is that a second glass of wine undermines your efforts to work after dinner.
  So you have an evaluative ranking concerning normal dinners: dinner with one glass of wine over dinner with two glasses.
  So far so good. The problem is that when you are in the middle of dinner, having had the first glass of wine, you frequently find yourself tempted.
  As you see it, your temptation is not merely a temporary, felt motivational pull in the direction of a second glass: if it were merely that we could simply say that, in at least one important sense, practical reason is on the side of your evaluative ranking.
  Your temptation, however, is more than that; or so, at least, it seems to you.
  Your temptation seems to involve a kind of evaluation, albeit an evaluation that is, you know, temporary.
  For a short period of time you seem to value the second glass of wine more highly than refraining from that second glass.
  It is not that you have temporarily come to value, quite generally, dinner with two glasses of wine over dinner with one glass.
  You still value an overall pattern of one glass over an overall pattern of two glasses; after all, productive after-dinner work remains of great importance to you.
  But in the middle of dinner, faced with the vivid and immediate prospect of a second glass this one time, you value two glasses over one glass just this one time.\nolinebreak
  \mbox{ }\hfill(\citeyear[257]{Bratman:2007ab})
\end{scenario}

% % Second Assignment


\citeauthor{Bratman:2007ab} proposes a conjecture to explain how agents can resist cases of temptation in which their evaluative ranking of present options at the time of action conflict with their general policies regarding their known evaluative ranking of options before and after the time of action.
This conjecture draws on two distinct aspects of temporally extended practical reasoning.

On the one had, the pragmatic role of policies regarding evaluative rankings in an agent's practical reasoning to enable complex, cross-temporal, and coordinating roles in order for an agent to satisfy their most important ends.
And, on the other hand, the role of certain evaluative judgements as having authority in allowing an agent to identify the structure of their practical reasoning as their own.

\citeauthor{Bratman:2007ab}'s conjecture is that in cases of temptation an on-balance judgement of practical rationality requires a convergence in these two aspects of an agent's practical reasoning, so that the agent sees their general policies of regarding evaluative rankings as their own, such that the conflicting evaluative ranking for the present option does not have a stronger claim to authority in reflecting what the agent values.

\citeauthor{Bratman:2007ab} makes a number of assumptions regarding practical reasoning.
\begin{itemize}
\item Practical reasoning involves weighing various pros and cons concerning alternative options.
\item Valuing something involves a policy of treating that thing as a justifying consideration in one's practical reasoning.
\item An \emph{on-balance} judgement states what is rational for you to do relative to all your relevant ends, valuings, and the like.
\item If an agent is able to reach an on-balance judgement, they will act in accordance with that judgement, so long as they act rationally.
\end{itemize}

Given these, it follows that in order for an agent to resist temptation, they must form an on-balance judgement they must form an on-balance judgement favouring resilience.
So, the two aspects of support \citeauthor{Bratman:2007ab} offers for his requirement (the pragmatic role of policies and agential authority) can be seen as insufficient but non-redundant parts of an unnecessary but sufficient condition for an on-balance judgement (which may be necessary in cases of temptation).

\citeauthor{Bratman:2007ab}'s argument for the pragmatic role of policies:

\begin{enumerate}
\item On-balance judgements are typically anchored in an agent's evaluative rankings of present options at the time of action.
\item Policies to act are, likewise, typically grounded in an agent's evaluative rankings.
\item\label{p:pol-role} Policies have complex, cross-temporal, and coordinating roles.
\item\label{p:bk-down} A break down in the roles described in \ref{p:pol-role} typically frustrate an agent's most important ends.
\item By \ref{p:bk-down}, an agent will not typically give evaluative weight to considerations (or value options) which conflict with their policies.
\end{enumerate}

The pragmatic role of policies, then, can help ensure that an agent does not succumb to reasoning about tempting options.
However, in cases of temptation an agent does give evaluative weight to considerations (or value options) which conflict with their policies and so this pragmatic role alone cannot explain how agents resist temptation.

\citeauthor{Bratman:2007ab}'s argument for the role of agential authority:

\begin{enumerate}
\item On-balance judgements of rationality belong to a framework of practical reasoning which is, in a strong sense, the agent's own framework.
\item An agent's present evaluative rankings have priority for on-balance judgements of rationality as they have the authority to articulate, in the face of conflict, the agent's relevant perspective on their present options.
\end{enumerate}

Agential authority likewise can help ensure that an agent does not succumb to reasoning about tempting options so long as the agent does not see their evaluative judgements as representing their own perspective.
However, in cases of temptation the agents evaluative rankings for the tempting option does have a claim to authority as they are the present evaluative rankings of the agent.

Still, in cases of temptation an agent's policies also have claim to authority.
Further, this is a distinctive claim to authority as the policy is grounded in cross-temporal judgements, while the tempting option is a singular judgement.
Therefore, the policy may have a greater claim to authority and provide the basis for an on-balance judgement, hence \citeauthor{Bratman:2007ab}'s conjecture.

\nocite{Mackie:1965aa}