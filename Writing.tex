\documentclass[10pt]{article}
% \usepackage[margin=1in]{geometry}
% \newcommand\hmmax{0}
% \newcommand\bmmax{0}

\usepackage{luatexbase} % While TeXLive is broken.

% % % Fonts% %
\usepackage[T1]{fontenc}
   % \usepackage{textcomp}
   % \usepackage{newtxtext}
   % \renewcommand\rmdefault{Pym} %\usepackage{mathptmx} %\usepackage{times}
\usepackage[complete, subscriptcorrection, slantedGreek, mtpfrak, mtpbbi, mtpcal]{mtpro2}
   \usepackage{bm}% Access to bold math symbols
   % \usepackage[onlytext]{MinionPro}
   \usepackage[no-math]{fontspec}
   \defaultfontfeatures{Ligatures=TeX,Numbers={Proportional}}
   \newfontfeature{Microtype}{protrusion=default;expansion=default;}
   \setmainfont[Ligatures=TeX]{TimesNRMTPro}
   \setsansfont[Microtype,Scale=MatchLowercase,Ligatures=TeX,BoldFont={* Semibold}]{Myriad Pro}
   \setmonofont[Scale=0.8]{Atlas Typewriter}
   % \usepackage{selnolig}% For suppressing certain typographic ligatures automatically
   \usepackage{microtype}
% % % % % % %
\usepackage{amsthm}         % (in part) For the defined environments
\usepackage{mathtools}      % Improves  on amsmaths/mtpro2
\usepackage{amsthm}         % (in part) For the defined environments
\usepackage{mathtools}      % Improves on amsmaths/mtpro2

% % % The bibliography % % %
\usepackage[backend=biber,
  style=authoryear-comp,
  bibstyle=authoryear,
  citestyle=authoryear-comp,
  uniquename=allinit,
  % giveninits=true,
  backref=false,
  hyperref=true,
  url=false,
  isbn=false,
]{biblatex}
\DeclareFieldFormat{postnote}{#1}
\DeclareFieldFormat{multipostnote}{#1}
% \setlength\bibitemsep{1.5\itemsep}
\addbibresource{Thesis.bib}

% % % % % % % % % % % % % % %

\usepackage[inline]{enumitem}
\setlist[itemize]{noitemsep}
\setlist[description]{noitemsep,style=unboxed,leftmargin=.5cm,font=\normalfont\space}
\setlist[enumerate]{noitemsep}

% % % The following section relates to theorems, etc. % % %
\usepackage{thmtools}

\declaretheoremstyle[
spaceabove=6pt, spacebelow=6pt,
headfont=\normalfont\bfseries,
notefont=\mdseries, notebraces={(}{)},
bodyfont=\normalfont,
% postheadspace=1em,
% qed=\qedsymbol
]{defstyle}

\declaretheoremstyle[
spaceabove=6pt, spacebelow=6pt,
headfont=\normalfont\bfseries,
notefont=\normalfont\bfseries, notebraces={}{},
bodyfont=\normalfont,
% postheadspace=1em,
% qed=\qedsymbol
]{defsstyle}


\declaretheoremstyle[
spaceabove=6pt, spacebelow=6pt,
headfont=\normalfont\bfseries,
notefont=\normalfont\bfseries, notebraces={}{},
bodyfont=\normalfont\color{red},
% postheadspace=1em,
qed=\qedsymbol
]{notestyle}

\declaretheorem[name=Theorem,numberwithin=section]{theorem}
\declaretheorem[sibling=theorem,style=remark]{remark}
\declaretheorem[sibling=theorem,name=Corollary]{corollary}
\declaretheorem[sibling=theorem,name=Lemma]{lemma}
\declaretheorem[sibling=theorem,name=Fact]{fact}
\declaretheorem[sibling=theorem,name=Proposition]{proposition}
\declaretheorem[sibling=theorem,name=Definition,style=defstyle]{definition}
\declaretheorem[name=Definitions,numbered=no,style=defsstyle]{definitions}
\declaretheorem[sibling=theorem,name=Example,style=defstyle]{example}
\declaretheorem[name=Note,style=notestyle]{note}
\declaretheorem[name=Ramble,style=notestyle]{ramble}
\declaretheorem[name=Scenario,style=defstyle]{scenario}
% % % % % % % % % % % % % % % % % % % % % % % % % % % % % %

% % % Misc packages % % %
\usepackage{setspace}
% \usepackage{refcheck} % Can be used for checking references
% \usepackage{lineno}   % For line numbers
% \usepackage{hyphenat} % For \hyp{} hyphenation command, and general hyphenation stuff

% % % % % % % % % % % % %

% % % Red Math % % %
    \usepackage[usenames, dvipsnames]{xcolor}
    % \usepackage{everysel}
    % \EverySelectfont{\color{black}}
    % \everymath{\color{red}}
    % \everydisplay{\color{black}}
% % % % % % % % % %

\usepackage{pifont}
\newcommand{\hand}{\ding{43}}
\usepackage{array}
\usepackage{epigraph}

\usepackage{titlesec}
\usepackage[hidelinks,breaklinks]{hyperref}



\titleclass{\subsubsubsection}{straight}[\subsection]

\newcounter{subsubsubsection}[subsubsection]
\renewcommand\thesubsubsubsection{\thesubsubsection.\arabic{subsubsubsection}}
\renewcommand\theparagraph{\thesubsubsubsection.\arabic{paragraph}} % optional; useful if paragraphs are to be numbered

\titleformat{\subsubsubsection}
  {\normalfont\normalsize\bfseries}{\thesubsubsubsection}{1em}{}
\titlespacing*{\subsubsubsection}
{0pt}{3.25ex plus 1ex minus .2ex}{1.5ex plus .2ex}

\makeatletter
\renewcommand\paragraph{\@startsection{paragraph}{5}{\z@}%
  {3.25ex \@plus1ex \@minus.2ex}%
  {-1em}%
  {\normalfont\normalsize\bfseries}}
\renewcommand\subparagraph{\@startsection{subparagraph}{6}{\parindent}%
  {3.25ex \@plus1ex \@minus .2ex}%
  {-1em}%
  {\normalfont\normalsize\bfseries}}
\def\toclevel@subsubsubsection{4}
\def\toclevel@paragraph{5}
\def\toclevel@paragraph{6}
\def\l@subsubsubsection{\@dottedtocline{4}{7em}{4em}}
\def\l@paragraph{\@dottedtocline{5}{10em}{5em}}
\def\l@subparagraph{\@dottedtocline{6}{14em}{6em}}
\makeatother



\setcounter{secnumdepth}{4}
\setcounter{tocdepth}{4}

% \titleclass{\todopar}{straight}[\section]
% \newcounter{todopar}
% \renewcommand{\thetodopar}{\Alph{todopar}.}
% \titleformat{\todopar}[runin]{\normalfont\normalsize\bfseries\color{WildStrawberry}}{\thesection.\thetodopar}{\wordsep}{}
% \titlespacing*{\todopar} {\parindent}{3.25ex plus 1ex minus .2ex}{1em}

\title{Second Year Paper: Practical Reasoning}
\author{Benjamin Sparkes}
% \date{ }

\begin{document}

\begin{quote}
  Note: I talk about judgements regarding what's `best', but I don't mean anything specific by this language.
  This can be replaced by desire, pro-attitude, etc.
\end{quote}

It seems to me that there is an important parallel between cases of temptation and cases of weakness of will, and that any proposed analysis of temptation cases should also respect intuitions regarding weakness of will.

For, stated broadly, in cases of both temptation and weakness of will an agent acts against their judgement about what is `best' for them.
This statement is very broad, and hides a key difference between the two types of cases.
In cases of temptation the agent's judgement about what is `best' is seen to be at some sort of fault, while in cases of weakness of will the agent's judgement about what is `best' fails to follow through into action.
In this respect, one may argue that the two cases call for distinct analyses.
However, even if this is right, the parallel may still constrain the scope of possible analysis.

For, consider an analysis of temptation cases which depends on the agent revising their initial judgement about what is `best' for them.
On such an analysis the agent does not act against their judgement about what is `best' for them when they act, only what is `best' for them while tempted.
So, strictly speaking the broad statement of temptation is wrong to hold that the agent acts against their judgement about what is `best' for them, but is right in so far as the agent acting against a salient judgement about what is `best' for them.
However, the success of such an analysis then seems to rest on the assumption that agent's act on the basis of their judgement about what is `best' for them at the moment of action, from which it then follows that in a case of weakness of will, and agent cannot (rationally) act against act judgement about what is `best', and hence it cannot be the case that their judgement about what is `best' fails to follow through into action.

Conversely, consider an analysis of weakness of will in which an agent acts against their judgement of what is `best' for them.
In this case, given that it is possible for an agent to act against their judgement, it does not seem necessary that an agent must revise their judgement about what is `best' in order to resist temptation.
Again, this rests on the same assumption noted above.

The key difficulty is that if an analysis is proposed on which an agent resists temptation through their judgement about what is `best' for them, it then seems difficult to explain how an agent can succumb to weakness of will.
Yet, weakness of will appears possible.
And, on the other hand, if an analysis is proposed on which weakness of will is possible, then it seems that an agent can resist temptation without revising their judgement about what is `best' for them, but it then seems that an agent can act against temptation without strength of will.
Or, in other words, weakness of will becomes a viable way to resist temptation.
Of course, it does not follow from the possibility of weakness of will that this is the only way to resist temptation, but likewise it is unclear that there are fine distinctions to be drawn if an agent is to act against their judgement.

This sketch of an argument needs to be worked out in a little more detail, but the basic upshot is that either an analysis of temptation cases in which an agent revises their judgement about what is `best' immediately prior acting is suspect or an analysis of weakness of will in which an agent does not revise their judgement about what is `best' is suspect.
Or, so it may seem.
The dilemma arises given the assumption that agent's act on their judgements about what is `best' immediately prior to acting.

I think there is scope for rejecting the assumption, but it's not obvious that the assumption is necessary for the dilemma to arise, and so there is no guarantee that any alternative assumption will not give rise to the same difficulty.
In broad outline, the idea is that an agent's judgements about what is `best' are the result of cognitive processes, and so an agent's judgements about what is `best' generalises over various contexts because the same may process feature in those contexts.
Agent's then (sometimes but perhaps always) act on the basis of what is `best' over general contexts, because how they act in specific contexts affects may affect how they will act in other contexts, and what judgements they form about what is `best' in other contexts.
This is very vague, and I don't have much more to say, but two examples may help.

For temptation, consider \citeauthor{Bratman:2007ab} two glasses of wine scenario, and assume that you really do value the second glass of wine over abstaining.
Still, you also value after dinner work, and you may reason that drinking the second glass this one time may lead to you drink a second glass on further occasions.
What you judge to be `best' in the context does not necessarily lead to the `best' overall pattern of behaviour.

For weakness of will, consider \citeauthor{Davidson:2001aa}'s toothbrush scenario, and assume that you really do value going to bed over brushing your teeth.
Still, one may also assume that you really do value brushing your teeth on the whole, and failing to brush your teeth this one time may lead you to neglecting to brush them on further occasions.
Again, what you judge to be `best' in the context does not necessarily lead to the `best' overall pattern of behaviour.

This doesn't explain the different intuitions we have about the respective cases.
However, I suspect that these intuitions arise from factors distinct from the core phenomena.
As something of a conjecture, it seems possible to turn cases of temptation into cases of weakness of will (e.g.\ by describing you as over-worked and in need of some relaxation) and similarly cases of weakness of will into cases of temptation (e.g.\ by describing you as fatigued from watching a movie late into the night).










\end{document}

Desires and conditions of satisfaction.
Have a proposition, and a certain attitude toward that proposition.
Assumption that this proposition is somehow represented in the mind of an agent, which I'm suspicious of, but an assumption that I'll endorse for now.
Conditions of satisfaction are too coarse, so we talk about degrees of satisfaction.
(Strength, valence, etc.)
So, useful to think of something like a function.
Proposition goes in, degrees come out.
Then, we get an ordering on propositions.
The ordering is what really matters, but it's a little simpler to think of values.

So, we have something like \(\delta(\rho) = v\).
Still, this may be thought to capture some kind of all-out-judgement.
\[\delta(\rho) = f(d_{1}(\rho), d_{2}(\rho), \dots)\]
We then get some kind of function from functions, where each of the functions captures particular desires, dispositions, etc.
Now, this is simply input and output.

One way of going beyond straightforward talk of inputs and outputs is to think of the function as computed.
So, the mind does something to arrive at a value judgement, strength, or whatever.
If we individuate desires by propositions, then this is close to what \textcite{Railton:2012aa} proposes.
We have something like a computation, which has at its core a feedback loop.
The strength of a desire, in \citeauthor{Railton:2012aa}'s case, isn't fixed, but changes.

However, even if we individuate desires by propositions, it doesn't follow that there needs to be a correspondence between desires and computations.
Instead, we may think of computations as determining the strengths of arbitrary collections of desires.
Even though the correspondence between desire and computations fails to hold by necessity, it's more helpful here to make \citeauthor{Pettit:1991aa}'s (\citeyear{Pettit:1991aa}) distinction between properties and prospects.
Every prospect involves the realisation of properties, and that desires primarily attach to properties.
Now, one can individuate computations by properties.
Can adapt \citeauthor{Railton:2012aa}'s proposal to this way of thinking about things.
We've now got something fairly fine grained, as you now how ways of composing desires with respect to properties.
And, following \citeauthor{Pettit:1991aa} it may be the case that comparisons between properties reduce to comparisons between properties, but this need not be the case.
Or, better, that the relation between the two is complex, and that one may be able to apply \citeauthor{Railton:2012aa}'s dynamic perspective to the composition of desires with respect to properties when determining desires over prospects.

Well, the distinction between properties and prospects doesn't seem too important, as for \citeauthor{Pettit:1991aa} it's about the structure of deliberation rather than anything else, and the fact that any property can be viewed as a certain kind of prospect makes the distinction above somewhat irrelevant.
In fact, \citeauthor{Railton:2012aa} is most naturally read as talking about properties, and this is really what everything must come down to.

Still, this helps capture the distinction I want to make.
Desires for properties, and prospects have these properties, so we opt for situations which fit desires.
Ah, but not quite, I want to mark a way in which this holds for properties as well.
Still, one way out of this is to deny that properties and prospects really are the same kinds of thing.
\emph{Then} I could start to recover the difference, but it's then unclear how prospects can have properties, 



``Run of the program'' Sipser Intro.\ to computation.
Automota theory. Moore automota. Büchi.


\end{document}

% Problem for an analysis of practical reasoning which want to find fault with the dynamics of the agent's reasoning.
% At best this can be done indirectly.
% We see this with \citeauthor{Bratman:1987aa}'s work.
% Need an explanation of why your current reasoning wouldn't allow you to desire, and why you privilege the desire.

By way of background, I'm assuming that desire (or any other pro-attitude) is a relation between an agent and a proposition and that the propositional component of the desire states conditions of satisfaction.

We can make a distinction between desire-satisfaction and agent-satisfaction (whether it is important to satisfy the desire held by an agent, or whether there's some independent fact of the matter about whether the agent will be satisfied, and hence about what they should desire).
Still, this gives us an evaluation of practical reasoning based on states, while I'll term state-satisfaction.\nolinebreak
\footnote{
  \citeauthor{Railton:2012aa} introduces some dynamics, but this is still a state-based theory.
  The agent updates their assessment of desire-satisfaction.
  This can either be in reference to agent-satisfaction, or this could be a process independent from reasoning, and so all that matters is desire-satisfaction.
}

Now, I'm somewhat interested in the distinction between desire-satisfaction and agent-satisfaction, but what I'm primarily interested in is direct assessments of the agent's reasoning, and the worry about state-satisfaction is that there's no direct connexion to the agent's reasoning.\nolinebreak
\footnote{
  From a certain point of view, state-satisfaction leads to variations on act-consequentialism, and focus on reasoning looks to be a form of indirect-consequentialism.
}
One may have the view that rational dynamics are selected on the basis of the states such dynamics bring about, and these dynamics determine whether an agent is rational or not.
This directly assesses the rationality of an agent's reasoning, but given that the starting point is satisfaction, it's unclear exactly how this works.
For it is remains the case that the satisfaction at the states which reasoning brings about, so it looks as though we can cut the evaluation of reasoning, and simply asses whether the agent brought about the appropriate states.
To escape this, it may seem that we need to appeal to something other than state-satisfaction.
However, this isn't quite so straightforward.

The see why something other than state-satisfaction is required, consider the temptation case.
Here, we consider the states of an agent over time, and we assume that at some intermediary choice point the agent's perspective on state-satisfaction is inconsistent with their overall perspective on state-satisfaction.
At the intermediary point, there's a difficulty in explaining how the agent's overall perspective is more important, and why this requires them to reason in a certain way.
So, we have \citeauthor{Bratman:1987aa}'s hybrid solution, where intentions generalise over states, and \citeauthor{Frankfurt:1971aa}ian higher-order attitudes endorse this generalisation.
To avoid temptation the agent identifies with some general form of state-satisfaction, and in this respect the agent has an indirect connexion to their own reasoning as they will derive satisfaction from their current tempted state by reasoning in a certain way, given certain background assumptions regarding the kind of reasoning available to the agent to ensure their general state-satisfaction at each choice point.
This background assumption that only certain forms of reasoning are available to the agent does significant work, but it's plausible to hold that this is true for agent's like us, and so we have an argument for adhering to an intention to drink a single glass of wine.

There is, however, another important assumption.
In the temptation case, whether you drink the second glass of wine direct affects whether you will be satisfied in certain states.
For, if you drink the second glass, you will not be able to engage in productive after dinner work.
It is less clear that this explanation can work if we assume that you only have a (defeasible) policy.
In this case you value productive after dinner work on the whole, but exceptions are permissible.
Given this, there's no trouble in drinking a second glass of wine on occasion.
And, it now seems as though we do not have an argument for adhering to an intention to drink a single glass of wine, as this rules out occasionally satisfying yourself by drinking a second glass of wine.
However, if you don't adopt an intention it seems as though you may drink a second glass on any given occasion, as this is compatible with refraining from the second glass on a sufficient number of subsequent occasions.
Still, one may appeal to similar constraints on the kind of reasoning available, and assume that the agent recognises that if they drink a second glass of wine too often, then they'll always drink the second glass of wine, and therefore violate their policy (which they reflectively endorse at each choice point).
Again, this seems to be a plausible restriction to hold for agents like us.

The general strategy has been to evaluate states in terms of state-satisfaction, add connexion between states, and observe that from some restricted class of rational dynamics, only certain dynamics are viable.
So, the agent's reasoning hasn't come under direct consideration.
Perhaps \citeauthor{Bratman:1987aa}'s hybrid view faces independent difficulties, but it is illustrative as an instance of this general strategy, and the strategy appears viable.

While the above strategy appears viable, I'm somewhat worried about the distinction between state-satisfaction and reasoning it presupposes.
One may think that state-satisfaction and reasoning are two sides of the same coin, so to speak; reasoning is a process, and this process can be `unravelled' into traces of satisfaction conditions for states.
In other words, to have a desire (or more broadly pro-attitude) is not to have conditions of satisfaction given independently of reasoning, but to be engaged in a process which identifies certain states as satisfactory.\nolinebreak
This might seem anti-Humean, but I don't think it is, as the notion of reasoning here is broad, and perhaps better put in terms of cognitive dynamics,\footnote{
  Perhaps it's also important to note here that I'm not assuming that there's a `single' process of reasoning, and that distinct concurrent processes may give rise to different traces.
}
as we don't need to assume that there is anything more to satisfaction than the conditions that result from unravelling the process, nor that certain process can be revised through judgements about what is an isn't the case.\nolinebreak
\footnote{
  This isn't necessarily a position I want to adopt.
  The hope is that there's something to this idea, but that it's relatively neutral with respect to whether one should be a Humean or anti-Humean.
}
A simple upshot of this way of thinking about things is that conditions of state-satisfaction are inherently linked to one another as traces of a process, so one doesn't need to coerce dependence between the satisfaction of distinct states.

This also gives us (the) two readings of \citeauthor{Bratman:1987aa}'s temptation case.
On the first reading, you have a single process, but due to environmental factors the standard dynamics aren't `functioning appropriately' (speaking very metaphorically).\nolinebreak
\footnote{
  Badly put, but the basic idea is that an agent's reasoning interacts with environment, and aspects of the environment are somewhat unexpected.
  In the specific setup of the scenario, the agent would realise that the process would interact in this way, but it may also be the case that this is recognised while in the environment (e.g.\ realising that one has drunk too much and probably shouldn't start a particular topic of conversation).
}
On the second, there are two distinct and conflicting processes, with a genuine question about which to `identify with' (again, speaking very metaphorically).

More broadly, the idea that you can unravel any process into a sequence of states suggests that most of what can be said in terms of state-satisfaction can be translated into talk about processes, and vice-versa, but it's unclear to me exactly what `most' amounts to.
Still, one aspect of this which relates to my other work is that it doesn't seem as though processes rely on (recognised) representational content, and so this way of thinking about things fits in with the puzzle about forgetting the ends to which you are undertaking the means, etc.










\end{document}

In \citeauthor{Bratman:2007ab}'s presentation of the Two Glasses of Wine scenario your reversal of evaluative ranking is hedged; `you \emph{seem} to value the second glass of wine more highly than refraining from that second glass' (\citeyear[emphasis added][257]{Bratman:2007ab}).
\citeauthor{Bratman:2007ab} goes on to argue that your temptation cannot be a simple inclination (\citeyear[cf.][258]{Bratman:2007ab}) and proposes a distinguished class evaluative rankings which have \emph{agential authority} such that under certain circumstances a ranking fail to constitute where an agent stands.
An evaluative ranking can the fail to command action, not in virtue of being a simple inclination, but in virtue of failing to have relevance in an agent's practical reasoning.

Broadly put, the straightforward characterisation of your evaluative rankings needs to be hedged as it does not distinguish between those which do and don't carry authority, and while your evaluative rankings are reversed in certain cases of temptation, this reversal need not be motivationally effective.
In this respect, your evaluative rankings are given and fixed, but where you stand in relation to these rankings is up to you.

I agree with \citeauthor{Bratman:2007ab} that your temptation is not mere inclination, but I disagree that your evaluative rankings should be taken as given and fixed.
Instead, I hold that your evaluative rankings are a result of your practical reasoning, and in cases of temptation your evaluative rankings change as a result of deliberation.
When you seem to value the second glass of wine more than refraining you really do value the glass more than refraining, and if your reasoning were to conclude at that moment you would drink the second glass.
However, you may engage in further reasoning, and this reasoning may lead to changes in your evaluative rankings.

It as clear that the evaluative rankings of agent's do change.
In \citeauthor{Bratman:2007ab}'s scenario you move from having an evaluative ranking in favour of a single glass to a ranking in favour of two.
Intuitively, this can be traced to a change in context.
You moved from a context in which you had not drunk a glass of wine, to a context in which you had.
Generally speaking, the relevant change in context and why it has an effect of your evaluative rankings may vary.
For example, you may rank one pair of shoes over another until you have the opportunity to try them on, and after gathering information about how they fit you, your ranking of the pairs may change.
In this case you receive new information which bears on your ranking, and your ranking changes in light of this information.
Still, in \citeauthor{Bratman:2007ab}'s scenario it does not appear that you gain new information.
You may come to know what it's like to have a glass of wine, but it is a reasonable assumption that you have drunk a glass of wine before, and so the change in context cannot be attributed to a change in information.
Instead, the relevant change in context appears to be how you reason with the information available to you, under the influence of the first glass of wine, you value a second glass more than refraining.

Supposing that a change in evaluative rankings is always due to a change in context, the ability for agent's to resist temptation in cases such as \citeauthor{Bratman:2007ab}'s scenario remains puzzling as after the first glass of wine there doesn't appear to be any further change in context.
We may suppose that your from the decision to resist before the effect of the first glass on your reasoning wears off, and likewise that nothing in your immediate environment changes.
If this is correct, as I will assume, then your resistance to temptation cannot (always) be straightforwardly explained by a change in context.
However, an agent's reasoning should not be taken as a function of the context in which they are in.
We noted that a change in reasoning may be attributed to a change in context, but agent's are also able to reason about the contexts they are in.
In this respect, after the first glass of wine you may reason about how you are reasoning given you have drunk a single glass of wine.
Put this way such reasoning may sound complex, but we routinely engage in such reasoning.
For example, we refrain from driving when we are tired as we recognise that the relevant decision making will be impaired, and we keep our distance from an opponent at the end of a game of tennis for fear of what our anger may lead us to do.
And, in line with these examples we may refrain from drinking a second glass of wine on recognition that our reasoning is unduly influenced by the first.

\citeauthor{Bratman:2007ab}'s proposal can be seen as an explanation of why we side with our alert, calmer, or sober, selves.
An alternative proposal is that we recognise that our evaluative rankings are the result of reasoning, and the reasoning of our alert, calmer, or sober, selves is better equipped to track what is of value to us, and so we may defer evaluative rankings to prior (or counterfactual) reasoning.
In cases of temptation, then, you may recognise that you are in a certain context, and change your reasoning on recognition of this.
Your evaluative rankings, then, may change due to how much confidence you have that your present reasoning is able to track what is of value to you.



\end{document}

% \citeauthor{Bratman:2007ab}'s solution then involves showing how there is complex value.




% \citeauthor{Lewis:1999ab} holds that folk psychology is an extensive, but tacit, shared understanding of how we work mentally which holds that `a system of beliefs and desires tends to cause behaviour that serves the subject's desires according to [their] beliefs'. (\citeyear[320]{Lewis:1999ab})
% In this respect folk psychology is a power instrument of prediction and explanation, and we can tell which predictions and explanations conform to its principles, even if we cannot systematically present those principles.
% Mental states are characterised by their causal role in a system, and the particular content that these mental states have when the system is applied to an agent explains and predicts the particular behaviour of the agent.

% \citeauthor{Lewis:1999ab} advances two important arguments about folk psychology.
% First, given that folk psychology identifies mental states by their causal role, folk psychology does not carry information about what (if anything) instantiates the mental.
% Second, even though folk psychology does not carry information about what instantiates causal roles (and hence is agnostic about how content works), it still places constraints on what the content of mental states can be, defined by the typical causal role of that content.

% The primary upshot of the above understanding of folk psychology is that we have a shared understanding of how we work mentally which doesn't carry substantive metaphysically commitments about what the mental is, but is secured by its explanatory and predictive power.
% This, in turn, allows us to construct a simple theory of the mental via conceptual analysis which can aid a metaphysical account of what there is.\nolinebreak
% \footnote{\citeauthor{Lewis:1999ab}'s broader argument in \citetitle{Lewis:1999ab} is that mental states are contingently identical to physical states, hence the focus on the relationship between folk psychology and metaphysics.}

% \citeauthor{Lewis:1999ab}'s argument that folk psychology does not carry information about what (if anything) instantiates the mental is a consequence of identifying mental states with causal roles in a certain kind of system.
% For something to occupy a certain role is for that thing to function as part of a system, and so long as that system does not make reference to how it is instantiated, it will not carry information about what occupies any of the roles it specifies.
% That folk psychology does not carry information to how it is instantiated can be seen as assumed by \citeauthor{Lewis:1999ab}'s characterisation of folk psychology as a \emph{theory}, if one takes theories to (roughly) be symbolic systems with rules for manipulation which seems implied by \citeauthor{Lewis:1999ab}, with a fixed interpretation with which the meaning of the symbolic expressions can be reconstructed (\citeyear[cf.][298]{Lewis:1999ab} and \citeauthor{Lewis:1999ab}'s discussion of Ramsification).
% However, \citeauthor{Lewis:1999ab} also provides a direct argument by arguing that our folk-psychological conceptions of mental states cannot discriminate between what actually instantiates the causal role and some other unactualized alternative.
% If folk psychology carried information about what instantiated causal roles we would be able to discriminate between distinct instantiations of the same role, but intuitively we cannot.
% For example, consider whether pain could be instantiated in multiple ways; intuitively we reference only a certain experience which typically leads to certain patterns of behaviour and do not make reference to anything more.
% Hence, as folk psychology concerns shared understanding, this intuition is evidence that folk psychology does not carry information about what (if anything) instantiates the mental.
% (\citeyear[304]{Lewis:1999ab})

% \citeauthor{Lewis:1999ab}'s argument that folk psychology places constraints on what the content of mental states can be denies that the content of mental states is given only by wide content; content which depends upon which external things you are suitably connected to be relations of acquaintance (\citeyear[312]{Lewis:1999ab}).
% \citeauthor{Lewis:1999ab} argues that wide content gives a suitable account of what is believed, but not \emph{how} it is believed.
% Narrow content, but contrast, is determined by the typical causal roles of mental states and gives information about how your behaviour is caused.
% It is independent of what one is acquainted with.
% For example, you and I may share the narrow content given by 'I want to go home', even though I (the agent) am neither acquainted with you nor your home, given that we both exhibit similar behaviour with respect to the contexts in which we act.
% So, given that folk psychology posits a shared system of mental states which cause behaviour, and our best sense of what causes our behaviour appears to be accounted for by shared beliefs and desires which are acquainted with different external things, folk psychology posits the same (narrow) content to our mental states.

% In summary, as folk psychology specifies both the function role of mental states and the content of mental states without reference to what instantiates these roles and their content, and so folk theory does not carry substantive metaphysical implications.

\vfill

\printbibliography

\end{document}


% % First Assignment

\paragraph{Writing Task}\mbox{ }

\citeauthor{Bratman:2007ab}'s Two Glasses of Wine scenario highlights a problem which arises from the combination of desire for general patterns of behaviour where on any given occasion the satisfaction of one desire leads to the frustration of the other deisre.
Below we present a simplified version of \citeauthor{Bratman:2007ab}'s scenario.
In the background we assume a principle which links desire to action by stating that if you have a desire and are able to perform an action to satisfy that desire, you will perform that action.

\begin{enumerate}
\item\label{w:gP} You have a desire for a general pattern of behaviour in which you have a pleasant dinner.
\item\label{w:gW} You have a desire for a general pattern of behaviour in which you engage in productive after dinner work.
\item\label{w:glass} If you have a glass of wine with your dinner, you will satisfy your desire for a pleasant dinner.
\item\label{w:condBad} If you have a second glass of wine, you will frustrate your desire to engage in productive after dinner work.
\item\label{w:oneGlass} Given~\ref{w:gP} and~\ref{w:glass}, on any given occasion you desire to have a glass of wine with each dinner.
\item\label{w:good} By~\ref{w:gW} and~\ref{w:condBad}, on any given occasion you desire not to have a second glass of wine with dinner.
\item\label{w:glassCond} On any given occasion, if you have a glass of wine with your dinner, you will desire a second glass of wine.
% \item\label{w:bad} By~\ref{w:condBad}, if you have a second glass of wine you will frustrate your desire to engage in productive after dinner work.
\item\label{w:rTwo} On any given occasion, if you have a glass of wine with your dinner, then by \ref{w:glassCond} and the background assumption you will have a second glass of wine with your dinner.
\item\label{w:conflct} By~\ref{w:oneGlass} and~\ref{w:rTwo}, on any given occasion if you satisfy your desire to have a pleasant dinner you will frustrate your desire not to have a second glass of wine with dinner (\ref{w:good}).
\item\label{w:notGood} Given that \ref{w:conflct} holds for any given occasion, you cannot satisfy your general desires given in~\ref{w:gP} and~\ref{w:gW}
\end{enumerate}

Steps \ref{w:glass} and \ref{w:condBad} link your behaviour on a given occasion to the satisfaction and frustration of your desires for general patterns of behavrious (premises \ref{w:gP} and~\ref{w:gW}, respectively).
Steps~\ref{w:oneGlass} and~\ref{w:good} relate your general desires to desires regarding particular occasions.
Premise \ref{w:glassCond} is the crucial premise to generating the problem, it is a premise about your desire on particular occasions, and does not directly conflict with your desires for general patterns of behaviour.
Step \ref{w:rTwo} uses step \ref{w:oneGlass} and the background assumption to state your behaviour on any given occasion given the assumption that you satisfy your desire for a glass of wine.
Step \ref{w:conflct} observes that your desires regarding particular occasions cannot be jointly satisfied, and step \ref{w:notGood} generalises this failure of satisfaction to a problem for your general pattern of behaviour.
The problem, then, is that two distinct desires for general patterns of behaviour may be conflict free when considered independently of individual instances of behaviour, but conflict on every instance.


\newpage
\paragraph{Original scenario}

\begin{scenario}[Wine]\label{sc:wine}
  Suppose you value both pleasant dinners and productive work after dinner.
  One pleasant aspect of dinner is a glass of wine.
  Indeed, two glasses would make the dinner even more pleasant.
  The problem is that a second glass of wine undermines your efforts to work after dinner.
  So you have an evaluative ranking concerning normal dinners: dinner with one glass of wine over dinner with two glasses.
  So far so good. The problem is that when you are in the middle of dinner, having had the first glass of wine, you frequently find yourself tempted.
  As you see it, your temptation is not merely a temporary, felt motivational pull in the direction of a second glass: if it were merely that we could simply say that, in at least one important sense, practical reason is on the side of your evaluative ranking.
  Your temptation, however, is more than that; or so, at least, it seems to you.
  Your temptation seems to involve a kind of evaluation, albeit an evaluation that is, you know, temporary.
  For a short period of time you seem to value the second glass of wine more highly than refraining from that second glass.
  It is not that you have temporarily come to value, quite generally, dinner with two glasses of wine over dinner with one glass.
  You still value an overall pattern of one glass over an overall pattern of two glasses; after all, productive after-dinner work remains of great importance to you.
  But in the middle of dinner, faced with the vivid and immediate prospect of a second glass this one time, you value two glasses over one glass just this one time.\nolinebreak
  \mbox{ }\hfill(\citeyear[257]{Bratman:2007ab})
\end{scenario}


% % Second Assignment


\citeauthor{Bratman:2007ab} proposes a conjecture to explain how agents can resist cases of temptation in which their evaluative ranking of present options at the time of action conflict with their general policies regarding their known evaluative ranking of options before and after the time of action.
This conjecture draws on two distinct aspects of temporally extended practical reasoning.

On the one had, the pragmatic role of policies regarding evaluative rankings in an agent's practical reasoning to enable complex, cross-temporal, and coordinating roles in order for an agent to satisfy their most important ends.
And, on the other hand, the role of certain evaluative judgements as having authority in allowing an agent to identify the structure of their practical reasoning as their own.

\citeauthor{Bratman:2007ab}'s conjecture is that in cases of temptation an on-balance judgement of practical rationality requires a convergence in these two aspects of an agent's practical reasoning, so that the agent sees their general policies of regarding evaluative rankings as their own, such that the conflicting evaluative ranking for the present option does not have a stronger claim to authority in reflecting what the agent values.

\citeauthor{Bratman:2007ab} makes a number of assumptions regarding practical reasoning.
\begin{itemize}
\item Practical reasoning involves weighing various pros and cons concerning alternative options.
\item Valuing something involves a policy of treating that thing as a justifying consideration in one's practical reasoning.
\item An \emph{on-balance} judgement states what is rational for you to do relative to all your relevant ends, valuings, and the like.
\item If an agent is able to reach an on-balance judgement, they will act in accordance with that judgement, so long as they act rationally.
\end{itemize}

Given these, it follows that in order for an agent to resist temptation, they must form an on-balance judgement they must form an on-balance judgement favouring resilience.
So, the two aspects of support \citeauthor{Bratman:2007ab} offers for his requirement (the pragmatic role of policies and agential authority) can be seen as insufficient but non-redundant parts of an unnecessary but sufficient condition for an on-balance judgement (which may be necessary in cases of temptation).

\citeauthor{Bratman:2007ab}'s argument for the pragmatic role of policies:

\begin{enumerate}
\item On-balance judgements are typically anchored in an agent's evaluative rankings of present options at the time of action.
\item Policies to act are, likewise, typically grounded in an agent's evaluative rankings.
\item\label{p:pol-role} Policies have complex, cross-temporal, and coordinating roles.
\item\label{p:bk-down} A break down in the roles described in \ref{p:pol-role} typically frustrate an agent's most important ends.
\item By \ref{p:bk-down}, an agent will not typically give evaluative weight to considerations (or value options) which conflict with their policies.
\end{enumerate}

The pragmatic role of policies, then, can help ensure that an agent does not succumb to reasoning about tempting options.
However, in cases of temptation an agent does give evaluative weight to considerations (or value options) which conflict with their policies and so this pragmatic role alone cannot explain how agents resist temptation.

\citeauthor{Bratman:2007ab}'s argument for the role of agential authority:

\begin{enumerate}
\item On-balance judgements of rationality belong to a framework of practical reasoning which is, in a strong sense, the agent's own framework.
\item An agent's present evaluative rankings have priority for on-balance judgements of rationality as they have the authority to articulate, in the face of conflict, the agent's relevant perspective on their present options.
\end{enumerate}

Agential authority likewise can help ensure that an agent does not succumb to reasoning about tempting options so long as the agent does not see their evaluative judgements as representing their own perspective.
However, in cases of temptation the agents evaluative rankings for the tempting option does have a claim to authority as they are the present evaluative rankings of the agent.

Still, in cases of temptation an agent's policies also have claim to authority.
Further, this is a distinctive claim to authority as the policy is grounded in cross-temporal judgements, while the tempting option is a singular judgement.
Therefore, the policy may have a greater claim to authority and provide the basis for an on-balance judgement, hence \citeauthor{Bratman:2007ab}'s conjecture.

\nocite{Mackie:1965aa}