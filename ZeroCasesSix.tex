\documentclass[10pt]{article}
% \usepackage[margin=1in]{geometry}
% \newcommand\hmmax{0}
% \newcommand\bmmax{0}
% % % Fonts% %
\usepackage{luatexja}

\usepackage[T1]{fontenc}
   % \usepackage{textcomp}
   % \usepackage{newtxtext}
   % \renewcommand\rmdefault{Pym} %\usepackage{mathptmx} %\usepackage{times}
\usepackage[complete, subscriptcorrection, slantedGreek, mtpfrak, mtpbb, mtpcal]{mtpro2}
   \usepackage{bm}% Access to bold math symbols
   % \usepackage[onlytext]{MinionPro}
   \usepackage[no-math]{fontspec}
   \defaultfontfeatures{Ligatures=TeX,Numbers={Proportional}}
   \newfontfeature{Microtype}{protrusion=default;expansion=default;}
   \setmainfont[Ligatures=TeX,BoldFont={*-Semibold}]{Source Serif Pro}
   \setsansfont[Microtype,Scale=MatchLowercase,Ligatures=TeX,BoldFont={*-Semibold}]{Source Sans Pro}
   \setmonofont[Scale=0.8]{Atlas Typewriter}
   % \usepackage{selnolig}% For suppressing certain typographic ligatures automatically
   \usepackage{microtype}
% % % % % % %
\usepackage{amsthm}         % (in part) For the defined environments
\usepackage{mathtools}      % Improves  on amsmaths/mtpro2
\usepackage{amsthm}         % (in part) For the defined environments
\usepackage{mathtools}      % Improves on amsmaths/mtpro2
\usepackage{xfrac}

% % % The bibliography % % %
\usepackage[backend=biber,
  style=authoryear-comp,
  bibstyle=authoryear,
  citestyle=authoryear-comp,
  uniquename=false,
  % allinit,
  % giveninits=true,
  backref=false,
  hyperref=true,
  url=false,
  isbn=false,
  useprefix=true,
  ]{biblatex}
\DeclareFieldFormat{postnote}{#1}
\DeclareFieldFormat{multipostnote}{#1}
% \setlength\bibitemsep{1.5\itemsep}
\newcommand{\noopsort}[1]{}
\addbibresource{Thesis.bib}

% % % % % % % % % % % % % % %

\usepackage[inline]{enumitem}
\setlist[enumerate]{noitemsep}
\setlist[description]{style=unboxed,leftmargin=\parindent,labelindent=\parindent,font=\normalfont\space}
\setlist[itemize]{noitemsep}

% % % Misc packages % % %
\usepackage{setspace}
% \usepackage{refcheck} % Can be used for checking references
% \usepackage{lineno}   % For line numbers
% \usepackage{hyphenat} % For \hyp{} hyphenation command, and general hyphenation stuff
\usepackage{subcaption}
% % % % % % % % % % % % %

% % % Red Math % % %
\usepackage[usenames, dvipsnames]{xcolor}
% \usepackage{everysel}
% \EverySelectfont{\color{black}}
% \everymath{\color{red}}
% \everydisplay{\color{black}}
\definecolor{fuchsia}{HTML}{FE4164}%Neon Fuchsia %{F535AA}%Neon Pink
% % % % % % % % % %

\usepackage{pifont}
\newcommand{\hand}{\ding{43}}
\usepackage{array}


\usepackage{multirow}
\usepackage{adjustbox}

\usepackage{titlesec}

\usepackage{multicol}

\setcounter{secnumdepth}{4}
\setcounter{tocdepth}{4}

\usepackage{tikz}
\usetikzlibrary{bending,arrows,positioning,calc}
\usetikzlibrary{arrows.meta}
\usepackage{tikz-qtree} %for simple tree syntax
% \usepgflibrary{arrows} %for arrow endings
% \usetikzlibrary{positioning,shapes.multipart} %for structured nodes
\usetikzlibrary{tikzmark}
\usetikzlibrary{patterns}


\usepackage{graphicx} % for images (png/jpeg etc.)
\usepackage{caption} % for \caption* command


\usepackage{tabularx}

\usepackage{bussalt}

\usepackage{Oblique} % Custom package for oblique commands
\usepackage{CustomTheorems}
\usepackage{FuturePromisedEvents}

\usepackage{svg}
\usepackage[off]{svg-extract}
\svgsetup{clean=true}

\usepackage{dashrule}

\newcommand{\hozline}[0]{%
  \noindent\hdashrule[0.5ex][c]{\textwidth}{.1pt}{}
  %\vspace{-10pt}
  % \noindent\rule{\textwidth}{.1pt}
}

\newcommand{\hozlinedash}[0]{%
  \noindent\hdashrule[0.5ex][c]{\textwidth}{.1pt}{2.5pt}
  %\vspace{-10pt}
}

\usepackage{contour}
% \usepackage{pdfrender}

\usepackage{extarrows}

% % % My commands % % %

% % % % % % % % % % % %

\usepackage[hidelinks,breaklinks]{hyperref}

\title{Promised Futures}
\author{Ben Sparkes}
% \date{ }


\begin{document}

\maketitle

\tableofcontents

\newpage

\begin{note}
  Style of paper:
  \begin{itemize}
  \item Cases in which the agent is confident that some reasoning can be done from some support to a conclusion.
  \item Goal is to provide a way to understand these cases, so that the reader can figure out how they may relate to things that the reader is interested in.
  \item So there are three lines of interest
    \begin{enumerate}
    \item The cases, and why some forms of explanation can be resisted.
    \item The understanding of the cases, and what this understanding depends on and/or is independent of.
      \begin{itemize}
      \item structural issues, etc.\
      \end{itemize}
    \item What kind of issues these cases link to.
      \begin{itemize}
      \item Reasoning.
      \item Normativity.
      \item Justification.
      \item Goal is to lay some foundations for approaching these issues in greater detail.
      \end{itemize}
    \end{enumerate}
  \end{itemize}
\end{note}



\section{Introduction}
\label{sec:introduction}

\subsection{Overview}
\label{sec:overview}

\begin{note}
  Identify and provide a framework.
  Intuitively, some instances are permissible and others impermissible.
  Will not make any strong normative claims, and instead sketch how the framework can be used to make an argument.
  (Basically, in the superman case no way to fulfil promise, while in the taxes case there is.)
\end{note}

\begin{itemize}
\item Main feature of cases of interest:
  \begin{enumerate}
  \item An agent is confident that some piece of reasoning would demonstrate that a particular conclusion follows from some support.
  \end{enumerate}
\item Three additional features.
  \begin{enumerate}[resume]
  \item Agent is confident that the support holds.
  \item The agent is confident that they (the agent) are able to reason from the support to the conclusion.
    \begin{itemize}
    \item Though, note somewhere (maybe here) that the agent may be assisted in this reasoning.
    \end{itemize}
  \item And, the agent adopts an attitude toward the conclusion on the basis of the prior features.
  \end{enumerate}
\item Term this \emph{speculation.}\nolinebreak
  \footnote{Something about how the term is used, esp.\ in CS.}
\item Goal is to provide a framework for thinking about a certain type of scenario which falls under this general description.
\end{itemize}

\subsection{Central case: Morse \& Lewis.}
\label{sec:central-case:-morse}

\begin{scenario}[Morse \& Lewis]
  More and Lewis are police officers.
  The pair have been working together for some time, and each consider st the other an equal (in their role as a police officer).
  The equality is supported their experience working together, and reflected in their working methodology:
  Any part of an investigation may be worked on by either Morse or Lewis, and no type of work has been done exclusively by either.
  Morse is confident that they could do any work Lewis does, and likewise Lewis is confident that they could do any work that Morse does.

  The pair are working on an investigation and share a casebook containing all the evidence they have gather so far.
  Lewis returns home from their day off to discover a message from Morse on their answering machine.
  Morse says, in so few words, that they, in the absence of any new evidence, have been reviewing the casebook and what evidence they do have supports bringing in Woodthorpe for questioning.
  Morse is not loquacious, and Morse does not explain their reasoning in the message.
  Lewis listens to the message and goes to bed.

  The following morning, and to Lewis' surprise, Lewis spots Woodthorpe on their way to the police station.
  And, given Morse's message, Lewis arrests Woodthorpe.
\end{scenario}

\begin{enumerate}
\item Lewis is confident that there is some piece of reasoning would demonstrate that Woodthorpe's arrest is supported by the evidence contained in the case book.
\item Lewis is confident that the contents of the casebook constitute evidence.
\item Lewis is confident that both they and Morse are able to reason to the support for Woodthorpe's arrest from the casebook.
\item Lewis adopted some attitude to the proposition that Woodthorpe being arrested, and the action of arresting Woodthorpe (in part) depended on Lewis forming that attitude.
\end{enumerate}

\begin{itemize}
\item In ideal case, Lewis would have done the reasoning.
\item Indeed, in the ideal case Lewis would not be unaware of what follows from their evidence.
\end{itemize}

\begin{itemize}
\item Might think that Lewis needs only be confident that Woodthorpe's guilt can be demonstrated.
\item So, Lewis does not need to be confident that Woodthorpe is guilty.
\item And, Lewis has testimony for the availability of a demonstration.
\item As Lewis does not have a demonstration, Lewis is not required to be confident that Woodthorpe is guilty.
\item I do not think there is an important difference.
\item For, it seems plausible that Lewis does come to be confident that Woodthorpe is guilty.
\item However, if there is an important difference, then add additional constraints so that Lewis would only make the arrest based on their confidence that Woodthorpe is guilty.
\item May argue that the possibility of a demonstration is sufficient for Lewis, but not interested in ideal case.
\item Want to understand certain types of cases, the kind of cases {\color{red} stated in the introduction}.
\item It is plausible that these types of cases happen, and the ability to recast agency so that these do not happen would not help understand how creatures like us reason and act.
\end{itemize}

\begin{itemize}
\item Other explanations are possible, discuss some of these in what follows.
\item Two things to note:
  \begin{itemize}
  \item Lewis has confidence of the possibility of reasoning, and on this basis forms an attitude an arrests.
    \begin{itemize}
    \item Could think that Lewis straightforwardly forms an attitude toward the proposition, understands Morse as providing testimony, of a sort.
    \item There may be good reason to make this move, but it comes with some difficulties.
      \begin{enumerate}
      \item Morse didn't explicitly provide testimony.
      \item However, it seems Morse would not be able to avoid providing testimony if pressed.
      \item Can add some complexity to the case, assume that there are some claims in the dossier that only Lewis is aware of the evidence for, and hence Morse is restricted to only making a claim about what follows.
      \end{enumerate}
    \item Bracket testimony for the paper.
    \item Confidence is key, and while this may be supported by testimony, it is not necessary.
      The equality and history establish this without directly appealing to testimony.
    \end{itemize}
  \item It is important that Lewis is able to follow the support for the arrest.
    \begin{itemize}
    \item There are many ways to explain why this would be important for Lewis, and there are many ways to explain why this would not be important for Lewis.
    \item Sometimes this may not matter for the agent.
      \begin{itemize}
      \item If Morse is expected to do the heavy lifting, then perhaps Lewis only needs to understand that there is reason for arresting Woodthorpe.
      \item Lewis doesn't really care.
      \end{itemize}
    \item However, assume that it is important here.
      \begin{itemize}
      \item Can assume that Lewis would not arrest if there wasn't evidence.
      \item Lewis is aware that they themselves will need to justify the arrest.
      \item Lewis expects to fully understand the case file, etc.\
      \end{itemize}
    \end{itemize}
  \end{itemize}
\item Mention issue of justification, and that there's some things that can be read into the scenario that go a little beyond the basics that I'll be focusing on to start with.
\item Also, may want to mention understanding, as this is somewhat related but not required.
  \begin{itemize}
  \item For example, Morse nor Lewis may be involved in a complex financial case, and are able to demonstrate that Woodthorpe is guilty due to relations between statutes, but still do not \emph{understand} which Woodthorpe in anything other than a formal sense.
  \end{itemize}
\end{itemize}

\begin{itemize}
\item Also mention conflict with Lord, i.e.\ add in that Lewis has a copy of the case file at hand, and does not appropriately respond to reasons (may need some additional changes to adequately make this argument).
\end{itemize}

\begin{itemize}
\item That the case file is an object is unimportant.
\item If needed, assume that Lewis (and Morse) have memorised the collection of propositions which constitutes the case file.
\item Lewis and Morse are bounded agents, so they have not necessarily have attitudes toward anything that follows from the case file.
\item Borrow the phrase `epistemic reach' from \cite{Egan:2007aa}.
\end{itemize}

\begin{itemize}
\item More can be said about the details.
\item For now, turn to an broad characterisation.
\item Goal is to provide an abstract account of the main features, allowing us to reason about the general features of these types of cases.
\item Framework!
\end{itemize}

\newpage


\section{Framework}
\label{sec:framework-1}

\begin{note}
  Propositional and doxastic justification may help capture the difference.
\end{note}

\begin{itemize}
\item Main feature of cases of interest:
  \begin{enumerate}
  \item An agent is confident that some reasoning demonstrates that a particular conclusion follows from some support.
  \end{enumerate}
\end{itemize}

\begin{itemize}
\item Lewis is confident that there is a way to demonstrate the guilt of Woodthorpe follows from the case file shared with Morse, as Morse has claimed the reasoning can be done.
\item Students in a logic class are confident that there is a way to demonstrate \(\forall x(Fx \leftrightarrow x = a)\) follows from \(\exists x(\forall y(Fy \leftrightarrow y = x) \land Rxa)\) and \(\forall x\forall y(Rxy \rightarrow y = x)\), as they are required to demonstrate the entailment as a homework exercise.
\item You are confident that there is a way to demonstrate that the speaker would like to know if the camera is honest follows from their asking 「カメラは正直?」, as you have been given a translation.\nolinebreak
  \footnote{Well, in context, as this could also be asking whether cameras (in general) are honest.}
\end{itemize}

\begin{note}
  These examples help set up the two different kinds of reasoning discussed later.
  \begin{itemize}
  \item Lewis is a speculative case.
  \item Logic is more difficult, as the student may be confident in their future reasoning, or they may remain doubtful even after providing the proof.
  \item Translation is hard to make sense of other than as incorporeal.
  \end{itemize}
\end{note}

\begin{itemize}
\item Agent does something with their confidence that some piece of reasoning would demonstrate that a particular conclusion follows from some support.
\item Agent is confident in a claim about reasoning
\item Initial interest is in understanding the claim about reasoning, and then using this to model what the agent does with the claim about reasoning given their confidence.
\end{itemize}

\subsection{Reasoning as a process}
\label{sec:reasoning-as-process}

\begin{itemize}
\item Reasoning as a process of drawing conclusions from premises.
\item Set aside normative issues.
\item Hence, reasoning is similar to buttering a piece of toast, hugging Caesar, or flying to the North Pole.
\item Neo-davidsonian.
\end{itemize}

Our interest is in reasoning as a \emph{process} of drawing a conclusion from premises.
The use of the term `reasoning' in English does not uniquely capture this process.
To illustrate, there are at least two distinct readings of the following:
\begin{enumerate}
\item\label{reasoning:reading:amb} Lewis thought about Morse's reasoning from the case file to the guilt of Woodthorpe.
\end{enumerate}

On a `factive' reading, Lewis is thinking about the state of affairs in which reasoning from the case file to the guilt of Woodthorpe has been done by Morse.
In an attempt to force the factive reading, we could create an independent \emph{that}-clause for Morse's reasoning, and state \emph{that} Morse's reasoning is a fact.
\begin{enumerate}
\item[\ref{reasoning:reading:amb}\(_{f}\).] It is a fact that Morse's reasoning from the case file to the guilt of Woodthorpe happened, and Lewis is thinking about that fact.
\end{enumerate}
Morse's reasoning may have been a process, but on a factive reading Morse's reasoning the role of the process limited to making a certain fact true.

% \begin{enumerate}
% \item Morse's reasoning from the case file to the guilt of Woodthorpe happened earlier today.
% \item It is true that Morse reasoning from the case file to the guilt of Woodthorpe happened earlier today. (Perfective aspect and past tense mix-up here --- `reasoned' is more natural.)
% \end{enumerate}

% Here, assigning a complex predicate to Morse.
% The predicate implies that some process took place, but this is really a straightforward proposition that is true or false.
% The `that' clause formulation helps make this clear.
% Even if an event (or process) is implied, that Morse reasoned is more-or-less a fact rather than an event.

On an `active' reading, Lewis is thinking about the action, event, or process of Morse reasoning from the case file to the guilt of Woodthorpe.
Similar to the forced factive reading, to ensure an active we may attempt to explicitly state that Morse's was a process.
\begin{enumerate}
\item[\ref{reasoning:reading:amb}\(_{a}\).] Morse's reasoning from the case file to the guilt of Woodthorpe was a process, and Lewis is thinking about that process.
\end{enumerate}

% \begin{enumerate}
% \item Morse was reasoning from the case file to Woodthorpe's guilt. (process)
% \item The reasoning from the case file to Woodthorpe's guilt by Morse. (an instance of the process)
% \end{enumerate}

% In the first some process was taking place.
% In the second, identify the thing that is Morse's reasoning.
% Both instance understood in terms of events or processes.
% It is this understand of reasoning that is important for us.

This distinction between fact and event is part of the motivation for \citeauthor{Davidson:2001aa}'s treatment of events.
An event need not consist of some fact.
(\citeyear[116]{Davidson:2001aa})

\begin{itemize}
\item Parallels to event semantics also suggest lines of philosophical interest, esp.\ understanding reasoning in terms of mental events/actions.
\end{itemize}

\hozlinedash

\begin{itemize}
\item Active reading is of primary interest.
\item Our approach is to capture factive aspects of reasoning by abstraction from active readings.
\item This is not to say that the active reading is strictly required.
\item Rather, things are simpler if we keep to a particular reading, and the active reading is useful.
\end{itemize}

\begin{itemize}
\item Neo-Davidsonian event semantics.
\item Events are things, and ascribe properties to events.
\item Neo-Davidsonian due to the use of thematic roles to identify and situate that participants of the event.
\end{itemize}

\hozlinedash

\begin{itemize}
\item ``Morse's reasoning from the case file to the guilt of Woodthorpe'' is structurally similar to ``Brutus hugged Caesar''.
\item ``Brutus hugged Caesar'' is analysed with\dots
  The event predicate `hug'.
  As `hug' is a transitive verb there is an `agent', a participant who hugs, and a `patient', a participant who is hugged.
  \[
    \exists e(\everb{hug} \land \eagent{Brutus} \land \epatient{Caesar})
  \]
  Our treatment of reasoning is structurally similar.
  An event predicate `reason' indicates that the event is a process of reasoning and the thematic role `agent' identifies the reasoner, and as the agent reasons from premises to a conclusion we introduce to additional thematic roles termed `in' and `out' for the premises and conclusion, respectively.
  \[
    \exists e(\everb{reason} \land \ein{\Sigma} \land \eout{\chi} \land \eagentm{a})
  \]
  Informally, the above captures \(a\)'s reasoning from \(\Sigma\) to \(\chi\).
  {\color{red} Typically} take propositions for the role of `in' and `out', with \(\Sigma\) standing for a set of propositions\nolinebreak
  \footnote{Here, note that propositions include events.
    This is why \citeauthor{Parsons:1990aa}'s terminology is avoided.

    Also, may want a conjunction of propositions, but then there's some difficulty distinguishing between \(\{\phi,\psi\}\) and \(\{\phi \land \psi\}\).
    Not building a rigorous formal system.
    Issues of object language and metalanguage, beyond the scope of the paper, really.
  Still, could introduce `premise marking parenthesis' to distinguish between \([\phi] \land [\psi]\) and \([\phi \land \psi]\) if desired.}
  which are the premises of an agent's reasoning, \(\chi\) for the proposition which is the conclusion of the agent's reasoning.
  % \[
  %   \exists e(\everb{reason} \land \ein{\text{Case file}} \land \eout{\text{Guilt of Woodthorpe}} \land \eagent{\text{Morse}})
  % \]
  \[
    \exists e
    \left(
      \begin{array}{l}
        \everb{reason} \land \\
        \ein{\text{Case file}} \land \\
        \eout{\text{Guilt of Woodthorpe}} \land \\
        \eagent{\text{Morse}}
      \end{array}
    \right)
  \]
% \item Brutus did not only hug Caesar.
% \item Brutus hugged and tickled Caesar.
%   \[
%     \exists e(\everb{hug} \land \everb{tickle} \land \eagent{\text{Brutus}} \land \epatient{\text{Caesar}})
%   \]
% \item No changes in the (thematic) roles of the participants, more detail added.
% \item And, not a separate event, the hugging was also a tickling, Brutus did not hug and also tickle.
%   Could do the same thing with a the novel-predicate \(\text{tickle-hug}\).
%   \[
%     \exists e(\everb{tickle-hug} \land \eagent{Brutus} \land \epatient{Caesar})
%   \]
% \item Same with reasoning.
\end{itemize}

\hozlinedash

\begin{itemize}
\item We introduce a function in order to capture the fact that there was reasoning from \(\Sigma\) to \(\chi\).\nolinebreak
  \footnote{\textcite[117]{Davidson:2001aa} considers an example involving the addition of 2 and 3 as an event.}
\item A function is something which relates an input to an output, and the type of reasoning we are interested in is characterised by an input (premises) and an output (conclusion).
\item Broadly stated, the specification of a function captures the factive aspect of reasoning at coarse grain.
\item Hence, some event of reasoning from premises \(\Sigma\) to a conclusion \(\chi\) ensures the existence of a function defined for input \(\Sigma\) and output \(\chi\).
  \[
    \exists e
    \left(
      \begin{array}{l}
        \everb{reason} \\
        \ein{\Sigma} \\
        \eout{\chi} \\
        \exists f(\efunc{f}) \\
          \eagentm{a}
      \end{array}
    \right)
  \]
  \item The existential captures our lack of information about whether it is possible for the agent to reason in the same way from different premises to a (not necessarily) different conclusion.
\item The event entails
  \[
    \exists f(f\Sigma = \chi)
  \]
  This may not specify a unique function, as only a single in-out pair constrains the possible witnesses of the existentially quantified function.
  To take a simple example, \(\exists f(f(2,2) = 4)\) can be witnessed by addition, multiplication, exponentiation, the constant function which always yields \(4\), and so on.
  \[
    f(2,2) = 4.
    f = +
    f = \times
    f = \exp
  \]
  When Lewis considers Morse's reasoning, the existence of a function is the relevant factive information, with additional background information which we omit for simplicity.
\item This function may be partial; defined for \(\Sigma\) and \(\chi\).
  However, whatever that function is, it can potentially be instantiated in other instances of reasoning.
\item For example with a known function, I may ask you what the median {\color{red} number of characters in a Shakespeare play} is.
    \[
    \exists e
    \left(
      \begin{array}{l}
        \everb{reason} \\
        \ein{\text{List of character counts in Shakespeare's plays}} \\
        \eout{32.5} \\
        (\efunc{\text{median}}) \\
        \eagentm{a}
      \end{array}
    \right)
  \]
\item An agent reasoned from \(\Sigma\) to \(\chi\) entails that there exists a function mapping \(\Sigma\) to \(\chi\).
\item \(\exists f(f\Sigma = \chi)\) or \(\text{mean}(\text{List of character counts in Shakespeare's plays}) = 32.5\).
\item We term the bare statement of a function relating some input to some output from some process of reasoning a \emph{mention} of the function.
\item The mention of a function contrasts with the \emph{use} of the function in the process of reasoning that function is abstracted from.
\item If inclined to think of reasoning as a rule governed activity, then this is the distinction between mentioning a rule and being guided by the rule.
\item \textcite{Simchen:2001aa} is quite nice.
\item Perhaps `demonstration of' and `reference to' a/the function.
\end{itemize}

\begin{itemize}
\item The distinction between the mention and the use of a function related to a process of reasoning is important for understanding Lewis' reasoning.
\item To give a quick example \dots
\item Propositional logic.
\item Agent reasons from \(\alpha \rightarrow \beta\) to \(\lnot \beta \rightarrow \lnot\alpha\) via contraposition.
      \[
    \exists e
    \left(
      \begin{array}{l}
        \everb{reason} \\
        \ein{\alpha \rightarrow \beta} \\
        \eout{\lnot \beta \rightarrow \lnot\alpha} \\
        \efunc{\text{Contraposition}}
      \end{array}
    \right)
  \]
\item This description does not state how the agent contraposed \(\alpha \rightarrow \beta\).
  However, by identifying the function we can be sure that if the agent reasoned from some other premise of the same form, e.g.\ \(\gamma \rightarrow \delta\), then the agent would have obtained the contrapositive; \(\lnot \delta \rightarrow \lnot \gamma\).
\item The appeal of propositional reasoning is that there is a conventional way to mention the agent's reasoning.
\item \((\alpha \rightarrow \beta) \rightarrow (\lnot \beta \rightarrow \lnot\alpha)\).
  \[
    \exists e
    \left(
      \begin{array}{l}
        \everb{reason} \\
        \ein{\{\alpha \rightarrow \beta, (\alpha \rightarrow \beta) \rightarrow (\lnot \beta \rightarrow \lnot\alpha)\}} \\
        \eout{\lnot \beta \rightarrow \lnot\alpha} \\
        \efunc{\text{Modus ponens}}
      \end{array}
    \right)
  \]
\item Convention encodes the complexities of explaining how a conditional captures the fact of some reasoning.
\item Avoid this for the rest of the paper, but useful to illustrate.
\end{itemize}

\hozlinedash

\begin{itemize}
\item Functions aren't \emph{pure}.
  May be side effects.
\end{itemize}

\hozlinedash

\paragraph{Summary of the section}

The central elements are:\nolinebreak
\footnote{
  \begin{itemize}
  \item This is similar to \textcite[24--27]{Marr:1982aa}
  \end{itemize}
  The specification of a function mapping the premises of the reasoning to conclusion of the reasoning corresponds to \citeauthor{Marr:1982aa}'s computational level.
  At this level, in \citeauthor{Marr:1982aa} words, `the performance of the device is characterized as a mapping from one kind of information to another' (\citeyear[24]{Marr:1982aa}).
  Likewise, the process of reasoning corresponds to the level of the implementation of the computation; how the computation is realised.

  Missing from our discussion is \citeauthor{Marr:1982aa}'s algorithmic level, which specifies `the choice of representation for the input and output and the algorithm to be used to transform one into the other' (\citeyear[24--25]{Marr:1982aa}).
  The specifics of representation are dealt with by however our representations are interpreted, but it is certainly natural to include a description of the method by which an agent reasons from premises to conclusion.
  For example, Brutus reasons from the premise of the event of Caesar crying to the conclusion that Brutus ought to given Caesar a hug.
  And coarsely, Brutus does so by reasoning in line with the categorical imperative.
  Stated in full:
  \[
    \exists e
    \left(
      \begin{array}{l}
        \everb{reason} \\
        \ein{\exists e'(\everb[e']{cry} \land \eagent[e']{Caesar})} \\
        \eout{\text{Ought}
        \left(\exists e''\left(
        \begin{array}{l}
          \everb[e'']{hug} \\
          \eagent[e'']{Brutus} \\
          \epatient[e'']{Caesar}
        \end{array}
        \right)
        \right)
        } \\
        \eagent{Brutus} \\
        \emethod{Categorical imperative}
      \end{array}
    \right)
  \]
  \begin{itemize}
  \item Function can have multiple methods, may be no relevant characterisation of method, characterisation of method may be too coarse grained to specify a particular function.
  \item For the most part, ignore method.
  \end{itemize}
}

\[
  \exists e(\text{reason}(e) \land \ein{\Sigma} \land \eout{\chi} \land \exists f(\efunc{f}))
\]

\begin{itemize}
\item Reasoning is understood as an event, and to abstract to a function which relates the input to the output.
\item Functions to abstractly capture reasoning.
\item Use and mention of a function.
\end{itemize}

\hozlinedash

\begin{itemize}
\item This is a simple view, and things can get messy.
  Especially if dealing with non-monotonic or ecological reasoning, where there may be no easy way to exhaustively capture the ins and outs of reasoning as a process.
\end{itemize}

\hozlinedash

\subsection{Key inference}
\label{sec:key-inference}

\begin{itemize}
\item Lewis is confident that Morse has reasoned from the case file to the guilt of Woodthorpe.
\item Lewis has arrested Woodthorpe, and Lewis has performed this arrest because Lewis is confident that Woodthorpe is guilty.
\item Still, Lewis has not reasoned from the case file to the guilt of Woodthorpe.
\end{itemize}

\begin{itemize}
\item Lewis' confidence that Woodthorpe is guilty is important.
\item Morse has no provided testimony that Woodthorpe is guilty, and we are assuming that Lewis does not coerce Morse's report on their reasoning as testimony.
\end{itemize}

\begin{itemize}
\item How is it the case that Lewis is confident that Woodthorpe is guilty?
\end{itemize}

\begin{itemize}
\item The event of Morse reasoning from the case file to the guilt of Woodthorpe establishes the existence of a function which captures the possibility of reasoning from the case file to the guilt of Woodthorpe.
\item As Lewis is confident that Morse and Lewis are equally matched, this is a function that Lewis can witness.
  \begin{itemize}
  \item Function is an abstract characterisation of Morse's reasoning, and at a certain level of generality Morse's reasoning is a particular but not exclusive witness.
  \end{itemize}
\item Hence, Lewis can be confident that Lewis \emph{is able} reason from the case file to the guilt of Woodthorpe.
\item And, if Lewis can reason, then there's an appropriate relation of support that Lewis is able to draw on.
\item This is the key.
\end{itemize}

\hozlinedash

\begin{itemize}
\item Lewis is confident that there is the appropriate relation of propositional support, and Lewis is confident that this can be transformed into doxastic support.
\end{itemize}

Propositional and doxastic support are non-committal versions of propositional and doxastic justification.
And agent has propositional justification for \(\phi\) is the agent having justification to believe \(\phi\), while and agent having doxastic justification for \(\phi\) is an agent justifiably believing \(\phi\).
Propositional justification does not require belief.

Contested.

Morse's reasoning from the case file to the guilt of Woodthorpe establishes that there is a relation of support between the case file and the guilt of Woodthorpe which holds independently of Morse's reasoning.
Morse's reasoning demonstrates what that relation of support is.

Even so, learn from the literature on propositional and doxastic justification that an independent relation of support between the case file and the guilt of Woodthorpe does not entail that Lewis can draw on the independent relation and confidence that relation exists to be confident that Woodthorpe is guilty.

\textcite{Turri:2010aa} discusses certain kinds of cases in which agents share the same {\color{red} epistemic reasons} and form the same belief, but do so in different ways.
To adopt \citeauthor{Turri:2010aa}'s example, if Lewis were to reason from thee case file to the guilt of Woodthorpe due to tea leaves making it overwhelmingly likely that Woodthorpe is guilty, it seems Lewis' confidence would not be supported by the case file.

\begin{itemize}
\item This is where the puzzle kicks in.
\item It doesn't seems as though Lewis is in water for being confident.
\item So, how is this explained?
\end{itemize}

Pair of examples which share the structure as the Morse and Lewis scenario help clarify.

\hozlinedash

\subsection{Practical contamination}
\label{sec:pract-cont}

\begin{itemize}
\item The argument here is that things are different from a practical point of view.
\item Winston and Julia is a little complex, and perhaps unintuitive, but I'm fond of the example.
\item To make the point a little clearer, introduce instructor scenario.
\item Suspect this can be exploited.
\end{itemize}

\hozlinedash

\begin{scenario}[Winston \& Julia]
  Winston works on a farm where visitors can come to pet animals.
  While looking through the bookings, Winston notices that Eric Blair is due to visit on the tenth of January.
  Winston says to Julia ``Eric Blair is due to visit soon\dots'', and after a pause adds, ``\dots and were you aware that `Eric Blair' is the actual name of `George Orwell'''.
  Julia replies ``No! But if that's true then it means George Orwell will be visiting us''.
  Winston reflects on Julia's reply, and comes to be confident that George Orwell is due to visit.
\end{scenario}

Winston is confident that Julia reasoned from the propositions that Eric Blair is due to visit and that `Eric Blair' and `George Orwell' are co-referential to the conclusion that George Orwell is due to visit.
And, on the basis of Winston's confidence that Eric Blair is due to visit and that `Eric Blair' and `George Orwell' are co-referential and Julia's reasoning from these propositions to the conclusion that George Orwell is due to visit, Winston is confident that George Orwell is due to visit.

Still, something is amiss.
The substitution of co-referring names in a referentially transparent proposition is a straightforward piece of reasoning.
And, Winston's reliance of Julia's reasoning suggests that Winston has not grasped the basics of referential transparency, and hence Winston is unable to replicate Julia's reasoning themselves.
Therefore, Winston's confidence that George Orwell will be visiting does not seem to be supported on the basis of Winston being confident that that Eric Blair is due to visit and that `Eric Blair' and `George Orwell' are co-referential.

\hozlinedash

\begin{itemize}
\item The role of Julia's reasoning from these propositions to the conclusion that George Orwell is due to visit.
\end{itemize}

\begin{itemize}
\item What is Winston reasons with:
  \begin{itemize}
  \item \(Vo, o = e, \exists f(f\{Vo, o = e\} = Ve)\).
  \end{itemize}
\item In this case, Winston is certainly not the sharpest, but is their confidence that \(Ve\) fine?
\item The difference is that I've added confidence that it is possible to reason to the mix.
\item Intuitively, there is nothing wrong with Winston's \emph{method} of reasoning.
\item And, might think that this is fine because there's no need for Winston to reason from \(Vo, o = e\).
\item That is, \(Ve\) is independent of Winston's reasoning from \(Vo, o = e\).
\item So, Winston can't be confident based only on their confidence, but the fact that they're confident that it's possible to reason is an additional consideration for Winston that they're able to make use of.
\end{itemize}

\begin{itemize}
\item So, if someone were to ask Winston, they'd appeal to Julia's reasoning.
\end{itemize}

\begin{itemize}
\item Suppose Julia says something like ``if that is so, then you can be confident that Eric Blair will be visiting us.''
\end{itemize}

\begin{itemize}
\item Now it seems as though Winston is in trouble.
\item For, Julia has informed Winston that Winston is able to reason from \(Vo\) and \(o = e\) to \(Ve\).
\item However, Winston doesn't see why this is the case.
\item So, it looks as though Julia is telling Winston that they can reason to \(Ve\) from \(\{Vo, o = e\}\).
\item And the problem for Winston is that they aren't able to do this.
\item So, Winston can't be confident.
\item The problem is that Julia has said something false, it seems.
\item Julia has overestimated Winston's ability.
  \begin{itemize}
  \item \(\rightarrow\) This is the important distinction between Julia's two utterances.
  \end{itemize}
\item It's true that it is possible to reason, but it is false that it is possible for \emph{Winston} to reason\dots
\item So, if this is the case, then there's a problem with the general inference from there is some reasoning\dots
\item This shouldn't be too surprising, as dependence is common, and the difference comes up in other instance of action.
\end{itemize}

\begin{itemize}
\item Could talk about different kinds of attitudes.
\item Easier to talk about confidence and it's support.
\item What the confidence depends on.
\end{itemize}

\begin{itemize}
\item Then there's the contrasting case.
\item Can do the reasoning, and so attitude depends.
\item Perhaps also include desire to illustrate, as here dependence does seem important.
\end{itemize}

\begin{itemize}
\item It is only in the case of Julia's dependency statement that the scenario parallels the kind of scenarios that \citeauthor{Worsnip:2018aa} deals with.
\end{itemize}

\hozlinedash

\begin{scenario}[Carmen \& Lawrence ({\color{red} Amelia, Richard})]
  Carmen and Lawrence have travelled to Estonia.
  Lawrence looks at the weather forecast in the morning newspaper, and the midday temperature is reported to be around 10 degrees Celsius.

  ``It says here the high will be around 10 Celsius today \dots'', Lawrence says to Carmen, ``\dots and if I multiply by 1.8 and add 32\dots''.

  ``If so \dots'' Carmen adds, ``\dots it won't get much warmer than 40 Fahrenheit''.

  Lawrence frowns and adds: ``It seems we'll need to buy warmer clothes.''
\end{scenario}

\hozlinedash

\begin{itemize}
\item Can potentially add to the Lawrence case a little, so that Lawrence does not communicate that it's 10 Celsius outside, only the relevant conversion.
\item Then, Amelia notes that \(10 = 40\).
\item What are you doing? Well [transformation]. Oh, so for example \(10 = 40\).
\item Oh, yeah. It's 40 outside.
\end{itemize}

\hozlinedash

\begin{itemize}
\item Similar situation.
\item Carmen doesn't state that the whether will be 40 Fahrenheit.
\item Can assume that Carmen is not aware of the conversion, and hence that Carmen simply reports the result of the process of reasoning that Lawrence was about to perform.
\item Does some reasoning and reports the result to Lawrence.
\item Lawrence understands the reasoning, and is confident that the temperature will be around 40 Fahrenheit.
\item It is clear that Lawrence could have done the reasoning, but Carmen was too quick.
\item In contrast to Winston \& Julia, Lawrence has stated an important part of how they would reason to the temperature in Fahrenheit.
\item May be the case that Lawrence's confidence that it will be 40 Fahrenheit is supported on the basis of Lawrence's confidence that it will be 20 Celsius.
\end{itemize}

\begin{itemize}
\item There's an importance difference between the two scenarios, as in one a basic pattern of reasoning is missing.
\item In the other, it is stated.
\item Lewis is likely located between these two extremes.
\item Some guess as the kind of reasoning Morse has done.
\item Lewis' confidence that they are on par with Morse splits this difference.
  Unlike Winston, Lewis does not need to improve their reasoning to demonstrate the guilt of Woodthorpe.
  Unlike Lawrence, Lewis does not have a grasp on the particular way of reasoning from the case file to the guilt of Woodthorpe.
  Like Lawrence, Lewis is confident that they can do the reasoning.
  Like Winston, Lewis is unsure of how the guilt of Woodthorpe can be demonstrated from the case file.
\end{itemize}

\citeauthor{Turri:2010aa}'s proposed analysis of propositional justification:\nolinebreak
\footnote{Not assuming that \citeauthor{Turri:2010aa}'s proposal is correct.
  However, it's useful in linking propositional and doxastic support.
  If propositional support does not require doxastic support, then whatever else it is that captures the difference\dots
}

\begin{quote}
  Necessarily, for all \(S\), \(p\), and \(t\), if \(p\) is propositionally justified for \(S\) at \(t\), then \(p\) is propositionally justified for \(S\) at \(t\) because \(S\) currently possesses at least one means of coming to believe \(p\) such that, were \(S\) to believe \(p\) in one of those ways, \(S\)'s belief would thereby be doxastically justified.\nolinebreak
  \mbox{}\hfill\mbox{(\citeyear[320]{Turri:2010aa})}
\end{quote}

Lawrence satisfied \citeauthor{Turri:2010aa}'s requirement for propositional justification.
Winston does not satisfy \citeauthor{Turri:2010aa}'s requirement for propositional justification.


This is a necessary condition.
So, can only answer whether Lewis satisfies the necessary requirements for (\citeauthor{Turri:2010aa}'s analysis) of propositional justification.
And, it seems Lewis does, at least from Lewis' perspective.
For, assume that Morse's belief that Woodthorpe is guilty is doxastically justified on the basis of Morse's reasoning from the case file to the guilt of Woodthorpe.
Then, as Morse and Lewis are equally matched, Morse reasoning from the case file to the guilt of Woodthorpe underwrites Lewis' ability to reason from the case file to the guilt of Woodthorpe.
Hence, Lewis possesses the same means as Morse, and were Lewis to reason in that way, Lewis would be doxastically justified.

The slight snag is that restricted to Lewis' perspective because we have no stated whether or not Lewis and Morse are equally matched.
This doesn't matter.
We are interested in Lewis' reasoning, not whether Lewis is correct that Woodthorpe is guilty.

\hozlinedash

\begin{itemize}
\item Lewis makes the move that Carrol's tortoise refuses to make.
\item ``the permissibility of the transition from p to q depends on the existence of a licence specifying that such a transition is indeed permissible'' (\cite[459]{Simchen:2001aa})
\end{itemize}

\hozlinedash

\paragraph{Summarising propositional but not doxastic}

\begin{itemize}
\item Lewis is confident of the status of the evidence in the case file.
\item Lewis cites this if questioned.
\item Suggestion is that Morse's reasoning from the case file to the guilt of Woodthorpe traces a relation between the case file and the guilt of Woodthorpe that is independent of Morse's reasoning for the case file to the guilt of Woodthorpe.
\item Relation of propositional support.
\item Morse's reasoning establishes doxastic support.
\item So, Lewis' confidence in the case file ensures that Lewis has propositional support for the guilt of Woodthorpe.
\item Without reasoning from the case file, Lewis does not have doxastic support on the basis of the case file.
\end{itemize}

\newpage

\subsection{The difficulty}
\label{sec:difficulty}

\begin{itemize}
\item \emph{Practical contamination}
\item There are different intuitions.
\end{itemize}




\begin{itemize}
\item The basic idea is that Lewis settles the guilt of Woodthorpe on the basis of Lewis' confidence that the case file constitutes evidence.
\item Morse's message is an important part of \emph{how} Lewis settles, but it is not \emph{why} Lewis settles.
\end{itemize}

\begin{itemize}
\item The problem is that Lewis hasn't done the reasoning, so Lewis cannot demonstrate why the case file settles Woodthorpe's guilt.
\item However, Lewis is confident that they can do the reasoning.
\end{itemize}

\begin{itemize}
\item To get the intuition, the reason why Lewis arrests Woodthorpe are those reasons that Lewis will write in the arrest report.
\item Morse's message is no good.
\item Morse's message is no different from Lewis' statement that there is a demonstration.
\item If that's all that is in the report, then Woodthorpe will be set free.
\item The problem is that Lewis doesn't yet have a grasp of what those reasons are.
\item This is the difficulty.
\item If the reasons why are constrained by how, then there is no way out.
\item I think it is plausible that in the case of ideal agents, this is true.
\item For agents like us, I think this is false.
\end{itemize}

\begin{itemize}
\item My model for this is fairly straightforward, as the agent stipulates some function, and may at some point demonstrate what this function is.
\end{itemize}

\hozlinedash

\begin{itemize}
\item If \(\exists f(f\Sigma = \chi), \Sigma\) is sufficient, then Winston is fine.
\item Winston doesn't seem fine, and the suggestion is that this is because Winston is unable to do the reasoning.
\end{itemize}

\hozline

\begin{itemize}
\item Lewis needs to demonstration.
\item The core question is how the demonstration relates to Lewis' reasoning.
\item The intuition is that Lewis should not be making an arrest without a demonstration in hand, so to speak.
\item Or, that Lewis' arrest of Woodthorpe is `premised' on providing a demonstration.
  \begin{itemize}
  \item Non-constructive reasoning premised on constructive reasoning.
  \end{itemize}
\item On the one hand, there's the incorporeal approach.
  \begin{itemize}
  \item Here, Lewis reflects on propositional justification, and adds something to this.
  \end{itemize}
\item On the other hand, there's the speculative approach.
  \begin{itemize}
  \item Here, Lewis speculates that Woodthorpe is guilty based on reasoning that they are confident that they will perform.
  \end{itemize}
\item The difference is, roughly, completeness.
  \begin{itemize}
  \item On the incorporeal approach, Lewis' confidence that Woodthorpe is guilty is a complete piece of reasoning.
    \begin{itemize}
    \item Lewis then guarantees that they'll provide a demonstration of the guilt of Woodthorpe from the case file.
    \item Lewis \emph{assures} that a demonstration will be provided.
    \end{itemize}
  \item On the speculative approach, Lewis' confidence that Woodthorpe is guilt is an incomplete piece of reasoning.
    \begin{itemize}
    \item For, Lewis' attitude toward the guilt of Woodthorpe is premised on a demonstration.
    \item Lewis takes for granted that they will provide a demonstration.
    \item Lewis \emph{assumes} that a demonstration will be provided.
    \end{itemize}
  \end{itemize}
\end{itemize}

\begin{itemize}
\item What is the difference between \emph{assuring} and \emph{assuming}?
\item {\color{red}
    These terms aren't ideal.
    For an assumption doesn't in general entail an assurance.
    However, I am assuming that this is the case.
  }
\item Well, and assumption requires an assurance.
  \begin{itemize}
  \item At least, in the cases of interest.
  \item For, in effect Lewis is taking for granted the result of what they assure.
  \item {\color{red}
      Aha!
      The assumption is like an additional step.
      Lewis can assure, but can also make the additional step of assuming the result of the assurance.
    }
  \end{itemize}
\end{itemize}

\begin{itemize}
\item The two options are:
  \begin{enumerate}
  \item Arrest because high confidence that Woodthorpe's guilt will be demonstrated.
  \item Arrest because unlikely that assumption that Woodthorpe's guilt will be demonstrated is mistaken.
  \end{enumerate}
\item It is easy to switch between one and the other.
\item The difference is the assumption \dots
\item If Lewis makes the assumption then Lewis presupposes providing a demonstration.
\end{itemize}


\hozlinedash

\begin{itemize}
\item An analogy is credit.
\item I do not have the funds to purchase a washing machine.
\item However, I will be paid at the end of the month.
\item So, I use the credit that I have to purchase.
\item I have not paid until the end of the month.
\item Bank guarantees my ability to pay.
\end{itemize}

\hozlinedash

\begin{itemize}
\item The existence of propositional support can be seen as a premise in Lewis' reasoning.
\item {\color{red} This is the inverse of the types of cases \citeauthor{Worsnip:2018aa} considers\dots}
\item \(\exists f(f\Sigma = \chi), \Sigma \vdash \chi\).
\item The reasoning is similar to the more familiar case of reasoning with existentially quantified objects in a first order setting.
\item If there is something that is not alive.
\item And, if everything that is mean spirited is alive.
\item Then, there is something that is not mean spirited.
\item \(\exists x \lnot Ax, \forall x (Mx \rightarrow Ax) \vdash \exists x \lnot Mx\).
\end{itemize}

\begin{itemize}
\item Indirect application of the function.
\item Incorporeal.
\end{itemize}

\begin{itemize}
\item In short, Lewis can be confident because Morse reasoned.
\end{itemize}

\begin{itemize}
\item Pause here.
\end{itemize}

\hozlinedash

\begin{itemize}
\item Lewis goes and does the reasoning.
\item Now there are two witnesses, one either side of Lewis' arrest.
\item Asymmetry, different roles.
\item Still, for the same reasoning; the demonstration of the guilt of Woodthorpe.
\item Lewis' future reasoning does not reach back in time to provide the witness required for confidence.
\item Morse's reasoning does not provide Lewis with a demonstration.
\end{itemize}

\begin{itemize}
\item Reasoning.
\item And modelling.
\item Have the problem, but unlikely that there is a unique solution.
\item Introduce a proposal, and see how well it works.
\end{itemize}

\begin{itemize}
\item Lewis can commit to reasoning from the case file to the guilt of Woodthorpe at some future point in time.
\item In short, Lewis can \emph{promise} an demonstration.
\item {\color{red} \dots}
  \begin{itemize}
  \item A commitment that Lewis will do the reasoning.
  \end{itemize}
\item This is problematic for the proposed analysis because Lewis' reasoning does not depend on any particular instance of the function.
\item As things stand, Lewis' reasoning is complete.
  \begin{itemize}
  \item The reasoning does not depend on a future witness.
  \item The idea is that Lewis' reasoning depends on a specific function.
  \item Specifically, the one that will be witnessed in Lewis' future reasoning.
  \item Lewis doesn't discharge the function.
  \item Morse --(propositional)--> Arrest <--(doxastic)-- Lewis
  \end{itemize}
\item {\color{red} Note, this is a precondition for Lewis writing out the report\dots}
\end{itemize}

\begin{note}[Core problem]
  Morse's reasoning has limited use.

  Morse's reasoning establishes propositional support.
  However, propositional support is limited.
  Maybe, but this is a hard argument to make.

  Alternative is to argue that Lewis wants doxastic support.
  And, issue is whether this is a separate requirement, or whether it's linked to Lewis' reasoning.

  ``I arrested Woodthorpe on the basis of being able to reason from the case file to the guilt of Woodthorpe, and I need to fill in what that reasoning is.''
  ``Here is how the guilt of Woodthorpe was supported by the case file.''
\end{note}

\begin{itemize}
\item What's important is that the reasoning can be done.
\item {\color{red} The existence of propositional support}, from a broader perspective.
\item That Morse has done the reasoning is what underwrites this, along with the parity, but it is the existence and possible witness that matters.
  \begin{itemize}
  \item So, difference between me telling you that there are exactly seven intermediate logic systems with a certain property, and then informing you of a proof of this, as an example of how this makes a difference.
  \end{itemize}
\item The key is a witness.
\item {\color{red} Question of the witness\dots}
  \begin{itemize}
  \item If the guarantee of a witness if what matters .
  \item Then, as Lewis is confident that they are able to do the reasoning, Lewis is confident that they are able to provide a witnessing piece of reasoning.
  \item Build on the existential object example, I'm confident that I can find some non living thing, so I call it Casper.
    And, I will, eventually, have a friendly ghost.
    Reference De dicto reference and de re.
    Example, Scully does not believe that there is something which is not mean spirited.
    Scully does not believe that Casper exists.
    Hm, something better than this is needed.
  \end{itemize}
\item Seems important that Lewis can reason.
  \begin{itemize}
  \item Generalising from confidence that Woodthorpe is guilty, it may be important that Lewis is able to demonstrate that Woodthorpe is guilty.
  \item Lewis will need to write a report.
  \item Investigate the idea that Lewis' confidence is premised on their own reasoning.
  \item Reasoning about Casper is premised on reference de re.
  \item Key difference is that Lewis' reasoning is incomplete.
  \end{itemize}
\item Suggests that Lewis could premise on their own reasoning.
\item Main object of interest.
\item Future.
\end{itemize}


\begin{itemize}
\item There are cases in which problems arise.
\item Morse doesn't reason to the guilt of Woodthorpe.
\item Lewis has a vendetta.
\item Lewis arrests Woodthorpe, convinced that they'll be able to find evidence, etc.
\end{itemize}

\[
  \exists e
  \left(
    \begin{array}{l}
      \everb{reason} \land
      \ein{\text{Case file}} \land
      \eout{\text{G}(w)} \land
      \eagentm{m}
    \end{array}
  \right)
\]

\[
  \lnot\exists e
  \left(
    \begin{array}{l}
      \everb{reason} \land
      \ein{\text{Case file}} \land
      \eout{\text{G}(w)} \land
      \eagentm{l}
    \end{array}
  \right)
\]

\begin{enumerate}
\item Because of the fact that Morse reasoned.
\item Because Lewis will reason.
\end{enumerate}


\begin{itemize}
\item \(\exists f(f\Sigma = \chi), \Sigma\), hence \(\chi\).
\item \(\chi\) is obtained by an application of the function to \(\Sigma\).
\item The function captures the fact that Morse reasoned from the case file to the guilt of Woodthorpe.
\item So, by appealing to the witnessing instance, Lewis appeals to the relation of propositional support which holds between the case file and the guilt of Woodthorpe.
\item If Woodthorpe is guilty, then Woodthorpe is guilty \emph{independently} of any demonstration of Woodthorpe's guilt from the case file.
\item Lewis is confident that Woodthorpe is guilty, but it's the ability to demonstrate Woodthorpe's guilt that allows Lewis to make an arrest.
\item Demonstration of Woodthorpe's guilt is obtained by an application of the function.
  \begin{itemize}
  \item Lewis is unable to provide additional information about how the function is witnessed, but this doesn't prevent Lewis' use of the function.
  \item {\color{red} This is difficult.}
    \begin{itemize}
    \item Lewis uses the fact that Morse reasoned to indirectly reason from the case file to Woodthorpe's guilt by the same relation of support that Morse's reasoning appealed to.
    \item Lewis, if pressed, could say they are confident that the case file demonstrates Woodthorpe's guilty, though they cannot at present provide a demonstration, but Morse's reasoning witnesses the demonstration.
      Along with a note about how Lewis did not (yet) have the opportunity to themselves demonstrate.
      No, Morse did not tell Lewis that Woodthorpe is guilty, but Morse did not need to, for the fact that Morse reasoned ensures a the possibility of a demonstration, and the ability to provide a demonstration of guilt is sufficient to arrest.
      If one also needed the demonstration, then a whole bunch of people should not have been arrested.
      The lack of direct testimony does not prevent Lewis from appealing to the case file that Lewis has and the testimony that one can reason from this to the guilt of Woodthorpe.
    \end{itemize}
  \end{itemize}
\item \(\exists f(f\Sigma = \chi) \approx \Sigma \Rightarrow \chi\), returning to propositional logic
\item And, \(\Sigma, \Sigma \Rightarrow \chi \vdash \chi\).
\item If the evidence in the case file is sound, then Woodthorpe is guilty.
\item On the one hand, the appeal to an existential may be seen as an artefact of how we understand reasoning.
  On the other hand, the use of a conditional obscures the fact that Lewis is drawing on Morse's reasoning.
\item Regardless, Lewis appeals to more than the relation of propositional support that holds between the case file and the guilt of Woodthorpe.
\item Lewis appeals to the process of Morse's reasoning as a witness.
\end{itemize}

\begin{itemize}
\item In both cases, indirect application of the function.
\item \(\exists f(f\Sigma = \chi), \Sigma\), hence \(\chi\).
\item Lewis is confident that there is an appropriate witness, which is part of why this works.
\item If Morse was known to read tea leaves, then the function would still exist, but it would not have relevant witness.
\item Likewise, Morse's confidence in the evidence doesn't really matter, and Morse may have explicitly reported only on their reasoning \emph{because} they did not have an appropriate attitude toward the case file.
  \begin{itemize}
  \item Example here with buying something, and the assistant giving a recommendation based on reported information.
  \item Assistant gives a conditional, because they're only sure that the item is suitable given the information provided, and cannot guarantee that the recommendation would hold given additional information that may have not been stated.
  \end{itemize}
\item These aspects aren't captured in the bare existential quantification, instead how Lewis' reasoned to the existential.
\item So, Lewis obtains their attitude toward the guilt of Woodthorpe by the function.
\item And, Lewis can do this because they the relevant in-out pair of the function.
\item This allows taking an arbitrary function, and doing a standard piece of reasoning.
\item This is blocked if the function appears in the conclusion of the reasoning.
  \begin{itemize}
  \item \(\exists f(f\Sigma = f\Sigma')\), \(\Sigma\), but Lewis can't draw anything from this.
  \item Well, a useless example if \(\exists f(f\Sigma = f\Sigma)\).
    Here, Morse tells Lewis that they've reasoned to a conclusion, but Lewis can't done anything with this.
    Lewis is confident that \(\exists f(f\Sigma = f\Sigma)\), but there is not function \(f\) such that Lewis is confident that \(f\Sigma\).
  \end{itemize}
\end{itemize}

\begin{note}
  Case in which the result depends on the function.

  Romeo came to believe that Juliet 

  A controversial reading of (a kind of) belief may work.
  Romeo, believing Juliet to be alive, discarded the poison.
  \[
    \exists e(\everb{reason} \land \ein{\phi} \land \eout{b\phi} \land \efunc{b} \land \eagent{a})
  \]

  Romeo's reasoning from believing that Juliet is dead to taking poison.\nolinebreak
  \footnote{
  \[
    \exists e
    \left(
      \begin{array}{l}
        \everb{reason} \\
        \ein{
        \exists e'
        \left(
        \begin{array}{l}
          \everb[e']{reason} \\
          \ein[e']{\text{Alive}(\text{Juliet})} \\
          \eout[e']{b(\text{Alive}(\text{Juliet}))} \\
          \efunc[e']{b} \\
          \eagent[e']{Romeo}
        \end{array}
        \right)
        } \\
        \eout{
        \text{Ought}
        \left(
        \exists e''
        \left(
        \begin{array}{l}
          \everb[e'']{discard} \\
          \eagent[e'']{Romeo} \\
          \epatient[e'']{poison}
        \end{array}
        \right)
        \right)
        } \\
        \eagent{Romeo}
      \end{array}
    \right)
  \]
}
  Here, believing is a process which involves some kind of reasoning.
  The input of the process is some proposition \(\phi\), and the output of the process is that the relevant kind of reasoning is occurring.
  Hence, \(b\phi = b\phi\).
  To believe \(\phi\) is just for some relevant kind of reason to occur.
  So, to say that \(a\) believes \(\phi\) is just to say that \(a\) is engaged in some reasoning about \(\phi\) in some particular way.
  Here, understand \(\phi\) as interpreted.
  Now generalise to attitudes, and this is a plausible non-trivial case in which the output depends on the function.
\end{note}


Lewis is confident that Morse's reasoning from the case file to the guilt of Woodthorpe happened.
Hence, Lewis is confident that some reasoning from the case file demonstrates the guilt of Woodthorpe.
The key feature of Morse's reasoning is that the guilt of Woodthorpe does not depend on Morse's reasoning.
If Woodthorpe is guilty, then Woodthorpe is guilty \emph{indenedently} of whether Morse reasons from the case file to Woodthorpe's guilt.
In this sense Morse's reasoning demonstrates Woodthorpe's guilt.

Not all reasoning demonstrates a conclusion that is independent of reasoning.
For example, in the process of reasoning from the case file Morse may have reflected on the activity they were engaged in and noted that they were demonstrating the guilt of Woodthorpe.
Here, the intermediate conclusion that Morse is demonstrating the guilt of Woodthorpe \emph{depends} on the activity that Morse is engaged in.

\begin{note}[Observation]
There's an important difference between the mere existential and the use of a future.
\end{note}

\begin{note}
  This relation is clear in the case of propositional logic and decidability.
  \(\Phi \vdash \psi\) can be read as stating that there is a relation, or that there is a function.

  Standard relation to think about is propositional justification.
  So, \(\Sigma\) and \(\chi\) stand in the relation of propositional justification.

  Well, function does more than stating the relation, and given a relation is doesn't always seem to follow that there's a function.
  At least, not a function of the appropriate kind.
  I mean, there's the incompleteness results which demonstrate this, and this is part of the reason why soundness and completeness results are so nice.

  So, function as the agent is confident that they can \emph{obtain} \(\chi\) from \(\Sigma\).
  However, the agent does not need to \emph{construct} \(\chi\) from \(\Sigma\).
  Any doxastic support is seen as a byproduct of applying the function, or at least \emph{a} kind of doxastic support can be seen in this way.
\end{note}


\begin{itemize}
\item Before going into the details of reasoning, there's a key inference.
\end{itemize}

\[
\exists f(f\Sigma = \chi), \Sigma \vDash \chi
\]

So long as \(\chi\) does not depend on the application of some function, then so long as one is sure of the existence of a function and some key-value pair of that function, then given the key one can also be sure of the value.

\begin{itemize}
\item Simple in the case of the use of functions in logic, and Skolemization.
  \begin{itemize}
  \item {\color{red} Stress the idea of `Skolemization' here, as this does not apply to futures and promises.}
  \item E.g.\ key inference relied on idea\dots and this does not apply to futures, in particular because a claim about a future is often a claim about a process.
  \end{itemize}
\item Extends to mathematics in a straightforward way.
\item And to other instances of reasoning.
\end{itemize}

\begin{itemize}
\item In the case of Lewis, the contrast is between Woodthorpe's guilt and the demonstration of Woodthorpe's guilt.
\item Lewis is confident that Woodthorpe is guilty, though they are not confident that they have demonstrated Woodthorpe's guilt.
\item The lack of demonstration is important, and it is also important to highlight that Lewis is not necessarily certain of Woodthorpe's guilt, hence nor the possibility of demonstrating Woodthorpe's guilt.
\item However, confidence that Woodthorpe is guilty is all that is needed to motivate an arrest.
  \begin{itemize}
  \item To see that the demonstration is not necessary for Lewis, run the same scenario with explicit testimony from Morse that Woodthorpe is guilty.
  \end{itemize}
\item One way to understand this is in terms of propositional support.
  \begin{itemize}
  \item If Morse has demonstrated the guilt of Woodthorpe from the case file, then Morse has doxastic support.
  \item In turn, there is propositional support.
  \item The function then relies of propositional support, and Lewis doesn't have doxastic support.
  \end{itemize}
\end{itemize}

\hozlinedash

\begin{itemize}
\item This assumes that the agent is confident that \(\exists f(f\Sigma = \chi)\).
\item Haven't yet shown why this is the case.
\item Quick answer in the case of Lewis is that \(\Sigma\) and \(\chi\) are the ins and outs of Morse's reasoning.
\item However, Morse's reasoning is some process, while the function is something witnessed by the process.
  \begin{itemize}
  \item Problem is that these a potentially two different types of things.
  \item Possible to argue for a reduction, but this isn't clear.
  \item Consider logic cases, where there are soundness and completeness theorems.
    Have semantic entailment, so there's some syntactic entailment, but some work needs to be done to show that this can be witnessed by some reasoning.
    For, it is possible that the syntactic proof requires resources beyond the reach of agents like us.
    Hence, may need to consider reasoning being witnessed by idealised agents, and so on.
  \item This does not rule out reduction, but to avoid taking on unnecessary burdens, function may be of a distinct kind.
  \item Further complications with method of reasoning, i.e.\ algorithm, as this too may be something distinct.
  \end{itemize}
\end{itemize}

\subsubsection{Two different types of reasoning, briefly characterised}
\label{sec:two-different-types}

\begin{figure}[h]
  \begin{subfigure}{.5\textwidth}
    \centering
    \begin{tikzpicture}[
      ->,
      >=stealth',
      % auto,
      node distance=0cm, every text node part/.style={align=center},
      ]

      \node [] (c) at (0,0) {};
      \node [] (d) at (-3,0) {};
      \node [] (e) at (3,0) {};
      \node [] (f) at (0,-2.1) {};

      \node (1) at (0,-.1) {\(\{\exists f (f\futin = \futout)\} \cup \futin\)};
      \node (2) at (0,-2) {\(\futout\)};

      \draw [->] (1.270) to [] node[left] {\(g\)} (2.90);
    \end{tikzpicture}
    \caption{Incorporeal \\ Gather up the resources along with the existential.}
    \label{fig:non-constructive}
  \end{subfigure}
  % \hfill
  \begin{subfigure}{.5\textwidth}
    \centering
    \begin{tikzpicture}[
      ->,
      >=stealth',
  % auto,
      node distance=0cm, every text node part/.style={align=center},
      ]

      \node [] (c) at (0,0) {};
      \node [] (d) at (-3,0) {};
      \node [] (e) at (3,0) {};
      \node [] (f) at (0,-2.1) {};

      \node (1) at (0,-.1) {\(\futin\)};
      \node (2) at (0,-2) {\(\futout\)};

      \node (x) at (-2,-1.05) {\(\exists f(f\futin = \futout)\)};

      \draw [->, dashed] (1.270) to node[left] (3) {\(\future{f}\)} (2.90);

      \node (4) [left of=3, xshift=-2cm] {}; % {\(\exists f (f\Phi = \psi)\)};
      \draw [-{Circle[open]}] (x.0) to (3.180);

    \end{tikzpicture}
    \caption{Speculative \\ Use the existential to secure a promise of a witness.}
    \label{fig:speculative}
  \end{subfigure}
\end{figure}

\begin{itemize}
\item Incorporeal:
  \begin{itemize}
  \item Lewis does some reasoning on the basis of there existing some reasoning which demonstrates the support for arresting Woodthorpe.
  \end{itemize}
\item Speculative:
  \begin{itemize}
  \item Lewis takes there to be support for arresting Woodthorpe on the basis of reasoning that Lewis hasn't yet done.
  \item Lewis \emph{uses} the function but defers the demonstration to some future reasoning.
    Hence, Lewis' reasoning is `incomplete'.
  \end{itemize}
\item The key idea with both types of reasoning is that the output of the function does not depend on the reasoning.
  Hence, one can, in principle, obtain \(\futout\) from \(\futin\), as explained.
  \begin{itemize}
  \item This is important to emphasise, as it rules out certain cases.
  \item Still, examples of where the output \emph{depends} on reasoning are hard to come by.
  \item Self-referential cases can be found, e.g.\ belief that one has reasoned to \(\futout\).
  \item For the most part, however, the ins and outs are not usually restricted in this way.
  \item May think that there's also a restriction on the attitudes that the agent may have to the output, though.
  \item This is plausible, but substantive.
    \begin{itemize}
    \item \citeauthor{Sinhababu:2017aa} gives an example with desire, and there may be a case for belief somewhere\dots
    \item As a rough heuristic, think of what can be done on the basis of testimony, as here one doesn't have access to the reasoning, and any case of testimony which details the ins and outs and makes a claim to reasoning can be recast in these terms.
    \end{itemize}
  \end{itemize}
\end{itemize}

\hozline


\subsection{Promises and futures}
\label{sec:promises-futures}

\begin{itemize}
\item Promises reverse the extraction of the function with argument and value specified.
\item Agent promises some function, and there is some process that conforms to the function.
\item This isn't so straightforward.
  \begin{itemize}
  \item For, given how things have been set up, processes, algorithms, and functions could all be promised.
  \item And, given that functions may be abstract key-value pairs, it's not necessarily going to be useful to provide a referent for the function used.
  \end{itemize}
\end{itemize}

\begin{note}[Promise]
  A promise is a characteristic function which may have an indeterminate truth value.
  The key idea is that a promise sets out a bunch of conditions that any argument must satisfy in order for the function to be true.
  So, the task is to specify something for which the function returns true.
  And, if the function returns true, then the argument is a way to fulfil the promise.

  They key idea is that if there's some reasoning that the agent is confident of, then we have something of the form

  \[\exists e(\Phi(e))\]

  And from this one can create the promise

  \[\text{promise}(e)\]

  Where

  \[\text{promise}(e) = \top \leftrightarrow \Phi(e)\]

  And, in the cases of interest there may be additional restrictions

  \[\text{promise}(e) = \top \leftrightarrow \Phi(e) \land \Psi(e)\]

  Hum, and alternative is to return the argument is it satisfies the constraints, so now the promise looks a whole lot like a test.
  Yeah, this is really nice, I think.


  \[\text{promise}(e) = e \leftrightarrow \Phi(e) \land \Psi(e)\]

  And then

  \[\exists e(\cdots \land \text{promise}(e) = e \land \cdots)\]

  This is a useful way to think about promises, as the standard idea with a test is that you run the test, and this allows one to continue or not, here, and here the idea is the same, it's simply a test, and the equality holds if the test passes.
  However, the additional idea is to project a future from the test.
  So, this is what we would have if the test is satisfied.

  \[\text{future}(\text{promise}(\dot{e})) = \ddot{e}\]
\end{note}

\begin{note}[Constructive?]
  Can consider two different types of tests, for different types of reference, respectively.
  On the one hand, could require something with direct reference, and on the other could allow the guarantee of the existence of something.
  
  And, this corresponds to two different types of promises.
\end{note}

\hozlinedash

So, promise the function and project a future, and then describe Lewis' reasoning as

\[
  \exists e(\everb{reason} \land \ein{\Sigma} \land \eout{\chi} \land \efunc{\future{f}})
\]

This isn't a standard case of reasoning, as \(\future{f}\) is a future.
Lewis can't properly reason via \(\future{f}\) as Lewis doesn't really know what this function is.
In some sense, the promise should be part of the ins of the reasoning.
But then the reasoning stated above can be seen as \emph{part} of the agent's reasoning.
Well, this is one way of thinking about things.
The other is to understand \(\future{f}\) as a particular kind of polymorphic function.
As it occurs here, it's a future, so the agent reasons with a future, and this doesn't really do anything in particular.

So, here, the type of the function tells us something about what the reasoning is.
And, as it's an unfulfilled promise, we know that there's very little going on.

Some time later, the agent does some additional work, and reasons from \(\Sigma\) to \(\chi\), and provides a witness.

\[
  \exists e(\everb{reason} \land \ein{\Sigma} \land \eout{\chi} \land \efunc{\underline{\future{f}}})
\]

The future now has a witness.

\[
  \text{promise}(f,f\Sigma = \chi)
\]

\[
  \text{future}(\text{promise}(f,f\Sigma = \chi)) = \future{f}
\]


{\color{red}
  Lewis can't reach back in time and provide the reasoning.
  However, Lewis can create something that refers to the reasoning that they will do.
  So, this refers, and has some important constraints.
  And, hence, \dots
}

\begin{note}
  Strictly speaking, on the understanding proposed, Lewis \emph{does} reason from \(\Sigma\) to \(\chi\).
  For, Lewis forms an attitude toward \(\chi\) based on their attitudes toward \(\Sigma\).
  Lewis' reasoning is based on the promise that their reasoning can witness a certain function.
  Lewis does not do work to demonstrate \(\chi\) on the basis of \(\Sigma\).
  However, Lewis promises the work.
  Function is just a collection of key-value pairs, and so the two components of reasoning are asynchronously distributed.
  So, with the promise there's little process to speak of.
  However, even if there's little of interest, there's still something.
\end{note}

\[
  \exists e(\everb{reason} \land \ein{\Sigma} \land \eout{\chi} \land \efunc{\future{f}} \land \emethod{speculate})
\]

\newpage

\mbox{ }

\newpage

\section{Less old notes}
\label{sec:less-old-notes}


\subsection{Logic}
\label{sec:logic}

\begin{note}
  Here I'm only dealing with the reasoning, so there's no discussion of justification and so on.
  All that's happening is an account of the reasoning that's going on in the case of promises.
\end{note}

This about some rules governing existentials.
Standard idea is to take a fresh constant, and show that something follows from this.
The existential guarantees that the term refers, but have no idea what to, and the fresh constant is used with the promise that it is referential.

The principle is similar to existential elimination rules.
Given a formula of the form \(\exists x Px\), fresh constant \(a\) and use this to reason about an object that \(P\) holds of.

% \begin{multicols}{2}
\begin{prooftree}
  \AxiomC{\(\exists x Px\)}
  \AxiomC{}
  \RightLabel{\scriptsize(1)}
  \UnaryInfC{\(Pa\)}
  \AxiomC{\(\forall x(Px \rightarrow Qx)\)}
  \RightLabel{\scriptsize \(\forall\) E}
  \UnaryInfC{\(Pa \rightarrow Qa\)}
  \RightLabel{\scriptsize \(\rightarrow\) E}
  \BinaryInfC{\(Qa\)}
  \RightLabel{\scriptsize \(\exists\) I}
  \UnaryInfC{\(\exists x Qx\)}
  \RightLabel{\scriptsize 1 \(\exists\) E}
  \BinaryInfC{\(\exists x Qx\)}
\end{prooftree}

This can be reformulated.

\begin{prooftree}
  \AxiomC{\(\exists x Px\)}
  \AxiomC{}
  \RightLabel{\scriptsize(1)}
  \UnaryInfC{\(\future{a}\)}
  \BinaryInfC{\(Pa\)}
  \AxiomC{\(\forall x(Px \rightarrow Qx)\)}
  \UnaryInfC{\(P\future{a} \rightarrow Q\future{a}\)}
  \BinaryInfC{\(Q\mathcal{a}\)}
  \UnaryInfC{\(\exists xQx\)}
\end{prooftree}



Quantification over elements is standard, but this can be extended to functions.

\begin{prooftree}
  \AxiomC{\(\exists f(fa = b)\)}
  \AxiomC{}
  \RightLabel{\scriptsize(1)}
  \UnaryInfC{\(\future{f}a = b\)}
  \AxiomC{\(Pb\)}
  \RightLabel{\scriptsize \(=\) E}
  \BinaryInfC{\(P\future{f}a\)}
  \RightLabel{\scriptsize \(\exists\) I}
  \UnaryInfC{\(\exists f Pfa\)}
  \RightLabel{\scriptsize 1 \(\exists\) E}
  \BinaryInfC{\(\exists f Pfa\)}
\end{prooftree}
% \end{multicols}

Here, need to reintroduce quantification over \(f\) because the premises do not provide a way of referring to \(f\).

Straying further from standard proof systems, introduce a function and bind this to the function that is stated to exist.

\begin{prooftree}
  \AxiomC{\(\exists f(fa = b)\)}
  \AxiomC{}
  \RightLabel{\scriptsize(1)}
  \UnaryInfC{\(\future{f}\)}
  \RightLabel{\scriptsize 1 B}
  \BinaryInfC{\(\future{f}a = b\)}
  \AxiomC{\(Pb\)}
  \RightLabel{\scriptsize \(=\) E}
  \BinaryInfC{\(P\future{f}a\)}
  \RightLabel{\scriptsize 1 \(\exists\) E}
  \UnaryInfC{\(\exists fP(fa)\)}
\end{prooftree}

In these examples, \(a\) and \(\future{f}\) are fresh, and \(\psi\) is inferred on the basis of these, and because no assumptions are made regarding \(a\) and \(\future{f}\), one can be sure that \(P\) holds of some transformation of \(a\).

Deductive system, hence the need to discharge assumptions.

Abstracting further, bind function and apply to something.
\(n B\) denotes the binding of the future \(n\).

\begin{prooftree}
  \AxiomC{\(\exists f (f\Phi = \psi)\)}
  \AxiomC{}
  \RightLabel{\scriptsize(1)}
  \UnaryInfC{\(\future{f}\)}
  \RightLabel{\scriptsize 1 B}
  \BinaryInfC{\(\future{f}\Phi = \psi\)}

  \AxiomC{}
  \RightLabel{\scriptsize(1)}
  \UnaryInfC{\(\future{f}\)}
  \AxiomC{\(\Phi\)}
  \RightLabel{\scriptsize 1 A}
  \BinaryInfC{\(\future{f}\Phi\)}

  \RightLabel{\scriptsize \(=\) E}
  \BinaryInfC{\(\psi\)}
  % \RightLabel{\scriptsize 1 \(\exists\) E}
  % \UnaryInfC{\(\psi\)}
\end{prooftree}

The idea remains the same, use the guarantee of reference to reason with a specific instance with a referring term.
As before, obtain \(\psi\), for if \(\Phi\) holds, and there is some transformation of \(\Phi\) which yields \(\psi\) then \(\psi\) holds.
\begin{note}
  Think about Skolemization.
\end{note}

The difficulty for the agents in the scenarios is the existential statement, and whether this is in fact true.


\subsection{Futures and promises}
\label{sec:futures-promises-1}

\begin{itemize}
\item Pause to sketch out the big picture.
\item Fairly brief, as this will be developed in more detail with comparison to other types of reasoning in a later section.
\end{itemize}

\section{Some applications}
\label{sec:some-applications}

\begin{itemize}
\item Use the basic framework to understand a few scenarios, before dealing with the abstraction some more.
\item Some applications are puzzling, others are more familiar.
\item Test of appealing to something as an excuse as an indicator of it doing some work.
  \begin{itemize}
  \item I.e.\ the exam case/cheaters defence.
  \item The cheater promises that they could have reasoned.
  \end{itemize}
\end{itemize}


\subsection{Scenarios}
\label{sec:scenarios}

\subsubsection{Shopping}
\label{sec:shopping-1}

\begin{scenario}[Shopping]
  Agent is shopping in a supermarket and has an end.
  On the shopping list is an item.
  The agent cannot immediately recall how the item relates to the end, but is confident that they put the item on the shopping list in service of the end.
  The agent is confident that purchasing the item is worthwhile means to the end, but they are not sure how.
\end{scenario}

\begin{note}
  \begin{itemize}
  \item Here, the key is an additional promise, and a shift in focus.
  \item For, the reasoning from the ends to the means does not seem too difficult, and hence it's specifying the ends which is key.
  \end{itemize}
\end{note}

\begin{prooftree}
  \AxiomC{\(\exists f \exists X(fX = \psi)\)}
  \AxiomC{}
  \RightLabel{\scriptsize(F1)}
  \UnaryInfC{\(\future{f}\)}
  \RightLabel{\scriptsize 1 B}
  \BinaryInfC{\(\future{f}X = \psi\)}
  \AxiomC{}
  \RightLabel{\scriptsize (F2)}
  \UnaryInfC{\(\future{X}\)}
  \RightLabel{\scriptsize 2 B}
  \BinaryInfC{\(\future{f}\future{X} = \psi\)}

  \AxiomC{}
  \RightLabel{\scriptsize(F1)}
  \UnaryInfC{\(\future{f}\)}
  \AxiomC{}
  \RightLabel{\scriptsize (F2)}
  \UnaryInfC{\(\future{X}\)}
  \RightLabel{\scriptsize 1 2 FA}
  \BinaryInfC{\(\future{f}\future{X}\)}

  \RightLabel{\scriptsize \(=\) E}
  \BinaryInfC{\(\psi\)}
\end{prooftree}


\section{Reasoning}
\label{sec:reasoning}

\begin{note}
  The reasoning section details the beliefs 
\end{note}

\begin{itemize}
\item promise stands in place of some to-be-done process.
\item I promise that when I will understand this, I'll call back.
\end{itemize}

A promise is a proxy for a result that it initially unknown.

\begin{itemize}
\item pending state
\item promises are fulfilled
\item A promise is a container for an as-yet-unknown value
\item extract the value out of the promise and give it to another process
\end{itemize}


This is a more-or-less technical term, especially as in these cases the promise is something that is not shared.
However, there's some intuition, especially when one considers interacting with the agent.

The idea is that the agent makes something like a promise.
They provide something which looks like a function, and this is used to obtain \(\phi\).

Here, with the existential, it's nothing new.
Taking a fresh variable, with the promise that this \emph{can} be given a referent.



\section{Precedent}
\label{sec:precedent}

\begin{itemize}
\item \citeauthor{Boghossian:2014aa}'s self-awareness condition (\citeauthor{Siegel:2019aa} provides a good overview).
\item \textcite{Pryor:2018aa}: Discussing the relationship between categorical and hypothetical justification, though this takes some work to line up.
\item \textcite{Siegel:2019aa}: \emph{Reckoning de dicto} S reckons that (for some G: having G supports Q).
\item The cases that motivate \textcite{Worsnip:2018aa}, \textcite{Fogal:2019aa}, and others are similar to those that motivate the kind of cases that I'm interested in, but a conclusion is prevented.
\end{itemize}

\subsection{Siegel}
\label{sec:siegel}

In \citeauthor{Siegel:2019aa}'s (\citeyear{Siegel:2019aa}) terminology, there is no `reckoning'.
The type of case is related, though distinct.
In the type of cases \citeauthor{Siegel:2019aa} discusses, agent's draw inferences in ignorance of the exact factors that they are responding to (\citeyear[8]{Siegel:2019aa}).
By contrast, in the types of cases under consideration the agent may be aware of the factors that they are responding to, and is aware that an inference to some proposition can be drawn, but is unaware of what the inference is.
Here, then, unclear that the interest is in inference, specifically.
I.e.\ not going to make the claim that the agent infers the relevant proposition \emph{and} it's possible for one to keep hold of the reckoning model that \citeauthor{Siegel:2019aa} argues against while puzzling about these types of cases.

\hozlinedash

\citeauthor{Siegel:2019aa} notes that if inference meets \citeauthor{Boghossian:2014aa}'s self-awareness condition, `then inferrers are never ignorant of the fact that they are responding to some of their psychological states, or why they are so responding.' (\citeyear[6]{Siegel:2019aa})

\begin{description}[font=\bfseries, leftmargin=.75cm, style=nextline]
\item[Self-awareness condition] Person-level reasoning [is] mental action that a person performs, in which he is either aware, or can become aware, of why he is moving from some beliefs to others.\nolinebreak
  \mbox{}\hfill\mbox{(\citeyear[16]{Boghossian:2014aa})}
\end{description}

\begin{itemize}
\item \citeauthor{Siegel:2019aa} considers scenarios in which an agent doesn't recognise what they're responding to.
\item This is different to the scenarios I consider because for me it is the failure to do the reasoning that is important.
\item So, in the scenarios that \citeauthor{Siegel:2019aa} considers, it seems to be the case that the agent does do the reasoning, but without self-awareness that they are doing the reasoning.
\end{itemize}


\section{Conflicts}
\label{sec:conflicts}

\begin{itemize}
\item Lord, and responding to reasons.
\item Time slice epistemology, as it seems there's something asynchronous at work, and hence there's some trouble reducing everything to the time-slice of an agent.
\end{itemize}

\newpage

\hozline

\section{Old notes}
\label{sec:old-notes}



\subsubsection{Incorporeal}
\label{sec:incorporeal}

\begin{note}
  Also to emphasise is that both the types of reasoning considered here can be seen as `time-slice' reasoning.
  This, then, allows a contrast to the distributed reasoning of promises.
\end{note}

If the agent is concerned with the reliability of the companion, then the agent needs some way to project the companions reliability with respect to questions for which the agent also settles to questions for which the agent does not settle.

The difficulties present are clear if we assume that the agent settles on a conflicting answer.
Does this suggest that the companion has been bluffing?
Has the companion made a mistake or overlooked a potential move?
Has the agent made a mistake?
Is the companion better, but previous experience has been unable to show this?
These questions can be answered, and the agent may have expectations regarding the likelihood of each of these scenarios.

Here, the proposition is whether the companion's response is truthful.

Here, that the agent can reason may be important, as it narrows down the reference class to which the companion's statement belongs.
However, the agent makes no commitment to provide reasoning as a witness to their ability to reason to the existence of a winning strategy.



\begin{itemize}
\item It is possible to reason from the board and the rules of chess to the existence of a winning strategy.
  \begin{itemize}
  \item No need for the agent and companion to be similarly matched.
  \end{itemize}
\item It is possible for the agent to reason from the board and the rules of chess to the existence of a winning strategy.
  \begin{itemize}
  \item Assumes that the agent and companion are similarly matched.
  \item It is unclear whether the agent's confidence that \emph{they} are able to reason to a winning strategy is relevant.
  \item The agent's reasoning is premised on confidence that there exists a witnessing processes, but the agent's reasoning only appeals to certain properties that a witnessing processes satisfies.
    \begin{itemize}
    \item The primary property is that the witnessing process demonstrates the existence of a winning strategy.
    \item Other properties are that the reasoning of the companion is a witness, and that the agent could provide a (perhaps distinct) witness.
    \item These latter properties are secondary because the agent only needs to be confident that a winning strategy exists.
    \item In short, the agent's reasoning that a winning strategy exists depends only on the existence of a witness.
    \item The possibility for the agent to reason is a byproduct of the support for the existence of a winning strategy that the agent appeals to.
  \item Perhaps the agent makes an attempt to reason to a winning strategy given their confidence that they \emph{can} reason to such a strategy, etc.
  \item Counterfactually, the agent could have entered the arrangement of pieces on the board into a chess program, or consulted a grand master.
  \item Of course, the way in which the agent establishes confidence does matter, and the details of the agent's reasoning would change in each of these counterfactual cases.
    However, the broad structure would remain the same: There is a way to demonstrate that a winning strategy follows from the arrangement of the pieces on the board and the rules of chess, and the existence of a winning strategy does not depend on demonstrating that it exists, therefore confidence that there is a demonstration of a winning strategy supports confidence that a winning strategy exists.
    \end{itemize}
  \end{itemize}
\end{itemize}


\subsubsection{Weights}
\label{sec:weights}

\begin{note}
  Important to note here where the uncertainty fits in.
  \begin{itemize}
  \item The uncertainty is part of what supports the agent settling on a particular action.
  \end{itemize}
\end{note}

\begin{itemize}
\item Agent considers outcome of actions given the existence or non-existence of a winning strategy, and uses their confidence of the existence of a winning strategy to weigh the possible outcomes.
\item In short, decision theoretic reasoning.
\item The idea that the agent considers their ability to reason to a winning strategy is reflected in outcomes.
\item Grant that the structure of the agent's reasoning is rational, and must argue over evaluation of outcomes.
\item If the agent recognises some tension, then it can only be due to their evaluation of outcomes.
  \begin{itemize}
  \item Incommensurability between different norms, or a different perspective (i.e.\ what would happen in the ideal case).
  \end{itemize}
\item Puzzle about how the possibility to reason fits into the evaluation of outcomes.
  \begin{itemize}
  \item I do not doubt that some explanation can be given, but it doesn't seem straightforward.
  \item And, it seems as though there may be some complexity in the agent's reasoning.
  \end{itemize}
\item Decision theoretic reasoning is not the only model of reasoning, so propose considering other options.
\end{itemize}

\hozlinedash

\subsection{Speculative}
\label{sec:speculative}

\begin{note}
  Here, first emphasis is on the agent's confidence that they are able to reason to the conclusion, paired with the observation that the above types of reasoning do not structurally depend on this, as they only deal with the existential.
\end{note}

\begin{note}
  There's the possibility of background norms, which should be noted somewhere.
  \begin{itemize}
  \item This links back to the idea that there's some pressure for the agent to reason to the conclusion.
  \item And, the promise does something odd here, as the agent doesn't completely ignore this pressure, but they do not immediately respond to it.
  \end{itemize}
\end{note}


\begin{note}
  \begin{itemize}
  \item Somewhat between the two types of incorporeal reasoning.
  \item As in the case of making an assumption, the agent uncertainty is ignored when reasoning, but in contrast the uncertainty continues to support the agent's reasoning, as the agent needs this to be low in order to guarantee the promise.
  \item Here, there's a clearer view of the puzzle, because the agent considers reasoning to the conclusion.
    \begin{itemize}
    \item This is an important contrast, as the two types of reasoning above do not depend on the agent having confidence that they are able to reason to the conclusion.
    \end{itemize}
  \item Whether the agent is really able to act with the promise provides a clearer view of the puzzle.
  \item For, if the agent makes the promise and fulfils it, then there's no problem in principle, in the same way that one trades futures.
  \item However, one may think that making the promise isn't permissible in this case, and that the agent needed to do the reasoning.
  \end{itemize}
\end{note}
\begin{itemize}
\item Agent creates a future with an associate promise.
\item Agent's confidence that there exists some reasoning from the board and the rules of chess to a winning strategy is reflected in the agent's confidence that the promise can be fulfilled.
\item Agent now assumes that they have demonstrated the existence of a winning strategy by the promised reasoning.
  \begin{itemize}
  \item This is not the same as reasoning, esp.\ clear as reasoning will provide additional information, such as specific moves.
  \end{itemize}
\item So, as the agent assumes they have demonstrated the existence of a winning strategy, the reasoning is similar to simply assuming the existential is true.
\item However, it differs in that the agent does not ignore the possibility of being false.
\end{itemize}

\begin{itemize}
\item In contrast to incorporeal certainty, the agent providing some reasoning would not be retroactive justification.
\end{itemize}

\begin{itemize}
\item Judgements given the promise of a future are complex.
\item Suppose the agent resigns, and the later demonstrates that there was a winning strategy.
\item The action of resigning was taken under the risk that the promise would not be fulfilled.
\item However, the promise was later fulfilled, and hence met the conditions set out by the promise.
\item Of course, the agent is unlikely to fulfil the promise, and this complicates matters further, as this doesn't invalidate the agent's promise; only a failure to demonstrate that they could have reasoned would do so.
\end{itemize}


There is a different proposition, whether the agent would have determined that a winning strategy exists.
Things are more straightforward.
The agent has often received the companion's response prior to settling, and it has always coincided with how the agent would have responded if truthful.
It is likely that a truthful interpretation of the companion's response reveals the answer that the agent would have settled on.

And, the agent has a high degree of confidence that the companion is truthful if the agent has settled on an answer.

The agent is confident that they could reason to a winning strategy.
There are two key thoughts here.
First, that the agent can do some reasoning.
Second, what they result of the reasoning would be.

Though the agent has not reasoned to the existence of a winning strategy, the agent adopts an attitude toward the existence of a winning strategy on the basis of the reasoning they are confident that they are able to do.
With some caveats, the agent has the same attitude toward the existence of a winning strategy that the agent would have if they were to reason to the existence of a winning strategy by some witness for the reasoning from the state of the board and the rules of chess to a winning strategy.

\subsubsection{Key features}
\label{sec:key-features}

The agent is confident that there exists a way to reason from the arrangement of pieces on the board and the rules of chess to a winning strategy.

\begin{enumerate}
\item Transfer of risk
\item Promise
\end{enumerate}


\subsubsection{Why?}
\label{sec:why}

\begin{itemize}
\item Well, the distinguishing feature is that \(\psi\) is now treated as the result of reasoning.
\end{itemize}

\begin{itemize}
\item Decision is made under an assumption that is uncertain.
  \begin{itemize}
  \item Here, then, the agent puts uncertainty to the background.
  \end{itemize}
\item Decision is made under uncertainty.
  \begin{itemize}
  \item The uncertainty is in the foreground.
  \end{itemize}
\end{itemize}

\begin{itemize}
\item One clear difference is that in the speculative case the agent may go back to reason, and whatever conclusions the agent drew will likely continue to hold, as these were drawn on the assumption of the agent reasoning to the existence of a winning strategy.
\item In contrast, if the agent reasons from confidence of existence alone, then whatever attitude the agent forms will continue to hold.
\end{itemize}

\begin{itemize}
\item The outcome may be the same, and one may argue that there shouldn't be a distinction.
\item However, this isn't obvious.
\item Imagine taking the agent to an interrogation room.
  \begin{itemize}
  \item Here, the agent works to demonstrate the existence of a winning strategy, and hence fill in their reasoning.
  \item By contrast, if the agent had reasoned from the existence, then they would not need to do this.
  \item Note, that the agent wouldn't invoke the existential in the reasoning that they would provide.
  \item And, even if reasoning by a promised future ensures that there is some line of reasoning from the existential, it's not clear that these are equally accessible.
  \end{itemize}
\item If focusing on the resignation, there's no too much of interest.
\item If focusing on justification there is a difference.
\end{itemize}

\begin{itemize}
\item Two important ideas.
  \begin{enumerate}
  \item Adoption of risk
  \item A guarantee to mitigate some of this risk.
  \end{enumerate}
\item These are the two distinguishing features.
  \begin{itemize}
  \item The promise doesn't do everything, and in some aspects it's not different to assuming the existential.
    However, there are distinguishing features that suggest this is of some interest, and in what follows the interaction between these two features and their relative emphasis is key.
  \end{itemize}
\end{itemize}

\subsection{Notes}
\label{sec:notes}

\begin{note}
  It seems as though there are cases where the distinction between witnessing the process and witnessing the algorithm may matter.
  For example, you inform my that you're reasoning from \(\futin\) to \(\futout\), but your reasoning was quite detailed.
  I may be confident that \(\futin\), and so form an attitude to \(\futout\).
  However, what matters to me is that your reasoning was adequate, not that I do the reasoning.
  Hence, I check your reasoning, but I do this in a piecemeal way, so I never instantiate the algorithm itself, but having verified that the algorithm works, I'm satisfied that my attitude toward \(\futout\) is adequately supported.
  \begin{itemize}
  \item Still, these types of cases can still be understood in terms of the simplified formalism.
  \end{itemize}
\end{note}

\hozlinedash

\subsection{Reflection}
\label{sec:reflection}

\begin{itemize}
\item Related to, but distinct from, reflection principles.
\item Reflection focuses on credences and conditionalization.
\item The agent is aware of how they will reason.
\item So, it is narrower than the broad account of reasoning.
\item Agent's future self.
\item So, also somewhat orthogonal.
\end{itemize}

\begin{itemize}
\item Right, this difference with reflection is that of a future self.
\item In the cases I'm thinking of, it's more about the limitations of one's current self.
\end{itemize}

\begin{itemize}
\item The easier distinction to make is that reflection is about evidence possessed, rather than reasoning that can be done.
\item So with reflection, the idea is to take on board future evidence given one's credence that one will learn the evidence.
\item Here, the idea is to reason on the basis of your confidence that you are able to reason in a certain way.
\end{itemize}

\section{Oughts?}
\label{sec:oughts}

\begin{itemize}
\item Idea is that there's a general principle of the form ``if reason to \(\phi\), then \(\phi\)''.
\item Roughly, defer to a better point of evaluation.
\item This is what happens with testimony, arguably, and with normative claims in general.
\item If this is so, then the argument that I want to make is that the agent is highly confident that speculating results in more closely satisfying the better point of evaluation.
\end{itemize}


\begin{itemize}
\item So, when we evaluate what the agent is to do, we adopt \emph{some} evaluative point of view.
\item The question is whether the agent's current reasoning fits into this.
\item This is a difficult question.
\item The supermarket case is interesting.
\item On the one hand, the agent should buy the carambola, because they desire starfruit.
\item On the other hand, if the agent never comes to recognise that carambola and starfruit are the same thing, then the purchase of the carambola will not serve as a means to the agent's end.
\item On the basis of the attitude that Lewis has towards the evidence, Lewis should arrest Woodthorpe.
\item However, if Lewis does not provide a demonstration of Woodthorpe's guilt, then the arrest does more harm than good.
\item I take it that these judgements are relatively clear.
\item The suggestion is then that the agent's in these scenarios are confident that the enhanced evaluative outlook is correct.
\item Hence, that performing the action is appropriate.
\item This is where the contrast to \citeauthor{Paul:2014aa} is clear, as with \citeauthor{Paul:2014aa} one is unsure about what the alternative outlook is, and for me the agent is confident.
\end{itemize}

\begin{itemize}
\item So, I'm claiming that there's a middle ground of permissibility.
\item Hence, there are things the agent ought to do.
\item And, there are things the agent ought not to do.
\item And, there are things that are permissible.
\item Hence, the thing I need to argue against is the idea that the agent speculating is something that they ought \emph{not} to do, rather than it not being something the agent ought to do.
\end{itemize}

\begin{itemize}
\item If preferred, term this `weak' rationality, which is the agent not doing things that they ought not to do.
\item Can use games to illustrate this.
\item There are ideal moves, then there are non-ideal moves, and then there are illegal moves.
\item There are probably better ways to do this\dots
\end{itemize}

\begin{itemize}
\item What is needed is an argument for evaluating from the point of view of the information state that the agent is confident that they could be in, were they to do the reasoning.
\item One argument is that this is what is needed for the kind of judgements that come in the cases that \citeauthor{Worsnip:2018aa} considers.
\item For, in the \citeauthor{Worsnip:2018aa} type cases, an agent has an attitude, and then becomes confident that this attitude is not supported on the basis of other attitudes.
\item Hence, if the agent is to revise against their held attitude, then they need to consider the evaluative viewpoint that shows the attitude is not supported.
\item If the agent can ignore this viewpoint, then they can keep hold of the attitude.
\item Yet, the intuition is that the agent cannot keep hold of the attitude on the basis that tqhey haven't demonstrated the inconsistency.
\item So, the same with being confident that \(\phi\) and being confident that \(\psi\) and then becoming confident that \(\lnot(\phi \land \psi)\).
\item An important distinction here is that in these types of cases, an agent revises away from holding an attitude, while in the types of cases I'm interested in, the agent revises toward an attitude.
\item So, this isn't quite general enough.
\item Testimony is an option, but the issue here is that testimony isn't typically about ability to reason, or rather there are cases of testimony which seem to be straightforward evidence, and it may be the case that all instances of testimony leading to the formation of an attitude can be analysed in this way.
\end{itemize}

\section{Not ought}
\label{sec:not-ought}

\begin{itemize}
\item One way of arguing would be to show that the agent ought to form the relevant attitude.
\item That is, to show that the slightly idealised point of evaluation generates a relevant ought statement, and pair this with an account of enkrasia.
\item That is, do what you are confident you ought to do.
\end{itemize}

\begin{itemize}
\item I think this may be true.
\item However, I will argue for something weaker.
\item The point of evaluation is not ruled out.
\end{itemize}

\hozlinedash

\begin{itemize}
\item The question is whether the agent's confidence that they are able to provide a witness allows the agent to form an attitude toward the conclusion based on their attitude toward the premises.
\end{itemize}

\begin{itemize}
\item If this is the question, then whether or not the agent ought to form the attitude isn't quite right, as we're not interested in whether or not the agent should form the attitude, but whether or not the agent is licensed to form the attitude in a particular way.
\item Right, and in all of the ideal worlds, the agent will do the reasoning.
\item However, this isn't the right point of view, because we are not interested in what happens in the ideal worlds.
\item Rather, we're interested in what happens given that we are in a non-ideal world.
\item The general claim that I need to make is that agent's rely on the claims of \emph{un-witnessed} reasoning.
  \begin{itemize}
  \item For, I'm in an okay position if the agent is confident that \(\chi = f\Sigma\) for some \(f\).
  \item The competing argument is that the agent can only have the attitude if the agent has a witness for \(f\).
  \item This is why \citeauthor{Lord:2018aa} is an interesting contrast, for \citeauthor{Lord:2018aa} requires a witness in order to \emph{correctly} respond to reasons.
  \end{itemize}
\item And, one way of understanding this is abstracting to a viewpoint in which there is a witness.
  \begin{itemize}
  \item Here, with Dowell, the almost-parallel is that the agent is not sure whether there is a witness for the diagnosis or not.
  \end{itemize}
\item \emph{the puzzle is about the witness}
\item In other words: \emph{What determines whether or not an inference is licensed}?
  \begin{itemize}
  \item Argument is that it is not solely determined by the agent's present reasoning.
  \end{itemize}
  \begin{itemize}
  \item This is where there's some similarity to \citeauthor{Siegel:2019aa}, as in \citeauthor{Siegel:2019aa}'s scenarios, there is a witness, but the agent isn't aware of what the witness is --- the agent doesn't have the appropriate form of self-awareness.
  \end{itemize}
\item Also note, the instructor case has the nice feature of informing the students that they can fill in a witness.
\end{itemize}


\subsection{Must}
\label{sec:must}

\begin{itemize}
\item A possible line of argument is that the agent's reasoning generates a ``must''-type statement.
\item For example:
  \begin{itemize}
  \item It must be the case the Woodthorpe is guilty.
  \item It must be the case that purchasing carambola is a means to some end on mine.
  \item It must be the case that Eric is due to visit soon.
  \item It must be the case that the weather will hit around 10 Celsius today.
  \end{itemize}
\item All of these seem quite natural, but I don't think the proposal works.
\item The problem is temptation cases.
\item The argument is that if the agent is interested in providing a witness, then these types of cases should work out well.
\item However, it seems as though the tipsy agent would think ``it must be the case that I go to the ATM to withdraw some funds'' as opposed to ``it must be the case that I ask for some water''.
\item In short, it seems as though `must' takes the agent's current reasoning as a parameter, and so won't straightforwardly apply to the agent's non-tipsy reasoning.
\item Same thing with the glasses of wine: ``I must resist a second glass'' doesn't seem appropriate, and it seems this is so because the agent has a way to reason to a second glass of wine.
\item The same thing does not happen in the case of `might' claims, however.
\item Following Egan's terminology of `epistemic reach', the intuition is that `must' is more restrictive about what is within an agent's reach.
\item So, while in the standard cases there may be some overlap, in the more interesting cases there is some divergence.
\end{itemize}

\subsection{Filters}
\label{sec:filters}

\begin{itemize}
\item From a certain point of view, the proposal is the inverse of \citeauthor{Bratman:1987aa}'s notion of a filter.
\item For, on \citeauthor{Bratman:1987aa}'s account an agent's present reasoning is constrained by the filter put in place by the agent's intentions.
\item While, on my view I have an ideal which allows an agent to form attitudes on reasoning that they haven't done.
\end{itemize}

\subsection{Attitudes}
\label{sec:attitudes}

\begin{itemize}
\item One argument is that it is only the agent's attitudes that matter.
\item A possible way of putting this is that the only thing that matters is what is supported by the agent's evidence in the purely epistemic cases.
\item However, this runs into the Winston scenario, where it seems as though Winston is not permitted to be confident that Eric Blair is due to visit.
\item This is a familiar problem.
\item \citeauthor{Lord:2018aa} deals with this, and broadly speaking it's an instance of the problem of believing any tautology.
\item So, it seems as though what the agent \emph{can} reason to is important.
\end{itemize}

\subsection{99\%}
\label{sec:99}

\begin{itemize}
\item Don't form attitudes on the basis of reasoning that one is not confident of.
\item Form attitudes on the basis of not reasoning that one is confident of.
\item There's not direct line of support here, though it does suggest that one's present reasoning is not `authoritative'.
\end{itemize}

\begin{itemize}
\item The argument in the 99\% case is a little tricky.
\item First, there is a difference between the agent being confident that their reasoning is problematic and being confident that they have a collection of rationally impermissible attitudes.
\item The 99\% case I'm interested in is only that the agent is 99\% confident that their reasoning is problematic.
\item So, it does not follow from this that the agent has a rationally impermissible collection of attitudes.
\item Therefore, if it is only the attitudes that matter, and if it is only the agent's actual reasoning that matters, then
\item The agent needs to demonstrate that the attitudes that they have reasoned to are rationally impermissible.
\item For, at present the agent is only confident that there may be a mistake.
\end{itemize}

\begin{itemize}
\item This still rests on intuition to some degree.
\item For, intuitively it seems that the issue is that they collection of attitudes that the agent has likely diverges from those that would be the case were the agent to have reasoned appropriately.
\item Hence, the agent is at least permitted to retract the result of their reasoning.
\item Hum \dots it's something of a mess.
\end{itemize}

\subsection{Commitment?}
\label{sec:commitment}

\begin{itemize}
\item Lewis is committed to confidence that Woodthorpe is guilty given Lewis' attitude toward the contents of the case file.
\item In the same way that an agent is committed to confidence that \(\phi \land \psi\) given their confidence that \(\phi\) and their confidence that \(\psi\).
\end{itemize}

\subsection{Looking at cases where the agent makes the inference}
\label{sec:looking-at-cases}

\begin{itemize}
\item This is the step of the argument that I need.
\item If this works, then there's a probability argument that the agent's confidence should be fairly high, as one intersects states where there's a witness for the reasoning and where all of \(\Sigma\) come out as true.
\item The question, then, is why we care about these cases in which the agent makes the inference, as we have established that in the actual world the agent has \emph{not} made the inference.
\item One part of the puzzle seems to be that the result does not depend on the agent's reasoning.
\item If the result does depend on the agent's reasoning, then \(\chi\) is certainly not the case in the actual world, because the agent has not made it the case that \(\chi\).
\item Hence, one could argue that the assumption of reasoning is informative about the current state of the world.
\end{itemize}

\begin{itemize}
\item The position I am arguing against claims that Lewis can only be confident that Woodthorpe is guilty on the basis of the case file if Lewis reasons from the case file to the guilt of Woodthorpe.
\item In other words, or on a specific basis, Lewis must correctly respond to the reasons that the case file presents to be confident on the basis of those reasons.
  \begin{itemize}
  \item Lewis would not be appropriately responding to reasons were Lewis to be confident without doing the relevant reasoning.
  \end{itemize}
\end{itemize}

\begin{itemize}
\item \(\chi\) is independent of the agent's reasoning.
\item Hence, \(\chi\) is or is not the case.
\item So, evaluating the worlds in which the agent does the reasoning is a guide to the way the world actually is.
\item Agent ends up with \(\chi = f\Sigma\) because this is just the agent's epistemic counterpart to the relation of propositional support that obtains in the actual world (more-or-less as the relation may be inaccessible).
\end{itemize}

\begin{itemize}
\item So, if the agent is required to reason, then this is a requirement against some kind of accuracy\dots
\end{itemize}

\subsection{Maybe argument}
\label{sec:maybe-argument}

\begin{enumerate}
\item Agent is confident that they can reason from \(\Sigma\) to \(\chi\).
  \begin{itemize}
  \item Understand as agent is confident that \(\chi = f\Sigma\) for some \(f\).
  \item Model this as there being states in which \(f\) has a reference, and then whether \(\chi = f\Sigma\) in those states.
  \item Problem is that all of these states are hypothetical.
  \item The agent has not reasoned in the actual state, and \(f\) does not have a reference.
  \item Note, however, this does not show that \(\chi \ne f\Sigma\), as \(f\) is not defined.
  \item However, as \(\chi\) and \(\Sigma\) are independent of the agent's current reasoning, all of these hypothetical states resemble the actual world.
  \item That is, the agent's reasoning does not affect whether \(\chi = f\Sigma\) when \(f\) is witnessed.
  \item The witness is only about the agent's reasoning.
  \end{itemize}
\item Were \(f\) to be given a witness, it would (likely) be the case that \(\chi = f\Sigma\).
\end{enumerate}

\begin{itemize}
\item One way of putting the goal is that I want \(f\) to be an available inferential path despite the agent not reasoning via \(f\).
\item In this respect, the agent's attitude toward \(\chi\) isn't super important.
\item Indeed, it's plausible that there are cases in which \(\chi = f\Sigma\) and so \(\lnot\Sigma\) on the basis of \(\lnot\chi\), and hence the agent uses their speculative reasoning to revise attitudes that they have.
\end{itemize}


\begin{itemize}
\item ``Bound by permissible reasoning.''
\item If reasoning is permissible, then structural constraint. (Note, the converse doesn't hold.)
  I do not claim that if structural constraint then permissible to reason.
  And, there may be arguments against this, except this is relations between attitudes, rather than new info and so on.
\item If no structural constraint, then reasoning is impermissible. (Contraposition of above.)
\item Permitted to reason, hence there is this structural constraint.
\item This is more or less where I am at with the argument.
\item Except, it's a little different, because modals are introduced.
\item Possible to reason, hence possible that there is a structural constraint.
\item Confident that can reason, hence confident that there is a structural constraint.
\item While possibility to reason may entail that there is a structural constraint, this does not follow with confidence.
\item Hence, in the scenarios I'm interested in, the agent can only be confident that there is a structural constraint.
\end{itemize}

\begin{itemize}
\item If confident that there is the structural constraint, then confident that violation is forbidden.
\item So, attitude toward \(\Sigma\) and \(\chi\) should align with the assumption that \(\chi = f\Sigma\)?
\item If these attitudes don't align, then the agent should be confident that they are irrational.
\item So, by akrasia, if the agent is confident that they can reason, then they should be confident that they are irrational to the degree that the constraint placed by \(f\) is violated.
\end{itemize}

\begin{itemize}
\item This says nothing about speculating on \(f\).
\item However, it shows that from the point of view of the agent's attitudes, the introduction of \(f\) does no harm, as the agent's attitudes are already constrained in this way.
\end{itemize}

\begin{itemize}
\item Objection is that the structural constraint assumes reasoning.
\item Therefore, the attitudes would only be constrained were the agent to reason.
\item 99\% case, permissible for agent to retract the result of their reasoning.
\item However agent has not demonstrated that this is a structural constraint.
\item The agent may indeed satisfy a structural constraint.
\item So, if retraction is okay, don't require reasoning.
\end{itemize}

\begin{itemize}
\item Perhaps I get to here and now the point is that the agent is constrained by \(f\), and it is a short step to assuming \(f\).
\end{itemize}

\begin{itemize}
\item Either reasoning determines structural constraints, or structural constraints are independent of reasoning.
\item Implausible that reasoning determines structural constraints.
\item I can't appeal to the cases of interest to support this, without begging the question.
\item 99\% offers a sufficiently different type of scenario to motivate the disjunction.
\item As does \(\phi, \psi\) then \(\phi \land \psi\).
\end{itemize}

\begin{itemize}
\item So, then paired with akrasia, get that agent is going to satisfy the structural constraints.
\item Hence, agent who speculates does no worse from the point of view of attitudes, assuming confidence holds.
\end{itemize}

\begin{itemize}
\item So, no argument for \(f\), but agent won't do any worse.
\item Is this enough for \(f\) to be permissible?
\item In certain aspects, \(f\) is disposable, but it allows a straightforward reading of the cases of interest.
\end{itemize}

\begin{itemize}
\item The argument relies only on their being structural requirements, it doesn't assume that these structural requirements are non-derivative, nor that these are independent of the specific \(\chi\)'s and \(\Sigma\)'s.
\end{itemize}


\hozlinedash

\begin{itemize}
\item In all of the cases I've been considering, the reasoning that the agent is confident that they are able to do can be seen as the recognition of a structural requirement that holds been the premises of the reasoning and the conclusion of the reasoning.
\item In part, this is because the agent is confident that they can reason from attitudes that they hold without any further information.
\item Winston is different in this respect, because it is not clear that Winston can reason the Eric Blair visiting given that Winston does not understand substitution of co-referential terms.
\item Idea is to argue that it plausibly follows from any account that commits agent's to satisfying structural commitments between attitudes that the agent should, to the extent that they are rational, have attitudes toward the premises and the conclusion that adhere to the structural requirement that this agent is confident obtains.
\item This argument is partial, as it does not show that the agent is permitted to assume that the structural requirement holds.
  \begin{itemize}
  \item The goal is to argue that if the agent's attitudes are required to fall in line with the structural requirement, then the agent is permitted to assume that the structural requirement holds.
  \item This doesn't quite beg the question.
    \begin{itemize}
    \item For, the premise of the argument is that the agent is confident that holding an attitude toward the premises without holding an attitude toward the conclusion is not rationally permitted.
    \item But this is tricky.
    \item Rather, it seems as though this is a disjunctive type of argument.
    \item Either confidence that structural requirement holds requires the agent to adopt attitudes.
    \item Or, there is an independent requirement and the agent ends up satisfying the structural requirement anyway.
      \begin{itemize}
      \item Yet, one may argue that the combination of attitudes for, e.g., Morse only holds because Morse has done the reasoning.
        Hence, on the latter independent requirement Lewis may not be required to satisfy the relevant structural constraint.
      \end{itemize}
    \end{itemize}
  \end{itemize}
\end{itemize}

\begin{description}
\item[Akratic Principle] No situation rationally permits any overall state containing both combination of attitudes \(\{\phi_{1},\dots,\phi_{n}\}\) and the belief that the combination of attitudes \(\{\phi_{1},\dots,\phi_{n}\}\) is rationally forbidden in one's current situation.
\end{description}

\begin{itemize}
\item If the agent recognises they have the combination of attitudes \(\{\phi_{1},\dots,\phi_{n}\}\), and if the agent is confident that the combination of attitudes \(\{\phi_{1},\dots,\phi_{n}\}\) is rationally forbidden in the agent's situation, then the agent should be confident that they are in a state that is not rationally permitted.
\end{itemize}

\begin{itemize}
\item It seems plausible that there is an argument from the akratic principle to the agent's confidence that they would in an impermissible state were they to have the relevant combination of attitudes paired with their confidence that the combination is not rationally permitted.
\end{itemize}

\begin{itemize}
\item Hence, for the agent to not be confident that they are in a state that is not rationally permitted, the agent needs to avoid the combination of states \(\{\phi_{1},\dots,\phi_{n}\}\).
\item So, in the Lewis case, the agent is confident that the case file constitutes evidence, and hence should be confident that Woodthorpe is guilty.
\item In this respect, Lewis will end up satisfying the structural constraint that would be placed on the relation between the case file and the guilt of Woodthorpe were they to reason from the case file to the guilt of Woodthorpe (and remain confident that the case file constitutes evidence).
\end{itemize}

\begin{itemize}
\item This does not establish that Lewis is permitted to hold that there is a structural relation between the case file and the guilt of Woodthorpe, but this does establish that the attitudes that Lewis holds to the case file and the guilt of Woodthorpe are more-or-less in line with the structural relation that would be witnessed by Lewis reasoning from the case file to the guilt of Woodthorpe.
\item In other words, this doesn't show that the agent is permitted to assume that the structural relation obtains, but this does not permit the agent to violate the structural constraint to a degree in line with the agent's confidence that the constraint holds.
\end{itemize}

\begin{itemize}
\item Potential argument from 99\% case is that the absence of a structural constraint permits the agent to revise their attitudes.
\item For, the agent is 99\% confident that their reasoning was impaired, and hence that they did not follow structural constraints.
\item This does not show that the agent has a collection of attitudes that are not rationally permitted, for there is a chance that the agent did reason appropriately.
\item However, the agent is confident that they satisfied structural constraints prior to reasoning, and therefore the agent is more confident than less that they would be in a better position were they to roll back the results of the reasoning that they are unsure about.
\item This argument differs from the one I am giving because it concerns confidence of the absence of structural constraints, rather than confidence that structural constraints obtain.
\end{itemize}

\begin{description}
\item[The JK Rule] One has justification to assert and to act on P if and only if one has justification to believe that one is in a position to know that P
\end{description}

\hozlinedash

\begin{itemize}
\item Confident that structural relation obtains.
\item Hence, confident that violating is not rationally permitted.
\item Deny that agent is able to appeal to structural relation without reasoning.
\item But this denies that the agent can be confident about what they can reason to.
\item In order to block confidence, would need to argue that agent is restricted to what they have reasoned to.
\item Implausible.
\item With confidence, things are a mess.
\item Hence, permissible.
\end{itemize}

\begin{itemize}
\item Agent is confident that they can reason from \(\Sigma\) to \(\chi\).
\item Hence, agent is confident that there is a structural relation between \(\Sigma\) and \(\chi\).
\item Assume that it is not permissible for the agent to speculate the structural relation between \(\Sigma\) and \(\chi\).
\item The agent is confident that a structural relation obtains, but it not permitted to speculate the structural relation.
\item Either confidence that a structural relation obtains restricts the agent's attitudes or it does not.
\item If structural relation does not restrict, then it is rationally permissible for the agent to be confident that they violate structural constraints.
  \begin{itemize}
  \item It follows that, it is rationally permissible for the agent to be confident that they have a collection of attitudes that are not rationally permissible.
  \item The stronger claim that the agent is not permitted to revise does not follow from this.
    Only have that the agent is permitted to not revise, despite confidence of that collection of states is not rationally permitted.
  \item Moore sentences: ``\(\phi\) and \(\psi\), but I am confident that \(\{\phi,\psi\}\) is not permitted''.
  \item Perhaps these are odd because we assume that the agent has done the reasoning.
  \item Akratic principle, surely.
  \item More carefully stated, it is rationally permissible for the agent to be confident that they have a collection of attitudes that are not rationally permissible according to the principles of reasoning that the agent is committed to.
  \item The agent is confident that any reasoning that they are committed to would require them to revise the relevant collection of attitudes.
    \begin{itemize}
    \item This, in part, is why the agent is committed to there being a structural relation.
    \item This is also a better way to put things, as it does not depend on the reasoning that the agent is now able to do, e.g.\ it avoids problems where the agent is confident that any reasoning that they `can' do would be unreliable.
    \end{itemize}
  \end{itemize}
\end{itemize}



\begin{itemize}
\item One way out of this problem is to note that if the agent reasons by their own lights, then so long as the agent reasons from those attitudes, the agent will revise the collection of attitudes appropriately.
  \begin{itemize}
  \item So, the agent is not allowed to revise without reasoning, but any reasoning that the agent does will lead to revision.
  \item The agent may not be able to do the reasoning, and these are the primary cases of interest.
  \item Trouble with 99\% cases \dots
  \item Also need to add that the agent is not allowed to assume the result of any future reasoning that the agent will do.
    \begin{itemize}
    \item This conflicts with part of the story about reflection.
    \item Part of the story is new evidence, but the other part of the story is that the agent will perform a certain kind of reasoning.
    \item Suspect that one who objects to speculation will also object to reflection, so this is a little bit of a dead end.
    \item Way out of the preface paradox, as the agent hasn't demonstrated a false claim.
    \end{itemize}
  \end{itemize}
\end{itemize}

\begin{itemize}
\item Credit is a potentially useful example here.
\item For, it is similar to speculation, in that one uses resources that one does not have.
\item But, it is an analogy, and it doesn't resolve anything.
\item Indeed, it is plausible to think that there are arguments against the permissibility of credit along these lines.
\end{itemize}

\newpage

\begin{itemize}
\item An important step is:
\item Confidence that \(\phi\) \(\rightarrow\) Confidence that assumption of \(\phi\) is not mistaken.
\item For the agent in the scenarios I'm considering is confident that some structural relation holds, and the issue is whether the agent is permitted to speculate on the basis of the structural relation, which is more-or-less assuming that the structural relation holds, and hence the agent's confidence that the structural relation holds is confidence that they have not made a mistake about the structural relation between their attitudes.
\end{itemize}

\begin{itemize}
\item Priority of committed structural relations over those witnessed by reasoning.
\item This is another way to look at things.
\item In the temptation cases, it seems as though, from the agent's point of view, that there are structural relations which support the action that is tempting.
\item The agent has reasoned to temptation, for sure.
\item The difference is that the agent is also confident that there are other structural relations in play, those work against temptation.
\item If reasoned relations win out, then temptation.
\item If they do not, then agent can resist temptation.
\end{itemize}


\newpage

\begin{itemize}
\item Agent is confident that structural principle constrains their attitudes, but is not permitted to appeal to the structural principle because they have not themselves provided a witness for the structural principle.
\item So, on the basis of Morse's message, Lewis' attitude toward the guilt of Woodthorpe is constrained by Lewis' confidence that they case file constitutes evidence, but Lewis is not permitted to hold that Woodthorpe is guilty on the basis of the case file constituting evidence because Lewis has not themselves demonstrated Woodthorpe's guilt on the basis of the case file.
\item Only permitted to appeal to a structural relation if one demonstrates the structural relation obtains.
\item Lewis is only permitted to appeal to the structural relation if them demonstrate the relation through reasoning.
\item So, Lewis' attitudes are constrained by a structural relation they cannot appeal to.
\item Well, here it is plausible that Lewis appeals to the combination of both Lewis' and Morse's states.
\item Yet, even if this is so, because Lewis is confident that both Morse and Lewis are alike, Lewis should be confident that Morse's state is simply a developed version of Lewis' state.
\item It is in this sense that Lewis is confident that if they were to reason then they would demonstrate the guilt of Woodthorpe.
\end{itemize}

\begin{itemize}
\item The logic case is particularly puzzling.
\item For, it seems as though the instructor denies that the student is permitted to be confident of the conclusion based on the students confidence of the basic rules of logic.
\item If this is the case, then it seems as though, if there are cases in which and agent is permitted to appeal to a structural relation, then the permissibility is pragmatically grounded.
\item However, I'm not sure this is the case.
\item There are a few things going on, however.
\item On the one hand, there are cases of complex proofs, along the lines of Fermat's Last Theorem, or whatever, in which the demonstration is outside my grasp, even if I am committed to all of the premises and rules that are used.
\item Here, then, the claim is that the student shouldn't be confident that they can do the reasoning.
\end{itemize}

\newpage

\begin{itemize}
\item Argument to conform to structural consequences.
\item However, plausible that there's only pressure if these are recognised.
\item So, in some cases the witness is provided by the reasoning of someone else.
\item For example, in cases of complex mathematical proofs.
\item Here, it seems that Godel's proof constrains my attitudes toward arithmetic and completeness, but the witness for my confidence that there is a structural relation is Godel's reasoning, or some reconstruction of Godel's reasoning.
\item Difference between being able to witness the structural relation, and being confident that there is a witness, roughly.
\item If there is a witness, then I cannot provide a counter-witness, and hence my attitudes are constrained because I'm confident that I am committed to the relevant structure that the demonstration makes use of.
\item This is what goes wrong in the Winston case.
\item In the Lawrence case, agent is confident that they themselves can provide a witness.
\end{itemize}

\begin{itemize}
\item \emph{Agentive modals}!
\item The idea is that the relevant understanding of `reason' is an ability to demonstrate a structural constraint.
\item That Morse reasoned from the case file to the guilt of Woodthorpe is relevant only in so far as Morse has the ability to demonstrate the guilt of Woodthorpe on the basis of the case file.
\item And, from Lewis' standpoint Lewis has the same abilities, hence Lewis is also able to demonstrate the guilt of Woodthorpe.
\item If this is not the case, then we're not interested in the agentive reading of `reason', but for sure it seems that we are.
\end{itemize}

\begin{itemize}
\item This is an argument for why Lewis is confident, and hence Lewis' confidence no longer needs to be built into the scenario.
\item Morse's report is about an ability rather than a specific demonstration, in short, and as Lewis is confident that they share the relevant abilities, Lewis is also able to demonstrate the guilt of Woodthorpe.
\end{itemize}

\begin{itemize}
\item I move or less want to argue that speculation is built into the understanding of reasoning report.
\item In temptation, so long as I am confident that I can reason, then things work out.
\item So, in this respect, speculation can fall into the background, though it helps to motivate the puzzle.
\end{itemize}

\begin{itemize}
\item Reason is a potential/agentive verb.
\item If potential, then structural constraints.
\item Akratic principle.
\item Hence, confidence requires alignment.
\end{itemize}

\begin{itemize}
\item This does not rule out alignment in general, and an agent may be claimed to be irrational by failing to have the appropriate attitudes.
\item I take no stance on this issue.
\item Of interest is the weak claim that the agent aligns when the recognise that there is a structural relation, and this holds when the agent has the potential to reason, because the potential to reason entails that the relevant structural constraints obtain and are recognisable.
\end{itemize}

\begin{itemize}
\item Arguing that the agent has not demonstrated the ability is difficult, because:
  \begin{enumerate}
  \item Demonstration only secures a weak reading of the verb, and hence does not establish what is required.
  \item Conflict with other instances of agentive verbs.
  \end{enumerate}
\end{itemize}

\begin{itemize}
\item \cite{Mandelkern:2017aa} appeal to an agent `trying'
\item \cite{Schwarz:2020aa} appeals to relevant `volitional states'
\item I don't see the reason for this.
\item The agent's volition is of contingent interest.
\item It is plausible that for many verbs, the agent's volition affects whether the completion of the verb is guaranteed, but this does not mean that it is built into the analysis.
\item Still, this is tricky, because I want to claim that in cases of reasoning volition isn't super important, in order to claim that in some cases an agent does have a rational failing despite not being willing to try to reason.
\item For example, suppose Kant is right, but a dictator is never willing to reason via the hypothetical imperative.
\item Would like to say that there's a problem, but if trying or volition is built in, then it is not true that the dictator can reason, because they will never voluntarily attempt the reasoning, they will never try.
\item Hence, if volition of some kind is built in, the scope of reasoning is severely restricted.
\item More locally, if the agent is under the influence, they will not try to reason, one may suppose.
\item If we are allowed to consider cases where the agent is able to do the reasoning and their volition doesn't matter, then things are sort of okay.
\item (Also, to strengthen the idea is the connexion \citeauthor{Schwarz:2020aa} makes between these modals and ought claims.)
\item So, stronger argument if the agent is required to think about what they could do independently of their volition, or whatever it is that `trying' implies, if anything.
\item One option is to deny that there is any clear upshot, and that the use of `try' and `volition' is an artefact of the propositional analysis that the authors give, whereas on an eventive analysis one can simply fix the relevant event to be of the right kind (possibly relevant to description, but this really seems to be an additional parameter, rather than being built into a particular reading of the modal).
\item A stronger option would be to show that there really is a problem with building such constraints into the modal.
\end{itemize}

\begin{itemize}
\item Super important! The idea that reasoning is reading as a potential verb means that a single instance of reasoning doesn't establish much.
\item So, this offers some progress on temptation cases, as in these it's not clear that the agent has reasoned in the appropriate sense.
\end{itemize}

\begin{itemize}
\item Idea is that for ability modals, the relevant ranking of that of adversity.
\item Intuitively, an agent needs to demonstrate they can witness the event even given adverse conditions.
\item Hence, the actual world may not present adverse conditions, but then as one weakens the verb, there are less plausible adverse conditions.
\item So, throwing a dart and hitting a dot is a hard thing to do, but throwing a dart is not.
\item One has to work a lot harder to demonstrate that one cannot throw a dart, hence fewer worlds are excluded, and hence worlds in which the dart is thrown and the dot is hit are possible\dots
\end{itemize}

\newpage

\begin{itemize}
\item Ability modals.
\item Relevant understanding of reasoning is in terms of ability.
\end{itemize}

\begin{itemize}
\item Reason is a verb, describing an activity that an agent does.
\item The activity moves from premises to conclusion.
\item Different ways this can be understood.
\item Weak reading, on which an agent participated in some activity.
\item Strong reading, on which the agent demonstrates that the conclusion follows from the premises.
\item The difference is whether there is a way to show that the conclusion does not follow from the premises.
\item In short, whether there are any relevant defeaters.
\item To illustrate, I reason that it would be good to get a cup of coffee.
\item However, were I to have remembered that I need to get up early tomorrow morning, I would not have concluded in favour of coffee.
\item I reasoned, but I did thereby demonstrate that it would be good to get a cup of coffee because there was a relevant defeater that I failed to consider.
\item So, I reasoned in the weak sense, but not in the strong sense.
\item By contrast, I carefully examine all of the boxes of cereal, and settle on Cornflakes.
\item I'm confident that I've worked through all of the relevant information.
\item Intuitively, the key is my ability to demonstrate (to myself, at least) that the conclusion follows from the premises.
\end{itemize}

\begin{itemize}
\item Consider ability modal.
\item It seems as though any instance of the event described by the modal witnesses the truth of the ability.
\item This is because, intuitively, this is an existential modal.
\item However, things are more nuanced.
\item I ran 5k, hence demonstrating that I am able to run 5k.
\item However, the course was downhill, and I had a strong supporting wind.
\item So, I demonstrated my ability to run 5k, in favourable conditions.
\item The point is that the straightforward reading of an event is highly permissive.
\item Often, however, we are interested in something slightly stronger.
\item Hitting the bullseye by luck demonstrates, in a sense, that the agent is able to hit the bullseye, but there is also a sense in which this does not demonstrate the agent's ability.
\item A clearer case is a first order logic proof.
\item It's fairly straightforward to piece together rules, and the quantifiers contain some subtleties.
\item Hence, agent pieces together a proof, and happened to get the subtleties right, but these did not enter their reasoning.
\item Here, some pressure to revise the description.
\item Paired with this is the worry of committing the no true scotsman fallacy.
\item Need to add something, hence I'm changing the rules.
\item Yet, there's no general linguistic mechanism to avoid this.
\item Restricting witnesses is hard.
\item But if witnesses aren't restricted, things are too easy.
\item But this is standard behaviour for modals.
\item Need to fix the relevant ordering source, and this is something that's done by the participants of the conversation.
\item My difficulty is not necessarily that there is no relevant line, but that communicating the line is a difficult task.
\item Modal makes it possible to restrict, restriction is not part of what is said.
\end{itemize}

\begin{itemize}
\item To emphasise here is the tricky relationship between ability requiring an existential, but the ordering source acting almost as a universal quantifier by eliminating possibilities.
\item And this is because there may be multiple options, more or less.
\item Intuitively, in the case of reason, there may only be a unique demonstration, but this is something about reason and not the modality in general.
\item And, there may be really difficult situations that are ruled out.
\item So, in a sense an error theory can be avoided by noting that the only cases of particular interest are those in which the demonstration is sound, for example.
\end{itemize}

\begin{itemize}
\item So, the goal here is to provide the framework.
\item The specifics of what are required for an agent to reason in the appropriate sense can be left under-specified.
\end{itemize}

\begin{itemize}
\item Adversarial orderings aren't so uncommon to be ruled out.
\item This, in short, is the response to \citeauthor{Schwarz:2020aa}, who thinks that the actual world cannot be ruled out of any ordering.
\end{itemize}

\newpage

\begin{itemize}
\item Interested in ability modals.
\item Success does not entail ability.
\item Failure does not entail inability.
\end{itemize}

\begin{enumerate}
\item The agent threw the dart and hit the centre of the board.
\item The agent threw the dart and did not hit the centre of the board.
\end{enumerate}

\begin{itemize}
\item The latter non-entailment is less controversial than the former.
\item A bird few across the agent's line of sight as the were making the throw.
\end{itemize}

\begin{enumerate}
\item The former non-entailment is more controversial.
\item The agent is bad at throwing objects and got lucky.
\end{enumerate}


\begin{itemize}
\item \cite{Boylan:2020aa} has a nice argument with disjunctions.
  \begin{itemize}
  \item Dart player who throws \(n\) darts which lands at \(n\) different points on the board.
  \item Does not entail that the agent is able to hit any of the \(n\) points.
  \end{itemize}
\item Problem with the former entailment is the tendency to redescribe.
\item \citeauthor{Schwarz:2020aa} makes use of descriptions in this way.
\item Hitting the centre of the board was outside of the control of the agent.
\item Still, this wasn't included in the statement that the agent can hit the board.
  \begin{enumerate}
  \item The agent can reliably hit the centre of the board.
  \end{enumerate}
\item \citeauthor{Schwarz:2020aa} distinguishes strong and weak readings.
\item However, possibility of redescription does not show that success entails ability.
\item The redescription works in both ways.
  \begin{enumerate}
  \item The agent can throw the dart, and it is then possible that the dart hits the centre of the board.
  \end{enumerate}
\item Both seems plausible, so if there is a distinction between strong and weak readings, it seems both may have a claim to the default reading of can.
\item Still, paraphrasing does not show ambiguity.
\item Counterfactuals.
\item Ordering is sensitive to surrounding context.
\end{itemize}

\begin{itemize}
\item The basic proposal is that the modal takes the event as an argument.
\item This is somewhat supported by Champolion's view of events.
\item Further, the event describes all the ways in which the event can happen.
\item For example, \(\exists e(\text{throw}(e))\) denotes all throwing events.
\item Hence, can see the modal as restricting to certain instances of the relevant event.
\item Argument is that these are adversarial instances, and the question is whether the agent can overcome the relevant adversity.
\item Trouble is that one can't pack all of the relevant adversities into a single event.
\item Hence, it seems that the modal cannot simply be an existential modal.
\item This then pushes toward a universal reading, rather than an existential reading.
\item This can be illustrated with the safe case.
\item As an adversary, we only get to choose a single combination.
\item Hence, we may choose the combination that the agent guesses.
\item Yet, this wouldn't demonstrate the agent's ability to open the safe.
\item It seems the agent needs to be able to open the safe given an arbitrary code.
\end{itemize}

\begin{itemize}
\item \citeauthor{Boylan:2020aa}'s argument for \emph{Can't-entails-won't} is primarily the strangeness of:
  \begin{enumerate}
  \item \# I can't hit the bullseye on this shot, but I might.
  \end{enumerate}
\item However, this seems weak.
\item For, it's unclear that this is given an agentive reading.
\item The following is also bad, but certainly not agentive:
  \begin{enumerate}
  \item \# This gas can't solidify above 10\(^{{\circ}}\)C, but it might.
  \end{enumerate}
\item Things are a little less clear with an alternative verb designed to highlight an agentive reading:
  \begin{enumerate}
  \item I do not have the ability to hit the bullseye on this shot, but I might.
  \end{enumerate}
\item The issue is that there is no clear reading of the agentive modal.
\item For, the antecedent is an agentive modal, and the consequent looks at a completion.
\item If the relevant adversaries are limited to what actually occurred when the agent performed the action, then assuming that the agent was successful ensures that the agent had the ability.
\item And, searching for a successful performance is the role of `might' in this context.
\item However, as this is a conditional there's no requirement that the agentive modal is read in this way, and hence it seems plausible to consider adversarial complications.
\item Consider also the contrapositive:
  \begin{enumerate}
  \item I might not hit the shot, but I have the ability to.
  \end{enumerate}
\item The same effects are assumed to hold, in the converse direction, `might not' searches for a defeating instance, and this voids the ability.
\item However, neither success nor failure are sufficient to demonstrate ability.
\item So, there's something of a problem regardless.
\item For, it should be the case that both:
  \begin{enumerate}
  \item I don't have the ability to hit the shot, and
  \item I might hit the shot.
  \end{enumerate}
  to both be true.
\item It seems that the problem is that a weak reading of the agentive modal is available.
\item One could try to force a strong reading:
  \begin{enumerate}
  \item If I am at the top of my game, I have the ability to hit the shot.
  \item Of course, I might not hit the shot, if I'm not at the top of my game.
  \item And, of course even if I am at the top of my game things don't always work out.
  \end{enumerate}
\item Even better:
  \begin{enumerate}
  \item I have the ability to hit the shot, but you never know\dots
  \item I don't have the ability to hit the shot, but fortune favours the brave\dots
  \end{enumerate}
\item So, I don't think the data is sufficiently clear here.
\end{itemize}

\newpage

\begin{itemize}
\item Agent is confident of ability to reason.
\item Hence, agent is confident that structural relation holds between attitudes.
\item Only need ability for this to be the case.
\item Ability grounds speculation.
  \begin{itemize}
  \item If ability grounds speculation then it doesn't matter whether the agent does the reasoning or not.
  \item For, there is no general entailment between instances and ability.
  \item In other words, challenging the agent on the basis of not having performed an action is looking at the wrong thing.
  \item One needs to challenge the agent's ability to perform the action.
  \end{itemize}
\item Note, Morse's communication is about Morse's ability, not that Morse has demonstrated.
\item Can imagine Morse making the arrest, the difference is that Morse has demonstrated their ability, but it is Morse's ability to demonstrate that does the work at the time of the arrest.
\item Morse `rests' on their ability.
\item So, the key part of the argument is that ability is relevant.
\end{itemize}

\begin{itemize}
\item Lewis has not demonstrated that Woodthorpe is guilty on the basis of the case file.
\item However, it does not follow from this that Lewis has not demonstrated their \emph{ability} to demonstrate the guilt of Woodthorpe on the basis of the case file.
\item So, it does not follow from the lack of prior demonstration that Lewis does not have the ability to demonstrate the guilt of Woodthorpe.
\end{itemize}

\begin{itemize}
\item Even if Lewis has the ability to demonstrate the guilt of Woodthorpe, it is still the case that Lewis has not put their ability to use by demonstrating the guilt of Woodthorpe.
\item Further, at the moment of the arrest Lewis is unable to demonstrate the guilt of Woodthorpe.
\item However, the same would may were Morse to have been at the corner shop.
\item The demonstration of Woodthorpe's guilt is not necessarily simple.
\item Woodthorpe's guilt may rest of a number of subtle principles or complex statutes.
\end{itemize}

\begin{itemize}
\item Difference in attitudes.
\item Morse's attitude toward the guilt of Woodthorpe was arrived at by reasoning.
\item Lewis' attitude toward the guilt of Woodthorpe relies on Morse' reasoning.
\item This does not show that Lewis' attitude toward the guilt of Woodthorpe does not \emph{depend} on Lewis' attitudes toward the contents of the case file.
\end{itemize}

\begin{itemize}
\item This is a common idea with respect to proofs.
\item What matters is the guarantee that the informal proof can be transformed into a formal proof.
\item I am not committed to this claim, however.
\item There may be no relevant difference between informal and formal proofs.
\item However, this is an option.
\end{itemize}

\begin{itemize}
\item Of course, it would be \emph{ideal} for Lewis to have demonstrated the guilt of Woodthorpe, just as it would be ideal for Lewis to do so at the moment of arrest.
\end{itemize}

\begin{itemize}
\item It remains important that the case file supports Woodthorpe's guilt independently of Lewis' reasoning.
\item For example, that a driver has the ability to win a race by driving a particular does not (in general) permit the driver to assume that they have won the race.
\item Yet, this is because winning the race depends on the driver driving the car.
\item The race is not won independently of the act being performed.
\item Now, it may be possible to argue that Woodthorpe's guilt does depend on reasoning in some way, but unfortunately I do not see why this would be the case.
\item The reasoning is a demonstration, and this does depend, but the guilt is a separate matter.
\item Lewis cannot be confident that they have demonstrated Woodthorpe's guilt.
  Yet, issue is not about whether Lewis has demonstrated Woodthorpe's guilt.
  Rather, the issue is about whether Lewis has the ability to demonstrate Woodthorpe's guilt.
\end{itemize}

\begin{itemize}
\item Interesting to note that `has the ability' seems to have some idealisation built into it.
\item This seems to be because there's no actuality entailment.
\item Perhaps this is something to consider in additional detail.
\end{itemize}

\begin{itemize}
\item In the Lewis scenario, there are clear(ish) requirements on when Lewis is required to demonstrate the ability.
\item In the Lawrence case, it does not seem as though Lawrence needs to demonstrate the ability.
\item In the Winston case, it does not seem as though Winston has the ability.
\item In the shopping case, it does not seem as though the agent needs at that moment to demonstrate the ability.
\item In the temptation case, it does not seem as though the agent needs at that moment to demonstrate the ability.
\item Hence, abilities allow for asynchronicity. (Slogan)

\end{itemize}

\newpage

\begin{itemize}
\item Agentive modals.
  \begin{enumerate}
  \item The agent can hit the dartboard with a dart.
  \item The agent is able to open the safe.
  \item The agent can reason from the premises to the conclusion.
  \end{enumerate}
\item Weak and strong readings:
  \begin{itemize}
  \item Weak: Successful performance of the action demonstrates ability.
    \begin{itemize}
    \item Agent throws the dart and it lands on the dartboard, hence the agent is able to hit the dartboard with a dart.
    \end{itemize}
  \item Strong: Successful performance of the action does not necessarily demonstrate ability.
    \begin{itemize}
    \item On the \(n\)th throw, the agent hits the dartboard.
      The agent hitting the dartboard by luck does not demonstrate ability.
    \end{itemize}
  \end{itemize}
\item Interested in strong readings of `is able to reason'.
\item There are no straightforward entailments between `is able to reason' in the strong sense and specific instances of reasoning.
  \begin{itemize}
  \item The agent may reason from premises to conclusion, but in doing so fails to notice a number of different ways in which a premise could be understood, and so does not demonstrate their ability to reason from the premises to conclusion.
    \begin{itemize}
    \item Successful reasoning does not entail ability to reason.
    \end{itemize}
  \item Likewise, agent may fail to reason from premises to conclusion because they accidentally overlook an important premises, but this does not demonstrate that the agent does not have the ability to reason from the premises to the conclusion.
        \begin{itemize}
    \item Failure to reason does not entail inability to reason.
    \end{itemize}
  \end{itemize}
\item Same for other agentive modals.
  \begin{itemize}
  \item Agent throws a dart and at the same time the crowd watching the football game behind cheers for a goal.
    Does not entail that the agent does not have the ability to hit the dartboard.
  \item As above, the agent hitting the board by luck does not entail that the agent has the ability to hit the dartboard.
  \end{itemize}
\end{itemize}

\begin{itemize}
\item Not too much work on the semantics of agentive modals.
  \begin{itemize}
  \item Determinate truth.
    \begin{itemize}
    \item Unclear why this shouldn't be understood by under of \(\Box\)- and \(\Diamond\)-type modalities.
    \item I.e.\ it's true or not that the modal is determinate/indeterminate.
    \end{itemize}
    \begin{itemize}
    \item \textcite{Mandelkern:2017aa}
      \[
        \sem{\text{S can }\phi}^{c,w} = 1 \text{ iff } \exists A \in \mathcal{A}_{S,c,w} \colon \sem{\phi(S)}^{c,f_{c}(\text{S tries to }A,w)} = 1
      \]
      \begin{quote}
        there is some practically available action \(A\) such that the closest world where \(S\) tries to do \(A\) is a world where \(S\) does \(\sem{\phi}^{c}\); in other words, just in case there is some practically available action such that if \(S\) tries to do it, she does \(\sem{\phi}^{c}\).
      \end{quote}
    \item \textcite{Boylan:2020aa} expands on \textcite{Mandelkern:2017aa} with a trivalent semantics for future contingents.
    \end{itemize}
  \item Performing an action under a particular description.
    \begin{itemize}
    \item \textcite{Schwarz:2020aa}
      \begin{quote}
        \begin{enumerate}[label=\alph*.]
        \item \(S\) can \(\phi\) (effectively) iff there are (highest-ranked) accessible worlds at which \(S\) \(\phi\)s.
        \item \(S\) can \(\phi\) (transparently) iff there are (highest-ranked) accessible worlds at which \(S\) \(\phi\)s transparently.
        \end{enumerate}
\(S\) \(\phi\)s \emph{transparently} iff \(S\) \(\phi\)s as a result of a volitional state that warrants believing that she will \(\phi\) provided that \(\phi\)ing is under her volitional control.
      \end{quote}
    \end{itemize}
  \item Generic situations.
    \begin{itemize}
    \item \textcite{Bhatt:2008aa}
    \item \textcite{Mandelkern:2017aa}
    \end{itemize}
  \end{itemize}
\end{itemize}

\begin{itemize}
\item I'm not sure about appealing to a distinction between determinate and indeterminate truth.
\item Nor am I sure about appealing to the agent performing an act under a particular description.
\item I'm very unsure about appealing to `trying' or `volitional states'.
  \begin{itemize}
  \item It seems as though in the case of reasoning, the ability for an agent to recognise that their attitudes are incoherent is sufficient for some kind of irrationality, whether or not the agent would try or have the volition to demonstrate the incoherence.
  \item For example, it seems plausible that there is something up with an agent who is able to demonstrate that their credences are probabilistically incoherent.
    Similarly, for an agent who intends an end without intending a necessary means.
  \end{itemize}
\end{itemize}

\begin{itemize}
\item Basic intuition I have is that agentive/ability modals are evaluated through an `adversarial ordering'.
\item In other words, an agent has an ability (in a strong sense) so long as they are able to counteract complications introduced in their performance of an action.
\item This rules out situations that make performance of the action `too easy', and likewise rules out situations that make the performance of the action `too difficult'.
\item Big-O analysis for average case complexity in computer science is an example.
  Imagine a demon who is able to manipulate the data prior to running of an algorithm as it pleases, but is unable to inspect the algorithm in order to exploit the specific ways in which the algorithm works.
  (I.e.\ prevent the demon from guaranteeing the worst case on every run of the algorithm.)
\end{itemize}

\begin{itemize}
\item Slightly more detailed idea is that the action verb denotes all of the events in which the verb is true.
\item Ability modal restricts the events of interest to sufficiently adversarial events.
\item An agent has an ability if there is a successful completion of the verb in all of the relevant adversarial events.
\end{itemize}


\begin{itemize}
\item Two things to argue for, roughly:
  \begin{enumerate}
  \item Agent has the ability.
  \item Agent is permitted to perform the relevant action on the basis of what they are able to demonstrate without demonstrating the ability.
    \begin{itemize}
    \item Demonstrating the ability \(\ne\) demonstrating that the agent has the ability.
    \end{itemize}
  \end{enumerate}
\end{itemize}

\begin{itemize}
\item In the Morse and Lewis example, Morse does not merely communicate that they did some reasoning and concluded that Woodthorpe is guilty.
\item Rather, Morse communicates that they are able to demonstrate Woodthorpe's guilt.
\item Given that Lewis and Morse are equally matches, Lewis concludes that Lewis is (also) able to demonstrate Woodthorpe's guilt.
\item Lewis does not need to demonstrate Woodthorpe's guilt at the moment of the arrest, just as Morse does not need to demonstrate Woodthorpe's guilt at the time of the arrest.
\item What matters is that both Morse and Lewis have the ability to demonstrate the guilt of Woodthorpe if/when a demonstration is required.
\item Similarly for Lawrence, who would be able to demonstrate that \(10\) Celsius = \(40\) Fahrenheit.
\item Different from Winston, who is not able to demonstrate that \(Pa \land (a = b) \Rightarrow  Pb\).
\item Unclear that an agent's failure to reason to abstention after a glass of wine is a counterexample to the agent's ability to reason to abstention (e.g.\ as the agent is tipsy, their ability to reason is impaired, and hence the actual world is `too adversarial').
\end{itemize}

\newpage


The basic steps involved in the kind of cases I've been thinking about are:

\begin{enumerate}
\item \(\Sigma\)
\item \(\exists f(f\Sigma = \chi)\)
\item \(\chi\)
\end{enumerate}

So, intuitively, the agent is confident that \(\Sigma\) is the case, is confident that there is a way to reason from \(\Sigma\) to \(\chi\), and comes to be confident that \(\chi\) is the case.

The question is whether \(\exists f(f\Sigma = \chi)\) is part of the basis of the agents confidence.
In other words, whether the agent is confident that \(\chi\) holds on the basis of \(\Sigma\) holding, or whether the agent must also cite their confidence that there is a way to reason from \(\Sigma\) to \(\chi\).

In the Lewis scenario, the issue is whether Lewis arrests Woodthorpe on the basis of the evidence, or whether Morse's message is also part of the evidence, roughly.

To take a standard piece of reasoning to illustrate.
I am confident that \(\phi, \phi \rightarrow \psi\), and on the basis of this confident that \(\psi\).
Here, my own reasoning is not part of my evidence for being confident that \(\psi\).
My own reasoning explains why I am confident that \(\psi\), for I would not have been confident that \(\psi\) without reasoning.
However, the reason for which I am confident that \(\psi\) is that \(\phi\) and that \(\phi \rightarrow \psi\).

This isn't fully satisfactory, though, as something about \(\phi, \phi \rightarrow \psi \vDash \psi\) is also important.
So, it is possible to distinguish between the use and mention of a rule.
Here, a mention of the rule used should plausibly be added.

\begin{enumerate}
\item \(\Sigma\)
\item \(\exists f(f\Sigma = \chi)\)
  \begin{itemize}
  \item \(\Sigma \vDash \chi\)
  \end{itemize}
\item \(\chi\)
\end{enumerate}

The key question is whether the second premise is used or mentioned.
For the agent to `use' the premise, it would need to be the case that the agent does the reasoning.
However, it's possible for the agent's reasoning to be `speculative'.

And, in order for this pattern to occur, it seems that the agent needs to be confident that they're able to do the reasoning.

\begin{enumerate}
\item \(\Sigma\)
\item \(\exists f(f\Sigma = \chi)\)
\item \(\future{f}\)
\item \(\chi = \future{f}\Sigma\)
\end{enumerate}

Compared to:

\begin{enumerate}
\item \(\Sigma\)
\item \(\exists f(f\Sigma = \chi)\)
\item \(\chi = g(\{\Sigma, \exists f(f\Sigma = \chi)\})\)
\end{enumerate}

In short, the agent needs to be confident that they're able to do the reasoning for the future to be available.
Of course, perhaps it is better to say that this is a sufficient condition, as that's also going to be true.
On the other hand, if the agent is not confident that \(\exists f(f\Sigma = \chi)\), then it does not seem as though there's a way to obtain \(\chi\) from \(\Sigma\), because it is possible that \(\Sigma \land \lnot\chi\) is the case.
For this reason, I doubt the argument can be made without the agent's confidence.
However, the structure of the argument is merely that speculating is permissible, and as such I do not need to argue that if speculating is permissible then confidence is required.

If premised on the agent's ability, then no `new' information is required.
This is somewhat tricky to specify, but the idea is that one only needs to remove `contingent' defeaters.

The basic argument is that there's little different between the two instances of reasoning.
Both involve appeal to reasoning that the agent does not do.
And, ah, right, it seems as though the agent would be appealing to Morse's reasoning in the \(g\) reasoning.

Either \(\exists f(f\Sigma = \chi)\) is a premise or supports speculation.
If it is a premise, then it is Morse's ability that secures the availability as a premise.
However, given that Morse and Lewis are equally matched, this is also supported by Lewis' reasoning.
Hence, there's a way to detach, so now Morse is part of the explanation of why this is available, but it is no longer the case that this is supporting the availability of the existential as a premise.

\newpage

\begin{itemize}
\item As things are, there has been no demonstration of the guilt of Woodthorpe.
\item Morse, indifferent to the evidential status of the case file, has reasoned that if the case file constitutes evidence then Woodthorpe is guilty.
\item While Lewis has established that the case file constitutes evidence.
\item So, if one were to fuse Morse and Lewis, one would have an agent able to demonstrate the guilt of Woodthorpe.
\item For, the Lewis-part would demonstrate that the case file constitutes evidence, and the Morse-part would demonstrate that Woodthorpe is guilt on the basis of the case file.
\item Of course, this fusion may happen.
  \begin{itemize}
  \item Lewis may demonstrate to Morse the evidential support for the case file.
  \item Morse may demonstrate to Lewis the reasoning that establishes the guilt of Woodthorpe.
  \end{itemize}
\item The situation is similar to the fusion of two proofs.
  \begin{itemize}
  \item One agent \(A\) demonstrated that \(\phi\) entails \(\psi\) and agent \(B\) has demonstrated that \(\psi\) entails \(\chi\).
  \item Neither agent has demonstrated that \(\phi\) entails \(\chi\).
  \item It seems both can be confident that \(\phi\) entails \(\chi\).
  \end{itemize}
\end{itemize}

\newpage

\begin{note}
  \begin{itemize}
  \item Changed `hold \(\chi\) on the basis of \(\Sigma\)' to `judge that \(\chi\) holds on the basis of \(\Sigma\)'.
    \begin{itemize}
    \item Main argument can focus on `can' and `ought' as applied to events.
    \end{itemize}
  \item Added argument to link the ought of events to the ought of states.
  \end{itemize}
\end{note}

\section{Argument sketch}
\label{sec:argument-sketch}

Goal is to show:
\begin{enumerate}[label=G\arabic*., ref=G\arabic*]
\item\label{goal} If an agent can reason from \(\Sigma\) to \(\chi\) then it is permissible for the agent to hold \(\chi\) on the basis of \(\Sigma\).
\end{enumerate}
Rewriting:
\begin{enumerate}
\item[\ref{goal}\('\).] If an agent can reason from \(\Sigma\) to \(\chi\) then it is not the case that the agent ought not to hold \(\chi\) on the basis of \(\Sigma\).
\end{enumerate}

What follows is an outline of a fairly straightforward argument, with a handful of assumptions that seem to depend on how `can' is understood.
\begin{itemize}
\item Understanding the semantics of `can' may not be necessary for the argument, though defending a particular interpretation is.
\end{itemize}

\begin{itemize}
\item \ref{goal} is primarily about whether an agent ought not do something; how \(\chi\) on the basis of \(\Sigma\).
\item In this respect, the interest of \ref{goal} is in cases in which an agent can reason from \(\Sigma\) to \(\chi\) and holds \(\chi\) on the basis of \(\Sigma\).
\item There are cases in which an agent can reason from premises to a conclusion without having already reasoned from those premises to the conclusion, and hence the antecedent of \ref{goal} will apply to those situations.
\end{itemize}

\begin{itemize}
\item Also assuming that \(\chi\) is independent of the agent's reasoning.
\item In short, if \(\Sigma\) holds then \(\chi\) holds regardless of whether the agent reasons from \(\Sigma\) to \(\chi\).
\end{itemize}


\subsection{The agent's point of view}
\label{sec:point-view}

\ref{goal} is stated in terms independent of an agent's point of view.
While, in the cases I am interested in, it is an agent's confidence that they are able to reason from \(\Sigma\) to \(\chi\) which leads the agent to be confident that \(\chi\) holds on the basis of \(\Sigma\).
Still, hope is it that:

\begin{enumerate}
\item[\ref{goal}\(''\).] If an agent (is confident that they) can reason from \(\Sigma\) to \(\chi\) then it is permissible for the agent to (be confident that) hold(s) \(\chi\) on the basis of \(\Sigma\).
\end{enumerate}

can be shown to follow from \ref{goal}.

\begin{itemize}
\item In the background is the assumption that \ref{goal} holds assuming that the agent has not reasoned from \(\Sigma\) to \(\chi\).
\item The following sketch does not assume that the agent reasons from \(\Sigma\) to \(\chi\).
  Still, it's not obvious that \ref{goal} is monotonic with respect to this assumption.
  For example, it does not seem to follow that:
  \begin{itemize}
  \item If an agent \emph{is confident that a premise of \(\Sigma\) is mistaken}, then [if the agent can reason from \(\Sigma\) to \(\chi\), then it is permissible for the agent to hold \(\chi\) on the basis of \(\Sigma\)].
  \end{itemize}
\item So, there is still some work to do in order to obtain:
  \begin{itemize}
  \item If an agent \emph{has not reasoned from \(\Sigma\) to \(\chi\)}, then [if the agent can reason from \(\Sigma\) to \(\chi\), then it is permissible for the agent to hold \(\chi\) on the basis of \(\Sigma\)].
  \end{itemize}
\end{itemize}

\hozlinedash

\newpage

\subsection{Simplifying the goal}
\label{sec:obtaining-G1}

\begin{itemize}
\item \ref{goal} relates an can of an event and the ought of a state.
  To make this explicit, one may rewrite \ref{goal} as follows:
  \begin{enumerate}[label=G\arabic*., ref=G\arabic*]
  \item[\ref{goal}\(_{v}\).] If an agent can witness the event of reasoning from \(\Sigma\) to \(\chi\) then it is not the case that the agent ought not to be in a state of holding \(\chi\) on the basis of \(\Sigma\).
  \end{enumerate}
\item The argument for \ref{goal} focuses on the relationship between can and ought.
  This relationship is complex, and the complexity is not reduced by dealing with events and states.
\item Therefore, show that \ref{goal} follows from a similar conditional which deals only with events, given an independently plausible principle linking events and states and a minor assumption about the relationship between reasoning to a conclusion from some premises and holding the conclusion on the basis of those premises.
\end{itemize}

Assume a minimal conditional relating the ought of a state and the ought of an event.
If a state is forbidden, then the agent is forbidden from performing any action that would yield that state.

\begin{enumerate}[label=S\arabic*., ref=S\arabic*]
\item\label{event-state-relation} If an agent ought not come to be in a state \(S\), then the agent ought not perform any action that would yield state \(S\).
\end{enumerate}

For example, if an agent ought not be in a restricted area, then the agent ought not perform any action that would result in the agent being in the restricted area.

\begin{enumerate}[label=S\arabic*., ref=S\arabic*, resume]
\item\label{event-state-relation-c} If it is not the case that an agent ought not perform all actions that would yield state \(S\), then it is not the case that the agent ought not come to be in state \(S\).
\end{enumerate}

Paraphrasing: If it is permissible for an agent to perform an action that would yield some state \(S\), then the agent is permitted to come to be in state \(S\).

Finally, \ref{goal} is formulated in terms of events:

\begin{enumerate}[label=G\arabic*\(_{e}\)., ref=G\arabic*\(_{e}\)]
\item\label{goal-event} If an agent can witness the event of reasoning from \(\Sigma\) to \(\chi\) then it is not the case that the agent ought not to witness the event of judging that \(\chi\) holds on the basis of \(\Sigma\).
\end{enumerate}

The argument from \ref{event-state-relation} and \ref{goal}\(_{e}\) is as follows:

\begin{enumerate}
\item\label{event-state-p1} An agent can witness the event of reasoning from \(\Sigma\) to \(\chi\).
\item\label{event-state-p2} It is not the case that the agent ought not to witness the event of judging that \(\chi\) holds on the basis of \(\Sigma\). \hfill (\ref{event-state-p1} \& \ref{goal-event})
\item Judging that \(\chi\) holds on the basis of \(\Sigma\) results in the agent coming to be in the state of holding \(\chi\) on the basis of \(\Sigma\). \hfill (assumption)
\item\label{event-state-p3} It is not the case that an agent ought not to perform all actions that would yield the state of holding \(\chi\) on the basis of \(\Sigma\). \hfill (\ref{event-state-p2}, \ref{event-state-p3}, logic)
\item It is not the case that the agent ought not come to be in state of holding \(\chi\) on the basis of \(\Sigma\). \hfill (\ref{event-state-relation-c})
\end{enumerate}

Therefore, as \ref{goal} follows from \ref{event-state-relation} and \ref{goal-event}, we can work with \ref{goal-event} and restrict attention to the can and ought of events.

\subsection{Ought and can}
\label{sec:ought-can}

Initial assumption is that ought implies can holds for arbitrary propositions.
Therefore, ought implies can applies to things not being the case.
And, in particular, ought implies can applies to an agent not holding a conclusion on the basis of premises.

\begin{itemize}
\item If an agent ought \(\phi\) then the agent can \(\phi\).
\item If an agent ought \(\lnot\phi\) then the agent can \(\lnot\phi\).
\end{itemize}

Substituting `hold \(\chi\) on the basis of \(\Sigma\)' for \(\phi\), the first premise of the argument is:

\begin{enumerate}[label=P\arabic*., ref=P\arabic*]
\item\label{oic:instance} If an agent ought not judge that \(\chi\) holds on the basis of \(\Sigma\) then the agent can not judge that holds \(\chi\) on the basis of \(\Sigma\).
\end{enumerate}

\begin{itemize}
\item \ref{oic:instance} may be obtained independently from a general ought implies can principle.
  Still, I take the ability to obtain \ref{oic:instance} from a general ought implies can principle to be sufficient motivation for the time being.
\end{itemize}

\begin{note}
  It is easy to read `can not' as `cannot', and if does this raises some problems. The consequent states that there is something that the agent is able to not do. However, if the agent `cannot hold \(\chi\) on the basis of \(\Sigma\)' then intuitively there is something that the agent is not able to do. To illustrate the difference, consider:
  \begin{itemize}
  \item I can not walk up the stairs. \hfill \(C \lnot \phi\)
    \begin{itemize}
    \item (Because I can take the elevator.)
    \end{itemize}
  \item I cannot walk up the stairs. \hfill  \(\lnot C \phi\)
    \begin{itemize}
    \item (Because I am recovering from a broken leg.)
    \item It is not the case that I can walk up the stairs.
    \end{itemize}
  \end{itemize}
  In general, it is not the case that \(C \lnot\phi \rightarrow \lnot C\phi\).
\end{note}

Strategy is to argue by contradiction, so assume agent can reason from \(\Sigma\) to \(\chi\) and ought not judge that \(\chi\) holds on the basis of \(\Sigma\).

\begin{enumerate}[label=P\arabic*., ref=P\arabic*, resume]
\item\label{p:can} Agent can reason from \(\Sigma\) to \(\chi\).
\item\label{p:oughtnot} Agent ought not judge that \(\chi\) holds on the basis of \(\Sigma\).
\end{enumerate}

Given the second assumption and the relevant instance of the ought implies can principle, obtain:

\begin{enumerate}[label=P\arabic*., ref=P\arabic*, resume]
\item\label{p:can-not} Agent can not judge that \(\chi\) holds on the basis of \(\Sigma\). \hfill (\ref{oic:instance} \& \ref{p:oughtnot})
\end{enumerate}

The issue is whether an agent can not judge that \(\chi\) holds on the basis of \(\Sigma\) when an agent is able to reason from \(\Sigma\) to \(\chi\).
(Whether the \ref{p:can} and \ref{p:can-not} entail a contradiction.)

\hozlinedash

% \newpage

Idea is that the following conditional holds:

\begin{enumerate}[label=H\arabic*., ref=H\arabic*]
\item\label{reason-hold} If an agent can reason from \(\Sigma\) to \(\chi\) then it is not the case that the agent can not judge that \(\chi\) holds on the basis of \(\Sigma\).
\end{enumerate}

\begin{itemize}
\item \ref{reason-hold} is somewhat hard to parse.
\item \ref{reason-hold-event-ability} is slightly easier.
\item Stick with \ref{reason-hold} for now.
\end{itemize}

\begin{enumerate}[label=H\arabic*\('\)., ref=H\arabic*\('\)]
\item\label{reason-hold-event-ability} If an agent can reason from \(\Sigma\) to \(\chi\) then it is not the case that the agent has the ability to not judge that \(\chi\) holds on the basis of \(\Sigma\).
\end{enumerate}

\begin{itemize}
\item Intuitively, the agent does not have the ability to conclude their reasoning without establishing \(\chi\) given \(\Sigma\).
\item For, if the agent does conclude their reasoning without establishing \(\chi\) given \(\Sigma\) while the agent is able to reason from \(\Sigma\) to \(\chi\), then it is unclear how the agent can reason to \(\chi\).
\end{itemize}

% Contraposing \ref{reason-hold}:

% \begin{enumerate}
% \item[\ref{reason-hold}\('\).] If an agent can not hold \(\chi\) on the basis of \(\Sigma\) then it is not the case that the agent can reason from \(\Sigma\) to \(\chi\).
% \end{enumerate}

To clarify the structure:

\begin{enumerate}
\item[\ref{reason-hold}\(_{c}\).] If an agent can [reason from \(\Sigma\) to \(\chi\)] then it is not the case that the agent can not [judge that \(\chi\) holds on the basis of \(\Sigma\)].
% \item[\ref{reason-hold}\('_{c}\).] If an agent can not [hold \(\chi\) on the basis of \(\Sigma\)] then it is not the case that the agent can [reason from \(\Sigma\) to \(\chi\)].
\end{enumerate}

\hozlinedash

In support:

\begin{enumerate}[label=R\arabic*., ref=R\arabic*.]
\item\label{r:1} If an agent can [reason from \(\Sigma\) to \(\chi\)], then the agent is not able to exercise their ability to reason and fail to establish that \(\chi\) holds on the basis of \(\Sigma\).
\end{enumerate}

And:

\begin{enumerate}[label=R\arabic*., ref=R\arabic*., resume]
\item\label{r:2} If an agent can not [judge that \(\chi\) holds on the basis of \(\Sigma\)], then the agent is able to exercise their ability to reason and fail to establish that \(\chi\) holds on the basis of \(\Sigma\).
\end{enumerate}

\begin{itemize}
\item In \ref{r:1} and \ref{r:2}, the agent's `exercise of their ability to reason' is understood relative to premises \(\Sigma\) and the issue of whether \(\chi\).
For, the agent may exercise their ability to reason from a different set of premises to a different conclusion (or fail to do so).
\item The intuition for these two conditionals is that an agent cannot simultaneously recognise that a conclusion follows from a collection of premises and resist drawing the conclusion on the basis of the premises.
\item For an example, consider an agent who holds \(\Sigma = \{a,b\}\) and is able to reason by conjunction introduction.
  Intuitively, the agent is not `able to avoid' holding \(a \land b\) on the basis of \(\{a,b\}\).
  For, intuitively still, if conjunction introduction holds there is no way to find a flaw in moving from \(\{a,b\}\) to \(a \land b\).
\end{itemize}

\begin{itemize}
\item \ref{reason-hold} is a kind of non-voluntarism about certain mental states/dependencies.
\item For, if \ref{reason-hold} is true, then there is are mental states/dependencies that an agent is not able to reject.
\end{itemize}

% \newpage

Given \ref{r:1} and \ref{r:2} it is then straightforward to establish the desired conditional:

\begin{enumerate}
\item[\ref{r:1}] If an agent can [reason from \(\Sigma\) to \(\chi\)], then the agent is not able to exercise their ability to reason and fail to establish that \(\chi\) holds on the basis of \(\Sigma\).
\item[\ref{r:2}] If an agent can not [judge that \(\chi\) holds on the basis of \(\Sigma\)], then the agent is able to exercise their ability to reason and fail to establish that \(\chi\) holds on the basis of \(\Sigma\).
  \begin{itemize}
  \item {
      \color{red}
      Note, this is compatible with the agent having insufficient premises to establish that \(\chi\) follows from \(\Sigma\).
      In other words, it does not require the agent to demonstrate that \(\Sigma \nvDash \chi\).
      In the sense that failing to show that \(\Sigma \vDash \chi\) leaves open the possibility that \(\Sigma \vDash \chi\).
      And, this should be the case, as the agent may not be able to fully work through all the consequences.
    }
  \item The idea behind \ref{r:2} is that
    \[
      \text{Agent can not judge that} \Sigma \vDash \chi \leftrightarrow \text{Agent can judge} \Sigma \nvDash \chi
    \]
    The idea is that the agent is performing some action by not judging, and the ability to do this is the ability to judge that \(\Sigma \nvDash \chi\).
    The issue is, that the ought statement is then requiring the agent to perform an action.
    And if this is the case, then it's not clear that I get the permissibility of an action by denying this.
    Instead, it is impermissible for the agent to judge \(\Sigma \vDash \chi\) \dots, but this would require
    \[
      \text{Not permissible for the agent to judge } \Sigma \vDash \chi \rightarrow \text{Agent can not judge } \Sigma \vDash \chi
    \]
    Which entails
    \[
      \text{Not permissible for the agent to judge } \Sigma \vDash \chi \rightarrow \text{Agent can judge } \Sigma \nvDash \chi
    \]
    This is a fairly strong requirement on impermissibility due to the biconditional.
    For, not permissible to \(\phi\) implying can \(\lnot\phi\) is okay.
    E.g.\ if it is not permissible for me to take from the tip jar, then I can do some action which does not take from the tip jar.
    So, read this way the first conditional is true just in case there is some action available to the agent which is not judging that \(\Sigma \vDash \chi\).
    However, the other way to read the condition is that, if it is not permissible for me to take from the tip jar, then I can refuse to take from the tip jar.
    So, I can intervene so that I do not take from the tip jar.
    That is, if some taking from the tip jar is in progress, I can intervene to stop this action.
    And this is the sense of can that I'm interest in.
    So, it's certainly a different sense of can.
    Roughly, if an agent can not do something, then the agent can intervene on any event leading to that thing.
    \begin{itemize}
    \item If it is not permissible for the agent to \(\phi\), then the agent has the ability to interrupt \(\phi\)ing.
    \item If the agent does not have the ability to interrupt \(\phi\)ing, then it is permissible for the agent to \(\phi\).
    \end{itemize}
    The question here is really about permissibility, and how this is understood, I think.
    For, the notion of can here is fairly intuitive.
    Still:
    \begin{itemize}
    \item If it is not permissible for me to take part in the heist then I am able to interrupt taking part in the heist.
      \begin{itemize}
      \item Here, I might have no choice, so I am not able to interrupt.
      \item Yet, it's not clear that it is permissible for me to take part in the heist.
      \item But this is tricky, as if there's no way to interrupt, then it's not clear why this is really impermissible.
      \item For, there may be permissible actions that are not what the agent ought do.
        \begin{itemize}
        \item E.g.\ It would be best to \(\phi\), but also permissible to \(\psi\), though \(\phi > \psi\).
        \end{itemize}
      \item 
      \end{itemize}
    \end{itemize}
  \end{itemize}
\item[\ref{p:can}.] Agent can [reason from \(\Sigma\) to \(\chi\)].
  \begin{enumerate}[ref=\alph*.]
  \item[\ref{p:can-not}.] Agent can not [judge that \(\chi\) holds on the basis of \(\Sigma\)].
  \item\label{c1} Agent is not able to exercise their ability to reason and fail to establish that \(\chi\) holds on the basis of \(\Sigma\). \hfill (\ref{p:can} \& \ref{r:1})
  \item\label{c2} Agent is able to exercise their ability to reason and fail to establish that \(\chi\) holds on the basis of \(\Sigma\). \hfill (\ref{p:can-not} \& \ref{r:2})
  \item\label{cbot} \(\bot\) (\ref{c1} \& \ref{c2})
  \end{enumerate}
\item[\(\lnot\)\ref{p:can-not}] It is not the case that agent can not [judge that \(\chi\) holds on the basis of \(\Sigma\)]. \hfill (\ref{p:can-not} -- \ref{cbot})
\end{enumerate}

The contradiction is that the agent is able to exercise their ability to \(\phi\) and the agent is not able to exercise their ability to \(\phi\) (\ref{c1} and \ref{c2}).
In short, if \(p \rightarrow r\) and \(q \rightarrow \lnot r\) then \(p \rightarrow \lnot q\).
Therefore:

\begin{itemize}
\item[\ref{reason-hold}\(_{c}\).] If an agent can [reason from \(\Sigma\) to \(\chi\)] then it is not the case that agent can not [judge that \(\chi\) holds on the basis of \(\Sigma\)].
\end{itemize}

\hozlinedash

\newpage

\subsection{Argument overview}
\label{sec:argument-overview}

Putting the main argument together:

\begin{enumerate}
\item[\ref{oic:instance}.] If an agent ought not [judge that \(\chi\) holds on the basis of \(\Sigma\)] then the agent can not [judge that \(\chi\) holds on the basis of \(\Sigma\)].
\item[\ref{p:can}.] Agent can reason from \(\Sigma\) to \(\chi\).
  \begin{enumerate}[ref=\alph*.]
  \item[\ref{p:oughtnot}.] Agent ought not [judge that \(\chi\) holds on the basis of \(\Sigma\)].
  \item[\ref{p:can-not}.] Agent can not [judge that \(\chi\) holds on the basis of \(\Sigma\)]. \hfill (\ref{oic:instance} \& \ref{p:oughtnot})
  \item[\ref{reason-hold}.] If an agent can [reason from \(\Sigma\) to \(\chi\)] then it is not the case that agent can not [judge that \(\chi\) holds on the basis of \(\Sigma\)].
  \item\label{dbot} It is not the case that the agent can not [judge that \(\chi\) holds on the basis of \(\Sigma\)].
  \mbox{ }\hfill (\ref{p:can-not} \& \ref{reason-hold})
  \item\label{dbott} \(\bot\) \hfill (\ref{p:can-not} \& \ref{dbot})
  \end{enumerate}
\item[\(\lnot\)\ref{p:oughtnot}.] It is not the case that the agent ought not hold \(\chi\) on the basis of \(\Sigma\).\linebreak
  \mbox{ }\hfill (\ref{p:oughtnot} -- \ref{dbott})
\end{enumerate}

Therefore:

\begin{enumerate}[label=G\arabic*., ref=G\arabic*]
\item[\ref{goal}\(_{e}\).] If an agent can witness the event of reasoning from \(\Sigma\) to \(\chi\) then it is not the case that the agent ought not to witness the event of judging that \(\chi\) holds on the basis of \(\Sigma\).
\end{enumerate}

And hence by the argument of section~\ref{sec:obtaining-G1}:

\begin{enumerate}[label=G\arabic*., ref=G\arabic*]
\item[\ref{goal}\(_{v}\).] If an agent can witness the event of reasoning from \(\Sigma\) to \(\chi\) then it is not the case that the agent ought not to be in a state of holding \(\chi\) on the basis of \(\Sigma\).
\end{enumerate}


\subsubsection{Messy}
\label{sec:messy}

\begin{itemize}
\item A real difficulty in the sketch given is that there are a whole bunch of ways to read `can'.
\item On the one hand, there is the `can' of available options.
  This is a `can' that is insensitive to rationality.
  For, on this sense of `can', an agent `can' believe two contradictory propositions.
\item Interested in a sense of `can' that is sensitive to rationality, 
\item Hence, for simplicity replace `can' with `has the ability to' in the cases of interest.
\item Or perhaps `is able to'.
\item Maybe `has the option to'.
\end{itemize}

\begin{enumerate}
\item[\ref{oic:instance}.] If an agent ought not [judge that \(\chi\) holds on the basis of \(\Sigma\)] then the agent is able to not [judge that \(\chi\) holds on the basis of \(\Sigma\)].
\item[\ref{p:can}.] Agent is able to reason from \(\Sigma\) to \(\chi\).
  \begin{enumerate}[ref=\alph*.]
  \item[\ref{p:oughtnot}.] Agent ought not [judge that \(\chi\) holds on the basis of \(\Sigma\)].
  \item[\ref{p:can-not}.] Agent is able to not [judge that \(\chi\) holds on the basis of \(\Sigma\)]. \hfill (\ref{oic:instance} \& \ref{p:oughtnot})
  \item[\ref{reason-hold}.] If an agent is able to [reason from \(\Sigma\) to \(\chi\)] then it is not the case that agent is able to not [judge that \(\chi\) holds on the basis of \(\Sigma\)].
  \item\label{dbot} It is not the case that the agent is able to not [judge that \(\chi\) holds on the basis of \(\Sigma\)].
  \mbox{ }\hfill (\ref{p:can-not} \& \ref{reason-hold})
  \item\label{dbott} \(\bot\) \hfill (\ref{p:can-not} \& \ref{dbot})
  \end{enumerate}
\item[\(\lnot\)\ref{p:oughtnot}.] It is not the case that the agent ought not hold \(\chi\) on the basis of \(\Sigma\).\linebreak
  \mbox{ }\hfill (\ref{p:oughtnot} -- \ref{dbott})
\end{enumerate}

\begin{itemize}
\item An agent ought not [judge that \(\chi\) holds on the basis of \(\Sigma\)] in the actual situation only if the agent is able to not [judge that \(\chi\) holds on the basis of \(\Sigma\)] in the actual situation.
\item Agent can reason from \(\Sigma\) to \(\chi\) in the actual situation.
\item If agent has the ability to reason from \(\Sigma\) to \(\chi\) in the actual situation then it is not the case that the agent is able to not judge that \(\chi\) holds on the basis of \(\Sigma\) in the actual situation.
  \begin{itemize}
  \item Not something that I am able to do, because one can line up the relevant defeaters fairly easily, viz.\ the way I would reason from \(\Sigma\) to \(\chi\).
  \end{itemize}
\end{itemize}

\begin{itemize}
\item So, the semantics of can that I want to appeal to involves generating a defeating scenario.
\item If there is a reasonable defeating scenario, then one can claim that the agent does not have the ability.
\item In other words, the agent has the ability if there is a way out of any potential defeater.
\item So:
  \begin{enumerate}
  \item Can \(\phi\) iff not (restricted) possible to force \(\lnot\phi\).
  \item Can \(\lnot\phi\) iff not (restricted) possible to force \(\phi\).
  \end{enumerate}
  The point here is that if the agent can reason from \(\Sigma\) to \(\chi\) then it is (restricted) possible to force judging \(\phi\) on the basis of \(\Sigma\).
\item Negated:
  \begin{enumerate}
  \item Not can \(\phi\) iff (restricted) possible to force \(\lnot\phi\).
  \item Not can \(\lnot\phi\) iff (restricted) possible to force \(\phi\)
  \end{enumerate}
  This is why the notion of can is quite strong.
  Because, the forcing is a kind of existential.
\item It's not clear that \(C\phi \lor C\lnot\phi\) holds.
  For, it may be possible to force \(\phi\) and force \(\lnot\phi\).
  Consider the agent at the dartboard.
  \(\phi\) is the agent hitting the dartboard.
  Need to be able to force a miss and to force a hit.
  The dart is magnetised, and a person is standing behind the dartboard with a giant magnet.
  If the throw is off target, the person behind moves the magnet behind the dartboard.
  If the throw is on target, the person behind moves the magnet away from the dartboard.
  So, it is not true that the agent can hit the dartboard, and it is also not true that the agent can miss the dartboard.
\item Still, there's something unintuitive here.
  For, it's not possible to force that it rains on Monday, and it's not possible to fore that it doesn't rain on Monday.
  This seems like a category mistake.
  As, interested in the completion of events for which the agent is a participant.
\end{itemize}

\begin{itemize}
\item This isn't quite right, though.
\item For, the agent can reason from \(\Sigma\) to \(\chi\), which is true just is case it is not possible to force the agent to \dots
\end{itemize}


\begin{enumerate}
\item Ought \(\lnot\) (judge \(\Sigma \vDash \chi\)) \(\rightarrow\) \(\langle \text{Can} \rangle\) \(\lnot\) (judge \(\Sigma \vDash \chi\))
\item \([\text{Can}]\) (Reason \(\Sigma \vDash \chi\))
\item \(\langle \text{Can} \rangle\) \(\lnot\) ( judge \(\Sigma \vDash \chi\))
\item \([\text{Can}]\) (Reason \(\Sigma \vDash \chi\)) \(\rightarrow\) \([\text{Can}]\) (judge \(\Sigma \vDash \chi\))
\item \([\text{Can}]\) (Reason \(\Sigma \vDash \chi\)) \(\rightarrow\) \(\lnot\langle \text{Can} \rangle\lnot\) (judge \(\Sigma \vDash \chi\))
\item \(\lnot\langle \text{Can} \rangle\lnot\) (judge \(\Sigma \vDash \chi\))
\item \(\lnot\) Ought \(\lnot\) (judge \(\Sigma \vDash \chi\))
\end{enumerate}

\begin{enumerate}
\item \(\langle \text{Can} \rangle \phi\) iff not (restricted) possible to force \(\lnot\phi\)
\item \([\text{Can}] \phi\) iff (restricted) possible to force \(\phi\)
\end{enumerate}

\begin{itemize}
\item So, two distinct, but related, uses of can are appealed to here.
\item However, I'm not so sure that this is a problem.
\item The natural language argument is really difficult, because it's super hard to distinguish these uses.
\item Still, that's a regular natural language problem.
\item And, intuitively different things are meant by the relevant instance of can.
\item So, the difficulty is in showing the relation between the two uses.
\item Right, as for sure it needs to be the case that one is the dual of the other!
  \begin{itemize}
  \item This really is the central point of the argument, viewed in this way.
  \item Hence, why the semantics of can is particularly important here.
  \end{itemize}
\end{itemize}

\begin{itemize}
\item \([\text{Can}]\)'t (make) me go clubbing.
  \begin{itemize}
  \item Not possible to force me to go clubbing
  \end{itemize}
\item \(\langle \text{Can} \rangle\)'t come into this club.
  \begin{itemize}
  \item Possible to force that you aren't in this club.
  \end{itemize}
\end{itemize}

\begin{itemize}
\item Sam \([\text{Can}]\) win the race. (If they put in the effort.)
\item Sam \(\langle \text{Can} \rangle\) win the race. (If they get lucky.)
\end{itemize}

\begin{itemize}
\item Sam \([\text{Can}]\)'t
\item \item Sam \(\langle \text{Can} \rangle\)'t
\end{itemize}

Something like the following seems to be what I want.

\begin{itemize}
\item \([\text{Can}]\phi\) iff \(\exists e((e \rightarrow \phi) \land \forall d(e \rightarrow \lnot d))\)
  \begin{itemize}
  \item There's a way that makes \(\phi\) true such that all defeaters are blocked.
  \end{itemize}
\item \(\langle \text{Can} \rangle\phi\) iff \(\forall e((e \rightarrow \lnot\phi) \rightarrow \exists d(e \land d))\)
  \begin{itemize}
  \item For all ways that make \(\phi\) false, there's a defeater that blocks the event.
  \end{itemize}
\end{itemize}

Here, the duality works out, though it is complex.
The diamond-can is read something along the lines of ``there's nothing stopping the agent'' while the box-can is something like ``it is not possible to stop the agent''.

\begin{itemize}
\item \([\text{Can}]\phi\) iff \(\exists e\)
  \begin{enumerate}
  \item \(e\) has culminated in \(\phi\), or
  \item \(e\) is progressing to \(\phi\) and for all stages, if \(e + e'\) stops the culmination, then there's \(e''\) at that stage such that \(e''\) continues the progression.
  \end{enumerate}
\end{itemize}

\begin{itemize}
\item So, one picks where the defeater is, and this is avoided.
\item This is still a little too messy.
\item The basic idea is that this is the sense in which a seed can grow into a flower.
\end{itemize}


\begin{itemize}
\item \(\nvDash [\text{Can}]\) go under par
  \begin{itemize}
  \item Because there are some circumstances that I don't have an answer for.
  \item No path that guarantees I go under.
  \end{itemize}
\item \(\nvDash \langle\text{Can}\rangle\) go under par
  \begin{itemize}
  \item Because there's no way to ensure the circumstances that I don't have an answer for obtain.
  \item No path that forces me a go over.
  \item So in other words, there is a path, but it is sensitive to defeaters.
  \item For, if you can't guarantee \(\lnot\phi\) then it can't be that every path ends in \(\lnot\phi\), else one can pick any path one wants to obtain \(\lnot\phi\).
  \item Hence, the `can' is somewhat implicit in this formulation.
  \item Furthermore, it makes additional sense with negation, where nothing is required of the agent.
  \item It may be that case that most instance of not doing something are read with this sense of can.
  \end{itemize}
\end{itemize}

\begin{itemize}
\item The intuition here seems okay, but the details are difficult to figure out.
\item Still, it seems something.
\item For, there is something of a link to the progressive.
\item For, if an agent is Ving, then the agent can V.
  \begin{itemize}
  \item However, this seems to be the weak sense of can, in that there is a continuation.
  \item Because, one can say that agent is Ving, but if I \dots then the agent will be forced to stop.
  \end{itemize}
\item This might not really hold.
\item Szabo has the example of `enumerating the primes' in the progressive, which seems true, but there's no way for an agent to complete the event of enumerating the primes.
  \begin{itemize}
  \item Example of this use by Kleene.
  \item One may argue that Kleene isn't talking about agents, but it's unclear why an agent would be part of the semantics of the modal, and not an additional parameter which helps to restrict the modal.
  \end{itemize}
\item However, if one opts for a stage-by-stage analysis, then it may be true, in a sense, that the agent can enumerate the primes, so long as there's a way to generate the next prime.
\end{itemize}

\begin{itemize}
\item If the two sense of `can' don't require completion, then it may be possible to apply the same sense of completion to both, without interfering with the duality.
\item Basically, this suggests a way to keep the logical form simple.
\end{itemize}

\begin{itemize}
\item You cannot not go to the office
  \begin{enumerate}
  \item \(\lnot \langle \text{Can} \rangle \lnot\) go to the office
  \item \([\text{Can}]\) go to the office
  \end{enumerate}
\item There's some kind of obligation modal hidden here, intuitively.
\item Still, basic idea is that there is no permitted action that blocks going to the office.
\item Hence, there's always an action that progresses going to the office.
\item Other examples:
  \begin{itemize}
  \item Cannot not work.
  \item Cannot not see (\dots password or grid).
  \item I cannot not read this in Frasier's voice.
  \item Quote by Charlotte Brontë: “I write because I cannot NOT write.”
  \item I cannot not choose. (Sarte?)
  \item I cannot not vote.
  \end{itemize}
\item In most of these, it seems that \(\phi\)ing is always something that's available, and in some cases this gets strengthened with the idea that therefore \(\phi\)ing must be done.
  Perhaps with the idea being that if \(\phi\)ing could not be done, then \(\phi\)ing wouldn't be available in the relevant sense.
\end{itemize}

\begin{itemize}
\item Maybe:
  \begin{enumerate}
  \item \([\text{Can}]\phi\) iff \(\exists e(e \Vdash \phi \land \forall e'\exists e''(e + e' + e'' \vDash \phi))\)
  \end{enumerate}
  This is quite strong.
  For, it ensures that there is an event, for which \(\phi\) is in progress, no matter how this event proceeds, there's a way to continue such that \(\phi\) remains in progress.
  To make this false, one either needs to ensure that \(\phi\) doesn't get started, or to find some continuation that blocks \(\phi\) from progressing.
  The lack of restriction on \(e'\) is useful, insofar as this includes cases where the agent stops progressing with \(\phi\), so long as the course is corrected.
\item The problem is that the dual then requires that the agent can `break out' of any problematic event, which is a little strong also.
\item Though this is difficult.
\item For, if it only applies when the agent is in a bad spot, then it seems right that there must be something the agent is able to do that breaks them out of the bad spot.
  \begin{enumerate}
  \item \(\langle \text{Can} \rangle \phi\) iff \(\forall e(e \Vdash \lnot\phi \rightarrow \exists e'\forall e''(e + e' + e'' \nvDash \lnot\phi))\)
  \end{enumerate}
  The difficulty here is the quantifiers and so on.
\item And, part of the problem here is the relevant verb.
\item `Solve the maze' is problematic, because it has some completion, whereas `solving the maze' does not.
  So, perhaps the lack of completion makes things particularly tricky here.
\item Here, then, take the modal and claim that it holds throughout a culminating sequence.
\item With this idea:
  \begin{enumerate}
  \item \([\text{Can}]\phi\) iff \(\exists e[e \Vdash \phi \land \forall e'\exists e''(e + e' + e'' \vDash \phi)]\)
  \end{enumerate}
  Would then require that \(\phi\) holds throughout the sequence.
  Still a mess.
  The logical form isn't right.
  And, this is likely because FOL isn't going to play easy with the kind of idea here.
  For, the attempt is to express something like a recursive function, where the internals of the function are duals.
  And the intuition here seems okay.
  For, box-can if there's a sure way to keep going so that things work out.
  And diamond-can if there's no sure way to keep going so that things don't work out.
  Or, there's a \dots
\end{itemize}

\begin{itemize}
\item Potentially something like:
  \[
    \exists e \forall d(e \subseteq d \rightarrow \exists d' \subseteq d \colon e \subseteq d' \subseteq e' \vDash \phi).
  \]
  So, there's an event, such that for all extensions, there is some initial part of that extension where the agent can do something so that, well, this same predicate remains true, perhaps.
\end{itemize}

\newpage


\subsection{Notes}
\label{sec:notes-1}

\subsubsection{Buridan cases}
\label{sec:buridan-cases}

\begin{itemize}
\item If an agent can [reason from \(\Sigma\) to \(\chi\)] then it is not the case that the agent can not [hold \(\chi\) on the basis of \(\Sigma\)].
\end{itemize}

\begin{itemize}
\item If an agent can [reason from \(\Sigma\) to \emph{left}] then it is not the case that the agent can not [hold \emph{left} on the basis of \(\Sigma\)].
\item If an agent can [reason from \(\Sigma\) to \emph{right}] then it is not the case that the agent can not [hold \emph{right} on the basis of \(\Sigma\)].
\item It is not the case that the agent can not [hold \emph{left} on the basis of \(\Sigma\)].
\item It is not the case that the agent can not [hold \emph{right} on the basis of \(\Sigma\)].
\item It is not the case that the agent can:
  \begin{enumerate}
  \item not [hold \emph{left} on the basis of \(\Sigma\)], and
  \item not [hold \emph{right} on the basis of \(\Sigma\)]
  \end{enumerate}
\item So, the agent is not able to `reject' \emph{left} nor is the agent able to `reject' \emph{right}.
\item In short, because the agent is able to reason that the left bale of hay is a means to their end, and likewise that the right bale of hay is a means to their end, the agent is not able to hold that the left bale of hay is not a means to their end, and likewise for the right bale of hay.
\item The agent would be mistaken for thinking that the premises that they hold determine one of the two bales of hay, and also mistaken for ruling out either bale of hay.
\end{itemize}

\begin{itemize}
\item The supermarket case is of particular interest.
\item For, one may assume that what is written on the shopping list is the result of a choice, and there are other items that would satisfy the same end.
\item Therefore, without some further restrictions (e.g.\ an intention), it cannot be claimed that the agent ought to purchase the item on their shopping list over any other item that would be a means to the relevant end.
  \begin{itemize}
  \item Assuming, here, that, if it is permissible for an agent to hold that an action is a(n available) means to an end, then it is permissible for the agent to perform the action as a means to the end.
  \end{itemize}
\item If this is right, then it seems that for an agent to exclude an option in cases of means-end reasoning, the agent must be sure that the option is not a means to their end.
\item However, this does not restrict the agent from choosing a particular end from an incomplete collection of recognised ends.
\end{itemize}

\subsubsection{Implication}
\label{sec:implication}

\begin{itemize}
\item If an agent can [reason from \(\Sigma\) to \(\chi\)] then it is not the case that the agent can not [hold \(\chi\) on the basis of \(\Sigma\)].
\item If an agent can [reason from \(\Sigma\) to \(\chi\)] then it is not the case that the agent can not [reason from \(\Sigma\) to \(\chi\)].
\end{itemize}

The latter is a consequence of the former, and perhaps should be seen as the relevant conditional.
For, it seems plausible that:

\begin{itemize}
\item If it is not the case that the agent can not [reason from \(\Sigma\) to \(\chi\)], then it is not the case that the agent can not [hold \(\chi\) on the basis of \(\Sigma\)].
\end{itemize}

Contraposed:

\begin{itemize}
\item If it is the case that the agent can not [hold \(\chi\) on the basis of \(\Sigma\)], then it is the case that the agent can not [reason from \(\Sigma\) to \(\chi\)].
\end{itemize}

Where these conditionals are relativised to the issue of whether \(\chi\) follows from \(\Sigma\).
(For, there may be many other things that the agent can reason to.)

% \subsubsection{Normative `can'?}
% \label{sec:normative-can}

% \begin{itemize}
% \item Might seem as though understanding of what an agent ought to do is informing the understanding of what an agent can do.
%   \begin{itemize}
%   \item To illustrate, possible to think of ought and can as two separate modalities, and then a restricted sense of ought is obtained by restricting the general sense of ought to those cases in which ought implies can holds.
%   \item In other words, ought looks as the best worlds independently of constraints, and is reigned in by additional modals.
%   \end{itemize}
% \item I don't think this is correct.
% \item Not the case that an agent can not reason from \(\Sigma\) to \(\chi\), hence no ought statement.
% \end{itemize}

\subsection{Can not reason to}
\label{sec:can-not-reason}

\begin{itemize}
\item Idea here is that if there's a path to \(\chi\), then that's all that's needed.
\item Hence, if one performs some act and does not obtain \(\chi\), that act is not reasoning.
\end{itemize}

\begin{itemize}
\item In other words, `can' may be an existential type of operator, but because the issue is an entailment relation, there is only one kind of possibility.
\item Yet, this is a problem, especially when contrasted with the Buridan conditionals.
  For, it then seems as though if the agent can reason to the left, then it is not the case that the agent can reason to the right.
  Hence, either there's something other than reasoning going on, or I've got a problem.
\end{itemize}

\subsection{Demonstration}
\label{sec:demonstration}

\begin{itemize}
\item The consequence of \ref{goal} only states that something is permissible.
\item Still, worry that there are cases in which an agent is unable to defend their appeal, etc.\ but still holds \(\chi\) on the basis of \(\Sigma\).
\item This seems mistaken.
\item The core idea is that there's no way to demonstrate that the agent is mistaken, so in these cases there's no way to show that the agent has made a mistake in holding \(\chi\) on the basis of \(\Sigma\).
\item However, the broader point to note is that \ref{goal} only permits the agent to hold \(\chi\) on the basis of \(\Sigma\), and there are various way to block the application of the agent holding \(\chi\) on the basis of \(\Sigma\) to some action.
\item In the Morse and Lewis scenario, for example, if Lewis is unable to provide a demonstration in time, then Lewis may be unable to arrest Woodthorpe, though Lewis may still hold Woodthorpe to be guilty on the basis of the evidence.
\end{itemize}

\hozlinedash

\begin{itemize}
\item Misleading evidence is a more difficult case.
\item The worry is that an agent may be permitted to hold that \(\chi\) follows from \(\Sigma\) while their evidence strongly (and misleadingly) supports that \(\chi\) does not follow from \(\Sigma\).
\end{itemize}

\begin{itemize}
\item This turns on the way in which the evidence is misleading.
\item If the misleading evidence prevents the agent from establishing \(\chi\) on the basis of \(\Sigma\), then it's not clear that the agent is able to reason to \(\chi\) on the basis of \(\Sigma\).
\item On the other hand, if the misleading evidence makes reasoning from \(\Sigma\) to \(\chi\) particularly difficult, then things are less clear.
  \begin{itemize}
  \item To argue for this one needs a case in which:
    \begin{enumerate}
    \item \(\chi\) follows from \(\Sigma\).
    \item\label{mle:2} The agent is able to reason from \(\Sigma\) to \(\chi\).
    \item\label{mle:3} An agent has misleading evidence that \(\chi\) does not follow from \(\Sigma\).
    \item\label{mle:4} The agent ought to hold that \(\chi\) does not follow from \(\Sigma\).
    \end{enumerate}
  \item I doubt that \ref{mle:2} and \ref{mle:4} can both be true.
    For it \ref{mle:2} is true, then the evidence in \ref{mle:3} cannot be conclusive, else the agent would not be able to reason from \(\Sigma\) to \(\chi\).
    And, it seems that if \ref{mle:2} and \ref{mle:4} are both true, then the agent ought to ignore some sound piece of reasoning.
  \end{itemize}
  % In a sense, this is a case of epistemic temptation.
\end{itemize}

% \subsection{The state and futures}
% \label{sec:state}

% \begin{itemize}
% \item The state of the agent holding isn't a standard kind of mental state.
% \item This is where ideas about futures come in.
% \item For applications that don't need the demonstration, the result is held and can be used.
% \item For applications that do need the demonstration, the agent's `can' do the reasoning.
% \end{itemize}


\newpage

\subsection{`Can'}
\label{sec:can}

\begin{itemize}
\item The sense of `can' or `is able' that I am interested in is relatively general.
\item This is in contrast to the can of `ability and opportunity' that is often of interest, and may be false even when an agent has a general ability.
  \begin{itemize}
  \item For example, an agent may have the general ability to steal.
  \item However, on a deserted island the agent lacks the opportunity to steal, and in the presence of peers lacks the ability due to being unable to resist the group pressure to act in an `appropriate' way.
  \item So, while the agent, in some sense, has the general ability, witnessed, e.g.\ by taking apples from a private orchard, the sense of `can' will take into account the abilities and opportunities of an agent given some context.
  \end{itemize}
\item The can of `ability and opportunity' is problematic as I am interested in cases where the agent may not have the ability nor opportunity in the context of evaluation.
  \begin{itemize}
  \item For example, Lewis does not have the opportunity to examine the case file before bumping into Woodthorpe.
  \item And, without recalling the carambola and star fruit are the same object, the agent may not be able to reason from their desire for carambola to their desire for star fruit.
  \end{itemize}
\item However, the sense of `can' that I am interested in does not permit enhancing the agent's ability to reason, nor does it permit providing the agent with arbitrary information.
\item Furthermore, it is not clear that there is any robust relation between these two sense of `can'.
  \begin{itemize}
  \item For example, an agent may lack a general ability to do something, but context may grant them that ability.
  \item An agent may not have the general ability to win a tennis tournament, but circumstances are such that the matches are paired in a very favourable way.
    Hence, on practically any other draw the agent would stand no chance, but given the particular draw everyone who could beat the agent will be beaten by someone who the agent can beat.
    So it may be that the agent has the `ability and opportunity' to win a tournament, while the agent lacks the general ability to win a tournament.
  \item Similarly to the examples given above, there are cases in which an agent may have a general ability which is masked by the particular context.
    Another example here is an agent who is out drinking and is so fuzzy that they would fail to do some relatively straightforward multiplication, nor are they given the opportunity to multiply some numbers.
  \end{itemize}
\item So, if there's no robust relation between these two senses of `can', then it's somewhat unclear whether there's much to be gained from thinking about the sense that I'm not interested in.
\end{itemize}

\begin{itemize}
\item Still, the interest is in rejecting ought claims.
\item So the worry is going to be that the general sense of ought allows the agent to reject certain oughts that specific situations require.
\item E.g.\ ought to do specific task \(\rightarrow\) has ability to do specific task.
\item Consequent is false, but agent has ability in specific context.
\end{itemize}

It seems plausible that if there are two senses of can, but one sense of ought, then clashes will arise.

\begin{itemize}
\item Lack of general ability blocks application of ought in a `specific' context.
\item Lack of specific ability blocks application of ought in a `general' context.
\end{itemize}

\begin{enumerate}
\item Multiple senses of ought.
  \begin{itemize}
  \item But then the conflict with abilities will generate a conflict of oughts.
  \item Could this be resolved by an all things considered ought?
  \end{itemize}
\item Reject a certain understanding of abilities --- e.g.\ only focus on `ability and opportunity'
\item Instances are compatible
  \begin{itemize}
  \item There's no general ought implies can validity\dots
  \end{itemize}
\item Something about reasoning?
\end{enumerate}

\begin{itemize}
\item Well, one point of contention here is that different senses of `can' may deny different instances of ought.
\item However, thinking about holding particular attitudes on the basis of others.
\item And, it's not clear that it is ever going to be the case that an agent ought to hold an attitude on the basis of some other attitude, and for the agent to lack either the specific or general ability.
\item So, specific and general may deny that an ought statement holds, but this doesn't matter if there is no instance in which an ought statement holds based on a single instance.
\end{itemize}

This doesn't seem right, perhaps.

\begin{itemize}
\item Agent is given some drug that enhances their ability to reason.
\item For whatever reason is it very important that the agent determines whether or not to hold \(\chi\) on the basis of \(\Sigma\).
\item For some reason the agent ought to hold \(\chi\) on the basis of \(\Sigma\).
  \begin{itemize}
  \item This doesn't need to be too strange.
  \item For example, someone has demonstrated to the agent that \(\chi\) follows from \(\Sigma\).
  \end{itemize}
\item The agent does not have the general ability.
\item It is not the case that the agent ought to hold \(\chi\) on the basis of \(\Sigma\).
\end{itemize}

\begin{itemize}
\item It seems to me that the issue here is whether the agent can have the specific ability without having the general ability.
\item Note that in the above it can't be that the agent is given access to some restricted information, for I don't think this plays into the general ability.
\item And, see below for argument against building in information to specific instance.
\end{itemize}

\begin{itemize}
\item The other instance would be  more problematic, in the sense that the situations are certainly quite common.
\item Agent has been drugged and is for the moment quite bad at reasoning.
\item Hence, as the agent doesn't have the ability and the opportunity it doesn't follow that the agent ought to hold \(\chi\) on the basis of \(\Sigma\).
\item Yet, how does one support the idea that the agent \emph{ought} to hold \(\chi\) on the basis of \(\Sigma\) in this case?
\item Plausible that the agent is not denied this \dots
\end{itemize}

\begin{itemize}
\item Ugh, but the ability and opportunity is really difficult.
\item For, there are many cases where the agent is only going to have the ability in a specific instance.
\item So, agent holds \(\chi\) on the basis of \(\Sigma\), but \(\Sigma\) contains some sensitive information and so the agent only has one opportunity to reason from \(\Sigma\) to \(\chi\).
\item Hence, agent does the reason, etc.\ and holds \(\chi\) on the basis of \(\Sigma\), but no long has the ability nor the opportunity, so it seems it cannot be the case that the agent ought to hold \(\chi\) on the basis of \(\Sigma\).
\item Yet, if \dots something about belief and evidence \dots then it seems the agent ought to hold \(\chi\) on the basis of \(\Sigma\).
\end{itemize}

\subsection{Different senses of `can'}
\label{sec:different-senses-can}

{
  \color{red}
  Part of the problem here is the framing of general ability, which is \emph{not} what I am after.
  I am interested in whether the agent can do the reasoning given the information available to them, and hence the focus is only on the reasoning that the agent can do.
}

\begin{itemize}
\item Ought \(\phi\) implies can \(\phi\).
\item If an agent ought to judge that \(\phi\) holds on the basis of \(\Sigma\) then the agent can judge that \(\chi\) holds on the basis of \(\Sigma\).
\end{itemize}

\begin{itemize}
\item Given that `can' is ambiguous, it is possible that different ways of reading `can' will restrict what the agent ought to do in different ways.
\item For example:
  \begin{enumerate}
  \item\label{oic:strong} If an agent does not have the general ability to judge that \(\phi\) holds on the basis of \(\Sigma\), then it is not the case that the agent ought to judge that \(\phi\) holds on the basis of \(\Sigma\).
  \item\label{oic:weak} If an agent does not have the specific opportunity and ability to judge that \(\phi\) holds on the basis of \(\Sigma\), then it is not the case that the agent ought to judge that \(\phi\) holds on the basis of \(\Sigma\).
  \end{enumerate}
\end{itemize}

\begin{itemize}
\item The difficulty here is that I endorse a general ought implies can principle, and to the extent that \ref{oic:strong} seems plausible, \ref{oic:weak} likewise seems plausible.
\item For, in the case of \ref{oic:weak} it does seems as though if an agent ought to judge that \(\chi\) follows from \(\Sigma\) only if the agent has the opportunity and ability to reason from \(\Sigma\) to \(\chi\) given the context in which the ought statement is taken to hold.
\end{itemize}

\begin{itemize}
\item For an issue to arise:
  \begin{enumerate}
  \item The same sense of ought applies to both \ref{oic:strong} and \ref{oic:weak}.
    \begin{itemize}
    \item For, if there are different senses of ought in \ref{oic:strong} and \ref{oic:weak}, respectively, then it seems there would need to be some more general sense of ought to adjudicate between the different respective senses of ought.
    \end{itemize}
  \item It is the case that the agent ought to judge that \(\chi\) follows from \(\Sigma\).
  \item And, either:
    \begin{enumerate}
    \item\label{oic:strong-a} The agent does not have the general ability to judge that \(\phi\) holds on the basis of \(\Sigma\).
    \item\label{oic:weak-a} The agent does not have the specific opportunity and ability to judge that \(\phi\) holds on the basis of \(\Sigma\).
    \end{enumerate}
    For either of \ref{oic:strong-a} and \ref{oic:weak-a} is sufficient to deny that it ought to be the case that the agent ought to judge that \(\chi\) follows from \(\Sigma\).
  \end{enumerate}
\end{itemize}

\begin{itemize}
\item It is difficult to consider cases in which an agent ought to judge that \(\chi\) follows from \(\Sigma\) without having the ability and opportunity to reason from \(\Sigma\) to \(\chi\).
\item Likewise, it is difficult to consider cases in which the agent does not have the general ability to judge that \(\phi\) holds on the basis of \(\Sigma\) but has the specific opportunity and ability to judge that \(\phi\) holds on the basis of \(\Sigma\).
  \begin{itemize}
  \item Some care is required in understanding the specific opportunity and ability, for if information factors into the understanding of opportunity, then there may well be cases in which the agent does not have the general ability to judge due to the required information being highly restricted, but does have the ability given special access.
    \begin{itemize}
    \item However, if opportunity involves information in this way and general ability does not,
    \end{itemize}
  \item Possible exceptions are cases in which an agent takes some kind of stimulant which allows them to reason
  \end{itemize}
\end{itemize}

\newpage

\section{Can}
\label{sec:can-1}

\begin{itemize}
\item Assumption is that the agent's reasoning traces some consequence relation \(\vDash\).
\item And, this consequence relation is such that \(\vDash\) and \(\nvDash\) are incompatible.
\item If \(\Sigma \vDash \chi\) then it is not the case that \(\Sigma \nvDash \chi\).
  \begin{itemize}
  \item Contraposed:
  \item If \(\Sigma \nvDash \chi\) then it is not the case that \(\Sigma \vDash \chi\).
  \end{itemize}
\item Does not assume that if it is not the case that \(\Sigma \nvDash \chi\) then \(\Sigma \vDash \chi\).
\item This really doesn't say too much, as \(\Sigma \nvDash \chi\) iff it is not the case that \(\Sigma \vDash \chi\).
\item Hence, the requirement amounts to \(\phi \rightarrow \lnot\lnot\phi\).
\item But don't assume that \(\lnot\lnot\phi \rightarrow \phi\), so something intuitionistic is okay.
\end{itemize}

\begin{itemize}
\item Idea is that, in some sense, \(\Sigma \vDash \chi\) in the actual world.
\item This should be okay.
\item Because, if monotonic, then nothing can go wrong, and if non-monotonic, then it should still be the case that the entailment holds even if there's a way to expand the premises such that \(\Sigma' \nvDash \chi\).
\item Hence, unlike some instance of `can' modals, there's going to be some kind of factive constraint.
\item Or, better, \(\Diamond\phi \rightarrow \Box\phi\), as this would sit more neatly with the standard semantics.
\item In other words, one can't mess with the consequence relation when searching for a witness.
\item This might be contentious, depending on how the consequence relation is understood, but on the other hand it might be fair to assume this as a further constraint on relevant consequence relations.
\item Of course, it's even easier to say that the consequence relation must satisfy the requirements of an \(S5\) modality.
  \begin{enumerate}
  \item \(\Diamond\phi \rightarrow \Box\phi\)
  \item \(\Box\phi \rightarrow \phi\)
  \item \(\Diamond\Diamond\phi \rightarrow \Diamond\phi\)
  \end{enumerate}
\item If this isn't satisfied, then the consequence relation is too weak to be of interest.
\item However, this doesn't say too much, as one may have the choice between selecting worlds first or selecting consequence relation first.
\item I think that the worlds first is appropriate, so this will rule out certain consequence relations.
\item However, I also don't think that this rules out too much, as it's really hard to think of interesting consequence relations which don't satisfy this constraint, esp.\ as the agent should be able to witness the event in the actual world, so to speak.
\item This is the familiar idea, for most modals of this kind, that the laws remain fixed.
\item There are ways to break this assumption, but it's something of a default assumption in the premises.
\end{itemize}

\begin{itemize}
\item So, \(\Sigma \vDash \chi\) holds in the actual world, and all relevant accessible worlds.
\item And, therefore \(\Sigma \nvDash \chi\) does not hold.
\item And, given this an the \(S5\) axioms, this pair also holds for any world in which one wants a witness.
\item Therefore, if the agent can reason, then there is a consequence relation, and it's not possible for the agent to witness \(\Sigma \nvDash \chi\) because there's no accessible world in which this holds.
\end{itemize}

\begin{itemize}
\item This is unlike other instance of `can', where \(S5\) constraints won't hold.
\item Hence, the relevant issue is whether the agent can \emph{complete} the instance of reasoning.
\item To contrast, it may be the case that one can go to the beach and one can go to the fair.
  Therefore, it is possible for one to go to the beach, but from this is does not follow that one necessarily goes to the beach.
  Else, it would be necessary that one goes to both the beach and the fair.
  This is more explicit if one can go the beach and one can not go to the beach.
\end{itemize}

\begin{itemize}
\item What's also needed is that \(\Sigma \nvDash \chi\) is not due to failure.
\item Plausibly, one is only interested in success cases of the verb.
\item For, the agent being unable to reason from \(\Sigma\) to \(\chi\) is not the same that the agent being able to reason that \(\Sigma \nvDash \chi\).
\item Whether or not \(\Sigma \vDash \chi\) could be outside of what the agent is able to do.
\end{itemize}

\begin{itemize}
\item Roughly speaking there are three distinct possibilities.
  \begin{enumerate}
  \item\label{reason:out:1} Agent establishes that \(\chi\) holds on the basis of \(\Sigma\)
  \item\label{reason:out:2} Agent establishes that \(\chi\) does not hold on the basis of \(\Sigma\)
  \item\label{reason:out:3} Agent fails to establish whether or not \(\chi\) holds on the basis of \(\Sigma\)
  \end{enumerate}
\item Exactly one of the above holds if the agent does some reasoning.
\item Fourth may be:
  \begin{enumerate}[resume]
  \item Agent does not reason.
  \end{enumerate}
  However, this isn't too interesting, and may be a case of \ref{reason:out:3}.
\end{itemize}

\begin{itemize}
\item The following three conditionals are equivalent:
  \begin{enumerate}
  \item \(C(\Sigma \vDash \chi) \rightarrow \lnot C \lnot (\Sigma \vDash \chi)\)
  \item \(C(\Sigma \vDash \chi) \rightarrow \lnot C (\Sigma \nvDash \chi)\)
  \item \(C(\Sigma \nvDash \chi) \rightarrow \lnot C (\Sigma \vDash \chi)\)
  \end{enumerate}
  Hence, only need to argue for one instance to establish exclusion on demonstrating entailment or non-entailment.
\end{itemize}

\begin{itemize}
\item This is a mistake.
\item I'm interested in cases where the agent is prevented from judging that \(\chi\) holds on the basis of \(\Sigma\).
\item This, in turn, is distinct from the agent judging that \(\chi\) does not hold on the basis of \(\Sigma\).
\item For example, it may be the case that the agent can not judge that \(\chi\) holds on the basis of \(\Sigma\) due to some lack of evidence.
\end{itemize}

\newpage

\section{Something puzzling}
\label{sec:something-puzzling}

\begin{itemize}
\item In order for it to be permissible for the agent to judge that \(\chi\) holds on the basis of \(\Sigma\), it need only be the case that the agent can not judge that \(\chi\) does not hold on the basis of \(\Sigma\).
\item For, the idea is that if the agent ought not then the agent can not.
\item So, if the agent can't provide a counterexample, then the agent can hold\dots
\item This, is remarkably weak, because it holds even when the agent is unable to demonstrate that \(\chi\) does follow from \(\Sigma\).
\item And, this seems unintuitive.
\item For, there are many things that I cannot demonstrate to not be the case.
\item For example, it seems I can hold arbitrary propositions about what's going on in other peoples minds, in some sense.
\item Is this right?
\item Ought not then can not.
\item Not can not, so not ought not.
\item Yes, this is odd.
\item So, not reason to think something is true, but also not can not demonstrate that it does not follow.
\end{itemize}

\begin{itemize}
\item Alternative, is to hold that if one ought not do something then one can do something else.
\item Ought \(\lnot\phi\) then can \(\phi\).
\item not can \(\phi\) then not ought not \(\phi\).
\item Not what I want.
\item Ought \(\lnot\phi\) then not can \(\phi\).
\item This would also work.
\item If an agent ought to not hold \(\chi\) on the basis of \(\Sigma\), then it is not the case that the agent can hold \(\chi\) on the basis of \(\Sigma\).
\end{itemize}

\begin{itemize}
\item Worry, roughly, is that I need a strong sense of can, and hence a fairly strong sense of ought.
\item But, this is fine.
\item For, I need the sense of ought work in the ought implies can statement, so I need any sense of ought that's going to say that holding the conclusion on the basis of the premises ought not be done.
\item For, if a weaker understanding is relevant, then I get the conclusion in any case, as on the weaker reading the ought statement does hold.
\item Hence, part of the difficulty for the opponent is, if I can make the argument work, finding a way to squeeze an intermediate sense of ought in between the two extremes.
\end{itemize}


\newpage

\begin{itemize}
\item Dealing with rationality.
\item So, the can is some kind of rational can.
\item And, not judging that the conclusion follows from the premises is not an rationally available option.
\item So the only rationally available option is to judge that the conclusion follows from the premises.
\item However, it need not be the case that one is rational only if one does every rationally available option.
\item So, there's a kind of unique rational option, which forms a kind of non-voluntarism.
\item However, something about the agent not needing to take a stance.
\item That's the key part that's missing.
\item The idea is that in not judging that the conclusion follows from the premises, there is some action that the agent does.
\item So, this is different from not pursuing any relationship between the conclusion and the premises.
\end{itemize}

\begin{itemize}
\item Whether \(\chi\) follows from \(\Sigma\) is at issue.
\item So, if the agent is to engage with \(?(\Sigma \vDash \chi)\).
\item Not the case that this is always at issue.
\item So I kind of want \(\lnot J (\Sigma \vDash \chi) \rightarrow J\Diamond (\Sigma \nvDash \chi)\).
\item But this isn't quite right.
\item Judging is the problem.
\item I'm interested in an attitude which considers the agent's reasoning.
\item Roughly, the attitude is judging that conclusion follows from premises given reasoning.
\item In this respect, the problem arises because this clearly conflicts with the agent being able to reason.
\end{itemize}

\begin{itemize}
\item If the agent can judge that they are not committed to \(\phi\), then there is a `fixed point' at which \(\phi\) does not hold.
\item So, there's a sequence of events, such that eventually there's no continuation for which \(\phi\) holds.
\item 
\end{itemize}


\begin{itemize}
\item Judging that one is committed to \(\phi\) does not lead to judging that \(\phi\) is true.
\item For, this could indicate that some premises is mistaken.
\item It's similar to \citeauthor{Harman:1986aa} on the conditional.
\item I'm committed to \(\phi\), but I judge \(\lnot\phi\), hence problem with premises/reasoning.
\item However, if agent is confident in both, then there is a kind of factivity inference.
\item And, in the cases of interest, these conditions are met.
\end{itemize}

\begin{itemize}
\item The attitude of interest, parsed as ``judge that committed to \(\phi\)''.
\item There are a number of ways to understand this.
\item Here, agent is committed to \(\phi\) is they have the ability to reason to \(\phi\).
\item So, could say ``judge that have the ability to reason to \(\phi\)'', instead.
\item Then, the argument is that this is permissible if the agent can do the reasoning.
\end{itemize}

The intuitive argument, is, roughly:

\begin{itemize}
\item Agent is permitted to hold that \(\chi\) follows on the basis of \(\Sigma\).
\item This is because, it is not true that the agent ought not hold that \(\chi\) follows on the basis of \(\Sigma\).
\item For, if the agent ought not hold that \(\chi\) follows on the basis of \(\Sigma\), then the agent can not hold that \(\chi\) follows on the basis of \(\Sigma\).
\item In other words, the agent has the ability to not hold that \(\chi\) follows on the basis of \(\Sigma\).
\item Yet, this means that the agent can intervene on any purported demonstration from \(\Sigma\) to \(\chi\).
\end{itemize}

The important stuff happens in the last two steps.
For, it's relatively simple to show that the agent cannot intervene granted that the agent is able to reason.

\begin{itemize}
\item One issue is that it seems the only thing that agent can do is hold that \(\chi\) follows on the basis of \(\Sigma\).
\item And this is to some extent true.
\item But this doesn't mean that the agent ought to hold that \(\chi\) follows on the basis of \(\Sigma\).
\item Yet, the agent not being able to not hold that \(\chi\) follows on the basis of \(\Sigma\) suggests this.
\item But, this assumes that the agent forms an attitude as to whether or not \(\chi\) follows from \(\Sigma\).
\item The problem here is that one could argue that the agent ought not form an attitude.
\item That is, there's the potential for a gentle murderer style paradox here.
\end{itemize}

\begin{itemize}
\item Another way of putting things is that I have ``as to whether \(?(\Sigma \vDash \chi)\), only option is \(\Sigma \vDash \chi\)''.
\item So, if one takes a stand, then there's only one option, but it's possible to recommend against taking a stand.
\item Agent does not have the ability to not hold that \(\chi\) follows on the basis of \(\Sigma\).
\end{itemize}

\begin{itemize}
\item The problem is that if the agent doesn't hold that \(\chi\) follows on the basis of \(\Sigma\), then this looks like a counterexample to the claim that it is not the case that the agent can not hold that \(\chi\) follows on the basis of \(\Sigma\).
\item Yet, intuitively this is quite plausible, because one doesn't expect the agent to be confident on everything that follows on the basis of reasoning that they are able to do.
\end{itemize}

\begin{itemize}
\item There's got to be some kind of response here.
\item And this is something along the lines of:
  \begin{itemize}
  \item It is not the cases that \(\lnot P \lnot\phi \rightarrow O \phi\)
  \end{itemize}
\item For, it doesn't seem to be the case that if not doing an action is impermissible then the agent ought to do the action.
\item Well, put like this it seems okay.
\item Not permissible for the agent not to hold that \(\Sigma \vDash \phi\), but this doesn't mean that the agent ought to hold that \(\Sigma \vDash \phi\).
\item {\color{red} Note that these points are for the following argument\dots}
\item It seems that things are a lot smoother if I work with impermissibility.
\item For, then the conclusion is simply that it is impermissible for the agent to hold that \(\Sigma \vDash \chi\).
\item Agent can't form this attitude, in short, rather than requiring the agent to form the attitude.
\end{itemize}

\begin{itemize}
\item To object, one runs the argument slightly differently.
\item If it's permissible for the agent not to hold that \(\chi\) follows on the basis of \(\Sigma\), then the agent can not hold that \(\chi\) follows on the basis of \(\Sigma\).
\item Yet, by the same reasoning it is not the case that the agent can not hold that \(\chi\) follows on the basis of \(\Sigma\).
\item Therefore, it is not permissible for the agent not to hold that \(\chi\) follows on the basis of \(\Sigma\).
\item So, the agent ought to hold that \(\chi\) follows on the basis of \(\Sigma\).
\end{itemize}

This objection is somewhat problematic.
However, the first premise isn't true.
Which is part of the problem.
It's not permissible for the agent to do this, hence the agent ought to hold that \(\chi\) follows on the basis of \(\Sigma\).
Though, in a sense this isn't so bad, in that it does seem true that this is what the agent ought to do in some sense.
The point of contention is that it's not clear that the agent is `required' to do what they ought to do in order to be of interest.
For, I'm looking only for permissibility.


Maybe I should run the argument on the permissibility of holding that \(\Sigma \nvDash \chi\).
Then, argue that one is prevented from speculating only if it is permitted to hold that \(\Sigma \nvDash \chi\).

Another option is:
\[
  P(\text{hold} \Sigma \vDash \chi) \lor P(\text{hold}\Sigma \nvDash \chi)
\]
Then the conclusion follows by disjunction elimination.
Not clear on why this should be true.
Ah, well, this probably isn't true.
For there should be a third conjunct, maybe.
Well, the issue is, whether \(\phi \lor \lnot\phi\) entails \(PX\phi \lor PX\lnot\phi\).
For, it's not simply \(\phi \lor \lnot\phi\), given the addition of holding in the scope of permissibility.

\[
  P(\text{hold} \Sigma \vDash \chi) \lor P(\text{hold}\Sigma \nvDash \chi) \lor P(\lnot\text{hold} \Sigma \vDash \chi) \lor P(\lnot\text{hold}\Sigma \nvDash \chi)
\]

So, there are also the options of not holding.
So, it seems I would like

\[
  P(\text{hold}\Sigma \nvDash \chi) \lor P(\text{speculate}\Sigma \vDash \chi)
\]

So, either it's permissible to deny that something follows, or it's permissible to speculate.
This is then all about speculation, which also gets messy, but isn't yet ruled out.
So, either it's permissible to hold that no witness for \(\Sigma \vDash \chi\) can be provided, or it is permissible to hold that there is a witness for \(\Sigma \vDash \chi\).
So, if I can reason, then I can hold that a witness may be provided.
And, then the significant details which follow from this concern the providing of the witness.
For, if there is a witness, then if I'm confident, then a quasi factivity inference applies.

This isn't quite one big circle, because the inference form being able to reason to the permissibility to stipulating a witness isn't obvious.

The sort of idea here is that the agent is not permitted to hold an attitude that would block speculation.

\begin{itemize}
\item Can \(\Sigma \vDash \chi\) then \(\lnot P\) to hold \(\Sigma \nvDash \chi\)
\item Can \(\Sigma \nvDash \chi\) then \(\lnot P\) to hold \(\Sigma \vDash \chi\)
\item So, \(\lnot P\) to hold \(\Sigma \nvDash \chi\) \(\lor\) \(\lnot P\) to hold \(\Sigma \vDash \chi\)
\item Which is \(\lnot (\) \(P\) to hold \(\Sigma \nvDash \chi\) \(\land\) \(P\) to hold \(\Sigma \vDash \chi\) \()\)
\item So, it's not both permissible to hold an entailment and to hold that the entailment fails.
\item This is similar to \(\lnot(PB\phi \land PB\lnot\phi)\).
\item I.e.\ denial of voluntarism.
\item But this doesn't require the agent to form either belief, or hold either entailment.
\end{itemize}

\begin{itemize}
\item It seems as though this is what is expected.
\end{itemize}


\newpage

The trouble with the ought implies can principle is that because I want to do modus tollens, I end up needing to argue for some kind of \(\lnot C \lnot \phi\).
That is:
\begin{itemize}
\item \(O\lnot\phi \rightarrow C\lnot\phi\)
\item \(\lnot C \lnot \phi \rightarrow \lnot O \lnot \phi\)
\end{itemize}
And, the \(\lnot C \lnot \phi\) argument is particularly difficult, because it then seems as though the agent is prevented from doing something.

On my understanding of `can', the idea is that \(C\phi\) is true just in case the agent has the ability to intervene so that \(\phi\) is true.
Hence, \(C \lnot\phi\) is the ability to intervene so that \(\lnot\phi\) is true, or so that \(\phi\) is not true.
And, so \(\lnot C \lnot \phi\) is the inability to intervene so that \(\lnot\phi\) is true, or so that \(\phi\) is not true.

So, the agent doesn't have the ability to intervene so that \(\lnot\phi\) is true, but this doesn't prevent the agent from performing any particular action, it just ensures that the action is not going to be successful.
Hence, this is where the idea of a weaker can comes into player, in the sense that \(\lnot C \lnot\phi\) means that the agent is not able to intervene so that \(\lnot\phi\) is true, and hence there's some action that the agent is able to do so that \(\lnot\phi\) is not true.


\newpage

Tentatively, the main claim is:

\begin{enumerate}
\item If it is not permissible for the agent to judge that they can witness reasoning from \(\Sigma\) to \(\chi\), then the agent has the ability to not witness reasoning from \(\Sigma\) to \(\chi\).
\item If the agent does not have the ability to not witness reasoning from \(\Sigma\) to \(\chi\), then it is permissible for the agent to judge that they can witness reasoning from \(\Sigma\) to \(\chi\).
\end{enumerate}

This is not super obvious.
It looks like this should really be the converse.
Or, it certainly seems as though the converse is sound.
For, if the agent has the ability to not witness reasoning from \(\Sigma\) to \(\chi\) then it doesn't seem permissible for the agent to judge that they can witness from \(\Sigma\) to \(\chi\), as the agent can't witness reasoning from \(\Sigma\) to \(\chi\).

Agent is not allowed to judge that they can witness.

\begin{enumerate}
\item An agent is not permitted to judge that they can witness reasoning from \(\Sigma\) to \(\chi\) only if the agent has the ability to not witness reasoning from \(\Sigma\) to \(\chi\).
\end{enumerate}

So, the agent is not permitted to judge that they have the ability to witness reasoning.
And, the agent does not have the ability not to witness the reasoning.
Hence, agent has the ability to witness reasoning.
So, agent has the ability to witness, but is not permitted to judge that they have the ability to witness.
There are cases like this.
For, I may have the ability, but from my perspective it seems as though I do not have the ability.
I.e., evidence suggests that I do not have the ability, but in fact I do.
Yet, on the other hand, this is not at all obvious either.
In these cases, it's not as though I can access the fact that I'm permitted.

\begin{enumerate}
\item If it is not permissible for the agent to judge that they can witness reasoning from \(\Sigma\) to \(\chi\), then the agent has the ability to not judge that they can witness reasoning from \(\Sigma\) to \(\chi\).
\end{enumerate}

Where, the ability to not judge is understood as the ability to resist judging that they can witness.

\begin{enumerate}
\item If it is not permissible for the agent to judge that they can witness reasoning from \(\Sigma\) to \(\chi\), then the agent has the ability to resist judging that they can witness reasoning from \(\Sigma\) to \(\chi\).
\item If the agent does not have the ability to resist (judging that they can witness) reasoning from \(\Sigma\) to \(\chi\), then it is permissible for the agent to judge that they can witness reasoning from \(\Sigma\) to \(\chi\)
\end{enumerate}

Getting to the conclusion that the agent is not permitted to judge that \(\Sigma \nvDash \chi\) is interested.
For, if the agent recognises this, then it can only be that \(\Sigma \vDash \chi\).
That is, only \(\Sigma \vDash \chi\) is available to the agent.
In a sense, this is what it is for the agent to be committed to \(\Sigma \vDash \chi\).
Because the agent's reasoning doesn't tolerate a gap here.
Hence, intuition is that I should be able to do \emph{something} with this premise.

\begin{itemize}
\item Not (permitted to judge \(\Sigma \nvDash \chi\) and permitted to judge \(\Sigma \vDash \chi\))
\item Not permitted to judge \(\Sigma \nvDash \chi\) or not permitted to judge \(\Sigma \vDash \chi\)
\end{itemize}

Well, these follow from permissibility of judging to can witness.

What I really want is:

\begin{itemize}
\item Permitted to judge \(\Sigma \nvDash \chi\) or permitted to judge \(\Sigma \vDash \chi\)
\item Not (not permitted to judge \(\Sigma \nvDash \chi\) and not permitted to judge \(\Sigma \vDash \chi\))
\end{itemize}

In defence of this, an agent may be unaware of what they are permitted to do.
Hence, the agent can't pick an arbitrary judgement.
And yet, it's still messy.
It seems unintuitive, that I'm not permitted to make a judgement, rather than being unaware of what I am permitted to do. 

\newpage

\begin{itemize}
\item If it is not permissible for the agent to \emph{expect} that \(\Sigma \vDash \chi\), then the agent has the ability to interrupt sequence of events that would establish \(\Sigma \vDash \chi\).
\item If the agent does not have the ability to interrupt any sequence of events that would establish \(\Sigma \vDash \chi\), then the agent is permitted to \emph{expect} that \(\Sigma \vDash \chi\).
\end{itemize}

This may only hold for reasoning.
Or, a little more general, with some \(\phi\) that follows from the action of the agent.
So, here, it is a necessary condition on this speculative attitude that the agent can prevent it from coming about.

Okay.
Then, I don't have the ability to interrupt a sequence of events that would lead to be getting fired based on my recent performance.
Therefore, it is permissible for me to expect that I will be fired based on my performance.
However, this sequence of events is highly unlikely.
Same for getting ill, which is a little clearer.
I don't have the ability to interrupt a sequence of events that leads to me being ill, but it doesn't seem as though it's permissible for me to expect that I will get ill.
Though, getting ill is not something that follows from my action.
Still, it's far from clear that there is a general principle here.

However, I may expect to get ill and I may also expect to stay healthy.
The difference with reasoning is that if I expect \(\Sigma \vDash \chi\), then I can not expect \(\Sigma \nvDash \chi\)?
Right, this might follow.
For, it seems I have an argument that shows it is not permissible for the agent to expect that \(\Sigma \nvDash \chi\).

Because:
\begin{itemize}
\item Permissible to expect that \(\Sigma \nvDash \chi\) only if agent has the ability to witness \(\Sigma \nvDash \chi\).
\item Not permissible to expect that \(\Sigma \vDash \chi\) only if the agent has the ability to deny witness for \(\Sigma \vDash \chi\).
\end{itemize}

Here, these two conditionals can be put together to obtain:
\begin{itemize}
\item Agent does not have the ability to deny witness only if the agent has the ability to witness.
\item If agent does not have the ability to witness then the agent has the ability to deny witness.
\end{itemize}

So I might want to deny the first conditional, at least for the broad notion of expectation, or actions.
But I'm not sure this is appropriate.
The notion of ability simply requires that there is some action that the agent can take.
So, there is some path that the agent can step through.
Therefore, it's not the case that every path ends in failure, so if the agent can expect something, then it seems as though they can expect \dots oh, it's the difference between all defeaters and some defeater.
This might be okay.

For, permissible to expect only if there is always some defeater that the agent can avoid.
This is denied if the agent can reason.
So, the first conditional is read with the \(\exists\) sense of can.
The second conditional is read with the \(\forall\) sense of can.

And so the second conditional only requires that they agent lacks the ability to intervene on \(\Sigma \vDash \chi\) to obtain permissibility of expectation.


\newpage

\printbibliography

\end{document}
