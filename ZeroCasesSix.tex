\documentclass[10pt]{article}
% \usepackage[margin=1in]{geometry}
% \newcommand\hmmax{0}
% \newcommand\bmmax{0}
% % % Fonts% %
\usepackage{luatexja}

\usepackage[T1]{fontenc}
   % \usepackage{textcomp}
   % \usepackage{newtxtext}
   % \renewcommand\rmdefault{Pym} %\usepackage{mathptmx} %\usepackage{times}
\usepackage[complete, subscriptcorrection, slantedGreek, mtpfrak, mtpbb, mtpcal]{mtpro2}
   \usepackage{bm}% Access to bold math symbols
   % \usepackage[onlytext]{MinionPro}
   \usepackage[no-math]{fontspec}
   \defaultfontfeatures{Ligatures=TeX,Numbers={Proportional}}
   \newfontfeature{Microtype}{protrusion=default;expansion=default;}
   \setmainfont[Ligatures=TeX,BoldFont={*-Semibold}]{Source Serif Pro}
   \setsansfont[Microtype,Scale=MatchLowercase,Ligatures=TeX,BoldFont={*-Semibold}]{Source Sans Pro}
   \setmonofont[Scale=0.8]{Atlas Typewriter}
   % \usepackage{selnolig}% For suppressing certain typographic ligatures automatically
   \usepackage{microtype}
% % % % % % %
\usepackage{amsthm}         % (in part) For the defined environments
\usepackage{mathtools}      % Improves  on amsmaths/mtpro2
\usepackage{amsthm}         % (in part) For the defined environments
\usepackage{mathtools}      % Improves on amsmaths/mtpro2
\usepackage{xfrac}

% % % The bibliography % % %
\usepackage[backend=biber,
  style=authoryear-comp,
  bibstyle=authoryear,
  citestyle=authoryear-comp,
  uniquename=false,
  % allinit,
  % giveninits=true,
  backref=false,
  hyperref=true,
  url=false,
  isbn=false,
  useprefix=true,
  ]{biblatex}
\DeclareFieldFormat{postnote}{#1}
\DeclareFieldFormat{multipostnote}{#1}
% \setlength\bibitemsep{1.5\itemsep}
\newcommand{\noopsort}[1]{}
\addbibresource{Thesis.bib}

% % % % % % % % % % % % % % %

\usepackage[inline]{enumitem}
\setlist[enumerate]{noitemsep}
\setlist[description]{style=unboxed,leftmargin=\parindent,labelindent=\parindent,font=\normalfont\space}
\setlist[itemize]{noitemsep}

% % % Misc packages % % %
\usepackage{setspace}
% \usepackage{refcheck} % Can be used for checking references
% \usepackage{lineno}   % For line numbers
% \usepackage{hyphenat} % For \hyp{} hyphenation command, and general hyphenation stuff
\usepackage{subcaption}
% % % % % % % % % % % % %

% % % Red Math % % %
\usepackage[usenames, dvipsnames]{xcolor}
% \usepackage{everysel}
% \EverySelectfont{\color{black}}
% \everymath{\color{red}}
% \everydisplay{\color{black}}
\definecolor{fuchsia}{HTML}{FE4164}%Neon Fuchsia %{F535AA}%Neon Pink
% % % % % % % % % %

\usepackage{pifont}
\newcommand{\hand}{\ding{43}}
\usepackage{array}


\usepackage{multirow}
\usepackage{adjustbox}

\usepackage{titlesec}

\usepackage{multicol}

\setcounter{secnumdepth}{4}
\setcounter{tocdepth}{4}

\usepackage{tikz}
\usetikzlibrary{bending,arrows,positioning,calc}
\usetikzlibrary{arrows.meta}
\usepackage{tikz-qtree} %for simple tree syntax
% \usepgflibrary{arrows} %for arrow endings
% \usetikzlibrary{positioning,shapes.multipart} %for structured nodes
\usetikzlibrary{tikzmark}
\usetikzlibrary{patterns}


\usepackage{graphicx} % for images (png/jpeg etc.)
\usepackage{caption} % for \caption* command


\usepackage{tabularx}

\usepackage{bussalt}

\usepackage{Oblique} % Custom package for oblique commands
\usepackage{CustomTheorems}
\usepackage{FuturePromisedEvents}

\usepackage{svg}
\usepackage[off]{svg-extract}
\svgsetup{clean=true}

\usepackage{dashrule}

\newcommand{\hozline}[0]{%
  \noindent\hdashrule[0.5ex][c]{\textwidth}{.1pt}{}
  %\vspace{-10pt}
  % \noindent\rule{\textwidth}{.1pt}
}

\newcommand{\hozlinedash}[0]{%
  \noindent\hdashrule[0.5ex][c]{\textwidth}{.1pt}{2.5pt}
  %\vspace{-10pt}
}

\usepackage{contour}
% \usepackage{pdfrender}

\usepackage{extarrows}

% % % My commands % % %

% % % % % % % % % % % %

\usepackage[hidelinks,breaklinks]{hyperref}

\title{Promised Futures}
\author{Ben Sparkes}
% \date{ }


\begin{document}

\maketitle

\tableofcontents

\newpage

\begin{note}
  Style of paper:
  \begin{itemize}
  \item Cases in which the agent is confident that some reasoning can be done from some support to a conclusion.
  \item Goal is to provide a way to understand these cases, so that the reader can figure out how they may relate to things that the reader is interested in.
  \item So there are three lines of interest
    \begin{enumerate}
    \item The cases, and why some forms of explanation can be resisted.
    \item The understanding of the cases, and what this understanding depends on and/or is independent of.
      \begin{itemize}
      \item structural issues, etc.\
      \end{itemize}
    \item What kind of issues these cases link to.
      \begin{itemize}
      \item Reasoning.
      \item Normativity.
      \item Justification.
      \item Goal is to lay some foundations for approaching these issues in greater detail.
      \end{itemize}
    \end{enumerate}
  \end{itemize}
\end{note}



\section{Introduction}
\label{sec:introduction}

\subsection{Overview}
\label{sec:overview}

\begin{note}
  Identify and provide a framework.
  Intuitively, some instances are permissible and others impermissible.
  Will not make any strong normative claims, and instead sketch how the framework can be used to make an argument.
  (Basically, in the superman case no way to fulfil promise, while in the taxes case there is.)
\end{note}

\begin{itemize}
\item Main feature of cases of interest:
  \begin{enumerate}
  \item An agent is confident that some piece of reasoning would demonstrate that a particular conclusion follows from some support.
  \end{enumerate}
\item Three additional features.
  \begin{enumerate}[resume]
  \item Agent is confident that the support holds.
  \item The agent is confident that they (the agent) are able to reason from the support to the conclusion.
    \begin{itemize}
    \item Though, note somewhere (maybe here) that the agent may be assisted in this reasoning.
    \end{itemize}
  \item And, the agent adopts an attitude toward the conclusion on the basis of the prior features.
  \end{enumerate}
\item Term this \emph{speculation.}\nolinebreak
  \footnote{Something about how the term is used, esp.\ in CS.}
\item Goal is to provide a framework for thinking about a certain type of scenario which falls under this general description.
\end{itemize}

\subsection{Central case: Morse \& Lewis.}
\label{sec:central-case:-morse}

\begin{scenario}[Morse \& Lewis]
  More and Lewis are police officers.
  The pair have been working together for some time, and each consider st the other an equal (in their role as a police officer).
  The equality is supported their experience working together, and reflected in their working methodology:
  Any part of an investigation may be worked on by either Morse or Lewis, and no type of work has been done exclusively by either.
  Morse is confident that they could do any work Lewis does, and likewise Lewis is confident that they could do any work that Morse does.

  The pair are working on an investigation and share a casebook containing all the evidence they have gather so far.
  Lewis returns home from their day off to discover a message from Morse on their answering machine.
  Morse says, in so few words, that they, in the absence of any new evidence, have been reviewing the casebook and what evidence they do have supports bringing in Woodthorpe for questioning.
  Morse is not loquacious, and Morse does not explain their reasoning in the message.
  Lewis listens to the message and goes to bed.

  The following morning, and to Lewis' surprise, Lewis spots Woodthorpe on their way to the police station.
  And, given Morse's message, Lewis arrests Woodthorpe.
\end{scenario}

\begin{enumerate}
\item Lewis is confident that there is some piece of reasoning would demonstrate that Woodthorpe's arrest is supported by the evidence contained in the case book.
\item Lewis is confident that the contents of the casebook constitute evidence.
\item Lewis is confident that both they and Morse are able to reason to the support for Woodthorpe's arrest from the casebook.
\item Lewis adopted some attitude to the proposition that Woodthorpe being arrested, and the action of arresting Woodthorpe (in part) depended on Lewis forming that attitude.
\end{enumerate}

\begin{itemize}
\item In ideal case, Lewis would have done the reasoning.
\item Indeed, in the ideal case Lewis would not be unaware of what follows from their evidence.
\end{itemize}

\begin{itemize}
\item Might think that Lewis needs only be confident that Woodthorpe's guilt can be demonstrated.
\item So, Lewis does not need to be confident that Woodthorpe is guilty.
\item And, Lewis has testimony for the availability of a demonstration.
\item As Lewis does not have a demonstration, Lewis is not required to be confident that Woodthorpe is guilty.
\item I do not think there is an important difference.
\item For, it seems plausible that Lewis does come to be confident that Woodthorpe is guilty.
\item However, if there is an important difference, then add additional constraints so that Lewis would only make the arrest based on their confidence that Woodthorpe is guilty.
\item May argue that the possibility of a demonstration is sufficient for Lewis, but not interested in ideal case.
\item Want to understand certain types of cases, the kind of cases {\color{red} stated in the introduction}.
\item It is plausible that these types of cases happen, and the ability to recast agency so that these do not happen would not help understand how creatures like us reason and act.
\end{itemize}

\begin{itemize}
\item Other explanations are possible, discuss some of these in what follows.
\item Two things to note:
  \begin{itemize}
  \item Lewis has confidence of the possibility of reasoning, and on this basis forms an attitude an arrests.
    \begin{itemize}
    \item Could think that Lewis straightforwardly forms an attitude toward the proposition, understands Morse as providing testimony, of a sort.
    \item There may be good reason to make this move, but it comes with some difficulties.
      \begin{enumerate}
      \item Morse didn't explicitly provide testimony.
      \item However, it seems Morse would not be able to avoid providing testimony if pressed.
      \item Can add some complexity to the case, assume that there are some claims in the dossier that only Lewis is aware of the evidence for, and hence Morse is restricted to only making a claim about what follows.
      \end{enumerate}
    \item Bracket testimony for the paper.
    \item Confidence is key, and while this may be supported by testimony, it is not necessary.
      The equality and history establish this without directly appealing to testimony.
    \end{itemize}
  \item It is important that Lewis is able to follow the support for the arrest.
    \begin{itemize}
    \item There are many ways to explain why this would be important for Lewis, and there are many ways to explain why this would not be important for Lewis.
    \item Sometimes this may not matter for the agent.
      \begin{itemize}
      \item If Morse is expected to do the heavy lifting, then perhaps Lewis only needs to understand that there is reason for arresting Woodthorpe.
      \item Lewis doesn't really care.
      \end{itemize}
    \item However, assume that it is important here.
      \begin{itemize}
      \item Can assume that Lewis would not arrest if there wasn't evidence.
      \item Lewis is aware that they themselves will need to justify the arrest.
      \item Lewis expects to fully understand the case file, etc.\
      \end{itemize}
    \end{itemize}
  \end{itemize}
\item Mention issue of justification, and that there's some things that can be read into the scenario that go a little beyond the basics that I'll be focusing on to start with.
\item Also, may want to mention understanding, as this is somewhat related but not required.
  \begin{itemize}
  \item For example, Morse nor Lewis may be involved in a complex financial case, and are able to demonstrate that Woodthorpe is guilty due to relations between statutes, but still do not \emph{understand} which Woodthorpe in anything other than a formal sense.
  \end{itemize}
\end{itemize}

\begin{itemize}
\item Also mention conflict with Lord, i.e.\ add in that Lewis has a copy of the case file at hand, and does not appropriately respond to reasons (may need some additional changes to adequately make this argument).
\end{itemize}

\begin{itemize}
\item That the case file is an object is unimportant.
\item If needed, assume that Lewis (and Morse) have memorised the collection of propositions which constitutes the case file.
\item Lewis and Morse are bounded agents, so they have not necessarily have attitudes toward anything that follows from the case file.
\item Borrow the phrase `epistemic reach' from \cite{Egan:2007aa}.
\end{itemize}

\begin{itemize}
\item More can be said about the details.
\item For now, turn to an broad characterisation.
\item Goal is to provide an abstract account of the main features, allowing us to reason about the general features of these types of cases.
\item Framework!
\end{itemize}

\newpage


\section{Framework}
\label{sec:framework-1}

\begin{note}
  Propositional and doxastic justification may help capture the difference.
\end{note}

\begin{itemize}
\item Main feature of cases of interest:
  \begin{enumerate}
  \item An agent is confident that some reasoning demonstrates that a particular conclusion follows from some support.
  \end{enumerate}
\end{itemize}

\begin{itemize}
\item Lewis is confident that there is a way to demonstrate the guilt of Woodthorpe follows from the case file shared with Morse, as Morse has claimed the reasoning can be done.
\item Students in a logic class are confident that there is a way to demonstrate \(\forall x(Fx \leftrightarrow x = a)\) follows from \(\exists x(\forall y(Fy \leftrightarrow y = x) \land Rxa)\) and \(\forall x\forall y(Rxy \rightarrow y = x)\), as they are required to demonstrate the entailment as a homework exercise.
\item You are confident that there is a way to demonstrate that the speaker would like to know if the camera is honest follows from their asking 「カメラは正直?」, as you have been given a translation.\nolinebreak
  \footnote{Well, in context, as this could also be asking whether cameras (in general) are honest.}
\end{itemize}

\begin{note}
  These examples help set up the two different kinds of reasoning discussed later.
  \begin{itemize}
  \item Lewis is a speculative case.
  \item Logic is more difficult, as the student may be confident in their future reasoning, or they may remain doubtful even after providing the proof.
  \item Translation is hard to make sense of other than as incorporeal.
  \end{itemize}
\end{note}

\begin{itemize}
\item Agent does something with their confidence that some piece of reasoning would demonstrate that a particular conclusion follows from some support.
\item Agent is confident in a claim about reasoning
\item Initial interest is in understanding the claim about reasoning, and then using this to model what the agent does with the claim about reasoning given their confidence.
\end{itemize}

\subsection{Reasoning as a process}
\label{sec:reasoning-as-process}

\begin{itemize}
\item Reasoning as a process of drawing conclusions from premises.
\item Set aside normative issues.
\item Hence, reasoning is similar to buttering a piece of toast, hugging Caesar, or flying to the North Pole.
\item Neo-davidsonian.
\end{itemize}

Our interest is in reasoning as a \emph{process} of drawing a conclusion from premises.
The use of the term `reasoning' in English does not uniquely capture this process.
To illustrate, there are at least two distinct readings of the following:
\begin{enumerate}
\item\label{reasoning:reading:amb} Lewis thought about Morse's reasoning from the case file to the guilt of Woodthorpe.
\end{enumerate}

On a `factive' reading, Lewis is thinking about the state of affairs in which reasoning from the case file to the guilt of Woodthorpe has been done by Morse.
In an attempt to force the factive reading, we could create an independent \emph{that}-clause for Morse's reasoning, and state \emph{that} Morse's reasoning is a fact.
\begin{enumerate}
\item[\ref{reasoning:reading:amb}\(_{f}\).] It is a fact that Morse's reasoning from the case file to the guilt of Woodthorpe happened, and Lewis is thinking about that fact.
\end{enumerate}
Morse's reasoning may have been a process, but on a factive reading Morse's reasoning the role of the process limited to making a certain fact true.

% \begin{enumerate}
% \item Morse's reasoning from the case file to the guilt of Woodthorpe happened earlier today.
% \item It is true that Morse reasoning from the case file to the guilt of Woodthorpe happened earlier today. (Perfective aspect and past tense mix-up here --- `reasoned' is more natural.)
% \end{enumerate}

% Here, assigning a complex predicate to Morse.
% The predicate implies that some process took place, but this is really a straightforward proposition that is true or false.
% The `that' clause formulation helps make this clear.
% Even if an event (or process) is implied, that Morse reasoned is more-or-less a fact rather than an event.

On an `active' reading, Lewis is thinking about the action, event, or process of Morse reasoning from the case file to the guilt of Woodthorpe.
Similar to the forced factive reading, to ensure an active we may attempt to explicitly state that Morse's was a process.
\begin{enumerate}
\item[\ref{reasoning:reading:amb}\(_{a}\).] Morse's reasoning from the case file to the guilt of Woodthorpe was a process, and Lewis is thinking about that process.
\end{enumerate}

% \begin{enumerate}
% \item Morse was reasoning from the case file to Woodthorpe's guilt. (process)
% \item The reasoning from the case file to Woodthorpe's guilt by Morse. (an instance of the process)
% \end{enumerate}

% In the first some process was taking place.
% In the second, identify the thing that is Morse's reasoning.
% Both instance understood in terms of events or processes.
% It is this understand of reasoning that is important for us.

This distinction between fact and event is part of the motivation for \citeauthor{Davidson:2001aa}'s treatment of events.
An event need not consist of some fact.
(\citeyear[116]{Davidson:2001aa})

\begin{itemize}
\item Parallels to event semantics also suggest lines of philosophical interest, esp.\ understanding reasoning in terms of mental events/actions.
\end{itemize}

\hozlinedash

\begin{itemize}
\item Active reading is of primary interest.
\item Our approach is to capture factive aspects of reasoning by abstraction from active readings.
\item This is not to say that the active reading is strictly required.
\item Rather, things are simpler if we keep to a particular reading, and the active reading is useful.
\end{itemize}

\begin{itemize}
\item Neo-Davidsonian event semantics.
\item Events are things, and ascribe properties to events.
\item Neo-Davidsonian due to the use of thematic roles to identify and situate that participants of the event.
\end{itemize}

\hozlinedash

\begin{itemize}
\item ``Morse's reasoning from the case file to the guilt of Woodthorpe'' is structurally similar to ``Brutus hugged Caesar''.
\item ``Brutus hugged Caesar'' is analysed with\dots
  The event predicate `hug'.
  As `hug' is a transitive verb there is an `agent', a participant who hugs, and a `patient', a participant who is hugged.
  \[
    \exists e(\everb{hug} \land \eagent{Brutus} \land \epatient{Caesar})
  \]
  Our treatment of reasoning is structurally similar.
  An event predicate `reason' indicates that the event is a process of reasoning and the thematic role `agent' identifies the reasoner, and as the agent reasons from premises to a conclusion we introduce to additional thematic roles termed `in' and `out' for the premises and conclusion, respectively.
  \[
    \exists e(\everb{reason} \land \ein{\Sigma} \land \eout{\chi} \land \eagentm{a})
  \]
  Informally, the above captures \(a\)'s reasoning from \(\Sigma\) to \(\chi\).
  {\color{red} Typically} take propositions for the role of `in' and `out', with \(\Sigma\) standing for a set of propositions\nolinebreak
  \footnote{Here, note that propositions include events.
    This is why \citeauthor{Parsons:1990aa}'s terminology is avoided.

    Also, may want a conjunction of propositions, but then there's some difficulty distinguishing between \(\{\phi,\psi\}\) and \(\{\phi \land \psi\}\).
    Not building a rigorous formal system.
    Issues of object language and metalanguage, beyond the scope of the paper, really.
  Still, could introduce `premise marking parenthesis' to distinguish between \([\phi] \land [\psi]\) and \([\phi \land \psi]\) if desired.}
  which are the premises of an agent's reasoning, \(\chi\) for the proposition which is the conclusion of the agent's reasoning.
  % \[
  %   \exists e(\everb{reason} \land \ein{\text{Case file}} \land \eout{\text{Guilt of Woodthorpe}} \land \eagent{\text{Morse}})
  % \]
  \[
    \exists e
    \left(
      \begin{array}{l}
        \everb{reason} \land \\
        \ein{\text{Case file}} \land \\
        \eout{\text{Guilt of Woodthorpe}} \land \\
        \eagent{\text{Morse}}
      \end{array}
    \right)
  \]
% \item Brutus did not only hug Caesar.
% \item Brutus hugged and tickled Caesar.
%   \[
%     \exists e(\everb{hug} \land \everb{tickle} \land \eagent{\text{Brutus}} \land \epatient{\text{Caesar}})
%   \]
% \item No changes in the (thematic) roles of the participants, more detail added.
% \item And, not a separate event, the hugging was also a tickling, Brutus did not hug and also tickle.
%   Could do the same thing with a the novel-predicate \(\text{tickle-hug}\).
%   \[
%     \exists e(\everb{tickle-hug} \land \eagent{Brutus} \land \epatient{Caesar})
%   \]
% \item Same with reasoning.
\end{itemize}

\hozlinedash

\begin{itemize}
\item We introduce a function in order to capture the fact that there was reasoning from \(\Sigma\) to \(\chi\).\nolinebreak
  \footnote{\textcite[117]{Davidson:2001aa} considers an example involving the addition of 2 and 3 as an event.}
\item A function is something which relates an input to an output, and the type of reasoning we are interested in is characterised by an input (premises) and an output (conclusion).
\item Broadly stated, the specification of a function captures the factive aspect of reasoning at coarse grain.
\item Hence, some event of reasoning from premises \(\Sigma\) to a conclusion \(\chi\) ensures the existence of a function defined for input \(\Sigma\) and output \(\chi\).
  \[
    \exists e
    \left(
      \begin{array}{l}
        \everb{reason} \\
        \ein{\Sigma} \\
        \eout{\chi} \\
        \exists f(\efunc{f}) \\
          \eagentm{a}
      \end{array}
    \right)
  \]
  \item The existential captures our lack of information about whether it is possible for the agent to reason in the same way from different premises to a (not necessarily) different conclusion.
\item The event entails
  \[
    \exists f(f\Sigma = \chi)
  \]
  This may not specify a unique function, as only a single in-out pair constrains the possible witnesses of the existentially quantified function.
  To take a simple example, \(\exists f(f(2,2) = 4)\) can be witnessed by addition, multiplication, exponentiation, the constant function which always yields \(4\), and so on.
  \[
    f(2,2) = 4.
    f = +
    f = \times
    f = \exp
  \]
  When Lewis considers Morse's reasoning, the existence of a function is the relevant factive information, with additional background information which we omit for simplicity.
\item This function may be partial; defined for \(\Sigma\) and \(\chi\).
  However, whatever that function is, it can potentially be instantiated in other instances of reasoning.
\item For example with a known function, I may ask you what the median {\color{red} number of characters in a Shakespeare play} is.
    \[
    \exists e
    \left(
      \begin{array}{l}
        \everb{reason} \\
        \ein{\text{List of character counts in Shakespeare's plays}} \\
        \eout{32.5} \\
        (\efunc{\text{median}}) \\
        \eagentm{a}
      \end{array}
    \right)
  \]
\item An agent reasoned from \(\Sigma\) to \(\chi\) entails that there exists a function mapping \(\Sigma\) to \(\chi\).
\item \(\exists f(f\Sigma = \chi)\) or \(\text{mean}(\text{List of character counts in Shakespeare's plays}) = 32.5\).
\item We term the bare statement of a function relating some input to some output from some process of reasoning a \emph{mention} of the function.
\item The mention of a function contrasts with the \emph{use} of the function in the process of reasoning that function is abstracted from.
\item If inclined to think of reasoning as a rule governed activity, then this is the distinction between mentioning a rule and being guided by the rule.
\item \textcite{Simchen:2001aa} is quite nice.
\item Perhaps `demonstration of' and `reference to' a/the function.
\end{itemize}

\begin{itemize}
\item The distinction between the mention and the use of a function related to a process of reasoning is important for understanding Lewis' reasoning.
\item To give a quick example \dots
\item Propositional logic.
\item Agent reasons from \(\alpha \rightarrow \beta\) to \(\lnot \beta \rightarrow \lnot\alpha\) via contraposition.
      \[
    \exists e
    \left(
      \begin{array}{l}
        \everb{reason} \\
        \ein{\alpha \rightarrow \beta} \\
        \eout{\lnot \beta \rightarrow \lnot\alpha} \\
        \efunc{\text{Contraposition}}
      \end{array}
    \right)
  \]
\item This description does not state how the agent contraposed \(\alpha \rightarrow \beta\).
  However, by identifying the function we can be sure that if the agent reasoned from some other premise of the same form, e.g.\ \(\gamma \rightarrow \delta\), then the agent would have obtained the contrapositive; \(\lnot \delta \rightarrow \lnot \gamma\).
\item The appeal of propositional reasoning is that there is a conventional way to mention the agent's reasoning.
\item \((\alpha \rightarrow \beta) \rightarrow (\lnot \beta \rightarrow \lnot\alpha)\).
  \[
    \exists e
    \left(
      \begin{array}{l}
        \everb{reason} \\
        \ein{\{\alpha \rightarrow \beta, (\alpha \rightarrow \beta) \rightarrow (\lnot \beta \rightarrow \lnot\alpha)\}} \\
        \eout{\lnot \beta \rightarrow \lnot\alpha} \\
        \efunc{\text{Modus ponens}}
      \end{array}
    \right)
  \]
\item Convention encodes the complexities of explaining how a conditional captures the fact of some reasoning.
\item Avoid this for the rest of the paper, but useful to illustrate.
\end{itemize}

\hozlinedash

\begin{itemize}
\item Functions aren't \emph{pure}.
  May be side effects.
\end{itemize}

\hozlinedash

\paragraph{Summary of the section}

The central elements are:\nolinebreak
\footnote{
  \begin{itemize}
  \item This is similar to \textcite[24--27]{Marr:1982aa}
  \end{itemize}
  The specification of a function mapping the premises of the reasoning to conclusion of the reasoning corresponds to \citeauthor{Marr:1982aa}'s computational level.
  At this level, in \citeauthor{Marr:1982aa} words, `the performance of the device is characterized as a mapping from one kind of information to another' (\citeyear[24]{Marr:1982aa}).
  Likewise, the process of reasoning corresponds to the level of the implementation of the computation; how the computation is realised.

  Missing from our discussion is \citeauthor{Marr:1982aa}'s algorithmic level, which specifies `the choice of representation for the input and output and the algorithm to be used to transform one into the other' (\citeyear[24--25]{Marr:1982aa}).
  The specifics of representation are dealt with by however our representations are interpreted, but it is certainly natural to include a description of the method by which an agent reasons from premises to conclusion.
  For example, Brutus reasons from the premise of the event of Caesar crying to the conclusion that Brutus ought to given Caesar a hug.
  And coarsely, Brutus does so by reasoning in line with the categorical imperative.
  Stated in full:
  \[
    \exists e
    \left(
      \begin{array}{l}
        \everb{reason} \\
        \ein{\exists e'(\everb[e']{cry} \land \eagent[e']{Caesar})} \\
        \eout{\text{Ought}
        \left(\exists e''\left(
        \begin{array}{l}
          \everb[e'']{hug} \\
          \eagent[e'']{Brutus} \\
          \epatient[e'']{Caesar}
        \end{array}
        \right)
        \right)
        } \\
        \eagent{Brutus} \\
        \emethod{Categorical imperative}
      \end{array}
    \right)
  \]
  \begin{itemize}
  \item Function can have multiple methods, may be no relevant characterisation of method, characterisation of method may be too coarse grained to specify a particular function.
  \item For the most part, ignore method.
  \end{itemize}
}

\[
  \exists e(\text{reason}(e) \land \ein{\Sigma} \land \eout{\chi} \land \exists f(\efunc{f}))
\]

\begin{itemize}
\item Reasoning is understood as an event, and to abstract to a function which relates the input to the output.
\item Functions to abstractly capture reasoning.
\item Use and mention of a function.
\end{itemize}

\hozlinedash

\begin{itemize}
\item This is a simple view, and things can get messy.
  Especially if dealing with non-monotonic or ecological reasoning, where there may be no easy way to exhaustively capture the ins and outs of reasoning as a process.
\end{itemize}

\hozlinedash

\subsection{Key inference}
\label{sec:key-inference}

\begin{itemize}
\item Lewis is confident that Morse has reasoned from the case file to the guilt of Woodthorpe.
\item Lewis has arrested Woodthorpe, and Lewis has performed this arrest because Lewis is confident that Woodthorpe is guilty.
\item Still, Lewis has not reasoned from the case file to the guilt of Woodthorpe.
\end{itemize}

\begin{itemize}
\item Lewis' confidence that Woodthorpe is guilty is important.
\item Morse has no provided testimony that Woodthorpe is guilty, and we are assuming that Lewis does not coerce Morse's report on their reasoning as testimony.
\end{itemize}

\begin{itemize}
\item How is it the case that Lewis is confident that Woodthorpe is guilty?
\end{itemize}

\begin{itemize}
\item The event of Morse reasoning from the case file to the guilt of Woodthorpe establishes the existence of a function which captures the possibility of reasoning from the case file to the guilt of Woodthorpe.
\item As Lewis is confident that Morse and Lewis are equally matched, this is a function that Lewis can witness.
  \begin{itemize}
  \item Function is an abstract characterisation of Morse's reasoning, and at a certain level of generality Morse's reasoning is a particular but not exclusive witness.
  \end{itemize}
\item Hence, Lewis can be confident that Lewis \emph{is able} reason from the case file to the guilt of Woodthorpe.
\item And, if Lewis can reason, then there's an appropriate relation of support that Lewis is able to draw on.
\item This is the key.
\end{itemize}

\hozlinedash

\begin{itemize}
\item Lewis is confident that there is the appropriate relation of propositional support, and Lewis is confident that this can be transformed into doxastic support.
\end{itemize}

Propositional and doxastic support are non-committal versions of propositional and doxastic justification.
And agent has propositional justification for \(\phi\) is the agent having justification to believe \(\phi\), while and agent having doxastic justification for \(\phi\) is an agent justifiably believing \(\phi\).
Propositional justification does not require belief.

Contested.

Morse's reasoning from the case file to the guilt of Woodthorpe establishes that there is a relation of support between the case file and the guilt of Woodthorpe which holds independently of Morse's reasoning.
Morse's reasoning demonstrates what that relation of support is.

Even so, learn from the literature on propositional and doxastic justification that an independent relation of support between the case file and the guilt of Woodthorpe does not entail that Lewis can draw on the independent relation and confidence that relation exists to be confident that Woodthorpe is guilty.

\textcite{Turri:2010aa} discusses certain kinds of cases in which agents share the same {\color{red} epistemic reasons} and form the same belief, but do so in different ways.
To adopt \citeauthor{Turri:2010aa}'s example, if Lewis were to reason from thee case file to the guilt of Woodthorpe due to tea leaves making it overwhelmingly likely that Woodthorpe is guilty, it seems Lewis' confidence would not be supported by the case file.

\begin{itemize}
\item This is where the puzzle kicks in.
\item It doesn't seems as though Lewis is in water for being confident.
\item So, how is this explained?
\end{itemize}

Pair of examples which share the structure as the Morse and Lewis scenario help clarify.

\hozlinedash

\subsection{Practical contamination}
\label{sec:pract-cont}

\begin{itemize}
\item The argument here is that things are different from a practical point of view.
\item Winston and Julia is a little complex, and perhaps unintuitive, but I'm fond of the example.
\item To make the point a little clearer, introduce instructor scenario.
\item Suspect this can be exploited.
\end{itemize}

\hozlinedash

\begin{scenario}[Winston \& Julia]
  Winston works on a farm where visitors can come to pet animals.
  While looking through the bookings, Winston notices that Eric Blair is due to visit on the tenth of January.
  Winston says to Julia ``Eric Blair is due to visit soon\dots'', and after a pause adds, ``\dots and were you aware that `Eric Blair' is the actual name of `George Orwell'''.
  Julia replies ``No! But if that's true then it means George Orwell will be visiting us''.
  Winston reflects on Julia's reply, and comes to be confident that George Orwell is due to visit.
\end{scenario}

Winston is confident that Julia reasoned from the propositions that Eric Blair is due to visit and that `Eric Blair' and `George Orwell' are co-referential to the conclusion that George Orwell is due to visit.
And, on the basis of Winston's confidence that Eric Blair is due to visit and that `Eric Blair' and `George Orwell' are co-referential and Julia's reasoning from these propositions to the conclusion that George Orwell is due to visit, Winston is confident that George Orwell is due to visit.

Still, something is amiss.
The substitution of co-referring names in a referentially transparent proposition is a straightforward piece of reasoning.
And, Winston's reliance of Julia's reasoning suggests that Winston has not grasped the basics of referential transparency, and hence Winston is unable to replicate Julia's reasoning themselves.
Therefore, Winston's confidence that George Orwell will be visiting does not seem to be supported on the basis of Winston being confident that that Eric Blair is due to visit and that `Eric Blair' and `George Orwell' are co-referential.

\hozlinedash

\begin{itemize}
\item The role of Julia's reasoning from these propositions to the conclusion that George Orwell is due to visit.
\end{itemize}

\begin{itemize}
\item What is Winston reasons with:
  \begin{itemize}
  \item \(Vo, o = e, \exists f(f\{Vo, o = e\} = Ve)\).
  \end{itemize}
\item In this case, Winston is certainly not the sharpest, but is their confidence that \(Ve\) fine?
\item The difference is that I've added confidence that it is possible to reason to the mix.
\item Intuitively, there is nothing wrong with Winston's \emph{method} of reasoning.
\item And, might think that this is fine because there's no need for Winston to reason from \(Vo, o = e\).
\item That is, \(Ve\) is independent of Winston's reasoning from \(Vo, o = e\).
\item So, Winston can't be confident based only on their confidence, but the fact that they're confident that it's possible to reason is an additional consideration for Winston that they're able to make use of.
\end{itemize}

\begin{itemize}
\item So, if someone were to ask Winston, they'd appeal to Julia's reasoning.
\end{itemize}

\begin{itemize}
\item Suppose Julia says something like ``if that is so, then you can be confident that Eric Blair will be visiting us.''
\end{itemize}

\begin{itemize}
\item Now it seems as though Winston is in trouble.
\item For, Julia has informed Winston that Winston is able to reason from \(Vo\) and \(o = e\) to \(Ve\).
\item However, Winston doesn't see why this is the case.
\item So, it looks as though Julia is telling Winston that they can reason to \(Ve\) from \(\{Vo, o = e\}\).
\item And the problem for Winston is that they aren't able to do this.
\item So, Winston can't be confident.
\item The problem is that Julia has said something false, it seems.
\item Julia has overestimated Winston's ability.
  \begin{itemize}
  \item \(\rightarrow\) This is the important distinction between Julia's two utterances.
  \end{itemize}
\item It's true that it is possible to reason, but it is false that it is possible for \emph{Winston} to reason\dots
\item So, if this is the case, then there's a problem with the general inference from there is some reasoning\dots
\item This shouldn't be too surprising, as dependence is common, and the difference comes up in other instance of action.
\end{itemize}

\begin{itemize}
\item Could talk about different kinds of attitudes.
\item Easier to talk about confidence and it's support.
\item What the confidence depends on.
\end{itemize}

\begin{itemize}
\item Then there's the contrasting case.
\item Can do the reasoning, and so attitude depends.
\item Perhaps also include desire to illustrate, as here dependence does seem important.
\end{itemize}

\begin{itemize}
\item It is only in the case of Julia's dependency statement that the scenario parallels the kind of scenarios that \citeauthor{Worsnip:2018aa} deals with.
\end{itemize}

\hozlinedash

\begin{scenario}[Carmen \& Lawrence ({\color{red} Amelia, Richard})]
  Carmen and Lawrence have travelled to Estonia.
  Lawrence looks at the weather forecast in the morning newspaper, and the midday temperature is reported to be around 10 degrees Celsius.

  ``It says here the high will be around 10 Celsius today \dots'', Lawrence says to Carmen, ``\dots and if I multiply by 1.8 and add 32\dots''.

  ``If so \dots'' Carmen adds, ``\dots it won't get much warmer than 40 Fahrenheit''.

  Lawrence frowns and adds: ``It seems we'll need to buy warmer clothes.''
\end{scenario}

\begin{itemize}
\item Similar situation.
\item Carmen doesn't state that the whether will be 40 Fahrenheit.
\item Can assume that Carmen is not aware of the conversion, and hence that Carmen simply reports the result of the process of reasoning that Lawrence was about to perform.
\item Does some reasoning and reports the result to Lawrence.
\item Lawrence understands the reasoning, and is confident that the temperature will be around 40 Fahrenheit.
\item It is clear that Lawrence could have done the reasoning, but Carmen was too quick.
\item In contrast to Winston \& Julia, Lawrence has stated an important part of how they would reason to the temperature in Fahrenheit.
\item May be the case that Lawrence's confidence that it will be 40 Fahrenheit is supported on the basis of Lawrence's confidence that it will be 20 Celsius.
\end{itemize}

\begin{itemize}
\item There's an importance difference between the two scenarios, as in one a basic pattern of reasoning is missing.
\item In the other, it is stated.
\item Lewis is likely located between these two extremes.
\item Some guess as the kind of reasoning Morse has done.
\item Lewis' confidence that they are on par with Morse splits this difference.
  Unlike Winston, Lewis does not need to improve their reasoning to demonstrate the guilt of Woodthorpe.
  Unlike Lawrence, Lewis does not have a grasp on the particular way of reasoning from the case file to the guilt of Woodthorpe.
  Like Lawrence, Lewis is confident that they can do the reasoning.
  Like Winston, Lewis is unsure of how the guilt of Woodthorpe can be demonstrated from the case file.
\end{itemize}

\citeauthor{Turri:2010aa}'s proposed analysis of propositional justification:\nolinebreak
\footnote{Not assuming that \citeauthor{Turri:2010aa}'s proposal is correct.
  However, it's useful in linking propositional and doxastic support.
  If propositional support does not require doxastic support, then whatever else it is that captures the difference\dots
}

\begin{quote}
  Necessarily, for all \(S\), \(p\), and \(t\), if \(p\) is propositionally justified for \(S\) at \(t\), then \(p\) is propositionally justified for \(S\) at \(t\) because \(S\) currently possesses at least one means of coming to believe \(p\) such that, were \(S\) to believe \(p\) in one of those ways, \(S\)'s belief would thereby be doxastically justified.\nolinebreak
  \mbox{}\hfill\mbox{(\citeyear[320]{Turri:2010aa})}
\end{quote}

Lawrence satisfied \citeauthor{Turri:2010aa}'s requirement for propositional justification.
Winston does not satisfy \citeauthor{Turri:2010aa}'s requirement for propositional justification.


This is a necessary condition.
So, can only answer whether Lewis satisfies the necessary requirements for (\citeauthor{Turri:2010aa}'s analysis) of propositional justification.
And, it seems Lewis does, at least from Lewis' perspective.
For, assume that Morse's belief that Woodthorpe is guilty is doxastically justified on the basis of Morse's reasoning from the case file to the guilt of Woodthorpe.
Then, as Morse and Lewis are equally matched, Morse reasoning from the case file to the guilt of Woodthorpe underwrites Lewis' ability to reason from the case file to the guilt of Woodthorpe.
Hence, Lewis possesses the same means as Morse, and were Lewis to reason in that way, Lewis would be doxastically justified.

The slight snag is that restricted to Lewis' perspective because we have no stated whether or not Lewis and Morse are equally matched.
This doesn't matter.
We are interested in Lewis' reasoning, not whether Lewis is correct that Woodthorpe is guilty.

\hozlinedash

\begin{itemize}
\item Lewis makes the move that Carrol's tortoise refuses to make.
\item ``the permissibility of the transition from p to q depends on the existence of a licence specifying that such a transition is indeed permissible'' (\cite[459]{Simchen:2001aa})
\end{itemize}

\hozlinedash

\paragraph{Summarising propositional but not doxastic}

\begin{itemize}
\item Lewis is confident of the status of the evidence in the case file.
\item Lewis cites this if questioned.
\item Suggestion is that Morse's reasoning from the case file to the guilt of Woodthorpe traces a relation between the case file and the guilt of Woodthorpe that is independent of Morse's reasoning for the case file to the guilt of Woodthorpe.
\item Relation of propositional support.
\item Morse's reasoning establishes doxastic support.
\item So, Lewis' confidence in the case file ensures that Lewis has propositional support for the guilt of Woodthorpe.
\item Without reasoning from the case file, Lewis does not have doxastic support on the basis of the case file.
\end{itemize}

\newpage

\subsection{The difficulty}
\label{sec:difficulty}

\begin{itemize}
\item \emph{Practical contamination}
\item There are different intuitions.
\end{itemize}




\begin{itemize}
\item The basic idea is that Lewis settles the guilt of Woodthorpe on the basis of Lewis' confidence that the case file constitutes evidence.
\item Morse's message is an important part of \emph{how} Lewis settles, but it is not \emph{why} Lewis settles.
\end{itemize}

\begin{itemize}
\item The problem is that Lewis hasn't done the reasoning, so Lewis cannot demonstrate why the case file settles Woodthorpe's guilt.
\item However, Lewis is confident that they can do the reasoning.
\end{itemize}

\begin{itemize}
\item To get the intuition, the reason why Lewis arrests Woodthorpe are those reasons that Lewis will write in the arrest report.
\item Morse's message is no good.
\item Morse's message is no different from Lewis' statement that there is a demonstration.
\item If that's all that is in the report, then Woodthorpe will be set free.
\item The problem is that Lewis doesn't yet have a grasp of what those reasons are.
\item This is the difficulty.
\item If the reasons why are constrained by how, then there is no way out.
\item I think it is plausible that in the case of ideal agents, this is true.
\item For agents like us, I think this is false.
\end{itemize}

\begin{itemize}
\item My model for this is fairly straightforward, as the agent stipulates some function, and may at some point demonstrate what this function is.
\end{itemize}

\hozlinedash

\begin{itemize}
\item If \(\exists f(f\Sigma = \chi), \Sigma\) is sufficient, then Winston is fine.
\item Winston doesn't seem fine, and the suggestion is that this is because Winston is unable to do the reasoning.
\end{itemize}

\hozline

\begin{itemize}
\item Lewis needs to demonstration.
\item The core question is how the demonstration relates to Lewis' reasoning.
\item The intuition is that Lewis should not be making an arrest without a demonstration in hand, so to speak.
\item Or, that Lewis' arrest of Woodthorpe is `premised' on providing a demonstration.
  \begin{itemize}
  \item Non-constructive reasoning premised on constructive reasoning.
  \end{itemize}
\item On the one hand, there's the incorporeal approach.
  \begin{itemize}
  \item Here, Lewis reflects on propositional justification, and adds something to this.
  \end{itemize}
\item On the other hand, there's the speculative approach.
  \begin{itemize}
  \item Here, Lewis speculates that Woodthorpe is guilty based on reasoning that they are confident that they will perform.
  \end{itemize}
\item The difference is, roughly, completeness.
  \begin{itemize}
  \item On the incorporeal approach, Lewis' confidence that Woodthorpe is guilty is a complete piece of reasoning.
    \begin{itemize}
    \item Lewis then guarantees that they'll provide a demonstration of the guilt of Woodthorpe from the case file.
    \item Lewis \emph{assures} that a demonstration will be provided.
    \end{itemize}
  \item On the speculative approach, Lewis' confidence that Woodthorpe is guilt is an incomplete piece of reasoning.
    \begin{itemize}
    \item For, Lewis' attitude toward the guilt of Woodthorpe is premised on a demonstration.
    \item Lewis takes for granted that they will provide a demonstration.
    \item Lewis \emph{assumes} that a demonstration will be provided.
    \end{itemize}
  \end{itemize}
\end{itemize}

\begin{itemize}
\item What is the difference between \emph{assuring} and \emph{assuming}?
\item {\color{red}
    These terms aren't ideal.
    For an assumption doesn't in general entail an assurance.
    However, I am assuming that this is the case.
  }
\item Well, and assumption requires an assurance.
  \begin{itemize}
  \item At least, in the cases of interest.
  \item For, in effect Lewis is taking for granted the result of what they assure.
  \item {\color{red}
      Aha!
      The assumption is like an additional step.
      Lewis can assure, but can also make the additional step of assuming the result of the assurance.
    }
  \end{itemize}
\end{itemize}

\begin{itemize}
\item The two options are:
  \begin{enumerate}
  \item Arrest because high confidence that Woodthorpe's guilt will be demonstrated.
  \item Arrest because unlikely that assumption that Woodthorpe's guilt will be demonstrated is mistaken.
  \end{enumerate}
\item It is easy to switch between one and the other.
\item The difference is the assumption \dots
\item If Lewis makes the assumption then Lewis presupposes providing a demonstration.
\end{itemize}


\hozlinedash

\begin{itemize}
\item An analogy is credit.
\item I do not have the funds to purchase a washing machine.
\item However, I will be paid at the end of the month.
\item So, I use the credit that I have to purchase.
\item I have not paid until the end of the month.
\item Bank guarantees my ability to pay.
\end{itemize}

\hozlinedash

\begin{itemize}
\item The existence of propositional support can be seen as a premise in Lewis' reasoning.
\item {\color{red} This is the inverse of the types of cases \citeauthor{Worsnip:2018aa} considers\dots}
\item \(\exists f(f\Sigma = \chi), \Sigma \vdash \chi\).
\item The reasoning is similar to the more familiar case of reasoning with existentially quantified objects in a first order setting.
\item If there is something that is not alive.
\item And, if everything that is mean spirited is alive.
\item Then, there is something that is not mean spirited.
\item \(\exists x \lnot Ax, \forall x (Mx \rightarrow Ax) \vdash \exists x \lnot Mx\).
\end{itemize}

\begin{itemize}
\item Indirect application of the function.
\item Incorporeal.
\end{itemize}

\begin{itemize}
\item In short, Lewis can be confident because Morse reasoned.
\end{itemize}

\begin{itemize}
\item Pause here.
\end{itemize}

\hozlinedash

\begin{itemize}
\item Lewis goes and does the reasoning.
\item Now there are two witnesses, one either side of Lewis' arrest.
\item Asymmetry, different roles.
\item Still, for the same reasoning; the demonstration of the guilt of Woodthorpe.
\item Lewis' future reasoning does not reach back in time to provide the witness required for confidence.
\item Morse's reasoning does not provide Lewis with a demonstration.
\end{itemize}

\begin{itemize}
\item Reasoning.
\item And modelling.
\item Have the problem, but unlikely that there is a unique solution.
\item Introduce a proposal, and see how well it works.
\end{itemize}

\begin{itemize}
\item Lewis can commit to reasoning from the case file to the guilt of Woodthorpe at some future point in time.
\item In short, Lewis can \emph{promise} an demonstration.
\item {\color{red} \dots}
  \begin{itemize}
  \item A commitment that Lewis will do the reasoning.
  \end{itemize}
\item This is problematic for the proposed analysis because Lewis' reasoning does not depend on any particular instance of the function.
\item As things stand, Lewis' reasoning is complete.
  \begin{itemize}
  \item The reasoning does not depend on a future witness.
  \item The idea is that Lewis' reasoning depends on a specific function.
  \item Specifically, the one that will be witnessed in Lewis' future reasoning.
  \item Lewis doesn't discharge the function.
  \item Morse --(propositional)--> Arrest <--(doxastic)-- Lewis
  \end{itemize}
\item {\color{red} Note, this is a precondition for Lewis writing out the report\dots}
\end{itemize}

\begin{note}[Core problem]
  Morse's reasoning has limited use.

  Morse's reasoning establishes propositional support.
  However, propositional support is limited.
  Maybe, but this is a hard argument to make.

  Alternative is to argue that Lewis wants doxastic support.
  And, issue is whether this is a separate requirement, or whether it's linked to Lewis' reasoning.

  ``I arrested Woodthorpe on the basis of being able to reason from the case file to the guilt of Woodthorpe, and I need to fill in what that reasoning is.''
  ``Here is how the guilt of Woodthorpe was supported by the case file.''
\end{note}

\begin{itemize}
\item What's important is that the reasoning can be done.
\item {\color{red} The existence of propositional support}, from a broader perspective.
\item That Morse has done the reasoning is what underwrites this, along with the parity, but it is the existence and possible witness that matters.
  \begin{itemize}
  \item So, difference between me telling you that there are exactly seven intermediate logic systems with a certain property, and then informing you of a proof of this, as an example of how this makes a difference.
  \end{itemize}
\item The key is a witness.
\item {\color{red} Question of the witness\dots}
  \begin{itemize}
  \item If the guarantee of a witness if what matters .
  \item Then, as Lewis is confident that they are able to do the reasoning, Lewis is confident that they are able to provide a witnessing piece of reasoning.
  \item Build on the existential object example, I'm confident that I can find some non living thing, so I call it Casper.
    And, I will, eventually, have a friendly ghost.
    Reference De dicto reference and de re.
    Example, Scully does not believe that there is something which is not mean spirited.
    Scully does not believe that Casper exists.
    Hm, something better than this is needed.
  \end{itemize}
\item Seems important that Lewis can reason.
  \begin{itemize}
  \item Generalising from confidence that Woodthorpe is guilty, it may be important that Lewis is able to demonstrate that Woodthorpe is guilty.
  \item Lewis will need to write a report.
  \item Investigate the idea that Lewis' confidence is premised on their own reasoning.
  \item Reasoning about Casper is premised on reference de re.
  \item Key difference is that Lewis' reasoning is incomplete.
  \end{itemize}
\item Suggests that Lewis could premise on their own reasoning.
\item Main object of interest.
\item Future.
\end{itemize}


\begin{itemize}
\item There are cases in which problems arise.
\item Morse doesn't reason to the guilt of Woodthorpe.
\item Lewis has a vendetta.
\item Lewis arrests Woodthorpe, convinced that they'll be able to find evidence, etc.
\end{itemize}

\[
  \exists e
  \left(
    \begin{array}{l}
      \everb{reason} \land
      \ein{\text{Case file}} \land
      \eout{\text{G}(w)} \land
      \eagentm{m}
    \end{array}
  \right)
\]

\[
  \lnot\exists e
  \left(
    \begin{array}{l}
      \everb{reason} \land
      \ein{\text{Case file}} \land
      \eout{\text{G}(w)} \land
      \eagentm{l}
    \end{array}
  \right)
\]

\begin{enumerate}
\item Because of the fact that Morse reasoned.
\item Because Lewis will reason.
\end{enumerate}


\begin{itemize}
\item \(\exists f(f\Sigma = \chi), \Sigma\), hence \(\chi\).
\item \(\chi\) is obtained by an application of the function to \(\Sigma\).
\item The function captures the fact that Morse reasoned from the case file to the guilt of Woodthorpe.
\item So, by appealing to the witnessing instance, Lewis appeals to the relation of propositional support which holds between the case file and the guilt of Woodthorpe.
\item If Woodthorpe is guilty, then Woodthorpe is guilty \emph{independently} of any demonstration of Woodthorpe's guilt from the case file.
\item Lewis is confident that Woodthorpe is guilty, but it's the ability to demonstrate Woodthorpe's guilt that allows Lewis to make an arrest.
\item Demonstration of Woodthorpe's guilt is obtained by an application of the function.
  \begin{itemize}
  \item Lewis is unable to provide additional information about how the function is witnessed, but this doesn't prevent Lewis' use of the function.
  \item {\color{red} This is difficult.}
    \begin{itemize}
    \item Lewis uses the fact that Morse reasoned to indirectly reason from the case file to Woodthorpe's guilt by the same relation of support that Morse's reasoning appealed to.
    \item Lewis, if pressed, could say they are confident that the case file demonstrates Woodthorpe's guilty, though they cannot at present provide a demonstration, but Morse's reasoning witnesses the demonstration.
      Along with a note about how Lewis did not (yet) have the opportunity to themselves demonstrate.
      No, Morse did not tell Lewis that Woodthorpe is guilty, but Morse did not need to, for the fact that Morse reasoned ensures a the possibility of a demonstration, and the ability to provide a demonstration of guilt is sufficient to arrest.
      If one also needed the demonstration, then a whole bunch of people should not have been arrested.
      The lack of direct testimony does not prevent Lewis from appealing to the case file that Lewis has and the testimony that one can reason from this to the guilt of Woodthorpe.
    \end{itemize}
  \end{itemize}
\item \(\exists f(f\Sigma = \chi) \approx \Sigma \Rightarrow \chi\), returning to propositional logic
\item And, \(\Sigma, \Sigma \Rightarrow \chi \vdash \chi\).
\item If the evidence in the case file is sound, then Woodthorpe is guilty.
\item On the one hand, the appeal to an existential may be seen as an artefact of how we understand reasoning.
  On the other hand, the use of a conditional obscures the fact that Lewis is drawing on Morse's reasoning.
\item Regardless, Lewis appeals to more than the relation of propositional support that holds between the case file and the guilt of Woodthorpe.
\item Lewis appeals to the process of Morse's reasoning as a witness.
\end{itemize}

\begin{itemize}
\item In both cases, indirect application of the function.
\item \(\exists f(f\Sigma = \chi), \Sigma\), hence \(\chi\).
\item Lewis is confident that there is an appropriate witness, which is part of why this works.
\item If Morse was known to read tea leaves, then the function would still exist, but it would not have relevant witness.
\item Likewise, Morse's confidence in the evidence doesn't really matter, and Morse may have explicitly reported only on their reasoning \emph{because} they did not have an appropriate attitude toward the case file.
  \begin{itemize}
  \item Example here with buying something, and the assistant giving a recommendation based on reported information.
  \item Assistant gives a conditional, because they're only sure that the item is suitable given the information provided, and cannot guarantee that the recommendation would hold given additional information that may have not been stated.
  \end{itemize}
\item These aspects aren't captured in the bare existential quantification, instead how Lewis' reasoned to the existential.
\item So, Lewis obtains their attitude toward the guilt of Woodthorpe by the function.
\item And, Lewis can do this because they the relevant in-out pair of the function.
\item This allows taking an arbitrary function, and doing a standard piece of reasoning.
\item This is blocked if the function appears in the conclusion of the reasoning.
  \begin{itemize}
  \item \(\exists f(f\Sigma = f\Sigma')\), \(\Sigma\), but Lewis can't draw anything from this.
  \item Well, a useless example if \(\exists f(f\Sigma = f\Sigma)\).
    Here, Morse tells Lewis that they've reasoned to a conclusion, but Lewis can't done anything with this.
    Lewis is confident that \(\exists f(f\Sigma = f\Sigma)\), but there is not function \(f\) such that Lewis is confident that \(f\Sigma\).
  \end{itemize}
\end{itemize}

\begin{note}
  Case in which the result depends on the function.

  Romeo came to believe that Juliet 

  A controversial reading of (a kind of) belief may work.
  Romeo, believing Juliet to be alive, discarded the poison.
  \[
    \exists e(\everb{reason} \land \ein{\phi} \land \eout{b\phi} \land \efunc{b} \land \eagent{a})
  \]

  Romeo's reasoning from believing that Juliet is dead to taking poison.\nolinebreak
  \footnote{
  \[
    \exists e
    \left(
      \begin{array}{l}
        \everb{reason} \\
        \ein{
        \exists e'
        \left(
        \begin{array}{l}
          \everb[e']{reason} \\
          \ein[e']{\text{Alive}(\text{Juliet})} \\
          \eout[e']{b(\text{Alive}(\text{Juliet}))} \\
          \efunc[e']{b} \\
          \eagent[e']{Romeo}
        \end{array}
        \right)
        } \\
        \eout{
        \text{Ought}
        \left(
        \exists e''
        \left(
        \begin{array}{l}
          \everb[e'']{discard} \\
          \eagent[e'']{Romeo} \\
          \epatient[e'']{poison}
        \end{array}
        \right)
        \right)
        } \\
        \eagent{Romeo}
      \end{array}
    \right)
  \]
}
  Here, believing is a process which involves some kind of reasoning.
  The input of the process is some proposition \(\phi\), and the output of the process is that the relevant kind of reasoning is occurring.
  Hence, \(b\phi = b\phi\).
  To believe \(\phi\) is just for some relevant kind of reason to occur.
  So, to say that \(a\) believes \(\phi\) is just to say that \(a\) is engaged in some reasoning about \(\phi\) in some particular way.
  Here, understand \(\phi\) as interpreted.
  Now generalise to attitudes, and this is a plausible non-trivial case in which the output depends on the function.
\end{note}


Lewis is confident that Morse's reasoning from the case file to the guilt of Woodthorpe happened.
Hence, Lewis is confident that some reasoning from the case file demonstrates the guilt of Woodthorpe.
The key feature of Morse's reasoning is that the guilt of Woodthorpe does not depend on Morse's reasoning.
If Woodthorpe is guilty, then Woodthorpe is guilty \emph{indenedently} of whether Morse reasons from the case file to Woodthorpe's guilt.
In this sense Morse's reasoning demonstrates Woodthorpe's guilt.

Not all reasoning demonstrates a conclusion that is independent of reasoning.
For example, in the process of reasoning from the case file Morse may have reflected on the activity they were engaged in and noted that they were demonstrating the guilt of Woodthorpe.
Here, the intermediate conclusion that Morse is demonstrating the guilt of Woodthorpe \emph{depends} on the activity that Morse is engaged in.

\begin{note}[Observation]
There's an important difference between the mere existential and the use of a future.
\end{note}

\begin{note}
  This relation is clear in the case of propositional logic and decidability.
  \(\Phi \vdash \psi\) can be read as stating that there is a relation, or that there is a function.

  Standard relation to think about is propositional justification.
  So, \(\Sigma\) and \(\chi\) stand in the relation of propositional justification.

  Well, function does more than stating the relation, and given a relation is doesn't always seem to follow that there's a function.
  At least, not a function of the appropriate kind.
  I mean, there's the incompleteness results which demonstrate this, and this is part of the reason why soundness and completeness results are so nice.

  So, function as the agent is confident that they can \emph{obtain} \(\chi\) from \(\Sigma\).
  However, the agent does not need to \emph{construct} \(\chi\) from \(\Sigma\).
  Any doxastic support is seen as a byproduct of applying the function, or at least \emph{a} kind of doxastic support can be seen in this way.
\end{note}


\begin{itemize}
\item Before going into the details of reasoning, there's a key inference.
\end{itemize}

\[
\exists f(f\Sigma = \chi), \Sigma \vDash \chi
\]

So long as \(\chi\) does not depend on the application of some function, then so long as one is sure of the existence of a function and some key-value pair of that function, then given the key one can also be sure of the value.

\begin{itemize}
\item Simple in the case of the use of functions in logic, and Skolemization.
  \begin{itemize}
  \item {\color{red} Stress the idea of `Skolemization' here, as this does not apply to futures and promises.}
  \item E.g.\ key inference relied on idea\dots and this does not apply to futures, in particular because a claim about a future is often a claim about a process.
  \end{itemize}
\item Extends to mathematics in a straightforward way.
\item And to other instances of reasoning.
\end{itemize}

\begin{itemize}
\item In the case of Lewis, the contrast is between Woodthorpe's guilt and the demonstration of Woodthorpe's guilt.
\item Lewis is confident that Woodthorpe is guilty, though they are not confident that they have demonstrated Woodthorpe's guilt.
\item The lack of demonstration is important, and it is also important to highlight that Lewis is not necessarily certain of Woodthorpe's guilt, hence nor the possibility of demonstrating Woodthorpe's guilt.
\item However, confidence that Woodthorpe is guilty is all that is needed to motivate an arrest.
  \begin{itemize}
  \item To see that the demonstration is not necessary for Lewis, run the same scenario with explicit testimony from Morse that Woodthorpe is guilty.
  \end{itemize}
\item One way to understand this is in terms of propositional support.
  \begin{itemize}
  \item If Morse has demonstrated the guilt of Woodthorpe from the case file, then Morse has doxastic support.
  \item In turn, there is propositional support.
  \item The function then relies of propositional support, and Lewis doesn't have doxastic support.
  \end{itemize}
\end{itemize}

\hozlinedash

\begin{itemize}
\item This assumes that the agent is confident that \(\exists f(f\Sigma = \chi)\).
\item Haven't yet shown why this is the case.
\item Quick answer in the case of Lewis is that \(\Sigma\) and \(\chi\) are the ins and outs of Morse's reasoning.
\item However, Morse's reasoning is some process, while the function is something witnessed by the process.
  \begin{itemize}
  \item Problem is that these a potentially two different types of things.
  \item Possible to argue for a reduction, but this isn't clear.
  \item Consider logic cases, where there are soundness and completeness theorems.
    Have semantic entailment, so there's some syntactic entailment, but some work needs to be done to show that this can be witnessed by some reasoning.
    For, it is possible that the syntactic proof requires resources beyond the reach of agents like us.
    Hence, may need to consider reasoning being witnessed by idealised agents, and so on.
  \item This does not rule out reduction, but to avoid taking on unnecessary burdens, function may be of a distinct kind.
  \item Further complications with method of reasoning, i.e.\ algorithm, as this too may be something distinct.
  \end{itemize}
\end{itemize}

\subsubsection{Two different types of reasoning, briefly characterised}
\label{sec:two-different-types}

\begin{figure}[h]
  \begin{subfigure}{.5\textwidth}
    \centering
    \begin{tikzpicture}[
      ->,
      >=stealth',
      % auto,
      node distance=0cm, every text node part/.style={align=center},
      ]

      \node [] (c) at (0,0) {};
      \node [] (d) at (-3,0) {};
      \node [] (e) at (3,0) {};
      \node [] (f) at (0,-2.1) {};

      \node (1) at (0,-.1) {\(\{\exists f (f\futin = \futout)\} \cup \futin\)};
      \node (2) at (0,-2) {\(\futout\)};

      \draw [->] (1.270) to [] node[left] {\(g\)} (2.90);
    \end{tikzpicture}
    \caption{Incorporeal \\ Gather up the resources along with the existential.}
    \label{fig:non-constructive}
  \end{subfigure}
  % \hfill
  \begin{subfigure}{.5\textwidth}
    \centering
    \begin{tikzpicture}[
      ->,
      >=stealth',
  % auto,
      node distance=0cm, every text node part/.style={align=center},
      ]

      \node [] (c) at (0,0) {};
      \node [] (d) at (-3,0) {};
      \node [] (e) at (3,0) {};
      \node [] (f) at (0,-2.1) {};

      \node (1) at (0,-.1) {\(\futin\)};
      \node (2) at (0,-2) {\(\futout\)};

      \node (x) at (-2,-1.05) {\(\exists f(f\futin = \futout)\)};

      \draw [->, dashed] (1.270) to node[left] (3) {\(\future{f}\)} (2.90);

      \node (4) [left of=3, xshift=-2cm] {}; % {\(\exists f (f\Phi = \psi)\)};
      \draw [-{Circle[open]}] (x.0) to (3.180);

    \end{tikzpicture}
    \caption{Speculative \\ Use the existential to secure a promise of a witness.}
    \label{fig:speculative}
  \end{subfigure}
\end{figure}

\begin{itemize}
\item Incorporeal:
  \begin{itemize}
  \item Lewis does some reasoning on the basis of there existing some reasoning which demonstrates the support for arresting Woodthorpe.
  \end{itemize}
\item Speculative:
  \begin{itemize}
  \item Lewis takes there to be support for arresting Woodthorpe on the basis of reasoning that Lewis hasn't yet done.
  \item Lewis \emph{uses} the function but defers the demonstration to some future reasoning.
    Hence, Lewis' reasoning is `incomplete'.
  \end{itemize}
\item The key idea with both types of reasoning is that the output of the function does not depend on the reasoning.
  Hence, one can, in principle, obtain \(\futout\) from \(\futin\), as explained.
  \begin{itemize}
  \item This is important to emphasise, as it rules out certain cases.
  \item Still, examples of where the output \emph{depends} on reasoning are hard to come by.
  \item Self-referential cases can be found, e.g.\ belief that one has reasoned to \(\futout\).
  \item For the most part, however, the ins and outs are not usually restricted in this way.
  \item May think that there's also a restriction on the attitudes that the agent may have to the output, though.
  \item This is plausible, but substantive.
    \begin{itemize}
    \item \citeauthor{Sinhababu:2017aa} gives an example with desire, and there may be a case for belief somewhere\dots
    \item As a rough heuristic, think of what can be done on the basis of testimony, as here one doesn't have access to the reasoning, and any case of testimony which details the ins and outs and makes a claim to reasoning can be recast in these terms.
    \end{itemize}
  \end{itemize}
\end{itemize}

\hozline


\subsection{Promises and futures}
\label{sec:promises-futures}

\begin{itemize}
\item Promises reverse the extraction of the function with argument and value specified.
\item Agent promises some function, and there is some process that conforms to the function.
\item This isn't so straightforward.
  \begin{itemize}
  \item For, given how things have been set up, processes, algorithms, and functions could all be promised.
  \item And, given that functions may be abstract key-value pairs, it's not necessarily going to be useful to provide a referent for the function used.
  \end{itemize}
\end{itemize}

\begin{note}[Promise]
  A promise is a characteristic function which may have an indeterminate truth value.
  The key idea is that a promise sets out a bunch of conditions that any argument must satisfy in order for the function to be true.
  So, the task is to specify something for which the function returns true.
  And, if the function returns true, then the argument is a way to fulfil the promise.

  They key idea is that if there's some reasoning that the agent is confident of, then we have something of the form

  \[\exists e(\Phi(e))\]

  And from this one can create the promise

  \[\text{promise}(e)\]

  Where

  \[\text{promise}(e) = \top \leftrightarrow \Phi(e)\]

  And, in the cases of interest there may be additional restrictions

  \[\text{promise}(e) = \top \leftrightarrow \Phi(e) \land \Psi(e)\]

  Hum, and alternative is to return the argument is it satisfies the constraints, so now the promise looks a whole lot like a test.
  Yeah, this is really nice, I think.


  \[\text{promise}(e) = e \leftrightarrow \Phi(e) \land \Psi(e)\]

  And then

  \[\exists e(\cdots \land \text{promise}(e) = e \land \cdots)\]

  This is a useful way to think about promises, as the standard idea with a test is that you run the test, and this allows one to continue or not, here, and here the idea is the same, it's simply a test, and the equality holds if the test passes.
  However, the additional idea is to project a future from the test.
  So, this is what we would have if the test is satisfied.

  \[\text{future}(\text{promise}(\dot{e})) = \ddot{e}\]
\end{note}

\begin{note}[Constructive?]
  Can consider two different types of tests, for different types of reference, respectively.
  On the one hand, could require something with direct reference, and on the other could allow the guarantee of the existence of something.
  
  And, this corresponds to two different types of promises.
\end{note}

\hozlinedash

So, promise the function and project a future, and then describe Lewis' reasoning as

\[
  \exists e(\everb{reason} \land \ein{\Sigma} \land \eout{\chi} \land \efunc{\future{f}})
\]

This isn't a standard case of reasoning, as \(\future{f}\) is a future.
Lewis can't properly reason via \(\future{f}\) as Lewis doesn't really know what this function is.
In some sense, the promise should be part of the ins of the reasoning.
But then the reasoning stated above can be seen as \emph{part} of the agent's reasoning.
Well, this is one way of thinking about things.
The other is to understand \(\future{f}\) as a particular kind of polymorphic function.
As it occurs here, it's a future, so the agent reasons with a future, and this doesn't really do anything in particular.

So, here, the type of the function tells us something about what the reasoning is.
And, as it's an unfulfilled promise, we know that there's very little going on.

Some time later, the agent does some additional work, and reasons from \(\Sigma\) to \(\chi\), and provides a witness.

\[
  \exists e(\everb{reason} \land \ein{\Sigma} \land \eout{\chi} \land \efunc{\underline{\future{f}}})
\]

The future now has a witness.

\[
  \text{promise}(f,f\Sigma = \chi)
\]

\[
  \text{future}(\text{promise}(f,f\Sigma = \chi)) = \future{f}
\]


{\color{red}
  Lewis can't reach back in time and provide the reasoning.
  However, Lewis can create something that refers to the reasoning that they will do.
  So, this refers, and has some important constraints.
  And, hence, \dots
}

\begin{note}
  Strictly speaking, on the understanding proposed, Lewis \emph{does} reason from \(\Sigma\) to \(\chi\).
  For, Lewis forms an attitude toward \(\chi\) based on their attitudes toward \(\Sigma\).
  Lewis' reasoning is based on the promise that their reasoning can witness a certain function.
  Lewis does not do work to demonstrate \(\chi\) on the basis of \(\Sigma\).
  However, Lewis promises the work.
  Function is just a collection of key-value pairs, and so the two components of reasoning are asynchronously distributed.
  So, with the promise there's little process to speak of.
  However, even if there's little of interest, there's still something.
\end{note}

\[
  \exists e(\everb{reason} \land \ein{\Sigma} \land \eout{\chi} \land \efunc{\future{f}} \land \emethod{speculate})
\]

\newpage

\mbox{ }

\newpage

\section{Less old notes}
\label{sec:less-old-notes}


\subsection{Logic}
\label{sec:logic}

\begin{note}
  Here I'm only dealing with the reasoning, so there's no discussion of justification and so on.
  All that's happening is an account of the reasoning that's going on in the case of promises.
\end{note}

This about some rules governing existentials.
Standard idea is to take a fresh constant, and show that something follows from this.
The existential guarantees that the term refers, but have no idea what to, and the fresh constant is used with the promise that it is referential.

The principle is similar to existential elimination rules.
Given a formula of the form \(\exists x Px\), fresh constant \(a\) and use this to reason about an object that \(P\) holds of.

% \begin{multicols}{2}
\begin{prooftree}
  \AxiomC{\(\exists x Px\)}
  \AxiomC{}
  \RightLabel{\scriptsize(1)}
  \UnaryInfC{\(Pa\)}
  \AxiomC{\(\forall x(Px \rightarrow Qx)\)}
  \RightLabel{\scriptsize \(\forall\) E}
  \UnaryInfC{\(Pa \rightarrow Qa\)}
  \RightLabel{\scriptsize \(\rightarrow\) E}
  \BinaryInfC{\(Qa\)}
  \RightLabel{\scriptsize \(\exists\) I}
  \UnaryInfC{\(\exists x Qx\)}
  \RightLabel{\scriptsize 1 \(\exists\) E}
  \BinaryInfC{\(\exists x Qx\)}
\end{prooftree}

This can be reformulated.

\begin{prooftree}
  \AxiomC{\(\exists x Px\)}
  \AxiomC{}
  \RightLabel{\scriptsize(1)}
  \UnaryInfC{\(\future{a}\)}
  \BinaryInfC{\(Pa\)}
  \AxiomC{\(\forall x(Px \rightarrow Qx)\)}
  \UnaryInfC{\(P\future{a} \rightarrow Q\future{a}\)}
  \BinaryInfC{\(Q\mathcal{a}\)}
  \UnaryInfC{\(\exists xQx\)}
\end{prooftree}



Quantification over elements is standard, but this can be extended to functions.

\begin{prooftree}
  \AxiomC{\(\exists f(fa = b)\)}
  \AxiomC{}
  \RightLabel{\scriptsize(1)}
  \UnaryInfC{\(\future{f}a = b\)}
  \AxiomC{\(Pb\)}
  \RightLabel{\scriptsize \(=\) E}
  \BinaryInfC{\(P\future{f}a\)}
  \RightLabel{\scriptsize \(\exists\) I}
  \UnaryInfC{\(\exists f Pfa\)}
  \RightLabel{\scriptsize 1 \(\exists\) E}
  \BinaryInfC{\(\exists f Pfa\)}
\end{prooftree}
% \end{multicols}

Here, need to reintroduce quantification over \(f\) because the premises do not provide a way of referring to \(f\).

Straying further from standard proof systems, introduce a function and bind this to the function that is stated to exist.

\begin{prooftree}
  \AxiomC{\(\exists f(fa = b)\)}
  \AxiomC{}
  \RightLabel{\scriptsize(1)}
  \UnaryInfC{\(\future{f}\)}
  \RightLabel{\scriptsize 1 B}
  \BinaryInfC{\(\future{f}a = b\)}
  \AxiomC{\(Pb\)}
  \RightLabel{\scriptsize \(=\) E}
  \BinaryInfC{\(P\future{f}a\)}
  \RightLabel{\scriptsize 1 \(\exists\) E}
  \UnaryInfC{\(\exists fP(fa)\)}
\end{prooftree}

In these examples, \(a\) and \(\future{f}\) are fresh, and \(\psi\) is inferred on the basis of these, and because no assumptions are made regarding \(a\) and \(\future{f}\), one can be sure that \(P\) holds of some transformation of \(a\).

Deductive system, hence the need to discharge assumptions.

Abstracting further, bind function and apply to something.
\(n B\) denotes the binding of the future \(n\).

\begin{prooftree}
  \AxiomC{\(\exists f (f\Phi = \psi)\)}
  \AxiomC{}
  \RightLabel{\scriptsize(1)}
  \UnaryInfC{\(\future{f}\)}
  \RightLabel{\scriptsize 1 B}
  \BinaryInfC{\(\future{f}\Phi = \psi\)}

  \AxiomC{}
  \RightLabel{\scriptsize(1)}
  \UnaryInfC{\(\future{f}\)}
  \AxiomC{\(\Phi\)}
  \RightLabel{\scriptsize 1 A}
  \BinaryInfC{\(\future{f}\Phi\)}

  \RightLabel{\scriptsize \(=\) E}
  \BinaryInfC{\(\psi\)}
  % \RightLabel{\scriptsize 1 \(\exists\) E}
  % \UnaryInfC{\(\psi\)}
\end{prooftree}

The idea remains the same, use the guarantee of reference to reason with a specific instance with a referring term.
As before, obtain \(\psi\), for if \(\Phi\) holds, and there is some transformation of \(\Phi\) which yields \(\psi\) then \(\psi\) holds.
\begin{note}
  Think about Skolemization.
\end{note}

The difficulty for the agents in the scenarios is the existential statement, and whether this is in fact true.


\subsection{Futures and promises}
\label{sec:futures-promises-1}

\begin{itemize}
\item Pause to sketch out the big picture.
\item Fairly brief, as this will be developed in more detail with comparison to other types of reasoning in a later section.
\end{itemize}

\section{Some applications}
\label{sec:some-applications}

\begin{itemize}
\item Use the basic framework to understand a few scenarios, before dealing with the abstraction some more.
\item Some applications are puzzling, others are more familiar.
\item Test of appealing to something as an excuse as an indicator of it doing some work.
  \begin{itemize}
  \item I.e.\ the exam case/cheaters defence.
  \item The cheater promises that they could have reasoned.
  \end{itemize}
\end{itemize}


\subsection{Scenarios}
\label{sec:scenarios}

\subsubsection{Shopping}
\label{sec:shopping-1}

\begin{scenario}[Shopping]
  Agent is shopping in a supermarket and has an end.
  On the shopping list is an item.
  The agent cannot immediately recall how the item relates to the end, but is confident that they put the item on the shopping list in service of the end.
  The agent is confident that purchasing the item is worthwhile means to the end, but they are not sure how.
\end{scenario}

\begin{note}
  \begin{itemize}
  \item Here, the key is an additional promise, and a shift in focus.
  \item For, the reasoning from the ends to the means does not seem too difficult, and hence it's specifying the ends which is key.
  \end{itemize}
\end{note}

\begin{prooftree}
  \AxiomC{\(\exists f \exists X(fX = \psi)\)}
  \AxiomC{}
  \RightLabel{\scriptsize(F1)}
  \UnaryInfC{\(\future{f}\)}
  \RightLabel{\scriptsize 1 B}
  \BinaryInfC{\(\future{f}X = \psi\)}
  \AxiomC{}
  \RightLabel{\scriptsize (F2)}
  \UnaryInfC{\(\future{X}\)}
  \RightLabel{\scriptsize 2 B}
  \BinaryInfC{\(\future{f}\future{X} = \psi\)}

  \AxiomC{}
  \RightLabel{\scriptsize(F1)}
  \UnaryInfC{\(\future{f}\)}
  \AxiomC{}
  \RightLabel{\scriptsize (F2)}
  \UnaryInfC{\(\future{X}\)}
  \RightLabel{\scriptsize 1 2 FA}
  \BinaryInfC{\(\future{f}\future{X}\)}

  \RightLabel{\scriptsize \(=\) E}
  \BinaryInfC{\(\psi\)}
\end{prooftree}


\section{Reasoning}
\label{sec:reasoning}

\begin{note}
  The reasoning section details the beliefs 
\end{note}

\begin{itemize}
\item promise stands in place of some to-be-done process.
\item I promise that when I will understand this, I'll call back.
\end{itemize}

A promise is a proxy for a result that it initially unknown.

\begin{itemize}
\item pending state
\item promises are fulfilled
\item A promise is a container for an as-yet-unknown value
\item extract the value out of the promise and give it to another process
\end{itemize}


This is a more-or-less technical term, especially as in these cases the promise is something that is not shared.
However, there's some intuition, especially when one considers interacting with the agent.

The idea is that the agent makes something like a promise.
They provide something which looks like a function, and this is used to obtain \(\phi\).

Here, with the existential, it's nothing new.
Taking a fresh variable, with the promise that this \emph{can} be given a referent.



\section{Precedent}
\label{sec:precedent}

\begin{itemize}
\item \citeauthor{Boghossian:2014aa}'s self-awareness condition (\citeauthor{Siegel:2019aa} provides a good overview).
\item \textcite{Pryor:2018aa}: Discussing the relationship between categorical and hypothetical justification, though this takes some work to line up.
\item \textcite{Siegel:2019aa}: \emph{Reckoning de dicto} S reckons that (for some G: having G supports Q).
\item The cases that motivate \textcite{Worsnip:2018aa}, \textcite{Fogal:2019aa}, and others are similar to those that motivate the kind of cases that I'm interested in, but a conclusion is prevented.
\end{itemize}

\subsection{Siegel}
\label{sec:siegel}

In \citeauthor{Siegel:2019aa}'s (\citeyear{Siegel:2019aa}) terminology, there is no `reckoning'.
The type of case is related, though distinct.
In the type of cases \citeauthor{Siegel:2019aa} discusses, agent's draw inferences in ignorance of the exact factors that they are responding to (\citeyear[8]{Siegel:2019aa}).
By contrast, in the types of cases under consideration the agent may be aware of the factors that they are responding to, and is aware that an inference to some proposition can be drawn, but is unaware of what the inference is.
Here, then, unclear that the interest is in inference, specifically.
I.e.\ not going to make the claim that the agent infers the relevant proposition \emph{and} it's possible for one to keep hold of the reckoning model that \citeauthor{Siegel:2019aa} argues against while puzzling about these types of cases.

\hozlinedash

\citeauthor{Siegel:2019aa} notes that if inference meets \citeauthor{Boghossian:2014aa}'s self-awareness condition, `then inferrers are never ignorant of the fact that they are responding to some of their psychological states, or why they are so responding.' (\citeyear[6]{Siegel:2019aa})

\begin{description}[font=\bfseries, leftmargin=.75cm, style=nextline]
\item[Self-awareness condition] Person-level reasoning [is] mental action that a person performs, in which he is either aware, or can become aware, of why he is moving from some beliefs to others.\nolinebreak
  \mbox{}\hfill\mbox{(\citeyear[16]{Boghossian:2014aa})}
\end{description}

\begin{itemize}
\item \citeauthor{Siegel:2019aa} considers scenarios in which an agent doesn't recognise what they're responding to.
\item This is different to the scenarios I consider because for me it is the failure to do the reasoning that is important.
\item So, in the scenarios that \citeauthor{Siegel:2019aa} considers, it seems to be the case that the agent does do the reasoning, but without self-awareness that they are doing the reasoning.
\end{itemize}


\section{Conflicts}
\label{sec:conflicts}

\begin{itemize}
\item Lord, and responding to reasons.
\item Time slice epistemology, as it seems there's something asynchronous at work, and hence there's some trouble reducing everything to the time-slice of an agent.
\end{itemize}

\newpage

\hozline

\section{Old notes}
\label{sec:old-notes}



\subsubsection{Incorporeal}
\label{sec:incorporeal}

\begin{note}
  Also to emphasise is that both the types of reasoning considered here can be seen as `time-slice' reasoning.
  This, then, allows a contrast to the distributed reasoning of promises.
\end{note}

If the agent is concerned with the reliability of the companion, then the agent needs some way to project the companions reliability with respect to questions for which the agent also settles to questions for which the agent does not settle.

The difficulties present are clear if we assume that the agent settles on a conflicting answer.
Does this suggest that the companion has been bluffing?
Has the companion made a mistake or overlooked a potential move?
Has the agent made a mistake?
Is the companion better, but previous experience has been unable to show this?
These questions can be answered, and the agent may have expectations regarding the likelihood of each of these scenarios.

Here, the proposition is whether the companion's response is truthful.

Here, that the agent can reason may be important, as it narrows down the reference class to which the companion's statement belongs.
However, the agent makes no commitment to provide reasoning as a witness to their ability to reason to the existence of a winning strategy.



\begin{itemize}
\item It is possible to reason from the board and the rules of chess to the existence of a winning strategy.
  \begin{itemize}
  \item No need for the agent and companion to be similarly matched.
  \end{itemize}
\item It is possible for the agent to reason from the board and the rules of chess to the existence of a winning strategy.
  \begin{itemize}
  \item Assumes that the agent and companion are similarly matched.
  \item It is unclear whether the agent's confidence that \emph{they} are able to reason to a winning strategy is relevant.
  \item The agent's reasoning is premised on confidence that there exists a witnessing processes, but the agent's reasoning only appeals to certain properties that a witnessing processes satisfies.
    \begin{itemize}
    \item The primary property is that the witnessing process demonstrates the existence of a winning strategy.
    \item Other properties are that the reasoning of the companion is a witness, and that the agent could provide a (perhaps distinct) witness.
    \item These latter properties are secondary because the agent only needs to be confident that a winning strategy exists.
    \item In short, the agent's reasoning that a winning strategy exists depends only on the existence of a witness.
    \item The possibility for the agent to reason is a byproduct of the support for the existence of a winning strategy that the agent appeals to.
  \item Perhaps the agent makes an attempt to reason to a winning strategy given their confidence that they \emph{can} reason to such a strategy, etc.
  \item Counterfactually, the agent could have entered the arrangement of pieces on the board into a chess program, or consulted a grand master.
  \item Of course, the way in which the agent establishes confidence does matter, and the details of the agent's reasoning would change in each of these counterfactual cases.
    However, the broad structure would remain the same: There is a way to demonstrate that a winning strategy follows from the arrangement of the pieces on the board and the rules of chess, and the existence of a winning strategy does not depend on demonstrating that it exists, therefore confidence that there is a demonstration of a winning strategy supports confidence that a winning strategy exists.
    \end{itemize}
  \end{itemize}
\end{itemize}


\subsubsection{Weights}
\label{sec:weights}

\begin{note}
  Important to note here where the uncertainty fits in.
  \begin{itemize}
  \item The uncertainty is part of what supports the agent settling on a particular action.
  \end{itemize}
\end{note}

\begin{itemize}
\item Agent considers outcome of actions given the existence or non-existence of a winning strategy, and uses their confidence of the existence of a winning strategy to weigh the possible outcomes.
\item In short, decision theoretic reasoning.
\item The idea that the agent considers their ability to reason to a winning strategy is reflected in outcomes.
\item Grant that the structure of the agent's reasoning is rational, and must argue over evaluation of outcomes.
\item If the agent recognises some tension, then it can only be due to their evaluation of outcomes.
  \begin{itemize}
  \item Incommensurability between different norms, or a different perspective (i.e.\ what would happen in the ideal case).
  \end{itemize}
\item Puzzle about how the possibility to reason fits into the evaluation of outcomes.
  \begin{itemize}
  \item I do not doubt that some explanation can be given, but it doesn't seem straightforward.
  \item And, it seems as though there may be some complexity in the agent's reasoning.
  \end{itemize}
\item Decision theoretic reasoning is not the only model of reasoning, so propose considering other options.
\end{itemize}

\hozlinedash

\subsection{Speculative}
\label{sec:speculative}

\begin{note}
  Here, first emphasis is on the agent's confidence that they are able to reason to the conclusion, paired with the observation that the above types of reasoning do not structurally depend on this, as they only deal with the existential.
\end{note}

\begin{note}
  There's the possibility of background norms, which should be noted somewhere.
  \begin{itemize}
  \item This links back to the idea that there's some pressure for the agent to reason to the conclusion.
  \item And, the promise does something odd here, as the agent doesn't completely ignore this pressure, but they do not immediately respond to it.
  \end{itemize}
\end{note}


\begin{note}
  \begin{itemize}
  \item Somewhat between the two types of incorporeal reasoning.
  \item As in the case of making an assumption, the agent uncertainty is ignored when reasoning, but in contrast the uncertainty continues to support the agent's reasoning, as the agent needs this to be low in order to guarantee the promise.
  \item Here, there's a clearer view of the puzzle, because the agent considers reasoning to the conclusion.
    \begin{itemize}
    \item This is an important contrast, as the two types of reasoning above do not depend on the agent having confidence that they are able to reason to the conclusion.
    \end{itemize}
  \item Whether the agent is really able to act with the promise provides a clearer view of the puzzle.
  \item For, if the agent makes the promise and fulfils it, then there's no problem in principle, in the same way that one trades futures.
  \item However, one may think that making the promise isn't permissible in this case, and that the agent needed to do the reasoning.
  \end{itemize}
\end{note}
\begin{itemize}
\item Agent creates a future with an associate promise.
\item Agent's confidence that there exists some reasoning from the board and the rules of chess to a winning strategy is reflected in the agent's confidence that the promise can be fulfilled.
\item Agent now assumes that they have demonstrated the existence of a winning strategy by the promised reasoning.
  \begin{itemize}
  \item This is not the same as reasoning, esp.\ clear as reasoning will provide additional information, such as specific moves.
  \end{itemize}
\item So, as the agent assumes they have demonstrated the existence of a winning strategy, the reasoning is similar to simply assuming the existential is true.
\item However, it differs in that the agent does not ignore the possibility of being false.
\end{itemize}

\begin{itemize}
\item In contrast to incorporeal certainty, the agent providing some reasoning would not be retroactive justification.
\end{itemize}

\begin{itemize}
\item Judgements given the promise of a future are complex.
\item Suppose the agent resigns, and the later demonstrates that there was a winning strategy.
\item The action of resigning was taken under the risk that the promise would not be fulfilled.
\item However, the promise was later fulfilled, and hence met the conditions set out by the promise.
\item Of course, the agent is unlikely to fulfil the promise, and this complicates matters further, as this doesn't invalidate the agent's promise; only a failure to demonstrate that they could have reasoned would do so.
\end{itemize}


There is a different proposition, whether the agent would have determined that a winning strategy exists.
Things are more straightforward.
The agent has often received the companion's response prior to settling, and it has always coincided with how the agent would have responded if truthful.
It is likely that a truthful interpretation of the companion's response reveals the answer that the agent would have settled on.

And, the agent has a high degree of confidence that the companion is truthful if the agent has settled on an answer.

The agent is confident that they could reason to a winning strategy.
There are two key thoughts here.
First, that the agent can do some reasoning.
Second, what they result of the reasoning would be.

Though the agent has not reasoned to the existence of a winning strategy, the agent adopts an attitude toward the existence of a winning strategy on the basis of the reasoning they are confident that they are able to do.
With some caveats, the agent has the same attitude toward the existence of a winning strategy that the agent would have if they were to reason to the existence of a winning strategy by some witness for the reasoning from the state of the board and the rules of chess to a winning strategy.

\subsubsection{Key features}
\label{sec:key-features}

The agent is confident that there exists a way to reason from the arrangement of pieces on the board and the rules of chess to a winning strategy.

\begin{enumerate}
\item Transfer of risk
\item Promise
\end{enumerate}


\subsubsection{Why?}
\label{sec:why}

\begin{itemize}
\item Well, the distinguishing feature is that \(\psi\) is now treated as the result of reasoning.
\end{itemize}

\begin{itemize}
\item Decision is made under an assumption that is uncertain.
  \begin{itemize}
  \item Here, then, the agent puts uncertainty to the background.
  \end{itemize}
\item Decision is made under uncertainty.
  \begin{itemize}
  \item The uncertainty is in the foreground.
  \end{itemize}
\end{itemize}

\begin{itemize}
\item One clear difference is that in the speculative case the agent may go back to reason, and whatever conclusions the agent drew will likely continue to hold, as these were drawn on the assumption of the agent reasoning to the existence of a winning strategy.
\item In contrast, if the agent reasons from confidence of existence alone, then whatever attitude the agent forms will continue to hold.
\end{itemize}

\begin{itemize}
\item The outcome may be the same, and one may argue that there shouldn't be a distinction.
\item However, this isn't obvious.
\item Imagine taking the agent to an interrogation room.
  \begin{itemize}
  \item Here, the agent works to demonstrate the existence of a winning strategy, and hence fill in their reasoning.
  \item By contrast, if the agent had reasoned from the existence, then they would not need to do this.
  \item Note, that the agent wouldn't invoke the existential in the reasoning that they would provide.
  \item And, even if reasoning by a promised future ensures that there is some line of reasoning from the existential, it's not clear that these are equally accessible.
  \end{itemize}
\item If focusing on the resignation, there's no too much of interest.
\item If focusing on justification there is a difference.
\end{itemize}

\begin{itemize}
\item Two important ideas.
  \begin{enumerate}
  \item Adoption of risk
  \item A guarantee to mitigate some of this risk.
  \end{enumerate}
\item These are the two distinguishing features.
  \begin{itemize}
  \item The promise doesn't do everything, and in some aspects it's not different to assuming the existential.
    However, there are distinguishing features that suggest this is of some interest, and in what follows the interaction between these two features and their relative emphasis is key.
  \end{itemize}
\end{itemize}

\subsection{Notes}
\label{sec:notes}

\begin{note}
  It seems as though there are cases where the distinction between witnessing the process and witnessing the algorithm may matter.
  For example, you inform my that you're reasoning from \(\futin\) to \(\futout\), but your reasoning was quite detailed.
  I may be confident that \(\futin\), and so form an attitude to \(\futout\).
  However, what matters to me is that your reasoning was adequate, not that I do the reasoning.
  Hence, I check your reasoning, but I do this in a piecemeal way, so I never instantiate the algorithm itself, but having verified that the algorithm works, I'm satisfied that my attitude toward \(\futout\) is adequately supported.
  \begin{itemize}
  \item Still, these types of cases can still be understood in terms of the simplified formalism.
  \end{itemize}
\end{note}

\hozlinedash

\subsection{Reflection}
\label{sec:reflection}

\begin{itemize}
\item Related to, but distinct from, reflection principles.
\item Reflection focuses on credences and conditionalization.
\item The agent is aware of how they will reason.
\item So, it is narrower than the broad account of reasoning.
\item Agent's future self.
\item So, also somewhat orthogonal.
\end{itemize}

\begin{itemize}
\item Right, this difference with reflection is that of a future self.
\item In the cases I'm thinking of, it's more about the limitations of one's current self.
\end{itemize}

\begin{itemize}
\item The easier distinction to make is that reflection is about evidence possessed, rather than reasoning that can be done.
\item So with reflection, the idea is to take on board future evidence given one's credence that one will learn the evidence.
\item Here, the idea is to reason on the basis of your confidence that you are able to reason in a certain way.
\end{itemize}

\section{Oughts?}
\label{sec:oughts}

\begin{itemize}
\item Idea is that there's a general principle of the form ``if reason to \(\phi\), then \(\phi\)''.
\item Roughly, defer to a better point of evaluation.
\item This is what happens with testimony, arguably, and with normative claims in general.
\item If this is so, then the argument that I want to make is that the agent is highly confident that speculating results in more closely satisfying the better point of evaluation.
\end{itemize}


\begin{itemize}
\item So, when we evaluate what the agent is to do, we adopt \emph{some} evaluative point of view.
\item The question is whether the agent's current reasoning fits into this.
\item This is a difficult question.
\item The supermarket case is interesting.
\item On the one hand, the agent should buy the carambola, because they desire starfruit.
\item On the other hand, if the agent never comes to recognise that carambola and starfruit are the same thing, then the purchase of the carambola will not serve as a means to the agent's end.
\item On the basis of the attitude that Lewis has towards the evidence, Lewis should arrest Woodthorpe.
\item However, if Lewis does not provide a demonstration of Woodthorpe's guilt, then the arrest does more harm than good.
\item I take it that these judgements are relatively clear.
\item The suggestion is then that the agent's in these scenarios are confident that the enhanced evaluative outlook is correct.
\item Hence, that performing the action is appropriate.
\item This is where the contrast to \citeauthor{Paul:2014aa} is clear, as with \citeauthor{Paul:2014aa} one is unsure about what the alternative outlook is, and for me the agent is confident.
\end{itemize}

\begin{itemize}
\item So, I'm claiming that there's a middle ground of permissibility.
\item Hence, there are things the agent ought to do.
\item And, there are things the agent ought not to do.
\item And, there are things that are permissible.
\item Hence, the thing I need to argue against is the idea that the agent speculating is something that they ought \emph{not} to do, rather than it not being something the agent ought to do.
\end{itemize}

\begin{itemize}
\item If preferred, term this `weak' rationality, which is the agent not doing things that they ought not to do.
\item Can use games to illustrate this.
\item There are ideal moves, then there are non-ideal moves, and then there are illegal moves.
\item There are probably better ways to do this\dots
\end{itemize}

\begin{itemize}
\item What is needed is an argument for evaluating from the point of view of the information state that the agent is confident that they could be in, were they to do the reasoning.
\item One argument is that this is what is needed for the kind of judgements that come in the cases that \citeauthor{Worsnip:2018aa} considers.
\item For, in the \citeauthor{Worsnip:2018aa} type cases, an agent has an attitude, and then becomes confident that this attitude is not supported on the basis of other attitudes.
\item Hence, if the agent is to revise against their held attitude, then they need to consider the evaluative viewpoint that shows the attitude is not supported.
\item If the agent can ignore this viewpoint, then they can keep hold of the attitude.
\item Yet, the intuition is that the agent cannot keep hold of the attitude on the basis that tqhey haven't demonstrated the inconsistency.
\item So, the same with being confident that \(\phi\) and being confident that \(\psi\) and then becoming confident that \(\lnot(\phi \land \psi)\).
\item An important distinction here is that in these types of cases, an agent revises away from holding an attitude, while in the types of cases I'm interested in, the agent revises toward an attitude.
\item So, this isn't quite general enough.
\item Testimony is an option, but the issue here is that testimony isn't typically about ability to reason, or rather there are cases of testimony which seem to be straightforward evidence, and it may be the case that all instances of testimony leading to the formation of an attitude can be analysed in this way.
\end{itemize}

\section{Not ought}
\label{sec:not-ought}

\begin{itemize}
\item One way of arguing would be to show that the agent ought to form the relevant attitude.
\item That is, to show that the slightly idealised point of evaluation generates a relevant ought statement, and pair this with an account of enkrasia.
\item That is, do what you are confident you ought to do.
\end{itemize}

\begin{itemize}
\item I think this may be true.
\item However, I will argue for something weaker.
\item The point of evaluation is not ruled out.
\end{itemize}

\hozlinedash

\begin{itemize}
\item The question is whether the agent's confidence that they are able to provide a witness allows the agent to form an attitude toward the conclusion based on their attitude toward the premises.
\end{itemize}

\begin{itemize}
\item If this is the question, then whether or not the agent ought to form the attitude isn't quite right, as we're not interested in whether or not the agent should form the attitude, but whether or not the agent is licensed to form the attitude in a particular way.
\item Right, and in all of the ideal worlds, the agent will do the reasoning.
\item However, this isn't the right point of view, because we are not interested in what happens in the ideal worlds.
\item Rather, we're interested in what happens given that we are in a non-ideal world.
\item The general claim that I need to make is that agent's rely on the claims of \emph{un-witnessed} reasoning.
  \begin{itemize}
  \item For, I'm in an okay position if the agent is confident that \(\chi = f\Sigma\) for some \(f\).
  \item The competing argument is that the agent can only have the attitude if the agent has a witness for \(f\).
  \item This is why \citeauthor{Lord:2018aa} is an interesting contrast, for \citeauthor{Lord:2018aa} requires a witness in order to \emph{correctly} respond to reasons.
  \end{itemize}
\item And, one way of understanding this is abstracting to a viewpoint in which there is a witness.
  \begin{itemize}
  \item Here, with Dowell, the almost-parallel is that the agent is not sure whether there is a witness for the diagnosis or not.
  \end{itemize}
\item \emph{the puzzle is about the witness}
\item In other words: \emph{What determines whether or not an inference is licensed}?
  \begin{itemize}
  \item Argument is that it is not solely determined by the agent's present reasoning.
  \end{itemize}
  \begin{itemize}
  \item This is where there's some similarity to \citeauthor{Siegel:2019aa}, as in \citeauthor{Siegel:2019aa}'s scenarios, there is a witness, but the agent isn't aware of what the witness is --- the agent doesn't have the appropriate form of self-awareness.
  \end{itemize}
\item Also note, the instructor case has the nice feature of informing the students that they can fill in a witness.
\end{itemize}

\newpage

\printbibliography

\end{document}
