\documentclass[10pt]{article}
% \usepackage[margin=1in]{geometry}
% \newcommand\hmmax{0}
% \newcommand\bmmax{0}
% % % Fonts% %
\usepackage{luatexja}

\usepackage[T1]{fontenc}
   % \usepackage{textcomp}
   % \usepackage{newtxtext}
   % \renewcommand\rmdefault{Pym} %\usepackage{mathptmx} %\usepackage{times}
\usepackage[complete, subscriptcorrection, slantedGreek, mtpfrak, mtpbb, mtpcal]{mtpro2}
   \usepackage{bm}% Access to bold math symbols
   % \usepackage[onlytext]{MinionPro}
   \usepackage[no-math]{fontspec}
   \defaultfontfeatures{Ligatures=TeX,Numbers={Proportional}}
   \newfontfeature{Microtype}{protrusion=default;expansion=default;}
   \setmainfont[Ligatures=TeX]{Source Serif Pro}
   \setsansfont[Microtype,Scale=MatchLowercase,Ligatures=TeX,BoldFont={* Semibold}]{Source Sans Pro}
   \setmonofont[Scale=0.8]{Atlas Typewriter}
   % \usepackage{selnolig}% For suppressing certain typographic ligatures automatically
   \usepackage{microtype}
% % % % % % %
\usepackage{amsthm}         % (in part) For the defined environments
\usepackage{mathtools}      % Improves  on amsmaths/mtpro2
\usepackage{amsthm}         % (in part) For the defined environments
\usepackage{mathtools}      % Improves on amsmaths/mtpro2
\usepackage{xfrac}

% % % The bibliography % % %
\usepackage[backend=biber,
  style=authoryear-comp,
  bibstyle=authoryear,
  citestyle=authoryear-comp,
  uniquename=false,
  % allinit,
  % giveninits=true,
  backref=false,
  hyperref=true,
  url=false,
  isbn=false,
  useprefix=true,
  ]{biblatex}
\DeclareFieldFormat{postnote}{#1}
\DeclareFieldFormat{multipostnote}{#1}
% \setlength\bibitemsep{1.5\itemsep}
\newcommand{\noopsort}[1]{}
\addbibresource{Thesis.bib}

% % % % % % % % % % % % % % %

\usepackage[inline]{enumitem}
\setlist[enumerate]{noitemsep}
\setlist[description]{style=unboxed,leftmargin=\parindent,labelindent=\parindent,font=\normalfont\space}
\setlist[enumerate]{noitemsep}

% % % Misc packages % % %
\usepackage{setspace}
% \usepackage{refcheck} % Can be used for checking references
% \usepackage{lineno}   % For line numbers
% \usepackage{hyphenat} % For \hyp{} hyphenation command, and general hyphenation stuff
\usepackage{subcaption}
% % % % % % % % % % % % %

% % % Red Math % % %
\usepackage[usenames, dvipsnames]{xcolor}
% \usepackage{everysel}
% \EverySelectfont{\color{black}}
% \everymath{\color{red}}
% \everydisplay{\color{black}}
\definecolor{fuchsia}{HTML}{FE4164}%Neon Fuchsia %{F535AA}%Neon Pink
% % % % % % % % % %

\usepackage{pifont}
\newcommand{\hand}{\ding{43}}
\usepackage{array}


\usepackage{multirow}
\usepackage{adjustbox}

\usepackage{titlesec}

\usepackage{multicol}

\setcounter{secnumdepth}{4}
\setcounter{tocdepth}{4}

\usepackage{tikz}
\usetikzlibrary{bending,arrows,positioning,calc}
\usetikzlibrary{arrows.meta}
\usepackage{tikz-qtree} %for simple tree syntax
% \usepgflibrary{arrows} %for arrow endings
% \usetikzlibrary{positioning,shapes.multipart} %for structured nodes
\usetikzlibrary{tikzmark}
\usetikzlibrary{patterns}


\usepackage{graphicx} % for images (png/jpeg etc.)
\usepackage{caption} % for \caption* command


\usepackage{tabularx}

\usepackage{bussalt}

\usepackage{Oblique} % Custom package for oblique commands
\usepackage{CustomTheorems}

\usepackage{svg}
\usepackage[off]{svg-extract}
\svgsetup{clean=true}

\usepackage{dashrule}

\newcommand{\hozline}[0]{%
  \noindent\hdashrule[0.5ex][c]{\textwidth}{.1pt}{}
  %\vspace{-10pt}
  % \noindent\rule{\textwidth}{.1pt}
}

\newcommand{\hozlinedash}[0]{%
  \noindent\hdashrule[0.5ex][c]{\textwidth}{.1pt}{2.5pt}
  %\vspace{-10pt}
}

\usepackage{contour}
% \usepackage{pdfrender}

\usepackage{extarrows}

% % % My commands % % %
\newcommand{\future}[1]{\ensuremath{\mathcal{#1}}}
\newcommand{\futpro}{\ensuremath{\mathcal{p}}}
\newcommand{\futin}{\ensuremath{\Sigma}}
\newcommand{\futout}{\ensuremath{\chi}}
% % % % % % % % % % % %

\usepackage[hidelinks,breaklinks]{hyperref}

\title{Promised Futures}
\author{Ben Sparkes}
% \date{ }


\begin{document}

\maketitle

\tableofcontents


\newpage


\begin{note}
  Style of paper:
  \begin{itemize}
  \item Cases in which the agent is confident that some reasoning can be done from some support to a conclusion.
  \item Goal is to provide a way to understand these cases, so that the reader can figure out how they may relate to things that the reader is interested in.
  \item So there are three lines of interest
    \begin{enumerate}
    \item The cases, and why some forms of explanation can be resisted.
    \item The understanding of the cases, and what this understanding depends on and/or is independent of.
      \begin{itemize}
      \item structural issues, etc.\
      \end{itemize}
    \item What kind of issues these cases link to.
      \begin{itemize}
      \item Reasoning.
      \item Normativity.
      \item Justification.
      \item Goal is to lay some foundations for approaching these issues in greater detail.
      \end{itemize}
    \end{enumerate}
  \end{itemize}
\end{note}



\section{Introduction}
\label{sec:introduction}

\begin{note}
  Identify and provide a framework.
  Intuitively, some instances are permissible and others impermissible.
  Will not make any strong normative claims, and instead sketch how the framework can be used to make an argument.
  (Basically, in the superman case no way to fulfil promise, while in the taxes case there is.)
\end{note}

\begin{itemize}
\item Main feature of cases of interest:
  \begin{enumerate}
  \item An agent is confident that some piece of reasoning would demonstrate that a particular conclusion follows from some support.
  \end{enumerate}
\item Three additional features.
  \begin{enumerate}[resume]
  \item Agent is confident that the support holds.
  \item The agent is confident that they (the agent) are able to reason from the support to the conclusion.
    \begin{itemize}
    \item Though, note somewhere (maybe here) that the agent may be assisted in this reasoning.
    \end{itemize}
  \item And, the agent adopts an attitude toward the conclusion on the basis of the prior features.
  \end{enumerate}
\item Term this \emph{speculation.}\nolinebreak
  \footnote{Something about how the term is used, esp.\ in CS.}
\item Goal is to provide a framework for thinking about a certain type of scenario which falls under this general description.
\item Central case: Morse \& Lewis.
\end{itemize}

\begin{scenario}
  More and Lewis are police officers.
  The pair have been working together for some time, and each consider st the other an equal (in their role as a police officer).
  The equality is supported their experience working together, and reflected in their working methodology:
  Any part of an investigation may be worked on by either Morse or Lewis, and no type of work has been done exclusively by either.
  Morse is confident that they could do any work Lewis does, and likewise Lewis is confident that they could do any work that Morse does.

  The pair are working on an investigation and share a casebook containing all the evidence they have gather so far.
  Lewis returns home from their day off to discover a message from Morse on their answering machine.
  Morse says, in so few words, that they, in the absence of any new evidence, have been reviewing the casebook and what evidence they do have supports bringing in Woodthorpe for questioning.
  Morse is not loquacious, and Morse does not explain their reasoning in the message.
  Lewis listens to the message and goes to bed.

  The following morning, and to Lewis' surprise, Lewis spots Woodthorpe on their way to the police station.
  And, Given Morse's message, Lewis arrests Woodthorpe.
\end{scenario}

\begin{enumerate}
\item Lewis is confident that there is some piece of reasoning would demonstrate that Woodthorpe's arrest is supported by the evidence contained in the case book.
\item Lewis is confident that the contents of the casebook constitute evidence.
\item Lewis is confident that both they and Morse are able to reason to the support for Woodthorpe's arrest from the casebook.
\item Lewis adopted some attitude to the proposition that Woodthorpe being arrested, and the action of arresting Woodthorpe (in part) depended on Lewis forming that attitude.
\end{enumerate}

\begin{itemize}
\item In ideal case, Lewis would have done the reasoning.
\item Indeed, in the ideal case Lewis would not be unaware of what follows from their evidence.
\end{itemize}

\begin{itemize}
\item Other explanations are possible, discuss some of these in what follows.
\item Two things to note:
  \begin{itemize}
  \item Lewis has confidence of the possibility of reasoning, and on this basis forms an attitude an arrests.
    \begin{itemize}
    \item Could think that Lewis straightforwardly forms an attitude toward the proposition, understands Morse as providing testimony, of a sort.
    \item There may be good reason to make this move, but it comes with some difficulties.
      \begin{enumerate}
      \item Morse didn't explicitly provide testimony.
      \item However, it seems Morse would not be able to avoid providing testimony if pressed.
      \item Can add some complexity to the case, assume that there are some claims in the dossier that only Lewis is aware of the evidence for, and hence Morse is restricted to only making a claim about what follows.
      \end{enumerate}
    \item Bracket testimony for the paper.
    \item Confidence is key, and while this may be supported by testimony, it is not necessary.
      The equality and history establish this without directly appealing to testimony.
    \end{itemize}
  \item It is important that Lewis is able to follow the support for the arrest.
    \begin{itemize}
    \item There are many ways to explain why this would be important for Lewis, and there are many ways to explain why this would not be important for Lewis.
    \item Sometimes this may not matter for the agent.
      \begin{itemize}
      \item If Morse is expected to do the heavy lifting, then perhaps Lewis only needs to understand that there is reason for arresting Woodthorpe.
      \item Lewis doesn't really care.
      \end{itemize}
    \item However, assume that it is important here.
      \begin{itemize}
      \item Can assume that Lewis would not arrest if there wasn't evidence.
      \item Lewis is aware that they themselves will need to justify the arrest.
      \item Lewis expects to fully understand the case file, etc.\
      \end{itemize}
    \end{itemize}
  \end{itemize}
\item Mention issue of justification, and that there's some things that can be read into the scenario that go a little beyond the basics that I'll be focusing on to start with.
\item Also, may want to mention understanding, as this is somewhat related but not required.
  \begin{itemize}
  \item For example, Morse nor Lewis may be involved in a complex financial case, and are able to demonstrate that Woodthorpe is guilty due to relations between statutes, but still do not \emph{understand} which Woodthorpe in anything other than a formal sense.
  \end{itemize}
\end{itemize}

\begin{itemize}
\item Also mention conflict with Lord, i.e.\ add in that Lewis has a copy of the case file at hand, and does not appropriately respond to reasons (may need some additional changes to adequately make this argument).
\end{itemize}

\begin{itemize}
\item More can be said about the details.
\item For now, turn to an broad characterisation.
\item Goal is to provide an abstract account of the main features, allowing us to reason about the general features of these types of cases.
\item Framework!
\end{itemize}

\newpage


\section{Framework}
\label{sec:framework-1}

\begin{note}
  Propositional and doxastic justification may help capture the difference.
\end{note}

\begin{itemize}
\item Main feature of cases of interest:
  \begin{enumerate}
  \item An agent is confident that some piece of reasoning would demonstrate that a particular conclusion follows from some support.
  \end{enumerate}
\end{itemize}

\begin{itemize}
\item Lewis is confident that there is a way to demonstrate the guilt of Woodthorpe follows from the case file shared with Morse, as Morse has claimed the reasoning can be done.
\item Students in a logic class are confident that there is a way to demonstrate \(\forall x(Fx \leftrightarrow x = a)\) follows from \(\exists x(\forall y(Fy \leftrightarrow y = x) \land Rxa)\) and \(\forall x\forall y(Rxy \rightarrow y = x)\), as they are required to demonstrate the entailment as a homework exercise.
\item You are confident that there is a way to demonstrate that the speaker would like to know if the camera is honest follows from their asking 「カメラは正直?」, as you have been given a translation.
\end{itemize}

\begin{note}
  These examples help set up the two different kinds of reasoning discussed later.
  \begin{itemize}
  \item Lewis is a speculative case.
  \item Logic is more difficult, as the student may be confident in their future reasoning, or they may remain doubtful even after providing the proof.
  \item Translation is hard to make sense of other than as incorporeal.
  \end{itemize}
\end{note}

\begin{itemize}
\item Agent does something with their confidence that some piece of reasoning would demonstrate that a particular conclusion follows from some support.
\item Agent is confident in a claim about reasoning
\item Initial interest is in understanding the claim about reasoning, and then using this to model what the agent does with the claim about reasoning given their confidence.
\end{itemize}



\subsection{Key inference}
\label{sec:key-inference}

\begin{note}
  Reasoning goes from premises to conclusion.
  There's some way to transition from attitudes toward the premises to attitudes toward the conclusion.
  So, abstractly can think of a function.
  Function is in the foreground, and a relation in the background.
  The reasoning is licensed due to some relation between the premises and the conclusion.

  This relation is clear in the case of propositional logic and decidability.
  \(\Phi \vdash \psi\) can be read as stating that there is a relation, or that there is a function.

  Standard relation to think about is propositional justification.
  So, \(\Sigma\) and \(\chi\) stand in the relation of propositional justification.

  Well, function does more than stating the relation, and given a relation is doesn't always seem to follow that there's a function.
  At least, not a function of the appropriate kind.
  I mean, there's the incompleteness results which demonstrate this, and this is part of the reason why soundness and completeness results are so nice.

  So, function as the agent is confident that they can \emph{obtain} \(\chi\) from \(\Sigma\).
  However, the agent does not need to \emph{construct} \(\chi\) from \(\Sigma\).
  Any doxastic support is seen as a byproduct of applying the function, or at least \emph{a} kind of doxastic support can be seen in this way.
\end{note}


\begin{itemize}
\item Before going into the details of reasoning, there's a key inference.
\end{itemize}

\[
\exists f(f\Sigma = \chi), \Sigma \vDash \chi
\]

So long as \(\chi\) does not depend on the application of some function, then so long as one is sure of the existence of a function and some key-value pair of that function, then given the key one can also be sure of the value.

\begin{itemize}
\item Simple in the case of the use of functions in logic, and Skolemization.
  \begin{itemize}
  \item {\color{red} Stress the idea of Skolemization here, as this does not apply to futures and promises.}
  \item E.g.\ key inference relied on idea\dots and this does not apply to futures, in particular because a claim about a future is often a claim about a process.
  \end{itemize}
\item Extends to mathematics in a straightforward way.
\item And to other instances of reasoning.
\end{itemize}

\begin{itemize}
\item In the case of Lewis, the contrast is between Woodthorpe's guilt and the demonstration of Woodthorpe's guilt.
\item Lewis is confident that Woodthorpe is guilty, though they are not confident that they have demonstrated Woodthorpe's guilt.
\item The lack of demonstration is important, and it is also important to highlight that Lewis is not necessarily certain of Woodthorpe's guilt, hence nor the possibility of demonstrating Woodthorpe's guilt.
\item However, confidence that Woodthorpe is guilty is all that is needed to motivate an arrest.
  \begin{itemize}
  \item To see that the demonstration is not necessary for Lewis, run the same scenario with explicit testimony from Morse that Woodthorpe is guilty.
  \end{itemize}
\item One way to understand this is in terms of propositional support.
  \begin{itemize}
  \item If Morse has demonstrated the guilt of Woodthorpe from the case file, then Morse has doxastic support.
  \item In turn, there is propositional support.
  \item The function then relies of propositional support, and Lewis doesn't have doxastic support.
  \end{itemize}
\end{itemize}

\begin{itemize}
\item This assume that the agent is confident that \(\exists f(f\Sigma = \chi)\).
\item Haven't yet shown why this is the case.
\item Quick answer in the case of Lewis is that \(\Sigma\) and \(\chi\) are the ins and outs of Morse's reasoning.
\item However, Morse's reasoning is some process, while the function is something witnessed by the process.
  \begin{itemize}
  \item Problem is that these a potentially two different types of things.
  \item Possible to argue for a reduction, but this isn't clear.
  \item Consider logic cases, where there are soundness and completeness theorems.
    Have semantic entailment, so there's some syntactic entailment, but some work needs to be done to show that this can be witnessed by some reasoning.
    For, it is possible that the syntactic proof requires resources beyond the reach of agents like us.
    Hence, may need to consider reasoning being witnessed by idealised agents, and so on.
  \item This does not rule out reduction, but to avoid taking on unnecessary burdens, function may be of a distinct kind.
  \item Further complications with method of reasoning, i.e.\ algorithm, as this too may be something distinct.
  \end{itemize}
\end{itemize}

\subsubsection{Two different types of reasoning, briefly characterised}
\label{sec:two-different-types}

\begin{itemize}
\item Two different types of reasoning, briefly characterised.
\end{itemize}

\begin{figure}[h]
  \begin{subfigure}{.5\textwidth}
    \centering
    \begin{tikzpicture}[
      ->,
      >=stealth',
      % auto,
      node distance=0cm, every text node part/.style={align=center},
      ]

      \node [] (c) at (0,0) {};
      \node [] (d) at (-3,0) {};
      \node [] (e) at (3,0) {};
      \node [] (f) at (0,-2.1) {};

      \node (1) at (0,-.1) {\(\{\exists f (f\futin = \futout)\} \cup \futin\)};
      \node (2) at (0,-2) {\(\futout\)};

      \draw [->] (1.270) to [] node[left] {\(g\)} (2.90);
    \end{tikzpicture}
    \caption{Incorporeal \\ Gather up the resources along with the existential.}
    \label{fig:non-constructive}
  \end{subfigure}
  % \hfill
  \begin{subfigure}{.5\textwidth}
    \centering
    \begin{tikzpicture}[
      ->,
      >=stealth',
  % auto,
      node distance=0cm, every text node part/.style={align=center},
      ]

      \node [] (c) at (0,0) {};
      \node [] (d) at (-3,0) {};
      \node [] (e) at (3,0) {};
      \node [] (f) at (0,-2.1) {};

      \node (1) at (0,-.1) {\(\futin\)};
      \node (2) at (0,-2) {\(\futout\)};

      \node (x) at (-2,-1.05) {\(\exists f(f\futin = \futout)\)};

      \draw [->] (1.270) to [] node[left] (3) {\(\future{f}\)} (2.90);

      \node (4) [left of=3, xshift=-2cm] {}; % {\(\exists f (f\Phi = \psi)\)};
      \draw [-{Circle[open]}] (x.0) to (3.180);

    \end{tikzpicture}
    \caption{Speculative \\ Use the existential to secure a promise of a witness.}
    \label{fig:speculative}
  \end{subfigure}
\end{figure}

\begin{itemize}
\item Incorporeal:
  \begin{itemize}
  \item Lewis does some reasoning on the basis of there existing some reasoning which demonstrates the support for arresting Woodthorpe.
  \end{itemize}
\item Speculative:
  \begin{itemize}
  \item Lewis takes there to be support for arresting Woodthorpe on the basis of reasoning that Lewis hasn't yet done.
  \end{itemize}
\item The key idea with both types of reasoning is that the output of the function does not depend on the reasoning.
  Hence, one can, in principle, obtain \(\futout\) from \(\futin\), as explained.
  \begin{itemize}
  \item This is important to emphasise, as it rules out certain cases.
  \item Still, examples of where the output \emph{depends} on reasoning are hard to come by.
  \item Self-referential cases can be found, e.g.\ belief that one has reasoned to \(\futout\).
  \item For the most part, however, the ins and outs are not usually restricted in this way.
  \item May think that there's also a restriction on the attitudes that the agent may have to the output, though.
  \item This is plausible, but substantive.
    \begin{itemize}
    \item \citeauthor{Sinhababu:2017aa} gives an example with desire, and there may be a case for belief somewhere\dots
    \item As a rough heuristic, think of what can be done on the basis of testimony, as here one doesn't have access to the reasoning, and any case of testimony which details the ins and outs and makes a claim to reasoning can be recast in these terms.
    \end{itemize}
  \end{itemize}
\end{itemize}

\subsection{Reasoning in more detail}
\label{sec:reas-more-deta}

\begin{itemize}
\item Problem is that reasoning claim can be understood in a number of different ways.
\item Distinguish(ed) three different ways to understand claim about reasoning.
  \begin{enumerate}
  \item Some specific process \(p\).
  \item Some method \(m\) which can be instantiated by some process(es) \(p\).
  \item Some relation which can be specified by some method(s) \(m\), and instantiated by some process(es) \(p\).
  \end{enumerate}
\item Used to thinking of reasoning in these different ways.
  \begin{enumerate}
  \item Conjunction as a function.
    Agent conjoined A and B.
  \item Particular method.
    Agent intersected two sets.
  \item Some process.
    Unclear.
    Whatever is said here would still be a method of a sort and remain in need of interpretation to point to a process.
  \end{enumerate}
\item Issue here is that there's no clear reduction between these.
  \begin{itemize}
  \item This is similar to \textcite{Marr:1982aa}
  \item This is also a simple view, and things can get messy.
    Especially if dealing with non-monotonic or ecological reasoning, where there may be no easy way to exhaustively capture the ins and outs of reasoning as a process.
  \end{itemize}
\item Polymorphic.
\item Eventually want to point to some processes, but this may be by some complex interpretation, and parts of this reduction may not be the focused on by the agent.
\item Process of reasoning is the eventual witness, but the agent could be thinking in terms of a method rather than any particular process.
  \begin{itemize}
  \item E.g.\ possible to reason via specific syntactic proof system.
  \end{itemize}
\end{itemize}

\begin{note}
  It seems as though there are cases where the distinction between witnessing the process and witnessing the algorithm may matter.
  For example, you inform my that you're reasoning from \(\futin\) to \(\futout\), but your reasoning was quite detailed.
  I may be confident that \(\futin\), and so form an attitude to \(\futout\).
  However, what matters to me is that your reasoning was adequate, not that I do the reasoning.
  Hence, I check your reasoning, but I do this in a piecemeal way, so I never instantiate the algorithm itself, but having verified that the algorithm works, I'm satisfied that my attitude toward \(\futout\) is adequately supported.
  \begin{itemize}
  \item Still, these types of cases can still be understood in terms of the simplified formalism.
  \end{itemize}
\end{note}

\begin{itemize}
\item Distinct categories, to avoid unnecessary commitments.
\item Eventual referent is a process, something that an agent does.
  \begin{itemize}
  \item In Morse and Lewis, there is some process.
  \item Still, want Lewis to also be able to do a process satisfying similar constraints.
  \end{itemize}
\item Functions and methods somehow relate.
\item Constrain the possible referent of the process.
\item Similarities to event semantics, \dots
  \begin{itemize}
  \item Process as an alternative characterisation of an event.
    \begin{itemize}
    \item \textcite{Koenig:2016aa} and \textcite{Link:1997aa} make this connexion.
    \item \textcite[117]{Davidson:2001aa} uses addition as an example!
      \[\exists x(x \text{ consists in the fact that } 2 + 2 = 5)\]

      \begin{quote}
        `\(2 + 3 = 5\)' becomes `\((\exists x) (x \text{ consists in the fact that } 2 + 3 = 5)\)'.
        Why not say `\(2 + 3 = 5\)' does not show its true colours until put through the machine?
        For that matter, are we finished when we get to the first step?
        Shouldn't we go on to `\((\exists y) (y \text{ consists in the fact that } (\exists x) (x \text{ consists in the fact that } 2 + 3 = 5)\)'?
        And so on.\nolinebreak
        \mbox{}\hfill\mbox{(\citeyear[117]{Davidson:2001aa})}
      \end{quote}
    \end{itemize}
  \end{itemize}
\end{itemize}

Agent is confident that:

\[\exists p(\text{in}(p) = \Sigma \land \text{out}(p) = \chi \land \exists f(\text{func}(p) = f))\]

This allows the agent to be confident that:

\[\exists f(f\Sigma = \chi)\]

\begin{itemize}
\item This is due to the fact that the relation between the ins and outs of the reasoning doesn't depend on the reasoning itself, as argued in section~\ref{sec:key-inference}.
\item This doesn't say much about what the function is.
\item Nor does this presuppose any particular relation between the function and the process.
  \begin{itemize}
  \item There's some way to interpret \(\text{func}(p)\), and leave this open.
  \end{itemize}
\item It is not necessarily the case that the agent reasons about some (other) process of reasoning.
  It may be the case that the function can be instantiated by various instances of reasoning.
\end{itemize}

\begin{itemize}
\item Flexibility allows specification of method, and so on.
\item Parallels also suggest lines of philosophical interest, esp.\ understanding reasoning in terms of mental events/actions.
\end{itemize}

\begin{itemize}
\item Obtaining function is now straightforward.
\item From
  \[\exists p(\text{in}(p) = \Sigma \land \text{out}(p) = \chi)\]
  One has the ins and outs of a function, hence introduce an existential to capture whatever the function is.
  \[\exists p(\text{in}(p) = \Sigma \land \text{out}(p) = \chi \land \exists f(\text{func}(p) = f))\]
  And then as one has the ins and outs, extract the function with argument and value specified.
  \[\exists f(f\Sigma = \chi)\]
\end{itemize}



\hozline


\subsection{Promises and futures}
\label{sec:promises-futures}

\begin{itemize}
\item Promises reverse the extraction of the function with argument and value specified.
\item Agent promises some function, and there is some process that conforms to the function.
\item This isn't so straightforward.
  \begin{itemize}
  \item For, given how things have been set up, processes, algorithms, and functions could all be promised.
  \item And, given that functions may be abstract key-value pairs, it's not necessarily going to be useful to provide a referent for the function used.
  \end{itemize}
\end{itemize}

\begin{note}[Promise]
  A promise is a characteristic function which may have an indeterminate truth value.
  The key idea is that a promise sets out a bunch of conditions that any argument must satisfy in order for the function to be true.
  So, the task is to specify something for which the function returns true.
  And, if the function returns true, then the argument is a way to fulfil the promise.

  They key idea is that if there's some reasoning that the agent is confident of, then we have something of the form

  \[\exists e(\Phi(e))\]

  And from this one can create the promise

  \[\text{promise}(e)\]

  Where

  \[\text{promise}(e) = \top \leftrightarrow \Phi(e)\]

  And, in the cases of interest there may be additional restrictions

  \[\text{promise}(e) = \top \leftrightarrow \Phi(e) \land \Psi(e)\]

  Hum, and alternative is to return the argument is it satisfies the constraints, so now the promise looks a whole lot like a test.
  Yeah, this is really nice, I think.


  \[\text{promise}(e) = e \leftrightarrow \Phi(e) \land \Psi(e)\]

  And then

  \[\exists e(\cdots \land \text{promise}(e) = e \land \cdots)\]

  This is a useful way to think about promises, as the standard idea with a test is that you run the test, and this allows one to continue or not, here, and here the idea is the same, it's simply a test, and the equality holds if the test passes.
  However, the additional idea is to project a future from the test.
  So, this is what we would have if the test is satisfied.

  \[\text{future}(\text{promise}(\dot{e})) = \ddot{e}\]
\end{note}

\begin{note}[Constructive?]
  Can consider two different types of tests, for different types of reference, respectively.
  On the one hand, could require something with direct reference, and on the other could allow the guarantee of the existence of something.
  
  And, this corresponds to two different types of promises.
\end{note}

\newpage

\section{Less old notes}
\label{sec:less-old-notes}


\begin{itemize}
\item The details can vary.
\item Example with addition, maybe.
  \begin{itemize}
  \item \(2 + 2 = 4\) tells us a little about the way in which \(4\) was obtained from \(2\) and \(2\).
    This is different from \(2 \times 2 = 4\), and different from \(2^{2} = 4\), etc.
    However, it doesn't go into detail on how addition was made.
    Can specify this in some more detail, etc.\
  \end{itemize}
\end{itemize}
Process witnesses/instantiates the function/algorithm.

\begin{itemize}
\item This is similar to \textcite{Marr:1982aa}
\item This is also a simple view, and things can get messy.
  Especially if dealing with non-monotonic or ecological reasoning, where there may be no easy way to exhaustively capture the ins and outs of reasoning as a process.
\end{itemize}


\begin{itemize}
\item This is an explicit representation, and allows functions, algorithms, and processes to all be distinct entities.
  \[\exists f \exists a \exists p \exists S \exists c(fS = c \land f \multimap p \land f \triangleright a)\]
\item Can ignore certain elements, such as the algorithm.
  \[\exists f \exists p(f\futin = \futout \land f \multimap p)\]
\item If the algorithm is ignored, can think of the process as implicitly specifying the function, and hence could simplify.
  \[\exists p(p\futin = \futout)\]
  \begin{itemize}
  \item Possible to introduce additional notation to directly link processes and algorithms here.
  \end{itemize}
\item Likewise, may assume that a function of the relevant kind is always witnessed by some process, and so adopt:
  \[\exists f(f\futin = \futout)\]
\item Here, can put algorithm back in.
  \[\exists f \exists a(f\futin = \futout \land f \triangleright a)\]
\item For simplicity, adopt the assumption that functions of the appropriate kind are witnessed by processes, and same for algorithms.
\item I have nothing interesting to say about how the process of reasoning works.
\item And, most of the time not too interested in the algorithm, so focus on:
  \[\exists f(f\futin = \futout)\]
\end{itemize}

\begin{note}
  In these cases our interest is in witnessing the reasoning.
  Enthymematic reasoning is related.
  Can imagine the agent quantifying over parts of some processes.
  However, if the agent is not interested in filling in the enthymematic parts of the reasoning, then this is distinct.
  For sure, there is no requirement on what form the process takes, so long as it does the work for the agent.
  In other words, the reasoning does not need to be complete, but whatever the reasoning is, the agent is interested in all of it.
\end{note}






\subsection{Logic}
\label{sec:logic}

\begin{note}
  Here I'm only dealing with the reasoning, so there's no discussion of justification and so on.
  All that's happening is an account of the reasoning that's going on in the case of promises.
\end{note}

This about some rules governing existentials.
Standard idea is to take a fresh constant, and show that something follows from this.
The existential guarantees that the term refers, but have no idea what to, and the fresh constant is used with the promise that it is referential.

The principle is similar to existential elimination rules.
Given a formula of the form \(\exists x Px\), fresh constant \(a\) and use this to reason about an object that \(P\) holds of.

% \begin{multicols}{2}
\begin{prooftree}
  \AxiomC{\(\exists x Px\)}
  \AxiomC{}
  \RightLabel{\scriptsize(1)}
  \UnaryInfC{\(Pa\)}
  \AxiomC{\(\forall x(Px \rightarrow Qx)\)}
  \RightLabel{\scriptsize \(\forall\) E}
  \UnaryInfC{\(Pa \rightarrow Qa\)}
  \RightLabel{\scriptsize \(\rightarrow\) E}
  \BinaryInfC{\(Qa\)}
  \RightLabel{\scriptsize \(\exists\) I}
  \UnaryInfC{\(\exists x Qx\)}
  \RightLabel{\scriptsize 1 \(\exists\) E}
  \BinaryInfC{\(\exists x Qx\)}
\end{prooftree}

This can be reformulated.

\begin{prooftree}
  \AxiomC{\(\exists x Px\)}
  \AxiomC{}
  \RightLabel{\scriptsize(1)}
  \UnaryInfC{\(\future{a}\)}
  \BinaryInfC{\(Pa\)}
  \AxiomC{\(\forall x(Px \rightarrow Qx)\)}
  \UnaryInfC{\(P\future{a} \rightarrow Q\future{a}\)}
  \BinaryInfC{\(Q\mathcal{a}\)}
  \UnaryInfC{\(\exists xQx\)}
\end{prooftree}



Quantification over elements is standard, but this can be extended to functions.

\begin{prooftree}
  \AxiomC{\(\exists f(fa = b)\)}
  \AxiomC{}
  \RightLabel{\scriptsize(1)}
  \UnaryInfC{\(\future{f}a = b\)}
  \AxiomC{\(Pb\)}
  \RightLabel{\scriptsize \(=\) E}
  \BinaryInfC{\(P\future{f}a\)}
  \RightLabel{\scriptsize \(\exists\) I}
  \UnaryInfC{\(\exists f Pfa\)}
  \RightLabel{\scriptsize 1 \(\exists\) E}
  \BinaryInfC{\(\exists f Pfa\)}
\end{prooftree}
% \end{multicols}

Here, need to reintroduce quantification over \(f\) because the premises do not provide a way of referring to \(f\).

Straying further from standard proof systems, introduce a function and bind this to the function that is stated to exist.

\begin{prooftree}
  \AxiomC{\(\exists f(fa = b)\)}
  \AxiomC{}
  \RightLabel{\scriptsize(1)}
  \UnaryInfC{\(\future{f}\)}
  \RightLabel{\scriptsize 1 B}
  \BinaryInfC{\(\future{f}a = b\)}
  \AxiomC{\(Pb\)}
  \RightLabel{\scriptsize \(=\) E}
  \BinaryInfC{\(P\future{f}a\)}
  \RightLabel{\scriptsize 1 \(\exists\) E}
  \UnaryInfC{\(\exists fP(fa)\)}
\end{prooftree}

In these examples, \(a\) and \(\future{f}\) are fresh, and \(\psi\) is inferred on the basis of these, and because no assumptions are made regarding \(a\) and \(\future{f}\), one can be sure that \(P\) holds of some transformation of \(a\).

Deductive system, hence the need to discharge assumptions.

Abstracting further, bind function and apply to something.
\(n B\) denotes the binding of the future \(n\).

\begin{prooftree}
  \AxiomC{\(\exists f (f\Phi = \psi)\)}
  \AxiomC{}
  \RightLabel{\scriptsize(1)}
  \UnaryInfC{\(\future{f}\)}
  \RightLabel{\scriptsize 1 B}
  \BinaryInfC{\(\future{f}\Phi = \psi\)}

  \AxiomC{}
  \RightLabel{\scriptsize(1)}
  \UnaryInfC{\(\future{f}\)}
  \AxiomC{\(\Phi\)}
  \RightLabel{\scriptsize 1 A}
  \BinaryInfC{\(\future{f}\Phi\)}

  \RightLabel{\scriptsize \(=\) E}
  \BinaryInfC{\(\psi\)}
  % \RightLabel{\scriptsize 1 \(\exists\) E}
  % \UnaryInfC{\(\psi\)}
\end{prooftree}

The idea remains the same, use the guarantee of reference to reason with a specific instance with a referring term.
As before, obtain \(\psi\), for if \(\Phi\) holds, and there is some transformation of \(\Phi\) which yields \(\psi\) then \(\psi\) holds.
\begin{note}
  Think about Skolemization.
\end{note}

The difficulty for the agents in the scenarios is the existential statement, and whether this is in fact true.


\subsection{Futures and promises}
\label{sec:futures-promises-1}

\begin{itemize}
\item Pause to sketch out the big picture.
\item Fairly brief, as this will be developed in more detail with comparison to other types of reasoning in a later section.
\end{itemize}

\section{Some applications}
\label{sec:some-applications}

\begin{itemize}
\item Use the basic framework to understand a few scenarios, before dealing with the abstraction some more.
\item Some applications are puzzling, others are more familiar.
\item Test of appealing to something as an excuse as an indicator of it doing some work.
  \begin{itemize}
  \item I.e.\ the exam case/cheaters defence.
  \item The cheater promises that they could have reasoned.
  \end{itemize}
\end{itemize}


\subsection{Scenarios}
\label{sec:scenarios}

\subsubsection{Shopping}
\label{sec:shopping-1}

\begin{scenario}[Shopping]
  Agent is shopping in a supermarket and has an end.
  On the shopping list is an item.
  The agent cannot immediately recall how the item relates to the end, but is confident that they put the item on the shopping list in service of the end.
  The agent is confident that purchasing the item is worthwhile means to the end, but they are not sure how.
\end{scenario}

\begin{note}
  \begin{itemize}
  \item Here, the key is an additional promise, and a shift in focus.
  \item For, the reasoning from the ends to the means does not seem too difficult, and hence it's specifying the ends which is key.
  \end{itemize}
\end{note}

\begin{prooftree}
  \AxiomC{\(\exists f \exists X(fX = \psi)\)}
  \AxiomC{}
  \RightLabel{\scriptsize(F1)}
  \UnaryInfC{\(\future{f}\)}
  \RightLabel{\scriptsize 1 B}
  \BinaryInfC{\(\future{f}X = \psi\)}
  \AxiomC{}
  \RightLabel{\scriptsize (F2)}
  \UnaryInfC{\(\future{X}\)}
  \RightLabel{\scriptsize 2 B}
  \BinaryInfC{\(\future{f}\future{X} = \psi\)}

  \AxiomC{}
  \RightLabel{\scriptsize(F1)}
  \UnaryInfC{\(\future{f}\)}
  \AxiomC{}
  \RightLabel{\scriptsize (F2)}
  \UnaryInfC{\(\future{X}\)}
  \RightLabel{\scriptsize 1 2 FA}
  \BinaryInfC{\(\future{f}\future{X}\)}

  \RightLabel{\scriptsize \(=\) E}
  \BinaryInfC{\(\psi\)}
\end{prooftree}


\section{Reasoning}
\label{sec:reasoning}

\begin{note}
  The reasoning section details the beliefs 
\end{note}

\begin{itemize}
\item promise stands in place of some to-be-done process.
\item I promise that when I will understand this, I'll call back.
\end{itemize}

A promise is a proxy for a result that it initially unknown.

\begin{itemize}
\item pending state
\item promises are fulfilled
\item A promise is a container for an as-yet-unknown value
\item extract the value out of the promise and give it to another process
\end{itemize}


This is a more-or-less technical term, especially as in these cases the promise is something that is not shared.
However, there's some intuition, especially when one considers interacting with the agent.

The idea is that the agent makes something like a promise.
They provide something which looks like a function, and this is used to obtain \(\phi\).

Here, with the existential, it's nothing new.
Taking a fresh variable, with the promise that this \emph{can} be given a referent.



\section{Precedent}
\label{sec:precedent}

\begin{itemize}
\item \citeauthor{Boghossian:2014aa}'s self-awareness condition (\citeauthor{Siegel:2019aa} provides a good overview).
\item \textcite{Pryor:2018aa}: Discussing the relationship between categorical and hypothetical justification, though this takes some work to line up.
\item \textcite{Siegel:2019aa}: \emph{Reckoning de dicto} S reckons that (for some G: having G supports Q).
\item The cases that motivate \textcite{Worsnip:2018aa}, \textcite{Fogal:2019aa}, and others are similar to those that motivate the kind of cases that I'm interested in, but a conclusion is prevented.
\end{itemize}

\subsection{Siegel}
\label{sec:siegel}

In \citeauthor{Siegel:2019aa}'s (\citeyear{Siegel:2019aa}) terminology, there is no `reckoning'.
The type of case is related, though distinct.
In the type of cases \citeauthor{Siegel:2019aa} discusses, agent's draw inferences in ignorance of the exact factors that they are responding to (\citeyear[8]{Siegel:2019aa}).
By contrast, in the types of cases under consideration the agent may be aware of the factors that they are responding to, and is aware that an inference to some proposition can be drawn, but is unaware of what the inference is.
Here, then, unclear that the interest is in inference, specifically.
I.e.\ not going to make the claim that the agent infers the relevant proposition \emph{and} it's possible for one to keep hold of the reckoning model that \citeauthor{Siegel:2019aa} argues against while puzzling about these types of cases.

\hozlinedash

\citeauthor{Siegel:2019aa} notes that if inference meets \citeauthor{Boghossian:2014aa}'s self-awareness condition, `then inferrers are never ignorant of the fact that they are responding to some of their psychological states, or why they are so responding.' (\citeyear[6]{Siegel:2019aa})

\begin{description}[font=\bfseries, leftmargin=.75cm, style=nextline]
\item[Self-awareness condition] Person-level reasoning [is] mental action that a person performs, in which he is either aware, or can become aware, of why he is moving from some beliefs to others.\nolinebreak
  \mbox{}\hfill\mbox{(\citeyear[16]{Boghossian:2014aa})}
\end{description}




\section{Conflicts}
\label{sec:conflicts}

\begin{itemize}
\item Lord, and responding to reasons.
\item Time slice epistemology, as it seems there's something asynchronous at work, and hence there's some trouble reducing everything to the time-slice of an agent.
\end{itemize}

\newpage

\hozline

\section{Old notes}
\label{sec:old-notes}



\subsubsection{Incorporeal}
\label{sec:incorporeal}

\begin{note}
  Also to emphasise is that both the types of reasoning considered here can be seen as `time-slice' reasoning.
  This, then, allows a contrast to the distributed reasoning of promises.
\end{note}

If the agent is concerned with the reliability of the companion, then the agent needs some way to project the companions reliability with respect to questions for which the agent also settles to questions for which the agent does not settle.

The difficulties present are clear if we assume that the agent settles on a conflicting answer.
Does this suggest that the companion has been bluffing?
Has the companion made a mistake or overlooked a potential move?
Has the agent made a mistake?
Is the companion better, but previous experience has been unable to show this?
These questions can be answered, and the agent may have expectations regarding the likelihood of each of these scenarios.

Here, the proposition is whether the companion's response is truthful.

Here, that the agent can reason may be important, as it narrows down the reference class to which the companion's statement belongs.
However, the agent makes no commitment to provide reasoning as a witness to their ability to reason to the existence of a winning strategy.



\begin{itemize}
\item It is possible to reason from the board and the rules of chess to the existence of a winning strategy.
  \begin{itemize}
  \item No need for the agent and companion to be similarly matched.
  \end{itemize}
\item It is possible for the agent to reason from the board and the rules of chess to the existence of a winning strategy.
  \begin{itemize}
  \item Assumes that the agent and companion are similarly matched.
  \item It is unclear whether the agent's confidence that \emph{they} are able to reason to a winning strategy is relevant.
  \item The agent's reasoning is premised on confidence that there exists a witnessing processes, but the agent's reasoning only appeals to certain properties that a witnessing processes satisfies.
    \begin{itemize}
    \item The primary property is that the witnessing process demonstrates the existence of a winning strategy.
    \item Other properties are that the reasoning of the companion is a witness, and that the agent could provide a (perhaps distinct) witness.
    \item These latter properties are secondary because the agent only needs to be confident that a winning strategy exists.
    \item In short, the agent's reasoning that a winning strategy exists depends only on the existence of a witness.
    \item The possibility for the agent to reason is a byproduct of the support for the existence of a winning strategy that the agent appeals to.
  \item Perhaps the agent makes an attempt to reason to a winning strategy given their confidence that they \emph{can} reason to such a strategy, etc.
  \item Counterfactually, the agent could have entered the arrangement of pieces on the board into a chess program, or consulted a grand master.
  \item Of course, the way in which the agent establishes confidence does matter, and the details of the agent's reasoning would change in each of these counterfactual cases.
    However, the broad structure would remain the same: There is a way to demonstrate that a winning strategy follows from the arrangement of the pieces on the board and the rules of chess, and the existence of a winning strategy does not depend on demonstrating that it exists, therefore confidence that there is a demonstration of a winning strategy supports confidence that a winning strategy exists.
    \end{itemize}
  \end{itemize}
\end{itemize}


\subsubsection{Weights}
\label{sec:weights}

\begin{note}
  Important to note here where the uncertainty fits in.
  \begin{itemize}
  \item The uncertainty is part of what supports the agent settling on a particular action.
  \end{itemize}
\end{note}

\begin{itemize}
\item Agent considers outcome of actions given the existence or non-existence of a winning strategy, and uses their confidence of the existence of a winning strategy to weigh the possible outcomes.
\item In short, decision theoretic reasoning.
\item The idea that the agent considers their ability to reason to a winning strategy is reflected in outcomes.
\item Grant that the structure of the agent's reasoning is rational, and must argue over evaluation of outcomes.
\item If the agent recognises some tension, then it can only be due to their evaluation of outcomes.
  \begin{itemize}
  \item Incommensurability between different norms, or a different perspective (i.e.\ what would happen in the ideal case).
  \end{itemize}
\item Puzzle about how the possibility to reason fits into the evaluation of outcomes.
  \begin{itemize}
  \item I do not doubt that some explanation can be given, but it doesn't seem straightforward.
  \item And, it seems as though there may be some complexity in the agent's reasoning.
  \end{itemize}
\item Decision theoretic reasoning is not the only model of reasoning, so propose considering other options.
\end{itemize}

\hozlinedash

It is plausible that the agent reasons in this way.
However, this struggles to make sense of the potential attempt of the agent to demonstrate that they could reason to a winning strategy.
For, any decision is made given uncertainty.
Could assume that the agent considers the possibility, that this is encoded into the decision tree.
Some time has passed, hence the additional assumption that the agent remains committed.

Possibility that there's nothing to criticise the agent for.
Or, at the very least nothing to criticise given the agent's perspective

\hozlinedash

Possible that this explains intuition.
For, this starts to mix doxastic and non-doxastic issues.
That is, from an idealised doxastic point of view, it seems clear that the agent would do better by attempting to resolve any uncertainty that they have over the existence of a winning strategy, and perhaps especially so given that the agent is also confident that they are able to do so.







\subsubsection{Assumption}
\label{sec:assumption}

\begin{note}
  Possible that the agent is sufficiently confident that the agent assumes a winning strategy exists.
  \begin{itemize}
  \item Simple contrast to `weights' in that the agent's uncertainty is no longer part of what supports the agent settling on a particular action.
  \item The uncertainty is ignored when reasoning from the existential, the board, and the rules of chess.
  \item However, this is now the only place to fault the agent, as if the assumption is granted then the agent's reasoning looks fine.
    \begin{itemize}
    \item So, this offers a little more flexibility, but still seems limited, as there doesn't seem to be much to say about the assumption.
    \item The agent retrospectively demonstrating can be seen as an attempt to license this, but this doesn't change their reasoning.
    \end{itemize}
  \end{itemize}
\end{note}

\begin{itemize}
\item Simple motivation is the threshold view of belief.
  \begin{itemize}
  \item Can assume that the agent has arbitrarily high confidence, so on the assumption that there is some threshold, assume that the agent passes this.
  \end{itemize}
\item Agent simplifies and assumes the existence of a winning strategy, and decides what to do based on this.
\item The agent may be able to provide reasons for doing so.
\item That the agent may fill in the reasoning at some later point in time, however, is irrelevant.
\item The agent's assumption, if licensed, is licensed on the basis of the agent's present situation.
  \begin{itemize}
  \item Suppose the agent resigns, and then later demonstrates that there was a winning strategy.
  \item This may excuse, but does not affect, the agent's reasoning given the assumption.
  \end{itemize}
\end{itemize}

\subsection{Speculative}
\label{sec:speculative}

\begin{note}
  Here, first emphasis is on the agent's confidence that they are able to reason to the conclusion, paired with the observation that the above types of reasoning do not structurally depend on this, as they only deal with the existential.
\end{note}

\begin{note}
  There's the possibility of background norms, which should be noted somewhere.
  \begin{itemize}
  \item This links back to the idea that there's some pressure for the agent to reason to the conclusion.
  \item And, the promise does something odd here, as the agent doesn't completely ignore this pressure, but they do not immediately respond to it.
  \end{itemize}
\end{note}


\begin{note}
  \begin{itemize}
  \item Somewhat between the two types of incorporeal reasoning.
  \item As in the case of making an assumption, the agent uncertainty is ignored when reasoning, but in contrast the uncertainty continues to support the agent's reasoning, as the agent needs this to be low in order to guarantee the promise.
  \item Here, there's a clearer view of the puzzle, because the agent considers reasoning to the conclusion.
    \begin{itemize}
    \item This is an important contrast, as the two types of reasoning above do not depend on the agent having confidence that they are able to reason to the conclusion.
    \end{itemize}
  \item Whether the agent is really able to act with the promise provides a clearer view of the puzzle.
  \item For, if the agent makes the promise and fulfils it, then there's no problem in principle, in the same way that one trades futures.
  \item However, one may think that making the promise isn't permissible in this case, and that the agent needed to do the reasoning.
  \end{itemize}
\end{note}
\begin{itemize}
\item Agent creates a future with an associate promise.
\item Agent's confidence that there exists some reasoning from the board and the rules of chess to a winning strategy is reflected in the agent's confidence that the promise can be fulfilled.
\item Agent now assumes that they have demonstrated the existence of a winning strategy by the promised reasoning.
  \begin{itemize}
  \item This is not the same as reasoning, esp.\ clear as reasoning will provide additional information, such as specific moves.
  \end{itemize}
\item So, as the agent assumes they have demonstrated the existence of a winning strategy, the reasoning is similar to simply assuming the existential is true.
\item However, it differs in that the agent does not ignore the possibility of being false.
\end{itemize}

\begin{itemize}
\item In contrast to incorporeal certainty, the agent providing some reasoning would not be retroactive justification.
\end{itemize}

\begin{itemize}
\item Judgements given the promise of a future are complex.
\item Suppose the agent resigns, and the later demonstrates that there was a winning strategy.
\item The action of resigning was taken under the risk that the promise would not be fulfilled.
\item However, the promise was later fulfilled, and hence met the conditions set out by the promise.
\item Of course, the agent is unlikely to fulfil the promise, and this complicates matters further, as this doesn't invalidate the agent's promise; only a failure to demonstrate that they could have reasoned would do so.
\end{itemize}


There is a different proposition, whether the agent would have determined that a winning strategy exists.
Things are more straightforward.
The agent has often received the companion's response prior to settling, and it has always coincided with how the agent would have responded if truthful.
It is likely that a truthful interpretation of the companion's response reveals the answer that the agent would have settled on.

And, the agent has a high degree of confidence that the companion is truthful if the agent has settled on an answer.

The agent is confident that they could reason to a winning strategy.
There are two key thoughts here.
First, that the agent can do some reasoning.
Second, what they result of the reasoning would be.

Though the agent has not reasoned to the existence of a winning strategy, the agent adopts an attitude toward the existence of a winning strategy on the basis of the reasoning they are confident that they are able to do.
With some caveats, the agent has the same attitude toward the existence of a winning strategy that the agent would have if they were to reason to the existence of a winning strategy by some witness for the reasoning from the state of the board and the rules of chess to a winning strategy.

\subsubsection{Key features}
\label{sec:key-features}

The agent is confident that there exists a way to reason from the arrangement of pieces on the board and the rules of chess to a winning strategy.

\begin{enumerate}
\item Transfer of risk
\item Promise
\end{enumerate}


\subsubsection{Why?}
\label{sec:why}

\begin{itemize}
\item Well, the distinguishing feature is that \(\psi\) is now treated as the result of reasoning.
\end{itemize}

\begin{itemize}
\item Decision is made under an assumption that is uncertain.
  \begin{itemize}
  \item Here, then, the agent puts uncertainty to the background.
  \end{itemize}
\item Decision is made under uncertainty.
  \begin{itemize}
  \item The uncertainty is in the foreground.
  \end{itemize}
\end{itemize}

\begin{itemize}
\item One clear difference is that in the speculative case the agent may go back to reason, and whatever conclusions the agent drew will likely continue to hold, as these were drawn on the assumption of the agent reasoning to the existence of a winning strategy.
\item In contrast, if the agent reasons from confidence of existence alone, then whatever attitude the agent forms will continue to hold.
\end{itemize}

\begin{itemize}
\item The outcome may be the same, and one may argue that there shouldn't be a distinction.
\item However, this isn't obvious.
\item Imagine taking the agent to an interrogation room.
  \begin{itemize}
  \item Here, the agent works to demonstrate the existence of a winning strategy, and hence fill in their reasoning.
  \item By contrast, if the agent had reasoned from the existence, then they would not need to do this.
  \item Note, that the agent wouldn't invoke the existential in the reasoning that they would provide.
  \item And, even if reasoning by a promised future ensures that there is some line of reasoning from the existential, it's not clear that these are equally accessible.
  \end{itemize}
\item If focusing on the resignation, there's no too much of interest.
\item If focusing on justification there is a difference.
\end{itemize}

\begin{itemize}
\item Two important ideas.
  \begin{enumerate}
  \item Adoption of risk
  \item A guarantee to mitigate some of this risk.
  \end{enumerate}
\item These are the two distinguishing features.
  \begin{itemize}
  \item The promise doesn't do everything, and in some aspects it's not different to assuming the existential.
    However, there are distinguishing features that suggest this is of some interest, and in what follows the interaction between these two features and their relative emphasis is key.
  \end{itemize}
\end{itemize}


% \subsection{Reference Classes}
% \label{sec:reference-classes}

% There's no requirement that the agent reasons in this way, but I take it to be permissible.
% Not clear that an ideal Bayesian agent would do better, as there's the reference class problem.
% There are a number of potential reference classes, and the agent's confidence may vary between these.
% And, the agent, roughly, is thinking that they are good reason to treat a certain reference class as indicative.
% This is \textcite{Hajek:2007aa}.

% \begin{note}
%   So, the argument is that this is a plausible reference class for the agent to appeal to.
% \end{note}

\newpage

\printbibliography

\end{document}
