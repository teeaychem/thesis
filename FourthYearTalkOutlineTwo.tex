\documentclass[10pt]{article}
% \usepackage[margin=1in]{geometry}
% \newcommand\hmmax{0}
% \newcommand\bmmax{0}
% % % Fonts% %
\usepackage[T1]{fontenc}
   % \usepackage{textcomp}
   % \usepackage{newtxtext}
   % \renewcommand\rmdefault{Pym} %\usepackage{mathptmx} %\usepackage{times}
\usepackage[complete, subscriptcorrection, slantedGreek, mtpfrak, mtpbb, mtpcal]{mtpro2}
   \usepackage{bm}% Access to bold math symbols
   % \usepackage[onlytext]{MinionPro}
   \usepackage[no-math]{fontspec}
   \defaultfontfeatures{Ligatures=TeX,Numbers={Proportional}}
   \newfontfeature{Microtype}{protrusion=default;expansion=default;}
   \setmainfont[Ligatures=TeX]{Minion 3}
   \setsansfont[Microtype,Scale=MatchLowercase,Ligatures=TeX,BoldFont={* Semibold}]{Myriad Pro}
   \setmonofont[Scale=0.8]{Atlas Typewriter}
   % \usepackage{selnolig}% For suppressing certain typographic ligatures automatically
   \usepackage{microtype}
% % % % % % %
\usepackage{amsthm}         % (in part) For the defined environments
\usepackage{mathtools}      % Improves  on amsmaths/mtpro2
\usepackage{amsthm}         % (in part) For the defined environments
\usepackage{mathtools}      % Improves on amsmaths/mtpro2

% % % The bibliography % % %
\usepackage[backend=biber,
  style=authoryear-comp,
  bibstyle=authoryear,
  citestyle=authoryear-comp,
  uniquename=false,%allinit,
  % giveninits=true,
  backref=false,
  hyperref=true,
  url=false,
  isbn=false,
  useprefix=true,
  ]{biblatex}
\DeclareFieldFormat{postnote}{#1}
\DeclareFieldFormat{multipostnote}{#1}
% \setlength\bibitemsep{1.5\itemsep}
\newcommand{\noopsort}[1]{}
\addbibresource{Thesis.bib}

% % % % % % % % % % % % % % %

\usepackage[inline]{enumitem}
% \setlist[itemize]{noitemsep}
\setlist[description]{style=unboxed,leftmargin=\parindent,labelindent=\parindent,font=\normalfont\space}
% \setlist[enumerate]{noitemsep}

% % % Misc packages % % %
\usepackage{setspace}
% \usepackage{refcheck} % Can be used for checking references
% \usepackage{lineno}   % For line numbers
% \usepackage{hyphenat} % For \hyp{} hyphenation command, and general hyphenation stuff
\usepackage{subcaption}
% % % % % % % % % % % % %

% % % Red Math % % %
\usepackage[usenames, dvipsnames]{xcolor}
% \usepackage{everysel}
% \EverySelectfont{\color{black}}
% \everymath{\color{red}}
% \everydisplay{\color{black}}
\definecolor{fuchsia}{HTML}{FE4164}%Neon Fuchsia %{F535AA}%Neon Pink
% % % % % % % % % %

\usepackage{pifont}
\newcommand{\hand}{\ding{43}}
\usepackage{array}


\usepackage{multirow}
\usepackage{adjustbox}

\usepackage{titlesec}

\makeatletter
\newcommand{\clabel}[2]{%
   \protected@write \@auxout {}{\string \newlabel {#1}{{#2}{\thepage}{#2}{#1}{}} }%
   \hypertarget{#1}{#2}
}
\makeatother

\usepackage{multicol}

\setcounter{secnumdepth}{4}
\setcounter{tocdepth}{4}

\usepackage{tikz}
\usetikzlibrary{arrows,positioning}
\usepackage{tikz-qtree} %for simple tree syntax
% \usepgflibrary{arrows} %for arrow endings
% \usetikzlibrary{positioning,shapes.multipart} %for structured nodes
\usetikzlibrary{tikzmark}
\usetikzlibrary{patterns}


\usepackage{graphicx} % for images (png/jpeg etc.)
\usepackage{caption} % for \caption* command


\usepackage{tabularx}

\usepackage{bussalt}

\usepackage{Oblique} % Custom package for oblique commands
\usepackage{CustomTheorems}

\usepackage{svg}
\usepackage[off]{svg-extract}
\svgsetup{clean=true}



\usepackage{dashrule}

\newcommand{\hozline}[0]{%
  \noindent\hdashrule[0.5ex][c]{\textwidth}{.1pt}{}
  %\vspace{-10pt}
  % \noindent\rule{\textwidth}{.1pt}
}

\newcommand{\hozlinedash}[0]{%
  \noindent\hdashrule[0.5ex][c]{\textwidth}{.1pt}{2.5pt}
  %\vspace{-10pt}
}

\usepackage{contour}
 % \usepackage{pdfrender}

\usepackage[hidelinks,breaklinks]{hyperref}

\title{Fourth Year Talk Outline}
\author{Ben Sparkes}
% \date{ }


\begin{document}
% \pagestyle{empty}

\textbf{Schema}:
An agent is rationally permitted to settle on a means (as a means) only if the agent [\underline{does something with}] a means-end relation from an end they have which supports taking the relevant means.

\hozlinedash

There are three parts to this schema:
\begin{enumerate}[label=\alph*., noitemsep]
\item An agent being rationally permitted to settle on a means (as a means)
\item A means-end relation from an end the agent has which supports taking the means.
\item The link the agent has to the specified means-end relation.
\end{enumerate}

\hozlinedash

\textbf{Two instances of the schema}:

An agent is rationally permitted to settle on a means (as a means) only if:
\begin{enumerate}[label=\roman*., ref=(\roman*)]
\item\label{schemaInstance:i} The agent \underline{\emph{reasons via}} a means-end relation from an end they have which supports taking the relevant means.
\item\label{schemaInstance:ii} The agent \underline{\emph{takes there to be}} a means-end relation from an end they have which supports taking the relevant means.
\end{enumerate}

Label these `reasoning' and `taking', respectively.

\hozlinedash

I will argue against \ref{schemaInstance:i} and for \ref{schemaInstance:ii}.

In full, I will be arguing:
\begin{enumerate}[label=\Roman*., ref=(\Roman*)]
\item\label{position:Against} \textbf{Against} the position that an agent is rationally permitted to settle on a means (as a means) only if the agent \underline{\emph{reasons via}} a means-end relation from an end they have which supports taking the relevant means.

\item\label{position:For} \textbf{For} the position that an agent is rationally permitted to settle on a means (as a means) only if the agent \underline{\emph{takes there to be}} a means-end relation from an end they have which supports taking the relevant means.
\end{enumerate}

\hozlinedash

% \begin{itemize}
% \item If \ref{position:Against} is correct then the `reasoning' instance of the schema does not apply to all instance of practical reasoning.

% \item And, if \ref{position:For} is correct then the `taking' instance of the schema applies to some instances of practical reasoning.

% \item I will suggest, but not argue, that the `taking' instance of the schema applies to all instances of practical reasoning.
% \end{itemize}

% \hozlinedash

\begin{itemize}
\item If `\emph{reasons via}' is replaced by `\emph{takes the to be by reasoning via}', then \ref{schemaInstance:i} entails \ref{schemaInstance:ii}.
\item I think this is a natural reading of `\emph{reasons via}'.
\item However:
  \begin{itemize}
  \item A logical connexion between \ref{schemaInstance:i} and~\ref{schemaInstance:ii} is not required for the argument I will make, so I will not build it into the respective positions.
  \item Making the logical connexion would not help the argument. For, \ref{schemaInstance:ii} would hold in all the cases that \ref{schemaInstance:i} holds, but this would say nothing about the cases in which \ref{schemaInstance:i} fails.
  \item There may be good reason to more sharply distinguish the two ways in which an agent may be linked to means-end relations.
  \end{itemize}
\end{itemize}



\newpage

\noindent\textbf{Argument:}

\begin{enumerate}[label=\arabic*., ref=(\arabic*)]

\item\label{scenarios:exist} There are cases in which agents recognise a means (as means) without being able to reason from an end they have to those means.

\item\label{scenarios:persmissible} And, in these cases the agent is rational in settling on the means.

\item[C\(_{\text{I}}\).]\label{scenario:no-reasoning} In order for an agent to be rationally permitted in settling on a means it cannot, in general, be required that the agent settling on those means is the result of their reasoning from an end they have to those means.

  \begin{itemize}
  \item From \ref{scenarios:exist} and~\ref{scenarios:persmissible}.
  \end{itemize}

\item\label{settle:worthwhile} For an agent to be rationally permitted to settle on an action, the agent must take the action to be worthwhile.

  \begin{itemize}
  \item Principle: Rationally settling on an action is explained by the action being considered both possible and worthwhile by the agent, perhaps in comparison to the same attributes to other actions.
  \end{itemize}

\item\label{m-e:dependence} If a rational agent considers a means \emph{only} as a means, there is no other way in which the means can be worthwhile other than as the means to an end.

  \begin{itemize}
  \item Principle: Whether a means (as a means) is worthwhile wholly depends on whether the end to the means is worthwhile.
    %\nolinebreak \mbox{ }\hfill(From \ref{m-e:dependence}, special case)
  \end{itemize}

\item[C\(_{\text{II}}\).] \label{together} If an agent is rationally permitted to settle on a means as a means, the agent must take there to be some relevant means-end relation from an end the agent has which supports taking the relevant means.
  \begin{itemize}
  \item For, by \ref{settle:worthwhile} the agent takes the means to be worthwhile.
  \item And by \ref{m-e:dependence} this can only be because there is an end the agent takes to be worthwhile.
  \end{itemize}
% \item[C\(_{\text{II}}\).] For the cases described in \ref{scenarios:exist} the agent must take their settling on the means to be supported by a means-end relation they are unable to reason about.
%   \begin{itemize}
%   \item By \ref{scenarios:exist} and \ref{scenarios:persmissible} these are cases in which it is permissible for an agent settles on means without being able to reason from an end to those means.
%   \item  And, from \ref{settle:worthwhile} to \ref{together} a means-end relation is required for the agent to settle on those means.
%   \end{itemize}
\end{enumerate}


\newpage

Illustration of case supporting premises \ref{scenarios:exist} and \ref{scenarios:persmissible}.

(This is separated into bullet-points on pages 5 and 6.)

\hozlinedash

It is Saturday morning and Professor Oblique has some time for uninterrupted research.
While the coffee is brewing Oblique decides to read some papers as a means to making progress on their research project.

Oblique logs on to their computer, and on the desktop is a folder named `papers'.
Inside the folder is an unorganised collection of academic papers.

Simply reading a paper is not a means to Oblique's end of making progress on their research project.
Some papers are read because they may point to new projects, others are to be discussed with colleagues, certain papers are part of syllabi, and so on.
Still, the papers in the folder may be a means for Oblique to make progress on their research project.
So, Oblique skims a few abstracts.

From the abstracts Oblique reads, it is unclear to Oblique that reading the papers would contribute to making progress on their research project.
However, Oblique is sure that the papers in the folder can only be of interest as a means to some end, as they of no intrinsic interest.
Still, the papers do not seem to point to new projects, nor do these seem to be the kind of papers their colleagues would be reading, and so on.

So, Oblique is unable to reason from their end of making progress on their research project to reading the papers in the folder as a means.
And, Oblique is confident that the papers can only be of interest as a means to some end.

Further, the computer is not autonomous and Oblique is the only user.
Oblique recognises that at some point in time they must have downloaded the papers, created the folder, and placed the papers in the folder --- the papers are not, for example, in a catch-all downloads folder.

The coffee finishes brewing, and Oblique beings to read a paper in the folder.

In settling on the paper, Oblique reasoned that they placed the folders in the paper for some reason --- as the result of some instance of practical reasoning.
And, the most plausible explanation is that they set these papers aside because they took them to be a means to making progress on their research project.

Perhaps the papers in the folder were cited in a papers that Oblique had been reading.
Or, perhaps the papers in the folder cited papers that Oblique had been reading.
Alternatively, the papers may have been recommends by a service such as PhilPapers or Google Scholar.
There are many ways in which the papers could stand in a means-end relation to their end of making progress on their research project, and although Oblique is unable reason through the relation, they take it to be that a relation exists, and it was on this basis they began reading.

\hozlinedash

\begin{itemize}
\item Oblique appears rationally permitted to settle on reading a paper in the folder (as a means).
  This is because:
  \begin{itemize}
  \item Oblique considers reading a paper only as a means to some end.
  \item There are many means-ends relations from Oblique's end of making progress on their research project which could support taking the means.
  \item Oblique takes it to be that (at least) one of the possible means-end relations holds.
  \end{itemize}
\item Oblique is unable to reason via an particular means-end relation from their end of making progress on their research to reading a paper in the folder.
\end{itemize}


\begin{itemize}[noitemsep]
\item The case supports premises \ref{scenarios:exist} and~\ref{scenarios:persmissible}, and in a counterexample to C\(_{\text{I}}\).

\item The case also provides some non-deductive support for C\(_{\text{II}}\), but it is only an instance of the general claim made by C\(_{\text{II}}\).
\end{itemize}




% Steps \ref{scenarios:exist} -- \ref{scenario:no-reasoning}

% Premises~\ref{scenarios:exist} and~\ref{scenarios:persmissible} are separated because

% Premise~\ref{scenario:no-reasoning} is the position that I am denying.


% The fourth premise states that means-end relations are necessary for settling on a means to be permissible.

% \newpage

% \noindent  \textbf{Suggestion}: See cases of practical reasoning in which an agent reasons from an end to a means as either
%   \begin{enumerate*}
%   \item constructing, or
%   \item checking
%   \end{enumerate*}
%   the relevant means-end relations.
% \linebreak

% \noindent Two questions:

% \begin{enumerate}[label=\alph*)]
% \item How does the conclusion relate to more complex cases, such as those involving shared activity, or gaslighting.
% \item Are there cases in which an agent is able to reason from ends to means, but settles what to do based on means-end relations that they are not able to reason about.
%   \begin{itemize}
%   \item Reasoning from ends to means, a further glass of wine settles what to do, but as the agent recognises they are tipsy, they take some water instead.
%   \end{itemize}
% \end{enumerate}

\newpage


\begin{itemize}[noitemsep]
\item It is Saturday morning and Professor Oblique has some time for uninterrupted research.
\item While the coffee is brewing Oblique decides to read some papers as a means to making progress on their research project.
\item Oblique logs on to their computer, and on the desktop is a folder named `papers'. Inside the folder is an unorganised collection of academic papers.
\item[]
\item Simply reading a paper is not a means to Oblique's end of making progress on their research project.
  \begin{itemize}
  \item Some papers are read because they may point to new projects.
  \item Other papers are to be discussed with colleagues
  \item Certain papers are part of syllabi, and so on.
  \item Still, the papers in the folder may be a means for Oblique to make progress on their research project.
  \item So, Oblique skims a few abstracts.
  \end{itemize}
\item[]
\item From the abstracts Oblique reads, it is unclear to Oblique that reading the papers would contribute to making progress on their research project.
\item However, Oblique is sure that the papers in the folder can only be of interest as a means to some end, as they of no intrinsic interest.
\item Still, the papers do not seem to point to new projects, nor do these seem to be the kind of papers their colleagues would be reading, and so on.
\item[]
\item So, Oblique is unable to reason from their end of making progress on their research project to reading the papers in the folder as a means.
\item And, Oblique is confident that the papers can only be of interest as a means to some end.
\item[]
\item Further, the computer is not autonomous and Oblique is the only user.
\item Oblique recognises that at some point in time they must have downloaded the papers, created the folder, and placed the papers in the folder --- the papers are not, for example, in a catch-all downloads folder.
\item[]
\item The coffee finishes brewing, and Oblique beings to read a paper in the folder.
\item[]
\item In settling on the paper, Oblique reasoned that they placed the folders in the paper for some reason --- as the result of some instance of practical reasoning.
\item And, the most plausible explanation is that they set these papers aside because they took them to be a means to making progress on their research project.
\item[]
\item There are many ways in which the papers could stand in a means-end relation to their end of making progress on their research project:
  \begin{itemize}
  \item Perhaps the papers in the folder were cited in a papers that Oblique had been reading.
  \item Or, perhaps the papers in the folder cited papers that Oblique had been reading.
  \item Alternatively, the papers may have been recommends by a service such as PhilPapers or Google Scholar.
  \end{itemize}
\item And although Oblique is unable reason through the relation, they take it to be that a relation exists, and it was on this basis they began reading.
\end{itemize}

\end{document}
