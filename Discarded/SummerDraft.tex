\documentclass[10pt]{article}
% \usepackage[margin=1in]{geometry}
% \newcommand\hmmax{0}
% \newcommand\bmmax{0}
% % % Fonts% %
\usepackage[T1]{fontenc}
   % \usepackage{textcomp}
   % \usepackage{newtxtext}
   % \renewcommand\rmdefault{Pym} %\usepackage{mathptmx} %\usepackage{times}
\usepackage[complete, subscriptcorrection, slantedGreek, mtpfrak, mtpbb, mtpcal]{mtpro2}
   \usepackage{bm}% Access to bold math symbols
   % \usepackage[onlytext]{MinionPro}
   \usepackage[no-math]{fontspec}
   \defaultfontfeatures{Ligatures=TeX,Numbers={Proportional}}
   \newfontfeature{Microtype}{protrusion=default;expansion=default;}
   \setmainfont[Ligatures=TeX]{Source Serif Pro}
   \setsansfont[Microtype,Scale=MatchLowercase,Ligatures=TeX,BoldFont={* Semibold}]{Source Sans Pro}
   \setmonofont[Scale=0.8]{Atlas Typewriter}
   % \usepackage{selnolig}% For suppressing certain typographic ligatures automatically
   \usepackage{microtype}
% % % % % % %
\usepackage{amsthm}         % (in part) For the defined environments
\usepackage{mathtools}      % Improves  on amsmaths/mtpro2
\usepackage{amsthm}         % (in part) For the defined environments
\usepackage{mathtools}      % Improves on amsmaths/mtpro2

% % % The bibliography % % %
\usepackage[backend=biber,
  style=authoryear-comp,
  bibstyle=authoryear,
  citestyle=authoryear-comp,
  uniquename=false,%allinit,
  % giveninits=true,
  backref=false,
  hyperref=true,
  url=false,
  isbn=false,
  useprefix=true,
  ]{biblatex}
\DeclareFieldFormat{postnote}{#1}
\DeclareFieldFormat{multipostnote}{#1}
% \setlength\bibitemsep{1.5\itemsep}
\newcommand{\noopsort}[1]{}
\addbibresource{Thesis.bib}

% % % % % % % % % % % % % % %

\usepackage[inline]{enumitem}
\setlist[itemize]{noitemsep}
\setlist[description]{style=unboxed,leftmargin=\parindent,labelindent=\parindent,font=\normalfont\space}
\setlist[enumerate]{noitemsep}

% % % Misc packages % % %
\usepackage{setspace}
% \usepackage{refcheck} % Can be used for checking references
% \usepackage{lineno}   % For line numbers
% \usepackage{hyphenat} % For \hyp{} hyphenation command, and general hyphenation stuff
\usepackage{subcaption}
% % % % % % % % % % % % %

% % % Red Math % % %
\usepackage[usenames, dvipsnames]{xcolor}
% \usepackage{everysel}
% \EverySelectfont{\color{black}}
% \everymath{\color{red}}
% \everydisplay{\color{black}}
\definecolor{fuchsia}{HTML}{FE4164}%Neon Fuchsia %{F535AA}%Neon Pink
% % % % % % % % % %

\usepackage{pifont}
\newcommand{\hand}{\ding{43}}
\usepackage{array}


\usepackage{multirow}
\usepackage{adjustbox}

\usepackage{titlesec}

\makeatletter
\newcommand{\clabel}[2]{%
   \protected@write \@auxout {}{\string \newlabel {#1}{{#2}{\thepage}{#2}{#1}{}} }%
   \hypertarget{#1}{#2}
}
\makeatother

\usepackage{multicol}

\setcounter{secnumdepth}{4}
\setcounter{tocdepth}{4}

\usepackage{tikz}
\usetikzlibrary{arrows,positioning}
\usepackage{tikz-qtree} %for simple tree syntax
% \usepgflibrary{arrows} %for arrow endings
% \usetikzlibrary{positioning,shapes.multipart} %for structured nodes
\usetikzlibrary{tikzmark}
\usetikzlibrary{patterns}


\usepackage{graphicx} % for images (png/jpeg etc.)
\usepackage{caption} % for \caption* command


\usepackage{tabularx}

\usepackage{bussalt}

\usepackage{Oblique} % Custom package for oblique commands
\usepackage{CustomTheorems}

\usepackage{svg}
\usepackage[off]{svg-extract}
\svgsetup{clean=true}



\usepackage{dashrule}

\newcommand{\hozline}[0]{%
  \noindent\hdashrule[0.5ex][c]{\textwidth}{.1pt}{}
  %\vspace{-10pt}
  % \noindent\rule{\textwidth}{.1pt}
}

\newcommand{\hozlinedash}[0]{%
  \noindent\hdashrule[0.5ex][c]{\textwidth}{.1pt}{2.5pt}
  %\vspace{-10pt}
}

\usepackage{contour}
 % \usepackage{pdfrender}

\usepackage[hidelinks,breaklinks]{hyperref}

\title{Means-end reasoning and means-end relations}
\author{Ben Sparkes}
% \date{ }


\begin{document}

\section{Scenario}
\label{sec:scenario-1}

\noindent Shopping lists?\newline
Yes, because shopping lists record information about the outcome of means-end reasoning, and we often use this information to settle on performing an action without reconstructing the relevant reasoning and this --- I will argue --- requires instrumental reasoning to be founded on confidence of means-end relations obtaining, rather than reasoning via means-end relations.

\hozlinedash

\begin{scenario}
  It's Tuesday evening and Oblique is in a supermarket with a shopping list taken from their fridge.
  Oblique lives alone, and only shops in the supermarket for themselves, so Oblique is confident that each item was added to the list by them.
  Further, Oblique has little interest in shopping.
  Oblique is confident that an item is on the list indicates that Oblique reasoned that having the item would be worthwhile, and Oblique does not find the act of looking at, or purchasing an item, worthwhile.
  So, if an item is on the shopping list it is almost certain that Oblique reasoned that purchasing the item is a means to some end that Oblique has.

  Oblique is in no hurry to return home, and decides to reason through each item on the shopping list.
  Some items are straightforward.
  For example, Oblique remembers being unable to eat cereal earlier in the day due to spoilt milk, and so purchasing milk will serve Oblique's end of eating cereal the following day.

  Other items are less straightforward, but Oblique is able to reason from an end to purchasing the item as a means.
  For example, Oblique cannot recall why they wrote bin liner on the shopping list.
  However, Oblique considers what it be like to have a lack bin liner, sufficient bin liner, and excess bin liner.
  Purchasing bin liner ensures Oblique will have either sufficient or excess bin liner, and on balance this is preferable to a lack of bin liner, and so Oblique purchases bin liner.

  And, some items are scratched from the shopping list.
  For example, probiotic yogurt is on the list, and Oblique recalls that they had been experimenting to see whether they the yogurt provided any health benefits, but Oblique is uncertain about whether they have experienced any health benefits, and the yogurt is no longer on sale, so Oblique decides not to buy the yogurt.

  The final item is star fruit, and unlike the others, as Oblique cannot reason from an end they have to star fruit.
  Oblique cannot recall why they wrote `star fruit' on the shopping list.
  Oblique does not have enough information to determine what impact having or lacking star fruit would have.
  And, in line with this lack of information, Oblique is unable to reason that purchasing star fruit wouldn't be worthwhile.
  Still, Oblique purchases some star fruit.
\end{scenario}

Oblique is unable to reason from some end they have which supports purchasing the star fruit as a means.
Note, however, that from this it does not follow that there is no supporting end.
Indeed, Oblique may have recalled that they have the end of trying carambola while searching their memory for clues as to whether or not to purchase the star fruit.
If so, Oblique is lacking the information that star fruit is carambola.
Given Oblique's end of trying carambola, and the fact that carambola is star fruit, purchasing some star fruit would be worthwhile for Oblique, but Oblique isn't able to perform this reasoning.

\begin{note}
  I've gone through various ways of constructing the puzzle, but I think this is the one to settle on.
  The core idea that the agent has some information that an action (purchasing star fruit) would be worthwhile as a means, but the agent is unable to reason from an end they have to the means.
  And, while I don't think the presence of the shopping list is essential, it seems less unnatural than an agent losing track of what they were doing while performing an action and basing their reasoning on evidence about what that action was.
  That carambola and star fruit are the same thing also provides a way to make it clear to the reader that the purchasing the star fruit could be worthwhile for the agent.
\end{note}

\begin{note}
  The goal of the paper is to locate the tension in Oblique's case.
  Whether or not Oblique is rational depends on the confidence that Oblique has in whether a means-end relation obtains.
  Oblique is not decisively irrational because they are unable to reason via a means-end relation.
\end{note}

\begin{note}
  Agent has information that they lack relevant information about what would be worthwhile.
  This information is a means-end relation.
  Goal of the paper is to argue that obtaining this information isn't required in order for the agent to act rationally.
  (In a restricted sense of rationality, as we're dealing with a bounded agent.)
\end{note}

\section{Arguments}
\label{sec:arguments-1}

\begin{enumerate}
\item Argument that if the agent considers an action (in part) as a means, then the agent must take there to be a supporting means end relation.
\end{enumerate}

\begin{note}
  This is the argument from the forth-year talk.
  The idea is to follow the example with this argument as it helps to establish what (I think) the puzzle is:

  (Roughly) there must be a supporting means-end relation in order for an agent to settle on a means, and so is it possible for an agent to settle on an action as a means without being able to reason from and end to a means without being able to `trace' the means-end relation in their reasoning?

  We briefly talked about the possibility of a weaker claim relying only on the existence of a means-end relation to rationalise Oblique's action in our last meeting, and I think this is important to consider.
  This should be mentioned somewhere, and while I think it is plausible, I do not think I need to argue for it or against it.
  Briefly, this is because the agent recognises that the action is a means to some end, and I use this to argue that the agent must recognise that there needs to be a supporting means-end relation.
  And, for now, I'm interested in what work this can do.
  It may be the case that the existence of an means-end relation is sufficient to ensure the action is rational, but this seems to license various actions that do not seem rational.
  For example, it may be that writing all of my notes with my non-dominant hand is a means to writing a good thesis, and perhaps there is research to show this, etc, but it's hard to see that I would be rational to start writing with my right-hand by virtue of the fact alone.
  In contrast, working from the agent's recognition of the action as a means to the need for a supporting means-end relation to the idea that the agent needs some confidence that this means-end relation obtains follows the (I think) more common idea that it is rational for an agent to perform an action if the are confident that it is worthwhile --- the puzzle is that this confidence is about a relation rather than an action.
  (For example, it is --- in a one sense of the term rational --- rational for me to stub my toe if I think that stubbing my toe is worthwhile.)
\end{note}

Following the idea of pressing the dilemma that Oblique is either irrational or reasoning via a means-end relation I've been trying to convince myself that one of these two disjuncts is correct.
Below are sketches of an argument for each of the disjuncts, and responses.

\begin{enumerate}[resume]
\item Oblique is irrational.
  \begin{itemize}
  \item I haven't been able to find a convincing argument for this.
  \item Oblique would be irrational if either there is no means-end relation from an end Oblique has to the means, or if Oblique does not have the end present in any supporting means-end relation.
  \item However, as developed after the scenario, it is straightforward to provide an end that Oblique has which would support purchasing the item.
  \item And, for Oblique to rule out the possibility of a means-end relation would require Oblique to be able to do an exhaustive search through all possible means-end relations.
  \item If Oblique fails to correctly recognise all possible means-end relations, then the possibility of a supporting means-end relation remains.
  \item Oblique likely recognises that they have missed or made a mistake in evaluating a means-end relation.
  \item Weakened version: Oblique would be irrational if they are confident that either there is no means-end relation from an end Oblique has to the means, or if Oblique does not have the end present in any supporting means-end relation.
  \item Weakened version may be true, but it does not necessarily apply to the scenario, as Oblique may not be confident that there is no means-end relation/Oblique does not have a supporting end.
  \end{itemize}
\end{enumerate}

\begin{enumerate}[resume]
\item If Oblique is rational, this is because Oblique reasons via a means-end relation.
\end{enumerate}

Similar to the above, I haven't been able to produce a convincing argument for this.
I suspect that there is no way to rule out that Oblique reasons via a means-end relation, and the issue is whether this is plausible.
I do, however, think that arguing from the assumption that Oblique is rational and the fact that Oblique settles on a means to the conclusion that Oblique reasons via a means-end argument fails.
I need to do some more work to motivate this strategy and to explain why it fails, but I'll try to provide the gist.

The basic idea is that if we assume that Oblique is rational, and that Oblique does some reasoning and settles on the means, then Oblique has done some reasoning which can be captured in premise conclusion form.
As this is practical reasoning, this reasoning cannot consist solely of `pure' beliefs --- some of these premises must be about what is worthwhile --- and those premises that aren't `pure' beliefs are the relevant ends.
So, Oblique's reasoning constitutes a means-end relation.
(As a background analogy, consider how any inference rule in propositional logic can be recast as an application of modus pones --- e.g.\ \(A \land B \vdash B\) can be recast as \(A, (A \land B) \rightarrow B \vdash B\).)

Roughly, the idea is that the agent reasons from ends to means, so we are given ends as input and the means as output.
And, as the agent will establish a relation from the ends to the means, the reasoning can either be part of means-end reasoning, or can be taken as an additional input.

There are two potential claims this kind of argument could be making.
Either:
\begin{enumerate*}[label=\alph*)]
\item All instances of practical reasoning are instances of reasoning via means-end relations, or
\item all instances of practical reasoning can be treated as instances of reasoning via a means-end relation.
\end{enumerate*}
The first claim is more problematic than the second, as the first denies that Oblique does anything other than reason via a means-end relation, while the second only raises the issue of whether there is anything useful to be gained by establishing that the agent does not \emph{strictly} engage in reasoning via a means-end relation.

However, I don't think argument works.
For, it is straightforward to see that an agent's practical reasoning does not, in general, provide a means-end relation.
If an agent settles on an action (in part) as a means, then the agent reasons that there is no `better' action available to them.
However, a means-end relation does not establish that there is no `better' action.
Taking a shower as a means to cleaning myself and taking a bath as a means to cleaning myself are both means-end relations, and depending on the circumstances, I may disfavour either taking a shower or taking a bath as a means to cleaning myself.
If water is scarce, then taking a shower is `better', and if water is plentiful and I also want to relax, then taking a bath is `better'.
So, reasoning from ends to an action shows that I consider the action `best', and as this judgement about an action being `best' relative to an agent's ends is part of the agents reasoning but not part of a means-end relation, an agent's means-end reasoning cannot be regarded as a means-end relation.
Hence, the agent's reasoning itself does not constitute a means-end relation, and further argument would be needed to `extract' a means-end relation from the agent's reasoning, which requires more than the observation that the agent settled on an end.

\section{Two additional notes}
\label{sec:two-additional-notes}

\begin{enumerate}
\item Doxastic cases of memory aren't quite analogous.
\end{enumerate}

From the shopping list one can form a belief about what one desired.
However, the belief is indexed to a point in time.
There's a difference between a shopping list held in one's hand while walking around a supermarket and a shopping list found at the bottom of a draw.
Looking at the shopping list in the bottom of the draw may inform the agent about their prior means and ends, but it does not automatically inform the agent about their present means and ends.


\begin{enumerate}[resume]
\item It is not the agent's inability to derive the means from an end that is important.
\end{enumerate}
For, consider interpersonal cases.
I want to bake a pie, and I ask a shop employee with type of flour I need, after explaining what kind of pie I want to make.
I am unable to reason from the kind of pie to the type of flour, but by listening to the shop employee I form a conditional which allows me to perform the relevant means-end reasoning.

This observation may help support the idea that the agent only needs to be confident that a means-end relation obtains, but for the moment it is only an observation.

\newpage

\begin{quote}
  Basic idea is that process information to determine what is possible and worthwhile.
  Processing information is costly, and there's a trade-off between the cost of processing more information and the difference that this information would provide.
  Memory allows agents to cache the results of processing information, the same information doesn't need to be processed every time an agent figures out whether they have an attitude.
\end{quote}


\begin{note}
  The thing I'm interested in is whether the persistence of attitudes can be reduces to the formation of attitudes.
  Practical case is potentially interesting because there's this in-built dependency.
  Think of intentions, the reduction seems to be assumed.
  And, this complicates the story about the formation of attitudes if the reduction fails.
\end{note}

Difference with epistemic stuff is that memory might not count as evidence for practical attitudes.
Right, with memory there's interest in what \emph{was} the case, lists and so on aren't necessarily instances of memory.
Lists are somewhat forward looking.
So, the parallel isn't particularly clear, though the positions may parallel.
For example, form the list, does it hold until I have reason to doubt it?
This is also where proofs differ, as these are `timeless'!

So, it does not seem as though memory is doing the work.
Perhaps it helps to think about contextual facts here\dots I remember travelling via the mountain, but that doesn't mean that I am travelling via the mountain, etc.
There's no parallel to rest on for the continued presence of an end being worthwhile.

However, that memory is not informative is a different thing.
I remember that a friend likes dogs.
From this I can't infer that my friend likes dogs, but it's reasonable.

This leads back to the idea that what's needed is a derivation from the end, as this is the only way to be sure\dots or something along these lines.

\begin{note}
  What matters is what is worthwhile \emph{for the agent at the time of performing the action}.
  (Roughly, at least --- \citeauthor{Bratman:2007ab} suggests this may not be completely correct, and \citeauthor{Frankfurt:1971aa} is similar in this respect; there are puzzles about what is worthwhile at any given index.)
  Still, `worthwhile' is sufficiently abstract that it can be used to capture the judgement, rather than the basis of the judgement.
  For example, in \citeauthor{Bratman:2007ab}'s case there's a change in what the agent values, but this is overridden, in a way, and that final judgement is the thing that can be identified with what is worthwhile for the agent.

  Still, there's a problem with this, as there are cases in which an action would be worthwhile and the agent is unable to reason to that action.
  This is still the same kind of case, which is neat.
  Hence, the issue isn't as simple as the agent not having the end.
\end{note}

\maketitle

\begin{note}
  Perhaps the agent can do some expected utility type stuff, but whether this is any better than the information that they have is unclear.
\end{note}

\begin{enumerate}
\item Central case
\item `Analysis'
\item `Dilemma' (two ways on the standard picture)
  \begin{itemize}
  \item Agent seems rational: belief
  \item Potential explanation of `evidence' is problematic, as this assumes the existence of the end.
  \end{itemize}
\item Information loss and information gain
  \begin{itemize}
  \item There is both information loss and information gain.
    Information loss is clear, information gain less so.
    The basis of the information is the same, nothing new is added, but given that there is a change, some bases provide distinct information.
    Well, this doesn't seem essential, the agent always has the shopping list.
  \end{itemize}
\item Parallels
\end{enumerate}

Rationality.
I understand this in terms of modelling, but this isn't standard.
It seems \citeauthor{Titelbaum:2013aa} might be a good resource, as \citeauthor{Titelbaum:2013aa} uses the Bayesian approach for modelling.


Can see the problem in two ways:
\begin{enumerate}
\item The means persists unless there is some reason to suspect that there is some reason to think that the shopping list is unreliable.
\item The shopping list provides information that needs to be supported.
\end{enumerate}

Whether there's an entailment from memory to fact.

So, if going by the first, then need an argument that inability to derive means is an indication that the means is no longer valuable.
This isn't clear.
The ability to represent is difficult.

If going by the second, then there is a problem, because it seems the relevant support cannot be obtained, unless there's an alternative way to estalbish this.

(I don't think I'm comitted to anything at present, the taking relation happens on both, it's whether this relation exists or whether the relation needs to be constructed.
However, it may be that the presentation given favours the latter, which is a problem.)

Presence of reasons to doubt veresus absence of reasons to trust. (\cite[703]{Weatherson:2015aa})

It seems as though intention may be able to some of the work here.
For, intention may be the thing that secures the need for presence of reasons to doubt.
Right, it's possible that there's no single attitude, and the shopping list could flip between the two.
It's not clear why this would be the case, though.
For, it isn't clear what work the division would do.



\newpage

\section{Anscombe}
\label{sec:anscombe}

Anscombe highlights two ways in which a shopping list can be used, and does so by distinguishing between two kinds of mistake.

\begin{quote}
  Let us consider a man going round a town with a shopping list in his hand.
  Now it is clear that the relation of this list to the things he actually buys is one and the same whether his wife gave him the list or it is his own list; and that there is a different relation when a list is made by a detective following him about.
  If he made the list itself, it was an expression of intention; if his wife gave it him, it has the role of an order.
  What then is the identical relation to what happens, in the order and the intention, which is not shared by the record?
  It is precisely this:
  if the list and the things that the man actually buys do not agree, and if this and this alone constitutes a \emph{mistake}, then the mistake is not in the list but in the man's performance (if his wife were to say: `Look, it says butter and you have bought margarine', he would hardly reply: `What a mistake! we must put that right' and alter the word on the list to `margarine'); whereas if the detective’s record and what the man actually buys do not agree, then the mistake is in the record.\nolinebreak
  \mbox{}\hfill\mbox(\citeyear[56]{Anscombe:1957aa})
\end{quote}

\citeauthor{Anscombe:1957aa} goes on to note that there are other things that can happen, such as the revision of intention, or a mistake in adding certain items to the list, and so on.

Anscombe uses `tackle for catching sharks' to illustrate a mistake in constructing the list.
\begin{quote}
  If I go out in Oxford with a shopping list including `tackle for catching sharks', no one will think of it as a mistake in performance that I fail to come back with it.\nolinebreak
  \mbox{}\hfill\mbox(\citeyear[56]{Anscombe:1957aa})
\end{quote}

Shopping list can be seen as a metaphor, but shopping lists are also quite common.
A few things happen:
\begin{enumerate}
\item Recall the reasoning with which the item was put on the shopping list.
\item Pick up the item without thought.
\item Fail to recall the reasoning with which the item was put on the shopping list.
\end{enumerate}

The first and second cases are straightforward to understand.
Means-end reasoning and intentions.
The third is less straightforward.




Suppose the shopping list has only a single direction of fit, then it doesn't seem as though the shopping list can inform the agent about what would be beneficial for them to do.
It seems it can only be a report on their past wishes.
So, it seems the agent would use the shopping list to aid their reconstruction of a piece of means-end reasoning.

\newpage

\citeauthor{Anscombe:1957aa} introduced shopping lists.
Metaphor.
Lists, really, as there's a shopping list and a list of shopping.
Not much has been said about shopping lists themselves.

In part, not much needs to be said.

However, cases where there the agent is unable to do the reasoning.



Absence of a pro-attitude doesn't entail that the situation is not worthwhile.
Didn't realise X = Y, etc.\
(This is likely important to have at the start, as it highlights the way I'm thinking about the lack of reasoning, and failure of reasoning is not the only thing.
There are more complex cases, such as temptation, but these aren't part of the main cases.)

\section{Scenario}
\label{sec:scenario}

Shopping lists are mundane, and the philosophy of action has a number of resources to explain these.
However, cases of a shopping list in which the agent is unable to do means-end reasoning and the item does not have the status of an intention.
Shopping list is interesting because the actions are almost always a means.
It is rare that the purchasing of the item is of intrinsic value.
Sometimes this is the same, comfort shopping happens.

\begin{itemize}
\item Store
\item Shopping list
\item Ingredient
\item Wonder about what you had planned
\item Can't reconstruct means-end reasoning
\item Is it rational for the agent to purchase the ingredient?
  \begin{itemize}
  \item Not ideal rationality, etc.
    It seems the agent already fails this, as they've forgotten something.
  \end{itemize}
\end{itemize}

\section{Ways in}
\label{sec:ways}

\subsection{Instrumental requirement}
\label{sec:instr-requ}

This is usually a conditional, and not a bi-conditional.
This is kind of surprising.
Could see this similar to conditionalization.
Abstract away from information loss, and so there's no need to consider cases in which the means may persist without the ability to do the relevant reasoning.
This isn't an argument, though.
And, it's a difficult position.
The idealising assumption limits the applicability of this, and we do so well with the shopping list in most cases.

\section{Arguments}
\label{sec:arguments}

\subsection{Ideal Agent}
\label{sec:ideal-agent}

\begin{note}
  \citeauthor{Hume:2011aa} might be a nice example here, as \citeauthor{Hume:2011aa} does talk about the recognition of failure.
  At least, this motivates the basic idea.
\end{note}

Argument for a ideally rational agent.
Means requires some end.
Agent has end available, or not end.
And, not hindered in the reasoning.
So, if the agent can't do it, it's irrational.
Something like this.

Three assumptions:
\begin{enumerate}
\item The agent is in possession of all relevant information.
\item The agent makes no mistakes in their reasoning.
\item The agent's reasoning is effective --- if there's a relation they'll find it.
\end{enumerate}

As the issue is the existence of a supporting means-end relation, the agent has the relevant information to establish this.
The agent will only reason via a genuine means-end relation.
The agent will not miss a genuine means-end relation.
So, if the agent is unable to reason via, then a relation does not exist.

Without these assumptions, a relation may exist.
Lack the relevant information, so unable to identify.
Make a mistake.
Miss a possibility.
This doesn't show that the agent is rational.
It's interesting that there's a lack of relation, so making a mistake is not about establishing something.
It's different to mistakenly reasoning to a means.
Still, can bracket this out.
But it's interesting.
For example, mistake that a proof won't work, so find something else.
This doesn't seem to suggest that there's something problematic with the result, though a mistake in the proof would.

\begin{note}
  The three conditions seem to be necessary and sufficient for the agent's inability to reason for an end to an action as a means to be informative about the absence of a relation.
  Parallels to doxastic attitudes may help here, esp.\ wrt.\ proofs.
\end{note}


\subsection{End}
\label{sec:end}

Perhaps there's some way to argue that the agent doesn't have the relevant end.

Cases where an agent has an end but is unable to reason to means, and the agent does have the end.
So, the ability to reason to the means doesn't seem to be part of what it is for an agent to have an end.

In certain cases, it does seem as though the issue may be that the agent lacks the ability to represent the end.
However, inferring from this that the agent does not have the end is difficult.
At a conference, want to meet someone.
Two descriptions, etc.\ note to self that the person is wearing etc.\ and then forget that the two descriptions match the same person.
Clear that you want to meet X, see X as Y, note doesn't distinguish between X and Y.


\subsection{Verification}
\label{sec:verification}

Idea is that an agent needs to be able to verify, or something.
Reasoning via means-end relations verifies, at least from the agent's perspective.



\section{?}

\begin{itemize}
\item Shopping list
  \begin{itemize}
  \item What does the shopping list do?
    \begin{itemize}
    \item Focusing options for practical (means-end) reasoning.
    \item Avoiding the need to redo means-end reasoning.
      \begin{itemize}
      \item This is how intentions in \textcite{Bratman:1987aa} work, more-or-less
      \end{itemize}
    \item Information about what is worthwhile.
      \begin{itemize}
      \item This is a weaker version of \citeauthor{Bratman:1987aa}'s intentions.
      \item Need some argument for this, maybe.
      \end{itemize}
    \end{itemize}
  \end{itemize}
\end{itemize}

Many things could be happening with an item that one can't reason to.
Cannot reason from ends to means to be sure that purchasing the item on the shopping list would be worthwhile.
This is the key part, the failure of means-end reasoning is the point of interest.
To say that the agent is unsure whether purchasing the item on the shopping list would be worthwhile isn't necessarily the same thing.
If means-end reasoning is required, then there's a case for this, but there is a straightforward difference.
Simple analogy, I did not prove a formula was a theorem by a syntactic proof, but it does not follow from this that I did not prove the theorem (I may have used a semantic proof).

Perhaps there's a case to be made.
Need that means-end reasoning and permissibility align.
A result of this would be that either there's an end the agent reasons from, or the agent is irrational.
Here, assuming that the agent is reasoning, as intentions, etc.
So, result of reasoning only if end to means.

Shopping lists are interesting because the reasoning is often quite easy to reconstruct.
\citeauthor{Anscombe:1957aa} has the example of bait for shark fishing.

Not all shopping lists are like this.


\subsubsection{Nails and bin bags}
\label{sec:nails-bin-bags}

{\color{red}
  Different items would be nice.
  Maybe apples instead of bin bags.
}

Bin bags: evaluate what it would be to have bin bags and what it would be like not to have bin bags.
Probability distribution, and done.
This seems like a paradigm case of re-evaluating.
Whatever the prior reason was, a new reason can be found.

Nails of a certain length:
Well, less clear.
Nails of the correct length, for sure.
And, the distribution is then over whether the length written down is the correct length.
Is there more?
The correct length isn't really something that can be used to do any work.
Can't reason to the particular length.
Means-end reasoning fails, though it gets close.
Situation is that the nails of the specified length are more expensive than those slightly longer and slightly shorter.


\subsection{Ingredient}
\label{sec:ingredient}

This seems like the best example.
It's on the shopping list, but it's unfamiliar.
Can't reconstruct the means-end reasoning.
Indeed, it's striking and so you make the attempt because you're interested in recalling what you had planned.
And now you're unable to continue.


\newpage

Shopping lists and the metaphor.
But \emph{shopping lists}.
Lists, really.

Instrumental rationality.
Items on list are means to end, purchasing the items is `necessary'.

Suppose I start to think and cannot recall.
Unable to reason from an end to the action as a means.

Assume that I did perform some means-end reasoning when I wrote the shopping list.
Persistence of the attitude assumes prior existence.
It may be unclear to the agent that there was prior existence.
And, it's not clear to us that there was prior existence.

Instrumental rationality doesn't clearly apply, as need the conditional.




Two modals.
The agent's reasoning may not provide a unique action.
Require a certain method of reasoning.

Basic answer:

Investigating what can provide the relevant information is too broad to cover in a single paper.

Question is whether prior information of an attitude can be distinguished from the novel formation of an attitude.
Epistemic parallels.
But, also disanalogies, as facts get indexed.
Issue isn't whether the means \emph{was} worthwhile, but whether it \emph{is} worthwhile.


\newpage

\begin{wodge}[Basic question]
  If an agent has information that they settled on performing an action (in part) as a means to an end, then is the agent required to be able to reason from that end to the action as a means in order to be permitted to settle on the action (in part) as a means?
  \begin{itemize}
  \item Notes:
    \begin{itemize}
    \item Not assuming that information is factive.
    \item Roughly, information is what the agent uses to rule out possibilities.
    \end{itemize}
  \end{itemize}
\end{wodge}

\begin{wodge}[Basic answer]
  What matters is the existence of a means-end relation.
  Reasoning is one way of establishing this, but it is not the only way.
\end{wodge}

\hozlinedash

\noindent Shopping lists?\newline
Yes, because shopping lists record information about the outcome of means-end reasoning, and we often use this information to settle on performing an action without reconstructing the relevant reasoning and this --- I will argue --- requires instrumental reasoning to be founded on confidence of means-end relations obtaining, rather than reasoning via means-end relations.

Example.

\begin{itemize}
\item Look at the item, recognise why.
  \begin{itemize}
  \item Ah, yeah, I wasn't able to eat cereal this morning because I didn't have any milk, and I'll have the same problem tomorrow if I don't buy some\dots
  \item I remember putting the item on the shopping list on the basis of this reasoning.
  \end{itemize}
\item Look at the item, do some reasoning.
  \begin{itemize}
  \item Not sure if I need bin bags, but between not having, having some, and having too many, I prefer the latter two options to the first, and so on balance I think purchasing bin bags is the thing to do.
  \item I do not remember putting the item of the shopping list on the basis of this reasoning, but regardless, this reasoning supports purchasing the item on the shopping list.
  \end{itemize}
\item Star fruit, hum\dots
  \begin{itemize}
  \item I don't recall the reasoning, nor am I able to reason from some end to purchasing the item.
  \end{itemize}
\end{itemize}

This is more than what usually goes on when shopping.
Focus is typically on completing the list, and doesn't reason about the relevant items.
Position of privilege, as don't need to worry about funds, special offers, etc.

Can't take the mere presence of star fruit on the shopping list as support for purchasing the item as being worthwhile.
\citeauthor{Anscombe:1957aa} points this out.
Could be that the item makes no sense, or that the agent is clear that there is no supporting end.

May be a supporting end.
Argument with carambola and star fruit.
What the agent lacks is the information that star fruit and carambola are the same thing.

Interested in things from the agent's perspective.
So, won't assume this sort of condition holds.

Agent recognises it's a means.
And, conceptual that this requires an end.
If there's no end, then the means isn't worthwhile.
Means, so there must be a supporting means-end relation.

This, then is the puzzle.
Relation must be there, but the agent isn't able to reason with it.
In other words, it's clear what the relevant justification is, but the agent doesn't have access to it.

This, then, is what I'm arguing for, that the agent is confident that the relation holds.
Well, this is a sketch.
Question is whether, and if so how, to make sense of this.


This is a common occurrence.
The phenomenon is the issue.
It is puzzling.
Rational or irrational.

Opposition is rational only if means-end.
So, not rational or no means-end.

Two quick arguments.

No means-end, so not rational.
Easy way to this is if the agent doesn't have the end, or there is no means-end relation.
But, this can't be inferred.
Agent has end under some other description, and the missing information may be clear.

Rational, so means-end.
Reasoning is polymorphic.
If the agent settles on the action, then they're done some means-end reasoning and this is the relevant means-end relation.
Don't have a grasp of what the reasoning is, but it constitutes a means-end relation.
Same way as with modus ponens.
This isn't right, means-end relations are different from means-end reasoning.





\newpage

\begin{wodge}[On the big picture]
  The shopping list captures that there is potentially a route from an end the agent has to the means.
  So, the agent doesn't necessarily need to for provide a resolution to the issue of what to do.
  Instead, the agent may apply a pre-existing resolution to the issue.

  This, however, is still a way of viewing the problem.
  Consider an analogy to a straightforward question.
  Write some fact down on a piece of paper, and someone asks a question.
  Typically, need to be able to link the fact to the question.
  It is not typically required that the agent is able to derive the fact.

  May seem as though the agent's past self has given the answers to the geography exam, and the agent's present self is now copying these down.
  The agent isn't answering the questions in any robust sense.

  In these cases, the agent lacks justification, right?
  There needs to be some account of \emph{why} the provided information resolves the issue.
  In the puzzle, it seems as though there is little issue as to what this is.

  This is, in part, a way to understand how \citeauthor{Bratman:2007ab}'s solution to temptation works --- \citeauthor{Bratman:2007ab} provides a distinct source of justification that the agent can appeal to in the present reasoning.

  But at the same time this doesn't seem right.
  Cases where the agent can't explain why, as in the case of another person providing the answer.
  In this case the agent doesn't have an explanation for why the information resolves the issue.
  Instead, the agent is confident that the information resolves the issue.

  \emph{Basic} idea as to why the agent \emph{could} settle on the item is that there is a case to be made for purchasing the item as worthwhile.
  And, this is all that's required.
  Well, someone objecting needs to show that there's something more than the action being potentially worthwhile.
  And this is why there seems to be a parallel with searching for information.

  Given some background, need to answer two questions.
  Is there a way that the agent could expect purchasing the item to be worthwhile.
  And, if so, is this more worthwhile than any alternative.
  If the means-end relation holds, then it would be worthwhile.
  If agent is confident that the means-end relation holds.

  If rejecting this, then there must be something more than the action being worthwhile that rules out.
  Doubt this is the case.

  In terms of structure, this argument after the two (simple) negative arguments.
  
\end{wodge}



Shopping lists.
Information about means, but do not contain information about ends.
This is the distinctive feature.
It is unusual to find a shopping list annotated with the ends the listed items are for.
Likewise, it is unusual for an agent to have the end of purchasing the items on the shopping list.
Even if the possession of the item is the end, purchasing is an intermediate step.
Wondering why an item is on a list isn't the same as removing and attempting to add the item from scratch --- the list persists.

\hozlinedash

\begin{wodge}[Big picture]
  The idea here is to identify what instrumental reasoning is doing, and the idea is that instrumental reasoning isn't too much like premise-conclusion reasoning.

  Well, broadly, there are two models of practical reasoning.
  The premise-conclusion model and the decision theoretic model.
  I don't think I need to take a position on which is correct, but I do think the decision theoretic model holds up better.
  Right, because on my understanding one does some reasoning, and there's a further question of whether this answers what to do.
  The agent in the cases I'm thinking about doesn't have an answer, but the question had already been answered, which is where the interest is, but this doesn't assume too much about the reasoning.
\end{wodge}


\begin{enumerate}
\item Basic case: shopping list.
  \begin{enumerate}
  \item Agent is unable to reason from an end to the item on the list.
  \end{enumerate}
\item In order for the item on the list to be a permissible means there must be an end that the agent has which supports purchasing the item.
  \begin{itemize}
  \item This follows from understanding the item on the list as a means.
    (Conceptual necessity.)
  \end{itemize}
\item Strong idealising assumptions are required to show that the agent's inability to reason from an end to the item on the list establishes that there is no relation.
  \begin{itemize}
  \item For simplicity, assume that means-end relations exist independently from the agent.
    Then, in order to show that no means-end relations exists, one needs to ensure an agent is able to do an exhaustive search through any candidate means-end relation.
    If the agent fails to do an exhaustive search, a supporting means-end relation may be in the collection of means-end relations the agent did not consider.
    So, Need to assume:
    \begin{enumerate}
    \item The agent has access to all the relevant information.
      \begin{itemize}
      \item If the agent lacks some information they may fail to notice a possible means-end relation.
      \end{itemize}
    \item The agent is able to reason effectively (is able to consider all possible means-end relations)
      \begin{itemize}
      \item If the agent is unable to consider all possible relations, a supporting means-end relation may be in the collection of unconsidered relations.
      \end{itemize}
    \item The agent makes no mistakes in their reasoning.
      \begin{itemize}
      \item If the agent cannot rule out mistakes, then the agent cannot be sure that the failed to recognise a supporting means-end relation.
      \end{itemize}
    \end{enumerate}
  \end{itemize}
\item The case does not require that the agent does not recognise the relevant end, only that the agent is unable to reason from the end to the means.
  \begin{itemize}
  \item Case of co-reference where the agent recognises the end under a certain description, but is unable to reason from this description to the means.
  \item `Star fruit' and `carambola'.
    Agent is British and has the end of trying carambola, but is in America and has written `star fruit'.
    The agent loses the information that `carambola' and `star fruit' are co-referential.
  \end{itemize}
\end{enumerate}




\begin{proposition}
  An agent may have an end and a means and be unable to reason from the end to the means.
  \begin{proof}
    Example involving co-reference.
    Agent wants to try star fruit.
    The agent is in England, and the British term for star fruit is `carambola'.
    So, the agent writes `carambola' on their shopping list.
    Between writing the shopping list and going to the shops, the agent loses the information that `star fruit' and `carambola' or co-referential.
    In the store the agent looks at their shopping list, and wonders why `carambola' is written on the list.
    The agent recalls that they want to try star fruit, but is unable to reason from this end to purchasing the object labelled `carambola'.
    Assumption: agent does not consider the possibility of co-reference (the agent does not have first hand experience with star fruit, so the shape does not provide a hint, \dots star fruit was recommended by a friend, etc.)
    [This example should be switched, as star fruit is a little too descriptive.]
  \end{proof}
  If the agent had the missing information, they would be able to reason from the end to the means.
  No change in what the agent thinks is worthwhile makes sense here.
\end{proposition}

\begin{proposition}
  Doxastic cases of memory aren't quite analogous.
  \begin{proof}
    From the shopping list one can form a belief about what one desired.
    However, the belief is indexed to a point in time.
    There's a difference between a shopping list held in one's hand while walking around a supermarket and a shopping list found at the bottom of a draw.
    Looking at the shopping list in the bottom of the draw may inform the agent about their prior means and ends, but it does not automatically inform the agent about their present means and ends.
  \end{proof}
\end{proposition}

\begin{proposition}
  It is not the agent's inability to derive the means from an end that is important.
  \begin{proof}
    Consider interpersonal cases.
    I want to bake a pie, and I ask a shop employee with type of flour I need, after explaining what kind of pie I want to make.
    I am unable to reason from the kind of pie to the type of flour, but by listening to the shop employee I form a conditional which allows me to perform the relevant means-end reasoning.
  \end{proof}
\end{proposition}

This seems like an important observation.
An action regarded as a means requires an end, and without this link the status of the action as a means seems suspect.
The agent is unable to link the action as a means to an end, and what is required is, potentially, something very weak; a simple conditional.

Individual cases are easy to come by:

\begin{proposition}
  Reasoning via doesn't require derivation.
  \begin{proof}
    Case in which agent works to figure out pleasing response.
    Agent remembers response, but forgets why it is pleasing.
    Agent retains conditional linking means to end, and this seems to be enough.
  \end{proof}
\end{proposition}

The agent is able to reason via, but is unable to derive a means-end relation.
This raises a puzzle.
For, one can have a belief that the shopping list is a means to the agent's end.
Then, there's a conditional, and the difficulty is that the agent's end isn't specified.
But this is very puzzling.
It seems that if the means is worthwhile, then it should follow from the agent's ends considered as a whole, and so this kind of conditional `should' be available.
Some of the ends may be redundant, etc.\ but this doesn't invalidate the conditional itself.

If this kind of conditional is available, then if the agent is considering and doesn't have a way of reasoning via a means-end relation, then it seems this conditional is unavailable to the agent, and so the agent does not believe that performing the means is required for the agent's ends.
So, this solves the problem.
If failure, then this conditional doesn't exist, hence agent doesn't have reason to pursue the means.
And, if no failure then this conditional does exist, and reason to pursue the means.
In either case, the `problem' is reduced to a familiar case of means-end reasoning.
It's the second disjunct that's most troubling, as it seems that even if I argue that the shopping list case is rational, then this explanation is possible.
So, I need an answer, and it seems the only way to do this is to argue against the general conditional working out.

Potential difference between support and a relation.
It is going to be the case that the agent's ends considered broadly support the means, but this isn't the same as a means-end relation obtaining between the agent's ends and the means.
This is an argument that the general ends to means belief isn't of the right kind.

Perhaps, there's no way to establish a belief of the right kind.
So, it's not possible to have the belief that there's a means end relation from one's ends to the means, because there's nothing that can support this belief.
Recognition of ends is partial, further reasoning could overturn anything established, could find something better, some alternative, etc.\ or that something else is more worthwhile.

Without idealising, there's no way for the agent to avoid this kind of non-monotonicity.
However, same problem with beliefs, but then in the case of non-monotonic beliefs, are relations different?
What goes on here?

In the practical case there are two different kinds of defeasibility.
There's the means-end relation, or whatever complex conditional this is, which is more-or-less doxastic.
And, there's the relevant means being a `best' thing for the agent to do.
These two things are independent.




The dependency isn't so straightforward, though.
A similar dependency arises in the case of effects and causes.
Something cannot be an effect without a cause, but I do not need to understand the cause to regard the state as an effect.




% \begin{proposition}
%   Loss of information is complex.
%   \begin{proof}
%     Case in which new information is added in one domain which leads to loss of information in another.
%     Unclear that this is the same as loss of information.
%   \end{proof}
% \end{proposition}

\begin{proposition}
  When opening up the list for further consideration, the agent is, in effect, discarding any old information that they have.
  So, their prior reasoning no longer provides information.
  If this is the case, there does seem to be a problem, as the agent can't establish the action as a means, in absence of the shopping list (the list is the important bit of information).
  But why should this be the case?
  The list doesn't disappear!
  Similar with proofs, failure to reprove doesn't invalidate what has been established in absence of additional information.
\end{proposition}

\begin{proposition}
  Chance of doing something not supported by end is too high.
  But this seems to beg the question against the position.
  Or, alternatively, requires something too strong.
  Cases where reasoning is clear, but risk is involved.
\end{proposition}

\begin{proposition}
  End of doing what my past self instructed.
  But I scratched something from the shopping list earlier.
  Past self wrote something down, and I reject this.
  Pizza, but I reject this for something healthy.
\end{proposition}

\begin{wodge}[Explaning the puzzle with Setiya]
  \textcite{Setiya:2014aa} seems a good way to explain the puzzle.
  \citeauthor{Setiya:2014aa}'s claim is:
  \begin{quote}
    Reasons: The fact that \emph{p} is a reason for A to \(\phi\) just in case A has a collection of psychological states, C, such that the disposition to be moved to \(\phi\) by C-and-the-belief-that-\emph{p} is a good disposition of practical thought, and C contains no false beliefs.\nolinebreak
    \mbox{}\hfill\mbox{(\citeyear[12]{Setiya:2007aa})}
  \end{quote}
  The gist is that: `Whether a given fact is a practical reason has to do with the soundness of practical reasoning from that fact.' (\citeyear[223]{Setiya:2014aa})

  Now, the issue is that the agent lacks the ability to reason from an end to the means, and it seems as though the reason required here is the means-end relation.
  If \emph{p} is a means, then this is because there is a supporting means-end relation, and this is (perhaps) the only relevant fact which could count if favour of \emph{p}.
  (Almost, the relation is a reason, but not one the agent is able to use in their reasoning.)

  If \citeauthor{Setiya:2007aa} is right, and perhaps with a little more argument, the means-end relation is not a reason for \emph{p}, and so it seems as though the agent has no reason to \emph{p}, as \emph{p} is required to form the disposition.
  Hum, this isn't quite clear.
  It seems clear that if the agent had the belief that the means-end relation holds, then it would be a good disposition, so \emph{p} is a reason, but it's not a recognised reason.
  So, from \citeauthor{Setiya:2007aa}'s analysis of reasons it seems to follow that the means-end relation \emph{is} a reason.
  But, as the disposition to be moved to \(\phi\) relies on the belief that \emph{p}, this isn't a `recognised' reason.

  It seems \citeauthor{Setiya:2007aa}'s principle simply provides grounds for thinking that the means-end relation is a reason, and it's the additional factor that this reason is unavailable that needs to do some work, but \citeauthor{Setiya:2007aa} makes no claims about this.

  What I want to argue against is that (re-)settling should be thought of as an instance of applying reasons.
  And, \citeauthor{Setiya:2007aa}'s account here doesn't require this.
  \citeauthor{Hieronymi:2011aa} might be a useful supplement\dots
\end{wodge}

\newpage

[Somewhere, should note that there's a conflict with \citeauthor{Broome:2002aa}'s account of reasoning, as there's no way for the agent to do a simple kind of deduction, as there's no appropriate representation of the relevant end.
The agent doesn't have \(\phi \leadsto \psi\), which is why switching to \(E \leadsto \psi\) is considered, and the idea here is that this can still work.]

Assume that the agent is rational in settling to purchase the item on the shopping list.
Then, the agent has engaged in some instrumental reasoning from their ends to the action (in part) as a means.
Therefore, the agent has the belief that the action (in part) as a means is the best way to achieve their ends, and this belief is the relevant conditional required for a standard case of means-end reasoning.
For, the agent has the belief that the action (in part) as a means is the best way to achieve their ends, and they have their ends, hence the agent can conclude that taking the action (in part) as a means is the thing to do.

This does not work.
If the agent's reasoning is required to take this form, then the first instance of reasoning does not constitute a belief that the action (in part) as a means is the best way to achieve their ends.
Instead, it is a belief that if the agent has the ends and believes that the action (in part) as a means is the best way to achieve their ends then the taking the action (in part) as a means is the thing to do.
The belief that the action (in part) as a means is the best way to achieve their ends is now embedded in the antecedent of a conditional.

Response may be to generalise the account of practical reasoning.
Allow agent to reason from ends to means, this is what is given.
So, the agent can form the belief that the action (in part) as a means is the best way to achieve their ends.
However, one may argue, this can be recast as the agent reasoning from their ends and the belief to the action (in part) as a means being the thing to do.
Direct reasoning from ends to means can be recast as reasoning via a belief, and though there may be some distinction between the two instances, this distinction is insubstantial.
[Analogy, \(A \land B \vdash B\) can be recast as \(A \land B, (A \land B) \rightarrow B \vdash B\) --- the material conditional internalises the consequence relation, so every deduction can be seen as an instance of modus ponens.]

Roughly, the idea is that the agent reasons from ends to means, so we are given ends as input and the means as output.
And, as the agent will establish a relation from the ends to the means, the reasoning can either be part of means-end reasoning, or can be taken as an additional input.

This isn't so straightforward.
Reasoning from ends to means, is something that depends on the structure of the agent's ends.
Additions to an agent's ends can break the applicability of a means-end conditional, so the reasoning cannot be as straightforward as matching the agent's ends to what are embedded in the antecedent of the conditional, unless this is an `exact' account of the agent's ends.
That is, \(E, E \leadsto m \vdash m\), but \(E, e', E \leadsto m \not\vdash m\), as the addition of \(e'\) may favour some other means --- \(m\) is only best relative to the agent's ends as they are.
If one focuses on the agent's \emph{intended} ends then this is different, as now the intention will track relevant differences, but this doesn't work with the kind of reasoning I'm interested in.
In other words, focusing on an agent's intended ends may ensure the relevant structure is in place.
This, seems dubious.
For, addition of compatible intended ends can require some revision.
If I intend to buy some flour, then driving to the local groceries is the best means to this end, but if I also intend to walk my dog then walking is the best means given both these intentions.
Could argue that I now have the intention to buy flour and walk the dog, but not the intention to buy flour nor the intention to walk the dog, and \citeauthor{Bratman:1987aa} does argue for some kind of agglomeration\dots
But, this is subtle.
Intention to have flour and to walk the dog, then putting these together.
But then I should do this with disparate intentions too; there should be a single intention.
I intend to take my dog for a walk and to submit a paper to a journal.
These can be put under the joint intention of having walked my dog and having submitted a paper to a journal, but this doesn't seem right, as the two intentions may come apart in various ways.

The easy difference is that means-end relations are about how, and means-end reasoning is about whether.
Indeed, shopping list is a statement of how, in a sense, and the issue is if it is whether.



The two instances are distinct.
Need detachment in the second, the conditional, whatever logic it has, consists of an antecedent and a consequent.

\newpage

\section{Two ways of thinking about utility}
\label{sec:two-ways-thinking}

\begin{enumerate}[label=\Alph*)]
\item Known utility and uncertainty about whether the act will provide this utility.
\item Know act, but uncertainty about the utility that the act will provide.
\end{enumerate}

Illustration:

Hunger, and know the utility of satisfying hunger, but unsure whether the chocolate bar will satisfy hunger.

Eating the chocolate bar will have an effect, but unsure what that effect will be.

This distinction applies to actions that are under the control of the agent.
If there is uncertainty due to nature, then the distinction still applies, but it's less relevant.

On the first way of thinking, if there's no recognition of what the outcome would do, then there's no utility.
The only utility that matters is the utility that's projected, more-or-less, and without any projection of utility the outcome is irrelevant to the agent's reasoning.
Here, each act will have multiple outcomes, that of performing the act and satisfying the relevant desire, and that of performing the act and failing to satisfy the relevant desire.
On the other way of thinking, there's only a single outcome, speaking here of acts that are wholly under the agent's control, and that's having done the act.

\newpage

\section{What and how}
\label{sec:what-how}

The agent does not represent obtaining the star fruit as worthwhile.
An agent is only rational in performing an action if that action is represented is evaluated as worthwhile.

???

As the agent cannot reason from their end of trying carambola to the means of purchasing star fruit, if the act of purchasing star fruit was represented as worthwhile, then this would not be on the basis of the end of trying carambola.

Need a favourable attitude toward representation.
Without m-e relation the relevant representation can't be found because the representation of the star fruit isn't favourable.
Ned the favourable attitude toward carambola to combine, but agent doesn't have this.
The belief in a means-end relation isn't important as forming the relevant representation.
It may seem this is intuitive, if you can't help but to represent something in a certain way, then it seems that the belief isn't going to do any work.

Whether something some act is worthwhile is a matter of \emph{how} it is represented.
Whether an agent's belief that a state of affairs obtains is justified is a matter of \emph{how} the state of affairs is represented.
Simple example with any other instance of co-reference.

These attitudes are used to understand agent's, and we need the how to understand action.
What only goes so far.
Distinction between what and how only matters for the communication of attitude ascription; we aren't able to capture the relevant information, but it remains essential.
Right, Perry takes about the \emph{essential} indexical.
Functional role of belief can't be captured by what is believed.

So, it's clear that \emph{how} something is believed is important, and the basic idea is that this generalises to desires.
Further, whether something is desirable is a function of \emph{how}, and there are various cases which seem to illustrate this.
That is, we have some intuition and independent philosophical argument supporting some explanation along these lines.

Upshot: Means-end relation isn't so much about belief, but about representing.
With a means-end relation, one is able to \emph{represent} an act as a means to a particular end, and by this relation the desirability of the end transfers to the means.
Really, it's a complex representation of some kind.

This explains what's going on in with the shopping list, why the agent isn't rational, and what the agent needs in order to be rational.
Further, this is a strong failure of rationality.
Compare this to bootstrapping of intentions.
Here, the bootstrapping is irrational, but after this is done, we're able to make sense of the agent.
This is, at least in part, what makes the potential for bootstrapping interesting.
Same holds with belief, form a bad belief and what follows may be rational given that.
Here, however, there's no way to make sense of the agent acting, because the required representation isn't present.
There's no way to condition on irrationality to find something rational.

Note, of some interest, is that this argument doesn't require that an agent cannot question their representation capacities.
Rather, it requires that an agent cannot use this questioning to motivate action.
This creates a symmetry with akrasia, where an agent doubts their representation, but can't do anything with this, at least on some potential understanding of the issue (the belief doesn't do enough to transform the representation).






Positive argument:
Advice.
Symmetry in means-end relation, asymmetry in reasoning.
Else, asymmetry in both cases.


\newpage



Questions about practical reasoning:
\begin{enumerate}
\item What makes something worthwhile?
  \begin{itemize}
  \item This gets pushed around in various ways.
  \item Basic difference between objective and subjective evaluations, etc.
  \item Difference between desires and reasons understood in some other way.
  \end{itemize}
\item Is instrumental reasoning really \emph{reasoning}?
  \begin{itemize}
  \item This comes up in \citeauthor{Hume:2011aa} and \citeauthor{Smith:2004aa}.
  \item The core idea is that there's nothing that could genuinely be regarded as reasoning.
  \item At best, one has something about mental states.
  \end{itemize}
\item Is all practical reasoning instrumental?
  \begin{itemize}
  \item Is all practical reasoning from ends to means?
  \item Are there instances of reasoning in which an agent does not reason about how to achieve the ends they have, etc?
  \end{itemize}
\item Are there things other than ends which govern reasoning?
  \begin{itemize}
  \item This could be other sources of value, ends, etc.
  \item Or, it may be the case that some form of rationality constrains a rational agent, such as with the instrumental principle.
  \item I.e.\ Are there norms governing the reasoning?
  \item On the above point, see \textcite{Raz:2005aa} and \textcite{Kolodny:2008aa,Kolodny:2007aa}.
  \end{itemize}
\item Can reasoning generate new ends?
  \begin{itemize}
  \item Some difficulty in stating this without qualifications.
  \item Taking about ends is fairly non-committal, as these can be seen to mark dependency, and little more.
  \item Can see to get a new desire for the means, so better to say new non-instrumental desires.
  \item Unclear, however, what counts as new, e.g.\ in cases of co-reference.
  \end{itemize}
\item Desire as belief?
  \begin{itemize}
  \item The status of the attitude, more-or-less.
  \item This doesn't have too much to do with the structure of reasoning, with the exception that there are established ways of thinking about belief.
  \end{itemize}


\end{enumerate}


\newpage

It is unclear whether the agent in the shopping list scenario is rational.
There are many things that could influence our judgements of rationality.
For example, the age of the shopping list, the agent's history of forgetting things, whether or not the agent is able to recall ends for other items on the list, on so on.
The question I am interested in is not whether the agent in the shopping list scenario \emph{is} rational, but whether the agent in the shopping list example \emph{might be} rational.

Broadly stated, the question is:

If an agent recognises that they settled on an action as a means to some end, and is unable to reason from an end to the means, is it possible for the agent to be rational in performing the means?

If `yes', then it is not the case that an agent is rational in performing a action as means only if the agent settles on the action as the result of reasoning from an end to the means.

In the cases of interest the agent recognises that they had settled on performing the action as a means to some end, and reasons that whether or not performing the action as a means is worthwhile depends on whether the end is worthwhile.
This, I think, suggests that (as a necessary condition) in order for the agent to be rational, the agent must be think the end is worthwhile and think that a means-end relation holds between the action and the relevant end.

My current task is to figure out whether this is sufficient.

Below are notes on sets of notes.
The first provides a slightly different example.
The second attempts to establish a symmetric example with belief.

\subsection{Variation}
\label{sec:variation}

Consider gifts.
A close friend has a gift for the agent.
The specifics of the gift are left to the reader, so long as they satisfy the following constraints:
\begin{enumerate}
\item\label{gift:means} The gift requires the agent to perform an action as a means.
\item\label{gift:relation} The agent is unable to reason from an end they have to the means.
\end{enumerate}

For example, the gift may be an idea contained in a book.
The agent is required to read the book in order to grasp the idea, and the agent is unable to establish that the book would be (or not be) worth reading by the information that the book provides about it's contents.
Or, perhaps the gift is a unique taste, and the agent is required to consume part of a meal in order to experience the taste, and the meal is not immediately appealing (nor unappealing).

The question is whether the agent can rationally perform the means required to receive the gift without reasoning from an end they have to performing the required means.

The idea here is that the may agent `outsource' their means-end reasoning to their close friend.
That is, the agent may reason as follows:
\begin{enumerate}
\item My close friend has a good understanding of what I think is worthwhile.
\item My close friend reasoned from information about what I think is worthwhile to the gift.
\item So, it is likely that the gift is worthwhile.
\item If the gift is worthwhile and there is a means-end relation between the gift and the required action, then it is worthwhile to perform the action as a means.
\item So, it is worthwhile to perform the action as a means.
\end{enumerate}

This scenario may be contrasted to the agent explaining something that they think is worthwhile to their friend and their friend recommending the action as a means to what they think is worthwhile.
In this contrasting case the agent may not be able to reason from whatever they think is worthwhile to performing the action as a means, but their friend states a specific means end relation, while in the present scenario the information about the specific end the agent's friend has reasoned from is not provided to the agent.

There are various ways in which the agent could potentially reason from an end they recognise to performing the means required to receive the gift.
For example, the agent may have the end of pleasing their friend, performing actions that close friends think would lead the agent to something the agent considers worthwhile, and so on.

It is, of course, possible to assume that the agent has the end of performing actions that close friends think would lead the agent to something the agent considers worthwhile.
This, however, is not part of the reasoning given above, reformulated with this end the agent would reason as follows:

\begin{enumerate}
\item My close friend has a good understanding of what I think is worthwhile.
\item I have the end of performing actions that close friends think would lead me to something I consider worthwhile.
\item So, it is worthwhile to perform the action as a means.
\end{enumerate}

In the agent's initial reasoning there is no appeal to an end the agent has \emph{other} than the end identified by the agent's friend.
The agent's friend does not inform the agent of which end the friend has identified, and it is assumed that the agent would not be able to reason from this to performing the action as a means even if the agent considers the relevant end when thinking about the means.

So, while there may be alternative explanations for why the agent may be rational in performing the means, the question is whether there is something problematic with the suggested reasoning.
That is, the task, is to demonstrate that the reasoning sketched above could not support the agent rationally performing the action as a means.

In the reasoning, it seems the agent takes their confidence that the relevant end is worthwhile to justify performing the means.
What is distinctive about the agent's reasoning is that the agent cannot specify what this justification is.

\citeauthor{Smith:2004aa} presents a sceptical, \citeauthor{Hume:2011aa}an argument against practical reasoning of this form, regardless of whether the agent is able to identify the appearance of justification or otherwise.
\begin{quote}
  \dots in the practical case we cannot remain faithful to the idea that the rationality of a psychological transition must have something to do with the possibility of there being reasons for making that psychological transition.

  \dots

  There is no such thing as means-end rationality: talk of ``means-end reasoning'' remains an oxymoron.
  There is simply the human habit of forming desires for means on the basis of desires for ends and beliefs about means, a habit that is underwritten by neither reasons nor rationality.\nolinebreak
  \mbox{}\hfill\mbox{\citeyear[88]{Smith:2004aa}}
\end{quote}

If means-end reasoning is an oxymoron, then the question of whether the agent is rational or not based on some means-end reasoning is mistaken.
And, if means-end reasoning is not an oxymoron, then this sceptical argument is too strong to isolate the kind of reasoning suggested above as irrational.

As \citeauthor{Smith:2004aa} notes, appeal to coherence relations does not require transmission of justification and may be compatible with \citeauthor{Hume:2011aa}'s sceptical arguments to the letter.
However, coherence relations may also be functionally equivalent to transmissions of justification and may therefore be incompatible with the spirit of \citeauthor{Hume:2011aa}'s sceptical arguments.

Bracketing scepticism, then either way (whether by transmission of justification or by relations of coherence) the task is to find a problem with the agent's reasoning.

\subsection{Creating a similar example with belief}
\label{sec:creat-simil-example}

I suspect that the puzzle about the agent's reasoning isn't due to the agent engaging in means-end reasoning.
The focus on means-end reasoning and means-end relations is useful because it is clear that the agent can settle on the means if they are aware of a means-end relation between the means and an end they have.
I'm less clear on intuitions about this kind of dependency relation in the case of belief.
Still, with some work I think a similar case can be constructed.
(I very much doubt this is the best case nor the simplest, and it's certainly not particularly realistic, but it does seem straightforward enough to puzzle over.)

The basic idea is to build a scenario based on illustrations of zero-knowledge proofs.
I'll start with the idea of these proofs, and then build the relevant scenario.

\subsubsection{Illustration of zero-knowledge proofs}
\label{sec:illustr-zero-knowl}

\citeauthor{Quisquater:1989aa} provide a simple example titled `The Strange Cave of Ali Baba'.
What follows is a rough outline.

We have a cave with a single entrance.
The entrance leads to a passage which forks, and both forks lead to the same door.
So, if you enter the cave and take the left fork and pass through the door you'll arrive back at the fork.
Likewise, if you enter the cave and take the right fork and pass through the door you'll arrive back at the fork.
The door, however, is guarded by a password.
Agent \(A\) claims to know the password for the door, and agent \(B\) wishes to establish this.
Agent \(A\) could tell agent \(B\) the password, but agent \(A\) does not want to reveal the password to agent \(B\).
Perhaps agent \(A\) doesn't want agent \(B\) to know the password, or perhaps agent \(A\) is worried that someone might overhear them saying the password to agent \(B\), these details aren't too important.
What matters is that agent \(A\) can demonstrate that they know the password without revealing what it is.

The method is straightforward.
Agent \(A\) enters the cave and takes either the left or right fork, after allowing agent \(A\) enough time to have taken a one of the forks, agent \(B\) then enters the cave.
Agent \(B\) then loudly shouts which fork they want agent \(A\) to appear from chosen at random.
If agent \(A\) has taken the fork agent \(B\) wants them to appear from, then agent \(A\) retraces their steps, but if agent \(A\) has taken the alternative fork then agent \(A\) must pass through the door which requires the password.

So, if agent \(A\) knows the password, then agent \(A\) will always be able to appear from the fork agent \(B\) requests, and if not then on each trip to the cave there is a 50\% chance that agent \(A\) will be unable to appear from the fork agent \(B\) requests.
Hence, the two agents can enter and exit the cave as many times as agent \(B\) requires in order to be confident that agent \(A\) knows the password, but agent \(B\) is never informed by agent \(A\) of what the password is.

It seems to me clear to me that by following this method agent \(B\) can form a justified belief that agent \(A\) knows the password.

\subsubsection{Scenario}
\label{sec:scenario-2}

In the case of means-end reasoning the agent has information that some action is a means, but is unable to recall the means-end relation that they used to establish the action as a means.
In the case of belief it's unclear that an agent forgetting the way in which they formed a belief should lead the agent to be puzzled about whether or not they have the belief.
Further, in the case of means-end reasoning the puzzle is whether the agent can adopt the means, and so for an appropriate parallel it seems I need a case in which an agent is unsure of what justification they have for a belief, but may still be rational in forming the belief.
So, the basic idea of the scenario is to adopt the illustration of zero-knowledge proofs given above to provide the information that they had justification for a belief without revealing what that information is.

Suppose we have an agent who has a passion for co-reference.
The agent has two endless list of failures and instance of co-reference in the form of a pair of names.
One of lists contains pairs the agent is aware of their justification for whether or not the pair of names is an instance of co-reference.
The other list contains pairs for which the agent is unaware of their justification for whether or not the pair of names is an instance of co-reference.
The agent's considers that they may have justification for an instance of co-reference, but is unable to identify whatever that justification is.
The question is whether the agent has justification for believing this particular instance of co-reference.\nolinebreak
\footnote{Similar to the case of means-end reasoning, the issue is not that the agent cannot possibly reason from the missing information to the belief/means, but that they are unable to in the context of the scenario.}

However, the agent has a companion who claims they keep track of whether the agent has justification for each of beliefs about co-reference that the agent has.
As the agent has an endless list of failures and instances of co-reference for which they are aware of whether or not they are justified.
Therefore, so long as the agent's companion correctly identifies which pair belongs to which list, the agent can be as confident as they wish that their companion is reliable (in line with the illustration of the sketch of zero-knowledge proofs given above).

Assuming the agent's companion does keep track of whether or not the agent has justification, then what is the agent to conclude from the agent's companion claiming that they have justification for the instance of co-reference that the agent is unsure about?

It seems the agent has a justified belief that their companion keeps track of whether or not the agent has justification, and that the agent therefore can come to have a justified belief that they have justification for the instance of co-reference that the agent is unsure about.
Therefore, it seems that the agent can come to have the belief that the pair of names is co-referential, yet the agent is (in the context of the scenario) unaware of what that justification is.

Perhaps there is a way to block the inference from the agent (recognisably) justified belief that they have (unrecognised) justification for believing the instance of co-reference to the belief that the instance of co-reference is true, but I'm not sure what would support this.

The simple way out is to argue that the agent's companion cannot answer `yes' to the agent's question of whether or not the agent has justification for the instance of co-reference they are unsure about.
This, however, does not seem reasonable.
I reason from some premises to a conclusion, and write this reasoning to paper, but I am undecided on whether I have a proof of the conclusion from the premises.
Peer review responds that the reasoning is indeed a proof.
It does not seem as though my failure to be able to verify my own reasoning cancelled the justification it provided for the conclusion, and likewise self-confidence would not have provided justification.
The difficulty isn't with justification, but my recognition of the justification I have.

It seems the underlying problem is the same, as the lists of instances of co-reference can be replaced with actions, and belief and justification with means and means-end relations.

\newpage

\printbibliography

\end{document}
