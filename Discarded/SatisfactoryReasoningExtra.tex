\documentclass[10pt]{article}
% \usepackage[margin=1in]{geometry}
% \newcommand\hmmax{0}
% \newcommand\bmmax{0}

% % % Fonts% %
\usepackage[T1]{fontenc}
   % \usepackage{textcomp}
   % \usepackage{newtxtext}
   % \renewcommand\rmdefault{Pym} %\usepackage{mathptmx} %\usepackage{times}
\usepackage[complete, subscriptcorrection, slantedGreek, mtpfrak, mtpbb, mtpcal]{mtpro2}
   \usepackage{bm}% Access to bold math symbols
   % \usepackage[onlytext]{MinionPro}
   \usepackage[no-math]{fontspec}
   \defaultfontfeatures{Ligatures=TeX,Numbers={Proportional}}
   \newfontfeature{Microtype}{protrusion=default;expansion=default;}
   \setmainfont[Ligatures=TeX]{Minion 3}
   \setsansfont[Microtype,Scale=MatchLowercase,Ligatures=TeX,BoldFont={* Semibold}]{Myriad Pro}
   \setmonofont[Scale=0.8]{Atlas Typewriter}
   % \usepackage{selnolig}% For suppressing certain typographic ligatures automatically
   \usepackage{microtype}
% % % % % % %
\usepackage{amsthm}         % (in part) For the defined environments
\usepackage{mathtools}      % Improves  on amsmaths/mtpro2
\usepackage{amsthm}         % (in part) For the defined environments
\usepackage{mathtools}      % Improves on amsmaths/mtpro2

% % % The bibliography % % %
\usepackage[backend=biber,
  style=authoryear-comp,
  bibstyle=authoryear,
  citestyle=authoryear-comp,
  uniquename=false,%allinit,
  % giveninits=true,
  backref=false,
  hyperref=true,
  url=false,
  isbn=false,
]{biblatex}
\DeclareFieldFormat{postnote}{#1}
\DeclareFieldFormat{multipostnote}{#1}
% \setlength\bibitemsep{1.5\itemsep}
\addbibresource{Thesis.bib}

% % % % % % % % % % % % % % %

\usepackage[inline]{enumitem}
\setlist[itemize]{noitemsep}
\setlist[description]{style=unboxed,leftmargin=\parindent,labelindent=\parindent,font=\normalfont\space}
\setlist[enumerate]{noitemsep}

% % % The following section relates to theorems, etc. % % %
\usepackage{thmtools}

\declaretheoremstyle[
spaceabove=6pt, spacebelow=6pt,
headfont=\normalfont\bfseries,
notefont=\mdseries, notebraces={(}{)},
bodyfont=\normalfont,
% postheadspace=1em,
% qed=\qedsymbol
]{defstyle}

\declaretheoremstyle[
spaceabove=6pt, spacebelow=6pt,
headfont=\normalfont\bfseries,
notefont=\normalfont\bfseries, notebraces={}{},
bodyfont=\normalfont,
% postheadspace=1em,
% qed=\qedsymbol
]{defsstyle}


\declaretheoremstyle[
spaceabove=6pt, spacebelow=6pt,
headfont=\normalfont\bfseries,
notefont=\normalfont\bfseries, notebraces={}{},
bodyfont=\normalfont\color{red},
% postheadspace=1em,
qed=\qedsymbol
]{notestyle}

\declaretheorem[name=Theorem,numberwithin=section]{theorem}
\declaretheorem[sibling=theorem,style=remark]{remark}
\declaretheorem[sibling=theorem,name=Corollary]{corollary}
\declaretheorem[sibling=theorem,name=Lemma]{lemma}
\declaretheorem[sibling=theorem,name=Fact]{fact}
\declaretheorem[sibling=theorem,name=Proposition]{proposition}
\declaretheorem[sibling=theorem,name=Definition,style=defstyle]{definition}
\declaretheorem[sibling=theorem,name=Assumption,style=defstyle]{assumption}
\declaretheorem[name=Definitions,numbered=no,style=defsstyle]{definitions}
\declaretheorem[sibling=theorem,name=Example,style=defstyle]{example}
\declaretheorem[name=Note,style=notestyle]{note}
\declaretheorem[name=Ramble,style=notestyle]{ramble}
\declaretheorem[name=Scenario,style=defstyle]{scenario}
% % % % % % % % % % % % % % % % % % % % % % % % % % % % % %

% % % Misc packages % % %
\usepackage{setspace}
% \usepackage{refcheck} % Can be used for checking references
% \usepackage{lineno}   % For line numbers
% \usepackage{hyphenat} % For \hyp{} hyphenation command, and general hyphenation stuff

% % % % % % % % % % % % %

% % % Red Math % % %
    \usepackage[usenames, dvipsnames]{xcolor}
    % \usepackage{everysel}
    % \EverySelectfont{\color{black}}
    % \everymath{\color{red}}
    % \everydisplay{\color{black}}
% % % % % % % % % %

\usepackage{pifont}
\newcommand{\hand}{\ding{43}}
\usepackage{array}
\usepackage{epigraph}


\usepackage{multirow}
\usepackage{adjustbox}
\usepackage{verse}


\usepackage{titlesec}



\makeatletter
\newcommand{\clabel}[2]{%
   \protected@write \@auxout {}{\string \newlabel {#1}{{#2}{\thepage}{#2}{#1}{}} }%
   \hypertarget{#1}{#2}
}
\makeatother

\newcommand{\boxarrow}{%
  \mathrel{\mathop\Box}\mathrel{\mkern-2.5mu}\rightarrow
}
\newcommand{\diamondarrow}{%
  \mathrel{\mathop\Diamond}\mathrel{\mkern-2.8mu}\rightarrow
}


\titleclass{\subsubsubsection}{straight}[\subsection]

\newcounter{subsubsubsection}[subsubsection]
\renewcommand\thesubsubsubsection{\thesubsubsection.\arabic{subsubsubsection}}
\renewcommand\theparagraph{\thesubsubsubsection.\arabic{paragraph}} % optional; useful if paragraphs are to be numbered

\titleformat{\subsubsubsection}
  {\normalfont\normalsize\bfseries}{\thesubsubsubsection}{1em}{}
\titlespacing*{\subsubsubsection}
{0pt}{3.25ex plus 1ex minus .2ex}{1.5ex plus .2ex}

\makeatletter
\renewcommand\paragraph{\@startsection{paragraph}{5}{\z@}%
  {3.25ex \@plus1ex \@minus.2ex}%
  {-1em}%
  {\normalfont\normalsize\bfseries}}
\renewcommand\subparagraph{\@startsection{subparagraph}{6}{\parindent}%
  {3.25ex \@plus1ex \@minus .2ex}%
  {-1em}%
  {\normalfont\normalsize\bfseries}}
\def\toclevel@subsubsubsection{4}
\def\toclevel@paragraph{5}
\def\toclevel@paragraph{6}
\def\l@subsubsubsection{\@dottedtocline{4}{7em}{4em}}
\def\l@paragraph{\@dottedtocline{5}{10em}{5em}}
\def\l@subparagraph{\@dottedtocline{6}{14em}{6em}}
\makeatother

\newcommand{\sem}[1]{\ensuremath{[\kern-.5mm[{#1}]\kern-.5mm]}}

\setcounter{secnumdepth}{4}
\setcounter{tocdepth}{4}

% \titleclass{\todopar}{straight}[\section]
% \newcounter{todopar}
% \renewcommand{\thetodopar}{\Alph{todopar}.}
% \titleformat{\todopar}[runin]{\normalfont\normalsize\bfseries\color{WildStrawberry}}{\thesection.\thetodopar}{\wordsep}{}
% \titlespacing*{\todopar} {\parindent}{3.25ex plus 1ex minus .2ex}{1em}

\usepackage{tikz}
\usetikzlibrary{arrows,positioning}

\usepackage[hidelinks,breaklinks]{hyperref}

\title{Aberrant desire}
\author{Benji Sparkes}
% \date{ }

\begin{document}

\maketitle


{\color{red}
  I want the following when I've got the talk of structure in place.
  It's that this `objective' nature of satisfaction allows us to talk about information in this way.
}
A difficulty in our talk about participation without representation is that the information available to an agent may be distinct from the information available to us when describing the agent.
Still, this parallels belief.
You may believe that it is raining in Z-land by believing that \emph{it is raining} without having any representation of Z-land given the isolated group to which you belong because your situation suitably fixes the participation of Z-land in your belief and we have the resources to make this participation explicit.
(\cite[cf.][]{Perry:1986aa})


{\color{red}
  The key difference, I think, between what I'm proposing and the ideas of \citeauthor{Pettit:1990aa} and \citeauthor{Schroeder:2007aa} is that for these authors desire is part of the agent's reasoning, it's something that's there.
  However, on the satisfaction view this isn't part of practical reasoning, it's something independent, and it's the orientation of reasoning that distinguishes it as practical.
}


On the view we endorse, practical reasoning can lead to the formation of desires through an agent reasoning that they would be satisfied were certain circumstances to obtain.
And, the puzzle of participation without representation reduces to a straightforward situation in which an agent is unable to infer the conditions under which they would be satisfied but has appropriate premises to expect that some aspect of a situation they are able to bring about will lead to their being satisfied.
{\color{red}
  Ah, so it's important that there's partial information.
  This, perhaps, links to cases of failures of introspection.
  Of course, failure of introspection here is something different.
  Though, I don't think I want to explore the issue in quite so much detail at this point in time.
  Right here the issue should be on some of the more Humean aspects.
}



{\color{red}
  Here I want to say something about having a desire to figure out what you desired.
  This can happen, but it doesn't need to, and the way I've set the case up is quite unreflective.
  But even if it does, this doesn't really explain anything, you have no way of representing that desire, and that's the problem.
  It's also that this doesn't get the satisfaction conditions correct.
  Either you then are satisfied when you find out, and you reform the desire or it was there all along.
}

\section{Desire Satisfaction}
\label{sec:desire-satisfaction}

There are two ways to look at the propositional account of desire.
The first is to see the basic attitude, and think that this is what constitutes the desire.
Satisfaction is then a placeholder for whatever it is that happens when the proposition desired is true.

This may derive from the old-school way of looking at things, but has been rejected by more modern Humean theories.
Still, it's there in guise-of-the-good type theories.
May also think that this is important to distinguish belief from acceptance, etc.
That is, what you believe and accept are both correct if the proposition is true (and there's a reference I can dig out here).

The second is to take satisfaction as basic.
Here, the conditions under which a proposition is true characterise the conditions under which an agent is (or expects to be) satisfied.
For an agent to desire a proposition is for an agent to recognise conditions under which they would be satisfied.
Satisfaction takes a central role.



\section{Schroeder's Taxonomy}
\label{sec:schroeders-taxonomy}

\begin{quote}
  \textbf{Standard Theory 1 (ST1)}:
  To desire that \(P\) is to be disposed to bring it about that \(P\).\nolinebreak
  \mbox{ }\hfill(\citeyear[11]{Schroeder:2004aa})
\end{quote}

So, good for following \citeauthor{Ryle:1949aa} where dispositional ascriptions license inferences as these don't need to attribute representations.

\begin{quote}
  To desire that \(P\) is to be disposed to act in ways that would tend to bring it about that \(P\) in a world in which one's beliefs, whatever they are, were true.\nolinebreak
  \mbox{ }\hfill(\citeyear[15]{Stalnaker:1984aa})
\end{quote}
{\color{red} correlative dispositional states of a potentially rational agent, the tendency-to-bring-about relation. But, (\citeyear[18]{Stalnaker:1984aa}) for \citeauthor{Stalnaker:1984aa} discussing the role of content. [22--23] also reinforces this.
    \begin{quote}
      It is essential to rational activities such as deliberation and investigation that the participants represent alternative possibilities, and it is essential to the role of beliefs and desires in the explanation of action that the contents of those attitudes distinguish between the alternative possibilities.\nolinebreak
      \mbox{ }\hfill(\citeyear[23]{Stalnaker:1984aa})
    \end{quote}
  }
  
\begin{quote}
  \textbf{ST2}:
  To desire that \(P\) is to be so disposed that, if one were to believe that taking action \(A\) would be an effective method for bringing it about that \(P\), then one would take \(A\).\nolinebreak
  \mbox{ }\hfill(\citeyear[17]{Schroeder:2004aa})
\end{quote}

\begin{quote}
  \textbf{ST3}:
  To desire that \(P\) is to have a structure inside one whose biological function is to bring it about that \(P\).\nolinebreak
  \mbox{ }\hfill(\citeyear[18]{Schroeder:2004aa})
\end{quote}

{\color{red}
  I don't really see any reason to go through with this account.
}

\begin{quote}
  \textbf{ST4}:
  To desire that \(P\) is to have a mental representation that \(P\) which plays a certain causal role, namely, that of disposing one to bring it about that \(P\).\nolinebreak
  \mbox{ }\hfill(\citeyear[24]{Schroeder:2004aa})
\end{quote}

\begin{quote}
  A person has a desire in the directed-attention sense that \(P\) if the thought of \(P\) keeps occurring to him or her in a favorable light, that is to say, if the person’s attention is directed insistently toward considerations that present themselves as counting in favor of \(P\).\nolinebreak
  \mbox{ }(\citeyear[39]{Scanlon:1998aa})
\end{quote}

However, here \citeauthor{Scanlon:1998aa} takes this account of desire to correspond to the intuitive sense, and as such this doesn't correspond to the placeholder pro-attitude view.

\citeauthor{Scanlon:1998aa} does offer a sort of account of attitudes around (\citeyear[20]{Scanlon:1998aa}), but this is heavily idealised, and it's not clear to me what to get from this.
But, I guess the problem is that now fixed it's not judgement sensitive.


\begin{quote}
  \textbf{Hedonic Theory 1 (HT1)}:
  To desire that \(P\) is to be so disposed that one will tend to feel pleasure if it seems that \(P\), and/or displeasure if it seems that not-\(P\).\nolinebreak
  \mbox{ }\hfill(\citeyear[27]{Schroeder:2004aa})
\end{quote}

\begin{quote}
  \textbf{HT2}:
  To desire that \(P\) is to contain some structure (this being the desire) which so disposes one that one will tend to feel pleasure if it seems that \(P\), and/or displeasure if it seems that not-\(P\).\nolinebreak
  \mbox{ }\hfill(\citeyear[27]{Schroeder:2004aa})
\end{quote}


\begin{quote}
  \textbf{Reward Theory of Desire (RTD)}:
  To have an intrinsic (positive) desire that \(P\) is to use the capacity to perceptually or cognitively represent that \(P\) to constitute \(P\) as a reward.
  To be averse to it being the case that \(P\) is to use the capacity to perceptually or cognitively represent that \(P\) to constitute \(P\) as a punishment.\nolinebreak
  \mbox{ }\hfill(\citeyear[131]{Schroeder:2004aa})
\end{quote}



\section{Slaves of the Passions}
\label{sec:slaves-passions}

\begin{description}
\item[Desire]
For \(X\) to have a desire whose object is \(P\) is for \(X\) to be in a psychological state grounding the following disposition: when for some action \(a\) and proposition \(r\) believed by \(X\), given \(X\)'s beliefs \(r\) obviously helps to explain why \(X\)'s doing \(a\) promotes \(P\), \(X\) finds \(r\) salient, and this tends to prompt \(X\) to do \(a\), and \(X\)'s attention is directed toward considerations like \(r\).
\end{description}

{\color{red}
  Okay, I need to get clear on exactly what's going on here.
  I've got a distinction between \emph{how} and \emph{what} and this is good.
  However, it now comes to stating what's going on with motivation.
  The desire cannot be the reason why you act, in the sense that it's not the thing you take to be going on in your practical reasoning.
  Yet, if there wasn't that desire you're reasoning wouldn't happen in the way it does.
  So, there's a difference between the how and the what, and here there's really not even a `how' in the standard sense of the term.
  So, the desire only has an indirect role in your practical reasoning.
  It gives you the means that you currently take to be relevant in determining what to do, and it also gives the conditions under which you'll be satisfied.
  However, clearly it cannot be part of your deliberation.
  So, this is the puzzle stated.
  The part I need to emphasise is that when you hear the song again you're able to discharge the means, but this might require some work on motivating some of the practical reasoning.
}

{\color{blue}
  The puzzle, then, is that given this analysis, there are desires which have a role in practical reasoning which is indirect.
  So, we won't get a good analysis of desire if we focus on the things you reason about.
  In other words, an analysis of reasons for action will not provide an analysis of practical reasoning.
  That is, there's no complete story of practical reasoning which pieces together reasons for action with some decision procedure.
  So, can't take desires as `givens'.
  Instead, need to reason about (expected) satisfaction.
  That's what's going on in the scenario.
  You're reasoning that there's something which you expect to satisfy you to which you have the means.
}

{\color{green}
  Oh, and in cases of akrasia you're reasoning about something which you expect to satisfy you but for which you don't have the means.
  This, is really nice.
  (Though also rather sad.)
}

{\color{blue}
  Well, then, given the above relations to propositions are still the core of desire, and the idea is that having access to representational content is what allows agents to actually do things.
  Well, it allows us to do thing in the way we do them.
  The motivational component comes first, and we use practical reasoning to guide this.
  (This is really what deals with the problem of akrasia, and the proposal only makes sense in this context.)
  And, the way the argument is going is that there's a way in which we can understand motivation without resorting to a given proposition.
  For better or worse we can't really reason without some representation content, and the interesting thing about us is that we are able to do this.
  
}


% What explains why you are searching through the menu at the restaurant is your desire for a particular meal.
% But your desire has no representational content.
% Had you mentioned to your partner that you'd been thinking about the particular meal they could have explained to you why you're there.
% Or, if you'd taken down the name of the person you talked to on the telephone you could have asked them about your request.
% Still, sitting as you are you can only that you desire a particular meal, but that you don't know which.


{\color{red} Problem is that the counterfactual isn't obviously true.
  For example, taking the dispositional states, there's no clear proposition that the agent stands in a relation to, but this shows that the desire has simply been corrupted.
  Then, it reduces to a problem of introspection.}

{\color{blue} Well, the general idea is that it seems as though this is a problem with introspection, and when we grant that introspection can fail, there's no problem with specifying various desires.
  So, it's the case that the puzzling desires really aren't the same desires without representational content, but that they're different desires.
  In short, the continuity is somewhat misplaced.}



\section{Intro}
\label{sec:intro}

Getting the TV channel wrong and enjoying a film.
This is a case where the extrinsic desire is fine, but there's a problem.
You think you're satisfying a desire, but really you're finding out that something else satisfies the same desire.
Or, two YouTube links or whatever, one copied and you fail to copy the second.
Either way, you now have more information about what satisfies you.

So, that's the proposal.
Desire something because one reasons that it may satisfy them.

In these examples you're reasoning about what would satisfy you, in the same way that you may reason about what is true when you try to recall some fact that you've forgotten.
However, if this is the case then desires can't be `givens'.

The key is that in these scenarios one is trying to recreate the content.
One deosn't use the other desires one has.
There's cognitive significance without content.
Though, this term isn't quite right.

Incohate desires.

Then, washing machine example.
You lose some of the relevant propositional information.

Conversely, some other example where you refine the desire into something concrete.
Of course, here we can speak of desires being formed, but there's an appealing symmetry.
Indeed, in the relevant example it's simply going to be reading through the spec sheet, desiring something with a something or other spin.


So, what's going on?
Well, either you recover the content, or you start to construct a new desire, arguably.


\section{Some observations}
\label{sec:some-observations}

Desires really aren't important for what I have in mind.
One can take anything they want which would serve the function of desire, so reasons for action in a broad sense.
The question is about the status of these.
And, the basic idea is that there's a problem with taking these as basic, and further understanding success conditions.
Taking them as basic does, for sure, allow one to state the success conditions.
But, this is not the way we reason.
It's the success conditions themselves which keep us ticking.
It's trying to get a handle on this.
And, this leads into the weak and strong theses.
The weak is that the have access to success conditions at a certain point in our reasoning, and additional reasoning can lead us away from these.
The strong is that sometimes we don't even have access to the success conditions.

This comes out in the case of belief.
It's the fixing the stopwatch via the pendulum and via the church.

Reasoning, then, is structuring information around all of this.
What I come to form regarding pro-attitudes based on this kind of stuff doesn't amount to an intention, it's not necessarily in the slightest robust, and that's part of the point (and this is how the concerns link into the earlier paper).
However, it does suggest that desire-belief based reasoning has a lot to say for it.

This, at least, is the start.
Then there's some deep question about how this works out in the case of pracitcal reasoning, which is certainly something more involved.

\section{The Idea}
\label{sec:idea}

The basic idea is that there's what goes on in the mind of an agent.
Here, it's not just beliefs and desires.
It's something to do with the flow of information.
But there's information from two distinct sources.
One is the world, the other is the self, so to speak.
Trouble is, this information is noisy.
But this is only part of the problem, the other part is that this information is sometimes sparse.
There's a difference between sailing a small boat on a lake where one can see the shore, and sailing a boat on the ocean.
This is like walking down a path in comparison to walking through some woods.
Something like this, what's interesting about these cases, in some respect, is that they're to do one's location, but I don't think this is an important feature.


The core of the idea is that there's transition between information states, and that there doesn't need to be a causal connexion between the information state and the world.


\subsection{Weak thesis}
\label{sec:weak-thesis}

Assuming that one forms desires and beliefs in a way which ensures that appropriate states are proper, so to speak.
That one can have transitions between `desire states' which can diverge from the underlying (or perhaps better put motivating) desire.
This, arguably, parallels cases of belief.

One way to view this week thesis is the idea that one recognises a desire to \(\phi\), but \(\phi\)ing is something in the future, and as the time of \(\phi\)ing approaches, one loses track of \(\phi\) but knows (or believes) that one of \(\psi_{i}\) will entail \(\phi\).
The question, then, is which \(\psi_{i}\) to choose.
This is a problem, and a belief will step in, but it's not a straightforward case of a means-end belief as there's a problem in determining what the end is.

One example is buying a present.
Or, to go with meals, remembering you ordered something that you really enjoyed, but not being able to recall the name.

You are, then, acting on a desire but you have no idea what the propositional content of that desire is.

The alternative way for accounting for this kind of case is by supposing that there's a desire for each of the toys or menu items, and your reasoning attempts to adjudicate between these.
But I find it hard to really understand how this works here.

The idea is to contrast the third person and the first person cases.

There's also the case where someone forms a desire, and then the beliefs on which they form this desire no longer hold.
Ray had faculty with boots, I had adviser and quoting Shakespeare.
The puzzling thing about this is that the person is able to cite the certain determinate conditions, but none of these capture the desire.
Any condition stated doesn't work, unless they say that they want red boots.
Perhaps this is the way to go, but then an instrumental desire has turned into a self-standing desire.
Ah, in this sense I like Shakespeare case, as what's different is that the desire looks in some sense irrational.
Right, and with the Shakespeare they've started quoting Milton outside the office of the new chair.

\subsection{Strong thesis}
\label{sec:strong-thesis}


The weak thesis, but also denies that there's this requirement for appropriate states.
In the case of belief, this is forming the attitude independently of any evidence, so the parallel is thinking I have a fix on the pendulum by looking at the church.

The strong thesis, though, is not a \emph{very} strong thesis.
It still allows that there are these appropriate states, in that one does get hungry, or latches on to a particular normative reason.

The problem, then, is why these should be allowed.
It's possible to posit further reasons, reasons `in the background' so to speak.
Be this as it may, this blocks the kind of reasoning that I've been describing.
So, what does this explain?
Well, it's a mix between the sort of learning theory that Dretske goes for and a more robust type of reasoning which Goldman endorses.

Potentially motivating cases here involve enhancements.


\subsection{Objection}
\label{sec:objection}

The key objection, at least to my understanding, is that it seems possible to have some instance of reasoning which is \emph{about} or \emph{concerns} satisfaction without this having some motivational component.
This, then, pushes the inclusion of some desire proper, which stands behind the reasoning, and explains its motivational force.

So, there's some quality of the mental which makes certain states motivationally effective.
And, while the above distinction does some work, it doesn't capture this distinction.

The analogy I then want is with cases in which certain recognised beliefs aren't motivationally effective.
That is, an argument that there's something missing in our talk of belief.
So, in a sense, beliefs which don't enter into the causal nexus of things, but are likewise indistinguishable from other beliefs.
Examples involving phobias seem to be a good candidate for this.
One has a very specific motivational desire.
But, despite believing that this thing isn't an instance of what they have a phobia toward, they still act as though it was.
It's quite plausible, I want to argue, that the belief isn't correctly hooked up to the causal mechanisms.
And, it doesn't make sense to explain this via some other desire.
Say, they've done this before.
Actually, right, the agent is there and avows all of the `right things'.
Further, could enrich so that these types of cases have happened before, and that eventually the belief does make it's way.





\section{Tests}
\label{sec:tests}

Typically we think of propositions as sets of worlds.
E.g.\ \(\phi = \{w_{i}, \dots, w_{k}, \dots\}\) where the it is the commonalities between \(w_{i}, \dots, w_{k}, \dots\) which determine the meaning of \(\phi\).
This, however, isn't the full story, as we can likewise think of propositions as \emph{tests}, functions from (sets of worlds) to truth values.
From a certain perspective, this is what's captured by the use of the lambda calculus.
One can take \(\lambda w.\phi(w) = \top\), etc.\

But, this isn't quite the full story.


\begin{figure}[h]
  \centering
  \begin{tikzpicture}
    \node[draw] (sat) {\(\mathcal{S}\)};

    \node[draw, above right=of sat, rectangle] (inf) {\(\sigma\)};

    \node[draw, left=of sat] (world) {\(\omega\)};

    \node[draw, above left=of inf] (exp) {\(\mathbb{E}\)};

    \node[above right=of inf] (ext) {};

    \node[above=of ext] (ext2) {};

    \draw[->, to path={-| (\tikztotarget)}] (sat) edge (inf);
    \draw[->, to path={|- (\tikztotarget)}] (world) edge (sat);
    \draw[->, to path={-- (\tikztotarget)}] (exp) edge (sat);
    \draw[->, to path={|- (\tikztotarget)}] (ext) edge (inf);
    \draw[->, to path={|- (\tikztotarget)}] (inf) edge (exp);
    \draw[to path={-- (\tikztotarget)}, dashed] (ext) edge (ext2);

  \end{tikzpicture}
  \caption{Cards on the table}
  \label{fig:cards}
\end{figure}



\newpage
\printbibliography

\end{document}