\documentclass[10pt]{article}
% \usepackage[margin=1in]{geometry}
% \newcommand\hmmax{0}
% \newcommand\bmmax{0}
% % % Fonts% %
\usepackage[T1]{fontenc}
   % \usepackage{textcomp}
   % \usepackage{newtxtext}
   % \renewcommand\rmdefault{Pym} %\usepackage{mathptmx} %\usepackage{times}
\usepackage[complete, subscriptcorrection, slantedGreek, mtpfrak, mtpbb, mtpcal]{mtpro2}
   \usepackage{bm}% Access to bold math symbols
   % \usepackage[onlytext]{MinionPro}
   \usepackage[no-math]{fontspec}
   \defaultfontfeatures{Ligatures=TeX,Numbers={Proportional}}
   \newfontfeature{Microtype}{protrusion=default;expansion=default;}
   \setmainfont[Ligatures=TeX]{Minion 3}
   \setsansfont[Microtype,Scale=MatchLowercase,Ligatures=TeX,BoldFont={* Semibold}]{Myriad Pro}
   \setmonofont[Scale=0.8]{Atlas Typewriter}
   % \usepackage{selnolig}% For suppressing certain typographic ligatures automatically
   \usepackage{microtype}
% % % % % % %
\usepackage{amsthm}         % (in part) For the defined environments
\usepackage{mathtools}      % Improves  on amsmaths/mtpro2
\usepackage{amsthm}         % (in part) For the defined environments
\usepackage{mathtools}      % Improves on amsmaths/mtpro2

% % % The bibliography % % %
\usepackage[backend=biber,
  style=authoryear-comp,
  bibstyle=authoryear,
  citestyle=authoryear-comp,
  uniquename=false,%allinit,
  % giveninits=true,
  backref=false,
  hyperref=true,
  url=false,
  isbn=false,
  useprefix=true,
  ]{biblatex}
\DeclareFieldFormat{postnote}{#1}
\DeclareFieldFormat{multipostnote}{#1}
% \setlength\bibitemsep{1.5\itemsep}
\newcommand{\noopsort}[1]{}
\addbibresource{Thesis.bib}

% % % % % % % % % % % % % % %

\usepackage[inline]{enumitem}
% \setlist[itemize]{noitemsep}
\setlist[description]{style=unboxed,leftmargin=\parindent,labelindent=\parindent,font=\normalfont\space}
% \setlist[enumerate]{noitemsep}

% % % Misc packages % % %
\usepackage{setspace}
% \usepackage{refcheck} % Can be used for checking references
% \usepackage{lineno}   % For line numbers
% \usepackage{hyphenat} % For \hyp{} hyphenation command, and general hyphenation stuff
\usepackage{subcaption}
% % % % % % % % % % % % %

% % % Red Math % % %
\usepackage[usenames, dvipsnames]{xcolor}
% \usepackage{everysel}
% \EverySelectfont{\color{black}}
% \everymath{\color{red}}
% \everydisplay{\color{black}}
\definecolor{fuchsia}{HTML}{FE4164}%Neon Fuchsia %{F535AA}%Neon Pink
% % % % % % % % % %

\usepackage{pifont}
\newcommand{\hand}{\ding{43}}
\usepackage{array}


\usepackage{multirow}
\usepackage{adjustbox}

\usepackage{titlesec}

\makeatletter
\newcommand{\clabel}[2]{%
   \protected@write \@auxout {}{\string \newlabel {#1}{{#2}{\thepage}{#2}{#1}{}} }%
   \hypertarget{#1}{#2}
}
\makeatother

\usepackage{multicol}

\setcounter{secnumdepth}{4}
\setcounter{tocdepth}{4}

\usepackage{tikz}
\usetikzlibrary{arrows,positioning}
\usepackage{tikz-qtree} %for simple tree syntax
% \usepgflibrary{arrows} %for arrow endings
% \usetikzlibrary{positioning,shapes.multipart} %for structured nodes
\usetikzlibrary{tikzmark}
\usetikzlibrary{patterns}


\usepackage{graphicx} % for images (png/jpeg etc.)
\usepackage{caption} % for \caption* command


\usepackage{tabularx}

\usepackage{bussalt}

\usepackage{Oblique} % Custom package for oblique commands
\usepackage{CustomTheorems}

\usepackage{svg}
\usepackage[off]{svg-extract}
\svgsetup{clean=true}



\usepackage{dashrule}

\newcommand{\hozline}[0]{%
  \noindent\hdashrule[0.5ex][c]{\textwidth}{.1pt}{}
  %\vspace{-10pt}
  % \noindent\rule{\textwidth}{.1pt}
}

\newcommand{\hozlinedash}[0]{%
  \noindent\hdashrule[0.5ex][c]{\textwidth}{.1pt}{2.5pt}
  %\vspace{-10pt}
}

\usepackage{contour}
 % \usepackage{pdfrender}

\usepackage[hidelinks,breaklinks]{hyperref}

\title{Fourth Year Talk Outline}
\author{Ben Sparkes}
% \date{ }


\begin{document}
\pagestyle{empty}

\noindent \textbf{I will be arguing}:
\begin{enumerate}[label=\roman*.]
\item Against the position that an agent is permitted to settle on a means only if the agent settling on those means as the result of their reasoning from some end to those means.

\item For the position that what permits the agent to settle on a means is the agent's recognition of the existence of a means-end relationship (from an end the agent has) which supports taking the relevant means.
\end{enumerate}

\noindent\textbf{Argument:}

\begin{enumerate}

\item\label{scenarios:exist} There are cases in which agents recognise a means (as means) without being able to reason from an end they have to those means.

\item\label{scenarios:persmissible} And, in these cases it is permissible for the agent to settle on the means.

\item[C\(_{\text{i}}\).]\label{scenario:no-reasoning} In order for an agent to be permitted in settling on a means it cannot, in general, by required that the agent settling on those means is the result of their reasoning from an end they have to those means.

  \begin{itemize}
  \item From \ref{scenarios:exist} and~\ref{scenarios:persmissible}.
  \end{itemize}

\item\label{settle:worthwhile} For an agent to be permitted to settle on an action, the action must be worthwhile for the agent.

  \begin{itemize}
  \item Principle: Settling on an action is (primarily) explained by the action being considered both possible and worthwhile by the agent, perhaps in comparison to the same attributes to other actions.
  \end{itemize}

\item\label{m-e:dependence} If an agent considers a means only as a means, there is no other way in which the means can be worthwhile other than as the means to an end.

  \begin{itemize}
  \item Principle: Whether a means (as a means) is worthwhile wholly depends on whether the end to the means is worthwhile.
    %\nolinebreak \mbox{ }\hfill(From \ref{m-e:dependence}, special case)
  \end{itemize}

\item\label{together} If an agent is permitted to settle on a means as a means, there existence of is some relevant means-end relation is required for the explanation of why this is permissible.
  \begin{itemize}
  \item For, by \ref{settle:worthwhile} the means is worthwhile.
  \item And by \ref{m-e:dependence} this is only because there is an end which is worthwhile.
  \end{itemize}
\item[C\(_{\text{ii}}\).] For the cases described in \ref{scenarios:exist}, the agent must take their settling on the means to be supported by a means-end relation they are unable to reason about.
  \begin{itemize}
  \item By \ref{scenarios:exist} and \ref{scenarios:persmissible} these are cases in which it is permissible for an agent settles on means without being able to reason from an end to those means.
  \item  And, from \ref{settle:worthwhile} to \ref{together} a means-end relation is required for the agent to settle on those means.
  \end{itemize}

\end{enumerate}

% Steps \ref{scenarios:exist} -- \ref{scenario:no-reasoning}

% Premises~\ref{scenarios:exist} and~\ref{scenarios:persmissible} are separated because

% Premise~\ref{scenario:no-reasoning} is the position that I am denying.


% The fourth premise states that means-end relations are necessary for settling on a means to be permissible.

% \newpage

% \noindent  \textbf{Suggestion}: See cases of practical reasoning in which an agent reasons from an end to a means as either
%   \begin{enumerate*}
%   \item constructing, or
%   \item checking
%   \end{enumerate*}
%   the relevant means-end relations.
% \linebreak

% \noindent Two questions:

% \begin{enumerate}[label=\alph*)]
% \item How does the conclusion relate to more complex cases, such as those involving shared activity, or gaslighting.
% \item Are there cases in which an agent is able to reason from ends to means, but settles what to do based on means-end relations that they are not able to reason about.
%   \begin{itemize}
%   \item Reasoning from ends to means, a further glass of wine settles what to do, but as the agent recognises they are tipsy, they take some water instead.
%   \end{itemize}
% \end{enumerate}

\end{document}
