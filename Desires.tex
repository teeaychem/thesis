\documentclass[10pt]{article}
\usepackage[margin=1in]{geometry}
% \newcommand\hmmax{0}
% \newcommand\bmmax{0}

% % % Fonts% %
\usepackage[T1]{fontenc}
   % \usepackage{textcomp}
   % \usepackage{newtxtext}
   % \renewcommand\rmdefault{Pym} %\usepackage{mathptmx} %\usepackage{times}
\usepackage[complete, subscriptcorrection, slantedGreek, mtpfrak, mtpbbi, mtpcal]{mtpro2}
   \usepackage{bm}% Access to bold math symbols
   % \usepackage[onlytext]{MinionPro}
   \usepackage[no-math]{fontspec}
   \defaultfontfeatures{Ligatures=TeX,Numbers={Proportional}}
   \newfontfeature{Microtype}{protrusion=default;expansion=default;}   \setmainfont[Ligatures=TeX]{Minion3}
   \setsansfont[Microtype,Scale=MatchLowercase,Ligatures=TeX,BoldFont={* Semibold}]{Myriad Pro}
   \setmonofont[Scale=0.8]{Atlas Typewriter}
   % \usepackage{selnolig}% For suppressing certain typographic ligatures automatically
   \usepackage{microtype}
% % % % % % %
\usepackage{amsthm}         % (in part) For the defined environments
\usepackage{mathtools}      % Improves  on amsmaths/mtpro2
\usepackage{amsthm}         % (in part) For the defined environments
\usepackage{mathtools}      % Improves on amsmaths/mtpro2

% % % The bibliography % % %
\usepackage[backend=biber,
  style=authoryear-comp,
  bibstyle=authoryear,
  citestyle=authoryear-comp,
  uniquename=allinit,
  % giveninits=true,
  backref=false,
  hyperref=true,
  url=false,
  isbn=false,
]{biblatex}
\DeclareFieldFormat{postnote}{#1}
\DeclareFieldFormat{multipostnote}{#1}
% \setlength\bibitemsep{1.5\itemsep}
\addbibresource{Thesis.bib}

% % % % % % % % % % % % % % %

\usepackage[inline]{enumitem}
\setlist[itemize]{noitemsep}
\setlist[description]{noitemsep,style=unboxed,leftmargin=.5cm,font=\normalfont\space}
\setlist[enumerate]{noitemsep}

% % % The following section relates to theorems, etc. % % %
\usepackage{thmtools}

\declaretheoremstyle[
spaceabove=6pt, spacebelow=6pt,
headfont=\normalfont\bfseries,
notefont=\mdseries, notebraces={(}{)},
bodyfont=\normalfont,
% postheadspace=1em,
% qed=\qedsymbol
]{defstyle}

\declaretheoremstyle[
spaceabove=6pt, spacebelow=6pt,
headfont=\normalfont\bfseries,
notefont=\normalfont\bfseries, notebraces={}{},
bodyfont=\normalfont,
% postheadspace=1em,
% qed=\qedsymbol
]{defsstyle}


\declaretheoremstyle[
spaceabove=6pt, spacebelow=6pt,
headfont=\normalfont\bfseries,
notefont=\normalfont\bfseries, notebraces={}{},
bodyfont=\normalfont\color{red},
% postheadspace=1em,
qed=\qedsymbol
]{notestyle}

\declaretheorem[name=Theorem,numberwithin=section]{theorem}
\declaretheorem[sibling=theorem,style=remark]{remark}
\declaretheorem[sibling=theorem,name=Corollary]{corollary}
\declaretheorem[sibling=theorem,name=Lemma]{lemma}
\declaretheorem[sibling=theorem,name=Fact]{fact}
\declaretheorem[sibling=theorem,name=Proposition]{proposition}
\declaretheorem[sibling=theorem,name=Definition,style=defstyle]{definition}
\declaretheorem[sibling=theorem,name=Assumption,style=defstyle]{assumption}
\declaretheorem[name=Definitions,numbered=no,style=defsstyle]{definitions}
\declaretheorem[sibling=theorem,name=Example,style=defstyle]{example}
\declaretheorem[name=Note,style=notestyle]{note}
\declaretheorem[name=Ramble,style=notestyle]{ramble}
\declaretheorem[name=Scenario,style=defstyle]{scenario}
% % % % % % % % % % % % % % % % % % % % % % % % % % % % % %

% % % Misc packages % % %
\usepackage{setspace}
% \usepackage{refcheck} % Can be used for checking references
% \usepackage{lineno}   % For line numbers
% \usepackage{hyphenat} % For \hyp{} hyphenation command, and general hyphenation stuff

% % % % % % % % % % % % %

% % % Red Math % % %
    \usepackage[usenames, dvipsnames]{xcolor}
    % \usepackage{everysel}
    % \EverySelectfont{\color{black}}
    % \everymath{\color{red}}
    % \everydisplay{\color{black}}
% % % % % % % % % %

\usepackage{pifont}
\newcommand{\hand}{\ding{43}}
\usepackage{array}
\usepackage{epigraph}

\usepackage{titlesec}
\usepackage[hidelinks,breaklinks]{hyperref}

\newcommand{\boxarrow}{%
  \mathrel{\mathop\Box}\mathrel{\mkern-2.5mu}\rightarrow
}
\newcommand{\diamondarrow}{%
  \mathrel{\mathop\Diamond}\mathrel{\mkern-2.8mu}\rightarrow
}


\titleclass{\subsubsubsection}{straight}[\subsection]

\newcounter{subsubsubsection}[subsubsection]
\renewcommand\thesubsubsubsection{\thesubsubsection.\arabic{subsubsubsection}}
\renewcommand\theparagraph{\thesubsubsubsection.\arabic{paragraph}} % optional; useful if paragraphs are to be numbered

\titleformat{\subsubsubsection}
  {\normalfont\normalsize\bfseries}{\thesubsubsubsection}{1em}{}
\titlespacing*{\subsubsubsection}
{0pt}{3.25ex plus 1ex minus .2ex}{1.5ex plus .2ex}

\makeatletter
\renewcommand\paragraph{\@startsection{paragraph}{5}{\z@}%
  {3.25ex \@plus1ex \@minus.2ex}%
  {-1em}%
  {\normalfont\normalsize\bfseries}}
\renewcommand\subparagraph{\@startsection{subparagraph}{6}{\parindent}%
  {3.25ex \@plus1ex \@minus .2ex}%
  {-1em}%
  {\normalfont\normalsize\bfseries}}
\def\toclevel@subsubsubsection{4}
\def\toclevel@paragraph{5}
\def\toclevel@paragraph{6}
\def\l@subsubsubsection{\@dottedtocline{4}{7em}{4em}}
\def\l@paragraph{\@dottedtocline{5}{10em}{5em}}
\def\l@subparagraph{\@dottedtocline{6}{14em}{6em}}
\makeatother

\newcommand{\sem}[1]{\ensuremath{[\kern-.5mm[{#1}]\kern-.5mm]}}

\setcounter{secnumdepth}{4}
\setcounter{tocdepth}{4}

% \titleclass{\todopar}{straight}[\section]
% \newcounter{todopar}
% \renewcommand{\thetodopar}{\Alph{todopar}.}
% \titleformat{\todopar}[runin]{\normalfont\normalsize\bfseries\color{WildStrawberry}}{\thesection.\thetodopar}{\wordsep}{}
% \titlespacing*{\todopar} {\parindent}{3.25ex plus 1ex minus .2ex}{1em}

\title{Desire}
\author{Benjamin Sparkes}
% \date{ }


\begin{document}

\maketitle
\epigraph{You're everything I never knew I always wanted.}{Fools Rush In}


Basic idea is that desire is to satisfaction as belief is to truth.


The recieved view has it that:

\begin{minipage}{0.45\linewidth}\mbox{ }
\begin{enumerate}[labelindent=\parindent,leftmargin=*,label=(RB\arabic*)]
\item\label{rb:1} Belief is a relation between a person and a proposition
\item\label{rb:2} \emph{S}’s belief that \emph{P} is true iff \emph{P} is true
\item\label{rb:3} \emph{S}’s belief that \emph{P} is false iff \emph{P} is not true
\end{enumerate}\mbox{ }
\end{minipage}\hfill
\begin{minipage}{0.45\linewidth}\mbox{ }
\begin{enumerate}[labelindent=\parindent,leftmargin=*,label=(RD\arabic*)]
\item\label{rd:1} Desire is a relation between a person and a proposition
\item\label{rd:2} \emph{S}’s desire that \emph{P} is satisfied iff \emph{P} is true
\item\label{rd:3} \emph{S}’s desire that \emph{P} is frustrated iff \emph{P} is not true
\end{enumerate}\mbox{ }
\end{minipage}

I think the received wisdom is correct in the~\cite{Lewis:1974aa}ian sense that it's at the core of our common sense theory of persons, and it's this common sense (or folk-theoretic) understanding of belief and desire that I'm interested in.
Still, even if correct \ref{rb:1}---\ref{rb:3} and \ref{rd:1}---\ref{rd:3} aren't particularly informative, and it's the way in which \ref{rd:1}---\ref{rd:3} are correct which I hope to give an account of here.

\begin{description}
\item[Basic idea] Practical reasoning (and agency) driven by satisfaction, rather than desires.
\end{description}

By analogy, satisfaction is to practical reasoning as truth is to theoretical reasoning, in that what I believe is what I take to be true and What is desire is what I take to be satisfactory.
This doesn't say anything substantive about the nature of truth, and I don't think this says anything substantive about the nature of satisfaction.
However, I will argue that this provides insight into the structure of practical reasoning.

Perhaps one could say that desire aims at satisfaction in the same way that belief aims at truth, but what I'm really interested in is desires, not desire \emph{per se}.
And, in addition, I don't quite understand what's going on with teleological/normative explanations.




Go through the menu example.
Note Frankfurt, Bratman, Aristotle, etc.
And then, we're talked about desire, but in that there is something impressed on the agent.
Reasoning is involved on in so far it indicates which desires one acts on.
There is no question as to whether acting on the impressed desire would satisfy.
But perhaps I reason on the basis of satisfaction.
I'm impressed by a variety of options, but this does not mean that these are the only options which would satisfy me.
There's also no sense in which I have an impression to try something new, but the nice thing about menus is that they put before me options that I would not have otherwise considered, so given that I expect the dish to satisfy me, I desire the dish.
I could have desired any other dish, and in this sense there's nothing special, except there must be something to explain why I ask the waiter for this one, this is still an extrinsic desire, but there it is.

Could now say the same thing about the healthy option.
Perhaps also something about explaining the significance of the waiters restriction (but this seems weak).


One may object with the idea that desires are cheap.
There's still some impression that these dishes have, and reasoning simply elevates these.
So, this denies that an agent can consider satisfaction in absence of impressions of desire.
Talk of satisfaction doesn't do anything.
So, there's a role for reasoning, but there's no concern with satisfaction in reasoning.

First argument against this is belief.
The same thing doesn't hold.
When I make a distinction between the appearance of something and my belief of what's going on, this seems rather sensible, and claiming that this doesn't hold for desire appears unmotivated.

Second, cognitive significance.
In a good case, desire X, find out X = Y, and can say a desire for Y.
This is tricky, though, as there are bad cases, and here I'm not sure what to say.

Third, you're everything I never knew I always wanted.
Find out about desires through satisfaction.
Nesquick, one could argue irrationality, or that the desire wasn't quite recognised.
This is hard, not quite a case of cog.\ sig.\ but close.

Context, really.
A simple mistrust of impression.
You can say that impressions come cheap, but this doesn't seem to capture the relevant issue.
It's that one genuinely reasons beyond the impressions of desire that they have.



\newpage




Desire menu examples.
This kind of effectively gets to the idea of context.
So, the claim is that what one desires is in part due to the context in which one is situated.

A different example with buying glasses.
Or, something like this.
The options are limited, and so I change the context.
But perhaps I don't reflect on the context, perhaps I have a stronger desire to look elsewhere.
Perhaps, and so the story is that although I desire a certain pair of glasses, it's much the same as the menu example, where I am able to turn the page, my desire is in part dependent on the belief about which options are available, and I have a stronger desire to explore alternative options.
Context has a role, but it's not an important role.

Well, the idea is to set this up, the glasses example that is, and to suggest that there's a case where I seem to have a desire, like a genuine desire, but I go and explore other options.
The argument then is that here it's not that I'm worrying about whether I actually desire the pair of glasses, instead it's a stronger desire to explore more options.
Desire comes in different strengths, and the thought that I may find an option which has even greater cognitive significance outweighs the desire for the particular pair.
The na\"{i}ve view, on the other hand, is that I don't associate my desire with the cognitive significance.
Perhaps, so let's then explore a different case.

The tournament case, and I end up frustrating my desire.
There's then a second tournament, and exactly the same thing happens.
However, I act differently.
Now, do I have a higher-order desire not to desire or simply a stronger desire to do nothing?
Or, do I recognise that acting on the cognitive significance of a desire might frustrate, and disassociate what I desire with what has the cognitive significance of a desire?

I think we should adhere to the received wisdom, and say that I do not desire, even though the cognitive significance remains the same.
What motivates is satisfaction, and that I can distinguish between what appears satisfactory and what would satisfy.

If the tournament case seems too bizarre, think about netflix and reading, or eating fast food and preparing a meal.
Is one really cultivating stronger desires, or is there a disconnect?

Temptation cases.
Here, the idea is that one realises.








Start with the simple idea that practical reasoning goes beyond desire and belief.
There are various ways to think about this \dots



Winning a game, as desiring different things in the same way.
I don't desire that \emph{Holmes} wins, just as you don't desire that \emph{Moriarty} wins, we both desire that \emph{I} win and the character of our desire does not by itself determine the conditions under which our desires are satisfied.



Satisfaction is to desire as truth is to belief.
The important point here is that one doesn't need to take a strong account on the nature of satisfaction to account for the role of desire in practical reasoning just as one doesn't need to take a strong account on the nature of truth.

What I mean by desire is fairly sparse, but it's most definitely the philosophical usage of desire, and more specifically the sort of folk-theoretic philosophical usage.
In short, the assumption I make about desire is that it's a propositional attitude which when paired with belief provides reasons for action.
This already allows for various positions to be sketched.
Humeanism, weak-Humeanism, and strong-Humeanism.
However, it seems to me that these should be recast as debates about the nature of satisfaction, and my purpose here is to show that a collection of interesting problems arise independently of what we take satisfaction to be.
In particular, the slogan that satisfaction is to desire as truth is to belief should be read to carry the implication that one does not have privelidged access to one's satisfaction.


There's an upshot to this in terms of second-order desires.





But this is not the full stroy.
An agent's belief has character, and it is the character of the agent's belief together with context which determines the content, or proposition of belief.

Simple examples arise from indexicals.
The character of `I am here now' stays fixed between different people and times (viz.\ contexts) but an utterance by a person at a time determines a different proposition.

Still, indexicals are a sympthom of character and are not constitutive of it.
What is captured by character and context is a distinction between \emph{what} and \emph{how}.

Following Perry distinguish between two types of cases.

Type A: The same thing is believed in different ways

Type B: Different things are believed in the same way.

There's a subtle shift here from language to mind.
It's one thing to argue that language depends on character and context, but it is another thing to argue that attitudes depend on character and context.
I'm interested in the latter, and while I think language provides insight into what goes on in the mind, I don't take any of the following to say anything about language \emph{per se}.

The above is to motivate context, but it feels as though I should be able to specify what I mean by context independently of this.



Context + character = content


\section{Belief}
\label{sec:belief}

So, something to the idea that belief has character.
This isn't too well put, but it makes some sense.
Then, put this together with context, and you get beliefs.
However, it's not as straightforward as this.
Beliefs are not simply a report on context.
Now I need an example of where I see something but I do not believe it.
I see a spider crawling along the floor, but I believe that it's a bundle of black thread moved by the draft.
I hear a person upstairs, but I believe the old house is creaking.
My keys are missing, but I believe they're where I left them.
I taste coffee, but I believe it's decaf.
I cannot find an error in my proof/fault in my argument, but I believe it is incorrect.
I see a familiar face on the train, but I do not believe it is my friend.
I am given a complement, but I believe that I have been insulted (or vice-versa).
The letter is signed by the defendant, but I believe it is their lawyers words.
The point here being that the character of my belief is such that it mediates context, and the beliefs which result are not reducible to context (for else I would believe things that I do not) nor are they reducible to belief (for without context these beliefs would not be formed).
It's a straightforward point, that our reasoning mediates what is reported to us.
I see problems of co-reference as particular instances of this broader class.
I believe that Lewis Carroll, but not Charles Dodgson, wrote Alice in Wonderland, etc.


{\color{red} Now, do problems of co-reference really occur as special cases of this class?
  It seems not, it seems as though these are cases of something different.
  Well, it's more that the relevant cases of co-reference are where one believes something under one name, and fails to believe, or believes something contrary, under a different name.
  But it's hard to see how this relates.
  The idea would be that, in some sense, names across contexts is a tricky thing, but I don't see this argument working out.
  Isn't it the problem that things appear to us in different contexts, and it takes some work to piece together information across contexts, I don't believe that A and B are the same thing because \dots I come across A in one context and I come across B in another, but I come across Charles Dodgson in many different contexts, and I believe these all refer to the same person.
  Well, often there's information which links context, and as it happens full names are quite good at this.
  Partial names are not, I don't believe that Sam refers to the same person every time I hear the name, which seems like a fair observation.
  So, in this sense these are opposite kinds of cases, in the above I have the same thing happening in multiple contexts, and so I don't form a belief, while in the name case I don't have the same thing in multiple contexts, and so I don't form a belief.
  Perhaps, then, it doesn't make sense to talk about the character of belief, as it seems as though what's happening here that one builds a somewhat context insensitive body of information and relies on that when forming beliefs.
  And, if this is the case then the same reasoning with desire goes though.
  However, in this sense it seems as though all one needs is individual beliefs.
  In some respect I should be fine with this.
}

So, there's a puzzle about exactly what I'm arguing for here.




A problem I have here is that I want to say something about correcting over other contexts, though perhaps this isn't quite the right way to go \dots or maybe I don't need to give a detailed argument for this at all, as it seems sufficiently intuitive that this is what's going on.

A different problem is that I wanted to draw on the idea that desire is a propositional attitude, and that this was a fairly straightforward thing.
I this respect, well, I really like the wax example, and it does seem as though I'm getting at something with that.
The idea then, would be to argue that satisfaction is to desire as truth is to belief, and the point I need to make is that the truth is in some sense independent of belief, even though we correct against

Well, there's something about belief which is not reducible to individual beliefs, and there's something about desire which isn't reducible to individual desires.
In a sense, this is what is captured when we talk about belief forming mechanisms, 

\section{A different start}
\label{sec:different-start}

Strength of desire.
Desire that you don't believe will/can be satisfied.
Instrumental desire that persists through different beliefs.

I think focusing on context is probably the best way to deal with this stuff.
So, start out and I say that look, there's something that persists through change.
The outward appearance of the wax does not truly reflect what the wax is.


\section{Enough with the wax}
\label{sec:enough-with-wax}





Perhaps it's better to start with the puzzles about desire.
There are these cases in which desire is mediated by belief.
But, it seems as though we can factor out those beliefs.
We can get to something which doesn't seem to depend on the beliefs that one has.
But this doesn't really work.
The food case is nice, as it seems as though things stop with the desire to have eaten.
However, I wonder about this, am I really hungry, does I really desire a meal, or do I lack some kind of nutrient, and so what I desire is not to eat something, but something with that nutrient.
Do I really desire to help you, or do I desire to help you in so far as I desire that you will help me, and helping you is a way to increase the chance that you will help me?
It's clear that we can't identify a desire with the proposition that's desired, but it also seems clear that we can't identify desire with what is taken to be desired.

The same holds for truth.
What I believe can't be identified with the proposition that's believed,



Oh, does the example work by showing that you believe something under a certain guise, and then as it happens you believe something further.
So, believing that A, but I don't just believe that A, I also believe that B.
It's somewhat misleading, because it doesn't track cognitive significance, but it makes sense, and it's what explains our response when we learn new information.

So, what I believe shouldn't be identified with the cognitive significance of what I believe, and what I desire shouldn't be identified with the cognitive significance of what I desire.

This is a point which can be made.
Difference between \emph{what} and \emph{how}.
It's the \emph{what} that's important.

Is this correct?
You're in a blind tournament, and you hear rumours about this person with a particular play style.
You desire that they loose their next game.
So, you desire that you loose your next game.
But this isn't right.

Well, how similar to is this with respect to cases where you don't believe that \(a = b\)?
I think the correct response here is to say that you do desire this.
You do desire something problematic.
Expand the case so that you anonymously engage in some fixing, and you end up loosing because of it.
You're desire is satisfied, but you're other desire is not.
You can have conflicting desire just as you can have conflicting beliefs.

Same with desiring to meet a particular person.
Your desire can be satisfied, even if you can't recognise how it's been satisfied.

And with the food example, you can resist associating what is cognitively significant to you with a desire, because you recognise what would satisfy your desire may be something else, you desire that.

This may be a little surprising, but there are similar cases with belief.

Now, in making this argument I've assumed nothing about what satisfaction amounts to.
All I've argued is that one can desire something, on the basis of satisfaction, which does not have the cognitive profile, so to speak, of a desire.

So, is this just a belief-desire pair?
In one sense, for sure.
In another sense, however, there's something problematic about thinking of things in this way, as it's definitely different from standard cases in which belief enters the picture.
Of course, belief is relevant in some respect, but I think then we're forced to say that there are never any cases of `simple' desires, because in order to represent these we must invoke beliefs.
To me, folk theory takes for granted a certain level of representation, and permits a certain level of reasoning, without requiring the invocation of belief.
I agree that there are puzzles here, but they are not puzzles at the folk-theoretic level.
Here, your reasoning about your desires can be separated from your beliefs.

Some borderline cases where beliefs are involved.
However, I don't think this is quite right.
It's not the belief that's doing the work in the relevant considerations, it's that there's something there.
Whatever you want to appeal to, it should be separated from the kind of belief that is at work when I put together my desires with my beliefs.

We're well aware that what appears to use as desirable may not be so, and satisfaction may be taken to require the desire of things which do not impress significance.

This may sound anti-Humean, but I don't think it is, at least not in spirit.
I'm not assuming anything about desire, and \emph{that} seems to be the key point.
I am, however, denying that the mind is `directed' by desire, it's rather the inverse, desire directs the mind, it's about satisfaction.
Or, perhaps better put, it seems right to say that one will not be satisfied because one believes they will be.


\section{Quotes}
\label{sec:quotes}

“[Desire] is produced in the margin which exists between the demand for the satisfaction of need and the demand for love” (Seminar V, 11.06.58., p.4).

Well, there's something here.
It'd be nice to include it.

``To paraphrase Lacan’s point, the joke shows that desire is like a ‘perverted’ form of need by the very process of expressing that need in the form of a demand. The young benefactor obviously needs food, but his hunger can obviously be satisfied by something less than the most expensive and elaborate meals. So we find desire in what ‘pushes beyond’ need in the very expression of a demand.''
http://www.lacanonline.com/index/2010/05/what-does-lacan-say-about-desire/

\section{Examples}
\label{sec:examples}

spotted-necked otter (Hydrictis maculicollis), or speckle-throated otter

https://plato.stanford.edu/entries/natural-kinds/
https://en.wikipedia.org/wiki/Conversion_of_units#Force

\end{document}