\documentclass[10pt]{article}
\usepackage[margin=1in]{geometry}
% \newcommand\hmmax{0}
% \newcommand\bmmax{0}

% % % Fonts% %
\usepackage[T1]{fontenc}
   % \usepackage{textcomp}
   % \usepackage{newtxtext}
   % \renewcommand\rmdefault{Pym} %\usepackage{mathptmx} %\usepackage{times}
\usepackage[complete, subscriptcorrection, slantedGreek, mtpfrak, mtpbbi, mtpcal]{mtpro2}
   \usepackage{bm}% Access to bold math symbols
   % \usepackage[onlytext]{MinionPro}
   \usepackage[no-math]{fontspec}
   \defaultfontfeatures{Ligatures=TeX,Numbers={Proportional}}
   \newfontfeature{Microtype}{protrusion=default;expansion=default;}   \setmainfont[Ligatures=TeX]{Minion3}
   \setsansfont[Microtype,Scale=MatchLowercase,Ligatures=TeX,BoldFont={* Semibold}]{Myriad Pro}
   \setmonofont[Scale=0.8]{Atlas Typewriter}
   % \usepackage{selnolig}% For suppressing certain typographic ligatures automatically
   \usepackage{microtype}
% % % % % % %
\usepackage{amsthm}         % (in part) For the defined environments
\usepackage{mathtools}      % Improves  on amsmaths/mtpro2
\usepackage{amsthm}         % (in part) For the defined environments
\usepackage{mathtools}      % Improves on amsmaths/mtpro2

% % % The bibliography % % %
\usepackage[backend=biber,
  style=authoryear-comp,
  bibstyle=authoryear,
  citestyle=authoryear-comp,
  uniquename=allinit,
  % giveninits=true,
  backref=false,
  hyperref=true,
  url=false,
  isbn=false,
]{biblatex}
\DeclareFieldFormat{postnote}{#1}
\DeclareFieldFormat{multipostnote}{#1}
% \setlength\bibitemsep{1.5\itemsep}
\addbibresource{Thesis.bib}

% % % % % % % % % % % % % % %

\usepackage[inline]{enumitem}
\setlist[itemize]{noitemsep}
\setlist[description]{style=unboxed,leftmargin=\parindent,labelindent=\parindent,font=\normalfont\space}
\setlist[enumerate]{noitemsep}

% % % The following section relates to theorems, etc. % % %
\usepackage{thmtools}

\declaretheoremstyle[
spaceabove=6pt, spacebelow=6pt,
headfont=\normalfont\bfseries,
notefont=\mdseries, notebraces={(}{)},
bodyfont=\normalfont,
% postheadspace=1em,
% qed=\qedsymbol
]{defstyle}

\declaretheoremstyle[
spaceabove=6pt, spacebelow=6pt,
headfont=\normalfont\bfseries,
notefont=\normalfont\bfseries, notebraces={}{},
bodyfont=\normalfont,
% postheadspace=1em,
% qed=\qedsymbol
]{defsstyle}


\declaretheoremstyle[
spaceabove=6pt, spacebelow=6pt,
headfont=\normalfont\bfseries,
notefont=\normalfont\bfseries, notebraces={}{},
bodyfont=\normalfont\color{red},
% postheadspace=1em,
qed=\qedsymbol
]{notestyle}

\declaretheorem[name=Theorem,numberwithin=section]{theorem}
\declaretheorem[sibling=theorem,style=remark]{remark}
\declaretheorem[sibling=theorem,name=Corollary]{corollary}
\declaretheorem[sibling=theorem,name=Lemma]{lemma}
\declaretheorem[sibling=theorem,name=Fact]{fact}
\declaretheorem[sibling=theorem,name=Proposition]{proposition}
\declaretheorem[sibling=theorem,name=Definition,style=defstyle]{definition}
\declaretheorem[sibling=theorem,name=Assumption,style=defstyle]{assumption}
\declaretheorem[name=Definitions,numbered=no,style=defsstyle]{definitions}
\declaretheorem[sibling=theorem,name=Example,style=defstyle]{example}
\declaretheorem[name=Note,style=notestyle]{note}
\declaretheorem[name=Ramble,style=notestyle]{ramble}
\declaretheorem[name=Scenario,style=defstyle]{scenario}
% % % % % % % % % % % % % % % % % % % % % % % % % % % % % %

% % % Misc packages % % %
\usepackage{setspace}
% \usepackage{refcheck} % Can be used for checking references
% \usepackage{lineno}   % For line numbers
% \usepackage{hyphenat} % For \hyp{} hyphenation command, and general hyphenation stuff

% % % % % % % % % % % % %

% % % Red Math % % %
    \usepackage[usenames, dvipsnames]{xcolor}
    % \usepackage{everysel}
    % \EverySelectfont{\color{black}}
    % \everymath{\color{red}}
    % \everydisplay{\color{black}}
% % % % % % % % % %

\usepackage{pifont}
\newcommand{\hand}{\ding{43}}
\usepackage{array}
\usepackage{epigraph}


\usepackage{multirow}
\usepackage{adjustbox}
\usepackage{verse}


\usepackage{titlesec}
\usepackage[hidelinks,breaklinks]{hyperref}


\makeatletter
\newcommand{\clabel}[2]{%
   \protected@write \@auxout {}{\string \newlabel {#1}{{#2}{\thepage}{#2}{#1}{}} }%
   \hypertarget{#1}{#2}
}
\makeatother



\newcommand{\boxarrow}{%
  \mathrel{\mathop\Box}\mathrel{\mkern-2.5mu}\rightarrow
}
\newcommand{\diamondarrow}{%
  \mathrel{\mathop\Diamond}\mathrel{\mkern-2.8mu}\rightarrow
}


\titleclass{\subsubsubsection}{straight}[\subsection]

\newcounter{subsubsubsection}[subsubsection]
\renewcommand\thesubsubsubsection{\thesubsubsection.\arabic{subsubsubsection}}
\renewcommand\theparagraph{\thesubsubsubsection.\arabic{paragraph}} % optional; useful if paragraphs are to be numbered

\titleformat{\subsubsubsection}
  {\normalfont\normalsize\bfseries}{\thesubsubsubsection}{1em}{}
\titlespacing*{\subsubsubsection}
{0pt}{3.25ex plus 1ex minus .2ex}{1.5ex plus .2ex}

\makeatletter
\renewcommand\paragraph{\@startsection{paragraph}{5}{\z@}%
  {3.25ex \@plus1ex \@minus.2ex}%
  {-1em}%
  {\normalfont\normalsize\bfseries}}
\renewcommand\subparagraph{\@startsection{subparagraph}{6}{\parindent}%
  {3.25ex \@plus1ex \@minus .2ex}%
  {-1em}%
  {\normalfont\normalsize\bfseries}}
\def\toclevel@subsubsubsection{4}
\def\toclevel@paragraph{5}
\def\toclevel@paragraph{6}
\def\l@subsubsubsection{\@dottedtocline{4}{7em}{4em}}
\def\l@paragraph{\@dottedtocline{5}{10em}{5em}}
\def\l@subparagraph{\@dottedtocline{6}{14em}{6em}}
\makeatother

\newcommand{\sem}[1]{\ensuremath{[\kern-.5mm[{#1}]\kern-.5mm]}}

\setcounter{secnumdepth}{4}
\setcounter{tocdepth}{4}

% \titleclass{\todopar}{straight}[\section]
% \newcounter{todopar}
% \renewcommand{\thetodopar}{\Alph{todopar}.}
% \titleformat{\todopar}[runin]{\normalfont\normalsize\bfseries\color{WildStrawberry}}{\thesection.\thetodopar}{\wordsep}{}
% \titlespacing*{\todopar} {\parindent}{3.25ex plus 1ex minus .2ex}{1em}

\title{Desire}
\author{Benji Sparkes}
% \date{ }


\begin{document}

\maketitle
\epigraph{You're everything I never knew I always wanted.}{Fools Rush In}



\section{Basic idea}
\label{sec:basic-idea}


\begin{center}
  Desire is to satisfaction as belief is to truth.

  \emph{Or to put it somewhat deeply:}

  Satisfaction is the aim of desire, as truth is the aim of belief.
\end{center}

The principal thought here is that `desire' does not by itself determine the conditions under which we are satisfied, and we engage in practical reasoning in order to figure out what to desire.\nolinebreak
\footnote{Desire occurs in quotes as I take desire to have a certain (folk-)theoretical role, distinct from standard use.
  The details of this are (hopefully) made somewhat clear below.}
This is contrasted with a view (which seems to me standard) in which desires are primitive, and practical reasoning adjudicates between these.

If this is right, then figuring out what to desire is an interesting problem and the structure of practical reasoning may have a different shape to that of adjudicating between desires.
However, as I'll attempt to show, there is a strong sense in which the standard view of desire is able to account for everything I wish to claim as an advantage for this view.

So, \nolinebreak
\begin{enumerate*}[label=\alph*)]
\item practical reasoning is driven by satisfaction and your desires reflect what you expect to be satisfied by.
\item satisfaction is not all-or-nothing; one can be satisfied to varying degrees,
\item I have no idea what satisfaction is, and in some respects this is the point,
\item satisfaction may arise from distinct (and potentially conflicting) sources, but
\item satisfaction is what motivates us (as agents, perhaps not as creatures)
\end{enumerate*}


\subsection{Some background}
\label{sec:some-background}

Received wisdom has it that:\nolinebreak
\footnote{The received wisdom regarding desire is lifted from \textcite{McDaniel:2008aa}.
  \citeauthor{McDaniel:2008aa} argues that there's a problem with the received wisdom as it does not account for \emph{conditional} desires.
Some details in what I say may hang on whether desires are conditional, but I don't think anything of much substance does.}

\begin{minipage}{0.45\linewidth}\mbox{ }
  \begin{enumerate}[labelindent=\parindent,leftmargin=*,label=(RB\arabic*)]
  \item\label{rd:1} Desire is a relation between a person and a proposition
  \item\label{rd:2} \emph{S}’s desire that \emph{P} is satisfied iff \emph{P} is true
  \item\label{rd:3} \emph{S}’s desire that \emph{P} is frustrated iff \emph{P} is not true
  \end{enumerate}\mbox{ }
\end{minipage}\hfill
\begin{minipage}{0.45\linewidth}\mbox{ }
  \begin{enumerate}[labelindent=\parindent,leftmargin=*,label=(RD\arabic*)]
  \item\label{rb:1} Belief is a relation between a person and a proposition
  \item\label{rb:2} \emph{S}’s belief that \emph{P} is true iff \emph{P} is true
  \item\label{rb:3} \emph{S}’s belief that \emph{P} is false iff \emph{P} is not true
  \end{enumerate}\mbox{ }
\end{minipage}

I think the received wisdom is correct a~\citeauthor{Lewis:1974aa}ian sense.
It's at the core of our common sense theory of persons, and it's this common sense/folk-theoretic understanding of belief and desire that I'm interested in.
Still, even if correct \ref{rb:1}---\ref{rb:3} and \ref{rd:1}---\ref{rd:3} aren't particularly informative, and it's the way in which \ref{rd:1}---\ref{rd:3} are correct which I hope to give an account of here, in line with the basic idea.\nolinebreak
\footnote{As an aside, one may argue that desires are for (conceivable) states of affairs rather than propositions, but there's nothing too important about this distinction, as propositions describe states of affairs, and even if this kind of correspondence turns out to be problematic, I don't think what I have to say hangs on this.}

Following \textcite{Schapiro:2014aa} (who in turn follows \textcite{Schueler:1995aa}) I'm treating desire as a ``placeholder'' or dummy concept.
The understanding of desire I care about is the sense in which whenever you act we can say that you had a `desire' to do what you did.
What I care about is attributing `motivation to the agent, as the conceptual correlate of action' in order to explain `how the agent, rather than something external to the agent' is the source of action.
\citeauthor{Schapiro:2014aa} (or \citeauthor{Schueler:1995aa}) distinguish this from a alternative, `substantive', understanding of desire in which `it is logically possible to do something without having a desire to do so', something which `does pick out something that plays a distinct, determinate role in the explanation of action' as given the logical possibility of doing something without a desire to do it, it cannot simply be whatever explains action. (\citeyear[136--137]{Schapiro:2014aa})
I'm not too interested in what this `substantive' sense of desire is, what I care about is whatever it is which leads to action.


\subsubsection{Abstract and nominal desire}
\label{sec:abstr-nomin-desire}

When explaining the basic idea I distinguished two understandings of desire.
Here, I'll try to make this distinction a little clearer.

So, let's start with a pair of labels.
I term the view I favour \emph{abstract} and the `standard' view \emph{nominal}.

The idea here is that the nominal---standard---view denies the existence of something which goes beyond individual desires, but what I mean by this is rather weak.
It does not deny that distinct desires may be desires for the same `thing', what it denies is that reasoning `grasps' that thing independently from particular desires and asserts that practical reasoning is concerned with particular desires.
By contrast, the abstract view takes there to be something which is grasped independently of particular desires and has a role in practical reasoning.

\begin{description}
\item[\textbf{Nominal}] There are particular desires and these are basic inputs to practical reasoning which adjudicates between these.
\item[\textbf{Abstract}] There is satisfaction, practical reasoning generates desires based on which propositions are taken to be satisfactory and resolves conflicts in satisfaction (or at least attempts to).
\end{description}

Following \textcite[133]{Schapiro:2014aa} I am not assuming that on the nominal view desire requires one to be `pushed or pulled by a psychological force', to have a `blind urge' toward some proposition, or involve `brute force' in any substantial sense.
This may be so, but the nominal view is also compatible with the idea that desires simply provide considerations which count in favour (or make appropriate) certain actions.
What matters is that reasoning (in so far as it is concerned with desires) takes particular desires as givens.

In line with this, I am also not assuming that the nominal view requires that in acting one must respond to desires.
Desire may operate `in the background' to shape our outlook on the world so that things appear to us in practically salient terms.
(\cite[133]{Pettit:1990aa,Schapiro:2014aa})
What matters for my understanding of the nominal view is that particular propositions appear desirable, practically salient, or whatever, and that practical reasoning cannot question this.
Practical reasoning can only weigh various considerations with respect to what desirable or practically salient propositions to act on.

The abstract view, in contrast, rejects the idea that there are fundamental particulars which serve as the foundation for practical reasoning as view satisfaction as a primitive unknown which motivates practical reasoning to form particular desires.
By analogy, satisfaction is to practical reasoning as truth is to theoretical reasoning, in that what I believe is what I take to be true and What is desire is what I take to be satisfactory.
This doesn't say anything substantive about the nature of truth, and I don't think this says anything substantive about the nature of satisfaction.
In this respect, `satisfaction is the aim of desire' is really a slogan, and what I mean by this is that there's a problem in understanding desire without considering the role of satisfaction, and in particular the distinction between a desire being satisfied and being satisfied by a desire being satisfied.
This is a distinction that the nominal view can't get.
Perhaps a better way of putting things is that there's truth, and our beliefs conform to this, and there is satisfaction, and our desire conforms to this.
There's no direct access to either, and what's puzzling about desire is that it seems as though we should have this kind of access.
For sure, given abstract desires one adjudicates between ways of satisfaction, but this is difference is that reasoning is driven by satisfaction, not by nominal desires.

\subsubsection{Impressed desires}
\label{sec:impressions}

The abstract view, as I understand it, does not need to deny that there are `particular desires' in a sense of the term in which certain propositions appear desirable, practically salient, or whatever.
For ease of talking I'll refer to these as `impressed desires'.
Further, such impressed desires may lead to action, etc.
What matters is that practical reasoning can appeal to something beyond such impressed desires.
In this respect, the abstract view treats such impressed desires as sources of information, but does not take these to be necessarily connected with what would provide satisfaction.


\subsection{Menu example}
\label{sec:menu-example}

There are a number of different aspects to the role of desire in practical reasoning, here's a scenario to help tease some of these apart:

\begin{scenario}
  I'm at a restaurant and before be lies a menu. I open it up and start to look though the options.

  I have a desire to lay down---it's been a hard day---but here this desire is \emph{standing}, it does not have any (perhaps significant or relevant) role in my reasoning about what I am doing here, in the restaurant. My desire to order from the menu, by contrast, is \emph{occurrent}. If I did not have the desire, I would not continue to look through the options. Still, ordering from the menu is not the only occurrent desire I have.
  As I read through certain dishes I have occurrent desires regarding these, but I do not have an \emph{intrinsic} desire to order from the menu, nor do I have an intrinsic desire for any particular dish---these desires are not held in part for their own sake, they are means to an end, and as such are \emph{instrumental}. Perhaps my desire to eat in intrinsic, but this does not mean that my current reasoning is driven exclusively by this desire. I may also be a foodie(?), and desire eating for its own sake.

  Further, as I read through the menu my desire for particular dishes vary in strength. I am more disposed to order certain dishes than others. This doesn't mean I necessarily have a strong desire for any particular dish, but I can compare the desires I have for different dishes.

  I may also have desires regarding my own choice. Perhaps I desire that I choose a healthy option, or that I relax and indulge in whatever strikes me as most appealing. This may be a simple higher order desire, or it might a~\citeauthor{Frankfurt:1971aa}ian \emph{volition}, where I desire that my desire to take a vegetarian option is a desire which succeeds in action. Or, perhaps I have a \citeauthor{Bratman:2003aa}ian higher-order policy which favours treating certain desires as motivationally effective in my practical reasoning.\nolinebreak
  \footnote{What I have to say does not depend on desire being the only consideration in practical reasoning.}

  So I look through the menu, and as I'm attempting to come to a decision between the Portobello Mushroom Stroganoff and the Artichoke Spinach Lasagne the waiter comes over and informs me that the restaurant is out of portobello mushrooms and \dots lasagne. Disappointed, I update my beliefs I recognise that I should not desire this dishes as a realizer of my intrinsic desire to live, or(?) to `eat', but the waiter helpfully points out that I've only looked at half the menu, and after turning over a few pages I'm again deciding between two options: the Vegetarian Bibimbap and the Raw Pad Thai.
  I ask the waiter for a recommendation, and use this information and form a stronger desire for the Bibimbap.
\end{scenario}

In the above scenario everything is described in terms of nominal desire, and the agent's practical reasoning adjudicates between these.
This is fine as far as things go, the nominal view is attractive, but I do not think that it's success in accounting for scenarios such as these means that it is unassailable.

Even though all that's in play in the scenario are my particular desires and beliefs and my reasoning is involved on in so far it indicates which desires one acts on, it seems the scenario can be enriched.
I have not considered whether what I desire would actually satisfy me.
And, this seems to me a fairly straightforward and simple thought.
I can wonder about whether what impresses me as desirable really is desirable.
\begin{enumerate*}[label=\arabic*.]
\item Am I really hungry?
\item Do I really desire a meal?
\item Or, do I lack some kind of nutrient?
\item Perhaps what I desire is not to eat something, but something with a particular nutrient that I'm lacking which is why I am impressed with a desire to eat?
\item Perhaps the additional protein form Black Beans and Rice would actually satisfy me?
\end{enumerate*}

Further, I'm impressed by a variety of options, but this does not mean that these are the only options which would satisfy me.
The wonderful thing about menus is that they put before me options that I would not have otherwise considered, so perhaps, given background beliefs about the quality of the chef, etc.\ I can expect any dish to satisfy me, as such \emph{form} a desire based on this line of reasoning.
In this sense, I could have desired any other dish, and so there's nothing special about the particular desire I have, except for the fact that there must be something to explain why I ask the waiter for a particular dish rather than any other.

\subsection{Note}
\label{sec:note}

In the following I mostly have in mind (I think) intrinsic desires, desires held in part for their own sake.
I'm not too careful about this in what I have written, and there may be some conceptual confusion also \dots

\newpage

\section{Objections}
\label{sec:objections}

The general picture is that there's more to practical reasoning than adjudicating between desires (in so far as practical reasoning is concerned with desire) and specifically practical reasoning can be concerned with satisfaction which is something that goes beyond impressed desires.

One may object to this in a number of ways.
Perhaps it's not exactly clear what the proposal is, perhaps satisfaction is too mysterious, or what I'm saying doesn't make any sense at all.
I'm happy to grant any of these, but they show that I need to do more work.
What worries me are objections that the kind of idea I'm pursuing is misguided from the outset.
Here, there seem to be two prominent objections.
The first two I think are misguided, but the second I take to be of genuine interest.

\subsection{Reasoning and rationality}
\label{sec:reas-rati}

One may worry that satisfaction sneaks in rationality, and as such that the received view is in conflict with \citeauthor{Hume:2011aa}ianism.
As \textcite{Smith:2004aa} shows, the exact status of \citeauthor{Hume:2011aa}ianism is somewhat unclear, and there's a sense in which \citeauthor{Hume:2011aa}ians can make space for practical rationality, but I don't think the abstract view is committed to rationality in any strong sense of the term.
What would satisfy you may be, in~\cite{Johnston:1989aa}'s term `brutally insane, psychopathically callous and demonically indifferent' (\citeyear[161]{Johnston:1989aa}).

As I suggested above, the abstract view doesn't make any commitment to what satisfaction is, nor as to the nature of reasoning premised on satisfaction, and it is hard to see why anything (substantial) about rationality should necessarily follow from this view.
Of course, the view is compatible with substantive rational requirements, but \citeauthor{Smith:2004aa}'s shows that this holds of moderate \citeauthor{Hume:2011aa}ianism too.

\subsection{The metaphysical objection}
\label{sec:metaphysics}

The first objection to this proposal is that satisfaction is simply an abstraction of nominal desires, and even if there's a distinction in our everyday/folk-theoretic understanding of (practical) reasoning, what's \emph{actually} going on is just adjudication between nominal desires, and I'm here confusing a conceptual understanding of reasoning with the metaphysics of reasoning (or however you want to mark a distinction between what we take to be going on and what is actually going on).

But this objection is misplaced, what I care about is a folk-theoretic understanding of (practical) reasoning, and I don't care if this understanding misrepresents what's actually going on in practical reasoning, so long as this misrepresentation is widespread amongst our common understanding of reasoning, and that this misunderstanding informs the way we reason about ourselves and others.
That is, if there even is a misunderstanding.


\subsection{The Humble Argument}
\label{sec:humble-argument}

The humble objection starts with the same idea as the metaphysical objection, and suggests that any talk of satisfaction is simply an abstraction of nominal desires, but stays at the conceptual or folk-theoretic level, and argues that whatever you come to desire through reflections on satisfaction can be regarded as a nominal desire.
In other words, reasoning about satisfaction is simply a way to recognise which nominal desires you hold.

The argument is humble in the sense that whatever the abstract view says, it is committed to desire being a relation between a person and a proposition, and holds that to appeal to anything more is in some sense extravagant.

There are various ways to make this work, but what I have in mind is some modification of \citeauthor{Schroeder:2007aa}'s (\citeyear{Schroeder:2007aa}) idea that reasons are cheap.
Desires, likewise, are cheap,\nolinebreak
\footnote{This is, of course, a consequence of \citeauthor{Schroeder:2007aa}'s view, as I understand it, but as \citeauthor{Schroeder:2007aa} frames their discussion in terms of reasons I'm going to ignore the details here.}
and so if it is possible for one to entertain a proposition, then it is possible for one to have a desire for that proposition.

This objection has force as the abstract view takes practical reasoning to be propositional, in that when reasoning about satisfaction an agent considers various propositions and forms a judgement about their potential to satisfy the agent.
So, goes the objection, whenever there is reasoning about satisfaction this is straightforwardly reasoning about the strength (or whatever) of the nominal desire for the proposition under consideration.

It is perhaps worth pointing out that one doesn't need to make nominal desires into chips\nolinebreak
\footnote{`Cheap as chips', apparently coined by The Duke (David Dickinson).}\nolinebreak
, nominal desires need only be `cheap enough' for practical reasoning to have something to adjudicate with---they don't need to be bobby dazzlers---and this seems, well, a defensible position.
If a decide to order a meal based on the thought that I'm lacking some nutritional content, then perhaps I really do have a nominal desire for that nutritional content.

\subsubsection{Potential lines of response}
\label{sec:potent-lines-resp}

I find the humble objection particularly worrying.
I take it that the nominal view is an entrenched position, and the argument suggests that, at least in principle, a defender of the nominal view will be able to give an explanation to whatever kind of scenario or problem a defender of the abstract view generates in order to distinguish the two views with regards to the nominal desirability or satisfaction of a particular proposition.

Further, it seems as though we're unlikely to learn anything new about practical reasoning by adopting the abstract view.
It may allow us to posit a different structure, but whatever this structure allows, the nominal view will be able to capture in terms of adjudication between nominal desires.

It's tempting, then, to seek analogies elsewhere, or in other words to align the abstract view with other aspects of reasoning and so to attempt to motivate the view via parity.

For example, I can make a distinction between the appearance of something and my belief of what's going on, which seems rather sensible, and claiming that this doesn't hold for desire appears unmotivated.
Evidence is cheap in the relevant sense, but evidence is distinguished from belief, and sometimes we go against `immediate' evidence which would seem to correspond to nominal desires in these cases.
Perhaps this reply goes too fast, and we shouldn't distinguish evidence from belief, but the point is that reasoning is an intermediate step, and the issue is whether something is true, so I think there is a disanalogy here, but perhaps more needs to be done to mark the distinction properly.

Perhaps a parity based argument can work, but a different line of thought considers the role of propositions and context, and I'd like to attempt to explore this in some detail.


\subsection{Propositions and Context}
\label{sec:prop-and-cntxt}

\begin{verse}[12em]
  There's an old trick played \\
  When the light and the wine conspire \\
  To make me think I'm fine \\
  \nopagebreak{\vspace{2ex}\raggedleft (\cite{Newson:2015aa})\par}
\end{verse}

I'm include to see the humble argument as showing that there's no way to make a clean argument to the conclusion that the abstract and nominal views will fail to account for whatever phenomena they are required to explain.
I only explicitly gave an argument that the nominal view could mimic the abstract view, but I think running the same kind of reasoning in reverse is plausible, and as such I don't think that relying on intuitions about particular desires is going to be of much help.

However, the abstract and nominal view certainly do differ in the way in which the account for desires, and the relationship between desires and propositions.
Recall, the nominal view states that there are only nominal desires, and so a desire must be identified (in some sense) with the proposition desired, the abstract view doesn't share this commitment.

I don't have the exact details of an argument worked out here, but clearly if two propositions are distinct and desired, then there must be two distinct desires.
The converse, however, is a little more complex.
I'm assuming that cognitive significance is built into the proposition desired, and in this respect while desiring that \emph{Lewis Carrol hadn't taken any pictures} and desiring that \emph{Charles Dodgson hadn't taken any pictures} are desires for the same proposition to be true, these are here treated as different propositions.\nolinebreak
\footnote{If it helps, think of \citeauthor{Kaplan:1989ab}'s (\citeyear{Kaplan:1989ab}) distinction between the \emph{character} and \emph{content} of an utterance.
As an aside, the concerns raised here largely arise from thinking about how this kind of distinction matters to desire.}

Still, even granting cognitive significance is part of what is desired, if there are two distinct desires, it does not straightforwardly follow that two distinct propositions are desired, as the nominal view may distinguish between \emph{senses} of desire.
For example, perhaps I desire \emph{that I eat} because
\begin{enumerate*}[label=\alph*)]
\item I desire to survive, and
\item I am a foodie.
\end{enumerate*}
However, here we have a case of an instrumental desire, which is supported by two desires for distinct propositions.
And, the case of non-instrumental desires is less clear.
It seems to me as though in non-instrumental cases there's not much of an argument to be made for distinguishing senses of desire, but I'm somewhat biased.
Regardless, I think there's potentially some interesting observations to make here with respect to \emph{what} is desired and \emph{how} it is desired.

\paragraph{ }
What missing here is an explicit account of how all of this matters from a folk-psychological point of view, and in particular how all of this ties into the explanation of action, but I'm hoping that you feel the pull \dots

% In an attempt to help ground the intuition here, consider the following accounts of desire lifted from \citeauthor{Schroeder:2017aa}'s SEP article on desire:

% \begin{itemize}
% \item For an organism to desire \emph{p} is for the organism to be disposed to act so as to bring about \emph{p}.
% \item For an organism to desire \emph{p} is for the organism to be disposed to take whatever actions it believes are likely to bring about \emph{p}.
% \item For an organism to desire \emph{p} is for the organism to be disposed to take pleasure in it seeming that \emph{p} and displeasure in it seeming that not-\emph{p}.
% \item For an organism to desire \emph{p} is for it to believe \emph{p} is good.
% \item For an organism to desire \emph{p} is for \emph{p} to appear good to the organism.
% \item For an organism to desire \emph{p} is for the thought of \emph{p} to keep occurring to the organism in a favorable light, so that its attention is directed insistently toward considerations that present themselves as counting in favor of \emph{p}.
% \item For an organism to desire \emph{p} is for it to be disposed to keep having its attention drawn to reasons to have \emph{p}, or to reasons to avoid not-\emph{p}.
% \item For an organism to desire \emph{p} is for it to use representations of \emph{p} to drive reward-based learning.
% \end{itemize}

% The point here is that you can take any of these accounts

\subsubsection{Proposition}
\label{sec:prop}

Here's a simple table, varying desires and propositions to contrast the abstract and nominal views.

\begin{center}
  {\setlength{\tabcolsep}{0.5em}%
      \renewcommand{\arraystretch}{1.5}%
    \begin{tabular}[h]{cc|cc}
      \multicolumn{2}{c}{Abstract} & \multicolumn{2}{c}{Nominal} \\
      \hline
      Different proposition & Same feature of satisfaction & \multicolumn{2}{c}{?} \\
      Different proposition & Different feature of satisfaction & Different proposition & Different desire \\
      Same proposition & Different feature of satisfaction & \multicolumn{2}{c}{?`} \\  % Same proposition & Different desire \\
      Same proposition & Same feature of satisfaction & Same proposition & Same desire \\
    \end{tabular}
}
\end{center}


\subsubsection{Context}
\label{sec:context}

And here's a more complex table, varying desires, context, and propositions to contrast the abstract and nominal views.

% Something is missing here too.
% There's some interesting cases.
% So, Sen's scenario is a good one.
% But, this is case where there's a switch, so in this sense it's similar to the different restaurant cases.
% But is there as case where we have a different context, and the same thing is desired in two different ways?

% So, in the context case, there's different context, same feature, different proposition.
% Hum, perhaps Sen's case is more complex, and perhaps this isn't the right table.
% As it's now to do with the relevant propositions, at least in the Sen case.
% So what I really want to do is expand the table.




\newcolumntype{R}{@{\extracolsep{10pt}}|@{\extracolsep{10pt}}}%


\begin{figure}[h]
  \centering
  \begin{adjustbox}{center}
    {\setlength{\tabcolsep}{0.5em}%
      \renewcommand{\arraystretch}{1.5}%
      \begin{tabular}[h]{cccr|c|lccc}
        \multicolumn{3}{c}{Abstract} & \multicolumn{3}{c}{ } & \multicolumn{3}{c}{Nominal} \\
        \hline\hline
        \multirow{4}{*}{Different proposition} & \multirow{2}{*}{Different context} & Different feature of satisfaction  &  & \mbox{ \clabel{case:a}{\emph{A}}} & & Different proposition & Different context & Different desire \\
        \cline{3-3}
                                     &  & Same feature of satisfaction  & & \mbox{ \clabel{case:b}{\emph{B}}} & & & ? & \\
        \cline{2-2}
                                     & \multirow{2}{*}{Same context} & Different feature of satisfaction  & & \mbox{ \clabel{case:c}{\emph{C}}} & &  Different proposition & Same context & Different desire \\
        \cline{3-3}
                                     &  & Same feature of satisfaction  & & \mbox{ \clabel{case:d}{\emph{D}}} & &  & ?` & \\
        \cline{1-1}
        \multirow{4}{*}{Same proposition} & \multirow{2}{*}{Different context} & Different feature of satisfaction  & & \mbox{ \clabel{case:e}{\emph{E}}} & & & ? & \\ % Same proposition & Different context & Different desire \\
        \cline{3-3}
                                     &  & Same feature of satisfaction & & \mbox{ \clabel{case:f}{\emph{F}}} & &  Same proposition & Different context & Same desire \\
        \cline{2-2}
                                     & \multirow{2}{*}{Same context} & Different feature of satisfaction  & & \mbox{ \clabel{case:g}{\emph{G}}} & &  & ?` & \\
        \cline{3-3}
                                     & & Same feature of satisfaction  & & \mbox{ \clabel{case:h}{\emph{H}}} & &  Same proposition & Same context & Same desire \\
        \hline
      \end{tabular}
    }
  \end{adjustbox}
  \caption{Combinations of proposition and context}
  \label{fig:comb-prop-ctxt}
\end{figure}



\begin{enumerate}
\item It seems to me that the difference between type \ref{case:a} and \ref{case:c} cases and between type \ref{case:f} and \ref{case:h} cases on the nominal view is only surface level, as change in context cannot mark a change in desire.
\item Type \ref{case:b} cases are particularly interesting, and are certainly ruled out by the nominal view.
  Think about your desire to go to a certain supermarket in order to buy a meal.
  Now, consider being somewhere else.
  Intuitively it seems as though it's the same `desire', and what has changed is how to realise this.
  However, on the nominal view there must be two distinct desires for the separate supermarkets, and the change in context indicates a change in salience for the respective desires.
\item Type \ref{case:d} cases, with Different proposition, Same context, Same feature are the standard Buridan type cases (as I understand things).
\item I don't have a good example for type \ref{case:e} cases, these are Buridan cases in which in different context the same proposition is desired on the basis of distinct aspects of satisfaction.
\item Whether type \ref{case:d} and \ref{case:g} cases can happen on the nominal view is something of an open question.
\end{enumerate}




% \subsection{Differences}
% \label{sec:differences}



% Second, cognitive significance and context.
% In a good case, desire X, find out X = Y, and can say a desire for Y.
% This is tricky, though, as there are bad cases, and here I'm not sure what to say.
% The idea here is to say that one had the desire, but that there was no impression or anything of the sort, but the problem is that this might not seem to be the right sense of desire.
% Okay, the problem is that when one tries to say that one can incorporate this worry into their reasoning, then it seems as though one allows for the possibility of an impression which can be latched onto by reasoning.


% Context, really.
% A simple mistrust of impression.
% You can say that impressions come cheap, but this doesn't seem to capture the relevant issue.
% It's that one genuinely reasons beyond the impressions of desire that they have.

\newpage
\section{Upshots}
\label{sec:upshots}

\subsection{Making intuitive sense of utterances in romantic comedies}
\label{sec:making-intu-sense}

\begin{center}
  \emph{You're everything I never knew I always wanted.}
\end{center}


You find out about desires through reasoning about satisfaction.
The abstract view has a straightforward way to account for what's going on here, while the alternative view needs to argue that there `were' nominal desires.
`Were', as the nominal view could argue that what's going on here is a kind of disguised counterfactual.
Had it been the case that you appeared to me and I had reasoned about my nominal desires as I have done so now, then I would have had a desire for you, just as I now have such a desire.


\subsection{Buridan cases}
\label{sec:buridan-cases-1}

As \citeauthor{Rescher:1960aa} points out, Buridan's ass never appears in Burdian's writing, though a discussion of a typical Buridan case does occur in unpublished commentary on Aristotle's \emph{De Caelo} in which a dog is dying of hunger between two equal portions of food (\citeyear[154]{Rescher:1960aa}).\nolinebreak
\footnote{\citeauthor{Rescher:1960aa} translates a wonderful passage from \emph{Prize Essay on the Freedom of the Will} in which Schopenhauer writes `I myself own an edition of his \emph{Sophismata}, apparently printed already in the fifteenth century \dots, in which I have repeatedly searched for it in vain, although asses occur as examples on virtually every page.' (\citeyear[153]{Rescher:1960aa})}


There is a rather nice passage from \citeauthor{Reid:1815aa} while illustrates the problem.

\begin{quote}
  Cases frequently occur, in which an end that is of some importance, may be answered equally well by any one of several different means.
  In such cases, a man who intends the end finds not the least difficulty in taking one of these means, though he be firmly persuaded that it has no title to be preferred to any of the others.
  To say that this is a case that cannot happen, is to contradict the experience of mankind; for surely a man who has occasion to lay out a shilling or a guinea, may have two hundred that are of equal value, both to the giver and to the receiver, any one of which will answer his purpose equally well.
  To say, that, if such a case should happen the man could not execute his purpose, is still more ridiculous, though it may have the authority of some of the schoolmen, who determined that the ass, between two equal bundles of hay would stand still till it died of hunger.\nolinebreak
  \mbox{ }\hfill(\citeyear[233--234]{Reid:1815aa})
\end{quote}

The gist of the idea is that there are two distinct ways to satisfy the same desire, not two distinct (nominal) desires which are equally tied---satisfaction can be achieved in two ways---and in this respect arbitrary choice is well motivated.
However, this observation only applies to simple cases.
The more general problem is that one can be equally tied between two options on the basis of different features of satisfaction, or alternatively conflicting nominal desires.
I'm fond of~\citeauthor{Montaigne:1965aa}'s take:

\begin{quote}
  It is a pleasant imagination, to conceive a spirit justly ballanced betweene two equal desires.
  For, it is not to be doubted, that he shall never be resolved upon any match:
  Foresomuch as the application and choise brings an inequality of prise:
  And who should place us betweene a Bottle of Wine and a Gammon of Bacon, with a equal appetite to eat and drinke, doubtlesse there were noe remedy but to die of thirst and hunger.
\end{quote}

Here, the contribution of the abstract view is not so straightforward as there are not two ways things which would satisfy the agent in the same way, but two things which satisfy the agent in two different ways.
I don't have anything substantial to say about this.
The nominal view is in no better position, but that's hardly comforting.

\subsection{Mistakes}
\label{sec:mistakes}

A thought I find comforting is that you on the abstract view one can be completely wrong about what would satisfy them.
For, practical reasoning can generate desires, and there's no necessary connexion between a desire being satisfied and satisfaction, but on the nominal view one must argue that in some sense the agent was satisfied, for there's nothing more to satisfaction that nominal desires, and therefore it must be that the agent derived some satisfaction, but perhaps no significant satisfaction.

Still, what I'm really interested in is the idea that context (in particular) can mislead an agent with respect to which propositions they reason will satisfy them, but oh fuck it's 17:10 \dots


\newpage
\printbibliography

\end{document}


\newpage




A different example with buying glasses.
Or, something like this.
The options are limited, and so I change the context.
But perhaps I don't reflect on the context, perhaps I have a stronger desire to look elsewhere.
Perhaps, and so the story is that although I desire a certain pair of glasses, it's much the same as the menu example, where I am able to turn the page, my desire is in part dependent on the belief about which options are available, and I have a stronger desire to explore alternative options.
Context has a role, but it's not an important role.

Well, the idea is to set this up, the glasses example that is, and to suggest that there's a case where I seem to have a desire, like a genuine desire, but I go and explore other options.
The argument then is that here it's not that I'm worrying about whether I actually desire the pair of glasses, instead it's a stronger desire to explore more options.
Desire comes in different strengths, and the thought that I may find an option which has even greater cognitive significance outweighs the desire for the particular pair.
The na\"{i}ve view, on the other hand, is that I don't associate my desire with the cognitive significance.
Perhaps, so let's then explore a different case.

The tournament case, and I end up frustrating my desire.
There's then a second tournament, and exactly the same thing happens.
However, I act differently.
Now, do I have a higher-order desire not to desire or simply a stronger desire to do nothing?
Or, do I recognise that acting on the cognitive significance of a desire might frustrate, and disassociate what I desire with what has the cognitive significance of a desire?

I think we should adhere to the received wisdom, and say that I do not desire, even though the cognitive significance remains the same.
What motivates is satisfaction, and that I can distinguish between what appears satisfactory and what would satisfy.

If the tournament case seems too bizarre, think about netflix and reading, or eating fast food and preparing a meal.
Is one really cultivating stronger desires, or is there a disconnect?





Winning a game, as desiring different context in the same way.
I don't desire that \emph{Holmes} wins, just as you don't desire that \emph{Moriarty} wins, we both desire that \emph{I} win and the character of our desire does not by itself determine the conditions under which our desires are satisfied.



Satisfaction is to desire as truth is to belief.
The important point here is that one doesn't need to take a strong account on the nature of satisfaction to account for the role of desire in practical reasoning just as one doesn't need to take a strong account on the nature of truth.

What I mean by desire is fairly sparse, but it's most definitely the philosophical usage of desire, and more specifically the sort of folk-theoretic philosophical usage.
In short, the assumption I make about desire is that it's a propositional attitude which when paired with belief provides reasons for action.
This already allows for various positions to be sketched.
Humeanism, weak-Humeanism, and strong-Humeanism.
However, it seems to me that these should be recast as debates about the nature of satisfaction, and my purpose here is to show that a collection of interesting problems arise independently of what we take satisfaction to be.
In particular, the slogan that satisfaction is to desire as truth is to belief should be read to carry the implication that one does not have privelidged access to one's satisfaction.


There's an upshot to this in terms of second-order desires.

But this is not the full stroy.
An agent's belief has character, and it is the character of the agent's belief together with context which determines the content, or proposition of belief.

Simple examples arise from indexicals.
The character of `I am here now' stays fixed between different people and times (viz.\ contexts) but an utterance by a person at a time determines a different proposition.

Still, indexicals are a sympthom of character and are not constitutive of it.
What is captured by character and context is a distinction between \emph{what} and \emph{how}.

Following Perry distinguish between two types of cases.

Type A: The same thing is believed in different ways

Type B: Different context are believed in the same way.

There's a subtle shift here from language to mind.
It's one thing to argue that language depends on character and context, but it is another thing to argue that attitudes depend on character and context.
I'm interested in the latter, and while I think language provides insight into what goes on in the mind, I don't take any of the following to say anything about language \emph{per se}.

The above is to motivate context, but it feels as though I should be able to specify what I mean by context independently of this.



Context + character = content


\section{Belief}
\label{sec:belief}

So, something to the idea that belief has character.
This isn't too well put, but it makes some sense.
Then, put this together with context, and you get beliefs.
However, it's not as straightforward as this.
Beliefs are not simply a report on context.
Now I need an example of where I see something but I do not believe it.
I see a spider crawling along the floor, but I believe that it's a bundle of black thread moved by the draft.
I hear a person upstairs, but I believe the old house is creaking.
My keys are missing, but I believe they're where I left them.
I taste coffee, but I believe it's decaf.
I cannot find an error in my proof/fault in my argument, but I believe it is incorrect.
I see a familiar face on the train, but I do not believe it is my friend.
I am given a complement, but I believe that I have been insulted (or vice-versa).
The letter is signed by the defendant, but I believe it is their lawyers words.
The point here being that the character of my belief is such that it mediates context, and the beliefs which result are not reducible to context (for else I would believe proposition that I do not) nor are they reducible to belief (for without context these beliefs would not be formed).
It's a straightforward point, that our reasoning mediates what is reported to us.
I see problems of co-reference as particular instances of this broader class.
I believe that Lewis Carroll, but not Charles Dodgson, wrote Alice in Wonderland, etc.


{\color{red} Now, do problems of co-reference really occur as special cases of this class?
  It seems not, it seems as though these are cases of something different.
  Well, it's more that the relevant cases of co-reference are where one believes something under one name, and fails to believe, or believes something contrary, under a different name.
  But it's hard to see how this relates.
  The idea would be that, in some sense, names across contexts is a tricky thing, but I don't see this argument working out.
  Isn't it the problem that things appear to us in different contexts, and it takes some work to piece together information across contexts, I don't believe that A and B are the same thing because \dots I come across A in one context and I come across B in another, but I come across Charles Dodgson in many different contexts, and I believe these all refer to the same person.
  Well, often there's information which links context, and as it happens full names are quite good at this.
  Partial names are not, I don't believe that Sam refers to the same person every time I hear the name, which seems like a fair observation.
  So, in this sense these are opposite kinds of cases, in the above I have the same thing happening in multiple contexts, and so I don't form a belief, while in the name case I don't have the same thing in multiple contexts, and so I don't form a belief.
  Perhaps, then, it doesn't make sense to talk about the character of belief, as it seems as though what's happening here that one builds a somewhat context insensitive body of information and relies on that when forming beliefs.
  And, if this is the case then the same reasoning with desire goes though.
  However, in this sense it seems as though all one needs is individual beliefs.
  In some respect I should be fine with this.
}

So, there's a puzzle about exactly what I'm arguing for here.

\section{Enough with the wax}
\label{sec:enough-with-wax}


Do I really desire to help you, or do I desire to help you in so far as I desire that you will help me, and helping you is a way to increase the chance that you will help me?
It's clear that we can't identify a desire with the proposition that's desired, but it also seems clear that we can't identify desire with what is taken to be desired.

This is a point which can be made.
Difference between \emph{what} and \emph{how}.
It's the \emph{what} that's important.

Is this correct?
You're in a blind tournament, and you hear rumours about this person with a particular play style.
You desire that they loose their next game.
So, you desire that you loose your next game.
But this isn't right.

Well, how similar to is this with respect to cases where you don't believe that \(a = b\)?
I think the correct response here is to say that you do desire this.
You do desire something problematic.
Expand the case so that you anonymously engage in some fixing, and you end up loosing because of it.
You're desire is satisfied, but you're other desire is not.
You can have conflicting desire just as you can have conflicting beliefs.

Same with desiring to meet a particular person.
Your desire can be satisfied, even if you can't recognise how it's been satisfied.



So, is this just a belief-desire pair?
In one sense, for sure.
In another sense, however, there's something problematic about thinking of things in this way, as it's definitely different from standard cases in which belief enters the picture.
Of course, belief is relevant in some respect, but I think then we're forced to say that there are never any cases of `simple' desires, because in order to represent these we must invoke beliefs.
To me, folk theory takes for granted a certain level of representation, and permits a certain level of reasoning, without requiring the invocation of belief.
I agree that there are puzzles here, but they are not puzzles at the folk-theoretic level.
Here, your reasoning about your desires can be separated from your beliefs.

Some borderline cases where beliefs are involved.
However, I don't think this is quite right.
It's not the belief that's doing the work in the relevant considerations, it's that there's something there.
Whatever you want to appeal to, it should be separated from the kind of belief that is at work when I put together my desires with my beliefs.

We're well aware that what appears to use as desirable may not be so, and satisfaction may be taken to require the desire of things which do not impress significance.

This may sound anti-Humean, but I don't think it is, at least not in spirit.
I'm not assuming anything about desire, and \emph{that} seems to be the key point.
I am, however, denying that the mind is `directed' by desire, it's rather the inverse, desire directs the mind, it's about satisfaction.
Or, perhaps better put, it seems right to say that one will not be satisfied because one believes they will be.


% \section{Quotes}
% \label{sec:quotes}

% “[Desire] is produced in the margin which exists between the demand for the satisfaction of need and the demand for love” (Seminar V, 11.06.58., p.4).

% Well, there's something here.
% It'd be nice to include it.

% ``To paraphrase Lacan’s point, the joke shows that desire is like a ‘perverted’ form of need by the very process of expressing that need in the form of a demand. The young benefactor obviously needs food, but his hunger can obviously be satisfied by something less than the most expensive and elaborate meals. So we find desire in what ‘pushes beyond’ need in the very expression of a demand.''
% http://www.lacanonline.com/index/2010/05/what-does-lacan-say-about-desire/

% \section{Examples}
% \label{sec:examples}

% spotted-necked otter (Hydrictis maculicollis), or speckle-throated otter

% \url{https://plato.stanford.edu/entries/natural-kinds/}

% \url{https://en.wikipedia.org/wiki/Conversion_of_units#Force}


\nocite{Bratman:1981aa,Bratman:1987aa,Bratman:2003aa,Chislenko:2016aa,Kaplan:1989aa,McDaniel:2008aa,Nottelmann:2011aa,Perry:1993aa,Rescher:1960aa,Schapiro:2014aa,Schroeder:2006aa,Schroeder:2007aa,Schroeder:2017aa,Sinhababu:2013aa,Sinhababu:2017aa,Smith:2004aa,Velleman:2000aa,Velleman:2007aa}






\end{document}