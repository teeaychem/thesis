\documentclass[noamssymb,
graphics,
% compress,
% handout,
]{beamer} % amssymb is incompatible with mtpro2



\usetheme{Antibes}
\usecolortheme{fuchsia}

\setbeamertemplate{navigation symbols}{}
\addtobeamertemplate{navigation symbols}{}{%
    \usebeamerfont{footline}%
    \usebeamercolor[fg]{footline}%
    \hspace{1em}%
    \insertframenumber/\insertmainframenumber
}
\setbeamertemplate{enumerate items}[default]
\setbeamertemplate{bibliography item}{}

% \usepackage{relsize,etoolbox}% http://ctan.org/pkg/{relsize,etoolbox}
% \AtBeginEnvironment{quote}{\smaller}% Step font down one size relative to current font.

\usepackage[T1]{fontenc}
   % \usepackage{textcomp}
   % \usepackage{newtxtext}
   % \renewcommand\rmdefault{Pym} %\usepackage{mathptmx} %\usepackage{times}
   % \usepackage{bm}% Access to bold math symbols
   \usepackage[no-math]{fontspec}
   \defaultfontfeatures{Ligatures=TeX,Numbers={Proportional}}
   \newfontfeature{Microtype}{protrusion=default;expansion=default;}
   \setmainfont[Ligatures=TeX,Scale=MatchLowercase]{Source Serif Pro}
   \setsansfont[Ligatures=TeX,Scale=MatchLowercase,BoldFont={* Semibold}]{Source Sans Pro}
   \setmonofont[Scale=MatchLowercase]{Source Code Pro}
   % \usepackage{selnolig}% For suppressing certain typographic ligatures automatically
   % \usepackage{microtype}
   \newfontfamily\myfont[Ligatures=TeX,Scale=MatchLowercase]{AdornS Pomander Bold}
   % % mtpro2 fix from:
   % https://tex.stackexchange.com/questions/374353/no-room-for-a-new-count
   \makeatletter
   \let\alloc@latex\alloc@
   \def\alloc@#1#2#3#4{\newcount}% the current \newcount doesn't use \alloc@
   \makeatother
   % % %
   \usepackage[complete, subscriptcorrection, slantedGreek, mtpfrak, mtpbb, mtpcal]{mtpro2}
% % % % % % %
\usepackage{amsthm}         % (in part) For the defined environments
\usepackage{mathtools}      % Improves on amsmaths/mtpro2

\usepackage{calc}

% % % The bibliography % % %
\usepackage[backend=biber,
  style=authoryear-comp,
  bibstyle=authoryear,
  citestyle=authoryear-comp,
  uniquename=false,%allinit,
  giveninits=true,
  backref=false,
  % hyperref=true,
  url=false,
  isbn=false,
  useprefix=true,
  ]{biblatex}
\DeclareFieldFormat{postnote}{#1}
\DeclareFieldFormat{multipostnote}{#1}
% \setlength\bibitemsep{1.5\itemsep}
\newcommand{\noopsort}[1]{}
\addbibresource{Thesis.bib}
% % % % % % % % % % % % % % %

% \usepackage[inline]{enumitem}
% \setlist[itemize]{noitemsep}
% \setlist[description]{style=unboxed,leftmargin=\parindent,labelindent=\parindent,font=\normalfont\space}
% \setlist[enumerate]{noitemsep}


\usepackage{pifont}
\newcommand{\hand}{\ding{43}}
\usepackage{array}

\usepackage{setspace}

\newcommand{\schemaName}[1]{\textsc{#1}}
\newcommand{\principleName}[1]{\textsc{#1}}
\newcommand{\dependencePrinciple}[0]{\textsf{Dependence Principle}}
% % %
\usepackage{dashrule}
\newcommand{\hozline}[0]{
  \noindent\hdashrule[0.5ex][c]{\textwidth}{.1pt}{}
}
\newcommand{\hozlinedash}[0]{
  \noindent\hdashrule[0.5ex][c]{\textwidth}{.1pt}{2.5pt}
}
\newcommand{\btVFill}{\vskip0pt plus 1filll}
\newcommand{\citeline}[0]{\btVFill\vspace{-8pt}\noindent\hdashrule[0.5ex][c]{.25\textwidth}{.1pt}{}\vspace{-8pt}}
%https://tex.stackexchange.com/questions/347228/footnote-without-numbering-and-without-indention-in-beamer-class
\newcommand\citenote[1]{%
  \pgfsetfillopacity{0.33}
  \tikz[remember picture,overlay]
  \draw (current page.south west) +(1in + \oddsidemargin,0.5em)
  node[anchor=south west,inner sep=0pt]{\parbox{\textwidth}{%
      \rlap{\rule{10em}{0.4pt}}\raggedright\scriptsize#1}};
}
% % %

\usepackage{tikz}
\usetikzlibrary{arrows,shapes}
\usetikzlibrary{fit}
\usetikzlibrary{fadings,patterns}


% https://tex.stackexchange.com/questions/227871/how-to-place-a-color-gradient-text-node-inside-a-tikzpicture-absolute-positioni
%******************************************************************
%
% Defining a new coordinate system for the page:
%
% --------------------------
% |(-1,1)    (0,1)    (1,1)|
% |                        |
% |(-1,0)    (0,0)    (1,0)|
% |                        |
% |(-1,-1)   (0,-1)  (1,-1)|
% --------------------------
\makeatletter
\def\parsecomma#1,#2\endparsecomma{\def\page@x{#1}\def\page@y{#2}}
\tikzdeclarecoordinatesystem{page}{
    \parsecomma#1\endparsecomma
    \pgfpointanchor{current page}{north east}
    % Save the upper right corner
    \pgf@xc=\pgf@x%
    \pgf@yc=\pgf@y%
    % save the lower left corner
    \pgfpointanchor{current page}{south west}
    \pgf@xb=\pgf@x%
    \pgf@yb=\pgf@y%
    % Transform to the correct placement
    \pgfmathparse{(\pgf@xc-\pgf@xb)/2.*\page@x+(\pgf@xc+\pgf@xb)/2.}
    \expandafter\pgf@x\expandafter=\pgfmathresult pt
    \pgfmathparse{(\pgf@yc-\pgf@yb)/2.*\page@y+(\pgf@yc+\pgf@yb)/2.}
    \expandafter\pgf@y\expandafter=\pgfmathresult pt
}
\makeatother
%******************************************************************

\newcommand\shadetext[2][]{%
  \setbox0=\hbox{{\special{pdf:literal 7 Tr }#2}}%
  \tikz[baseline=0]\path [#1] \pgfextra{\rlap{\copy0}} (0,-\dp0) rectangle (\wd0,\ht0);%
}
% % % % % % % % % % % %




\usepackage{svg}

% % % Color % % %
% \usepackage[usenames, dvipsnames]{xcolor}
% \usepackage{everysel}
\definecolor{fuchsia}{HTML}{FE4164}%Neon Fuchsia %{F535AA}%Neon Pink
\definecolor{olga}{HTML}{eebdda}
\definecolor{adam}{HTML}{96d3e6}
\usecolortheme[named=fuchsia]{structure}
\colorlet{structure}{fuchsia}
\usestructuretemplate{\color{structure}}{}
% % % % % % % % % %

% Set quotes to non-italic
\setbeamerfont{quote}{shape=\upshape,family=\rmfamily}



\setbeamercovered{transparent}

% https://tex.stackexchange.com/questions/28654/beamer-table-of-contents-display-all-subsections-below-section
\AtBeginSection[]
{
  {
    \setbeamertemplate{navigation symbols}{}
    \begin{frame}[noframenumbering]
      \frametitle{Structure}
      \setbeamertemplate{navigation symbols}{ }
      \tableofcontents[
      currentsection,
      % hideothersubsections,
      % sectionstyle=show/hide,
      % subsectionstyle=show/shaded,
      subsubsectionstyle=hide,
      ]
    \end{frame}
  }
}

\makeatletter
\newlength\beamerleftmargin
\setlength\beamerleftmargin{\Gm@lmargin}
\makeatother

\title{Means-end reasoning and means-end relations}
\author{Ben Sparkes}
% \institute{Stanford}
\date{June 17, 2020}

\begin{document}
\tikzstyle{every picture}+=[remember picture]
\def\beamertemplatetransparentcoveredmedium{\setbeamercovered{transparent=70}}
\beamertemplatetransparentcoveredmedium

{
  \setbeamertemplate{navigation symbols}{}
  \begin{frame}[plain, noframenumbering]
    \hspace*{-\beamerleftmargin}%
    % \frametitle{Congratulations}
    % \begin{center}
    % {\myfont\Huge Congratulations!}
    % \end{center}
    {\tikz[remember picture,overlay]
      \node[scale=2.75] at (.5\paperwidth,.25) {\myfont \shadetext[left color=adam, right color=olga]{\space Congratulations!\space}};
    }
    \only<1>{
      {
        \tikz[remember picture,overlay]
        \draw (current page.south west) +(.275\linewidth, 3em)
        node[anchor=south west,inner sep=0pt, opacity=1]{\def\svgwidth{.3\linewidth}
          \includeinkscape{images/kitty-colour}};
      }
      {
        \tikz[remember picture,overlay]
        \draw (current page.south west) +(.575\linewidth, 3em)
        node[anchor=south west,inner sep=0pt, opacity=1]{\def\svgwidth{.3\linewidth}
          \includeinkscape{images/daniel-colour}};
      }
    }
    \only<2>{
      {
        \tikz[remember picture,overlay]
        \draw (current page.south west) +(.415\linewidth, 2.8em)
        node[anchor=south west,inner sep=0pt, opacity=.5]{\def\svgwidth{.3\linewidth}
          \includeinkscape{images/daniel-colour}};
      }
      {
        \tikz[remember picture,overlay]
        \draw (current page.south west) +(.42\linewidth, 3em)
        node[anchor=south west,inner sep=0pt, opacity=.5]{\def\svgwidth{.3\linewidth}
          \includeinkscape{images/kitty-colour}};
      }
      {
        \tikz[remember picture,overlay]
        \draw (current page.south west) +(.415\linewidth, 2.8em)
        node[anchor=south west,inner sep=0pt, opacity=.5]{\def\svgwidth{.3\linewidth}
          \includeinkscape{images/daniel-colour}};
      }
      {
        \tikz[remember picture,overlay]
        \draw (current page.south west) +(.42\linewidth, 3em)
        node[anchor=south west,inner sep=0pt, opacity=.5]{\def\svgwidth{.3\linewidth}
          \includeinkscape{images/kitty-colour}};
      }
      {
        \tikz[remember picture,overlay]
        \draw (current page.south west) +(.415\linewidth, 2.8em)
        node[anchor=south west,inner sep=0pt, opacity=.5]{\def\svgwidth{.3\linewidth}
          \includeinkscape{images/daniel-colour}};
      }
      {
        \tikz[remember picture,overlay]
        \draw (current page.south west) +(.42\linewidth, 3em)
        node[anchor=south west,inner sep=0pt, opacity=.5]{\def\svgwidth{.3\linewidth}
          \includeinkscape{images/kitty-colour}};
      }
    }
  \end{frame}
}

\begin{frame}[noframenumbering]
  \setbeamertemplate{navigation symbols}{ }
  \titlepage
\end{frame}


\section*{Overview}
\label{sec:overview}

\begin{frame}
  \frametitle{Overview}
  \begin{itemize}
  \item Means-end reasoning and means-end relations.
  \item \only<1>{When is a rational agent permitted to settle on an action (in part) as a means?}
  \only<2->{What connexion holds between a rational agent and a means-end relation when they agent is permitted to settle on an action (in part) as a means?}
  % \item Understand how means-end relations are used in practical reasoning.
  \only<3->{
    \begin{enumerate}
    \item<3,5> It is \textcolor<3->{fuchsia}{\emph{not}} the case that the agent needs to \textcolor<3->{fuchsia}{\emph{reason via}} a means-end relation from an end they have supporting them action.
    \item<4,5> It \textcolor<4->{fuchsia}{\emph{is}} the case that the agent needs to \textcolor<4->{fuchsia}{\emph{takes there to be}} a means-end relation from an end they have supporting the action.
    \end{enumerate}
\only<1-4>{%\color{white}
\item[] {\color{white} Goal is to highlight that an agent taking there to be a means-end relation is always available as an explanatory resource, and to motivate the use of this resource by showing that a recognised resource fails in certain cases.}
}
}
\only<5>{
\item Goal is to highlight that an agent taking there to be a means-end relation is always available as an explanatory resource, and to motivate the use of this resource by showing that a recognised resource fails in certain cases.
  }
\end{itemize}

% \hozlinedash

% \begin{itemize}
% \item The purpose of the talk is to convince you of these two claims.
% \item Putting the positive claim to work will be left for some other time, perhaps the Q\&A!
% \item My goal is establish the positive claim so that you can have some fun putting it to work.
% \item I want \emph{you} to apply the positive claim to your favoured theory or model of practical reasoning and show \emph{me} the interesting stuff it can do!
% \item Though, so long as things go okay today, I'll do \emph{some} of this work in the thesis\dots
% \end{itemize}

% \hozlinedash

% \begin{itemize}
% \item I will suggest a few ways in which the perspective could be useful at the end of the talk, but the focus of the talk is on the basics.
% \item In most cases of an agent settling on a means both conditions could be argued for.
% \item So, showing that it is always the case that agent takes there to be a means-end relation and showing that in some cases the agent taking there to be a means-end relation does important work is a way to establish the explanatory potential of the alternative link.
% \end{itemize}

\end{frame}


\begin{frame}[noframenumbering]
  \frametitle{Overview (structure)}
  \setbeamertemplate{navigation symbols}{ }
  \tableofcontents[
  % currentsection,
  % hideothersubsections,
  % sectionstyle=show/hide,
  % subsectionstyle=show/shaded,
  subsubsectionstyle=hide,
  ]
\end{frame}



\section{Means-end schema}
\label{sec:schema}

\begin{frame}
  \frametitle{The means-end schema}

  \begin{block}<1->{The Schema}
    \textcolor<3>{fuchsia}{A rational agent} is \textcolor<4>{fuchsia}{permitted to settle on an action (in part) as a means}
    \newline
    \mbox{ }\hfill\emph{only if}\hfill\mbox{ }
    \newline
    \textcolor<3>{fuchsia}{The agent} [\textcolor<6>{fuchsia}{does something with}] \textcolor<5>{fuchsia}{a means-end relation from an end they have which supports taking the relevant means}.
  \end{block}

  \onslide<2->{Four parts to the schema:}
  \begin{enumerate}
  \item<3-> \textcolor<3>{fuchsia}{A rational agent}
  \item<4-> \textcolor<4>{fuchsia}{The agent being permitted to settle on an action (in part) as a means.}
  \item<5-> \textcolor<5>{fuchsia}{A means-end relation from an end the agent has which supports taking the means.}
  \item<6-> \textcolor<6->{fuchsia}{\emph<7>{A link between the agent has to the specified means-end relation.}}
  \end{enumerate}
\end{frame}

\begin{frame}
  \frametitle{Example}

  \begin{quote}
    You might reason like this:\newline
    \mbox{}\quad \textcolor<4>{fuchsia}{I am going to buy a boat} \hfill {(1a)}\newline
    and \only<6->{\textcolor<6>{fuchsia}{\hfill\(\mbox{\mid\hspace{-.75pt}\mid}\)\hspace{5.25pt}\mbox{}}}\newline
    \mbox{}\quad \textcolor<5>{fuchsia}{For me to buy a boat, a necessary means is to borrow money} \hfill {(1b)}\newline
    so \only<6>{\textcolor<6>{fuchsia}{\hfill\(\Downarrow\)\hspace{3.75pt}\mbox{}}}\newline
    \mbox{}\quad \textcolor<3>{fuchsia}{I shall borrow money.} \hfill {(1c)}
  \end{quote}
  \vspace{-10pt}
  \hozlinedash
  \vspace{-5pt}
  \onslide<2->{\vspace{-10pt}
    \begin{block}{\only<1-5>{The Schema}\only<6>{\schemaName{Reasoning}}}
      A rational agent is \textcolor<3>{fuchsia}{permitted to settle on an action (in part) as a means}
      \newline
      \mbox{ }\hfill\emph{only if}\hfill\mbox{ }
      \newline
      the agent
      \only<-5>{\makebox[\widthof{[does something with]}][c]{[does something with]}}\nolinebreak
      \only<6->{\makebox[\widthof{[does something with]}][c]{\textcolor{fuchsia}{\emph{reasons via}}}} \textcolor<5>{fuchsia}{a means-end relation from} \textcolor<4-5>{fuchsia}{an end they have} \textcolor<5>{fuchsia}{which supports taking the relevant means}.
    \end{block}
    \citenote{\fullcite{Broome:2002aa}}
  }
\end{frame}

\begin{frame}
  \frametitle{\schemaName{Reasoning}}

  \begin{block}{\schemaName{Reasoning}}
    A rational agent is permitted to settle on an action (in part) as a means
    \newline
    \mbox{ }\hfill\emph{only if}\hfill\mbox{ }
    \newline
    the agent \textcolor{fuchsia}{\emph{reasons via}} a means-end relation from an end they have which supports taking the relevant means.
  \end{block}

  \begin{itemize}
  \item Agent is required to connect means to end.
  \item Many cases of reasoning satisfy this requirement.
  \item Standard way of thinking about means-end reasoning.
  \end{itemize}
\end{frame}


\begin{frame}
  \frametitle{Example}

  \only<1>{
    \begin{quote}
      I am going to buy a boat \hfill (1a)\newline
      For me to buy a boat, a necessary means is to borrow money \hfill (1b)\newline
      I shall borrow money. \hfill (1c)
    \end{quote}
    \hozlinedash
  }
  \only<2->{
    \vspace{-28pt}
    \begin{block}{\schemaName{Reasoning}}
      A rational agent is permitted to settle on an action (in part) as a means
      \newline
      \mbox{ }\hfill\emph{only if}\hfill\mbox{ }
      \newline
      the agent \textcolor{fuchsia}{\emph{reasons via}} a means-end relation from an end they have which supports taking the relevant means \emph{by reasoning via the relation}.
    \end{block}
    \hozlinedash
  }
  \only<1-2>{
    \begin{quote}
      Your reasoning process is a particular type of practical reasoning.
      It
      % \textcolor<2>{fuchsia}{\emph<2>{is}}
      \only<1>{\makebox[\widthof{is}][c]{is}}
      \only<2>{\makebox[\widthof{is}][c]{\textcolor{fuchsia}{\emph{is}}}}
      \only<1>{\makebox[\widthof{instrumental}][c]{\emph{instrumental}}}
      \only<2>{\makebox[\widthof{instrumental}][c]{instrumental}}
      reasoning, which means it is concerned with taking an appropriate means to an end.%\nolinebreak
      % \mbox{}\hfill\mbox{(\citeyear[86]{Broome:2002aa})}
    \end{quote}
    \vspace{13pt}
    \citenote{\fullcite{Broome:2002aa}}%\citeauthor{Broome:2002aa} (\citeyear{Broome:2002aa}) \citetitle{Broome:2002aa}
  }
  \only<3>{
    \begin{quote}
      \emph{Desire–Belief Theory of Reasoning:}\newline
      Desire is affected as the conclusion of reasoning if and \textcolor{fuchsia}{only if} desire that E is combined with belief that M would raise E's probability, constituting desire that M.
    \end{quote}
    \citenote{\fullcite{Sinhababu:2017aa}}
  }
\end{frame}

\begin{frame}
  \frametitle{\schemaName{Taking}}

  \begin{block}{\schemaName{Taking}}
    A rational agent is permitted to settle on an action (in part) as a means
    \newline
    \mbox{ }\hfill\emph{only if}\hfill\mbox{ }
    \newline
    the agent \textcolor{fuchsia}{\emph{takes there to be}} a means-end relation from an end they have which supports taking the relevant means.
  \end{block}

  \only<2>{
    \begin{itemize}
    \item An existential-like requirement that does not require the agent to reason to the means from an end.
    \item Some cases of reasoning may satisfy this requirement.
      \vspace{+64.5pt}
    \end{itemize}
  }
  \only<3>{
    \vspace{-10pt}
    \begin{exampleblock}{}
      \begin{itemize}
        \setlength\itemsep{-.1em}
      \item On your desk is a stack of papers.
      \item You recognise that you created the stack for some end, but you cannot recall what that end is.
      \item You are confident that the end remains relevant to you, and you begin to read the papers.
      \item As you read, you begin to recall the research project that led you to forming the stack.
      \end{itemize}
    \end{exampleblock}
  }
\end{frame}


\begin{frame}
  \frametitle{I will argue \emph{for}:}

  \begin{block}{\schemaName{Taking}}
    A rational agent is permitted to settle on an action (in part) as a means
    \newline
    \mbox{ }\hfill\emph{only if}\hfill\mbox{ }
    \newline
    the agent \textcolor{fuchsia}{\emph{takes there to be}} a means-end relation from an end they have which supports taking the relevant means.
  \end{block}
\end{frame}


\begin{frame}
  \frametitle{I will argue \emph{against}:}

  \begin{block}{\schemaName{Reasoning}}
    A rational agent is permitted to settle on an action (in part) as a means
    \newline
    \mbox{ }\hfill\emph{only if}\hfill\mbox{ }
    \newline
    the agent \textcolor{fuchsia}{\emph{reasons via}} a means-end relation from an end they have which supports taking the relevant means.
  \end{block}
\end{frame}

\begin{frame}
  \frametitle{No logical entailment between the principles}

  \begin{block}{\schemaName{Reasoning}}
    A rational agent is permitted to settle on an action (in part) as a means
    \newline
    \mbox{ }\hfill\emph{only if}\hfill\mbox{ }
    \newline
    the agent \textcolor{fuchsia}{\emph{reasons via}} a means-end relation from an end they have which supports taking the relevant means.
  \end{block}

  {\Large \mbox{ }\hfill \(\textcolor{fuchsia}{\not}\hspace{-4pt}\Uparrow\) \qquad \(\textcolor{fuchsia}{\not}\hspace{-4pt}\Downarrow\) \hfill\mbox{ }}

  \begin{block}{\schemaName{Taking}}
    A rational agent is permitted to settle on an action (in part) as a means
    \newline
    \mbox{ }\hfill\emph{only if}\hfill\mbox{ }
    \newline
    the agent \textcolor{fuchsia}{\emph{takes there to be}} a means-end relation from an end they have which supports taking the relevant means.
  \end{block}

\end{frame}


\begin{frame}
  \frametitle{Possible logical entailment from \schemaName{Reasoning} to \schemaName{Taking}}

  \begin{block}{(taking by) \schemaName{Reasoning}}
   A rational agent is permitted to settle on an action (in part) as a means
    \newline
    \mbox{ }\hfill\emph{only if}\hfill\mbox{ }
    \newline
    the agent \textcolor{fuchsia}{\emph{takes there to be}} a means-end relation from an end they have which supports taking the relevant means \textcolor{fuchsia}{\emph{(by reasoning via the relation)}}.
  \end{block}

  {\Large \mbox{ }\hfill\(\Downarrow\)\hfill\mbox{ }}

  \begin{block}{\schemaName{Taking}}
    A rational agent is permitted to settle on an action (in part) as a means
    \newline
    \mbox{ }\hfill\emph{only if}\hfill\mbox{ }
    \newline
    the agent \textcolor{fuchsia}{\emph{takes there to be}} a means-end relation from an end they have which supports taking the relevant means.
  \end{block}

\end{frame}


\section{Argument}
\label{sec:argument}


\subsection{Negative part (against \emph{reasoning via})}% \schemaName{Reasoning})}
\label{sec:negative}

\begin{frame}
  \frametitle{Negative argument}

  {%
  \setbeamertemplate{enumerate item}{N\arabic{enumi}.}
  \begin{enumerate}
  \item\label{scenarios:exist} There are cases in which rational agents are permitted to settle on an action (in part) as means without being able to reason from an end they have to the action.
  \item\label{scenarios:persmissible} And, in these cases the agent is rational in settling on the action (in part) as a means.
  \item \schemaName{Reasoning} does not hold.
    % \item[C\(_{\text{I}}\).]\label{scenario:no-reasoning} In order for an agent to be rationally permitted in settling on a means it cannot be required that the agent reasons via a means-end relation from an end they have which supports taking the relevant means.
    \begin{itemize}
    \item From \ref{scenarios:exist} and~\ref{scenarios:persmissible}.
    \end{itemize}
  \end{enumerate}
  }

  \hozlinedash
  {
    \begin{block}{\schemaName{Reasoning}}
      A rational agent is permitted to settle on an action (in part) as a means
      \newline
      \mbox{ }\hfill\emph{only if}\hfill\mbox{ }
      \newline
      the agent \textcolor{fuchsia}{\emph{reasons via}} a means-end relation from an end they have which supports taking the relevant means.
    \end{block}
  }
\end{frame}

\subsubsection{Counterexample}
\label{sec:counterexample}

\begin{frame}
  \frametitle{Oblique (counterexample to \schemaName{Reasoning})}

  \only<1>{\rmfamily
    % \begin{spacing}{1.3}
    \begin{itemize}
      % \setbeamertemplate{itemize items}{}
      \item Oblique regains awareness after drifting off for a second.%\newline
      \item They are in a supermarket, and their hand is stretched toward an item on a shelf.%\newline
      \item Oblique is unable to recall why they are in the supermarket and is unable to recognise what the item on the shelf would be for.%\newline
      \item Oblique does not have a shopping list.%\newline
      \item However, Oblique's outstretched hand indicates to them that they were about to purchase the item, and so they settle on doing so.
      \end{itemize}
    % \end{spacing}
    }
    \only<2>{\rmfamily
    % \begin{spacing}{1.3}
    \begin{itemize}
      % \setbeamertemplate{itemize items}{}
      \item Oblique regains awareness after drifting off for a second.%\newline
      \item They are in a supermarket, and their hand is stretched toward an item on a shelf.%\newline
      \item Oblique is unable to recall why they are in the supermarket and is unable to recognise what the item on the shelf would be for.%\newline
      \item[\hand] Oblique does not have a shopping list.%\newline
      \item However, Oblique's outstretched hand indicates to them that they were about to purchase the item, and so they settle on doing so.
      \end{itemize}
    % \end{spacing}
  }
\end{frame}

\begin{frame}
  \frametitle{Oblique settled without satisfying \schemaName{Reasoning}}

  \begin{itemize}
  \item Oblique settles on purchasing the item as a means to some end.
  \item Oblique's outstretched hand provides information that purchasing the item was something they settled on.
  \item[\hand] Oblique is unable to reason via a means-end relation from an end they have which supports purchasing the item.
    \begin{itemize}
    \item Oblique is unable to reason about the relevant end.
    \item Without the relevant end Oblique is unable to reason about means-end relations from the end to the means.
    \end{itemize}
  \end{itemize}

\end{frame}


\begin{frame}
  \frametitle{Oblique is permitted to settle on the means}

  \begin{itemize}
  \item<+> Oblique's outstretched hand provides information that purchasing the item was something they settled on.
  \item<+> Oblique is unable to settle on purchasing the item given the means-end relations they are able to reason via.
  \end{itemize}
  \hozlinedash
  \vspace{-15pt}
  \begin{itemize}
  \item<+-> Two ways to motivate that Oblique was permitted to settle on purchasing the item:
    \begin{enumerate}
    \item<+> It seems absurd to send Oblique all the way home to look at the shopping list they left on their fridge.
    \item<+-> Purchasing the item is supported by a means-end relation.
      \begin{itemize}
      \item<.-> If Oblique settled on the purchasing the item, this was supported by a means-end relation and some end.
      \item<.-> And, as Oblique only nodded off for a moment, Oblique reasons that the relevant means-end relation and end continue to hold.
      \end{itemize}
    \end{enumerate}
  \end{itemize}

\end{frame}

\begin{frame}
  \frametitle{Key features of the counterexample}

  \begin{enumerate}[A)]
  \item<+> Oblique has enough information to determine that an act would be worthwhile as a means to some end.
  \item<+> Oblique does not have sufficient information to determine how a means-end relation from the end would support the means.
  \item<+> Oblique has sufficient information to determine that a supporting means-end relation from the end they have holds.
  \end{enumerate}

  \onslide<+>{
  \hozlinedash
  \begin{itemize}
  \item Oblique's is unable to recall the end ensures Oblique cannot reason via a means-end relation.
  \item[\hand] Oblique's inability to reason via a means-end relation is key.
  \end{itemize}
  }
\end{frame}

\begin{frame}
  \frametitle{Negative argument (recap)}

  {%
  \setbeamertemplate{enumerate item}{N\arabic{enumi}.}
  \begin{enumerate}
    \item There are cases in which rational agents are permitted to settle on an action (in part) as means without being able to reason from an end they have to the action.
  \item And, in these cases the agent is rational in settling on the action (in part) as a means.
  \item \schemaName{Reasoning} does not hold.
    % \item[C\(_{\text{I}}\).]\label{scenario:no-reasoning} In order for an agent to be rationally permitted in settling on a means it cannot be required that the agent reasons via a means-end relation from an end they have which supports taking the relevant means.
    \begin{itemize}
    \item From \ref{scenarios:exist} and~\ref{scenarios:persmissible}.
    \end{itemize}
  \end{enumerate}
  }

  \hozlinedash
  {%\footnotesize
    \begin{block}{\schemaName{Reasoning}}
      A rational agent is permitted to settle on an action (in part) as a means
      \newline
      \mbox{ }\hfill\emph{only if}\hfill\mbox{ }
      \newline
      the agent \textcolor{fuchsia}{\emph{reasons via}} a means-end relation from an end they have which supports taking the relevant means.
    \end{block}
  }
  % \begin{itemize}
  % \item<2>[\hand] Oblique settled on taking the item as a means to some end without being able to recall the end, and hence without being about to reason via a means-end relation from the end to taking the item.
  % \end{itemize}
\end{frame}

\subsection{Positive part (for \emph{taking there to be})}%\schemaName{Taking})}
\label{sec:positive}

\begin{frame}
  \frametitle{I now will argue \emph{for}:}

  \begin{block}{\schemaName{Taking}}
    A rational agent is permitted to settle on an action (in part) as a means
    \newline
    \mbox{ }\hfill\emph{only if}\hfill\mbox{ }
    \newline
    the agent \textcolor{fuchsia}{\emph{takes there to be}} a means-end relation from an end they have which supports taking the relevant means.
  \end{block}
\end{frame}

\begin{frame}
  \frametitle{Positive argument sketch}
  \setbeamertemplate{enumerate item}{\Roman{enumi}.}
  \setbeamertemplate{enumerate subitem}{\Roman{enumi}(\alph{enumii}).}
  \begin{enumerate}
  \item Assume a rational agent settles on an action (in part) as a means.
  \item A rational agent settles on an action (in part) as a means only if
    \begin{enumerate}
    \item the agent evaluates the action as a means, and
    \item the evaluation supports settling on the action (as a means).
    \end{enumerate}
  \item Evaluating an action as a means requires a supporting means-end relation and an evaluation of the end.
    % \begin{enumerate}
    % \item Whether the means is worthwhile depends on whether the end is worthwhile.
    % \end{enumerate}
  \item As the agent evaluates the action as a means, there must be a supporting means-end relation and an evaluation of the end.
  \item As the agent's evaluation of the means depends on their evaluation of the end, and the agent's evaluation of the means supports settling on the action (as a means), the agent's evaluation of the end also supports settling on the action (as a means).
  \item \schemaName{Taking} holds.
  \end{enumerate}
\end{frame}

\begin{frame}
  \frametitle{Positive argument}
  \tikzstyle{na} = [baseline=-.5ex]
  \only<1-7>{
    \setbeamertemplate{enumerate item}{P\arabic{enumi}.}
    \setbeamertemplate{enumerate subitem}{P\arabic{enumi}\alph{enumii}.}
    \begin{enumerate}
    \item \onslide<1,2>{A rational agent is permitted to settle on an action (in part) as a means \textcolor<2>{fuchsia}{only if}}
      \onslide<1,4->{
      \begin{enumerate}
      \item\tikz[na] \node[coordinate, xshift=-3em] (a1) {}; \textcolor<4>{fuchsia}{\emph{the agent evaluates the action as a means}}, and
      \item\tikz[na] \node[coordinate, xshift=-3em] (b1) {}; \textcolor<5>{fuchsia}{\emph{the evaluation supports settling on the action (as a means)}}.
      \end{enumerate}
      }
    \item \onslide<1,3,4->{An evaluation of an action as a means requires a supporting means-end relation and evaluation of the end.}
    \item \onslide<1,2>{A rational agent is permitted to settle on an action (in part) as a means \textcolor<2>{fuchsia}{only if}}
      \onslide<1,4->{
      \begin{enumerate}
      \item\tikz[na] \node[coordinate, xshift=-3em] (a2) {}; \textcolor<4,7>{fuchsia}{\emph{the agent takes there to be a means-end relation supporting the means and}} \textcolor<4>{fuchsia}{\emph{evaluates the end}}, and
      \item\tikz[na] \node[coordinate, xshift=-3em] (b2) {}; \textcolor<5,7>{fuchsia}{\emph{the evaluation of the end supports settling on the action}} \textcolor<5>{fuchsia}{\emph{as a means to the end}}.
      \end{enumerate}
      }
    \item \textcolor<7>{fuchsia}{\schemaName{Taking} holds}.
    \end{enumerate}
  }
  \only<8>{
    \setbeamertemplate{enumerate item}{P\arabic{enumi}.}
    \setbeamertemplate{enumerate subitem}{P\arabic{enumi}\alph{enumii}.}
    \begin{enumerate}
      \setcounter{enumi}{2}
    \item A rational agent is permitted to settle on an action (in part) as a means only if
      \begin{enumerate}
      \item \emph{the agent takes there to be a means-end relation supporting the means and evaluates the end}, and
      \item \emph{the evaluation of the end supports settling on the action as a means to the end}.
      \end{enumerate}
    \item \schemaName{Taking} holds.
    \end{enumerate}
    \hozlinedash
    \begin{block}{\schemaName{Taking}}
      A rational agent is permitted to settle on an action (in part) as a means
      \newline
      \mbox{ }\hfill\emph{only if}\hfill\mbox{ }
      \newline
      the agent \textcolor{fuchsia}{\emph{takes there to be}} a means-end relation from an end they have which supports taking the relevant means.
    \end{block}
  }
  \begin{tikzpicture}[overlay]
    \path[->]<4> (a1) edge [very thick, draw=fuchsia, opacity=.75, bend right=75] (a2);
    \path[->]<5> (b1) edge [very thick, draw=fuchsia, opacity=.75, bend right=75] (b2);
  \end{tikzpicture}
\end{frame}

\begin{frame}
  \frametitle{Premise P1}
  {%
    \setbeamertemplate{enumerate item}{P\arabic{enumi}.}
    \setbeamertemplate{enumerate subitem}{P\arabic{enumi}\alph{enumii}.}
    \begin{enumerate}
    \item A rational agent is permitted to settle on an action (in part) as a means only if
      \begin{enumerate}
      \item \emph{the agent evaluates the action as a means}, and
      \item \emph{the evaluation supports settling on the action (as a means)}.
      \end{enumerate}
    \end{enumerate}
  }
\end{frame}

\begin{frame}
  \frametitle{Premise P1a}

  \begin{block}{\principleName{Principle A: Settling}}
    A rational agent is permitted to settle on an action \emph{on the basis of some reasoning about the action} if and only if the agent evaluates
    \begin{enumerate}
    \item whether the action is possible, and \hfill \only<2->{(\(\approx\)\emph{mind-to-world})}
    \item whether the action is worthwhile. \hfill \only<2->{(\(\approx\)\emph{world-to-mind})}
    \end{enumerate}
  \end{block}
  \btVFill
    \only<1>{
    \setbeamertemplate{enumerate item}{P\arabic{enumi}.}
    \setbeamertemplate{enumerate subitem}{P\arabic{enumi}\alph{enumii}.}
    \begin{enumerate}
    \item A rational agent is permitted to settle on an action (in part) as a means only if
      \begin{enumerate}
      \item \textcolor{fuchsia}{\emph{the agent evaluates the action as a means}}, and
      \item \emph{the evaluation supports settling on the action (as a means)}.
      \end{enumerate}
    \end{enumerate}
  }
  \only<2>{
    \begin{itemize}
    \item If an agent is permitted to settle on an action \emph{on the basis of some reasoning about the action}, the action is evaluated.
    \item The `if and only if' of \principleName{Settling} does not exclude other considerations contributing to an agent's evaluation of an action, but these are not required.
      \begin{itemize}
      \item[\hand] This does not commit the agent to evaluating the action \emph{as} possible and worthwhile.
      \end{itemize}
    \end{itemize}
  }
  \only<3->{
    \vspace{-5pt}
    \begin{quote}
      Belief and desire \dots\space are correlative dispositional states of a potentially rational agent.
      To desire that \emph{P} is to be disposed to act in ways that would tend to bring it about that \emph{P} in a world in which one's beliefs, whatever they are, were true.
      To believe that \emph{P} is to be disposed to act in ways that would tend to satisfy one's desires, whatever they are, in a world in which \emph{P} (together with one’s other beliefs) were true.
    \end{quote}
    \citenote{\fullcite{Stalnaker:1984aa}}
  }
  \btVFill
\end{frame}

\begin{frame}
  \frametitle{Premise P1b}

  {%
    \setbeamertemplate{enumerate item}{P\arabic{enumi}.}
    \setbeamertemplate{enumerate subitem}{P\arabic{enumi}\alph{enumii}.}
    \begin{enumerate}
    \item A rational agent is permitted to settle on an action (in part) as a means only if
      \begin{enumerate}
      \item \emph{the agent evaluates the action as a means}, and
      \item \textcolor{fuchsia}{\emph{the evaluation supports settling on the action (as a means)}}.
      \end{enumerate}
    \end{enumerate}
  }
  \hozlinedash
  \btVFill
  \only<2>{
    \begin{itemize}
    \item[\hand] If an action is settled on (in part) as a means then conceptualisation of the action as a means \emph{does some work} in establishing that settling on the action is permissible.
    \item The work is the role of the action as a means counts in favour of the action; in contributing to settling the issue of acting.
    \item An evaluation is required for the relevant work to be done.
      \begin{itemize}
      \item Consider Oblique in the supermarket, if they did not consider the action as a means, they would not settle on purchasing the item.
      \end{itemize}
    \end{itemize}
  }
  \only<3>{
    \begin{exampleblock}{}
      I reason that settling on listening to Taeko Ohnuki's \emph{Sunshower} in part as a means to relaxing to be permissible.
      \begin{itemize}
      \item Relaxing does some work in my reasoning, even if I could have reasoned that listening to \emph{Sunshower} is permissible based on how nice it sounds.
      \item If I reasoned that listening to \emph{Sunshower} would be permissible based only on how nice it sounds, then I would not have settling on the action in part as a means to relaxing.
      \end{itemize}
    \end{exampleblock}
  }
  \btVFill
\end{frame}

\begin{frame}
  \frametitle{Premise P2}
  {%
    \setbeamertemplate{enumerate item}{P\arabic{enumi}.}
    \setbeamertemplate{enumerate subitem}{P\arabic{enumi}\alph{enumii}.}
    \begin{enumerate}
      \setcounter{enumi}{1}
    \item \emph{An evaluation of an action as a means requires \textcolor<2>{fuchsia}{a supporting means-end relation} and \textcolor<3>{fuchsia}{evaluation of the end}.}
    \end{enumerate}
  }
  \hozlinedash
  \onslide<2>{
    \begin{block}{\principleName{Principle B: Relation}}
      An action is (in part) a means only if it is part of a supporting means-end relation.
    \end{block}
  }
  \onslide<3>{
    \begin{block}{\principleName{Principle C: Dependence}}
      Whether an action is worthwhile \emph{as a means} depends on whether the end is possible and worthwhile.
    \end{block}
  }
\end{frame}

\begin{frame}
  \frametitle{Means-end relation}

  \begin{block}{\principleName{Principle B: Relation}}
    An action is (in part) a means only if it is part of a supporting means-end relation.
  \end{block}
  \btVFill
  \vspace{-10pt}

  \begin{itemize}
  \item To understand an action as a means is to understand the action supports achieving something in some way.
  \item That something is the end.
  \item And, that the means achieves the end in some way establishes the relation.
  \end{itemize}
\btVFill
\end{frame}

\begin{frame}
  \frametitle{Means-end dependence}

  \begin{block}{\principleName{Principle C: Dependence}}
    Whether an action is worthwhile \emph{as a means} depends on whether the end is possible and worthwhile.
  \end{block}
  \btVFill
  \vspace{-10pt}
  \only<+>{
    There are two key parts this principle:
    \begin{enumerate}
    \item The evaluation of end as possible.
    \item The evaluation of end as worthwhile
    \end{enumerate}

    \begin{itemize}
    \item If the end is not possible, then the means is not worthwhile.
    \item If the end is not worthwhile, then the relevant action is not worthwhile as a means.
    \end{itemize}
  }
  \only<+>{

    \begin{quote}
      the value of the means derives from the value of the ends \dots
      If there are reasons to take the means, they must be none other than the reasons to pursue the ends, or at least they must derive from them.
    \end{quote}
    \citenote{\fullcite{Raz:2005aa}\newline\mbox{}}
  }
  \only<+>{
    \begin{quote}
      I may will the performance of certain actions as means of obtaining any desired good; but as my willing of these actions is only secondary, and founded on the supposition, that they are causes of the proposed effect; as soon as I discover the falshood of that supposition, they must become indifferent to me.
    \end{quote}
    \citenote{\fullcite{Hume:2011aa}}
  }
  \only<+->{
    \begin{enumerate}
    \item<+> By \principleName{Principle A: Settling}, if an agent evaluates an action as a means, the agent evaluates whether the action is worthwhile as a means.
    \item<+> By \principleName{Principle B: Relation}, understanding an action as a means requires understanding the action as part of a supporting means-end relation.
    \item<+> By \principleName{Principle C: Dependence}, an evaluation of an action as a means requires an evaluation of whether the end is possible and weather the end is worthwhile.
    \item<+> By \principleName{Principle A: Settling}, an evaluation of whether the end is possible and weather the end is worthwhile entails an evaluation of the end.
    \end{enumerate}

  }
  \btVFill
\end{frame}

\begin{frame}
  \frametitle{Positive argument}
  \tikzstyle{na} = [baseline=-.5ex]
    \setbeamertemplate{enumerate item}{P\arabic{enumi}.}
    \setbeamertemplate{enumerate subitem}{P\arabic{enumi}\alph{enumii}.}
    \begin{enumerate}
    \item A rational agent is permitted to settle on an action (in part) as a means only if
      \begin{enumerate}
      \item\tikz[na] \node[coordinate, xshift=-3em] (a1) {}; \textcolor{fuchsia}{\emph{the agent evaluates the action as a means}}, and
      \item\tikz[na] \node[coordinate, xshift=-3em] (b1) {}; \textcolor{fuchsia}{\emph{the evaluation supports settling on the action (as a means)}}.
      \end{enumerate}
    \item An evaluation of an action as a means requires a supporting means-end relation and evaluation of the end.
    \item A rational agent is permitted to settle on an action (in part) as a means only if
      \begin{enumerate}
      \item\tikz[na] \node[coordinate, xshift=-3em] (a2) {}; \textcolor{fuchsia}{\emph{the agent takes there to be a means-end relation supporting the means and}} \textcolor{fuchsia}{\emph{evaluates the end}}, and
      \item\tikz[na] \node[coordinate, xshift=-3em] (b2) {}; \textcolor{fuchsia}{\emph{the evaluation of the end supports settling on the action}} \textcolor{fuchsia}{\emph{as a means to the end}}.
      \end{enumerate}
    \item \schemaName{Taking} holds.
    \end{enumerate}

  \begin{tikzpicture}[overlay]
    \path[->] (a1) edge [very thick, draw=fuchsia, opacity=.5, bend right=75] (a2);
    \path[->] (b1) edge [very thick, draw=fuchsia, opacity=.5, bend right=75] (b2);
  \end{tikzpicture}
\end{frame}


\begin{frame}
  \frametitle{Premise P3}
  {%
    \setbeamertemplate{enumerate item}{P\arabic{enumi}.}
    \setbeamertemplate{enumerate subitem}{P\arabic{enumi}\alph{enumii}.}
    \begin{enumerate}
      \setcounter{enumi}{2}
    \item A rational agent is permitted to settle on an action (in part) as a means only if
      \begin{enumerate}
      \item \textcolor<2>{fuchsia}{\emph{the agent takes there to be a means-end relation supporting the means and evaluates the end}}, and
      \item \textcolor<3>{fuchsia}{\emph{the evaluation of the end supports settling on the action as a means to the end}}.
      \end{enumerate}
    \end{enumerate}
  }
  \hozlinedash
  \btVFill
  \only<1-3>{
    \begin{enumerate}
    \item \textcolor<2>{fuchsia}{In order for the agent to evaluate the action as a means the agent must take there to be a means-end relation supporting the means and evaluate the end}.
    \item \textcolor<3>{fuchsia}{The evaluation supports settling on the action as a means, then the evaluation supports settling on the action as a means to the end}.
    \end{enumerate}
  }
  \only<4-5>{
    \begin{itemize}
    \item The agent `has' the end in the sense that their evaluation of the end supports settling on the action (in part) as a means.
      \onslide<5>{
      \item[\hand] The evaluation of the end \emph{does work} in establishing actions as permissible.
      }
    \end{itemize}
  }
  \only<6>{
    {%\footnotesize
      \begin{block}{\schemaName{Taking}}
        A rational agent is permitted to settle on an action (in part) as a means
        \newline
        \mbox{ }\hfill\emph{only if}\hfill\mbox{ }
        \newline
        the agent \textcolor{fuchsia}{\emph{takes there to be}} a means-end relation from an end they have which supports taking the relevant means.
      \end{block}
    }
  }
  \btVFill
\end{frame}

\begin{frame}
  \frametitle{Positive argument (recap)}
    \tikzstyle{na} = [baseline=-.5ex]
    \setbeamertemplate{enumerate item}{P\arabic{enumi}.}
    \setbeamertemplate{enumerate subitem}{P\arabic{enumi}\alph{enumii}.}
    \begin{enumerate}
    \item A rational agent is permitted to settle on an action (in part) as a means only if
      \begin{enumerate}
      \item\tikz[na] \node[coordinate, xshift=-3em] (a1) {}; \emph{the agent evaluates the action as a means}, and
      \item\tikz[na] \node[coordinate, xshift=-3em] (b1) {}; \emph{the evaluation supports settling on the action (as a means)}.
      \end{enumerate}
    \item An evaluation of an action as a means requires a supporting means-end relation and evaluation of the end.
    \item A rational agent is permitted to settle on an action (in part) as a means only if
      \begin{enumerate}
      \item\tikz[na] \node[coordinate, xshift=-3em] (a2) {}; \emph{the agent takes there to be a means-end relation supporting the means and} \emph{evaluates the end}, and
      \item\tikz[na] \node[coordinate, xshift=-3em] (b2) {}; \emph{the evaluation of the end supports settling on the action} \emph{as a means to the end}.
      \end{enumerate}
    \item \schemaName{Taking} holds.
    \end{enumerate}

  \begin{tikzpicture}[overlay]
    \path[->] (a1) edge [very thick, draw=fuchsia, opacity=.5, bend right=75] (a2);
    \path[->] (b1) edge [very thick, draw=fuchsia, opacity=.5, bend right=75] (b2);
  \end{tikzpicture}
\end{frame}


\section{Summary}
\label{sec:summary}


\begin{frame}
  \frametitle{Summary}
  \begin{itemize}
  \item Goal has been to highlight that an agent taking there to be a means-end relation is always available as an explanatory resource, and to motivate the use of this resource by showing that a recognised resource fails in certain cases.
    \begin{itemize}
    \item There are cases in which a rational agent is unable to settle on a means by reason via a means-end relation.
    \item In all cases, if a rational agent settles on a means the agent must take there to be a supporting means-end relation.
    \end{itemize}
  \item Provides a way to understand Oblique.
  \item What use can \schemaName{Taking} be put to as an explanatory resource?
  \end{itemize}
  \hozlinedash
  \vspace{-15pt}
  \begin{itemize}
  \item Reasoning about means-end relations is important!
  \item Reasoning via a means-end relation allows an agent to `audit' the relation.
  \item The relevant `audit' doesn't need to occur when settling on an action (in part) as a means.
    % \begin{itemize}
    % \item It must be the case that the means supports achieving the end in some way.
    % \end{itemize}
    % \item Commitment to the end: If the end has a role in certain instances, then there's questions about whether it should also have a role in other instances of practical reasoning.
  \end{itemize}
\end{frame}

\section{Applications}
\label{sec:future-work}

\begin{frame}
  \frametitle{Applications}

  \begin{enumerate}
  \item Akraisa
  \item Ethics
  \item Shared ends
  \end{enumerate}
\end{frame}

\begin{frame}
  \frametitle{Akrasia}

  \begin{itemize}
  \item Acting against one's better judgement.
  \item Consider a case in which there is divergence between
    \begin{enumerate}[a.]
    \item what an agent is permitted to settle on given the means-end relations they are able to reason via, and
    \item what the agent is permitted to settle on given the means-end relations they take there to be.
    \end{enumerate}
  \end{itemize}
  \only<1>{
    \hozlinedash
    \vspace{-15pt}
    \begin{itemize}
    \item[]
    \item[]
    \item[]
    \item[]
    \end{itemize}
  }
  \only<2>{
    \hozlinedash
    \vspace{-15pt}
    \begin{itemize}
    \item A second glass of wine and abstaining.
    \item Able to reason from end of enjoying meal to a second glass.
    \item Unable to reason from end of after-dinner work to abstaining.
    \item Perhaps the agent doubts the quality of their means-end reasoning?
    \end{itemize}

    {
      \tikz[remember picture,overlay]
      \draw (current page.south east) +(-.5\linewidth, 2.2em)
      node[anchor=south west,inner sep=0pt, opacity=0.25]{\def\svgwidth{.5\linewidth}
        \includeinkscape{images/wine}};
    }
    \citenote{\fullcite{Bratman:2007ab}}
  }
\end{frame}

\begin{frame}
  \frametitle{Ethics}
  \only<1>{
    \begin{itemize}
    \item What is required for a rational agent to be permitted to settle on an action (in part) as a means?
    \item Are there cases in which an agent taking there to be a means-end relation is an important/useful resource?
    \item Cases in which agents settle on doing the right thing by a taking relation there to be a means-end relation from an end they have, but are unable to reason from the end to the means.
      \begin{itemize}
      \item \citeauthor{Arpaly:2003aa}'s reading of Huckleberry Finn.
      \item \citeauthor{Fricker:2007aa}'s notion of a hermeneutical lacuna.
      \end{itemize}
    \end{itemize}
  }
  \only<2>{
    \begin{itemize}
    \item Cases in which an agent is pressured to reason via means-end relations while the resources that would allow the agent to perform the reasoning are sabotaged.
      \begin{itemize}
        % \item Pettigrew: agents lack an important concept or a body of evidence that would allow the agent to internally justify beliefs.
        % \item Agent has a lose grasp of a concept, and second guesses their ability to apply it to reach a conclusion.
        % \item ``the efficacy of [gaslighting] relies on the targets responding to their evidence by forming or retaining only those beliefs that are internally justified''
      \item Gaslighting?
      % \item Agent lacks resources that would allow the agent to reason from an end to a means.
      % \item Prevent agent from settling on a means by requiring an agent to reason from an end to means in order for the means to be permissible.
      \item \emph{Pat and Mike} (1952)
        \begin{itemize}
        \item Pat is an aspiring golfer.
        \item Collier wants Pat to give up their golf career for marriage.
        \item Pat loses a tournament and has a crisis of confidence.
        \item Collier refuses to support Pat, and persistently suggests doubt in Pat's ability to make judgements.
        \end{itemize}
        \item Among other issues, the gaslighter pressuring for the agent to reason via means-end relations may be unwarranted.
      \end{itemize}
    \end{itemize}
    {
      % @JANENORTHCOTE
      \tikz[remember picture,overlay]
      \draw (current page.south east) +(-.35\linewidth, 1em) %
      node[anchor=south west,inner sep=0pt, opacity=0.25]{\def\svgwidth{.3\linewidth}
        \includeinkscape{images/gas-c}};
    }
    {
      \tikz[remember picture,overlay, every node/.style={scale=0.6}]
      \draw (current page.south east) +(-.1em, .75em) node[anchor=south east,inner sep=0pt, opacity=0.3] {\tiny \textcolor{fuchsia}{@JANENORTHCOTE}};
    }
    \citenote{\fullcite{Spear:2020aa}}
  }
\end{frame}


\begin{frame}
  \frametitle{Shared ends}
  \begin{itemize}
  \item Rat and Mole have the shared end of reaching the shore.
  \item Rat is rowing and Mole is the coxswain.
    \begin{itemize}
    \item Mole can see obstacles, but cannot control the boat.
    \item Rat cannot see obstacles, but can control the boat.
    \end{itemize}
  \item Mole observes and obstacle, and tells Rat to turn.
  \item Rat cannot observe the obstacle.
  \item Given Mole's instruction, Rat takes there to be a means-end relation that supports turning.
  \item Perhaps Rat and Mole's ends are not aligned, and Rat turns the other direction!
  \end{itemize}

  {
    \tikz[remember picture,overlay]
    \draw (current page.south east) +(-.5\linewidth, 0em)
    node[anchor=south west,inner sep=0pt, opacity=0.25]{\def\svgwidth{.5\linewidth}
      \includeinkscape{images/witw}};
  }
\end{frame}

% \subsection{Additional scenarios}
% \label{sec:final-case}

\begin{frame}
  \frametitle{Multiple agents}
  \vspace{-40pt}
  {\rmfamily
    Outside his house he found Piglet, jumping up and down trying to reach the knocker.\newline
    ``Hallo, Piglet,'' he said.\newline
    ``Hallo, Pooh,'' said Piglet.\newline
    ``What are \emph{you} trying to do?''\newline
    ``I was trying to reach the knocker,'' said Piglet. ``I just came round---''\newline
    ``Let me do it for you,'' said Pooh kindly.\newline
    So he reached up and knocked at the door.
  }
  {
    \tikz[remember picture,overlay]
    \draw (current page.south west) +(5em, 2em)
    node[anchor=south west,inner sep=0pt, opacity=0.25]{
      \def\svgwidth{.9\linewidth}
      \includeinkscape{images/pandpf}};
  }
  \citenote{\fullcite{Milne:2009aa}}
\end{frame}


\appendix
\setbeamertemplate{navigation symbols}{}

\section*{Thanks!}
{
  \setbeamertemplate{navigation symbols}{@aries_bug}
\begin{frame}[noframenumbering]
  \frametitle{}
  {
    \tikz[remember picture,overlay]
    \draw (current page.south west) +(.35\linewidth, .5)
    node[anchor=south west,inner sep=0pt, opacity=0.5]{\def\svgwidth{.5\linewidth}
      \includeinkscape{images/pika}};
  }
\end{frame}
}

% \begin{frame}[noframenumbering]
%   \frametitle{}
%   {
%     \tikz[remember picture,overlay]
%     \draw (current page.south west) +(-1.4, -.75)
%     node[anchor=south west,inner sep=0pt, opacity=0.5]{\def\svgwidth{1.5\linewidth}
%       \includeinkscape{images/e-c}};
%   }
% \end{frame}

% Remove structure slides between sections
\AtBeginSection[]{}

\begin{frame}[noframenumbering]
  \frametitle{References}
  {
    \renewcommand*{\bibfont}{\scriptsize}
    \printbibliography
  }
\end{frame}

% \begin{frame}
%   \frametitle{More cases}

%   Persistent objects:
% \begin{itemize}
% \item I drink Yakult because I had \emph{something} in mind when I bought a months worth.
% \item I still won't open some bottle of whisky because I'm saving it for some purpose.
% \item You see a gift for a friend in a store but have to go to a meeting, and after you end up searching the store for the gift.
%   \begin{itemize}
%   \item \dots and sometimes you might come across the gift again but are sure that you found something better.
%   \end{itemize}
% \end{itemize}
% \end{frame}


\begin{frame}
  \frametitle{Supermarket}

  {\rmfamily

    Snow dances and random flicker turns to rigid form as the agent regains focus.

    The supermarket aisle is empty, and their left hand is stretched toward an item as if ready to impart life.

    \emph{Crikey}!

    Forgetting the shopping list, and now forgetting consciousness.

    What were they doing?

    Their position suggests they were about to obtain the item.

    No basket. So, perhaps this is all they came in for.

    But what would the item be for? What is the item a means to?

  }
\end{frame}

\begin{frame}
  \frametitle{Nixon}

  {\rmfamily
    Static monitors shimmer and enamelled wood creaks as Nixon recovers consciousness.

    A big red button is inside an open suitcase pierced with keys.

    Safety protocols have been followed.

    Nixon presses down and an electrical signal \dots
  }

  % \pause
  % \hozlinedash

  % \begin{itemize}
  % \item<+-> Did Nixon evaluate a nuclear strike as worthwhile?
  %   \begin{itemize}
  %   \item<+-> Settled on the means, so by settling principle Nixon took this to be worthwhile.
  %   \item<+-> And, this commits Nixon to the evaluation of the end as worthwhile.
  %   \item<+-> Nixon, however, was unable to recall that the end was a nuclear strike!
  %     \begin{itemize}
  %     \item<+-> Different representations of the same outcome.
  %     \end{itemize}
  %   \end{itemize}
  % \end{itemize}
\end{frame}


\begin{frame}
  \frametitle{Ethics}
  \begin{quote}
    Huckleberry constantly perceives data (never deliberated upon) that amount to the message that Jim is a person, just like him.\newline
    \dots\newline
    While Huckleberry never reflects on these facts, they do prompt him to act toward Jim, more and more, in the same way he would have acted toward any other friend.\newline
    \dots\newline
    Huckleberry Finn \dots does not have the belief that what he does is right anywhere in his head—this moral insight is exactly what eludes him.
  \end{quote}
  {
    \tikz[remember picture,overlay]
    \draw (current page.south east) +(-.5\linewidth, 2em)
    node[anchor=south west,inner sep=0pt, opacity=0.25]{\def\svgwidth{.5\linewidth}
      \includeinkscape{images/huckleberry}};
  }
  \citenote{\fullcite{Arpaly:2003aa}}
\end{frame}

% \section{Relating the principles}

% \begin{frame}
%   \frametitle{Relating the principles}

%   \begin{block}{Taking (by reasoning)}
%     A rational agent is permitted to settle on an action (in part) as a means
%     \newline
%     \mbox{ }\hfill\emph{only if}\hfill\mbox{ }
%     \newline
%     the agent \emph{takes there to be} a means-end relation from an end they have which supports taking the relevant means \emph{(by reasoning via the relation)}.
%   \end{block}

%   \begin{enumerate}
%   \item An entailment is not required for the argument.
%   \item Requiring an entailment would not help the argument.
%     \begin{itemize}
%     \item \schemaName{Taking} would hold in all the cases in which \schemaName{Reasoning} does, but this would say nothing about the cases in which \schemaName{Reasoning} fails.
%     \end{itemize}
%   \item There may be good reason to \schemaName{Taking} and \schemaName{Reasoning} which would break the entailment.
%   \end{enumerate}
% \end{frame}

% \section{Types of cases}

% \begin{frame}
%   \frametitle{Types of cases (aside)}

%   Professor Oblique is reasoning about an end, and considers a means (initially to some end) on the basis of a persistent object, but is unable to reason from the means to the end.
%   \begin{itemize}
%   \item Three parameters to vary:
%     \begin{enumerate}
%     \item Whether the agent is able to reason about the end.
%     \item Whether the agent is able to reason about the means-end relation.
%     \item Whether the agent is able to recall information about the end/means/means-end relation `internally' or `externally'
%     \end{enumerate}
%   \item Example questions:
%     \begin{itemize}
%     \item Are there cases in which an agent recognises a means, but is unable to recall what the means is an end to, but would easily be able to reason from the end to the means were they able to recall the end?
%     \item Could an agent rationally settle on a means if they have neither an end nor a candidate means-end relation?
%     \end{itemize}
%   \end{itemize}
% \end{frame}

% \section{Akrasia}
% \label{sec:akrasia}

% \begin{frame}
%   \frametitle{Akrasia}

%   {\rmfamily
%     After dinner work and enjoyable meals.

%     A single glass of wine makes the meal more enjoyable and does not obstruct later work.

%     However, after the first glass, you can only make sense of having another.

%     Any lost productivity can be made up for elsewhere, but this dinner must end at some point \dots
%   }

%   \hozlinedash

% \end{frame}

% \begin{frame}
%   \frametitle{An instance of \schemaName{Reasoning}}

% The `only if' direction of \citeauthor{Sinhababu:2017aa}'s \emph{Desire–Belief Theory of Reasoning} may be seen as an instance of \schemaName{Reasoning}:
% \vspace{10pt}
% \begin{quote}
%   The Desire–Belief Theory of Reasoning: Desire is affected as the conclusion of reasoning if and only if desire that [end] is combined with belief that [means] would raise [end]'s probability, constituting desire that [means].
% \end{quote}
% \vspace{10pt}
% Here \citeauthor{Sinhababu:2017aa} takes A, E, and M to suggest ``action'', ``end'', and ``means''.
% % The relevant missing premise is that a rational agent is permitted to settle on an action only if the agent desires the action.

% \citenote{\fullcite{Sinhababu:2017aa}}
% \end{frame}

\end{document}