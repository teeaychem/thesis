\documentclass[noamssymb,
graphics,
% compress,
% handout,
]{beamer} % amssymb is incompatible with mtpro2

\usetheme{Antibes}
\usecolortheme{fuchsia}

\setbeamertemplate{navigation symbols}{}
\setbeamertemplate{enumerate items}[default]

\usepackage[T1]{fontenc}
   % \usepackage{textcomp}
   % \usepackage{newtxtext}
   % \renewcommand\rmdefault{Pym} %\usepackage{mathptmx} %\usepackage{times}
   % \usepackage{bm}% Access to bold math symbols
   \usepackage[no-math]{fontspec}
   \defaultfontfeatures{Ligatures=TeX,Numbers={Proportional}}
   \newfontfeature{Microtype}{protrusion=default;expansion=default;}
   \setmainfont[Ligatures=TeX,Scale=MatchLowercase]{Source Serif Pro}
   \setsansfont[Ligatures=TeX,Scale=MatchLowercase,BoldFont={* Semibold}]{Source Sans Pro}
   \setmonofont[Scale=MatchLowercase]{Source Code Pro}
   % \usepackage{selnolig}% For suppressing certain typographic ligatures automatically
   % \usepackage{microtype}

   % % mtpro2 fix from:
   % https://tex.stackexchange.com/questions/374353/no-room-for-a-new-count
   \makeatletter
   \let\alloc@latex\alloc@
   \def\alloc@#1#2#3#4{\newcount}% the current \newcount doesn't use \alloc@
   \makeatother
   % % %
   \usepackage[complete, subscriptcorrection, slantedGreek, mtpfrak, mtpbb, mtpcal]{mtpro2}
% % % % % % %
\usepackage{amsthm}         % (in part) For the defined environments
\usepackage{mathtools}      % Improves  on amsmaths/mtpro2


% % % The bibliography % % %
\usepackage[backend=biber,
  style=authoryear-comp,
  bibstyle=authoryear,
  citestyle=authoryear-comp,
  uniquename=false,%allinit,
  giveninits=true,
  backref=false,
  % hyperref=true,
  url=false,
  isbn=false,
  useprefix=true,
  ]{biblatex}
\DeclareFieldFormat{postnote}{#1}
\DeclareFieldFormat{multipostnote}{#1}
% \setlength\bibitemsep{1.5\itemsep}
\newcommand{\noopsort}[1]{}
\addbibresource{Thesis.bib}
% % % % % % % % % % % % % % %

% \usepackage[inline]{enumitem}
% \setlist[itemize]{noitemsep}
% \setlist[description]{style=unboxed,leftmargin=\parindent,labelindent=\parindent,font=\normalfont\space}
% \setlist[enumerate]{noitemsep}


\usepackage{pifont}
\newcommand{\hand}{\ding{43}}
\usepackage{array}

\usepackage{setspace}

\newcommand{\schemaName}[1]{\textsf{#1}}
\newcommand{\principleName}[1]{\textsf{#1}}
\newcommand{\dependencePrinciple}[0]{\textsf{Dependence Principle}}
% % %
\usepackage{dashrule}
\newcommand{\hozline}[0]{
  \noindent\hdashrule[0.5ex][c]{\textwidth}{.1pt}{}
}
\newcommand{\hozlinedash}[0]{
  \noindent\hdashrule[0.5ex][c]{\textwidth}{.1pt}{2.5pt}
}
\newcommand{\btVFill}{\vskip0pt plus 1filll}
\newcommand{\citeline}[0]{\btVFill\vspace{-8pt}\noindent\hdashrule[0.5ex][c]{.25\textwidth}{.1pt}{}\vspace{-8pt}}
%https://tex.stackexchange.com/questions/347228/footnote-without-numbering-and-without-indention-in-beamer-class
\newcommand\citenote[1]{%
  \tikz[remember picture,overlay]
  \draw (current page.south west) +(1in + \oddsidemargin,0.5em)
  node[anchor=south west,inner sep=0pt]{\parbox{\textwidth}{%
      \rlap{\rule{10em}{0.4pt}}\raggedright\scriptsize#1}};
}
% % %

\usepackage{tikz}
\usepackage{svg}

% % % Color % % %
% \usepackage[usenames, dvipsnames]{xcolor}
% \usepackage{everysel}
\definecolor{fuchsia}{HTML}{FE4164}%Neon Fuchsia %{F535AA}%Neon Pink
\usecolortheme[named=fuchsia]{structure}
\colorlet{structure}{fuchsia}
\usestructuretemplate{\color{structure}}{}
% % % % % % % % % %

% Set quotes to non-italic
\setbeamerfont{quote}{shape=\upshape,family=\rmfamily}

\addtobeamertemplate{navigation symbols}{}{%
    \usebeamerfont{footline}%
    \usebeamercolor[fg]{footline}%
    \hspace{1em}%
    \insertframenumber/\insertmainframenumber
}

\setbeamercovered{transparent}

% https://tex.stackexchange.com/questions/28654/beamer-table-of-contents-display-all-subsections-below-section
\AtBeginSection[]
{
\begin{frame}[noframenumbering]
  \frametitle{Structure}
  \setbeamertemplate{navigation symbols}{ }
  \tableofcontents[
  currentsection,
  % hideothersubsections,
  % sectionstyle=show/hide,
  % subsectionstyle=show/shaded,
  ]
\end{frame}
}



\title{Means-end reasoning and means-end relations}
\author{Ben Sparkes}
% \institute{Stanford}
% \date{}

\begin{document}

\begin{frame}[noframenumbering]
  \setbeamertemplate{navigation symbols}{ }
  \titlepage
\end{frame}


\section*{Overview}
\label{sec:overview}

\begin{frame}
  \frametitle{Overview}

  \begin{itemize}
  \item Argue against the idea that settling on means requires an agent reasons from ends to means.
  \item Argue for the idea that settling on means requires an agent to take a means-end relation to exist from an end they have.
  \end{itemize}
\end{frame}


\section{Means-end schema}
\label{sec:schema}

\begin{frame}
  \frametitle{The means-end schema}

  \begin{block}<1->{The Schema}
    \textcolor<3>{fuchsia}{A rational agent} is \textcolor<4>{fuchsia}{permitted to settle on an action (in part) as a means}
    \newline
    \mbox{ }\hfill\emph{only if}\hfill\mbox{ }
    \newline
    \textcolor<3>{fuchsia}{The agent} [\textcolor<6>{fuchsia}{does something with}] \textcolor<5>{fuchsia}{a means-end relation from an end they have which supports taking the relevant means}.
  \end{block}

  \onslide<2-5>{Four parts to this schema:}
  \begin{enumerate}%[label=\alph*., noitemsep]
  \item<3-> \textcolor<3>{fuchsia}{A rational agent}
  \item<4-> \textcolor<4>{fuchsia}{The agent being permitted to settle on an action (in part) as a means.}
  \item<5-> \textcolor<5>{fuchsia}{A means-end relation from an end the agent has which supports taking the means.}
  \item<6-> \textcolor<6->{fuchsia}{\emph<7>{A link between the agent has to the specified means-end relation.}}
  \end{enumerate}
\end{frame}

\begin{frame}
  \frametitle{Example}

  \begin{quote}
    You might reason like this:\newline
    \mbox{}\quad \textcolor<4>{fuchsia}{I am going to buy a boat} \hfill (1a)\newline
    and\newline
    \mbox{}\quad \textcolor<5>{fuchsia}{For me to buy a boat, a necessary means is to borrow money} \hfill (1b)\newline
    so\newline
    \mbox{}\quad \textcolor<3>{fuchsia}{I shall borrow money.} \hfill (1c)
  \end{quote}
  \hozlinedash
  \onslide<2->{\vspace{-10pt}
    \begin{block}{The Schema}
      A rational agent is \textcolor<3>{fuchsia}{permitted to settle on an action (in part) as a means}
      \newline
      \mbox{ }\hfill\emph{only if}\hfill\mbox{ }
      \newline
      the agent
      \only<-5>{[does something with]}
      \only<6->{\textcolor{fuchsia}{reasons via}} \textcolor<5>{fuchsia}{a means-end relation from} \textcolor<4-5>{fuchsia}{an end they have} \textcolor<5>{fuchsia}{which supports taking the relevant means}.
    \end{block}

    \citenote{\citeauthor{Broome:2002aa} (\citeyear{Broome:2002aa}) \citetitle{Broome:2002aa}}
  }
\end{frame}

\begin{frame}
  \frametitle{Reasoning}

  \begin{block}{Reasoning}
    A rational agent is permitted to settle on an action (in part) as a means
    \newline
    \mbox{ }\hfill\emph{only if}\hfill\mbox{ }
    \newline
    the agent \textcolor{fuchsia}{\emph{reasons via}} a means-end relation from an end they have which supports taking the relevant means \emph{by reasoning via the relation}.
  \end{block}

  \begin{itemize}
  \item Agent is required to connect means to end.
  \item Many cases of reasoning satisfy this requirment.
    \begin{itemize}
    \item I have the end of waking up, and I combine all sorts of information I have about myself and coffee to settle on drinking a cup.
    \end{itemize}
  \end{itemize}
\end{frame}


\begin{frame}
  \frametitle{Example}

  \only<1>{
  \begin{quote}
    I am going to buy a boat \hfill (1a)\newline
    For me to buy a boat, a necessary means is to borrow money \hfill (1b)\newline
    I shall borrow money. \hfill (1c)
  \end{quote}
  \hozlinedash
}
\only<2>{
  \vspace{-28pt}
  \begin{block}{Reasoning}
    A rational agent is permitted to settle on an action (in part) as a means
    \newline
    \mbox{ }\hfill\emph{only if}\hfill\mbox{ }
    \newline
    the agent \textcolor{fuchsia}{\emph{reasons via}} a means-end relation from an end they have which supports taking the relevant means \emph{by reasoning via the relation}.
  \end{block}
  \hozlinedash
}

  \begin{quote}
    Your reasoning process is a particular type of practical reasoning.
    It \textcolor<2>{fuchsia}{\emph<2>{is}} \emph<1>{instrumental} reasoning, which means it is concerned with taking an appropriate means to an end.%\nolinebreak
    % \mbox{}\hfill\mbox{(\citeyear[86]{Broome:2002aa})}
  \end{quote}
\citenote{\citeauthor{Broome:2002aa} (\citeyear{Broome:2002aa}) \citetitle{Broome:2002aa}}

\end{frame}

\begin{frame}
  \frametitle{\schemaName{Taking}}

  \begin{block}{Taking}
    A rational agent is permitted to settle on an action (in part) as a means
    \newline
    \mbox{ }\hfill\emph{only if}\hfill\mbox{ }
    \newline
    the agent \textcolor{fuchsia}{\emph{takes there to be}} a means-end relation from an end they have which supports taking the relevant means.
  \end{block}

  \only<2>{
    \begin{itemize}
    \item An existential-like requirement that does not require the agent to reason to the means from an end.
    \item Some cases of reasoning may satisfy this requirement.
      \begin{itemize}
      \item At some point I purchased a collection of Ligeti recordings, and I have no idea why, but it wasn't an acident and so I will listen to them.
      \end{itemize}
    \end{itemize}
  }
  \only<3>{
    \vspace{-10pt}
    \begin{exampleblock}{}
      \begin{itemize}
        \setlength\itemsep{-.1em}
      \item On your desk is a stack of papers.
      \item You recognise that you created the stack for some end, but you cannot recall what that end is.
      \item You are confident that the end remains relevant to you, and you begin to work through the stack.
      \item As you work through the stack, you begin to recall the research project that led you to forming the stack.
      \end{itemize}
    \end{exampleblock}
  }
\end{frame}


\begin{frame}
  \frametitle{I will argue \emph{for}:}

  \begin{block}{Taking}
    A rational agent is permitted to settle on an action (in part) as a means
    \newline
    \mbox{ }\hfill\emph{only if}\hfill\mbox{ }
    \newline
    the agent \textcolor{fuchsia}{\emph{takes there to be}} a means-end relation from an end they have which supports taking the relevant means.
  \end{block}
\end{frame}


\begin{frame}
  \frametitle{I will argue \emph{against}:}

  \begin{block}{Reasoning}
    A rational agent is permitted to settle on an action (in part) as a means
    \newline
    \mbox{ }\hfill\emph{only if}\hfill\mbox{ }
    \newline
    the agent \textcolor{fuchsia}{\emph{reasons via}} a means-end relation from an end they have which supports taking the relevant means.
  \end{block}
\end{frame}

\begin{frame}
  \frametitle{No entailment between the principles}

  \begin{block}{Reasoning}
    A rational agent is permitted to settle on an action (in part) as a means
    \newline
    \mbox{ }\hfill\emph{only if}\hfill\mbox{ }
    \newline
    the agent \textcolor{fuchsia}{\emph{reasons via}}  a means-end relation from an end they have which supports taking the relevant means.
  \end{block}

  {\Large \mbox{ }\hfill \(\textcolor{fuchsia}{\not}\hspace{-4pt}\Uparrow\) \qquad \(\textcolor{fuchsia}{\not}\hspace{-4pt}\Downarrow\) \hfill\mbox{ }}

  \begin{block}{Taking}
    A rational agent is permitted to settle on an action (in part) as a means
    \newline
    \mbox{ }\hfill\emph{only if}\hfill\mbox{ }
    \newline
    the agent \textcolor{fuchsia}{\emph{takes there to be}} a means-end relation from an end they have which supports taking the relevant means.
  \end{block}

\end{frame}


\begin{frame}
  \frametitle{Possible entailment from \schemaName{Reasoning} to \schemaName{Taking}}

  \begin{block}{(Taking by) Reasoning}
   A rational agent is permitted to settle on an action (in part) as a means
    \newline
    \mbox{ }\hfill\emph{only if}\hfill\mbox{ }
    \newline
    the agent \textcolor{fuchsia}{\emph{takes there to be}}  a means-end relation from an end they have which supports taking the relevant means \textcolor{fuchsia}{\emph{(by reasoning via the relation)}}.
  \end{block}

  {\Large \mbox{ }\hfill\(\Downarrow\)\hfill\mbox{ }}

  \begin{block}{Taking}
    A rational agent is permitted to settle on an action (in part) as a means
    \newline
    \mbox{ }\hfill\emph{only if}\hfill\mbox{ }
    \newline
    the agent \textcolor{fuchsia}{\emph{takes there to be}} a means-end relation from an end they have which supports taking the relevant means.
  \end{block}

\end{frame}


\section{Argument}
\label{sec:argument}


\subsection{Negative part (against \schemaName{Reasoning})}
\label{sec:negative}

\begin{frame}
  \frametitle{Negative argument}

  {%
  \setbeamertemplate{enumerate item}{N\arabic{enumi}.}
  \begin{enumerate}
  \item\label{scenarios:exist} There are cases in which agents recognise a means (as means) without being able to reason from an end they have to those means.
  \item\label{scenarios:persmissible} And, in these cases the agent is rational in settling on the means.

  \item \schemaName{Reasoning} does not hold.
    % \item[C\(_{\text{I}}\).]\label{scenario:no-reasoning} In order for an agent to be rationally permitted in settling on a means it cannot be required that the agent reasons via a means-end relation from an end they have which supports taking the relevant means.
    \begin{itemize}
    \item From \ref{scenarios:exist} and~\ref{scenarios:persmissible}.
    \end{itemize}
  \end{enumerate}
  }

  \hozlinedash
  {\footnotesize
    \begin{block}{Reasoning}
      A rational agent is permitted to settle on an action (in part) as a means
      \newline
      \mbox{ }\hfill\emph{only if}\hfill\mbox{ }
      \newline
      the agent \textcolor{fuchsia}{\emph{reasons via}}  a means-end relation from an end they have which supports taking the relevant means.
    \end{block}
  }
\end{frame}

\subsubsection{Counterexample}
\label{sec:counterexample}

\begin{frame}
  \frametitle{Oblique (counterexample to \schemaName{Reasoning})}

  {\rmfamily
    % \begin{spacing}{1.3}
    \begin{itemize}
      \setbeamertemplate{itemize items}{}
      \item Oblique regains consciousness after a moment of darkness.%\newline
      \item They are in a supermarket, and their hand is stretched toward an item on a shelf.%\newline
      \item Oblique is unable to recall why they are in the supermarket and is unable to recognise what the item on the shelf would be for.%\newline
      \item Oblique does not have a shopping list.%\newline
      \item However, Oblique's outstretched hand indicates to them that they were about to purchase the item, and so they settle on doing so.
      \end{itemize}
    % \end{spacing}
  }
\end{frame}

\begin{frame}
  \frametitle{Oblique settled without satisfying \schemaName{Reasoning}}

  \begin{itemize}
  \item Oblique settles on purchasing the item as a means to some end.
  \item Oblique's outstretched hand provides information that purchasing the item was something they settled on.
  \item Oblique is unable to reason via a means-end relation from an end they have which supports purchasing the item.
    \begin{itemize}
    \item Obliqe is unable to reason about the relevant end.
    \item Without the relevant end Oblique is unable to reason about means-end relations from the end to the means.
    \end{itemize}
  \end{itemize}

\end{frame}


\begin{frame}
  \frametitle{Oblique is rationally permitted to settle on the means}

  \begin{itemize}
  \item Oblique's outstretched hand provides information that purchasing the item was something they settled on.
  \item Two arguments:
    \begin{enumerate}
    \item Purchasing the item is supported by a means-end relation.
      \begin{itemize}
      \item If Oblique settled on the purchasing the item, this was supported by a means-end relation and some end.
      \item So, purchasing the item was supported by a means-end relation and some end Oblique had.
      \item And, as Oblique only lost consciousness for a moment they reason that the relevant means-end relation and end continue to hold.
      \end{itemize}
    \item It seems asburd to send Oblique all the way home to look at the shopping list they left on their fridge.
    \end{enumerate}
  \end{itemize}

\end{frame}

\begin{frame}
  \frametitle{Key features of the counterexample}

  \begin{enumerate}[A)]
  \item The agent has enough information to determine that an act would be worthwhile as a means to some end.
  \item The agent does not have sufficient information to determine how a means-end relation from the end would support the means.
  \item The agent has sufficient information to determine that a supporting means-end relation from the end they have holds.
  \end{enumerate}

  \hozlinedash

  \begin{itemize}
  \item That Oblique cannot recall the end is not an essential feature, but does a significant amount of work.
  \end{itemize}
\end{frame}

\begin{frame}
  \frametitle{Negative argument (recap)}

  {%
  \setbeamertemplate{enumerate item}{N\arabic{enumi}.}
  \begin{enumerate}
  \item\label{scenarios:exist} There are cases in which agents recognise a means (as means) without being able to reason from an end they have to those means.
  \item\label{scenarios:persmissible} And, in these cases the agent is rational in settling on the means.

  \item \schemaName{Reasoning} does not hold.
    % \item[C\(_{\text{I}}\).]\label{scenario:no-reasoning} In order for an agent to be rationally permitted in settling on a means it cannot be required that the agent reasons via a means-end relation from an end they have which supports taking the relevant means.
    \begin{itemize}
    \item From \ref{scenarios:exist} and~\ref{scenarios:persmissible}.
    \end{itemize}
  \end{enumerate}
  }

  \hozlinedash
  {\footnotesize
    \begin{block}{Reasoning}
      A rational agent is permitted to settle on an action (in part) as a means
      \newline
      \mbox{ }\hfill\emph{only if}\hfill\mbox{ }
      \newline
      the agent \textcolor{fuchsia}{\emph{reasons via}}  a means-end relation from an end they have which supports taking the relevant means.
    \end{block}
  }
  \begin{itemize}
  \item<2>[\hand] Oblique settled on taking the item as a means to some end without being able to recall the end, and hence without being about to reason via a means-end relation from the end to taking the item.
  \end{itemize}
\end{frame}

\subsection{Positive part (for \schemaName{Taking})}
\label{sec:positive}

\begin{frame}
  \frametitle{I now will argue \emph{for}:}

  \begin{block}{Taking}
    A rational agent is permitted to settle on an action (in part) as a means
    \newline
    \mbox{ }\hfill\emph{only if}\hfill\mbox{ }
    \newline
    the agent \textcolor{fuchsia}{\emph{takes there to be}} a means-end relation from an end they have which supports taking the relevant means.
  \end{block}
\end{frame}

\subsubsection{Two principles}
\label{sec:two-principles}

\begin{frame}
  \frametitle{Positive Argument}
  \only<1>{

    {%
      \setbeamertemplate{enumerate item}{P\arabic{enumi}.}
      \setbeamertemplate{enumerate subitem}{P\arabic{enumi}\alph{enumii}.}
      \begin{enumerate}
      \item A rational agent is permitted to settle on an action (in part) as a means only if
        \begin{enumerate}
        \item \emph{the agent evaluates the action as a means}, and
        \item \emph{the evaluation supports settling on the action as a means}.
        \end{enumerate}
      \item \emph{An evaluation of an action as a means requires a supporting means-end relation and evaluation of the end.}
      \item A rational agent is permitted to settle on an action (in part) as a means only if
        \begin{enumerate}
        \item \emph{the agent takes there to be a means-end relation supporting the means and evaluates the end}, and
        \item \emph{the evaluation of the end supports settling on the action as a means to the end}.
        \end{enumerate}
      \end{enumerate}
    }
  }
  \only<2>{
    {%
      \setbeamertemplate{enumerate item}{P\arabic{enumi}.}
      \setbeamertemplate{enumerate subitem}{P\arabic{enumi}\alph{enumii}.}
      \begin{enumerate}
        \setcounter{enumi}{2}
      \item A rational agent is permitted to settle on an action (in part) as a means only if
        \begin{enumerate}
        \item \emph{the agent takes there to be a means-end relation supporting the means and evaluates the end}, and
        \item \emph{the evaluation of the end supports settling on the action as a means to the end}.
        \end{enumerate}
      \end{enumerate}
    }
    \hozlinedash
    {%\footnotesize
      \begin{block}{Taking}
        A rational agent is permitted to settle on an action (in part) as a means
        \newline
        \mbox{ }\hfill\emph{only if}\hfill\mbox{ }
        \newline
        the agent \textcolor{fuchsia}{\emph{takes there to be}}  a means-end relation from an end they have which supports taking the relevant means.
      \end{block}
    }
  }

\end{frame}

\begin{frame}
  \frametitle{Evaluation}
  \begin{block}{Principle: Evlauation}
    An agent evaluates an action if and only if the agent evaluates
    \begin{enumerate}
    \item whether the action is possible, and \hfill \only<2->{(\(\approx\)\emph{mind-to-world})}
    \item whether the action is worthwhile. \hfill \only<2->{(\(\approx\)\emph{world-to-mind})}
    \end{enumerate}
  \end{block}
  \btVFill
  \only<3->{
    The `if and only if' of principle \principleName{Evaluation} does not exclude other considerations contributing to an agent's evaluation of an action, but these are not required.

    \begin{itemize}
    \item[\hand] This does not commit the agent to evaluating the action \emph{as} possible and worthwhile.
    \end{itemize}
  }
  \btVFill
\end{frame}

\begin{frame}
  \frametitle{Premise P1}

  {%
    \setbeamertemplate{enumerate item}{P\arabic{enumi}.}
    \setbeamertemplate{enumerate subitem}{P\arabic{enumi}\alph{enumii}.}
    \begin{enumerate}
    \item A rational agent is permitted to settle on an action (in part) as a means only if
      \begin{enumerate}
      \item \emph{the agent evaluates the action as a means}, and
      \item \emph{the evaluation supports settling on the action as a means}.
      \end{enumerate}
    \end{enumerate}
  }
  \hozlinedash
  \btVFill
  \only<2>{
    \begin{itemize}
    \item The role of the action as a means \emph{does some work} in establishing that settling on the action is permissible.
    \item The role of the action as a means counts in favour of the action; it positively contributes to settling the issue of acting.
      \begin{itemize}
      \item Consider Oblique in the supermarket, if they did not consider the action as a means, they would not settle on purchasing the item.
      \end{itemize}
    \end{itemize}
  }
  \only<3>{
    \begin{example}
      I reason that settling on listing to Taeko Ohnuki's \emph{Sunshower} in part as a means to relaxing to be permissible.
      Relaxing then does some work in my reasoning, even if I would have reasoned that listing to \emph{Sunshower} would be permissible based on how nice it sounds.
      If I reasoned that listing to \emph{Sunshower} would be permissible based only on how nice it sounds, then I would not have settling on the action in part as a means to relaxing.
    \end{example}
  }
  \btVFill
\end{frame}

\begin{frame}
  \frametitle{Premise P2}
  {%
    \setbeamertemplate{enumerate item}{P\arabic{enumi}.}
    \setbeamertemplate{enumerate subitem}{P\arabic{enumi}\alph{enumii}.}
    \begin{enumerate}
      \setcounter{enumi}{1}
    \item \emph{An evaluation of an action as a means requires a supporting means-end relation and evaluation of the end.}
    \end{enumerate}
  }
  \hozlinedash

  \begin{block}{Principle: Means-end relation}
    An action is (in part) a means only if it is part of a supporting means-end relation.
  \end{block}

  \begin{block}{Principle: Means-end dependence}
    Whether an action is worthwhile as a means depends on whether the end is possible and worthwhile.
  \end{block}

\end{frame}

\begin{frame}
  \frametitle{Means-end relation}

  \begin{block}{Principle: Means-end relation}
    An action is (in part) a means only if it is part of a supporting means-end relation.
  \end{block}
  \btVFill
  \vspace{-10pt}

  To understand an action as a means is to understand the action supports achieving something in some way.
That something is the end.
And, that the means achieves the end in some way establishes the relation.

\btVFill
\end{frame}

\begin{frame}
  \frametitle{Means-end dependence}

  \begin{block}{Principle: Means-end dependence}
    Whether an action is worthwhile as a means depends on whether the end is possible and worthwhile.
  \end{block}
  \btVFill
  \vspace{-10pt}
  \only<+>{
    Three are two key parts this principle:
    \begin{enumerate}
    \item The evaluation of end as possible.
    \item The evaluation of end as worthwhile
    \end{enumerate}

    \begin{itemize}
    \item If the end is not possible, then the means is not worthwhile.
    \item If the end is not worthwhile, then the relevant action is not worthwhile as a means.
    \end{itemize}
  }
  \only<+>{

    \begin{quote}
      the value of the means derives from the value of the ends \dots
      If there are reasons to take the means, they must be none other than the reasons to pursue the ends, or at least they must derive from them.
    \end{quote}
    \citenote{\fullcite{Raz:2005aa}\newline\mbox{}}
  }
  \only<+>{
    \begin{quote}
      I may will the performance of certain actions as means of obtaining any desired good; but as my willing of these actions is only secondary, and founded on the supposition, that they are causes of the proposed effect; as soon as I discover the falshood of that supposition, they must become indifferent to me.
    \end{quote}
    \citenote{\fullcite{Hume:2011aa}}
  }
  \btVFill
\end{frame}

\begin{frame}
  \frametitle{Premise P3}
  {%
    \setbeamertemplate{enumerate item}{P\arabic{enumi}.}
    \setbeamertemplate{enumerate subitem}{P\arabic{enumi}\alph{enumii}.}
    \begin{enumerate}
      \setcounter{enumi}{2}
    \item A rational agent is permitted to settle on an action (in part) as a means only if
      \begin{enumerate}
      \item \emph{the agent takes there to be a means-end relation supporting the means and evaluates the end}, and
      \item \emph{the evaluation of the end supports settling on the action as a means to the end}.
      \end{enumerate}
    \end{enumerate}
  }
  \hozlinedash
  \btVFill
  \only<2>{
    \begin{enumerate}
    \item In order for the agent to evaluate the action as a means the agent must take there to be a means-end relation supporting the means and evaluates the end.
    \item The evaluation supports settling on the action as a means, then the evaluation supports settling on the action as a means to the end.
    \end{enumerate}

    The agent `has' the end as the evaluation of the end supports settling on the action as a means (to the end).
    The agent's evaluation of the end does work in establishing actions as permissible.
  }
  \only<3>{
    {\footnotesize
      \begin{block}{Taking}
        A rational agent is permitted to settle on an action (in part) as a means
        \newline
        \mbox{ }\hfill\emph{only if}\hfill\mbox{ }
        \newline
        the agent \textcolor{fuchsia}{\emph{takes there to be}}  a means-end relation from an end they have which supports taking the relevant means.
      \end{block}
    }
  }
  \btVFill
\end{frame}

\begin{frame}
  \frametitle{Positive Argument (recap)}
  \only<1>{
    {%
      \setbeamertemplate{enumerate item}{P\arabic{enumi}.}
      \setbeamertemplate{enumerate subitem}{P\arabic{enumi}\alph{enumii}.}
      \begin{enumerate}
      \item A rational agent is permitted to settle on an action (in part) as a means only if
        \begin{enumerate}
        \item \emph{the agent evaluates the action as a means}, and
        \item \emph{the evaluation supports settling on the action as a means}.
        \end{enumerate}
      \item \emph{An evaluation of an action as a means requires a supporting means-end relation and evaluation of the end.}
      \item A rational agent is permitted to settle on an action (in part) as a means only if
        \begin{enumerate}
        \item \emph{the agent takes there to be a means-end relation supporting the means and evaluates the end}, and
        \item \emph{the evaluation of the end supports settling on the action as a means to the end}.
        \end{enumerate}
      \end{enumerate}
    }
  }
  \only<2>{
    {%
      \setbeamertemplate{enumerate item}{P\arabic{enumi}.}
      \setbeamertemplate{enumerate subitem}{P\arabic{enumi}\alph{enumii}.}
      \begin{enumerate}
        \setcounter{enumi}{2}
      \item A rational agent is permitted to settle on an action (in part) as a means only if
        \begin{enumerate}
        \item \emph{the agent takes there to be a means-end relation supporting the means and evaluates the end}, and
        \item \emph{the evaluation of the end supports settling on the action as a means to the end}.
        \end{enumerate}
      \end{enumerate}
    }
    \hozlinedash
    {\footnotesize
      \begin{block}{Taking}
        A rational agent is permitted to settle on an action (in part) as a means
        \newline
        \mbox{ }\hfill\emph{only if}\hfill\mbox{ }
        \newline
        the agent \textcolor{fuchsia}{\emph{takes there to be}}  a means-end relation from an end they have which supports taking the relevant means.
      \end{block}
    }
  }

\end{frame}


\section{Summary}
\label{sec:summary}


\begin{frame}
  \frametitle{Summary}


\end{frame}


\begin{frame}
  \frametitle{Nixon}

  {\rmfamily
    Static monitors shimmer and enamelled wood creaks as Nixon recovers consciousness.

    A big red button is inside an open suitcase pierced with keys.

    Safety protocols have been followed.

    Nixon presses down and an electrical signal \dots
  }

  \pause
  \hozlinedash

  \begin{itemize}
  \item<+-> Did Nixon evaluate a nuclear strike as worthwhile?
    \begin{itemize}
    \item<+-> Setted on the means, so by settling principle Nixon took this to be worthwhile.
    \item<+-> And, this commits Nixon to the evaluation of the end as worthwhile.
    \item<+-> Nixon, however, was unable to recall that the end was a nuclear strike!
      \begin{itemize}
      \item<+-> Different representations of the same outcome.
      \end{itemize}
    \end{itemize}
  \end{itemize}
\end{frame}



\section{Future work}
\label{sec:future-work}




\subsection{Additional scenarios}
\label{sec:final-case}


\begin{frame}
  \frametitle{Multiple agents}

  \begin{tikzpicture}
    \def\svgwidth{1\linewidth}
    \node[inner sep=0, opacity=0.25] (image) at (.25,-.75) {\includeinkscape{images/pandpf}};
    \node[align=left, font={\rmfamily}] at (0,1) {
      Outside his house he found Piglet, jumping up and down trying to\\ reach the knocker.\\

      ``Hallo, Piglet,'' he said.\\

      ``Hallo, Pooh,'' said Piglet.\\

      ``What are \emph{you} trying to do?''\\

      ``I was trying to reach the knocker,'' said Piglet. ``I just came round---''\\

      ``Let me do it for you,'' said Pooh kindly.\\
      So he reached up and knocked at the door.};
  \end{tikzpicture}

  \citenote{\citeauthor{Milne:2009aa} (\citeyear{Milne:2009aa}) \citetitle{Milne:2009aa}. With Decorations by Ernest H.\ Shepard.}
\end{frame}


\appendix
% Remove structure slides between sections
\AtBeginSection[]
{
}

\begin{frame}[noframenumbering]
  \frametitle{References}
  \printbibliography
\end{frame}

\begin{frame}
  \frametitle{More cases}

  Persistent objects:
\begin{itemize}
\item I drink Yakult because I had \emph{something} in mind when I bought a months worth.
\item I still won't open some bottle of whisky because I'm saving it for some purpose.
\item You see a gift for a friend in a store but have to go to a meeting, and after you end up searching the store for the gift.
  \begin{itemize}
  \item \dots and sometimes you might come across the gift again but are sure that you found something better.
  \end{itemize}
\end{itemize}
\end{frame}


\begin{frame}
  \frametitle{Supermarket}

  {\rmfamily

    Snow dances and random flicker turns to rigid form as the agent regains focus.

    The supermarket aisle is empty, and their left hand is stretched toward an item as if ready to impart life.

    \emph{Crikey}!

    Forgetting the shopping list, and now forgetting consciousness.

    What were they doing?

    Their position suggests they were about to obtain the item.

    No basket. So, perhaps this is all they came in for.

    But what would the item be for?  What is the item a means to?

  }
\end{frame}

\section{Relating the principles}

\begin{frame}
  \frametitle{Relating the principles}

  \begin{block}{Taking (by reasoning)}
    A rational agent is permitted to settle on an action (in part) as a means
    \newline
    \mbox{ }\hfill\emph{only if}\hfill\mbox{ }
    \newline
    the agent \emph{takes there to be}  a means-end relation from an end they have which supports taking the relevant means \emph{(by reasoning via the relation)}.
  \end{block}

  \begin{enumerate}
  \item An entailment is not required for the argument.
  \item Requiring an entailment would not help the argument.
    \begin{itemize}
    \item \schemaName{Taking} would hold in all the cases in which \schemaName{Reasoning} does, but this would say nothing about the cases in which \schemaName{Reasoning} fails.
    \end{itemize}
  \item There may be good reason to \schemaName{Taking} and \schemaName{Reasoning} which would break the entailment.
  \end{enumerate}
\end{frame}

\section{Types of cases}

\begin{frame}
  \frametitle{Types of cases (aside)}

  Professor Oblique is reasoning about an end, and considers a means (initially to some end) on the basis of a persistent object, but is unable to reason from the means to the end.
  \begin{itemize}
  \item Three paramaters to vary:
    \begin{enumerate}
    \item Whether the agent is able to reason about the end.
    \item Whether the agent is able to reason about the means-end relation.
    \item Whether the agent is able to recall information about the end/means/means-end relation `internally' or `externally'
    \end{enumerate}
  \item Example questions:
    \begin{itemize}
    \item Are there cases in which an agent recognises a means, but is unable to recall what the means is an end to, but would easily be able to reason from the end to the means were they able to recall the end?
    \item Could an agent rationally settle on a means if they have neither an end nor a candidate means-end relation?
    \end{itemize}
  \end{itemize}
\end{frame}



\section{Akrasia}
\label{sec:akrasia}

\begin{frame}
  \frametitle{Akrasia}

  \begin{itemize}
  \item[\hand] Why think that in cases where an agent \emph{can} reason from an end to a means that the agent \emph{must} settle on the means they are able to reason to?
  \end{itemize}

  Here's something of an inductive argument (needs work).

\begin{enumerate}
\item Settling on means requires an appropriate means-end relation.
\item Sometimes an agent may have a `bad grasp' on the relevant means-end relations or recognise their inability to adequately reason from ends to means.
\item Sometimes it is permissible for an agent to rely on means-end relations they are unable to reason though and hence settle on a means that they are unable to reason to.
\end{enumerate}
\end{frame}

\begin{frame}
  \frametitle{Akrasia}

  {\rmfamily
    After dinner work and enjoyable meals.

    A single glass of wine makes the meal more enjoyable and does not obstruct later work.

    However, after the first glass, you can only make sense of having another.

    Any lost productivity can be made up for elsewhere, but this dinner must end at some point.
  }

  \hozlinedash

  Example, involving reasoning.
\end{frame}

\begin{frame}
  \frametitle{Akrasia}

  `How possible'.
\end{frame}


\end{document}