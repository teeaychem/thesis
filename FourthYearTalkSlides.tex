\documentclass[noamssymb, compress, handout]{beamer} % amssymb is incompatible with mtpro2

\usetheme{Antibes}
\usecolortheme{beaver}

\setbeamertemplate{enumerate items}[default]

\usepackage[T1]{fontenc}
   % \usepackage{textcomp}
   % \usepackage{newtxtext}
   % \renewcommand\rmdefault{Pym} %\usepackage{mathptmx} %\usepackage{times}
   % \usepackage{bm}% Access to bold math symbols
   \usepackage[no-math]{fontspec}
   \defaultfontfeatures{Ligatures=TeX,Numbers={Proportional}}
   \newfontfeature{Microtype}{protrusion=default;expansion=default;}
   \setmainfont[Ligatures=TeX,Scale=MatchLowercase]{Source Serif Pro}
   \setsansfont[Ligatures=TeX,Scale=MatchLowercase,BoldFont={* Semibold}]{Source Sans Pro}
   \setmonofont[Scale=0.8]{Source Code Pro}
   % \usepackage{selnolig}% For suppressing certain typographic ligatures automatically
   % \usepackage{microtype}

   % % mtpro2 fix from:
   % https://tex.stackexchange.com/questions/374353/no-room-for-a-new-count
   \makeatletter
   \let\alloc@latex\alloc@
   \def\alloc@#1#2#3#4{\newcount}% the current \newcount doesn't use \alloc@
   \makeatother
   % % %
   \usepackage[complete, subscriptcorrection, slantedGreek, mtpfrak, mtpbb, mtpcal]{mtpro2}
% % % % % % %
\usepackage{amsthm}         % (in part) For the defined environments
\usepackage{mathtools}      % Improves  on amsmaths/mtpro2


% % % The bibliography % % %
\usepackage[backend=biber,
  style=authoryear-comp,
  bibstyle=authoryear,
  citestyle=authoryear-comp,
  uniquename=false,%allinit,
  % giveninits=true,
  backref=false,
  hyperref=true,
  url=false,
  isbn=false,
  useprefix=true,
  ]{biblatex}
\DeclareFieldFormat{postnote}{#1}
\DeclareFieldFormat{multipostnote}{#1}
% \setlength\bibitemsep{1.5\itemsep}
\newcommand{\noopsort}[1]{}
\addbibresource{Thesis.bib}
% % % % % % % % % % % % % % %

% \usepackage[inline]{enumitem}
% \setlist[itemize]{noitemsep}
% \setlist[description]{style=unboxed,leftmargin=\parindent,labelindent=\parindent,font=\normalfont\space}
% \setlist[enumerate]{noitemsep}

\usepackage{pifont}
\newcommand{\hand}{\ding{43}}
\usepackage{array}

\usepackage{setspace}

\usepackage{dashrule}

\newcommand{\hozline}[0]{%
  \noindent\hdashrule[0.5ex][c]{\textwidth}{.1pt}{}
  %\vspace{-10pt}
  % \noindent\rule{\textwidth}{.1pt}
}

\newcommand{\hozlinedash}[0]{%
  \noindent\hdashrule[0.5ex][c]{\textwidth}{.1pt}{2.5pt}
  %\vspace{-10pt}
}

% % % Color % % %
% \usepackage[usenames, dvipsnames]{xcolor}
% \usepackage{everysel}
\definecolor{fuchsia}{HTML}{FE4164}%Neon Fuchsia %{F535AA}%Neon Pink
\usecolortheme[named=fuchsia]{structure}
\colorlet{structure}{fuchsia}
\usestructuretemplate{\color{structure}}{}
% % % % % % % % % %

% Set quotes to non-italic
\setbeamerfont{quote}{shape=\upshape,family=\rmfamily}

\addtobeamertemplate{navigation symbols}{}{%
    \usebeamerfont{footline}%
    \usebeamercolor[fg]{footline}%
    \hspace{1em}%
    \insertframenumber/\insertmainframenumber
}

\setbeamercovered{transparent}

% https://tex.stackexchange.com/questions/28654/beamer-table-of-contents-display-all-subsections-below-section
\AtBeginSection[]
{
\begin{frame}[noframenumbering]
  \frametitle{Structure}
  \setbeamertemplate{navigation symbols}{ }
  \tableofcontents[
  currentsection,
  % hideothersubsections,
  % sectionstyle=show/hide,
  % subsectionstyle=show/shaded,
  ]

\end{frame}
}



\title{Means-end reasoning and means-end relations}
\author{Ben Sparkes}
% \institute{Stanford}
% \date{2014}



\begin{document}

\begin{frame}[noframenumbering]
  \setbeamertemplate{navigation symbols}{ }
  \titlepage
\end{frame}


\section*{Overview}
\label{sec:overview}

\begin{frame}
  \frametitle{Overview}

  \begin{itemize}
  \item Argue against the idea that settling on means requires an agent reasons from ends to means.
  \item Argue for the idea that settling on means requires an agent to take a means-end relation to exist.
  \end{itemize}
  
\end{frame}


\section{Means-end schema}
\label{sec:schema}

% \begin{frame}
%   \frametitle{Schema}

%   \begin{center}
%     An agent is rationally permitted to settle on a means (as a means)

%     \mbox{ }

%     \emph{only if}

%     \mbox{ }

%     The agent [\underline{does something with}] a means-end relation from an end they have which supports taking the relevant means.
%   \end{center}
% \end{frame}


\begin{frame}
  \frametitle{The means-end Schema}

  % \begin{block}<1->
  %   {\footnotesize An agent is rationally permitted to settle on a means (as a means) \emph{only if} \newline The agent [\underline{?}] a means-end relation from an end they have which supports taking the means.}
  % \end{block}
  \begin{block}<1->{The Schema}
    An agent is \textcolor<3>{fuchsia}{rationally permitted to settle on a means (as a means)}
    \newline
    \mbox{ }\hfill\emph{only if}\hfill\mbox{ }
    \newline
    The agent [\textcolor<5>{fuchsia}{does something with}] \textcolor<4>{fuchsia}{a means-end relation from an end they have which supports taking the relevant means}.
  \end{block}

  \onslide<2-5>{Three parts to this schema:}
  \begin{enumerate}%[label=\alph*., noitemsep]
  \item<3-5> \textcolor<3>{fuchsia}{An agent being rationally permitted to settle on a means (as a means).}
  \item<4-5> \textcolor<4>{fuchsia}{A means-end relation from an end the agent has which supports taking the means.}
  \item<5-5> \textcolor<5>{fuchsia}{A link the agent has to the specified means-end relation.}
  \end{enumerate}
\end{frame}

\begin{frame}
  \frametitle{Reasoning}

  \begin{block}{Reasoning}
    An agent is rationally permitted to settle on a means (as a means)
    \newline
    \mbox{ }\hfill\emph{only if}\hfill\mbox{ }
    \newline
    The agent \textcolor{fuchsia}{\emph{reasons via}} a means-end relation from an end they have which supports taking the relevant means \emph{by reasoning via the relation}.
  \end{block}

  \begin{itemize}
  \item Agent is required to connect means to end.
  \item Many cases of reasoning satisfy this requirment.
    \begin{itemize}
    \item I have the end of waking up, and I combine all sorts of information I have about myself and coffee to settle on drinking a cup.
    \end{itemize}
  \end{itemize}
\end{frame}

  \begin{frame}
    \frametitle{Taking}

  \begin{block}{Taking}
    An agent is rationally permitted to settle on a means (as a means)
    \newline
    \mbox{ }\hfill\emph{only if}\hfill\mbox{ }
    \newline
    The agent \textcolor{fuchsia}{\emph{takes there to be}} a means-end relation from an end they have which supports taking the relevant means.
  \end{block}

  \begin{itemize}
  \item An existential-like requirement that does not require the agent to connect the means to the end.
  \item Some cases of reasoning may satisfy this requirement.
    \begin{itemize}
    \item At some point I purchased a collection of Ligeti recordings, and I have no idea why, but it wasn't an acident and so I will listen to them.
    \end{itemize}
  \end{itemize}
\end{frame}

\begin{frame}
  \frametitle{I will argue \emph{for}:}

  \begin{block}{Taking}
    An agent is rationally permitted to settle on a means (as a means)
    \newline
    \mbox{ }\hfill\emph{only if}\hfill\mbox{ }
    \newline
    The agent \textcolor{fuchsia}{\emph{takes there to be}} a means-end relation from an end they have which supports taking the relevant means.
  \end{block}
\end{frame}


\begin{frame}
  \frametitle{I will argue \emph{against}:}

  \begin{block}{Reasoning}
    An agent is rationally permitted to settle on a means (as a means)
    \newline
    \mbox{ }\hfill\emph{only if}\hfill\mbox{ }
    \newline
    The agent \textcolor{fuchsia}{\emph{reasons via}} a means-end relation from an end they have which supports taking the relevant means.
  \end{block}
\end{frame}

\begin{frame}
  \frametitle{No entailment between the principles}

  \begin{block}{Reasoning}
    An agent is rationally permitted to settle on a means (as a means)
    \newline
    \mbox{ }\hfill\emph{only if}\hfill\mbox{ }
    \newline
    The agent \textcolor{fuchsia}{\emph{reasons via}}  a means-end relation from an end they have which supports taking the relevant means.
  \end{block}

  {\Large \mbox{ }\hfill \(\not\Uparrow\) \qquad \(\not\Downarrow\) \hfill\mbox{ }}

  \begin{block}{Taking}
    An agent is rationally permitted to settle on a means (as a means)
    \newline
    \mbox{ }\hfill\emph{only if}\hfill\mbox{ }
    \newline
    The agent \textcolor{fuchsia}{\emph{takes there to be}} a means-end relation from an end they have which supports taking the relevant means.
  \end{block}

\end{frame}


\begin{frame}
  \frametitle{Possible entailment from `reasoning' to `taking'}

  \begin{block}{Taking (by reasoning via)}
    An agent is rationally permitted to settle on a means (as a means)
    \newline
    \mbox{ }\hfill\emph{only if}\hfill\mbox{ }
    \newline
    The agent \textcolor{fuchsia}{\emph{takes there to be}}  a means-end relation from an end they have which supports taking the relevant means \textcolor{fuchsia}{\emph{(by reasoning via the relation)}}.
  \end{block}

  {\Large \mbox{ }\hfill\(\Downarrow\)\hfill\mbox{ }}

  \begin{block}{Taking}
    An agent is rationally permitted to settle on a means (as a means)
    \newline
    \mbox{ }\hfill\emph{only if}\hfill\mbox{ }
    \newline
    The agent \textcolor{fuchsia}{\emph{takes there to be}} a means-end relation from an end they have which supports taking the relevant means.
  \end{block}

\end{frame}


\section{Argument}
\label{sec:argument}


\subsection{Negative part}
\label{sec:negative}

\begin{frame}
  \frametitle{Negative argument}

  \begin{enumerate}
  \item\label{scenarios:exist} There are cases in which agents recognise a means (as means) without being able to reason from an end they have to those means.
  \item\label{scenarios:persmissible} And, in these cases the agent is rational in settling on the means.

  \item[C\(_{\text{i}}\)] `Reasoning' does not hold.
    % \item[C\(_{\text{I}}\).]\label{scenario:no-reasoning} In order for an agent to be rationally permitted in settling on a means it cannot be required that the agent reasons via a means-end relation from an end they have which supports taking the relevant means.
    \begin{itemize}
    \item From \ref{scenarios:exist} and~\ref{scenarios:persmissible}.
    \end{itemize}
  \end{enumerate}

  \hozlinedash
  {\footnotesize
    \begin{block}{Reasoning}
      An agent is rationally permitted to settle on a means (as a means)
      \newline
      \mbox{ }\hfill\emph{only if}\hfill\mbox{ }
      \newline
      The agent \textcolor{fuchsia}{\emph{reasons via}}  a means-end relation from an end they have which supports taking the relevant means.
    \end{block}
  }
\end{frame}

\subsubsection{Scenario}
\label{sec:case}

\begin{frame}
  \frametitle{Oblique (counterexample to `Reasoning')}

  {\rmfamily
    % \begin{spacing}{1.3}
    \begin{itemize}
      \setbeamertemplate{itemize items}{}
      \item Oblique regains consciousness after a moment of darkness.%\newline
      \item They are in a supermarket, and their hand is stretched toward an item on a shelf.%\newline
      \item Oblique is unable to recall why they are in the supermarket and is unable to recognise what the item on the shelf would be for.%\newline
      \item Oblique does not have a shopping list.%\newline
      \item However, Oblique's outstretched hand indicates to them that they were about to purchase the item, and so they settle on doing so.
      \end{itemize}
    % \end{spacing}
  }
\end{frame}

\begin{frame}
  \frametitle{Oblique settled without satisfying `Reasoning'}

  \begin{itemize}
  \item Oblique settles on purchasing the item as a means to some end.
  \item Oblique's outstretched hand provides information that purchasing the item was something they settled on.
  \item Oblique is unable to reason via a means-end relation from an end they have which supports purchasing the item.
    \begin{itemize}
    \item Obliqe is unable to reason about the relevant end.
    \item Without the relevant end Oblique is unable to reason about means-end relations from the end to the means.
    \end{itemize}
  \end{itemize}

\end{frame}


\begin{frame}
  \frametitle{Oblique is rationally permitted to settle on the means}

  \begin{itemize}
  \item Oblique's outstretched hand provides information that purchasing the item was something they settled on.
  \item Two arguments:
    \begin{enumerate}
    \item Purchasing the item is supported by a means-end relation.
      \begin{itemize}
      \item If Oblique settled on the purchasing the item, this was supported by a means-end relation and some end.
      \item So, purchasing the item was supported by a means-end relation and some end Oblique had.
      \item And, as Oblique only lost consciousness for a moment they reason that the relevant means-end relation and end continue to hold.
      \end{itemize}
    \item It seems asburd to send Oblique all the way home to look at the shopping list they left on their fridge.
    \end{enumerate}
  \end{itemize}

\end{frame}

\begin{frame}
  \frametitle{Key features of the counterexample}

  \begin{enumerate}[A)]
  \item The agent has enough information to determine that an act would be worthwhile as a means to some end.
  \item The agent does not have sufficient information to determine how a means-end relation from the end would support the means.
  \item The agent has sufficient information to determine that a supporting means-end relation from the end they have holds.
  \end{enumerate}

  \hozlinedash

  \begin{itemize}
  \item That Oblique cannot recall the end is not a key feature.
  \end{itemize}
\end{frame}

\begin{frame}
  \frametitle{Negative argument (recap)}

  \begin{enumerate}
  \item\label{scenarios:exist} There are cases in which agents recognise a means (as means) without being able to reason from an end they have to those means.
  \item\label{scenarios:persmissible} And, in these cases the agent is rational in settling on the means.

  \item[C\(_{\text{i}}\)] `Reasoning' does not hold.
    % \item[C\(_{\text{I}}\).]\label{scenario:no-reasoning} In order for an agent to be rationally permitted in settling on a means it cannot be required that the agent reasons via a means-end relation from an end they have which supports taking the relevant means.
    \begin{itemize}
    \item From \ref{scenarios:exist} and~\ref{scenarios:persmissible}.
    \end{itemize}
  \end{enumerate}

  \hozlinedash
  {\footnotesize
    \begin{block}{Reasoning}
      An agent is rationally permitted to settle on a means (as a means)
      \newline
      \mbox{ }\hfill\emph{only if}\hfill\mbox{ }
      \newline
      The agent \textcolor{fuchsia}{\emph{reasons via}}  a means-end relation from an end they have which supports taking the relevant means.
    \end{block}
  }

\end{frame}

\begin{frame}
  \frametitle{I have argued \emph{against}:}

  \begin{block}{Reasoning}
    An agent is rationally permitted to settle on a means (as a means)
    \newline
    \mbox{ }\hfill\emph{only if}\hfill\mbox{ }
    \newline
    The agent \textcolor{fuchsia}{\emph{reasons via}} a means-end relation from an end they have which supports taking the relevant means.
  \end{block}

  \begin{itemize}
  \item[\hand] Professor Oblique read some papers as a means to making progress on their research even though Oblique could not see how the papers would serve to make progress.
  \end{itemize}
\end{frame}



\subsection{Positive part}
\label{sec:positive}

\begin{frame}
  \frametitle{I now will argue \emph{for}:}

  \begin{block}{Taking}
    An agent is rationally permitted to settle on a means (as a means)
    \newline
    \mbox{ }\hfill\emph{only if}\hfill\mbox{ }
    \newline
    The agent \textcolor{fuchsia}{\emph{takes there to be}} a means-end relation from an end they have which supports taking the relevant means.
  \end{block}
\end{frame}

\subsubsection{Two principles}
\label{sec:two-principles}

\begin{frame}
    \frametitle{Positive Argument}

  \begin{enumerate}
    \setcounter{enumi}{2}
  \item\label{settle:worthwhile} For an agent to be rationally permitted to settle on an action, the agent must take the action to be worthwhile.
    % \begin{itemize}
    % \item Principle: Rationally settling on an action is explained by the action being considered both possible and worthwhile by the agent, perhaps in comparison to the same attributes to other actions.
    % \end{itemize}
  \item\label{m-e:dependence} If a rational agent considers a means \emph{only} as a means then: \newline Taking the means to be worthwhile amounts to taking the means to be part of a means-end relation with an end they take to be worthwhile.
    % \begin{itemize}
    % \item Principle: Whether a means (as a means) is worthwhile wholly depends on whether the end to the means is worthwhile.
    %   % \nolinebreak \mbox{ }\hfill(From \ref{m-e:dependence}, special case)
    % \end{itemize}
    \item[C\(_{\text{ii}}\)] `Taking' does hold.
  % \item[C\(_{\text{II}}\).] \label{together} If an agent is rationally permitted to settle on a means as a means, the agent must take there to be some relevant means-end relation from an end the agent has which supports taking the relevant means.
    \begin{itemize}
    \item By \ref{settle:worthwhile} the agent takes the means to be worthwhile.
    \item And, by \ref{m-e:dependence} this can only be because the agent takes the means to be part of a means-end relation with a worthwhile end.
    \end{itemize}
  \end{enumerate}

  (Note: Still not quite right.)

  \hozlinedash

  {\footnotesize
    \begin{block}{Taking}
      An agent is rationally permitted to settle on a means (as a means)
      \newline
      \mbox{ }\hfill\emph{only if}\hfill\mbox{ }
      \newline
      The agent \textcolor{fuchsia}{\emph{takes there to be}}  a means-end relation from an end they have which supports taking the relevant means.
    \end{block}
  }
\end{frame}


\begin{frame}
  \frametitle{Premise 3 / Settling Principle}

  \begin{enumerate}
    \setcounter{enumi}{2}
  \item For an agent to be rationally permitted to settle on an action, the agent must take the action to be worthwhile.
    \begin{itemize}
    \item Principle: Rationally settling on an action is explained by the action being considered both possible and worthwhile by the agent, perhaps in comparison to the same attributes to other actions.
    \end{itemize}
  \end{enumerate}

  \begin{itemize}
  \item Standard idea about belief and desire, etc.
  \item Taking as worthwhile is an abstraction of the desire part of this.
  \end{itemize}
\end{frame}

\begin{frame}
  \frametitle{Premise 4 / Means-end principle}

  \begin{enumerate}
    \setcounter{enumi}{3}
  \item If a rational agent considers a means \emph{only} as a means then taking the means to be worthwhile amounts to taking the means to be part of a means-end relation with a worthwhile end.
    \begin{itemize}
    \item Principle: Whether a means (as a means) is worthwhile wholly depends on whether the end to the means is worthwhile.
    \end{itemize}
  \end{enumerate}

\end{frame}



\begin{frame}

% \begin{quote}
%   Who wills the end, wills (so far as reason has a decisive influence on his actions) also the means which are indispensably necessary and in his power.\nolinebreak
%   \mbox{ }\hfill\mbox{(\citeauthor[Ak 417]{Kant:1948aa})}
% \end{quote}

% \hozlinedash

\begin{quote}
  I may will the performance of certain actions as means of obtaining any desired good; but as my willing of these actions is only secondary, and founded on the supposition, that they are causes of the proposed effect; as soon as I discover the falshood of that supposition, they must become indifferent to me.\nolinebreak
  \mbox{ }\hfill\mbox{\hfill(\citeauthor{Hume:2011aa}, T2.3.3)}
\end{quote}

\end{frame}


\begin{frame}
  \begin{itemize}
  \item Key idea: the two principles are fairly common.
  \item The relations are necessary, and do some work.
  \item The conclusion does not rely on the agent being able to reason about the relevant end, as was the case with Oblique.
  \end{itemize}
\end{frame}

\begin{frame}
  \frametitle{Positive Argument (recap)}

  \begin{enumerate}
    \setcounter{enumi}{2}
  \item\label{settle:worthwhile} For an agent to be rationally permitted to settle on an action, the agent must take the action to be worthwhile.
    % \begin{itemize}
    % \item Principle: Rationally settling on an action is explained by the action being considered both possible and worthwhile by the agent, perhaps in comparison to the same attributes to other actions.
    % \end{itemize}
  \item\label{m-e:dependence} If a rational agent considers a means \emph{only} as a means then: \newline Taking the means to be worthwhile amounts to taking the means to be part of a means-end relation with an end they take to be worthwhile.
    % \begin{itemize}
    % \item Principle: Whether a means (as a means) is worthwhile wholly depends on whether the end to the means is worthwhile.
    %   % \nolinebreak \mbox{ }\hfill(From \ref{m-e:dependence}, special case)
    % \end{itemize}
    \item[C\(_{\text{ii}}\)] `Taking' does hold.
  % \item[C\(_{\text{II}}\).] \label{together} If an agent is rationally permitted to settle on a means as a means, the agent must take there to be some relevant means-end relation from an end the agent has which supports taking the relevant means.
    \begin{itemize}
    \item By \ref{settle:worthwhile} the agent takes the means to be worthwhile.
    \item And, by \ref{m-e:dependence} this can only be because the agent takes the means to be part of a means-end relation with a worthwhile end.
    \end{itemize}
  \end{enumerate}

  (Note: Still not quite right.)

  \hozlinedash

  {\footnotesize
    \begin{block}{Taking}
      An agent is rationally permitted to settle on a means (as a means)
      \newline
      \mbox{ }\hfill\emph{only if}\hfill\mbox{ }
      \newline
      The agent \textcolor{fuchsia}{\emph{takes there to be}}  a means-end relation from an end they have which supports taking the relevant means.
    \end{block}
  }
\end{frame}

\begin{frame}
  \frametitle{I have argued \emph{for}:}
    \begin{block}{Taking}
    An agent is rationally permitted to settle on a means (as a means)
    \newline
    \mbox{ }\hfill\emph{only if}\hfill\mbox{ }
    \newline
    The agent \textcolor{fuchsia}{\emph{takes there to be}} a means-end relation from an end they have which supports taking the relevant means.
  \end{block}
\end{frame}

\section{Summary}
\label{sec:summary}


\begin{frame}
  \frametitle{The schema}



  \begin{itemize}
  \item Somewhat weak necessary condition.
    
  \end{itemize}
\end{frame}


\begin{frame}
  \frametitle{Two principles}

  \begin{itemize}
  \item Principle: Rationally settling on an action is explained by the action being considered both possible and worthwhile by the agent, perhaps in comparison to the same attributes to other actions.
  \item Principle: Whether a means (as a means) is worthwhile wholly depends on whether the end to the means is worthwhile.
  \end{itemize}

  \hozlinedash

  \begin{itemize}
  \item If an agent is settled on a means (as a means), these ensure that an agent takes there to be supporting means-end structure.
  \item Necessary, but perhaps not sufficient.
  \end{itemize}

  \begin{itemize}
  \item Two important questions:
    \begin{enumerate}
    \item Is there an intermediate position which is more informative about the role of means-end relations in practical reasoning?
    \item How could the argument be generalised to cover cases in which an agent evlauates a means along multiple dimensions?
    \end{enumerate}
  \end{itemize}

 \end{frame}


 \begin{frame}
  \frametitle{Nixon}

  {\rmfamily
    Static monitors shimmer and enamelled wood creaks as Nixon recovers consciousness.

    A big red button is inside an open suitcase pierced with keys.

    Safety protocols have been followed.

    Nixon presses down and an electrical signal \dots
  }

  \pause
  \hozlinedash

  \begin{itemize}
  \item<+-> Did Nixon evaluate a nuclear strike as worthwhile?
    \begin{itemize}
    \item<+-> Setted on the means, so by settling principle Nixon took this to be worthwhile.
    \item<+-> And, this commits Nixon to the evaluation of the end as worthwhile.
    \item<+-> Nixon, however, was unable to recall that the end was a nuclear strike!
      \begin{itemize}
      \item<+-> Different representations of the same outcome.
      \end{itemize}
    \end{itemize}
  \end{itemize}
\end{frame}



\section{Future work}
\label{sec:future-work}


\subsection{Akrasia}
\label{sec:akrasia}

\begin{frame}
  \frametitle{Akrasia}

  \begin{itemize}
  \item[\hand] Why think that in cases where an agent \emph{can} reason from an end to a means that the agent \emph{must} settle on the means they are able to reason to?
  \end{itemize}

  Here's something of an inductive argument (needs work).

\begin{enumerate}
\item Settling on means requires an appropriate means-end relation.
\item Sometimes an agent may have a `bad grasp' on the relevant means-end relations or recognise their inability to adequately reason from ends to means.
\item Sometimes it is permissible for an agent to rely on means-end relations they are unable to reason though and hence settle on a means that they are unable to reason to.
\end{enumerate}
\end{frame}

\begin{frame}
  \frametitle{Akrasia}

  {\rmfamily
    After dinner work and enjoyable meals.

    A single glass of wine makes the meal more enjoyable and does not obstruct later work.

    However, after the first glass, you can only make sense of having another.

    Any lost productivity can be made up for elsewhere, but this dinner must end at some point.
  }

  \hozlinedash

  Example, involving reasoning.
\end{frame}

\begin{frame}
  \frametitle{Akrasia}

  `How possible'.
\end{frame}


\subsection{Additional scenarios}
\label{sec:final-case}

\begin{frame}
  \frametitle{Multiple agents}

  % \begin{quote}
    {\rmfamily
    Outside his house he found Piglet, jumping up and down trying to reach the knocker.

    ``Hallo, Piglet,'' he said.

    ``Hallo, Pooh,'' said Piglet.

    ``What are \emph{you} trying to do?''

    ``I was trying to reach the knocker,'' said Piglet. ``I just came round---''

    ``Let me do it for you,'' said Pooh kindly.
    So he reached up and knocked at the door.}\linebreak
    \mbox{ }\hfill\mbox{(\cite[77--78]{Milne:2009aa})}
  % \end{quote}
\end{frame}


\appendix

\begin{frame}[noframenumbering]
  \frametitle{References}
  \printbibliography
\end{frame}

\begin{frame}
  \frametitle{More cases}

  Persistent objects:
\begin{itemize}
\item I drink Yakult because I had \emph{something} in mind when I bought a months worth.
\item I still won't open some bottle of whisky because I'm saving it for some purpose.
\item You see a gift for a friend in a store but have to go to a meeting, and after you end up searching the store for the gift.
  \begin{itemize}
  \item \dots and sometimes you might come across the gift again but are sure that you found something better.
  \end{itemize}
\end{itemize}
\end{frame}


\begin{frame}
  \frametitle{Supermarket}

  {\rmfamily

    Snow dances and random flicker turns to rigid form as the agent regains focus.

    The supermarket aisle is empty, and their left hand is stretched toward an item as if ready to impart life.

    \emph{Crikey}!

    Forgetting the shopping list, and now forgetting consciousness.

    What were they doing?

    Their position suggests they were about to obtain the item.

    No basket. So, perhaps this is all they came in for.

    But what would the item be for?  What is the item a means to?

  }
\end{frame}


\begin{frame}
  \frametitle{Relating the principles}

  \begin{block}{Taking (by reasoning)}
    An agent is rationally permitted to settle on a means (as a means)
    \newline
    \mbox{ }\hfill\emph{only if}\hfill\mbox{ }
    \newline
    The agent \emph{takes there to be}  a means-end relation from an end they have which supports taking the relevant means \emph{(by reasoning via the relation)}.
  \end{block}

  \begin{enumerate}
  \item An entailment is not required for the argument.
  \item Requiring an entailment would not help the argument.
    \begin{itemize}
    \item `Taking' would hold in all the cases in which `reasoning' does, but this would say nothing about the cases in which `reasoning' fails.
    \end{itemize}
  \item There may be good reason to `taking' and `reasoning' which would break the entailment.
  \end{enumerate}
\end{frame}


\begin{frame}
  \frametitle{Types of cases (aside)}

  Professor Oblique is reasoning about an end, and considers a means (initially to some end) on the basis of a persistent object, but is unable to reason from the means to the end.
  \begin{itemize}
  \item Three paramaters to vary:
    \begin{enumerate}
    \item Whether the agent is able to reason about the end.
    \item Whether the agent is able to reason about the means-end relation.
    \item Whether the agent is able to recall information about the end/means/means-end relation `internally' or `externally'
    \end{enumerate}
  \item Example questions:
    \begin{itemize}
    \item Are there cases in which an agent recognises a means, but is unable to recall what the means is an end to, but would easily be able to reason from the end to the means were they able to recall the end?
    \item Could an agent rationally settle on a means if they have neither an end nor a candidate means-end relation?
    \end{itemize}
  \end{itemize}
\end{frame}


\end{document}