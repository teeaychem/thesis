\chapter{\requ{3} and \tC{}}
\label{sec:typicalRequs}



\begin{note}
  \begin{proposition}[\tC{2} and \requ{1}]
    \label{prop:hinge}
    \vspace{-\baselineskip}
    \begin{itenum}
    \item[\emph{If}:]
      Clauses~\ref{prop:hinge:A:rep},~\ref{prop:hinge:A:tI},~and~\ref{prop:hinge:A:novel} jointly hold for \(e\):
      \begin{enumerate}[label=\arabic*., ref=(\arabic*)]
      \item
        \label{prop:hinge:A:rep}
        \(T'\) is a \tRep{} of \vAgent{} \tCV{} \(\pv{\phi}{v}\) from \(\Phi\) by type \(T\) in \(e\).
      \item
        \label{prop:hinge:A:tI}
        \(\pvp{\psi}{v'}{\Psi}\) is a \tI{} of \(T'\)
      \item
        \label{prop:hinge:A:novel}
        The following conditional is true:
        \begin{itemize}
        \item[\emph{If}:]
          There is some available action \(a\) such that \vAgent{} is concluding \(\pv{\psi}{v'}\) from \(\Psi\), when \vAgent{} does \(a\).
        \item[\emph{Then}:]
          There is some available action \(a'\) such that \vAgent{} is concluding \(\pv{\psi}{v'}\) from \(\Psi\) without use of any novel information obtained by doing \(a'\), when \vAgent{} does \(a'\).
        \end{itemize}
      \end{enumerate}
    \item[\emph{Then}:]
      \(\pvp{\psi}{v'}{\Psi}\) is a \requ{} of \(e\) being an event in which \vAgent{} is \tCV{} \(\pv{\phi}{v}\) from \(\Phi\).
    \end{itenum}
    \vspace{-\baselineskip}
  \end{proposition}

  \noindent%
  \autoref{prop:hinge}, sufficient conditions for \requ{} when \tCV{}.
  Useful proposition with respect to counterexamples to \issueConstraint{}.

  Intuitive motivation for \autoref{prop:hinge} is simple.
  \autoref{prop:tC-and-fc} (\autopageref{prop:tC-and-fc}).

  If clauses hold and agent is \tCV{}, then \fc{}.
  This is exactly what is required for \requ{}.

  In detail:

  \begin{argument}{prop:hinge}
    Assume~ clauses~\ref{prop:hinge:A:rep},~\ref{prop:hinge:A:tI},~and~\ref{prop:hinge:A:novel} jointly hold.
    Suppose \tCV{}.
    Then, by \autoref{prop:tC-and-fc}, \fc{}.

    Then by \autoref{def:requ}, \requ{}.
  \end{argument}

  Recap.

  \tCV{}.
  Intuitively, various other things conclude.
  \tRep{}, requires events.
  Conditional of \autoref{prop:hinge:A:novel} strengthens so no novel information.

  As get \fc{}, then \requ{}.

  Importance of \autoref{prop:hinge} is existence of \requ{}.
  \tCV{} without novel information.
\end{note}



\section{\tC{2} and \issueConstraint{}}
\label{sec:tpyically-concluding}

\begin{note}
  Understand \requ{} and counterexamples to \issueConstraint{}.

  Introduced \tCV{} to motivate \requ{}.

  In this section, recast previous section in terms of \tCV{}.
\end{note}

\subsection{\tCon{2}}
\label{cha:binding:tCon}

\begin{note}
  The \tCon{0} parallel the \rCon{0}.
  Difference is third.
  Rather than \requ{}, conditions sufficient for \requ{} in terms of \tCV{}.

  \begin{definition}[The \tCon{0}]
    \label{def:tCon}
    \vspace{-\baselineskip}
    \begin{itemize}
    \item
      The \emph{\tCon{0}} hold of \(e\) with respect to \(\langle \vAgent{}, \pvp{\phi}{v}{\Phi}, \pvp{\psi}{v'}{\Psi}, e^{\flat} \rangle\).
    \end{itemize}

    \emph{If and only if}:

    \begin{itemize}
    \item
      Conditions~%
      \ref{def:tCon:C}~and~%
      \ref{def:tCon:Cing}~%
      jointly hold:
      \begin{enumerate}[label=\arabic*., ref=(\arabic*)]
      \item
        \label{def:tCon:C}
        \(e\) is an event in which \vAgent{} concludes \(\pv{\phi}{v}\) from \(\Phi\).
      \item
        \label{def:tCon:Cing}
        There is some sub-event \(e^{\flat}\) of \(e\) such that \ref{def:tCon:sR} and \ref{def:tCon:Cing:itc} jointly hold:
        \begin{enumerate}[label=\alph*., ref=(\arabic{enumi}\alph*)]
        \item
          \label{def:tCon:Cing:itc}
          The following conditional is true:
          \begin{itenum}
          \item[\emph{If}:]
            \(e^{\flat}\) is not an event in which \vAgent{} is \tCV{} \(\pv{\phi}{v}\) from \(\Phi\).
          \item[\emph{Then}:]
            \(e^{\flat}\) is not, or does not develop into, an event in which \vAgent{} concludes \(\pv{\phi}{v}\) from \(\Phi\).
          \end{itenum}
        \item
          \label{def:tCon:sR}
          Conditions%
          ~\ref{def:tCon:sR:rep},%
          ~\ref{def:tCon:sR:tI}, and%
          ~\ref{def:tCon:sR:itc} %
          jointly hold:
          \begin{enumerate}[label=\roman*., ref=(\arabic{enumi}\alph{enumii}.\roman*)]
          \item
            \label{def:tCon:sR:rep}
            \(T'\) is a \tRep{} of \vAgent{} \tCV{} \(\pv{\phi}{v}\) from \(\Phi\) by type \(T\) in \(e^{\flat}\).
          \item
            \label{def:tCon:sR:tI}
            \(\pvp{\psi}{v'}{\Psi}\) is a \tI{} of \(T'\)
          \item
            \label{def:tCon:sR:itc}
            The following conditional is true:
            \begin{itemize}
            \item[\emph{If}:]
              There is some available action \(a\) such that \vAgent{} is concluding \(\pv{\psi}{v'}\) from \(\Psi\), when \vAgent{} does \(a\).
            \item[\emph{Then}:]
              There is some available action \(a'\) such that \vAgent{} is concluding \(\pv{\psi}{v'}\) from \(\Psi\) without use of any novel information obtained by doing \(a'\), when \vAgent{} does \(a'\).
            \end{itemize}
          \end{enumerate}
        \end{enumerate}
      \end{enumerate}
    \end{itemize}
    \vspace{-1.5\baselineskip}
  \end{definition}

  \noindent%
  Similar to previous conditions.
  Two key differences.
  \ref{def:tCon:Cing}: \tCV{} rather than concluding.
  \ref{def:tCon:sR}: Rather than \requ{}, sufficient conditions for \requ{} so long as an agent is \tCV{}.

  In short, we slightly strengthen \ref{def:tCon:Cing} to provide a more specific account of the \scen{1} of interest.
\end{note}

\subsection{\tCon{2}, \qWhyV{} and \issueConstraint{}}
\label{sec:tccon2-qwhyv-}

\begin{note}
  \begin{proposition}[\tCV{3} and \qWhyV{}]
    \label{prop:tCV-WhyV}
    \vspace{-\baselineskip}
    \begin{itenum}
    \item[\emph{If}:]
      The \tCon{0} hold of \(e\) with respect to \(\langle \vAgent{}, \pvp{\phi}{v}{\Phi}, \pvp{\psi}{v'}{\Psi}, e^{\flat} \rangle\).
    \item[\emph{Then}:]
      \(\pvp{\psi}{v'}{\Psi}\), in part, answers \qWhyV{}.
    \end{itenum}
    \vspace{-1.5\baselineskip}
  \end{proposition}

  Instead of \requ{} directly, strengthen to \tCV{} and sufficient conditions for \requ{} given \tCV{}.

  \begin{argument}{prop:tCV-WhyV}
    The argument parallels \autoref{prop:requ-WhyV} with  minor adjustment.

    Rather appeal to Sub-condition~\ref{def:rCs:Cing:requ} of the \rCon{} to observe \(e^{\flat}\) is not an event in which \vAgent{} is concluding \(\pv{\phi}{v}\) from \(\Phi\), appeal to Sub-condition~\ref{def:tCon:sR} of the \tCon{} to observe \(e^{\flat}\) is not an event in which \vAgent{} is \tCV{} \(\pv{\phi}{v}\) from \(\Phi\).
    And, with this observation in hand appeal to Sub-condition~\ref{def:tCon:Cing:itc} to observe \(e^{\flat}\) is not, or does not develop into, an event in which \vAgent{} concludes \(\pv{\phi}{v}\) from \(\Phi\).
  \end{argument}

  \noindent%
  With \autoref{prop:tCV-WhyV} in hand, we observe the relevant parallel to \autoref{prop:requ-WhyV-ces}:

  \begin{proposition}[\tCV{3} and \issueConstraint{}]
    \label{prop:tCV-WhyV-ces}
    \vspace{-\baselineskip}
    \begin{itenum}
    \item[\emph{If}:]
      The \tCon{0} hold of \(e\) with respect to \(\langle \vAgent{}, \pvp{\phi}{v}{\Phi}, \pvp{\psi}{v'}{\Psi}, e^{\flat} \rangle\).
    \item[\emph{And}:]
      \vAgent{} does not have a \wit{} for a \ros{} between \(\pv{\phi}{v}\) and \(\Psi\) when \vAgent{} concludes \(\pv{\phi}{v}\) from \(\Phi\).
    \item[\emph{Then}:]
      \(\pvp{\psi}{v'}{\Psi}\) is a counterexample to \issueConstraint{}.
    \end{itenum}
    \vspace{-\baselineskip}
  \end{proposition}

  \begin{argument}{prop:tCV-WhyV-ces}
    Adapt the argument for \autoref{prop:requ-WhyV-ces} with \autoref{prop:tCV-WhyV}.
  \end{argument}
\end{note}

\subsection{Additional upshots}

\begin{note}
  \color{red}
  \tCV{2} motivates, and also takes care of a potential objection, which is somewhat standard.

  Something like, explanation supervenes on facts.
  If possible to change facts while explanation holds, then those facts are not explanatory.

  Things hinge of \requ{}.
  Change whether something is a \requ{}.
  Intuitively, agent still concludes.

  In short, it's not the case that \ros{} between \(\pv{\psi}{v'}\) and \(\Psi\).
  For, possible for \ros{} to fail while agent concludes \(\pv{\phi}{v}\) from \(\Phi\).

  In general this may be true.

  However, in this case, the agent is not \tCV{}.

  For, if \ros{} fails then not \fc{}, and hence agent is not \tCV{}.
\end{note}

\begin{note}
  Second, agency.

  \tCV{}, then agency.

  Else, it's accidental to what the agent is doing that they're \tCV{}.
  And, in some cases I think this may be right, but not in general.

  The agent \emph{is} \tCV{}.
  It is not merely the case that the agent \emph{happens to be} \tCV{}.
\end{note}



\paragraph*{Sound rules}

\begin{note}
  \phantlabel{squish-elimination-proof}

  \begin{illustration}[Squish elimination]%
    \label{scen:squish}%
    It is late morning on a sunny day.
    I ate a good breakfast, and drank some nice coffee.
    I have completed a handful of syntactic proofs for entailments of propositional logic using the basic rules of inference in a Fitch-style system.

    I create the following syntactic proof:
    \begin{center}
      \begin{fitch}
        \phantlabel{illu:sP:1}\fa (P \rightarrow Q) \rightarrow P \\
        \phantlabel{illu:sP:2}\fj R \\
        \phantlabel{illu:sP:3}\fa P & \sqE{}:\hyperref[illu:sP:1]{1} \\
        \phantlabel{illu:sP:4}\fa P \land R & \(\land\)\textbf{Intro:} \hyperref[illu:sP:2]{2},\hyperref[illu:sP:3]{3}
      \end{fitch}
    \end{center}

    Still, I haven't yet concluded \((P \rightarrow Q) \rightarrow P, R \vdash P \land R\).

    For, \sqE{} may not be a sound rule of inference.

    I go on to conclude \((P \rightarrow Q) \rightarrow P, R \vdash P \land R\).
  \end{illustration}

  \begin{definition}[\sqE{}]%
    \label{def:sque}%
    \sqE{} is the following rule:
    \begin{center}
      \begin{fitch}
        \ftag{\text{\scriptsize \emph{i}}}{\fa (\phi \rightarrow \psi) \rightarrow \phi} \\
        \ftag{\text{\scriptsize }}{\fa \vdots } \\
        \ftag{\text{\scriptsize \emph{j}}}{\fa \phi } & \sqE{}:\emph{i} \\
      \end{fitch}
    \end{center}
  \end{definition}

  \sqE{} is sound.%
  \footnote{
    \label{prop:sqE-sound}
    Rather than prove \sqE{} is sound (which would require a detailed statement of the proof system in question), we prove that the corresponding semantic entailment holds:

    Let \(v\) be an arbitrary (truth-functional) valuation, and assume \(v((\phi \rightarrow \psi) \rightarrow \phi) = \valI{True}\).
    Further, assume for contradiction \(v(\phi) = \valI{False}\).

    As \(v(\phi) = \valI{False}\), it immediately follows that \(v(\phi \rightarrow \psi) = \valI{True}\).
    Therefore, by the first assumption, it must be the case that \(v(\phi) = \valI{True}\).
    This contradictions the second assumption.
    % Hence, \((\phi \rightarrow \psi) \rightarrow \phi \vDash \phi\).
  }

  The relevant propositions, values, and \pool{1} are as follows:
  \begin{itemize}[noitemsep]
  \item
    I am the agent.
  \item
    \(\phi\) is the proposition: \(\mathsf{(P \rightarrow Q) \rightarrow P, R \vdash P \land R}\).
  \item
    \(\psi\) is the proposition: \propI{\sqE{} is sound}
  \item
    Both \(v\) and \(v'\) are the value: \valI{True}.
    And,
  \item
    The pools of premises \(\Phi\) and \(\Psi\) are left unspecified.%
    \footnote{
      \((P \rightarrow Q) \rightarrow P\) and \(R\) are premises of the deduction, and not elements of \(\Phi\).

      For, my conclusion is \pv{\mathsf{(P \rightarrow Q) \rightarrow P, R \vdash P \land R}}{\valI{True}}.
      My conclusion is not \pv{\mathsf{P \land R}}{\valI{True}}.
    }
  \item
    Key observations:
    \begin{itemize}[noitemsep]
    \item
      sunny day, good breakfast, nice coffee.
      I.e.\ as ideal as situations may be for syntactic proofs.
    \item
      Proved \sqE{} numerous times before.
    \end{itemize}
  \end{itemize}

  Hence, the relevant instance of the conditional by which a \requ{} is defined is:

  \begin{quote}
    \begin{itenum}
    \item[\emph{If}:]
      If \pv{\propI{\sqE{} is sound}}{\valI{True}} from \(\Psi\) is not a \fc{}.
    \item[\emph{Then}:]
      I am not concluding \(\pv{\mathsf{(P \rightarrow Q) \rightarrow P, R \vdash P \land R}}{\valI{True}}\) from \(\Phi\).
    \end{itenum}
  \end{quote}

  In contrast to \autoref{illu:lost-key}, \autoref{scen:squish} leads to a conclusion.

  At issue is whether \requ{}.
  By \autoref{prop:hinge}, three things.
  \begin{itemize}
  \item
    \tRep{}
  \item
    \sqE{} is \tI{} of \(T'\).
  \item
    Conditional.
  \end{itemize}

  Plausibly, four \prop{0}-\val{0}-\pool{0} pairs in type:

  \begin{center}
    \begin{tabular}{R{.45\textwidth} L{.45\textwidth}}
      \prop{2}-\val{0} pair & \pool{2} \\
      \hline
      \pv{(\phi \rightarrow \phi) \rightarrow \psi \space/\space \phi}{\textover[c]{\valI{Sound}}{\valI{Unsound}}} & \dots \\
      \pv{\phi \rightarrow (\psi \rightarrow \phi) \space/\space \phi}{\valI{Unsound}} & \dots \\
      \pv{\psi \rightarrow (\phi \rightarrow \psi) \space/\space \phi}{\valI{Unsound}} & \dots \\
      \pv{(\psi \rightarrow \phi) \rightarrow \psi \space/\space \phi}{\valI{Unsound}} & \dots \\
    \end{tabular}
  \end{center}

  Throughout \autoref{scen:squish} I \emph{know} \sqE{} is sound.
  Prior to \autoref{scen:squish} I have proved \sqE{} is sound on various occasions using the same basic observations made in the argument for \autoref{prop:sqE-sound}.

  However, there is a distinction between \emph{knowing} \sqE{} is sound and \emph{proving} \sqE{} is sound.
  For example, if I have just drunk a considerable amount of wine, or woken from a night of tormented sleep.
  Generally said, I may not be thinking straight.

  Of course, I may conclude \(\pv{\phi}{v}\) from \(\Phi\) regardless of whether \(\pv{\psi}{v'}\) from \(\Psi\) is a \fc{}.
  Indeed, given some particularly good wine I may conclude \(P \land C \vdash O\).%
  \footnote{
    For, \(P \land C\) reads `Pac', \(O\) looks like a pellet, and Pacman likes to eat pellets.
  }
  However, then not \tCV{}.
\end{note}

\begin{note}
  Note, does not depend on \agpe{my}.
\end{note}

\section*{Summary}

\begin{note}
  Introduced \requ{1}.

  Defined in terms of \tC{}.

  Motivated by \tC{}.
\end{note}

\begin{note}
  Smashing definitions.
\end{note}

\begin{note}
  Also, objection.
\end{note}

\begin{note}
  Still not \scen{1} for any of this to be amount to a counterexample to \issueConstraint{}.
\end{note}




% \begin{note}[Problems of induction]
%   Hence, the sketch does not apply to black ravens.
%   I wouldn't conclude all ravens are black if I saw a white raven.

%   I may worry about shortly seeing a white raven when concluding all ravens are black, and I may refuse to entertain the possibility that the sun will rise tomorrow when planning to mow the grass.

%   However, it's not possible to reason to seeing a white raven, nor is it possible to reason to the sun not rising tomorrow.

%   Abstractly, at issue in~\autoref{illu:lost-key} is the possibility of failing to a reason to some proposition-value pair given \emph{present} information, rather than the possibility of failing to a reason to some proposition-value pair given \emph{new} information.

%   To the extent that problems of induction arise from receiving new information, what is at issue is distinct.%
%   \footnote{
%     See~\textcite{Henderson:2020wb} for more on the problem of induction.
%   }

%   Similar points for external world scepticism.
%   Would not conclude that I have hand if disembodied brain in a vat.

%   However, conclusion is out of reach.
% \end{note}

    %   \footnote{
    %     The present point is similar to issues raised by \citeauthor{Harman:1973ww} (\citeyear{Harman:1973ww}) regarding the proposed equivalence between reasons for which an agent believes something with reasons the agent would offer if asked to justify their belief.
    %     As \citeauthor{Harman:1973ww} notes, an agent may offer reasons because they think they will convince their audience, not because the agent is compelled by the reasons, etc.
    %     (\citeyear[Ch.2]{Harman:1973ww})

    %     To the extent that \citeauthor{Harman:1973ww}'s point is that what holds from an \agpe{} need not actually be the case, the point in the same.
    %     However, to the extent that \citeauthor{Harman:1973ww} relies on an under-specification of what holds from an \agpe{} --- i.e.\ the distinction between whether \(\phi\) has value \(v\) from the \agpe{} or whether the agent evaluates as true the proposition that their audience is responsive to \(\phi\) having value \(v\), the point is distinct.
    %   }


%%% Local Variables:
%%% mode: latex
%%% TeX-master: "master"
%%% TeX-engine: luatex
%%% End:

