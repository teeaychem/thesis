\chapter{\requ{3} and \tC{}}
\label{sec:typicalRequs}


\begin{note}
  Chapter links \tC{0} and \requ{1}.
\end{note}



\section{\tC{2} and \requ{1}}
\label{sec:tc2-requ1}


\begin{note}
  \autoref{prop:hinge}, sufficient conditions for \requ{} when \tCV{}.

  \begin{proposition}[\tC{2} and \requ{1}]
    \label{prop:hinge}
    \vspace{-\baselineskip}
    \begin{itenum}
    \item[\emph{If}:]
      Conditions \ref{prop:hinge:tC} and \ref{prop:hinge:A} jointly hold:
      \begin{enumerate}[label=\arabic*., ref=\arabic*]
      \item
        \label{prop:hinge:tC}
        \(\ed{}\) is an event in which \vAgent{} is \tCp{} \(\pv{\phi}{v}\) from \(\Phi\) by type \(T\).
      \item
        \label{prop:hinge:A}
        Conditions~\ref{prop:hinge:A:rep},~\ref{prop:hinge:A:tI},~and~\ref{prop:hinge:A:novel} jointly with respect to some \se{} \(\ed{\flat}\) of \(\ed{}\):
        \begin{enumerate}[label=\alph*., ref=\theenumi\alph*]
      \item
        \label{prop:hinge:A:rep}
        \(T'\) is a \tRep{} of \vAgent{} \tCV{} \(\pv{\phi}{v}\) from \(\Phi\) by type \(T\) in \(\ed{}\).
      \item
        \label{prop:hinge:A:tI}
        \(\pvp{\psi}{v'}{\Psi}\) is a \tI{} of \(T'\)
      \item
        \label{prop:hinge:A:novel}
          For each \prop{0}-\val{0} pair \(\pv{\psi'}{v''}\) in \(\Psi\), \vAgent{} \evals{} \(\psi'\) as having value \(v''\) prior to \(\ed{\flat}\).
      \end{enumerate}
    \end{enumerate}
  \item[\emph{Then}:]
    \(\pv{\psi}{v'}\) being a \fc{} from \(\Psi\) throughout \(\ed{\flat}\) is a \requ{} of \(\ed{}\).
    \end{itenum}
  \end{proposition}

  \begin{argument}{prop:hinge}
    Assume conditions \ref{prop:hinge:tC} and \ref{prop:hinge:A} jointly hold.

    In particular, consider some \se{} \(\ed{\flat}\) of \(\ed{}\) such that conditions \ref{prop:hinge:A:rep}, \ref{prop:hinge:A:tI}, and \ref{prop:hinge:A:novel}.
    By Conditions \ref{prop:hinge:tC} \(\ed{}\) is an event in which \vAgent{} \tCp{} \(\pv{\phi}{v}\) from \(\Phi\) by type \(T\).
    And, by assumption \(\ed{\flat}\) is a \se{} of \(\ed{}\).
    So, \(\ed{\flat}\) is an event in which \vAgent{} is \tCV{} \(\pv{\phi}{v}\) from \(\Phi\) by type \(T\).
    Pairing this observation with conditions \ref{prop:hinge:A:rep}, \ref{prop:hinge:A:tI}, and \ref{prop:hinge:A:novel}, it follows \(\pv{\psi}{v'}\) is a \fc{} from \(\Psi\) throughout \(\ed{\flat}\) by \autoref{prop:tC-and-fc} (\autopageref{prop:tC-and-fc}).

    And, by assumption \(\ed{\flat}\) is a \se{} of \(\ed{}\).
    So, \(\pv{\psi}{v'}\) being a \fc{} from \(\Psi\) throughout \(\ed{\flat}\) is a \requ{} of \(\ed{}\) is immediate by \autoref{def:requ}.%
    \footnote{
      All the conditionals stated in definitions, propositions, etc.\ of this document are material conditionals.
      So, if \(A\) and \(B\) are true, \emph{if} \(A\) \emph{then} \(B\) is likewise true.
      If you don't like this appeal to the material condition, then adjust Condition~\ref{prop:hinge:A} to state that \(\ed{\flat}\) is an event in which the agent is \tC{} \(\pv{\phi}{v}\) from \(\Phi\) by type \(T\).
      \autoref{prop:tC-and-fc} ensures \(\pv{\psi}{v'}\) is a \fc{} from \(\Psi\) by the revised constraint on \(\ed{\flat}\) and conditions, \ref{prop:hinge:A:rep}, \ref{prop:hinge:A:tI}, and \ref{prop:hinge:A:novel} alone.
      Hence, holds under the additional assumption that \(\ed{\flat}\) is a \se{} of \(\ed{}\).
    }
  \end{argument}
\end{note}


\begin{note}
  Recap.

  \se{} to focus on sub-events in which agent is concluding so that if agent concludes, concludes as a result.

  \requ{} states that if \se{} then \fc{}.
  In other words, \se{} \emph{only if} \fc{}.

  Key idea with \tC{} is that agent is \tC{} only if conclusions from various \prop{0}-\val{0}-\pool{0}.
  However, as \fc{1} are defined with respect to actions, additional constraints on relevant \prop{0}-\val{0}-\pool{0} to ensure \tC{} requires the \prop{0}-\val{0}-\pool{0} is a \fc{}.

  Hence, conditions~\ref{prop:hinge:A:rep},~\ref{prop:hinge:A:tI},~and~\ref{prop:hinge:A:novel} capture sufficient conditions for \fc{}.
  And, \autoref{prop:hinge} simply observes that as \fc{}, then trivially the case that if \se{}, then \requ{}.
\end{note}



\paragraph*{Sound rules}


\begin{note}
  \phantlabel{squish-elimination-proof}

  \begin{illustration}[Squish elimination]%
    \label{scen:squish}%
    It is late morning on a sunny day.
    I ate a good breakfast, and drank some nice coffee.
    I have completed a handful of syntactic proofs for entailments of propositional logic using the basic rules of inference in a Fitch-style system.

    I create the following syntactic proof:
    \begin{center}
      \begin{fitch}
        \phantlabel{illu:sP:1}\fa (P \rightarrow Q) \rightarrow P \\
        \phantlabel{illu:sP:2}\fj R \\
        \phantlabel{illu:sP:3}\fa P & \sqE{}:\hyperref[illu:sP:1]{1} \\
        \phantlabel{illu:sP:4}\fa P \land R & \(\land\)\textbf{Intro:} \hyperref[illu:sP:2]{2},\hyperref[illu:sP:3]{3}
      \end{fitch}
    \end{center}

    Still, I haven't yet concluded \((P \rightarrow Q) \rightarrow P, R \vdash P \land R\).

    For, \sqE{} may not be a sound rule of inference.

    I go on to conclude \((P \rightarrow Q) \rightarrow P, R \vdash P \land R\).
  \end{illustration}

  \begin{definition}[\sqE{}]%
    \label{def:sque}%
    \sqE{} is the following rule:
    \begin{center}
      \begin{fitch}
        \ftag{\text{\scriptsize \emph{i}}}{\fa (\phi \rightarrow \psi) \rightarrow \phi} \\
        \ftag{\text{\scriptsize }}{\fa \vdots } \\
        \ftag{\text{\scriptsize \emph{j}}}{\fa \phi } & \sqE{}:\emph{i} \\
      \end{fitch}
    \end{center}
  \end{definition}

  \sqE{} is sound.%
  \footnote{
    \label{prop:sqE-sound}
    Rather than prove \sqE{} is sound (which would require a detailed statement of the proof system in question), we prove that the corresponding semantic entailment holds:

    Let \(v\) be an arbitrary (truth-functional) valuation, and assume \(v((\phi \rightarrow \psi) \rightarrow \phi) = \valI{True}\).
    Further, assume for contradiction \(v(\phi) = \valI{False}\).

    As \(v(\phi) = \valI{False}\), it immediately follows that \(v(\phi \rightarrow \psi) = \valI{True}\).
    Therefore, by the first assumption, it must be the case that \(v(\phi) = \valI{True}\).
    This contradictions the second assumption.
  }
  Of interest is whether the soundness of \sqE{} being a \fc{} from the agent from some \pool{} is a \requ{} of the agent \tCV{} \pv{\propM{(P \rightarrow Q) \rightarrow P, R \vdash P \land R}}{\valI{True}} from the relevant \pool{}.

  Given \autoref{prop:hinge}, we need to establish two conditions.

  The first condition is that the agent is \tC{} \pv{\propM{(P \rightarrow Q) \rightarrow P, R \vdash P \land R}}{\valI{True}}.
  This, we assume.

  The sub-conditions of the second condition.

  \begin{itemize}
  \item
    \tRep{} and \sqE{} is \tI{} of \(T'\).
    For present purposes, we need only consider \(T'\) with soundness of \sqE{}.
    There may be other types which are \tRep{1}.
    However, \autoref{prop:hinge} does not place any constraints on the \tRep{} of interest.%
    \footnote{
      For example, the type may also include \(\phi \rightarrow (\psi \rightarrow \phi) \space/\space \phi\), \(\psi \rightarrow (\phi \rightarrow \psi) \space/\space \phi\), and \((\psi \rightarrow \phi) \rightarrow \psi \space/\space \phi\) are unsound, along with various other syntactic derivations.
    }

    Plausibly.
    The agent is working through a syntactic proof.
    However, soundness of a rule is distinct from syntactic proof.
    Rather than apply syntactic rules, the relevant reasoning is semantic.
    Assume the assumptions of the rule are true for any model, and show the result of applying the rule must be true.

    However, if the agent has a basic grasp of (the meta-theory of) propositional logic, the soundness of \sqE{} is quick (see \autoref{prop:sqE-sound}, above).

    You could narrow, but common-sense.
    If fail at soundness of \sqE{} then perhaps the agent is doing a little better than creating logic soup, but the agent's reasoning is off.
  \item
    That the relevant \pool{}.
    This should be the case.
    The \agents{} understanding of (the meta-theory of) propositional logic.
  \end{itemize}

  Throughout \autoref{scen:squish} I (think I) \emph{know} \sqE{} is sound.
  Prior to \autoref{scen:squish} I have proved \sqE{} is sound on various occasions using the same basic observations made in the argument for \autoref{prop:sqE-sound}.

  However, there is a distinction between thinking one \emph{knows} \sqE{} is sound and proving \sqE{} is sound.
  I may not be thinking straight.
  For example, I may have just woken from a night of tormented sleep.

  Of course, I may conclude \(\pv{\phi}{v}\) from \(\Phi\) regardless of whether \(\pv{\psi}{v'}\) from \(\Psi\) is a \fc{}.
  Indeed, given some very tormented sleep I may conclude \pv{\phi \rightarrow (\psi \rightarrow \phi) \space/\space \phi}{\valI{Sound}}.
  However, then not \tCV{}.
  I have some logic soup, and later in the day I'll (hopefully) check my work.
\end{note}



\section{\tC{2} and \issueConstraint{}}
\label{sec:tpyically-concluding}

\begin{note}
  Understand \requ{} and counterexamples to \issueConstraint{}.

  Introduced \tCV{} to motivate \requ{}.

  In this section, recast a key proposition from \autoref{cha:requs} (\autoref{prop:requ-WhyV-ces}) in terms of \tCV{}.
\end{note}


% \begin{note}
%   \begin{definition}[The \tCon{0}]
%     \label{def:tCon}
%     \vspace{-\baselineskip}
%     \begin{itemize}
%     \item
%       The \emph{\tCon{0}} hold of \(e\) with respect to \(\langle \vAgent{}, \pvp{\phi}{v}{\Phi}, \pvp{\psi}{v'}{\Psi}, e \rangle\).
%     \end{itemize}

%     \emph{If and only if}:

%         \vspace{-1.5\baselineskip}
%   \end{definition}
% \end{note}

% \begin{note}
%   \begin{proposition}[\tCV{3} and \qWhyV{}]
%     \label{prop:tCV-WhyV}
%     \vspace{-\baselineskip}
%     \begin{itenum}
%     \item[\emph{If}:]
%       The \tCon{0} hold of \(e\) with respect to \(\langle \vAgent{}, \pvp{\phi}{v}{\Phi}, \pvp{\psi}{v'}{\Psi}, e \rangle\).
%     \item[\emph{Then}:]
%       A \ros{} between \(\pv{\psi}{v'}\) and \(\Psi\) answers \qWhyV{}.
%     \end{itenum}
%     \vspace{-1.5\baselineskip}
%   \end{proposition}

%   Instead of \requ{} directly, strengthen to \tCV{} and sufficient conditions for \requ{} given \tCV{}.

%   \begin{argument}{prop:tCV-WhyV}
%     The argument parallels the argument for \autoref{prop:requ-WhyV} (\autopageref{prop:requ-WhyV}) with minor adjustment.

%     % Rather appeal to Sub-condition~\ref{def:rCs:Cing:requ} of the \rCon{} to observe \(e^{\flat}\) is not an event in which \vAgent{} is concluding \(\pv{\phi}{v}\) from \(\Phi\), appeal to Sub-condition~\ref{def:tCon:sR} of the \tCon{} to observe \(e^{\flat}\) is not an event in which \vAgent{} is \tCV{} \(\pv{\phi}{v}\) from \(\Phi\).
%     % And, with this observation in hand appeal to Sub-condition~\ref{def:tCon:Cing:itc} to observe \(e^{\flat}\) is not, or does not develop into, an event in which \vAgent{} concludes \(\pv{\phi}{v}\) from \(\Phi\).
% \end{argument}
% \end{document}

\begin{note}
  The relevant parallel to \autoref{prop:requ-WhyV-ces}:

  \begin{proposition}[\tCV{3} and \issueConstraint{}]
    \label{prop:tCV-WhyV-ces}
    \vspace{-\baselineskip}
    \begin{itenum}
    \item[\emph{If}:]
      Conditions \ref{prop:hinge:X:tC}, \ref{prop:hinge:A}, and \ref{def:tCon:nWit} jointly hold:
      \begin{enumerate}[label=\arabic*., ref=\arabic*]
      \item
        \label{prop:hinge:X:tC}
        \(\ed{}\) is an event in which \vAgent{} is \tCp{} \(\pv{\phi}{v}\) from \(\Phi\) by type \(T\).
      \item
        \label{prop:hinge:X:A}
        Conditions~\ref{prop:hinge:X:A:rep},~\ref{prop:hinge:X:A:tI},~and~\ref{prop:hinge:X:A:novel} jointly with respect to some \se{} \(\ed{\flat}\) of \(\ed{}\):
        \begin{enumerate}[label=\alph*., ref=\theenumi\alph*]
      \item
        \label{prop:hinge:X:A:rep}
        \(T'\) is a \tRep{} of \vAgent{} \tCV{} \(\pv{\phi}{v}\) from \(\Phi\) by type \(T\) in \(\ed{}\).
      \item
        \label{prop:hinge:X:A:tI}
        \(\pvp{\psi}{v'}{\Psi}\) is a \tI{} of \(T'\)
      \item
        \label{prop:hinge:X:A:novel}
        For each \prop{0}-\val{0} pair \(\pv{\psi'}{v''}\) in \(\Psi\), \vAgent{} \evals{} \(\psi'\) as having value \(v''\) prior to \(\ed{\flat}\).
      \end{enumerate}
    \item
      \label{def:tCon:nWit}
        \vAgent{} does not have a \wit{} for a \ros{} between \(\pv{\psi}{v'}\) and \(\Psi\) when \vAgent{} concludes \(\pv{\phi}{v}\) from \(\Phi\).
      \end{enumerate}
    \item[\emph{Then}:]
      \issueConstraint{} does not hold.
    \end{itenum}
    \vspace{-\baselineskip}
  \end{proposition}

  \begin{argument}{prop:tCV-WhyV-ces}
    Assume conditions \ref{prop:hinge:X:tC}, \ref{prop:hinge:A}, and \ref{def:tCon:nWit} jointly hold.

    From conditions \ref{prop:hinge:X:tC}, \ref{prop:hinge:A}, and \autoref{prop:hinge} (\autopageref{prop:hinge}) it follows there is some \se{} \(\ed{\flat}\) of \(\ed{}\) such that \(\pv{\psi}{v'}\) being a \fc{} from \(\Psi\) throughout \(\ed{\flat}\) is a \requ{} of \(\ed{}\).

    So, from this observation and Condition~\ref{prop:hinge:X:tC}, a \ros{} between \(\pv{\psi}{v'}\) and \(\Psi\) answers \qWhyV{} by \autoref{prop:requ-WhyV} (\autopageref{prop:requ-WhyV}).

    However, by Condition~\ref{def:tCon:nWit}, the agent does not have a \wit{} for the \ros{} between \(\pv{\psi}{v'}\) and \(\Psi\) when \vAgent{} concludes \(\pv{\phi}{v}\) from \(\Phi\).

    So, \issueConstraint{} fails to hold.
  \end{argument}
\end{note}




\section*{Summary}

\begin{note}
  Introduced \requ{1}.

  Defined in terms of \tC{}.

  Motivated by \tC{}.
\end{note}

\begin{note}
  Smashing definitions.
\end{note}

\begin{note}
  Also, objection.
\end{note}

\begin{note}
  Still not \scen{1} for any of this to be amount to a counterexample to \issueConstraint{}.
\end{note}




% \begin{note}[Problems of induction]
%   Hence, the sketch does not apply to black ravens.
%   I wouldn't conclude all ravens are black if I saw a white raven.

%   I may worry about shortly seeing a white raven when concluding all ravens are black, and I may refuse to entertain the possibility that the sun will rise tomorrow when planning to mow the grass.

%   However, it's not possible to reason to seeing a white raven, nor is it possible to reason to the sun not rising tomorrow.

%   Abstractly, at issue in~\autoref{illu:lost-key} is the possibility of failing to a reason to some proposition-value pair given \emph{present} information, rather than the possibility of failing to a reason to some proposition-value pair given \emph{new} information.

%   To the extent that problems of induction arise from receiving new information, what is at issue is distinct.%
%   \footnote{
%     See~\textcite{Henderson:2020wb} for more on the problem of induction.
%   }

%   Similar points for external world scepticism.
%   Would not conclude that I have hand if disembodied brain in a vat.

%   However, conclusion is out of reach.
% \end{note}

    %   \footnote{
    %     The present point is similar to issues raised by \citeauthor{Harman:1973ww} (\citeyear{Harman:1973ww}) regarding the proposed equivalence between reasons for which an agent believes something with reasons the agent would offer if asked to justify their belief.
    %     As \citeauthor{Harman:1973ww} notes, an agent may offer reasons because they think they will convince their audience, not because the agent is compelled by the reasons, etc.
    %     (\citeyear[Ch.2]{Harman:1973ww})

    %     To the extent that \citeauthor{Harman:1973ww}'s point is that what holds from an \agpe{} need not actually be the case, the point in the same.
    %     However, to the extent that \citeauthor{Harman:1973ww} relies on an under-specification of what holds from an \agpe{} --- i.e.\ the distinction between whether \(\phi\) has value \(v\) from the \agpe{} or whether the agent evaluates as true the proposition that their audience is responsive to \(\phi\) having value \(v\), the point is distinct.
    %   }


%%% Local Variables:
%%% mode: latex
%%% TeX-master: "master"
%%% TeX-engine: luatex
%%% End:

