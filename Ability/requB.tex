\chapter{\tC{2} and \requ{1}}
\label{sec:typicalRequs}


\begin{note}
  This chapter links \tC{} to \requ{}, and focuses on two propositions.

  The first is a sufficient condition for a \requ{} when an agent is \tC{}.

  The second is a variation on \autoref{prop:requ-WhyV-ces} from \autoref{cha:requs} and characterises when \issueConstraint{} fails to hold via an agent \tCV{} \(\pv{\phi}{v}\) from \(\Phi\).
\end{note}


\begin{note}
  At the close of the chapter all the idea relating to counterexamples to \issueConstraint{} will have been stated, and the following chapter (\autoref{cha:ces}) provides some counter-samples.
\end{note}



\section{\tC{2}, \requ{1}}
\label{sec:tc2-requ1}


\begin{note}
  \autoref{prop:hinge}, sufficient conditions for \requ{} when \tCV{}.

  \begin{proposition}[\tC{2} and \requ{1}]
    \label{prop:hinge}
    \vspace{-\baselineskip}
    \begin{itenum}
    \item[\emph{If}:]
      Conditions \ref{prop:hinge:tC} and \ref{prop:hinge:A} jointly hold:
      \begin{enumerate}[label=\arabic*., ref=\arabic*]
      \item
        \label{prop:hinge:tC}
        \(\ed{}\) is an event in which \vAgent{} is \tCp{} \(\pv{\phi}{v}\) from \(\Phi\) by type \(T\).
      \item
        \label{prop:hinge:A}
        Conditions~\ref{prop:hinge:A:rep}, and \ref{prop:hinge:A:tI} jointly with respect to some \se{} \(\ed{\flat}\) of \(\ed{}\):
        \begin{enumerate}[label=\alph*., ref=\theenumi\alph*]
      \item
        \label{prop:hinge:A:rep}
        \(T'\) is a \tRep{} of \vAgent{} \tCV{} \(\pv{\phi}{v}\) from \(\Phi\) by type \(T\) in \(\ed{}\).
      \item
        \label{prop:hinge:A:tI}
        \(\pvp{\psi}{v'}{\Psi}\) is a \tI{} of \(T'\),
      \end{enumerate}
    \end{enumerate}
  \item[\emph{Then}:]
    \(\pv{\psi}{v'}\) being a \fc{} from \(\Psi\) throughout \(\ed{\flat}\) is a \requ{} of \(\ed{}\).
    \end{itenum}
  \end{proposition}

  \noindent%
  \autoref{prop:hinge} is a more-or-less a direct consequence of \autoref{prop:tC-and-fc} (\autopageref{prop:tC-and-fc}).

  In short, given \autoref{prop:tC-and-fc}, conditions~\ref{prop:hinge:A:rep} and \ref{prop:hinge:A:tI} capture sufficient conditions for \fc{}.
  Hence, it is trivially the case that if the event to which conditions~\ref{prop:hinge:A:rep} and \ref{prop:hinge:A:tI} is also a \se{} of an event in which the agent \tCV{} \(\pv{\phi}{v}\) from \(\Phi\), then the \fc{} satisfies the definition of a \requ{}.

  In long:

  \begin{argument}{prop:hinge}
    Assume conditions \ref{prop:hinge:tC} and \ref{prop:hinge:A} jointly hold.

    In particular, consider some \se{} \(\ed{\flat}\) of \(\ed{}\) such that conditions \ref{prop:hinge:A:rep} and \ref{prop:hinge:A:tI}hold.
    By Conditions \ref{prop:hinge:tC} \(\ed{}\) is an event in which \vAgent{} \tCp{} \(\pv{\phi}{v}\) from \(\Phi\) by type \(T\).
    And, by assumption \(\ed{\flat}\) is a \se{} of \(\ed{}\).
    So, \(\ed{\flat}\) is an event in which \vAgent{} is \tCV{} \(\pv{\phi}{v}\) from \(\Phi\) by type \(T\).
    Pairing this observation with conditions \ref{prop:hinge:A:rep} and \ref{prop:hinge:A:tI}, it follows \(\pv{\psi}{v'}\) is a \fc{} from \(\Psi\) throughout \(\ed{\flat}\) by \autoref{prop:tC-and-fc}.

    And, by assumption \(\ed{\flat}\) is a \se{} of \(\ed{}\).
    So, \(\pv{\psi}{v'}\) being a \fc{} from \(\Psi\) throughout \(\ed{\flat}\) is a \requ{} of \(\ed{}\) is immediate by \autoref{def:requ}.%
    \footnote{
      All the conditionals stated in definitions, propositions, etc.\ of this document are material conditionals.
      So, if \(A\) and \(B\) are true, \emph{if} \(A\) \emph{then} \(B\) is likewise true.
      If you don't like this appeal to the material condition, then adjust Condition~\ref{prop:hinge:A} to state that \(\ed{\flat}\) is an event in which the agent is \tC{} \(\pv{\phi}{v}\) from \(\Phi\) by type \(T\).
      \autoref{prop:tC-and-fc} ensures \(\pv{\psi}{v'}\) is a \fc{} from \(\Psi\) by the revised constraint on \(\ed{\flat}\) and conditions \ref{prop:hinge:A:rep} and \ref{prop:hinge:A:tI} alone.
      Hence, holds under the additional assumption that \(\ed{\flat}\) is a \se{} of \(\ed{}\).
    }
  \end{argument}
\end{note}


\begin{note}
  Paired with \autoref{prop:requ-WhyV} (\autopageref{prop:requ-WhyV}), \autoref{prop:hinge} provides sufficient conditions for answers to \qWhyV{}.
  Still, our key interest is with the failure of \issueConstraint{}, and \autoref{prop:hinge}, likewise, provides sufficient conditions:

    \begin{proposition}[\tCV{3} and \issueConstraint{}]
    \label{prop:tCV-WhyV-ces}
    \vspace{-\baselineskip}
    \begin{itenum}
    \item[\emph{If}:]
      Conditions \ref{prop:hinge:X:tC}, \ref{prop:hinge:A}, and \ref{def:tCon:nWit} jointly hold:
      \begin{enumerate}[label=\arabic*., ref=\arabic*]
      \item
        \label{prop:hinge:X:tC}
        \(\ed{}\) is an event in which \vAgent{} is \tCp{} \(\pv{\phi}{v}\) from \(\Phi\) by type \(T\).
      \item
        \label{prop:hinge:X:A}
        Conditions~\ref{prop:hinge:X:A:rep} and \ref{prop:hinge:X:A:tI} jointly with respect to some \se{} \(\ed{\flat}\) of \(\ed{}\):
        \begin{enumerate}[label=\alph*., ref=\theenumi\alph*]
      \item
        \label{prop:hinge:X:A:rep}
        \(T'\) is a \tRep{} of \vAgent{} \tCV{} \(\pv{\phi}{v}\) from \(\Phi\) by type \(T\) in \(\ed{}\).
      \item
        \label{prop:hinge:X:A:tI}
        \(\pvp{\psi}{v'}{\Psi}\) is a \tI{} of \(T'\)
      \end{enumerate}
    \item
      \label{def:tCon:nWit}
        \vAgent{} does not have a \wit{} for a \ros{} between \(\pv{\psi}{v'}\) and \(\Psi\) when \vAgent{} concludes \(\pv{\phi}{v}\) from \(\Phi\).
      \end{enumerate}
    \item[\emph{Then}:]
      \issueConstraint{} does not hold.
    \end{itenum}
    \vspace{-\baselineskip}
  \end{proposition}

  \begin{argument}{prop:tCV-WhyV-ces}
    Assume conditions \ref{prop:hinge:X:tC}, \ref{prop:hinge:A}, and \ref{def:tCon:nWit} jointly hold.

    From conditions \ref{prop:hinge:X:tC}, \ref{prop:hinge:A}, and \autoref{prop:hinge} (\autopageref{prop:hinge}) it follows there is some \se{} \(\ed{\flat}\) of \(\ed{}\) such that \(\pv{\psi}{v'}\) being a \fc{} from \(\Psi\) throughout \(\ed{\flat}\) is a \requ{} of \(\ed{}\).

    So, from this observation and Condition~\ref{prop:hinge:X:tC}, a \ros{} between \(\pv{\psi}{v'}\) and \(\Psi\) answers \qWhyV{} by \autoref{prop:requ-WhyV} (\autopageref{prop:requ-WhyV}).

    However, by Condition~\ref{def:tCon:nWit}, the agent does not have a \wit{} for the \ros{} between \(\pv{\psi}{v'}\) and \(\Psi\) when \vAgent{} concludes \(\pv{\phi}{v}\) from \(\Phi\).

    So, \issueConstraint{} fails to hold.
  \end{argument}
\end{note}





\section{An \illu{0}}


\begin{note}
  \autoref{prop:hinge} may be applied to the \scen{1} of \autoref{cha:typical}  used to illustrate \tCN{}.
  
\end{note}


\begin{note}
  \phantlabel{squish-elimination-proof}

  \begin{scenario}[Squish elimination]%
    \label{scen:squish}%
    Some time ago the agent showed \sqE{} is sound.

    It is now late morning on a sunny day.
    The agent ate a good breakfast, and drank some nice coffee and does the following syntactic proof:
    %
    \begin{center}
      \begin{fitch}
        \phantlabel{illu:sP:1}\fa (P \rightarrow Q) \rightarrow P \\
        \phantlabel{illu:sP:2}\fj R \\
        \phantlabel{illu:sP:3}\fa P & \sqE{}:\hyperref[illu:sP:1]{1} \\
        \phantlabel{illu:sP:4}\fa P \land R & \(\land\)\textbf{Intro:} \hyperref[illu:sP:2]{2},\hyperref[illu:sP:3]{3}
      \end{fitch}
    \end{center}
    %
    The agent concludes \((P \rightarrow Q) \rightarrow P, R \vdash P \land R\).
  \end{scenario}

  \noindent%
  Intuitively, agent \tCN{} \((P \rightarrow Q) \rightarrow P, R \vdash P \land R\) from some \pool{} which captures the \agents{} understanding of the relevant Fitch-style proof system \emph{and} the \torN{} captures the \agents{} understanding of the relevant Fitch-style proof system.



  
  Still, the agent uses \sqE{}.
  And, \sqE{} is a non-standard rule.%
  \footnote{
    \label{prop:sqE-sound}
    Rather than prove \sqE{} is sound (which would require a detailed statement of the proof system in question), we show that the key corresponding semantic entailment holds:

    Let \(v\) be an arbitrary (truth-functional) valuation, and assume \(v((\phi \rightarrow \psi) \rightarrow \phi) = \valI{True}\).
    Further, assume for contradiction \(v(\phi) = \valI{False}\).

    As \(v(\phi) = \valI{False}\), it immediately follows that \(v(\phi \rightarrow \psi) = \valI{True}\).
    Therefore, by the first assumption, it must be the case that \(v(\phi) = \valI{True}\).
    This contradictions the second assumption.
  }

  Specifically:

  \begin{definition}[\sqE{}]%
    \label{def:sque}%
    \sqE{} is the following rule:
    \begin{center}
      \begin{fitch}
        \ftag{\text{\scriptsize \emph{i}}}{\fa (\phi \rightarrow \psi) \rightarrow \phi} \\
        \ftag{\text{\scriptsize }}{\fa \vdots } \\
        \ftag{\text{\scriptsize \emph{j}}}{\fa \phi } & \sqE{}:\emph{i} \\
      \end{fitch}
    \end{center}
  \end{definition}

  If the agent concludes by their understanding of the relevant Fitch-style proof system, then \sqE{} follows from their understanding of the proof system.
  Plausibly, to be \tC{}, \sqE{} follows from the \agents{} understanding of the proof system.

  \autoref{scen:squish} states the agent has showed \sqE{} is sound.
  However, \dots

  \begin{illustration}[Squash elimination]%
    \label{scen:squish}%
    Some time ago the agent showed \sqE{} is sound.

    It is now late morning on a sunny day.
    The agent ate a good breakfast, and drank some nice coffee and does the following syntactic proof:
    %
    \begin{center}
      \begin{fitch}
        \phantlabel{illu:sP:1}\fa P \rightarrow (Q \rightarrow P) \\
        \phantlabel{illu:sP:2}\fj R \\
        \phantlabel{illu:sP:3}\fa P & \sqE{}:\hyperref[illu:sP:1]{1} \\
        \phantlabel{illu:sP:4}\fa P \land R & \(\land\)\textbf{Intro:} \hyperref[illu:sP:2]{2},\hyperref[illu:sP:3]{3}
      \end{fitch}
    \end{center}
    %
    The agent concludes \(P \rightarrow (Q \rightarrow P), R \vdash P \land R\).
  \end{illustration}

  \(P \rightarrow (Q \rightarrow P), R \vdash P \land R\) is logic soup.
  The \agents{} application of \sqE{} is not an instance of \sqE{}.
  Instead, it is an instance of \sqaE{}.%
  \footnote{
    In shorthand: \(\phi \rightarrow (\psi \rightarrow \phi) \space/\space \phi\).
  }
  And, \sqaE{} is not sound.%
  \footnote{
    If \(v(P) = \valI{False}\) then \(v(P \rightarrow (Q \rightarrow P)) = \valI{True}\).
    So, \(P\) does not follow from \(P \rightarrow (Q \rightarrow P)\).
  }
  The same is true of \sqoE{} and \sqeE{}.%
  \footnote{
    \(\psi \rightarrow (\phi \rightarrow \psi) \space/\space \phi\), and \((\psi \rightarrow \phi) \rightarrow \psi \space/\space \phi\), respectively.
  }
\end{note}


\begin{note}
  Intuitively, \tCV{}.
  For some \tRep{}, soundness of \sqE{} is a \tI{} of the \tRep{} and likewise for \sqaE{}, \sqoE{}, and \sqeE{} being unsound.

  Given \autoref{prop:hinge}, we need to establish two conditions.

  The first condition is that the agent is \tC{} \pv{\propM{(P \rightarrow Q) \rightarrow P, R \vdash P \land R}}{\valI{True}}.
  This is given.

  The sub-conditions of the second condition.

  \begin{itemize}
  \item
    \tRep{} and \sqE{} is \tI{} of \(T'\).
    For present purposes, we need only consider \(T'\) with soundness of \sqE{}.
    There may be other types which are \tRep{1}.
    However, \autoref{prop:hinge} does not place any constraints on the \tRep{} of interest.%
    \footnote{
      For example, the type may also include  are unsound, along with various other syntactic derivations.
    }

    Plausibly.

      Two options.
  Either directly show \sqE{} follows from the \agents{} understanding of the proof system via constructing a meta-proof of \sqE{} or indirectly show \sqE{} follows by providing a semantic argument and combine with completeness result.
  If neither direct or indirect, then the \agents{} use of \sqE{} does not follow from the \agents{} understanding of relevant Fitch-style proof system.
  \item
    That the relevant \pool{}.
    This should be the case.
    The \agents{} understanding of (the meta-theory of) propositional logic.
  \end{itemize}
\end{note}



\begin{note}
  So, by \autoref{prop:tC-and-fc}, \fc{}.
  Hence, by \autoref{prop:hinge} a \requ{}.
  
\end{note}



\begin{note}
  \autoref{def:sque} is the same as 
\end{note}



\section*{Summary}

\begin{note}
  Introduced \requ{1}.

  Defined in terms of \tC{}.

  Motivated by \tC{}.
\end{note}

\begin{note}
  Smashing definitions.
\end{note}

\begin{note}
  Also, objection.
\end{note}

\begin{note}
  Still not \scen{1} for any of this to be amount to a counterexample to \issueConstraint{}.
\end{note}




% \begin{note}[Problems of induction]
%   Hence, the sketch does not apply to black ravens.
%   I wouldn't conclude all ravens are black if I saw a white raven.

%   I may worry about shortly seeing a white raven when concluding all ravens are black, and I may refuse to entertain the possibility that the sun will rise tomorrow when planning to mow the grass.

%   However, it's not possible to reason to seeing a white raven, nor is it possible to reason to the sun not rising tomorrow.

%   Abstractly, at issue in~\autoref{illu:lost-key} is the possibility of failing to a reason to some proposition-value pair given \emph{present} information, rather than the possibility of failing to a reason to some proposition-value pair given \emph{new} information.

%   To the extent that problems of induction arise from receiving new information, what is at issue is distinct.%
%   \footnote{
%     See~\textcite{Henderson:2020wb} for more on the problem of induction.
%   }

%   Similar points for external world scepticism.
%   Would not conclude that I have hand if disembodied brain in a vat.

%   However, conclusion is out of reach.
% \end{note}

    %   \footnote{
    %     The present point is similar to issues raised by \citeauthor{Harman:1973ww} (\citeyear{Harman:1973ww}) regarding the proposed equivalence between reasons for which an agent believes something with reasons the agent would offer if asked to justify their belief.
    %     As \citeauthor{Harman:1973ww} notes, an agent may offer reasons because they think they will convince their audience, not because the agent is compelled by the reasons, etc.
    %     (\citeyear[Ch.2]{Harman:1973ww})

    %     To the extent that \citeauthor{Harman:1973ww}'s point is that what holds from an \agpe{} need not actually be the case, the point in the same.
    %     However, to the extent that \citeauthor{Harman:1973ww} relies on an under-specification of what holds from an \agpe{} --- i.e.\ the distinction between whether \(\phi\) has value \(v\) from the \agpe{} or whether the agent evaluates as true the proposition that their audience is responsive to \(\phi\) having value \(v\), the point is distinct.
    %   }


%%% Local Variables:
%%% mode: latex
%%% TeX-master: "master"
%%% TeX-engine: luatex
%%% End:

