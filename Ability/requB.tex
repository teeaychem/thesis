\chapter{\tC{2} and \requ{1}}
\label{sec:typicalRequs}


\begin{note}
  This chapter links \tC{} to \requ{1}, and hence \tC{} to counterexamples to \issueInclusion{}.
  The content of this chapter amounts to little more than connecting \autoref{prop:tCV-fc} (\autopageref{prop:tCV-fc}) from \autoref{cha:typical} with the results from \autoref{cha:requs}.

  Given the thematic change the links have been separated from the discussion of \tCV{} in \autoref{cha:typical} and the counter-samples to \issueInclusion{} to follow in \autoref{cha:ces}.
\end{note}



\section{\tC{2}, \requ{1}}
\label{sec:tc2-requ1}


\begin{note}
  \autoref{prop:hinge}, sufficient conditions for \requ{1} in connexion to an agent \tCV{}.

  \begin{proposition}[\tC{2} and \requ{1}]
    \label{prop:hinge}
    \vspace{-\baselineskip}
    \begin{itenum}
    \item[\emph{If}:]
      Conditions \ref{prop:hinge:tC} and \ref{prop:hinge:e} hold:
      \begin{enumerate}[label=\arabic*., ref=\arabic*]
      \item
        \(\ed{}\) is an event in which \vAgent{} is \tCs{} \(\pv{\phi}{v}\) from \(\Phi\) by \torNa{} \(T\).
      \item
        \label{prop:hinge:tC}
        \(\ed{\flat}\) is an event in which \vAgent{} is \tCV{} \(\pv{\phi}{v}\) from \(\Phi\) by \torNa{} \(T\).
      \item
        \label{prop:hinge:e:act:se}
        \(\ed{i}\) is a \se{} of \(\ed{}\).
      \item
        \label{prop:hinge:e}
        For some event \(\edn{i}\) in some partition of \(\ed{}\) into sub-events \(\edn{1}, \dots, \edn{k}\) and description \(\edo{i}\):
        \begin{enumerate}[label=\roman*., ref=\theenumi\roman*]
        \item
          There is some possible event \(\edn{\sharp}\) in \(\mathcal{E}\) and \prop{0}-\val{0}-\pool{0} pairing \(\pvp{\psi}{v'}{\Psi}\) in \(\mathcal{X}\) such that:
          \begin{enumerate}[label=\alph*., ref=\theenumi\theenumii\alph*]
          \item
            \label{prop:hinge:e:act:i}
            \(\edn{\sharp}\) is the result of an action \(a\) done by \vAgent{} in \(\ed{i}\)
          \item
            \label{prop:hinge:e:act:ii}
            \vAgent{} \evals{} \(\psi'\) as having value \(v''\) prior to doing \(a\), for each \prop{0}-\val{0} pair \(\pv{\psi'}{v''}\) in \(\Psi\).
          \end{enumerate}
        \end{enumerate}
      \end{enumerate}
    \item[\emph{Then}:]
        \(\pv{\psi}{v'}\) being a \fc{} from \(\Psi\) through \(\ed{i}\) is a \requ{} of \(\ed{}\).
    \end{itenum}
    \vspace{-\baselineskip}
  \end{proposition}

  \begin{argument}{prop:hinge}
    We show clauses~\ref{def:requ:se} and~\ref{def:requ:fc} of the definition of a \requ{} (\autoref{def:requ}, \autopageref{def:requ}) are satisfied with respect to \(\pv{\psi}{v'}\), \(\Psi\), and \(\ed{i}\).
    \medskip

    \noindent%
    Clause~\ref{def:requ:se} is satisfied by assumption.
    For, by Clause~\ref{prop:hinge:e:act:se} \(\ed{i}\) is a \se{} of \(\ed{}\).
    \medskip

    \noindent%
    Clause~\ref{def:requ:fc} is satisfied by observing conditions \ref{prop:hinge:tC} and \ref{prop:hinge:e} include the relevant conditions of \autoref{prop:tCV-fc} (\autopageref{prop:tCV-fc}).
    Hence, \(\pv{\psi}{v'}\) is a \fc{0} from \(\Psi\) for the agent through \(\ed{i}\).
    \medskip

    \noindent%
    So, \(\pv{\psi}{v'}\) being a \fc{} from \(\Psi\) through \(\ed{i}\) is a \requ{} of \(\ed{}\) by \autoref{def:requ}.
  \end{argument}
\end{note}



\section{\tCV{2} and \issueInclusion{}}


\begin{note}
  Paired with \autoref{prop:requ-WhyV} (\autopageref{prop:requ-WhyV}), \autoref{prop:hinge} provides sufficient conditions for answers to \qWhy{}.
  Still, our main interest is with the failure of \issueInclusion{}, and \autoref{prop:hinge}, likewise, provides sufficient conditions:

  \begin{proposition}[\tCV{3} and \issueInclusion{}]
    \label{prop:tCV-WhyV-ces}
    \vspace{-\baselineskip}
    \begin{itenum}
    \item[\emph{If}:]
      Conditions \ref{prop:hinge:X:tC} and \ref{prop:hinge:X:e} hold:
      \begin{enumerate}[label=\arabic*., ref=\arabic*]
      \item
        \label{prop:hinge:X:tC}
        \(\ed{}\) is an event in which \vAgent{} is \tCV{} \(\pv{\phi}{v}\) from \(\Phi\) by \torNa{} \(T\).
      \item
        \label{prop:hinge:X:e}
        For some event \(\edn{i}\) in some partition of \(\ed{}\) into sub-events \(\edn{1}, \dots, \edn{k}\) and description \(\edo{i}\):
        \begin{enumerate}[label=\roman*., ref=\theenumi\roman*]
        \item
          There is some possible event \(\edn{\sharp}\) in \(\mathcal{E}\) and \prop{0}-\val{0}-\pool{0} pairing \(\pvp{\psi}{v'}{\Psi}\) in \(\mathcal{X}\) such that:
          \begin{enumerate}[label=\alph*., ref=\theenumi\theenumii\alph*]
          \item
            \label{prop:hinge:X:e:act:i}
            \(\edn{\sharp}\) is the result of an action \(a\) done by \vAgent{} in \(\ed{}\)
          \item
            \label{prop:hinge:X:e:act:ii}
            \vAgent{} \evals{} \(\psi'\) as having value \(v''\) prior to doing \(a\), for each \prop{0}-\val{0} pair \(\pv{\psi'}{v''}\) in \(\Psi\).
          \end{enumerate}
        \item
          \label{prop:hinge:X:e:se}
          \(\edn{i}\) is a \se{} of \(\ed{}\).
        \item
          \label{def:tCon:nWit}
          \vAgent{} does not have a \wit{} for a \ros{} between \(\pv{\psi}{v'}\) and \(\Psi\) when \vAgent{} concludes \(\pv{\phi}{v}\) from \(\Phi\).
        \end{enumerate}
      \end{enumerate}
    \item[\emph{Then}:]
      \issueInclusion{} does not hold.
    \end{itenum}
    \vspace{-\baselineskip}
  \end{proposition}

  \noindent%
  The antecedent of \autoref{prop:tCV-WhyV-ces} includes the same conditions as \autoref{prop:hinge} and differs only by extending Condition~\ref{prop:hinge:e} to assume the agent does not have a \wit{} for the relevant \ros{}.
  And, the consequent of \autoref{prop:tCV-WhyV-ces} is more-or-less immediate from prior definitions.

  \begin{argument}{prop:tCV-WhyV-ces}
    Suppose conditions \ref{prop:hinge:X:tC} and \ref{prop:hinge:X:e} hold.

    Conditions \ref{prop:hinge:X:tC} and \ref{prop:hinge:X:e} of \autoref{prop:tCV-WhyV-ces} include conditions \ref{prop:hinge:tC} and \ref{prop:hinge:e} of \autoref{prop:hinge}.
    Hence, we may repeat the argument for \autoref{prop:hinge} to establish \(\pv{\psi}{v'}\) being a \fc{} from \(\Psi\) through \(\ed{i}\) is a \requ{} of \(\ed{}\).
    And, by Condition~\ref{def:tCon:nWit} the agent does not have a \wit{} for a \ros{} between \(\pv{\psi}{v'}\) and \(\Psi\) when the agent concludes \(\pv{\phi}{v}\) from \(\Phi\).
    So, \issueInclusion{} fails to hold.
  \end{argument}
\end{note}



\section*{Summary}





% \begin{note}[Problems of induction]
%   Hence, the sketch does not apply to black ravens.
%   I wouldn't conclude all ravens are black if I saw a white raven.

%   I may worry about shortly seeing a white raven when concluding all ravens are black, and I may refuse to entertain the possibility that the sun will rise tomorrow when planning to mow the grass.

%   However, it's not possible to reason to seeing a white raven, nor is it possible to reason to the sun not rising tomorrow.

%   Abstractly, at issue in~\autoref{illu:lost-key} is the possibility of failing to a reason to some proposition-value pair given \emph{present} information, rather than the possibility of failing to a reason to some proposition-value pair given \emph{new} information.

%   To the extent that problems of induction arise from receiving new information, what is at issue is distinct.%
%   \footnote{
%     See~\textcite{Henderson:2020wb} for more on the problem of induction.
%   }

%   Similar points for external world scepticism.
%   Would not conclude that I have hand if disembodied brain in a vat.

%   However, conclusion is out of reach.
% \end{note}

    %   \footnote{
    %     The present point is similar to issues raised by \citeauthor{Harman:1973ww} (\citeyear{Harman:1973ww}) regarding the proposed equivalence between reasons for which an agent believes something with reasons the agent would offer if asked to justify their belief.
    %     As \citeauthor{Harman:1973ww} notes, an agent may offer reasons because they think they will convince their audience, not because the agent is compelled by the reasons, etc.
    %     (\citeyear[Ch.2]{Harman:1973ww})

    %     To the extent that \citeauthor{Harman:1973ww}'s point is that what holds from an \agpe{} need not actually be the case, the point in the same.
    %     However, to the extent that \citeauthor{Harman:1973ww} relies on an under-specification of what holds from an \agpe{} --- i.e.\ the distinction between whether \(\phi\) has value \(v\) from the \agpe{} or whether the agent evaluates as true the proposition that their audience is responsive to \(\phi\) having value \(v\), the point is distinct.
    %   }


%%% Local Variables:
%%% mode: latex
%%% TeX-master: "master"
%%% TeX-engine: luatex
%%% End:

