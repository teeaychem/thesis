\chapter{\requ{3} and \tC{}}
\label{sec:typicalRequs}


\begin{note}
  Chapter links \tC{0} and \requ{1}.
\end{note}



\section{\tC{2} and \requ{1}}
\label{sec:tc2-requ1}


\begin{note}
  \autoref{prop:hinge}, sufficient conditions for \requ{} when \tCV{}.

  \begin{proposition}[\tC{2} and \requ{1}]
    \label{prop:hinge}
    \vspace{-\baselineskip}
    \begin{itenum}
    \item[\emph{If}:]
      Conditions~\ref{prop:hinge:A:rep},~\ref{prop:hinge:A:tI},~and~\ref{prop:hinge:A:novel} jointly with respect to \(\ed{}\):
      \begin{enumerate}[label=\arabic*., ref=\arabic*]
      \item
        \label{prop:hinge:A:rep}
        \(T'\) is a \tRep{} of \vAgent{} \tCV{} \(\pv{\phi}{v}\) from \(\Phi\) by type \(T\) in \(\ed{}\).
      \item
        \label{prop:hinge:A:tI}
        \(\pvp{\psi}{v'}{\Psi}\) is a \tI{} of \(T'\)
      \item
        \label{prop:hinge:A:novel}
        {
          \color{red}
          For each \prop{0}-\val{0} pair \(\pv{\phi'}{v'}\) in \(\Phi\), \vAgent{} \evals{} \(\phi'\) as having value \(v'\).
        }
      \end{enumerate}
    \item[\emph{Then}:]
      \(\pv{\psi}{v'}\) being a \fc{} from \(\Psi\) is a \requ{} of \(\ed{}\) being an event in which \vAgent{} is \tCV{} \(\pv{\phi}{v}\) from \(\Phi\).
    \end{itenum}
    \vspace{-\baselineskip}
  \end{proposition}

  \begin{argument}{prop:hinge}
    \color{blue}
    Assume conditions~\ref{prop:hinge:A:rep},~\ref{prop:hinge:A:tI},~and~\ref{prop:hinge:A:novel} jointly hold.

    Assume \(\ed{}\) is an event in which the agent is \tCV{} \(\pv{\phi}{v}\) from \(\Phi\).
    By \autoref{prop:tC-and-fc} (\autopageref{prop:tC-and-fc}), \(\pv{\phi}{v}\) is a \fc{} from \(\Phi\) throughout \(\ed{}\).

    Hence, consider an event \(\ed{\flat}\) such that \(\ed{\flat}\) is a \se{} of \(\ed{}\).
    Then, by \autoref{def:requ}, \(\pv{\phi}{v}\) is a \fc{} from \(\Phi\) is a \requ{}.
  \end{argument}
\end{note}


\begin{note}
  Recap.

  \se{} to focus on sub-events in which agent is concluding so that if agent concludes, concludes as a result.

  \requ{} states that if \se{} then \fc{}.
  In other words, \se{} \emph{only if} \fc{}.

  Key idea with \tC{} is that agent is \tC{} only if conclusions from various \prop{0}-\val{0}-\pool{0}.
  However, as \fc{1} are defined with respect to actions, additional constraints on relevant \prop{0}-\val{0}-\pool{0} to ensure \tC{} requires the \prop{0}-\val{0}-\pool{0} is a \fc{}.

  Hence, clauses~\ref{prop:hinge:A:rep},~\ref{prop:hinge:A:tI},~and~\ref{prop:hinge:A:novel} capture sufficient conditions for \fc{}.
  And, \autoref{prop:hinge} simply observes that as \fc{}, then trivially the case that if \se{}, then \requ{}.
\end{note}


\paragraph*{Sound rules}

\begin{note}
  \color{red}
\end{note}

\begin{note}
  \phantlabel{squish-elimination-proof}

  \begin{illustration}[Squish elimination]%
    \label{scen:squish}%
    It is late morning on a sunny day.
    I ate a good breakfast, and drank some nice coffee.
    I have completed a handful of syntactic proofs for entailments of propositional logic using the basic rules of inference in a Fitch-style system.

    I create the following syntactic proof:
    \begin{center}
      \begin{fitch}
        \phantlabel{illu:sP:1}\fa (P \rightarrow Q) \rightarrow P \\
        \phantlabel{illu:sP:2}\fj R \\
        \phantlabel{illu:sP:3}\fa P & \sqE{}:\hyperref[illu:sP:1]{1} \\
        \phantlabel{illu:sP:4}\fa P \land R & \(\land\)\textbf{Intro:} \hyperref[illu:sP:2]{2},\hyperref[illu:sP:3]{3}
      \end{fitch}
    \end{center}

    Still, I haven't yet concluded \((P \rightarrow Q) \rightarrow P, R \vdash P \land R\).

    For, \sqE{} may not be a sound rule of inference.

    I go on to conclude \((P \rightarrow Q) \rightarrow P, R \vdash P \land R\).
  \end{illustration}

  \begin{definition}[\sqE{}]%
    \label{def:sque}%
    \sqE{} is the following rule:
    \begin{center}
      \begin{fitch}
        \ftag{\text{\scriptsize \emph{i}}}{\fa (\phi \rightarrow \psi) \rightarrow \phi} \\
        \ftag{\text{\scriptsize }}{\fa \vdots } \\
        \ftag{\text{\scriptsize \emph{j}}}{\fa \phi } & \sqE{}:\emph{i} \\
      \end{fitch}
    \end{center}
  \end{definition}

  \sqE{} is sound.
  If not the case that pause and show \sqE{} is sound, then not \tC{}.%
  \footnote{
    \label{prop:sqE-sound}
    Rather than prove \sqE{} is sound (which would require a detailed statement of the proof system in question), we prove that the corresponding semantic entailment holds:

    Let \(v\) be an arbitrary (truth-functional) valuation, and assume \(v((\phi \rightarrow \psi) \rightarrow \phi) = \valI{True}\).
    Further, assume for contradiction \(v(\phi) = \valI{False}\).

    As \(v(\phi) = \valI{False}\), it immediately follows that \(v(\phi \rightarrow \psi) = \valI{True}\).
    Therefore, by the first assumption, it must be the case that \(v(\phi) = \valI{True}\).
    This contradictions the second assumption.
  }

  % The relevant propositions, values, and \pool{1} are as follows:
  % \begin{itemize}[noitemsep]
  % \item
  %   I am the agent.
  % \item
  %   \(\phi\) is the proposition: \(\mathsf{(P \rightarrow Q) \rightarrow P, R \vdash P \land R}\).
  % \item
  %   \(\psi\) is the proposition: \propI{\sqE{} is sound}
  % \item
  %   Both \(v\) and \(v'\) are the value: \valI{True}.
  %   And,
  % \item
  %   The pools of premises \(\Phi\) and \(\Psi\) are left unspecified.%
  %   \footnote{
  %     \((P \rightarrow Q) \rightarrow P\) and \(R\) are premises of the deduction, and not elements of \(\Phi\).

  %     For, my conclusion is \pv{\mathsf{(P \rightarrow Q) \rightarrow P, R \vdash P \land R}}{\valI{True}}.
  %     My conclusion is not \pv{\mathsf{P \land R}}{\valI{True}}.
  %   }
  % \item
  %   Key observations:
  %   \begin{itemize}[noitemsep]
  %   \item
  %     sunny day, good breakfast, nice coffee.
  %     I.e.\ as ideal as situations may be for syntactic proofs.
  %   \item
  %     Proved \sqE{} numerous times before.
  %   \end{itemize}
  % \end{itemize}

  % Hence, the relevant instance of the conditional by which a \requ{} is defined is:

  % \begin{quote}
  %   \begin{itenum}
  %   \item[\emph{If}:]
  %     If \pv{\propI{\sqE{} is sound}}{\valI{True}} from \(\Psi\) is not a \fc{}.
  %   \item[\emph{Then}:]
  %     I am not concluding \(\pv{\mathsf{(P \rightarrow Q) \rightarrow P, R \vdash P \land R}}{\valI{True}}\) from \(\Phi\).
  %   \end{itenum}
  % \end{quote}

  % In contrast to \autoref{illu:lost-key}, \autoref{scen:squish} leads to a conclusion.

  At issue is whether \requ{}.
  By \autoref{prop:hinge}, three things.
  \begin{itemize}
  \item
    \tRep{}
  \item
    \sqE{} is \tI{} of \(T'\).
  \item
    Conditional.
  \end{itemize}

  Plausibly, four \prop{0}-\val{0}-\pool{0} pairs in type:

  \begin{center}
    \begin{tabular}{R{.45\textwidth} L{.45\textwidth}}
      \prop{2}-\val{0} pair & \pool{2} \\
      \hline
      \pv{(\phi \rightarrow \phi) \rightarrow \psi \space/\space \phi}{\textover[c]{\valI{Sound}}{\valI{Unsound}}} & \dots \\
      \pv{\phi \rightarrow (\psi \rightarrow \phi) \space/\space \phi}{\valI{Unsound}} & \dots \\
      \pv{\psi \rightarrow (\phi \rightarrow \psi) \space/\space \phi}{\valI{Unsound}} & \dots \\
      \pv{(\psi \rightarrow \phi) \rightarrow \psi \space/\space \phi}{\valI{Unsound}} & \dots \\
    \end{tabular}
  \end{center}

  Throughout \autoref{scen:squish} I \emph{know} \sqE{} is sound.
  Prior to \autoref{scen:squish} I have proved \sqE{} is sound on various occasions using the same basic observations made in the argument for \autoref{prop:sqE-sound}.

  However, there is a distinction between \emph{knowing} \sqE{} is sound and \emph{proving} \sqE{} is sound.
  For example, if I have just drunk a considerable amount of wine, or woken from a night of tormented sleep.
  Generally said, I may not be thinking straight.

  Of course, I may conclude \(\pv{\phi}{v}\) from \(\Phi\) regardless of whether \(\pv{\psi}{v'}\) from \(\Psi\) is a \fc{}.
  Indeed, given some particularly good wine I may conclude \(P \land C \vdash O\).%
  \footnote{
    For, \(P \land C\) reads `Pac', \(O\) looks like a pellet, and Pacman likes to eat pellets.
  }
  However, then not \tCV{}.
\end{note}



\section{\tC{2} and \issueConstraint{}}
\label{sec:tpyically-concluding}

\begin{note}
  Understand \requ{} and counterexamples to \issueConstraint{}.

  Introduced \tCV{} to motivate \requ{}.

  In this section, recast a key proposition from \autoref{cha:requs} (\autoref{prop:requ-WhyV-ces}) in terms of \tCV{}.
\end{note}


% \begin{note}
%   \begin{definition}[The \tCon{0}]
%     \label{def:tCon}
%     \vspace{-\baselineskip}
%     \begin{itemize}
%     \item
%       The \emph{\tCon{0}} hold of \(e\) with respect to \(\langle \vAgent{}, \pvp{\phi}{v}{\Phi}, \pvp{\psi}{v'}{\Psi}, e \rangle\).
%     \end{itemize}

%     \emph{If and only if}:

%         \vspace{-1.5\baselineskip}
%   \end{definition}
% \end{note}

% \begin{note}
%   \begin{proposition}[\tCV{3} and \qWhyV{}]
%     \label{prop:tCV-WhyV}
%     \vspace{-\baselineskip}
%     \begin{itenum}
%     \item[\emph{If}:]
%       The \tCon{0} hold of \(e\) with respect to \(\langle \vAgent{}, \pvp{\phi}{v}{\Phi}, \pvp{\psi}{v'}{\Psi}, e \rangle\).
%     \item[\emph{Then}:]
%       A \ros{} between \(\pv{\psi}{v'}\) and \(\Psi\) answers \qWhyV{}.
%     \end{itenum}
%     \vspace{-1.5\baselineskip}
%   \end{proposition}

%   Instead of \requ{} directly, strengthen to \tCV{} and sufficient conditions for \requ{} given \tCV{}.

%   \begin{argument}{prop:tCV-WhyV}
%     The argument parallels the argument for \autoref{prop:requ-WhyV} (\autopageref{prop:requ-WhyV}) with minor adjustment.

%     % Rather appeal to Sub-condition~\ref{def:rCs:Cing:requ} of the \rCon{} to observe \(e^{\flat}\) is not an event in which \vAgent{} is concluding \(\pv{\phi}{v}\) from \(\Phi\), appeal to Sub-condition~\ref{def:tCon:sR} of the \tCon{} to observe \(e^{\flat}\) is not an event in which \vAgent{} is \tCV{} \(\pv{\phi}{v}\) from \(\Phi\).
%     % And, with this observation in hand appeal to Sub-condition~\ref{def:tCon:Cing:itc} to observe \(e^{\flat}\) is not, or does not develop into, an event in which \vAgent{} concludes \(\pv{\phi}{v}\) from \(\Phi\).
% \end{argument}
% \end{document}

\begin{note}
  The relevant parallel to \autoref{prop:requ-WhyV-ces}:

  \begin{proposition}[\tCV{3} and \issueConstraint{}]
    \label{prop:tCV-WhyV-ces}
    \vspace{-\baselineskip}
    \begin{itenum}
    \item[\emph{If}:]
      Conditions \ref{def:tCon:C}, \ref{def:tCon:Cing}, and \ref{def:tCon:nWit} jointly hold:
      \begin{enumerate}[label=\arabic*., ref=\arabic*]
      \item
        \label{def:tCon:C}
        \(\ed{}\) is an event in which \vAgent{} concludes \(\pv{\phi}{v}\) from \(\Phi\).
      \item
        \label{def:tCon:Cing}
        There is some sub-event \(\ed{\flat}\) of \(\ed{}\) such that conditions%
          ~\ref{def:tCon:sR:rep},%
          ~\ref{def:tCon:sR:tI}, and%
          ~\ref{def:tCon:sR:itc} %
          jointly hold:
          \begin{enumerate}[label=\roman*., ref=\arabic{enumi}\roman*]
          \item
            \label{def:tCon:sR:rep}
            \(T'\) is a \tRep{} of \vAgent{} \tCV{} \(\pv{\phi}{v}\) from \(\Phi\) by type \(T\) in \(\ed{\flat}\).
          \item
            \label{def:tCon:sR:tI}
            \(\pvp{\psi}{v'}{\Psi}\) is a \tI{} of \(T'\)
          \item
            \label{def:tCon:sR:itc}
            {
              \color{red}
              For each \prop{0}-\val{0} pair \(\pv{\phi'}{v'}\) in \(\Phi\), \vAgent{} \evals{} \(\phi'\) as having value \(v'\).
            }
        \end{enumerate}
      \item
        \label{def:tCon:nWit}
        \vAgent{} does not have a \wit{} for a \ros{} between \(\pv{\psi}{v'}\) and \(\Psi\) when \vAgent{} concludes \(\pv{\phi}{v}\) from \(\Phi\).
      \end{enumerate}
    \item[\emph{Then}:]
      \issueConstraint{} does not hold.
    \end{itenum}
    \vspace{-\baselineskip}
  \end{proposition}

  \begin{argument}{prop:tCV-WhyV-ces}
    Assume conditions \ref{def:tCon:C}, \ref{def:tCon:Cing}, and \ref{def:tCon:nWit} jointly hold.

    Given Condition \ref{def:tCon:nWit}, consider sub-event \(\ed{\flat}\) of \(\ed{}\) such that conditions \ref{def:tCon:sR:rep}, \ref{def:tCon:sR:tI}, and \ref{def:tCon:sR:itc} jointly hold.
    From \autoref{prop:hinge} (\autopageref{prop:hinge}), it follows that \(\pv{\psi}{v'}\) being a \fc{} from \(\Psi\) is a \requ{} of \(\ed{}\) being an event in which \vAgent{} is \tCV{} \(\pv{\phi}{v}\) from \(\Phi\).

    Hence, by \autoref{prop:requ-WhyV} (\autopageref{prop:requ-WhyV}), a \ros{} between \(\pv{\psi}{v'}\) and \(\Psi\) answers \qWhyV{}.

    However, by Condition~\ref{def:tCon:nWit}, the agent does not have a \wit{} for the \ros{} between \(\pv{\psi}{v'}\) and \(\Psi\) when \vAgent{} concludes \(\pv{\phi}{v}\) from \(\Phi\).

    So, \issueConstraint{} does not hold.
  \end{argument}
\end{note}






\subsection{Additional upshots}

\begin{note}
  \color{red}
  \tCV{2} motivates, and also takes care of a potential objection, which is somewhat standard.

  Something like, explanation supervenes on facts.
  If possible to change facts while explanation holds, then those facts are not explanatory.

  Things hinge of \requ{}.
  Change whether something is a \requ{}.
  Intuitively, agent still concludes.

  In short, it's not the case that \ros{} between \(\pv{\psi}{v'}\) and \(\Psi\).
  For, possible for \ros{} to fail while agent concludes \(\pv{\phi}{v}\) from \(\Phi\).

  In general this may be true.

  However, in this case, the agent is not \tCV{}.

  For, if \ros{} fails then not \fc{}, and hence agent is not \tCV{}.
\end{note}

\begin{note}
  Second, agency.

  \tCV{}, then agency.

  Else, it's accidental to what the agent is doing that they're \tCV{}.
  And, in some cases I think this may be right, but not in general.

  The agent \emph{is} \tCV{}.
  It is not merely the case that the agent \emph{happens to be} \tCV{}.
\end{note}




\section*{Summary}

\begin{note}
  Introduced \requ{1}.

  Defined in terms of \tC{}.

  Motivated by \tC{}.
\end{note}

\begin{note}
  Smashing definitions.
\end{note}

\begin{note}
  Also, objection.
\end{note}

\begin{note}
  Still not \scen{1} for any of this to be amount to a counterexample to \issueConstraint{}.
\end{note}




% \begin{note}[Problems of induction]
%   Hence, the sketch does not apply to black ravens.
%   I wouldn't conclude all ravens are black if I saw a white raven.

%   I may worry about shortly seeing a white raven when concluding all ravens are black, and I may refuse to entertain the possibility that the sun will rise tomorrow when planning to mow the grass.

%   However, it's not possible to reason to seeing a white raven, nor is it possible to reason to the sun not rising tomorrow.

%   Abstractly, at issue in~\autoref{illu:lost-key} is the possibility of failing to a reason to some proposition-value pair given \emph{present} information, rather than the possibility of failing to a reason to some proposition-value pair given \emph{new} information.

%   To the extent that problems of induction arise from receiving new information, what is at issue is distinct.%
%   \footnote{
%     See~\textcite{Henderson:2020wb} for more on the problem of induction.
%   }

%   Similar points for external world scepticism.
%   Would not conclude that I have hand if disembodied brain in a vat.

%   However, conclusion is out of reach.
% \end{note}

    %   \footnote{
    %     The present point is similar to issues raised by \citeauthor{Harman:1973ww} (\citeyear{Harman:1973ww}) regarding the proposed equivalence between reasons for which an agent believes something with reasons the agent would offer if asked to justify their belief.
    %     As \citeauthor{Harman:1973ww} notes, an agent may offer reasons because they think they will convince their audience, not because the agent is compelled by the reasons, etc.
    %     (\citeyear[Ch.2]{Harman:1973ww})

    %     To the extent that \citeauthor{Harman:1973ww}'s point is that what holds from an \agpe{} need not actually be the case, the point in the same.
    %     However, to the extent that \citeauthor{Harman:1973ww} relies on an under-specification of what holds from an \agpe{} --- i.e.\ the distinction between whether \(\phi\) has value \(v\) from the \agpe{} or whether the agent evaluates as true the proposition that their audience is responsive to \(\phi\) having value \(v\), the point is distinct.
    %   }


%%% Local Variables:
%%% mode: latex
%%% TeX-master: "master"
%%% TeX-engine: luatex
%%% End:

