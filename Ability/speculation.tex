\chapter{Speculation}
\label{cha:speculation}

\paragraph{Sudoku}

\begin{note}
  Fill in the puzzle with hints.
  \ros{3} which answer \qWhyV{}?
  Well, absent hints.

  Same as initial \scen{0}.
  Is could just be testimony, but at the same time, is it only testimony?
\end{note}

\section{Deduction theorem}

\begin{note}[Conditionals, a point of interest]
  More generally, we have the following result.
  If appeal to some conditional which links premises to a conclusion, such that agent may reason from premises to conclusion, then the agent has always concluded premises from conclusion.

  This is interesting.
  If agent has concluded from conditional in this way, then in effect the conditional drops out as a premise.

  If \(\Sigma, \phi \rightarrow \psi \vdash \psi\) then \(\Sigma, \phi \vdash \psi\).

  If \(\Sigma, \phi \vdash \phi \rightarrow \psi\) then \(\Sigma, \phi \vdash \psi\).

  The second is close to a restricted form of the deduction theorem.

  Note, this is only when the conditional has a special \requ{}.

  In cases where no checking the conditional, then the elimination does not hold.

  Whether anything of independent interest follows from this, unclear.
  One example, responsibility.
  Then, to another person, not only reasoning with conditional, but also full reasoning.
  No way to distinguish between the two cases.
  Or, from the converse perspective, no need to add any additional clauses to account of responsibility.
  However, issues of this kind are far beyond the scope of this document.
\end{note}

\begin{note}
  Somewhere at the end, or perhaps on a speculative chapter:
  Deduction theorem for reasoning.
  And, support, so why not conclude from without witnessing the reasoning.
  This would just be witnessing a foregone-conclusion.
\end{note}

\begin{note}
  \citeauthor{Easwaran:2009tm} on proofs from discarded?
\end{note}

\section{Foregone-concluding}

\begin{note}
  Tentative suggestion.
\end{note}

\begin{note}[Foregone-concluding]
  Pair this with a key idea.

  \begin{restatable}[Foregone-concluding]{idea}{ideaForegoneCing}
    \label{idea:reassignment}
    If \fc{}, then may conclude.
    %\vspace{-\baselineskip}
  \end{restatable}

  Cases where concluding by witnessing reduces to witnessing an \fc{}.
  \emph{Concluding \(\pv{\psi}{v'}\) from \(\Psi\) is just witnessing \fc{}.}
  So, reduction, in certain cases.
  Further, if \fc{}, then conclude.
  At least, in certain cases.
\end{note}


%%% Local Variables:
%%% mode: latex
%%% TeX-master: "master"
%%% End:
