\chapter{Speculation}
\label{cha:speculation}

\section{Desire}
\label{sec:desire}

\begin{note}
  Variation on Setiya.

  \begin{quote}
    Reasons: The fact that p is a reason for A to / just in case A has a collection of psychological states, C, such that the disposition to be moved to / by C-and- the-belief-that-p is a good disposition of practical thought, and C contains no false beliefs.
  \end{quote}

  Here, \fc{}.

  Well, really Smith.
  Here, it's kind of interesting.
  It's related to Smith.
  And, oddly, we see the phenomena elsewhere.
  Desires and so on are just a special case.
\end{note}

\subsection{Temptation}
\label{sec:temptation}

\begin{note}
  Consider temptation.

  One way to think about things.

  Thing is, the agent doesn't conclude to resist temptation.

  However, does this entail that the agent is not concluding to resist temptation?

  What to make of the progressive?

  If progressive is true, then get a \fc{}.
  And, if \fc{} then possibility of a \requ{}.

  In this sense, temptation is compatible with basic understanding of things.
\end{note}

\begin{note}
  Desires more generally.

  Interest here is with \fc{1} which expand on the desire.

  This gets to a nice mundane observation.

  It does seem fine to say the agent's reason was to reproduce, or whatever.
  Implicit suggestion is that agent concludes, ah, I guess that is it.
  Though, it's obviously not clear that this really is a \fc{}.
\end{note}


\section{New information}

\begin{note}
  \begin{scenario}
    Wason selection task.

    If X then Y.
  \end{scenario}

  This is really nice, if flip over the card, then would be concluding.

  Kind of trivial, huh.

  Of course, don't need to do this to conclude.
  However, expected.
  Indeed, something surprising if it is not a \fc{}.

  No \wit{}, clearly, as you have no seen the other side of the card.

  If any doubts, then clear problem.

  The difference here is that this requires new information.
  Need to turn over the card.
\end{note}

\begin{note}
  \citeauthor{Easwaran:2009tm} on proofs from discarded?
\end{note}

\section{Foregone-concluding}

\begin{note}[Foregone-concluding]
  So, generally seen that there's nothing too distinctive about \wit{1}.
  Hence, why care for these?

  \begin{idea}[Foregone-concluding]
    \label{idea:reassignment}
    If \fc{}, then may conclude.
    %\vspace{-\baselineskip}
  \end{idea}

  Cases where concluding by witnessing reduces to witnessing an \fc{}.
  \emph{Concluding \(\pv{\psi}{v'}\) from \(\Psi\) is just witnessing \fc{}.}
  So, reduction, in certain cases.
  Further, if \fc{}, then conclude.
  At least, in certain cases.
\end{note}

% \section{Deduction theorem}

% \begin{note}[Conditionals, a point of interest]
%   More generally, we have the following result.
%   If appeal to some conditional which links premises to a conclusion, such that agent may reason from premises to conclusion, then the agent has always concluded premises from conclusion.

%   This is interesting.
%   If agent has concluded from conditional in this way, then in effect the conditional drops out as a premise.

%   If \(\Sigma, \phi \rightarrow \psi \vdash \psi\) then \(\Sigma, \phi \vdash \psi\).

%   If \(\Sigma, \phi \vdash \phi \rightarrow \psi\) then \(\Sigma, \phi \vdash \psi\).

%   The second is close to a restricted form of the deduction theorem.

%   Note, this is only when the conditional has a special \requ{}.

%   In cases where no checking the conditional, then the elimination does not hold.

%   Whether anything of independent interest follows from this, unclear.
%   One example, responsibility.
%   Then, to another person, not only reasoning with conditional, but also full reasoning.
%   No way to distinguish between the two cases.
%   Or, from the converse perspective, no need to add any additional clauses to account of responsibility.
%   However, issues of this kind are far beyond the scope of this document.
% \end{note}


%%% Local Variables:
%%% mode: latex
%%% TeX-master: "master"
%%% TeX-engine: luatex
%%% End:
