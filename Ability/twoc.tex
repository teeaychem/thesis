\chapter{Two ways of concluding}
\label{chap:twoc}

\begin{note}[Goal]
  The goal here is to motivate a distinction between how and why.

  Introduce idea of an \itp{}.

  Suggest the possibility of negative resolution to \issueInclusion{}, though not to \issueConstraint{}.

  In part, motivation for focusing on \issueConstraint{}.

  However, sufficient motivation as negative to \issueConstraint{} entails negative to \issueInclusion{}.
\end{note}

\begin{note}[Two ways of concluding]
  In this section, broad idea.
  \adA{} and \adB{}.
\end{note}

\section{Broad}
\label{chap:twoc:broad}

\subsection{The first type of reasoning: \adA{}}

\begin{note}
  \begin{restatable}[\adA{}]{definition}{defADA}
    \label{AR:adA}
    \label{def:adA}
    \vAgent{} concludes \(\pv{\phi}{v}\) from \(\Phi\) by `\adA{}' if:
    \begin{enumerate}[label=\textsf{S:\arabic*}., ref=(\textsf{S}:\arabic*)]
    \item
      \label{def:adA:psi}
      \vAgent{} concludes \(\pv{\phi}{v}\) by witnessing reasoning from some pool of premises \(\pv{\phi_{1}}{v_{1}},\dots,\pv{\phi_{k}}{v_{k}}\)
    \item
      \(\Phi\) is the collection of all and only the premises \(\pv{\phi_{i}}{v_{i}}\).
    \end{enumerate}
    \vspace{-\baselineskip}
  \end{restatable}

  \adA{} is straightforward.

  As before, concern may be raised about what the relevant premises are, and whether the relevant agent identifies those premises as premises.
  However, granting that an agent always concludes from some collection of premises, the relevant collection exists.

  The restriction to the \emph{exact} collection of premises the agent reasons from is for convenience.
  Nothing in particular hangs on this distinction, but equally nothing much is gained by allowing the inclusion of redundant proposition-value pairs.
\end{note}

\begin{note}[\illu{1}]
  Time from positions of hands on a clock and understanding of how time is represented by such a clock.

  Whether to make a bet from tolerance for risk, distribution of cards in a pack, cards in hand, and rules of the game.
\end{note}

\begin{note}
  \adA{} does not outline a specific way of reasoning.
  Deductive, inductive, etc.\
\end{note}

\begin{note}
  \phantlabel{abstract-adA}
  Basic (abstract) instance of \adA{}:

  {
    \small
    \begin{enumerate}[label=\arabic*., ref=\arabic*, noitemsep]
    \item\label{def:adA:ex:C:Cp} I have concluded \(\phi\) has value \(v\).
    \item\label{def:adA:ex:C:p} So, \(\phi\) has value \(v\). \hfill(From~\ref{def:adA:ex:C:Cp})
    \item\label{def:adA:ex:C:Cps} Likewise, I have concluded \(\psi\) has value \(v'\) when \(\phi\) has value \(v\).
    \item\label{def:adA:ex:C:ps} So, \(\psi\) has value \(v'\) when \(\phi\) has value \(v\). \hfill(From~\ref{def:adA:ex:C:Cps})
    \item\label{def:adA:ex:C:T} If \(\psi\) has value \(v'\) when \(\phi\) has value \(v\) and \(\phi\) has value \(v\), then it must be the case that \(\psi\) has value \(v'\). \hfill (From understanding of `if\dots then\dots')
    \item\label{def:adA:ex:C:s} Hence, \(\psi\) has value \(v'\).\newline
      \mbox{}\hfill (From \ref{def:adA:ex:C:p},~\ref{def:adA:ex:C:ps}~and~\ref{def:adA:ex:C:T})
    \item Therefore, I conclude \(\psi\) has value \(v'\). \hfill (From \ref{def:adA:ex:C:Cp} -- \ref{def:adA:ex:C:s})
    \end{enumerate}
  }
  From this reasoning, two clear premises.
  \(\pv{CS(\pv{\phi}{v})}{\top}\) and \(\pv{CS(\pv{\pv{\phi}{v} \Rightarrow \pv{\psi}{v'}}{\top})}{\top}\).
  Witness reasoning from these premises.
  The reasoning is verbose, premises are that the agent has concluded.
  Here, concluding is not factive.

  % (Consider parallel reasoning with knowledge.%
  % \footnote{The parallel reasoning in full:
  %   \begin{enumerate}[label=\arabic*., ref=\arabic*]
  %   \item\label{def:adA:ex:K:Kp} I know \(\phi\) has value \(v\).
  %   \item\label{def:adA:ex:K:p} So, \(\phi\) has value \(v\). \hfill (From~\ref{def:adA:ex:K:Kp})
  %   \item\label{def:adA:ex:K:Kps} I know \(\psi\) has value \(v'\) when \(\phi\) has value \(v\).
  %   \item\label{def:adA:ex:K:ps} So, \(\psi\) has value \(v'\) when \(\phi\) has value \(v\). \hfill(From~\ref{def:adA:ex:K:Kps})
  %   \item\label{def:adA:ex:K:T} If \(\psi\) has value \(v'\) when \(\phi\) has value \(v\) and \(\phi\) has value \(v\), then it must be the case that \(\psi\) has value \(v'\). \hfill (From understanding of `if\dots then\dots')
  %   \item\label{def:adA:ex:K:s} Hence, \(\psi\) has value \(v'\). \hfill (From \ref{def:adA:ex:C:p},~\ref{def:adA:ex:C:ps}~and~\ref{def:adA:ex:C:T})
  %   \item So, I know \(\psi\) has value \(v'\) as \(\psi\) having value \(v'\) follows from~(\ref{def:adA:ex:K:Kp}) and~(\ref{def:adA:ex:K:Kps}).
  %     \mbox{}\hfill (From \ref{def:adA:ex:K:Kp} -- \ref{def:adA:ex:K:s})
  %   \end{enumerate}
  % }%
  % )
\end{note}


\subsection{The second type of reasoning: \adB{}}

\begin{note}[Turning to \adB{}]
  We now turn to the second type of reasoning: `\adB{}'.

  We begin with a definition of \adB{}.
  However, our attention will quickly turn to a pair of helper definitions which relate some proposition-value pair we identify as an `\itp{}' to some other proposition-value pair and pool of premises.
\end{note}

\begin{note}
  \begin{restatable}[\adB{}]{definition}{defADB}
    \label{def:adB}
    \vAgent{} concludes \(\pv{\phi}{v}\) from \(\Phi\) by `\adB{}' if:
    \begin{enumerate}[label=\textsf{I}:\arabic*., ref=(\textsf{I}:\arabic*)]
    \item
      \label{def:adB:itp}
      \vAgent{} has concluded \(\pv{\mu}{v}\) and  \(\pv{\mu}{v}\) is either:
      \begin{enumerate}
      \item
        \label{def:adB:itp:between}
        An \itp{} \emph{between} \(\pv{\phi}{v}\) and \(\Phi\), or:
      \item
        \label{def:adB:itp:for}
        An \itp{} \emph{for} \(\pv{\phi}{v}\), with \(\Phi\) as the relevant pool of premises.
      \end{enumerate}
    \item
      \label{def:adB:conclude}
      \vAgent{} concludes \(\pv{\phi}{v}\) by appeal to the premises \(\pv{\phi_{1}}{v_{1}},\dots,\pv{\phi_{k}}{v_{k}}\) from the pool of premises \(\Phi\) via the possibility of witnessing the relevant reasoning from \(\pv{\mu}{v}\).
    \end{enumerate}
    \vspace{-\baselineskip}
  \end{restatable}
\end{note}

\begin{note}
  Without definitions of what an \itp{0} between \(\pv{\phi}{v}\) and \(\Phi\) is, or what an \itp{0} for \(\pv{\phi}{v}\) is,~\autoref{def:adB} is incomplete.
  We will shortly turn to the relevant helper definitions.

  Though, working backwards from~\autoref{def:adB} gives a hint.
  An \itp{} should contain information that it is possible for the agent to witness reasoning that conclude \(\pv{\phi}{v}\) from some pool of premises \(\Phi\).
\end{note}

\begin{note}
  Still, before turning to the pair of helper definitions, let me stress a key aspect of~\autoref{def:adB}:
  From~\ref{def:adB:conclude}, the agent concludes \(\pv{\phi}{v}\) from \(\Phi\) and \(\Phi\) alone (without witnessing the relevant reasoning).
  The agent does not conclude \(\pv{\phi}{v}\) from \(\Phi\) and \(\pv{\mu}{v}\).
  From the perspective of defining \adB{}, the latter option may be included, but the possibility of the former will be important when we turn to tension.
\end{note}

\subsection{Contrast}

\begin{note}
  The key difference between \adA{} and \adB{}:
  \begin{itemize}
  \item \adA{} involves the agent appealing to \(\pv{\mu}{v}\) in order to conclude \(\pv{\phi}{v}\), while
  \item \adB{} does not involve the agent appealing to \(\pv{\mu}{v}\) to conclude \(\pv{\phi}{v}\).
    Instead, the role of \(\phi\) is to highlight \(\rho_{1},\dots,\rho_{k}\) and the agent appeals to propositions \(\rho_{1},\dots,\rho_{k}\) to claim support for \(\psi\).
  \end{itemize}

  For the definition to be satisfied, \(\phi\) needs only be involved to the extent that it provides the link.
  Hence, \(\phi\) is not irrelevant.
  Still, the agent does not appeal to \(\phi\).
\end{note}

\paragraph*{\adB{}: Helper definitions}

\begin{note}[]
  We briefly noted, working backwards from~\autoref{def:adB}, that an \itp{} should contain information that it is possible for the agent to witness reasoning that conclude \(\pv{\phi}{v}\) from some pool of premises \(\Phi\).
  We now detail what an \itp{0}.
  Or, rather, what \itp{1} are.

  There are two cases.
  First, an \itp{0} \emph{between} \(\pv{\phi}{v}\) and \(\Phi\).
  Second, an \itp{0} \emph{for} \(\pv{\phi}{v}\).
  The distinction between these cases is whether the relevant \itp{0} identifies a particular pool of premises.

  In practice, we will blur the distinction, but from a definitional perspective the latter is best seen as a generalisation of the former.
\end{note}

\subparagraph*{An \itp{0} between \(\pv{\phi}{v}\) and \(\Phi\)}

\begin{note}[\itp{} between]
  \begin{definition}[An \itp{0} between \(\pv{\phi}{v}\) and \(\Phi\) \hfill \named{I.b}]
    \label{def:itp:b}
    \(\pv{\mu}{v}\) is an \itp{} \emph{between} \(\pv{\phi}{v}\) and \(\Phi\) if and only if:

    \begin{enumerate}[label=\arabic*., ref=\named{\textsf{I.b}:\arabic*}]
    \item
      \label{def:itp:b:pR}
      \(\mu\) having value \(v\) ensures:
      \begin{itemize}
      \item
        It is possible for \vAgent{} to conclude \(\phi\) has value \(v\) by witnessing reasoning from \(\Phi\) to \(\pv{\phi}{v}\), given \vAgent{}'s present epistemic state.
      \end{itemize}
    \item
      \label{def:itp:b:distinct}
      \(\pv{\mu}{v}\) is not equivalent to any \(\pv{\phi_{i}}{v_{i}}\), given \vAgent{}'s present epistemic state.
    \end{enumerate}
    \vspace{-\baselineskip}
  \end{definition}
\end{note}

\begin{note}[Plan]
  The definition of an \itp{} between \(\pv{\phi}{v}\) and \(\Phi\) consists of two components:~\ref{def:itp:b:pR} and~\ref{def:itp:b:distinct}.

  \ref{def:itp:b:pR} is the core of the definition, while~\ref{def:itp:b:distinct} narrows the definition to cases of interest.

  We begin by expanding on~\ref{def:itp:b:pR}, and then motivate the restriction given by~\ref{def:itp:b:distinct}.
\end{note}

\begin{note}[Expanding on~\ref{def:itp:b:pR}]
  Intuitively, think of an \itp{} between \(\pv{\phi}{v}\) and \(\Phi\) as a particular kind of conditional of the (rough) form `if \(\Phi\) then \(\pv{\phi}{v}\)'.

  Indeed, an `\emph{if} \dots \emph{then} \dots' statement between \(\Phi\) and \(\pv{\phi}{v}\) may be constructed from any \itp{} between \(\pv{\phi}{v}\) and \(\Phi\).

  For, if it is possible for an agent to conclude \(\phi\) has value \(v\) by witnessing reasoning from \(\Phi\) to \(\pv{\phi}{v}\), then (from the agent's perspective at least), \(\pv{\phi}{v}\) whenever \(\pv{\phi_{i}}{v_{i}}\) for each \(\pv{\phi_{i}}{v_{i}}\) in \(\Phi\).
  So, if every proposition \(\phi_{i}\) in \(\Phi\) has it's respective value \(v_{i}\), then \(\phi\) also has value \(v\).
  Or, more colloquially, if \(\Phi\) then \(\pv{\phi}{v}\).

  However, an \itp{} between \(\pv{\phi}{v}\) and \(\Phi\) is stronger than `if \(\Phi\) then \(\pv{\phi}{v}\)'.
  For, not only is it the case that \(\pv{\phi}{v}\) whenever \(\pv{\phi_{i}}{v_{i}}\) for each \(\pv{\phi_{i}}{v_{i}}\) in \(\Phi\), but in addition it is possible for the agent to conclude \(\pv{\phi}{v}\) from \(\Phi\) given the agent's present epistemic state.

  Naturally, the possibility for the agent to conclude \(\pv{\phi}{v}\) from \(\Phi\) goes beyond a plain conditional between \(\pv{\phi}{v}\) and \(\Phi\).

  Breaking down \autoref{def:itp:b:pR}, observe we have an `inner' statement:
  \begin{quote}
     It is possible for \vAgent{} to conclude \(\phi\) has value \(v\) by witnessing reasoning from \(\Phi\) to \(\pv{\phi}{v}\).
  \end{quote}
  And, a qualifier:
  \begin{quote}
    [G]iven \vAgent{}'s present epistemic state.
  \end{quote}

  The statement is simple.
  The relevant possibility is just for the agent to conclude \(\pv{\phi}{v}\) from \(\Phi\) by an instance of \adA{}.
  Indeed, the relevant instances of \EAS{} we motivate by developing tension will always involve an \itp{}, and hence will always involve the possibility of witnessing reasoning to the relevant conclusion from some pool of premises.

  Turn now to the qualifier:
  \begin{quote}
    [G]iven \vAgent{}'s present epistemic state.
  \end{quote}
  This is a qualifier on possible witnessing.

  Generally speaking, it may be possible for an agent to conclude \(\pv{\phi}{v}\) from \(\Phi\) by an instance of \adA{} from a distinct epistemic state.
  For example, if the agent were to learn some \(\pv{\phi_{i}}{v_{i}}\) in \(\Phi\) is the case, or if the agent were to improve their reasoning skills.
  However, the innermost qualifier ensures that it possible for an agent to conclude \(\pv{\phi}{v}\) from \(\Phi\) without any revision to the agent's epistemic state.

  An important consequence of this qualifier is that the agent must already hold that for each \(\pv{\phi_{i}}{v_{i}}\) in \(\Phi\), \(\phi_{i}\) has value \(v_{i}\).
  For, if not, then \(\pv{\phi_{i}}{v_{i}}\) would not be available as a premise.
  Of course, the definition of an \itp{} between \(\pv{\phi}{v}\) and \(\Phi\) may be given without this assumption, but we have no use for any more general definition.
\end{note}

\begin{note}[\illu{2}]
  \color{red}
    For example, consider being informed that the first player in a game of tic-tac-toe may always guarantee a draw.
  No premises are specified, but on reflection it is clear to see that one may reason through all the possible games to identify the strategy.
  The relevant \itp{0}, then, is the combination of the novel information and one's understanding of tic-tac-toe, the premises, some general results about tic-tac-toe, and the conclusion the required strategy.

  What guarantees the possibility of concluding --- general properties of tic-tac-toe which follow from the rules --- is intuitively distinct from the relevant pool of premises, which likely be limited to the rules themselves combined with the \dots

\end{note}

\begin{note}
  A quick observation.

  \(\pv{\mu}{v}\) being an \itp{} between \(\pv{\phi}{v}\) and \(\Phi\) depend on whether or not it is actually possible for the agent to conclude \(\pv{\phi}{v}\) from \(\Phi\), given their epistemic state.
  If it is not possible for the agent to witness reasoning from \(\Phi\) to \(\pv{\phi}{v}\), then no such \itp{} will exist.

  However, whether or not an agent \emph{concludes} \(\pv{\mu}{v}\) is an \itp{} between \(\pv{\phi}{v}\) and \(\Phi\) does not depend on whether or not it is actually possible for the agent to conclude \(\pv{\phi}{v}\) from \(\Phi\), given their epistemic state.
  We do not assume that an agent concludes \(\pv{\phi}{v}\) only if \(\phi\) actually has value \(v\).
  And, our interest with \itp{1} will typically be from the perspective of the agent's present epistemic state.
\end{note}

\begin{note}[Expanding on~\ref{def:itp:b:distinct}]
  The above has expanded on~\ref{def:itp:b:pR}.
  Finally, we turn to~\ref{def:itp:b:distinct}.

  In short,~\ref{def:itp:b:distinct} ensures that if \(\pv{\mu}{v}\) is an \itp{0} between \(\pv{\phi}{v}\) and \(\Phi\), then \(\pv{\mu}{v}\) is not a premise that the agent would appeal to when witnessing the reasoning from \(\Phi\) to \(\pv{\phi}{v}\) captured by~\ref{def:itp:b:pR}.

  More strictly, not only is \(\pv{\mu}{v}\) not a premise, but is not equivalent to any \(\pv{\phi_{i}}{v_{i}}\) in \(\Phi\).
  Where, again, equivalence is evaluated from the perspective of the agent.

  From an abstract perspective, if \(\pv{\mu}{v}\) is an \itp{0} between \(\pv{\phi}{v}\) and \(\Phi\), then \(\pv{\mu}{v}\) is purely descriptive of the relationship between \(\pv{\phi}{v}\) and \(\Phi\).

  In other words, \(\pv{\mu}{v}\) is not required to conclude \(\pv{\phi}{v}\) from \(\Phi\).

  Now,~\ref{def:itp:b:distinct} is a somewhat arbitrary restriction.

  In general, it is plausible that some \(\pv{\mu}{v}\) may both inform an agent that they may conclude \(\pv{\phi}{v}\) from \(\Phi\), but is also a member of \(\Phi\).

  Indeed, consider the conditional `if \(\pv{\alpha}{v}\) then \(\pv{\beta}{v'}\)'.
  Granting the conditional allows detachment, then it is surely possible for an agent to reason from the pool of premises \(\{\pv{\alpha}{v}, \text{if} \pv{\alpha}{v} \text{ then } \pv{\beta}{v'}\}\) to \(\pv{\beta}{v'}\).%
  \footnote{
    So long as the agent has already concluded \(\pv{\alpha}{v}\).
  }

  Indeed, without~\ref{def:itp:b:distinct}, \itp{1} would be abundant.
  Hence,~\ref{def:itp:b:distinct} narrows our attention to cases of interest, cases where \(\pv{\mu}{v}\) merely describes --- and does not partake --- in the reasoning of interest.%
  \footnote{
    Our course, ruling out an abundance of potential \itp{1} via~\ref{def:itp:b:distinct} carries risk of ruling out interesting \itp{1}.
    I encourage further investigation.
    Still, as \itp{1} are only of indirect interest, simplicity via arbitrary restrictions is favoured over complexity from general definitions.
  }
\end{note}

\begin{note}[Summary of \itp{0} between]
  \dots
\end{note}

\subparagraph*{An \itp{} for \(\pv{\phi}{v}\)}

\begin{note}[\itp{2} for]
  We now turn to the second helper definition, that of an \itp{0} for some proposition-value pair.
  In short, an \itp{0} \emph{between} \(\pv{\phi}{v}\) and \(\Phi\) ensures the possibility of the agent concluding \(\pv{\phi}{v}\) from \(\Phi\).
  And, by contrast, an \itp{0} \emph{for} \(\pv{\phi}{v}\) ensures there is some \(\Phi\) such that it is possible for the agent to conclude \(\pv{\phi}{v}\) from \(\Phi\).
\end{note}

\begin{note}[def: \itp{2} for]
  \begin{definition}[An \itp{0} for \(\pv{\phi}{v}\) \dots --- \named{I.f}]
    \label{def:itp:f}
    \(\pv{\mu}{v}\) is an \itp{} \emph{for} \(\pv{\phi}{v}\) if and only if:
    \begin{enumerate}[label=\arabic*., ref=\named{I.f:\arabic*}]
    \item
      \label{def:itp:f:pR}
      \(\mu\) having value \(v\) ensures there is some pool of proposition-value pairs \(\Phi\) and proposition-value pair \(\pv{\mu'}{v'}\) such that:
      \begin{itemize}
      \item
        \(\pv{\mu'}{v'}\) is an \itp{} between \(\pv{\phi}{v}\) and \(\Phi\).
      \end{itemize}
    \item
      \label{def:itp:f:distinct}
      \(\pv{\mu}{v}\) is not equivalent to any \(\pv{\phi_{i}}{v_{i}}\), given \vAgent{}'s present epistemic state.
    \end{enumerate}
    \vspace{-\baselineskip}
  \end{definition}
\end{note}

\begin{note}
  As with~\autoref{def:itp:b},~\autoref{def:itp:f} contains two components; a core and a restriction.
\end{note}

\begin{note}
  The core is straightforward.
  \ref{def:itp:f:pR} requires \(\pv{\mu}{v}\) to do ensure two things:
  \begin{enumerate}
  \item The existence of some pool of proposition-value pairs \(\Phi\), and
  \item The existence of an \itp{0} between \(\pv{\phi}{v}\) and \(\Phi\).
  \end{enumerate}
  In other words, then, an \itp{0} \emph{for} \(\pv{\phi}{v}\) just is the guarantee of an \itp{0} \emph{between} \(\pv{\phi}{v}\) and some pool of premises \(\Phi\).

  Simply as it may be, the definition of a \itp{0} for will prove quite useful, as we avoid the need to specify any particular pool of premises.

  Further, any \itp{0} between \(\pv{\phi}{v}\) and \(\Phi\) is always an \itp{0} for \(\pv{\phi}{v}\).
  For, suppose \(\pv{\mu}{v}\) is an \itp{0} between \(\pv{\phi}{v}\) and \(\Phi\).
  Then, \(\Phi\) is the relevant pool of premises and \(\pv{\mu}{v}\) itself is the relevant \(\pv{\mu'}{v'}\), and by definition \(\pv{\mu}{v}\) is not equivalent to any \(\pv{\phi_{i}}{v_{i}}\) in \(\Phi\), given the agent's present epistemic state.

  Note, however, that in general, if \(\pv{\mu}{v}\) is an \itp{0} for \(\pv{\phi}{v}\) then \(\pv{\mu}{v}\) need not be an \itp{0} between \(\pv{\phi}{v}\) and some pool of premises.
  Simply, though any \itp{0} between is also an \itp{0} for, the converse does not hold.
  For, an \itp{0} requires a pool of premises to be specified.
  Therefore, \(\pv{\mu}{v}\) and \(\pv{\mu'}{v'}\) must, in general, be distinct.

  Of course, an \itp{0} for \(\pv{\phi}{v}\) may have the same general statement as an \itp{0} between \(\langle \phi,\Phi \rangle\).

  Consider again tic-tac-toe.
  Above, we considered the \itp{0} between the existence of a strategy for first player in a game to guarantee a draw from the rules of tic-tac-toe.
  Though, with a moments reflection the statement that there existence of a strategy for first player in a game to guarantee a draw will typically lead to an \itp{0} for the existence of the relevant strategy.
  For, some premises must exist, and given the simplicity of tic-tac-toe these are surely within the grasp of an agent who understands the rules of tic-tac-toe.
\end{note}

\begin{note}
  Finally, the restriction~\ref{def:itp:f:distinct} functions in parallel to the restriction \ref{def:itp:b:distinct} of~\autoref{def:itp:b}.
  An \itp{0} for \(\pv{\phi}{v}\) is purely descriptive, and does not participate in concluding \(\pv{\phi}{v}\) from the relevant pool of premises.
  A more general definition may be given without this restriction, but such a definition is beyond present interest.
\end{note}

\paragraph*{\adB{}}

\begin{note}
  With the two helper definitions in hand, let us return to the definition of \adB{}.

  Recall two components:~\ref{def:adB:itp} and~\ref{def:adB:conclude}.

  \ref{def:adB:itp} stated that the agent has concluded \(\pv{\mu}{v}\) where \(\pv{\mu}{v}\) is either:
  \begin{enumerate}[label=(\alph*)]
  \item An \itp{} \emph{between} \(\pv{\phi}{v}\) and \(\Phi\), or
  \item An \itp{} \emph{for} \(\pv{\phi}{v}\), with \(\Phi\) as the relevant pool of premises
  \end{enumerate}
  We have seen the relevant definitions.

  So, given the agent has concluded \(\pv{\mu}{v}\), for some \itp{0} \(\pv{\mu}{v}\) then it is possible for the agent to conclude \(\pv{\phi}{v}\) from some pool of premises \(\Phi\).

  \ref{def:adB:conclude} is key.

  The agent concludes \(\pv{\phi}{v}\) by appeal to the pool of premises \(\Phi\) \emph{via} the possibility of witnessing the relevant reasoning from \(\Phi\) to \(\pv{\phi}{v}\) given by the \itp{0} \(\pv{\mu}{v}\).

  Hence, the agent does \emph{not} conclude \(\pv{\phi}{v}\) from \(\pv{\mu}{v}\) or indeed from some pool of premises for which \(\pv{\mu}{v}\) is a member.
  Indeed, the latter point follows from~\ref{def:itp:b:distinct} and~\ref{def:itp:f:distinct} --- an \itp{0} is always purely descriptive of some possible reasoning.

  Instead, the pool of premises the agent concludes \(\pv{\phi}{v}\) from is the collection of those premises that they agent would appeal to if they were to witness the relevant instance of reasoning given by the \itp{0}.

  Of course, the presence of an \itp{0} is, intuitively, crucial.
  For, without an \itp{0} the agent would lack information that it is possible to conclude \(\pv{\phi}{v}\) from some pool of premises.
\end{note}

\paragraph{A pair of \illu{3}}

\begin{note}
  To \illu{0} \adA{} and \adB{} we work through two \illu{1} in some detail.
  Both \illu{1} share two components:
  \begin{enumerate}[label=\alph*., ref=(\alph*)]
  \item
    \label{adX:illu:struc:mem}
    Memory of creating a syntactic proof for some first order formula.
  \item
    \label{adX:illu:struc:concl}
    Concluding that the relevant formula is a theorem of first-order logic.
  \end{enumerate}

  The key difference between the two \illu{1} is whether the memory~\ref{adX:illu:struc:mem} serves as a premise or an \itp{0} for the conclusion~\ref{adX:illu:struc:concl}.

\end{note}
\begin{note}[Two premises]
  \begin{quote}
    \begin{enumerate}[%
      label={(Mem)},%
      ref={(Mem)}%
      ]
    \item
      \label{ill:Eproof:mem}
      I remember having created a syntactic proof of \formula{\forall x Px \rightarrow \lnot \exists x \lnot P x} (using a sound first-order system).%
      \footnote{
        We use the phrasing `having created' rather than `creating' to imply completion.
      }
    \end{enumerate}
  \end{quote}
  And:
  \begin{quote}
    \begin{enumerate}[%
      label={(\(\exists\mathord{\vdash}{,}\top\))},%
      ref={(\(\exists\mathord{\vdash}{,}\top\))}%
      ]
    \item
      \label{ill:Eproof:def}
      The existence of a syntactic proof of a formula (using a sound first-order system) is sufficient to establish the formula is a (syntactic) theorem of first-order logic.
    \end{enumerate}
  \end{quote}
\end{note}

\paragraph{First \illu{0} (\adA{})}

\begin{note}
  \begin{illustration}[\adA{}]
    \label{ill:ad:proof:mem}
    \mbox{}
    \vspace{-\baselineskip}
    \begin{enumerate}[%
      label=\arabic*.,%
      ref=({I}.{\ref{ill:ad:proof:mem}}:\arabic*)%
      ]
    \item
      \illEproofMem{} \hfill \ref{ill:Eproof:mem}
    \item
      \label{ill:Eproof:exP}
      So, there exists a syntactic proof of \formula{\forall x Px \rightarrow \lnot \exists x \lnot P x} (using a sound first-order system)
    \item
      \label{ill:Eproof:thm}
      Hence, by \ref{ill:Eproof:def}, \formula{\forall x Px \rightarrow \lnot \exists x \lnot P x} is a theorem of first-order logic.
    \end{enumerate}
    \vspace{-\baselineskip}
  \end{illustration}
\end{note}

\begin{note}[Discussion of \autoref{ill:ad:proof:mem}]
  I take \autoref{ill:ad:proof:mem} to be a straightforward case of concluding.
  \ref{ill:Eproof:mem}, memory,


  and \ref{ill:Eproof:def}, to recast the existence of a proof from \ref{ill:Eproof:mem} in terms of the formula being a theorem.

  Neither premise from anything more basic, and without either premise the conclusion would not be obtained.
  For, without \ref{ill:Eproof:mem} no proof, and without \ref{ill:Eproof:def} no recasting.
\end{note}

\begin{note}
  It seems sufficient, generally speaking, to conclude some proposition has value \(v\) by appeal to memory, hence the agent claims support that there was some event which culminated in a syntactic proof of the formula.

  Of course, the agent may have misremembered.
  Still, we do not require that any agent concludes \(\pv{\phi}{v}\) from \(\Phi\) only if \(\phi\) \emph{actually} has value \(v\) and each \(\pv{\phi_{i}}{v_{i}}\) in \(\Phi\), \(\phi_{i}\) \emph{actually} has value \(v_{i}\).

  Following, this allows the agent to conclude a syntactic proof of the formula exists.
  As before, the agent may have failed to \emph{actually} create a syntactic proof of the formula
  Still, from the same perspective this does not prevent the agent from concluding they did (actually) create such a proof.

  Hence, finally, the agent claims support that the formula is a (syntactic) theorem of first-order logic.
\end{note}

\begin{note}
  To concisely summarise, we may say that the agent conclude the is a (syntactic) theorem of first-order logic \emph{because} of their understanding of syntactic theorem-hood and their memory of proving the formula.

  For sure,~\autoref{ill:ad:proof:mem} is designed to be as straightforward as possible.
  Of interest is not whether the agent claims support, but how the role the agent gives to their memory in claiming support.

  The agent appeals to their memory to establish that there exists a syntactic proof of the formula, and then combines the existence of a syntactic proof with~\ref{ill:Eproof:def} to claim support that the formula is a theorem.
  Hence, the agent's memory is directly involved in their claimed support for the formula being a theorem.%
    \footnote{
      \color{red}
      Whether proving is an unsatisfied \requ{}.
      However, recall that allowed a \requ{} to be satisfied by some instance of concluding.
      And, memory of concluding.
      Still question about original proof, but no problem with memory.
      Also, method.
    }
\end{note}

\paragraph{Second \illu{0} (\adB{})}

\begin{note}
  \begin{illustration}[\adB{}]
    \label{ill:ad:proof:eve}
    \mbox{}
    \vspace{-\baselineskip}
    \begin{enumerate}
    \item \illEproofMem{} \hfill \ref{ill:Eproof:mem}
    \item
      \label{ill:ad:proof:eve:app}
      In creating the syntactic proof I appealed to various aspects of some sound first-order system.
    \item
      \label{ill:ad:proof:eve:pos}
      As I created a proof, those various aspects of the sound first-order system are sufficient to ensure there exists a proof.
    \item
      Hence, by \ref{ill:Eproof:def}, \formula{\forall x Px \rightarrow \lnot \exists x \lnot P x} is a theorem of first-order logic.
    \end{enumerate}
    \vspace{-\baselineskip}
  \end{illustration}

  {
    \color{red}
    Premises are rules are part of a sound system, may be combined in this way.%
    \footnote{
      Indeed, relative simplicity is why we chose a syntactic rather than semantic proof.
      With semantic, need an argument which covers all models, and while premises plausibly exist, these have no common codification.
    }
  }

  As with~\autoref{ill:ad:proof:mem}, the agent's memory has a role in~\autoref{ill:ad:proof:eve}, but the role is quite different.
  Above, the agent claimed support for the formula being a theorem primarily \emph{because} they remembered creating a proof.
  By contrast, here the agent claims support for the formula being a theorem primarily because of the properties of some sound first-order system.

  Step~\ref{ill:ad:proof:eve:app} appeals to various aspects of some sound first-order system and, in turn, step~\ref{ill:ad:proof:eve:app} observes that those aspects are sufficient to ensure a proof exists.
  The agent claims support for the existence of a proof by appeal to the various aspects of some first-order system they appealed to when constructing the proof, rather than their memory of constructing the proof.
\end{note}

\begin{note}
  To help clarify, let's fix a particular syntactic proof using the Fitch-style proof system of~\textcite[557--560]{Barwise:1999tu}:

  \begin{figure}[H]
    \centering
    \begin{quote}
      \fitchprf{}{
        \subproof{\pline[1.]{\forall x P x}}{
          \subproof{\pline[2.]{\exists x \lnot Px}}{
            \boxedsubproof[3.]{a}{\lnot Pa}{
              \pline[4.]{Pa}[\lalle{1}] \\
              \pline[5.]{\bot}[\lfalsei{3}{4}]
            }
            \pline[6.]{\bot}[\lexie{2}{3--5}]
          }
          \pline[7.]{\lnot \exists x \lnot Px}[\lnoti{2--6}]
        }
        \pline[8.]{\forall x Px \rightarrow \lnot \exists x \lnot Px}[\lifi{1--7}]
      }
    \end{quote}
    \caption{A syntactic proof}\label{fig:syntx-prf}
  \end{figure}

  The proof consists of single instances of five introduction or elimination rules.
  Each rule is part of the Fitch-style proof system, and the specific application of the rules constitute the proof.
\end{note}


\begin{note}[Before\dots]
  Before returning to~\autoref{ill:ad:proof:eve}, let us observe that with the proof in hand one may claim support that a proof of the formula exists via the contents of~\autoref{fig:syntx-prf}.

  Broadly stated:

  \begin{enumerate}
  \item The proof is constructed from a sound first-order proof system.
  \item And, the particular application of some rules of the system to formulae is such that the proof begins with no assumptions and the last line of the proof is not part of any assumption made during the course of the proof.
  \end{enumerate}
\end{note}

\begin{note}
  Note, appeal to creation of the proof involves appeal to various aspects of the Fitch-style proof system.

  The object itself is mute to whether or not it is a proof.

  For example, adding `\formula{Ba}' as an assumption would void the proof, but you would need to observe that the appeal to existential elimination on line 6 requires that `\formula{a}' does not appear in the proof prior to its introduction on line 3 in order to claim support that the proof is void.

  Indeed, the proof consists of eight steps, each step is permitted by the first-order system, the proof begins with no assumptions, the last line of the proof is not part of any assumption made during the course of the proof and the proof, and so on.

  Sparing the details, claimed support that~\autoref{fig:syntx-prf} is a syntactic proof of \formula{\forall x Px \rightarrow \lnot \exists x \lnot P x} from the creation of~\autoref{fig:syntx-prf} is a matter of claiming support for each step of the creation.

  Indeed, to spare the details in general, let us instead talk of some collection of propositions and steps of reasoning.
  Claiming support that a proof exists from the some creation in the way under discussion is an instance of reasoning from details of the creation to the conclusion that a proof exists.
  Hence, as an instance of reasoning involves certain premises and steps of reasoning.
  And, whatever these turn out to be, the proceed from the creation of the proof rather than from some other source such as memory, testimony, and so on.
\end{note}

\begin{note}
  In other words, one may claim support that a proof of \formula{\forall x Px \rightarrow \lnot \exists x \lnot P x} exists (primarily) \emph{because} of their reasoning from some collection of premises and steps of reasoning concerning the creation to the existence of a proof of \formula{\forall x Px \rightarrow \lnot \exists x \lnot P x}.
\end{note}

\begin{note}[Return to \ref{ill:ad:proof:eve}]
  Now let us return to the reasoning of~\autoref{ill:ad:proof:eve}, and in particular steps~\ref{ill:ad:proof:eve:app} and~\ref{ill:ad:proof:eve:pos}:
  \begin{quote}
    \begin{enumerate}
      \setcounter{enumi}{1}
    \item In creating the syntactic proof I appealed to various aspects of some sound first-order system.
    \item As I created a proof, those various aspects of the sound first-order system are sufficient to ensure there exists a proof.
    \end{enumerate}
  \end{quote}
  Given that the agent remembers having created a syntactic proof, the `various aspects of some sound first-order system' of step~\ref{ill:ad:proof:eve} may be taken as those aspects of the first-order system that were appealed to in the premises and steps of reasoning when the agent created the proof.
  And step \ref{ill:ad:proof:eve}, in turn, appeals to how those various aspects of some sound first-order system were sufficient for the agent to claim support that a proof exists by the reasoning that occurred.

  In short, the agent remembers creating a syntactic proof and claiming support that a proof exists from the creation.
  The instance of claiming support involved reasoning from premises via steps to the relevant conclusion.
  Hence, it is possible to claim support for the conclusion by those premises and steps of reasoning.
  So, in~\ref{ill:ad:proof:eve} the agent observes that those premises and steps of reasoning are sufficient to claim support by way of their memory, and in turn appeals to those premises and steps of reasoning to claim support for the relevant conclusion.
\end{note}

\begin{note}
  {
    \color{red}
    Propositional support.
    (If I talk about this, it should be after the definitions.)
  }
\end{note}

\begin{note}
  Generalising, the way in which the agent claims support in~\autoref{ill:ad:proof:eve} is of interest because the agent appeals to premises and steps of reasoning that are not `part' of their present reasoning.
  The role of memory in the \illu{0} is (merely) a way for the agent to recognise that there are such premises and steps of reasoning.
  And, in the definitions that follow, we will abstract from any particular way in which the recognises that relevant premises and steps of reasoning are available.

  Still, even though memory is contingent, we may briefly observe that the way in which the agent claim support in~\autoref{ill:ad:proof:eve} is compatible with \ESU{}.
  For, \ESU{} requires that an agent may claim support for some conclusion from premises and steps of reasoning only if the agent has witnessed reasoning to the conclusion from those premises via those steps of reasoning.
  So, if the initial instance of claiming support conformed to \ESU{} then the agent will have witnessed reasoning from those steps and premises to the conclusion --- the instance of claiming support in~\autoref{ill:ad:proof:eve} does not involve such witnessing, but the agent's memory would be about how the relevant premises and steps were used to claim support.

  Of course, the way in which the agent claim support in~\autoref{ill:ad:proof:eve} is incompatible with a strengthened variant of \ESU{} which requires the agent to use any premises and steps they appeal to in the \emph{present} instance of reasoning, but the point for the moment is that the way in which the agent claims support in~\autoref{ill:ad:proof:eve} does not already require what we are arguing against: \ESU{}.
\end{note}

\begin{note}
  \color{red}
  In this case, \adB{} is not incompatible with \ESU{}.
  For, the agent has witnessed reasoning (granting memory).
  So, \ESU{} does not lead to an immediate rejection of \adB{}.

  Oh, this is noted.
\end{note}

\subsubsection{Additional illustrations}

\begin{note}

  \begin{illustration}
    \mbox{}
    \vspace{-\baselineskip}
    \begin{itemize}
    \item If bag are overweight then they can't be taken on the flight.
    \item Machine reads\dots
    \item Bag can't be taken on the flight.
    \end{itemize}
  \end{illustration}
  Contents of the bag are overweight.

  Combined weight of the items versus the combination of the individual weights.

  Compare, filling the bag and weighing it, versus summing the weight of the items as you fill the bag.

  Now, seems possible to fill the bag and weight it, then appeal to the sum of the items.

  So, this is a little more subtle.
  The bag has been weighed, and the distinction is between the weight of the contents of the bag, and the combined weight of the items that make up the contents of the bag.

  This is particularly interesting.
  Because, it seems clear that something is strange if someone talks about the weight of the contents of the bag without recognising that this is a function of the combined weight of all the individual elements of the bag.
  However, no idea what the contents of the bag are.

  So, claiming support from what is has been observed, the combined weight, rather than what must be the case in order to have made the observation.
\end{note}

%%% Local Variables:
%%% mode: latex
%%% TeX-master: "master"
%%% End:
