\chapter{Events, in progress}
\label{cha:events-progress}


\begin{note}
  Our interest is understanding the way an \eiw{0} an agent concludes happens.

  This chapter outlines our understanding of both events and events (such that some other event is) in progress.

  This understanding is then applied to an \eiw{1} an agent concludes and \eiw{0} an agent concludes is in progress --- or colloquially an \eiw{0} an agent is concluding --- throughout the rest of the document.

  A key takeaway of this chapter is `\progEx{}' which expands on \autoref{idea:why} (\autoref{cha:intro}, \autopageref{idea:why}) and characterises the sense of `why' present in \qWhy{} in terms of features of an event in progress.
\end{note}


\section{Events}
\label{sec:events}

\begin{note}
  We understand events in a broadly (Neo-)\citeauthor{Davidson:1967aa}ian framework.
  In short:
  Events are things we refer to by way of descriptions which are true of the event.%
  \footnote{
    Keeping track of both events and descriptions gives me a headache, and perhaps it gives you a headache too.
    I tried to avoid the complication, but the headache was almost unbearable.

    Notation introduced below is designed so that a description is easy to ignore when it is of little importance.
    And, when a description is important attention will be drawn to its importance.
    Unfortunately, descriptions are fairly important through this chapter, though less so in later chapters.
  }

  For example, a natural language sentence such as:
  % 
  \begin{enumerate}[label=\arabic*., ref=(\arabic*), series=ESERIES]
  \item
    \label{ESERIES:toast}
    Sam buttered some toast in the kitchen.
  \end{enumerate}
  % 
  Is understood as saying there is some event \(\edn{}\) such that \(\edn{}\) is a butter event, the agent of \(\edn{}\) is Sam, the theme of \(\edn{}\) is some toast, and the location of \(\edn{}\) is the kitchen.%
  \footnote{
    Alternatively:
    \(\exists e [\textsc{butter}(e)\text{ \& }\textsc{agent}(e, \text{Sam})\text{ \& }\exists x(\textsc{theme}(e, \text{toast}(x)))\text{ \& }\textsc{in}(e, \text{the kitchen})]\)
  }

  Likewise:
  % 
  \begin{enumerate}[label=\arabic*., ref=(\arabic*), resume*=ESERIES]
  \item
    \label{ESERIES:gistCalcEq}
    Max concludes \gistCalcEq{} has value \valI{True}.
  \end{enumerate}
  % 
  Is understood as sating there is some event \(\edn{}\) such that \(\edn{}\) is a conclude event, the agent of \(\edn{}\) is Max, the \prop{0} of \(\edn{}\) is \gistCalcEq{} and the \val{} is \valI{True}.

  As events are referred to by descriptions true of the event, a description captures a specific event only if there is a unique event which satisfies the description.

  So, neither \ref{ESERIES:toast} nor \ref{ESERIES:gistCalcEq} explicitly refer to a unique event.
  For example, the description of \ref{ESERIES:toast} is compatible with Sam buttering three or five pieces of toast and doing so yesterday or a few months ago.
  Likewise \ref{ESERIES:gistCalcEq} is compatible with Max using a calculator or their understanding of arithmetic to conclude \gistCalcEq{} has value \valI{True}.

  Still, when discussing events we will almost always have a specific event.
  For our purposes the role of descriptions is to highlight particular features of events.%
  \footnote{
    In particular, both \qWhy{} and \qHow{} are asked with respect to a specific event.
  }
  So, to ease the way we talk about events, we make use of following notation:

  \begin{notation}[Events and descriptions]%
    \label{assu:HaUniqueD}%
    \vspace{-\baselineskip}
    \begin{itemize}
    \item
      \(\edn{}\) picks out a specific event (i.e.\ an event under some unique description).
    \item
      \(\ed{}\) captures (the specific event) \(\edn{}\) under the particular description \(\edo{}\).
    \end{itemize}
    \vspace{-\baselineskip}
  \end{notation}

  \noindent%
  For example, given \autoref{assu:HaUniqueD} we may re-express \ref{ESERIES:gistCalcEq} to capture a specific event by writing:
  % 
  \begin{enumerate}[label=\arabic*\('\)., ref=(\arabic*\('\))]
    \setcounter{enumi}{1}
  \item
    The event \(\edn{}\) under description \(\edo{}\) `Max concludes \gistCalcEq{} has value \valI{True}'.
  \end{enumerate}
  % 
  Where \(\edn{}\) captures all features of the event, such as when the event took place, the method by which Max concluded \gistCalcEq{} has value \valI{True}, how many times Max blinked while concluding, and so on.
\end{note}



\begin{note}
  Some technicalities (such as our use of \prop{1}, \val{1}, and \pool{1}) influence the descriptions we use, but something like `the agent concluded \gistCalcEq{} is true by using a calculator' is about the level of detail I have in mind, though what follows is compatible with additional detail.
\end{note}




\section{Events in progress}
\label{sec:events-progress}


\begin{note}
  Our interest is understanding the way an \eiw{0} an agent concludes happens.
  And, the idea of an event is progress is a key idea with respect to our understanding of the way an \eiw{0} an agent concludes happens.

  This section briefly characterises the way we understand events in progress, and states two important assumptions we may about events in progress.

  Stated a little more carefully, our interest is with events such that the event is described as being an \eiw{0} some other event is in progress.
  Still, `events in progress' is a little easier to parse.
\end{note}



\subsection{The progressive}
\label{sec:progressive}


\begin{note}
  Events in progress are intuitively understood via the progressive aspect.

  For example:
  % 
  \begin{enumerate}
  \item
    The agent is making soup.\newline
    \mbox{ } \hfill \(\leadsto\) An \eiw{0} the agent makes soup is in progress.
  \item
    The agent is reading Henley's `Invictus'.\newline
    \mbox{ } \hfill \(\leadsto\) An \eiw{0} the agent reads Henley's `Invictus' is in progress.
  \item
    The agent is riding the slope.\newline
    \mbox{ } \hfill \(\leadsto\) An \eiw{0} the agent rides the slope is in progress.
  \end{enumerate}
  % 
  Specifically:

  \begin{intuition}[Events in progress and the progressive]
    \label{def:es-in-prog}
    \vspace{-\baselineskip}
    \begin{itemize}
    \item
      \(\ed{\flat}\) is an \eiw{0} \(\ed{}\) is in progress.
    \end{itemize}
    % 
    \emph{If and only if}:
    % 
    \begin{itemize}
    \item
      \(\edn{}\) is a (maybe non-actual) development of \(\edn{\flat}\).
    \item
      \(\edo{\flat}\) entails \(\edo{}\) is happening.\newline
      \mbox{ }\hfill (Where \(\edo{}\) is happening is understood in terms of the progressive aspect.)
    \end{itemize}
    \vspace{-.5\baselineskip}
  \end{intuition}

  \noindent%
  For example, going by \autoref{def:es-in-prog}, the following are equivalent:
  \begin{itemize}
  \item
    \(\ed{\flat}\) is an \eiw{0} Max concludes \pv{\propM{\gistCalcEq{}}}{\valI{True}} is in progress.
  \item
    \(\edo{\flat}\) is true of \(\edn{\flat}\) and entails Max concludes \pv{\propM{\gistCalcEq{}}}{\valI{True}} is happening.
  \item
    \(\edo{\flat}\) is true of \(\edn{\flat}\) and entails Max is concluding \propM{\gistCalcEq{}} is \valI{True}.
  \end{itemize}

  \noindent%
  We assume an implicit understanding of the progressive aspect,%
  \footnote{
    \nocite{Portner:1998um}
    \nocite{Engelberg:1999vi}
    Note, though, that English does not have a quick, unambiguous, way of expressing events in progress.
    For, consider the sentence:
    \begin{enumerate}[label=\arabic*., ref=(\arabic*)]
    \item
      \label{prog:abmig}
      \textquote{John is studying for an exam}.
    \end{enumerate}
    \ref{prog:abmig} may be understand to express either the continuous or progressive aspect.

    Under the continuous aspect, \ref{prog:abmig} captures something about John, rather than something about an event happening.
    Hence, it need to be the case that John is engaged in an event of studying when \ref{prog:abmig} is said.
    For example, we may expand:
    \textquote{Sam is studying for an exam, but is taking a short nap.}

    By contrast, \ref{prog:abmig} under the progressive captures an event in where John studying is in progress.
    For example, we may expand:
    \textquote{Sam is studying for an exam, so they aren't taking a nap.}

    See,~\textcite{Richards:1981wo},~\textcite{Portner:2011vi}, etc for a general overview of the progressive.
    In particular, I suggest \textcite{Landman:1992wh} as an nice technical introduction.
    \citeauthor{Szabo:2004ul} (\citeyear[34]{Szabo:2004ul}) provides a concise summary:
    \begin{quote}
      [A] progressive sentence is true at some time just in case some event occurs at that time, and if we follow the development of the event (within our world as long as it goes, then jumping into a nearby world, and iterating the process within the limits of reasonability) we will reach a possible world where the perfective correlate is true of the continuation.
    \end{quote}
    For a more in depth summary see (\cite[764--766]{Portner:1998um}) and for some issues with \citeauthor{Landman:1992wh}'s account, see
    (\cite{Bonomi:1997uq}),
    (\cite[49--50]{Engelberg:1999vi}),
    (\cite[35]{Szabo:2004ul}),
    (\cite[767]{Portner:1998um}),
    and (\cite[1256]{Portner:2011vi}).
  }
  and explicitly assume a common feature of analyses of the progressive holds of events in progress:%
  \footnote{
    See, e.g.:
    (\cite{Bennett:1972uw}),
    (\cite{Dowty:1979vq}),
    (\cite{Parsons:1990aa}),
    (\cite{Landman:1992wh}), and
    (\cite{Portner:1998um}).

    \assuPP{2} is sometimes motivated by the imperfective paradox (\cite[Ch.3.1]{Dowty:1979vq}, \cite[12]{Bach:1986tb}).
  }

  \begin{assumption}[\assuPP{2}]%
    \label{assu:PP}%
    \vspace{-\baselineskip}
    \begin{itenum}
    \item[\emph{If}:]
      \(\ed{\flat}\) is an \eiw{0} \(\ed{}\) is in progress.
    \item[\emph{Then}:]
      There is some \progAdj{0} event \(\edn{\sharp}\) such that~\ref{assu:PP:pe:dev} and~\ref{assu:PP:pe:verb} are both true:
      \begin{enumerate}[label=\roman*., ref=(\roman*)]
      \item
        \label{assu:PP:pe:dev}
        \(\edn{\sharp}\) is a development of \(\edn{\flat}\).
      \item
        \label{assu:PP:pe:verb}
        \(\edo{\sharp}\) is true of \(\edn{\sharp}\), where \(\edo{\sharp}\) is the perfective correlate of \(\edo{}\).
      \end{enumerate}
    \end{itenum}
    \vspace{-\baselineskip}
  \end{assumption}
\end{note}

\begin{note}[Interest with the progressive]
  The immediate role of \assuPP{} is to help fix intuitions about events in progress by reflecting on the sense of possibility at issue.%
  \footnote{
    \assuPP{2} is denied by some.
    For example, \citeauthor{Szabo:2004ul} argues:
    \textquote{Sometimes we are \emph{doing} things even though there is no real chance that we could get them \emph{done}, and this is true even if we abstract away from the possibility of miraculous intervention.}
    (\citeyear[40]{Szabo:2004ul})
    E.g., \citeauthor{Szabo:2004ul} denies~\ref{Szabo:Arch} is necessarily false:
    \begin{quote}
      \begin{enumerate}[label=(\arabic*), ref=(\arabic*)]
        \setcounter{enumi}{12}
      \item
        \label{Szabo:Arch}
        As the architect was building the cathedral he knew that, although he would be building it for another year or so, he couldn't possibly complete it.%
        \mbox{ }\hfill\mbox{(\citeyear[38]{Szabo:2004ul})}
      \end{enumerate}
    \end{quote}
    Though,~\ref{Szabo:Arch} seems always false to me.
    The only sense with which I read~\ref{Szabo:Arch} as true under the progressive requires factivity of knowledge to fail, thus allowing the cathedral to be built.
    I'd rephrase \autoref{Szabo:Arch} to state the architect was building \emph{part of} the cathedral and knew they couldn't possible complete the entire thing.

    See (\cite[1245]{Portner:2011vi}) for additional, distinct, discussion of (\cite{Szabo:2004ul}).
  }
  The deferred role of \assuPP{} is to help motivate a couple of ideas introduced in later chapters, and ease a handful of arguments.%
  \footnote{
    Specifically, \fc{1} in \autoref{cha:fcs} and \fingfr{1} in \autoref{cha:ros}.
  }

  We make three observations to help firm intuition.

  \begin{observation}[\assuPP{2} and existential modality]%
    \label{obs:prog-not-reg-poss}%
    The sense of possibility in \assuPP{} does not reduce to existential \{logical, metaphysical, nomic, \dots\} possibility.
  \end{observation}
  \begin{motivation}{obs:prog-not-reg-poss}
    Suppose an agent is sitting a multiple choice exam.
    To pass the exam the agent only needs to make a sufficient number of correct choices.
    It is certainly logically, metaphysically, and nomically possible that the agent chooses a sufficient number of correct choices.
    However, it does not follow that the agent is passing the exam, as there need be no guarantee the agent is makes a sufficient number of correct choices.%
    \footnote{
      See also Igal Kvart's example of Mary wiping out the Roman army (\cite[18]{Landman:1992wh}).
    }
  \end{motivation}

  \begin{observation}[\assuPP{2} and counterfactuals]%
    \label{obs:prog-not-cfs}%
    There is no simple relation between the sense of possibility in \assuPP{} and a `close-world' analyses of counterfactuals.%
    \footnote{
      E.g.\ as found in (\cite{Todd:1964aa}), (\cite{Stalnaker:1968vt}), and (\cite{Lewis:1973th}).
      See (\cite[\S2]{Starr:2022aa}) for an overview.
    }
  \end{observation}
  \begin{motivation}{obs:prog-not-cfs}
    Suppose an agent is passing an exam without external help.
    Then, a classmate passes the agent answers, which the agent uses.
    The agent is no longer passing the exam without external help.
    And, any close possible world where the classmate does not pass answers, it may be that some other classmate does.
  \end{motivation}

  \begin{observation}[\assuPP{2} and uniqueness]%
    \label{obs:prog-no-unique}%
    An event may be in progress without the event being sufficiently developed to `indicate' a unique outcome.\newline
  \end{observation}
  \begin{motivation}{obs:prog-no-unique}
    Suppose an agent has drawn a straight line on a piece of paper.
    It may be true that the agent is drawing a triangle.
    However, the straight line is compatible with the agent drawing an \(n\)-sided polygon, for any \(n\) within a some reasonable bound.%
    \footnote{
      This observation is inspired by \citeauthor{Dowty:1979vq}'s example involving a circle and a triangle (\citeyear[133]{Dowty:1979vq}).
    }
  \end{motivation}

  \noindent%
  Loosely paraphrased, if an event is in progress, then intuitively there is \emph{something} about the way things are which ensures the existence of a possible completion event (\autoref{obs:prog-no-unique}) which is robust against external influence (\autoref{obs:prog-not-cfs}) and does not require luck (\autoref{obs:prog-not-reg-poss}).
\end{note}


\begin{note}
  Finally, a brief observation highlights a quirk \autoref{assu:HaUniqueD}:

  \begin{observation}[Events in progress and common satisfaction]%
    \label{obs:eip-partial}%
    Sometimes, when \(\ed{}\) is an \eiw{0} \(\ed{\ast}\) is in progress, \(\edo{\ast}\) may be satisfied by multiple events.
  \end{observation}

  \begin{motivation}{obs:eip-partial}
    Consider an agent flipping a coin until it lands heads or lands tails ten times in a row.
    Given sufficient determination from the agent, it is true that an \eiw{0} the agent flips a coin until it lands heads or lands tails ten times in a row is in progress.

    Now, a unique \eiw{0} the agent flips a coin until it lands heads or lands tails ten times in a row involves an \(i\)th throw the coin lands heads on, or ten throws where the coin lands tails.
    However, by \autoref{cons:no-f-ref}, it must be true of \(\ed{}\) that \(\ed{\ast}\) is in progress.
    And, as coin flips are quite random (\cite{Gelman:2002ww}), \(\ed{}\) may fail to entail that the coin lands heads on the \(i\)th, or that the coin lands tails on all ten throws.
    Hence, \(\ed{\ast}\) may be satisfied an such event.
  \end{motivation}
\end{note}


\begin{note}
  Note, given \autoref{assu:HaUniqueD}, \(\ed{\ast}\) being an \eiw{0} \(\ed{}\) is in progress is distinct from \(\ed{\ast}\) being an \eiw{0} \(\edn{}\) is in progress.
  For, \(\ed{}\) being in progress only requires the features of \(\edn{}\) as captured by description \(\edo{}\) are in progress.
  And, by contrast, \(\edn{}\) being in progress requires all the features of \(\edn{}\) are in progress.

  For example, suppose \(\edn{}\) is an \eiw{0} an agent completes a crossword without using a dictionary and an earlier event \eiw[\(e^{\ast}\)]{0} the agent is completing the crossword.
  Given the description \(\edo{\ast}\) `the agent is completing the crossword' it follows \(\edn{}\) under the description `the agent completes the crossword' is in progress.
  However, it does not follow that the agent completing the crossword without using a dictionary is in progress.
  For, without expanding \(\edo{\ast}\) we have no guarantee the agent may solve the crossword without using a dictionary.

  In this respect, it may seem a little odd to mention \(\edn{}\).
  For, to say \(\edn{}\) is in progress is really just to say there is \emph{some} event in progress such that \(\edo{}\) is true of the event.
  However, the option of referencing features of \(\edn{}\) not captured by \(\edo{}\) is used later.
\end{note}




\subsection{A constraint and an assumption}
\label{sec:assumptions-1}



\paragraph{The constraint}


\begin{note}
  We place one important constraint on descriptions:

  \begin{constraint}{Fog}{Fog}%
    \label{cons:no-f-ref}%
    A description \(\edo{}\) of an event \(\edn{}\) is limited to what is true of \(\edn{}\) when \(\edn{}\) happens.
  \end{constraint}

  \noindent%
  \autoref{cons:no-f-ref} rules out describing an event by using information what happens after the relevant event happens.
  And, \autoref{cons:no-f-ref} is important given the way hindsight interacts with descriptions of events.
  To illustrate, consider the following passage from \citeauthor{Bhatt:2008aa} (\citeyear{Bhatt:2008aa}) with respect to ability:%
  \footnote{
    Similarly, \citeauthor{Austin:1961vz} remarks `it follows merely from the premiss that he does it, that he has the ability to do it, according to ordinary English' (\citeyear[175]{Austin:1961vz}).
    See also \citeauthor{Boylan:2020aa}'s discussion of `Past Success' (\citeyear[\S1.1]{Boylan:2020aa}).
  }
  % 
  \begin{quote}
    \begin{enumerate}[label=(\arabic*)]
      \setcounter{enumi}{314}
    \item
      (from~\cite{Thalberg:1969ta})
      \begin{enumerate}[label=\alph*., ref=(315\alph*)]
      \item
        \label{Bhatt:Thal-a}
        Yesterday, Brown hit three bulls-eyes in a row.
        Before he hit three bulls-eyes, he fired 600 rounds, without coming close to the bullseye; and his subsequent tries were equally wild.
      \item
        \label{Bhatt:Thal-b}
        Brown was able to hit three bulls-eyes in a row.
      \item
        \label{Bhatt:Thal-c}
        Brown had the ability to hit three bulls-eyes in a row.
      \end{enumerate}
    \end{enumerate}
    % 
    From~\ref{Bhatt:Thal-a}, we can conclude~\ref{Bhatt:Thal-b} but not~\ref{Bhatt:Thal-c}.
    Brown could have hit the target three times in a row by pure chance and he does not need to have had any ability for~\ref{Bhatt:Thal-b} to be true.\newline
    \mbox{ }\hfill\mbox{(\citeyear[167]{Bhatt:2008aa})}
  \end{quote}
  % 
  Personally, I read \ref{Bhatt:Thal-b} and \ref{Bhatt:Thal-c} as synonyms.
  Still, the distinction \citeauthor{Bhatt:2008aa} highlights is clear.
  Expressed in terms of events in progress:
  When Brown was throwing darts it was not the Brown was throwing three bulls-eyes in a row, as Brown may have missed the dartboard on any throw.
  However, in hindsight one may say Brown was hitting three bulls-eyes in a row, as Brown hit three bulls-eyes in a row.

  Given \autoref{cons:no-f-ref} Brown was hitting three bulls-eyes only if something about the \eiw{0} Brown was throwing darts made it the case that Brown was hitting three bulls-eyes in a row regardless of what may be said with hindsight.

  The same applies to conclusions.
  For example, suppose an agent has completed a crossword.
  Intuitively, with hindsight an \eiw{0} the agent completes the crossword puzzle was in progress while the agent worked on the puzzle.
  However, it may not have been the case that an \eiw{0} the agent completes the crossword puzzle was in progress in the sense of interest.
  For, the agent may only have figured out a world by randomly stringing together vowels and consonants, and the agent may have failed to hit upon the right string to jog their memory.
\end{note}


\begin{note}
  In broader context, our interest in this document is understanding the way an \eiw{0} an agent concludes happens.
  The role of events in progress is to help understand the way an \eiw{0} an agent concludes happens, and \autoref{cons:no-f-ref} ensures any explanation given is not informed by hindsight.

  In short, the explanations of interest do not do so with the benefit of hindsight.
\end{note}



% \begin{note}
%   \nocite{Hackl:1998tt}
%   May be helpful to think of ability.
%   Still, fraught.

%   \nocite{Maier:2018uo}
%   First, in the terminology of \citeauthor{Whittle:2010wr} of interest are specific (or local) ability which concern `what the agent is able to do now, in some particular circumstances', rather than general (or global) abilities which concern `what an agent is able to do in a large range of circumstances'. (\citeyear[2]{Whittle:2010wr}).%
%   \footnote{
%   See also \citeauthor{Austin:1961vz}'s (\citeyear{Austin:1961vz}) distinction between `categorical' abilities and opportunities.

%   In particular, this sets aside accounts of ability such as~\citeauthor{Carter:2021wd}'s~(\citeyear{Carter:2021wd}) `fallibilist',~\citeauthor{Kikkert:2022wp}'s~(\citeyear{Kikkert:2022wp}) `robust', and \citeauthor{Maier:2013vk}'s (\citeyear{Maier:2013vk}) `general' account, among others.
% }

%   Second, specific ability is often accoppandied by a `control' intution.
%   For example,
%   `[A]n agent has the ability to \(\phi\) iff there are accessible worlds at which she \(\phi\)s simply by deciding to \(\phi\).' (\citeyear[19]{Schwarz:2020aa})

%   This is far stronger than an event in progress.

%   At issue is that a conclusion may be in progress even through there is no action available to the agent which results in the conclusion.
%   In particular, to adhere to some strategy is not to perform a single action.
%   For example, consider an agent working on a crossword.
%   So long as the agent is quite good at crosswords and the crossword is not too difficult, it is plausible an \eiw{0} the agent concludes all the words is in progress.
%   However, conclusion is not the result of a single act.
%   For, as the agent works through the crossword they develop additional hints, and reasoning builds on those hints.

%   Of course, \assuPP{} gets a world, but not necessarily that this is a result of deciding or any other action that may be specified in advance.%
%   \footnote{
%   Similar accounts of the control intuition are found in (\cite{Brown:1988tl}), (\cite{Horty:1995wu}), (\cite{Jaster:2020wv}), and (\cite{Kikkert:2022wp}).
%   Further, the control intuition is a key feature of both \citeauthor{Mandelkern:2017aa}'s (\citeyear{Mandelkern:2017aa}) and \citeauthor{Boylan:2020aa}'s (\citeyear{Boylan:2020aa}) `act conditional' analysis of ability.
%   Abstracting from a few minor details, the act conditional analysis of ability states:
%     %   
%   \begin{quote}
%     `\(S\) is able to \(\varphi\)' is true just in case there is some action \(A\) available to \(S\) such that if \(S\) tried to \(A\) then S would \(\varphi\).
%   \end{quote}
%     %   
%   Formally:
%     %   
%   \[%
%     \sem[c,w]{\text{S is able to }\varphi} = 1\text{ iff }\exists A \in \mathcal{A}_{S,c,w,t}\colon \forall v \in f_{c}(\text{S does }A,w),  \sem[c,v]{\varphi(S)} = 1%
%   \]
%     %   
%   Where:
%     %   
%   \begin{itemize}
%   \item
%     \(f_{c}\) is a selection function from proposition-world pairings to set of worlds.
%   \item
%     \(\mathcal{A}_{S,c,w}\) is the set of actions that are available to \(S\) in context \(c\) and world \(w\).
%   \end{itemize}
%     %   
%   In other words, \(S\text{ is able to }\varphi\) is true at some world \(w\) in context \(c\), just in case there is some action available to the agent, such that for every world for which it is true that \(S\text{ tries to}A\) determined by the selection function \(f_{c}\), it is the case that \(S \varphi\text{s}\).

%   The primary difference between the analyses of \citeauthor{Mandelkern:2017aa} and \citeauthor{Boylan:2020aa} is the specification of \(f_{c}\).
%   For \citeauthor{Mandelkern:2017aa},
%   \(f_{c}\) is~\citeauthor{Stalnaker:1968vt}'s selection function.
%   I.e.\ \(f_{c}(\psi, w) = \{\text{\emph{the} `closest' world to }w\text{ where }\psi\text{ is true.}\}\).
%   (\citeyear[Cf.][314]{Mandelkern:2017aa}).
%   (Though, a `\citeauthor{Lewis:1973th}ian' approach that allows multiple close worlds seems viable.)

%   In contrast, \citeauthor{Boylan:2020aa}'s \(f_{c}\) selects all (possibly indeterminate) worlds which are identical to \(w\) up until time \(t\) (in which \(S\) does \(A\)).
%   (\citeyear[\S3.3]{Boylan:2020aa})

%   For our purposes, the key part of the act conditional analysis is not the details of the relevant selection function, but the observation that \(S \varphi\)s \emph{follows from} \(S\text{ does }A\) in all worlds captured by the selection function.
%   This is the control intuition --- the agent doing some action results in \(\phi\) being the case.
% }
% \end{note}


\paragraph{The assumption}


\begin{note}
  Given \autoref{def:es-in-prog} the progressive helps fix intuitions about what it is for an event (under some description) to be such that some other event (under some other description) is in progress.
  However, we may the following key assumption which may conflict with intuitions about the progressive:

  \begin{assumption}[Exclusive progress]
    \label{assu:p:ex}
    \vspace{-\baselineskip}
    \begin{itenum}
    \item[\emph{If}:]
      It is not possible for \(\edo{\alpha}\) and \(\edo{\omega}\) to both be true of an event.
    \item[\emph{Then}:]
      For any events \(e\) and \(\ed{\flat}\):
      \begin{itemize}
      \item
        It is not possible for \(\ed{\flat}\) to be such that \(\ed[\alpha]{}\) \emph{and} \(\ed[\omega]{}\) are in progress.
      \end{itemize}
    \end{itenum}
    \vspace{-\baselineskip}
  \end{assumption}

  \noindent%
  Paraphrased, \autoref{assu:p:ex} assumes that if an event is in progress, then no incompatible event is (also) in progress.

  For example:
  It is not possible for both Team A wins the game of hockey and for Team B to wins the game of hockey to be true of an event.
  So, by \autoref{assu:p:ex}, it is not possible for Team A to be winning the game of hockey and for Team B to be winning the game of hockey.
  Hence, if Team A is winning the game of hockey, Team B is not winning the game of hockey.
\end{note}


\begin{note}
  I take \autoref{assu:p:ex} to be intuitive.
  Still, \autoref{assu:p:ex} does not clearly hold of the progressive.
  For example, \citeauthor{Landman:1992wh} writes:%
  \footnote{
    See \textcite{Bonomi:1997uq} for additional discussion.
  }

  \begin{quote}
    Suppose I was on a plane to Boston which got hijacked and landed in Bismarck, North Dakota.
    What was going on before the plane was hijacked?
    One thing I can say is:
    `I was flying to Boston when the plane was hijacked'
    This is reasonable.
    But another thing I could say is:
    `I was flying to Boston.
    Well, in fact, I wasn't, I was flying to Bismarck, but I didn't know that at the time'
    And this is also reasonable.%
    \mbox{ }\hfill\mbox{(\citeyear[30--31]{Landman:1992wh})}
  \end{quote}
  % 
  \citeauthor{Landman:1992wh}'s observation is about the progressive, and is only incompatible with \autoref{assu:p:ex} to the extent if it is reasonable to say `I was flying to Boston' \emph{and} `I was flying to Bismarck', then it was the case an event in which I was flying to Boston \emph{and} I was flying to Bismarck was in progress.

  However, I don't think this entailment makes sense.
  It is not possible to have flown to both Boston and Bismarck at the same time in the same way it is not possible for team A and team B to have won a game of hockey at the same time --- having flown to Boston entails one is not in Bismarck and a win for Team A entails a loss for Team B.

  Indeed, \citeauthor{Landman:1992wh}'s observation as stated may be compatible with this.
  For, \citeauthor{Landman:1992wh}'s first statement is explicitly clarified as given with respect to what they knew while \citeauthor{Landman:1992wh}'s second statement takes into account details about what happened after the event (cf.\ our \autoref{cons:no-f-ref}).
  \citeauthor{Landman:1992wh} does not make both statements from the same perspective, and only the second statement involves knowledge.

  Still, I see no real benefit in raising \autoref{def:es-in-prog} to a definition by digressing into an account of the progressive.
  Even if \autoref{assu:p:ex} does hold of the progressive,%
  \footnote{
    An account of the progressive which ties the progressive to causation, such as \citeauthor{Szabo:2004ul}'s (\citeyear{Szabo:2004ul}) account, arguably entails \autoref{assu:p:ex}.
  }
  an argument would amount to a significant digression.
  So, \autoref{assu:p:ex} captures an important way our understanding of events in progress \emph{may} differ from the progressive.
\end{note}



\section{\se{3} and \progEx{}}
\label{sec:se3-progex}


\begin{note}
  Events, events in progress.

  The role of events in progress is to help understand the way an \eiw{0} an agent concludes happens.
  This section outlines the way events in progress help understand the way an \eiw{0} an agent concludes happens.

  This section is split into two sub-sections.

  The first sub-section introduces the idea of a `\se{}'.
  And, with the idea of a \se{} in hand, the second section links \se{} to explanations which involve `why' with the sense of `why' present in \qWhy{} by a pair of propositions we collectively term `\progEx{}'.
\end{note}



\subsection{\se{3}}

\begin{note}
  An \se{} is an event \(\edn{\flat}\) under some description \(\edo{\flat}\) such that some other event \(\edn{}\) under description \(\edo{}\) is in progress and \(\ed{}\) (in part) as a result \(\ed{\flat}\).%
  \footnote{
    The `p' in `\se{}' stands for `progress' and the `r' stands for `result'.
  }

  We define \se{1}, make a few observations, and then apply the definition of a \se{} to \autoref{illu:gist:roots:F}.
\end{note}


\begin{note}
  \begin{definition}[\se{3}]
    \label{def:se}
    \vspace{-\baselineskip}
    \begin{itemize}
    \item
      \(\ed{\flat}\) is a \emph{\se{0}} of \(\ed{}\).
    \end{itemize}
    \emph{If and only if}:
    \begin{itemize}
    \item
      Clauses~\ref{assu:p:se:prog} and \ref{assu:p:se:hCon} hold:
      \begin{enumerate}[label=\Alph*., ref=\Alph*]
      \item
        \label{assu:p:se:prog}
        \(\ed{\flat}\) is such that \(\ed{}\) is in progress.
      \item
        \label{assu:p:se:hCon}
        \(\ed{}\) partly happens as a result of \(\ed{\flat}\).
      \end{enumerate}
    \end{itemize}
    \vspace{-\baselineskip}
  \end{definition}

  \noindent%
  \(\ed{\flat}\) being a \emph{\se{0}} of \(\ed{}\) concerns specific events \(\edn{\flat}\) and \(\edn{}\).
  And, of interest is whether specific descriptions \(\edo{}\) and \(\edo{\flat}\) capture a particular connexion between \(\edn{\flat}\) and \(\edn{}\).

  Still, our interest is only with parts of \(\edn{\flat}\) and \(\edn{}\), respectively.
  Intuitively:
  % 
  \begin{itemize}
  \item
    Clause~\ref{assu:p:se:prog} `looks forward':

    The description \(\edo{\flat}\) of \(\edn{\flat}\) captures that an event described by \(\edo{}\) is in progress.
  \item
    Clause~\ref{assu:p:se:hCon} `looks backward':

    \(\edn{}\) as described by \(\edo{}\) happened in part as a result of \(\edn{\flat}\) as described by \(\edo{\flat}\).
  \end{itemize}

  The role of Clause~\ref{assu:p:se:prog} is to ensure the event \(\edn{}\) is favoured over some other event.
  For, by \autoref{assu:p:ex} as \(\ed{}\) is in progress, it is not the case that \(\ed[x]{}\) is in progress, grating it is not possible for \(\edo{}\) and \(\edo{x}\) to both be true of an event.

  And, the role of Clause~\ref{assu:p:se:hCon} is to ensure \(\edn{}\) happens as a result of being favoured.
\end{note}


\begin{note}
  The role of clauses~\ref{assu:p:se:prog} and \ref{assu:p:se:hCon} combined to give answers to `why' (in the sense of \qWhy{}) \(\ed{}\) happened.
  Still, for the moment our interest is with what clauses~\ref{assu:p:se:prog} and \ref{assu:p:se:hCon} say rather than their role.

  As Clause~\ref{assu:p:se:prog} simply places a constraint on the relevant description \(\edo{\flat}\) our discussion focuses on Clause~\ref{assu:p:se:hCon}.
\end{note}


\begin{note}
  I have little to say in abstract about what is it is for an event to `partly' happen `as a result of' some other event.
  Roughly, it was not possible for \(\ed{}\) to happen without \(\ed{\flat}\) happening.
  In this respect, \(\ed{}\) only happened --- in part --- as a result \(\ed{\flat}\).%
  \footnote{
    I suspect `causally depends on' may be substituted in place of `partly happens as a result of'.

    For example, if \(\ed{}\) causally depends on \(\ed{\flat}\) \emph{just in case} \(\ed{}\) would not occur were \(\ed{\flat}\) not occur (\cite[cf.][1.1]{Menzies:2020aa}), then the agent buttering toast causally depends on the bread being toasted.

    However, in contrast to an event being partly being the result of some other event I doubt there is much pre-theoretical intuition for what causal dependence amounts to.

    A number of Aesop's fable, in translation at least, talk about events being results of other events and I have not found a (translation of a) fable that talks about causal dependence.
    E.g.:
    % 
    \begin{quote}
      A deer had fallen ill and was resting on the grassy plain.
      When the other animals came to see her, they ate up all the grass in her pasture.
      As a result, when the deer recovered from her illness, she ended up dying since her pasture had come to an end.%
      \mbox{ }\hfill\mbox{(\cite[124]{Aesop:2002aa})}
    \end{quote}
    % 
    Still, Clause~\ref{assu:p:se:prog} requires \(\ed{}\) is in progress, and if the progressive is understood via causation as suggested, e.g. \textcite{Szabo:2004ul}, then both clauses may then be expressed via causation.
  }

  For example, it is not possible for an agent to be paddling in the ocean unless the agent steps foot into the ocean.
  Hence, the agent paddling in the ocean partly happened as the result of the agent stepping foot into the ocean.

  Likewise, it is not possible for an agent to butter some toast without some bread being toasted.
  Hence, the agent buttering toast partly happened as a result of the bread being toasted.
\end{note}


\begin{note}
  The following observations and proposition highlight important features of Clause~\ref{assu:p:se:hCon}.
\end{note}


\begin{note}
  First, we observe \(\ed{}\) partly happening as a result of \(\ed{\flat}\) does not follow from \(\ed{\flat}\) being such that an event \(\ed{}\) is in progress:

  \begin{observation}[Irrelevant progress]%
    \label{obs:se-need-hCon}%
    It may be the case that:
    \begin{itemize}
    \item
      \(\ed{\flat}\) is such that an event \(\ed{}\) is in progress and \(\ed{}\) happens.
    \item
      \(\ed{}\) does not partly happen as a result of \(\ed{\flat}\).
    \end{itemize}
    \vspace{-\baselineskip}
  \end{observation}

  \noindent%
  I.e., Clause~\ref{assu:p:se:hCon} of \autoref{def:se} does not follow from Clause~\ref{assu:p:se:prog}.

  \begin{motivation}{obs:se-need-hCon}
    Suppose an agent is passing an exam without external help.
    Then, a classmate passes the agent some answers, which the agent uses.
    The agent passes the exam, and was passing an exam without external help but the agent does not pass the exam as a result passing an exam without external help.
  \end{motivation}
\end{note}


\begin{note}
  Second, observe \(\ed{}\) partly happening as a result of \(\ed{\flat}\) concerns \(\edn{}\) under description \(\edo{}\) and \(\edn{\flat}\) under description \(\edo{\flat}\).
  These descriptions capture features of the relevant events, and so of interest is whether the features of \(\edn{}\) as captured by description \(\edo{}\) partly happened as a result of the feature of \(\edn{\flat}\) as captured by description \(\edo{\flat}\).
  This brief observation has the important consequence:

  \begin{proposition}[Limited descriptions]
    \label{prop:se-d-lim}
    \vspace{-\baselineskip}
    \begin{itenum}
    \item[\emph{If}:]
      \(\ed{}\) partly happens as a result of \(\ed{\flat}\).
    \item[\emph{Then}:]
      \(\edo{\flat}\) does not include features of \(\edn{\flat}\) that \(\ed{}\) does not partly happen as a result of.
    \end{itenum}
    \vspace{-\baselineskip}
  \end{proposition}

  \begin{argument}{prop:se-d-lim}
    As highlighted above, our interest is in whether \(\edn{}\) as described by \(\edo{}\) partly happens as a result of \(\edn{\flat}\) as described by \(\edo{\flat}\).

    Our interest is \emph{not} with whether \(\edn{}\) as described by \(\edo{}\) partly happens as a result of \(\edn{\flat}\) as described by \emph{some part of} \(\edo{\flat}\).

    In this respect, \(\ed{}\) must be a result of everything captured by \(\edo{\flat}\).
  \end{argument}

  \noindent%
  For example, suppose \(\edo{}\) is the description `an agent shuffles a deck so that \mainCard{} is \mainCardPos{} by sleight of hand', and \(\edo{}\) is true of \(\edn{}\).
  Further, suppose \(\edo{\flat}\) is the description `the agent marks the \mainCard{} so they do not lose the card in a shuffle' and \(\edo{\flat}\) is true of \(\edn{\flat}\).

  Now, intuitively \(\ed{}\) partly happens as a result of \(\ed{\flat}\).
  For, \(\ed{\flat}\) captures part of the sleight of hand.
  It is not possible for the agent to shuffle a deck so that \mainCard{} is \mainCardPos{} by sleight of hand without ensuring they do not lose \mainCard{} in the shuffle.%
  \footnote{
    Of course, the agent may randomly shuffle a deck so that \mainCard{} is \mainCardPos{}, but then the agent has not shuffled a deck \emph{so that} \mainCard{} is \mainCardPos{}.
  }

  However, \(\edo{\flat}\) may be expanded to include further details.
  For example, let \(\edo{\flat +}\) be the description `the agent marks the \mainCard{} so they do not lose the card in a shuffle while finding a hole in their sock'.
  Intuitively, it is not the case that \(\ed{}\) partly happens as a result of \(\ed{\flat +}\).
  For, the agent finding a hole in their sock had no influence on whether the agent marked \mainCard{}.

  Though, if \(\edn{}\) is described by \(\edo{+}\) as shuffles the deck by performing sleight of hand while coming to regret not buying a new pair of socks in the January sale, then \(\ed{+}\) plausibly happens as a result of \(\ed{\flat +}\).
\end{note}


\begin{note}
  Finally, our interest is not with whether \emph{any} event captured by description \(\edo{}\) is a result of any event captured by description \(\edo{\flat}\).
  In particular, features of \(\edn{}\) which are not captured by \(\edo{}\) and features of \(\edn{\flat}\) which are not captured by \(\edo{\flat}\) may be relevant to \(\ed{}\) partly happening as a result of \(\ed{\flat}\).
  This is highlighted by the following observation:

  \begin{observation}%
    \label{obs:theEventsHuh}%
    It may be the case that \(\ed{}\) partly happens as a result of \(\ed{\flat}\) only given details of \(\edn{}\) and \(\edn{\flat}\) which are not captured by \(\edo{}\) and \(\edo{\flat}\).
  \end{observation}

  \begin{motivation}{obs:theEventsHuh}
    Consider the example following \autoref{prop:se-d-lim}.

    \(\edo{}\) is the description `an agent shuffles a deck so that \mainCard{} is \mainCardPos{} by sleight of hand', and \(\edo{}\) is true of \(\edn{}\).
    And, \(\edo{\flat}\) is the description `the agent marks the \mainCard{} so they do not lose the card in a shuffle' and \(\edo{\flat}\) is true of \(\edn{\flat}\).

    \(\ed{}\) intuitively partly happens as a result of \(\ed{\flat}\).
    However, \(\edo{\flat}\) does not guarantee that the agent successfully completes the shuffle.
    For example, it is consistent with \(\edo{\flat}\) that the agent loses their mark.
    Still, the implicit understanding of \(\edn{}\) and \(\edn{\flat}\) is that the two shuffles are connected.
    And, so long as the two shuffles are connected \(\ed{}\) plausibly does partly happen as a result of \(\ed{\flat}\) for the motivated given above.
  \end{motivation}
\end{note}


\begin{note}
  \autoref{obs:theEventsHuh} may be seen to pair with \autoref{prop:se-d-lim}.
  For, our interest is in whether \(\edn{}\) as described by \(\edo{}\) partly happens as a result of \(\edn{\flat}\) as described by \(\edo{\flat}\).
  This means that everything in the description \(\edo{\flat}\) must be relevant to whether \(\edn{}\) as described by \(\edo{}\) happens.
  However, this does not means that everything relevant to whether \(\edn{}\) as described by \(\edo{}\) happens is captured by \(\edo{\flat}\).
\end{note}


\begin{note}
  In short, I take the idea \(\ed{}\) partly happening as a result of \(\ed{\flat}\) to be intuitive.

  Our interest is in the feature of \(\edn{}\) captured by description \(\edo{}\) partly being a result of the feature of \(\edn{\flat}\) captured by description \(\edo{\flat}\).
  Hence, \(\edo{\flat}\) is limited to what is relevant to \(\edn{}\) as captured by description \(\edo{}\) happening (\autoref{prop:se-d-lim}) though \(\edo{\flat}\) need not exhaust what is relevant to \(\edn{}\) as captured by description \(\edo{}\) happening (\autoref{obs:theEventsHuh}).
\end{note}


\begin{note}
  To close this section we provide an argument for \se{} with respect to \autoref{illu:gist:roots:F}.

  \begin{application}[A \se{} of \autoref{illu:gist:roots:F}]
    \label{obs:se-inst}%
    Given:
    % 
    \begin{itemize}
    \item
      \(\edn{}\) is the event described by \autoref{illu:gist:roots:F}.
    \item
      \(\Phi\) includes the \agents{} understanding of factorisation prior to \(\edn{}\).
    \item
      \(\edo{}\) is the description:
      `The agent concludes \pv{\propM{\rootsCon{}}}{\valI{True}} from \(\Phi\)'.
    \item
      \(\edn{\flat}\) covers Step~\ref{illu:gist:roots:F:factor} of the \agents{} reasoning in \autoref{illu:gist:roots:F}
    \item
      \(\edo{\flat}\) is the description:
      `The agent figures out \rootsConEqExV{6}{3}{2} with the aim to identify the factors of \rootsConEq{}'.
    \end{itemize}
    % 
    It is the case that:
    % 
    \begin{itemize}
    \item
      \(\ed{\flat}\) is a \se{} of \(\ed{}\).
    \end{itemize}
    \vspace{-\baselineskip}
  \end{application}

  \noindent%
  Note, Step~\ref{illu:gist:roots:F:factor} of \autoref{illu:gist:roots:F} amounts to the agent figuring out \rootsConEqExV{6}{3}{2}.
  However, \autoref{illu:gist:roots:F} did not explicitly characterise the agent attempting to identifying the factors of \rootsConEq{}.
  Still, we take it to be implicit the agent was attempting to identifying the factors of \rootsConEq{}.
  Alternatively, consider this a retcon.
  Either way, \(\edn{}\) is true of \(\edn{}\) and \(\edo{\flat}\) is true of \(\edn{\flat}\).
\end{note}


\begin{note}
  The detail for \autoref{obs:se-inst} are somewhat complex.
  I sort of recommend working the detail for \autoref{obs:se-inst} as it highlights the way events and event descriptions interact.
  Still, if you would like to skip ahead the very brief argument is:

  \begin{itemize}
  \item
    Clause~\ref{assu:p:se:prog} of \autoref{def:se} is satisfied as an agent figuring out \rootsConEqExV{6}{3}{2} with the aim to identify the factors of \rootsConEq{} is clearly working towards an \eiw{0} the agent concludes \pv{\propM{\rootsCon{}}}{\valI{True}} from \(\Phi\).
  \item
    Clause~\ref{assu:p:se:hCon} of \autoref{def:se} is satisfied as it makes no sense for an agent to conclude \pv{\propM{\rootsCon{}}}{\valI{True}} from a \pool{} which includes the \agents{} understanding of factorisation if the conclusion does not partly happen as a result of factorisation, which is exactly what Step~\ref{illu:gist:roots:F:factor} of \autoref{illu:gist:roots:F} captures.

    Further, it is plausibly the case that the \agents{} conclusion is party a result of the agent aiming to figure out the factors of \rootsConEq{}.

    So, the \agents{} conclusion is party a result of factoring with the aim to figure out the factors of \rootsConEq{}.
  \end{itemize}

  \begin{dets}{obs:se-inst}
    Our goal is to show clauses~\ref{assu:p:se:prog} and \ref{assu:p:se:hCon} of \autoref{def:se} hold.
    \medskip

    \noindent%
    In order for Clause~\ref{assu:p:se:prog} to hold it must be the case \(\ed{\flat}\) is such that \(\ed{}\) is in progress.
    In other words, form \(\edo{\flat}\) being true of \(\edn{\flat}\) it must be the case that \(\edn{}\) as described by \(\edo{}\) is in progress.

    Now, \(\edn{\flat}\) as described by \(\edo{\flat}\) is an \eiw{0} the agent figures out \rootsConEqExV{6}{3}{2} with the aim of figuring out the relevant factors.
    And, by assumption the agent has a good understanding of factorisation.

    So, as the agent figures out \rootsConEqExV{6}{3}{2}, the only reasoning which remains is captured by steps \ref{illu:gist:roots:F:zero} to \ref{illu:gist:roots:F:factor:done} of \autoref{illu:gist:roots:F}.
    These steps are more-or-less straightforward.
    Hence, I take it to be clear an \eiw{0} the agent concludes \pv{\propM{\rootsCon{}}}{\valI{True}} from \(\Phi\) \emph{may} be in progress.

    Further, as by \(\edo{\flat}\) the agent has the aim to identify the factors of \rootsConEq{} the previous observation may be strengthened to state an \eiw{0} the agent concludes \pv{\propM{\rootsCon{}}}{\valI{True}} from \(\Phi\) \emph{is} be in progress.

    In short, the reasoning which remains to conclude \pv{\propM{\rootsCon{}}}{\valI{True}} from \(\Phi\) is more-or-less straightforward and the agent plans to complete the relevant reasoning.
    So, \(\ed{\flat}\) is building to an \eiw{0} the agent concludes \pv{\propM{\rootsCon{}}}{\valI{True}} from \(\Phi\).

    This is what we wanted to show, Clause~\ref{assu:p:se:prog} of \autoref{def:se} holds.
    \medskip

    \noindent
    Note, in particular, our goal was only to show \(\edn{}\) under the description `the agent concludes \pv{\propM{\rootsCon{}}}{\valI{True}} from \(\Phi\)' is in progress.
    Our goal was not to show \(\edn{}\) under any other description is in progress.

    And, note the importance of the \agents{} aim to identify the factors of \rootsConEq{}.
    If \(\edn{\flat}\) is only described as an \eiw{0} the agent figures out \rootsConEqExV{6}{3}{2} then there is nothing to guarantee the agent puts the factorisation of \rootsConEq{} to any further use.
    \bigskip

    \noindent%
    Clause~\ref{assu:p:se:hCon} is a little more involved.
    We start by arguing for a slightly distinct claim:
    % 
    \begin{enumerate}[label=X., ref=(X)]
    \item
      \label{obs:se-inst:phew}
      For any event \(\edn{+}\) such that \(\edo{}\) is true of \(\edn{+}\), clauses \label{obs:se-inst:phew:d} and \label{obs:se-inst:phew:e} hold:
      \begin{enumerate}[label=\alph*., ref=\alph*]
      \item
        \label{obs:se-inst:phew:d}
        There is some description \(\edo{\sharp}\) such that:
        \begin{itemize}
        \item
          \(\edo{\sharp}\) is the description: `The agent figures out \rootsConEqExV{6}{3}{2}'
        \end{itemize}
      \item
        \label{obs:se-inst:phew:e}
        There is some event \(\edn{\times}\) such that clauses \ref{obs:se-inst:phew:e:t} and \ref{obs:se-inst:phew:e:h} hold:
        \begin{enumerate}[label=\roman*., ref=\roman*]
        \item
          \label{obs:se-inst:phew:e:t}
          \(\edo{\sharp}\) is true of \(\edn{\times}\)
        \item
          \label{obs:se-inst:phew:e:h}
          \(\ed[]{+}\) partly happens as a result of \(\ed[\sharp]{\times}\).
        \end{enumerate}
      \end{enumerate}
    \end{enumerate}
    % 
    In short, \ref{obs:se-inst:phew} states an \agents{} conclusion of \pv{\propM{\rootsCon{}}}{\valI{True}} from a \pool{} which includes the \agents{} understanding of factorisation prior to \(\edn{}\) partly happens as the result of the agent figuring out \rootsConEqExV{6}{3}{2}.
    \medskip

    \noindent%
    The argument for \ref{obs:se-inst:phew} is by contradiction.

    Suppose \(\edn{+}\) such that \(\edo{}\) is true of \(\edn{+}\) and either Clause~\ref{obs:se-inst:phew:d} or Clause~\ref{obs:se-inst:phew:e} fails to hold.

    Clause~\ref{obs:se-inst:phew:d} only concerns the existence of a description, and it is clear \(\edo{\sharp}\) is a description.
    Hence, given our assumption Clause~\ref{obs:se-inst:phew:e} must fail to hold.
    In particular, there is no event \(\edn{\times}\) such that Clause~\ref{obs:se-inst:phew:e:t} or Clause~\ref{obs:se-inst:phew:e:h} fails to hold.

    Now, consider the collection of events that \(\ed[]{+}\) partly happens as a result of.
    As Clause~\ref{obs:se-inst:phew:e:h} is true of every such event, Clause~\ref{obs:se-inst:phew:e:t} must fail to be true of every such event.

    In short, \(\ed{}\) does not partly happen as a result of an \eiw{0} the agent figures out \rootsConEqExV{6}{3}{2}.

    This is a contradiction.
    For, \(\edo{}\) describes \(\edn{+}\) as an \eiw{0} the agent concludes \pv{\propM{\rootsCon{}}}{\valI{True}} from \(\Phi\).
    And, \(\Phi\) includes the \agents{} understanding of factorisation.
    Yet, given the assumption made for a contradiction, \(\edo{}\) does not partly happen as a result of the agent factoring.
    And, if \(\edo{}\) does not partly happen as a result of the agent factoring then it is not the case that \(\edn{}\) is an \eiw{0} the agent concludes anything by their understanding of factorisation.
    For, the \agents{} understanding of factorisation is clearly irrelevant to whatever happened in \(\edn{+}\), and so \(\edo{}\) is not true of \(\edn{+}\).

    In short, it is built in to description being true of conclusion by factorisation that the agent factors.%
    \footnote{
      Indeed, it is built into \(\edn{+}\) being described as an \eiw{0} the agent concludes \pv{\propM{\rootsCon{}}}{\valI{True}} from \(\Phi\) that \(\edn{+}\) partly happens as a result of reasoning which amounts to steps~\ref{illu:gist:roots:F:eq}~to~\ref{illu:gist:roots:F:factor:done} of \autoref{illu:gist:roots:F}.
    }
    \medskip

    \noindent%
    With \ref{obs:se-inst:phew} in hand we return to \(\ed{\flat}\) and \(\ed{}\).
    Note, \(\edo{}\) is true of \(\edn{}\) and so clauses~\ref{obs:se-inst:phew:d}~and~\ref{obs:se-inst:phew:e} hold with respect to \(\edn{}\).

    Intuitively, \(\edn{\flat}\) is the relevant instance of \(\edn{\times}\).
    I do not argue for this point.%
    \footnote{
      However, note that if \(\edn{\flat}\) is \emph{not} the relevant instance of \(\edn{\times}\) then by \ref{obs:se-inst:phew} there must be some other \eiw{0} the agent figures out \rootsConEqExV{6}{3}{2} that \(\ed{}\) partly happens as a result of.
      And, \autoref{illu:gist:roots:F} does not involve the agent (re-)figuring out \rootsConEqExV{6}{3}{2} after \(\edn{\flat}\) so any such event must happen prior to \(\edn{\flat}\).
      So, the relevant event must happen prior to \(\edn{\flat}\).
    }
    And, if you are not convinced, note we may retcon our account of \autoref{illu:gist:roots:F} to state the agent performs the reasoning of Step~\ref{illu:gist:roots:F:factor} for the first time.%
    \footnote{
      Note in particular, nothing hangs on the relevant description givens to \(\edn{}\), \(\edn{\flat}\) or any other event here.
      For, \ref{obs:se-inst:phew} concerns the existence of a event for which a particular description is true.
    }
    \medskip

    \noindent%
    Now, as this point we have establish \(\ed{}\) partly happens as a result of \(\ed[\sharp]{\flat}\).
    However, \(\edo{\sharp}\) and \(\edo{\flat}\) are not equivalent.
    For, \(\edo{\flat}\) adds to \(\edo{\sharp}\) that the agent figures out \rootsConEqExV{6}{3}{2} with the aim to identify the factors of \rootsConEq{}.
    And, by \autoref{prop:se-d-lim} \(\ed{}\) does not partly happens as a result of \(\ed{\flat}\) if \(\edo{\flat}\) includes features of \(\edn{\flat}\) that \(\ed{}\) does not happen as a result of.

    So, that the agent has the aim to identify the factors of \rootsConEq{} is a feature that \(\ed{}\) happens as a result of.

    However, this final part of the argument may be made more-or-less immediate by expanding on the agent.

    Agent concludes, rather than conclusion happens to the agent, and false that the agent concludes without the aim of doing so.%
    \footnote{
      Note, this is consistent with \autoref{prop:se-d-lim}.
      Does not include features does not partly happen as a result of.
      However, this does not entail includes all features.
    }

    So, Clause~\ref{assu:p:se:hCon} of \autoref{def:se} is satisfied.
  \end{dets}

  \noindent%

  % The key idea of \autoref{obs:se-inst} is somewhat straightforward:
  % Observe it makes no sense for \(\edo{}\) to be true of \(\edn{}\) if it is not the case that \(\ed{}\) partly happens as a result of \(\edo{\flat}\) being true of \(\edn{\flat}\).

  % In particular, most of the detail of \autoref{obs:se-inst} amounted to arguing it makes no sense for an agent to conclude \pv{\propM{\rootsCon{}}}{\valI{True}} from \(\Phi\) without the conclusion partly happening as a result of the agent figuring out \rootsConEqExV{6}{3}{2}.
\end{note}



\subsection{\progEx{2}}
\label{sec:ProgEx}


\begin{note}
  We now argue \se{1} (\autoref{def:se}, \autopageref{def:se}) to provide explanations about why an event happened --- given the sense of `why' present in \qWhy{}.

  The argument is split into two propositions, titled `\progExI{}' and `\progExII{}', respectively.
  We refer to both propositions by the term `\progEx{}'.
\end{note}


\begin{note}
  Before starting, recall we understand the sense of `why' present in \qWhy{} in terms of why did the \eiw[\(\ed{}\)]{0} an agent concludes some \prop{0} \(\phi\) has \val{0} \(v\) from some \pool{0} \(\Phi\) happen, rather than any incompatible event --- e.g., an \eiw{0} the agent failed to conclude \(\pv{\phi}{v}\) from \(\Phi\).
\end{note}



\subsubsection{\progExI{}}


\begin{note}
  \begin{proposition}[\progExI{}]%
    \label{prop:PEbasic}%
    Given \(\ed{}\) is an \eiw{0} \vAgent{} does \(a\), and the sense of `why' present in \qWhy{}:

    \begin{itenum}
    \item[\emph{If}:]
      \(\ed{\flat}\) is a \se{} of \(\ed{}\).
    \item[\emph{Then:}]
      \(\edo{\flat}\) being true of \(\edn{\flat}\) explains `why' \(\ed{}\) happened.
    \end{itenum}
    \vspace{-\baselineskip}
  \end{proposition}

  % \noindent%
  % The argument for \autoref{prop:PEbasic} primarily follows from the way we understand the sense of `why' present in \qWhy{}, the definition of a \se{} (\autoref{def:se}) and our assumption that it is not possible for two incompatible events to be in progress (\autoref{assu:p:ex}).

  \begin{motivation}{prop:PEbasic}
    Assume \(\ed{\flat}\) as a \se{} of \(\ed{}\).
    \medskip

    \noindent%
    By Clause~\ref{assu:p:se:prog} of \autoref{def:se} (\autopageref{def:se}), \(\ed{\flat}\) is such that \(\ed{}\) is in progress.
    So, by \autoref{assu:p:ex}:
    For any description \(\ed{\ast}\) such that it is not possible for \(\edo{}\) and \(\edo{\ast}\) to both be true of an event, it is not the case that \(\ed[\ast]{}\) is in progress, given .
    So, \(\ed{\flat}\) being such that \(\ed{}\) is in progress favours \(\ed{}\) happening over any other incompatible event (\(\ed[\ast]{}\)) happening.
    \medskip

    \noindent%
    Further, by Clause~\ref{assu:p:se:hCon} of \autoref{def:se}, \(\ed{}\) partly happens as a result of \(\ed{\flat}\).
    So, \(\ed{}\) partly happened as a result of something with favoured \(\ed{}\) happening over any other incompatible event happening.
    \medskip

    \noindent%
    In this respect, \(\edo{\flat}\) being true of \(\ed{}\) explains `why' \(\ed{}\) happened, as opposed any other event.
    For, \(\edo{\flat}\) being true of \(\edn{\flat}\) gets that \(\ed{}\) was favoured over any other (incompatible) event and \(\ed{}\) partly happened as a result of \(\ed{\flat}\).\newline
  \end{motivation}

  \noindent%
  In other words, given \(\ed{\flat}\) is a \se{} of \(\ed{}\), \(\edo{\flat}\) being true of \(\ed{}\) explains `why' \(\ed{}\) happened in a basic sense, as by Clause~\ref{assu:p:se:hCon} \(\ed{}\) partly happened as a result of \(\ed{\flat}\).
  And, Clause~\ref{assu:p:se:prog} allows us to expand this observation to see \(\ed{}\) partly happened as a result of something which favoured \(\ed{}\) happening over any (incompatible) event.
\end{note}


\begin{note}
  For example, \autoref{obs:se-inst} established the description \(\edo{\flat}\) such that the event \(\edn{\flat}\) which covers Step~\ref{illu:gist:roots:F:factor} of the \agents{} reasoning is a \se{} of \(\ed{}\).
  Specifically, \(\edo{\flat}\) was `the agent is concluding \pv{\propM{\rootsCon{}}}{\valI{True}} from \(\Phi\) in \(\edn{\flat}\) and figures out \rootsConEqExV{6}{3}{2}'
  Hence, by \autoref{prop:PEbasic}, \(\edo{\flat}\) being true of \(\edn{\flat}\) explains `why' the agent concluded \pv{\propM{\rootsCon{}}}{\valI{True}} from \(\Phi\).

  Specifically, given the argument for \autoref{obs:se-inst}, \(\edo{\flat}\) being true of \(\edn{\flat}\) explains `why' the agent concluded \pv{\propM{\rootsCon{}}}{\valI{True}} from \(\Phi\) as the \agents{} conclusion happened as a result of the agent figuring out \rootsConEqExV{6}{3}{2} and as the agent was concluding \pv{\propM{\rootsCon{}}}{\valI{True}} from \(\Phi\) while figuring out \rootsConEqExV{6}{3}{2}.
\end{note}


\begin{note}%
  \nocite{Bromberger:1966aa}%
  Setting the particular way \progEx{} is motivated aside, I take the basic idea to be intuitive.
  For example, consider \citeauthor{Hempel:1965aa}'s Deductive-Nomological account of scientific explanation:
  \phantlabel{mention:Hempel:1}
  % 
  \begin{quote}
    [A Deductive-Nomological] explanation answers the question
    `\emph{Why} did the explanandum-phenomenon occur?'
    by showing that the phenomenon resulted from certain particular circumstances, specified in \(C_{1}, C_{2}, \dots C_{k}\), in accordance with the laws \(L_{1}, L_{2}, \dots L_{\gamma}\).
    By pointing this out, the argument shows that, given the particular circumstances and the laws in question, the occurrence of the phenomenon \emph{was to be expected}; and it is in this sense that the explanation enables us to \emph{understand why} the phenomenon occurred.%
    \mbox{ }\hfill\mbox{(\citeyear[337]{Hempel:1965aa})}
  \end{quote}
  % 
  Observe, \citeauthor{Hempel:1965aa} stress a connexion with observing an event \textquote{was to be expected} and an explanation allowing us to \textquote{understand why} the event occurred (the italics are \citeauthor{Hempel:1965aa}'s).
  For \citeauthor{Hempel:1965aa} the relevant expectation follows from circumstances and laws while for us the relevant expectation follows from identifying some event such that the event of interest is in progress (and partly happens as a result of).

  Indeed, circumstances and laws capture events in progress, so long as the laws are true.
  For, if some event is to happen given certain circumstances and laws, then the circumstances make it the case the event is in progress.
  Further, if the relevant laws are causal, the event happens partly as a result of the circumstances, and a \se{} may be identified.

  Still, our account of events in progress and \se{} does not rest on the existence of laws, and in this respect \progExI{} is distinct than \citeauthor{Hempel:1965aa}'s account of Deductive-Nomological explanation --- not to mention the emphasis we place of descriptions of the relevant events.%
  \footnote{
    Looking ahead, our full method for constructing counterexamples to \issueInclusion{} will appeal to a law-like idea (specifically the idea of an agent \tC{} in \autoref{cha:typical}).
    However, the law-like idea we appeal to is not used to provide additional motivation with respect to what explains.
    Rather, that law-like idea is used to extract additional features of particular descriptions.
  }
\end{note}



\subsubsection{\progExII{}}


\begin{note}
  \progExI{} observes the key way a description of a \se{} of some event explains `why' the event happens.
  However, \qWhy{} asks about \fingfr{1}, and we have not connected \fingfr{1} to \se{1}.
  There are (at least) two possible way to argue for a connexion:
  % 
  \begin{itemize}
  \item
    Argue \fingfr{1} are part of some descriptions which explains `why' some event happened.
  \item
    Argue that in some cases what is entailed by a description which explains `why' some event happened \emph{also} explains `why' the event happened, and that \fingfr{1} are entailed by certain descriptions which explain `why'.
  \end{itemize}
  % 
  We opt for the second option.
  For, the second option allows us to easily make general observations about when a \fingfr{} answers \qWhy{} based on descriptions which do not (directly) appeal to \fingfr{1}.

  \progExII{} expands on \progExI{} and argues what is entailed by a description which explains `why' some event happened \emph{also} explains `why' the event happened.
  And later, in \autoref{cha:ros}, we establish the way \fingfr{1} are entailed by certain descriptions.
  Still, we explicitly highlight the relevant connexion between \progExII{} and \fingfr{} after the statement of and argument for \progExII{}.
\end{note}


\begin{note}
  \begin{proposition}[\progExII{}]%
    \label{prop:PEentail}%
    Given \(\ed{}\) is an \eiw{0} \vAgent{} does \(a\), and the sense of `why' present in \qWhy{}:

    \begin{itenum}
    \item[\emph{If}:]
      There is some \se{} \(\ed{\flat}\) of \(\ed{}\) such that:
      \begin{itemize}
      \item
        \(\edo{\flat}\) being true of \(\edn{\flat}\) (non-vacuously) entails feature \(f\) is true of \(\edn{\flat}\).
      \end{itemize}
    \item[\emph{Then:}]
      Feature \(f\) being true of \(\edn{\flat}\) explains `why' \(\ed{}\) happened.
    \end{itenum}
    \vspace{-\baselineskip}
  \end{proposition}


  \begin{argument}{prop:PEentail}
    Assume \(\ed{\flat}\) is a \se{} of \(\ed{}\), and feature \(f\) such that \(f\) being true of \(\edn{\flat}\) is (non-vacuously) entailed by \(\edo{\flat}\) being true of \(\edn{\flat}\).

    Now, consider the description which combines of \(\edo{\flat}\) and \(f\): \(\edo{\flat + f}\).
    We claim \(\edo{\flat + f}\) is a \se{} of \(\ed{}\), and then argue \(f\) being true of \(\edn{\flat}\) alone explains `why' \(\ed{}\) happened, in the sense of `why' present in \qWhy{}.
    \medskip

    To see that \(\ed[\flat + f]{\flat}\) is a \se{} of \(\ed{}\), we check both clauses of \autoref{def:se} are satisfied:

    \begin{itemize}
    \item
      Clause~\ref{assu:p:se:prog} is satisfied.

      For, by assumption \(\ed{\flat}\) is such that \(\ed{}\) is in progress.
      So, as \(\edo{\flat + f}\) combines \(\edo{\flat}\) and \(f\), \(\ed[\flat + f]{\flat}\) must be such that \(\ed{}\) is in progress
    \item
      Clause~\ref{assu:p:se:hCon} is satisfied.

      For, \(f\) is (non-vacuously) entailed by \(\edo{\flat}\).
      So, \(\edo{\flat + f}\) captures no more of \(\edn{\flat}\) than \(\edo{\flat}\) implicitly captures.
      Hence, as \(\ed{}\) partly happens as a result of \(\ed{\flat}\), it must be the case that \(\ed{}\) partly happens as a result of \(\ed[\flat + f]{\flat}\).
    \end{itemize}
    % 
    So, \(\ed[\flat + f]{\flat}\) is a \se{} of \(\ed{}\).
    \medskip

    Now, as \(\ed[\flat + f]{\flat}\) is a \se{} of \(\ed{}\) it follows by \autoref{prop:PEbasic} that \(\edo{\flat + f}\) being true of \(\edn{\flat}\) explains `why' \(\ed{}\) happened, in the sense of `why' present in \qWhy{}.

    Finally, observe that for some thing to explain `why' \(\ed{}\) happened, given the sense of `why' present in \qWhy{}, the thing need not amount to a complete explanation of `why' \(\ed{}\) happened.
    Hence, and particular part of \(\edo{\flat + f}\) being true of \(\edn{\flat}\) explains `why' \(\ed{}\) happened, and specifically \(f\) being true of \(\edn{\flat}\) explains `why' \(\ed{}\) happened.
  \end{argument}

  Given \autoref{prop:PEentail}, if \(\edo{\flat}\) is a \se{} of \(\ed{}\), then any part of \(\edo{\flat}\) explains `why' \(\ed{}\) happened.
  Hence, given \autoref{obs:se-inst} the agent figuring out \rootsConEqExV{6}{3}{2} explains why the agent concluded \pv{\propM{\rootsCon{}}}{\valI{True}} from \(\Phi\) in \autoref{illu:gist:roots:F}.
\end{note}



\subsubsection{\progEx{2} and \qWhy{}}


\begin{note}
  An simple consequence of \autoref{prop:PEentail} clarifies the way we obtain answers to \qWhy{}:

  \begin{proposition}[\progEx{2}, conclusions, and \fingfr{1}]%
    \label{sketch:PE:cROS}%
    Given \(\ed{}\) is an \eiw{0} \vAgent{} concludes \(\phi\) has value \(v\) from \(\Phi\):

    \begin{itenum}
    \item[\emph{If}:]
      Conditions \ref{sketch:PE:cROS:a} and \ref{sketch:PE:cROS:b} both hold:
      % 
      \begin{enumerate}[label=\arabic*., ref=\arabic*]
      \item
        \label{sketch:PE:cROS:a}
        \(\ed{}\) is an \eiw{0} agent concludes \(\pv{\phi}{v}\) from \(\Phi\).
      \item
        \label{sketch:PE:cROS:b}
        There is some \se{} \(\ed{\flat}\) of \(\ed{}\) such that:
        \begin{itemize}
        \item
          \(\edo{\flat}\) entails it is true of \(\edn{\flat}\) that:
          \fofb{\(\pv{\psi}{v'}\)}{\(\Psi\)} through \(\ed{\flat}\).
        \end{itemize}
      \end{enumerate}
    \item[\emph{Then:}]
      \fingfb{\(\pv{\psi}{v'}\)}{\(\Psi\)} answers \qWhy{}.
    \end{itenum}
    \vspace{-\baselineskip}
  \end{proposition}

  \begin{argument}{sketch:PE:cROS}
    Assume conditions \ref{sketch:PE:cROS:a} and \ref{sketch:PE:cROS:b} hold.

    By Condition~\ref{sketch:PE:cROS:b}, \fingfb{\(\pv{\psi}{v'}\)}{\(\Psi\)} through \(\ed{\flat}\) explains `why' \(\ed{}\) happened, in the sense of `why' present in \qWhy{}.
    And, by Condition~\ref{sketch:PE:cROS:a}, \(\ed{}\) just is an \eiw{0} agent concludes \(\pv{\phi}{v}\) from \(\Phi\).
    Hence, \fingfb{\(\pv{\psi}{v'}\)}{\(\Psi\)} answers \qWhy{}.
  \end{argument}
\end{note}


\begin{note}
  Given \autoref{sketch:PE:cROS}, the answers to \qWhy{} of interest may be recast as answers to a slightly different, but perhaps a little more precise question `\qWhyV{}':

  \begin{question}{questionWhyV}{\qWhyV{}}%
    Given \(\ed{}\) is an \eiw{0} \vAgent{} concludes \(\pv{\phi}{v}\) from \(\Phi\):

    \begin{itemize}
    \item
      Which \fingfr{1} are such that:
      \begin{itemize}
      \item
        There is some \se{} \(\ed{\flat}\) of \(\ed{}\) such that:
        \begin{itemize}
        \item
          \(\edo{\flat}\) entails is true of \(\edn{\flat}\) that:
          \fofb{\(\pv{\psi}{v'}\)}{\(\Psi\)} through \(\ed{\flat}\).
        \end{itemize}
      \end{itemize}
    \end{itemize}
    \vspace{-1.5\baselineskip}
  \end{question}

  For, answer to \qWhyV{} satisfies conditions \ref{sketch:PE:cROS:a} and \ref{sketch:PE:cROS:b} of \autoref{sketch:PE:cROS}.
  Hence, by \autoref{sketch:PE:cROS} the answer to \qWhyV{} is also an answer to \qWhy{}.

  Still, to reduce the overhead of keeping track of variant questions and constraints we stick with \qWhy{}.
\end{note}


\section*{Summary}
\label{sec:summary}


\begin{note}
  \begin{itemize}
  \item
    An \eiw{0} an agent is concluding is an \eiw{0} an \eiw{0} the agent concludes is in progress.
  \end{itemize}
\end{note}


\begin{note}
  \autoref{sketch:PE:cROS} structures the argument to follow.
  Broadly stated, three tasks remain:

  \begin{enumerate}[label=\arabic*., ref=\arabic*]
  \item
    \label{eip:task:1}
    Demonstrate a description of some event entails a \fingfr{} holds.
  \item
    \label{eip:task:2}
    Show some \se{} of an \eiw{0} an agent concludes some \prop{0} \(\phi\) has \val{0} from some \pool{0} \(\Phi\) entails a \fingfr{} holds.
  \item
    \label{eip:task:3}
    Show some \se{} entails a \fofb{\(\pv{\psi}{v'}\)}{\(\Psi\)} even when an agent has not concluded \(\pv{\psi}{v'}\) from \(\Psi\) when the agent concludes \(\pv{\phi}{v}\) from \(\Phi\).
  \end{enumerate}

  Task~\ref{eip:task:1} establishes a \fingfr{0} may explain `why' an event happened, given \progExII{}/\autoref{sketch:PE:cROS}.

  Task~\ref{eip:task:2} establishes \fingfr{1} do explain `why' an event happened, given \progExII{}/\autoref{sketch:PE:cROS} and hence establishes certain answers \qWhy{}.

  And, Task~\ref{eip:task:3} establishes \issueInclusion{} fails to hold.
  For, by \autoref{sketch:PE:cROS} the relevant \fingfr{} between \(\pv{\psi}{v'}\) and \(\Psi\) answers \qWhy{}, and \issueInclusion{} holds only if the agent has also concluded \(\pv{\psi}{v'}\) from \(\Psi\) when the agent concludes \(\pv{\phi}{v}\) from \(\Phi\).
\end{note}


% \subsection{Ex}

% {
% \color{red}
% Here, this example is to be reworked to highlight interesting features of events in progress and \progEx{}.

% Basically, when scoring darts, event in progress through a variety of different things happen.
% Then, as play goes on, this constrains what the agent may do (as the need to score 501).
% E.g. if it's the last throw and the agent has a score of 481, then they need to hit a 20.

% In turn, one the game is over, understand why agent won, as, e.g. they were hitting a 20 on the last throw.
% }

%   \begin{note}
%     \begin{illustration}[Darts]
%       Agent wins at darts just in case there is some action available to the agent, such that if the agent were to perform the action they would be winning at darts.

%       Winning is a complex action.
%       An agent has three dart throws to lower their score from 501 to 0 before play switches to the other player, and play continues until neither player may lower their score further on their next turn (without going past 0).
%       Playing a game is a complex action, as the region of a dartboard an agent wishes to hit changes according to previous throws.
%       For example, if the agent's score is 51 with three throws remaining, the agent will not wish to hit bullseye, as there is no way to reduce their score by a single point using two darts.
%       If the agent goes on to hit 20, then the score of the remaining to darts should equal 31, and so on.
%     \end{illustration}

%     Note, the initial sequence of actions may be more or less arbitrary.
%     It is not possible to score 501 in three or six dart throws, so an agent \emph{could} start by throwing a few darts blindly, so long as they have sufficient skill to recover on subsequent throws.

%     Of course, throwing darts is quite different from concluding, but this note extends.
%     An agent may be concluding a theorem is true even though their first line of enquiry turns out to be a dead end, etc.

%     This is the appeal of the progressive.
%   \end{note}



%%%   Local Variables:
%%%   mode: latex
%%%   TeX-master: "master"
%%%   TeX-engine: luatex
%%%   End:
