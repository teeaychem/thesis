\chapter{Events, in progress}
\label{cha:events-progress}


\begin{note}
  Our interest is understanding the way an event in which an agent concludes happens.

  This chapter briefly outlines the way we understand both events and events in progress.
  This understanding is then applied to an events in which an agent concludes and event in which an agent concludes is in progress --- or colloquially an event in which an agent is concluding --- throughout the rest of the document.

  A key takeaway of this chapter is `\progEx{}' (\autoref{obs:PE}, \autopageref{obs:PE}) which expands on \autoref{idea:why} (\autoref{cha:intro}, \autopageref{idea:why}) and characterises the sense of `why' present in \qWhy{} in terms of features of an event in progress.

  Important assumptions made about events in progress are highlighted prior to the statement of \progEx{}.
  And, following the statement of \progEx{} we sketch the way in which \progEx{} applies to an event in which an agent concludes happens.
\end{note}


\section{Events}
\label{sec:events}

\begin{note}
  We understand events in a broadly (Neo-)\citeauthor{Davidson:1967aa}ian framework.
  In short:
  Events are things we refer to by way of descriptions which are true of the event.%
    \footnote{
      Keeping track of both events and descriptions gives me a headache, and perhaps it gives you a headache too.
      I tried to avoid the complication, but the headache was almost unbearable.

      Notation introduced below is designed so that a description is easy to ignore when it is of little importance.
      And, when a description is important attention will be drawn to its importance.
      Unfortunately, descriptions are fairly important through this chapter, though less so in later chapters.
    }

  For example, a natural language sentence such as:
  % 
  \begin{enumerate}[label=\arabic*., ref=(\arabic*), series=ESERIES]
  \item
    \label{ESERIES:toast}
    Sam buttered some toast in the kitchen.
  \end{enumerate}
  % 
  Is understood as saying there is some event \(\edn{}\) such that \(\edn{}\) is a butter event, the agent of \(\edn{}\) is Sam, the theme of \(\edn{}\) is some toast, and the location of \(\edn{}\) is the kitchen.%
  \footnote{
    Alternatively:
    \(\exists e [\textsc{butter}(e)\text{ \& }\textsc{agent}(e, \text{Sam})\text{ \& }\exists x(\textsc{theme}(e, \text{toast}(x)))\text{ \& }\textsc{in}(e, \text{the kitchen})]\)
  }

  Likewise:
  % 
  \begin{enumerate}[label=\arabic*., ref=(\arabic*), resume*=ESERIES]
  \item
    \label{ESERIES:gistCalcEq}
    Max concludes \gistCalcEq{} has value \valI{True}.
  \end{enumerate}
  % 
  Is understood as sating there is some event \(\edn{}\) such that \(\edn{}\) is a conclude event, the agent of \(\edn{}\) is Max, the \prop{0} of \(\edn{}\) is \gistCalcEq{} and the \val{} of \(\edn{}\) is \valI{True}.

  As events are referred to by descriptions true of an event, a description captures a specific event only if there is a unique event which satisfies the description.

  Neither \ref{ESERIES:toast} nor \ref{ESERIES:gistCalcEq} refer to a unique event without additional background information.
  For example, the description of \ref{ESERIES:toast} is compatible with Sam buttering three pieces of toast and with Sam buttering five pieces of toast.
  Hence, if Sam has buttered three pieces of toast and Sam has buttered five pieces of toast, the reference of \ref{ESERIES:toast} is under-determined.
  Likewise \ref{ESERIES:gistCalcEq} is compatible with Max using a calculator or their understanding of arithmetic to conclude \gistCalcEq{} has value \valI{True}.

  Still, when considering the way an event in which an agent concludes happens, we have a specific event in mind.
  Hence, we make the following assumption:

  \begin{assumption}[Happened uniquely]%
    \label{assu:HaUniqueD}%
    A description of an event which has happened is satisfied by a unique event.
  \end{assumption}

  \noindent%
  For example, if \ref{ESERIES:gistCalcEq} has happened, then \ref{ESERIES:gistCalcEq} is assumed to include information about the time when, or method by which, Max concluded \gistCalcEq{} has value \valI{True}.

  Note, \autoref{assu:HaUniqueD} only concerns events which have happened.
  For example, if \ref{ESERIES:gistCalcEq} has not happened, then \ref{ESERIES:gistCalcEq} is compatible with Max using a calculator or their understanding of arithmetic to conclude \gistCalcEq{} has value \valI{True}.
  We clarify the importance of this note with \autoref{obs:eip-partial}, below.

  Still, in the case of \qWhy{} and \qHow{}, both questions are asked with respect to an event which has happened.
  So, by \autoref{assu:HaUniqueD} both \qWhy{} and \qHow{} concern a unique event, no matter the description given of the event.
\end{note}


\begin{note}
  \begin{notationList}
  \item
    For ease of expression `\(\ed{}\)' abbreviates `\(\edn{}\) under description \(\edo{}\)'.
  \end{notationList}
\end{note}

\begin{note}
  We place one important constraint on descriptions:

  \begin{constraint}{Fog}{Fog}%
    \label{cons:no-f-ref}%
    A description of an event is limited to what is true of the event when the event happened.
  \end{constraint}

  \noindent%
  \autoref{cons:no-f-ref} rules out describing an event by using information about the way in which the event develops.

  For example, consider an event \(\edn{}\) under description \(\edo{}\) in which an agent has rolled nine using a two dice.
  Now, consider the event \(\edn{-}\) under description \(\edo{-}\) in which the dice are tumbling around in the agent's hands before being released on the table.
  By \autoref{cons:no-f-ref}, it is not the case that \(\edo{-}\) includes the information that \(\edn{-}\) develops into \(\edn{}\).
  And, if \(\edo{-}\) is limited a description of \(\edn{-}\) as an event in which dice are tumbling around in the agent's hands before being released on the table through some particular period of time, then it is not possible to observe via \(\edo{-}\) that \(\edn{-}\) is an event which develops into an event in which the agent rolls a nine.

  Of course, \(\edo{-}\) may be more detailed, and granting the present determines the future it may be the case that \(\edo{-}\) entails \(\edn{-}\) is an event which develops into an event in which the agent rolls a nine.
  However, this entailment rests on what is true of \(\edn{}\) being determined by what is true of \(\edn{-}\).

  Note, however, that even if one grants the present determines the future, it need not be the case that \(\edo{-}\) is sufficiently detailed to entails the agent rolls a nine.
  And, this is the case even if \(\edo{-}\) is uniquely satisfied by \(\edn{-}\).
  For, an description may be ensure unique satisfaction by pointing to a specific piece of space-time without describing in any further detail what occurs during the piece of space-time.
\end{note}


\begin{note}
  In general, while we assume a description of an event which has happened is satisfied by a unique event (\autoref{assu:HaUniqueD}) and that a description of an event is limited to what is true of the event when the event happened (\autoref{cons:no-f-ref}), we do not assume a description captures everything true of an event.

  In particular, when considering an event in which an agent concludes, we allow for the description to be common.
  Some technicalities (such as our use of \prop{1}, \val{1}, and \pool{1}) constrain the descriptions we use, but something like `the agent concluded \gistCalcEq{} has value \valI{True} for the first time by using a calculator' is about the level of detail I have in mind, though what follows is compatible with additional detail.
\end{note}



\section{Events in progress}
\label{sec:events-progress}


\begin{note}
  Our interest is understanding the way an event in which an agent concludes happens.
  And, the idea of an event is progress is a key idea with respect to our understanding of the way an event in which an agent concludes happens.

  This section briefly characterises the way in which we understand events in progress, and states two important assumptions we may about events in progress.
\end{note}



\subsection{Events in progress}


\begin{note}
  Events in progress are intuitively understood in terms of the progressive aspect.

  For example:
  % 
  \begin{enumerate}
  \item
    The agent is making soup.\newline
    \mbox{ } \hfill \(\leadsto\) An event in which the agent makes soup is in progress.
  \item
    The agent is reading Henley's `Invictus'.\newline
    \mbox{ } \hfill \(\leadsto\) An event in which the agent reads Henley's `Invictus' is in progress.
  \item
    The agent is riding the slope.\newline
    \mbox{ } \hfill \(\leadsto\) An event in which the agent rides the slope is in progress.
  \end{enumerate}
  % 
  Intuitively:

  \begin{intuition}[Events in progress and the progressive]
    \label{def:es-in-prog}
    \vspace{-\baselineskip}
    \begin{itemize}
    \item
      \(\ed{}\) is an event in which \(\ed{\ast}\) is in progress.
    \end{itemize}
    % 
    \emph{If and only if}:
    % 
    \begin{itemize}
    \item
      \(\text{Prog}[\ed{\ast}]\) is entailed by \(\edo{}\).\newline
      \mbox{ }\hfill (Where \(\text{Prog}[\ed{\ast}]\) is \(\edo{\ast}\) under the progressive aspect.)
    \end{itemize}
    \vspace{-\baselineskip}
  \end{intuition}

  \noindent%
  For example, going by \autoref{def:es-in-prog}, the following are equivalent:
  \begin{itemize}
  \item
    \(\ed{}\) is an event in which Max concludes \propM{\gistCalcEq{}} is \valI{True} is in progress.
  \item
    \(\edo{}\) is true of \(\edn{}\) and \(\edo{}\) entails \(\text{Prog}[\text{Max concludes \propM{\gistCalcEq{}} has value \valI{True}}]\).
  \item
    \(\edo{}\) is true of \(\edn{}\) and \(\edo{}\) entails Max is concluding \propM{\gistCalcEq{}} is \valI{True}.
  \end{itemize}

  \noindent%
  We assume an implicit understanding based on progressive aspect.%
  \footnote{
    \nocite{Portner:1998um}
    \nocite{Engelberg:1999vi}
    Note, though, that English does not have a quick, unambiguous, way of expressing events in progress.
    For, consider the sentence:
    \begin{enumerate}[label=\arabic*., ref=(\arabic*)]
    \item
      \label{prog:abmig}
      \textquote{John is studying for an exam}.
    \end{enumerate}
    \ref{prog:abmig} may be understand to express either the continuous or progressive aspect.

    Under the continuous aspect, \ref{prog:abmig} captures something about John, rather than something about an event happening.
    Hence, it need to be the case that John is engaged in an event of studying when \ref{prog:abmig} is said.
    For example, we may expand:
    \textquote{Sam is studying for an exam, but is taking a short nap.}

    By contrast, \ref{prog:abmig} under the progressive captures an event in where John studying is in progress.
    For example, we may expand:
    \textquote{Sam is studying for an exam, so they aren't taking a nap.}

    See,~\textcite{Richards:1981wo},~\textcite{Portner:2011vi}, etc for a general overview of the progressive.
    In particular, I suggest \textcite{Landman:1992wh} as an nice technical introduction.
    % \citeauthor{Landman:1992wh} considers a variety of important considerations in a straightforward way and
    % And, \citeauthor{Landman:1992wh}'s the way in which \citeauthor{Landman:1992wh} wrangles closeness to provide an account of the progressive is illustrative of the difficulties and does not require much technical background.
    \citeauthor{Szabo:2004ul} (\citeyear[34]{Szabo:2004ul}) provides a concise summary:
    \begin{quote}
      [A] progressive sentence is true at some time just in case some event occurs at that time, and if we follow the development of the event (within our world as long as it goes, then jumping into a nearby world, and iterating the process within the limits of reasonability) we will reach a possible world where the perfective correlate is true of the continuation.
    \end{quote}
    % \citeauthor{Portner:1998um} (\citeyear[764--766]{Portner:1998um}) provides a more in depth summary.
    For some of the issues with \citeauthor{Landman:1992wh}'s account, see
    (\cite{Bonomi:1997uq}),
    (\cite[49--50]{Engelberg:1999vi}),
    (\cite[35]{Szabo:2004ul}),
    (\cite[767]{Portner:1998um}),
    and (\cite[1256]{Portner:2011vi}).
  }
\end{note}


\begin{note}
  Fix our understanding of events in progress via the progressive aspect due to a common feature of analyses of the progressive holds of events in progress:%
  \footnote{
    See, e.g.:
    (\cite{Bennett:1972uw}),
    (\cite{Dowty:1979vq}),
    (\cite{Parsons:1990aa}),
    (\cite{Landman:1992wh}),
    (\cite{Portner:1998um}).

    \assuPP{} is often motivated by the `imperfective paradox' (\cite[cf.][Ch.3.1]{Dowty:1979vq}).

    \citeauthor{Bach:1986tb} summarises:
    \begin{quote}
      [H]ow can we characterize the meaning of a progressive sentences like \ref{Bach:impP:17} on the basis of the meaning of a simple sentence like \ref{Bach:impP:18} when \ref{Bach:impP:17} can be true of a history without \ref{Bach:impP:18} ever being true?
      \begin{enumerate}[label=(\arabic*), ref=(\arabic*)]
        \setcounter{enumi}{16}
      \item
        \label{Bach:impP:17}
        John was crossing the street.
      \item
        \label{Bach:impP:18}
        John crossed the street.%
        \mbox{ }\hfill\mbox{(\citeyear[12]{Bach:1986tb})}
      \end{enumerate}
    \end{quote}
    % 
    The `paradox' amounts to two seemingly incompatible observations:
    \begin{enumerate}[noitemsep]
    \item
      \ref{Bach:impP:17} entails an event in which John is crosses the street is in progress.
    \item
      There need not be an (actual) event in which John is crosses the street.%
    \end{enumerate}
    % 
    For example, John may have been hit by a bus.
    % In parallel, it may be true that you are falling asleep before a fire alarm is set off.
    The intuitive solution to this `paradox' is that the event in which the agent completes the action need not be an actual event.
  }

  \begin{assumption}[\assuPP{2}]%
    \label{assu:PP}%
    \vspace{-\baselineskip}
    \begin{itenum}
    \item[\emph{If}:]
      \(\ed{}\) is an event in which \(\ed{\ast}\) is in progress.
    \item[\emph{Then}:]
      There is some \progAdj{0} event \(e^{\sharp}\) such that~\ref{assu:PP:pe:dev} and~\ref{assu:PP:pe:verb} are both true:
      \begin{enumerate}[label=\roman*., ref=(\roman*)]
      \item
        \label{assu:PP:pe:dev}
        \(\edn{\sharp}\) is a development of \(\edn{}\).
      \item
        \label{assu:PP:pe:verb}
        \(\edo{\sharp}\) is true of \(e^{\sharp}\), where \(\edo{\sharp}\) is the perfective correlate of \(\edo{\ast}\).
      \end{enumerate}
    \end{itenum}
    \vspace{-\baselineskip}
  \end{assumption}
\end{note}

\begin{note}[Interest with the progressive]
  The immediate role of \assuPP{} is to help fix intuitions about events in progress by reflecting on the sense of possibility at issue.%
  \footnote{
    \assuPP{0} is denied by some.
    For example, \citeauthor{Szabo:2004ul} argues:
    \textquote{Sometimes we are \emph{doing} things even though there is no real chance that we could get them \emph{done}, and this is true even if we abstract away from the possibility of miraculous intervention.}
    (\citeyear[40]{Szabo:2004ul})
    E.g., \citeauthor{Szabo:2004ul} denies~\ref{Szabo:Arch} is necessarily false:
    \begin{quote}
      \begin{enumerate}[label=(\arabic*), ref=(\arabic*)]
        \setcounter{enumi}{12}
      \item
        \label{Szabo:Arch}
        As the architect was building the cathedral he knew that, although he would be building it for another year or so, he couldn't possibly complete it.%
        \mbox{ }\hfill\mbox{(\citeyear[38]{Szabo:2004ul})}
      \end{enumerate}
    \end{quote}
    Though,~\ref{Szabo:Arch} seems always false to me.
    The only sense with which I read~\ref{Szabo:Arch} as true under the progressive requires factivity of knowledge to fail, thus allowing the cathedral to be built.

    See (\cite[1245]{Portner:2011vi}) for additional, distinct, discussion of (\cite{Szabo:2004ul}).
  }
  The deferred role of \assuPP{} is to help intuitively motivate a couple of ideas introduced in later chapters, and ease a handful of arguments.%
  \footnote{
    Specifically, \fc{1} in \autoref{cha:fcs} and \ros{1} in \autoref{cha:ros}.
  }

  We make three observations to help firm intuition.

  \begin{observation}[\assuPP{2} and existential modality]%
    \label{obs:prog-not-reg-poss}%
    The sense of possibility in \assuPP{} does not reduce to existential \{logical, metaphysical, nomic, \dots\} possibility.
  \end{observation}
  \begin{motivation}{obs:prog-not-reg-poss}
    Suppose an agent is sitting a multiple choice exam.
    To pass the exam the agent only needs to chose some number of correct choices.
    It is certainly logically, metaphysically, and nomically possible that the agent chooses a sufficient number of correct choices.
    However, it does not follow that the agent is passing the exam.%
    \footnote{
      See also Igal Kvart's example of Mary wiping out the Roman army (\cite[18]{Landman:1992wh}).
    }
  \end{motivation}

  \begin{observation}[\assuPP{2} and counterfactuals]%
    \label{obs:prog-not-cfs}%
    There is no simple relation between the sense of possibility in \assuPP{} and counterfactuals.
  \end{observation}
  \begin{motivation}{obs:prog-not-cfs}
    Suppose an agent is passing an exam without external help.
    Then, a classmate passes the agent some answers, which the agent uses.
    The agent is no longer passing the exam without external help.
    And, any close possible world where the classmate does not pass answers, some other classmate may pass the same answers.
  \end{motivation}

  \begin{observation}[\assuPP{} and uniqueness]%
    \label{obs:prog-no-unique}%
    The progressive may be true of an event without the event being sufficiently developed to `indicate' a unique outcome.
  \end{observation}
  \begin{motivation}{obs:prog-no-unique}
    Suppose an agent has drawn a straight line on a piece of paper.
    It may be true that the agent is drawing a triangle.
    However, the straight line is compatible with the agent drawing an \(n\)-sided polygon, for any \(n\) within a some reasonable bound.%
    \footnote{
      This observation is inspired by \citeauthor{Dowty:1979vq}'s example involving a circle and a triangle (\citeyear[133]{Dowty:1979vq}).
    }
  \end{motivation}

  \noindent%
  Loosely paraphrased, if an event is in progress, then intuitively there is \emph{something} about the way things are which ensures the existence of a possible completion event (\autoref{obs:prog-no-unique}) which is robust against external influence (\autoref{obs:prog-not-cfs}) and does not require luck (\autoref{obs:prog-not-reg-poss}).
\end{note}


\begin{note}
  A brief observation, which contrasts with \autoref{assu:HaUniqueD}:

  \begin{observation}[Events in progress and common satisfaction]%
    \label{obs:eip-partial}%
    Sometimes, when \(\ed{}\) is an event in which \(\ed{\ast}\) is in progress, \(d'\) may be satisfied by multiple events.
  \end{observation}

  \begin{motivation}{obs:eip-partial}
    Consider an agent flipping a coin until it lands heads or lands tails ten times in a row.
    Given sufficient determination from the agent, it is true that an event in which the agent flips a coin until it lands heads or lands tails ten times in a row is in progress.

    Now, a unique event in which the agent flips a coin until it lands heads or lands tails ten times in a row involves an \(i\)th throw the coin lands heads on, or ten throws in which the coin lands tails.
    However, by \autoref{cons:no-f-ref}, it must be true of \(\ed{}\) that \(\ed{\ast}\) is in progress.
    And, as coin flips are quite random (\cite{Gelman:2002ww}), \(\ed{}\) may fail to entail that the coin lands heads on the \(i\)th throw the coin lands heads, or that the coin lands tails on all ten throws.

    Hence, if \(\ed{}\) fails to entail either that the coin lands heads on the \(i\)th throw the coin lands heads, or that the coin lands tails on all ten throws, then \(\ed{\ast}\) may be satisfied by either event.
  \end{motivation}
\end{note}



\subsection{An assumption}
\label{sec:assumptions-1}


\begin{note}
  Events in progress, rough intuition.
  For, following assumption.

  \begin{assumption}[Exclusivity]
    \label{assu:p:ex}
    \vspace{-\baselineskip}
    \begin{itenum}
    \item[\emph{If}:]
      It is not possible for \(d^{\#}\) and \(d^{\$}\) to both be true of an event.
    \item[\emph{Then}:]
      It is not possible for \(\ed{}\) to be an event in which \(\ed{\#}\) to be in progress and \(\ed{\$}\) is in progress.
    \end{itenum}
    \vspace{-\baselineskip}
  \end{assumption}

  \noindent%
  Paraphrased, \autoref{assu:p:ex} assumes that if an event is in progress, then no incompatible event is (also) in progress.

  For example:
  It is not possible for both team A wins the game of hockey and for team B to wins the game of hockey to be true of an event.
  So, by \autoref{assu:p:ex}, it is not possible for team A to be winning the game of hockey and for team B to be winning the game of hockey.
  Hence, if team A is winning the game of hockey, team B is not winning the game of hockey.
\end{note}


\begin{note}
  \autoref{assu:p:ex} is about events in progress.
  And, the progressive aspect helps fix what events in progress are (\autoref{def:es-in-prog}).
  However, \autoref{assu:p:ex} does not clearly hold of the progressive aspect.
  For example, \citeauthor{Landman:1992wh} writes:%
  \footnote{
    See \textcite{Bonomi:1997uq} for additional discussion.
  }

  \begin{quote}
    Suppose I was on a plane to Boston which got hijacked and landed in Bismarck, North Dakota.
    What was going on before the plane was hijacked?
    One thing I can say is:
    `I was flying to Boston when the plane was hijacked'
    This is reasonable.
    But another thing I could say is:
    `I was flying to Boston.
    Well, in fact, I wasn't, I was flying to Bismarck, but I didn't know that at the time'
    And this is also reasonable.%
    \mbox{ }\hfill\mbox{(\citeyear[30--31]{Landman:1992wh})}
  \end{quote}
  % 
  \citeauthor{Landman:1992wh}'s observation is compatible with \autoref{assu:p:ex} to the extent that \citeauthor{Landman:1992wh} does not suggest it is reasonable to say `I was flying to Boston \emph{and} I was flying to Bismarck'.
  However, \citeauthor{Landman:1992wh}'s observation is compatible with \autoref{assu:p:ex} to the extent that \citeauthor{Landman:1992wh} suggests it is reasonable to say `I was flying to Boston' \emph{and} `I was flying to Bismarck'.

  More could be said.
  For example, \citeauthor{Landman:1992wh}'s first statement is implicitly (and the explicitly clarified as) given with respect to what they knew, and \citeauthor{Landman:1992wh}'s second statement takes into account details about what happened after the event.
  And, it is reasonable to say various things, regardless of whether those things are true.
  However, I see no real benefit in raising \autoref{def:es-in-prog} to a definition by digressing into an account of the progressive.%
  \footnote{
    Though, an account of the progressive which ties the progressive to causation, such as \citeauthor{Szabo:2004ul}'s (\citeyear{Szabo:2004ul}) account, arguably entails \autoref{assu:p:ex}.
  }
  Hence, \autoref{assu:p:ex} captures an important way in which our understanding of an event in progress may differ from the progressive aspect.
\end{note}





\section{\se{3} and \progEx{}}
\label{sec:se3-progex}


\begin{note}
  Now, events in progress and answers to the sense of `why' present in \qWhy{}.

  Definition of a \se{}.
\end{note}


\subsection{\se{3}}

\begin{note}
  \begin{definition}[\se{3}]
    \label{def:se}
    \vspace{-\baselineskip}
    \begin{itemize}
    \item
      \(\ed{\flat}\) is a \emph{\se{0}} of \(\ed{}\).
    \end{itemize}
    \emph{If and only if}:
    \begin{itemize}
    \item
      Conditions~\ref{assu:p:se:prog}, \ref{assu:p:se:desc}, and \ref{assu:p:se:hCon} hold:
      \begin{enumerate}[label=\Alph*., ref=\Alph*]
      \item
        \label{assu:p:se:prog}
        \(\ed{\flat}\) is such that \(\edn{}\) under description \(d'\) is in progress.
      \item
        \label{assu:p:se:desc}
        \(\edo{}\) is true of \(\edn{}\) \emph{only if} \(d'\) is true of \(\edn{}\).
      \item
        \label{assu:p:se:hCon}
        The following conditional is true:
        \begin{itenum}
        \item[\emph{If}:]
          \(\ed{}\) happens.
        \item[\emph{Then}:]
          \(\ed{}\) happens as a result of \(\ed{\flat}\).
        \end{itenum}
      \end{enumerate}
    \end{itemize}
    \vspace{-\baselineskip}
  \end{definition}

  \noindent%
  Paraphrased, a \se{} is an event such that an event is in progress and, in turn, if the event in progress happens, it happens as a result of the \se{}.

  Still, given \autoref{assu:HaUniqueD} and \autoref{obs:eip-partial}, the paraphrase of a \se{} may be misleading.
  For, by \autoref{assu:HaUniqueD}, if \(\ed{}\) happens then \(\edo{}\) is satisfied by a unique event.
  And, by \autoref{obs:eip-partial} if \(\ed{}\) is in progress, then \(\edo{}\) may be satisfied by multiple events.
  Hence, \autoref{def:se} distinguishes the event in progress and links the event in progress to the event which happens by Condition~\autoref{assu:p:se:desc}.

  Paraphrased a little more carefully, a \se{} is an event such that an event is in progress and, in turn, if an event happens such that the event in progress happens, the event happens as a result of the \se{}.
\end{note}


\begin{note}
  A quick variation of \autoref{obs:prog-not-cfs} highlights the importance of Condition~\autoref{assu:p:se:hCon}:

  \begin{observation}[Condition \ref{assu:p:se:hCon}!]%
    \label{obs:se-need-hCon}%
    It is not the case that:
    \begin{itenum}
    \item[\emph{If}:]
      \(\ed{\flat}\) is an event such that conditions~\ref{assu:p:se:prog} and \ref{assu:p:se:desc} hold of \(\ed{\flat}\) and \(\ed{}\) happens.
    \item[\emph{Then}:]
      \(\ed{}\) happens as a result of \(\ed{\flat}\).
    \end{itenum}
    \vspace{-\baselineskip}
  \end{observation}

  \begin{motivation}{obs:se-need-hCon}
    Suppose an agent is passing an exam without external help.
    Then, a classmate passes the agent some answers, which the agent uses.
    The agent passes the exam, and was passing an exam without external help but the agent does not pass the exam as a result passing an exam without external help.
  \end{motivation}
\end{note}


\begin{note}
  Alternative, substitute `causes' in place of `happens as a result of'.%
  \footnote{
    For link between progressive and causation, \textcite{Szabo:2004ul}.
  }
\end{note}

\subsection{\progEx{2}}
\label{sec:what-these-do}

\begin{note}
  With \autoref{assu:p:ex} and \autoref{def:se} in hand, an observation about events in progress an explanations about why an event happened --- given the sense of `why' present in \qWhy{} --- follows.

  \begin{observation}[\progEx{2}]%
    \label{obs:PE}%
    Given \(\ed{}\) is an event in which \vAgent{} does \(a\):

    \begin{itenum}
    \item[\emph{If}:]
      There is some \se{} \(\ed{\flat}\) of \(\ed{}\) such that:
      \begin{itemize}
      \item
        \(\ed{\flat}\) is such that \(\ed{}\) is in progress \emph{only if} feature \(f\) of \(\edo{\flat}\) holds throughout \(\ed{\flat}\).
      \end{itemize}
    \item[\emph{Then:}]
      Feature \(f\) explains `why' \(\ed{}\) happened, in the sense of `why' present in \qWhy{}.
    \end{itenum}
    \vspace{-\baselineskip}
  \end{observation}

  \begin{motivation}{obs:PE}
    The motivation for \autoref{obs:PE} follows from the way in which we understand the sense of `why' present in \qWhy{} and the way we understand \se{} such that an agent is concluding given \autoref{def:se} and \autoref{assu:p:ex}.

    For, we understand the sense of `why' present in \qWhy{} as the did the event \(\ed{}\) in which an agent concludes \(\pv{\phi}{v}\) from \(\Phi\) happen, rather than any incompatible event --- e.g., an event in which the agent cleans the lint from their pocket, etc.

    Hence, if it is possible to point to something that favours \(\ed{}\) happening over any other event happening, then that thing explains `why' \(\ed{}\) happened, as opposed any other event.
    \medskip

    \noindent%
    Now, consider \(\ed{\flat}\) as a \se{} of \(\ed{}\).

    As \(\ed{}\) happened and \(\ed{\flat}\) is a \se{} of \(\ed{}\), then by \autoref{def:se}:
      \(\ed{\flat}\) is such that \(\ed{}\) is in progress \emph{and}
      \(\ed{}\) happened as a result of \(\ed{\flat}\).

      Further, consider any event \(\ed{\ast}\) such that it is not possible for \(\edo{}\) and \(\edo{\ast}\) to both be true of an event.
    Then, by \autoref{assu:p:ex}, it is not the case that \(\ed{\ast}\) is in progress.
    Therefore, \(\ed{\flat}\) being such that \(\ed{}\) is in progress favours \(\ed{}\) happening over any other incompatible event happening.

    And, as \(\ed{}\) happened as a result of \(\ed{\flat}\), \(\ed{}\) happened as the result of something which favoured \(\ed{}\) happening over any other incompatible event happening.

    So, whatever features of \(\ed{\flat}\) make it the case that \(\ed{\flat}\) is a \se{} of \(\ed{}\) explain `why' \(\ed{}\) happened, in the sense of `why' present in \qWhy{}.
    For, \(\ed{\flat}\) favoured it being the case that \(\ed{}\) happened, and \(\ed{}\) happened as a result of \(\ed{\flat}\).
    \medskip

    \noindent%
    Finally, then, consider some feature \(f\) such that \(\ed{}\) is in progress \emph{only if} feature \(f\) of \(\edo{\flat}\) holds throughout \(\edn{\flat}\).
    By \autoref{def:se}, if feature \(f\) fails to hold, then \(\ed{\flat}\) is not a \se{} of \(\ed{}\).
    Hence, \(f\) is part of what makes it the case that \(\ed{\flat}\) is a \se{} of \(\ed{}\).\newline
  \end{motivation}

  \noindent%
  Setting the particular way \progEx{} is motivated aside, I take \progEx{} to be intuitive.
  For example, consider \citeauthor{Hempel:1965aa}'s Deductive-Nomological account of scientific explanation:
  \phantlabel{mention:Hempel:1}
  % 
  \begin{quote}
    [A Deductive-Nomological] explanation answers the question
    `\emph{Why} did the explanandum-phenomenon occur?'
    by showing that the phenomenon resulted from certain particular circumstances, specified in \(C_{1}, C_{2}, \dots C_{k}\), in accordance with the laws \(L_{1}, L_{2}, \dots L_{\gamma}\).
    By pointing this out, the argument shows that, given the particular circumstances and the laws in question, the occurrence of the phenomenon \emph{was to be expected}; and it is in this sense that the explanation enables us to \emph{understand why} the phenomenon occurred.%
    \mbox{ }\hfill\mbox{(\citeyear[337]{Hempel:1965aa})}
  \end{quote}
  % 
  Observe, \citeauthor{Hempel:1965aa} stress a connexion with observing an event \textquote{was to be expected} and an explanation allowing us to \textquote{understand why} the event occurred (the italics are \citeauthor{Hempel:1965aa}'s).
  Though, while \citeauthor{Hempel:1965aa} details the link with circumstances and laws, while we detail the link in terms of a \se{}.
\end{note}


\begin{note}
  \progEx{2} is a key part of the arguments to follow.
  For, \progEx{0} provides, in some detail, sufficient conditions for a \ros{} to be an answer to \qWhy{}:

  \begin{sketch}[\progEx{2}, conclusions, and \ros{1}]%
    \label{sketch:PE:cROS}%
    Given \(\ed{}\) is an event in which \vAgent{} concludes \(\phi\) has value \(v\) from \(\Phi\):

    \begin{itenum}
    \item[\emph{If}:]
      There is some \se{} \(\ed{\flat}\) of \(\ed{}\) such that:
      \begin{itemize}
      \item
        \(\ed{\flat}\) is such that \(\ed{}\) is in progress \emph{only if} a \ros{} between \(\psi\), \(v'\) and \(\Psi\) holds throughout \(\ed{\flat}\).
      \end{itemize}
    \item[\emph{Then:}]
      The \ros{} between \(\psi\), \(v'\) and \(\Psi\) explains `why' \(\ed{}\) happened, in the sense of `why' present in \qWhy{}.
    \end{itenum}
    \vspace{-\baselineskip}
  \end{sketch}

  \begin{motivation}{sketch:PE:cROS}
    \autoref{sketch:PE:cROS} follows from \autoref{obs:PE}, by taking the  action \(a\) to be a conclusion that \(\phi\) has value \(v\) from \(\Phi\), and feature \(f\) to be a \ros{} between \(\psi\), \(v'\) and \(\Psi\).
  \end{motivation}

  \noindent%
  In \autoref{cha:var} we re-express \autoref{sketch:PE:cROS} as a sufficient condition for answers to \qWhy{}, after clarifying the way in which we understand conclusions and \ros{}.

  Still, \autoref{sketch:PE:cROS} hints at the overall argument to follow.
  For, given an event in which an agent concludes \(\phi\) has value \(v\) from \(\Phi\) then if the antecedent is satisfied by a \ros{} between \(\psi\), \(v'\) and \(\Psi\) and there is no past or present event in which the agent concludes \(\psi\) has value \(v'\) from \(\Psi\), then \issueInclusion{} fails to hold.
\end{note}



\section*{Summary}
\label{sec:summary}


\begin{note}
  \begin{itemize}
  \item
    An event in which an agent is concluding is an event in which an event in which the agent concludes is in progress.
  \end{itemize}
\end{note}



% \subsection{Ex}

% {
% \color{red}
% Here, this example is to be reworked to highlight interesting features of events in progress and \progEx{}.

% Basically, when scoring darts, event in progress through a variety of different things happen.
% Then, as play goes on, this constrains what the agent may do (as the need to score 501).
% E.g. if it's the last throw and the agent has a score of 481, then they need to hit a 20.

% In turn, one the game is over, understand why agent won, as, e.g. they were hitting a 20 on the last throw.
% }

%   \begin{note}
%     \begin{illustration}[Darts]
%       Agent wins at darts just in case there is some action available to the agent, such that if the agent were to perform the action they would be winning at darts.

%       Winning is a complex action.
%       An agent has three dart throws to lower their score from 501 to 0 before play switches to the other player, and play continues until neither player may lower their score further on their next turn (without going past 0).
%       Playing a game is a complex action, as the region of a dartboard an agent wishes to hit changes according to previous throws.
%       For example, if the agent's score is 51 with three throws remaining, the agent will not wish to hit bullseye, as there is no way to reduce their score by a single point using two darts.
%       If the agent goes on to hit 20, then the score of the remaining to darts should equal 31, and so on.
%     \end{illustration}

%     Note, the initial sequence of actions may be more or less arbitrary.
%     It is not possible to score 501 in three or six dart throws, so an agent \emph{could} start by throwing a few darts blindly, so long as they have sufficient skill to recover on subsequent throws.

%     Of course, throwing darts is quite different from concluding, but this note extends.
%     An agent may be concluding a theorem is true even though their first line of enquiry turns out to be a dead end, etc.

%     This is the appeal of the progressive.
%   \end{note}



%%% Local Variables:
%%% mode: latex
%%% TeX-master: "master"
%%% TeX-engine: luatex
%%% End:
