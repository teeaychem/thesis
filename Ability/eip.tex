\chapter{Events, in progress}
\label{cha:events-progress}


\begin{note}
  Our interest is understanding the way an event in which an agent concludes happens.

  This chapter briefly outlines the way we understand both events and events in progress.
  This understanding is then applied to an events in which an agent concludes and event in which an agent concludes is in progress --- or colloquially an event in which an agent is concluding --- throughout the rest of the document.

  A key takeaway of this chapter is `\progEx{}' (\autoref{obs:PE}, \autopageref{obs:PE}) which expands on \autoref{idea:why} (\autoref{cha:intro}, \autopageref{idea:why}) and characterises the sense of `why' present in \qWhy{} in terms of features of an event in progress.

  Important assumptions made about events in progress are highlighted prior to the statement of \progEx{}.
  And, following the statement of \progEx{} we sketch the way in which \progEx{} applies to an event in which an agent concludes happens.
\end{note}


\section{Events}
\label{sec:events}

\begin{note}
  We understand events in a broadly (Neo-)\citeauthor{Davidson:1967aa}ian framework.
  In short:
  Events are things we refer to by way of descriptions which are true of the event.%
    \footnote{
      Keeping track of both events and descriptions gives me a headache, and perhaps it gives you a headache too.
      I tried to avoid the complication, but the headache was almost unbearable.

      Notation introduced below is designed so that a description is easy to ignore when it is of little importance.
      And, when a description is important attention will be drawn to its importance.
      Unfortunately, descriptions are fairly important through this chapter, though less so in later chapters.
    }

  For example, a natural language sentence such as:
  % 
  \begin{enumerate}[label=\arabic*., ref=(\arabic*), series=ESERIES]
  \item
    \label{ESERIES:toast}
    Sam buttered some toast in the kitchen.
  \end{enumerate}
  % 
  Is understood as saying there is some event \(\edn{}\) such that \(\edn{}\) is a butter event, the agent of \(\edn{}\) is Sam, the theme of \(\edn{}\) is some toast, and the location of \(\edn{}\) is the kitchen.%
  \footnote{
    Alternatively:
    \(\exists e [\textsc{butter}(e)\text{ \& }\textsc{agent}(e, \text{Sam})\text{ \& }\exists x(\textsc{theme}(e, \text{toast}(x)))\text{ \& }\textsc{in}(e, \text{the kitchen})]\)
  }

  Likewise:
  % 
  \begin{enumerate}[label=\arabic*., ref=(\arabic*), resume*=ESERIES]
  \item
    \label{ESERIES:gistCalcEq}
    Max concludes \gistCalcEq{} has value \valI{True}.
  \end{enumerate}
  % 
  Is understood as sating there is some event \(\edn{}\) such that \(\edn{}\) is a conclude event, the agent of \(\edn{}\) is Max, the \prop{0} of \(\edn{}\) is \gistCalcEq{} and the \val{} of \(\edn{}\) is \valI{True}.

  As events are referred to by descriptions true of an event, a description captures a specific event only if there is a unique event which satisfies the description.

  Neither \ref{ESERIES:toast} nor \ref{ESERIES:gistCalcEq} refer to a unique event without additional background information.
  For example, the description of \ref{ESERIES:toast} is compatible with Sam buttering three pieces of toast and with Sam buttering five pieces of toast.
  Hence, if Sam has buttered three pieces of toast and Sam has buttered five pieces of toast, the reference of \ref{ESERIES:toast} is under-determined.
  Likewise \ref{ESERIES:gistCalcEq} is compatible with Max using a calculator or their understanding of arithmetic to conclude \gistCalcEq{} has value \valI{True}.

  Still, when considering the way an event in which an agent concludes happens, we have a specific event in mind.
  Hence, we make the following assumption:

  \begin{assumption}[Happened uniquely]%
    \label{assu:HaUniqueD}%
    \begin{itemize}
    \item
      \(\edn{}\) picks out the event \(\edn{}\) under a `maximal' description.
    \item
      \(\ed{}\) captures \(\edn{}\) under the particular description \(\edo{}\).
    \end{itemize}
  \end{assumption}

  \noindent%
  For example,
  {
    \color{red}
    \ref{ESERIES:gistCalcEq} captures an event under a particular description.
    However, \ref{ESERIES:gistCalcEq} captures a particular event, and so it is also true that the event captured by \ref{ESERIES:gistCalcEq} happens at a particular time, and involves a particular method by which Max concludes \gistCalcEq{} has value \valI{True}, etc.
  }

  In order to talk about an event which satisfies the description, explicitly mark that of interest is any event which satisfies the description.
  For example, the description of \ref{ESERIES:gistCalcEq} is compatible with Max using a calculator or their understanding of arithmetic to conclude \gistCalcEq{} has value \valI{True}, and so on.
  We clarify the importance of this note with \autoref{obs:eip-partial}, below.

  Still, in the case of \qWhy{} and \qHow{}, both questions are asked with respect to a specific event.
  However, what is of interest is only a particular feature of the event --- the conclusion.
  {
    \color{blue}
    Flexible.
    Description of the conclusion may be more or less detailed.
    For the most part, our interest is with a `minimal' description.
    `The agent concluded' etc.
    Adding more detail adds to possible answers to \qWhy{}.
  }
\end{note}


\begin{note}
  We place one important constraint on descriptions:

  \begin{constraint}{Fog}{Fog}%
    \label{cons:no-f-ref}%
    A description \(\edo{}\) of an event \(\edn{}\) is limited to what is true of \(\edn{}\) when \(\edn{}\) happens.
  \end{constraint}

  \noindent%
  \autoref{cons:no-f-ref} rules out describing an event by using information what happens after the relevant event happens.

  For example, consider an event \(\edn{}\) under description \(\edo{}\) in which an agent has rolled nine using a two dice.
  Now, consider the event \(\edn{-}\) under description \(\edo{-}\) in which the dice are tumbling around in the agent's hands before being released on the table.
  By \autoref{cons:no-f-ref}, it is not the case that \(\edo{-}\) includes the information that \(\edn{-}\) develops into \(\edn{}\).
  And, if \(\edo{-}\) is limited a description of \(\edn{-}\) as an event in which dice are tumbling around in the agent's hands before being released on the table through some particular period of time, then it is not possible to observe via \(\edo{-}\) that \(\edn{-}\) is an event which develops into an event in which the agent rolls a nine.

  Of course, \(\edo{-}\) may be more detailed, and granting the present determines the future it may be the case that \(\edo{-}\) entails \(\edn{-}\) is an event which develops into an event in which the agent rolls a nine.
  However, this entailment rests on what is true of \(\edn{}\) being determined by what is true of \(\edn{-}\).

  Note, however, that even if one grants the present determines the future, it need not be the case that \(\edo{-}\) is sufficiently detailed to entail the agent rolls a nine.%
\end{note}


\begin{note}
  In general, while we assume a description of an event which has happened is satisfied by a unique event (\autoref{assu:HaUniqueD}) and that a description of an event is limited to what is true of the event when the event happened (\autoref{cons:no-f-ref}), we do not assume a description captures everything true of an event.

  In particular, when considering an event in which an agent concludes, we allow for the description to be common.
  Some technicalities (such as our use of \prop{1}, \val{1}, and \pool{1}) constrain the descriptions we use, but something like `the agent concluded \gistCalcEq{} has value \valI{True} for the first time by using a calculator' is about the level of detail I have in mind, though what follows is compatible with additional detail.
\end{note}



\section{Events in progress}
\label{sec:events-progress}


\begin{note}
  Our interest is understanding the way an event in which an agent concludes happens.
  And, the idea of an event is progress is a key idea with respect to our understanding of the way an event in which an agent concludes happens.

  This section briefly characterises the way in which we understand events in progress, and states two important assumptions we may about events in progress.
\end{note}



\subsection{Events in progress}


\begin{note}
  Events in progress are intuitively understood in terms of the progressive aspect.

  For example:
  % 
  \begin{enumerate}
  \item
    The agent is making soup.\newline
    \mbox{ } \hfill \(\leadsto\) An event in which the agent makes soup is in progress.
  \item
    The agent is reading Henley's `Invictus'.\newline
    \mbox{ } \hfill \(\leadsto\) An event in which the agent reads Henley's `Invictus' is in progress.
  \item
    The agent is riding the slope.\newline
    \mbox{ } \hfill \(\leadsto\) An event in which the agent rides the slope is in progress.
  \end{enumerate}
  % 
  Intuitively:

  \begin{intuition}[Events in progress and the progressive]
    \label{def:es-in-prog}
    \vspace{-\baselineskip}
    \begin{itemize}
    \item
      \(\ed{}\) is an event in which \(\ed{\ast}\) is in progress.
    \end{itemize}
    % 
    \emph{If and only if}:
    % 
    \begin{itemize}
    \item
      \(\edo{}\) entails:
      \(\text{Prog}[\ed[\ast]{}]\).\newline
      \mbox{ }\hfill (Where \(\text{Prog}[\ed{\ast}]\) is \(\edo{\ast}\) under the progressive aspect.)
    \end{itemize}
    \vspace{-.5\baselineskip}
  \end{intuition}

  {
    \color{blue}
    In other words, we get an additional description of \(\edn{}\).

    Note, \(\ed{}\) being an event in which \(\ed{\ast}\) is in progress is distinct from \(\ed{}\) being an event in which \(\edn{\ast}\) is in progress.
    Here, there is far more involved.
    The point is, we're only looking at \(\edn{\ast}\) under \(\edo{\ast}\).
    The description is limited.

    In this respect, disingenuous to mention \(\edn{\ast}\).
    However, relation between two specific events is important.
  }
  

  \noindent%
  For example, going by \autoref{def:es-in-prog}, the following are equivalent:
  \begin{itemize}
  \item
    \(\ed{}\) is an event in which Max concludes \propM{\gistCalcEq{}} is \valI{True} is in progress.
  \item
    \(\edo{}\) is true of \(\edn{}\) and \(\edo{}\) entails \(\text{Prog}[\text{Max concludes \propM{\gistCalcEq{}} has value \valI{True}}]\).
  \item
    \(\edo{}\) is true of \(\edn{}\) and \(\edo{}\) entails Max is concluding \propM{\gistCalcEq{}} is \valI{True}.
  \end{itemize}

  \noindent%
  We assume an implicit understanding based on progressive aspect.%
  \footnote{
    \nocite{Portner:1998um}
    \nocite{Engelberg:1999vi}
    Note, though, that English does not have a quick, unambiguous, way of expressing events in progress.
    For, consider the sentence:
    \begin{enumerate}[label=\arabic*., ref=(\arabic*)]
    \item
      \label{prog:abmig}
      \textquote{John is studying for an exam}.
    \end{enumerate}
    \ref{prog:abmig} may be understand to express either the continuous or progressive aspect.

    Under the continuous aspect, \ref{prog:abmig} captures something about John, rather than something about an event happening.
    Hence, it need to be the case that John is engaged in an event of studying when \ref{prog:abmig} is said.
    For example, we may expand:
    \textquote{Sam is studying for an exam, but is taking a short nap.}

    By contrast, \ref{prog:abmig} under the progressive captures an event in where John studying is in progress.
    For example, we may expand:
    \textquote{Sam is studying for an exam, so they aren't taking a nap.}

    See,~\textcite{Richards:1981wo},~\textcite{Portner:2011vi}, etc for a general overview of the progressive.
    In particular, I suggest \textcite{Landman:1992wh} as an nice technical introduction.
    % \citeauthor{Landman:1992wh} considers a variety of important considerations in a straightforward way and
    % And, \citeauthor{Landman:1992wh}'s the way in which \citeauthor{Landman:1992wh} wrangles closeness to provide an account of the progressive is illustrative of the difficulties and does not require much technical background.
    \citeauthor{Szabo:2004ul} (\citeyear[34]{Szabo:2004ul}) provides a concise summary:
    \begin{quote}
      [A] progressive sentence is true at some time just in case some event occurs at that time, and if we follow the development of the event (within our world as long as it goes, then jumping into a nearby world, and iterating the process within the limits of reasonability) we will reach a possible world where the perfective correlate is true of the continuation.
    \end{quote}
    % \citeauthor{Portner:1998um} (\citeyear[764--766]{Portner:1998um}) provides a more in depth summary.
    For some of the issues with \citeauthor{Landman:1992wh}'s account, see
    (\cite{Bonomi:1997uq}),
    (\cite[49--50]{Engelberg:1999vi}),
    (\cite[35]{Szabo:2004ul}),
    (\cite[767]{Portner:1998um}),
    and (\cite[1256]{Portner:2011vi}).
  }
\end{note}


\begin{note}
  Fix our understanding of events in progress via the progressive aspect due to a common feature of analyses of the progressive holds of events in progress:%
  \footnote{
    See, e.g.:
    (\cite{Bennett:1972uw}),
    (\cite{Dowty:1979vq}),
    (\cite{Parsons:1990aa}),
    (\cite{Landman:1992wh}),
    (\cite{Portner:1998um}).

    \assuPP{} is often motivated by the `imperfective paradox' (\cite[cf.][Ch.3.1]{Dowty:1979vq}).

    \citeauthor{Bach:1986tb} summarises:
    \begin{quote}
      [H]ow can we characterize the meaning of a progressive sentences like \ref{Bach:impP:17} on the basis of the meaning of a simple sentence like \ref{Bach:impP:18} when \ref{Bach:impP:17} can be true of a history without \ref{Bach:impP:18} ever being true?
      \begin{enumerate}[label=(\arabic*), ref=(\arabic*)]
        \setcounter{enumi}{16}
      \item
        \label{Bach:impP:17}
        John was crossing the street.
      \item
        \label{Bach:impP:18}
        John crossed the street.%
        \mbox{ }\hfill\mbox{(\citeyear[12]{Bach:1986tb})}
      \end{enumerate}
    \end{quote}
    % 
    The paradox amounts to two seemingly incompatible observations:
    \begin{enumerate}[noitemsep]
    \item
      \ref{Bach:impP:17} entails an event in which John is crosses the street is in progress.
    \item
      There need not be an (actual) event in which John is crosses the street.%
    \end{enumerate}
    % 
    For example, John may have been hit by a bus.
    % In parallel, it may be true that you are falling asleep before a fire alarm is set off.
    The intuitive solution to this `paradox' is that the event in which the agent completes the action need not be an actual event.
  }

  \begin{assumption}[\assuPP{2}]%
    \label{assu:PP}%
    \vspace{-\baselineskip}
    \begin{itenum}
    \item[\emph{If}:]
      \(\ed{}\) is an event in which \(\ed{\ast}\) is in progress.
    \item[\emph{Then}:]
      There is some \progAdj{0} event \(e^{\sharp}\) such that~\ref{assu:PP:pe:dev} and~\ref{assu:PP:pe:verb} are both true:
      \begin{enumerate}[label=\roman*., ref=(\roman*)]
      \item
        \label{assu:PP:pe:dev}
        \(\edn{\sharp}\) is a development of \(\edn{}\).
      \item
        \label{assu:PP:pe:verb}
        \(\edo{\sharp}\) is true of \(e^{\sharp}\), where \(\edo{\sharp}\) is the perfective correlate of \(\edo{\ast}\).
      \end{enumerate}
    \end{itenum}
    \vspace{-\baselineskip}
  \end{assumption}
\end{note}

\begin{note}[Interest with the progressive]
  The immediate role of \assuPP{} is to help fix intuitions about events in progress by reflecting on the sense of possibility at issue.%
  \footnote{
    \assuPP{0} is denied by some.
    For example, \citeauthor{Szabo:2004ul} argues:
    \textquote{Sometimes we are \emph{doing} things even though there is no real chance that we could get them \emph{done}, and this is true even if we abstract away from the possibility of miraculous intervention.}
    (\citeyear[40]{Szabo:2004ul})
    E.g., \citeauthor{Szabo:2004ul} denies~\ref{Szabo:Arch} is necessarily false:
    \begin{quote}
      \begin{enumerate}[label=(\arabic*), ref=(\arabic*)]
        \setcounter{enumi}{12}
      \item
        \label{Szabo:Arch}
        As the architect was building the cathedral he knew that, although he would be building it for another year or so, he couldn't possibly complete it.%
        \mbox{ }\hfill\mbox{(\citeyear[38]{Szabo:2004ul})}
      \end{enumerate}
    \end{quote}
    Though,~\ref{Szabo:Arch} seems always false to me.
    The only sense with which I read~\ref{Szabo:Arch} as true under the progressive requires factivity of knowledge to fail, thus allowing the cathedral to be built.

    See (\cite[1245]{Portner:2011vi}) for additional, distinct, discussion of (\cite{Szabo:2004ul}).
  }
  The deferred role of \assuPP{} is to help intuitively motivate a couple of ideas introduced in later chapters, and ease a handful of arguments.%
  \footnote{
    Specifically, \fc{1} in \autoref{cha:fcs} and \ros{1} in \autoref{cha:ros}.
  }

  We make three observations to help firm intuition.

  \begin{observation}[\assuPP{2} and existential modality]%
    \label{obs:prog-not-reg-poss}%
    The sense of possibility in \assuPP{} does not reduce to existential \{logical, metaphysical, nomic, \dots\} possibility.
  \end{observation}
  \begin{motivation}{obs:prog-not-reg-poss}
    Suppose an agent is sitting a multiple choice exam.
    To pass the exam the agent only needs to chose some number of correct choices.
    It is certainly logically, metaphysically, and nomically possible that the agent chooses a sufficient number of correct choices.
    However, it does not follow that the agent is passing the exam.%
    \footnote{
      See also Igal Kvart's example of Mary wiping out the Roman army (\cite[18]{Landman:1992wh}).
    }
  \end{motivation}

  \begin{observation}[\assuPP{2} and counterfactuals]%
    \label{obs:prog-not-cfs}%
    There is no simple relation between the sense of possibility in \assuPP{} and counterfactuals.
  \end{observation}
  \begin{motivation}{obs:prog-not-cfs}
    Suppose an agent is passing an exam without external help.
    Then, a classmate passes the agent some answers, which the agent uses.
    The agent is no longer passing the exam without external help.
    And, any close possible world where the classmate does not pass answers, some other classmate may pass the same answers.
  \end{motivation}

  \begin{observation}[\assuPP{} and uniqueness]%
    \label{obs:prog-no-unique}%
    The progressive may be true of an event without the event being sufficiently developed to `indicate' a unique outcome.
  \end{observation}
  \begin{motivation}{obs:prog-no-unique}
    Suppose an agent has drawn a straight line on a piece of paper.
    It may be true that the agent is drawing a triangle.
    However, the straight line is compatible with the agent drawing an \(n\)-sided polygon, for any \(n\) within a some reasonable bound.%
    \footnote{
      This observation is inspired by \citeauthor{Dowty:1979vq}'s example involving a circle and a triangle (\citeyear[133]{Dowty:1979vq}).
    }
  \end{motivation}

  \noindent%
  Loosely paraphrased, if an event is in progress, then intuitively there is \emph{something} about the way things are which ensures the existence of a possible completion event (\autoref{obs:prog-no-unique}) which is robust against external influence (\autoref{obs:prog-not-cfs}) and does not require luck (\autoref{obs:prog-not-reg-poss}).
\end{note}


\begin{note}
  A brief observation, which contrasts with \autoref{assu:HaUniqueD}:

  \begin{observation}[Events in progress and common satisfaction]%
    \label{obs:eip-partial}%
    Sometimes, when \(\ed{}\) is an event in which \(\ed{\ast}\) is in progress, \(\edo{\ast}\) may be satisfied by multiple events.
  \end{observation}

  \begin{motivation}{obs:eip-partial}
    Consider an agent flipping a coin until it lands heads or lands tails ten times in a row.
    Given sufficient determination from the agent, it is true that an event in which the agent flips a coin until it lands heads or lands tails ten times in a row is in progress.

    Now, a unique event in which the agent flips a coin until it lands heads or lands tails ten times in a row involves an \(i\)th throw the coin lands heads on, or ten throws in which the coin lands tails.
    However, by \autoref{cons:no-f-ref}, it must be true of \(\ed{}\) that \(\ed{\ast}\) is in progress.
    And, as coin flips are quite random (\cite{Gelman:2002ww}), \(\ed{}\) may fail to entail that the coin lands heads on the \(i\)th throw the coin lands heads, or that the coin lands tails on all ten throws.

    Hence, if \(\ed{}\) fails to entail either that the coin lands heads on the \(i\)th throw the coin lands heads, or that the coin lands tails on all ten throws, then \(\ed{\ast}\) may be satisfied by either event.
  \end{motivation}
\end{note}



\subsection{An assumption}
\label{sec:assumptions-1}


\begin{note}
  Events in progress, rough intuition.
  For, following assumption.

  \begin{assumption}[Exclusivity]
    \label{assu:p:ex}
    \vspace{-\baselineskip}
    \begin{itenum}
    \item[\emph{If}:]
      It is not possible for \(d^{\#}\) and \(d^{\$}\) to both be true of an event.
    \item[\emph{Then}:]
      It is not possible for \(\ed{}\) to be an event in which \(\ed{\#}\) to be in progress and \(\ed{\$}\) is in progress.
    \end{itenum}
    \vspace{-\baselineskip}
  \end{assumption}

  \noindent%
  Paraphrased, \autoref{assu:p:ex} assumes that if an event is in progress, then no incompatible event is (also) in progress.

  For example:
  It is not possible for both team A wins the game of hockey and for team B to wins the game of hockey to be true of an event.
  So, by \autoref{assu:p:ex}, it is not possible for team A to be winning the game of hockey and for team B to be winning the game of hockey.
  Hence, if team A is winning the game of hockey, team B is not winning the game of hockey.
\end{note}


\begin{note}
  \autoref{assu:p:ex} is about events in progress.
  And, the progressive aspect helps fix what events in progress are (\autoref{def:es-in-prog}).
  However, \autoref{assu:p:ex} does not clearly hold of the progressive aspect.
  For example, \citeauthor{Landman:1992wh} writes:%
  \footnote{
    See \textcite{Bonomi:1997uq} for additional discussion.
  }

  \begin{quote}
    Suppose I was on a plane to Boston which got hijacked and landed in Bismarck, North Dakota.
    What was going on before the plane was hijacked?
    One thing I can say is:
    `I was flying to Boston when the plane was hijacked'
    This is reasonable.
    But another thing I could say is:
    `I was flying to Boston.
    Well, in fact, I wasn't, I was flying to Bismarck, but I didn't know that at the time'
    And this is also reasonable.%
    \mbox{ }\hfill\mbox{(\citeyear[30--31]{Landman:1992wh})}
  \end{quote}
  % 
  \citeauthor{Landman:1992wh}'s observation is compatible with \autoref{assu:p:ex} to the extent that \citeauthor{Landman:1992wh} does not suggest it is reasonable to say `I was flying to Boston \emph{and} I was flying to Bismarck'.
  However, \citeauthor{Landman:1992wh}'s observation is compatible with \autoref{assu:p:ex} to the extent that \citeauthor{Landman:1992wh} suggests it is reasonable to say `I was flying to Boston' \emph{and} `I was flying to Bismarck'.

  More could be said.
  For example, \citeauthor{Landman:1992wh}'s first statement is implicitly (and the explicitly clarified as) given with respect to what they knew, and \citeauthor{Landman:1992wh}'s second statement takes into account details about what happened after the event.
  And, it is reasonable to say various things, regardless of whether those things are true.
  However, I see no real benefit in raising \autoref{def:es-in-prog} to a definition by digressing into an account of the progressive.%
  \footnote{
    Though, an account of the progressive which ties the progressive to causation, such as \citeauthor{Szabo:2004ul}'s (\citeyear{Szabo:2004ul}) account, arguably entails \autoref{assu:p:ex}.
  }
  Hence, \autoref{assu:p:ex} captures an important way in which our understanding of an event in progress may differ from the progressive aspect.
\end{note}





\section{\se{3} and \progEx{}}
\label{sec:se3-progex}


\begin{note}
  Now, events in progress and answers to the sense of `why' present in \qWhy{}.

  Definition of a \se{}.
\end{note}


\subsection{\se{3}}

\begin{note}
  \begin{definition}[\se{3}]
    \label{def:se}
    \vspace{-\baselineskip}
    \begin{itemize}
    \item
      \(\ed{\flat}\) is a \emph{\se{0}} of \(\ed{}\).
    \end{itemize}
    \emph{If and only if}:
    \begin{itemize}
    \item
      Clauses~\ref{assu:p:se:prog} and \ref{assu:p:se:hCon} hold:
      \begin{enumerate}[label=\Alph*., ref=\Alph*]
      \item
        \label{assu:p:se:prog}
        \(\ed{\flat}\) is such that \(\ed{}\) is in progress.
      \item
        \label{assu:p:se:hCon}
        \(\ed{}\) happens --- in part --- as a result of \(\ed{\flat}\).
      \end{enumerate}
    \end{itemize}
    \vspace{-\baselineskip}
  \end{definition}

  \noindent%
  \(\ed{\flat}\) being a \emph{\se{0}} of \(\ed{}\) concerns specific events \(\edn{\flat}\) and \(\edn{}\).
  And, of interest is whether specific descriptions \(\edo{}\) and \(\edo{\flat}\) capture a particular connexion between \(\edn{\flat}\) and \(\edn{}\).

  Still, our interest is only with parts of \(\edn{\flat}\) and \(\edn{}\), respectively.
  Intuitively:
  %
  \begin{itemize}
  \item
    Clause~\ref{assu:p:se:prog} `looks forward':

    The description \(\edo{\flat}\) of \(\edn{\flat}\) captures that an event described by \(\edo{}\) is in progress.
  \item
    Clause~\ref{assu:p:se:hCon} `looks backward':

    \(\edn{}\) as described by \(\edo{}\) happened in part as a result of \(\edn{\flat}\) as described by \(\edo{\flat}\).
  \end{itemize}

  The role of Clause~\ref{assu:p:se:prog} is to ensure the event \(\edn{}\) is favoured over some other event.
  For, by \autoref{assu:p:ex} as \(\ed{}\) is in progress, it is not the case that \(\ed[x]{}\) is in progress, grating it is not possible for \(\edo{}\) and \(\edo{x}\) to both be true of an event.

  And, the role of Clause~\ref{assu:p:se:hCon} is to ensure \(\edn{}\) happens as a result of being favoured.
\end{note}


\begin{note}
  The role of clauses~\ref{assu:p:se:prog} and \ref{assu:p:se:hCon} combined to give answers to `why' (in the sense of \qWhy{}) \(\ed{}\) happened.
  Still, for the moment our interest is with what clauses~\ref{assu:p:se:prog} and \ref{assu:p:se:hCon} say rather than their role.

  As Clause~\ref{assu:p:se:prog} simply places a constraint on the relevant description \(\edo{\flat}\) our discussion focuses on Clause~\ref{assu:p:se:hCon}.
  We briefly discuss what is it is for an event to happen `--- in part --- as a result of' some other event.
  Following this, we provide a handful of observations (and a proposition) which important features of Clause~\ref{assu:p:se:hCon}.
  Finally, before moving on, we provide an argument for \se{} with respect to \autoref{illu:gist:roots:F}.
\end{note}


\begin{note}
  I have little to say in abstract about what is it is for an event to happen `--- in part --- as a result of' some other event.
  Roughly, it was not possible for \(\ed{}\) to happen without \(\ed{\flat}\).
  In this respect, \(\ed{}\) only happened --- in part --- as a result \(\ed{\flat}\).%
  \footnote{
    I suspect `causally depends on' may be substituted in place of `happens --- in part --- as a result of'.

    For example, if \(\ed{}\) causally depends on \(\ed{\flat}\) \emph{just in case} \(\ed{}\) would not occur were \(\ed{\flat}\) not occur (\cite[cf.][1.1]{Menzies:2020aa}), then the agent buttering toast causally depends on the bread being toasted.

    However, in contrast to an event being --- in part --- the result of some other event I doubt there is much pre-theoretical intuition for what causal dependence amounts to.

    A number of Aesop's fable, in translation at least, talk about events being results of other events and I have not found a (translation of a) fable that talks about causal dependence.
    E.g.:
    %
    \begin{quote}
      A deer had fallen ill and was resting on the grassy plain.
      When the other animals came to see her, they ate up all the grass in her pasture.
      As a result, when the deer recovered from her illness, she ended up dying since her pasture had come to an end.%
      \mbox{ }\hfill\mbox{(\cite[124]{Aesop:2002aa})}
    \end{quote}
    %
    Still, Clause~\ref{assu:p:se:prog} requires \(\ed{}\) is in progress, and if the progressive is understood via causation as suggested, e.g. \textcite{Szabo:2004ul}, then both clauses may then be expressed via causation.
  }

  For example, it is not possible for an agent to be paddling in the ocean unless the agent steps foot into the ocean.
  Hence, the agent paddling in the ocean happened --- in part --- as the result of the agent stepping foot into the ocean.

  Likewise, it is not possible for an agent to butter some toast without some bread being toasted.
  Hence, the agent buttering toast happened --- in part --- as a result of the bread being toasted.

  Note, however, \(\ed{}\) happening --- in part --- as a result of \(\ed{\flat}\) does not follow from \(\ed{\flat}\) being such that an event \(\ed{}\) is in progress:

  \begin{observation}[Condition \ref{assu:p:se:hCon}!]%
    \label{obs:se-need-hCon}%
    It may be the case that:
    \begin{itemize}
    \item
      \(\ed{\flat}\) is such that an event \(\ed{}\) is in progress and \(\ed{}\) happens.
    \item
      \(\ed{}\) does not happen as a result of \(\ed{\flat}\).
    \end{itemize}
    \vspace{-\baselineskip}
  \end{observation}

  \begin{motivation}{obs:se-need-hCon}
    Suppose an agent is passing an exam without external help.
    Then, a classmate passes the agent some answers, which the agent uses.
    The agent passes the exam, and was passing an exam without external help but the agent does not pass the exam as a result passing an exam without external help.
  \end{motivation}
\end{note}


\begin{note}
  \begin{proposition}[Limited descriptions]
    \label{prop:se-d-lim}
    \vspace{-\baselineskip}
    \begin{itenum}
    \item[\emph{If}:]
      \(\ed{}\) happens --- in part --- as a result of \(\ed{\flat}\).
    \item[\emph{Then}:]
      \(\edo{\flat}\) does not include features of \(\edn{\flat}\) that \(\ed{}\) does not happen as a result of.
    \end{itenum}
    \vspace{-\baselineskip}
  \end{proposition}

  \noindent%
  In other words, \(\edn{}\) as captured by \(\edo{}\) constrains which descriptions of \(\edn{\flat}\) are of interest.

  \begin{argument}{prop:se-d-lim}
    Expanded, our interest is in whether \(\edn{}\) as described by \(\edo{}\) happens --- in part --- as a result of \(\edn{\flat}\) as described by \(\edo{\flat}\).

    Our interest is \emph{not} with whether \(\edn{}\) as described by \(\edo{}\) happens --- in part --- as a result of \(\edn{\flat}\) as described by \emph{some part of} \(\edo{\flat}\).

    In this respect, \(\ed{}\) must be a result of everything captured by \(\edo{\flat}\).
  \end{argument}

  \noindent%
  For example, suppose \(\edo{}\) is the description `an agent shuffles a deck so that \mainCard{} is \mainCardPos{} by sleight of hand', and \(\edo{}\) is true of \(\edn{}\).
  Further, suppose \(\edo{\flat}\) is the description `the agent marks the \mainCard{} so they do not lose the card in a shuffle' and \(\edo{\flat}\) is true of \(\edn{\flat}\).

  Now, intuitively \(\ed{}\) happens --- in part --- as a result of \(\ed{\flat}\).
  For, \(\ed{\flat}\) captures part of the sleight of hand.
  It is not possible for the agent to shuffle a deck so that \mainCard{} is \mainCardPos{} by sleight of hand without ensuring they do not lose \mainCard{} in the shuffle.%
  \footnote{
    Of course, the agent may randomly shuffle a deck so that \mainCard{} is \mainCardPos{}, but then the agent has not shuffled a deck \emph{so that} \mainCard{} is \mainCardPos{}.
  }

  However, \(\edo{\flat}\) may be expanded to include further details.
  For example, let \(\edo{\flat +}\) be the description `the agent marks the \mainCard{} so they do not lose the card in a shuffle while finding a hole in their sock'.
  Intuitively, it is not the case that \(\ed{}\) happens --- in part --- as a result of \(\ed{\flat +}\).
  For, the agent finding a hole in their sock had no influence on whether the agent marked \mainCard{}.

  Though, if \(\edn{}\) is described by \(\edo{+}\) as shuffles the deck by performing sleight of hand while coming to regret not buying a new pair of socks in the January sale, then \(\ed{+}\) plausibly happens as a result of \(\ed{\flat +}\).
\end{note}


\begin{note}
  Our interest is not with whether \emph{any} event captured by description \(\edo{}\) is a result of any event captured by description \(\edo{\flat}\).
  This is highlighted by the following observation.

  \begin{observation}
    \label{obs:theEventsHuh}
    It may be the case that \(\ed{}\) happens --- in part --- as a result of \(\ed{\flat}\) only given details of \(\edn{}\) and \(\edn{\flat}\) which are not captured by \(\edo{}\) and \(\edo{\flat}\).
  \end{observation}

  \begin{motivation}{obs:theEventsHuh}
    Consider the example just given following \autoref{prop:se-d-lim}.

    \(\edo{}\) is the description `an agent shuffles a deck so that \mainCard{} is \mainCardPos{} by sleight of hand', and \(\edo{}\) is true of \(\edn{}\).
    And, \(\edo{\flat}\) is the description `the agent marks the \mainCard{} so they do not lose the card in a shuffle' and \(\edo{\flat}\) is true of \(\edn{\flat}\).

    \(\ed{}\) intuitively happens --- in part --- as a result of \(\ed{\flat}\).
    However, \(\edo{\flat}\) does not guarantee that the agent successfully completes the shuffle.
    For example, it is consistent with \(\edo{\flat}\) that the agent loses their mark.
    Still, the implicit understanding of \(\edn{}\) and \(\edn{\flat}\) is that the two shuffles are connected.
    And, so long as the two shuffles are connected \(\ed{}\) plausibly does happen --- in part --- as a result of \(\ed{\flat}\) for the motivated given above.
  \end{motivation}
\end{note}


\begin{note}
  \autoref{obs:theEventsHuh} may be seen to pair with \autoref{prop:se-d-lim}.
  For, our interest is in whether \(\edn{}\) as described by \(\edo{}\) happens --- in part --- as a result of \(\edn{\flat}\) as described by \(\edo{\flat}\).
  This means that everything in the description \(\edo{\flat}\) must be relevant to whether \(\edn{}\) as described by \(\edo{}\) happens.
  However, this does not means that everything relevant to whether \(\edn{}\) as described by \(\edo{}\) happens is captured by \(\edo{\flat}\).
\end{note}



\begin{note}
  In short, I take the idea \(\ed{}\) happening --- in part --- as a result of \(\ed{\flat}\) to be intuitive.

  Our interest is in the feature of \(\edn{}\) captured by description \(\edo{}\) being --- in part --- a result of the feature of \(\edn{\flat}\) captured by description \(\edo{\flat}\).
  Hence, \(\edo{\flat}\) is limited to what is relevant to \(\edn{}\) as captured by description \(\edo{}\) happening (\autoref{prop:se-d-lim}) though \(\edo{\flat}\) need not exhaust what is relevant to \(\edn{}\) as captured by description \(\edo{}\) happening (\autoref{obs:theEventsHuh}).
\end{note}


\begin{note}
  To close this section we provide an argument for \se{} with respect to \autoref{illu:gist:roots:F}.

  \begin{observation}[A \se{} of \autoref{illu:gist:roots:F}]
    \label{obs:se-inst}%
    Given the \(\edn{}\) is the event described by \autoref{illu:gist:roots:F} and \(\edo{}\) is the description `the agent concludes \propM{\rootsCon{}} has \val{0} \valI{True} from \(\Phi\)', where captures the \agents{} understanding of factorisation:

    \begin{itemize}
    \item
      There is some description \(\edo{\flat}\) such that the event \(\edn{\flat}\) which covers Step~\ref{illu:gist:roots:F:factor} of the \agents{} reasoning is a \se{} of \(\ed{}\).
    \end{itemize}
    \vspace{-\baselineskip}
  \end{observation}

  \begin{motivation}{obs:se-inst}
    Let \(\edn{\flat}\) be the event which covers Step~\ref{illu:gist:roots:F:factor} of the \agents{} reasoning.

    \autoref{illu:gist:roots:F} describes \(\edn{\flat}\) as an event in which the agent is attempting to find possible values for \(x\) such that \(2x^{2} - x - 1 = 0\) and figures out \((2x + 1)(x - 1) = 0\).
    I take it to be sufficiently clear that the agent \emph{is} (regardless of any description) concluding \propM{\rootsCon{}} has \val{0} \valI{True} from \(\Phi\) in \(\edn{\flat}\).
    Therefore, we fix a description \(\edo{\flat}\) of \(\edn{\flat}\) which states, roughly, `the agent is concluding \propM{\rootsCon{}} has \val{0} \valI{True} from \(\Phi\) in \(\edn{\flat}\) and figures out \((2x + 1)(x - 1) = 0\)'.%
    \footnote{
      We have fixed a description such that clauses~\ref{assu:p:se:prog} and \ref{assu:p:se:desc} are mostly immediate.
      This is in keeping with the spirit of \autoref{def:se}.
      For, the role of \(d'\) is to avoid capturing irrelevant details of \(\edn{}\) --- e.g.\ that the agent concludes \propM{\rootsCon{}} has \val{0} \valI{True} from \(\Phi\) without furrowing their brow.
    }
    \medskip

    Okay, there's a trick here.
    We're interested in a specific event \(\edn{}\).
    This is not any event \(\edn{\sharp}\) such that \(\edo{}\) is true of \(\edn{\sharp}\).
    Hence, though we want to show \(\edn{}\) under description \(\edo{}\) happens as a result of \(\ed{\flat}\), we may still appeal to certain features of \(\edn{}\) which are not captured by \(\edo{}\).

    For our purposes, \(\edn{}\) is that there is no event after \(\edn{\flat}\) in which the agent re-figures out \((2x + 1)(x - 1) = 0\).%
    \footnote{
      This constraint is motivated by the observation that the agent may do Step~\ref{illu:gist:roots:F:factor}, be interrupted, discard the result and then factor \(2x^{2} - x - 1 = 0\) again.
      In this case, Step~\ref{illu:gist:roots:F:factor:done} plausibly need not happen as a result of Step~\ref{illu:gist:roots:F:factor}.
      This does not follow from \(\edo{\flat}\) alone.
      An event in which the agent concludes \propM{\rootsCon{}} has \val{0} \valI{True} from \(\Phi\) is in progress (cf.\ \autoref{def:es-in-prog}) and by \assuPP{} possible event.
      However, \dots
    }

    The argument is by contradiction.

    Suppose \(\edn{}\) does not happen as a result of \(\ed{\flat}\).
    This is a contradiction.

    For, by supposition \(\edn{}\) does not happen as a result of an event in which the agent figures out \((2x + 1)(x - 1) = 0\).
    And, \(\edn{}\) is an event in which the agent concludes \propM{\rootsCon{}} has \val{0} \valI{True} from a \pool{} of premises which captures their understanding of factorisation.
    Yet, as the agent did not conclude by factoring \(2x^{2} - x - 1 = 0\), it cannot be the case that the \agents{} concluded by their understanding of factorisation.
    The agent must have concluded \propM{\rootsCon{}} has \val{0} \valI{True} by some other \pool{}.
    \medskip

    So, the proposition.
    This is satisfied.
    Does happen as a result of figuring out.

  \end{motivation}

  \noindent%
  The key idea of \autoref{obs:se-inst} is somewhat straightforward:
  Observe it makes no sense for \(\edo{}\) to be true of \(\edn{}\) if it is not the case that \(\ed{}\) happens --- in part --- as a result of \(\edo{\flat}\) being true of \(\edn{\flat}\).
\end{note}



\subsection{\progEx{2}}
\label{sec:what-these-do}

\begin{note}
  With \autoref{assu:p:ex} and \autoref{def:se} in hand, an observation about events in progress an explanations about why an event happened --- given the sense of `why' present in \qWhy{} --- follows.

  \begin{observation}[\progEx{2}]%
    \label{obs:PE}%
    Given \(\ed{}\) is an event in which \vAgent{} does \(a\):

    \begin{itenum}
    \item[\emph{If}:]
      There is some \se{} \(\ed{\flat}\) of \(\ed{}\) such that:
      \begin{itemize}
      \item
        \(\edo{\flat}\) being true of \(\edn{\flat}\)  (non-vacuously) entails \(f\) is true of \(\edn{\flat}\).
      \end{itemize}
    \item[\emph{Then:}]
      \(f\) explains `why' \(\ed{}\) happened, in the sense of `why' present in \qWhy{}.
    \end{itenum}
    \vspace{-\baselineskip}
  \end{observation}

  \noindent%
  The motivation for \autoref{obs:PE} follows from the way in which we understand the sense of `why' present in \qWhy{}, the definition of a \se{} (\autoref{def:se}) and our assumption that it is not possible for two incompatible events to be in progress (\autoref{assu:p:ex}).

  In particular, we understand the sense of `why' present in \qWhy{} in terms of why did the event \(\ed{}\) in which an agent concludes some \prop{0} \(\phi\) has \val{0} \(v\) from some \pool{0} \(\Phi\) happen, rather than any incompatible event --- e.g., an event in which the agent failed to conclude \(\phi\) has \val{0} \(v\) from \(\Phi\).

  Our interest is with explanations of `why' \(\ed{}\) happened over any other event.
  However, our interest is not with \emph{complete} explanations of `why' \(\ed{}\) happened over any other event.
  Hence, if it is possible to point to something that is involved in \(\ed{}\) being favoured over any other incompatible event happening, then that thing explains `why' \(\ed{}\) happened, for our purposes.

  In this respect, things which explain `why' \(\ed{}\) happened over any other event do not necessarily need to be particularly interesting, they just need to be sufficiently involved in `why' it is the case that \(\ed{}\) happened over any other event.

  {
    \color{blue}
    In particular, our interest is with \ros{}.
    These are designed to be an abstraction.
    Hence, it is not possible to argue that a \ros{} has got to be part of any description.
    This is what motivate entailment.
    This abstraction is something which we get from a \se{} and in turn due to \se{1} answering why, \ros{} also answers why.
    In other words, \progEx{0} provides, in some detail, sufficient conditions for a \ros{} to be an answer to \qWhy{} so long as entailment.
    }


  \begin{motivation}{obs:PE}
    Consider \(\ed{\flat}\) as a \se{} of \(\ed{}\).
    \medskip

    \noindent%
    By Clause~\ref{assu:p:se:prog}, \(\ed{\flat}\) is such that \(\ed{}\) is in progress.
    So, \autoref{assu:p:ex}, it is not the case that \(\ed{\ast}\) is in progress, given it is not possible for \(\edo{}\) and \(\edo{\ast}\) to both be true of an event.
    So, \(\ed{\flat}\) being such that \(\ed{}\) is in progress favours \(\ed{}\) happening over any other incompatible event happening.
    \medskip

    \noindent%
    By Clause~\ref{assu:p:se:hCon}, \(\ed{}\) happens --- in part --- as a result of \(\ed{\flat}\).
    So, \(\ed{}\) happened --- in part --- as a result of something with favoured \(\ed{}\) happening over any other incompatible event happening.
    \medskip

    \noindent%
    In other words, Clause~\ref{assu:p:se:hCon} tells us \(\ed{}\) happened --- in part --- as a result of \(\ed{\flat}\) and that happened and Clause~\ref{assu:p:se:prog} allows us to expand this observation to see \(\ed{}\) happened --- in part --- as a result of something which favoured \(\ed{}\) happening over any other (incompatible) event.

    In this respect, \(\edo{\flat}\) being true of \(\ed{}\) explains `why' \(\ed{}\) happened, as opposed any other event.
    {
      \color{blue}
      For, happen and happen over any other event.
    }
    \medskip

    \noindent%
    Finally, consider \(f\) such that \(f\) is (non-vacuously) entailed by \(\edo{\flat}\).


    By \autoref{prop:se-d-lim} \(\edo{\flat}\) does not include features of \(\edn{\flat}\) that \(\ed{}\) does not happen as a result of.
    So, as \(f\) is (non-vacuously) entailed by \(\edo{\flat}\) and \(\edo{\flat}\) does not include features of \(\edn{\flat}\) that \(\ed{}\) does not happen as a result of, it is not possible to capture the features of \(\edn{\flat}\) of \(\ed{}\) captured by \(\edn{\flat}\) without \(f\) being the case.
    In short, \(f\) must be the case in order for \(\ed{}\) to happen --- in part --- as a result of \(\ed{\flat}\).
    \smallskip

    Now, it is not necessarily the case that \(f\) is a \se{}.
    However, when we turn to why, we're not interested in capturing everything about \(\edo{\flat}\).
    What we're interested in understanding way in which event happened.
    Answers to why themselves do not need to be such that \(\ed{}\) is a result of those things.
    But, everything involved in an \se{} provides understanding, and in this respect explain why.

    As such, \(f\) (also) explains `why' \(\ed{}\) happened, as opposed any other event.
    For, \(f\) is just part of the explanation of `why' \(\ed{}\) happened, as opposed any other event given by \(\edo{\flat}\).
  \end{motivation}
\end{note}




\begin{note}
  Given previous:
  
  \begin{observation}[\progEx{2}, conclusions, and \ros{1}]%
    \label{sketch:PE:cROS}%
    Given \(\ed{}\) is an event in which \vAgent{} concludes \(\phi\) has value \(v\) from \(\Phi\):

    \begin{itenum}
    \item[\emph{If}:]
      Both true:
      \begin{itemize}
      \item
        \(\ed{}\) is event in which agent concludes \(\phi\) has \val{0} \(v\) from \(\Phi\).
      \item
        There is some \se{} \(\ed{\flat}\) of \(\ed{}\) such that:
        \begin{itemize}
        \item
          \(\edo{\flat}\) entails it is true of \(\edn{\flat}\) that:
          A \ros{} between \(\psi\), \(v'\) and \(\Psi\) holds throughout \(\ed{\flat}\).
        \end{itemize}
      \end{itemize}
    \item[\emph{Then:}]
      The \ros{} between \(\psi\), \(v'\) and \(\Psi\) explains `why' \(\ed{}\) happened, in the sense of `why' present in \qWhy{}.
    \end{itenum}
    \vspace{-\baselineskip}
  \end{observation}

  \begin{motivation}{sketch:PE:cROS}
    \autoref{sketch:PE:cROS} follows from \autoref{obs:PE}, by taking the  action \(a\) to be a conclusion that \(\phi\) has value \(v\) from \(\Phi\), and \(f\) to be a \ros{} between \(\psi\), \(v'\) and \(\Psi\).
  \end{motivation}

  \noindent%
  In \autoref{cha:var} we re-express \autoref{sketch:PE:cROS} as a sufficient condition for answers to \qWhy{}, after clarifying the way in which we understand conclusions and \ros{}.

  Still, \autoref{sketch:PE:cROS} hints at the overall argument to follow.
  For, given an event in which an agent concludes \(\phi\) has value \(v\) from \(\Phi\) then if the antecedent is satisfied by a \ros{} between \(\psi\), \(v'\) and \(\Psi\) and there is no past or present event in which the agent concludes \(\psi\) has value \(v'\) from \(\Psi\), then \issueInclusion{} fails to hold.
\end{note}



\begin{note}
  Setting the particular way \progEx{} is motivated aside, I take the basic idea of \progEx{} to be intuitive.
  For example, consider \citeauthor{Hempel:1965aa}'s Deductive-Nomological account of scientific explanation:
  \phantlabel{mention:Hempel:1}
  % 
  \begin{quote}
    [A Deductive-Nomological] explanation answers the question
    `\emph{Why} did the explanandum-phenomenon occur?'
    by showing that the phenomenon resulted from certain particular circumstances, specified in \(C_{1}, C_{2}, \dots C_{k}\), in accordance with the laws \(L_{1}, L_{2}, \dots L_{\gamma}\).
    By pointing this out, the argument shows that, given the particular circumstances and the laws in question, the occurrence of the phenomenon \emph{was to be expected}; and it is in this sense that the explanation enables us to \emph{understand why} the phenomenon occurred.%
    \mbox{ }\hfill\mbox{(\citeyear[337]{Hempel:1965aa})}
  \end{quote}
  % 
  Observe, \citeauthor{Hempel:1965aa} stress a connexion with observing an event \textquote{was to be expected} and an explanation allowing us to \textquote{understand why} the event occurred (the italics are \citeauthor{Hempel:1965aa}'s).
  {
    \color{blue}
    Though, while \citeauthor{Hempel:1965aa} details the link with circumstances and laws, while we detail the link in terms of a \se{}.
  }
\end{note}


\section*{Summary}
\label{sec:summary}


\begin{note}
  \begin{itemize}
  \item
    An event in which an agent is concluding is an event in which an event in which the agent concludes is in progress.
  \end{itemize}
\end{note}



% \subsection{Ex}

% {
% \color{red}
% Here, this example is to be reworked to highlight interesting features of events in progress and \progEx{}.

% Basically, when scoring darts, event in progress through a variety of different things happen.
% Then, as play goes on, this constrains what the agent may do (as the need to score 501).
% E.g. if it's the last throw and the agent has a score of 481, then they need to hit a 20.

% In turn, one the game is over, understand why agent won, as, e.g. they were hitting a 20 on the last throw.
% }

%   \begin{note}
%     \begin{illustration}[Darts]
%       Agent wins at darts just in case there is some action available to the agent, such that if the agent were to perform the action they would be winning at darts.

%       Winning is a complex action.
%       An agent has three dart throws to lower their score from 501 to 0 before play switches to the other player, and play continues until neither player may lower their score further on their next turn (without going past 0).
%       Playing a game is a complex action, as the region of a dartboard an agent wishes to hit changes according to previous throws.
%       For example, if the agent's score is 51 with three throws remaining, the agent will not wish to hit bullseye, as there is no way to reduce their score by a single point using two darts.
%       If the agent goes on to hit 20, then the score of the remaining to darts should equal 31, and so on.
%     \end{illustration}

%     Note, the initial sequence of actions may be more or less arbitrary.
%     It is not possible to score 501 in three or six dart throws, so an agent \emph{could} start by throwing a few darts blindly, so long as they have sufficient skill to recover on subsequent throws.

%     Of course, throwing darts is quite different from concluding, but this note extends.
%     An agent may be concluding a theorem is true even though their first line of enquiry turns out to be a dead end, etc.

%     This is the appeal of the progressive.
%   \end{note}



%%% Local Variables:
%%% mode: latex
%%% TeX-master: "master"
%%% TeX-engine: luatex
%%% End:
