\chapter{Events, in progress}
\label{cha:events-progress}


\section{Events}
\label{sec:events}


\begin{note}
  \begin{assumption}
    An event \(e\) is always understood in terms of some description \(d\).
  \end{assumption}

  The role of description is to fix a particular characterisation of an event.
  For example, distinction between an event in which an agent concludes and an event in which an agent concludes by using a calculator.
  Former is compatible with it being the case that the agent concludes by their understanding of arithmetic.

  It is safe to assume the description is detailed.
  For example, concludes, information about method and so on.

  {
    \color{red}
    Hence, in the case of \qWhy{} and \qHow{}, intuitive the description is detailed.
    We're not just asking why and how event is such that the agent concludes, but additional details such as method and so on.
  }

  In most cases, I expect the relevant description is clear.
  However, in certain cases explicitly stating the relevant description is important for clarity.

  \begin{notationList}
  \item
    For ease of expression `\(e_{d}\)' abbreviate `\(e\) under description \(d\)'.

    And, `\(e\)' abbreviates `\(e\) under any description \(d\)'
  \end{notationList}

  We place one important constraint on descriptions:

  \begin{constraint}{Fog}{Fog}
    Description does not reference future events.
  \end{constraint}

  For example, coin flip.
  The coin flip lands heads.
  It is not the case that \(d\) states that coin flip which lands heads.

  However, description may contain information which entails something happens.
  For example, hit the billiard ball.
  Then, determinism, missed the pocket.
\end{note}

\begin{note}

\end{note}



\section{Events in progress}
\label{sec:events-progress}


\begin{note}
  Events in progress are understood in terms of the progressive aspect.

  \begin{definition}[Events in progress]
    \vspace{-\baselineskip}
    \begin{itemize}
    \item
      \(e_{d}\) is an event in which \vAgent{} does action \(a\) is in progress.
    \end{itemize}
    % 
    \emph{If and only if}:
    % 
    \begin{itemize}
    \item
      It is true that \(e_{d}\) is an event in which \vAgent{} is \(a\)ing.
    \end{itemize}
    \vspace{-\baselineskip}
  \end{definition}

  \noindent%
  Assume an implicit understanding based on progressive aspect.%
  \footnote{
    \nocite{Portner:1998um}
    \nocite{Engelberg:1999vi}
    I suggest \textcite{Landman:1992wh} as an introduction, as \citeauthor{Landman:1992wh} considers a variety of important considerations in a straightforward way.
    Further, \citeauthor{Landman:1992wh}'s account of the progressive is developed in terms of closeness between worlds, and the way in which \citeauthor{Landman:1992wh} wrangles closeness to provide an account of the progressive is illustrative of the difficulties and does not require much technical background.

    \citeauthor{Szabo:2004ul} (\citeyear[34]{Szabo:2004ul}) provides a concise summary:
    \begin{quote}
      [A] progressive sentence is true at some time just in case some event occurs at that time, and if we follow the development of the event (within our world as long as it goes, then jumping into a nearby world, and iterating the process within the limits of reasonability) we will reach a possible world where the perfective correlate is true of the continuation.
    \end{quote}
    See also (\cite[764--766]{Portner:1998um}) for a more in depth summary.

    For some of the issues with \citeauthor{Landman:1992wh}'s account, see (\cite{Bonomi:1997uq}), (\cite[49--50]{Engelberg:1999vi}), (\cite[35]{Szabo:2004ul}), (\cite[767]{Portner:1998um}), and (\cite[esp.][1256]{Portner:2011vi}).
  }

\end{note}



\begin{note}
  \noindent%
  English does not have a quick, unambiguous, way of expressing events in progress.
  For, consider the sentence:
  \begin{enumerate}[label=\arabic*., ref=(\arabic*)]
  \item
    \label{prog:abmig}
    \textquote{John is studying for an exam}.
  \end{enumerate}
  \ref{prog:abmig} may be understand to express either the continuous or progressive aspect.

  Under the continuous aspect, \ref{prog:abmig} captures something about John, rather than something about an event happening.
  Hence, it need to be the case that John is engaged in an event of studying when \ref{prog:abmig} is said.
  For example, we may expand:
  \textquote{Sam is studying for an exam, but is taking a short nap.}

  By contrast, \ref{prog:abmig} under the progressive captures an event in where John studying is in progress.%
  \footnote{
    See,~\textcite{Richards:1981wo},~\textcite{Portner:2011vi}, etc.
  }
  For example, we may expand:
  \textquote{Sam is studying for an exam, so they aren't taking a nap.}
\end{note}


\subsection{Assumptions}
\label{sec:assumptions-1}


\begin{note}
  \begin{assumption}[Exclusivity]
    \label{assu:p:ex}
    \vspace{-\baselineskip}
    \begin{itenum}
    \item[\emph{If}:]
      It is not possible for \(d\) and \(d'\) to simultaneously be true of any event \(e\).
    \item[\emph{Then}:]
      \(e_{d}\) is in progress, it is not the case that \(e'_{d'}\) is in progress.
    \end{itenum}
    \vspace{-\baselineskip}
  \end{assumption}
\end{note}



\begin{note}
  \begin{assumption}[\assuPP{2}]%
    \label{assu:PP}%
    For any event \(e\) and action description \(\alpha\):
    \begin{itenum}
    \item[\emph{If}:]
      \begin{enumerate}[label=\alph*., ref=(\alph*)]
      \item
        \(e\) is an event in which \(\alpha\) is in progress.
      \end{enumerate}
    \item[\emph{Then}:]
      \begin{enumerate}[label=\alph*., ref=(\alph*), resume]
      \item
        There is some \progAdj{0} event \(e'\) such that~\ref{assu:PP:pe:dev} and~\ref{assu:PP:pe:verb} are both true:
        \begin{enumerate}[label=\roman*., ref=(\roman*)]
        \item
          \label{assu:PP:pe:dev}
          \(e'\) is a development of \(e\).
        \item
          \label{assu:PP:pe:verb}
          \(\alpha\) is true of \(e'\).
        \end{enumerate}
      \end{enumerate}
    \end{itenum}
    \vspace{-\baselineskip}
  \end{assumption}

  \assuPP{} has a key role in parts of the argument to follow.

  Question is, what it is for \(e'\) to be a development of \(e\).
  Do not offer an answer.
  Captures a constraint on \progAdj{0} event of interest.


  \assuPP{2} is a common feature of analyses of the progressive.%
  \footnote{
    See, e.g.:
    \cite{Bennett:1972uw},
    \cite{Dowty:1979vq},
    \cite{Parsons:1990aa},
    \cite{Landman:1992wh},
    \cite{Portner:1998um}.

    \assuPP{} is often motivated by the `imperfective paradox' (\cite[cf.][Ch.3.1]{Dowty:1979vq}).

    \citeauthor{Bach:1986tb} summarises:
    \begin{quote}
      [H]ow can we characterize the meaning of a progressive sentences like \ref{Bach:impP:17} on the basis of the meaning of a simple sentence like \ref{Bach:impP:18} when \ref{Bach:impP:17} can be true of a history without \ref{Bach:impP:18} ever being true?
      \begin{enumerate}[label=(\arabic*), ref=(\arabic*)]
        \setcounter{enumi}{16}
      \item
        \label{Bach:impP:17}
        John was crossing the street.
      \item
        \label{Bach:impP:18}
        John crossed the street.%
        \mbox{ }\hfill\mbox{(\citeyear[12]{Bach:1986tb})}
      \end{enumerate}
    \end{quote}
    % 
    The `paradox' amounts to two seemingly incompatible observations:
    \begin{enumerate}[noitemsep]
    \item
      \ref{Bach:impP:17} entains an event in which John is crosses the street is in progress.
    \item
      There need not be an (actual) event in which John is crosses the street.%
    \end{enumerate}
    % 
    For example, John may have been hit by a bus.
    In parallel, it may be true that you are falling asleep before a fire alarm is set off.
    The `solution' to this paradox is that the event in which the agent completes the action need not be an actual event.%\footnotemark

    Still,~\assuPP{0} is denied by some.
    For example, \citeauthor{Szabo:2004ul} argues:
    \textquote{Sometimes we are \emph{doing} things even though there is no real chance that we could get them \emph{done}, and this is true even if we abstract away from the possibility of miraculous intervention.}
    (\citeyear[40]{Szabo:2004ul})
    E.g., \citeauthor{Szabo:2004ul} denies~\ref{Szabo:Arch} is necessarily false:
    \begin{quote}
      \begin{enumerate}[label=(\arabic*), ref=(\arabic*)]
        \setcounter{enumi}{12}
      \item
        \label{Szabo:Arch}
        As the architect was building the cathedral he knew that, although he would be building it for another year or so, he couldn't possibly complete it.%
        \mbox{ }\hfill\mbox{(\citeyear[38]{Szabo:2004ul})}
      \end{enumerate}
    \end{quote}
    Though,~\ref{Szabo:Arch} seems always false to me.
    The only sense with which I read~\ref{Szabo:Arch} as true under the progressive requires factivity of knowledge to fail, thus allowing the cathedral to be built.

    See \cite[1245]{Portner:2011vi}) for additional, distinct, discussion of (\cite{Szabo:2004ul}).
  }
  % \footnotetext{
  % Note, however, possible does not necessarily entail `non-actual'.

  % Contrast \textquote{listening to a radio drama} and \textquote{listening to an entire radio drama}.
  % Suppose the radio drama accidentally stops halfway through.
  % The former does not require a non-actual possible event, as the event in which the agent was listening to a radio drama is an event in which the agent listens to a radio drama.
  % To listen does not require to listen until completion.
  % The latter, by contrast, requires a non-actual event, as there is no actual event in which the agent listens to the entire radio drama.
  % In parallel, concluding \(\pv{\phi}{v}\) from \(\Phi\) requires completion.
  % }
\end{note}

\begin{note}[Interest with the progressive]
  Still, we make a few observations to highlight some features.

  \begin{observation}[\assuPP{2} and existential modality]%
    \label{obs:prog-not-reg-poss}%
    The sense of possibility in \assuPP{} does not reduce to existential \{logical, metaphysical, nomic, \dots\} possibility.
  \end{observation}
  \begin{motivation}{obs:prog-not-reg-poss}
    Suppose John is sitting a multiple choice exam.
    To pass the exam John only needs to chose some number of correct choices.
    It is certainly logically, metaphysically, and nomically possible that John chooses a sufficient number of correct choices.
    However, it does not follow that John is passing the exam.%
    \footnote{
      See also Igal Kvart's example of Mary wiping out the Roman army (\cite[18]{Landman:1992wh}).
    }
  \end{motivation}

  \begin{observation}[\assuPP{2} and counterfactuals]%
    \label{obs:prog-not-cfs}%
    There is no simple relation between the sense of possibility in \assuPP{} and counterfactuals.
  \end{observation}
  \begin{motivation}{obs:prog-not-cfs}
    Suppose John is passing an exam without external help.
    Then, a classmate slips John some answers, which John then uses.
    It is no longer true that John is passing the exam without external help.
    And, in the closest possible world where the classmate does not slip John answers, it need not be true that John passes the exam without external help.
    For, if John is surrounded by students of a similar mindset then it is plausible that the in closest possible world a different classmate slips John the same answers.
  \end{motivation}

  \begin{observation}[\assuPP{} and uniqueness]%
    \label{obs:prog-no-unique}%
    The progressive may be true of an event without the event being sufficiently developed to `indicate' a unique outcome.
  \end{observation}
  \begin{motivation}{obs:prog-no-unique}
    Suppose John has drawn a straight line on a piece of paper.
    It may be true that John is drawing a triangle.
    However, the straight line is compatible with John drawing an \(n\)-sided polygon, for any \(n\) within a some reasonable bound.%
    \footnote{
      This observation is inspired by \citeauthor{Dowty:1979vq}'s example involving a circle and a triangle (\citeyear[133]{Dowty:1979vq}).
    }
  \end{motivation}

  \noindent%
  Loosely paraphrased, if an event is in progress, then there is something about the way things are which ensures the existence of a possible completion event (\autoref{obs:prog-no-unique}) which is robust against external influence (\autoref{obs:prog-not-cfs}) and does not require luck (\autoref{obs:prog-not-reg-poss}).
\end{note}


\section{\se{3}}

\begin{note}
  \begin{definition}[\se{3}]
    \label{assu:p:se}
    \vspace{-\baselineskip}
    \begin{itemize}
    \item
      \(e^{\flat}_{d^{\flat}}\) is a \emph{\se{0}} of \(e_{d}\).
    \end{itemize}
    \emph{If and only if}:
    \begin{itemize}
    \item
      Both \ref{assu:p:se:prog} and \ref{assu:p:se:hCon} hold:
    \begin{enumerate}[label=\arabic*., ref=(\arabic*)]
    \item
      \label{assu:p:se:prog}
      \(e^{\flat}_{d^{\flat}}\) is such that \(e_{d}\) is in progress.
    \item
      \label{assu:p:se:hCon}
      The following conditional is true:
      \begin{itenum}
      \item[\emph{If}:]
        \(e_{d}\) happens.
      \item[\emph{Then}:]
        \(e_{d}\) happens as a result of \(e^{\flat}_{d^{\flat}}\).
      \end{itenum}
    \end{enumerate}
  \end{itemize}
  \vspace{-\baselineskip}
  \end{definition}

  \noindent%
  The role of \autoref{assu:p:se} is to identify events such that an event is in progress and, in turn, the event happens as a result of the event in progress.

  Alternative, substitute `causes' in place of `happens as a result of'.%
  \footnote{
    For link between progressive and causation, \textcite{Szabo:2004ul}!

    \begin{quote}
      This captures the idea that in the course of events leading up to Mary’s being on the other side of the street there must be a point beyond which there is no more causally relevant external interference with the progress of Mary’s crossing.
    \end{quote}
  }
\end{note}


\begin{note}
  For example, working on an exam problem.
  Agent answers the exam problem correctly.
  Well versed in material.
  This gets that the event in which agent answers the exam problem correctly is in progress.
  However, agent is interrupted.
  Handed the answer on a piece of paper.
  Agent looks at the piece of paper and writes down the answer.

  Now, initial description, event in which the agent correctly answers is in progress.
  However, not the case that event in which agent correctly answers by aid of a piece of paper is in progress.

  So, look, when the agent is going by the understanding of the material, event in which the agent concludes is in progress.
  And, by \assuPP{} there is a possible event in which agent concludes.
  But, it is not \(e\).
\end{note}




\section{Progressive explanation}
\label{sec:what-these-do}

\begin{note}
  With \autoref{assu:p:ex} and \autoref{assu:p:se} in hand, an observation about events in progress an explanations about why an event happened --- given the sense of `why' present in \qWhy{} --- follows.

  \begin{observation}[Progressive explanation]%
    \label{obs:PE}%
    Given \(e_{d}\) is an event in which \vAgent{} does \(a\):

    \begin{itenum}
    \item[\emph{If}:]
      There is some \se{} \(e^{\flat}_{d^{\flat}}\) of \(e_{d}\) such that:
      Conditions \ref{obs:PE:prog} and \ref{obs:PE:ros}:
      \begin{enumerate}[label=\arabic*., ref=\arabic*]
      \item
        \label{obs:PE:prog}
        \(e^{\flat}_{d^{\flat}}\) is such that \(e_{d}\) is in progress.
      \item
        \label{obs:PE:ros}
        \(e^{\flat}_{d^{\flat}}\) is such that \(e_{d}\) is in progress \emph{only if} feature \(f\) of \(d_{\flat}\) holds throughout \(e^{\flat}_{d^{\flat}}\).
      \end{enumerate}
    \item[\emph{Then:}]
      Feature \(f\) explains `why' \(e_{d}\) happened, if the sense of `why' present in \qWhy{}.
    \end{itenum}
    \vspace{-\baselineskip}
  \end{observation}

  \begin{motivation}{obs:PE}
    The motivation for \autoref{obs:PE} follows from the way in which we understand the sense of `why' present in \qWhy{} and the way we understand \se{} such that an agent is concluding given \autoref{assu:p:se} and \autoref{assu:p:ex}.

    For, we understand the sense of `why' present in \qWhy{} as the did the event \(e_{d}\) in which an agent concludes \(\pv{\phi}{v}\) from \(\Phi\) happen, rather than an event in which the agent cleans the lint from their pocket, etc.

    Hence, if it is possible to point to something that favours \(e_{d}\) happening over any other event happening, then that thing explains `why' \(e_{d}\) happened, as opposed any other event.

    Now, consider \(e^{\flat}_{d^{\flat}}\) as a \se{} of \(e_{d}\).

    From \ref{obs:PE:prog}, why happened.
    For, \autoref{assu:p:se}, \(e_{d}\), and \(e^{\flat}_{d^{\flat}}\) is a \se{}.
    Therefore, \(e_{d}\) happened as a result of \(e^{\flat}_{d^{\flat}}\).

    In this respect, if there is something about \(e^{\flat}_{d^{\flat}}\) which favours \(e_{d}\), then `why'.
    For, result of something which favoured \(e_{d}\) over any other event.

    And, \autoref{assu:p:ex}, favours.

    If event happens as a result of event in progress, then the event explains `why' the event happened.

    Clause~\ref{obs:PE:ros} highlights a specific feature \(f\) of \(d_{\flat}\).
  \end{motivation}

  {
    \color{red}
    Recall card shuffle.
  }

  This is a simple observation.%
  \footnote{
    \phantlabel{mention:Hempel:1}
    Consider \autoref{obs:PE} with respect to \citeauthor{Hempel:1965aa}'s Deductive-Nomological account of scientific explanation:

    \begin{quote}
      [A Deductive-Nomological] explanation answers the question
      `\emph{Why} did the explanandum-phenomenon occur?'
      by showing that the phenomenon resulted from certain particular circumstances, specified in \(C_{1}, C_{2}, \dots C_{k}\), in accordance with the laws \(L_{1}, L_{2}, \dots L_{\gamma}\).
      By pointing this out, the argument shows that, given the particular circumstances and the laws in question, the occurrence of the phenomenon \emph{was to be expected}; and it is in this sense that the explanation enables us to \emph{understand why} the phenomenon occurred.%
      \mbox{ }\hfill\mbox{(\citeyear[337]{Hempel:1965aa})}
    \end{quote}

    Something about circumstances and laws, \emph{expected}.
  }
  If we want to understand why, then anything that favours over any other event, that's enough.
  What it is for an event to be in progress is the narrow \(e\) from range of possible events.
\end{note}



\subsection{Ex}

\begin{note}
  \begin{illustration}[Darts]
    Agent wins at darts just in case there is some action available to the agent, such that if the agent were to perform the action they would be winning at darts.

    Winning is a complex action.
    An agent has three dart throws to lower their score from 501 to 0 before play switches to the other player, and play continues until neither player may lower their score further on their next turn (without going past 0).
    Playing a game is a complex action, as the region of a dartboard an agent wishes to hit changes according to previous throws.
    For example, if the agent's score is 51 with three throws remaining, the agent will not wish to hit bullseye, as there is no way to reduce their score by a single point using two darts.
    If the agent goes on to hit 20, then the score of the remaining to darts should equal 31, and so on.
  \end{illustration}

  Note, the initial sequence of actions may be more or less arbitrary.
  It is not possible to score 501 in three or six dart throws, so an agent \emph{could} start by throwing a few darts blindly, so long as they have sufficient skill to recover on subsequent throws.

  Of course, throwing darts is quite different from concluding, but this note extends.
  An agent may be concluding a theorem is true even though their first line of enquiry turns out to be a dead end, etc.

  This is the appeal of the progressive.
\end{note}





\section*{Summary}
\label{sec:summary}


\begin{note}
  \begin{itemize}
  \item
    An event in which an agent is concluding is an event in which an event in which the agent concludes is in progress.

    By \assuPP{} (\autoref{assu:PP}), if an agent is concluding there is a possible event in which the agent concludes (\autoref{prp:peventC}).

  \end{itemize}
\end{note}



%%% Local Variables:
%%% mode: latex
%%% TeX-master: "master"
%%% TeX-engine: luatex
%%% End:
