\chapter{\qWhyV{} and \qHowV{}}
\label{cha:var}

\begin{note}
  This chapter states variations on \qWhy{}, \qHow{}, and \issueInclusion{}.
  The role of the variations is to establish a (sufficiently) precise way in which a relation may answer \qWhy{} without a corresponding answer to \qHow{}.

  Variants, rather than substitutes.
  \qWhyV{} as variant to \qWhy{} and \qHowV{} and variant to \qHow{}.
  Variants are connected to initial questions by linking conditionals.

  \begin{TOCEnum}
  \item
    \TOCLine{cha:var:qwhyvnp}

    Answers to \qWhyV{} are answers to \qWhy{}.
  \item
    \TOCLine{cha:var:qhowv}

    Answers to \qHow{} are answers to \qHowV{}.
  \item
    \TOCLine{cha:var:issue}

    If links hold, and \issueConstraint{} holds, then this entails variant constraint.
  Provide a recepie for generating counterexamples and then either links fail, or \issueConstraint{}.
  \end{TOCEnum}
\end{note}

\begin{note}
  Ensuring links hold is important.
  For, \issueInclusion{} is plausibly doing work to narrow the relevant sense of why captured by \qWhy{}.

  In other words, there may be plausible answers to \qWhy{} which are not also answers to \qHow{} but fail to be counterexamples to \issueInclusion{} as the relevant sense of `why' and `how' are not the senses of `why' and `how' that \issueInclusion{} (intuitively) holds with respect to.

  And, as a constraint on answers to \qWhy{} in terms of \qHow{}, \issueInclusion{} may plausibly have a role in narrowing the relevant senses of `why' and `how' at issue.
  And, if \issueInclusion{} is rejected, then, \issueInclusion{} cannot perform this role.
\end{note}

\section{\qWhyV{}}
\label{cha:var:qwhyvnp}

\begin{note}
  \qWhyV{} as variant of \qWhy{}.

  The difference between \qWhyV{} and \qWhy{} is that \qWhyV{} substitutes the truth of a conditional for  the instance of `why' in \qWhy{}.

  Question.

  Illustrations.

  Link.

  Observations.
\end{note}

\subsection{The question}
\label{cha:var:qwhyvnp:question}

\begin{note}
  \qWhyV{} is as follows:%
  \footnote{
    When speaking abstractly, \qWhyV{} always applies to \(\phi\), \(v\), and \(\Phi\).
  }

  \begin{question}{questionWhyV}{\qWhyV{}}%
    Given \(e\) is an event in which \vAgent{} concludes \(\pv{\phi}{v}\) from \(\Phi\):

    \begin{itemize}
    \item
      Which proposition-value-\pool{} pairs \(\pvp{\psi}{v'}{\Psi}\) are such that:
      \begin{itemize}
      \item
        For some sub-event \(e^{\flat}\) of \(e\):
        \begin{itenum}
        \item[\emph{If}:]
          A \ros{0} between \(\pv{\psi}{v'}\) and \(\Psi\) fails to hold from \agpe{\vAgent{}'} through \(e^{\flat}\).
        \item[\emph{Then}:]
          \(e^{\flat}\) is not, or does not develop into, an event in which \vAgent{} concludes \(\pv{\phi}{v}\) from \(\Phi\).
        \end{itenum}
      \end{itemize}
    \end{itemize}
    \vspace{-1.5\baselineskip}
  \end{question}

  \noindent%
  In short, \qWhyV{} seeks \ros{} without which the event in which the agent concluded \(\pv{\phi}{v}\) from \(\Phi\) did not happen.

  The \itc{} of \qWhyV{} is a material conditional,%
  \footnote{
    Hence, the \itc{} is true if and only if: \emph{Either} a \ros{0} between \(\pv{\psi}{v'}\) and \(\Psi\) holds for \vAgent{} at \(e'\) \emph{or} \(e'\) is not, or does not develop into, an event in which \vAgent{} concludes \(\pv{\phi}{v}\) from \(\Phi\).
  }
  and builds on two ideas regarding events:
  \begin{enumerate}
  \item
    An event may or may not develop into some other event.
  \item
    Something about an event may influence which other events it may develop into.
  \end{enumerate}
  %
  To illustrate, consider the following \scen{0}:

  \begin{scenario}[Coin flip]%
    \label{illu:coinT}%
    Agent \(B\) has accepted gamble on coin toss from Agent \(A\).
    The gamble is captured by the following pair of conditionals:
    %
    \begin{itemize}
    \item
      If the coin lands heads, Agent \(B\) gives \texteuro{}5 to Agent \(A\).
    \item
      If the coin lands tails, Agent \(A\) gives \texteuro{}5 to Agent \(B\).
    \end{itemize}
    %
    Agent \(A\) has tossed the coin, and the result is hidden between Agent \(A\)'s hands.
  \end{scenario}

  \noindent%
  Intuitively, the gamble of \autoref{illu:coinT} influences any development of \autoref{illu:coinT}.
  In particular, a pair of conditional follow from the gamble:
  %
  \begin{itemize}
  \item
    If the coin has landed heads, then agent \(B\) will soon be \texteuro{}5 poorer.
  \item
    If the coin has landed tails, then agent \(A\) will soon be \texteuro{}5 poorer.
  \end{itemize}
  %
  Likewise, given the gamble, it is not the case that Agent \(A\) notices the coin has landed heads and receives \texteuro{}5 from Agent \(B\).%
  \footnote{
    \label{fn:desc-con-caveat}
    Note the distinction between our description of the event and the event.
    It is consistent with our description of the event that Agent \(A\) is dishonest and will run before handing anything to \(B\), or that neither agent really has \texteuro{}5 at hand and are hoping they win, or \dots
    However, at issue is the event.
  }

  The \itc{} of \qWhyV{} parallels to the way the observed conditionals hold of the event as given in~\autoref{illu:coinT}.

  For, consider an event in which an agent concludes \(\pv{\phi}{v}\) from \(\Phi\).
  This event may be broken down into various sub-events, and for any sub-event may or may not develop into the event in which the agent concludes \(\pv{\phi}{v}\) from \(\Phi\).
  As \qWhyV{} is stated with respect to an event in which the concludes \(\pv{\phi}{v}\) from \(\Phi\), it follows that sub-event may develop into an event in which the concludes \(\pv{\phi}{v}\) from \(\Phi\).
  However, something about an sub-event of \(e\) may have required a \ros{} to hold between \(\pv{\psi}{v'}\) and \(\Psi\) for \(e\) to have developed into an event in which the agent has concludes \(\pv{\phi}{v}\) from \(\Phi\).
  Though this idea is naturally expressed using a subjunctive conditional, the \itc{} is a material conditional and is evaluated by considering the sub-event, in parallel to the gamble of~\autoref{illu:coinT}.
\end{note}


\subsection{A link between \qWhyV{} and \qWhy{}}
\label{cha:var:qwhyvnp:link}

\begin{note}
  \begin{link}[\qWhyV{} and \qWhy{}]%
    \label{link:why:support:pvpp}%
    \vspace{-\baselineskip}
    \begin{itenum}
    \item[\emph{If}:]
      \(\pvp{\psi}{v'}{\Psi}\) is an answer to \qWhyV{}.
    \item[\emph{Then}:]
      \(\pvp{\psi}{v'}{\Psi}\) is an answer to \qWhy{}.
    \end{itenum}
    \vspace{-\baselineskip}
  \end{link}

  \noindent%
  Expanded, \qWhyV{} considers whether an sub-event does not develop into an event in which the agent concludes if a \ros{} failed to hold.
  And, answers to \qWhyV{} are proposition-value-\pool{} pairs such that a \ros{} failing to hold is sufficient to ensure the sub-event does not develop into an event in which the agent concludes.
  In this respect the \ros{} answers, in part, why an agent concluded \(\pv{\phi}{v}\) from \(\Phi\).
  For, without the \ros{} some other event happened.
\end{note}

\begin{note}
  Whether or not \autoref{link:why:support:pvpp} holds depends on whether why an event happens is explained by the presence of something whose absence leads to an different event happening.

  Consider \citeauthor{Hieronymi:2011aa}'s example of the extreme heat or the faulty construction as an explanatory reasons for why the engine failed.
  (\citeyear[409]{Hieronymi:2011aa}, mentioned on \autopageref{HierExpR})

  Absence of extreme heat leads to a different event happening.
  The engine may still fail, but the engine fails in a different way --- for example as a result of faulty construction.
  Likewise, absence of faulty construction leads to a different event happening (though the engine may still fail as a result of extreme heat).
\end{note}

\begin{note}
  \phantlabel{mention:Hempel:1}
  From a broader perspective, consider \citeauthor{Hempel:1965aa}'s Deductive-Nomological account of scientific explanation:

  \begin{quote}
    [A Deductive-Nomological] explanation answers the question
    `\emph{Why} did the explanandum-phenomenon occur?'
    by showing that the phenomenon resulted from certain particular circumstances, specified in \(C_{1}, C_{2}, \dots C_{k}\), in accordance with the laws \(L_{1}, L_{2}, \dots L_{\gamma}\).
    By pointing this out, the argument shows that, given the particular circumstances and the laws in question, the occurrence of the phenomenon \emph{was to be expected}; and it is in this sense that the explanation enables us to \emph{understand why} the phenomenon occurred.%
    \mbox{ }\hfill\mbox{(\citeyear[337]{Hempel:1965aa})}
  \end{quote}
  %
  Presence of something whose absence leads to an different event happening shows, in part, the event was to be expected, rather than the different event.

  Further, the \itc{} of \qWhyV{} intuitively functions as a a lawlike constraint, as it applies regardless of whether or not the agent goes on to conclude \(\pv{\phi}{v}\) from \(\Phi\).%
  \footnote{
    \nocite{Tichy:1976tp}%
    In the literature on subjunctive conditionals, the conditionals which constrain the development of events are sometimes termed `laws' (\cite{Chisholm:1955aa,Lewis:1979vm,Veltman:2005tj}) or thing that `lump together' certain facts (\cite{Kratzer:1981aa,Kratzer:1989aa}).
  }
\end{note}

\begin{note}
  In place of additional abstract motivation for~\autoref{link:why:support:pvpp}, we present an analysis of a \scen{1}.

  \begin{scenario}[Countersign]
    \label{scen:countS}
    \indent The captain mumbled, ``I come from Miran.''

    The man returned the gambit, grimly.
    ``Miran is early this year.''

    The captain said, ``No earlier than last year.''

    But the man did not step aside.
    He said, ``Who are you?''

    ``Aren't you Fox?''

    ``Do you always answer by asking?''

    The captain took an imperceptibly longer breath, and then said calmly,
    ``I am Han Pritcher, Captain of the Fleet, and member of the Democratic Underground Party.
    Will you let me in?''%
    \mbox{ }\hfill\mbox{(\cite[70]{Asimov:1945aa})}%
    \newline
  \end{scenario}

  \noindent%
  In the event described by \autoref{scen:countS} Fox concludes \propI{The person Fox is talking to is a fellow member of the Democratic Underground Party} is \valI{True}.
  Pritcher calmly tells Fox the are a fellow member of the party at the end of the \scen{}.
  Still, Fox has already drawn the conclusion when Fox asks `Who are you?' via the sequence of countersigning.
  And, intuitively, Pritcher saying, No earlier than last year' in response to Fox saying `Miran is early this year' explains why the event develop in an event in which Fox concludes Pritcher is a fellow member of the party.

  Consider the sub-event \(e^{flat}\) from Pritcher's response to `Miran is early this year' with `No earlier than last year' and Fox asking `Who are you?'.
  The following conditional holds, from \agpe{Fox's}:
  %
  \begin{itenum}
  \item[\emph{If}:]
    They didn't appropriately respond to `Miran is early this year'.
  \item[\emph{Then}:]
    They're not engaging in countersign.
  \end{itenum}
  %
  And, if Fox doesn't think Pritcher is engaging in countersign, it seems clear Fox does not conclude Pritcher a fellow member of the party.
  Hence, phrased in terms of \ros{}, we have:
  \begin{itenum}
  \item[\emph{If}:]
    A \ros{} between \pv{\propI{What they said was an appropriate response}}{\valI{True}} and some \pool{} \(\Psi\) fails to hold from \agpe{Fox's} through \(e^{\flat}\).
  \item[\emph{Then}:]
    \(e^{\flat}\) is not, or does not develop into an event in which Fox concludes \propI{The person Fox is talking to is a fellow member of the Democratic Underground Party} is \valI{True} from some \pool{} \(\Phi\).
  \end{itenum}
  In this respect, \ros{} answers \qWhy{}.
\end{note}

\subsection*{A proposition and an observation}

\begin{note}
  \begin{proposition}[A guaranteed answer to \qWhyV{}]%
    \label{prop:ros-always-answer}%
    A \ros{} between \(\pv{\phi}{v}\) from \(\Phi\) is always an answer to \qWhyV{}.
  \end{proposition}

  \begin{argument}{prop:ros-always-answer}
    Suppose \(e\) is an event in which an agent concludes \(\pv{\phi}{v}\) from \(\Phi\).
    We need to show there is some sub-event \(e^{\flat}\) of \(e\) such that:
    %
    \begin{itenum}
    \item[\emph{If}:]
      A \ros{0} between \(\pv{\phi}{v}\) and \(\Phi\) fails to hold from \agpe{\vAgent{}'} through \(e^{\flat}\).
    \item[\emph{Then}:]
      \(e^{\flat}\) is not, or does not develop into, an event in which \vAgent{} concludes \(\pv{\phi}{v}\) from \(\Phi\).
    \end{itenum}
    %
    And, by \supportI{}, a \ros{} holds between \(\pv{\phi}{v}\) and \(\Phi\), for the agent, when the agent pairs \(\phi\) with \(v\) as a sub-event of an event in which the agent concludes \(\pv{\phi}{v}\) from \(\Phi\).
    So it is immediate that if a \ros{} between \(\pv{\phi}{v}\) and \(\Phi\) fails to hold when the agent concludes, the agent does not conclude \(\pv{\phi}{v}\) from \(\Phi\).
  \end{argument}
\end{note}

\begin{note}
  \begin{observation}%
    \label{obs:qWhyV:rosNNecAns}%
    A \ros{} between \(\pv{\psi}{v'}\) and \(\Psi\) may answer \qWhyV{} though the \ros{} is unrelated to why \(\pv{\phi}{v}\) follows from \(\Phi\) from the \agpe{\agents{}}.
  \end{observation}

  \begin{motivation}{obs:qWhyV:rosNNecAns}
    Consider a \scen{0} in which an agent concludes \(\pv{\phi}{v}\) from \(\Phi\) where partway through the agent concludes to brew a cup of tea.

    It may well be the case that the agent fails to conclude \(\pv{\phi}{v}\) without the aid of the tea.
    For, without a warm cup of reassurance, the agent gives up.
    and, if so, a \ros{} between \pv{\propI{Brew a cup of tea!}}{\valI{True}} and some \pool{} \(\Psi\) answers \qWhyV{}.

    Still, from the \agpe{\agents{}}, the cup of tea was only a nice thing to have.
  \end{motivation}

  \noindent%
  Intuitively it is not be the case that the \ros{} and the conclusion to get some tea answers why \emph{the agent} concluded some theorem is true from some \pool{}.%
    \footnote{
      \citeauthor{Armstrong:1968vh} (\citeyear[195--196]{Armstrong:1968vh}) discusses a similar example, and suggests further discussion of this issue may also be found in \textcite{Moore:1962up}.
      See also \citeauthor{Sanford:1989aa} (\citeyear{Sanford:1989aa}) for a discussion of dependence as captured by subjunctive conditions (esp.\ pp.\ 192--193).
    }

    Still, \qWhy{} and \qWhyV{} are not stated from the \agpe{\agents{}}.
    Rather, both \qWhy{} and \qWhyV{} concern the way in which an event in which an agent concludes develops, and various \ros{} which hold during an event when an agent concludes may answer this.

    Note, however, for a \ros{} to answer \qWhyV{} it must be the case that
    \begin{enumerate*}[label=(\alph*), ref=(\alph*)]
    \item
      the \ros{} holds throughout some sub-event of the event in which the agent concludes, and
    \item
      absence of the \ros{} for the sub-event leads to an event in which the agent does not conclude \(\pv{\phi}{v}\) from \(\Phi\).
    \end{enumerate*}

    Hence, \qWhy{} is somewhat stringent.
    For example, a \ros{} between \pv{\propI{Brew a cup of coffee!}}{\valI{True}} and some \pool{} \(\Psi'\) does not answer \qWhyV{} as the agent concluded to get tea, rather than coffee (and even if the choice between tea and coffee was a figurative coin flip).
    And, the \ros{} between \pv{\propI{Brew a cup of tea!}}{\valI{True}} and \(\Psi\) fails to answer \qWhyV{} if the agent may have go on to conclude \(\pv{\phi}{v}\) from \(\Phi\) without the aid of the cup of tea.
\end{note}


\section{\qHowV{}}
\label{cha:var:qhowv}

\begin{note}
  \qHow{} is a broad question which asks, quite generally, for an account of how an agent concluded \(\pv{\phi}{v}\) in terms of what has happened.
\end{note}

\subsection{The question}
\label{cha:var:qhowv:question}

\begin{note}
  The variant of \qHow{} is as follows:

  \begin{question}{questionHowV}{\qHowV{}}%
    \label{q:how:v}%
    Given \(e\) is an event which \vAgent{} concludes \(\pv{\phi}{v}\) from \(\Phi\):
    \begin{itemize}
    \item
      Which events \(e^{\sharp}\) are such that:
      \begin{itemize}
      \item
        For some \(\pvp{\psi}{v'}{\Psi}\) such that a \ros{} between \(\pv{\psi}{v'}\) and \(\Psi\) held during or prior to some sub-event \(e^{\flat}\) of \(e\):
        \begin{itemize}
        \item
          \(e^{\sharp}\) is a \wit{0} for a \ros{} between \(\pv{\psi}{v'}\) and \(\Psi\).
        \end{itemize}
      \end{itemize}
    \end{itemize}
    \vspace{-\baselineskip}
  \end{question}

  Event which \wit{} a \ros{} while held when \vAgent{} concluded \(\pv{\phi}{v}\) from \(\Phi\), or held prior to \vAgent{}' conclusion.
\end{note}

\begin{note}
  \qHowV{} is graceless.
  Which events are such that agent concluded some proposition has some value from some \pool{} prior to conclusion of \(\pv{\phi}{v}\) from \(\Phi\).
  In contrast to \qHow{}, there is nothing that directly connected answers to \qHowV{} to the event in which the agent concludes \(\pv{\phi}{v}\) from \(\Phi\).

  There is nothing which prevents refinement.
  \ros{} is an answer to \qWhyV{}, for example.

  However, little interest.
  Role of \qHowV{} is to constrain answers to \qWhyV{}.
  Does this in the way that matters.
  \supportII{}, no need for \wit{}.
  If constrains, then need \wit{}.
\end{note}

\begin{note}
  So, have \wit{}?

  \qHow{} seeks events which explain how.
  \qHowV{} narrows attention to \ros{} and seeks events which \wit{} relevant \ros{1}.

  Key here is that \supportII{}, need not be the case that \wit{}.
\end{note}

\begin{note}
  \begin{proposition}[A guaranteed answer to \qHowV{}]%
    \label{prop:phi-always-how}%
    Any prior or present event \(e\) in which \vAgent{} concludes \(\pv{\phi}{v}\) from \(\Phi\) is an answer to \qHowV{}.
  \end{proposition}

  \begin{argument}{prop:phi-always-how}
    Let \(e\) be an event in which an agent concludes \(\pv{\phi}{v}\) from \(\Phi\).

    Then, immediate by \autoref{def:witnessing}.
  \end{argument}
\end{note}

\subsection{A link between \qHow{} and \qHowV{}}
\label{cha:var:qhowv:sec:link}

\begin{note}
  The link between \qHowV{} and \qHow{} is more straightforward than the link between \qWhyV{} and \qWhy{}.
  In short, \qHowV{} captures a particular sub-collection of answers to \qHow{}:

  \begin{link}[\qHowV{} and \qHow{}]%
    \label{link:how-witnessing}%
    \vspace{-\baselineskip}
      \begin{itenum}
      \item[\emph{If}:]
        Both \ref{link:how-witnessing:a:1} and \ref{link:how-witnessing:a:2} are true:
        \begin{enumerate}[label=\alph*., ref=(\alph*)]
        \item
          \label{link:how-witnessing:a:1}
        \(e\) answers \qHow{}.
      \item
        \label{link:how-witnessing:a:2}
        \(e\) is an event in which an agent concludes a \prop{} has a \val{}.
        \end{enumerate}
      \item[\emph{Then}:]
        \(e\) answers \qHowV{}.
      \end{itenum}
    \vspace{-\baselineskip}
  \end{link}

  Notice, with the link between \qWhyV{} and \qWhy{}, answers to \qWhyV{} sufficient for answers to \qWhy{}.
  With \autoref{link:how-witnessing}, answers to \qHow{} are sufficient for answers to \qHowV{}.

  Indeed, motivation is entailment.
  Restriction so that \(e\) is an event in which an agent concludes a \prop{} has a \val{}.
  Hence, answer to \qHowV{}.
\end{note}

\section{\issueConstraint{}}
\label{cha:var:issue}

\begin{note}
  \autoref{cha:var:qwhyvnp} developed \qWhyV{}, a variation of \qWhy{}.
  \autoref{cha:var:qhowv} developed \qHowV{}, a variation of \qHow{}.
  In both sections we established links between the variant --- \qWhyV{}, \qHowV{}  --- and initial --- \qWhy{}, \qHow{} --- questions.

  Develop \issueConstraint{}, a variation on \issueInclusion{}.
\end{note}

\begin{note}
 We being with a proposition:

  \begin{proposition}[\qWhyV{}-\qWhy{}-\qHow{}-\qHowV{}]%
    \label{prop:support-and-witnessing}%
      \linkW{}, \linkH{}, and \issueInclusion{} (jointly) entail the following conditional:
      \begin{itenum}
      \item[\emph{If}:]
      \(\pvp{\psi}{v'}{\Psi}\) answers \qWhyV{}.
      \item[\emph{Then}:]
        An event \(e\) which \wit[es]{0} a \ros{} between \(\pv{\psi}{v'}\) and \(\Psi\) answers \qHowV{}.
      \end{itenum}
    % \end{itemize}
    \vspace{-\baselineskip}
  \end{proposition}

  \begin{argument}{prop:support-and-witnessing}
    Suppose \qWhyV{} is answered, in part, by a \ros{} between \(\pv{\psi}{v'}\) and \(\Psi\).
    From~\linkW{} it is immediately follows that \(\pvp{\psi}{v'}{\Psi}\) answers, in part, \qWhy{}.
    And, given \issueInclusion{}, \(\pvp{\psi}{v'}{\Psi}\) answers, in part, \qHow{}.

    Further, \(\pvp{\psi}{v'}{\Psi}\) answers, in part, \qHow{} due to a \ros{} between \(\pv{\phi}{v}\) and \(\Psi\) being, in part, an answer to \qWhyV{}.
    Therefore, by \linkH{}, there is some \(e\) such that \(e\) is a \wit{0} for the \ros{} between \(\pv{\psi}{v'}\) and \(\Psi\).
  \end{argument}
\end{note}


\begin{note}
  As~\autoref{prop:support-and-witnessing} follows from \linkW{}, \linkH{}, and \issueInclusion{}, we consider the content of~\autoref{prop:support-and-witnessing} to be a parallel constraint to \issueInclusion{}:

  \begin{constraint}{consConstraint}{\issueConstraint{}}
    % \cenLine{
    %   \begin{VAREnum}
    %   \item
    %     Agent: \vAgent{}
    %   \item
    %     Propositions: \(\phi\), \(\psi\)
    %   \item
    %     Values: \(v\), \(v'\)
    %   \item
    %     \pool{3}: \(\Phi\), \(\Psi\)
    %   \item
    %     \mbox{ }
    %   \end{VAREnum}
    % }
    % \medskip
    % \begin{itemize}
    % \item
    \vspace{-\baselineskip}
    \begin{itenum}
    \item[\emph{If}:]
      \(\pvp{\psi}{v'}{\Psi}\) answers \qWhyV{}.
    \item[\emph{Then}:]
      \vAgent{} has \wit[ed]{} a \ros{} between \(\pv{\psi}{v'}\) and \(\Psi\).
    \end{itenum}
    % \end{itemize}
    \vspace{-\baselineskip}
  \end{constraint}

  As with \issueInclusion{}, if \issueConstraint{} holds, then answers to \qWhyV{} are constrained by answers to \qHowV{}.%
  \footnote{
    \phantlabel{fn:past-witness}
    To illustrate, consider an agent working on some mathematical problem.

    As part of their work on the problem the agent concludes the hypotenuse of some right-angled triangle is \(\sqrt{74}\text{cm}\) by use of the Pythagorean theorem.
    Further, the agent has, at some point in the past proved the Pythagorean theorem from more basic principles.

    Perhaps the agent concludes the hypotenuse of the triangle is \(\sqrt{74}\text{cm}\), in part, from those more basic principles.
    Perhaps in general it is true that if a agent concluded \(X\) from \(Y\) and \(Y\) from \(Z\), then the agent, in part at least, concluded \(X\) from \(Z\).
    On the other hand perhaps the more basic principles have no role explanatory role in the present.
    The agent only appealed to the theorem, rather than any basic principles.

    Though we will not take a stand on whether a relevant \wit{0} for some conclusion is distinct from the event in which the agent concludes, the first option highlights an issue with \autoref{def:witnessing}.
    Consider an agent working through a proof of some theorem \(\theta\), and let \(\Theta\) be the relevant \pool{}.
    Our interest is with the conclusion \(\pv{\theta}{\valI{True}}\) from \(\Theta\).

    Suppose the agent reasons to \(\pv{\theta}{\valI{True}}\) from \(\Theta\).
    Further, suppose the \agents{} reasoning is sound.
    However, the agent is worried about some parts of their reasoning.
    Hence, given their worries, \emph{reasons} to --- but does not conclude --- \(\pv{\theta}{\valI{True}}\) from \(\Theta\).
    Some time later the agent resolves their worries and concludes the theorem is true.
    I see no issue with the possibility that:
    %
    \begin{itemize}[noitemsep]
    \item
      When the agent revisited the proof, they concluded \(\pv{\theta}{\valI{True}}\) from \(\Theta\).
    \item
      In part, a \ros{} between \(\pv{\theta}{\valI{True}}\) and \(\Theta\), from the \agpe{}, answers why the agent concluded \(\pv{\theta}{\valI{True}}\) from \(\Theta\).
    \item
      The event in which the agent reasoned to \(\pv{\theta}{\valI{True}}\) from \(\Theta\) answers, in part, how the agent \(\pv{\theta}{\valI{True}}\) from \(\Theta\) by being a \wit{} for the \ros{} between \(\pv{\theta}{\valI{True}}\) and \(\Theta\).
    \end{itemize}
    %
    However, the idea is incompatible with the way a \wit{0} is understood.
    For, by the definition, an event which is a \wit{0} must be an event in which the agent \emph{concludes}.
    And, by construction of the \scen{0}, the \agents{} worries prevent the agent from forming the relevant conclusion.

    Perhaps \autoref{def:witnessing} should be revised so that an event \(e\) may be \wit{} so long as the agent adequately reasons (an optionally concludes).
    % However, providing an adequate characterisation of the relevant event is difficult.
    % That the agent \emph{reasoned} to \(\pv{\theta}{\valI{True}}\) from \(\Theta\) is insufficient.
    % 
    % For example, consider a variation of the \scen{} in which the agent identifies a problem with the proof.
    % Given the presence of a problem, there may be no \ros{} for the agent to have a \wit{0} for.
    Still,
    % rather than attempt to characterise \ros{1} independently of conclusions,
    we keep the present definition of a \wit{} for simplicity.
  }

  Our direct goal is to develops counterexamples to \issueConstraint{}.
  For, if there are counterexamples to \issueConstraint{}, then it immediately follows by \autoref{prop:support-and-witnessing} that either \linkW{}, \linkH{}, or \issueInclusion{} fails to hold.

  We defended \linkW{} when developing \qWhyV{} in~\autoref{cha:var:qwhyvnp}.
  And, likewise, we defended \linkH{} when developing \qHowV{} in~\autoref{cha:var:qhowv}.
  And, though there are some difficulties with \qWhyV{}, \qHowV{}, \linkW{}, and \linkH{}, I consider the most plausible point of failure to be \issueInclusion{}.
\end{note}

\section{\issueConstraint{} and \fc{1}}

\begin{note}
  To clarify the importance of \autoref{prop:fcs-only-if-pot-support}:

  \begin{proposition}%
    \label{prop:fc-wit}%
    \vspace{-\baselineskip}
    \begin{itenum}
    \item[\emph{If}:]
      Both~\ref{prop:fc-wit:fc} and~\ref{prop:fc-wit:noC} hold:
      \begin{enumerate}[label=\alph*., ref=(\alph*)]
      \item
        \label{prop:fc-wit:fc}
        \(\pv{\psi}{v'}\) is a \fc{0} from \(\Psi\), for \vAgent{}
      \item
        \label{prop:fc-wit:noC}
        \vAgent{} has not concluded \(\pv{\psi}{v'}\) from \(\Psi\).
      \end{enumerate}
    \item[\emph{Then}:]
      Both~\ref{prop:fc-wit:ros} and~\ref{prop:fc-wit:noW} hold:
      \begin{enumerate}[label=\alph*\('\)., ref=(\alph*\('\))]
      \item
        \label{prop:fc-wit:ros}
        A \ros{} between \(\pv{\psi}{v'}\) and \(\Psi\), for \vAgent{}.
      \item
        \label{prop:fc-wit:noW}
        \vAgent{} doesn't have a \wit{} for the \ros{} between \(\pv{\psi}{v'}\) and \(\Psi\).
      \end{enumerate}
    \end{itenum}
    \vspace{-\baselineskip}
  \end{proposition}

  \begin{argument}{prop:fc-wit}
    \ref{prop:fc-wit:fc} entails \ref{prop:fc-wit:ros} by \autoref{prop:fcs-only-if-pot-support}.

    \noindent \ref{prop:fc-wit:noC} entails \ref{prop:fc-wit:noW} by the definition of a \wit{} (\witpage{}).
  \end{argument}

  For, in order to argue against \issueConstraint{}, need some \(\pvp{\psi}{v'}{\Psi}\) such that answers \qWhyV{}.
  Consider \autoref{scen:calc}.
  \fc{}.
  However, intuitively does not answer \qWhy{} nor \qWhyV{}.
\end{note}

\section*{Summary}

\begin{note}
  Overall argument.
  Links then answer to \qWhyV{} which is not constrained by \qHowV{}, then \issueInclusion{} fails.

  Three broad ways in which the overall argument may fail:
  \begin{enumerate}[label=\arabic*., ref=(\arabic*), noitemsep]
  \item
    The link between \qWhyV{} and \qWhy{} fails to hold.
  \item
    The link between \qHowV{} and \qHow{} fails to hold.
  \item
    We fail to develop counterexamples to \issueConstraint{}.
  \end{enumerate}

  Still, I hope to have developed \qWhyV{}, \qHowV{}, and \issueConstraint{} in such a way that both questions are of some interest independent of link.
\end{note}


% \section{The role of variant questions}
% \label{cha:var:sec:wiggling}

% \begin{note}
%   \autoref{cha:introduction} introduced \qWhy{}, \qHow{}, and \issueInclusion{}.


%   The overall goal of this document is to argue \issueInclusion{} does not hold.

%   Hence, without establishing a clear understanding of the way in which the instances are to be understood, it is unclear how to develop counterexamples to \issueInclusion{}.
% \end{note}

% \begin{note}
%   In broad outline, we use the idea of a `\ros{0}' to provide variations on \qWhy{}, \qHow{}, and \issueInclusion{}.
%   The way in which we understand \ros{1} is minimal and tightly connected to an event in which an agent concludes \(\pv{\phi}{v}\) from \(\Phi\).
%   Specifically, we put forward three ideas in relation to \ros{1}:
%   \begin{enumerate}
%   \item
%     If an agent concludes \(\pv{\phi}{v}\) from \(\Phi\), then a \ros{0} between \(\pv{\phi}{v}\) and \(\Phi\), for the agent, when the agent pairs \(\phi\) with \(v\).
%   \item
%     If an agent has concluded \(\pv{\phi}{v}\) from \(\Phi\), then the event in which the agent concluded \(\pv{\phi}{v}\) from \(\Phi\) functions as a \wit{0} for a \ros{0} between \(\pv{\phi}{v}\) and \(\Phi\).
%   \item
%     It is possible for a \ros{0} between \(\pv{\phi}{v}\) and \(\Phi\) to hold, from an \agpe{}, without their being an \wit{0} for the \ros{0}.
%   \end{enumerate}

%   \autoref{cha:var:ros} will develop and discuss each idea in detail.

%   For the moment, the motivation for abstracting to \ros{1} is to capture, in a abstract way, the way in which \(\pv{\phi}{v}\) and \(\Phi\) are related from an \agpe{} when the agent concludes \(\pv{\phi}{v}\) from \(\Phi\).

%   Rather than directly capturing some relevant sense of `why' or `how' our goal is to use \ros{1} to construct variations on \qWhy{} and \qHow{} which are \emph{roughly} `extensionally adequate'.
%   Where, we understand the term `extensionally adequate' more-or-less in line with \citeauthor{Sumner:1987aa} (\citeyear{Sumner:1987aa}):

%   \begin{quote}
%     [A] conception of a concept is extensionally adequate when it includes every item which seems pre-analytically to be an instance of the concept and excludes every item which does not.%
%     \mbox{ }\hfill\mbox{(\citeyear[49]{Sumner:1987aa})}
%   \end{quote}

%   Adapted to our case, our interest with \qWhy{} and \qHow{} is with respect to intuitive answers to \qWhy{} and \qHow{} (and in particular the intuition that \issueInclusion{} holds).
%   And, the variations of both \qWhy{} and \qHow{} may be seen as `conceptions of a question' such that any answer to \qWhy{} is an answer to the variation of \qWhy{}, vice-versa, and the same with respect to \qHow{}.

%   Still, the way in which something answers \qWhy{} need not be equivalent to the way in which that thing answers the variation to \qWhy{}.%
%   \footnote{
%     In this respect, the variation to \qWhy{} need not be \emph{intensionally} adequate.
%     Where the variation of \qWhy{} (or \qHow{}) would be intensionally adequate just in case the variation captured the way something \emph{intuitively} answers \qWhy{}.
%   }

%   And, we are only interested in `rough' extensional adequacy.
%   In particular, we are only interested in answers to \qWhy{} and \qHow{} to the extent that \issueInclusion{} plausibly holds.
%   Hence, we will ignore intuitive answers to \qWhy{} and \qHow{} which extend beyond \issueInclusion{}.

%   Further, to the extent that \issueInclusion{} is intuitive, the variations \emph{may} conflict with this intuition.
% \end{note}

% \begin{note}
%   With the aid of \ros{1} we develop variations of \qWhy{} and \qHow{}:

%   \begin{itemize}
%   \item
%     Interpret `why' from \qWhy{} in terms of the \ros{1} the \agents{} conclusion of \(\pv{\phi}{v}\) from \(\Phi\) depended on.
%   \item
%     Interpret `how' from \qHow{} in terms of events which \wit{0} any \ros{} that the \agents{} conclusion of \(\pv{\phi}{v}\) from \(\Phi\) depended on.
%   \end{itemize}

%   % So, the variation to \qWhy{} is expected to be extensionally adequate for:

%   % If conclusion does not depend on \ros{}, then plausible that it is possible to answer \qWhy{} without citing the proposition-value-premises pair.
%   % For, event would have occurred regardless of whether paired.

%   % If conclusion does depend on \ros{}, then proposition-value-premises pair answers, in part, \qWhy{}.
%   % For, event would not have occurred regardless of whether paired.

%   % Variation to \qHow{}.
%   % Developed with respect to variation on \qWhy{}.

%   % If \ros{}, then if event \wit{} \ros{}, then of interest.
%   % If event which does not lead to \ros{}, then event is of no interest.
% \end{note}

% \begin{note}
%   An significant consequence of both variations will be as follows:

%   \begin{itemize}
%   \item
%     When an agent concludes \(\pv{\phi}{v}\) from \(\Phi\):
%     \begin{itemize}
%     \item
%       The \agents{} conclusion of \(\pv{\phi}{v}\) from \(\Phi\) depends on a \ros{0} between \(\pv{\phi}{v}\) and \(\Phi\) holding, for the agent.
%     \item
%       The event in which the agent concludes \(\pv{\phi}{v}\) from \(\Phi\) serves as a \wit{0} for the \ros{0} between \(\pv{\phi}{v}\) and \(\Phi\).
%     \end{itemize}
%   \end{itemize}
%   Hence, a \ros{} between \(\pvp{\phi}{v}{\Phi}\) will always be, in part, an answer to the variation of \qWhy{} and the event in which the agent concludes \(\pv{\phi}{v}\) from \(\Phi\) will always be, in part, an answer to the variation of \qHow{}.
% \end{note}

% \begin{note}
%   A variant to \issueInclusion{} follows from the variations to \qWhy{} and \qHow{}.
%   Roughly:

%   \begin{itemize}
%   \item
%     A conclusion of \(\pv{\phi}{v}\) from \(\Phi\) depends on some \ros{} between \(\pv{\psi}{v'}\) and \(\Psi\) holding for the agent

%     \emph{Only if}:

%     The agent has a \wit{} for the \ros{0} between \(\pv{\psi}{v'}\) and \(\Psi\).
%   \end{itemize}
% \end{note}

% \begin{note}
%   With an initial understanding of the variations to \qWhy{}, \qHow{}, and \issueInclusion{} in hand, we now return to the overall argument of this document.
%   Our goal is to develop counterexamples to \issueInclusion{}.
%   And, given the variation to \issueInclusion{} which follows from the variations to \qWhy{} and \qHow{}, we will do so by showing there are cases in which an agent concludes \(\pv{\phi}{v}\) from \(\Phi\) such that:
%   \begin{itemize}
%   \item
%     The agent pairing \(\phi\) and \(v\) depended on a \ros{0} between \(\pv{\psi}{v'}\) and \(\Psi\) holding, for the agent.
%   \item
%     The agent did not have a \wit{0} for the \ros{0} between \(\pv{\psi}{v'}\) and \(\Psi\) when the agent paired \(\phi\) and \(v\).
%   \end{itemize}
% \end{note}

% \begin{note}
  % As with \qWhyV{} our interest is with extensional adequacy, and specifically extensional adequacy with respect to \ros{1}.

  % Consider any event.
  % There are two cases.
  % \ros{} which answers \qWhyV{} or no \ros{}.

  % If no \ros{} then the \agents{} conclusion of \(\pv{\phi}{v}\) from \(\Phi\) does not depend on what happened.
  % Hence, event is not of direct interest with respect to answering `how' an agent concluded \(\pv{\phi}{v}\) from \(\Phi\).

  % If \ros{} which \qWhyV{}, then included as an answer to \qHowV{}.

  % So, though we have abstracted to \ros{1} to avoid any account of what conclusion amounts to, the existence of a \wit{0} intuitively captures whatever it is of relevance that happened when the agent concluded \(\pv{\phi}{v}\) from \(\Phi\).

  % And, as discussed in \autoref{cha:var:ros:W}, the existence of a \wit{0} allows for \ros{} which answer \qWhyV{} to be constrained by answers to \qHowV{}, even if the relevant \wit{0} occurs at some point prior to the event in which the agent concludes \(\pv{\phi}{v}\) from \(\Phi\).
% \end{note}

  % \begin{argument}{prop:phi-always-how}
  %   Suppose \(e\) is the event in which \vAgent{} concludes \(\pv{\psi}{v'}\) from \(\Psi\).
  %   By \autoref{prop:ros-always-answer} established that the \ros{} between \(\pv{\phi}{v}\) and \(\Phi\) is always an answer to \qWhyV{}.
  %   So, Clause~\ref{q:how:v:a} is satisfied.

  %   Likewise, by assumption \(e\) is the event in which \vAgent{} concludes \(\pv{\psi}{v'}\) from \(\Psi\).
  %   Hence, by \autoref{def:witnessing}, \(e\) is a \wit{} for the \ros{} between \(\pv{\psi}{v'}\) and \(\Psi\).
  %   So, Clause~\ref{q:how:v:b} is satisfied.

  %   And, as both Clause~\ref{q:how:v:a} and Clause~\ref{q:how:v:b} are satisfied, \(e\) is an answer to \qHowV{}.
  % \end{argument}

  % \begin{observation}
  %   \label{obs:qWhyV-description}
  %   Some \ros{1} matter.
  % \end{observation}
  % \begin{motivation}{obs:qWhyV-description}
  %   Situation.
  %   Flip coin.
  %   If heads, then by understanding of arithmetic.
  %   If tails, then by calculator.

  %   Now, coin lands heads.
  %   Well, sure, it seems this is needed.
  %   However, suppose coin lands tails.
  %   Then the agent would decide otherwise.
  %   So, really, the coin flip didn't matter.

  %   The point is that any event, and the description didn't rule out reversing.

  %   Does this matter?
  %   Not really, expand description.
  %   For any problematic event which differs from the way things are, rule out with description.
  % \end{motivation}

  % Whether there is something which rules out things being otherwise.

  % \begin{observation}
  %   \label{obs:do-mai}
  %   Some \ros{1} do not matter.
  % \end{observation}
  % \begin{motivation}{obs:do-mai}
  %   This tea tastes cold.

  %   Conclusion, so \ros{}.

  %   Consider event in which agent concludes.
  %   Unless deterministic and build in everything, then consider event in which agent doesn't take a sip.
  %   Does not conclude.
  %   Still concludes \(\pv{\phi}{v}\) from \(\Phi\).
  %   For, so much focus didn't even go for tea.
  % \end{motivation}

% \begin{figure}[H]
%     \centering
%     \begin{tikzpicture}
%       \tikzset{ansStyle/.style={%
%           draw=gray,%
%           text width=.5\textwidth,%
%           rounded corners=2pt,%
%         }%
%       }
%       %
%       \node[ansStyle] (whyO) at (0,0) %
%       {\qWhyV{} is answered by a \ros{0} between \(\pv{\psi}{v'}\) and \(\Psi\).};
%       %
%       \node[ansStyle] (whyA) at (1.933,-1.5) %
%       {\qWhy{} is answered by \(\pvp{\psi}{v'}{\Psi}\).};
%       %
%       \node[ansStyle] (howA) at (3.866,-3) %
%       {\qHow{} is answered by \(\pvp{\psi}{v'}{\Psi}\).};
%       %
%       \node[ansStyle] (witA) at (5.8,-4.5) %
%       {\qHowV{} is answered by event which \wit{1} sppt.\ btw.\ \(\pv{\psi}{v'}\) and \(\Psi\).};
%       %
%       \path[->] ($(whyO.south)!0.9!(whyO.south west)$) edge [out=270, in=180] (whyA);
%       \path[->] ($(whyA.south)!0.9!(whyA.south west)$) edge [out=270, in=180] (howA);
%       \path[->] ($(howA.south)!0.9!(howA.south west)$) edge [out=270, in=180] (witA);
%       %
%       \node[text width=.5\textwidth] (1) at (1,-.8) {\linkW{}};
%       \node[text width=.75\textwidth] (2) at (4.5,-2.25) {\issueInclusion{}};
%       \node[text width=.5\textwidth] (3) at (5,-3.625) {\linkH{}};
%     \end{tikzpicture}%
%     \caption{Visual representation of~\autoref{prop:support-and-witnessing} and \issueConstraint{}}
%     \label{fig:relations-between-whys-and-hows}
%   \end{figure}

%%% Local Variables:
%%% mode: latex
%%% TeX-master: "master"
%%% TeX-engine: luatex
%%% End:
