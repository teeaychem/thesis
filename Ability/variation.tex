\chapter{\qWhyV{} and \qHowV{}}
\label{cha:var}

\begin{note}
  This chapter states variations on \qWhy{}, \qHow{}, and \issueInclusion{}.
  The role of the variations is to establish a (sufficiently) precise way in which a relation may answer \qWhy{} without a corresponding answer to \qHow{}.

  Variants, rather than substitutes.
  \qWhyV{} as variant to \qWhy{} and \qHowV{} and variant to \qHow{}.
  Variants are connected to initial questions by linking conditionals.

  \begin{TOCEnum}
  \item
    \TOCLine{cha:var:qwhyvnp}

    Answers to \qWhyV{} are answers to \qWhy{}.
  \item
    \TOCLine{cha:var:qhowv}

    Answers to \qHow{} are answers to \qHowV{}.
  \item
    \TOCLine{cha:var:issue}

    If links hold, and \issueConstraint{} holds, then this entails variant constraint.
  Provide a recepie for generating counterexamples and then either links fail, or \issueConstraint{}.
  \end{TOCEnum}
\end{note}

\begin{note}
  Ensuring links hold is important.
  For, \issueInclusion{} is plausibly doing work to narrow the relevant sense of why captured by \qWhy{}.

  In other words, there may be plausible answers to \qWhy{} which are not also answers to \qHow{} but fail to be counterexamples to \issueInclusion{} as the relevant sense of `why' and `how' are not the senses of `why' and `how' that \issueInclusion{} (intuitively) holds with respect to.

  And, as a constraint on answers to \qWhy{} in terms of \qHow{}, \issueInclusion{} may plausibly have a role in narrowing the relevant senses of `why' and `how' at issue.
  And, if \issueInclusion{} is rejected, then, naturally, \issueInclusion{} cannot perform this role.

  Further, to the extent that various theories of conclusion, or sufficiently related phenomena, either implicitly or explicitly endorse \issueInclusion{}, it should be the case that the same theories either implicitly or explicitly motivate the variation of \issueConstraint{} constraining answers to the variation of \qWhy{} in terms of answers to \qHow{}.
\end{note}

\section{\qWhyV{}}
\label{cha:var:qwhyvnp}

% \begin{note}
%   Recall \qWhy{}:

%   \reQuestion{questionWhy}

%   Rather than asking for proposition-value-premises pairings associated with explanations of `why' an agent concluded \(\pv{\phi}{v}\) from \(\Phi\), the variation of \qWhy{} queries which \ros{1} are such that the conclusion of \(\pv{\phi}{v}\) from \(\Phi\) does not happen without presence of some \ros{0}.
% \end{note}

\subsection{The question}
\label{cha:var:qwhyvnp:question}

\begin{note}
  \qWhyV{} is as follows:

  \begin{question}{questionWhyV}{\qWhyV{}}
    \cenLine{
      \begin{VAREnum}
      \item
        Agent: \vAgent{}
      \item
        Prop.: \(\phi\)
      \item
        Value: \(v\)
      \item
        \pool{2}: \(\Phi\)
      \item
        Event: \(e\)
      % \item
      %   Event description: \(d\)
      \item
        \mbox{ }
      \end{VAREnum}
    }
    \medskip

    Given \(e\) is an event in which \vAgent{} concludes \(\pv{\phi}{v}\) from \(\Phi\):

    \begin{itemize}
    \item
      Which proposition-value-\pool{} pairs \(\pvp{\psi}{v'}{\Psi}\) are such that:
      \begin{itemize}
      \item
        There exists some (incomplete) description \(d\) such that:
        \begin{enumerate}[label=\arabic*., ref=(\arabic*)]
        \item
          There is some sub-event \(e^{\flat}\) of \(e\), such that \(d\) is true of \(e^{\flat}\).
        \item
          For any event \(e'\), \(d\) being true of \(e'\) entails:
          \begin{enumerate}
          \item[\emph{If}:]
            A \ros{0} between \(\pv{\psi}{v'}\) and \(\Psi\) fails to hold for \vAgent{} at \(e'\).
          \item[\emph{Then}:]
            \(e'\) is not, or does not develop into, an event in which \vAgent{} concludes \(\pv{\phi}{v}\) from \(\Phi\).
          \end{enumerate}
        \end{enumerate}
      \end{itemize}
    \end{itemize}
    \vspace{-\baselineskip}
  \end{question}
\end{note}

\begin{note}
  \qWhy{} seeks relations which explain \emph{why} an agent concluded \(\pv{\phi}{v}\) from \(\Phi\).
  \qWhyV{}, by contrast, seeks \ros{} without which the event in which the agent concluded \(\pv{\phi}{v}\) from \(\Phi\) did not happen.
\end{note}

\begin{note}
  \begin{notation}
  \item
    When speaking abstractly, \qWhyV{} always applies to \(\phi\), \(v\), and \(\Phi\).
  \end{notation}
\end{note}

\begin{note}
  Below we link \qWhyV{} to \qWhy{} by stating any answer to \qWhyV{} is an answer to \qWhy{}.
  (I.e., answers to \qWhyV{} explain why an agent concluded.)
  First, we clarify \qWhyV{}.
\end{note}

\begin{note}
  The \itc{} of \qWhyV{} builds on two basic ideas regarding events:
  \begin{enumerate}
  \item
    An event may or may not develop into some other event.

    For example, the event in which I take hold of a regular balloon may not develop into an event in which I float off into the sky.
    (I'm a little to heavy.)
  \item
    An event may be broken down into sub-events.

    For example, the event in which I anxiously anticipate floating off into the sky has an initial sub-event in which I take hold of a balloon.
  \end{enumerate}

  Now, consider an event in which an agent concludes \(\pv{\phi}{v}\) from \(\Phi\).
  This event may be broken down into various sub-events, and for any sub-event may or may not develop into the event in which the agent concludes \(\pv{\phi}{v}\) from \(\Phi\).
  One part is easy.
  The agent concludes \(\pv{\phi}{v}\) from \(\Phi\), and so each sub-event may develop into the event in which the agent concludes.
  However, some sub-events may not.
  For example, the event in which I take hold of a regular balloon may develop into a happy day (so long as I forget about floating off into the sky).

  Specifically, \qWhyV{} considers whether an sub-event may not have developed into an event in which the agent concludes if a \ros{} failed to hold.
  And, answers to \qWhyV{} are proposition-value-\pool{} pairing such that a \ros{} failing to hold is sufficient to ensure the sub-event does not develop into an event in which the agent concludes.
  In this respect the \ros{} answers, in part, why an agent concluded \(\pv{\phi}{v}\) from \(\Phi\).
  For, without the \ros{} some other event happened.
\end{note}

\begin{note}
  The role of descriptions and quantification over events is to reduce ambiguity.
  Natural language conditionals are difficult.%
  \footnote{
    ~\cite{Edgington:2020aa}
  }
  Therefore, the \itc{} of \qWhyV{} is a material conditional.
  So long as it is \emph{not} the case that the antecedent is true while the consequent is false, then the conditional is true.
  This is straightforward.

  Observe, event does develop.
  So, if conditional is true then by contraposition \ros{} holds at sub-event.

  However, this provides no information about what happens if the conditional fails to hold.
  Hence, description.

  Description captures parts of an event, and consider events which satisfy the description.
  If conditional follows from description then there's no event where fails and concludes.

  Incomplete description.
  Don't simply capture everything true of the sub-event.
  This is a little vague.
  However, tolerable.
  At issue is not application, but the threshold.
\end{note}

\begin{note}
  To get a feel for the way descriptions work, consider the following \scen{0}:

  \begin{scenario}[Coin flip]
    \label{illu:coinT}
    Agent \(B\) has accepted gable on coin toss from agent \(A\), captured by the following pair of conditionals:
    \begin{itemize}[noitemsep]
    \item
      If the coin lands heads, then agent \(B\) gives \texteuro{}5 to agent \(A\).
    \item
      If the coin lands tails, then agent \(A\) gives \texteuro{}5 to agent \(B\).
    \end{itemize}

    Agent \(A\) has tossed the coin, but the result of the toss is hidden between agent \(A\)'s hands.
  \end{scenario}

  Our presentation of the \scen{} is a description, and from the pair of conditionals two further conditionals follow:%
  \footnote{
    This is not immediate.
  For, consider adding to the description of~\autoref{illu:coinT} that agent \(A\) is dishonest, and will run before handing anything to agent \(B\), that neither agent really has \texteuro{}5 at hand and are hoping they win, or \dots
  The point is, unless description \(d\) (logically) entails description \(d'\), there is some what for description \(d\) to be true and description \(d'\) to be false.
  Hence, it may be argued that \(d'\) does not follow from \(d\).

  However, enhance the description.
  So long as there is something about the event which rules out, then add to description.
  }

  \begin{itemize}[noitemsep]
  \item
    If the coin has landed heads, then agent \(B\) will soon be \texteuro{}5 poorer.
  \item
    If the coin has landed tails, then agent \(A\) will soon be \texteuro{}5 poorer.
  \end{itemize}

  So, these conditionals are same as \itc{} from \qWhyV{}.%
  \footnote{
    \nocite{Tichy:1976tp}%
    Likewise, extend to event of \autoref{illu:coinT} so that the result of the toss has been revealed, then counterfactual conditionals, depending on the result.

    \begin{itemize}[noitemsep]
    \item
      If the coin had landed heads, agent \(B\) would have been \texteuro{}5 poorer.
    \item
      If the coin had landed tails, agent \(B\) would have been \texteuro{}5 richer.
    \end{itemize}

    Indeed, with respect to subjunctive conditionals, say the conditionals which specify the gamble are `laws' (\cite{Chisholm:1955aa,Lewis:1979vm,Veltman:2005tj}) or `lump together' the facts regarding the coin and relative wealth of agents \(A\) and \(B\) (\cite{Kratzer:1981aa,Kratzer:1989aa}).
  }

  The role of descriptions is to observe that there is something about the event which generalises.

  Returning to \qWhyV{}, the \itc{} follows from the specified description \(d\) in parallel to the way the observed conditionals follow from the description of the event as given in~\autoref{illu:coinT}.
  The agent has concluded, but to evaluate whether or not \(\pvp{\psi}{v'}{\Psi}\) answers, rewind to sub-event.
  Parallel, gamble has been settled, but rewind.%
  \footnote{
    One concern is overdetermination.

    It just so happens to be that both \(A\) and \(B\) have been visited by a pickpocket.
    For, we may vary the presence of the pickpocket while maintaining the description of the \scen{0}, and given the gamble, the relevant transfer of wealth occurs.

    Though, add to the description to rule out the presence of a pickpocket.
    So long as there's no pickpocket as the event happens, the description remains true.

    May worry about disconnect between description and event.
    Material, so it could be that in all other cases gamble, but in this case pickpocket.
  }
\end{note}


\begin{note}
  In short, \qWhyV{}, captures the phenomenon where it is not possible to describe \(e\) as opposed to some other event without \ros{}.

  In contrast to \qWhy{}, where relation explains why agent concludes, \qWhyV{} queries which \ros{1} are such that the conclusion of \(\pv{\phi}{v}\) from \(\Phi\) does not happen without presence of some \ros{0}.

  
\end{note}

\subsection{\illu{3}}
\label{sec:two-scen0}

\begin{note}
  \qWhyV{} is a little complex.
  Apply ideas of \qWhyV{} to two \scen{1}.

  The first \scen{0} abstracts from conclusions to consider the core ideas.
  The second \scen{0} returns to conclusions and \ros{1}.
\end{note}

\paragraph*{General form}

\begin{note}
  \begin{scenario}[A cup of tea]%
    \label{scen:cup-of-tea}%
    A moment ago I:
    \begin{itemize}[noitemsep]
    \item
      Boiled some water.
    \item
      Placed a tea bag into a cup.
    \item
      Poured the water into the cup.
    \item
      Let the tea bag rest in the water for a while.
    \item
      Removed the tea bag.
    \end{itemize}
    \vspace{-\baselineskip}
  \end{scenario}

  \autoref{scen:cup-of-tea} captures an event in which I made a cup of tea.

  Each step is a sub-event, and the presentation is an adequate description of the sub-event.
  Now consider the sub-event in which I pour water into the cup.
  It seems the truth of the following conditional more is-or-less immediate:
  \begin{itemize}
  \item
    If the water isn't boiled, tea isn't made.
  \end{itemize}

  And, the ways in which the conditional is less immediate may be ruled out by adding to the description.
  For example, I may notice the temperature of a splash and put the water on to boil.
  However, if I'm short for time, then I have no choice but to abandon tea.
  So, as long as the conditional is true for the sub-event as it happened, we may add to our description of the sub-event to secure the truth.

  {
    \color{red}
    Boiled water.
    Without this, wouldn't have made a cup of tea.
  }

  So, \qWhyV{}.
  Tea bag in tea cup to make tea.
  \ros{2} between \(\pv{\psi}{v'}\) and \(\Phi\) to conclude \(\pv{\phi}{v}\) from \(\Phi\).
\end{note}

\begin{note}
  There is a straightforward sense with which the description given to the event captures \emph{how} I made a cup of tea.
  Further, there is also a sense with which the description also captures \emph{why} I made a cup of tea.

  There is also a sense with which the description fails to capture why I made a cup of tea.
  Placing a tea bag in a cup does not usually motivate making a cup of tea.%
  \footnote{
    Though we may imagine someone whose only pleasure is finding excuses to place tea bag into cups.
    (\cite[Cf.][379--380]{Rawls:1999aa})
  }

  Why the event resulted in a cup of tea versus why I wanted the event to result in a cup of tea.

  The former parallels our interest with conclusions.
  Not interested with why an agent wanted to reach a conclusion, but why an agent concluded \(\phi\) has value \(v\) as opposed to any other proposition-value pair, and why the agent concluded \(\phi\) has value \(v\) from \(\Phi\) as opposed to some other pool of premises.
\end{note}

\paragraph{A \scen{0}}

\begin{note}
  Without \ros{}, the event does not develop into an event in which the agent concludes.

  \begin{scenario}[Countersign]
    \indent The captain mumbled, ``I come from Miran.''

    The man returned the gambit, grimly.
    ``Miran is early this year.''

    The captain said, ``No earlier than last year.''

    But the man did not step aside.
    He said, ``Who are you?''

    ``Aren't you Fox?''

    ``Do you always answer by asking?''

    The captain took an imperceptibly longer breath, and then said calmly,
    ``I am Han Pritcher, Captain of the Fleet, and member of the Democratic Underground Party.
    Will you let me in?''%
    \mbox{ }\hfill\mbox{(\cite[70]{Asimov:1945aa})}%
    \newline
  \end{scenario}

  Fox concluded the person they are talking to is a fellow member of the Democratic Underground Party via the countersigning.

  So, sub-event, `No earlier than last year'

  \ros{2}:
  \[
    \pv{\prop{`No earlier than last year' was an appropriate response to my sign}}{\val{True}}
  \]

  Description, Pritcher is attempting to countersign.
  Sequence, `I come from Miran.', `Miran is early this year.', `No earlier than last year.', \dots.

  Conditional, from \agpe{Fox's}:
  \begin{itemize}
  \item
    If conversation partner does not respond to `Miran is early this year' with `No earlier than last year', then the partner is not engaging in countersign.
  \end{itemize}

  If \ros{} failed to hold, then conversation partner did not respond to `I come from Miran' with `Miran is early this year'.
  Events are compatible with description.
  Attempting to countersign, so events in which attempts fail.
  Some event where Pritcher responds `Miran overslept', etc.
\end{note}

\begin{note}
  Understand why Fox concluded member of the Democratic Underground Party.
  Though, do not understand why Fox did not (conclude to immediately) let Pritcher in.
\end{note}

\begin{note}
  Not everything matters.
  \[
    \pv{\prop{`I come from Miran' was mumbled}}{\val{True}}
  \]
  Still, event in which Pritcher does not mumble.
  Countersign goes through, all fine.
\end{note}

\subsection{Link}
\label{cha:var:qwhyvnp:link}

\begin{note}
  \begin{link}[\qWhyV{} and \qWhy{}]
    \label{link:why:support:pvpp}
    \cenLine{
      \begin{VAREnum}
      \item
        Agent: \vAgent{}
      \item
        Proposition: \(\psi\)
      \item
        Value: \(v'\)
      \item
        \pool{2}: \(\Psi\)
      \item
        \mbox{ }
      \end{VAREnum}
    }
    \begin{itemize}
    \item[\emph{If}:]
      \begin{enumerate}[label=\alph*., ref=(\alph*)]
      \item
        \(\pvp{\psi}{v'}{\Psi}\) is an answer to \qWhyV{}.
      \end{enumerate}
    \item[\emph{Then}:]
      \begin{enumerate}[label=\alph*., ref=(\alph*), resume]
      \item
        \(\pvp{\psi}{v'}{\Psi}\) is an answer to \qWhy{}.
      \end{enumerate}
    \end{itemize}
    \vspace{-\baselineskip}
  \end{link}

  \linkW{} states that \textbf{dependence} captured by \qWhyV{} is sufficient to explain why an agent concluded \(\pv{\phi}{v}\) from \(\Phi\).
\end{note}

\begin{note}
  Motivation for \linkW{} has been highlighted throughout clarification of \qWhyV{}:
  Why in the sense of why this event rather than any other event.
\end{note}

\subsection{Observations}

\begin{note}
  \begin{proposition}[Guaranteed answers to \qWhyV{}]
    \label{prop:ros-always-answer}
    \cenLine{
      \begin{VAREnum}
      \item
        Agent: \vAgent{}
      \item
        Proposition: \(\phi\)
      \item
        Value: \(v\)
      \item
        \pool{2}: \(\Phi\)
      \item
        \mbox{ }
      \end{VAREnum}
    }
    \begin{itemize}
    \item
      The \ros{} between \(\pv{\phi}{v}\) from \(\Phi\) is always an answer to \qWhyV{}.
    \end{itemize}
    \vspace{-\baselineskip}
  \end{proposition}

  \begin{argument}{prop:ros-always-answer}
    By \supportI{}, a \ros{} holds between \(\pv{\phi}{v}\) and \(\Phi\), for the agent, when the agent pairs \(\phi\) with \(v\) as a sub-event of an event in which the agent concludes \(\pv{\phi}{v}\) from \(\Phi\).

    Hence, by restricting \qWhyV{} to the sub-event, we ensure that the \ros{} between \(\pv{\phi}{v}\) from \(\Phi\) holds, for the agent, and so may answer, in part, \qWhyV{}.

    And, the \ros{} between \(\pv{\phi}{v}\) from \(\Phi\) is guaranteed to be an answer to \qWhyV{} because, if the \ros{} between \(\pv{\phi}{v}\) from \(\Phi\) did not hold, for the agent, then the relevant event would be an event in which the agent concludes \(\pv{\phi}{v}\) from \(\Phi\).
  \end{argument}
\end{note}

\begin{note}
  Focused on boiled water.
  However, various redundant things.
  For example, cup.
  Selected cup from shelf, and variety to choose from.
  But, picked one.
  Does particular cup make a difference?
  Intuitively no.

  \begin{observation}
    \label{obs:qWhyV:rosNNecAns}
    \cenLine{
      \begin{VAREnum}
      \item
        Agent: \vAgent{}
      \item
        Prop.: \(\phi\)
      \item
        Value: \(v\)
      \item
        \pool{2}: \(\Phi\)
      \item
        Event: \(e\)
      \item
        \mbox{ }
      \end{VAREnum}
    }
    \medskip

    Given \(e\) is an event in which \vAgent{} concludes \(\pv{\phi}{v}\) from \(\Phi\):
    \begin{itemize}
    \item
      There is no simple relation between \ros{1} which hold at some sub-event \(e^{\flat}\) of \(e\) and answers to \qWhyV{}.
    \end{itemize}
    \vspace{-\baselineskip}
  \end{observation}

  \begin{motivation}{obs:qWhyV:rosNNecAns}
    In short, depends on description, and failure of \ros{} to hold blocks development.

    Suppose \(d\) only captures conclusion.
    Now, over determined.


    \begin{itemize}
    \item
      Suppose coin flip.
      In the style of \citeauthor{Tichy:1976tp}.
      Here, the coin flip makes no difference.
    \item
      For, consider a \scen{0} in which an agent concludes \(23 \times 15 = 345\) from their understanding of arithmetic, such that some time shortly before the agent did the calculation, the agent concluded it would be good to brew a cup of tea.

      Intuitively it should not be the case that the \ros{} and the conclusion that it would be good to get some coffee answers, in part, why the agent concluded some theorem is true from some \pool{}.
      Though, it may well be the case that the agent would have been too tired to conclude the theorem without the aid of the coffee.%
      \footnote{
        \citeauthor{Armstrong:1968vh} (\citeyear[195--196]{Armstrong:1968vh}) discusses a similar example, and suggests further discussion of this issue may also be found in \textcite{Moore:1962up}.
        See also \citeauthor{Sanford:1989aa} (\citeyear{Sanford:1989aa}) for a broader discussion of dependence as captured by subjunctive conditions (esp.\ pp.\ 192--193).
      }

      However, if the agent is tired, then would there have been something else to help them focus?
    \end{itemize}
    \vspace{-\baselineskip}
  \end{motivation}

  \autoref{obs:qWhyV:rosNNecAns}.
  Here, then, \ros{} holds.
  Issue whether \ros{} also answers \qWhy{}.
  Not necessarily clear from motivation.
  May be other cases.
  However, shouldn't be too worrying.
  \qWhyV{} is only sufficient.
  Hence, does not rule out these answers.
  Under-generating, and that's fine.
\end{note}

\begin{note}
  \begin{observation}
    Maybe over-generates.

    For, description so fine that everything about event matters.
    This comes down to understanding of events.
    However, I don't think this is too problematic.
    For, all of these things are captured by \qHow{}.
  \end{observation}

\end{note}

\begin{note}
  Still, we will proceed with \qWhyV{}.

  Motivated:
  Theory neutral.

  The imprecision arising from the \itc{} is in part by design.

  Abstracting to a conditional allows the relevant link between~the antecedent and consequent to be developed in accordance with specific details given by some account of what concluding amounts to.
\end{note}


\section{\qHowV{}}
\label{cha:var:qhowv}

\begin{note}
  \qHow{} is a broad question which asks, quite generally, for an account of how an agent concluded \(\pv{\phi}{v}\) in terms of what has happened.
\end{note}

\subsection{Question}
\label{cha:var:qhowv:sec:question}

\begin{note}
  The variant on \qHow{} is as follows:

  \begin{question}{questionHowV}{\qHowV{}}
    \label{q:how:v}
    \cenLine{
      \begin{VAREnum}
      \item
        Agent: \vAgent{}
      \item
        Proposition: \(\phi\)
      \item
        Value: \(v\)
      \item
        \pool{2}: \(\Phi\)
      \item
        Event: \(e\)
      \item
        \mbox{ }
      \end{VAREnum}
    }
    \medskip

    Given \(e\) is an event which \vAgent{} concludes \(\pv{\phi}{v}\) from \(\Phi\):
    \begin{itemize}
    \item
      Which events \(e'\) are such that:
      \begin{itemize}
      \item
        For some \(\pv{\psi}{v'}{\Phi}\) such that \ros{} which holds when \vAgent{} concludes, or held prior to \vAgent{} conclusion, of \(\pv{\phi}{v}\) from \(\Phi\):
        \begin{itemize}
        \item
          \(e'\) is a \wit{0} for a \ros{} between \(\pv{\psi}{v'}\) and \(\Psi\).
        \end{itemize}
      \end{itemize}
    \end{itemize}
    \vspace{-\baselineskip}
  \end{question}

  \qHowV{} is graceless.
  Which events are such that agent concluded some proposition has some value from some \pool{} prior to conclusion of \(\pv{\phi}{v}\) from \(\Phi\).
  In contrast to \qHow{}, there is nothing that directly connected answers to \qHowV{} to the event in which the agent concludes \(\pv{\phi}{v}\) from \(\Phi\).

  There is nothing which prevents refinement.
  \ros{} is an answer to \qWhyV{}, for example.

  However, little interest.
  Role of \qHowV{} is to constrain answers to \qWhyV{}.
  Does this in the way that matters.
  \supportII{}, no need for \wit{}.
  If constrains, then need \wit{}.
\end{note}

\begin{note}
  So, have \wit{}?

  \qHow{} seeks events which explain how.
  \qHowV{} narrows attention to \ros{} and seeks events which \wit{} relevant \ros{1}.

  Key here is that \supportII{}, need not be the case that \wit{}.
\end{note}

\begin{note}
  \begin{proposition}[Guaranteed answers to \qHowV{}]
    \label{prop:phi-always-how}
    \cenLine{
      \begin{VAREnum}
      \item
        Agent: \vAgent{}
      \item
        Proposition: \(\phi\)
      \item
        Value: \(v\)
      \item
        \pool{2}: \(\Phi\)
      \item
        \mbox{ }
      \end{VAREnum}
    }

    \begin{itemize}
    \item
      For any event \(e\) in which \vAgent{} concludes \(\pv{\phi}{v}\) from \(\Phi\):
      \begin{itemize}
      \item
        \(e\) is an answer to \qHowV{}.
      \end{itemize}
    \end{itemize}
    \vspace{-\baselineskip}
  \end{proposition}

  \begin{argument}{prop:phi-always-how}
    Let \(e\) be an event in which an agent concludes \(\pv{\phi}{v}\) from \(\Phi\).

    Then, immediate by \autoref{def:witnessing}.
  \end{argument}
\end{note}

\subsection{Link}
\label{cha:var:qhowv:sec:link}

\begin{note}
  \begin{link}[\qHowV{} and \qHow{}]
    \label{link:how-witnessing}
    \cenLine{
      \begin{VAREnum}
      \item
        Proposition: \(\psi\)
      \item
        Value: \(v'\)
      \item
        \pool{2}: \(\Psi\)
      \item
        \mbox{ }
      \end{VAREnum}
    }

    \begin{itemize}
    \item[\emph{If}:]
      \begin{enumerate}[label=\alph*., ref=(\alph*)]
      \item
        \(e\) answers \qHow{} (conclusion).
      \end{enumerate}
    \item[\emph{Then}:]
      \begin{enumerate}[label=\alph*., ref=(\alph*), resume]
      \item
        \(e\) answers \qHowV{}.
      \end{enumerate}
    \end{itemize}
    \vspace{-\baselineskip}
  \end{link}

  Answer to \qHow{}.
  Still, \qHow{} is not of direct interest.
  Instead, \issueInclusion{}.
  Constrain that \qHow{} places on \qWhy{}.

  Build on \autoref{prop:phi-always-how}.
  Event in which agent concludes is always an answer to \qHowV{}.
  Therefore, in order for \linkH{} to fail, must be the case that there is some event which is not a sub-event of the event in which an agent concludes.

  Intuitively, answers to \qHow{} are events in which an agent concludes.
  So, qualification is not needed.
  However, no clear argument.
  So, include.

  As \qHowV{} is weaker than \qHow{}, weaker constraint.
  And, as we are looking for counterexamples, this restricts possible counterexamples.
\end{note}

\section{\issueConstraint{}}
\label{cha:var:issue}

\begin{note}
  In~\autoref{cha:var:qwhyvnp} we developed \qWhyV{}, a variation of \qWhy{}.
  And, in~\autoref{cha:var:qhowv} we developed \qHowV{}, a variation of \qHow{}.
  In both sections we established links between the variant --- \qWhyV{}, \qHowV{}  --- and initial --- \qWhy{}, \qHow{} --- questions.

  Indeed, the purpose of developing the variant questions is to clarify the initial questions to a sufficient degree so that counterexamples to the intuitive constraint between the initial questions --- \issueInclusion{} --- may be developed.

  In this section, we bring together the variant and initial questions and \issueInclusion{} to state, in some detail, how everything fits together.
\end{note}

\begin{note}
 We being with a proposition:

  \begin{proposition}[\qWhyV{}-\qWhy{}-\qHow{}-\qHowV{}]
    \label{prop:support-and-witnessing}
    \cenLine{
      \begin{VAREnum}
      \item
        Agent: \vAgent{}
      \item
        Propositions: \(\phi\), \(\psi\)
      \item
        Values: \(v\), \(v'\)
      \item
        \pool{3}: \(\Phi\), \(\Psi\)
      \item
        \mbox{ }
      \end{VAREnum}
    }
    \begin{itemize}[leftmargin=*]
    \item
    \begin{itemize}
    \item[\emph{If}:]
      \linkW{}, \linkH{}, and \issueInclusion{} (jointly) hold.
      \item[\emph{Then}:]
        \begin{itemize}
        \item
          \qWhyV{} is answered, in part, by a \ros{} between \(\pv{\psi}{v'}\) and \(\Psi\).
        \end{itemize}

        \emph{Only if}

        \begin{itemize}
        \item
          \qHowV{} is answered, in part, by \vAgent{}' \wit{0} for the \ros{} between \(\pv{\psi}{v'}\) and \(\Psi\).
        \end{itemize}
      \end{itemize}
    \end{itemize}
    \vspace{-\baselineskip}
    \end{proposition}

  \begin{argument}{prop:support-and-witnessing}
    Suppose \qWhyV{} is answered, in part, by a \ros{} between \(\pv{\psi}{v'}\) and \(\Psi\).
    From~\linkW{} it is immediately follows that \(\pvp{\psi}{v'}{\Psi}\) answers, in part, \qWhy{}.
    And, given \issueInclusion{}, \(\pvp{\psi}{v'}{\Psi}\) answers, in part, \qHow{}.

    Further, \(\pvp{\psi}{v'}{\Psi}\) answers, in part, \qHow{} due to a \ros{} between \(\pv{\phi}{v}\) and \(\Psi\) being, in part, an answer to \qWhyV{}.
    Therefore, by \linkH{}, there is some \(e\) such that \(e\) is a \wit{0} for the \ros{} between \(\pv{\psi}{v'}\) and \(\Psi\).
  \end{argument}


  \autoref{fig:relations-between-whys-and-hows} is a a visual representation of the argument for~\autoref{prop:support-and-witnessing}.
\end{note}

\begin{figure}[H]
  \centering
  \begin{tikzpicture}
    \tikzset{ansStyle/.style={%
        draw=gray,%
        text width=.5\textwidth,%
        rounded corners=2pt,%
      }%
    }
    %
    \node[ansStyle] (whyO) at (0,0) %
    {\qWhyV{} is answered by a \ros{0} between \(\pv{\psi}{v'}\) and \(\Psi\).};
    %
    \node[ansStyle] (whyA) at (1.933,-1.5) %
    {\qWhy{} is answered by \(\pvp{\psi}{v'}{\Psi}\).};
    %
    \node[ansStyle] (howA) at (3.866,-3) %
    {\qHow{} is answered by \(\pvp{\psi}{v'}{\Psi}\).};
    %
    \node[ansStyle] (witA) at (5.8,-4.5) %
    {\qHowV{} is answered by event which \wit{1} sppt.\ btw.\ \(\pv{\psi}{v'}\) and \(\Psi\).};
    %
    \path[->] ($(whyO.south)!0.9!(whyO.south west)$) edge [out=270, in=180] (whyA);
    \path[->] ($(whyA.south)!0.9!(whyA.south west)$) edge [out=270, in=180] (howA);
    \path[->] ($(howA.south)!0.9!(howA.south west)$) edge [out=270, in=180] (witA);
    %
    \node[text width=.5\textwidth] (1) at (1,-.8) {\linkW{}};
    \node[text width=.75\textwidth] (2) at (4.5,-2.25) {\issueInclusion{}};
    \node[text width=.5\textwidth] (3) at (5,-3.625) {\linkH{}};
  \end{tikzpicture}%
  \caption{Visual representation of~\autoref{prop:support-and-witnessing}}
  \label{fig:relations-between-whys-and-hows}
\end{figure}

\begin{note}
  As~\autoref{prop:support-and-witnessing} follows from \linkW{}, \linkH{}, and \issueInclusion{}, we consider the content of~\autoref{prop:support-and-witnessing} to be a parallel constraint to \issueInclusion{}:

  \begin{constraint}{consConstraint}{\issueConstraint{}}
    \cenLine{
      \begin{VAREnum}
      \item
        Agent: \vAgent{}
      \item
        Propositions: \(\phi\), \(\psi\)
      \item
        Values: \(v\), \(v'\)
      \item
        \pool{3}: \(\Phi\), \(\Psi\)
      \item
        \mbox{ }
      \end{VAREnum}
    }
    \begin{itemize}
    \item
      \qWhyV{} is answered, in part, by a \ros{} between \(\pv{\psi}{v'}\) and \(\Psi\).
    \end{itemize}

    \emph{Only if}

    \begin{itemize}
    \item
      \qHowV{} is answered, in part, by \vAgent{}' \wit{0} for the \ros{} between \(\pv{\psi}{v'}\) and \(\Psi\).
    \end{itemize}
    \vspace{-\baselineskip}
  \end{constraint}

  As with \issueInclusion{}, if \issueConstraint{} holds, then answers to \qWhyV{} are constrained by answers to \qHowV{}.

  Our direct goal is to develops counterexamples to \issueConstraint{}.
  For, if there are counterexamples to \issueConstraint{}, then it immediately follows by \autoref{prop:support-and-witnessing} that either \linkW{}, \linkH{}, or \issueInclusion{} fails to hold.

  We defended \linkW{} when developing \qWhyV{} in~\autoref{cha:var:qwhyvnp}.
  And, likewise, we defended \linkH{} when developing \qHowV{} in~\autoref{cha:var:qhowv}.
  And, though there are some difficulties with \qWhyV{}, \qHowV{}, \linkW{}, and \linkH{}, I consider the most plausible point of failure to be \issueInclusion{}.
\end{note}

% \begin{note}
%   Still, \issueConstraint{} may be considered as a direct constraint on answers to \qWhyV{} in terms of answers to \qHow{}.
%   In particular, \autoref{prop:constraint-rewrite} expands our construction of \qWhyV{}  and \qHowV{} to provide an alternative statement of \issueConstraint{}.

  % \begin{proposition}[\issueConstraint{}, rewritten]
  %   \label{prop:constraint-rewrite}
  %   For an agent \vAgent{}, proposition-value pairs \(\pv{\phi}{v}\), \(\pv{\psi}{v'}\), and \pool{1} \(\Phi\), \(\Psi\):

  %   \issueConstraint{} is equivalent to the following conditional:

  %   \begin{enumerate}
  %   \item[\emph{If}:]
  %     \begin{enumerate}[label=\alph*., ref=(\alph*)]
  %     \item
  %       \label{constraint-rewrite:a}
  %       \vAgent{} concluded \(\pv{\phi}{v}\) from \(\Phi\).
  %     \end{enumerate}
  %   \item[\emph{And}:]
  %     \begin{enumerate}[label=\alph*., ref=(\alph*), resume]
  %     \item
  %       \label{constraint-rewrite:b}
  %       \vAgent{} would not have concluded \(\pv{\phi}{v}\) from \(\Phi\), if a \ros{} between \(\pv{\psi}{v'}\) and \(\Psi\) failed to hold, from \agpe{\vAgent{}'}.
  %     \end{enumerate}
  %   \item[\emph{Then}:]
  %     \begin{enumerate}[label=\alph*., ref=(\alph*), resume]
  %     \item
  %       \label{constraint-rewrite:c}
  %       \vAgent{} has a \wit{0} for the \ros{} between \(\pv{\psi}{v'}\) and \(\Psi\).
  %     \end{enumerate}
  %   \end{enumerate}
  %   \vspace{-\baselineskip}
  % \end{proposition}
  % %
  % \begin{argument}{prop:constraint-rewrite}
  %   Immediate by \autoref{prop:support-and-witnessing} and the construction to \qWhyV{} and \qHowV{}.

  %   The antecedent of the conditional~---~\ref{constraint-rewrite:a}~\&~\ref{constraint-rewrite:b}~---~correspond to a \ros{} being, in part, an answer to \qWhyV{} while the consequent~---~\ref{constraint-rewrite:c}~---~corresponds to the \wit{0} which answers, in part, \qHowV{}.
  %   The conditional then follows from \autoref{prop:support-and-witnessing}.
  % \end{argument}

%   So, directly, \issueConstraint{} amounts to the constraint that in order for an \agents{} conclusion of \(\pv{\phi}{v}\) to depend on some \ros{} between \(\pv{\psi}{v'}\) and \(\Psi\), the agent must have concluded \(\pv{\psi}{v'}\) from \(\Psi\).
% \end{note}

\section{Summary}
\label{cha:var:summary}

\begin{note}
  Overall argument.
  Links then answer to \qWhyV{} which is not constrained by \qHowV{}, then \issueInclusion{} fails.

  Three broad ways in which the overall argument may fail:
  \begin{enumerate}[label=\arabic*., ref=(\arabic*), noitemsep]
  \item
    The link between \qWhyV{} and \qWhy{} fails to hold.
  \item
    The link between \qHowV{} and \qHow{} fails to hold.
  \item
    We fail to develop counterexamples to \issueConstraint{}.
  \end{enumerate}

  Still, I hope to have developed \qWhyV{}, \qHowV{}, and \issueConstraint{} in such a way that both questions are of some interest independent of link

  In \autoref{cha:clar:sec:literature} we suggest how a handful of accounts of conclusion, or related, may be understood in terms of \qWhyV{} and \qHowV{}, and how the accounts motivate \issueConstraint{} as a constraint on answers to \qWhyV{} in terms of answers to \qHowV{}.
\end{note}

%%% Local Variables:
%%% mode: latex
%%% TeX-master: "master"
%%% End:


% \section{The role of variant questions}
% \label{cha:var:sec:wiggling}

% \begin{note}
%   \autoref{cha:introduction} introduced \qWhy{}, \qHow{}, and \issueInclusion{}.


%   The overall goal of this document is to argue \issueInclusion{} does not hold.

%   Hence, without establishing a clear understanding of the way in which the instances are to be understood, it is unclear how to develop counterexamples to \issueInclusion{}.
% \end{note}

% \begin{note}
%   In broad outline, we use the idea of a `\ros{0}' to provide variations on \qWhy{}, \qHow{}, and \issueInclusion{}.
%   The way in which we understand \ros{1} is minimal and tightly connected to an event in which an agent concludes \(\pv{\phi}{v}\) from \(\Phi\).
%   Specifically, we put forward three ideas in relation to \ros{1}:
%   \begin{enumerate}
%   \item
%     If an agent concludes \(\pv{\phi}{v}\) from \(\Phi\), then a \ros{0} between \(\pv{\phi}{v}\) and \(\Phi\), for the agent, when the agent pairs \(\phi\) with \(v\).
%   \item
%     If an agent has concluded \(\pv{\phi}{v}\) from \(\Phi\), then the event in which the agent concluded \(\pv{\phi}{v}\) from \(\Phi\) functions as a \wit{0} for a \ros{0} between \(\pv{\phi}{v}\) and \(\Phi\).
%   \item
%     It is possible for a \ros{0} between \(\pv{\phi}{v}\) and \(\Phi\) to hold, from an \agpe{}, without their being an \wit{0} for the \ros{0}.
%   \end{enumerate}

%   \autoref{cha:var:ros} will develop and discuss each idea in detail.

%   For the moment, the motivation for abstracting to \ros{1} is to capture, in a abstract way, the way in which \(\pv{\phi}{v}\) and \(\Phi\) are related from an \agpe{} when the agent concludes \(\pv{\phi}{v}\) from \(\Phi\).

%   Rather than directly capturing some relevant sense of `why' or `how' our goal is to use \ros{1} to construct variations on \qWhy{} and \qHow{} which are \emph{roughly} `extensionally adequate'.
%   Where, we understand the term `extensionally adequate' more-or-less in line with \citeauthor{Sumner:1987aa} (\citeyear{Sumner:1987aa}):

%   \begin{quote}
%     [A] conception of a concept is extensionally adequate when it includes every item which seems pre-analytically to be an instance of the concept and excludes every item which does not.%
%     \mbox{ }\hfill\mbox{(\citeyear[49]{Sumner:1987aa})}
%   \end{quote}

%   Adapted to our case, our interest with \qWhy{} and \qHow{} is with respect to intuitive answers to \qWhy{} and \qHow{} (and in particular the intuition that \issueInclusion{} holds).
%   And, the variations of both \qWhy{} and \qHow{} may be seen as `conceptions of a question' such that any answer to \qWhy{} is an answer to the variation of \qWhy{}, vice-versa, and the same with respect to \qHow{}.

%   Still, the way in which something answers \qWhy{} need not be equivalent to the way in which that thing answers the variation to \qWhy{}.%
%   \footnote{
%     In this respect, the variation to \qWhy{} need not be \emph{intensionally} adequate.
%     Where the variation of \qWhy{} (or \qHow{}) would be intensionally adequate just in case the variation captured the way something \emph{intuitively} answers \qWhy{}.
%   }

%   And, we are only interested in `rough' extensional adequacy.
%   In particular, we are only interested in answers to \qWhy{} and \qHow{} to the extent that \issueInclusion{} plausibly holds.
%   Hence, we will ignore intuitive answers to \qWhy{} and \qHow{} which extend beyond \issueInclusion{}.

%   Further, to the extent that \issueInclusion{} is intuitive, the variations \emph{may} conflict with this intuition.
% \end{note}

% \begin{note}
%   With the aid of \ros{1} we develop variations of \qWhy{} and \qHow{}:

%   \begin{itemize}
%   \item
%     Interpret `why' from \qWhy{} in terms of the \ros{1} the \agents{} conclusion of \(\pv{\phi}{v}\) from \(\Phi\) depended on.
%   \item
%     Interpret `how' from \qHow{} in terms of events which \wit{0} any \ros{} that the \agents{} conclusion of \(\pv{\phi}{v}\) from \(\Phi\) depended on.
%   \end{itemize}

%   % So, the variation to \qWhy{} is expected to be extensionally adequate for:

%   % If conclusion does not depend on \ros{}, then plausible that it is possible to answer \qWhy{} without citing the proposition-value-premises pairing.
%   % For, event would have occurred regardless of whether paired.

%   % If conclusion does depend on \ros{}, then proposition-value-premises pairing answers, in part, \qWhy{}.
%   % For, event would not have occurred regardless of whether paired.

%   % Variation to \qHow{}.
%   % Developed with respect to variation on \qWhy{}.

%   % If \ros{}, then if event \wit{} \ros{}, then of interest.
%   % If event which does not lead to \ros{}, then event is of no interest.
% \end{note}

% \begin{note}
%   An significant consequence of both variations will be as follows:

%   \begin{itemize}
%   \item
%     When an agent concludes \(\pv{\phi}{v}\) from \(\Phi\):
%     \begin{itemize}
%     \item
%       The \agents{} conclusion of \(\pv{\phi}{v}\) from \(\Phi\) depends on a \ros{0} between \(\pv{\phi}{v}\) and \(\Phi\) holding, for the agent.
%     \item
%       The event in which the agent concludes \(\pv{\phi}{v}\) from \(\Phi\) serves as a \wit{0} for the \ros{0} between \(\pv{\phi}{v}\) and \(\Phi\).
%     \end{itemize}
%   \end{itemize}
%   Hence, a \ros{} between \(\pvp{\phi}{v}{\Phi}\) will always be, in part, an answer to the variation of \qWhy{} and the event in which the agent concludes \(\pv{\phi}{v}\) from \(\Phi\) will always be, in part, an answer to the variation of \qHow{}.
% \end{note}

% \begin{note}
%   A variant to \issueInclusion{} follows from the variations to \qWhy{} and \qHow{}.
%   Roughly:

%   \begin{itemize}
%   \item
%     A conclusion of \(\pv{\phi}{v}\) from \(\Phi\) depends on some \ros{} between \(\pv{\psi}{v'}\) and \(\Psi\) holding for the agent

%     \emph{Only if}:

%     The agent has a \wit{} for the \ros{0} between \(\pv{\psi}{v'}\) and \(\Psi\).
%   \end{itemize}
% \end{note}

% \begin{note}
%   With an initial understanding of the variations to \qWhy{}, \qHow{}, and \issueInclusion{} in hand, we now return to the overall argument of this document.
%   Our goal is to develop counterexamples to \issueInclusion{}.
%   And, given the variation to \issueInclusion{} which follows from the variations to \qWhy{} and \qHow{}, we will do so by showing there are cases in which an agent concludes \(\pv{\phi}{v}\) from \(\Phi\) such that:
%   \begin{itemize}
%   \item
%     The agent pairing \(\phi\) and \(v\) depended on a \ros{0} between \(\pv{\psi}{v'}\) and \(\Psi\) holding, for the agent.
%   \item
%     The agent did not have a \wit{0} for the \ros{0} between \(\pv{\psi}{v'}\) and \(\Psi\) when the agent paired \(\phi\) and \(v\).
%   \end{itemize}
% \end{note}

% \begin{note}
  % As with \qWhyV{} our interest is with extensional adequacy, and specifically extensional adequacy with respect to \ros{1}.

  % Consider any event.
  % There are two cases.
  % \ros{} which answers \qWhyV{} or no \ros{}.

  % If no \ros{} then the \agents{} conclusion of \(\pv{\phi}{v}\) from \(\Phi\) does not depend on what happened.
  % Hence, event is not of direct interest with respect to answering `how' an agent concluded \(\pv{\phi}{v}\) from \(\Phi\).

  % If \ros{} which \qWhyV{}, then included as an answer to \qHowV{}.

  % So, though we have abstracted to \ros{1} to avoid any account of what conclusion amounts to, the existence of a \wit{0} intuitively captures whatever it is of relevance that happened when the agent concluded \(\pv{\phi}{v}\) from \(\Phi\).

  % And, as discussed in \autoref{cha:var:ros:W}, the existence of a \wit{0} allows for \ros{} which answer \qWhyV{} to be constrained by answers to \qHowV{}, even if the relevant \wit{0} occurs at some point prior to the event in which the agent concludes \(\pv{\phi}{v}\) from \(\Phi\).
% \end{note}

  % \begin{argument}{prop:phi-always-how}
  %   Suppose \(e\) is the event in which \vAgent{} concludes \(\pv{\psi}{v'}\) from \(\Psi\).
  %   By \autoref{prop:ros-always-answer} established that the \ros{} between \(\pv{\phi}{v}\) and \(\Phi\) is always an answer to \qWhyV{}.
  %   So, Clause~\ref{q:how:v:a} is satisfied.

  %   Likewise, by assumption \(e\) is the event in which \vAgent{} concludes \(\pv{\psi}{v'}\) from \(\Psi\).
  %   Hence, by \autoref{def:witnessing}, \(e\) is a \wit{} for the \ros{} between \(\pv{\psi}{v'}\) and \(\Psi\).
  %   So, Clause~\ref{q:how:v:b} is satisfied.

  %   And, as both Clause~\ref{q:how:v:a} and Clause~\ref{q:how:v:b} are satisfied, \(e\) is an answer to \qHowV{}.
  % \end{argument}

  % \begin{observation}
  %   \label{obs:qWhyV-description}
  %   Some \ros{1} matter.
  % \end{observation}
  % \begin{motivation}{obs:qWhyV-description}
  %   Situation.
  %   Flip coin.
  %   If heads, then by understanding of arithmetic.
  %   If tails, then by calculator.

  %   Now, coin lands heads.
  %   Well, sure, it seems this is needed.
  %   However, suppose coin lands tails.
  %   Then the agent would decide otherwise.
  %   So, really, the coin flip didn't matter.

  %   The point is that any event, and the description didn't rule out reversing.

  %   Does this matter?
  %   Not really, expand description.
  %   For any problematic event which differs from the way things are, rule out with description.
  % \end{motivation}

  % Whether there is something which rules out things being otherwise.

  % \begin{observation}
  %   \label{obs:do-mai}
  %   Some \ros{1} do not matter.
  % \end{observation}
  % \begin{motivation}{obs:do-mai}
  %   This tea tastes cold.

  %   Conclusion, so \ros{}.

  %   Consider event in which agent concludes.
  %   Unless deterministic and build in everything, then consider event in which agent doesn't take a sip.
  %   Does not conclude.
  %   Still concludes \(\pv{\phi}{v}\) from \(\Phi\).
  %   For, so much focus didn't even go for tea.
  % \end{motivation}
