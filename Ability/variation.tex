\chapter{Variant Questions}
\label{cha:var}

\begin{note}
  The overall goal of this document is to argue that \issueInclusion{} fails to hold.
  To do so we will provide a recipe for generating counterexamples to \issueInclusion{}.

  Given \issueInclusion{}, counterexamples take the form of some proposition-value-\pool{} pair which answers \qWhy{} but does not answer \qHow{}.

  \begin{quote}
    \textbf{Constraint 1 (\issueInclusion{})}.

    Given an agent \vAgent{}, proposition \(\phi\), value \(v\), and event \(e\) in which \vAgent{} concludes \(\phi\) has value \(v\):
    \begin{itemize}
    \item
      Some proposition-value-premises pair explains \emph{why} \(e\) is such that \vAgent{} concluded \(\phi\) has value \(v\).
    \end{itemize}
    \emph{Only if}:
    \begin{itemize}
    \item
      There is an event in which the agent concludes the proposition has the value from the pool of premises.
    \end{itemize}
  \end{quote}

  Still, both \qWhy{} and \qHow{} are broad questions which turn on the way in which the respective instances of `why' and `how' are understood.

  The function this chapter is to introduce variations on \qWhy{}, \qHow{}, and \issueInclusion{}.
  The role of the variations is to establish a (sufficiently) precise way in which a proposition-value-\pool{} pair may answer \qWhy{} without a corresponding answer to \qHow{}.
\end{note}

\begin{note}
  Do this by variant questions.

  \begin{itemize}
  \item
    Variant to \qWhy{} such that answers to \qWhyV{} are answers to \qWhy{}.
  \item
    Variant to \qHow{} such that answers to \qHow{} are answers to \qHowV{}.
  \end{itemize}

  If links hold, and \issueConstraint{} holds, then this entails variant constraint.
  Provide a recepie for generating counterexamples and then either links fail, or \issueConstraint{}.

  In this respect, variant questions will not be `developments' of \qWhy{} and \qHow{}.
  Instead, function is to get constraint.
\end{note}

\begin{note}
  Need four things:
  \begin{enumerate}[label=\alph*., ref=(\alph*)]
  \item
    \label{vars:constraint:int}
    Links hold.

    Hence, if variant constraint fails, this is due to failure of \issueInclusion{}.
  \item
    Sufficiently precise to allow clear account of the way in which failure occurs.
  \item
    \label{vars:constraint:ce}
    The variations to \qWhy{} and \qHow{} allow for the possibility for the variation of \issueInclusion{} to fail to hold.
  \item
    Theory neutral.
  \end{enumerate}

  Indeed, Constraint~\ref{vars:constraint:int} is a source of significant concern.
  For, \issueInclusion{} is plausibly doing work to narrow the relevant sense of why captured by \qWhy{}.

  In other words, there may be plausible answers to \qWhy{} which are not also answers to \qHow{} but fail to be counterexamples to \issueInclusion{} as the relevant sense of `why' and `how' are not the senses of `why' and `how' that \issueInclusion{} (intuitively) holds with respect to.

  And, as a constraint on answers to \qWhy{} in terms of \qHow{}, \issueInclusion{} may plausibly have a role in narrowing the relevant senses of `why' and `how' at issue.
  And, if \issueInclusion{} is rejected, then, naturally, \issueInclusion{} cannot perform this role.

  Further, to the extent that various theories of conclusion, or sufficiently related phenomena, either implicitly or explicitly endorse \issueInclusion{}, it should be the case that the same theories either implicitly or explicitly motivate the variation of \issueConstraint{} constraining answers to the variation of \qWhy{} in terms of answers to \qHow{}.
\end{note}

\begin{note}
  Key abstraction is \ros{1}.
\end{note}

\paragraph*{Outline}

\begin{note}
  This chapter is divided into two main sections:
  \begin{enumerate}[label=, leftmargin=*]
  \item
    \TOCLine{cha:var:ros}

    Detailed discussion of \ros{1}, foundation of how we develop variations.
    Includes initial discussion of \fc{1}.
  \item
    \TOCLine{cha:var:sec:vars}

    Variations to \qWhy{}, \qHow{}, and \issueInclusion{} with discussion.
  \item
    % \TOCLine{cha:clar:sec:literature}

    % Look at how to connect variant to various account of concluding, and related phenomena, from the literature.
  \end{enumerate}
\end{note}

% \section{The role of variant questions}
% \label{cha:var:sec:wiggling}

% \begin{note}
%   \autoref{cha:introduction} introduced \qWhy{}, \qHow{}, and \issueInclusion{}.


%   The overall goal of this document is to argue \issueInclusion{} does not hold.

%   Hence, without establishing a clear understanding of the way in which the instances are to be understood, it is unclear how to develop counterexamples to \issueInclusion{}.
% \end{note}

% \begin{note}
%   In broad outline, we use the idea of a `\ros{0}' to provide variations on \qWhy{}, \qHow{}, and \issueInclusion{}.
%   The way in which we understand \ros{1} is minimal and tightly connected to an event in which an agent concludes \(\pv{\phi}{v}\) from \(\Phi\).
%   Specifically, we put forward three ideas in relation to \ros{1}:
%   \begin{enumerate}
%   \item
%     If an agent concludes \(\pv{\phi}{v}\) from \(\Phi\), then a \ros{0} between \(\pv{\phi}{v}\) and \(\Phi\), for the agent, when the agent pairs \(\phi\) with \(v\).
%   \item
%     If an agent has concluded \(\pv{\phi}{v}\) from \(\Phi\), then the event in which the agent concluded \(\pv{\phi}{v}\) from \(\Phi\) functions as a \wit{0} for a \ros{0} between \(\pv{\phi}{v}\) and \(\Phi\).
%   \item
%     It is possible for a \ros{0} between \(\pv{\phi}{v}\) and \(\Phi\) to hold, from an \agpe{}, without their being an \wit{0} for the \ros{0}.
%   \end{enumerate}

%   \autoref{cha:var:ros} will develop and discuss each idea in detail.

%   For the moment, the motivation for abstracting to \ros{1} is to capture, in a abstract way, the way in which \(\pv{\phi}{v}\) and \(\Phi\) are related from an \agpe{} when the agent concludes \(\pv{\phi}{v}\) from \(\Phi\).

%   Rather than directly capturing some relevant sense of `why' or `how' our goal is to use \ros{1} to construct variations on \qWhy{} and \qHow{} which are \emph{roughly} `extensionally adequate'.
%   Where, we understand the term `extensionally adequate' more-or-less in line with \citeauthor{Sumner:1987aa} (\citeyear{Sumner:1987aa}):

%   \begin{quote}
%     [A] conception of a concept is extensionally adequate when it includes every item which seems pre-analytically to be an instance of the concept and excludes every item which does not.%
%     \mbox{ }\hfill\mbox{(\citeyear[49]{Sumner:1987aa})}
%   \end{quote}

%   Adapted to our case, our interest with \qWhy{} and \qHow{} is with respect to intuitive answers to \qWhy{} and \qHow{} (and in particular the intuition that \issueInclusion{} holds).
%   And, the variations of both \qWhy{} and \qHow{} may be seen as `conceptions of a question' such that any answer to \qWhy{} is an answer to the variation of \qWhy{}, vice-versa, and the same with respect to \qHow{}.

%   Still, the way in which something answers \qWhy{} need not be equivalent to the way in which that thing answers the variation to \qWhy{}.%
%   \footnote{
%     In this respect, the variation to \qWhy{} need not be \emph{intensionally} adequate.
%     Where the variation of \qWhy{} (or \qHow{}) would be intensionally adequate just in case the variation captured the way something \emph{intuitively} answers \qWhy{}.
%   }

%   And, we are only interested in `rough' extensional adequacy.
%   In particular, we are only interested in answers to \qWhy{} and \qHow{} to the extent that \issueInclusion{} plausibly holds.
%   Hence, we will ignore intuitive answers to \qWhy{} and \qHow{} which extend beyond \issueInclusion{}.

%   Further, to the extent that \issueInclusion{} is intuitive, the variations \emph{may} conflict with this intuition.
% \end{note}

% \begin{note}
%   With the aid of \ros{1} we develop variations of \qWhy{} and \qHow{}:

%   \begin{itemize}
%   \item
%     Interpret `why' from \qWhy{} in terms of the \ros{1} the \agents{} conclusion of \(\pv{\phi}{v}\) from \(\Phi\) depended on.
%   \item
%     Interpret `how' from \qHow{} in terms of events which \wit{0} any \ros{} that the \agents{} conclusion of \(\pv{\phi}{v}\) from \(\Phi\) depended on.
%   \end{itemize}

%   % So, the variation to \qWhy{} is expected to be extensionally adequate for:

%   % If conclusion does not depend on \ros{}, then plausible that it is possible to answer \qWhy{} without citing the proposition-value-premises pairing.
%   % For, event would have occurred regardless of whether paired.

%   % If conclusion does depend on \ros{}, then proposition-value-premises pairing answers, in part, \qWhy{}.
%   % For, event would not have occurred regardless of whether paired.

%   % Variation to \qHow{}.
%   % Developed with respect to variation on \qWhy{}.

%   % If \ros{}, then if event \wit{} \ros{}, then of interest.
%   % If event which does not lead to \ros{}, then event is of no interest.
% \end{note}

% \begin{note}
%   An significant consequence of both variations will be as follows:

%   \begin{itemize}
%   \item
%     When an agent concludes \(\pv{\phi}{v}\) from \(\Phi\):
%     \begin{itemize}
%     \item
%       The \agents{} conclusion of \(\pv{\phi}{v}\) from \(\Phi\) depends on a \ros{0} between \(\pv{\phi}{v}\) and \(\Phi\) holding, for the agent.
%     \item
%       The event in which the agent concludes \(\pv{\phi}{v}\) from \(\Phi\) serves as a \wit{0} for the \ros{0} between \(\pv{\phi}{v}\) and \(\Phi\).
%     \end{itemize}
%   \end{itemize}
%   Hence, a \ros{} between \(\pvp{\phi}{v}{\Phi}\) will always be, in part, an answer to the variation of \qWhy{} and the event in which the agent concludes \(\pv{\phi}{v}\) from \(\Phi\) will always be, in part, an answer to the variation of \qHow{}.
% \end{note}

% \begin{note}
%   A variant to \issueInclusion{} follows from the variations to \qWhy{} and \qHow{}.
%   Roughly:

%   \begin{itemize}
%   \item
%     A conclusion of \(\pv{\phi}{v}\) from \(\Phi\) depends on some \ros{} between \(\pv{\psi}{v'}\) and \(\Psi\) holding for the agent

%     \emph{Only if}:

%     The agent has a \wit{} for the \ros{0} between \(\pv{\psi}{v'}\) and \(\Psi\).
%   \end{itemize}
% \end{note}

% \begin{note}
%   With an initial understanding of the variations to \qWhy{}, \qHow{}, and \issueInclusion{} in hand, we now return to the overall argument of this document.
%   Our goal is to develop counterexamples to \issueInclusion{}.
%   And, given the variation to \issueInclusion{} which follows from the variations to \qWhy{} and \qHow{}, we will do so by showing there are cases in which an agent concludes \(\pv{\phi}{v}\) from \(\Phi\) such that:
%   \begin{itemize}
%   \item
%     The agent pairing \(\phi\) and \(v\) depended on a \ros{0} between \(\pv{\psi}{v'}\) and \(\Psi\) holding, for the agent.
%   \item
%     The agent did not have a \wit{0} for the \ros{0} between \(\pv{\psi}{v'}\) and \(\Psi\) when the agent paired \(\phi\) and \(v\).
%   \end{itemize}
% \end{note}

\section{\ros{3}}
\label{cha:var:ros}

\begin{note}
  The present section develops and discusses \ros{1} in detail.

  The role of \ros{1} within this document is to capture, in a abstract way, the way in which a proposition-value pair \(\pv{\phi}{v}\) and \pool{} \(\Phi\) are related from an \agpe{} when the agent concludes \(\pv{\phi}{v}\) from \(\Phi\).
\end{note}

\begin{note}
  Our understanding of a `\ros{1}' is given in terms of three ideas, and subsections will develop and discuss each idea in detail:

  \begin{enumerate}[label=, leftmargin=*]
  \item
    \TOCLine{cha:var:ros:I}

    An event in which an agent concludes \(\pv{\phi}{v}\) from \(\Phi\) is sufficient for a \ros{} to hold, for the agent.
  \item
    \TOCLine{cha:var:ros:W}

    An event in which an agent concludes \(\pv{\phi}{v}\) from \(\Phi\) provides an agent with a \wit{0} for a \ros{}.
  \item
    \TOCLine{cha:var:ros:II}

    It is possible for a \ros{} to hold, from an \agpe{} without the agent having a \wit{} for the \ros{}.
  \end{enumerate}
\end{note}

\begin{note}
  We speak in terms of \ros{1} holding from an \agpe{}.
  However, given some agent \vAgent{}, proposition \(\phi\), value \(v\), and \pool{} \(\Phi\), we do not draw any particular distinction between:

  \begin{enumerate}[label=\alph*., ref=(\alph*)]
  \item
    \label{ros:ap:maybe:a}
    A \ros{} between \(\pv{\phi}{v}\) and \(\Phi\) holds, from an \agpe{}.
  \item
    \label{ros:ap:maybe:b}
    The pairing of the proposition `A \ros{} holds between \(\pv{\phi}{v}\) and \(\Phi\)' with the value `True' by the agent.

    I.e.\ the proposition-value pair:\newline
    \mbox{ }\hfill%
    \(\pv{\text{A \ros{} holds between } \pv{\phi}{v}\text{ and }\Phi}{\text{True}}\)
  \end{enumerate}

  Any significant distinction between~\ref{ros:ap:maybe:a} and~\ref{ros:ap:maybe:b} would turn on details too specific for the degree of abstraction we target.
  For, what holds from an \agpe{} may just amount to which propositions are paired with the value `True' by the agent.
\end{note}

\subsection{\supportI{}}
\label{cha:var:ros:I}

\begin{note}
  \supportI{} states, roughly, that the event in which an agent concludes \(\pv{\phi}{v}\) from \(\Phi\) is sufficient for a \ros{} to hold between \(\pv{\phi}{v}\) and \(\Phi\).

  \begin{idea}[\supportI{}]
    \label{idea:support}
    For an agent \vAgent{}, a proposition-value pair \(\pv{\phi}{v}\), \pool{} \(\Phi\), and event \(e\):

    \begin{itemize}
    \item[\emph{If}:]
      \begin{enumerate}[label=\alph*., ref=(\alph*)]
      \item
        \(e\) is an event in which \vAgent{} concludes \(\pv{\phi}{v}\) from \(\Phi\).
      \end{enumerate}
    \item[\emph{Then}:]
      \begin{enumerate}[label=\alph*., ref=(\alph*), resume]
      \item
        When \vAgent{} pairs \(\phi\) with \(v\) as a sub-event of \(e\):
        \begin{itemize}
        \item
          A \emph{\ros{}} between \(\pv{\phi}{v}\) and \(\Phi\) holds, from \agpe{\vAgent{}'}.
        \end{itemize}
      \end{enumerate}
    \end{itemize}
    \vspace{-\baselineskip}
  \end{idea}

  The focus on the sub-event in which the agent pairs \(\phi\) with \(v\) is to allow for the \ros{} to, in part, explain why the agent concludes \(\pv{\phi}{v}\) from \(\Phi\) without requiring that the \ros{} holds, for the agent prior to the agent forming the conclusion that \(\phi\) has value \(v\).

  In this respect, a \ros{} between \(\pv{\phi}{v}\) and \(\Phi\) may be regarded as a static account of how the agent has come to pair \(\phi\) with \(v\).
  In other words, the \ros{} between \(\pv{\phi}{v}\) and \(\Phi\) just captures whatever it is, for the agent, that led to the agent concluding \(\pv{\phi}{v}\) from \(\Phi\).

  Still, \supportI{} is only a sufficient condition, and the suggestion --- however the details work out --- is intended as intuition for when the agent concludes \(\pv{\phi}{v}\) from \(\Phi\).
  As we will see when discussing \supportII{}, we will deny the converse of \supportI{}.
  Therefore, the intuition is not suitable to capture in general what a \ros{} holding, from an \agpe{}, amounts to.

  Generalised, what it is, that has, is, or will, relate \(\pv{\phi}{v}\) and \(\Phi\), for the agent.
\end{note}

\begin{note}
  Given \supportI{}, we will always qualify that a \ros{} holds \emph{from an \agpe{}}.
  Our interest is with conclusions, parings of propositions with values by an agent.
  Hence, we have no interest in whatever the idea of a \ros{} between \(\pv{\phi}{v}\) and \(\Phi\) simpliciter.
  As outlined in \autoref{cha:clar:sec:CCC:pvp}, we place no restrictions on conclusions.
  Hence, if an agent concludes that the ratio of the long side to the short side of a piece of paper is not \(\sqrt{2}\) from some \pool{} \(\Phi\), then, by \supportI{}, a \ros{} holds between:
  \(\pv{\text{The ratio of the long side to the short side of a piece of paper is not }\sqrt{2}}{\text{True}}\) and \(\Phi\).
\end{note}

\begin{note}
 \supportI{} is similar to, but designed to be distinct from,~\citeauthor{Boghossian:2014aa}'s Taking Condition:%
  \footnote{
    There are various objections to~\citeauthor{Boghossian:2014aa}'s Taking Condition, though we take no stance on whether~\citeauthor{Boghossian:2014aa}'s Taking Condition holds.

    See, for example,~\textcite{Hlobil:2014tq}, \textcite{McHugh:2016vp}, and~\textcite{Wright:2014tt}.

    \citeauthor{Hlobil:2014tq} argues against the Taking Condition as it distracts from what accounts of reasoning out to explain, rather than arguing against the Taking Condition directly.

    \citeauthor{McHugh:2016vp} summarise various objects to the taking condition, and present district arguments against against (distinct) ideas in favour of the taking condition.
    In particular,~\supportI{} is closer to what \citeauthor{McHugh:2016vp} term the `Consequence Condition' (\citeyear[cf.][316]{McHugh:2016vp}), which \citeauthor{McHugh:2016vp} also (indirectly) argue against.
    However, \citeauthor{McHugh:2016vp} does not consider an alternative account of what distinguishes concluding from any other action, and as~\supportI{} is designed to capture this distinction, it is unclear to me whether \citeauthor{McHugh:2016vp}'s arguments apply to~\supportI{} (if, indeed, they are sound).

    \citeauthor{Wright:2014tt} denies that reasoning must involve a state which connects premises to conclusions, as discussed in the main body of this section. (\citeyear[Cf.][33-34]{Wright:2014tt})
  }

  \begin{quote}
    (Taking Condition):
    Inferring necessarily involves the thinker \emph{taking} his premises to support his conclusion and drawing his conclusion because of that fact.%
    \mbox{}\hfill\mbox{(\citeyear[5]{Boghossian:2014aa})}
  \end{quote}

  There is an immediate superficial difference in that~\citeauthor{Boghossian:2014aa} states the Taking Condition in terms of inferring.
  However, `a conclusion' may be substituted for `inferring' and an important distinction remains.
  For, `taking' is understood by \citeauthor{Boghossian:2014aa} to be {
    \color{red}
    explanatory.

    However, for our purposes, \ros{1} will not have a direct explanatory role.
  Instead, \ros{1} are an abstraction.
}
\end{note}


\begin{note}
  \citeauthor{Boghossian:2014aa} illustrates with the following \scen{}:
  \begin{quote}
    On waking up one morning I recall that:

    \begin{enumerate}[label=(\arabic*), ref=(\arabic*), series=BogEx]
    \item
      \label{BogEx:1}
      It rained last night.
    \end{enumerate}

    I combine this with my knowledge that

    \begin{enumerate}[label=(\arabic*), ref=(\arabic*), resume*=BogEx]
    \item
      \label{BogEx:2}
      If it rained last night, then the streets are wet.
    \end{enumerate}

    to conclude:

    So,

    \begin{enumerate}[label=(\arabic*), ref=(\arabic*), resume*=BogEx]
    \item
      \label{BogEx:3}
      The streets are wet.
    \end{enumerate}
    This belief then affects my choice of footwear.%
    \mbox{ }\hfill\mbox{(\citeyear[2]{Boghossian:2014aa})}
  \end{quote}

  And \citeauthor{Boghossian:2014aa} expands as follows:

  \begin{quote}
    [M]y inferring from~\ref{BogEx:1} and~\ref{BogEx:2} to~\ref{BogEx:3} must involve my arriving at the judgment that~\ref{BogEx:3} in part \emph{because} I \emph{take} the presumed truth of~\ref{BogEx:1} and~\ref{BogEx:2} to provide support for~\ref{BogEx:3}.
    Let us call this insistence that an account of inference must in this way incorporate a notion of ``taking'' the Taking Condition on inference.%
    \mbox{ }\hfill\mbox{(\citeyear[4]{Boghossian:2014aa})}
  \end{quote}

  Hence, for \citeauthor{Boghossian:2014aa}, the Taking Condition captures something \emph{in addition} to~\ref{BogEx:3} being a conclusion from a \pool{} which includes~\ref{BogEx:1} and~\ref{BogEx:2}.
  The presence of `taking' has a distinctive role in classifying the move from~\ref{BogEx:1} and~\ref{BogEx:2} to~\ref{BogEx:3} as an inference (or as a conclusion).

  In contrast, we do not require that a \ros{} has any particular role \emph{for the agent} in event in which an agent concludes \(\pv{\phi}{v}\) from \(\Phi\).
  If an agent concludes~\ref{BogEx:3} from~\ref{BogEx:1} and~\ref{BogEx:2}, then a \ros{} holds between~\ref{BogEx:3} and \(\{\ref{BogEx:1}, \ref{BogEx:2}\}\), for the agent ---~\ref{BogEx:3} and \(\{\ref{BogEx:1}, \ref{BogEx:2}\}\) are related, in some way by the agent.
  However, the \ros{} between~\ref{BogEx:3} and \(\{\ref{BogEx:1}, \ref{BogEx:2}\}\) need not itself have a role in the \agents{} conclusion of~\ref{BogEx:3} from \(\{\ref{BogEx:1}, \ref{BogEx:2}\}\).%
  \footnote{
    Also, about the type of reasoning by which the agent concludes.
    This comes from \textcite{Boghossian:2008vf,Boghossian:2012vb}.
    Rule following, taking gets account of rule.
  }
\end{note}

\begin{note}
  \phantlabel{Wright-simple-supportI}
  Indeed, our intuitive understanding of \ros{} is close to \citeauthor{Wright:2014tt}'s (\citeyear{Wright:2014tt}) `Simple Proposal':
  \begin{quote}
    [C]onsider instead the proposal, not that the status of the transition as inferential depends on the thinker's judgments about his reasons, but that it depends on \emph{what his reasons are}.
    We want his acceptance of the premises to supply his \emph{actual} reasons for accepting the conclusion.
    \dots

    Call this the Simple Proposal.
    It says that a thinker infers q from p\(_{1}\) \(\cdots\) p\(_{\text{n}}\) when he accepts each of p\(_{1}\) \(\cdots\) p\(_{\text{n}}\), moves to accept q, and does so for the reason that he accepts p\(_{1}\) \(\cdots\) p\(_{\text{n}}\).%
    \mbox{}\hfill\mbox{(\citeyear[33]{Wright:2014tt})}
  \end{quote}

  \citeauthor{Wright:2014tt}'s simple proposal is that, for the agent, the relation between a conclusion and some \pool{} need not be part of what moves the agent to conclude the conclusion from the \pool{}.

  \begin{quote}
    What is needed, then, is an account of, or at least some insight into, what it is for certain intentional states of a thinker to be his actual reasons for his transition to another intentional state.
    \dots
    We need to avoid committing to the notion that doing something for certain reasons must involve a state that somehow registers those reasons as reasons for what one does.%
    \mbox{}\hfill\mbox{(\citeyear[34]{Wright:2014tt})}
  \end{quote}

  Still, anticipating the role of \ros{} in construction a variation to \qWhy{}, it intuitively remains the case that a \ros{} explains, in part, why an agent concluded the conclusion from the \pool{} for, the \ros{} captures \emph{that} the agent accepted each of the premises and moved to accept the conclusion (in \citeauthor{Wright:2014tt}'s terminology).

  Still, following the discussion above, there is an important between~\supportI{} and \citeauthor{Wright:2014tt}'s Simple Proposal.
  For,~\supportI{} is an entailment, while \citeauthor{Wright:2014tt}'s Simple Proposal is an identity statement.
  Inferring, on the Simple Proposal, is an agent accepting some conclusion for the reason that they accept premises from some \pool{}.
  \supportI{} does not entail that concluding is nothing more than moving to accept \(\pv{\phi}{v}\) as a result of accepting each element of \(\Phi\).
\end{note}

\subsection{\wit{3} for \ros{1}}
\label{cha:var:ros:W}

\begin{note}
  \autoref{cha:var:ros:I} introduced a sufficient condition for a \ros{} between \(\pv{\phi}{v}\) and \(\Phi\) to hold, from an \agpe{}:
  The agent concluded \(\pv{\phi}{v}\) from \(\Phi\).

  In general, if an agent has concluded \(\pv{\phi}{v}\) from \(\Phi\), then we will say the agent has a \wit{} for the \ros{} between \(\pv{\phi}{v}\) and \(\Phi\).
  In full:

  \begin{definition}[A \wit{2} for a \ros{0}]
    \label{def:witnessing}
    For an agent \vAgent{}, proposition-value pair \(\pv{\phi}{v}\), and \pool{} \(\Phi\):

    \begin{enumerate}[label=]
    \item
      \begin{enumerate}[label=\alph*., ref=(\alph*), series=WitnessDef]
      \item
        \(e\) is \emph{\wit{0}} for \ros{} between \(\pv{\phi}{v}\) and \(\Phi\), for \vAgent{}.
      \end{enumerate}
    \item
      \emph{If and only if:}
    \item
      \begin{enumerate}[label=\alph*., ref=(\alph*), resume*=WitnessDef]
      \item
        \(e\) is an event in which \vAgent{} concludes \(\pv{\phi}{v}\) from \(\Phi\).
      \end{enumerate}
    \end{enumerate}
    \vspace{-\baselineskip}
  \end{definition}

  Shorthand, we say that \emph{an agent has} a \wit{0} for the relevant \ros{}.
\end{note}

\begin{note}
  An important, but trivial, case of \autoref{def:witnessing} is when an agent concludes \(\pv{\phi}{v}\) from \(\Phi\).
  For, if an agent concludes \(\pv{\phi}{v}\) from \(\Phi\) then it is immediate that there is some event in which the agent concludes \(\pv{\phi}{v}\) from \(\Phi\) --- the very same event --- and hence the agent has a \wit{} for the \ros{} between \(\pv{\phi}{v}\) and \(\Phi\).

  Hence, joining \supportI{} with \autoref{def:witnessing}, we have the following:

  \begin{proposition}[Concludes, then witnessed \support{}]
    \label{prop:cws}
    For an agent \vAgent{}, proposition-value pair \(\pv{\phi}{v}\) and \pool{} \(\Phi\):
    \begin{itemize}
    \item
      If \(e\) is an event in which \vAgent{} concludes \(\pv{\phi}{v}\) from \(\Phi\) then:
      \begin{itemize}
      \item
        When \vAgent{} pairs \(\phi\) with \(v\) as a sub-event of \(e\), a \ros{} between \(\pv{\phi}{v}\) and \(\Phi\) holds, from \agpe{\vAgent{}'}.
      \item
        There is an event \(e'\) such that \(e'\) is a \wit{} for the \ros{} between \(\pv{\phi}{v}\) and \(\Phi\).
      \end{itemize}
    \end{itemize}
    \vspace{-\baselineskip}
  \end{proposition}

  \begin{argument}{prop:cws}
    Immediate for assuming the antecedent and appealing to \supportI{} and \autoref{def:witnessing}, respectively.
  \end{argument}

  \autoref{prop:cws} is of interest with respect to \qWhy{}, \qHow{}, \issueInclusion{}, and the variations to follow in \autoref{cha:var:sec:vars}.

  For, our variant to \qWhy{} will involve \ros{1}.
  Our variant to \qHow{} will involve \wit{1}.
  And, our variant to \issueInclusion{} will hold that a \ros{} is, in part, an answer to why an agent concluded only if the agent has a \wit{} for the \ros{}.

  Hence, \autoref{prop:cws} ensures that so long as there is an event in which the agent concludes \(\pv{\phi}{v}\) from \(\Phi\), then an answer to `why' will always have a corresponding answer to `how'.

  At issue is whether it is always the case that an agent has a \wit{} for a \ros{} which is, in part, an answer to why the agent concluded \(\pv{\phi}{v}\) from \(\Phi\).

  And, given \autoref{prop:cws} it is immediate that any such \ros{} must be distinct from the \ros{} between \(\pv{\phi}{v}\) and \(\Phi\).
\end{note}

\begin{note}
  Note, when we talk of \wit{1} we talk in terms of `having a \wit{0}'.
  In the case of \autoref{prop:cws}, the event in which the agent concludes and the event which secures the relevant \wit{} are identical.

  However, event \(e\) may be an event in which an agent concludes \(\pv{\phi}{v}\) from \(\Phi\) such that throughout the event \(e\), the agent has a \wit{} for a \ros{} between \(\pv{\psi}{v'}\) and \(\Psi\), such that the event \(e'\) which \wit{1} the \ros{} between \(\pv{\psi}{v'}\) and \(\Psi\) is distinct from \(e\).

  Hence, our understanding of `having a \wit{0}' allows for the possibility that some \ros{} between \(\pv{\psi}{v'}\) and \(\Psi\), in part, `answers why' an agent concludes \(\pv{\phi}{v}\) from \(\Phi\) though the relevant \wit{0} for the \ros{} between \(\pv{\psi}{v'}\) and \(\Psi\) is distinct.

  If you think there may be such cases, then the variant to \issueInclusion{} that we develop will be compatible with such cases.
  And, if you think there are no such cases, then it is safe to ignore this possibility.
  We will not directly, at least, consider such cases or take a stand either way in the main argument.%
  \footnote{
    \phantlabel{fn:past-witness}
    To illustrate, consider an agent working on some mathematical problem.

    As part of their work on the problem the agent concludes the hypotenuse of some right-angled triangle is \(\sqrt{74}\text{cm}\) by use of the Pythagorean theorem.

    Further, the agent has, at some point in the past proved the Pythagorean theorem from more basic principles.

    Now, generally speaking, it may be the case that the agent concludes the hypotenuse of the triangle is \(\sqrt{74}\text{cm}\), in part, from those more basic principles.
    For example, the agent may have just completed their proof of the Pythagorean theorem and the reasoning from the more basic principles to the hypotenuse of the triangle may be considered a single unified instances of reasoning, with an intermediary conclusion.

    Further, suppose the agent proved the Pythagorean theorem some years ago.

    Perhaps the \agents{} reasoning from more basic principles continues to provide, in part, an answer to how the agent concluded the hypotenuse of the triangle is \(\sqrt{74}\text{cm}\).
    Perhaps, regardless of the gap, the agent used the Pythagorean theorem \emph{because} they concluded the theorem from more basic principles.

    On the other hand, one may be inclined to hold that the more basic principles have no role explanatory role in the present.
    At best, the \agents{} \emph{memory} of --- rather than the event of --- concluding answers, in part, why the agent concluded hypotenuse of the triangle is \(\sqrt{74}\text{cm}\).
  }
\end{note}

\begin{note}
  Though we will not take a stand on whether a relevant \wit{0} for some conclusion is distinct from the event in which the agent concludes, the possibility of separation highlights an plausible issue with \autoref{def:witnessing}.

  For, if separation may occur, it seems there may be instances where an agent reasoned to \(\pv{\phi}{v}\) but did not conclude \(\phi\) has value \(v\) such that the event in which the agent reasoned to \(\pv{\phi}{v}\) serves as a \wit{0} to a \ros{} between \(\pv{\phi}{v}\) and \(\Phi\).

  As \autoref{def:witnessing} requires the event to be such that the agent concludes \(\pv{\phi}{v}\) from \(\Phi\), such events are excluded from being \wit{1}.

  To illustrate, consider an agent working through a proof of some theorem.

  Abstractly, let \(\theta\) be the state of affairs characterised by the theorem, and let \(\Theta\) be the relevant \pool{}.
  Our interest is with the conclusion \(\pv{\theta}{\text{True}}\) from \(\Theta\).

  Suppose the agent reasons to \(\pv{\theta}{\text{True}}\) from \(\Theta\).
  Further, suppose the \agents{} reasoning is sound.
  However, the agent is worried about some parts of their reasoning.
  Hence, given their worries, \emph{reasons} to --- but does not conclude --- \(\pv{\theta}{\text{True}}\) from \(\Theta\).

  Some time later the agent revisits proof, resolves their worries, and concludes the theorem is true.

  I see no issue with the \emph{idea} that:
  \begin{itemize}[noitemsep]
  \item
    When the agent revisited the proof, they concluded \(\pv{\theta}{\text{True}}\) from \(\Theta\).
  \item
    In part, a \ros{} between \(\pv{\theta}{\text{True}}\) and \(\Theta\), for the agent, answers why the agent concluded \(\pv{\theta}{\text{True}}\) from \(\Theta\).
  \item
    The event in which the agent reasoned to \(\pv{\theta}{\text{True}}\) from \(\Theta\) answers, in part, how the agent \(\pv{\theta}{\text{True}}\) from \(\Theta\) by being a \wit{} for the \ros{} between \(\pv{\theta}{\text{True}}\) and \(\Theta\).
  \end{itemize}

  However, the idea is incompatible with the way we understand a \wit{0}.
  For, by definition, the relevant event which serves as a \wit{0} must be an event in which the agent \emph{concludes} \(\pv{\theta}{\text{True}}\) from \(\Theta\).
  And, by construction of the \scen{0}, the \agents{} worries prevent the agent from forming the relevant conclusion.

  There are various ways to square the \scen{0} with our understanding of a \wit{0}.
  For example, one may consider the extended event in which the agent reasons, returns, and concludes.
  Or, one may hold that when the agent concluded \(\pv{\theta}{\text{True}}\), the agent concluded \(\pv{\theta}{\text{True}}\) not from \(\Theta\), but from some \pool{} \(\Theta'\) which include the adequacy of the \agents{} prior reasoning as a premise.

  Still, it is not clear to me that either of the options suggested --- or any other option --- is preferable to weakening \autoref{def:witnessing} in such a way that an event \(e\) which serves as a \wit{} to some \ros{} falls short of being an event in which an agent concludes.

  The difficulty is providing an adequate characterisation of the relevant event.
  That the agent \emph{reasoned} to \(\pv{\theta}{\text{True}}\) from \(\Theta\) is insufficient in general.

  For example, consider a variation of the \scen{} in which the agent identifies a problem with the proof.
  Given the presence of a problem, there is --- intuitively --- no \ros{} for the agent to have a \wit{0} for.

  Maintaining (some) intuition with regards to what it is for an event to be a \wit{0} for a \ros{0} is our priority.
  Strictly, the way in which we put \autoref{def:witnessing} to work is fully compatible with substituting `reasons to' in place of `concludes', but an overly narrow definition is preferably to an unintuitive definition.
  % Hence, in order to avoid a lengthy digression into sufficient conditions for an event to `\wit{0}' a \ros{} without the event being such that the agent concludes we simply require the event is such that the agent concludes.
\end{note}

\subsection{\supportII{}}
\label{cha:var:ros:II}

\begin{note}
  \supportI{} states that a \ros{} holds between \(\pv{\phi}{v}\) and \(\Phi\), for the agent, when an agent concludes \(\pv{\phi}{v}\).
  \supportII{}, in short, denies the converse of \supportI{} is necessarily the case:

  \begin{idea}[\supportII{}]
    \label{idea:support:possible}
    For an agent \vAgent{}, a proposition-value pair \(\pv{\phi}{v}\), and \pool{} \(\Phi\):

    \begin{itemize}
    \item
      It is possible for both~\ref{idea:support:possible:a} and~\ref{idea:support:possible:b} be true:
      \begin{enumerate}[label=\alph*., ref=(\alph*)]
      \item
        \label{idea:support:possible:a}
        A \ros{} between \(\pv{\phi}{v}\) and \(\Phi\) holds, from \agpe{\vAgent{}'}.
      \item
        \label{idea:support:possible:b}
        \vAgent{} does not have a \wit{} for the \ros{} between \(\pv{\phi}{v}\) and \(\Phi\).
      \end{enumerate}
    \end{itemize}
    \vspace{-\baselineskip}
  \end{idea}

  If an agent does not have a \wit{} for a \ros{} between \(\pv{\phi}{v}\) and \(\Phi\), then there is no event in which the agent concludes \(\pv{\phi}{v}\) from \(\Phi\).
  So, \supportI{} states that an event in which the agent concludes \(\pv{\phi}{v}\) from \(\Phi\) is sufficient for a \ros{} between \(\pv{\phi}{v}\) and \(\Phi\) to hold, for the agent.
  In contrast, \supportII{} denies that an event in which the agent concludes \(\pv{\phi}{v}\) from \(\Phi\) is necessary for a \ros{} between \(\pv{\phi}{v}\) and \(\Phi\) to hold, for the agent.
\end{note}

\begin{note}
  \supportII{} has a key role in the overall argument for this document.
  For, as indicated, answers to the variant of \qWhy{} will concern \ros{}, and the variant of \qHow{} will concern whether the agent has a \wit{} for the relevant \ros{}.
  \supportII{}, then, allows for the \emph{possibility} that the kind of thing which answers, in part, why an agent concluded is not constrained by how the agent concluded.

  However, our motivation for \supportII{} is independent of the success of the overall argument of this document.

  In short, any constraint on answers to why an agent concludes by answers to how an agent concludes is substantial.
  Denying that there is any instance in which a \ros{} may answer, in part, why an agent concluded \(\pv{\phi}{v}\) from \(\Phi\) without the agent having a \wit{} for the \ros{} amounts to a substantive constraint.

  Indeed, \supportII{} should not be of any immediate concern.
  For, there is a significant gap between:

  \begin{itemize}[noitemsep]
  \item
    There being a \ros{} between \(\pv{\psi}{v'}\) and \(\Psi\) from an \agpe{} without the agent having a \wit{} for the \ros{}.
  \item
    The \ros{} between \(\pv{\psi}{v'}\) and \(\Psi\), for the agent, answering, in part, and in some sense, why the agent concluded \(\pv{\phi}{v}\) from \(\Phi\).
  \end{itemize}
\end{note}

\subsection{Summary}
\label{cha:var:ros:summary}

\begin{note}
  \supportI{}.

  \wit{3}.

  \supportII{}.
\end{note}

\paragraph{Basing}

\begin{note}
  Think of \ros{} in terms of doxastic justification.

  \begin{quote}
    S's belief that p is doxastically justified (i.e. S's belief is held in an epistemically permissible fashion) if and only if S believes p in the right kind of way, on an epistemically appropriate basis.%
    \mbox{ }\hfill\mbox{(\citeyear{Bondy:2018tk})}
  \end{quote}

  However, there are difficulties:

  From \supportII{}, \ros{} without conclusion.

  This means don't need to believe, at least with understanding on which beliefs are sort of explicit.

  This may suggest propositional justification.%
  \footnote{
    See (\cite{Firth:1978vi}) and (\cite[esp.\ fn.1]{Silva:2020aa}).
  }
  And, there are accounts of propositional justification which work in this way.
  For example, GOLDMAN and \citeauthor{Turri:2010aa}.

  However, \supportI{} is then at issue.
  For, a conclusion does not, in general, entail propositional justification.
\end{note}

\begin{note}
  Further, we have no interest in justification.
  Again, this is not because anything will turn on cases where an agent lacks justification.
  Rather, do not wish to make the distinction.
  Ideas apply to cases in which an agent has justification, and lacks justification.
\end{note}

\section{\qWhyV{}, \qHowV{}, and \issueConstraint{}}
\label{cha:var:sec:vars}

\begin{note}
  \autoref{cha:var:ros} outlined how we understand \ros{1}.
  In the present section we develop variations of \qWhy{}, \qHow{}, and \issueInclusion{} in terms of \ros{}.

  We write the variants as `\qWhyV{}', `\qHowV{}', and `\issueConstraint{}', and the section is split into subsections which develop and discuss each variation.

  \begin{enumerate}[label=]
  \item
    \TOCLine{cha:var:sec:vars:qwhyvnp}
  \item
    \TOCLine{cha:var:sec:vars:qhowv}
  \item
    \TOCLine{cha:var:sec:vars:issue}
  \end{enumerate}

  % We term \qWhyV{} and \qHowV{} as \emph{variant} questions because \qWhyV{} and \qHowV{} are, respectively, too narrow and too broad to function as substitutes for \qWhy{} and \qHow{}.
  % Hence, to clarify the way in which \qWhyV{} is a variant of \qWhy{} and the way in which \qHowV{} is a variant of \qHow{} we will explicitly link the variant to the initial question.
  % We will then build the variant to \issueInclusion{} from \issueInclusion{} and the links between the variant questions and the initial questions.
\end{note}

\subsection{\qWhyV{}}
\label{cha:var:sec:vars:qwhyvnp}

\begin{note}
  Eliminate `why' from \qWhy{} in favour of whether pairing \(\phi\) and \(v\) depends on a \ros{} holding between proposition-value-\pool{} pairs.

  Recall:
  \begin{quote}%
    \vspace{-1.5\baselineskip}%
    \questionWhyBasic*
  \end{quote}

  Rather than asking for proposition-value-premises pairings associated with explanations of `why' an agent concluded \(\pv{\phi}{v}\) from \(\Phi\), the variation of \qWhy{} queries which \ros{1} are such that the conclusion of \(\pv{\phi}{v}\) from \(\Phi\) \emph{depended} on the \ros{1} holding, for the agent.
  \end{note}

\subsubsection{Question}
\label{cha:var:sec:vars:qwhyvnp:question}

\begin{note}
  The variation of \qWhy{}, \qWhyV{} is as follows:

  \begin{restatable}[\qWhyV{}]{question}{questionWhyV}
    \label{q:why:v}
    \cenLine{
      \begin{itemize*}[noitemsep, label=\(\circ\)]
      \item
        Agent: \vAgent{}
      \item
        Proposition: \(\phi\)
      \item
        Value: \(v\)
      \item
        \pool{2}: \(\Phi\)
      \item
        Event: \(e\)
      \item
        Event description: \(d\)
      \item
        \mbox{ }
      \end{itemize*}
    }
    \medskip

    Given \(e\) under \(d\) is an event in which \vAgent{} concludes \(\pv{\phi}{v}\) from \(\Phi\):

    \begin{itemize}
    \item
      Which proposition-value-\pool{} pairs \(\pvp{\psi}{v'}{\Psi}\) are such that:
      \begin{itemize}
      \item
        There is some sub-event \(e^{\flat}\) of \(e\), such that \(d\) implies:

        \begin{enumerate}[label=]
        \item
          \begin{enumerate}[label=\alph*., ref=(\alph*), series=qWhyVdef]
          \item
            \label{q:why:v:a}
            A \ros{0} between \(\pv{\psi}{v'}\) and \(\Psi\) holds for \vAgent{} at \(e^{\flat}\).
          \end{enumerate}
        \end{enumerate}
        %
      \item
        And:
        %
        \begin{enumerate}
        \item[\emph{If}:]
          \begin{enumerate}[label=\alph*., ref=(\alph*), resume*=qWhyVdef]
          \item
            \label{q:why:v:if}
            The \ros{0} between \(\pv{\psi}{v'}\) and \(\Psi\) failed to hold for \vAgent{} at \(e^{\flat}\).
          \end{enumerate}
        \item[\emph{Then}:]
          \begin{enumerate}[label=\alph*., ref=(\alph*), resume*=qWhyVdef]
          \item
            \label{q:why:v:then}
            \(e^{\flat}\) did not develop into an event \(e'\) such that \(e'\) is an event in which \vAgent{} concluded \(\pv{\phi}{v}\) from \(\Phi\).
          \end{enumerate}
        \end{enumerate}
      \end{itemize}
    \end{itemize}
    \vspace{-\baselineskip}
  \end{restatable}

  
\end{note}

\begin{note}
  \autoref{q:why:v}, captures the phenomenon where it is not possible to describe \(e\) without \ros{}.
  In short, absence of \ros{} impedes event from developing.

  Basic idea.
  Have an event in which the agent concludes.
  This event may be broken down into sub-events.
  And, failure for something to be the case in a sub-event in certain cases impedes event from developing.
\end{note}


\begin{note}
  Consider the general form of \qWhy{} with respect to making a cup of tea.

  \begin{scenario}[A cup of tea]
    \label{scen:cup-of-tea}
    A moment ago I:
    \begin{itemize}[noitemsep]
    \item
      Boiled some water.
    \item
      Placed a tea bag into a cup.
    \item
      Poured the water into the cup.
    \item
      Let the tea bag rest in the water for a while.
    \item
      Removed the tea bag.
    \end{itemize}
    I made myself a cup of tea.
  \end{scenario}

  \autoref{scen:cup-of-tea} captures an event in which I made a cup of tea.

  There is a straightforward sense with which the description given to the event captures \emph{how} I made a cup of tea.
  Further, there is also a sense with which the description also captures \emph{why} I made a cup of tea.

  For example, suppose I forgot to place a tea bag in the cup.
  The remainder of the description still captures some event, but the end result is a cup of warm water, rather than a cup of tea.

  There is also a sense with which the description fails to capture why I made a cup of tea.
  Placing a tea bag in a cup does not usually motivate making a cup of tea.%
  \footnote{
    Though we may imagine someone whose only pleasure is finding excuses to place tea bag into cups.
    (\cite[Cf.][379--380]{Rawls:1999aa})
  }

  Why the event resulted in a cup of tea versus why I wanted the event to result in a cup of tea.

  The former parallels our interest with conclusions.
  Not interested with why an agent wanted to reach a conclusion, but why an agent concluded \(\phi\) has value \(v\) as opposed to any other proposition-value pair, and why the agent concluded \(\phi\) has value \(v\) from \(\Phi\) as opposed to some other pool of premises.
\end{note}

\begin{note}
  Before considering \qWhyV{} in detail, apply the basic idea to a distinct domain.
  Recall \autoref{scen:cup-of-tea}, in which I made a cup of tea.

  As discussed, sub-event in which boiled water captures why I made a cup of tea as opposed to a cup of failure.

  So, we have a sub-event.
  Place tea bag.

  Parallels Clause~\autoref{q:why:v:a}.

  Parallel of \itc{}.

  If tea bag failed to be in cup, then did not develop into an event in which I made a cup of tea.

  In this sense, captures why.
  Why the event was making tea rather than warm water.

  % However, some caution.
  % For, it needs to be the case that event did not develop.
  % This is not guaranteed.
  % For example, perhaps I walked off and someone notices.

  % Description of \autoref{scen:cup-of-tea} did not provide any background details, so this is not ruled out.

  % However, add additional detail to the description to \autoref{scen:cup-of-tea} without presupposing.
  % For example, home alone.
\end{note}

\begin{note}
  Without \ros{}, the event does not develop into an event in which the agent concludes.

  \begin{scenario}[Countersign]
    The captain mumbled, ``I come from Miran.''

    The man returned the gambit, grimly.
    ``Miran is early this year.''

    The captain said, ``No earlier than last year.''

    But the man did not step aside.
    He said, ``Who are you?''

    ``Aren't you Fox?''%
    \mbox{ }\hfill\mbox{(\cite[70]{Asimov:1945aa})}
  \end{scenario}

  Captain Pritcher concluded the person they are talking to is a fellow member of the Democratic Underground Party via the countersign to their sign.

  So, sub-event, ``Miran is early this year'' is appropriate response to ``I come from Miran''.
  So, \ros{}.
  Goes on to conclude ``No earlier than last year'' is appropriate to respond with.
  Without first \ros{}, Pritcher does not conclude this, as not in a sign/countersign routine.

  However, keep in mind the function of \ros{} is an abstraction.
  Hot water, something that is.
  Not so with \ros{1}.
\end{note}

\begin{note}
  How to evaluate something failing to hold.

  Simplest is counterfactual.
  What would happen if failed to boil water.
  Well, poured the water, and got regret.

  Go back, remove \ros{}, and then get failure.

  However, distinction between counterfactual and what makes this true.

  However, equally guaranteed by some lawlike conditional.
  Similar to structural equations.

  In any case, description is doing work.
  What is absent from the description is things that could otherwise have been.

  Note, implication rather than entailment.
  Background things constrain whether or not follows.

  No getting tea without hot water.

  Description may rule out possibility of helper demon, or the closet possible world where don't boil being a world in which water is stiff sufficiently warm from previous use.

  No getting conclusion without \ros{}.
\end{note}

\begin{note}
  The role of description is to allow capture things which do not just happen to be.
  In this respect, fix as much as you want.

  In this respect, limit on counterfactuals.
  Must be the case that description holds.
\end{note}

\begin{note}
  \begin{notation}
  \item
    Only interested in single cases.
    So, rather than specify agent, etc.\ we write \qWhyV{}.

    When speaking abstractly, \(\phi\) \(v\) \(\Phi\) \qWhyV{} always applies to \(\phi\), \(v\), and \(\Phi\).
  \end{notation}
\end{note}


\subsubsection{Link}
\label{cha:var:sec:vars:qwhyvnp:link}

\begin{note}
  We developed \qWhyV{} through the idea that:

  If an event in which an agent concludes \(\pv{\phi}{v}\) from \(\Phi\) depends on a \ros{1} between \(\pv{\psi}{v'}\) and \(\Psi\) holding, for the agent, then the \ros{1} between \(\pv{\psi}{v'}\) and \(\Psi\) explains, in part, why the agent concluded \(\pv{\phi}{v}\) from \(\Phi\).

  The connexion between dependence and explanation of why links \qWhyV{} and \qWhy{} in the following way:

  \begin{link}[\qWhyV{} and \qWhy{}]
    \label{link:why:support:pvpp}
    For an agent \vAgent{}, and proposition-value-premises pairings \(\pvp{\psi}{v'}{\Psi}\):

    \begin{itemize}
    \item[\emph{If}:]
      \begin{enumerate}[label=\alph*., ref=(\alph*)]
      \item
        A \ros{0} between \(\pv{\psi}{v'}\) and \(\Psi\) is, in part, an answer to \qWhyV{}.
      \end{enumerate}
    \item[\emph{Then}:]
      \begin{enumerate}[label=\alph*., ref=(\alph*), resume]
      \item
        \(\pvp{\psi}{v'}{\Psi}\) is, in part, an answer to \qWhy{}.
      \end{enumerate}
    \end{itemize}
    \vspace{-\baselineskip}
  \end{link}

  In short, \linkW{} states that dependence captured by \qWhyV{} is sufficient to explain why an agent concluded \(\pv{\phi}{v}\) from \(\Phi\).
\end{note}

\begin{note}
  Though, given the discussion of entanglement in \autoref{cha:var:sec:vars:qwhyvnp:question}, we take the conditional and to be implicitly restricted by some commonsense.

  We will have no interest in the converse of \linkW{}.
  And, there may be cases in which \(\pvp{\psi}{v'}{\Psi}\) answers, in part, \qWhy{} while there is no corresponding \ros{} between \(\pv{\psi}{v'}\) and \(\Psi\) that answers, in part, \qWhyV{}.
  Indeed, \qWhy{} is not restricted to \ros{1}.
\end{note}

\subsubsection{Observations}

\begin{note}
  \begin{proposition}
    \label{prop:ros-always-answer}
    For an agent \vAgent{}, proposition-value pair \(\pv{\phi}{v}\) and \pool{} \(\Phi\):

    \begin{itemize}
    \item
      The \ros{} between \(\pv{\phi}{v}\) from \(\Phi\) is always an answer to \qWhyV{}.
    \end{itemize}
    \vspace{-\baselineskip}
  \end{proposition}

  \begin{argument}{prop:ros-always-answer}
    By \supportI{}, a \ros{} holds between \(\pv{\phi}{v}\) and \(\Phi\), for the agent, when the agent pairs \(\phi\) with \(v\) as a sub-event of an event in which the agent concludes \(\pv{\phi}{v}\) from \(\Phi\).

    Hence, by restricting \qWhyV{} to the sub-event, we ensure that the \ros{} between \(\pv{\phi}{v}\) from \(\Phi\) holds, for the agent, and so may answer, in part, \qWhyV{}.

    And, the \ros{} between \(\pv{\phi}{v}\) from \(\Phi\) is guaranteed to be an answer to \qWhyV{} because, if the \ros{} between \(\pv{\phi}{v}\) from \(\Phi\) did not hold, for the agent, then the relevant event would be an event in which the agent concludes \(\pv{\phi}{v}\) from \(\Phi\).
  \end{argument}
\end{note}

\begin{note}
  Focused on boiled water.
  However, various redundant things.
  For example, cup.
  Selected cup from shelf, and variety to choose from.
  But, picked one.
  Does particular cup make a difference?
  Intuitively no.

  \begin{observation}
    \label{obs:qWhyV:rosNNecAns}
    \cenLine{
      \begin{itemize*}[noitemsep, label=\(\circ\)]
      \item
        Agent: \vAgent{}
      \item
        Proposition: \(\phi\)
      \item
        Value: \(v\)
      \item
        \pool{2}: \(\Phi\)
      \item
        Event: \(e\)
      \item
        Event description: \(d\)
      \item
        \mbox{ }
      \end{itemize*}
    }
    \medskip

    Given \(e\) under \(d\) is an event in which \vAgent{} concludes \(\pv{\phi}{v}\) from \(\Phi\):
    \begin{itemize}
    \item
      It is not necessarily the case that a \ros{} which holds at some sub-event \(e^{\flat}\) of \(e\) answers \qWhyV{}.
    \end{itemize}
    \vspace{-\baselineskip}
  \end{observation}

  \begin{motivation}{obs:qWhyV:rosNNecAns}
    In short, depends on description, and failure of \ros{} to hold blocks development.

    Suppose \(d\) only captures conclusion.
    Now, over determined.
  \end{motivation}

  \autoref{obs:qWhyV:rosNNecAns}
\end{note}

\begin{note}
  \qWhyV{}.
  Hopefully clear.

  And, with \autoref{prop:ros-always-answer} passes basic test.

  Link, sufficiency.
  This is all we will need, but is also all we get.
  Follows are {
    \color{red}
    n
  }
  observations.
\end{note}

\begin{note}
  \begin{observation}
    \label{obs:qWhyV:underGen}
    It is not necessarily the case that a proposition-value-\pool{} pair which holds and intuitively answers \qWhy{} answers \qWhyV{}.
  \end{observation}

  So, this is highlighting that the converse to the link fails to hold.

  \begin{motivation}{obs:qWhyV:underGen}
    Suppose math problem.
    Flip a coin for calculator vs.\ mental arithmetic.
    If keep coin flip, then dependence.
    However, if do not keep coin flip, then no-dependence.
  \end{motivation}

  This shouldn't be too worrying.
  \qWhyV{} is only sufficient.
  Hence, does not rule out these answers.
  Under-generating, and that's fine.

  Still, over-generating.

  \begin{observation}
    \label{obs:qWhyV:overGen}
    Over-generating.
  \end{observation}

  \begin{motivation}{obs:qWhyV:overGen}
     For, consider a \scen{0} in which an agent concludes \(23 \times 15 = 345\) from their understanding of arithmetic, such that some time shortly before the agent did the calculation, the agent concluded it would be good to brew a cup of tea.

    Intuitively it should not be the case that the \ros{} and the conclusion that it would be good to get some coffee answers, in part, why the agent concluded some theorem is true from some \pool{}.
  Though, it may well be the case that the agent would have been too tired to conclude the theorem without the aid of the coffee.%
  \footnote{
    \citeauthor{Armstrong:1968vh} (\citeyear[195--196]{Armstrong:1968vh}) discusses a similar example, and suggests further discussion of this issue may also be found in \textcite{Moore:1962up}.
    See also \citeauthor{Sanford:1989aa} (\citeyear{Sanford:1989aa}) for a broader discussion of dependence as captured by subjunctive conditions (esp.\ pp.\ 192--193).
  }
  \end{motivation}

  This is less than idea.
  Depends on understanding of events involved and so on.
  For example, keep fixed the agent has had coffee.
  Then, it doesn't matter whether the \ros{} is preserved.

  Still, event.
  So, could be big.
  But then narrow.
\end{note}

\begin{note}
  Still, we will proceed with \qWhyV{}.

  Motivated:
  Theory neutral.

  The imprecision arising from the \itc{} is in part by design.

  We noted above that the sense of dependence captured by the \itc{} is not, in general, tightly connected to answering why questions.
  And, ideally, we would have a detailed account of the connexion between conclusions and \ros{1} which clarifies when link, captured by the conditional, between~\ref{q:why:v:if} and~\ref{q:why:v:then} holds.
  Still, we abstracted to \ros{1} to avoid any detailed account of what it is for an agent to conclude \(\pv{\phi}{v}\) from \(\Phi\).
  Hence, the abstracting to a conditional allows the relevant link between~\ref{q:why:v:if} and~\ref{q:why:v:then} to be developed in accordance with specific details given by some account of what concluding amounts to.
\end{note}


\subsection{\qHowV{}}
\label{cha:var:sec:vars:qhowv}

\begin{note}
  \qHow{} is a broad question which asks, quite generally, for an account of how an agent concluded \(\pv{\phi}{v}\) in terms of what has happened.
  Recall:

  \questionHowBasic*

  As with \qWhy{} and \qWhyV{}, `\qHowV{}' refers to the variation of \qHow{}.
\end{note}

\subsubsection{Question}
\label{cha:var:sec:vars:qhowv:sec:question}

\begin{note}
  The variant on \qHow{} is as follows:

  \begin{restatable}[\qHowV{}]{question}{questionHowV}
    \label{q:how:v}
    \cenLine{
      \begin{itemize*}[noitemsep, label=\(\circ\)]
      \item
        Agent: \vAgent{}
      \item
        Proposition: \(\phi\)
      \item
        Value: \(v\)
      \item
        \pool{2}: \(\Phi\)
      \item
        Event: \(e\)
      \item
        \mbox{ }
      \end{itemize*}
    }

    \begin{itemize}
    \item
      Given \(e\) is an event which \vAgent{} concludes \(\pv{\phi}{v}\) from \(\Phi\):
      \begin{itemize}
      \item
        Which events \(e'\) are such that:
        \begin{itemize}
        \item
          For any proposition-value pair \(\pv{\psi}{v'}\) such that a \ros{} between \(\pv{\psi}{v'}\) and \(\Psi\) holds when when \vAgent{} pairs \(\phi\) with \(v\):
          \begin{itemize}
          \item
            There exists an event \(e'\) such that \(e'\) is a \wit{0} for a \ros{} between \(\pv{\psi}{v'}\) and \(\Psi\).
          \end{itemize}
        \end{itemize}
      \end{itemize}
    \end{itemize}
    % \begin{itemize}
    % \item
    %   Given \(e\) is an event which \vAgent{} concludes \(\pv{\phi}{v}\) from \(\Phi\):

    %   \begin{itemize}
    %   \item
    %     Which events \(e'\) are such that, when \vAgent{} pairs \(\phi\) with \(v\), both~\ref{q:how:v:a} and~\ref{q:how:v:b} hold:
    %     \begin{enumerate}[label=\alph*., ref=(\alph*), , series=qHowVdef]
    %     \item
    %       \label{q:how:v:a}
    %       A \ros{} between \(\pv{\psi}{v'}\) and \({\Psi}\) answers \qWhyV{}.
    %     \item
    %       \label{q:how:v:b}
    %       \(e'\) is a \wit{0} for a \ros{} between \(\pv{\psi}{v'}\) and \(\Psi\).
    %     \end{enumerate}
    %   \end{itemize}
    % \end{itemize}
    \vspace{-\baselineskip}
  \end{restatable}

  Intuitively, \qHowV{} presupposes and answer to \qWhyV{} and returns an event which witnesses \ros{}, if such an event exists.
\end{note}

\begin{note}
  \begin{proposition}
    \label{prop:phi-always-how}
            \cenLine{
      \begin{itemize*}[noitemsep, label=\(\circ\)]
      \item
        Agent: \vAgent{}
      \item
        Proposition: \(\phi\)
      \item
        Value: \(v\)
      \item
        \pool{2}: \(\Phi\)
      \item
        \mbox{ }
      \end{itemize*}
    }

    \begin{itemize}
    \item
      For any event \(e\) in which \vAgent{} concludes \(\pv{\phi}{v}\) from \(\Phi\):
      \begin{itemize}
      \item
        \(e\) is an answer to \qHowV{}.
      \end{itemize}
    \end{itemize}
    \vspace{-\baselineskip}
  \end{proposition}

  \begin{argument}{prop:phi-always-how}
    Let \(e\) be an event in which an agent concludes \(\pv{\phi}{v}\) from \(\Phi\).

    Then, immediate by \autoref{def:witnessing}.
  \end{argument}


  % \begin{argument}{prop:phi-always-how}
  %   Suppose \(e\) is the event in which \vAgent{} concludes \(\pv{\psi}{v'}\) from \(\Psi\).
  %   By \autoref{prop:ros-always-answer} established that the \ros{} between \(\pv{\phi}{v}\) and \(\Phi\) is always an answer to \qWhyV{}.
  %   So, Clause~\ref{q:how:v:a} is satisfied.

  %   Likewise, by assumption \(e\) is the event in which \vAgent{} concludes \(\pv{\psi}{v'}\) from \(\Psi\).
  %   Hence, by \autoref{def:witnessing}, \(e\) is a \wit{} for the \ros{} between \(\pv{\psi}{v'}\) and \(\Psi\).
  %   So, Clause~\ref{q:how:v:b} is satisfied.

  %   And, as both Clause~\ref{q:how:v:a} and Clause~\ref{q:how:v:b} are satisfied, \(e\) is an answer to \qHowV{}.
  % \end{argument}

\end{note}

% \begin{note}
  % As with \qWhyV{} our interest is with extensional adequacy, and specifically extensional adequacy with respect to \ros{1}.

  % Consider any event.
  % There are two cases.
  % \ros{} which answers \qWhyV{} or no \ros{}.

  % If no \ros{} then the \agents{} conclusion of \(\pv{\phi}{v}\) from \(\Phi\) does not depend on what happened.
  % Hence, event is not of direct interest with respect to answering `how' an agent concluded \(\pv{\phi}{v}\) from \(\Phi\).

  % If \ros{} which \qWhyV{}, then included as an answer to \qHowV{}.

  % So, though we have abstracted to \ros{1} to avoid any account of what conclusion amounts to, the existence of a \wit{0} intuitively captures whatever it is of relevance that happened when the agent concluded \(\pv{\phi}{v}\) from \(\Phi\).

  % And, as discussed in \autoref{cha:var:ros:W}, the existence of a \wit{0} allows for \ros{} which answer \qWhyV{} to be constrained by answers to \qHowV{}, even if the relevant \wit{0} occurs at some point prior to the event in which the agent concludes \(\pv{\phi}{v}\) from \(\Phi\).
% \end{note}

\subsubsection{Link}
\label{cha:var:sec:vars:qhowv:sec:link}

\begin{note}
    Now, from \autoref{prop:phi-always-how} we have that the event in which an agent concludes \(\pv{\phi}{v}\) from \(\Phi\) is always an answer to \qHowV{}.
  However, the way in which \qHowV{} queries `how' an agent concludes \(\pv{\phi}{v}\) from \(\Phi\) is highly imprecise.

  For \qHowV{} searches for any \wit{}.
  The key, however, is that we are guaranteed to at least have the event.

  Further, so long as the agent has \wit{} for \ros{} which answers \qWhyV{}, then that event is answer to \qHowV{}.

  Still, \qHow{} is not of direct interest.
  Instead, \issueInclusion{}.
  Constrain that \qHow{} places on \qWhy{}.

  As \qHowV{} is weaker than \qHow{}, weaker constraint.
  And, as we are looking for counterexamples, this restricts possible counterexamples.
\end{note}

\begin{note}
  % Given that \qWhyV{} has a key role in the construction of \qHowV{}, the link between \qHow{} and \qHowV{} is a little more involved that the link between \qWhyV{} and \qWhy{}.
  \begin{link}[\qHowV{} and \qHow{}]
    \label{link:how-witnessing}
    For any proposition-value-premises pairing \(\pvp{\psi}{v'}{\Psi}\):
    \begin{itemize}
    \item[\emph{If}:]
      \begin{enumerate}[label=\alph*., ref=(\alph*)]
      \item
        \(\pvp{\psi}{v'}{\Psi}\) is, in part, an answer \qHow{} due to \ros{} between \(\pv{\psi}{v'}\) and \(\Psi\) being, in part, an answer to \qWhyV{}.
      \end{enumerate}
    \item[\emph{Then}:]
      \begin{enumerate}[label=\alph*., ref=(\alph*), resume]
      \item
        There exists some event \(e\) such that \(e\) is an event in which \vAgent{} concludes \(\pv{\psi}{v'}\) from \(\Psi\) and \(e\) is, in part, an answer \qHowV{}.
      \end{enumerate}
    \end{itemize}
    \vspace{-\baselineskip}
  \end{link}
  % We understand  \(\pvp{\psi}{v'}{\Psi}\) being, in part, an answer \qHow{} `due to' \ros{} between \(\pv{\psi}{v'}\) and \(\Psi\) being, in part, an answer to \qWhyV{} in terms of \linkW{} and \issueInclusion{}.

  % For, suppose \(\pv{\psi}{v'}\) and \(\Psi\) is, in part, an answer to \qWhyV{}.
  % Then, \(\pvp{\psi}{v'}{\Psi}\) is, in part, an answer to \qWhy{} via \linkW{}.
  % And, if \issueInclusion{} holds, then \(\pvp{\psi}{v'}{\Psi}\) is also, in part, an answer to \qHow{}.
  % Therefore, \(\pvp{\psi}{v'}{\Psi}\) is, in part, an answer to \qHow{} `due to' \(\pv{\psi}{v'}\) and \(\Psi\) being, in part, an answer to \qWhyV{}.
\end{note}

% \begin{note}
%   To summarise the preceding discussion, \linkH{} is designed to capture the idea that \qHow{} is concerned with the process by which an agent concludes \(\pv{\phi}{v}\) from \(\Phi\).
%   Hence, if \(\pvp{\psi}{v'}{\Psi}\) is, in part, an answer \qHow{} due to a \ros{} between \(\pv{\psi}{v'}\) and \(\Psi\) being, in part, an answer to \qWhyV{} then, there is some process\dots

%   Then, intuitively, relationship between \(\pv{\phi}{v}\) and \(\Psi\) implicitly captures some part of the process by which the agent concluded \(\pv{\phi}{v}\) from \(\Phi\).
%   Hence, as answers to \qHowV{} are given in terms of events, extract the event which corresponds to the pairing.

%   Given \qHowV{} is designed to relate to \qWhyV{}, restrict pairings of interest to those which result in a \ros{0}.
% \end{note}

\subsection{\issueConstraint{}}
\label{cha:var:sec:vars:issue}

\begin{note}
  In~\autoref{cha:var:sec:vars:qwhyvnp} we developed \qWhyV{}, a variation of \qWhy{}.
  And, in~\autoref{cha:var:sec:vars:qhowv} we developed \qHowV{}, a variation of \qHow{}.
  In both sections we established links between the variant --- \qWhyV{}, \qHowV{}  --- and initial --- \qWhy{}, \qHow{} --- questions.

  Indeed, the purpose of developing the variant questions is to clarify the initial questions to a sufficient degree so that counterexamples to the intuitive constraint between the initial questions --- \issueInclusion{} --- may be developed.

  In this section, we bring together the variant and initial questions and \issueInclusion{} to state, in some detail, how everything fits together.
\end{note}

\begin{note}
 We being with a proposition:

  \begin{restatable}[]{proposition}{propVariationsAndInclusion}
    \label{prop:support-and-witnessing}
    Assuming \linkW{}, \linkH{}, and \issueInclusion{} hold:

    For an agent \vAgent{}, proposition-value pairs \(\pv{\phi}{v}\), \(\pv{\psi}{v'}\), and \pool{1} \(\Phi\), \(\Psi\):

    \begin{itemize}
    \item
      \qWhyV{} is answered, in part, by a \ros{} between \(\pv{\psi}{v'}\) and \(\Psi\).
    \end{itemize}

    \emph{Only if}

    \begin{itemize}
    \item
      \qHowV{} is answered, in part, by \vAgent{}' \wit{0} for the \ros{} between \(\pv{\psi}{v'}\) and \(\Psi\).
    \end{itemize}
    \vspace{-\baselineskip}
  \end{restatable}

  \begin{argument}{prop:support-and-witnessing}
    Suppose \qWhyV{} is answered, in part, by a \ros{} between \(\pv{\psi}{v'}\) and \(\Psi\).
    From~\linkW{} it is immediately follows that \(\pvp{\psi}{v'}{\Psi}\) answers, in part, \qWhy{}.
    And, given \issueInclusion{}, \(\pvp{\psi}{v'}{\Psi}\) answers, in part, \qHow{}.

    Further, \(\pvp{\psi}{v'}{\Psi}\) answers, in part, \qHow{} due to a \ros{} between \(\pv{\phi}{v}\) and \(\Psi\) being, in part, an answer to \qWhyV{}.
    Therefore, by \linkH{}, there is some \(e\) such that \(e\) is a \wit{0} for the \ros{} between \(\pv{\psi}{v'}\) and \(\Psi\).
  \end{argument}


  \autoref{fig:relations-between-whys-and-hows} is a a visual representation of the argument for~\autoref{prop:support-and-witnessing}.
\end{note}

\begin{figure}[H]
  \centering
  \begin{tikzpicture}
    \tikzset{ansStyle/.style={%
        draw=gray,%
        text width=.5\textwidth,%
        rounded corners=2pt,%
      }%
    }
    %
    \node[ansStyle] (whyO) at (0,0) %
    {\qWhyV{} is answered by a \ros{0} between \(\pv{\psi}{v'}\) and \(\Psi\).};
    %
    \node[ansStyle] (whyA) at (1.933,-1.5) %
    {\qWhy{} is answered by \(\pvp{\psi}{v'}{\Psi}\).};
    %
    \node[ansStyle] (howA) at (3.866,-3) %
    {\qHow{} is answered by \(\pvp{\psi}{v'}{\Psi}\).};
    %
    \node[ansStyle] (witA) at (5.8,-4.5) %
    {\qHowV{} is answered by event which \wit{1} sppt.\ btw.\ \(\pv{\psi}{v'}\) and \(\Psi\).};
    %
    \path[->] ($(whyO.south)!0.9!(whyO.south west)$) edge [out=270, in=180] (whyA);
    \path[->] ($(whyA.south)!0.9!(whyA.south west)$) edge [out=270, in=180] (howA);
    \path[->] ($(howA.south)!0.9!(howA.south west)$) edge [out=270, in=180] (witA);
    %
    \node[text width=.5\textwidth] (1) at (1,-.8) {\linkW{}};
    \node[text width=.75\textwidth] (2) at (4.5,-2.25) {\issueInclusion{}};
    \node[text width=.5\textwidth] (3) at (5,-3.625) {\linkH{}};
  \end{tikzpicture}%
  \caption{Visual representation of~\autoref{prop:support-and-witnessing}}
  \label{fig:relations-between-whys-and-hows}
\end{figure}

\begin{note}
  As~\autoref{prop:support-and-witnessing} follows from \linkW{}, \linkH{}, and \issueInclusion{}, we consider the content of~\autoref{prop:support-and-witnessing} to be a parallel constraint to \issueInclusion{}:

  \begin{restatable}[\issueConstraint{}]{constraint}{issueConstraintStatement}
    \label{issue:has-witnessed}
    For an agent \vAgent{}, proposition-value pairs \(\pv{\phi}{v}\), \(\pv{\psi}{v'}\), and \pool{1} \(\Phi\), \(\Psi\):

    \begin{itemize}
    \item
      \qWhyV{} is answered, in part, by a \ros{} between \(\pv{\psi}{v'}\) and \(\Psi\).
    \end{itemize}

    \emph{Only if}

    \begin{itemize}
    \item
      \qHowV{} is answered, in part, by \vAgent{}' \wit{0} for the \ros{} between \(\pv{\psi}{v'}\) and \(\Psi\).
    \end{itemize}
    \vspace{-\baselineskip}
  \end{restatable}

  As with \issueInclusion{}, if \issueConstraint{} holds, then answers to \qWhyV{} are constrained by answers to \qHowV{}.

  Our direct goal is to develops counterexamples to \issueConstraint{}.
  For, if there are counterexamples to \issueConstraint{}, then it immediately follows by \autoref{prop:support-and-witnessing} that either \linkW{}, \linkH{}, or \issueInclusion{} fails to hold.

  We defended \linkW{} when developing \qWhyV{} in~\autoref{cha:var:sec:vars:qwhyvnp}.
  And, likewise, we defended \linkH{} when developing \qHowV{} in~\autoref{cha:var:sec:vars:qhowv}.
  And, though there are some difficulties with \qWhyV{}, \qHowV{}, \linkW{}, and \linkH{}, I consider the most plausible point of failure to be \issueInclusion{}.
\end{note}

% \begin{note}
%   Still, \issueConstraint{} may be considered as a direct constraint on answers to \qWhyV{} in terms of answers to \qHow{}.
%   In particular, \autoref{prop:constraint-rewrite} expands our construction of \qWhyV{}  and \qHowV{} to provide an alternative statement of \issueConstraint{}.

  % \begin{proposition}[\issueConstraint{}, rewritten]
  %   \label{prop:constraint-rewrite}
  %   For an agent \vAgent{}, proposition-value pairs \(\pv{\phi}{v}\), \(\pv{\psi}{v'}\), and \pool{1} \(\Phi\), \(\Psi\):

  %   \issueConstraint{} is equivalent to the following conditional:

  %   \begin{enumerate}
  %   \item[\emph{If}:]
  %     \begin{enumerate}[label=\alph*., ref=(\alph*)]
  %     \item
  %       \label{constraint-rewrite:a}
  %       \vAgent{} concluded \(\pv{\phi}{v}\) from \(\Phi\).
  %     \end{enumerate}
  %   \item[\emph{And}:]
  %     \begin{enumerate}[label=\alph*., ref=(\alph*), resume]
  %     \item
  %       \label{constraint-rewrite:b}
  %       \vAgent{} would not have concluded \(\pv{\phi}{v}\) from \(\Phi\), if a \ros{} between \(\pv{\psi}{v'}\) and \(\Psi\) failed to hold, from \agpe{\vAgent{}'}.
  %     \end{enumerate}
  %   \item[\emph{Then}:]
  %     \begin{enumerate}[label=\alph*., ref=(\alph*), resume]
  %     \item
  %       \label{constraint-rewrite:c}
  %       \vAgent{} has a \wit{0} for the \ros{} between \(\pv{\psi}{v'}\) and \(\Psi\).
  %     \end{enumerate}
  %   \end{enumerate}
  %   \vspace{-\baselineskip}
  % \end{proposition}
  % %
  % \begin{argument}{prop:constraint-rewrite}
  %   Immediate by \autoref{prop:support-and-witnessing} and the construction to \qWhyV{} and \qHowV{}.

  %   The antecedent of the conditional~---~\ref{constraint-rewrite:a}~\&~\ref{constraint-rewrite:b}~---~correspond to a \ros{} being, in part, an answer to \qWhyV{} while the consequent~---~\ref{constraint-rewrite:c}~---~corresponds to the \wit{0} which answers, in part, \qHowV{}.
  %   The conditional then follows from \autoref{prop:support-and-witnessing}.
  % \end{argument}

%   So, directly, \issueConstraint{} amounts to the constraint that in order for an \agents{} conclusion of \(\pv{\phi}{v}\) to depend on some \ros{} between \(\pv{\psi}{v'}\) and \(\Psi\), the agent must have concluded \(\pv{\psi}{v'}\) from \(\Psi\).
% \end{note}

\subsection{Summary}
\label{cha:var:sec:vars:summary}

\begin{note}
  Overall argument.
  Links then answer to \qWhyV{} which is not constrained by \qHowV{}, then \issueInclusion{} fails.

  Three broad ways in which the overall argument may fail:
  \begin{enumerate}[label=\arabic*., ref=(\arabic*), noitemsep]
  \item
    The link between \qWhyV{} and \qWhy{} fails to hold.
  \item
    The link between \qHowV{} and \qHow{} fails to hold.
  \item
    We fail to develop counterexamples to \issueConstraint{}.
  \end{enumerate}

  Still, I hope to have developed \qWhyV{}, \qHowV{}, and \issueConstraint{} in such a way that both questions are of some interest independent of link

  In \autoref{cha:clar:sec:literature} we suggest how a handful of accounts of conclusion, or related, may be understood in terms of \qWhyV{} and \qHowV{}, and how the accounts motivate \issueConstraint{} as a constraint on answers to \qWhyV{} in terms of answers to \qHowV{}.
\end{note}

%%% Local Variables:
%%% mode: latex
%%% TeX-master: "master"
%%% End:
