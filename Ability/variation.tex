\chapter{Variations}
\label{cha:var}

\begin{note}
  The primary role of the present chapter is to introduce variations on \qWhy{}, \qHow{}, and \issueInclusion{}, as introduced in \autoref{cha:introduction}.

  The overall goal of this document is to argue that \issueInclusion{} fails to hold, and the role of the variations is to refine \qWhy{}, \qHow{}, and \issueInclusion{} in such a way that the way in which the goal is to be achieved is clear.

  Hence, with variations fresh in hand, we will also sketch, in outline, the general way in which we will develop counterexamples to \issueInclusion{} while anticipate certain difficulties.
\end{note}

\begin{note}
  This chapter is divided into sections as follows:
  \begin{enumerate}[label=, leftmargin=*]
  \item
    \TOCLine{cha:var:sec:wiggling}

    Describe approach to variations.
    Highlight difficulty, and how variations will be developed.
  \item
    \TOCLine{cha:var:ros}

    Detailed discussion of \ros{1}, foundation of how we develop variations.
    Includes initial discussion of \fc{1}.
  \item
    \TOCLine{cha:var:sec:vars}

    Variations to \qWhy{}, \qHow{}, and \issueInclusion{} with discussion.
  \item
    \TOCLine{cha:clar:sec:literature}

    Look at how to connect variant to various account of concluding, and related phenomena, from the literature.
  \item
    \TOCLine{cha:var:wrang}

    Method for counterexamples, and anticipated difficulties.
  \end{enumerate}
\end{note}

\section{Wiggling}
\label{cha:var:sec:wiggling}

\begin{note}
  \autoref{cha:introduction} introduced \qWhy{}, \qHow{}, and \issueInclusion{}.

  \begin{quote}
    \vspace{-1.5\baselineskip}
    \questionWhyBasic*
  \end{quote}

  \begin{quote}
    \vspace{-1.5\baselineskip}
    \questionHowBasic*
  \end{quote}

  \begin{quote}
    \vspace{-1.5\baselineskip}
    \issueInclusionFirst*
  \end{quote}
  The overall goal of this document is to argue \issueInclusion{} does not hold.
  And, to do so we will develop a general type of counterexample to \issueInclusion{}.

  Still, both \qWhy{} and \qHow{} are broad questions which turn on the way in which the respective instances of `why' and `how' are understood.
  Hence, without establishing a clear understanding of the way in which the instances are to be understood, it is unclear how to develop counterexamples to \issueInclusion{}.

  However, these variations must obey to competing constraints:

  \begin{enumerate}[label=\alph*., ref=(\alph*)]
  \item
    \label{vars:constraint:ce}
    The variations to \qWhy{} and \qHow{} allow for the possibility for the variation of \issueInclusion{} to fail to hold.
  \item
    \label{vars:constraint:int}
    As \issueInclusion{} intuitively constrains answers to \qWhy{} in terms of answers to \qHow{}, a variation on \issueInclusion{} should intuitively constrain answers to the variation to \qWhy{} in terms of the variation to \qHow{}.
  \end{enumerate}

  Indeed, Constraint~\ref{vars:constraint:int} is a source of significant concern.

  For, there may be plausible answers to \qWhy{} which are not also answers to \qHow{} but fail to be counterexamples to \issueInclusion{} as the relevant sense of `why' and `how' are not the senses of `why' and `how' that \issueInclusion{} (intuitively) holds with respect to.

  And, as a constraint on answers to \qWhy{} in terms of \qHow{}, \issueInclusion{} may plausibly have a role in narrowing the relevant senses of `why' and `how' at issue.
  And, if \issueInclusion{} is rejected, then, naturally, \issueInclusion{} cannot perform this role.

  Hence, if the variations to \qWhy{}, \qHow{} or \issueInclusion{} fail to preserve intuitions regarding \qWhy{}, \qHow{} and \issueInclusion{}, then any counterexample to the variations will be of little to no interest.

  Further, to the extent that various theories of conclusion, or sufficiently related phenomena, either implicitly or explicitly endorse \issueInclusion{}, it should be the case that the same theories either implicitly or explicitly motivate the variation of \issueConstraint{} constraining answers to the variation of \qWhy{} in terms of answers to \qHow{}.
\end{note}

\begin{note}
  In broad outline, we use the idea of a `\ros{0}' to provide variations on \qWhy{}, \qHow{}, and \issueInclusion{}.
  The way in which we understand \ros{1} is minimal and tightly connected to an event in which an agent concludes \(\pv{\phi}{v}\) from \(\Phi\).
  Specifically, we put forward three ideas in relation to \ros{1}:
  \begin{enumerate}
  \item
    If an agent concludes \(\pv{\phi}{v}\) from \(\Phi\), then a \ros{0} between \(\pv{\phi}{v}\) and \(\Phi\), from the \agpe{}, when the agent pairs \(\phi\) with \(v\).
  \item
    If an agent has concluded \(\pv{\phi}{v}\) from \(\Phi\), then the event in which the agent concluded \(\pv{\phi}{v}\) from \(\Phi\) functions as a \wit{0} for a \ros{0} between \(\pv{\phi}{v}\) and \(\Phi\).
  \item
    It is possible for a \ros{0} between \(\pv{\phi}{v}\) and \(\Phi\) to hold, from an \agpe{}, without their being an \wit{0} for the \ros{0}.
  \end{enumerate}

  \autoref{cha:var:ros} will develop and discuss each idea in detail.

  For the moment, the motivation for abstracting to \ros{1} is to capture, in a abstract way, the way in which \(\pv{\phi}{v}\) and \(\Phi\) are related from an \agpe{} when the agent concludes \(\pv{\phi}{v}\) from \(\Phi\).

  Rather than directly capturing some relevant sense of `why' or `how' our goal is to use \ros{1} to construct variations on \qWhy{} and \qHow{} which are \emph{roughly} `extensionally adequate'.
  Where, we understand the term `extensionally adequate' more-or-less in line with \citeauthor{Sumner:1987aa} (\citeyear{Sumner:1987aa}):

  \begin{quote}
    [A] conception of a concept is extensionally adequate when it includes every item which seems pre-analytically to be an instance of the concept and excludes every item which does not.%
    \mbox{ }\hfill\mbox{(\citeyear[49]{Sumner:1987aa})}
  \end{quote}

  Adapted to our case, our interest with \qWhy{} and \qHow{} is with respect to intuitive answers to \qWhy{} and \qHow{} (and in particular the intuition that \issueInclusion{} holds).
  And, the variations of both \qWhy{} and \qHow{} may be seen as `conceptions of a question' such that any answer to \qWhy{} is an answer to the variation of \qWhy{}, vice-versa, and the same with respect to \qHow{}.

  Still, the way in which something answers \qWhy{} need not be equivalent to the way in which that thing answers the variation to \qWhy{}.%
  \footnote{
    In this respect, the variation to \qWhy{} need not be \emph{intensionally} adequate.
    Where the variation of \qWhy{} (or \qHow{}) would be intensionally adequate just in case the variation captured the way something \emph{intuitively} answers \qWhy{}.
  }

  And, we are only interested in `rough' extensional adequacy.
  In particular, we are only interested in answers to \qWhy{} and \qHow{} to the extent that \issueInclusion{} plausibly holds.
  Hence, we will ignore intuitive answers to \qWhy{} and \qHow{} which extend beyond \issueInclusion{}.

  Further, to the extent that \issueInclusion{} is intuitive, the variations \emph{may} conflict with this intuition.
\end{note}

\begin{note}
  With the aid of \ros{1} we develop variations of \qWhy{} and \qHow{}:

  \begin{itemize}
  \item
    `Why' is eliminated from \qWhy{} in favour of whether the agent conclusion of \(\pv{\phi}{v}\) from \(\Phi\) depended on a \ros{0} holding between \(\pv{\psi}{v'}\) and \(\Psi\).
  \item
    `How' is eliminated from \qHow{} in favour of events which serve to \wit{0} any \ros{} that the agent's conclusion of \(\pv{\phi}{v}\) from \(\Phi\) depended on.
  \end{itemize}

  So, the variation to \qWhy{} is expected to be extensionally adequate for:

  If conclusion does not depend on \ros{}, then plausible that it is possible to answer \qWhy{} without citing the proposition-value-premises pairing.
  For, event would have occurred regardless of whether paired.

  If conclusion does depend on \ros{}, then proposition-value-premises pairing answers, in part, \qWhy{}.
  For, event would not have occurred regardless of whether paired.

  Variation to \qHow{}.
  Developed with respect to variation on \qWhy{}.

  If \ros{}, then if event \wit{} \ros{}, then of interest.
  If event which does not lead to \ros{}, then event is of no interest.
\end{note}

\begin{note}
  An significant consequence of both variations will be as follows:

  \begin{itemize}
  \item
    When an agent concludes \(\pv{\phi}{v}\) from \(\Phi\):
    \begin{itemize}
    \item The agent's conclusion of \(\pv{\phi}{v}\) from \(\Phi\) depends on a \ros{0} between \(\pv{\phi}{v}\) and \(\Phi\) holding, from the \agpe{}.
    \item The event in which the agent concludes \(\pv{\phi}{v}\) from \(\Phi\) serves as a \wit{0} for the \ros{0} between \(\pv{\phi}{v}\) and \(\Phi\).
    \end{itemize}
  \end{itemize}
  Hence, a \ros{} between \(\pvp{\phi}{v}{\Phi}\) will always be, in part, an answer to the variation of \qWhy{} and the event in which the agent concludes \(\pv{\phi}{v}\) from \(\Phi\) will always be, in part, an answer to the variation of \qHow{}.
\end{note}

\begin{note}
  A variant to \issueInclusion{} follows from the variations to \qWhy{} and \qHow{}.
  Roughly:

  \begin{itemize}
  \item
    A conclusion of \(\pv{\phi}{v}\) from \(\Phi\) depends on some \ros{} between \(\pv{\psi}{v'}\) and \(\Psi\) holding from the \agpe{}

    \emph{Only if}:

    The agent has a \wit{} for the \ros{0} between \(\pv{\psi}{v'}\) and \(\Psi\).
  \end{itemize}
\end{note}

\begin{note}
  With an initial understanding of the variations to \qWhy{}, \qHow{}, and \issueInclusion{} in hand, we now return to the overall argument of this document.
  Our goal is to develop counterexamples to \issueInclusion{}.
  And, given the variation to \issueInclusion{} which follows from the variations to \qWhy{} and \qHow{}, we will do so by showing there are cases in which an agent concludes \(\pv{\phi}{v}\) from \(\Phi\) such that:
  \begin{itemize}
  \item
    The agent pairing \(\phi\) and \(v\) depended on a \ros{0} between \(\pv{\psi}{v'}\) and \(\Psi\) holding, from the \agpe{}.
  \item
    The agent did not have a \wit{0} for the \ros{0} between \(\pv{\psi}{v'}\) and \(\Psi\) when the agent paired \(\phi\) and \(v\).
  \end{itemize}
\end{note}

\section{\ros{3}}
\label{cha:var:ros}

\begin{note}
  In \autoref{cha:var:sec:wiggling} we briefly sketched the role \ros{1} will have in this document.
  The present section develops and discusses \ros{1} in detail.

  The role of \ros{1} within this document is to capture, in a abstract way, the way in which a proposition-value pair \(\pv{\phi}{v}\) and \poP{} \(\Phi\) are related from an \agpe{} when the agent concludes \(\pv{\phi}{v}\) from \(\Phi\).
\end{note}

\begin{note}
  Our understanding of a `\ros{1}' is given in terms of three ideas, and sub-sections will develop and discuss each idea in detail:

  \begin{enumerate}[label=, leftmargin=*]
  \item
    \TOCLine{cha:var:ros:I}

    An event in which an agent concludes \(\pv{\phi}{v}\) from \(\Phi\) is sufficient for a \ros{} to hold, from the \agpe{}.
  \item
    \TOCLine{cha:var:ros:W}

    An event in which an agent concludes \(\pv{\phi}{v}\) from \(\Phi\) provides an agent with a \wit{0} for a \ros{}.
  \item
    \TOCLine{cha:var:ros:II}

    It is possible for a \ros{} to hold, from an \agpe{} without the agent having a \wit{} for the \ros{}.
  \end{enumerate}

  \autoref{cha:var:ros:II} will also briefly introduce and discuss the way in which a \ros{} may hold, from an \agpe{} without the agent having a \wit{} from the \ros{}.
  The core idea is that of \(\pv{\phi}{v}\) being a `\fc{0}' from \(\Psi\).
  And, that if \(\pv{\phi}{v}\) is a \fc{0} from \(\Psi\), then a \ros{1} holds between \(\pv{\psi}{v'}\) and \(\Psi\), from the \agpe{}.

  A final subsection, \autoref{cha:var:ros:Emb}, will introduce and discuss a useful feature of abstracting to \ros{1} which we term `embedding'.
\end{note}

\begin{note}
  We speak in terms of \ros{1} holding from an \agpe{}.
  However, given some agent \vAgent{}, proposition \(\phi\), value \(v\), and \poP{} \(\Phi\), we do not draw any particular distinction between:

  \begin{enumerate}[label=\alph*., ref=(\alph*)]
  \item
    \label{ros:ap:maybe:a}
    A \ros{} between \(\pv{\phi}{v}\) and \(\Phi\) holds, from an \agpe{}.
  \item
    \label{ros:ap:maybe:b}
    The pairing of the proposition `A \ros{} holds between \(\pv{\phi}{v}\) and \(\Phi\)' with the value `True' by the agent.

    I.e.\ the proposition-value pair:\newline
    \mbox{ }\hfill%
    \(\pv{\text{A \ros{} holds between } \pv{\phi}{v}\text{ and }\Phi}{\text{True}}\)
  \end{enumerate}

  Any significant distinction between~\ref{ros:ap:maybe:a} and~\ref{ros:ap:maybe:b} would turn on details too specific for the degree of abstraction we target.
  For, what holds from an \agpe{} may just amount to which propositions are paired with the value `True' by the agent.
  % As~\ref{ros:ap:maybe:a} is shorter, we will only use~\ref{ros:ap:maybe:b} when we wish to explicitly consider 

  % Still, the difference in presentation between~\ref{ros:ap:maybe:a} and~\ref{ros:ap:maybe:b} is suggestive:
  % \begin{itemize}
  % \item
  %   \ref{ros:ap:maybe:b} explicitly captures a proposition-value pair.

  %   Hence, \ref{ros:ap:maybe:b} explicitly identifies something which may be a premises from which a conclusion is drawn.
  % \item
  %   By contrast,~\ref{ros:ap:maybe:a} does not explicitly capture a proposition-value pair.

  %   Hence,~\ref{ros:ap:maybe:a} only implicitly identifies something which may be a premises from which a conclusion is draw.

  %   And, if~\ref{ros:ap:maybe:a} and~\ref{ros:ap:maybe:b} are not equivalent,~\ref{ros:ap:maybe:a} may fail to identify something which may be a premises from which a conclusion is draw
  % \end{itemize}

  % Hence, we will use the presentation of \ref{ros:ap:maybe:b} when we wish to highlight the possibility that the relevant \ros{} may function as a premise for some conclusion.
  % And, likewise, we will use the presentation of~\ref{ros:ap:maybe:a} when we wish to highlight the possibility that the relevant \ros{} may not function as a premise.

  % Indeed, we will favour the presentation of~\ref{ros:ap:maybe:a} to remain neutral on whether \ros{}

  % In general, presentation of~\ref{ros:ap:maybe:a}.
  % For, it may not be possible to reduce certain things captured by \ros{} to premise.
  % For example, consider a strong distinction between the role of premises and rules of inference following \citeauthor{Carroll:1895uj}'s \citetitle{Carroll:1895uj}.

  % Make this distinction clearer by a discussion of embedding.
\end{note}

\subsection{\supportI{}}
\label{cha:var:ros:I}

\begin{note}
  \supportI{} states, roughly, that the event in which an agent concludes \(\pv{\phi}{v}\) from \(\Phi\) is sufficient for a \ros{} to hold between \(\pv{\phi}{v}\) and \(\Phi\).

  \begin{idea}[\supportI{}]
    \label{idea:support}
    For an agent \vAgent{}, a proposition-value pair \(\pv{\phi}{v}\), \poP{} \(\Phi\), and event \(e\):

    \begin{itemize}
    \item[\emph{If}:]
      \begin{enumerate}[label=\alph*., ref=(\alph*)]
      \item
        \(e\) is an event in which \vAgent{} concludes \(\pv{\phi}{v}\) from \(\Phi\).
      \end{enumerate}
    \item[\emph{Then}:]
      \begin{enumerate}[label=\alph*., ref=(\alph*), resume]
      \item
        When \vAgent{} pairs \(\phi\) with \(v\) as a sub-event of \(e\):
        \begin{itemize}
        \item
          A \emph{\ros{}} between \(\pv{\phi}{v}\) and \(\Phi\) holds, from \agpe{\vAgent{}'}.
        \end{itemize}
      \end{enumerate}
    \end{itemize}
    \vspace{-\baselineskip}
  \end{idea}

  The focus on the sub-event in which the agent pairs \(\phi\) with \(v\) is to allow for the \ros{} to, in part, explain why the agent concludes \(\pv{\phi}{v}\) from \(\Phi\) without requiring that the \ros{} holds, from the \agpe{} prior to the agent forming the conclusion that \(\phi\) has value \(v\).

  In this respect, a \ros{} between \(\pv{\phi}{v}\) and \(\Phi\) may be regarded as a static account of how the agent has come to pair \(\phi\) with \(v\).
  In other words, the \ros{} between \(\pv{\phi}{v}\) and \(\Phi\) just captures whatever it is, from the \agpe{}, that led to the agent concluding \(\pv{\phi}{v}\) from \(\Phi\).

  Still, \supportI{} is only a sufficient condition, and the suggestion --- however the details work out --- is intended as intuition for when the agent concludes \(\pv{\phi}{v}\) from \(\Phi\).
  As we will see when discussing \supportII{}, we will deny the converse of \supportI{}.
  Therefore, the intuition is not suitable to capture in general what a \ros{} holding, from an \agpe{}, amounts to.

  Generalised, what it is, that has, is, or will, relate \(\pv{\phi}{v}\) and \(\Phi\), from the \agpe{}.
\end{note}

\begin{note}
  Given \supportI{}, we will always qualify that a \ros{} holds \emph{from an \agpe{}}.
  Our interest is with conclusions, parings of propositions with values by an agent.
  Hence, we have no interest in whatever the idea of a \ros{} between \(\pv{\phi}{v}\) and \(\Phi\) simpliciter.
  As outlined in \autoref{cha:clar:sec:CCC:pvp}, we place no restrictions on conclusions.
  Hence, if an agent concludes that the ratio of the long side to the short side of a piece of paper is not \(\sqrt{2}\) from some \poP{} \(\Phi\), then, by \supportI{}, a \ros{} holds between:
  \(\pv{\text{The ratio of the long side to the short side of a piece of paper is not }\sqrt{2}}{\text{True}}\) and \(\Phi\).
\end{note}

\begin{note}
 \supportI{} is similar to, but designed to be distinct from,~\citeauthor{Boghossian:2014aa}'s Taking Condition:%
  \footnote{
    There are various objections to~\citeauthor{Boghossian:2014aa}'s Taking Condition, though we take no stance on whether~\citeauthor{Boghossian:2014aa}'s Taking Condition holds.

    See, for example,~\textcite{Hlobil:2014tq}, \textcite{McHugh:2016vp}, and~\textcite{Wright:2014tt}.

    \citeauthor{Hlobil:2014tq} argues against the Taking Condition as it distracts from what accounts of reasoning out to explain, rather than arguing against the Taking Condition directly.

    \citeauthor{McHugh:2016vp} summarise various objects to the taking condition, and present district arguments against against (distinct) ideas in favour of the taking condition.
    In particular,~\supportI{} is closer to what \citeauthor{McHugh:2016vp} term the `Consequence Condition' (\citeyear[cf.][316]{McHugh:2016vp}), which \citeauthor{McHugh:2016vp} also (indirectly) argue against.
    However, \citeauthor{McHugh:2016vp} does not consider an alternative account of what distinguishes concluding from any other action, and as~\supportI{} is designed to capture this distinction, it is unclear to me whether \citeauthor{McHugh:2016vp}'s arguments apply to~\supportI{} (if, indeed, they are sound).

    \citeauthor{Wright:2014tt} denies that reasoning must involve a state which connects premises to conclusions, as discussed in the main body of this section. (\citeyear[Cf.][33-34]{Wright:2014tt})
  }

  \begin{quote}
    (Taking Condition):
    Inferring necessarily involves the thinker \emph{taking} his premises to support his conclusion and drawing his conclusion because of that fact.%
    \mbox{}\hfill\mbox{(\citeyear[5]{Boghossian:2014aa})}
  \end{quote}

  There is an immediate superficial difference in that~\citeauthor{Boghossian:2014aa} states the Taking Condition in terms of inferring.
  However, `a conclusion' may be substituted for `inferring' and an important distinction remains.
  For, `taking' is understood by \citeauthor{Boghossian:2014aa} to be substantive.

  \citeauthor{Boghossian:2014aa} illustrates with the following \scen{}:
  \begin{quote}
    On waking up one morning I recall that:

    \begin{enumerate}[label=(\arabic*), ref=(\arabic*), series=BogEx]
    \item
      \label{BogEx:1}
      It rained last night.
    \end{enumerate}

    I combine this with my knowledge that

    \begin{enumerate}[label=(\arabic*), ref=(\arabic*), resume*=BogEx]
    \item
      \label{BogEx:2}
      If it rained last night, then the streets are wet.
    \end{enumerate}

    to conclude:

    So,

    \begin{enumerate}[label=(\arabic*), ref=(\arabic*), resume*=BogEx]
    \item
      \label{BogEx:3}
      The streets are wet.
    \end{enumerate}
    This belief then affects my choice of footwear.%
    \mbox{ }\hfill\mbox{(\citeyear[2]{Boghossian:2014aa})}
  \end{quote}

  And \citeauthor{Boghossian:2014aa} expands as follows:

  \begin{quote}
    [M]y inferring from~\ref{BogEx:1} and~\ref{BogEx:2} to~\ref{BogEx:3} must involve my arriving at the judgment that~\ref{BogEx:3} in part \emph{because} I \emph{take} the presumed truth of~\ref{BogEx:1} and~\ref{BogEx:2} to provide support for~\ref{BogEx:3}.
    Let us call this insistence that an account of inference must in this way incorporate a notion of ``taking'' the Taking Condition on inference.%
    \mbox{ }\hfill\mbox{(\citeyear[4]{Boghossian:2014aa})}
  \end{quote}

  Hence, for \citeauthor{Boghossian:2014aa}, the Taking Condition captures something \emph{in addition} to~\ref{BogEx:3} being a conclusion from a \poP{} which includes~\ref{BogEx:1} and~\ref{BogEx:2}.
  The presence of `taking' has a distinctive role in classifying the move from~\ref{BogEx:1} and~\ref{BogEx:2} to~\ref{BogEx:3} as an inference (or as a conclusion).

  In contrast, we do not require that a \ros{} has any particular role \emph{for the agent} in event in which an agent concludes \(\pv{\phi}{v}\) from \(\Phi\).
  If an agent concludes~\ref{BogEx:3} from~\ref{BogEx:1} and~\ref{BogEx:2}, then a \ros{} holds between~\ref{BogEx:3} and \(\{\ref{BogEx:1}, \ref{BogEx:2}\}\), from the \agpe{} ---~\ref{BogEx:3} and \(\{\ref{BogEx:1}, \ref{BogEx:2}\}\) are related, in some way by the agent.
  However, the \ros{} between~\ref{BogEx:3} and \(\{\ref{BogEx:1}, \ref{BogEx:2}\}\) need not itself have a role in the agent conclusion of~\ref{BogEx:3} from \(\{\ref{BogEx:1}, \ref{BogEx:2}\}\).
\end{note}

\begin{note}
  \phantlabel{Wright-simple-supportI}
  Indeed, our intuitive understanding of \ros{} is close to \citeauthor{Wright:2014tt}'s (\citeyear{Wright:2014tt}) `Simple Proposal':
  \begin{quote}
    [C]onsider instead the proposal, not that the status of the transition as inferential depends on the thinker's judgments about his reasons, but that it depends on \emph{what his reasons are}.
    We want his acceptance of the premises to supply his \emph{actual} reasons for accepting the conclusion.
    \dots

    Call this the Simple Proposal.
    It says that a thinker infers q from p\(_{1}\) \(\cdots\) p\(_{\text{n}}\) when he accepts each of p\(_{1}\) \(\cdots\) p\(_{\text{n}}\), moves to accept q, and does so for the reason that he accepts p\(_{1}\) \(\cdots\) p\(_{\text{n}}\).%
    \mbox{}\hfill\mbox{(\citeyear[33]{Wright:2014tt})}
  \end{quote}

  \citeauthor{Wright:2014tt}'s simple proposal is that, from the \agpe{}, the relation between a conclusion and some \poP{} need not be part of what moves the agent to conclude the conclusion from the \poP{}.

  \begin{quote}
    What is needed, then, is an account of, or at least some insight into, what it is for certain intentional states of a thinker to be his actual reasons for his transition to another intentional state.
    \dots
    We need to avoid committing to the notion that doing something for certain reasons must involve a state that somehow registers those reasons as reasons for what one does.%
    \mbox{}\hfill\mbox{(\citeyear[34]{Wright:2014tt})}
  \end{quote}

  Still, anticipating the role of \ros{} in construction a variation to \qWhy{}, it intuitively remains the case that a \ros{} explains, in part, why an agent concluded the conclusion from the \poP{} for, the \ros{} captures \emph{that} the agent accepted each of the premises and moved to accept the conclusion (in \citeauthor{Wright:2014tt}'s terminology).

  Still, following the discussion above, there is an important between~\supportI{} and \citeauthor{Wright:2014tt}'s Simple Proposal.
  For,~\supportI{} is an entailment, while \citeauthor{Wright:2014tt}'s Simple Proposal is an identity statement.
  Inferring, on the Simple Proposal, is an agent accepting some conclusion for the reason that they accept premises from some \poP{}.
  \supportI{} does not entail that concluding is nothing more than moving to accept \(\pv{\phi}{v}\) as a result of accepting each element of \(\Phi\).
\end{note}

\subsection{A \wit{0} for a \ros{0}}
\label{cha:var:ros:W}

\begin{note}
  We define a \wit{3} for \ros{1} as follows:

  \begin{definition}[A \wit{2}]
    \label{def:witnessing}
    For an agent \vAgent{}, proposition-value pair \(\pv{\phi}{v}\), and \poP{} \(\Phi\):

    \begin{enumerate}[label=]
    \item
      \begin{enumerate}[label=\alph*., ref=(\alph*), series=WitnessDef]
      \item
        \vAgent{} has a \emph{\wit{0}} for \ros{} between \(\pv{\phi}{v}\) and \(\Phi\).
      \end{enumerate}
    \item
      \emph{If and only if:}
    \item
      \begin{enumerate}[label=\alph*., ref=(\alph*), resume*=WitnessDef]
      \item
        There is some event \(e\) such that \(e\) is an event in which \vAgent{} concludes \(\pv{\phi}{v}\) from \(\Phi\).
      \end{enumerate}
    \end{enumerate}
    \vspace{-\baselineskip}
  \end{definition}

  \ros{3} hold from an \agpe{}.
  Hence, we say that \emph{an agent has} a \wit{0} for some \ros{} to implicitly capture that \ros{} are agent relative.
\end{note}

\begin{note}
  An important, but trivial, case of \autoref{def:witnessing} is when an agent concludes \(\pv{\phi}{v}\) from \(\Phi\).
  For, if an agent concludes \(\pv{\phi}{v}\) from \(\Phi\) then it is immediate that there is some event in which the agent concludes \(\pv{\phi}{v}\) from \(\Phi\) --- the very same event --- and hence the agent has a \wit{} for the \ros{} between \(\pv{\phi}{v}\) and \(\Phi\).

  Hence, joining \supportI{} with \autoref{def:witnessing}, we have the following:

  \begin{proposition}[Concludes, then witnessed \support{}]
    \label{prop:cws}
    For an agent \vAgent{}, proposition-value pair \(\pv{\phi}{v}\) and \poP{} \(\Phi\):
    \begin{itemize}
    \item
      If \(e\) is an event in which \vAgent{} concludes \(\pv{\phi}{v}\) from \(\Phi\) then:
      \begin{itemize}
      \item
        When \vAgent{} pairs \(\phi\) with \(v\) as a sub-event of \(e\), a \ros{} between \(\pv{\phi}{v}\) and \(\Phi\) holds, from \agpe{\vAgent{}'}.
      \item
        \vAgent{} has a \wit{} for the \ros{} between \(\pv{\phi}{v}\) and \(\Phi\).
      \end{itemize}
    \end{itemize}
    \begin{argument}
      Immediate for assuming the antecedent and appealing to \supportI{} and \autoref{def:witnessing}, respectively.
    \end{argument}
  \end{proposition}

  \autoref{prop:cws} is of interest with respect to \qWhy{}, \qHow{}, \issueInclusion{}, and the variations to follow in \autoref{cha:var:sec:vars}.

  For, our variant to \qWhy{} will involve \ros{1}.
  Our variant to \qHow{} will involve \wit{1}.
  And, our variant to \issueInclusion{} will hold that a \ros{} is, in part, an answer to why an agent concluded only if the agent has a \wit{} for the \ros{}.

  Hence, \autoref{prop:cws} ensures that so long as there is an event in which the agent concludes \(\pv{\phi}{v}\) from \(\Phi\), then an answer to `why' will always have a corresponding answer to `how'.

  At issue is whether it is always the case that an agent has a \wit{} for a \ros{} which is, in part, an answer to why the agent concluded \(\pv{\phi}{v}\) from \(\Phi\).

  And, given \autoref{prop:cws} it is immediate that any such \ros{} must be distinct from the \ros{} between \(\pv{\phi}{v}\) and \(\Phi\).
\end{note}

\begin{note}
  Note, when we talk of \wit{1} we talk in terms of `having a \wit{0}'.
  In the case of \autoref{prop:cws}, the event in which the agent concludes and the event which secures the relevant \wit{} are identical.

  However, event \(e\) may be an event in which an agent concludes \(\pv{\phi}{v}\) from \(\Phi\) such that throughout the event \(e\), the agent has a \wit{} for a \ros{} between \(\pv{\psi}{v'}\) and \(\Psi\), such that the event \(e'\) which \wit{1} the \ros{} between \(\pv{\psi}{v'}\) and \(\Psi\) is distinct from \(e\).

  Hence, our understanding of `having a \wit{0}' allows for the possibility that some \ros{} between \(\pv{\psi}{v'}\) and \(\Psi\), in part, `answers why' an agent concludes \(\pv{\phi}{v}\) from \(\Phi\) though the relevant \wit{0} for the \ros{} between \(\pv{\psi}{v'}\) and \(\Psi\) is distinct.

  If you think there may be such cases, then the variant to \issueInclusion{} that we develop will be compatible with such cases.
  And, if you think there are no such cases, then it is safe to ignore this possibility.
  We will not directly, at least, consider such cases or take a stand either way in the main argument.%
  \footnote{
    \phantlabel{fn:past-witness}
    To illustrate, consider an agent working on some mathematical problem.

    As part of their work on the problem the agent concludes the hypotenuse of some right-angled triangle is \(\sqrt{74}\text{cm}\) by use of the Pythagorean theorem.

    Further, the agent has, at some point in the past proved the Pythagorean theorem from more basic principles.

    Now, generally speaking, it may be the case that the agent concludes the hypotenuse of the triangle is \(\sqrt{74}\text{cm}\), in part, from those more basic principles.
    For example, the agent may have just completed their proof of the Pythagorean theorem and the reasoning from the more basic principles to the hypotenuse of the triangle may be considered a single unified instances of reasoning, with an intermediary conclusion.

    Further, suppose the agent proved the Pythagorean theorem some years ago.

    Perhaps the agent's reasoning from more basic principles continues to provide, in part, an answer to how the agent concluded the hypotenuse of the triangle is \(\sqrt{74}\text{cm}\).
    Perhaps, regardless of the gap, the agent used the Pythagorean theorem \emph{because} they concluded the theorem from more basic principles.

    On the other hand, one may be inclined to hold that the more basic principles have no role explanatory role in the present.
    At best, the agent's \emph{memory} of --- rather than the event of --- concluding answers, in part, why the agent concluded hypotenuse of the triangle is \(\sqrt{74}\text{cm}\).
  }
\end{note}

\begin{note}
  Though we will not take a stand on whether a relevant \wit{0} for some conclusion is distinct from the event in which the agent concludes, the possibility of separation highlights an plausible issue with \autoref{def:witnessing}.

  For, if separation may occur, it seems there may be instances where an agent reasoned to \(\pv{\phi}{v}\) but did not conclude \(\phi\) has value \(v\) such that the event in which the agent reasoned to \(\pv{\phi}{v}\) serves as a \wit{0} to a \ros{} between \(\pv{\phi}{v}\) and \(\Phi\).

  As \autoref{def:witnessing} requires the event to be such that the agent concludes \(\pv{\phi}{v}\) from \(\Phi\), such events are excluded from being \wit{1}.

  To illustrate, consider an agent working through a proof of some theorem.

  Abstractly, let \(\theta\) be the state of affairs characterised by the theorem, and let \(\Theta\) be the relevant \poP{}.
  Our interest is with the conclusion \(\pv{\theta}{\text{True}}\) from \(\Theta\).

  Suppose the agent reasons to \(\pv{\theta}{\text{True}}\) from \(\Theta\).
  Further, suppose the agent's reasoning is sound.
  However, the agent is worried about some parts of their reasoning.
  Hence, given their worries, \emph{reasons} to --- but does not conclude --- \(\pv{\theta}{\text{True}}\) from \(\Theta\).

  Some time later the agent revisits proof, resolves their worries, and concludes the theorem is true.

  I see no issue with the \emph{idea} that:
  \begin{itemize}[noitemsep]
  \item
    When the agent revisited the proof, they concluded \(\pv{\theta}{\text{True}}\) from \(\Theta\).
  \item
    In part, a \ros{} between \(\pv{\theta}{\text{True}}\) and \(\Theta\), from the \agpe{}, answers why the agent concluded \(\pv{\theta}{\text{True}}\) from \(\Theta\).
  \item
    The event in which the agent reasoned to \(\pv{\theta}{\text{True}}\) from \(\Theta\) answers, in part, how the agent \(\pv{\theta}{\text{True}}\) from \(\Theta\) by being a \wit{} for the \ros{} between \(\pv{\theta}{\text{True}}\) and \(\Theta\).
  \end{itemize}

  However, the idea is incompatible with the way we understand a \wit{0}.
  For, by definition, the relevant event which serves as a \wit{0} must be an event in which the agent \emph{concludes} \(\pv{\theta}{\text{True}}\) from \(\Theta\).
  And, by construction of the \scen{0}, the agent's worries prevent the agent from forming the relevant conclusion.

  There are various ways to square the \scen{0} with our understanding of a \wit{0}.
  For example, one may consider the extended event in which the agent reasons, returns, and concludes.
  Or, one may hold that when the agent concluded \(\pv{\theta}{\text{True}}\), the agent concluded \(\pv{\theta}{\text{True}}\) not from \(\Theta\), but from some \poP{} \(\Theta'\) which include the adequacy of the agent's prior reasoning as a premise.

  Still, it is not clear to me that either of the options suggested --- or any other option --- is preferable to weakening \autoref{def:witnessing} in such a way that an event \(e\) which serves as a \wit{} to some \ros{} falls short of being an event in which an agent concludes.

  The difficulty is providing an adequate characterisation of the relevant event.
  That the agent \emph{reasoned} to \(\pv{\theta}{\text{True}}\) form \(\Theta\) is insufficient in general.

  For example, consider a variation of the \scen{} in which the agent identifies a problem with the proof.
  Given the presence of a problem, the is --- intuitively --- no \ros{} for the agent to have a \wit{0} for.

  Maintaining (some) intuition with regards to what it is for an event to be a \wit{0} for a \ros{0} is our priority.
  Strictly, the way in which we put \autoref{def:witnessing} to work is fully compatible with substituting `reasons to' in place of `concludes', but an overly narrow definition is preferably to an unintuitive definition.
  Hence, in order to avoid a lengthy digression into sufficient conditions for an event to `\wit{0}' a \ros{} without the event being such that the agent concludes we simply require the event is such that the agent concludes.
\end{note}

\subsection{\supportII{}}
\label{cha:var:ros:II}

\begin{note}
  \supportI{} states that a \ros{} holds between \(\pv{\phi}{v}\) and \(\Phi\), from the \agpe{}, when an agent concludes \(\pv{\phi}{v}\).
  \supportII{}, in short, denies the converse of \supportI{} is the case.
  In full:

  \begin{idea}[\supportII{}]
    \label{idea:support:possible}
    For an agent \vAgent{}, a proposition-value pair \(\pv{\phi}{v}\), and \poP{} \(\Phi\):

    \begin{itemize}
    \item
      It is possible for both~\ref{idea:support:possible:a} and~\ref{idea:support:possible:b} be true:
      \begin{enumerate}[label=\alph*., ref=(\alph*)]
      \item
        \label{idea:support:possible:a}
        A \ros{} between \(\pv{\phi}{v}\) and \(\Phi\) holds, from \agpe{\vAgent{}'}.
      \item
        \label{idea:support:possible:b}
        \vAgent{} does not have a \wit{} for the \ros{} between \(\pv{\phi}{v}\) and \(\Phi\).
      \end{enumerate}
    \end{itemize}
    \vspace{-\baselineskip}
  \end{idea}

  If an agent does not have a \wit{} for a \ros{} between \(\pv{\phi}{v}\) and \(\Phi\), then there is no event in which the agent concludes \(\pv{\phi}{v}\) from \(\Phi\).
  So, \supportI{} states that an event in which the agent concludes \(\pv{\phi}{v}\) from \(\Phi\) is sufficient for a \ros{} between \(\pv{\phi}{v}\) and \(\Phi\) to hold, from the \agpe{}.
  In contrast, \supportII{} denies that an event in which the agent concludes \(\pv{\phi}{v}\) from \(\Phi\) is necessary for a \ros{} between \(\pv{\phi}{v}\) and \(\Phi\) to hold, from the \agpe{}.
\end{note}

\begin{note}
  \supportII{} has a key role in the overall argument for this document.
  For, as indicated, answers to the variant of \qWhy{} will concern \ros{}, and the variant of \qHow{} will concern whether the agent has a \wit{} for the relevant \ros{}.
  \supportII{}, then, allows for the \emph{possibility} that the kind of thing which answers, in part, why an agent concluded is not constrained by how the agent concluded.

  However, our motivation for \supportII{} is independent of the success of the overall argument of this document.

  In short, any constraint on answers to why an agent concludes by answers to how an agent concludes is substantial.
  And, given the sketch of the way in which we develop variations of \qWhy{} and \qHow{} from \autoref{cha:var:sec:wiggling}, denying that there is any instance in which a \ros{} may answer, in part, why an agent concluded \(\pv{\phi}{v}\) from \(\Phi\) without the agent having a \wit{} for the \ros{} amounts to a substantive constraint.

  Indeed, \supportII{} should not be of any immediate concern.
  For, there is a significant gap between:

  \begin{itemize}[noitemsep]
  \item
    There being a \ros{} between \(\pv{\psi}{v'}\) and \(\Psi\) from an \agpe{} without the agent having a \wit{} for the \ros{}.
  \item
    The \ros{} between \(\pv{\psi}{v'}\) and \(\Psi\), from the \agpe{}, answering, in part, and in some sense, why the agent concluded \(\pv{\phi}{v}\) from \(\Phi\).
  \end{itemize}
\end{note}


\subsubsection{\fc{3}}
\label{cha:var:ros:II:fcs}

\begin{note}
  In this section we briefly sketch plausible \scen{1} in which Clauses~\ref{idea:support:possible:a}~and~\ref{idea:support:possible:b} of \supportII{} are both true.

  In these \scen{1} some proposition-value pair \(\pv{\psi}{v'}\) is a `\fc{}'%
  \footnote{
    A hyphen between `foregone' and `conclusion' to indicate the specific interpretation of the term
  }%
  \(^{,}\)%
  \footnote{
    Related to \citeauthor{Firth:1978vi}'s (\citeyear{Firth:1978vi}) distinction between doxastic and propositional justification (or warrant).
    See also \citeauthor{Silva:2020aa} (\citeyear{Silva:2020aa}) --- esp.\ fn.\ 1.

    {\color{red}
      Compare \citeauthor{Firth:1978vi}'s example with Holmes and Watson (\citeyear[218]{Firth:1978vi}).
      Watson is presented with all the evidence Holmes used to that the coachman committed the murder, and that this provides Watson with sufficient epistemic reasons regardless of whether or not Watson forms any attitude, but it is not clear that Watson has the understanding to piece together the evidence laid before them.
    }

    However, no interest in justification.
  }
  from some \poP{} \(\Psi\).
\end{note}

\begin{note}
  For example, recall \autoref{illu:gist:calc}.
  The agent concluded \(23 \times 15 = 345\) from the testimony of a calculator.
  Still, \(23 \times 15\) is more-or-less basic arithmetic.
  And, the agent may have no difficulty doing such arithmetic.
  % Hence, after receiving the testimony of the calculator it may be clear to the agent that not only does \(23 \times 15 = 345\) follow from the testimony of the calculator, but \(23 \times 15 = 345\) also follows from the agent's understanding of arithmetic.
  So, given the opportunity, the agent would conclude \(23 \times 15 = 345\) from their understanding of arithmetic.
  And, by \supportI{}, if the agent were to take the opportunity, a \ros{} would hold between \(23 \times 15 = 345\) and some \poP{} that would be employed if the agent were to multiply \(23\) by \(15\).
  Hence, obtaining the \ros{} between \(23 \times 15 = 345\) and the relevant \poP{} from \supportI{} is pending only on the agent taking up the opportunity.
  

  In this respect, \ros{} is equally foregone.
  For, \ros{} follows obtains when the agent concludes \(\pv{\phi}{v}\).
  Hence, \ros{} without \wit{0}.
\end{note}


\begin{note}
  Above, described a general type of \scen{0}.
  Characterised \scen{0} by saying \(\pv{\phi}{v}\) is a \fc{0} from \(\Phi\).

  In this section, we briefly sketch \fc{0}.
\end{note}

\begin{note}
  As expressed above, the basic idea of \(\pv{\phi}{v}\) being a \fc{} from \(\Phi\) for some agent is that, given the opportunity, the agent would conclude \(\pv{\phi}{v}\) from \(\Phi\).

  We capture this basic idea via the progressive:

  \begin{sketch}[\fc{3}]
    For an agent \vAgent{}, proposition-value pair \(\pv{\phi}{v}\) and \poP{} \(\Phi\):

    \begin{enumerate}[label=]
    \item
      \begin{itemize}
      \item
        \(\pv{\phi}{v}\) is a \emph{\fc{0}} from \(\Phi\), for \vAgent{}.
      \end{itemize}
    \item
      \emph{If and only if}:
    \item
      \begin{itemize}
      \item
        Both~\ref{sketch:fc:exp:1}~and~\ref{sketch:fc:exp:2} are true:
        \begin{enumerate}[label=\alph*., ref=(\alph*)]
        \item
          \label{sketch:fc:exp:1}
          \vAgent{} has the option to do some action \(\mathcal{A}\).
        \item
          \label{sketch:fc:exp:2}
          The event in which \vAgent{} does \(\mathcal{A}\) is an event in which \vAgent{} is concluding \(\pv{\phi}{v}\) from \(\Phi\).
        \end{enumerate}
      \end{itemize}
    \end{enumerate}
    \vspace{-\baselineskip}
  \end{sketch}

  Note, \ref{sketch:fc:exp:2} only requires that the event \(e\) in which the agent does \(\mathcal{A}\) is such that \(e\) is an event in which the agent is \emph{concluding} \(\pv{\phi}{v}\) from \(\Phi\).
  Hence, \ref{sketch:fc:exp:2} does not require --- though is compatible with --- \(e\) being an event in which the agent concludes \(\pv{\phi}{v}\) from \(\Phi\).

  The difference is as follows:
  \begin{itemize}
  \item
    If \(e\) is an event in which an agent \emph{concludes} \(\pv{\phi}{v}\) from \(\Phi\), then \(e\) includes as a sub-event an event \(e^{-}\) in which the agent pairs \(\phi\) with \(v\).
  \item
    If \(e\) is an event in which an agent is \emph{concluding} \(\pv{\phi}{v}\) from \(\Phi\), then \(e\) does not necessarily include as a sub-event an event \(e^{-}\) in which the agent pairs \(\phi\) with \(v\).

    However, if \(e\) were allowed to develop, then \(e\) would develop into an event \(e^{+}\) such that \(e^{+}\) is an event in which the agent concludes \(\pv{\phi}{v}\) from \(\Phi\).
  \end{itemize}

  To illustrate the contrast,

  drawing a dog.

  draws a dog.

  If fire alarm goes off, was drawing a dog, even though the drawing is never completed.

  Likewise, as it happens, remarks that it's a nice drawing of a cat.
  Well, sure, maybe it would work as a cat.

  Note, does not start out as drawing a dog.
  Only after initial doodle does the event develop sufficient inertia.

  Concluding is the same.
\end{note}

\begin{note}
  Extends, \(\pv{\phi}{v}\) is a \fc{3} from \(\Phi\), \emph{from the \agpe{}}.
  In general, there is no entailment from \agpe{} to what is the case.
  However, our attention will be on cases in which agent \emph{knows} \(\pv{\phi}{v}\) is a \fc{3} from \(\Phi\).

  \autoref{illu:gist:calc}, again.
  Given testimony of the calculator and skill with arithmetic, agent knows that they would be concluding \(23 \times 15 = 345\) if they were to start.
\end{note}

\begin{note}

  \begin{sketch}
    \label{sketch:fc-then-ros}
    If \fc{} then \ros{}.
  \end{sketch}

  \fc{}.
  Then, action available to the agent.
  Concluding.
  Develops, concludes.
  Concludes, then \ros{}.

  Everything is in place.
  The agent need only perform the action.
  Hence, \ros{}.
\end{note}

\begin{note}
  \autoref{sketch:fc-then-ros} is important for obtaining \ros{1} without \wit{1}.
  However, the converse of \autoref{sketch:fc-then-ros} will also be important:

  \begin{itemize}
  \item
    If no \ros{} then not a \fc{}.
  \end{itemize}
\end{note}

\begin{note}
  Abstract.

    Instances in which an agent knows that concluding would be in progress exist.
  For example, consider basic arithmetic.
  Whether or not \(4131 + 1533 = 5664\) is a \fc{}.%
  \footnote{
    More generally, take any \(n\) and \(m\) such that the process is adding \(n\) and \(m\) would not take too long.
  }
  If you started, would be determining whether or not the equality holds.

  In most cases we will push a little further than addition.
  However, share the same pattern of the conclusion following from the application of some collection of rules which an agent knows.

  For example, enrich the collection of mathematical operations to include subtraction, multiplication, division, square-roots and so on.

  Beyond mathematics, but close, formal logic.
  In particular, theoretical results such tautologies of propositional logic, or meta-theoretical results which are generated from a common method, such as completeness proofs of various modal logics via canonical models.

  And, finally games.

  If you know the appropriate strategy, play first, and so desire, then it is a \fc{0} that any game of noughts and crosses will either end in a win for you or a draw.%
  \footnote{
    For details, see~(\cite[94--96]{Gardner:1983wn}).
  }

  Sudoku puzzles.
  Rules are simple, and I expect that if you have some experience with solving Sudoku puzzles, then you know that the solution to the Sudoku puzzle is a \fc{}.
  In the worst case scenario, you have the option to brute force the solution to the puzzle.

  A slightly more interesting example is chess problems.
  In particular, there is plausibly some bound where solution to a problem fails to be a \fc{}.
  However, any problem within the bound is a \fc{}.
  May not know where the bound is.
  Yet, solutions to some problems within the bound are know to be \fc{1}.
  For example, whether or not there is an available move for some piece is a \fc{1}, but whether there is a sequence of move that will result in checkmate for either player is often not (known) to be a \fc{0}.
\end{note}

\subsection{Embedding \ros{1}}
\label{cha:var:ros:Emb}

\begin{note}
  The previous sections detailed the three key ideas by which we understand \ros{1} in this document.

  The present section concerns, for some arbitrary proposition-value pair \(\pv{\phi}{v}\) and \poP{} \(\Psi\), the distinction between:

  \begin{enumerate}[label=\arabic*., ref=(\arabic*)]
  \item
    \label{Embed:no}
    A \ros{0} between \(\pv{\psi}{v'}\) and \(\Psi\).
  \item
    \label{Embed:yes}
    A \ros{0} between \(\Phi\) and \(\pv{\phi}{v}\), where:
    \begin{itemize}
    \item
      \(\Phi\) contains the proposition-value-premises pairing:
      \begin{itemize}
      \item
        \(\pv{\text{A \ros{} between }\pv{\psi}{v'}\text{ and }\Psi}{\text{True}}\)
      \end{itemize}
    \end{itemize}
  \end{enumerate}

  \ref{Embed:no} is a \ros{} between \(\pv{\psi}{v'}\) and \(\Psi\).
  Likewise, \ref{Embed:yes} involves a \ros{} between \(\pv{\psi}{v'}\) and \(\Psi\).
  However, the \ros{} between \(\pv{\psi}{v'}\) and \(\Psi\) is itself a premise in a \ros{} between \(\pv{\phi}{v}\) and \(\Phi\).
  In this respect, we will say the \ros{} between \(\pv{\psi}{v'}\) and \(\Psi\) is \emph{embedded} within a \ros{} --- specifically the \ros{} between \(\pv{\phi}{v}\) and \(\Phi\).
\end{note}

\begin{note}
  Embedding of this kind will be:
  \begin{itemize}[noitemsep]
  \item
    Useful for relating the variations of \qWhy{} and \qHow{} to theories of concluding, or related phenomena. And,
  \item
    Important for clarifying a difficult for developing counterexamples to \issueInclusion{}.
  \end{itemize}
\end{note}

\subsubsection{Definitions}
\label{cha:var:ros:Emb:defs}

\begin{note}
  In full, we define embedding in the following way:%
  \footnote{
    We assume, in general, that if a proposition \(\phi\) includes some other proposition \(\phi'\), then if \(\pv{\phi}{\text{True}} \in \Phi\) then \(\pv{\phi'}{\text{True}} \in \Phi\).
    In other words, if \(\pv{\phi'\text{ and }\phi''}{\text{True}} \in \Phi\) then both \(\pv{\phi'}{\text{True}} \in \Phi\) and \(\pv{\phi''}{\text{True}} \in \Phi\).
    We do not extend this assumption to values other than `True'.
  }

  \begin{definition}[Degree of embedding withing a \ros{}]
    \label{def:embedding:degree}
    For a proposition-value pairs \(\pv{\psi}{v'}\), \(\pv{\phi}{v}\), \poP{} \(\Phi\), and \(i \in \mathbb{N}\):

    \begin{itemize}
    \item
      \(\pv{\psi}{v'}\) has a \emph{degree of embedding \(1\)} with respect to a \ros{} between \(\pv{\phi}{v}\) and \(\Phi\) if and only if \(\pv{\psi}{v'} \in \Phi\).
    \item
      \(\pv{\psi}{v'}\) is has a \emph{degree of embedding \(i\)} with respect to a \ros{} between \(\pv{\phi}{v}\) and \(\Phi\) if and only if:
      \begin{itemize}
      \item
        There exists some \(\pv{\theta}{v''}\) and \(\Theta\) such that:
        \begin{itemize}
        \item
          \(\pv{\psi}{v'} \in \Theta\)
        \item
          \(\pv{\text{A \ros{} between }\pv{\theta}{v''}\text{ and }\Theta}{\text{True}}\) is an \(i - 1\) embedding with respect to the \ros{} between \(\pv{\phi}{v}\) and \(\Phi\).
        \end{itemize}
      \end{itemize}
    \end{itemize}
    \vspace{-\baselineskip}
  \end{definition}

  The cases of interest to us are where \(\pv{\psi}{v'}\) is embedded within in a \ros{} between \(\pv{\phi}{v}\) and \(\Phi\), no matter the degree of embedding:

  \begin{definition}[Embedding within a \ros{}]
    \label{def:embedding}
    For a proposition-value pairs \(\pv{\psi}{v'}\), \(\pv{\phi}{v}\), and a \poP{} \(\Phi\):


    \begin{itemize}
    \item
      \(\pv{\psi}{v'}\) is \emph{embedded} within in a \ros{} between \(\pv{\phi}{v}\) and \(\Phi\)
    \end{itemize}

    \emph{If and only if:}

    \begin{itemize}
    \item
      \(\pv{\psi}{v'}\) is has a degree of embedding \(i\) with respect to the \ros{} between \(\pv{\phi}{v}\) and \(\Phi\), for some \(i \in \mathbb{N}\).
    \end{itemize}
    \vspace{-\baselineskip}
  \end{definition}

  The definition of an embedding covers arbitrary proposition-value pairs.
  However, the cases of embedding of interest to us are where \ros{1} are embedded within a \ros{}.
  A final definition captures when this is the case:

  \begin{definition}[Embedded \ros{1}]
    For a proposition-value pairs \(\pv{\psi}{v'}\), \(\pv{\phi}{v}\), and \poP{1} \(\Phi\), \(\Psi\):

    \begin{itemize}
    \item
      The \ros{} between \(\pv{\psi}{v'}\) and \(\Psi\) is embedded within the \ros{} between \(\pv{\phi}{v}\) and \(\Phi\).
    \end{itemize}

    \emph{If and only if}

    \begin{itemize}[noitemsep]
    \item
      \(\chi\) is the proposition `\(\text{A \ros{} between }\pv{\psi}{v'}\text{ and }\Psi\)'.
    \item
      \(v''\) is the value `True'.
    \item
      \(\pv{\chi}{v''}\) is embedded within a \ros{} between \(\pv{\phi}{v}\) and \(\Phi\).
    \end{itemize}
    \vspace{-\baselineskip}
  \end{definition}
\end{note}

\subsubsection{Interpretation}
\label{cha:var:ros:Emb:interpretation}

\begin{note}
  Understanding in hand, turn to the function of distinguishing embedded from unembedded \ros{}.

  Simple.

  Whether the \ros{} was a premise from which the agent concluded \(\phi\) has value \(v\).
\end{note}

\begin{note}
  Idea is somewhat familiar from propositional logic.
  Certain kind of equivalence between proof and conditional.
  It is possible to find a corresponding conditional to any proof with a finite number of premises, proof captures derivation of conclusion from premises.

  Corresponding conditional is not a premise, nor any part, of the proof.

  For example, consider a proof from \(P\) and \(P \rightarrow Q\) to \(Q\) by conditional detachment.
  Corresponding conditional is \((P \land (P \rightarrow Q)) \rightarrow Q\).
  However, not part of the proof.

  Intuitive distinction between what a proof and a conditional refer to.
  However, informally there is no difficulty in treating a proof as a premise.
  \(P\), and I have a proof of \(P \rightarrow Q\), therefore \(Q\).
\end{note}

\begin{note}
  Discussion by \citeauthor{Carroll:1895uj} in \citetitle{Carroll:1895uj}.

  \begin{quote}
    My paradox \dots turns on the fact that, in a Hypothetical, the \emph{truth} of the Protasis, the \emph{truth} of the Apodosis, and the \emph{validity of the sequence}, are 3 distinct Propositions.

    \mbox{}\hfill\(\vdots\)\hfill\mbox{}

    Suppose I say ``I grant~\ref{AatT:a} and~\ref{AatT:b} and~\ref{AatT:c}, but I do \emph{not} grant that I am thereby \emph{obliged} to grant~\ref{AatT:z}.''
    Surely, my granting~\ref{AatT:z} must \emph{wait} until I have been made to see the validity of this sequence: i.e.\ in order to grant~\ref{AatT:z}, I must grant~\ref{AatT:a},~\ref{AatT:b},~\ref{AatT:c}, and~\ref{AatT:d}! And so on.%
    \mbox{ }\hfill\mbox{(\citeyear[472]{Carroll:1977wl})}
  \end{quote}

  \citeauthor{Carroll:1895uj} is slightly different.
  For, rather than \ros{}, \citeauthor{Carroll:1895uj} focuses on valid inferences.

  My understanding of \citeauthor{Carroll:1895uj} is in terms of the relation between general and specific.

  Inference from \(A\) and \(A \rightarrow B\) to \(B\).
  Valid inference.
  However, valid inference if and only if holds for all \emph{A} and \emph{B}.
  Point, then, is that given \(A\) and \(A \rightarrow B\) still need validity.
  Even if grant \(B\) is true, this only gives specific instance.
  However, at issue is general.
  Does \(D\) follow from \(C, C \rightarrow D\)?

  How do we get a general rule without already being sure of the specific instances of the general rule.

  Immediate observation is that general rule does not correspond to any specific instance to which the general rule applies.

  Hence, motivates the same idea.
  In some cases, these two things are different.
\end{note}

\begin{note}
  Start to see a difference by considering \(\pv{\phi}{v}\) and \(\Phi\).
  For, \ros{} between \(\pv{\phi}{v}\) and \(\Phi\).
  However, immediate difficulties if the proposition that there is a \ros{} between \(\pv{\phi}{v}\) and \(\Phi\) paired with the value `True' is a member of \(\Phi\).

  Likewise, \ros{} between the single premise and \(\pv{\phi}{v}\) is, intuitively, distinct from \ros{} between \(\pv{\phi}{v}\) and \(\Phi\).

  \ros{} between \(\pv{\phi}{v}\) and \(\Phi\) results from an event in which an agent concludes \(\pv{\phi}{v}\) from \(\Phi\).
  By contrast, \ros{} between \(\pv{\phi}{v}\) and \(\Phi\) results from an event in which an agent concludes \(\pv{\phi}{v}\) from the \ros{} between \(\pv{\phi}{v}\) and \(\Phi\).
\end{note}

\begin{note}
  \citeauthor{Boghossian:2008vf}:
  \begin{quote}
    [W]e would have to take as primitive a \emph{general (often conditional) content serving as the reason for which one believes something}, without this being mediated by inference of any kind.%
    \mbox{ }\hfill\mbox{(\citeyear[500]{Boghossian:2008vf})}
  \end{quote}
\end{note}

\begin{note}
  Significance for arguments is that dependence captured by \qWhyVnP{} does not amount to the \ros{} being a premise.
\end{note}

\subsection{Summary}
\label{cha:var:ros:summary}

\begin{note}
  \supportI{}.

  \wit{3}.

  \supportII{}.

  \fc{3}.
\end{note}

\begin{note}
  As indicated in \autoref{cha:var:sec:wiggling}, \autoref{cha:var:sec:vars} will use \ros{1} and \wit{3} to develop variations of \qWhy{}, \qHow{}, and \issueInclusion{} from \autoref{cha:introduction}.
\end{note}


\section{\qWhyVnP{}, \qHowV{}, and \issueConstraint{}}
\label{cha:var:sec:vars}

\begin{note}
  In \autoref{cha:var:sec:wiggling} we outlined how we will developed variations of \qWhy{}, \qHow{}, and \issueInclusion{} in terms of \ros{}.
  \autoref{cha:var:ros} outlined how we understand \ros{1}.
  In the present section we state the variations.
  We write the variants as `\qWhyVnP{}', `\qHowV{}', and `\issueConstraint{}', and the section is split into sub-sections which develop and discuss each variation.

  % \begin{enumerate}[label=]
  % \item
  %   \TOCLine{cha:var:sec:vars:qwhyvnp}
  % \item
  %   \TOCLine{cha:var:sec:vars:qhowv}
  % \item
  %   \TOCLine{cha:var:sec:vars:issue}
  % \end{enumerate}

  We term \qWhyVnP{} and \qHowV{} as \emph{variant} questions because \qWhyVnP{} and \qHowV{} are, respectively, too narrow and too broad to function as substitutes for \qWhy{} and \qHow{}.
  Hence, to clarify the way in which \qWhyVnP{} is a variant of \qWhy{} and the way in which \qHowV{} is a variant of \qHow{} we will explicitly link the variant to the initial question.
  We will then build the variant to \issueInclusion{} from \issueInclusion{} and the links between the variant questions and the initial questions.
\end{note}

\subsection{\qWhyVnP{}}
\label{cha:var:sec:vars:qwhyvnp}

\begin{note}
  As forecast in \autoref{cha:var:sec:wiggling}, eliminate `why' in favour of whether pairing \(\phi\) and \(v\) depends on \ros{}.

  Recall:
  \begin{quote}%
    \vspace{-1.5\baselineskip}%
    \questionWhyBasic*
  \end{quote}

  Rather than asking for proposition-value-premises pairings associated with explanations of `why' an agent concluded \(\pv{\phi}{v}\) from \(\Phi\), the variation of \qWhy{} queries which \ros{1} are such that the conclusion of \(\pv{\phi}{v}\) from \(\Phi\) \emph{depended} on the \ros{1} holding, from the \agpe{}.
  Where the relevant sense of `depends' is captured, roughly, by following the sketch:
  \phantlabel{dependence:rough}
  \nocite{Lewis:1973aa}

  \begin{sketch}[Dependence]
    \label{sketch:dependence}
    For things \(\mathbb{A}\) and \(\mathbb{B}\):
    \begin{itemize}
    \item
      \(\mathbb{B}\) depends on \(\mathbb{A}\)
    \end{itemize}
    \emph{If and only if}
    \begin{itemize}
    \item
      If \(\mathbb{A}\) were not the case, \(\mathbb{B}\) would not be the case.
    \end{itemize}
    \vspace{-\baselineskip}
  \end{sketch}

  In general, link between dependence and answers to why is difficult.
  For example,

  \begin{quote}
    If Smith was fired, Wilson was fired
  \end{quote}
  Wilson being employed depends on Smith being employed.

  However, as~\cite{Sanford:1989aa} observes, it may be the case that:

  \begin{quote}
    The firings of Smith and Wilson are completely independent in that neither resulted from the other, and there is nothing from which each resulted.
    But Wilson definitely was or will be the first to be fired.
    If anyone was fired, Wilson was fired.%
    \mbox{ }\hfill\mbox{(\citeyear[192--193]{Sanford:1989aa})}
  \end{quote}

  So, dependence in this sense is difficult.
  However, we are only interested in this sense of dependence with respect to an event in which an agent concludes \(\pv{\phi}{v}\) from \(\Phi\) and some \ros{}.

  Filling in the schema:

  \begin{quote}
    \begin{enumerate}
    \item[\emph{If}:]
      The \ros{0} between failed to hold, from \agpe{}.
    \item[\emph{Then}:]
      The agent would not have concluded \(\pv{\phi}{v}\) from \(\Phi\).
    \end{enumerate}
  \end{quote}


  Multiplication.

  Testimony of the calculator.

  Some instance of desire.

  Keep in mind, the dependence is between the \emph{event} and a \ros{}.
  For example, calculator, could have performed the relevant calculation.
  Hence, it seems plausible that the conclusion does not depend on the testimony of the calculator.
  However, the event itself, depended on the testimony of calculator.

  Indeed, trivially so.
  For, by \supportI{}, event, \ros{}.
  Hence, of no \ros{} then no event.
\end{note}

\begin{note}
  For clarity, the variant to \qWhy{} shifts perspective from after the agent has concluded \(\pv{\phi}{v}\) from \(\Phi\) to the (sub-)event in which the agent pairs \(\phi\) with \(v\).

  The basic idea is that a \ros{} between \(\pv{\psi}{v'}\) and \(\Psi\), in part, explains why the agent pairs \(\phi\) with \(v\).
  For, given dependence, the agent would not have paired \(\phi\) with \(v\) without the \ros{} holding, from the \agpe{}.
  We will explicitly capture this idea with \linkW{}.
\end{note}

\subsubsection{Question}
\label{cha:var:sec:vars:qwhyvnp:question}

\begin{note}
  The variation of \qWhy{}, \qWhyVnP{} is as follows:

  \begin{restatable}[\qWhyVnP{}]{question}{questionWhyVnP}
    \label{q:why:v:nP}
    Given an agent \vAgent{}, proposition-value pair \(\pv{\phi}{v}\), \poP{} \(\Phi\), and event \(e\) in which \vAgent{} concludes \(\pv{\phi}{v}\) from \(\Phi\).

    \begin{quote}
      Which proposition-value-premises pairings \(\pvp{\psi}{v'}{\Psi}\) are such that, when \vAgent{} pairs \(\phi\) with \(v\):

      \begin{enumerate}[label=]
      \item
        \begin{enumerate}[label=\alph*., ref=(\alph*), series=qWhyVnPdef]
        \item
          \label{q:why:v:a}
          A \ros{0} between \(\pv{\psi}{v'}\) and \(\Psi\) holds, from \agpe{\vAgent{}'}.
        \end{enumerate}
      \end{enumerate}

      And:

      \begin{enumerate}
      \item[\emph{If}:]
        \begin{enumerate}[label=\alph*., ref=(\alph*), resume*=qWhyVnPdef]
        \item
          \label{q:why:v:if}
          The \ros{0} between \(\pv{\psi}{v'}\) and \(\Psi\) failed to hold, from \agpe{\vAgent{}'}.
        \end{enumerate}
      \item[\emph{Then}:]
        \begin{enumerate}[label=\alph*., ref=(\alph*), resume*=qWhyVnPdef]
        \item
          \label{q:why:v:then}
          \vAgent{} would not have paired \(\phi\) with \(v\) as a sub-event of \(e\).
        \end{enumerate}
      \end{enumerate}
    \end{quote}
    \vspace{-\baselineskip}
  \end{restatable}

  The core of \qWhyVnP{} is the \qWVitc{} from~\ref{q:why:v:if} to~\ref{q:why:v:then}.%
  \footnote{
    Clause~\ref{q:why:v:a} is only separated from the \qWVitc{} for ease of reference.

    I.e.\ substituting `A' for `The' in Clause~\ref{q:why:v:if} would yield an equivalent statement, so long as the \qWVitc{} is read as a counterfactual.
  }
  For, it is the \qWVitc{} which (somewhat roughly) captures the relevant sense of dependence.

  In contrast to the brief sketch of dependence \hyperref[dependence:rough]{above}, note that the \qWVitc{} applies when the agent pairs \(\phi\) with \(v\).

  The role of this restriction is to ensure the following proposition is true:

  \begin{proposition}
    \label{prop:ros-always-answer}
    For an agent \vAgent{}, proposition-value pair \(\pv{\phi}{v}\) and \poP{} \(\Phi\):

    \begin{itemize}
    \item
      The \ros{} between \(\pv{\phi}{v}\) from \(\Phi\) is always an answer to \qWhyVnP{}.
    \end{itemize}
    \begin{argument}
      By \supportI{}, a \ros{} holds between \(\pv{\phi}{v}\) and \(\Phi\), from the \agpe{}, when the agent pairs \(\phi\) with \(v\) as a sub-event of an event in which the agent concludes \(\pv{\phi}{v}\) from \(\Phi\).

      Hence, by restricting \qWhyVnP{} to the sub-event, we ensure that the \ros{} between \(\pv{\phi}{v}\) from \(\Phi\) holds, from the \agpe{}, and so may answer, in part, \qWhyVnP{}.

  And, the \ros{} between \(\pv{\phi}{v}\) from \(\Phi\) is guaranteed to be an answer to \qWhyVnP{} because, if the \ros{} between \(\pv{\phi}{v}\) from \(\Phi\) did not hold, from the \agpe{}, then the relevant event would be an event in which the agent concludes \(\pv{\phi}{v}\) from \(\Phi\).
    \end{argument}
  \end{proposition}

  Note, however, that though Clause~\ref{q:why:v:then} is restricted to the sub-event in which the agent pairs \(\phi\) with \(v\), Clause~\ref{q:why:v:then} remains tied to the super-event in which the agent concludes \(\pv{\phi}{v}\) from \(\Phi\).
  That is, in order for a \ros{} to answer \qWhyVnP{}, it must be the case that the agent would not have concluded \(\pv{\phi}{v}\) from \(\Phi\), not merely fail to pair \(\phi\) with \(v\).

  To illustrate, consider again \autoref{illu:gist:calc}, in which an agent concluded \(23 \times 15 = 345\) from the testimony of a calculator.
  If we only considered the event in which the agent pairs \(23 \times 15 = 345\) with the value True, the \ros{} between \(\pv{23 \times 15 = 345}{\text{True}}\) and a \poP{} which includes the testimony of the calculator may \emph{fail} to answer, in part, \qWhyVnP{}.
  For, it may still have the case that the agent paired \(23 \times 15 = 345\) with the value True due to the agent performing the calculation by their understanding of arithmetic.
  However, such an event would not be a sub-event of an event in which the agent concludes \(\pv{23 \times 15 = 345}{\text{True}}\) from a \poP{} which includes the testimony of the calculator.
\end{note}

\begin{note}
  Though I take \qWhyVnP{} to express a somewhat intuitive idea, the question is imprecise in two significant ways:
  \begin{itemize}[noitemsep]
  \item
    First, we have not specified in detail what \ros{} are.
  \item
    Second, \qWVitc{} is a subjunctive conditional.
  \end{itemize}
\end{note}

\begin{note}
  The imprecision arising from \ros{1} is by design.
  Recall, a \ros{} between \(\pv{\phi}{v}\) and \(\Phi\) is designed to capture, in an abstract way, some distinctive relation which holds between \(\pv{\phi}{v}\) and \(\Phi\) from the \agpe{}.
  Hence, by abstracting to \ros{1} we avoid any particular account of what is it for an agent to conclude \(\pv{\phi}{v}\) from \(\Phi\).

  Though, with the caveat that a \ros{} may hold between \(\pv{\phi}{v}\) and \(\Phi\) without there being an event in which the agent concludes \(\pv{\phi}{v}\) from \(\Phi\).%
  \footnote{
    I.e., recall \supportI{} and \supportII{} from \autoref{cha:var:ros}.
  }
\end{note}

\begin{note}
  The imprecision arising from the \qWVitc{} is in part by design.

  We noted above that the sense of dependence captured by the \qWVitc{} is not, in general, tightly connected to answering why questions.
  And, ideally, we would have a detailed account of the connexion between conclusions and \ros{1} which clarifies when link, captured by the conditional, between~\ref{q:why:v:if} and~\ref{q:why:v:then} holds.
  Still, we abstracted to \ros{1} to avoid any detailed account of what it is for an agent to conclude \(\pv{\phi}{v}\) from \(\Phi\).
  Hence, the abstracting to a conditional allows the relevant link between~\ref{q:why:v:if} and~\ref{q:why:v:then} to be developed in accordance with specific details given by some account of what concluding amounts to.

  Still, even granting the abstract degree of abstraction at which the \qWVitc{} holds, imprecision arises which is not be design.
  Further, the kind of imprecision which arises suggests significant difficulties for any detailed account of the \qWVitc{}.

  \begin{itemize}
  \item
    Entanglement.
  \end{itemize}

  For, the \qWVitc{} concerns two events, \(e\) in which the agent concludes \(\pv{\phi}{v}\) from \(\Phi\), and the sub-event \(e^{-}\) in which the agent pairs \(\phi\) with \(v\).
  As events, complex.

  A somewhat detailed account of the events is required to ensure that event depends on \ros{}.
  Recall, distinction between testimony of calculator and understanding of arithmetic.

  However, given required detail, \ros{1} may be entangled with the events in problematic ways.

  The most straightforward way to see how entanglement arises is by considering how prior \ros{1} \dots

  For, consider a \scen{0} in which an agent concludes \(23 \times 15 = 345\) from their understanding of arithmetic, such that some time shortly before the agent did the calculation, the agent concluded it would be good to get some coffee.

  For example, intuitively it should not be the case that the \ros{} and the conclusion that it would be good to get some coffee answers, in part, why the agent concluded some theorem is true from some \poP{}.
  Though, it may well be the case that the agent would have been too tired to conclude the theorem without the aid of the coffee.%
  \footnote{
    \citeauthor{Armstrong:1968vh} (\citeyear[195--196]{Armstrong:1968vh}) discusses a similar example, and suggests a discussion of this issue may also be found in \textcite{Moore:1962up}.
  }

  Perhaps closest world, decided on some other stimulant.
  However, I don't think this works in general.
  Background so that coffee is the only thing available.

  Yet, problem arises regardless of the availability of coffee.
  For, it's plausible the agent may have decided otherwise.
  And, given tired, decided to use a calculator instead.%
  \footnote{
    Further, now distinction \ros{1} between \poP{1} and the conclusion to do arithmetic or use a calculator.
  }
  So, not just circumstances surrounding event, but features of the event which are important for capturing those \ros{1} which do, intuitively, explain, in part, why the agent concludes \(\pv{\phi}{v}\) from \(\Phi\).

  Worry.
  In short, it is not clear how dependence, as captured by the \qWVitc{}, handles certain simple cases where intuitions seem clear due to the way various \ros{1} may be entangled in an event in which an agent concludes.

  Whether or not there is a way to refine the \qWVitc{} to clearly following intuition regarding entanglement is unclear to me.
  Still, we will proceed with \qWhyVnP{}.
  For, it seems \qWhyVnP{} does capture, even if imprecisely, something about of substance regarding why an agent concludes \(\pv{\phi}{v}\) from \(\Phi\).
  Hence, any refinement or variant of \qWhyVnP{} will agree with \qWhyVnP{} on a collection of `core' \scen{1}.
  And, our interest is only with respect to identifying instances in which a \ros{} answers, in part, \qWhyVnP{}.
  Hence, so long as the \scen{1} we consider belong to the `core', the imprecision of \qWhyVnP{} will not detract from the overall argument.
\end{note}



\begin{note}
  By \supportI{}, trivially holds for \ros{} between \(\pv{\phi}{v}\) and \(\Phi\).
  For, if \support{} failed to hold, then the event would not be an event in which the agent concludes \(\pv{\phi}{v}\) from \(\Phi\).
  Though, without a clear account of what depending amounts to, there is no deductive argument that holds in general.

  By \supportII{}, rule out the subjunctive is not trivially true because event in which agent concludes is necessary for \ros{}.
\end{note}

\subsubsection{Link}
\label{cha:var:sec:vars:qwhyvnp:link}

\begin{note}
  We developed \qWhyVnP{} through the idea that:

  If an event in which an agent concludes \(\pv{\phi}{v}\) from \(\Phi\) depends on a \ros{1} between \(\pv{\psi}{v'}\) and \(\Psi\) holding, from the \agpe{}, then the \ros{1} between \(\pv{\psi}{v'}\) and \(\Psi\) answers, in part, why the agent concluded \(\pv{\phi}{v}\) from \(\Phi\).

  Hence, we link \qWhyVnP{} and \qWhy{} in the following way:

  \begin{restatable}[\qWhyVnP{} and \qWhy{}]{link}{linkSupportWhy}
    \label{link:why:support:pvpp}
    For an agent \vAgent{}, and proposition-value-premises pairings \(\pvp{\psi}{v'}{\Psi}\):

    \begin{itemize}
    \item[\emph{If}:]
      \begin{enumerate}[label=\alph*., ref=(\alph*)]
      \item
        A \ros{0} between \(\pv{\psi}{v'}\) and \(\Psi\) is, in part, an answer to \qWhyVnP{}.
      \end{enumerate}
    \item[\emph{Then}:]
      \begin{enumerate}[label=\alph*., ref=(\alph*), resume]
      \item
        \(\pvp{\psi}{v'}{\Psi}\) is, in part, an answer to \qWhy{}.
      \end{enumerate}
    \end{itemize}
    \vspace{-\baselineskip}
  \end{restatable}

  In short, \linkW{} states the \ros{} \(\pv{\psi}{v'}\) and \(\Psi\) which, in part, answers \qWhyVnP{} is sufficient for \(\pvp{\psi}{v'}{\Psi}\) to answer \qWhy{}, in part.

  Though, given the discussion of entanglement in \autoref{cha:var:sec:vars:qwhyvnp:question}, we take the conditional and to be implicitly restricted by some commonsense.

  We will have no interest in the converse of \linkW{}.
  And, there may be cases in which \(\pvp{\psi}{v'}{\Psi}\) answers, in part, \qWhy{} while there is no corresponding \ros{} between \(\pv{\psi}{v'}\) and \(\Psi\) that answers, in part, \qWhyVnP{}.
  Indeed, \qWhy{} is not restricted to \ros{1}.
\end{note}

\subsection{\qHowV{}}
\label{cha:var:sec:vars:qhowv}

\begin{note}
  \qHow{} is a broad question which asks, quite generally, for an explanation of how an agent concluded \(\pv{\phi}{v}\) in terms of what has happened.
  Recall:

  \begin{quote}%
    \vspace{-1.5\baselineskip}%
    \questionHowBasic*
  \end{quote}

  However, out interest with \qHow{} is limited to the way in which certain answers to \qHow{}, intuitively, constrain answers to \qWhy{} via \issueInclusion{}:

  \begin{quote}%
    \vspace{-1.5\baselineskip}%
    \issueInclusionFirst*
  \end{quote}

  Hence, \issueInclusion{} narrows interest to event which relate to proposition-value-premises pairings which answer \qWhy{}.
  In \autoref{cha:var:sec:vars:qwhyvnp} we developed a variation of \qWhy{} --- \qWhyVnP{} --- in terms of \ros{}.
  In turn, we develop a variation of \qHow{} in terms of \wit{1} to \ros{} which answer, in part, \qWhyVnP{}.

  As with \qWhy{} and \qWhyVnP{}, `\qHowV{}' refers to the variation of \qHow{}.
\end{note}

\begin{note}
  As \qHowV{} is developed in light of the way in which \issueInclusion{} intuitively constrains answers to \qWhy{} in terms of answers to \qHow{}, the link between \qHowV{} and \qHow{} is, likewise, restricted to be between those events which intuitively constrain answers to \qWhy{} via \issueInclusion{}.
\end{note}

\begin{note}
  As we will see with~\autoref{prop:phi-always-how}, the event in which the agent concludes \(\pv{\phi}{v}\) from \(\Phi\) will always answer, in part, the variation to \qHow{}.
\end{note}

\subsubsection{Question}
\label{cha:var:sec:vars:qhowv:sec:question}

\begin{note}
  In full our variant on \qHow{} is as follows:

  \begin{restatable}[\qHowV{}]{question}{questionHowV}
    \label{q:how:v}
    For an agent \vAgent{}, a proposition-value pair \(\pv{\phi}{v}\), and \poP{} \(\Phi\), and an event \(e\) in which \vAgent{} concludes \(\pv{\phi}{v}\) from \(\Phi\).

    \begin{enumerate}[label=]
    \item
      Which events \(e'\) are such that, when \vAgent{} pairs \(\phi\) with \(v\), both~\ref{q:how:v:a} and~\ref{q:how:v:b} hold:
      \begin{enumerate}[label=\alph*., ref=(\alph*), , series=qHowVdef]
      \item
        \label{q:how:v:a}
        A \ros{} between \(\pv{\psi}{v'}\) and \(\Psi\) answers \qWhyVnP{}.
      \item
        \label{q:how:v:b}
        \(e'\) is a \wit{0} for the \ros{} between \(\pv{\psi}{v'}\) and \(\Psi\).
      \end{enumerate}
    \end{enumerate}
    \vspace{-\baselineskip}
  \end{restatable}

  As with \qWhyVnP{}, we focus on the sub-event in which the agent pairs \(\phi\) with \(v\).
  And, as with \qWhyVnP{}, this is to ensure that a \ros{} holds, from the \agpe{}, by \supportI{},
  And hence, by \autoref{def:witnessing}, that the agent has a \wit{} for the \ros{} between \(\pv{\phi}{v}\) and \(\Phi\).
  Indeed, these two observations combine to ensure that the event in which an agent concludes \(\pv{\phi}{v}\) from \(\Phi\) is always an answer to \qHowV{}.

  \begin{proposition}
    \label{prop:phi-always-how}
    For an agent \vAgent{}, proposition-value pair \(\pv{\phi}{v}\) and \poP{} \(\Phi\):

    \begin{itemize}
    \item
      If \(e\) is the event in which \vAgent{} concludes \(\pv{\phi}{v}\) from \(\Phi\), then \(e\) is an answer to \qHowV{}.
    \end{itemize}
    \begin{argument}
      Suppose \(e\) is the event in which \vAgent{} concludes \(\pv{\psi}{v'}\) from \(\Psi\).
      By \autoref{prop:ros-always-answer} established that the \ros{} between \(\pv{\phi}{v}\) and \(\Phi\) is always an answer to \qWhyVnP{}.
      So, Clause~\ref{q:how:v:a} is satisfied.

      Likewise, by assumption \(e\) is the event in which \vAgent{} concludes \(\pv{\psi}{v'}\) from \(\Psi\).
      Hence, by \autoref{def:witnessing}, \(e\) is a \wit{} for the \ros{} between \(\pv{\psi}{v'}\) and \(\Psi\).
      So, Clause~\ref{q:how:v:b} is satisfied.

      And, as both Clause~\ref{q:how:v:a} and Clause~\ref{q:how:v:b} are satisfied, \(e\) is an answer to \qHowV{}.
    \end{argument}
  \end{proposition}
\end{note}

\begin{note}
  Now, from \autoref{prop:phi-always-how} we have that the event in which an agent concludes \(\pv{\phi}{v}\) from \(\Phi\) is always an answer to \qHowV{}.
  However, the way in which \qHowV{} queries `how' an agent concludes \(\pv{\phi}{v}\) from \(\Phi\) is somewhat obscure.

  As with \qWhyVnP{} our interest is with extensional adequacy, and specifically extensional adequacy with respect to \ros{1}.

  Consider any event.
  There are two cases.
  \ros{} which answers \qWhyVnP{} or no \ros{}.

  If no \ros{} then the agent's conclusion of \(\pv{\phi}{v}\) from \(\Phi\) does not depend on what happened.
  Hence, event is not of direct interest with respect to answering `how' an agent concluded \(\pv{\phi}{v}\) from \(\Phi\).

  If \ros{} which \qWhyVnP{}, then included as an answer to \qHowV{}.

  So, though we have abstracted to \ros{1} to avoid any account of what conclusion amounts to, the existence of a \wit{0} intuitively captures whatever it is of relevance that happened when the agent concluded \(\pv{\phi}{v}\) from \(\Phi\).

  And, as discussed in \autoref{cha:var:ros:W}, the existence of a \wit{0} allows for \ros{} which answer \qWhyVnP{} to be constrained by answers to \qHowV{}, even if the relevant \wit{0} occurs at some point prior to the event in which the agent concludes \(\pv{\phi}{v}\) from \(\Phi\).
\end{note}

\subsubsection{Link}
\label{cha:var:sec:vars:qhowv:sec:link}

\begin{note}
  Given that \qWhyVnP{} has a key role in the construction of \qHowV{}, the link between \qHow{} and \qHowV{} is a little more involved that the link between \qWhyVnP{} and \qWhy{}.

  \begin{restatable}[\qHowV{} and \qHow{}]{link}{linkHowWitnessing}
    \label{link:how-witnessing}
    For any proposition-value-premises pairing \(\pvp{\psi}{v'}{\Psi}\):
    \begin{itemize}
    \item[\emph{If}:]
      \begin{enumerate}[label=\alph*., ref=(\alph*)]
      \item
        \(\pvp{\psi}{v'}{\Psi}\) is, in part, an answer \qHow{} due to \ros{} between \(\pv{\psi}{v'}\) and \(\Psi\) being, in part, an answer to \qWhyVnP{}.
      \end{enumerate}
    \item[\emph{Then}:]
      \begin{enumerate}[label=\alph*., ref=(\alph*), resume]
      \item
        There exists some event \(e\) such that \(e\) is an event in which \vAgent{} concludes \(\pv{\psi}{v'}\) from \(\Psi\) and \(e\) is, in part, an answer \qHowV{}.
      \end{enumerate}
    \end{itemize}
    \vspace{-\baselineskip}
  \end{restatable}

  We understand  \(\pvp{\psi}{v'}{\Psi}\) being, in part, an answer \qHow{} `due to' \ros{} between \(\pv{\psi}{v'}\) and \(\Psi\) being, in part, an answer to \qWhyVnP{} in terms of \linkW{} and \issueInclusion{}.

  For, suppose \(\pv{\psi}{v'}\) and \(\Psi\) is, in part, an answer to \qWhyVnP{}.
  Then, \(\pvp{\psi}{v'}{\Psi}\) is, in part, an answer to \qWhy{} via \linkW{}.
  And, if \issueInclusion{} holds, then \(\pvp{\psi}{v'}{\Psi}\) is also, in part, an answer to \qHow{}.
  Therefore, \(\pvp{\psi}{v'}{\Psi}\) is, in part, an answer to \qHow{} `due to' \(\pv{\psi}{v'}\) and \(\Psi\) being, in part, an answer to \qWhyVnP{}.
\end{note}

\begin{note}
  To summarise the preceding discussion, \linkH{} is designed to capture the idea that \qHow{} is concerned with the process by which an agent concludes \(\pv{\phi}{v}\) from \(\Phi\).
  Hence, if \(\pvp{\psi}{v'}{\Psi}\) is, in part, an answer \qHow{} due to a \ros{} between \(\pv{\psi}{v'}\) and \(\Psi\) being, in part, an answer to \qWhyVnP{} then, there is some process 


  Then, intuitively, relationship between \(\pv{\phi}{v}\) and \(\Psi\) implicitly captures some part of the process by which the agent concluded \(\pv{\phi}{v}\) from \(\Phi\).
  Hence, as answers to \qHowV{} are given in terms of events, extract the event which corresponds to the pairing.

  Given \qHowV{} is designed to relate to \qWhyVnP{}, restrict pairings of interest to those which result in a \ros{0}.
\end{note}

\subsection{\issueConstraint{}}
\label{cha:var:sec:vars:issue}

\begin{note}
  In~\autoref{cha:var:sec:vars:qwhyvnp} we developed \qWhyVnP{}, a variation of \qWhy{}.
  And, in~\autoref{cha:var:sec:vars:qhowv} we developed \qHowV{}, a variation of \qHow{}.
  In both sections we established links between the variant --- \qWhyVnP{}, \qHowV{}  --- and initial --- \qWhy{}, \qHow{} --- questions.

  Indeed, the purpose of developing the variant questions is to clarify the initial questions to a sufficient degree so that counterexamples to the intuitive constraint between the initial questions --- \issueInclusion{} --- may be developed.

  In this section, we bring together the variant and initial questions and \issueInclusion{} to state, in some detail, how everything fits together.
\end{note}

\begin{note}
 We being with a proposition:

  \begin{restatable}[]{proposition}{propVariationsAndInclusion}
    \label{prop:support-and-witnessing}
    Assuming \linkW{}, \linkH{}, and \issueInclusion{} hold:

    For an agent \vAgent{}, proposition-value pairs \(\pv{\phi}{v}\), \(\pv{\psi}{v'}\), and \poP{1} \(\Phi\), \(\Psi\):

    \begin{itemize}
    \item
      \qWhyVnP{} is answered, in part, by a \ros{} between \(\pv{\psi}{v'}\) and \(\Psi\).
    \end{itemize}

    \emph{Only if}

    \begin{itemize}
    \item
      \qHowV{} is answered, in part, by \vAgent{}' \wit{0} for the \ros{} between \(\pv{\psi}{v'}\) and \(\Psi\).
    \end{itemize}
    \begin{argument}
      Suppose \qWhyVnP{} is answered, in part, by a \ros{} between \(\pv{\psi}{v'}\) and \(\Psi\).
      From~\linkW{} it is immediately follows that \(\pvp{\psi}{v'}{\Psi}\) answers, in part, \qWhy{}.
      And, given \issueInclusion{}, \(\pvp{\psi}{v'}{\Psi}\) answers, in part, \qHow{}.

      Further, \(\pvp{\psi}{v'}{\Psi}\) answers, in part, \qHow{} due to a \ros{} between \(\pv{\phi}{v}\) and \(\Psi\) being, in part, an answer to \qWhyVnP{}.
      Therefore, by \linkH{}, there is some \(e\) such that \(e\) is a \wit{0} for the \ros{} between \(\pv{\psi}{v'}\) and \(\Psi\).
    \end{argument}
  \end{restatable}
  \autoref{fig:relations-between-whys-and-hows} is a a visual representation of the argument for~\autoref{prop:support-and-witnessing}.
\end{note}

\begin{figure}[H]
  \centering
  \begin{tikzpicture}
    \tikzset{ansStyle/.style={%
        draw=gray,%
        text width=.45\textwidth,%
        rounded corners=2pt,%
      }%
    }
    %
    \node[ansStyle] (whyO) at (0,0) %
    {\qWhyVnP{} is answered by a \ros{0} between \(\pv{\psi}{v'}\) and \(\Psi\).};
    %
    \node[ansStyle] (whyA) at (2,-1.5) %
    {\qWhy{} is answered by \(\pvp{\psi}{v'}{\Psi}\).};
    %
    \node[ansStyle] (howA) at (4,-3) %
    {\qHow{} is answered by \(\pvp{\psi}{v'}{\Psi}\).};
    %
    \node[ansStyle] (witA) at (6,-4.5) %
    {\qHowV{} is answered by event which \wit{1} sppt.\ btw.\ \(\pv{\psi}{v'}\) and \(\Psi\).};
    %
    \path[->] ($(whyO.south)!0.9!(whyO.south west)$) edge [out=270, in=180] (whyA);
    \path[->] ($(whyA.south)!0.9!(whyA.south west)$) edge [out=270, in=180] (howA);
    \path[->] ($(howA.south)!0.9!(howA.south west)$) edge [out=270, in=180] (witA);
    %
    \node[text width=.5\textwidth] (1) at (1,-.8) {\linkW{}};
    \node[text width=.75\textwidth] (2) at (4.5,-2.25) {\issueInclusion{}};
    \node[text width=.5\textwidth] (3) at (5,-3.625) {\linkH{}};
  \end{tikzpicture}%
  \caption{Visual representation of~\autoref{prop:support-and-witnessing}}
  \label{fig:relations-between-whys-and-hows}
\end{figure}

\begin{note}
  As~\autoref{prop:support-and-witnessing} follows from \linkW{}, \linkH{}, and \issueInclusion{}, we consider the content of~\autoref{prop:support-and-witnessing} to be a parallel constraint to \issueInclusion{}:

  \begin{restatable}[\issueConstraint{}]{constraint}{issueConstraintStatement}
    \label{issue:has-witnessed}
    For an agent \vAgent{}, proposition-value pairs \(\pv{\phi}{v}\), \(\pv{\psi}{v'}\), and \poP{1} \(\Phi\), \(\Psi\):

    \begin{itemize}
    \item
      \qWhyVnP{} is answered, in part, by a \ros{} between \(\pv{\psi}{v'}\) and \(\Psi\).
    \end{itemize}

    \emph{Only if}

    \begin{itemize}
    \item
      \qHowV{} is answered, in part, by \vAgent{}' \wit{0} for the \ros{} between \(\pv{\psi}{v'}\) and \(\Psi\).
    \end{itemize}
    \vspace{-\baselineskip}
  \end{restatable}

  As with \issueInclusion{}, if \issueConstraint{} holds, then answers to \qWhyVnP{} are constrained by answers to \qHowV{}.

  Our direct goal is to develops counterexamples to \issueConstraint{}.
  For, if there are counterexamples to \issueConstraint{}, then it immediately follows by \autoref{prop:support-and-witnessing} that either \linkW{}, \linkH{}, or \issueInclusion{} fails to hold.

  We defended \linkW{} when developing \qWhyVnP{} in~\autoref{cha:var:sec:vars:qwhyvnp}.
  And, likewise, we defended \linkH{} when developing \qHowV{} in~\autoref{cha:var:sec:vars:qhowv}.
  And, though there are some difficulties with \qWhyVnP{}, \qHowV{}, \linkW{}, and \linkH{}, I consider the most plausible point of failure to be \issueInclusion{}.
\end{note}

\begin{note}
  Still, \issueConstraint{} may be considered as a direct constraint on answers to \qWhyVnP{} in terms of answers to \qHow{}.
  In particular, \autoref{prop:constraint-rewrite} expands our construction of \qWhyVnP{}  and \qHowV{} to provide an alternative statement of \issueConstraint{}.

  \begin{proposition}[\issueConstraint{}, rewritten]
    \label{prop:constraint-rewrite}
    For an agent \vAgent{}, proposition-value pairs \(\pv{\phi}{v}\), \(\pv{\psi}{v'}\), and \poP{1} \(\Phi\), \(\Psi\):

    \issueConstraint{} is equivalent to the following conditional:

    \begin{enumerate}
    \item[\emph{If}:]
      \begin{enumerate}[label=\alph*., ref=(\alph*)]
      \item
        \label{constraint-rewrite:a}
        \vAgent{} concluded \(\pv{\phi}{v}\) from \(\Phi\).
      \end{enumerate}
    \item[\emph{And}:]
      \begin{enumerate}[label=\alph*., ref=(\alph*), resume]
      \item
        \label{constraint-rewrite:b}
        \vAgent{} would not have concluded \(\pv{\phi}{v}\) from \(\Phi\), if a \ros{} between \(\pv{\psi}{v'}\) and \(\Psi\) failed to hold, from \agpe{\vAgent{}'}.
      \end{enumerate}
    \item[\emph{Then}:]
      \begin{enumerate}[label=\alph*., ref=(\alph*), resume]
      \item
        \label{constraint-rewrite:c}
        \vAgent{} has a \wit{0} for the \ros{} between \(\pv{\psi}{v'}\) and \(\Psi\).
      \end{enumerate}
    \end{enumerate}
    \begin{argument}
      Immediate by \autoref{prop:support-and-witnessing} and the construction to \qWhyVnP{} and \qHowV{}.

      The antecedent of the conditional~---~\ref{constraint-rewrite:a}~\&~\ref{constraint-rewrite:b}~---~correspond to a \ros{} being, in part, an answer to \qWhyVnP{} while the consequent~---~\ref{constraint-rewrite:c}~---~corresponds to the \wit{0} which answers, in part, \qHowV{}.
      The conditional then follows from \autoref{prop:support-and-witnessing}.
    \end{argument}
  \end{proposition}

  So, directly, \issueConstraint{} amounts to the constraint that in order for an agent's conclusion of \(\pv{\phi}{v}\) to depend on some \ros{} between \(\pv{\psi}{v'}\) and \(\Psi\), the agent must have concluded \(\pv{\psi}{v'}\) from \(\Psi\).
\end{note}

\subsubsection{Embeddings}
\label{cha:var:sec:vars:issue:embeddings}

\begin{note}
  Above, we observed how~\supportI{} and~\supportII{} may combine in such a way that \support{} holds between \(\pv{\psi}{v'}\) and \(\Psi\) (from an \agpe{}), though the agent has does not have a \wit{0} for the \ros{0} between \(\pv{\psi}{v'}\) from \(\Psi\) --- or more naturally stated, though the agent has not concluded \(\pv{\psi}{v'}\) from \(\Psi\).

  We have not yet seen any examples to suggest that there are cases in which a conclusion of \(\pv{\phi}{v}\) from \(\Phi\) depends on \ros{} between \(\pv{\psi}{v'}\) and \(\Psi\)%
  \footnote{
    Where either \(\phi \ne \psi\), \(v \ne v'\), or \(\Phi \ne \Psi\).
  }
  Still, I have expressed my intent.

  It need not be the case that an agent has a \wit{0} for a \ros{0} in order for \ros{} to be involved in answering \qWhyVnP{}.
\end{note}

\begin{note}
    \begin{enumerate}[label=\arabic*., ref=(\arabic*)]
  \item
    \label{Embed:again:no}
    A \ros{0} between \(\pv{\psi}{v'}\) and \(\Psi\).
  \item
    \label{Embed:again:yes}
    A \ros{0} between \(\Phi\) and \(\pv{\phi}{v}\), where:
    \begin{itemize}
    \item
      \(\Phi\) contains the proposition-value-premises pairing:
      \begin{itemize}
      \item
        \(\pv{\text{A \ros{} between }\pv{\psi}{v'}\text{ and }\Psi}{\text{True}}\)
      \end{itemize}
    \end{itemize}
  \end{enumerate}

  Now, suppose an agent does not have a \wit{0} for the \ros{} between \(\pv{\psi}{v'}\) and \(\Psi\).
  The upshot of the distinction between~\ref{Embed:again:no} and~\ref{Embed:again:yes} is as follows:

  \begin{itemize}
  \item
    If the \ros{0} of \ref{Embed:again:no} is, in part, an answer to \qWhyVnP{} then the \ros{0} is a counterexample to \issueConstraint{}.
  \item
    If the \ros{0} of \ref{Embed:again:yes} is, in part, an answer to \qWhyVnP{} then the \ros{0} is \emph{not} a counterexample to \issueConstraint{}.
  \end{itemize}


  Again, the difference is \emph{the way in which} the \ros{} functions with respect to the agent pairing \(\phi\) with \(v\).
  Whether the \ros{} functions as a premise when the agent concludes \(\pv{\phi}{v}\), or whether the \ros{} functions in a way that is different to a premise.
\end{note}

\subsection{Summary}
\label{cha:var:sec:vars:summary}

\begin{note}
  Overall argument.
  Links then answer to \qWhyVnP{} which is not constrained by \qHowV{}, then \issueInclusion{} fails.

  Three broad ways in which the overall argument may fail:
  \begin{enumerate}[label=\arabic*., ref=(\arabic*), noitemsep]
  \item
    The link between \qWhyVnP{} and \qWhy{} fails to hold.
  \item
    The link between \qHowV{} and \qHow{} fails to hold.
  \item
    We fail to develop counterexamples to \issueConstraint{}.
  \end{enumerate}

  Still, I hope to have developed \qWhyVnP{}, \qHowV{}, and \issueConstraint{} in such a way that both questions are of some interest independent of link

  In \autoref{cha:clar:sec:literature} we suggest how a handful of accounts of conclusion, or related, may be understood in terms of \qWhyVnP{} and \qHowV{}, and how the accounts motivate \issueConstraint{} as a constraint on answers to \qWhyVnP{} in terms of answers to \qHowV{}.
\end{note}



\section{\issueConstraint{} in the literature}
\label{cha:clar:sec:literature}

\begin{note}
  \autoref{cha:var:sec:vars} introduced \qWhyVnP{}, \qHowV{}, and \issueConstraint{}, variants to \qWhy{}, \qHow{}, and \issueInclusion{}, respectively.

  Linked together.

  In short, having a \wit{0} for a \ros{} is necessary for the \ros{} to, in part, explain why an agent concluded.

  Primary motivation is intuition.
  \scen{1} such as \autoref{illu:gist:calc}.

  Motivated via \citeauthor{Davidson:1963aa}.
  Though, \citeauthor{Davidson:1963aa} was a little more general.

  In this section we collect a handful of extracts from the literature which suggest this intuition is not merely an intuition, but a common theoretical constraint.

  Provide extract.
  Observe how plausibly understood in terms of \wit{} and constraint.

  In many cases, constraint is given by identification with \wit{}.

  In turn, understand this to implicitly constrain an answer to why.

  With causal accounts, more or less immediate.
  Nothing that is non-causal
\end{note}


\subsection{Causal}
\label{cha:clar:sec:literature:causal}

\begin{note}
  Causal theories of reasoning.

  Broadly, premises stand in a causal relation to conclusion.

  Cause is something which happens, therefore \wit{}.
  What is relevant to the activity is given why causal process..
\end{note}

\subsubsection{\textcite{Armstrong:1968vh}}

\begin{note}
  We being with~\cite{Armstrong:1968vh}'s (\citeyear{Armstrong:1968vh}) account of inferring.

  \begin{quote}
    We are not concerned here with logicians' questions about inference, but solely with the psychological process of inferring.
    The primary sense of the word is that in which it involves acquiring a belief on the basis of a belief already held.

    \mbox{}\hfill\(\vdots\)\hfill\mbox{}

    \dots to say that A infers \emph{p} from \emph{q} is simply to say that A's believing \emph{q} \emph{causes} him to acquire the belief \emph{p}.
    And the sense of `cause' employed here is the common or billiardball sense of `cause', whatever that sense is.%
    \mbox{}\hfill\mbox{(\citeyear[194]{Armstrong:1968vh})}
  \end{quote}

  \citeauthor{Armstrong:1968vh} tightens the account of inference a little in order to avoid including any belief \emph{r} in the causal chain leading to an agent acquiring the belief \emph{p} as a premise, but we set these details aside.
  (\citeyear[195--197]{Armstrong:1968vh})%
  \footnote{
    Indeed, any disagreement with \citeauthor{Armstrong:1968vh}'s restriction is of no real interest to us.
    For, if we grant that the relevant instance of causation provides a \wit{} for a \ros{}, then, re-expressed, \citeauthor{Armstrong:1968vh}'s revisions narrow down the \wit{1} of interest.
  }

  \citeauthor{Armstrong:1968vh} talks about `inference' rather than `conclusion'.
  However, plausibly the same thing.
  Belief \(\phi\) only if pair \(\phi\) with value `true'.
\end{note}

\begin{note}
  Of some interest.
  \citeauthor{Armstrong:1968vh} traces the causal account of inference to ~\citeauthor{Moore:1962up} and Hume.
\end{note}

\subsubsection{\textcite{Boghossian:2014aa}}

\begin{note}
  Seen above.
  Causal process.

  \citeauthor{Boghossian:2014aa}'s Taking Condition narrows the relevant causal processes.
  However, causal processes, and hence to matter, causally involved.
  If causally involved, then witness.
\end{note}

\subsection{Indeterminate}
\label{cha:var:sec:indeterminate}

\subsubsection{\textcite{Wright:2014tt}}

\begin{note}
  \citeauthor{Wright:2014tt}'s (\citeyear{Wright:2014tt}) `Simple Proposal'.
  Appealed to the Simple Proposal above on \autopageref{Wright-simple-supportI} to help clarify \supportI{}.

  Here, observation is on acceptance and move.
  Recall, the core of the Simple Proposal is the following idea:

  \begin{quote}
    [A] thinker infers q from p\(_{1}\) \(\cdots\) p\(_{\text{n}}\) when he accepts each of p\(_{1}\) \(\cdots\) p\(_{\text{n}}\), moves to accept q, and does so for the reason that he accepts p\(_{1}\) \(\cdots\) p\(_{\text{n}}\).%
    \mbox{}\hfill\mbox{(\citeyear[33]{Wright:2014tt})}
  \end{quote}

  Accepting and moving provides a \wit{}, and inferring is understood via the \wit{}.
\end{note}

\subsubsection{\textcite{Broome:2002aa}}

\begin{note}
  The same observation extends to \citeauthor{Broome:2002aa}'s (\citeyear{Broome:2013aa}) rule following account of (active) reasoning.

  \begin{quote}
    Active reasoning is a particular sort of process by which conscious premise-attitudes cause you to acquire a conclusion-attitude.
    The process is that you operate on the contents of your premise-attitudes following a rule, to construct the conclusion, which is the content of a new attitude of yours that you acquire in the process.\newline
    \mbox{ }\hfill\mbox{(\citeyear[234]{Broome:2002aa})}
  \end{quote}

  Understand \ros{0} entailed by an event in which an agent concludes \(\pv{\phi}{v}\) from \(\Phi\) by~\supportI{} in terms of having followed a rule.
  However, if this entailment holds then the converse entailment, that the agent concluded by following the rules that gave rise to \ros{0} does not hold.
\end{note}

\subsection{Non-causal}
\label{cha:var:sec:non-causal}

\subsubsection{\textcite{Harman:1973ww}}

\begin{note}
  \begin{quote}
    Reasons may or may not be causes; but explanation by reasons is not causal or deterministic explanation.
    It describes the sequence of considerations that led to belief in a conclusion without supposing that the sequence was determined.%
    \mbox{ }\hfill\mbox{(\citeyear[52]{Harman:1973ww})}
  \end{quote}

  Sequence of considerations provides a \wit{}.
\end{note}

\subsubsection{\textcite{Hieronymi:2011aa}}

\begin{note}
  More broadly, though similar.
  \citeauthor{Hieronymi:2011aa}'s (\citeyear{Hieronymi:2011aa}) account of acting for reasons.

  \begin{quote}
    The proposal starts with this simple thought: whenever an agent acts for reasons, the agent, in some sense, takes certain considerations to settle the question of whether so to act, therein intends so to act, and executes that intention in action.

    If this much is uncontroversial (and, under some interpretation, I believe it must be), we can use it as a form for filling out.
    I propose, then, that we explain an event that is an action done for reasons by appealing to the fact that the agent took certain considerations to settle the question of whether to act in some way, therein intended so to act, and successfully executed that intention in action.
    I suggest that \emph{this} complex fact, \dots is the reason that rationalizes the action---that explains the action by giving the agent’s reason for acting.\newline
    \mbox{ }\hfill\mbox{(\citeyear[431]{Hieronymi:2011aa})}
  \end{quote}

  So, reason is the complex fact.
  Complex fact gives the reason the agent acted, and so content of constituent considerations from \agpe{}.

  In particular, note here that everything is directed at the question.
  Premise-conclusion relationship.

  As with \citeauthor{Harman:1973ww}, capture the trace, which is given by a \wit{}.
\end{note}

\subsection{Normative}

\subsubsection{\textcite{Lord:2018aa}}

\begin{note}[Responding to reasons]
  Consider the proposal at the core of \citeauthor{Lord:2018aa}'s (\citeyear{Lord:2018aa}) thesis that being rational is to correctly respond to reasons.

  \begin{quote}
    \textbf{Correctly Responding:} What it is for A's \(\phi\)-ing to be ex post rational is for A to possess sufficient reason S to \(\phi\) and for A's \(\phi\)-ing to be a manifestation of knowledge about how to use S as sufficient reason to \(\phi\).%
    \mbox{}\hfill\mbox{(\citeyear[143]{Lord:2018aa})}
  \end{quote}

  An agent's action is rational only if the action is a manifestation of some know-how.
  \citeauthor{Lord:2018aa} summaries:

  \begin{quote}
    \dots when one manifests one's know-how, dispositions that are directly sensitive to normative facts are manifesting. Thus, the competences involved in the relevant know-how make one directly sensitive to the normative facts%
    \mbox{}\hfill\mbox{(\citeyear[16]{Lord:2018aa})}
  \end{quote}

  For our purposes, following example of manifesting know-how directly relates to reasoning:

  \begin{quote}
    The most salient disposition [when appealing to \emph{p} as a reason]%
    \footnote{
      Note, \citeauthor{Lord:2018aa} (explicitly) not talking about believing that \emph{p} is a reason, but argues that the cited disposition to present both when appealing to p as a reason and believing that \emph{p} is a reason.
    }
    is the disposition to (competently) use \emph{p} as a premise in reasoning.\newline
    \mbox{}\hfill\mbox{(\citeyear[25]{Lord:2018aa})}
  \end{quote}

  Hence, suppose an agent concludes.
  Then, if the agent does not witness reasoning from \poP{}, it seems the agent does not manifest know-how, which is required for the appeal to meet \citeauthor{Lord:2018aa}'s account of rational action.

  Of course, that the noted disposition is the most salient does not rule out alternative, less noteworthy, dispositions.
  However, issues is \emph{manifesting} know-how without a \wit{}.
\end{note}

\begin{note}[Illustration]
  Clear that there is no manifestation of understanding of arithmetic.
\end{note}

\begin{note}
  Whether or not argument to be develop is of any difficulty turns on manifesting.
  In various cases, plausible that \ros{1} at issue would arise from the same disposition the agent manifests when the agent concludes.
\end{note}

\subsection{Embedding}
\label{cha:var:sec:embedding}

\begin{note}
  So, in initial cases, plausible that constraint in terms of having a \wit{}.

  In other cases, embedding.
\end{note}

\begin{note}
  An initial borderline case is \citeauthor{Boghossian:2014aa}'s taking condition.
  Depending on how `taking' is understood, embedded within \ros{}.
  In particular, distinguished proposition-value pairs from attitudes, and hence this does not amount to reducing the taking condition to a doxastic condition.
\end{note}


\subsubsection{\textcite{Thomson:1965vv}}

\begin{note}
  \citeauthor{Thomson:1965vv} suggests a an account of reasoning such that an agent reasons from \(\phi\) to \(\psi\) just in case the agent believes that \(\phi\) is a reason for \(\psi\).
  \begin{quote}
    The claim which the 'formula' of p.\ 285\nolinebreak
    \footnote{
      The `formula' in question:
      \begin{quote}
        Now reasoning should surely involve drawing a conclusion from a set of premisses.
        But you can't be said to draw the conclusion that \emph{q} from \emph{p} if for all you know in knowing that \emph{p} it would at best be a matter of luck if \emph{q} as well.
        So to ``reason'' from \emph{p} by itself to \emph{q} isn't really to be reasoning; it's like saying one thing, and then taking a chance on it that something else is also true---like taking a leap in the dark, or more prosaically, like guessing.'
        (From here on I shall refer to this as the `\emph{formula}'.)\nolinebreak
        \mbox{}\hfill\mbox{(\citeyear[285]{Thomson:1965vv})}
      \end{quote}
    }
    above was to support was this:
    suppose \emph{p} does not imply \emph{q}, and suppose a man says `\emph{p}, so \emph{q}';
    then he is not reasoning in saying this unless he believes that \emph{r}, where the conjunction of \emph{p} and \emph{r} implies \emph{q}, and \emph{r} is a suppressed premiss of his reasoning.\par
    But suppose such a man believes that \emph{p} is reason for \emph{q}; would this not be enough?
    `It would if ``\emph{p} is reason for \emph{q}'' were construed as a suppressed premiss of his argument'.
    Then let us so construe it.\newline
    \mbox{}\hfill\mbox{(\citeyear[294]{Thomson:1965vv})}
  \end{quote}
  Causation is absent from \citeauthor{Thomson:1965vv}.
  Does not imply that \citeauthor{Thomson:1965vv}'s proposal is independent of causation, but motivated does not appeal to causation.
\end{note}

\subsubsection{\textcite{Longino:1978wv}}

\begin{note}
  \citeauthor{Longino:1978wv}'s (\citeyear{Longino:1978wv}) account of inferring seems explicit.
  \begin{quote}
    S infers at t that p from x if and only if
    \begin{enumerate}[label=\arabic*]
    \item
      S at t comes to believe that p, and
    \item
      S's epistemic reason for believing that p at t is x, i.e.,
      \begin{enumerate}[label=\alph*]
      \item
        S takes x to be evidence that p, and
      \item
        S's taking x to be evidence that p causes S to believe that p.\newline
        \mbox{}\hfill\mbox{(\citeyear[22]{Longino:1978wv})}
      \end{enumerate}
    \end{enumerate}
  \end{quote}
  Causation between mental states, but explanatory relation between things.
\end{note}

\subsection{\ros{3}?}

\begin{note}
  A recent account of reasoning given by \cite{Valaris:2014un} (\citeyear{Valaris:2014un}) is separate from \wit{} and embedding.%
  \footnote{
    For simplicity we ignore \citeauthor{Valaris:2014un}'s distinction between basic and non-basic instances of reasoning.
    The excerpts concern non-basic reasoning.
  }

    \begin{quote}
    Suppose that one believes \emph{R} and that \emph{p} follows from \emph{R}.
    What else might it take for one to count as believing \emph{p} by reasoning from \emph{R}?
    The crucial point here is that, if one believes both \emph{R} and that \emph{p} follows from \emph{R}, then --- barring inattention or irrationality --- one thereby believes \emph{p}.
    \dots
    In general, the relation between believing that one has conclusive evidence for a proposition and believing that proposition is constitutive, not merely causal.%
    \mbox{ }\hfill\mbox{(\citeyear[110 ]{Valaris:2014un})}
  \end{quote}

  Distinct from \wit{} because only interest is belief.
  Distinct from embedding because constitutive.
  In other words, the agent does not reason from their belief.
  Rather, reasoning is the belief.

  Inclined to understand in terms of \ros{}.
  For our purposes, role of \ros{} is to capture relationship between premises and conclusion.
  However, understood in a particular way, may amount to a belief.

  Still, not so straightforward.
  How does on get the belief that \emph{p} follows from \emph{R}?
  This, to my mind, is what is at issue.
  However, it seems that this is not how things are for \citeauthor{Valaris:2014un}.

  \begin{quote}
    [R]easoning just is believing that one’s conclusion follows from one’s premisses, and thereby believing one’s conclusion.%
    \mbox{ }\hfill\mbox{(\citeyear[112]{Valaris:2014un})}
  \end{quote}

  Unless belief is process, then it seems reasoning understood in this way is instantaneous.

  I am not sure what to make of this.
\end{note}


\section{Wrangling}
\label{cha:var:wrang}

\begin{note}
  \autoref{cha:introduction} introduced two questions, \qWhy{} and \qHow{}, and motivated a constraint between answers to \qWhy{} and \qHow{}.

  \autoref{cha:var:sec:vars} introduced variants of \qWhy{} and \qHow{}, and a variant constraint.

  \autoref{cha:clar:sec:literature}, in addition to intuition, constraint seems to often be a theoretical assumption.

  Purpose of variants is to motivate counterexamples to constraint.
  Specifically in terms of answers to \qWhyVnP{} which are not answers to \qHowV{}.
  In other words, \ros{} such that \ros{} explains, in part, why agent concludes but is such that the agent does not have a \wit{} for the \ros{}.

  In this section we outline in rough form how we will (attempt) to provide counterexamples.

  In short, need:
  An agent, event in which agent concludes \(\pv{\phi}{v}\) from \(\Phi\), and \ros{} between \(\pv{\psi}{v'}\) and \(\Psi\) such that:

  \begin{itemize}
  \item
    The agent does not have a \wit{} for the \ros{} between \(\pv{\psi}{v'}\) and \(\Psi\).
  \item
    The \ros{} between \(\pv{\psi}{v'}\) and \(\Psi\), in part, answers \qWhyVnP{}.
  \end{itemize}

  Our goal is motivate a general method for generating examples in which some \ros{} for which an agent does not have a \wit{} such that the \ros{} answers \qWhyVnP{}.
\end{note}

\subsection{\ros{3} without a \wit{}}

\begin{note}
  Immediate that an agent may not have a \wit{} for some \ros{}.

  Novel conclusions, as understood in this document, are common.
  Pair a proposition with some value.

  For example, enumerate all the tautologies of propositional logic.
  As \citeauthor{Harman:1973ww} notes, clutter, and there may be little point in deriving the tautologies.
  However, regardless of worth, it is not the case that have a \wit{} for most.

  Likewise, conclusions with respect to actions.
  For example, which particular style of coffee would like as the queue shortens and the time to place an order approaches.
\end{note}

\begin{note}
  More difficult, is \ros{}.

  As sketched in \autoref{cha:var:ros}, idea of a \fc{}.
\end{note}

\begin{note}
  \fc{} is such that action such that after performing action, event in which concludes is in progress.
  In other worlds, the agent would be concluding by performing the action.

  Delicacy.
  \fc{1} and \ros{} from the \agpe{}.

  Two issues:

  \begin{enumerate}
  \item
    If independent of \agpe{}, then it may be the case that \fc{} without any prior recognition from agent.
  \item
    If dependent on \agpe{}, then it may be the case that not a \fc{}.
  \end{enumerate}

  Two issues are important.
  Without \agpe{}, then unclear that get \ros{} of interest for answer to \qWhyVnP{}.
  Given \agpe{}, unclear that we get a \ros{}.
\end{note}

\begin{note}
  Second issue is particularly pressing.
  For, explanations.
  \emph{Factive}.

  Get \ros{} by \fc{}.
  However, may not be the case that \fc{}.
\end{note}

\begin{note}
  Our strategy is to avoid both problems by focusing on cases in which the agent \emph{knows} that \(\pv{\psi}{v'}\) from \(\Psi\) is a \fc{}.

  Hence, suitable link to the \agpe{}, as the agent is aware that \fc{}.
  And, avoid failure of \fc{} by factivity of knowledge.

  In this respect the understanding of \fc{1} in terms of action such that concluding is important.
  With the exception of more-or-less instantaneous actions, future may develop in surprising ways.

  For example, plausible that an agent knows when they strike the cue ball in a certain way, a particular red ball will land in a pocket.
  However, not plausible that the agent knows where the cue ball will come to rest after the red ball lands in the pocket.
  Hence, agent does not know their following more, and so on.

  In parallel, an agent may have no guarantee that they will not be interrupted, etc.
  Hence, in most cases it seems implausible that an agent knows they will concluded.
  Yet, to be concluding does not require completion.

  With respect to \fc{}, whether event in which the agent concludes would be in progress.

  As seen in \autoref{cha:var:ros:II:fcs} \dots

  \fc{1} obtain instances of a \ros{1} holding from an \agpe{} without the agent having a \wit{0} for the \ros{0}.
  Hence, candidate \ros{} that may be answers to \qWhyVnP{} such that the agent does not have a \wit{0} for the \ros{0}.

  However, in order for \ros{} to be an answer, in part, to \qWhyVnP{}, dependence.
\end{note}

\begin{note}
  This does not provide a complete solution to problem of factivity.
  For, what distinguishes one case from the other?

  However, this is nothing unique to cases under consideration, so long as relevant instances of \fc{} are plausibly knowledge.

  Though, this still differs from attitudes.
  Here is where embedding is interesting.
  If under constraint of \agpe{}, and \fc{}, then why not embed \fc{} as a premise.
\end{note}

\begin{note}
  Following, distinction between \ros{} answering, in part, \qWhyVnP{} and a \ros{} being embedded in some \ros{}.
  In particular, \ros{} between some \(\pv{\psi}{v'}\) and \(\Psi\) such that distinct from \ros{} between \(\pv{\phi}{v}\) and \(\Phi\) being embedded within the \ros{} between \(\pv{\phi}{v}\) and \(\Phi\) as a premise.
\end{note}

\begin{note}
  Rule out possibility of embedding proposed answers in this way.

  However, this will turn on the way in which dependence holds.
\end{note}

\subsection{Dependence}

\begin{note}
  Now, need it to be the case that if \ros{} failed to hold, then would not conclude.

  Suitable link between conclusion of \(\pv{\phi}{v}\) form \(\Phi\) and \ros{} between \(\pv{\psi}{v'}\) and \(\Psi\).

  \fc{3} serve an equally important role.

  For, if \fc{} then \ros{}.
  Conversely, if no \ros{} then not \fc{}.
  Hence, if fails to be \ros{}, then fails to be \fc{}.
  In turn, agent concludes only due to \fc{}, then suitable link.
\end{note}

\begin{note}
  Difficulty.
  How is it the case that the agent concludes only due to \fc{}?

  Method is to consider dependence of \qWhyVnP{}, from the \agpe{}:

  \begin{restatable}[\qWhyV{}]{question}{questionWhyV}
    \label{q:why:v}
    Given an agent \vAgent{}, proposition-value pair \(\pv{\phi}{v}\), \poP{} \(\Phi\), and event \(e\) in which \vAgent{} concludes \(\pv{\phi}{v}\) from \(\Phi\):

    \begin{quote}
      Which proposition-value-premises pairings \(\pvp{\psi}{v'}{\Psi}\) are such that, when \vAgent{} pairs \(\phi\) with \(v\):

      \begin{enumerate}[label=]
      \item
        \begin{enumerate}[label=\alph*., ref=(\alph*), series=qWhyVDef]
        \item
          A \ros{0} between \(\pv{\psi}{v'}\) and \(\Psi\) holds, from \agpe{\vAgent{}'}.
        \end{enumerate}
      \end{enumerate}

      And, from \agpe{\vAgent{}'}:

      \begin{enumerate}
      \item[\emph{If}:]
        \begin{enumerate}[label=\alph*., ref=(\alph*), resume*=qWhyVDef]
        \item
          The \ros{0} between \(\pv{\psi}{v'}\) and \(\Psi\) when pairing \(\phi\) with \(v\) failed to hold, from \agpe{\vAgent{}'}.%
          \footnote{
            It seems more natural to omit `from \agpe{\vAgent{}'}':
            \begin{itemize}
            \item
              The \ros{0} between \(\pv{\psi}{v'}\) and \(\Psi\) when pairing \(\phi\) with \(v\) failed to hold.
            \end{itemize}
            Indeed, evaluating conditional from \agpe{}.
            However, distinction between whether \ros{} and whether \ros{} from \agpe{}.
            Latter, and keeping the qualifier helps clarify.
          }
        \end{enumerate}
      \item[\emph{Then}:]
        \begin{enumerate}[label=\alph*., ref=(\alph*), resume*=qWhyVDef]
        \item
          \(e\) would not have been an event in which \vAgent{} concluded \(\pv{\phi}{v}\) from \(\Phi\).
        \end{enumerate}
      \end{enumerate}
    \end{quote}
    \vspace{-\baselineskip}
  \end{restatable}
\end{note}

\begin{note}
  The idea is:

  Counterexamples.

  One option is to specify (apparent) counterexamples so that the truth of the \emph{if-then} conditional follows.
  Then, at issue is whether analysis is correct.
  It may be that the truth of the conditional does not follow.

  Other option, specify (apparent) counterexamples so that the \emph{if-then} conditional holds from the \agpe{}.
  Then, at issue is whether the \agpe{} is correct.

  Trade a issue about analysis of counterexamples for a issue about what the counterexample achieves.

  Preference for the second option is ease of specifying examples.
  Build up an understanding of how and why such examples arise, and then try to figure out whether they result in anything substantial rather than attempting defend position that analysis of an example counters.

  In short:

  On first, whether analysis is correct.

  On second, whether agent is correct.

  Preference for whether agent is correct.
  Though, equivalent.
\end{note}

\begin{note}
  In order for \qWhyV{} to be of interest, must be cases where the \agpe{} is correct.

  \begin{proposition}
    \label{prop:why-n-p-link}
    Instances where \(\pvp{\psi}{v'}{\Psi}\) answers \qWhyVnP{} in virtue of answering \qWhyV{}
  \end{proposition}

  Find instances which witness the truth of \autoref{prop:why-n-p-link}.

  Stress, \autoref{prop:why-n-p-link} does not amount to an entailment.
  They may be cases where \agpe{} returns an answer to \qWhyV{} which is not an answer to \qWhyVnP{}.
\end{note}

\subsubsection{\fc{3}}
\label{sec:fc3}

\begin{note}
  \fc{3} will do the work:

  \begin{itemize}
  \item
    \ros{1} between \(\pv{\psi}{v'}\) and \(\Psi\) when pairing \(\phi\) with \(v\) failed to hold, from \agpe{\vAgent{}'}.
  \item
    \(\pv{\psi}{v'}\) failed to be a \fc{} from \(\Psi\).
  \item
    \(e\) would not have been an event in which \vAgent{} concluded \(\pv{\phi}{v}\) from \(\Phi\).
  \end{itemize}

  If \fc{} failed to hold, then no conclusion.
  Agent, implicitly, recognises link between \ros{1} and \fc{1}.
  However, given that \ros{1} are something of an abstraction, interest is really in whether \fc{}.

  So, dependence is captured from the \agpe{} and task is to construct \scen{1} such that if no \fc{0} then no conclusion.

  We will not sketch \scen{1} here.
\end{note}

\subsubsection{Concerns}
\label{sec:pitfalls}

\begin{note}
  Subdivide into two concerns.

  \begin{enumerate}
  \item
    Is \agpe{} informative?
  \item
    Granting true from \agpe{}, is informative, may it still be the case that \ros{} fails to answer \qWhyVnP{}.
  \end{enumerate}
\end{note}

\paragraph{Perspective alone}

\begin{note}
  It may be the case that that \emph{if-then} conditional holds from the \agpe{} but does not hold independently of the \agpe{}.
  Hence, the sense of dependence captured by \qWhyV{} is not equivalent with the intuitive sense of dependence captured by considering whether or not the \emph{if-then} conditional holds independently of the \agpe{}.

  The observation that the \emph{if-then} conditional may hold from the \agpe{} while failing to hold independently of the \agpe{} is clearest when considering conditionals more general.

  For example, suppose an agent has taken a gamble on a coin landing heads.
  The coin lands heads, and the agent receives a prize.
  From the \agpe{}, if the coin failed to lands heads, then the agent would not have received the prize.
  However, the agent was set to receive the prize for participating in the gamble, regardless of whether the coin landed heads.%
  \footnote{
    The present point is similar to issues raised by \citeauthor{Harman:1973ww} (\citeyear{Harman:1973ww}) regarding the proposed equivalence between reasons for which an agent believes something with reasons the agent would offer if asked to justify their belief.
  As \citeauthor{Harman:1973ww} notes, an agent may offer reasons because they think they will convince their audience, not because the agent is compelled by the reasons, etc.
  (\citeyear[Ch.2]{Harman:1973ww})

  To the extent that \citeauthor{Harman:1973ww}'s point is that what holds from an \agpe{} need not actually be the case, the point in the same.
  However, to the extent that \citeauthor{Harman:1973ww} relies on an under-specification of what holds from an \agpe{} --- i.e.\ the distinction between whether \(\phi\) has value \(v\) from the \agpe{} or whether the agent evaluates as true the proposition that their audience is responsive to \(\phi\) having value \(v\), the point is distinct.
  }

  So, switching back to \qWhyV{}, it may be the case that, though from the \agpe{} they would not have concluded \(\pv{\phi}{v}\) from \(\Phi\) if \support{} failed to hold between \(\pv{\psi}{v'}\) and \(\Psi\), the agent would have concluded \(\pv{\phi}{v}\) from \(\Phi\) regardless.
\end{note}

\begin{note}
  Consider \citeauthor{Davidson:1963aa}.
  Account of reason is in terms of attitudes.
  \emph{Not} what is the case from the \agpe{}.
  Attitudes, rather than the contents of attitudes.

  \begin{quote}
    Davidson[ asserts] a demand for a more ordinary form of explanation:
    an explanation which shows, not merely what, from another's point of view, \emph{could} count in favour of acting, but why that person did, in fact, act.%
    \mbox{}\hfill\mbox{(\cite[417]{Hieronymi:2011aa})}
  \end{quote}

  In general, difficult to make the switch.
  \textcite{Dancy:2000aa} argues for contents.
  However, problem of non-factivity.

  \citeauthor{Dancy:2000aa}'s position is difficult.
  On the one hand, reason from the \agpe{}.
  If this is the case, then compatible.
  In this respect, when shift to the \agpe{}, it is not the agent's belief, but what the agent believes.
  In short, reason from \agpe{} need not be the same as reason independent of \agpe{}.

  However, understanding from literature is that \citeauthor{Dancy:2000aa} holds that reason from \agpe{} \emph{is} reason.
  This, I find strange.
  And, this is not what we are interested in.
  Though, it is a delicate balance.
  Important thing to keep in mind is that we are two levels deep at this point.
  If dependence holds, along with proposition, then still in terms of \ros{} from \agpe{}.
  So, in dependence, then not pushing as far as \citeauthor{Dancy:2000aa}.
  No general entailment.
\end{note}

\begin{note}
  So, the only thing to do is ensure the \agpe{} is correct.

  However, motivation in similar style to \fc{1}.
  \fc{3} focus on whether agent would conclude.
  Pair is knowing that the agent would not conclude.
  So, in parallel fashion, knowledge.
\end{note}

\begin{note}
  So, avoid failure of factivity for a second time.
  However, problem of deviance.
  General disconnect.
  Agent is right, in general, but way in which they would not conclude is tangential.
  I have no solution to this.
  Grant that deviance does not occur.
\end{note}

\paragraph{Quarantine}

\begin{note}
  Granting \agpe{}, still fail to answer \qWhyVnP{}.
  Here, we return to the prospect of embedding \ros{1} within \ros{1} as seen in~\autoref{cha:var:ros:Emb}.

  If the \agpe{} is correct, then \fc{} matters.
  Hence, \ros{} matters.
  However, need not follow that conclusion of \(\pv{\phi}{v}\) from \(\Phi\) `directly' depends on \ros{} between \(\pv{\psi}{v'}\) and \(\Psi\).
  For, \ros{} between \(\pv{\psi}{v'}\) and \(\Psi\) may be embedded within the \ros{} between \(\pv{\phi}{v}\) from \(\Phi\).
\end{note}

\begin{note}
  So, embedding.

  Basically, concern about not being a \fc{}.
  Then, adding premise that is a \fc{} doesn't do any work.
\end{note}

\begin{note}
  Still, this isn't quite enough.
  What about some other explanation.

  If there is some general method to quarantine, then avoid worries about the link breaking.
  Deny the proposition holds.
  Separate \agpe{} from what happens.

  However, if no general method to quarantine.
  Then, more difficult.
  Something problematic about agent.

  Whether break the \agpe{} from what matters.

  Task: \fc{} matters from \agpe{}, but does not matter apart.
  Argue that this is not possible.
  To do this, relevant \fc{} does not embed.

  Does not function as an attitude.
\end{note}

\begin{note}
  Therefore, require a \wit{} in order to ensure that the \ros{} exists.
\end{note}

\begin{note}
  Factive and perspective is correct.
  If this is the case, then need a method of reducing to something factive.
  The only plausible way to reduce to something factive is to embed within a \ros{}.
  No embedding in relations of support.
  So, either reject factive or reject perspective.

  This is interesting, because typically the case that motivate factive by observing that it preserves the role of the \agpe{}.
\end{note}

\subsection{\qWhyVnP{} and \qWhyV{}}
\label{cha:var:expand:qWhy:variant}

\subsubsection{The \ros{} between \(\pv{\phi}{v}\) and \(\Phi\)}

\begin{note}
  Before turning to dependence, let us briefly observe that the \ros{} between \(\pv{\phi}{v}\) and \(\Phi\) should always, intuitively, be, in part, an answer to \qWhyV{}.

  For, when agent pairs \(\phi\) with \(v\), then by \supportI{}, it is the case that \ros{} between \(\pv{\phi}{v}\) and \(\Phi\) holds from the \agpe{}.
  Hence, \ref{q:why:v:a} must be true.

  Likewise, immediately the case that if \support{} failed to hold, then the agent would not have concluded \(\pv{\phi}{v}\) from \(\Phi\).
  For, by the same reasoning, when the agent pairs \(\phi\) with \(v\), then by \supportI{}, it is the case that \ros{} between \(\pv{\phi}{v}\) and \(\Phi\) holds from the \agpe{}.
  Therefore, if \support{} fails to hold between \(\pv{\phi}{v}\) and \(\Phi\), then \(e\) is not an event in which the agent concludes \(\pv{\phi}{v}\) from \(\Phi\).

  Now, the conditional when set aside from the \agpe{} may be true, it need not be the case that the conditional is true from the \agpe{}.
  However, we are working at some level of abstraction, and hence we assume the agent recognises the truth of the \emph{if-then} conditional.
  Indeed, we have merely expressed in artificial terms a truism:
  The agent would not have concluded \(\pv{\phi}{v}\) from \(\Phi\) if the agent had failed to conclude \(\pv{\phi}{v}\) from \(\Phi\).
\end{note}

\begin{note}
  With the core case of \support{} between \(\pv{\phi}{v}\) and \(\Phi\) being an answer to \qWhyV{} in hand, we turn to an in depth discussion of the kind of dependence captured by the \emph{if-then} conditional of \qWhyV{}.

  Our task is to balance a form of dependence that \emph{may} lead to counterexamples to \issueInclusion{} with an account of dependence that is compatible with the intuition that motivates \issueInclusion{}.
  In other words, the dependence should be such that answers to \qWhyV{} are constrained by the condition that the agent has a \wit{} for the relevant \ros{}.
\end{note}

\subsection{Difficulties}
\label{sec:qwhyv-subs-paragr}

\begin{note}
  Might be suspect, given voluntarism about concluding.
  But, this, I think, is a mistake.
  There's no suggestion that agent may choose whether or not to conclude in these types of cases.
  No choice over whether conditional is true.
\end{note}

\subsection{\citeauthor{Owens:2006tw}}

\begin{note}
  For example, \citeauthor{Owens:2006tw} argues for a belief expression model of assertion in which the rationality of a belief formed by an agent via testimony is connected to justification of the testifier:

  \begin{quote}
    Trusting an expression of belief by accepting what a speaker says involves entering a state of mind which gets its rationality from the rationality of the belief expressed.
    This state's rationality depends on the speaker's justification for the belief he expresses, not on his justification for the action of expressing it.
    And to hear a speaker as making a sincere assertion, as expressing a belief, is \emph{ceteris paribus} to feel able to tap into \emph{that} justification (whether or not his assertion was directed at you) by accepting what he says.%
    \mbox{}\hfill\mbox{(\citeyear[123]{Owens:2006tw})}
  \end{quote}

  On the view advanced by \citeauthor{Owens:2006tw}, justification.
  View in terms of \support{}.

  \support{} directly.
  Rationality of agent is rationality of speaker.

  However, `depends'.

  Distinction between rationality of state, and relation between rationality of state and rationality of state.

  Inclined to think \citeauthor{Owens:2006tw} is arguing for the former.%
  \footnote{
    \begin{quote}
      If we are to believe what the speaker indicates he believes, either the speaker must justify this belief to us, or we must supply some justification of our own
      \dots
      Neither act can be part of a rationality preserving mechanism for belief.%
      \mbox{ }\hfill\mbox{(\citeyear[123--124]{Owens:2006tw})}
    \end{quote}
  }
  Though, it is not clear to me that embedded isn't a viable option.

  Regardless, distinction that is important.
\end{note}

\begin{note}
  Same distinction holds for answers to \qWhyV{}.

  It may be the case that \support{} between \(\pv{\psi}{v'}\) and \(\Phi\) is, from the \agpe{}, involved in concluding \(\pv{\phi}{v}\) from \(\Phi\).

  However, no immediate move from this to \support{} being, in part, an answer to \qWhyV{}.
\end{note}

\subsection{Summary}
\label{cha:var:expand:issue:summary}



%%% Local Variables:
%%% mode: latex
%%% TeX-master: "master"
%%% End:
