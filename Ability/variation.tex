\chapter{Variant Questions}
\label{cha:var}

\begin{note}
  The primary role of the present chapter is to introduce variations on \qWhy{}, \qHow{}, and \issueInclusion{}, as introduced in \autoref{cha:introduction}.

  The overall goal of this document is to argue that \issueInclusion{} fails to hold.
  The role of the variations to \qWhy{}, \qHow{}, and \issueInclusion{} is to clarify the way in which the goal will be achieved.
\end{note}

\begin{note}
  This chapter is divided into sections as follows:
  \begin{enumerate}[label=, leftmargin=*]
  \item
    \TOCLine{cha:var:sec:wiggling}

    Describe approach to variations.
    Highlight difficulty, and how variations will be developed.
  \item
    \TOCLine{cha:var:ros}

    Detailed discussion of \ros{1}, foundation of how we develop variations.
    Includes initial discussion of \fc{1}.
  \item
    \TOCLine{cha:var:sec:vars}

    Variations to \qWhy{}, \qHow{}, and \issueInclusion{} with discussion.
  \item
    % \TOCLine{cha:clar:sec:literature}

    % Look at how to connect variant to various account of concluding, and related phenomena, from the literature.
  \end{enumerate}
\end{note}

\section{The role of variant questions}
\label{cha:var:sec:wiggling}

\begin{note}
  \autoref{cha:introduction} introduced \qWhy{}, \qHow{}, and \issueInclusion{}.

  \begin{quote}
    \vspace{-1.5\baselineskip}
    \questionWhyBasic*
  \end{quote}

  \begin{quote}
    \vspace{-1.5\baselineskip}
    \questionHowBasic*
  \end{quote}

  \begin{quote}
    \vspace{-1.5\baselineskip}
    \issueInclusionFirst*
  \end{quote}
  The overall goal of this document is to argue \issueInclusion{} does not hold.
  And, to do so we will develop a general type of counterexample to \issueInclusion{}.

  Still, both \qWhy{} and \qHow{} are broad questions which turn on the way in which the respective instances of `why' and `how' are understood.
  Hence, without establishing a clear understanding of the way in which the instances are to be understood, it is unclear how to develop counterexamples to \issueInclusion{}.

  However, these variations must obey to competing constraints:

  \begin{enumerate}[label=\alph*., ref=(\alph*)]
  \item
    \label{vars:constraint:ce}
    The variations to \qWhy{} and \qHow{} allow for the possibility for the variation of \issueInclusion{} to fail to hold.
  \item
    \label{vars:constraint:int}
    As \issueInclusion{} intuitively constrains answers to \qWhy{} in terms of answers to \qHow{}, a variation on \issueInclusion{} should intuitively constrain answers to the variation to \qWhy{} in terms of the variation to \qHow{}.
  \end{enumerate}

  Indeed, Constraint~\ref{vars:constraint:int} is a source of significant concern.

  For, there may be plausible answers to \qWhy{} which are not also answers to \qHow{} but fail to be counterexamples to \issueInclusion{} as the relevant sense of `why' and `how' are not the senses of `why' and `how' that \issueInclusion{} (intuitively) holds with respect to.

  And, as a constraint on answers to \qWhy{} in terms of \qHow{}, \issueInclusion{} may plausibly have a role in narrowing the relevant senses of `why' and `how' at issue.
  And, if \issueInclusion{} is rejected, then, naturally, \issueInclusion{} cannot perform this role.

  Hence, if the variations to \qWhy{}, \qHow{} or \issueInclusion{} fail to preserve intuitions regarding \qWhy{}, \qHow{} and \issueInclusion{}, then any counterexample to the variations will be of little to no interest.

  Further, to the extent that various theories of conclusion, or sufficiently related phenomena, either implicitly or explicitly endorse \issueInclusion{}, it should be the case that the same theories either implicitly or explicitly motivate the variation of \issueConstraint{} constraining answers to the variation of \qWhy{} in terms of answers to \qHow{}.
\end{note}

\begin{note}
  In broad outline, we use the idea of a `\ros{0}' to provide variations on \qWhy{}, \qHow{}, and \issueInclusion{}.
  The way in which we understand \ros{1} is minimal and tightly connected to an event in which an agent concludes \(\pv{\phi}{v}\) from \(\Phi\).
  Specifically, we put forward three ideas in relation to \ros{1}:
  \begin{enumerate}
  \item
    If an agent concludes \(\pv{\phi}{v}\) from \(\Phi\), then a \ros{0} between \(\pv{\phi}{v}\) and \(\Phi\), for the agent, when the agent pairs \(\phi\) with \(v\).
  \item
    If an agent has concluded \(\pv{\phi}{v}\) from \(\Phi\), then the event in which the agent concluded \(\pv{\phi}{v}\) from \(\Phi\) functions as a \wit{0} for a \ros{0} between \(\pv{\phi}{v}\) and \(\Phi\).
  \item
    It is possible for a \ros{0} between \(\pv{\phi}{v}\) and \(\Phi\) to hold, from an \agpe{}, without their being an \wit{0} for the \ros{0}.
  \end{enumerate}

  \autoref{cha:var:ros} will develop and discuss each idea in detail.

  For the moment, the motivation for abstracting to \ros{1} is to capture, in a abstract way, the way in which \(\pv{\phi}{v}\) and \(\Phi\) are related from an \agpe{} when the agent concludes \(\pv{\phi}{v}\) from \(\Phi\).

  Rather than directly capturing some relevant sense of `why' or `how' our goal is to use \ros{1} to construct variations on \qWhy{} and \qHow{} which are \emph{roughly} `extensionally adequate'.
  Where, we understand the term `extensionally adequate' more-or-less in line with \citeauthor{Sumner:1987aa} (\citeyear{Sumner:1987aa}):

  \begin{quote}
    [A] conception of a concept is extensionally adequate when it includes every item which seems pre-analytically to be an instance of the concept and excludes every item which does not.%
    \mbox{ }\hfill\mbox{(\citeyear[49]{Sumner:1987aa})}
  \end{quote}

  Adapted to our case, our interest with \qWhy{} and \qHow{} is with respect to intuitive answers to \qWhy{} and \qHow{} (and in particular the intuition that \issueInclusion{} holds).
  And, the variations of both \qWhy{} and \qHow{} may be seen as `conceptions of a question' such that any answer to \qWhy{} is an answer to the variation of \qWhy{}, vice-versa, and the same with respect to \qHow{}.

  Still, the way in which something answers \qWhy{} need not be equivalent to the way in which that thing answers the variation to \qWhy{}.%
  \footnote{
    In this respect, the variation to \qWhy{} need not be \emph{intensionally} adequate.
    Where the variation of \qWhy{} (or \qHow{}) would be intensionally adequate just in case the variation captured the way something \emph{intuitively} answers \qWhy{}.
  }

  And, we are only interested in `rough' extensional adequacy.
  In particular, we are only interested in answers to \qWhy{} and \qHow{} to the extent that \issueInclusion{} plausibly holds.
  Hence, we will ignore intuitive answers to \qWhy{} and \qHow{} which extend beyond \issueInclusion{}.

  Further, to the extent that \issueInclusion{} is intuitive, the variations \emph{may} conflict with this intuition.
\end{note}

\begin{note}
  With the aid of \ros{1} we develop variations of \qWhy{} and \qHow{}:

  \begin{itemize}
  \item
    Interpret `why' from \qWhy{} in terms of the \ros{1} the \agents{} conclusion of \(\pv{\phi}{v}\) from \(\Phi\) depended on.
  \item
    Interpret `how' from \qHow{} in terms of events which \wit{0} any \ros{} that the \agents{} conclusion of \(\pv{\phi}{v}\) from \(\Phi\) depended on.
  \end{itemize}

  So, the variation to \qWhy{} is expected to be extensionally adequate for:

  If conclusion does not depend on \ros{}, then plausible that it is possible to answer \qWhy{} without citing the proposition-value-premises pairing.
  For, event would have occurred regardless of whether paired.

  If conclusion does depend on \ros{}, then proposition-value-premises pairing answers, in part, \qWhy{}.
  For, event would not have occurred regardless of whether paired.

  Variation to \qHow{}.
  Developed with respect to variation on \qWhy{}.

  If \ros{}, then if event \wit{} \ros{}, then of interest.
  If event which does not lead to \ros{}, then event is of no interest.
\end{note}

\begin{note}
  An significant consequence of both variations will be as follows:

  \begin{itemize}
  \item
    When an agent concludes \(\pv{\phi}{v}\) from \(\Phi\):
    \begin{itemize}
    \item The \agents{} conclusion of \(\pv{\phi}{v}\) from \(\Phi\) depends on a \ros{0} between \(\pv{\phi}{v}\) and \(\Phi\) holding, for the agent.
    \item The event in which the agent concludes \(\pv{\phi}{v}\) from \(\Phi\) serves as a \wit{0} for the \ros{0} between \(\pv{\phi}{v}\) and \(\Phi\).
    \end{itemize}
  \end{itemize}
  Hence, a \ros{} between \(\pvp{\phi}{v}{\Phi}\) will always be, in part, an answer to the variation of \qWhy{} and the event in which the agent concludes \(\pv{\phi}{v}\) from \(\Phi\) will always be, in part, an answer to the variation of \qHow{}.
\end{note}

\begin{note}
  A variant to \issueInclusion{} follows from the variations to \qWhy{} and \qHow{}.
  Roughly:

  \begin{itemize}
  \item
    A conclusion of \(\pv{\phi}{v}\) from \(\Phi\) depends on some \ros{} between \(\pv{\psi}{v'}\) and \(\Psi\) holding for the agent

    \emph{Only if}:

    The agent has a \wit{} for the \ros{0} between \(\pv{\psi}{v'}\) and \(\Psi\).
  \end{itemize}
\end{note}

\begin{note}
  With an initial understanding of the variations to \qWhy{}, \qHow{}, and \issueInclusion{} in hand, we now return to the overall argument of this document.
  Our goal is to develop counterexamples to \issueInclusion{}.
  And, given the variation to \issueInclusion{} which follows from the variations to \qWhy{} and \qHow{}, we will do so by showing there are cases in which an agent concludes \(\pv{\phi}{v}\) from \(\Phi\) such that:
  \begin{itemize}
  \item
    The agent pairing \(\phi\) and \(v\) depended on a \ros{0} between \(\pv{\psi}{v'}\) and \(\Psi\) holding, for the agent.
  \item
    The agent did not have a \wit{0} for the \ros{0} between \(\pv{\psi}{v'}\) and \(\Psi\) when the agent paired \(\phi\) and \(v\).
  \end{itemize}
\end{note}

\section{\ros{3}}
\label{cha:var:ros}

\begin{note}
  In \autoref{cha:var:sec:wiggling} we briefly sketched the role \ros{1} will have in this document.
  The present section develops and discusses \ros{1} in detail.

  The role of \ros{1} within this document is to capture, in a abstract way, the way in which a proposition-value pair \(\pv{\phi}{v}\) and \poP{} \(\Phi\) are related from an \agpe{} when the agent concludes \(\pv{\phi}{v}\) from \(\Phi\).
\end{note}

\begin{note}
  Our understanding of a `\ros{1}' is given in terms of three ideas, and sub-sections will develop and discuss each idea in detail:

  \begin{enumerate}[label=, leftmargin=*]
  \item
    \TOCLine{cha:var:ros:I}

    An event in which an agent concludes \(\pv{\phi}{v}\) from \(\Phi\) is sufficient for a \ros{} to hold, for the agent.
  \item
    \TOCLine{cha:var:ros:W}

    An event in which an agent concludes \(\pv{\phi}{v}\) from \(\Phi\) provides an agent with a \wit{0} for a \ros{}.
  \item
    \TOCLine{cha:var:ros:II}

    It is possible for a \ros{} to hold, from an \agpe{} without the agent having a \wit{} for the \ros{}.
  \end{enumerate}
\end{note}

\begin{note}
  We speak in terms of \ros{1} holding from an \agpe{}.
  However, given some agent \vAgent{}, proposition \(\phi\), value \(v\), and \poP{} \(\Phi\), we do not draw any particular distinction between:

  \begin{enumerate}[label=\alph*., ref=(\alph*)]
  \item
    \label{ros:ap:maybe:a}
    A \ros{} between \(\pv{\phi}{v}\) and \(\Phi\) holds, from an \agpe{}.
  \item
    \label{ros:ap:maybe:b}
    The pairing of the proposition `A \ros{} holds between \(\pv{\phi}{v}\) and \(\Phi\)' with the value `True' by the agent.

    I.e.\ the proposition-value pair:\newline
    \mbox{ }\hfill%
    \(\pv{\text{A \ros{} holds between } \pv{\phi}{v}\text{ and }\Phi}{\text{True}}\)
  \end{enumerate}

  Any significant distinction between~\ref{ros:ap:maybe:a} and~\ref{ros:ap:maybe:b} would turn on details too specific for the degree of abstraction we target.
  For, what holds from an \agpe{} may just amount to which propositions are paired with the value `True' by the agent.
\end{note}

\subsection{\supportI{}}
\label{cha:var:ros:I}

\begin{note}
  \supportI{} states, roughly, that the event in which an agent concludes \(\pv{\phi}{v}\) from \(\Phi\) is sufficient for a \ros{} to hold between \(\pv{\phi}{v}\) and \(\Phi\).

  \begin{idea}[\supportI{}]
    \label{idea:support}
    For an agent \vAgent{}, a proposition-value pair \(\pv{\phi}{v}\), \poP{} \(\Phi\), and event \(e\):

    \begin{itemize}
    \item[\emph{If}:]
      \begin{enumerate}[label=\alph*., ref=(\alph*)]
      \item
        \(e\) is an event in which \vAgent{} concludes \(\pv{\phi}{v}\) from \(\Phi\).
      \end{enumerate}
    \item[\emph{Then}:]
      \begin{enumerate}[label=\alph*., ref=(\alph*), resume]
      \item
        When \vAgent{} pairs \(\phi\) with \(v\) as a sub-event of \(e\):
        \begin{itemize}
        \item
          A \emph{\ros{}} between \(\pv{\phi}{v}\) and \(\Phi\) holds, from \agpe{\vAgent{}'}.
        \end{itemize}
      \end{enumerate}
    \end{itemize}
    \vspace{-\baselineskip}
  \end{idea}

  The focus on the sub-event in which the agent pairs \(\phi\) with \(v\) is to allow for the \ros{} to, in part, explain why the agent concludes \(\pv{\phi}{v}\) from \(\Phi\) without requiring that the \ros{} holds, for the agent prior to the agent forming the conclusion that \(\phi\) has value \(v\).

  In this respect, a \ros{} between \(\pv{\phi}{v}\) and \(\Phi\) may be regarded as a static account of how the agent has come to pair \(\phi\) with \(v\).
  In other words, the \ros{} between \(\pv{\phi}{v}\) and \(\Phi\) just captures whatever it is, for the agent, that led to the agent concluding \(\pv{\phi}{v}\) from \(\Phi\).

  Still, \supportI{} is only a sufficient condition, and the suggestion --- however the details work out --- is intended as intuition for when the agent concludes \(\pv{\phi}{v}\) from \(\Phi\).
  As we will see when discussing \supportII{}, we will deny the converse of \supportI{}.
  Therefore, the intuition is not suitable to capture in general what a \ros{} holding, from an \agpe{}, amounts to.

  Generalised, what it is, that has, is, or will, relate \(\pv{\phi}{v}\) and \(\Phi\), for the agent.
\end{note}

\begin{note}
  Given \supportI{}, we will always qualify that a \ros{} holds \emph{from an \agpe{}}.
  Our interest is with conclusions, parings of propositions with values by an agent.
  Hence, we have no interest in whatever the idea of a \ros{} between \(\pv{\phi}{v}\) and \(\Phi\) simpliciter.
  As outlined in \autoref{cha:clar:sec:CCC:pvp}, we place no restrictions on conclusions.
  Hence, if an agent concludes that the ratio of the long side to the short side of a piece of paper is not \(\sqrt{2}\) from some \poP{} \(\Phi\), then, by \supportI{}, a \ros{} holds between:
  \(\pv{\text{The ratio of the long side to the short side of a piece of paper is not }\sqrt{2}}{\text{True}}\) and \(\Phi\).
\end{note}

\begin{note}
  Think of \supportI{} in terms of doxastic justification.

  \begin{quote}
    S's belief that p is doxastically justified (i.e. S's belief is held in an epistemically permissible fashion) if and only if S believes p in the right kind of way, on an epistemically appropriate basis.%
    \mbox{ }\hfill\mbox{(\citeyear{Bondy:2018tk})}
  \end{quote}

  However, we have no particular interest in justification.
\end{note}

\begin{note}
 \supportI{} is similar to, but designed to be distinct from,~\citeauthor{Boghossian:2014aa}'s Taking Condition:%
  \footnote{
    There are various objections to~\citeauthor{Boghossian:2014aa}'s Taking Condition, though we take no stance on whether~\citeauthor{Boghossian:2014aa}'s Taking Condition holds.

    See, for example,~\textcite{Hlobil:2014tq}, \textcite{McHugh:2016vp}, and~\textcite{Wright:2014tt}.

    \citeauthor{Hlobil:2014tq} argues against the Taking Condition as it distracts from what accounts of reasoning out to explain, rather than arguing against the Taking Condition directly.

    \citeauthor{McHugh:2016vp} summarise various objects to the taking condition, and present district arguments against against (distinct) ideas in favour of the taking condition.
    In particular,~\supportI{} is closer to what \citeauthor{McHugh:2016vp} term the `Consequence Condition' (\citeyear[cf.][316]{McHugh:2016vp}), which \citeauthor{McHugh:2016vp} also (indirectly) argue against.
    However, \citeauthor{McHugh:2016vp} does not consider an alternative account of what distinguishes concluding from any other action, and as~\supportI{} is designed to capture this distinction, it is unclear to me whether \citeauthor{McHugh:2016vp}'s arguments apply to~\supportI{} (if, indeed, they are sound).

    \citeauthor{Wright:2014tt} denies that reasoning must involve a state which connects premises to conclusions, as discussed in the main body of this section. (\citeyear[Cf.][33-34]{Wright:2014tt})
  }

  \begin{quote}
    (Taking Condition):
    Inferring necessarily involves the thinker \emph{taking} his premises to support his conclusion and drawing his conclusion because of that fact.%
    \mbox{}\hfill\mbox{(\citeyear[5]{Boghossian:2014aa})}
  \end{quote}

  There is an immediate superficial difference in that~\citeauthor{Boghossian:2014aa} states the Taking Condition in terms of inferring.
  However, `a conclusion' may be substituted for `inferring' and an important distinction remains.
  For, `taking' is understood by \citeauthor{Boghossian:2014aa} to be substantive.

  \citeauthor{Boghossian:2014aa} illustrates with the following \scen{}:
  \begin{quote}
    On waking up one morning I recall that:

    \begin{enumerate}[label=(\arabic*), ref=(\arabic*), series=BogEx]
    \item
      \label{BogEx:1}
      It rained last night.
    \end{enumerate}

    I combine this with my knowledge that

    \begin{enumerate}[label=(\arabic*), ref=(\arabic*), resume*=BogEx]
    \item
      \label{BogEx:2}
      If it rained last night, then the streets are wet.
    \end{enumerate}

    to conclude:

    So,

    \begin{enumerate}[label=(\arabic*), ref=(\arabic*), resume*=BogEx]
    \item
      \label{BogEx:3}
      The streets are wet.
    \end{enumerate}
    This belief then affects my choice of footwear.%
    \mbox{ }\hfill\mbox{(\citeyear[2]{Boghossian:2014aa})}
  \end{quote}

  And \citeauthor{Boghossian:2014aa} expands as follows:

  \begin{quote}
    [M]y inferring from~\ref{BogEx:1} and~\ref{BogEx:2} to~\ref{BogEx:3} must involve my arriving at the judgment that~\ref{BogEx:3} in part \emph{because} I \emph{take} the presumed truth of~\ref{BogEx:1} and~\ref{BogEx:2} to provide support for~\ref{BogEx:3}.
    Let us call this insistence that an account of inference must in this way incorporate a notion of ``taking'' the Taking Condition on inference.%
    \mbox{ }\hfill\mbox{(\citeyear[4]{Boghossian:2014aa})}
  \end{quote}

  Hence, for \citeauthor{Boghossian:2014aa}, the Taking Condition captures something \emph{in addition} to~\ref{BogEx:3} being a conclusion from a \poP{} which includes~\ref{BogEx:1} and~\ref{BogEx:2}.
  The presence of `taking' has a distinctive role in classifying the move from~\ref{BogEx:1} and~\ref{BogEx:2} to~\ref{BogEx:3} as an inference (or as a conclusion).

  In contrast, we do not require that a \ros{} has any particular role \emph{for the agent} in event in which an agent concludes \(\pv{\phi}{v}\) from \(\Phi\).
  If an agent concludes~\ref{BogEx:3} from~\ref{BogEx:1} and~\ref{BogEx:2}, then a \ros{} holds between~\ref{BogEx:3} and \(\{\ref{BogEx:1}, \ref{BogEx:2}\}\), for the agent ---~\ref{BogEx:3} and \(\{\ref{BogEx:1}, \ref{BogEx:2}\}\) are related, in some way by the agent.
  However, the \ros{} between~\ref{BogEx:3} and \(\{\ref{BogEx:1}, \ref{BogEx:2}\}\) need not itself have a role in the \agents{} conclusion of~\ref{BogEx:3} from \(\{\ref{BogEx:1}, \ref{BogEx:2}\}\).
\end{note}

\begin{note}
  \phantlabel{Wright-simple-supportI}
  Indeed, our intuitive understanding of \ros{} is close to \citeauthor{Wright:2014tt}'s (\citeyear{Wright:2014tt}) `Simple Proposal':
  \begin{quote}
    [C]onsider instead the proposal, not that the status of the transition as inferential depends on the thinker's judgments about his reasons, but that it depends on \emph{what his reasons are}.
    We want his acceptance of the premises to supply his \emph{actual} reasons for accepting the conclusion.
    \dots

    Call this the Simple Proposal.
    It says that a thinker infers q from p\(_{1}\) \(\cdots\) p\(_{\text{n}}\) when he accepts each of p\(_{1}\) \(\cdots\) p\(_{\text{n}}\), moves to accept q, and does so for the reason that he accepts p\(_{1}\) \(\cdots\) p\(_{\text{n}}\).%
    \mbox{}\hfill\mbox{(\citeyear[33]{Wright:2014tt})}
  \end{quote}

  \citeauthor{Wright:2014tt}'s simple proposal is that, for the agent, the relation between a conclusion and some \poP{} need not be part of what moves the agent to conclude the conclusion from the \poP{}.

  \begin{quote}
    What is needed, then, is an account of, or at least some insight into, what it is for certain intentional states of a thinker to be his actual reasons for his transition to another intentional state.
    \dots
    We need to avoid committing to the notion that doing something for certain reasons must involve a state that somehow registers those reasons as reasons for what one does.%
    \mbox{}\hfill\mbox{(\citeyear[34]{Wright:2014tt})}
  \end{quote}

  Still, anticipating the role of \ros{} in construction a variation to \qWhy{}, it intuitively remains the case that a \ros{} explains, in part, why an agent concluded the conclusion from the \poP{} for, the \ros{} captures \emph{that} the agent accepted each of the premises and moved to accept the conclusion (in \citeauthor{Wright:2014tt}'s terminology).

  Still, following the discussion above, there is an important between~\supportI{} and \citeauthor{Wright:2014tt}'s Simple Proposal.
  For,~\supportI{} is an entailment, while \citeauthor{Wright:2014tt}'s Simple Proposal is an identity statement.
  Inferring, on the Simple Proposal, is an agent accepting some conclusion for the reason that they accept premises from some \poP{}.
  \supportI{} does not entail that concluding is nothing more than moving to accept \(\pv{\phi}{v}\) as a result of accepting each element of \(\Phi\).
\end{note}

\subsection{\wit{3} for \ros{1}}
\label{cha:var:ros:W}

\begin{note}
  \autoref{cha:var:ros:I} introduced a sufficient condition for a \ros{} between \(\pv{\phi}{v}\) and \(\Phi\) to hold, from an \agpe{}:
  The agent concluded \(\pv{\phi}{v}\) from \(\Phi\).

  In general, if an agent has concluded \(\pv{\phi}{v}\) from \(\Phi\), then we will say the agent has a \wit{} for the \ros{} between \(\pv{\phi}{v}\) and \(\Phi\).
  In full:

  \begin{definition}[A \wit{2} for a \ros{0}]
    \label{def:witnessing}
    For an agent \vAgent{}, proposition-value pair \(\pv{\phi}{v}\), and \poP{} \(\Phi\):

    \begin{enumerate}[label=]
    \item
      \begin{enumerate}[label=\alph*., ref=(\alph*), series=WitnessDef]
      \item
        \vAgent{} has a \emph{\wit{0}} for \ros{} between \(\pv{\phi}{v}\) and \(\Phi\).
      \end{enumerate}
    \item
      \emph{If and only if:}
    \item
      \begin{enumerate}[label=\alph*., ref=(\alph*), resume*=WitnessDef]
      \item
        There is some event \(e\) such that \(e\) is an event in which \vAgent{} concludes \(\pv{\phi}{v}\) from \(\Phi\).
      \end{enumerate}
    \end{enumerate}
    \vspace{-\baselineskip}
  \end{definition}

  \ros{3} hold from an \agpe{}.
  Hence, we say that \emph{an agent has} a \wit{0} for some \ros{} to implicitly capture that \ros{} are agent relative.
\end{note}

\begin{note}
  An important, but trivial, case of \autoref{def:witnessing} is when an agent concludes \(\pv{\phi}{v}\) from \(\Phi\).
  For, if an agent concludes \(\pv{\phi}{v}\) from \(\Phi\) then it is immediate that there is some event in which the agent concludes \(\pv{\phi}{v}\) from \(\Phi\) --- the very same event --- and hence the agent has a \wit{} for the \ros{} between \(\pv{\phi}{v}\) and \(\Phi\).

  Hence, joining \supportI{} with \autoref{def:witnessing}, we have the following:

  \begin{proposition}[Concludes, then witnessed \support{}]
    \label{prop:cws}
    For an agent \vAgent{}, proposition-value pair \(\pv{\phi}{v}\) and \poP{} \(\Phi\):
    \begin{itemize}
    \item
      If \(e\) is an event in which \vAgent{} concludes \(\pv{\phi}{v}\) from \(\Phi\) then:
      \begin{itemize}
      \item
        When \vAgent{} pairs \(\phi\) with \(v\) as a sub-event of \(e\), a \ros{} between \(\pv{\phi}{v}\) and \(\Phi\) holds, from \agpe{\vAgent{}'}.
      \item
        \vAgent{} has a \wit{} for the \ros{} between \(\pv{\phi}{v}\) and \(\Phi\).
      \end{itemize}
    \end{itemize}
    \vspace{-\baselineskip}
  \end{proposition}

  \begin{argument}{prop:cws}
    Immediate for assuming the antecedent and appealing to \supportI{} and \autoref{def:witnessing}, respectively.
  \end{argument}

  \autoref{prop:cws} is of interest with respect to \qWhy{}, \qHow{}, \issueInclusion{}, and the variations to follow in \autoref{cha:var:sec:vars}.

  For, our variant to \qWhy{} will involve \ros{1}.
  Our variant to \qHow{} will involve \wit{1}.
  And, our variant to \issueInclusion{} will hold that a \ros{} is, in part, an answer to why an agent concluded only if the agent has a \wit{} for the \ros{}.

  Hence, \autoref{prop:cws} ensures that so long as there is an event in which the agent concludes \(\pv{\phi}{v}\) from \(\Phi\), then an answer to `why' will always have a corresponding answer to `how'.

  At issue is whether it is always the case that an agent has a \wit{} for a \ros{} which is, in part, an answer to why the agent concluded \(\pv{\phi}{v}\) from \(\Phi\).

  And, given \autoref{prop:cws} it is immediate that any such \ros{} must be distinct from the \ros{} between \(\pv{\phi}{v}\) and \(\Phi\).
\end{note}

\begin{note}
  Note, when we talk of \wit{1} we talk in terms of `having a \wit{0}'.
  In the case of \autoref{prop:cws}, the event in which the agent concludes and the event which secures the relevant \wit{} are identical.

  However, event \(e\) may be an event in which an agent concludes \(\pv{\phi}{v}\) from \(\Phi\) such that throughout the event \(e\), the agent has a \wit{} for a \ros{} between \(\pv{\psi}{v'}\) and \(\Psi\), such that the event \(e'\) which \wit{1} the \ros{} between \(\pv{\psi}{v'}\) and \(\Psi\) is distinct from \(e\).

  Hence, our understanding of `having a \wit{0}' allows for the possibility that some \ros{} between \(\pv{\psi}{v'}\) and \(\Psi\), in part, `answers why' an agent concludes \(\pv{\phi}{v}\) from \(\Phi\) though the relevant \wit{0} for the \ros{} between \(\pv{\psi}{v'}\) and \(\Psi\) is distinct.

  If you think there may be such cases, then the variant to \issueInclusion{} that we develop will be compatible with such cases.
  And, if you think there are no such cases, then it is safe to ignore this possibility.
  We will not directly, at least, consider such cases or take a stand either way in the main argument.%
  \footnote{
    \phantlabel{fn:past-witness}
    To illustrate, consider an agent working on some mathematical problem.

    As part of their work on the problem the agent concludes the hypotenuse of some right-angled triangle is \(\sqrt{74}\text{cm}\) by use of the Pythagorean theorem.

    Further, the agent has, at some point in the past proved the Pythagorean theorem from more basic principles.

    Now, generally speaking, it may be the case that the agent concludes the hypotenuse of the triangle is \(\sqrt{74}\text{cm}\), in part, from those more basic principles.
    For example, the agent may have just completed their proof of the Pythagorean theorem and the reasoning from the more basic principles to the hypotenuse of the triangle may be considered a single unified instances of reasoning, with an intermediary conclusion.

    Further, suppose the agent proved the Pythagorean theorem some years ago.

    Perhaps the \agents{} reasoning from more basic principles continues to provide, in part, an answer to how the agent concluded the hypotenuse of the triangle is \(\sqrt{74}\text{cm}\).
    Perhaps, regardless of the gap, the agent used the Pythagorean theorem \emph{because} they concluded the theorem from more basic principles.

    On the other hand, one may be inclined to hold that the more basic principles have no role explanatory role in the present.
    At best, the \agents{} \emph{memory} of --- rather than the event of --- concluding answers, in part, why the agent concluded hypotenuse of the triangle is \(\sqrt{74}\text{cm}\).
  }
\end{note}

\begin{note}
  Though we will not take a stand on whether a relevant \wit{0} for some conclusion is distinct from the event in which the agent concludes, the possibility of separation highlights an plausible issue with \autoref{def:witnessing}.

  For, if separation may occur, it seems there may be instances where an agent reasoned to \(\pv{\phi}{v}\) but did not conclude \(\phi\) has value \(v\) such that the event in which the agent reasoned to \(\pv{\phi}{v}\) serves as a \wit{0} to a \ros{} between \(\pv{\phi}{v}\) and \(\Phi\).

  As \autoref{def:witnessing} requires the event to be such that the agent concludes \(\pv{\phi}{v}\) from \(\Phi\), such events are excluded from being \wit{1}.

  To illustrate, consider an agent working through a proof of some theorem.

  Abstractly, let \(\theta\) be the state of affairs characterised by the theorem, and let \(\Theta\) be the relevant \poP{}.
  Our interest is with the conclusion \(\pv{\theta}{\text{True}}\) from \(\Theta\).

  Suppose the agent reasons to \(\pv{\theta}{\text{True}}\) from \(\Theta\).
  Further, suppose the \agents{} reasoning is sound.
  However, the agent is worried about some parts of their reasoning.
  Hence, given their worries, \emph{reasons} to --- but does not conclude --- \(\pv{\theta}{\text{True}}\) from \(\Theta\).

  Some time later the agent revisits proof, resolves their worries, and concludes the theorem is true.

  I see no issue with the \emph{idea} that:
  \begin{itemize}[noitemsep]
  \item
    When the agent revisited the proof, they concluded \(\pv{\theta}{\text{True}}\) from \(\Theta\).
  \item
    In part, a \ros{} between \(\pv{\theta}{\text{True}}\) and \(\Theta\), for the agent, answers why the agent concluded \(\pv{\theta}{\text{True}}\) from \(\Theta\).
  \item
    The event in which the agent reasoned to \(\pv{\theta}{\text{True}}\) from \(\Theta\) answers, in part, how the agent \(\pv{\theta}{\text{True}}\) from \(\Theta\) by being a \wit{} for the \ros{} between \(\pv{\theta}{\text{True}}\) and \(\Theta\).
  \end{itemize}

  However, the idea is incompatible with the way we understand a \wit{0}.
  For, by definition, the relevant event which serves as a \wit{0} must be an event in which the agent \emph{concludes} \(\pv{\theta}{\text{True}}\) from \(\Theta\).
  And, by construction of the \scen{0}, the \agents{} worries prevent the agent from forming the relevant conclusion.

  There are various ways to square the \scen{0} with our understanding of a \wit{0}.
  For example, one may consider the extended event in which the agent reasons, returns, and concludes.
  Or, one may hold that when the agent concluded \(\pv{\theta}{\text{True}}\), the agent concluded \(\pv{\theta}{\text{True}}\) not from \(\Theta\), but from some \poP{} \(\Theta'\) which include the adequacy of the \agents{} prior reasoning as a premise.

  Still, it is not clear to me that either of the options suggested --- or any other option --- is preferable to weakening \autoref{def:witnessing} in such a way that an event \(e\) which serves as a \wit{} to some \ros{} falls short of being an event in which an agent concludes.

  The difficulty is providing an adequate characterisation of the relevant event.
  That the agent \emph{reasoned} to \(\pv{\theta}{\text{True}}\) from \(\Theta\) is insufficient in general.

  For example, consider a variation of the \scen{} in which the agent identifies a problem with the proof.
  Given the presence of a problem, there is --- intuitively --- no \ros{} for the agent to have a \wit{0} for.

  Maintaining (some) intuition with regards to what it is for an event to be a \wit{0} for a \ros{0} is our priority.
  Strictly, the way in which we put \autoref{def:witnessing} to work is fully compatible with substituting `reasons to' in place of `concludes', but an overly narrow definition is preferably to an unintuitive definition.
  % Hence, in order to avoid a lengthy digression into sufficient conditions for an event to `\wit{0}' a \ros{} without the event being such that the agent concludes we simply require the event is such that the agent concludes.
\end{note}

\subsection{\supportII{}}
\label{cha:var:ros:II}

\begin{note}
  \supportI{} states that a \ros{} holds between \(\pv{\phi}{v}\) and \(\Phi\), for the agent, when an agent concludes \(\pv{\phi}{v}\).
  \supportII{}, in short, denies the converse of \supportI{} is the case.
  In full:

  \begin{idea}[\supportII{}]
    \label{idea:support:possible}
    For an agent \vAgent{}, a proposition-value pair \(\pv{\phi}{v}\), and \poP{} \(\Phi\):

    \begin{itemize}
    \item
      It is possible for both~\ref{idea:support:possible:a} and~\ref{idea:support:possible:b} be true:
      \begin{enumerate}[label=\alph*., ref=(\alph*)]
      \item
        \label{idea:support:possible:a}
        A \ros{} between \(\pv{\phi}{v}\) and \(\Phi\) holds, from \agpe{\vAgent{}'}.
      \item
        \label{idea:support:possible:b}
        \vAgent{} does not have a \wit{} for the \ros{} between \(\pv{\phi}{v}\) and \(\Phi\).
      \end{enumerate}
    \end{itemize}
    \vspace{-\baselineskip}
  \end{idea}

  If an agent does not have a \wit{} for a \ros{} between \(\pv{\phi}{v}\) and \(\Phi\), then there is no event in which the agent concludes \(\pv{\phi}{v}\) from \(\Phi\).
  So, \supportI{} states that an event in which the agent concludes \(\pv{\phi}{v}\) from \(\Phi\) is sufficient for a \ros{} between \(\pv{\phi}{v}\) and \(\Phi\) to hold, for the agent.
  In contrast, \supportII{} denies that an event in which the agent concludes \(\pv{\phi}{v}\) from \(\Phi\) is necessary for a \ros{} between \(\pv{\phi}{v}\) and \(\Phi\) to hold, for the agent.
\end{note}

\begin{note}
  \supportII{} has a key role in the overall argument for this document.
  For, as indicated, answers to the variant of \qWhy{} will concern \ros{}, and the variant of \qHow{} will concern whether the agent has a \wit{} for the relevant \ros{}.
  \supportII{}, then, allows for the \emph{possibility} that the kind of thing which answers, in part, why an agent concluded is not constrained by how the agent concluded.

  However, our motivation for \supportII{} is independent of the success of the overall argument of this document.

  In short, any constraint on answers to why an agent concludes by answers to how an agent concludes is substantial.
  And, given the sketch of the way in which we develop variations of \qWhy{} and \qHow{} from \autoref{cha:var:sec:wiggling}, denying that there is any instance in which a \ros{} may answer, in part, why an agent concluded \(\pv{\phi}{v}\) from \(\Phi\) without the agent having a \wit{} for the \ros{} amounts to a substantive constraint.

  Indeed, \supportII{} should not be of any immediate concern.
  For, there is a significant gap between:

  \begin{itemize}[noitemsep]
  \item
    There being a \ros{} between \(\pv{\psi}{v'}\) and \(\Psi\) from an \agpe{} without the agent having a \wit{} for the \ros{}.
  \item
    The \ros{} between \(\pv{\psi}{v'}\) and \(\Psi\), for the agent, answering, in part, and in some sense, why the agent concluded \(\pv{\phi}{v}\) from \(\Phi\).
  \end{itemize}
\end{note}

\paragraph{Basing}

\begin{note}
  Above, suggested \supportI{} in terms of basing.
  \supportII{} suggests propositional justification.

  In this respect, \supportII{} parallels accounts of propositional justification as advocated for by GOLDMAN and \citeauthor{Turri:2010aa}.

  Still, we have no interest in justification.
  Again, this is not because anything will turn on cases where an agent lacks justification.
  Rather, do not wish to make the distinction.
  Ideas apply to cases in which an agent has justification, and lacks justification.
\end{note}


% \subsubsection{\fc{3}}
% \label{cha:var:ros:II:fcs}

% \begin{note}
%   In this section we briefly sketch plausible \scen{1} in which Clauses~\ref{idea:support:possible:a}~and~\ref{idea:support:possible:b} of \supportII{} are both true.

%   In these \scen{1} some proposition-value pair \(\pv{\psi}{v'}\) is a `\fc{}'%
%   \footnote{
%     A hyphen between `foregone' and `conclusion' to indicate the specific interpretation of the term
%   }%
%   \(^{,}\)%
%   \footnote{
%     Related to \citeauthor{Firth:1978vi}'s (\citeyear{Firth:1978vi}) distinction between doxastic and propositional justification (or warrant).
%     See also \citeauthor{Silva:2020aa} (\citeyear{Silva:2020aa}) --- esp.\ fn.\ 1.

%     {\color{red}
%       Compare \citeauthor{Firth:1978vi}'s example with Holmes and Watson (\citeyear[218]{Firth:1978vi}).
%       Watson is presented with all the evidence Holmes used to that the coachman committed the murder, and that this provides Watson with sufficient epistemic reasons regardless of whether or not Watson forms any attitude, but it is not clear that Watson has the understanding to piece together the evidence laid before them.
%     }

%     However, no interest in justification.
%   }
%   from some \poP{} \(\Psi\).
% \end{note}

% \begin{note}
%   For example, recall \autoref{illu:gist:calc}.
%   The agent concluded \(23 \times 15 = 345\) from the testimony of a calculator.
%   Still, \(23 \times 15\) is more-or-less basic arithmetic.
%   And, the agent may have no difficulty doing such arithmetic.
%   % Hence, after receiving the testimony of the calculator it may be clear to the agent that not only does \(23 \times 15 = 345\) follow from the testimony of the calculator, but \(23 \times 15 = 345\) also follows from the \agents{} understanding of arithmetic.
%   So, given the opportunity, the agent would conclude \(23 \times 15 = 345\) from their understanding of arithmetic.
%   And, by \supportI{}, if the agent were to take the opportunity, a \ros{} would hold between \(23 \times 15 = 345\) and some \poP{} that would be employed if the agent were to multiply \(23\) by \(15\).
%   Hence, obtaining the \ros{} between \(23 \times 15 = 345\) and the relevant \poP{} from \supportI{} is pending only on the agent taking up the opportunity.

%   In this respect, \ros{} is equally foregone.
%   For, \ros{} follows obtains when the agent concludes \(\pv{\phi}{v}\).
%   Hence, \ros{} without \wit{0}.
% \end{note}


% \begin{note}
%   Above, described a general type of \scen{0}.
%   Characterised \scen{0} by saying \(\pv{\phi}{v}\) is a \fc{0} from \(\Phi\).

%   In this section, we briefly sketch \fc{0}.
% \end{note}

% \begin{note}
%   As expressed above, the basic idea of \(\pv{\phi}{v}\) being a \fc{} from \(\Phi\) for some agent is that, given the opportunity, the agent would conclude \(\pv{\phi}{v}\) from \(\Phi\).

%   We capture this basic idea via the progressive:

%   \begin{sketch}[\fc{3}]
%     For an agent \vAgent{}, proposition-value pair \(\pv{\phi}{v}\) and \poP{} \(\Phi\):

%     \begin{enumerate}[label=]
%     \item
%       \begin{itemize}
%       \item
%         \(\pv{\phi}{v}\) is a \emph{\fc{0}} from \(\Phi\), for \vAgent{}.
%       \end{itemize}
%     \item
%       \emph{If and only if}:
%     \item
%       \begin{itemize}
%       \item
%         Both~\ref{sketch:fc:exp:1}~and~\ref{sketch:fc:exp:2} are true:
%         \begin{enumerate}[label=\alph*., ref=(\alph*)]
%         \item
%           \label{sketch:fc:exp:1}
%           \vAgent{} has the opportunity to do some action \(\mathcal{A}\).
%         \item
%           \label{sketch:fc:exp:2}
%           The event in which \vAgent{} does \(\mathcal{A}\) is an event in which \vAgent{} is concluding \(\pv{\phi}{v}\) from \(\Phi\).
%         \end{enumerate}
%       \end{itemize}
%     \end{enumerate}
%     \vspace{-\baselineskip}
%   \end{sketch}

%   Note, \ref{sketch:fc:exp:2} only requires that the event \(e\) in which the agent does \(\mathcal{A}\) is such that \(e\) is an event in which the agent is \emph{concluding} \(\pv{\phi}{v}\) from \(\Phi\).
%   Hence, \ref{sketch:fc:exp:2} does not require --- though is compatible with --- \(e\) being an event in which the agent concludes \(\pv{\phi}{v}\) from \(\Phi\).

%   The difference is as follows:
%   \begin{itemize}
%   \item
%     If \(e\) is an event in which an agent \emph{concludes} \(\pv{\phi}{v}\) from \(\Phi\), then \(e\) includes as a sub-event an event \(e^{-}\) in which the agent pairs \(\phi\) with \(v\).
%   \item
%     If \(e\) is an event in which an agent is \emph{concluding} \(\pv{\phi}{v}\) from \(\Phi\), then \(e\) does not necessarily include as a sub-event an event \(e^{-}\) in which the agent pairs \(\phi\) with \(v\).

%     However, if \(e\) were allowed to develop, then \(e\) would develop into an event \(e^{+}\) such that \(e^{+}\) is an event in which the agent concludes \(\pv{\phi}{v}\) from \(\Phi\).
%   \end{itemize}

%   To illustrate the contrast,

%   drawing a dog.

%   draws a dog.

%   If fire alarm goes off, was drawing a dog, even though the drawing is never completed.

%   Likewise, as it happens, remarks that it's a nice drawing of a cat.
%   Well, sure, maybe it would work as a cat.

%   Note, does not start out as drawing a dog.
%   Only after initial doodle does the event develop sufficient inertia.

%   Concluding is the same.
% \end{note}

% \begin{note}
%   Extends, \(\pv{\phi}{v}\) is a \fc{3} from \(\Phi\), \emph{for the agent}.
%   In general, there is no entailment from \agpe{} to what is the case.
%   However, our attention will be on cases in which agent \emph{knows} \(\pv{\phi}{v}\) is a \fc{3} from \(\Phi\).

%   \autoref{illu:gist:calc}, again.
%   Given testimony of the calculator and skill with arithmetic, agent knows that they would be concluding \(23 \times 15 = 345\) if they were to start.
% \end{note}

% \begin{note}

%   \begin{sketch}
%     \label{sketch:fc-then-ros}
%     If \fc{} then \ros{}.
%   \end{sketch}

%   \fc{}.
%   Then, action available to the agent.
%   Concluding.
%   Develops, concludes.
%   Concludes, then \ros{}.

%   Everything is in place.
%   The agent need only perform the action.
%   Hence, \ros{}.
% \end{note}

% \begin{note}
%   \autoref{sketch:fc-then-ros} is important for obtaining \ros{1} without \wit{1}.
%   However, the converse of \autoref{sketch:fc-then-ros} will also be important:

%   \begin{itemize}
%   \item
%     If no \ros{} then not a \fc{}.
%   \end{itemize}
% \end{note}

% \begin{note}
%   Immediate that an agent may not have a \wit{} for some \ros{}.

%   Novel conclusions, as understood in this document, are common.
%   Pair a proposition with some value.

%   For example, enumerate all the tautologies of propositional logic.
%   As \citeauthor{Harman:1973ww} notes, clutter, and there may be little point in deriving the tautologies.
%   However, regardless of worth, it is not the case that have a \wit{} for most.

%   Likewise, conclusions with respect to actions.
%   For example, which particular style of coffee would like as the queue shortens and the time to place an order approaches.
% \end{note}

% \begin{note}
%   Abstract.

%     Instances in which an agent knows that concluding would be in progress exist.
%   For example, consider basic arithmetic.
%   Whether or not \(4131 + 1533 = 5664\) is a \fc{}.%
%   \footnote{
%     More generally, take any \(n\) and \(m\) such that the process is adding \(n\) and \(m\) would not take too long.
%   }
%   If you started, would be determining whether or not the equality holds.

%   In most cases we will push a little further than addition.
%   However, share the same pattern of the conclusion following from the application of some collection of rules which an agent knows.

%   For example, enrich the collection of mathematical operations to include subtraction, multiplication, division, square-roots and so on.

%   Beyond mathematics, but close, formal logic.
%   In particular, theoretical results such tautologies of propositional logic, or meta-theoretical results which are generated from a common method, such as completeness proofs of various modal logics via canonical models.

%   And, finally games.

%   If you know the appropriate strategy, play first, and so desire, then it is a \fc{0} that any game of noughts and crosses will either end in a win for you or a draw.%
%   \footnote{
%     For details, see~(\cite[94--96]{Gardner:1983wn}).
%   }

%   Sudoku puzzles.
%   Rules are simple, and I expect that if you have some experience with solving Sudoku puzzles, then you know that the solution to the Sudoku puzzle is a \fc{}.
%   In the worst case scenario, you have the option to brute force the solution to the puzzle.

%   A slightly more interesting example is chess problems.
%   In particular, there is plausibly some bound where solution to a problem fails to be a \fc{}.
%   However, any problem within the bound is a \fc{}.
%   May not know where the bound is.
%   Yet, solutions to some problems within the bound are know to be \fc{1}.
%   For example, whether or not there is an available move for some piece is a \fc{1}, but whether there is a sequence of move that will result in checkmate for either player is often not (known) to be a \fc{0}.
% \end{note}


\subsection{Summary}
\label{cha:var:ros:summary}

\begin{note}
  \supportI{}.

  \wit{3}.

  \supportII{}.
\end{note}

\begin{note}
  As indicated in \autoref{cha:var:sec:wiggling}, \autoref{cha:var:sec:vars} will use \ros{1} and \wit{3} to develop variations of \qWhy{}, \qHow{}, and \issueInclusion{} from \autoref{cha:introduction}.
\end{note}


\section{\qWhyV{}, \qHowV{}, and \issueConstraint{}}
\label{cha:var:sec:vars}

\begin{note}
  In \autoref{cha:var:sec:wiggling} we outlined how we will developed variations of \qWhy{}, \qHow{}, and \issueInclusion{} in terms of \ros{}.
  \autoref{cha:var:ros} outlined how we understand \ros{1}.
  In the present section we state the variations.
  We write the variants as `\qWhyV{}', `\qHowV{}', and `\issueConstraint{}', and the section is split into sub-sections which develop and discuss each variation.

  % \begin{enumerate}[label=]
  % \item
  %   \TOCLine{cha:var:sec:vars:qwhyvnp}
  % \item
  %   \TOCLine{cha:var:sec:vars:qhowv}
  % \item
  %   \TOCLine{cha:var:sec:vars:issue}
  % \end{enumerate}

  We term \qWhyV{} and \qHowV{} as \emph{variant} questions because \qWhyV{} and \qHowV{} are, respectively, too narrow and too broad to function as substitutes for \qWhy{} and \qHow{}.
  Hence, to clarify the way in which \qWhyV{} is a variant of \qWhy{} and the way in which \qHowV{} is a variant of \qHow{} we will explicitly link the variant to the initial question.
  We will then build the variant to \issueInclusion{} from \issueInclusion{} and the links between the variant questions and the initial questions.
\end{note}

\subsection{\qWhyV{}}
\label{cha:var:sec:vars:qwhyvnp}

\begin{note}
  As forecast in \autoref{cha:var:sec:wiggling}, eliminate `why' in favour of whether pairing \(\phi\) and \(v\) depends on \ros{}.

  Recall:
  \begin{quote}%
    \vspace{-1.5\baselineskip}%
    \questionWhyBasic*
  \end{quote}

  Rather than asking for proposition-value-premises pairings associated with explanations of `why' an agent concluded \(\pv{\phi}{v}\) from \(\Phi\), the variation of \qWhy{} queries which \ros{1} are such that the conclusion of \(\pv{\phi}{v}\) from \(\Phi\) \emph{depended} on the \ros{1} holding, for the agent.
  \end{note}

\subsubsection{Question}
\label{cha:var:sec:vars:qwhyvnp:question}

\begin{note}
  The variation of \qWhy{}, \qWhyV{} is as follows:

  \begin{restatable}[\qWhyV{}]{question}{questionWhyV}
    \label{q:why:v}
    \cenLine{
      \begin{itemize*}[noitemsep, label=\(\circ\)]
      \item
        Agent: \vAgent{}
      \item
        Proposition: \(\phi\)
      \item
        Value: \(v\)
      \item
        \poP{2}: \(\Phi\)
      \item
        Event: \(e\)
      \item
        \mbox{ }
      \end{itemize*}
    }

    Given \(e\) under description \(d\) is an event in which \vAgent{} concludes \(\pv{\phi}{v}\) from \(\Phi\):

    \begin{itemize}
    \item
      Which proposition-value-\poP{} pairs \(\pvp{\psi}{v'}{\Psi}\) are such that, under \(d\):

      \begin{enumerate}[label=]
      \item
        \begin{enumerate}[label=\alph*., ref=(\alph*), series=qWhyVdef]
        \item
          \label{q:why:v:a}
          A \ros{0} between \(\pv{\psi}{v'}\) and \(\Psi\) holds, for \vAgent{}, at some sub-event \(e^{\flat}\) of \(e\).
        \end{enumerate}
      \end{enumerate}

      And:

      \begin{enumerate}
      \item[\emph{If}:]
        \begin{enumerate}[label=\alph*., ref=(\alph*), resume*=qWhyVdef]
        \item
          \label{q:why:v:if}
          The \ros{0} between \(\pv{\psi}{v'}\) and \(\Psi\) failed to hold, for \vAgent{}, at \(e^{\flat}\).
        \end{enumerate}
      \item[\emph{Then}:]
        \begin{enumerate}[label=\alph*., ref=(\alph*), resume*=qWhyVdef]
        \item
          \label{q:why:v:then}
          \(e^{\flat}\) did not develop into \(e\).
        \end{enumerate}
      \end{enumerate}
    \end{itemize}
    \vspace{-\baselineskip}
  \end{restatable}

  The core of \qWhyV{} is the \itc{} from~\ref{q:why:v:if} to~\ref{q:why:v:then}.%
  \footnote{
    Clause~\ref{q:why:v:a} is only separated from the \itc{} for ease of reference.

    I.e.\ substituting `A' for `The' in Clause~\ref{q:why:v:if} would yield an equivalent statement, so long as the \itc{} is read as a counterfactual.
  }
  For, it is the \itc{} which (somewhat roughly) captures the relevant sense of dependence.
\end{note}

\begin{note}
    The basic idea is that a \ros{} between \(\pv{\psi}{v'}\) and \(\Psi\), in part, explains why the agent pairs \(\phi\) with \(v\).
  For, given dependence, the agent would not have paired \(\phi\) with \(v\) without the \ros{} holding, for the agent.
  We will explicitly capture this idea with \linkW{}.
\end{note}

\begin{note}
  \qWhyV{} shifts perspective from after the agent has concluded \(\pv{\phi}{v}\) from \(\Phi\) to the (sub-)event in which the agent pairs \(\phi\) with \(v\).

  The role of this restriction is to ensure the following proposition is true:

  \begin{proposition}
    \label{prop:ros-always-answer}
    For an agent \vAgent{}, proposition-value pair \(\pv{\phi}{v}\) and \poP{} \(\Phi\):

    \begin{itemize}
    \item
      The \ros{} between \(\pv{\phi}{v}\) from \(\Phi\) is always an answer to \qWhyV{}.
    \end{itemize}
    \vspace{-\baselineskip}
  \end{proposition}

  \begin{argument}{prop:ros-always-answer}
    By \supportI{}, a \ros{} holds between \(\pv{\phi}{v}\) and \(\Phi\), for the agent, when the agent pairs \(\phi\) with \(v\) as a sub-event of an event in which the agent concludes \(\pv{\phi}{v}\) from \(\Phi\).

    Hence, by restricting \qWhyV{} to the sub-event, we ensure that the \ros{} between \(\pv{\phi}{v}\) from \(\Phi\) holds, for the agent, and so may answer, in part, \qWhyV{}.

    And, the \ros{} between \(\pv{\phi}{v}\) from \(\Phi\) is guaranteed to be an answer to \qWhyV{} because, if the \ros{} between \(\pv{\phi}{v}\) from \(\Phi\) did not hold, for the agent, then the relevant event would be an event in which the agent concludes \(\pv{\phi}{v}\) from \(\Phi\).
    \end{argument}


  Note, however, that though Clause~\ref{q:why:v:then} is restricted to the sub-event in which the agent pairs \(\phi\) with \(v\), Clause~\ref{q:why:v:then} remains tied to the super-event in which the agent concludes \(\pv{\phi}{v}\) from \(\Phi\).
  That is, in order for a \ros{} to answer \qWhyV{}, it must be the case that the agent would not have concluded \(\pv{\phi}{v}\) from \(\Phi\), not merely fail to pair \(\phi\) with \(v\).

  To illustrate, consider again \autoref{illu:gist:calc}, in which an agent concluded \(23 \times 15 = 345\) from the testimony of a calculator.
  If we only considered the event in which the agent pairs \(23 \times 15 = 345\) with the value True, the \ros{} between \(\pv{23 \times 15 = 345}{\text{True}}\) and a \poP{} which includes the testimony of the calculator may \emph{fail} to answer, in part, \qWhyV{}.
  For, it may still have the case that the agent paired \(23 \times 15 = 345\) with the value True due to the agent performing the calculation by their understanding of arithmetic.
  However, such an event would not be a sub-event of an event in which the agent concludes \(\pv{23 \times 15 = 345}{\text{True}}\) from a \poP{} which includes the testimony of the calculator.
\end{note}

\subsubsection{Link}
\label{cha:var:sec:vars:qwhyvnp:link}

\begin{note}
  We developed \qWhyV{} through the idea that:

  If an event in which an agent concludes \(\pv{\phi}{v}\) from \(\Phi\) depends on a \ros{1} between \(\pv{\psi}{v'}\) and \(\Psi\) holding, for the agent, then the \ros{1} between \(\pv{\psi}{v'}\) and \(\Psi\) explains, in part, why the agent concluded \(\pv{\phi}{v}\) from \(\Phi\).

  The connexion between dependence and explanation of why links \qWhyV{} and \qWhy{} in the following way:

  \begin{restatable}[\qWhyV{} and \qWhy{}]{link}{linkSupportWhy}
    \label{link:why:support:pvpp}
    For an agent \vAgent{}, and proposition-value-premises pairings \(\pvp{\psi}{v'}{\Psi}\):

    \begin{itemize}
    \item[\emph{If}:]
      \begin{enumerate}[label=\alph*., ref=(\alph*)]
      \item
        A \ros{0} between \(\pv{\psi}{v'}\) and \(\Psi\) is, in part, an answer to \qWhyV{}.
      \end{enumerate}
    \item[\emph{Then}:]
      \begin{enumerate}[label=\alph*., ref=(\alph*), resume]
      \item
        \(\pvp{\psi}{v'}{\Psi}\) is, in part, an answer to \qWhy{}.
      \end{enumerate}
    \end{itemize}
    \vspace{-\baselineskip}
  \end{restatable}

  In short, \linkW{} states that dependence captured by \qWhyV{} is sufficient to explain why an agent concluded \(\pv{\phi}{v}\) from \(\Phi\).
\end{note}

\begin{note}
  Though, given the discussion of entanglement in \autoref{cha:var:sec:vars:qwhyvnp:question}, we take the conditional and to be implicitly restricted by some commonsense.

  We will have no interest in the converse of \linkW{}.
  And, there may be cases in which \(\pvp{\psi}{v'}{\Psi}\) answers, in part, \qWhy{} while there is no corresponding \ros{} between \(\pv{\psi}{v'}\) and \(\Psi\) that answers, in part, \qWhyV{}.
  Indeed, \qWhy{} is not restricted to \ros{1}.
\end{note}

\subsubsection{Imprecision}

\begin{note}
  Though I consider \qWhyV{} to express a somewhat intuitive idea, the question is imprecise in two significant ways:
  \begin{itemize}[noitemsep]
  \item
    First, we have not specified in detail what \ros{} are.
  \item
    Second, dependence is captured by a subjunctive conditional.
  \end{itemize}
\end{note}

\paragraph{\ros{3}}

\begin{note}
  The imprecision arising from \ros{1} is by design.
  Recall, a \ros{} between \(\pv{\phi}{v}\) and \(\Phi\) is designed to capture, in an abstract way, some distinctive relation which holds between \(\pv{\phi}{v}\) and \(\Phi\) for the agent.
  Hence, by abstracting to \ros{1} we avoid any particular account of what is it for an agent to conclude \(\pv{\phi}{v}\) from \(\Phi\).

  Though, with the caveat that a \ros{} may hold between \(\pv{\phi}{v}\) and \(\Phi\) without there being an event in which the agent concludes \(\pv{\phi}{v}\) from \(\Phi\).%
  \footnote{
    I.e., recall \supportI{} and \supportII{} from \autoref{cha:var:ros}.
  }
\end{note}

\paragraph{Dependence}

\nocite{Lewis:1973aa}

\begin{note}
  The \itc{} concerns two events, \(e\) in which the agent concludes \(\pv{\phi}{v}\) from \(\Phi\), and the sub-event \(e^{-}\) in which the agent pairs \(\phi\) with \(v\).
  As events, complex.

  A somewhat detailed account of the events is required to ensure that event depends on \ros{}.
  Recall, distinction between testimony of calculator and understanding of arithmetic.

  However, given required detail, \ros{1} may be entangled with the events in problematic ways.

  The most straightforward way to see how entanglement arises is by considering how prior \ros{1} \dots

  For, consider a \scen{0} in which an agent concludes \(23 \times 15 = 345\) from their understanding of arithmetic, such that some time shortly before the agent did the calculation, the agent concluded it would be good to get some coffee.

  For example, intuitively it should not be the case that the \ros{} and the conclusion that it would be good to get some coffee answers, in part, why the agent concluded some theorem is true from some \poP{}.
  Though, it may well be the case that the agent would have been too tired to conclude the theorem without the aid of the coffee.%
  \footnote{
    \citeauthor{Armstrong:1968vh} (\citeyear[195--196]{Armstrong:1968vh}) discusses a similar example, and suggests further discussion of this issue may also be found in \textcite{Moore:1962up}.
    See also \citeauthor{Sanford:1989aa} (\citeyear{Sanford:1989aa}) for a broader discussion of dependence as captured by subjunctive conditions (esp.\ pp.\ 192--193).
  }

  Perhaps closest world, decided on some other stimulant.
  However, I don't think this works in general.
  Background so that coffee is the only thing available.

  Yet, problem arises regardless of the availability of coffee.
  For, it's plausible the agent may have decided otherwise.
  And, given tired, decided to use a calculator instead.%
  \footnote{
    Further, now distinction \ros{1} between \poP{1} and the conclusion to do arithmetic or use a calculator.
  }
  So, not just circumstances surrounding event, but features of the event which are important for capturing those \ros{1} which do, intuitively, explain, in part, why the agent concludes \(\pv{\phi}{v}\) from \(\Phi\).

  Worry.
  In short, it is not clear how dependence, as captured by the \itc{}, handles certain simple cases where intuitions seem clear due to the way various \ros{1} may be entangled in an event in which an agent concludes.

  Whether or not there is a way to refine the \itc{} to clearly following intuition regarding entanglement is unclear to me.
\end{note}

\begin{note}
  Still, we will proceed with \qWhyV{}.

  Motivated in two ways:
\end{note}

\begin{note}[Way A]
  For, it seems \qWhyV{} does capture, even if imprecisely, something about of substance regarding why an agent concludes \(\pv{\phi}{v}\) from \(\Phi\).
  Hence, any refinement or variant of \qWhyV{} will agree with \qWhyV{} on a collection of `core' \scen{1}.
  And, our interest is only with respect to identifying instances in which a \ros{} answers, in part, \qWhyV{}.
  Hence, so long as the \scen{1} we consider belong to the `core', the imprecision of \qWhyV{} will not detract from the overall argument.
\end{note}

\begin{note}[Way B]
  Above, highlighted difficulties with dependence.
  However, positives.
  For, theory neutral.

  The imprecision arising from the \itc{} is in part by design.

  We noted above that the sense of dependence captured by the \itc{} is not, in general, tightly connected to answering why questions.
  And, ideally, we would have a detailed account of the connexion between conclusions and \ros{1} which clarifies when link, captured by the conditional, between~\ref{q:why:v:if} and~\ref{q:why:v:then} holds.
  Still, we abstracted to \ros{1} to avoid any detailed account of what it is for an agent to conclude \(\pv{\phi}{v}\) from \(\Phi\).
  Hence, the abstracting to a conditional allows the relevant link between~\ref{q:why:v:if} and~\ref{q:why:v:then} to be developed in accordance with specific details given by some account of what concluding amounts to.
\end{note}


\subsection{\qHowV{}}
\label{cha:var:sec:vars:qhowv}

\begin{note}
  \qHow{} is a broad question which asks, quite generally, for an account of how an agent concluded \(\pv{\phi}{v}\) in terms of what has happened.
  Recall:

  \begin{quote}%
    \vspace{-1.5\baselineskip}%
    \questionHowBasic*
  \end{quote}

  % However, out interest with \qHow{} is limited to the way in which certain answers to \qHow{}, intuitively, constrain answers to \qWhy{} via \issueInclusion{}:

  % \begin{quote}%
  %   \vspace{-1.5\baselineskip}%
  %   \issueInclusionFirst*
  % \end{quote}

  % Hence, \issueInclusion{} narrows interest to event which relate to proposition-value-premises pairings which answer \qWhy{}.
  % In \autoref{cha:var:sec:vars:qwhyvnp} we developed a variation of \qWhy{} --- \qWhyV{} --- in terms of \ros{}.
  % In turn, we develop a variation of \qHow{} in terms of \wit{1} to \ros{} which answer, in part, \qWhyV{}.

  As with \qWhy{} and \qWhyV{}, `\qHowV{}' refers to the variation of \qHow{}.
\end{note}

% \begin{note}
%   As \qHowV{} is developed in light of the way in which \issueInclusion{} intuitively constrains answers to \qWhy{} in terms of answers to \qHow{}, the link between \qHowV{} and \qHow{} is, likewise, restricted to be between those events which intuitively constrain answers to \qWhy{} via \issueInclusion{}.
% \end{note}

% \begin{note}
%   As we will see with~\autoref{prop:phi-always-how}, the event in which the agent concludes \(\pv{\phi}{v}\) from \(\Phi\) will always answer, in part, the variation to \qHow{}.
% \end{note}

\subsubsection{Question}
\label{cha:var:sec:vars:qhowv:sec:question}

\begin{note}
  The variant on \qHow{} is as follows:

  \begin{restatable}[\qHowV{}]{question}{questionHowV}
    \label{q:how:v}
    For an agent \vAgent{}, a proposition-value pair \(\pv{\phi}{v}\), a \poP{} \(\Phi\), and an event \(e\) in which \vAgent{} concludes \(\pv{\phi}{v}\) from \(\Phi\).

    Given \(e\) is an event which \vAgent{} concludes \(\pv{\phi}{v}\) from \(\Phi\), under description \(d\):

    \begin{itemize}
    \item
      Which events \(e'\) are such that, when \vAgent{} pairs \(\phi\) with \(v\), both~\ref{q:how:v:a} and~\ref{q:how:v:b} hold:

      \begin{enumerate}[label=\alph*., ref=(\alph*), , series=qHowVdef]
      \item
        \label{q:how:v:a}
        A \ros{} between \(\pv{\psi}{v'}\) and \({\Psi}\) answers \qWhyV{}.
      \item
        \label{q:how:v:b}
        There exists some description \(d'\) such that under \(d'\), \(e'\) is a \wit{0} for a \ros{} between \(\pv{\psi}{v'}\) and \(\Psi\).
      \end{enumerate}
    \end{itemize}
    %
    \vspace{-\baselineskip}
  \end{restatable}

  Intuitively, \qHowV{} presupposes and answer to \qWhyV{} and returns an event which witnesses \ros{}, if such an event exists.
\end{note}

\begin{note}
  % As with \qWhyV{}, we focus on the sub-event in which the agent pairs \(\phi\) with \(v\).
  % And, as with \qWhyV{}, this is to ensure that a \ros{} holds, for the agent, by \supportI{},
  % And hence, by \autoref{def:witnessing}, that the agent has a \wit{} for the \ros{} between \(\pv{\phi}{v}\) and \(\Phi\).
  % Indeed, these two observations combine to ensure that the event in which an agent concludes \(\pv{\phi}{v}\) from \(\Phi\) is always an answer to \qHowV{}.

  \begin{proposition}
    \label{prop:phi-always-how}
    For an agent \vAgent{}, proposition-value pair \(\pv{\phi}{v}\) and \poP{} \(\Phi\):

    \begin{itemize}
    \item
      If \(e\) is the event in which \vAgent{} concludes \(\pv{\phi}{v}\) from \(\Phi\), then \(e\) is an answer to \qHowV{}.
    \end{itemize}
    \vspace{-\baselineskip}
  \end{proposition}

  \begin{argument}{prop:phi-always-how}
    Suppose \(e\) is the event in which \vAgent{} concludes \(\pv{\psi}{v'}\) from \(\Psi\).
    By \autoref{prop:ros-always-answer} established that the \ros{} between \(\pv{\phi}{v}\) and \(\Phi\) is always an answer to \qWhyV{}.
    So, Clause~\ref{q:how:v:a} is satisfied.

    Likewise, by assumption \(e\) is the event in which \vAgent{} concludes \(\pv{\psi}{v'}\) from \(\Psi\).
    Hence, by \autoref{def:witnessing}, \(e\) is a \wit{} for the \ros{} between \(\pv{\psi}{v'}\) and \(\Psi\).
    So, Clause~\ref{q:how:v:b} is satisfied.

    And, as both Clause~\ref{q:how:v:a} and Clause~\ref{q:how:v:b} are satisfied, \(e\) is an answer to \qHowV{}.
  \end{argument}

\end{note}

\begin{note}
  Now, from \autoref{prop:phi-always-how} we have that the event in which an agent concludes \(\pv{\phi}{v}\) from \(\Phi\) is always an answer to \qHowV{}.
  However, the way in which \qHowV{} queries `how' an agent concludes \(\pv{\phi}{v}\) from \(\Phi\) is somewhat obscure.

  As with \qWhyV{} our interest is with extensional adequacy, and specifically extensional adequacy with respect to \ros{1}.

  Consider any event.
  There are two cases.
  \ros{} which answers \qWhyV{} or no \ros{}.

  If no \ros{} then the \agents{} conclusion of \(\pv{\phi}{v}\) from \(\Phi\) does not depend on what happened.
  Hence, event is not of direct interest with respect to answering `how' an agent concluded \(\pv{\phi}{v}\) from \(\Phi\).

  If \ros{} which \qWhyV{}, then included as an answer to \qHowV{}.

  So, though we have abstracted to \ros{1} to avoid any account of what conclusion amounts to, the existence of a \wit{0} intuitively captures whatever it is of relevance that happened when the agent concluded \(\pv{\phi}{v}\) from \(\Phi\).

  And, as discussed in \autoref{cha:var:ros:W}, the existence of a \wit{0} allows for \ros{} which answer \qWhyV{} to be constrained by answers to \qHowV{}, even if the relevant \wit{0} occurs at some point prior to the event in which the agent concludes \(\pv{\phi}{v}\) from \(\Phi\).
\end{note}

\subsubsection{Link}
\label{cha:var:sec:vars:qhowv:sec:link}

\begin{note}
  Given that \qWhyV{} has a key role in the construction of \qHowV{}, the link between \qHow{} and \qHowV{} is a little more involved that the link between \qWhyV{} and \qWhy{}.

  \begin{restatable}[\qHowV{} and \qHow{}]{link}{linkHowWitnessing}
    \label{link:how-witnessing}
    For any proposition-value-premises pairing \(\pvp{\psi}{v'}{\Psi}\):
    \begin{itemize}
    \item[\emph{If}:]
      \begin{enumerate}[label=\alph*., ref=(\alph*)]
      \item
        \(\pvp{\psi}{v'}{\Psi}\) is, in part, an answer \qHow{} due to \ros{} between \(\pv{\psi}{v'}\) and \(\Psi\) being, in part, an answer to \qWhyV{}.
      \end{enumerate}
    \item[\emph{Then}:]
      \begin{enumerate}[label=\alph*., ref=(\alph*), resume]
      \item
        There exists some event \(e\) such that \(e\) is an event in which \vAgent{} concludes \(\pv{\psi}{v'}\) from \(\Psi\) and \(e\) is, in part, an answer \qHowV{}.
      \end{enumerate}
    \end{itemize}
    \vspace{-\baselineskip}
  \end{restatable}

  We understand  \(\pvp{\psi}{v'}{\Psi}\) being, in part, an answer \qHow{} `due to' \ros{} between \(\pv{\psi}{v'}\) and \(\Psi\) being, in part, an answer to \qWhyV{} in terms of \linkW{} and \issueInclusion{}.

  For, suppose \(\pv{\psi}{v'}\) and \(\Psi\) is, in part, an answer to \qWhyV{}.
  Then, \(\pvp{\psi}{v'}{\Psi}\) is, in part, an answer to \qWhy{} via \linkW{}.
  And, if \issueInclusion{} holds, then \(\pvp{\psi}{v'}{\Psi}\) is also, in part, an answer to \qHow{}.
  Therefore, \(\pvp{\psi}{v'}{\Psi}\) is, in part, an answer to \qHow{} `due to' \(\pv{\psi}{v'}\) and \(\Psi\) being, in part, an answer to \qWhyV{}.
\end{note}

\begin{note}
  To summarise the preceding discussion, \linkH{} is designed to capture the idea that \qHow{} is concerned with the process by which an agent concludes \(\pv{\phi}{v}\) from \(\Phi\).
  Hence, if \(\pvp{\psi}{v'}{\Psi}\) is, in part, an answer \qHow{} due to a \ros{} between \(\pv{\psi}{v'}\) and \(\Psi\) being, in part, an answer to \qWhyV{} then, there is some process 


  Then, intuitively, relationship between \(\pv{\phi}{v}\) and \(\Psi\) implicitly captures some part of the process by which the agent concluded \(\pv{\phi}{v}\) from \(\Phi\).
  Hence, as answers to \qHowV{} are given in terms of events, extract the event which corresponds to the pairing.

  Given \qHowV{} is designed to relate to \qWhyV{}, restrict pairings of interest to those which result in a \ros{0}.
\end{note}

\subsection{\issueConstraint{}}
\label{cha:var:sec:vars:issue}

\begin{note}
  In~\autoref{cha:var:sec:vars:qwhyvnp} we developed \qWhyV{}, a variation of \qWhy{}.
  And, in~\autoref{cha:var:sec:vars:qhowv} we developed \qHowV{}, a variation of \qHow{}.
  In both sections we established links between the variant --- \qWhyV{}, \qHowV{}  --- and initial --- \qWhy{}, \qHow{} --- questions.

  Indeed, the purpose of developing the variant questions is to clarify the initial questions to a sufficient degree so that counterexamples to the intuitive constraint between the initial questions --- \issueInclusion{} --- may be developed.

  In this section, we bring together the variant and initial questions and \issueInclusion{} to state, in some detail, how everything fits together.
\end{note}

\begin{note}
 We being with a proposition:

  \begin{restatable}[]{proposition}{propVariationsAndInclusion}
    \label{prop:support-and-witnessing}
    Assuming \linkW{}, \linkH{}, and \issueInclusion{} hold:

    For an agent \vAgent{}, proposition-value pairs \(\pv{\phi}{v}\), \(\pv{\psi}{v'}\), and \poP{1} \(\Phi\), \(\Psi\):

    \begin{itemize}
    \item
      \qWhyV{} is answered, in part, by a \ros{} between \(\pv{\psi}{v'}\) and \(\Psi\).
    \end{itemize}

    \emph{Only if}

    \begin{itemize}
    \item
      \qHowV{} is answered, in part, by \vAgent{}' \wit{0} for the \ros{} between \(\pv{\psi}{v'}\) and \(\Psi\).
    \end{itemize}
    \vspace{-\baselineskip}
  \end{restatable}

  \begin{argument}{prop:support-and-witnessing}
    Suppose \qWhyV{} is answered, in part, by a \ros{} between \(\pv{\psi}{v'}\) and \(\Psi\).
    From~\linkW{} it is immediately follows that \(\pvp{\psi}{v'}{\Psi}\) answers, in part, \qWhy{}.
    And, given \issueInclusion{}, \(\pvp{\psi}{v'}{\Psi}\) answers, in part, \qHow{}.

    Further, \(\pvp{\psi}{v'}{\Psi}\) answers, in part, \qHow{} due to a \ros{} between \(\pv{\phi}{v}\) and \(\Psi\) being, in part, an answer to \qWhyV{}.
    Therefore, by \linkH{}, there is some \(e\) such that \(e\) is a \wit{0} for the \ros{} between \(\pv{\psi}{v'}\) and \(\Psi\).
  \end{argument}


  \autoref{fig:relations-between-whys-and-hows} is a a visual representation of the argument for~\autoref{prop:support-and-witnessing}.
\end{note}

\begin{figure}[H]
  \centering
  \begin{tikzpicture}
    \tikzset{ansStyle/.style={%
        draw=gray,%
        text width=.5\textwidth,%
        rounded corners=2pt,%
      }%
    }
    %
    \node[ansStyle] (whyO) at (0,0) %
    {\qWhyV{} is answered by a \ros{0} between \(\pv{\psi}{v'}\) and \(\Psi\).};
    %
    \node[ansStyle] (whyA) at (1.933,-1.5) %
    {\qWhy{} is answered by \(\pvp{\psi}{v'}{\Psi}\).};
    %
    \node[ansStyle] (howA) at (3.866,-3) %
    {\qHow{} is answered by \(\pvp{\psi}{v'}{\Psi}\).};
    %
    \node[ansStyle] (witA) at (5.8,-4.5) %
    {\qHowV{} is answered by event which \wit{1} sppt.\ btw.\ \(\pv{\psi}{v'}\) and \(\Psi\).};
    %
    \path[->] ($(whyO.south)!0.9!(whyO.south west)$) edge [out=270, in=180] (whyA);
    \path[->] ($(whyA.south)!0.9!(whyA.south west)$) edge [out=270, in=180] (howA);
    \path[->] ($(howA.south)!0.9!(howA.south west)$) edge [out=270, in=180] (witA);
    %
    \node[text width=.5\textwidth] (1) at (1,-.8) {\linkW{}};
    \node[text width=.75\textwidth] (2) at (4.5,-2.25) {\issueInclusion{}};
    \node[text width=.5\textwidth] (3) at (5,-3.625) {\linkH{}};
  \end{tikzpicture}%
  \caption{Visual representation of~\autoref{prop:support-and-witnessing}}
  \label{fig:relations-between-whys-and-hows}
\end{figure}

\begin{note}
  As~\autoref{prop:support-and-witnessing} follows from \linkW{}, \linkH{}, and \issueInclusion{}, we consider the content of~\autoref{prop:support-and-witnessing} to be a parallel constraint to \issueInclusion{}:

  \begin{restatable}[\issueConstraint{}]{constraint}{issueConstraintStatement}
    \label{issue:has-witnessed}
    For an agent \vAgent{}, proposition-value pairs \(\pv{\phi}{v}\), \(\pv{\psi}{v'}\), and \poP{1} \(\Phi\), \(\Psi\):

    \begin{itemize}
    \item
      \qWhyV{} is answered, in part, by a \ros{} between \(\pv{\psi}{v'}\) and \(\Psi\).
    \end{itemize}

    \emph{Only if}

    \begin{itemize}
    \item
      \qHowV{} is answered, in part, by \vAgent{}' \wit{0} for the \ros{} between \(\pv{\psi}{v'}\) and \(\Psi\).
    \end{itemize}
    \vspace{-\baselineskip}
  \end{restatable}

  As with \issueInclusion{}, if \issueConstraint{} holds, then answers to \qWhyV{} are constrained by answers to \qHowV{}.

  Our direct goal is to develops counterexamples to \issueConstraint{}.
  For, if there are counterexamples to \issueConstraint{}, then it immediately follows by \autoref{prop:support-and-witnessing} that either \linkW{}, \linkH{}, or \issueInclusion{} fails to hold.

  We defended \linkW{} when developing \qWhyV{} in~\autoref{cha:var:sec:vars:qwhyvnp}.
  And, likewise, we defended \linkH{} when developing \qHowV{} in~\autoref{cha:var:sec:vars:qhowv}.
  And, though there are some difficulties with \qWhyV{}, \qHowV{}, \linkW{}, and \linkH{}, I consider the most plausible point of failure to be \issueInclusion{}.
\end{note}

\begin{note}
  Still, \issueConstraint{} may be considered as a direct constraint on answers to \qWhyV{} in terms of answers to \qHow{}.
  In particular, \autoref{prop:constraint-rewrite} expands our construction of \qWhyV{}  and \qHowV{} to provide an alternative statement of \issueConstraint{}.

  \begin{proposition}[\issueConstraint{}, rewritten]
    \label{prop:constraint-rewrite}
    For an agent \vAgent{}, proposition-value pairs \(\pv{\phi}{v}\), \(\pv{\psi}{v'}\), and \poP{1} \(\Phi\), \(\Psi\):

    \issueConstraint{} is equivalent to the following conditional:

    \begin{enumerate}
    \item[\emph{If}:]
      \begin{enumerate}[label=\alph*., ref=(\alph*)]
      \item
        \label{constraint-rewrite:a}
        \vAgent{} concluded \(\pv{\phi}{v}\) from \(\Phi\).
      \end{enumerate}
    \item[\emph{And}:]
      \begin{enumerate}[label=\alph*., ref=(\alph*), resume]
      \item
        \label{constraint-rewrite:b}
        \vAgent{} would not have concluded \(\pv{\phi}{v}\) from \(\Phi\), if a \ros{} between \(\pv{\psi}{v'}\) and \(\Psi\) failed to hold, from \agpe{\vAgent{}'}.
      \end{enumerate}
    \item[\emph{Then}:]
      \begin{enumerate}[label=\alph*., ref=(\alph*), resume]
      \item
        \label{constraint-rewrite:c}
        \vAgent{} has a \wit{0} for the \ros{} between \(\pv{\psi}{v'}\) and \(\Psi\).
      \end{enumerate}
    \end{enumerate}
  \end{proposition}

  \begin{argument}{prop:constraint-rewrite}
    Immediate by \autoref{prop:support-and-witnessing} and the construction to \qWhyV{} and \qHowV{}.

    The antecedent of the conditional~---~\ref{constraint-rewrite:a}~\&~\ref{constraint-rewrite:b}~---~correspond to a \ros{} being, in part, an answer to \qWhyV{} while the consequent~---~\ref{constraint-rewrite:c}~---~corresponds to the \wit{0} which answers, in part, \qHowV{}.
    The conditional then follows from \autoref{prop:support-and-witnessing}.
  \end{argument}


  So, directly, \issueConstraint{} amounts to the constraint that in order for an \agents{} conclusion of \(\pv{\phi}{v}\) to depend on some \ros{} between \(\pv{\psi}{v'}\) and \(\Psi\), the agent must have concluded \(\pv{\psi}{v'}\) from \(\Psi\).
\end{note}


\subsection{Summary}
\label{cha:var:sec:vars:summary}

\begin{note}
  Overall argument.
  Links then answer to \qWhyV{} which is not constrained by \qHowV{}, then \issueInclusion{} fails.

  Three broad ways in which the overall argument may fail:
  \begin{enumerate}[label=\arabic*., ref=(\arabic*), noitemsep]
  \item
    The link between \qWhyV{} and \qWhy{} fails to hold.
  \item
    The link between \qHowV{} and \qHow{} fails to hold.
  \item
    We fail to develop counterexamples to \issueConstraint{}.
  \end{enumerate}

  Still, I hope to have developed \qWhyV{}, \qHowV{}, and \issueConstraint{} in such a way that both questions are of some interest independent of link

  In \autoref{cha:clar:sec:literature} we suggest how a handful of accounts of conclusion, or related, may be understood in terms of \qWhyV{} and \qHowV{}, and how the accounts motivate \issueConstraint{} as a constraint on answers to \qWhyV{} in terms of answers to \qHowV{}.
\end{note}

%%% Local Variables:
%%% mode: latex
%%% TeX-master: "master"
%%% End:
