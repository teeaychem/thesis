\chapter{Variations}
\label{cha:var}

\begin{note}
  The primary role of the present chapter is to introduce variations on \qWhy{}, \qHow{}, and \issueInclusion{}, as introduced in \autoref{cha:introduction}.

  The overall goal of this document is to argue that \issueInclusion{} fails to hold, and the role of the variations is to refine \qWhy{}, \qHow{}, and \issueInclusion{} in such a way that the way in which the goal is to be achieved is clear.

  Hence, with variations fresh in hand, we will also sketch, in outline, method and anticipate certain difficulties.
\end{note}

\section{\qWhy{}, \qHow{}, and \issueInclusion{}}
\label{cha:var:sec:q-no-v}

\begin{note}
   introduced \qWhy{}, \qHow{}, and \issueInclusion{}.

  Recall:
  \vspace{-\baselineskip}
  \begin{quote}
    \questionWhyBasic*
  \end{quote}

  \begin{quote}
    \questionHowBasic*
  \end{quote}

  \begin{quote}
    \issueInclusionFirst*
  \end{quote}
\end{note}

\begin{note}
    Example, pairing of \(23 \times 15 = 345\) and the testimony of the calculator that \(23 \times 15 = 345\) answers, in part, why the agent concluded \(23 \times 15 = 345\) in \autoref{illu:gist:calc}.
    Slightly more natural to say `the testimony of the calculator that \(23 \times 15 = 345\)', but \emph{paring}.

    Observed that, intuitively, pairing of \(23 \times 15 = 345\) and whatever pool of premises would be associated with the agent applying their understanding of arithmetic does not answer, in part, why the agent concluded \(23 \times 15 = 345\).
\end{note}

\section{Wiggling}
\label{cha:var:sec:wiggling}

\begin{note}
  For, both \qWhy{} and \qHow{} are broad questions which turn on the way in which the respective instances of `why' and `how' are understood.
  Hence, without establishing a clear understanding of the way in which the instances are to be understood, it is unclear how to motivate a rejection of \issueInclusion{}.

  For example, there may be plausible answers to \qWhy{} which are not also answers to \qHow{} but fail to be counterexamples to \issueInclusion{} as the relevant sense of `why' and `how' are not the senses of `why' and `how' that \issueInclusion{} (intuitively) holds with respect to.

  Indeed, as a simultaneous constraint on answers to \qWhy{} and \qHow{}, \issueInclusion{} has a role in narrowing the possible senses of `why' and `how' at issue.
  However, if \issueInclusion{} is rejected, then, naturally, \issueInclusion{} cannot perform this role.
\end{note}

\begin{note}
  Two constraints which are in tension.
  \begin{itemize}
  \item
    Variations are such that plausible, and that constrain remains plausible.
  \item
    Variations should be such that constraint may fail to hold.
  \end{itemize}

  If variations fail or constraint fail, then failure of the constraint is of no interest.
  On the other hand, avoid building in the constraint.
  Else, there is nothing to argue for.

  Two aspects to plausibility:
  \begin{itemize}
  \item
    Intuition.

    Variations to \qWhy{} and \qHow{} should be such that answers to \qWhy{} are constrained by \qHow{}.
  \item
    Theory.

    If a theoretical account of concluding, or a sufficient related phenomena --- such as inferring or reasoning --- plausibly contains answers to \qWhy{} in terms of \qHow{} then should be possible to motivate constraint from the theory.
  \end{itemize}

  Theory is somewhat difficult.
  For, on the one hand different accounts of concluding, or sufficiently related phenomena, are developed in different ways.
  Hence, there will be no exact correspondence.
  On the other hand, constraint is rarely --- if ever --- made explicit.
\end{note}

\begin{note}
  \autoref{cha:var:sec:q-no-v}
\end{note}

\begin{note}
  Central idea we will use to provide variations on \qWhy{}, \qHow{}, and \issueInclusion{} is that of a \ros{} between some proposition-value pair \(\pv{\phi}{v}\) and some pool of premises \(\Phi\).

  In broad outline:

  Given an agent has concluded \(\pv{\phi}{v}\) from \(\Phi\):
  Consider some arbitrary proposition-value pair \(\pv{\psi}{v'}\) and pool of premises \(\Psi\):
  \begin{itemize}
  \item
    Eliminate the relevant instance of `why' in favour of whether pairing \(\phi\) and \(v\) depended on a relation of \support{} holding between \(\pv{\phi}{v}\) and \(\Psi\).
  \item
    Eliminate the relevant instance of `how' in favour of whether or not there is an event in which agent concludes \(\pv{\psi}{v'}\) from \(\Psi\), and hence `\witness{1}' the \ros{}.
  \end{itemize}


  The way in which we understand \ros{1} will be minimal and  tightly connected to an event in which an agent concludes
  In short, serve as a static characterisation of the result of the dynamic process of concluding \(\pv{\phi}{v}\) from \(\Phi\).

 If agent concludes, then relation of \support{} when pairing.

  Hence, establishing a relation of \support{} amounts to an event in which an agent concludes.
  Likewise, depends.

  To get counterexample, depends on \ros{} such that there is no relevant event in which the agent concludes \(\pv{\psi}{v'}\) from \(\Psi\).

  Possibility will be given by two things:

  \ros{} does not necessarily require witnessing.

  Way in which dependence is understood.
\end{note}


\section{\ros{3}}
\label{cha:var:support}

{
  \color{red} Include some examples.
}

\begin{note}
  The present section introduces \ros{} in detail.

  We will not define a \ros{} in any specific way.
  Rather, our understanding of \ros{1} will be implicitly understood via two ideas termed `\supportI{}' and `\supportII{}'.

  \supportI{} states that \ros{} holds between \(\pv{\phi}{v}\) and \(\Phi\) when agent pairs \(\phi\) and \(v\).

  A \witness{} for a \ros{}.

  \supportII{} states that \ros{} may hold independently of a \witness{}.
\end{note}

\begin{note}
  \supportII{}.
  \fc{3}.
\end{note}

\begin{note}
  \supportI{} functions, in part, so that \ros{1} are answers to \qWhy{}.
  Though, delicate.

  So, tied to an event in which the agent concludes \(\pv{\phi}{v}\) from \(\Phi\).
  In this respect, \ros{} between \(\pv{\phi}{v}\) and \(\Phi\) is from the \agpe{}.

  Distinguish way in which \ros{} answer from the role of support in something like \citeauthor{Boghossian:2014aa}'s (\citeyear{Boghossian:2014aa}) Taking Condition.
  \ros{} is from \agpe{}.
  However, it does not need to be the case that \ros{} answers from the \agpe{}.

  Distinguish between:
  \begin{enumerate}[label=\alph*., ref=(\alph*)]
  \item
    A \ros{} between \(\pv{\phi}{v}\) and \(\Phi\), from an \agpe{}.
  \item
    Pairing of the proposition `A \ros{} holds between \(\pv{\phi}{v}\) and \(\Phi\)' with the value `True', from the \agpe{}.
  \end{enumerate}

  Whether this distinction amounts to anything substantial, well \dots
  Concludes, \ros{}, hence true.
  If \ros{} without concluding and has some role, then likely true from the \agpe{}.

  However, distinguish role.
  When proposition \ros{} and value `True', from the \agpe{}, then \ros{} has some role in the agent pairing \(\phi\) with \(v\).
  When \ros{}, then need not have a role, from the \agpe{}.

  Most of the time, when role, then rather than \ros{} directly, some proxy.
  Indeed, proxy will be \fc{}.

  However, this distinction embedding.
  propositions paired with values, function as premises.
  Distinguish between \ros{} answering \qWhy{} `directly' and \ros{} `indirectly' answering \qWhy{} as a premise.
\end{note}

\begin{note}
  In outline.
  \begin{itemize}
  \item
    \supportI{} \hfill \autoref{cha:var:support:I}
  \item
    \witness{} \hfill \autoref{cha:var:support:W}
  \item
    \supportII{} \hfill \autoref{cha:var:support:II}
  \item
    Embedding \hfill \autoref{cha:var:support:Emb}
  \end{itemize}
\end{note}

\subsection{\supportI{}}
\label{cha:var:support:I}

\begin{note}
  \supportI{} states, roughly, that the event in which an agent concludes \(\pv{\phi}{v}\) from \(\Phi\) is sufficient for a \ros{} to hold between \(\pv{\phi}{v}\) and \(\Phi\).

  \begin{idea}[\supportI{}]
    \label{idea:support}
    For an agent \vAgent{}, a proposition-value pair \(\pv{\phi}{v}\), pool of premises \(\Phi\), and event \(e\):

    \begin{itemize}
    \item[\emph{If}]
      \begin{enumerate}[label=\alph*., ref=(\alph*)]
      \item
        \(e\) is an event in which \vAgent{} concludes \(\pv{\phi}{v}\) from \(\Phi\).
      \end{enumerate}
    \item[\emph{then}]
      \begin{enumerate}[label=\alph*., ref=(\alph*), resume]
      \item
        When \vAgent{} pairs \(\phi\) with \(v\) as a sub-event of \(e\):
        \begin{itemize}
        \item
          A \emph{\ros{}} between \(\pv{\phi}{v}\) and \(\Phi\) holds, from \agpe{\vAgent{}'}.
        \end{itemize}
      \end{enumerate}
    \end{itemize}
    \vspace{-\baselineskip}
  \end{idea}

  The focus on the sub-event in which the agent pairs \(\phi\) with \(v\) is to allow for the \ros{} to, in part, explain why the agent concludes \(\pv{\phi}{v}\) from \(\Phi\) without requiring that the \ros{} holds, from the \agpe{} prior to the agent forming the conclusion that \(\phi\) has value \(v\).

  In this respect, a \ros{} between \(\pv{\phi}{v}\) and \(\Phi\) may be regarded as a static account of how the agent has come to pair \(\phi\) with \(v\).
  In other words, the \ros{} between \(\pv{\phi}{v}\) and \(\Phi\) just captures whatever it is, from the \agpe{}, that led to the agent concluding \(\pv{\phi}{v}\) from \(\Phi\).

  Still, \supportI{} is only a sufficient condition, and the suggestion --- however the details work out --- is indented as intuition for when the agent concludes \(\pv{\phi}{v}\) from \(\Phi\).
  As we will see when discussing \supportII{}, we will deny the converse of \supportI{}.
  Therefore, the intuition is not suitable to capture in general what a \ros{} holding, from an \agpe{}, amounts to.

  Generalised, what it is, that has, is, or will, relate \(\pv{\phi}{v}\) and \(\Phi\), from the \agpe{}.
\end{note}

\begin{note}
 ~\supportI{} is distinct from~\citeauthor{Boghossian:2014aa}'s Taking Condition:%
  \footnote{
    There are various objections to~\citeauthor{Boghossian:2014aa}'s Taking Condition, though we take no stance on whether~\citeauthor{Boghossian:2014aa}'s Taking Condition holds.

    See, for example,~\textcite{Hlobil:2014tq}, \textcite{McHugh:2016vp}, and~\textcite{Wright:2014tt}.

    \citeauthor{Hlobil:2014tq} argues against the Taking Condition as it distracts from what accounts of reasoning out to explain, rather than arguing against the Taking Condition directly.

    \citeauthor{McHugh:2016vp} summarise various objects to the taking condition, and present district arguments against against (distinct) ideas in favour of the taking condition.
    In particular,~\supportI{} is closer to what \citeauthor{McHugh:2016vp} term the `Consequence Condition' (\citeyear[cf.][316]{McHugh:2016vp}), which \citeauthor{McHugh:2016vp} also (indirectly) argue against.
    However, \citeauthor{McHugh:2016vp} does not consider an alternative account of what distinguishes concluding from any other action, and as~\supportI{} is designed to capture this distinction, it is unclear to me whether \citeauthor{McHugh:2016vp}'s arguments apply to~\supportI{} (if, indeed, they are sound).

    \citeauthor{Wright:2014tt} denies that reasoning must involve a state which connects premises to conclusions. (\citeyear[Cf.][33-34]{Wright:2014tt})
  }

  \begin{quote}
    (Taking Condition):
    Inferring necessarily involves the thinker \emph{taking} his premises to support his conclusion and drawing his conclusion because of that fact.%
    \mbox{}\hfill\mbox{(\citeyear[5]{Boghossian:2014aa})}
  \end{quote}

  \begin{quote}
    The intuition behind the Taking Condition is that no causal process counts as inference, unless it consists in an attempt to arrive at a belief by figuring out what, in some suitably broad sense, is supported by other things one believes.%
    \mbox{}\hfill\mbox{(\citeyear[5]{Boghossian:2014aa})}
  \end{quote}

  In short, \citeauthor{Boghossian:2014aa} assumes that inferring is a causal process, and the role of the Taking Condition is to distinguish the specific process of inferring from some other causal process.
  Of course, \citeauthor{Boghossian:2014aa} states the Taking Condition in terms of inferring, but extends to concluding.

  In contrast, we do not require that \support{} has any particular role an event in which an agent concludes \(\pv{\phi}{v}\) from \(\Phi\).
  As mentioned, static (and partial) perspective on dynamic process.
\end{note}

\begin{note}
  \phantlabel{Wright-simple-supportI}
  Indeed, our understanding of \support{} is close to \citeauthor{Wright:2014tt}'s (\citeyear{Wright:2014tt}) `Simple Proposal':
  \begin{quote}
    \dots consider instead the proposal, not that the status of the transition as inferential depends on the thinker's judgments about his reasons, but that it depends on \emph{what his reasons are}.
    We want his acceptance of the premises to supply his \emph{actual} reasons for accepting the conclusion.

    \mbox{}\hfill\(\vdots\)\hfill\mbox{}

    Call this the Simple Proposal.
    It says that a thinker infers q from p\(_{1}\) \(\cdots\) p\(_{\text{n}}\) when he accepts each of p\(_{1}\) \(\cdots\) p\(_{\text{n}}\), moves to accept q, and does so for the reason that he accepts p\(_{1}\) \(\cdots\) p\(_{\text{n}}\).%
    \mbox{}\hfill\mbox{(\citeyear[33]{Wright:2014tt})}
  \end{quote}

  \citeauthor{Wright:2014tt}'s simple proposal is that, from the \agpe{}, the relation between pool of premises and conclusion need not be part of why agent moves to accept conclusion.
  Considering how the agent concludes \(\pv{\phi}{v}\) from \(\Phi\) is sufficient.

  \begin{quote}
      What is needed, then, is an account of, or at least some insight into, what it is for certain intentional states of a thinker to be his actual reasons for his transition to another intentional state.

      [Which avoids] committing to the notion that doing something for certain reasons must involve a state that somehow registers those reasons as reasons for what one does.%
      \mbox{}\hfill\mbox{(\citeyear[34]{Wright:2014tt})}
    \end{quote}

  Still, following the discussion above, there is an important between~\supportI{} and \citeauthor{Wright:2014tt}'s Simple Proposal.
  For,~\supportI{} is an entailment, while \citeauthor{Wright:2014tt}'s Simple Proposal is an identity statement.
  Inferring, on the Simple Proposal, is an agent accepting some conclusion for the reason that they accept some (pool of) premises.
 ~\supportI{} does not entail that concluding is nothing more than pairing \(\phi\) with value \(v\) as a result of \(\Phi\).
\end{note}

\subsection{A \witness{0} for a \ros{0}}
\label{cha:var:support:W}

\begin{note}
  We define \witness{} as follows:

  \begin{definition}[A \witness{2}]
    \label{def:witnessing}
    For an agent \vAgent{}, proposition-value pair \(\pv{\phi}{v}\), and pool of premises \(\Phi\):

    \begin{enumerate}[label=]
    \item
      \begin{enumerate}[label=\alph*., ref=(\alph*), series=WitnessDef]
      \item
        \vAgent{} has a \emph{\witness{0}} for \ros{} between \(\pv{\phi}{v}\) and \(\Phi\).
      \end{enumerate}
    \item
      \emph{If and only if:}
    \item
      \begin{enumerate}[label=\alph*., ref=(\alph*), resume*=WitnessDef]
      \item
        There is some event \(e\) such that \(e\) is an event in which \vAgent{} concludes \(\pv{\phi}{v}\) from \(\Phi\).
      \end{enumerate}
    \end{enumerate}
    \vspace{-\baselineskip}
  \end{definition}

  If \(e\) is event of concluding \(\pv{\phi}{v}\) from \(\Phi\), then when agent pairs \(\phi\) with \(v\), the agent has a \witness{} for \ros{} between \(\pv{\phi}{v}\) and \(\Phi\).
\end{note}

\begin{note}
  An important, but trivial, case of \autoref{def:witnessing} is when an agent concludes \(\pv{\phi}{v}\) from \(\Phi\).
  For, if an agent concludes \(\pv{\phi}{v}\) from \(\Phi\) then it is immediate that there is some event in which the agent concludes \(\pv{\phi}{v}\) from \(\Phi\) --- the very same event --- and hence the agent has a \witness{} for the \ros{} between \(\pv{\phi}{v}\) and \(\Phi\).

  Hence, joining \supportI{} with \autoref{def:witnessing}, we have the following:

  \begin{proposition}[Concludes, then witnessed \support{}]
    \label{prop:cws}
    For an agent \vAgent{}, proposition-value pair \(\pv{\phi}{v}\) and pool of premises \(\Phi\):
    \begin{itemize}
    \item
      If \(e\) is an event in which \vAgent{} concludes \(\pv{\phi}{v}\) from \(\Phi\) then:
      \begin{itemize}
      \item
        When \vAgent{} pairs \(\phi\) with \(v\) as a sub-event of \(e\), a \ros{} between \(\pv{\phi}{v}\) and \(\Phi\) holds, from \agpe{\vAgent{}'}.%
        \mbox{ }\hfill(\supportI{})
      \item
        \vAgent{} has a \witness{} for the \ros{} between \(\pv{\phi}{v}\) and \(\Phi\), via \(e\).%
        \mbox{ }\hfill(\autoref{def:witnessing})
      \end{itemize}
    \end{itemize}
    \vspace{-\baselineskip}
  \end{proposition}

  \autoref{prop:cws} is of interest with respect to \qWhy{}, \qHow{}, \issueInclusion{}, and the variations to follow in \autoref{sec:variants-initial}.

  For, our variant to \qWhy{} will involve \ros{1}.
  Our variant to \qHow{} will involve \witness{1}.
  And, our variant to \issueInclusion{} will hold that a \ros{} is, in part, an answer to why an agent concluded only if the agent has a \witness{} for the \ros{}.

  Hence, \autoref{prop:cws} ensures that so long as there is an event in which the agent concludes \(\pv{\phi}{v}\) from \(\Phi\), then an answer to `why' will always have a corresponding answer to `how'.

  At issue is whether it is always the case that an agent has a \witness{} for a \ros{} which is, in part, an answer to why the agent concluded \(\pv{\phi}{v}\) from \(\Phi\).

  And, given \autoref{prop:cws} it is immediate that any such \ros{} must be distinct from the \ros{} between \(\pv{\phi}{v}\) and \(\Phi\).
\end{note}

\begin{note}
  Before proceeding to \supportII{}, we make a brief note and consider a slight objection.
\end{note}

\begin{note}
  Note:

  When we talk of \witness{1} we talk in terms of `having a \witness{0}'.
  In the case of \autoref{prop:cws}, the event in which the agent concludes and the event which secures the relevant \witness{} are identical.

  However, event \(e\) may be an event in which an agent concludes \(\pv{\phi}{v}\) from \(\Phi\) such that throughout the event \(e\), the agent has a \witness{} for a \ros{} between \(\pv{\psi}{v'}\) and \(\Psi\), such that the event \(e'\) which \witness{1} the \ros{} between \(\pv{\psi}{v'}\) and \(\Psi\) is distinct from \(e\).

  Hence, our understanding of `having a \witness{0}' allows for the possibility that some \ros{} between \(\pv{\psi}{v'}\) and \(\Psi\), in part, `answers why' an agent concludes \(\pv{\phi}{v}\) from \(\Phi\) though the relevant \witness{0} for the \ros{} between \(\pv{\psi}{v'}\) and \(\Psi\) is distinct.

  If you think there may be such cases, then the variant to \issueInclusion{} that we develop will be compatible with such cases.
  And, if you think there are no such cases, then it is safe to ignore this possibility.
  We will not directly, at least, consider such cases or take a stand either way in the main argument.%
  \footnote{
    Though, we will tentatively suggest a \scen{0} in \autoref{fn:past-witness} on \autopageref{fn:past-witness}.
  }
  However, the possibility suggests a objection.
\end{note}

\begin{note}
  Objection:

  There may be instances where an agent reasoned to \(\pv{\phi}{v}\) but did not conclude \(\phi\) has value \(v\) such that the event in which the agent reasoned to \(\pv{\phi}{v}\) serves as a \witness{} to a \ros{} between \(\pv{\phi}{v}\) and \(\Phi\).

  As \autoref{def:witnessing} requires the event to be such that the agent concludes \(\pv{\phi}{v}\) from \(\Phi\), such events are excluded from being \witness{1}.

  To illustrate, consider an agent working through a proof of some theorem.
  Suppose the agent completes the proof, and the agent's reasoning is sound, but the agent is worried about a particular step in the proof.
  Given the worry, the agent completes their reasoning, but does not conclude the theorem is true.
  However, some time later the agent returns to investigate the step further, resolves their worries, and concludes the theorem is true.

  It seems, 

  One way of viewing this \scen{} is in terms of an extended event.
  The agent completes their reasoning, in a sense, but the event in which the agent concludes carries on, and hence the agent's initial reasoning still serves as a \witness{}.
  Still, there may be a significant gap in time between the reasoning and investigation


  Investigate step, and strengthen.
  Strictly, have not concluded \(\pv{\phi}{v}\) from \(\Phi\).
  Rather, from previous reasoning and premise that reasoning was sound.

  Still, further proof, and would not conclude if prior proof was not sound.

  Plausible to weaken \autoref{def:witnessing} to reasoning.
  Hence, reasoning later seen to be sound would qualify.

  We do not weaken \autoref{def:witnessing} in order to allow for an intuitive understanding of the relevant event \(e\) being a \witness{} for the \ros{} between \(\pv{\phi}{v}\) and \(\Phi\).

  Given the example, no problem.
  However, discovered some issue with the proof.
  Without providing a more detailed characterisation of \ros{1}, it is difficult --- if not impossible --- to distinguish these cases.
\end{note}

\subsection{\supportII{}}
\label{cha:var:support:II}

\begin{note}
  \begin{idea}[\supportII{}]
    \label{idea:support:possible}
    For an agent \vAgent{}, a proposition-value pair \(\pv{\phi}{v}\), and pool of premises \(\Phi\):

    \begin{itemize}
    \item
      It is possible for both~\ref{idea:support:possible:a} and~\ref{idea:support:possible:b} be true:
      \begin{enumerate}[label=\alph*., ref=(\alph*)]
      \item
        \label{idea:support:possible:a}
        A \ros{} between \(\pv{\phi}{v}\) and \(\Phi\) holds, from \agpe{\vAgent{}'}.
      \item
        \label{idea:support:possible:b}
        \vAgent{} does not have a \witness{} for the \ros{} between \(\pv{\phi}{v}\) and \(\Phi\).
      \end{enumerate}
    \end{itemize}
    \vspace{-\baselineskip}
  \end{idea}

  Note, \supportII{} does not require there exist instances of \support{}.
  \autoref{cha:fcs} will focus on argument that there are instances of \support{} without event of in which agent concludes.
  These we will term \fc{1}.%
  \footnote{
    \citeauthor{Firth:1978vi}'s (\citeyear{Firth:1978vi}) distinction between doxastic and propositional justification (or warrant).
    See also \citeauthor{Silva:2020aa} (\citeyear{Silva:2020aa}) --- esp.\ fn.\ 1.

    {\color{red}
      Compare \citeauthor{Firth:1978vi}'s example with Holmes and Watson (\citeyear[218]{Firth:1978vi}).
      Watson is presented with all the evidence Holmes used to that the coachman committed the murder, and that this provides Watson with sufficient epistemic reasons regardless of whether or not Watson forms any attitude, but it is not clear that Watson has the understanding to piece together the evidence laid before them.
    }
  }

  \supportII{} establishes the \emph{possibility} of \support{} answering why, without the proposition-value-premises pairing being involved in how.

  Still, looking ahead a little, \supportII{} should not be of any immediate concern.
  For, suppose consider an instance of \support{} without event in which agent concludes.
  There mere existence of such an relation does not show that the relation is relevant to why.
\end{note}

\subsubsection{\fc{3}}
\label{cha:var:support:II:fcs}

\begin{note}
  \supportI{} and \supportII{}.

  To avoid mystery, possibility given by \fc{3}.
  \fc{} just in case, roughly, some action such that event results from performing action is an event in which concludes in progress.
\end{note}

\begin{note}
  Also important, converse: If no \ros{} then not a \fc{}.
\end{note}


\subsection{Embedding \ros{1}}
\label{cha:var:support:Emb}

\begin{note}
  Given some proposition-value pair \(\pv{\phi}{v}\) and pool of premises \(\Psi\), our interest is with the following distinction:

  \begin{enumerate}[label=\arabic*., ref=(\arabic*)]
  \item
    \label{Embed:no}
    A \ros{0} between \(\pv{\psi}{v'}\) and \(\Psi\).
  \item
    \label{Embed:yes}
    A \ros{0} between \(\Phi\) and \(\pv{\phi}{v}\), where:
    \begin{itemize}
    \item
      \(\Phi\) contains the proposition-value-premises pairing:
      \begin{itemize}
      \item
        \(\pv{\text{A \ros{} between }\pv{\psi}{v'}\text{ and }\Psi}{\text{True}}\)
      \end{itemize}
    \end{itemize}
  \end{enumerate}

  \ref{Embed:no} is a \ros{} between \(\pv{\psi}{v'}\) and \(\Psi\).
  Likewise, \ref{Embed:yes} involves a \ros{} between \(\pv{\psi}{v'}\) and \(\Psi\).
  However, the \ros{} between \(\pv{\psi}{v'}\) and \(\Psi\) is itself a premise in a \ros{} between \(\pv{\phi}{v}\) and \(\Phi\).
  In this respect, we will say the \ros{} between \(\pv{\psi}{v'}\) and \(\Psi\) is \emph{embedded} within a \ros{} --- specifically the \ros{} between \(\pv{\phi}{v}\) and \(\Phi\).

  In general, our understanding of `embedding' just requires that a proposition-value pair is `somewhere' in a pool of premises.
  In full, we define embedding in the following way:%
  \footnote{
    We assume, in general, that if a proposition \(\phi\) includes some other proposition \(\phi'\), then if \(\pv{\phi}{\text{True}} \in \Phi\) then \(\pv{\phi'}{\text{True}} \in \Phi\).
    In other words, if \(\pv{\phi'\text{ and }\phi''}{\text{True}} \in \Phi\) then both \(\pv{\phi'}{\text{True}} \in \Phi\) and \(\pv{\phi''}{\text{True}} \in \Phi\).
    We do not extend this assumption to values other than `True'.
  }

  \begin{definition}[Embedded within a \ros{}]
    \label{def:embedding}
    For a proposition-value pairs \(\pv{\chi}{v''}\), \(\pv{\phi}{v}\), and a pool of premises \(\Phi\):


    \begin{itemize}
    \item
      \(\pv{\chi}{v''}\) is \emph{embedded} within in a \ros{} between \(\pv{\phi}{v}\) and \(\Phi\)
    \end{itemize}

    \emph{If and only if:}

    \begin{itemize}
    \item
      \(\pv{\chi}{v''}\) is has a degree of embedding \(i\) with respect to \(\pvp{\phi}{v}{\Phi}\), for some \(i \in \mathbb{N}\).

    Where:
    \begin{itemize}
    \item
      \(\pv{\chi}{v''}\) is has a degree of embedding \(1\) with respect to \(\pvp{\phi}{v}{\Phi}\) if and only if \(\pv{\chi}{v''} \in \Phi\).
    \item
      \(\pv{\chi}{v''}\) is has a degree of embedding \(j\) with respect to \(\pvp{\phi}{v}{\Phi}\) if and only if:
      \begin{itemize}
      \item
        There exists some \(\pv{\theta}{v^{i}}\) and \(\Theta\) such that:
        \begin{itemize}
        \item
          \(\pv{\chi}{v''} \in \Theta\)
        \item
          \(\pv{\text{A \ros{} between }\pv{\theta}{v^{i}}\text{ and }\Theta}{\text{True}}\) is a \(j - 1\) embedding with respect to \(\pvp{\phi}{v}{\Phi}\).
        \end{itemize}
      \end{itemize}
    \end{itemize}
  \end{itemize}
  \vspace{-\baselineskip}
  \end{definition}

  Given this broad definition, the cases of particular interest to us are where:
  \begin{itemize}
  \item
    \(\pv{\chi}{v''}\) is embedded within in a \ros{} between \(\pv{\phi}{v}\) and \(\Phi\), no matter the degree of embedding.
  \item
    \(\chi\) is the proposition `\(\text{A \ros{} between }\pv{\psi}{v'}\text{ and }\Psi\)' and \(v''\) is the value `True'.
  \end{itemize}
\end{note}

\begin{note}
  Understanding in hand, turn to the function of distinguishing embedded from unembedded \ros{}.

  Simple.

  Whether the \ros{} was a premise from which the agent concluded \(\phi\) has value \(v\).
\end{note}

\begin{note}
  Idea is somewhat familiar from propositional logic.
  Certain kind of equivalence between proof and conditional.
  It is possible to find a corresponding conditional to any proof with a finite number of premises, proof captures derivation of conclusion from premises.

  Corresponding conditional is not a premise, nor any part, of the proof.

  For example, consider a proof from \(P\) and \(P \rightarrow Q\) to \(Q\) by conditional detachment.
  Corresponding conditional is \((P \land (P \rightarrow Q)) \rightarrow Q\).
  However, not part of the proof.

  Intuitive distinction between what a proof and a conditional refer to.
  However, informally there is no difficulty in treating a proof as a premise.
  \(P\), and I have a proof of \(P \rightarrow Q\), therefore \(Q\).
\end{note}

\begin{note}
  Discussion by \citeauthor{Carroll:1895uj} in \citetitle{Carroll:1895uj}.

  \begin{quote}
    My paradox \dots turns on the fact that, in a Hypothetical, the \emph{truth} of the Protasis, the \emph{truth} of the Apodosis, and the \emph{validity of the sequence}, are 3 distinct Propositions.

    \mbox{}\hfill\(\vdots\)\hfill\mbox{}

    Suppose I say ``I grant~\ref{AatT:a} and~\ref{AatT:b} and~\ref{AatT:c}, but I do \emph{not} grant that I am thereby \emph{obliged} to grant~\ref{AatT:z}.''
    Surely, my granting~\ref{AatT:z} must \emph{wait} until I have been made to see the validity of this sequence: i.e.\ in order to grant~\ref{AatT:z}, I must grant~\ref{AatT:a},~\ref{AatT:b},~\ref{AatT:c}, and~\ref{AatT:d}! And so on.%
    \mbox{ }\hfill\mbox{(\citeyear[472]{Carroll:1977wl})}
  \end{quote}

  \citeauthor{Carroll:1895uj} is slightly different.
  For, rather than \ros{}, \citeauthor{Carroll:1895uj} focuses on valid inferences.

  My understanding of \citeauthor{Carroll:1895uj} is in terms of the relation between general and specific.

  Inference from \(A\) and \(A \rightarrow B\) to \(B\).
  Valid inference.
  However, valid inference if and only if holds for all \emph{A} and \emph{B}.
  Point, then, is that given \(A\) and \(A \rightarrow B\) still need validity.
  Even if grant \(B\) is true, this only gives specific instance.
  However, at issue is general.
  Does \(D\) follow from \(C, C \rightarrow D\)?

  How do we get a general rule without already being sure of the specific instances of the general rule.

  Immediate observation is that general rule does not correspond to any specific instance to which the general rule applies.

  Hence, motivates the same idea.
  In some cases, these two things are different.
\end{note}

\begin{note}
  Start to see a difference by considering \(\pv{\phi}{v}\) and \(\Phi\).
  For, \ros{} between \(\pv{\phi}{v}\) and \(\Phi\).
  However, immediate difficulties if the proposition that there is a \ros{} between \(\pv{\phi}{v}\) and \(\Phi\) paired with the value `True' is a member of \(\Phi\).

  Likewise, \ros{} between the single premise and \(\pv{\phi}{v}\) is, intuitively, distinct from \ros{} between \(\pv{\phi}{v}\) and \(\Phi\).

  \ros{} between \(\pv{\phi}{v}\) and \(\Phi\) results from an event in which an agent concludes \(\pv{\phi}{v}\) from \(\Phi\).
  By contrast, \ros{} between \(\pv{\phi}{v}\) and \(\Phi\) results from an event in which an agent concludes \(\pv{\phi}{v}\) from the \ros{} between \(\pv{\phi}{v}\) and \(\Phi\).
\end{note}

\begin{note}
  Significance for arguments is that dependence captured by \qWhyVnP{} does not amount to the \ros{} being a premise.
\end{note}


\section{\qWhyVnP{} and \qHowV{}, and \issueConstraint{}}
\label{sec:variants-initial}

\begin{note}
  In \autoref{cha:var:sec:wiggling} we outlined how we will developed variations of \qWhy{} and \qHow{} in terms of \ros{}.
  And, in \autoref{cha:var:support}, \support{}.
  This section states the variations in full.

  The section is split into three sub-sections.

  \begin{itemize}
  \item
    \qWhyVnP{}%
    \mbox{ }\hfill\autoref{sec:variants-initial:qwhyvnp}

    Variation on \qWhy{}.
  \item
    \qHowV{}%
    \mbox{ }\hfill\autoref{sec:variants-initial:qhowv}

    Variation on \qHow{}.
  \item
    \issueConstraint{}%
    \mbox{ }\hfill\autoref{sec:variants-initial:issue}

    Variation on \issueInclusion{}.
  \end{itemize}

  Variations.
  \qWhy{} and \qHow{} are broad.
  Variations will not serve as substitutes.
  Instead, link \qWhyVnP{} to \qWhy{} and link \qHowV{} to \qHow{}.

  \issueConstraint{}, the variation on \issueInclusion{} will then, primarily, be seen as a consequence of assuming \issueInclusion{} holds with respect to \qWhy{} and \qHow{}, and the two links between the variant questions and the original.

  Still, I hope to have developed \qWhyVnP{} and \qHowV{} in such a way that both questions are of some interest independent of link, and that \issueConstraint{} follows the same intuition that motivates \issueInclusion{}.

  Indeed, in \autoref{cha:clar:sec:literature} we suggest how a handful of accounts of conclusion, or related, may be understood in terms of \qWhyVnP{} and \qHowV{}, and how the accounts motivate \issueConstraint{} as a constraint on answers to \qWhyVnP{} in terms of answers to \qHowV{}.

\end{note}

\subsection{\qWhyVnP{}}
\label{sec:variants-initial:qwhyvnp}

\begin{note}
  As forecast in \autoref{cha:var:sec:qov}, eliminate `why' in favour of whether pairing \(\phi\) and \(v\) depends on \ros{}.

  \begin{restatable}[\qWhyVnP{}]{question}{questionWhyVnP}
    \label{q:why:v:nP}
    Given an agent \vAgent{}, proposition-value pair \(\pv{\phi}{v}\), pool of premises \(\Phi\), and event \(e\) in which \vAgent{} concludes \(\pv{\phi}{v}\) from \(\Phi\):

    \begin{quote}
      Which proposition-value-premises pairings \(\pvp{\psi}{v'}{\Psi}\) are such that, when \vAgent{} pairs \(\phi\) with \(v\):

      \begin{enumerate}[label=]
      \item
        \begin{enumerate}[label=\alph*., ref=(\alph*), series=qWhyVnPdef]
        \item
          \label{q:why:v:a}
          \support{2} between \(\pv{\psi}{v'}\) and \(\Psi\) holds, from \agpe{\vAgent{}'}.
        \end{enumerate}
      \end{enumerate}

      And:

      \begin{enumerate}
      \item[\emph{If}:]
        \begin{enumerate}[label=\alph*., ref=(\alph*), resume*=qWhyVnPdef]
        \item
          \label{q:why:v:if}
          \support{2} between \(\pv{\psi}{v'}\) and \(\Psi\) failed to hold, from \agpe{\vAgent{}'}.
        \end{enumerate}
      \item[\emph{Then}:]
        \begin{enumerate}[label=\alph*., ref=(\alph*), resume*=qWhyVnPdef]
        \item
          \label{q:why:v:then}
          \vAgent{} would not have concluded \(\pv{\phi}{v}\) from \(\Phi\).
        \end{enumerate}
      \end{enumerate}
    \end{quote}
    \vspace{-\baselineskip}
  \end{restatable}

  \qWhyVnP{} captures when a \ros{} is, in part, why an agent concludes \(\pv{\phi}{v}\) from \(\Phi\) by two conditions.
  First, simple, \ros{} holds.
  Second, \emph{if-then} conditional.

  \ref{q:why:v:a} expresses the simple idea that \support{} holds from the \agpe{}.
  Take this as basic necessary condition that \support{} holds.
  \emph{If-then} conditional captures dependence, and \ref{q:why:v:a} ensures that what conclusion depends on is present.

  \emph{If-then} conditional.
  Between \support{} and whether or not the event is such that the agent concludes.
  This captures are simple idea of dependence.
  Defer discussion to \autoref{sec:dependence}.

  Role of the \emph{if-then} conditional is to capture sense in which concluding \(\pv{\phi}{v}\) from \(\Phi\) depends on \ros{} holding between \(\pv{\phi}{v}\) and \(\Phi\).

  If dependence, then \ros{}, in part, answers `why' the agent concluded \(\pv{\phi}{v}\) from \(\Phi\).

  For present purposes, key observation is that answers to \qWhyVnP{} are given in terms of \ros{1}.

  Setting aside what \ros{1} amount to, the consequent of the \emph{if-then} conditional \ref{q:why:v:then} is a subjunctive, and evaluating subjunctives is quite difficult.
  Indeed, I have no general account of how to evaluate \ref{q:why:v:then} when \ref{q:why:v:then} is set apart from an \agpe{}.%
  \footnote{
    For example, intuitively it should not be the case that the relation of support and the conclusion that it would be good to get some coffee answers, in part, why the agent concluded some theorem is true from some pool of premises.
    Though, it may well be the case that the agent would have been too tired to concludes the theorem without the aid of the coffee.
  }
  This is, in part, by design.
  The way in which \issueInclusion{} constrains answers to \qWhy{} in terms of answers to \qHow{} is taken to be substantial.
  In this respect, similar constraint is seen to be substantial due to the way in which the constraint constrains the understanding of the \emph{if-then} conditional.

  Indeed, argue against a parallel constraint, and therefore my understanding of the \emph{if-then} conditional would be of little interest at this point.
  For, given in full it would be the case that the constraint fails to hold.

  The core idea is that \emph{if} dependence as captured by the \emph{if-then} conditional, then answer to \qWhy{}.
\end{note}

\begin{note}
  Plausible holds for basic cases.
  If testimony was no good, would not conclude.
  And various other cases.

  Without a clear account of what depending amounts to, there is no deductive argument that holds in general.
  However, intention is that trivially holds for \ros{} between \(\pv{\phi}{v}\) and \(\Phi\).
\end{note}

\begin{note}
  Intuitively, \support{} between \(\pv{\phi}{v}\) and \(\Phi\) is always an answer to \qWhyVnP{}.
  For, if \support{} failed to hold, then the event would not be an event in which the agent concludes \(\pv{\phi}{v}\) from \(\Phi\).
\end{note}

\begin{note}
  We link \qWhyVnP{} and \qWhy{} in the following way:

  \begin{restatable}[\qWhy{} and \qWhyVnP{}]{link}{linkSupportWhy}
    \label{link:why:support:pvpp}
    For an agent \vAgent{}, and proposition-value-premises pairings \(\pvp{\psi}{v'}{\Psi}\):

    \begin{itemize}
    \item[\emph{If}:]
      \begin{enumerate}[label=\alph*., ref=(\alph*)]
      \item
        \support{2} between \(\pv{\psi}{v'}\) and \(\Psi\) is, in part, an answer to \qWhyVnP{}.
      \end{enumerate}
    \item[\emph{Then}:]
      \begin{enumerate}[label=\alph*., ref=(\alph*), resume]
      \item
        \(\pvp{\psi}{v'}{\Psi}\) is, in part, an answer to \qWhy{}.
      \end{enumerate}
    \end{itemize}
    \vspace{-\baselineskip}
  \end{restatable}

  Key thing is dependence captured links to why.
  Agent would not conclude \(\pv{\phi}{v}\) from \(\Phi\) unless \ros{} holds been \(\pv{\psi}{v'}\) and \(\Psi\) from the \agpe{}.

  In short, \autoref{link:why:support:pvpp} states that \support{} between \(\pv{\psi}{v'}\) and \(\Psi\) which answers \qWhyVnP{} is sufficient for \(\pvp{\psi}{v'}{\Psi}\) to answer \qWhy{}, in part.

  Note, \autoref{link:why:support:pvpp} only amounts to sufficiently, and only captures a partial answer to \qWhy{}.

  Sufficiency, converse does not hold, there may be cases where \(\pvp{\psi}{v'}{\Psi}\) answers \qWhy{} but there is no corresponding \ros{} between \(\pv{\psi}{v'}\) and \(\Psi\) that answers \qWhyVnP{}.
  Indeed, \qWhy{} is not restricted to \ros{1}.

  Partiality, in the case of \qWhy{}  language is used to allow for other things to answer.
  In case of \qWhyVnP{}, for multiple \ros{}.

  Hence, we are not committed to relations of \support{} capturing answers to \qWhy{}.
\end{note}

\subsection{\qHowV{}}
\label{sec:variants-initial:qhowv}

\begin{note}
  Variant on \qHow{} is as follows:

  \begin{restatable}[\qHowV{}]{question}{questionHowV}
    \label{q:how:v}
    Given event \(e\) where agent concludes \(\pv{\phi}{v}\) from \(\Phi\), which proposition-value-premises pairing are such that:

    There is an event \(e'\) either prior or identical to \(e\), such that \(e'\) is event in which \vAgent{} witnesses \support{} between \(\pv{\psi}{v'}\) from \(\Psi\).
  \end{restatable}

  Following \ros{1} as answers to \qWhyVnP{}, \qHowV{} concerns \witness{1} for \ros{1}.

  In this respect, parallel to \issueInclusion{}
  \ros{} is only an answer to \qWhyVnP{} if the agent has a \witness{}.
\end{note}

\begin{note}
  \begin{restatable}[\qHow{} and \qHowV{}]{link}{linkHowWitnessing}
    \label{link:how-witnessing}
    For any proposition-value-premises pairing \(\pvp{\psi}{v'}{\Psi}\):
    \begin{itemize}
    \item[\emph{If}]
      \begin{enumerate}[label=\alph*., ref=(\alph*)]
      \item
        \(\pvp{\psi}{v'}{\Psi}\) is, in part, an answer \qHow{} due to \support{} holding between \(\pv{\psi}{v'}\) and \(\Psi\).
      \end{enumerate}
    \item[\emph{then}]
      \begin{enumerate}[label=\alph*., ref=(\alph*), resume]
      \item
        \(\pvp{\psi}{v'}{\Psi}\) is, in part, an answer \qHowV{}.
      \end{enumerate}
    \end{itemize}
    \vspace{-\baselineskip}
  \end{restatable}
\end{note}

\begin{note}
  Has a \witness{0} is weak.
  Event in which \dots

  View having a \witness{0} as a necessary constraint.
  Constraint, without being too constraining.
  Do not want to rule out plausible accounts of concluding.
  In this respect, think of things which are sufficient.

  For example, causal theory.
  If causal, then constrain why by causes.
  And, if cause then has a \witness{0}.

  Rules, then \witness{0} to rules.

  Contrast, exercising disposition.
  In this case, exercise of the disposition results in a \witness{0}.

  So long as account requires having a \witness{0}, then variant to \qHow{} captures way in which account is of interest.
\end{note}

\begin{note}
  However, allows for the \witness{} to be distinct from event.%
  \footnote{
    \phantlabel{fn:past-witness}
    To illustrate, consider an agent working on some mathematical problem.

    As part of their work on the problem the agent concludes the hypotenuse of some right-angled triangle is \(\sqrt{74}\text{cm}\) by use of the Pythagorean theorem.

    Further, the agent has, at some point in the past proved the Pythagorean theorem from more basic principles.

    Now, generally speaking, it seems to me it may be the case that the agent concludes the hypotenuse of the triangle is \(\sqrt{74}\text{cm}\), in part, from those more basic principles.
    For example, the agent may have just completed their proof of the Pythagorean theorem and the reasoning from the more basic principles to the hypotenuse of the triangle may be considered a single unified instances of reasoning, with an intermediary conclusion.

    Still, suppose the agent proved the Pythagorean theorem some years ago.

    Perhaps the agent's reasoning from more basic principles continues to provide, in part, an answer to how the agent concluded the hypotenuse of the triangle is \(\sqrt{74}\text{cm}\).
    There may be a gap of some years, but it may be the case that the agent uses the Pythagorean theorem (in some sense of the word) \emph{because} they concluded the theorem from more basic principles.

    {
      \color{red}
      Hence, \support{}, and this in part answers \qWhyVnP{}.
    }

    On the other hand, one may be inclined to hold that the more basic principles the agent proved the Pythagorean theorem from have no role in explaining how the agent concluded the hypotenuse of the triangle is \(\sqrt{74}\text{cm}\) in the present.
    Rather, the Pythagorean theorem is a more-or-less fundamental premise of the agent's present reasoning.

    At best, the agent's memory of reasoning from more basic principles to the Pythagorean theorem may, in part, answer how the agent concluded hypotenuse of the triangle is \(\sqrt{74}\text{cm}\).
    The reasoning from the more basic principles, given that it happened so long ago, is irrelevant.
  }
\end{note}

\subsection{\issueConstraint{}}
\label{sec:variants-initial:issue}

\begin{note}
  Case of \(\pv{\phi}{v}\) from \(\Phi\).
  \ros{} between \(\pv{\phi}{v}\) and \(\Phi\) satisfies account of conditions of \qWhyVnP{}, and hence is an answer to \qWhyVnP{}.
  Further, as agent concludes, the agent has a \witness{0} for the \ros{} between \(\pv{\phi}{v}\) and \(\Phi\).
  Therefore, answer to \qHowV{}.
\end{note}

\begin{note}
  \issueConstraint{}:

  \begin{restatable}[\issueConstraint{}]{issue}{rIssueConstraint}
    \label{issue:has-witnessed}
    For an agent \vAgent{}, proposition-value pairs \(\pv{\phi}{v}\), \(\pv{\psi}{v'}\), and pools of premises \(\Phi\), \(\Psi\):

    \begin{enumerate}
    \item[\emph{If}:]
      \begin{enumerate}[label=\alph*., ref=(\alph*)]
      \item \vAgent{} concluded \(\pv{\phi}{v}\) from \(\Phi\).
      \end{enumerate}
    \item[\emph{And}:]
      \begin{enumerate}[label=\alph*., ref=(\alph*), resume]
      \item
        \vAgent{} would not have concluded \(\pv{\phi}{v}\) from \(\Phi\), if \support{} between \(\pv{\psi}{v'}\) and \(\Psi\) failed to hold, from \agpe{\vAgent{}'}.
      \end{enumerate}
    \item[\emph{Then}:]
      \begin{enumerate}[label=\alph*., ref=(\alph*), resume]
      \item
        \vAgent{} has a \witness{0} for the \ros{} between \(\pv{\psi}{v'}\) and \(\Psi\).
      \end{enumerate}
    \end{enumerate}
    \vspace{-\baselineskip}
  \end{restatable}

  In short, \issueConstraint{} holds that a \ros{} between \(\pv{\psi}{v'}\) and \(\Psi\) is an answer to \qWhyV{} \emph{only if} the \ros{} is (also) an answer to \qHowV{}.
  In other words, being an answer to \qHowV{} is a necessary condition on being an answer to \qWhyV{}.
\end{note}

\begin{note}
  \color{red}
  Key thing, this is `substantial'.
  Sense of dependence captured by \qWhyVnP{} is not trivially that of having a \witness{}.
  Still, plausible that dependence is so captured.
\end{note}

\begin{note}
  View \issueConstraint{} as a `direct' constraint on answers \qWhyVnP{} and \qWhy{}.
  Or, as a consequence of \issueInclusion{} and proposed links between \qWhy{} and \qWhyVnP{} and \qHowV{} and \qHow{}.
\end{note}

\begin{note}
 Following:

  \begin{restatable}[]{proposition}{propVariationsAndInclusion}
    \label{prop:support-and-witnessing}
    Grating a positive resolution to \issueInclusion{}, and given \autoref{link:why:support:pvpp} and \autoref{link:how-witnessing}:
    \begin{enumerate}
    \item[\emph{If}:]
      \begin{enumerate}[label=\alph*., ref=(\alph*)]
      \item \vAgent{} concluded \(\pv{\phi}{v}\) from \(\Phi\).
      \end{enumerate}
    \item[\emph{And}:]
      \begin{enumerate}[label=\alph*., ref=(\alph*), resume]
      \item
        \vAgent{} would not have concluded \(\pv{\phi}{v}\) from \(\Phi\), if \support{} between \(\pv{\psi}{v'}\) and \(\Psi\) failed to hold, from \agpe{\vAgent{}'}.
      \end{enumerate}
    \item[\emph{Then}:]
      \begin{enumerate}[label=\alph*., ref=(\alph*), resume]
      \item
        \vAgent{} has witnessed \support{} between \(\pv{\psi}{v'}\) and \(\Psi\).
      \end{enumerate}
    \end{enumerate}
  \end{restatable}

  \autoref{prop:support-and-witnessing} is straightforward.
  A visual representation is given in~\autoref{fig:relations-between-whys-and-hows}.
\end{note}

\begin{figure}[H]
  \centering
  \begin{tikzpicture}
    \tikzset{ansStyle/.style={
        draw=gray,
        text width=.45\textwidth,
        rounded corners=2pt,
      }
    }
    %
    \node[ansStyle] (whyO) at (0,0) %
    {\qWhyV{} is answered by support between \(\pv{\psi}{v'}\) and \(\Psi\).};
    %
    \node[ansStyle] (whyA) at (2,-1.5) %
    {\qWhy{} is answered by \(\pvp{\psi}{v'}{\Psi}\).};
    %
    \node[ansStyle] (howA) at (4,-3) %
    {\qHow{} is answered by \(\pvp{\psi}{v'}{\Psi}\).};
    %
    \node[ansStyle] (witA) at (6,-4.5) %
    {\qHowV{} is answered by witnessed \support{} between \(\pv{\psi}{v'}\) and \(\Psi\).};
    %
    \path[->] ($(whyO.south)!0.9!(whyO.south west)$) edge [out=270, in=180] (whyA);
    \path[->] ($(whyA.south)!0.9!(whyA.south west)$) edge [out=270, in=180] (howA);
    \path[->] ($(howA.south)!0.9!(howA.south west)$) edge [out=270, in=180] (witA);
    %
    \node[text width=.5\textwidth] (1) at (1,-.8) %
    {Via~\autoref{link:why:support:pvpp}.};
    %
    \node[text width=.75\textwidth] (2) at (4.5,-2.25) %
    {Via a positive resolution to~\issueInclusion{}.};
    %
    \node[text width=.5\textwidth] (3) at (5,-3.625) %
    {Via~\autoref{link:how-witnessing}.};
    %
    % \draw[->, gray] ($(whyA.north)!0.9!(whyA.north east)$) to [out=90, in=0] ($(whyO.east)$);
    %
    % \node[text width=.5\textwidth, text=gray] (1p) at (8,-.8) %
    % {Via~\autoref{link:why:pvpp:support}.};
  \end{tikzpicture}%
  \caption{Relation between positive answers to questions.}
  \label{fig:relations-between-whys-and-hows}
\end{figure}

\begin{note}
  As with \issueInclusion{}, \issueConstraint{} distinguishes classes of theories.
  A positive resolution to \issueInclusion{} will not directly provide general answer to \qWhy{} or \qHow{}.
  Though, a positive answer will rule out certain answers.
\end{note}

\subsubsection{Embeddings}
\label{sec:embeddings}

\begin{note}
  Above, we observed how~\supportI{} and~\supportII{} may combine in such a way that \support{} holds between \(\pv{\psi}{v'}\) and \(\Psi\) (from an \agpe{}), though the agent has does not have a \witness{0} for the \ros{0} between \(\pv{\psi}{v'}\) from \(\Psi\) --- or more naturally stated, though the agent has not concluded \(\pv{\psi}{v'}\) from \(\Psi\).

  We have not yet seen any examples to suggest that there are cases in which concluding \(\pv{\phi}{v}\) from \(\Phi\) depends on \ros{} between \(\pv{\psi}{v'}\) and \(\Psi\)%
  \footnote{
    Where either \(\phi \ne \psi\), \(v \ne v'\), \(\Phi \ne \Psi\).
  }
  Still, I have expressed my intent.

  It need not be the case that an agent has a \witness{0} for a \ros{0} in order for \ros{} to be involved in answering \qWhyVnP{}.
\end{note}

\begin{note}
  Now, suppose an agent does not have a \witness{0} for the \ros{} between \(\pv{\psi}{v'}\) and \(\Psi\).
  The upshot of the distinction between~\ref{Embed:no} and~\ref{Embed:yes} is as follows:

  \begin{itemize}
  \item
    If the \ros{0} of \ref{Embed:no} is, in part, an answer to \qWhyVnP{} then the \ros{0} is a counterexample to \issueConstraint{}.
  \item
    If the \ros{0} of \ref{Embed:yes} is, in part, an answer to \qWhyVnP{} then the \ros{0} is \emph{not} a counterexample to \issueConstraint{}.
  \end{itemize}


  Rather, the difference is \emph{how} the \ros{} functions with respect to the agent pairing \(\phi\) with \(v\).
  Whether the \ros{} functions as a premise when the agent concludes \(\pv{\phi}{v}\), or whether the \ros{} functions in a way that is different to a premise.

  What we capture by \ros{} as answer to \qWhyVnP{} and proposition-value pair which captures \ros{} is not necessarily a difference in reference, but a difference in function.
\end{note}



\section{\issueConstraint{} in the literature}
\label{cha:clar:sec:literature}

\begin{note}
  \autoref{sec:variants-initial} introduced variants to \qWhy{}, \qHow{}, and \issueInclusion{}.

  Linked together.

  In short, having a \witness{0} for a \ros{} is necessary for the \ros{} to, in part, explain why an agent concluded.

  Primary motivation is intuition.
  \scen{1} such as \autoref{illu:gist:calc}.

  Motivated via \citeauthor{Davidson:1963aa}.
  Though, \citeauthor{Davidson:1963aa} was a little more general.

  In this section we collect a handful of extracts from the literature which suggest this intuition is not merely an intuition, but a common theoretical constraint.

  Provide extract.
  Observe how plausibly understood in terms of \witness{} and constraint.

  In many cases, constraint is given by identification with \witness{}.
  In turn, understand this to implicitly constrain an answer to why.

  With causal accounts, more or less immediate.
  Nothing that is non-causal
\end{note}


\subsection{Causal}
\label{cha:clar:sec:literature:causal}

\begin{note}
  Causal theories of reasoning.

  Broadly, premises stand in a causal relation to conclusion.

  Cause is something which happens, therefore \witness{}.
  What is relevant to the activity is given why causal process..
\end{note}

\subsubsection{\textcite{Armstrong:1968vh}}

\begin{note}
  We being with~\cite{Armstrong:1968vh}'s (\citeyear{Armstrong:1968vh}) account of inferring.

  \begin{quote}
    We are not concerned here with logicians' questions about inference, but solely with the psychological process of inferring.
    The primary sense of the word is that in which it involves acquiring a belief on the basis of a belief already held.

    \mbox{}\hfill\(\vdots\)\hfill\mbox{}

    \dots to say that A infers \emph{p} from \emph{q} is simply to say that A's believing \emph{q} \emph{causes} him to acquire the belief \emph{p}.
    And the sense of `cause' employed here is the common or billiardball sense of `cause', whatever that sense is.%
    \mbox{}\hfill\mbox{(\citeyear[194]{Armstrong:1968vh})}
  \end{quote}

  \citeauthor{Armstrong:1968vh} tightens the account of inference a little in order to avoid including any belief \emph{r} in the causal chain leading to an agent acquiring the belief \emph{p} as a premise, but we set these details aside.
  (\citeyear[195--197]{Armstrong:1968vh})%
  \footnote{
    Indeed, any disagreement with \citeauthor{Armstrong:1968vh}'s restriction is of no real interest to us.
    For, if we grant that the relevant instance of causation provides a \witness{} for a \ros{}, then, re-expressed, \citeauthor{Armstrong:1968vh}'s revisions narrow down the \witness{1} of interest.
  }

  \citeauthor{Armstrong:1968vh} talks about `inference' rather than `conclusion'.
  However, plausibly the same thing.
  Belief \(\phi\) only if pair \(\phi\) with value `true'.
\end{note}

\begin{note}
  Of some interest.
  \citeauthor{Armstrong:1968vh} traces the causal account of inference to ~\citeauthor{Moore:1962up} and Hume.
\end{note}

\subsubsection{\textcite{Boghossian:2014aa}}

\begin{note}
  Seen above.
  Causal process.

  \citeauthor{Boghossian:2014aa}'s Taking Condition narrows the relevant causal processes.
  However, causal processes, and hence to matter, causally involved.
  If causally involved, then witness.
\end{note}

\subsection{Indeterminate}
\label{sec:indeterminate}

\subsubsection{\textcite{Wright:2014tt}}

\begin{note}
  \citeauthor{Wright:2014tt}'s (\citeyear{Wright:2014tt}) `Simple Proposal'.
  Appealed to the Simple Proposal above on \autopageref{Wright-simple-supportI} to help clarify \supportI{}.

  Here, observation is on acceptance and move.
  Recall, the core of the Simple Proposal is the following idea:

  \begin{quote}
    [A] thinker infers q from p\(_{1}\) \(\cdots\) p\(_{\text{n}}\) when he accepts each of p\(_{1}\) \(\cdots\) p\(_{\text{n}}\), moves to accept q, and does so for the reason that he accepts p\(_{1}\) \(\cdots\) p\(_{\text{n}}\).%
    \mbox{}\hfill\mbox{(\citeyear[33]{Wright:2014tt})}
  \end{quote}

  Accepting and moving provides a \witness{}, and inferring is understood via the \witness{}.
\end{note}

\subsubsection{\textcite{Broome:2002aa}}

\begin{note}
  The same observation extends to \citeauthor{Broome:2002aa}'s (\citeyear{Broome:2013aa}) rule following account of (active) reasoning.

  \begin{quote}
    Active reasoning is a particular sort of process by which conscious premise-attitudes cause you to acquire a conclusion-attitude.
    The process is that you operate on the contents of your premise-attitudes following a rule, to construct the conclusion, which is the content of a new attitude of yours that you acquire in the process.\newline
    \mbox{ }\hfill\mbox{(\citeyear[234]{Broome:2002aa})}
  \end{quote}

  Understand relation of \support{} entailed by an event in which an agent concludes \(\pv{\phi}{v}\) from \(\Phi\) by~\supportI{} in terms of having followed a rule.
  However, if this entailment holds then the converse entailment, that the agent concluded by following the rules that gave rise to relation of \support{} does not hold.
\end{note}

\subsection{Non-causal}
\label{sec:non-causal}

\subsubsection{\textcite{Harman:1973ww}}

\begin{note}
  \begin{quote}
    Reasons may or may not be causes; but explanation by reasons is not causal or deterministic explanation.
    It describes the sequence of considerations that led to belief in a conclusion without supposing that the sequence was determined.%
    \mbox{ }\hfill\mbox{(\citeyear[52]{Harman:1973ww})}
  \end{quote}

  Sequence of considerations provides a \witness{}.
\end{note}

\subsubsection{\textcite{Hieronymi:2011aa}}

\begin{note}
  More broadly, though similar.
  \citeauthor{Hieronymi:2011aa}'s (\citeyear{Hieronymi:2011aa}) account of acting for reasons.

  \begin{quote}
    The proposal starts with this simple thought: whenever an agent acts for reasons, the agent, in some sense, takes certain considerations to settle the question of whether so to act, therein intends so to act, and executes that intention in action.

    If this much is uncontroversial (and, under some interpretation, I believe it must be), we can use it as a form for filling out.
    I propose, then, that we explain an event that is an action done for reasons by appealing to the fact that the agent took certain considerations to settle the question of whether to act in some way, therein intended so to act, and successfully executed that intention in action.
    I suggest that \emph{this} complex fact, \dots is the reason that rationalizes the action---that explains the action by giving the agent’s reason for acting.\newline
    \mbox{ }\hfill\mbox{(\citeyear[431]{Hieronymi:2011aa})}
  \end{quote}

  So, reason is the complex fact.
  Complex fact gives the reason the agent acted, and so content of constituent considerations from \agpe{}.

  In particular, note here that everything is directed at the question.
  Premise-conclusion relationship.

  As with \citeauthor{Harman:1973ww}, capture the trace, which is given by a \witness{}.
\end{note}

\subsection{Normative}

\subsection{\textcite{Lord:2018aa}}

\begin{note}[Responding to reasons]
  Consider the proposal at the core of \citeauthor{Lord:2018aa}'s (\citeyear{Lord:2018aa}) thesis that being rational is to correctly respond to reasons.

  \begin{quote}
    \textbf{Correctly Responding:} What it is for A's \(\phi\)-ing to be ex post rational is for A to possess sufficient reason S to \(\phi\) and for A's \(\phi\)-ing to be a manifestation of knowledge about how to use S as sufficient reason to \(\phi\).%
    \mbox{}\hfill\mbox{(\citeyear[143]{Lord:2018aa})}
  \end{quote}

  An agent's action is rational only if the action is a manifestation of some know-how.
  \citeauthor{Lord:2018aa} summaries:

  \begin{quote}
    \dots when one manifests one's know-how, dispositions that are directly sensitive to normative facts are manifesting. Thus, the competences involved in the relevant know-how make one directly sensitive to the normative facts%
    \mbox{}\hfill\mbox{(\citeyear[16]{Lord:2018aa})}
  \end{quote}

  For our purposes, following example of manifesting know-how directly relates to reasoning:

  \begin{quote}
    The most salient disposition [when appealing to \emph{p} as a reason]%
    \footnote{
      Note, \citeauthor{Lord:2018aa} (explicitly) not talking about believing that \emph{p} is a reason, but argues that the cited disposition to present both when appealing to p as a reason and believing that \emph{p} is a reason.
    }
    is the disposition to (competently) use \emph{p} as a premise in reasoning.\newline
    \mbox{}\hfill\mbox{(\citeyear[25]{Lord:2018aa})}
  \end{quote}

  Hence, suppose an agent concludes.
  Then, if the agent does not witness reasoning from pool of premises, it seems the agent does not manifest know-how, which is required for the appeal to meet \citeauthor{Lord:2018aa}'s account of rational action.

  Of course, that the noted disposition is the most salient does not rule out alternative, less noteworthy, dispositions.
  However, issues is \emph{manifesting} know-how without a \witness{}.
\end{note}

\begin{note}[Illustration]
  Clear that there is no manifestation of understanding of arithmetic.
\end{note}

\begin{note}
  Whether or not argument to be develop is of any difficulty turns on manifesting.
  In various cases, plausible that \ros{1} at issue would arise from the same disposition the agent manifests when the agent concludes.
\end{note}

\subsection{Embedding}
\label{sec:embedding}

\begin{note}
  So, in initial cases, plausible that constraint in terms of having a \witness{}.

  In other cases, embedding.
\end{note}

\begin{note}
  An initial borderline case is \citeauthor{Boghossian:2014aa}'s taking condition.
  Depending on how `taking' is understood, embedded within \ros{}.
  In particular, distinguished proposition-value pairs from attitudes, and hence this does not amount to reducing the taking condition to a doxastic condition.
\end{note}


\subsubsection{\textcite{Thomson:1965vv}}

\begin{note}
  \citeauthor{Thomson:1965vv} suggests a an account of reasoning such that an agent reasons from \(\phi\) to \(\psi\) just in case the agent believes that \(\phi\) is a reason for \(\psi\).
  \begin{quote}
    The claim which the 'formula' of p.\ 285\nolinebreak
    \footnote{
      The `formula' in question:
      \begin{quote}
        Now reasoning should surely involve drawing a conclusion from a set of premisses.
        But you can't be said to draw the conclusion that \emph{q} from \emph{p} if for all you know in knowing that \emph{p} it would at best be a matter of luck if \emph{q} as well.
        So to ``reason'' from \emph{p} by itself to \emph{q} isn't really to be reasoning; it's like saying one thing, and then taking a chance on it that something else is also true---like taking a leap in the dark, or more prosaically, like guessing.'
        (From here on I shall refer to this as the `\emph{formula}'.)\nolinebreak
        \mbox{}\hfill\mbox{(\citeyear[285]{Thomson:1965vv})}
      \end{quote}
    }
    above was to support was this:
    suppose \emph{p} does not imply \emph{q}, and suppose a man says `\emph{p}, so \emph{q}';
    then he is not reasoning in saying this unless he believes that \emph{r}, where the conjunction of \emph{p} and \emph{r} implies \emph{q}, and \emph{r} is a suppressed premiss of his reasoning.\par
    But suppose such a man believes that \emph{p} is reason for \emph{q}; would this not be enough?
    `It would if ``\emph{p} is reason for \emph{q}'' were construed as a suppressed premiss of his argument'.
    Then let us so construe it.\newline
    \mbox{}\hfill\mbox{(\citeyear[294]{Thomson:1965vv})}
  \end{quote}
  Causation is absent from \citeauthor{Thomson:1965vv}.
  Does not imply that \citeauthor{Thomson:1965vv}'s proposal is independent of causation, but motivated does not appeal to causation.
\end{note}

\subsubsection{\textcite{Longino:1978wv}}

\begin{note}
  \citeauthor{Longino:1978wv}'s (\citeyear{Longino:1978wv}) account of inferring seems explicit.
  \begin{quote}
    S infers at t that p from x if and only if
    \begin{enumerate}[label=\arabic*]
    \item
      S at t comes to believe that p, and
    \item
      S's epistemic reason for believing that p at t is x, i.e.,
      \begin{enumerate}[label=\alph*]
      \item
        S takes x to be evidence that p, and
      \item
        S's taking x to be evidence that p causes S to believe that p.\newline
        \mbox{}\hfill\mbox{(\citeyear[22]{Longino:1978wv})}
      \end{enumerate}
    \end{enumerate}
  \end{quote}
  Causation between mental states, but explanatory relation between things.
\end{note}

\subsection{\ros{3}?}

\begin{note}
  A recent account of reasoning given by \cite{Valaris:2014un} (\citeyear{Valaris:2014un}) is separate from \witness{} and embedding.%
  \footnote{
    For simplicity we ignore \citeauthor{Valaris:2014un}'s distinction between basic and non-basic instances of reasoning.
    The excerpts concern non-basic reasoning.
  }

    \begin{quote}
    Suppose that one believes \emph{R} and that \emph{p} follows from \emph{R}.
    What else might it take for one to count as believing \emph{p} by reasoning from \emph{R}?
    The crucial point here is that, if one believes both \emph{R} and that \emph{p} follows from \emph{R}, then --- barring inattention or irrationality --- one thereby believes \emph{p}.
    \dots
    In general, the relation between believing that one has conclusive evidence for a proposition and believing that proposition is constitutive, not merely causal.%
    \mbox{ }\hfill\mbox{(\citeyear[110 ]{Valaris:2014un})}
  \end{quote}

  Distinct from \witness{} because only interest is belief.
  Distinct from embedding because constitutive.
  In other words, the agent does not reason from their belief.
  Rather, reasoning is the belief.

  Inclined to understand in terms of \ros{}.
  For our purposes, role of \ros{} is to capture relationship between premises and conclusion.
  However, understood in a particular way, may amount to a belief.

  Still, not so straightforward.
  How does on get the belief that \emph{p} follows from \emph{R}?
  This, to my mind, is what is at issue.
  However, it seems that this is not how things are for \citeauthor{Valaris:2014un}.

  \begin{quote}
    [R]easoning just is believing that one’s conclusion follows from one’s premisses, and thereby believing one’s conclusion.%
    \mbox{ }\hfill\mbox{(\citeyear[112]{Valaris:2014un})}
  \end{quote}

  Unless belief is process, then it seems reasoning understood in this way is instantaneous.

  I am not sure what to make of this.
\end{note}


\section{Wrangling}
\label{cha:var:wrang}

\begin{note}
  \autoref{cha:introduction} introduced two questions, \qWhy{} and \qHow{}, and motivated a constraint between answers to \qWhy{} and \qHow{}.

  \autoref{sec:variants-initial} introduced variants of \qWhy{} and \qHow{}, and a variant constraint.

  \autoref{cha:clar:sec:literature}, in addition to intuition, constraint seems to often be a theoretical assumption.

  Purpose of variants is to motivate counterexamples to constraint.
  Specifically in terms of answers to \qWhyVnP{} which are not answers to \qHowV{}.
  In other words, \ros{} such that \ros{} explains, in part, why agent concludes but is such that the agent does not have a \witness{} for the \ros{}.

  In this section we outline in rough form how we will (attempt) to provide counterexamples.

  In short, need:
  An agent, event in which agent concludes \(\pv{\phi}{v}\) from \(\Phi\), and \ros{} between \(\pv{\psi}{v'}\) and \(\Psi\) such that:

  \begin{itemize}
  \item
    The agent does not have a \witness{} for the \ros{} between \(\pv{\psi}{v'}\) and \(\Psi\).
  \item
    The \ros{} between \(\pv{\psi}{v'}\) and \(\Psi\), in part, answers \qWhyVnP{}.
  \end{itemize}

  Our goal is motivate a general method for generating examples in which some \ros{} for which an agent does not have a \witness{} such that the \ros{} answers \qWhyVnP{}.
\end{note}

\subsection{\ros{3} without a \witness{}}

\begin{note}
  Immediate that an agent may not have a \witness{} for some \ros{}.

  Novel conclusions, as understood in this document, are common.
  Pair a proposition with some value.

  For example, enumerate all the tautologies of propositional logic.
  As \citeauthor{Harman:1973ww} notes, clutter, and there may be little point in deriving the tautologies.
  However, regardless of worth, it is not the case that have a \witness{} for most.

  Likewise, conclusions with respect to actions.
  For example, which particular style of coffee would like as the queue shortens and the time to place an order approaches.
\end{note}

\begin{note}
  More difficult, is \ros{}.

  As sketched in \autoref{cha:var:support}, idea of a \fc{}.
\end{note}

\begin{note}
  \fc{} is such that action such that after performing action, event in which concludes is in progress.
  In other worlds, the agent would be concluding by performing the action.

  Delicacy.
  \fc{1} and \ros{} from the \agpe{}.

  Two issues:

  \begin{enumerate}
  \item
    If independent of \agpe{}, then it may be the case that \fc{} without any prior recognition from agent.
  \item
    If dependent on \agpe{}, then it may be the case that not a \fc{}.
  \end{enumerate}

  Two issues are important.
  Without \agpe{}, then unclear that get \ros{} of interest for answer to \qWhyVnP{}.
  Given \agpe{}, unclear that we get a \ros{}.
\end{note}

\begin{note}
  Second issue is particularly pressing.
  For, explanations.
  \emph{Factive}.

  Get \ros{} by \fc{}.
  However, may not be the case that \fc{}.
\end{note}

\begin{note}
  Our strategy is to avoid both problems by focusing on cases in which the agent \emph{knows} that \(\pv{\psi}{v'}\) from \(\Psi\) is a \fc{}.

  Hence, suitable link to the \agpe{}, as the agent is aware that \fc{}.
  And, avoid failure of \fc{} by factivity of knowledge.

  In this respect the understanding of \fc{1} in terms of action such that concluding is important.
  With the exception of more-or-less instantaneous actions, future may develop in surprising ways.

  For example, plausible that an agent knows when they strike the cue ball in a certain way, a particular red ball will land in a pocket.
  However, not plausible that the agent knows where the cue ball will come to rest after the red ball lands in the pocket.
  Hence, agent does not know their following more, and so on.

  In parallel, an agent may have no guarantee that they will not be interrupted, etc.
  Hence, in most cases it seems implausible that an agent knows they will concluded.
  Yet, to be concluding does not require completion.

  With respect to \fc{}, whether event in which the agent concludes would be in progress.

  Instances in which an agent knows that concluding would be in progress exist.
  For example, consider basic arithmetic.
  Whether or not \(4131 + 1533 = 5664\) is a \fc{}.%
  \footnote{
    More generally, take any \(n\) and \(m\) such that the process is adding \(n\) and \(m\) would not take too long.
  }
  If you started, would be determining whether or not the equality holds.

  In most cases we will push a little further than addition.
  However, share the same pattern of the conclusion following from the application of some collection of rules which an agent knows.

  For example, enrich the collection of mathematical operations to include subtraction, multiplication, division, square-roots and so on.

  Beyond mathematics, but close, formal logic.
  In particular, theoretical results such tautologies of propositional logic, or meta-theoretical results which are generated from a common method, such as completeness proofs of various modal logics via canonical models.

  And, finally games.

  If you know the appropriate strategy, play first, and so desire, then it is a \fc{0} that any game of noughts and crosses will either end in a win for you or a draw.%
  \footnote{
    For details, see~(\cite[94--96]{Gardner:1983wn}).
  }

  Sudoku puzzles.
  Rules are simple, and I expect that if you have some experience with solving Sudoku puzzles, then you know that the solution to the Sudoku puzzle is a \fc{}.
  In the worst case scenario, you have the option to brute force the solution to the puzzle.

  A slightly more interesting example is chess problems.
  In particular, there is plausibly some bound where solution to a problem fails to be a \fc{}.
  However, any problem within the bound is a \fc{}.
  May not know where the bound is.
  Yet, solutions to some problems within the bound are know to be \fc{1}.
  For example, whether or not there is an available move for some piece is a \fc{1}, but whether there is a sequence of move that will result in checkmate for either player is often not (known) to be a \fc{}.
\end{note}

\begin{note}
  So, via \fc{1} obtain instances of a \ros{1} holding from an \agpe{} without the agent having a \witness{0} for the \ros{0}.

  Hence, candidate \ros{} that may be answers to \qWhyVnP{} such that the agent does not have a \witness{0} for the \ros{0}.

  However, in order for \ros{} to be an answer, in part, to \qWhyVnP{}, dependence.
\end{note}

\begin{note}
  This does not provide a complete solution to problem of factivity.
  For, what distinguishes one case from the other?

  However, this is nothing unique to cases under consideration, so long as relevant instances of \fc{} are plausibly knowledge.

  Though, this still differs from attitudes.
  Here is where embedding is interesting.
  If under constraint of \agpe{}, and \fc{}, then why not embed \fc{} as a premise.
\end{note}

\begin{note}
  Following, distinction between \ros{} answering, in part, \qWhyVnP{} and a \ros{} being embedded in some \ros{}.
  In particular, \ros{} between some \(\pv{\psi}{v'}\) and \(\Psi\) such that distinct from \ros{} between \(\pv{\phi}{v}\) and \(\Phi\) being embedded within the \ros{} between \(\pv{\phi}{v}\) and \(\Phi\) as a premise.
\end{note}

\begin{note}
  Rule out possibility of embedding proposed answers in this way.

  However, this will turn on the way in which dependence holds.
\end{note}

\subsection{Dependence}

\begin{note}
  Now, need it to be the case that if \ros{} failed to hold, then would not conclude.

  Suitable link between conclusion of \(\pv{\phi}{v}\) form \(\Phi\) and \ros{} between \(\pv{\psi}{v'}\) and \(\Psi\).

  \fc{3} serve an equally important role.

  For, if \fc{} then \ros{}.
  Conversely, if no \ros{} then not \fc{}.
  Hence, if fails to be \ros{}, then fails to be \fc{}.
  In turn, agent concludes only due to \fc{}, then suitable link.
\end{note}

\begin{note}
  Difficulty.
  How is it the case that the agent concludes only due to \fc{}?

  Method is to consider a variant of \qWhyVnP{}.
\end{note}

\begin{note}
  \begin{restatable}[\qWhyV{}]{question}{questionWhyV}
    \label{q:why:v}
    Given an agent \vAgent{}, proposition-value pair \(\pv{\phi}{v}\), pool of premises \(\Phi\), and event \(e\) in which \vAgent{} concludes \(\pv{\phi}{v}\) from \(\Phi\):

    \begin{quote}
      Which proposition-value-premises pairings \(\pvp{\psi}{v'}{\Psi}\) are such that, when \vAgent{} pairs \(\phi\) with \(v\):

      \begin{enumerate}[label=]
      \item
        \begin{enumerate}[label=\alph*., ref=(\alph*), series=qWhyVDef]
        \item
          \support{2} between \(\pv{\psi}{v'}\) and \(\Psi\) holds, from \agpe{\vAgent{}'}.
        \end{enumerate}
      \end{enumerate}

      And, from \agpe{\vAgent{}'}:

      \begin{enumerate}
      \item[\emph{If}:]
        \begin{enumerate}[label=\alph*., ref=(\alph*), resume*=qWhyVDef]
        \item
          \support{2} failed to hold between \(\pv{\psi}{v'}\) and \(\Psi\) when pairing \(\phi\) with \(v\)(, from \agpe{\vAgent{}'}).
        \end{enumerate}
      \item[\emph{Then}:]
        \begin{enumerate}[label=\alph*., ref=(\alph*), resume*=qWhyVDef]
        \item
          \(e\) would not have been an event in which \vAgent{} concluded \(\pv{\phi}{v}\) from \(\Phi\).
        \end{enumerate}
      \end{enumerate}
    \end{quote}
    \vspace{-\baselineskip}
  \end{restatable}

  \fc{3} provide an account of how get the link:

  \begin{itemize}
  \item
    \support{2} failed to hold between \(\pv{\psi}{v'}\) and \(\Psi\) when pairing \(\phi\) with \(v\)(, from \agpe{\vAgent{}'}).
  \item
    \(\pv{\psi}{v'}\) failed to be a \fc{} from \(\Psi\).
  \item
    \(e\) would not have been an event in which \vAgent{} concluded \(\pv{\phi}{v}\) from \(\Phi\).
  \end{itemize}

  Given that \ros{1} are something of an abstraction, interest is really in whether \fc{}.
\end{note}

\begin{note}
  The idea is:

  Counterexamples.

  One option is to specify (apparent) counterexamples so that the truth of the \emph{if-then} conditional follows.
  Then, at issue is whether analysis is correct.
  It may be that the truth of the conditional does not follow.

  Other option, specify (apparent) counterexamples so that the \emph{if-then} conditional holds from the \agpe{}.
  Then, at issue is whether the \agpe{} is correct.

  Trade a issue about analysis of counterexamples for a issue about what the counterexample achieves.

  Preference for the second option is ease of specifying examples.
  Build up an understanding of how and why such examples arise, and then try to figure out whether they result in anything substantial rather than attempting to analyse examples.

  On first, whether analysis is correct.

  On second, whether agent is correct.
\end{note}

\begin{note}
  So, dependence is captured from the \agpe{} and task is to construct \scen{1} such that this is the case.

  Two issues:
  \begin{enumerate}
  \item
    Is it the case that \scen{1} exists?
    Do\fc{1} ever matter in this way?
  \item
    Is the \agpe{} lead to truth of conditional from \qWhyVnP{}?
  \end{enumerate}

  We will not sketch \scen{1} here.
  Interest is with the relation between \agpe{} on conditional and conditional.
\end{note}

\subsubsection{Perspective}

\begin{note}
  Subdividing problem further, there are two concerns.

  \begin{enumerate}
  \item
    Is \agpe{} informative?
  \item
    Granting true from \agpe{}, is informative, may it still be the case that \ros{} fails to answer \qWhyVnP{}.
  \end{enumerate}

  Roughly, motivation to think \agpe{} tells us anything about what actually happens.
  And, even if it does, do things work out in as they do from the \agpe{}.

  Develop the concerns in a little more detail.
\end{note}


\paragraph{Perspective alone}

\begin{note}
  The \emph{if-then} conditional is characterised from the \agpe{}, in part, to allow for ease considering possible answers to \qWhyV{}.
  And, a significant portion of this document will focus on providing examples.

  However, as the \emph{if-then} conditional is characterised from the \agpe{} a problem arises.
  For, it may be the case that that \emph{if-then} conditional holds from the \agpe{} but does not hold independently of the \agpe{}.
  Hence, the sense of dependence captured by \qWhyV{} is not equivalent with the intuitive sense of dependence captured by considering whether or not the \emph{if-then} conditional holds independently of the \agpe{}.

  The observation that the \emph{if-then} conditional may hold from the \agpe{} while failing to hold independently of the \agpe{} is clearest when considering conditionals more general.

  For example, suppose an agent has taken a gamble on a coin landing heads.
  The coin lands heads, and the agent receives a prize.
  From the \agpe{}, if the coin failed to lands heads, then the agent would not have received the prize.
  However, the agent was set to receive the prize for participating in the gamble, regardless of whether the coin landed heads.%
  \footnote{
    The present point is similar to issues raised by \citeauthor{Harman:1973ww} (\citeyear{Harman:1973ww}) regarding the proposed equivalence between reasons for which an agent believes something with reasons the agent would offer if asked to justify their belief.
  As \citeauthor{Harman:1973ww} notes, an agent may offer reasons because they think they will convince their audience, not because the agent is compelled by the reasons, etc.
  (\citeyear[Ch.2]{Harman:1973ww})

  To the extent that \citeauthor{Harman:1973ww}'s point is that what holds from an \agpe{} need not actually be the case, the point in the same.
  However, to the extent that \citeauthor{Harman:1973ww} relies on an under-specification of what holds from an \agpe{} --- i.e.\ the distinction between whether \(\phi\) has value \(v\) from the \agpe{} or whether the agent evaluates as true the proposition that their audience is responsive to \(\phi\) having value \(v\), the point is distinct.
  }

  So, switching back to \qWhyV{}, it may be the case that, though from the \agpe{} they would not have concluded \(\pv{\phi}{v}\) from \(\Phi\) if \support{} failed to hold between \(\pv{\psi}{v'}\) and \(\Psi\), the agent would have concluded \(\pv{\phi}{v}\) from \(\Phi\) regardless.
\end{note}


\begin{note}
  In order for \qWhyV{} to be of interest, must be cases where the \agpe{} is correct.

  \begin{proposition}
    \label{prop:why-n-p-link}
    Instances where \(\pvp{\psi}{v'}{\Psi}\) answers \qWhyVnP{} in virtue of answering \qWhyV{}
  \end{proposition}

  Find instances which witness the truth of \autoref{prop:why-n-p-link}.

  Stress, \autoref{prop:why-n-p-link} does not amount to an entailment.
  They may be cases where \agpe{} returns an answer to \qWhyV{} which is not an answer to \qWhyVnP{}.
\end{note}

\begin{note}
  So, the only thing to do is ensure the \agpe{} is correct.

  However, motivation in similar style to \fc{1}.
  \fc{3} focus on whether agent would conclude.
  Pair is knowing that the agent would not conclude.
  So, in parallel fashion, knowledge.
\end{note}

\paragraph{Quarantine}

\begin{note}
  So, embedding.

  Basically, concern about not being a \fc{}.
  Then, adding premise that is a \fc{} doesn't do any work.
\end{note}

\begin{note}
  Still, this isn't quite enough.
  What about some other explanation.

  If there is some general method to quarantine, then avoid worries about the link breaking.
  Deny the proposition holds.
  Separate \agpe{} from what happens.

  However, if no general method to quarantine.
  Then, more difficult.
  Something problematic about agent.

  Whether break the \agpe{} from what matters.

  Task: \fc{} matters from \agpe{}, but does not matter apart.
  Argue that this is not possible.
  To do this, relevant \fc{} does not embed.

  Does not function as an attitude.
\end{note}

\paragraph{Unsorted}

\begin{note}
  Start with \citeauthor{Davidson:1963aa}.
  Observe, attitudes, rather than the contents of attitudes.
  Here, this relates to \citeauthor{Dancy:2000aa} and \citeauthor{Collins:1997wn}.
  Whether the attitude explains or whether the contents of the attitude explain.

  However, \citeauthor{Davidson:1963aa} is pushed further, whether the attitude explains in the right way.
  Deviant causal chains.
\end{note}

\begin{note}
  Now, our interest is from the \agpe{}.
  Hence, we are not looking for anything beyond the agent's attitude, following \citeauthor{Davidson:1963aa}.

  However, in contrast to \citeauthor{Davidson:1963aa}, \ros{} does not reduce to an attitude.
  Connexion between \(\pv{\phi}{v}\) and \(\Psi\) captured by \ros{} parallels the causal relation \citeauthor{Davidson:1963aa} identifies between reason and action.

  Paraphrase \citeauthor{Davidson:1963aa} in our account by saying the \ros{} holds between content of reason and action.

  Therefore, have a middle ground.
  On the one hand, not looking for \ros{1} to explain --- \ros{1} are only of interest from an \agpe{}.
  On the other hand, \ros{} without \witness{} --- it must still be the case that \ros{} matters, even though the agent has not concluded \(\pv{\psi}{v'}\) from \(\Psi\).
\end{note}

\begin{note}
  As a middle ground, inherit the problematic aspect of \citeauthor{Dancy:2000aa}.
  It may be the case that the \emph{if-then} conditional fails.
  And, may be tempted to generalise in the style of \citeauthor{Davidson:1963aa}.
  Agent's perspective that the \emph{if-then} conditional holds, and in turn this is viewed as a premise.
  As we have seen above, as a premise, of no direct interest.
\end{note}

\begin{note}
  So, look.
  The thing is that in contrast to \citeauthor{Dancy:2000aa}, I am not looking for an entailment.

  Instead, the point is that \agpe{} provides an understanding of how dependence holds, and it turns out that the \agpe{} is correct.

  If so, the issue is motivating this and avoiding embedding.
\end{note}

\begin{note}
  Therefore, require a \witness{} in order to ensure that the \ros{} exists.
\end{note}

\begin{note}
  Factive and perspective is correct.
  If this is the case, then need a method of reducing to something factive.
  The only plausible way to reduce to something factive is to embed within a \ros{}.
  No embedding in relations of support.
  So, either reject factive or reject perspective.

  This is interesting, because typically the case that motivate factive by observing that it preserves the role of the \agpe{}.
\end{note}

\subsection{\qWhyVnP{} and \qWhyV{}}
\label{cha:var:expand:qWhy:variant}

\subsubsection{The \ros{} between \(\pv{\phi}{v}\) and \(\Phi\)}

\begin{note}
  Before turning to dependence, let us briefly observe that the \ros{} between \(\pv{\phi}{v}\) and \(\Phi\) should always, intuitively, be, in part, an answer to \qWhyV{}.

  For, when agent pairs \(\phi\) with \(v\), then by \supportI{}, it is the case that \ros{} between \(\pv{\phi}{v}\) and \(\Phi\) holds from the \agpe{}.
  Hence, \ref{q:why:v:a} must be true.

  Likewise, immediately the case that if \support{} failed to hold, then the agent would not have concluded \(\pv{\phi}{v}\) from \(\Phi\).
  For, by the same reasoning, when the agent pairs \(\phi\) with \(v\), then by \supportI{}, it is the case that \ros{} between \(\pv{\phi}{v}\) and \(\Phi\) holds from the \agpe{}.
  Therefore, if \support{} fails to hold between \(\pv{\phi}{v}\) and \(\Phi\), then \(e\) is not an event in which the agent concludes \(\pv{\phi}{v}\) from \(\Phi\).

  Now, the conditional when set aside from the \agpe{} may be true, it need not be the case that the conditional is true from the \agpe{}.
  However, we are working at some level of abstraction, and hence we assume the agent recognises the truth of the \emph{if-then} conditional.
  Indeed, we have merely expressed in artificial terms a truism:
  The agent would not have concluded \(\pv{\phi}{v}\) from \(\Phi\) if the agent had failed to conclude \(\pv{\phi}{v}\) from \(\Phi\).
\end{note}

\begin{note}
  With the core case of \support{} between \(\pv{\phi}{v}\) and \(\Phi\) being an answer to \qWhyV{} in hand, we turn to an in depth discussion of the kind of dependence captured by the \emph{if-then} conditional of \qWhyV{}.

  Our task is to balance a form of dependence that \emph{may} lead to counterexamples to \issueInclusion{} with an account of dependence that is compatible with the intuition that motivates \issueInclusion{}.
  In other words, the dependence should be such that answers to \qWhyV{} are constrained by the condition that the agent has a \witness{} for the relevant \ros{}.
\end{note}

\subsection{Difficulties}
\label{sec:qwhyv-subs-paragr}

\begin{note}
  Might be suspect, given voluntarism about concluding.
  But, this, I think, is a mistake.
  There's no suggestion that agent may choose whether or not to conclude in these types of cases.
  No choice over whether conditional is true.
\end{note}

\subsection{\citeauthor{Owens:2006tw}}

\begin{note}
  For example, \citeauthor{Owens:2006tw} argues for a belief expression model of assertion in which the rationality of a belief formed by an agent via testimony is connected to justification of the testifier:

  \begin{quote}
    Trusting an expression of belief by accepting what a speaker says involves entering a state of mind which gets its rationality from the rationality of the belief expressed.
    This state's rationality depends on the speaker's justification for the belief he expresses, not on his justification for the action of expressing it.
    And to hear a speaker as making a sincere assertion, as expressing a belief, is \emph{ceteris paribus} to feel able to tap into \emph{that} justification (whether or not his assertion was directed at you) by accepting what he says.%
    \mbox{}\hfill\mbox{(\citeyear[123]{Owens:2006tw})}
  \end{quote}

  On the view advanced by \citeauthor{Owens:2006tw}, justification.
  View in terms of \support{}.

  \support{} directly.
  Rationality of agent is rationality of speaker.

  However, `depends'.

  Distinction between rationality of state, and relation between rationality of state and rationality of state.

  Inclined to think \citeauthor{Owens:2006tw} is arguing for the former.%
  \footnote{
    \begin{quote}
      If we are to believe what the speaker indicates he believes, either the speaker must justify this belief to us, or we must supply some justification of our own
      \dots
      Neither act can be part of a rationality preserving mechanism for belief.%
      \mbox{ }\hfill\mbox{(\citeyear[123--124]{Owens:2006tw})}
    \end{quote}
  }
  Though, it is not clear to me that embedded isn't a viable option.

  Regardless, distinction that is important.
\end{note}

\begin{note}
  Same distinction holds for answers to \qWhyV{}.

  It may be the case that \support{} between \(\pv{\psi}{v'}\) and \(\Phi\) is, from the \agpe{}, involved in concluding \(\pv{\phi}{v}\) from \(\Phi\).

  However, no immediate move from this to \support{} being, in part, an answer to \qWhyV{}.
\end{note}

\subsection{Summary}
\label{cha:var:expand:issue:summary}

\begin{note}
  Three key things.

  Support.
  Witnessing.
  Issue.
\end{note}

\begin{note}
  Focus on \issueConstraint{}.
  \vspace{-\baselineskip}
  \begin{quote}
    \rIssueConstraint*
  \end{quote}
  Sufficient clarity on both `why?' and `how?'.
  Link we have argued for.
  And, further, independently of argument, it seems to me that a positive resolution to \issueConstraint{} is equally compelling as positive resolution to \issueInclusion{}.
\end{note}

\begin{note}
  This is the important thing, and in this respect it doesn't matter whether past or present.
  Whether a \support{} holds only if witnessed.
  Whether resolution to \qWhy{} only if the agent has witnessed.
\end{note}

\begin{note}
  Difficulty with all of this is that the accounts seem to be consistent, but do not explicitly motivate this constraint.
\end{note}

%%% Local Variables:
%%% mode: latex
%%% TeX-master: "master"
%%% End:
