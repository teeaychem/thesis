\chapter{\qWhyV{} and \qHowV{}}
\label{cha:var}


\begin{note}
  We have now detailed the way in which we understand conclusions (\autoref{cha:clar}) and \ros{} (\autoref{cha:ros}).
  The present chapter returns to \qWhy{} and \qHow{}, and \issueInclusion{} (\autoref{cha:intro}).

  Nominally, this chapter states a sufficient condition for answer to \qWhy{}, a necessary condition for certain answers to \qHow{}.
  Though, the function of this chapter is to summarise and simplify.

  At the close of the chapter variant questions to \qWhy{} and \qHow{} will have been stated (\qWhyV{} and \qHowV{}, respectively) and a our attention will turn to counterexamples to a constraints between the variant questions which parallels \issueInclusion{} (\issueConstraint{}).
\end{note}

\section{\qWhyV{}}
\label{cha:var:qwhyvnp}


\subsection{A sufficient condition}
\label{sec:sufficient-condition}

\begin{note}
  \begin{proposition}[\ros{3} and progressive explanation]%
    \label{obs:qWhyEIP}%
    Given \(\ed{}\) is an event in which \vAgent{} concludes \(\pv{\phi}{v}\) from \(\Phi\):

    \begin{itenum}
    \item[\emph{If}:]
      There is some \se{} \(\ed{\flat}\) of \(\ed{}\) such that conditions \ref{obs:qWhyEIP:prog} and \ref{obs:qWhyEIP:ros} hold:
      \begin{enumerate}[label=\arabic*., ref=\arabic*]
      \item
        \label{obs:qWhyEIP:prog}
        \(\ed{\flat}\) is such that \(\ed{}\) is in progress.
      \item
        \label{obs:qWhyEIP:ros}
        \(\ed{\flat}\) is such that \(\ed{}\) is in progress \emph{only if} a \ros{} between \(\pv{\psi}{v'}\) and \(\Psi\) holds throughout \(\ed{\flat}\).
      \end{enumerate}
    \item[\emph{Then:}]
      The \ros{} between \(\pv{\psi}{v'}\) and \(\Psi\) answers \qWhy{}.
    \end{itenum}
    \vspace{-2\baselineskip}
  \end{proposition}

  \begin{argument}{obs:qWhyEIP}
    As we have clarified the way in which we understand conclusions and \ros{}, \autoref{obs:qWhyEIP} is \autoref{sketch:PE:cROS} (\autopageref{sketch:PE:cROS}), re-expressed as stating a sufficient condition.

    In other words, \autoref{obs:qWhyEIP} is an instance of \progEx{}.
  \end{argument}

  \noindent%
  In short, \autoref{obs:qWhyEIP} ensures a \ros{} between \(\pv{\psi}{v'}\) and \(\Psi\) is sufficiently entwined in \(d^{\flat}\) so that it is not possible to characterise \(\ed{\flat}\) as an event in which concluding without the a \ros{} between \(\pv{\psi}{v'}\) and \(\Psi\) holding from the \agpe{}.
\end{note}


\begin{note}
  \qWhyV{} seeks \ros{} which satisfy the antecedent of \autoref{obs:qWhyEIP}:

  \begin{question}{questionWhyV}{\qWhyV{}}%
    Given \(\ed{}\) is an event in which \vAgent{} concludes \(\pv{\phi}{v}\) from \(\Phi\):

    \begin{itemize}
    \item
      Which \ros{1} are such that:
      \begin{itemize}
      \item
        For some \se{0} \(\ed{\flat}\) of \(\ed{}\) such that \(\ed{}\) is in progress:
        \begin{itenum}
        \item[\emph{If}:]
          A \ros{0} between \(\pv{\psi}{v'}\) and \(\Psi\) fails to hold from \agpe{\vAgent{}'} through \(\ed{\flat}\).
        \item[\emph{Then}:]
          \(\ed{\flat}\) is not an event such that \(\ed{}\) is in progress.
        \end{itenum}
      \end{itemize}
    \end{itemize}
    \vspace{-1.5\baselineskip}
  \end{question}

  \noindent%
  Hence, the link between \qWhy{} and \qWhyV{} holds:

  \begin{link}[\qWhyV{} and \qWhy{}]%
    \label{link:why:support:pvpp}%
    \vspace{-\baselineskip}
    \begin{itenum}
    \item[\emph{If}:]
      A \ros{} between \(\pv{\psi}{v'}\) and \(\Psi\) is an answer to \qWhyV{}.
    \item[\emph{Then}:]
      The \ros{} between \(\pv{\psi}{v'}\) and \(\Psi\) is an answer to \qWhy{}.
    \end{itenum}
    \vspace{-\baselineskip}
  \end{link}
\end{note}


\paragraph{An illustration}
% \label{sec:an-illustration}

\begin{note}
  Return to \autoref{scen:countS}:

  \reScenario{scen:countS}

  \noindent%
  In the event described by \autoref{scen:countS} Fox concludes \propI{The person Fox is talking to is a fellow member of the Democratic Underground Party} is \valI{True}.
  Pritcher calmly tells Fox the are a fellow member of the party at the end of the \scen{}.
  Still, Fox has already drawn the conclusion when Fox asks `Who are you?' via the sequence of countersigning.
  And, intuitively, Pritcher saying, No earlier than last year' in response to Fox saying `Miran is early this year' explains why the event develop in an event in which Fox concludes Pritcher is a fellow member of the party.

  Consider the \se{0} \(\ed{\flat}\) from Pritcher's response to `Miran is early this year' with `No earlier than last year' and Fox asking `Who are you?'.
  The following conditional holds, from \agpe{Fox's}:
  %
  \begin{itenum}
  \item[\emph{If}:]
    They didn't appropriately respond to `Miran is early this year'.
  \item[\emph{Then}:]
    They're not engaging in countersign.
  \end{itenum}
  %
  And, if Fox doesn't think Pritcher is engaging in countersign, it seems clear Fox does not conclude Pritcher a fellow member of the party.
  Hence, phrased in terms of \ros{}, we have:
  \begin{itenum}
  \item[\emph{If}:]
    A \ros{} between \pv{\propI{What they said was an appropriate response}}{\valI{True}} and some \pool{} \(\Psi\) fails to hold from \agpe{Fox's} through \(\ed{\flat}\).
  \item[\emph{Then}:]
    \(\ed{\flat}\) is not, or does not develop into an event in which Fox concludes \propI{The person Fox is talking to is a fellow member of the Democratic Underground Party} is \valI{True} from some \pool{} \(\Phi\).
  \end{itenum}
  In this respect, \ros{} answers \qWhy{}.
\end{note}

% \footnote{
%     \nocite{Tichy:1976tp}%
%     The \itc{} of \qWhyV{} captures the presence of a lawlike constraint, as it applies regardless of whether or not the agent goes on to conclude \(\pv{\phi}{v}\) from \(\Phi\).

%     In the literature on subjunctive conditionals, the conditionals which constrain the development of events are sometimes termed `laws' (\cite{Chisholm:1955aa,Lewis:1979vm,Veltman:2005tj}) or thing that `lump together' certain facts (\cite{Kratzer:1981aa,Kratzer:1989aa}).
%   }

\begin{note}
  To close this section we make a brief observation:

  \begin{observation}%
    \label{obs:qWhyV:rosNNecAns}%
    A \ros{} between \(\pv{\psi}{v'}\) and \(\Psi\) may answer \qWhyV{} though the \ros{} is unrelated to why \(\pv{\phi}{v}\) follows from \(\Phi\) from the \agpe{\agents{}}.
  \end{observation}

  \begin{motivation}{obs:qWhyV:rosNNecAns}
    Consider a \scen{0} in which an agent concludes \(\pv{\phi}{v}\) from \(\Phi\) where partway through the agent concludes to brew a cup of tea.

    It may well be the case that the agent fails to conclude \(\pv{\phi}{v}\) without the aid of the tea.
    For, without a warm cup of reassurance, the agent gives up.
    and, if so, a \ros{} between \pv{\propI{Brew a cup of tea}}{\valI{Do}} and some \pool{} \(\Psi\) answers \qWhyV{}.

    Still, from the \agpe{\agents{}}, the cup of tea was only a nice thing to have.
  \end{motivation}

  \noindent%
  Intuitively it is not be the case that the \ros{} and the conclusion to get some tea answers why \emph{the agent} concluded some theorem is true from some \pool{}.%
    \footnote{
      \citeauthor{Armstrong:1968vh} (\citeyear[195--196]{Armstrong:1968vh}) discusses a similar example, and suggests further discussion of this issue may also be found in \textcite{Moore:1962up}.
      See also \citeauthor{Sanford:1989aa} (\citeyear{Sanford:1989aa}) for a discussion of dependence as captured by subjunctive conditions (esp.\ pp.\ 192--193).
    }

    Still, \qWhy{} and \qWhyV{} concern the way in which an event in which an agent concludes develops, and various \ros{} which hold during an event when an agent concludes may answer this.

    Note, however, it must be the case that a \ros{} is required to hold in order for a \se{} to be a event in which the agent is concluding.
    Hence, a \ros{} between \pv{\propI{Brew a cup of coffee}}{\valI{Do}} and some \pool{} \(\Psi'\) does not answer \qWhyV{} as the agent concluded to get tea, rather than coffee (and even if the choice between tea and coffee was a figurative coin flip).
    And, a \ros{} between \pv{\propI{Brew a cup of tea}}{\valI{Do}} and \(\Psi\) fails to answer \qWhyV{} if the agent may have concluded \(\pv{\phi}{v}\) from \(\Phi\) without the aid of the cup of tea.
\end{note}



\section{\qHowV{}}
\label{cha:var:qhowv}


\begin{note}
  In parallel to the sufficient condition for answers to \qWhy{}, we observe a necessary condition for certain answers to \qHow{}, and the necessary condition allows us to state a variant of \qHow{} without reference to `how'.
  In perpendicular to sufficient condition for answers to \qWhy{}, the necessary condition for certain answers to \qHow{} is more-or-less immediate.
  The role of the condition and variant question is to help formulate a nice variant of \issueInclusion{}.
\end{note}

% \subsection{Necessary condition}
% \label{sec:necessary-condition}

% \begin{note}
%   The necessary condition for certain answers to \qHow{} is as follows:

%   \begin{condition}[\wit{3}]%
%     \label{cond:nH}%
%     For any event \(\ed{}\):
%     \begin{itenum}
%     \item[\emph{If}:]
%       Both \ref{cond:nH:aH} and \ref{cond:nH:cPVP} are true:
%       \begin{enumerate}[label=\arabic*., ref=\arabic*]
%       \item
%         \label{cond:nH:aH}
%         \(\ed{}\) answers \qHow{}.
%       \item
%         \label{cond:nH:cPVP}
%         \(\ed{}\) is an event in which \vAgent{} concludes \(\pv{\psi}{v'}\) from \(\Psi\).
%       \end{enumerate}
%     \item[\emph{Then}:]
%       \(\ed{}\) is a \wit{} for a \ros{} between \(\pv{\psi}{v'}\) and \(\Psi\).
%     \end{itenum}
%     \vspace{-\baselineskip}
%   \end{condition}
% \end{note}

\begin{note}
  The variant of \qHow{} is as follows:

  \begin{question}{questionHowV}{\qHowV{}}%
    \label{q:how:v}%
    Given \(\ed{}\) is an event which \vAgent{} concludes \(\pv{\phi}{v}\) from \(\Phi\):
    \begin{itemize}
    \item
      Which events \(\ed{\prime}\) are such that both \ref{q:how:v:bos} and \ref{q:how:v:wit} hold:
      \begin{enumerate}[label=\Alph*., ref=\Alph*]
      \item
        \label{q:how:v:bos}
        \(\ed{\prime}\) happens before or at the same time as \(\ed{}\).
      \item
        \label{q:how:v:wit}
        \(\ed{\prime}\) is a \wit{0} for a \ros{} between \(\pv{\psi}{v'}\) and \(\Psi\).
      \end{enumerate}
    \end{itemize}
    \vspace{-\baselineskip}
  \end{question}

  \noindent%
  In short, \qHowV{} asks which \ros{1} an agent has a \wit{} for when the agent concludes \(\pv{\phi}{v}\) from \(\Phi\).
\end{note}

\begin{note}
  \linkH{} connects answers to \qHow{} to answers to \qHowV{}:

  \begin{link}[\qHowV{} and \qHow{}]%
    \label{link:how-witnessing}%
    \vspace{-\baselineskip}
    \begin{itenum}
    \item[\emph{If}:]
      Both \ref{link:how-witnessing:a:1} and \ref{link:how-witnessing:a:2} are true:
      \begin{enumerate}[label=\arabic*., ref=\arabic*]
      \item
        \label{link:how-witnessing:a:1}
        \(\ed{\prime}\) answers \qHow{}.
      \item
        \label{link:how-witnessing:a:2}
        \(\ed{\prime}\) is an event in which an agent concludes \(\pv{\psi}{v'}\) from \(\Psi\).
      \end{enumerate}
    \item[\emph{Then}:]
      \(\ed{\prime}\) answers \qHowV{}.
    \end{itenum}
    \vspace{-\baselineskip}
  \end{link}

  \begin{argument}{link:how-witnessing}
    Suppose both \ref{link:how-witnessing:a:1} and \ref{link:how-witnessing:a:2} are true.

    From Condition~\ref{link:how-witnessing:a:1}, \(\ed{\prime}\) answers \qHow{}.
    Hence, \(\ed{\prime}\) happens before or at the same time as \(\ed{}\).
    And, from Condition~\ref{link:how-witnessing:a:2}, \(\ed{\prime}\) is an event in which an agent concludes \(\pv{\psi}{v'}\) from \(\Psi\).
    Therefore, \(\ed{\sharp}\) is a \wit{0} for a \ros{} between \(\pv{\psi}{v'}\) and \(\Psi\) by the definition of a \wit{}.
    So, \(\ed{\prime}\) answers \qHowV{}.
  \end{argument}
\end{note}



\section{\issueConstraint{}}
\label{cha:var:issue}

\begin{note}
  A constraint between \qWhyV{} and \qHowV{} follows from \linkW{}, \linkH{}, and \issueInclusion{}:

  \begin{proposition}[\qWhyV{}-\qWhy{}-\qHow{}-\qHowV{}]%
    \label{prop:support-and-witnessing}%
      \linkW{}, \linkH{}, and \issueInclusion{} (jointly) entail the following conditional:
      \begin{itenum}
      \item[\emph{If}:]
        A \ros{} between \(\pv{\psi}{v'}\) and \(\Psi\) is an answer to \qWhyV{}.
      \item[\emph{Then}:]
        An event \(\ed{\prime}\) which \wit[es]{0} a \ros{} between \(\pv{\psi}{v'}\) and \(\Psi\) answers \qHowV{}.
      \end{itenum}
    \vspace{-\baselineskip}
  \end{proposition}

  \begin{argument}{prop:support-and-witnessing}
    Suppose \qWhyV{} is answered, in part, by a \ros{} between \(\pv{\psi}{v'}\) and \(\Psi\).
    Then:

    By~\linkW{} the \ros{} between \(\pv{\psi}{v'}\) and \(\Psi\) answers \qWhy{}.
    So, by \issueInclusion{}, an event in which the agent concludes \(\pv{\psi}{v'}\) from \(\Psi\) answers \qHow{}.
    And finally, by \linkH{}, the event in which the agent concludes \(\pv{\psi}{v'}\) from \(\Psi\) answers \qHowV{}.
  \end{argument}
\end{note}


\begin{note}
  \color{blue}
  Intuitively, given an event in which an agent concludes \(\pv{\phi}{v}\) from \(\Phi\) then a \ros{} between \(\pv{\phi}{v}\) and \(\Psi\) answers \qWhy{} and the event in which the agent concludes \(\pv{\phi}{v}\) from \(\Phi\) answers \qHow{}.
  Hence, it should be the case that a \ros{} between \(\pv{\phi}{v}\) and \(\Psi\) answers \qWhyV{} and the event in which the agent concludes \(\pv{\phi}{v}\) from \(\Phi\) answers \qHowV{}.
  This is the case:

  \begin{proposition}[Sanity check]%
    \label{prop:vSantity}%
    \vspace{-\baselineskip}
    \begin{itenum}
      \item[\emph{If}:]
        \(\ed{}\) is an event in which an agent concludes \(\pv{\phi}{v}\) from \(\Phi\) as a result of concluding \(\pv{\phi}{v}\) from \(\Phi\) where \fc{}.
      \item[\emph{Then}]
        Conditions \ref{prop:vSantity:why} and \ref{prop:vSantity:how} both hold:
      \begin{enumerate}[label=\arabic*., ref=\arabic*]
      \item
        \label{prop:vSantity:why}%
        A \ros{} between \(\pv{\phi}{v}\) from \(\Phi\) answers \qWhyV{}.
      \item
        \label{prop:vSantity:how}%
        An event in which the agent concludes \(\pv{\phi}{v}\) from \(\Phi\) is always an answer to \qHowV{}.
      \end{enumerate}
    \end{itenum}
    \vspace{-\baselineskip}
  \end{proposition}

  \begin{argument}{prop:vSantity}
    \(\ed{}\) is an event in which an agent concludes \(\pv{\phi}{v}\) from \(\Phi\) as a result of concluding \(\pv{\phi}{v}\) from \(\Phi\).

    As the agent concludes \(\pv{\phi}{v}\) from \(\Phi\) concludes \(\pv{\phi}{v}\) from \(\Phi\) as a result of concluding \(\pv{\phi}{v}\) from \(\Phi\) there is an event in which the agent is concluding \(\pv{\phi}{v}\) from \(\Phi\).

    Take \(\ed{\flat}\) as the event in which the agent is concluding \(\pv{\phi}{v}\) from \(\Phi\).

    \(\ed{\flat}\) is a \se{} of \(\ed{}\).
    For, it is clear that \(\ed{}\) is in progress (conclusion when).
    And, \(\ed{}\) happens as a result of \(\ed{\flat}\).

    Further, \(\ed{\flat}\) is in progress \emph{only if} a \ros{} between \(\pv{\phi}{v}\) and \(\Phi\) holds throughout \(\ed{\flat}\).
    For, by \supportII{} a \ros{} between \(\pv{\phi}{v}\) and \(\Phi\) must hold if \(\ed{\flat}\) is an event in which the agent is concluding \(\pv{\phi}{v}\) from \(\Phi\).

    Hence, the \ros{} between \(\pv{\phi}{v}\) and \(\Phi\) answers \qWhyV{}.

    \medskip

    Now, \(\ed{}\) is an event in which the agent concludes \(\pv{\phi}{v}\) from \(\Phi\).
    Hence, \wit{} for the \ros{} between \(\pv{\phi}{v}\) and \(\Phi\) answers \qWhyV{}.
  \end{argument}
\end{note}


\begin{note}
  \autoref{prop:support-and-witnessing} constrains answers to \qWhyV{} in terms of answers to \qHowV{} and these variant questions pass a basic sanity check.
  Hence, we reformulate \autoref{prop:support-and-witnessing} as a constraint that parallels \issueInclusion{}:

  \begin{constraint}{consConstraint}{\issueConstraint{}}
    \vspace{-\baselineskip}
    \begin{itenum}
    \item[\emph{If}:]
      A \ros{} between \(\pv{\psi}{v'}\) and \(\Psi\) answers \qWhyV{}.
    \item[\emph{Then}:]
      \vAgent{} has a \wit{} for the \ros{} between \(\pv{\psi}{v'}\) and \(\Psi\) when \vAgent{} concludes \(\pv{\phi}{v}\) from \(\Phi\).
    \end{itenum}
    \vspace{-\baselineskip}
  \end{constraint}

  \noindent%
  The remainder of the document focuses on (counterexamples to) \issueConstraint{}.
  For, if there are counterexamples to \issueConstraint{}, then by \autoref{prop:support-and-witnessing} either \linkW{}, \linkH{}, or \issueInclusion{} fails to hold.

  I take \linkH{} to be trivial, and \linkW{} to be sufficiently well-motivated by \progEx{}.
  Hence, it seems if there are counterexamples then \issueInclusion{} fails to hold.
\end{note}




\section*{Summary}

\begin{note}
  If you like, you may now forget about \qWhy{} and \qHow{} and \issueInclusion{}.
\end{note}


% \section{The role of variant questions}
% \label{cha:var:sec:wiggling}

  % \begin{observation}
  %   \label{obs:qWhyV-description}
  %   Some \ros{1} matter.
  % \end{observation}
  % \begin{motivation}{obs:qWhyV-description}
  %   Situation.
  %   Flip coin.
  %   If heads, then by understanding of arithmetic.
  %   If tails, then by calculator.

  %   Now, coin lands heads.
  %   Well, sure, it seems this is needed.
  %   However, suppose coin lands tails.
  %   Then the agent would decide otherwise.
  %   So, really, the coin flip didn't matter.

  %   The point is that any event, and the description didn't rule out reversing.

  %   Does this matter?
  %   Not really, expand description.
  %   For any problematic event which differs from the way things are, rule out with description.
  % \end{motivation}

  % \footnote{
  %   \phantlabel{fn:past-witness}
  %   To illustrate, consider an agent working on some mathematical problem.

  %   As part of their work on the problem the agent concludes the hypotenuse of some right-angled triangle is \(\sqrt{74}\text{cm}\) by use of the Pythagorean theorem.
  %   Further, the agent has, at some point in the past proved the Pythagorean theorem from more basic principles.

  %   Perhaps the agent concludes the hypotenuse of the triangle is \(\sqrt{74}\text{cm}\), in part, from those more basic principles.
  %   Perhaps in general it is true that if a agent concluded \(X\) from \(Y\) and \(Y\) from \(Z\), then the agent, in part at least, concluded \(X\) from \(Z\).
  %   On the other hand perhaps the more basic principles have no role explanatory role in the present.
  %   The agent only appealed to the theorem, rather than any basic principles.

  %   Though we will not take a stand on whether a relevant \wit{0} for some conclusion is distinct from the event in which the agent concludes, the first option highlights an issue with \autoref{def:witnessing}.
  %   Consider an agent working through a proof of some theorem \(\theta\), and let \(\Theta\) be the relevant \pool{}.
  %   Our interest is with the conclusion \(\pv{\theta}{\valI{True}}\) from \(\Theta\).

  %   Suppose the agent reasons to \(\pv{\theta}{\valI{True}}\) from \(\Theta\).
  %   Further, suppose the \agents{} reasoning is sound.
  %   However, the agent is worried about some parts of their reasoning.
  %   Hence, given their worries, \emph{reasons} to --- but does not conclude --- \(\pv{\theta}{\valI{True}}\) from \(\Theta\).
  %   Some time later the agent resolves their worries and concludes the theorem is true.
  %   I see no issue with the possibility that:
  %   %
  %   \begin{itemize}[noitemsep]
  %   \item
  %     When the agent revisited the proof, they concluded \(\pv{\theta}{\valI{True}}\) from \(\Theta\).
  %   \item
  %     In part, a \ros{} between \(\pv{\theta}{\valI{True}}\) and \(\Theta\), from the \agpe{}, answers why the agent concluded \(\pv{\theta}{\valI{True}}\) from \(\Theta\).
  %   \item
  %     The event in which the agent reasoned to \(\pv{\theta}{\valI{True}}\) from \(\Theta\) answers, in part, how the agent \(\pv{\theta}{\valI{True}}\) from \(\Theta\) by being a \wit{} for the \ros{} between \(\pv{\theta}{\valI{True}}\) and \(\Theta\).
  %   \end{itemize}
  %   %
  %   However, the idea is incompatible with the way a \wit{0} is understood.
  %   For, by the definition, an event which is a \wit{0} must be an event in which the agent \emph{concludes}.
  %   And, by construction of the \scen{0}, the \agents{} worries prevent the agent from forming the relevant conclusion.

  %   Perhaps \autoref{def:witnessing} should be revised so that an event \(\ed{\prime}\) may be \wit{} so long as the agent adequately reasons (an optionally concludes).
  %   % However, providing an adequate characterisation of the relevant event is difficult.
  %   % That the agent \emph{reasoned} to \(\pv{\theta}{\valI{True}}\) from \(\Theta\) is insufficient.
  %   % 
  %   % For example, consider a variation of the \scen{} in which the agent identifies a problem with the proof.
  %   % Given the presence of a problem, there may be no \ros{} for the agent to have a \wit{0} for.
  %   Still,
  %   % rather than attempt to characterise \ros{1} independently of conclusions,
  %   we keep the present definition of a \wit{} for simplicity.
  % }

%%% Local Variables:
%%% mode: latex
%%% TeX-master: "master"
%%% TeX-engine: luatex
%%% End:
