\chapter{Plan for tension}

\begin{note}
  \color{red}
  New plan.

  The problem, really, is even getting to general ability.
  For, there are always going to be checks, in the form of specific abilities.
  If you want to use general to conclude specific, then the problem is getting to general first.

  Here, this is no doubt a non-deductive step.
  However, it is a completely reasonable non-deductive step.

  There are plenty of examples of this.
  Logic, solving problems in basic textbooks.
  Reading, going through YA books.
  Chess, solving basic problems.
  And so on.

  The problem is making this leap, so to speak.

  Now, there are two perspectives on general ability.
  First, is this property.
  Second, is witnessing.

  With property, there is trouble.
  For, with property, specific is going to always be a \requ{}.

  However, with witnessing, the idea here is that you get confident enough with premises and steps that you apply these to the relevant specific cases.

  What matters, then, is concluding that the basics are good enough.
  Then, this extends to all the specifics.
\end{note}

\begin{note}
  Here, to include how the argument is going to work.
  That is, the distinction matrix, and what this will amount to.

  So, matrix.

  For this we need three two things.
  Cases.
  Ability.
  Different types of reasoning.

  So, develop these, and then work through the problems.
\end{note}

\section{Structure of argument}
\label{sec:structure-argument}

\begin{note}[Structure of argument]
  Two lines of argument for endorsing~\EAS{}, and hence denying~\ESU{}.
  \begin{enumerate}[label=(L\arabic*), ref=(L\arabic*)]
  \item\label{arg:line:1} Motivate~\EAS{} as resolution to tension resulting from~\ESU{}.\newline
    Specifically:
    \begin{enumerate}[label=(L1\alph*)]
    \item\label{arg:line:1:a} Provide recipe for generating scenarios where~\ESU{} is in tension with particular scenarios involving information that an agent has the ability reason to some conclusion and a further claim regarding when it permissible for an agent to claim support for a proposition.
    \item\label{arg:line:1:b} Motivate~\EAS{} as a resolution to the tension.
    \end{enumerate}
  \item\label{arg:line:2} Argue that granting~\EAS{} as an exception to~\ESU{} allows for an intuitive understanding of cases in which agent has the option of appealing to ability, even if there are alternative ways of interpreting the scenario in line with~\ESU{}.
  \end{enumerate}
  These two lines of argument work together.
  The tension of~\ref{arg:line:1} generates interest in witnessing that may be flatly rejected by prior endorsement of~\ESU{}.
  The intuitive understanding of scenarios involving ability of~\ref{arg:line:2} suggests there's more to witnessing than resolving the tension in narrow cases.
\end{note}

\begin{note}[Details of \ref{arg:line:1}]
  The initial focus is on the first line of argument,~\ref{arg:line:1}.
  The tension developed in part~\ref{arg:line:1:a} is delicate, but hopefully informative.
\end{note}

\begin{note}
  \color{red}
  The relevant instance of \(\phi\) in \ref{fig:saMtxInterpreted} is that the agent has some specific ability.
\end{note}

\begin{note}[Table]
  \begin{figure}[H]
    \saMtxEmpty{}
    \caption{Distinction matrix for interpretations of \aben{the}. \\ Rows are interpretations of ability, columns are type of reasoning regarding ability.}
    \label{fig:saMtxEmpty}
  \end{figure}
\end{note}

\begin{note}
  \begin{figure}[H]
    \centering
    \saMtxInterpreted{}
    \caption{Distinction matrix with \aben{the}}
    \label{fig:saMtxInterpreted}
  \end{figure}
\end{note}

\begin{note}[Matrix, ruled out]
  \begin{figure}[H]
    \centering
    \saMtxRuledOut{}
    \caption{Distinction matrix}
    \label{fig:saMtxRuledOut}
  \end{figure}
\end{note}

%%% Local Variables:
%%% mode: latex
%%% TeX-master: "master"
%%% End: