\chapter{Plan for tension}

\begin{note}
  Here, to include how the argument is going to work.
  That is, the distinction matrix, and what this will amount to.

  So, matrix.

  For this we need three two things.
  Cases.
  Ability.
  Different types of reasoning.

  So, develop these, and then work through the problems.
\end{note}

\section{Structure of argument}
\label{sec:structure-argument}

\begin{note}[Structure of argument]
  Two lines of argument for endorsing~\EAS{}, and hence denying~\ESU{}.
  \begin{enumerate}[label=(L\arabic*), ref=(L\arabic*)]
  \item\label{arg:line:1} Motivate~\EAS{} as resolution to tension resulting from~\ESU{}.\newline
    Specifically:
    \begin{enumerate}[label=(L1\alph*)]
    \item\label{arg:line:1:a} Provide recipe for generating scenarios where~\ESU{} is in tension with particular scenarios involving information that an agent has the ability reason to some conclusion and a further claim regarding when it permissible for an agent to claim support for a proposition.
    \item\label{arg:line:1:b} Motivate~\EAS{} as a resolution to the tension.
    \end{enumerate}
  \item\label{arg:line:2} Argue that granting~\EAS{} as an exception to~\ESU{} allows for an intuitive understanding of cases in which agent has the option of appealing to ability, even if there are alternative ways of interpreting the scenario in line with~\ESU{}.
  \end{enumerate}
  These two lines of argument work together.
  The tension of~\ref{arg:line:1} generates interest in witnessing that may be flatly rejected by prior endorsement of~\ESU{}.
  The intuitive understanding of scenarios involving ability of~\ref{arg:line:2} suggests there's more to witnessing than resolving the tension in narrow cases.
\end{note}

\begin{note}[Details of \ref{arg:line:1}]
  The initial focus is on the first line of argument,~\ref{arg:line:1}.
  The tension developed in part~\ref{arg:line:1:a} is delicate, but hopefully informative.
\end{note}

\begin{note}[Table]
  \begin{figure}[H]
    \centering
    \begin{tblr}{abovesep=8pt, belowsep=8pt, width=0.95\textwidth, colspec={Q[c,m]|Q[c,m]|Q[1.8,c,m]|Q[1.8,c,m]}}
      \multicolumn{2}{c}{} & \adA{} & \adB{} \\
      \hline
      \multicolumn{2}{c}{\WR{}} & ? & ? \\
      \hline
      \multirow{2}{*}{\AR{}} & Basic & ? & ? \\
      \cline[dashed]{2-4}
      & Derived & ?  & ? \\
    \end{tblr}
    \caption{Distinction matrix for interpretations of \aben{the}. \\ Rows are interpretations of ability, columns are type of reasoning regarding ability.}
  \end{figure}
\end{note}

\begin{note}
  \begin{figure}[H]
    \centering
    \begin{tblr}{abovesep=8pt, belowsep=8pt, width=0.95\textwidth, colspec={Q[c,m]|Q[c,m]|Q[1.8,c,m]|Q[1.8,c,m]}}
      \multicolumn{2}{c}{} & \adA{} & \adB{} \\
      \hline
      \multicolumn{2}{c}{\WR{}} & That there is an event in which \emph{S} \emph{V}s that \(\phi\) entails \(\phi\) & Parts of an event in which \emph{S} \emph{V}s that \(\phi\) entail \(\phi\) \\
      \hline
      \multirow[c]{2}{*}{\AR{}} & Basic  & That \emph{S} has the ability (to \emph{V} that \(\phi\)) entails \(\phi\) & --- \\
      \cline[dashed]{2-4}
      & Derived & That there is some property \emph{P} (from \emph{S} having the ability to \emph{V} that \(\phi\)) entails \(\phi\) & Parts of some property \emph{P} (from \emph{S} having the ability to \emph{V} that \(\phi\)) entails \(\phi\) \\
    \end{tblr}
    \caption{Distinction matrix with \aben{the}}
  \end{figure}
\end{note}

\begin{note}[Matrix]
  \begin{figure}[H]
    \centering
    \begin{tblr}{abovesep=8pt, belowsep=8pt, width=0.95\textwidth, colspec={Q[c,m]|Q[c,m]|Q[1.8,c,m]|Q[1.8,c,m]}}
      \multicolumn{2}{c}{} & \adA{} & \adB{} \\
      \hline
      \multicolumn{2}{c}{\WR{}} & Ruled out by \nI{}  & Ruled out by \ESU{} \\
      \hline
      \multirow{2}{*}{\AR{}} & Basic  & Ruled out by \nI{}  & ---  \\
      \cline[dashed]{2-4}
      & Derived & Ruled out by \nI{}  & Ruled out by \ESU{} \\
    \end{tblr}
    \caption{Distinction matrix}
  \end{figure}
\end{note}

%%% Local Variables:
%%% mode: latex
%%% TeX-master: "master"
%%% End: