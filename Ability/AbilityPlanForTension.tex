\chapter{Tension sensed}

\begin{note}
  Three things
  \begin{enumerate}
  \item \ideaCS{}
  \item \ESU{}
  \item It is possible to claim support for general ability without witnessing each specific instance of the general ability.
  \end{enumerate}

  Develop tension with these three things, roughly.

  \begin{itemize}
  \item So, \ideaCS{} really does the work.
    This sets up two instance of a \requ{}.
    In the simple case, clearly need to deal with specifics before getting general.
    In the more difficult case, where not appealing to general, then always something to check.
  \end{itemize}
\end{note}

\begin{note}
  Now, developing the tension is one thing.
  We don't really require much more of an understanding of claiming support, reasoning, and ability than what has been given.

  However, our goal is not only to establish tension.
  Rather, motivate a way out by rejecting \ESU{}.
  To do this, some more things about claiming support and ability need to be said.
  In particular, how reasoning may avoid \ESU{}, and specifically with respect to ability.

  For this reason, we will limit tension to a brief sketch for now.
  The full argument will be given in {\color{red} ???}.
  First, however, we will go to ability and reasoning.

  Though not needed for tension, needed for way out.
  Hence, rather than tension then complexities, complexities then tension.
  For, then, don't need to worry about re-verifying tension in light of complexities.
\end{note}

\begin{note}
  So, these three ideas are going to be in tension.
  Motivated \ideaCS{} and \ESU{}.
  Ability, have not yet considered.
  Indeed, part of what's interesting here is that we don't need a detailed account of ability.

  Still, interest is in giving up \ESU{}, hence do need some account of ability to retain the possibility.
  That is, it can't be that reasoning with ability is an instance of \ESU{}.
  
\end{note}

\begin{note}
  \color{red}
  New plan.

  The problem, really, is even getting to general ability.
  For, there are always going to be checks, in the form of specific abilities.
  If you want to use general to conclude specific, then the problem is getting to general first.

  Here, this is no doubt a non-deductive step.
  However, it is a completely reasonable non-deductive step.

  There are plenty of examples of this.
  Logic, solving problems in basic textbooks.
  Reading, going through YA books.
  Chess, solving basic problems.
  And so on.

  The problem is making this leap, so to speak.

  Now, there are two perspectives on general ability.
  First, is this property.
  Second, is witnessing.

  With property, there is trouble.
  For, with property, specific is going to always be a \requ{}.

  However, with witnessing, the idea here is that you get confident enough with premises and steps that you apply these to the relevant specific cases.

  What matters, then, is concluding that the basics are good enough.
  Then, this extends to all the specifics.

  So, tension is arrived at by focusing on claiming support for having the general ability.
  The two pressures are
  \begin{enumerate}
  \item Via \ideaCS{}, no \requ{1}.
  \item Via \ESU{}, that the relevant premises are `used' --- for any premise, the agent reasons from that premise.
  \end{enumerate}
  Here's how they work.

  First, going to general.
  So, it's going to be the case that we get something stronger.
  Hence, there's all these specific instances of the general ability.
  And, as these are weaker, these function as \requ{1} (specifically, \crequ{1}).
  Hence, by \ideaCS{}, the agent may not claim support for having the general ability given presence of \requ{1}.

  Hence, the general ability won't work to get the specific instances.

  In other words, in order for the agent to conclude that they have the general ability, the agent must have already concluded that they have the specific instance of the general ability.

  There is always some antecedent check.

  Further, there are \emph{lots} of antecedent checks.

  Now, by \ESU{}, if the agent appeals to some premise, then the agent must reason from that premise.

  So, in order to conclude for all specific instances of the general ability, the agent must have some premise(s) that they witness reasoning from.

  Now, what's tempting here is a uniqueness assumption.
  There's a single premise that works for all cases.
  But I don't think I need that.
  
\end{note}

\begin{note}
  So, set up is lots of experience.
  Very good at propositional logic.

  Now, novel problem (of which there are countably many).
  Here, this is a \requ{}.
  Why is it the case that I wouldn't arrive at some other conclusion, or fail.
  Well, now, this seems quite obvious.

  However, \ESU{} got to witness reasoning from the relevant premise.
  Yet, the general ability as a premise won't do.
  Because, the specific ability is a \requ{}.
\end{note}

\begin{note}
  Here, to include how the argument is going to work.
  That is, the distinction matrix, and what this will amount to.

  So, matrix.

  For this we need three two things.
  Cases.
  Ability.
  Different types of reasoning.

  So, develop these, and then work through the problems.
\end{note}

\section{Structure of argument}
\label{sec:structure-argument}

\begin{note}[Structure of argument]
  Two lines of argument for endorsing~\EAS{}, and hence denying~\ESU{}.
  \begin{enumerate}[label=(L\arabic*), ref=(L\arabic*)]
  \item\label{arg:line:1} Motivate~\EAS{} as resolution to tension resulting from~\ESU{}.\newline
    Specifically:
    \begin{enumerate}[label=(L1\alph*)]
    \item\label{arg:line:1:a} Provide recipe for generating scenarios where~\ESU{} is in tension with particular scenarios involving information that an agent has the ability reason to some conclusion and a further claim regarding when it permissible for an agent to claim support for a proposition.
    \item\label{arg:line:1:b} Motivate~\EAS{} as a resolution to the tension.
    \end{enumerate}
  \item\label{arg:line:2} Argue that granting~\EAS{} as an exception to~\ESU{} allows for an intuitive understanding of cases in which agent has the option of appealing to ability, even if there are alternative ways of interpreting the scenario in line with~\ESU{}.
  \end{enumerate}
  These two lines of argument work together.
  The tension of~\ref{arg:line:1} generates interest in witnessing that may be flatly rejected by prior endorsement of~\ESU{}.
  The intuitive understanding of scenarios involving ability of~\ref{arg:line:2} suggests there's more to witnessing than resolving the tension in narrow cases.
\end{note}

\begin{note}[Details of \ref{arg:line:1}]
  The initial focus is on the first line of argument,~\ref{arg:line:1}.
  The tension developed in part~\ref{arg:line:1:a} is delicate, but hopefully informative.
\end{note}

\begin{note}
  \color{red}
  The relevant instance of \(\phi\) in \ref{fig:saMtxInterpreted} is that the agent has some specific ability.
\end{note}

\begin{note}[Table]
  \begin{figure}[H]
    \saMtxEmpty{}
    \caption{Distinction matrix for interpretations of \aben{the}. \\ Rows are interpretations of ability, columns are type of reasoning regarding ability.}
    \label{fig:saMtxEmpty}
  \end{figure}
\end{note}

\begin{note}
  \begin{figure}[H]
    \centering
    \saMtxInterpreted{}
    \caption{Distinction matrix with \aben{the}}
    \label{fig:saMtxInterpreted}
  \end{figure}
\end{note}

\begin{note}[Matrix, ruled out]
  \begin{figure}[H]
    \centering
    \saMtxRuledOut{}
    \caption{Distinction matrix}
    \label{fig:saMtxRuledOut}
  \end{figure}
\end{note}

%%% Local Variables:
%%% mode: latex
%%% TeX-master: "master"
%%% End: