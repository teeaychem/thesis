\chapter{Temp}

  % \begin{restatable}[\zSN{2} --- \zS{}]{idea}{ideazs}
  %   \label{idea:zs}
  %   An agent \vAgent{} has \emph{\zSAb{-}} for a proposition-value pair \(\pv{\phi}{v}\) with respect to some pool of premises \(\Phi\) when concluding \(\pv{\phi}{v}\) from \(\Phi\) just in case:
  %   \begin{itemize}
  %   \item
  %     When concluding \(\pv{\phi}{v}\) from \(\Phi\):
  %     \begin{enumerate}[label=\arabic*., ref=\named{\zSAb{}:\arabic*}]
  %     \item
  %       For any proposition-value-premises pairing \(\pvp{\psi}{v'}{\Psi}\) which is a \requ{} of concluding \(\pv{\phi}{v}\) from \(\Phi\):
  %       \begin{enumerate}[label=\alph*., ref=\named{\zSAb{-}:1.\alph*}]
  %       \item
  %         \vAgent{} has `settled' that they would conclude \(\pv{\psi}{v'}\) from \(\Psi\).
  %       \item
  %         \vAgent{} simultaneously `settles' that they would conclude \(\pv{\psi}{v'}\) from \(\Psi\) when concluding \(\pv{\phi}{v}\) from \(\Phi\).
  %       \end{enumerate}
  %     \end{enumerate}
  %   \end{itemize}
  %   \vspace{-\baselineskip}
  % \end{restatable}

\subsection{Deviance}
\label{sec:deviance}

\begin{note}
  Here, causal deviance.
\end{note}

\begin{note}
  Problem is, there's no way to guarantee a link between positive answer to \qzS{} and the agent concluding or not refraining from concluding.
\end{note}

\begin{note}
  Argument relies on tying content to explanation.

  In this respect, there is room for an objection.
  Deviant causal chains.
  Point here is that there are cases where these come apart.

  This isn't only a problem for causal theories of reasoning.
  The point is, some instantiation, and so long as act may be caused by something else, then possibly caused by the instantiation.

  So, possible here.

  Well, hold on.
  What is need is the relevance of the content.
  For this objection to work, need to take a theoretical perspective.
  See, in Davidson's case, the idea is fusing these two things together.
  We answer two different questions with a common thing viewed in two ways.

  Still, I think the objection can be pressed!
  Only \emph{really} an explanation is no deviance.
  To the same extent that potential event matters, it matters to the agent that there is no deviance.

  {
    \color{red}
    Resolution is, if deviance, then no agency.
  }

  I think this makes sense, or at least makes enough sense.
  Answers to `why', on this understanding, are tentative.

  Or, rest on presupposition that agent performed the action.

  So, contingent on showing there is no causal deviance.

  This is different to error.
  With error, thing appealed to isn't the case, but appeal still did work.
  Here, it doesn't matter whether or not the case, no work is done.

  In contrast to more typical instances of the problem, don't need to rule out deviant causal chains.
  Instead, just need one instance to fail to hold.
  One instance of non-deviousness.

  Still a problem for a compatible account which avoids.
  For, here, there can't be any direct link from perspective to reason.

  For example, \citeauthor{Hieronymi:2011aa}

    \begin{quote}
      [W]e explain an event that is an action done for reasons by appealing to the fact that the agent took certain considerations to settle the question of whether to act in some way, therein intended so to act, and successfully executed that intention in action.
    [\emph{T}]\emph{his} complex fact, [\dots] is the reason that rationalizes the action---that explains the action by giving the agent's reason for acting.%
    \mbox{ }\hfill\mbox{(\citeyear[431]{Hieronymi:2011aa})}
  \end{quote}

  So, here, considerations which settle question, and in so settling question.
  Link between settling the question and acting.

  Following \citeauthor{Hieronymi:2011aa}, no room for deviance.
  Too tight.

  In other words, so long as this fact holds, there is no distinction between settling the question and acting.
  Therefore, no deviance.

  Compatible, I think.
  Question is whether in resolving \qzS{} is sufficiently tied to resolving the question \citeauthor{Hieronymi:2011aa} identifies.
  And, plausibly is.
  This is what the motivation for \qzS{} did.

  Trouble is, for our purposes, need at least sufficient conditions for when this complex fact obtains.
  And, no account of this.

  \citeauthor{Hieronymi:2011aa} notes the gaps.

  Some tension.
  These considerations aren't premises.
\end{note}

\begin{note}
  So, the other option is to embrace deviant causal chains.
  Have the content, but this doesn't work in the way the agent thinks it does.

  Example from Davidson.

  The trouble here is that the content and resulting action match.
  So, things make sense from the agent's perspective.

  Deviant, but maybe not so deviant here.

  Systematic deviance, where content is separated from role of mental state.

  But, I see no motivation for this.

  Solution to causal chains doesn't get round this, because the result is a restricted account.
  So, there's no guaranteed trade-off here.
  Trouble is, it seems hard to see a case where this wouldn't be the case.
\end{note}

\paragraph{When an agent has \zS{}}

\begin{note}
  Now, two basic propositions follow.

  \begin{proposition}[When a agent has \zS{}]
    Agent and proposition-value-premises pairing.

    \begin{itemize}
    \item
      Agent \emph{has} \zS{} for \(\pv{\phi}{v}\) after concluding \(\pv{\phi}{v}\) from \(\Phi\).
    \end{itemize}

    \emph{if and only if}

    \begin{itemize}
    \item
      Either:
      \begin{enumerate}[label=(\alph*), ref=\alph*]
      \item
        No \requ{1} of concluding \(\pv{\phi}{v}\) from \(\Phi\) from the agent's perspective.
      \item
        For any \requ{} \(\pvp{\psi}{v'}{\Psi}\) of concluding \(\pv{\phi}{v}\) from \(\Phi\), the agent would not fail to conclude \(\pv{\psi}{v'}\) from \(\Psi\), from the agent's perspective.
      \end{enumerate}
    \end{itemize}
  \end{proposition}

  Second, when an agent does not have \zS{}.

  \begin{proposition}[When a agent does not have \zS{}]
    Agent and proposition-value-premises pairing.
    \begin{itemize}
    \item
      Agent \emph{does not} have \zS{} for \(\pv{\phi}{v}\) after concluding \(\pv{\phi}{v}\) from \(\Phi\).
    \end{itemize}

    \emph{if and only if}

    \begin{itemize}
    \item
      There is some proposition-value-premises pairing \(\pvp{\psi}{v'}{\Psi}\) such that, from the agent's perspective, the agent may fail to conclude \(\pv{\psi}{v'}\) from \(\Psi\).
    \end{itemize}
  \end{proposition}
\end{note}

\paragraph{Lazy argument}

\begin{note}
  Three premises:
  \begin{enumerate}
  \item
    \label{lazy:evidence-constraint}
    If evidence for \(\pv{\phi}{v}\), then no evidence for conflicting \(\pv{\psi}{v'}\).
  \item
    \label{lazy:evidence}
    Support only if evidence.
  \item
    \label{lazy:reason}
    Support to \(\pv{\phi}{v}\) from \(\Phi\) only if possible to conclude from \(\pv{\phi}{v}\) to \(\Phi\) from present perspective.
  \end{enumerate}

  Second, reasons are evidence.
  General expression found in \hyperlink{cite.Hume:1748tp}{Hume}'s \hyperlink{cite.Hume:1748tp}{Enquiry}:
  \begin{quote}
    A wise man, \dots, proportions his belief to the evidence.
    In such conclusions as are founded on an infallible experience, he expects the event with the last degree of assurance, and regards his past experience as a full proof of the future existence of that event.%
    \mbox{ }\hfill\mbox{(\hyperlink{cite.Hume:1748tp}{E.10.1.4})}
  \end{quote}

  Though, stronger formulation termed `evidentialism'~\citeauthor{Feldman:1985wx}:

  \begin{quote}
    \begin{enumerate}
    \item[EJ] Doxastic attitude \emph{D} toward proposition \emph{p} is epistemically justified for \emph{S} at \emph{t} if and only if having \emph{D} toward \emph{p} fits the evidence \emph{S} has at \emph{r}.%
      \mbox{ }\hfill\mbox{(\citeyear[15]{Feldman:1985wx})}
    \end{enumerate}
  \end{quote}

  Read, reason only if evidence.

  Evidence is consistent.
  Take as given.

  Here we cite~\citeauthor{Achinstein:2001ub} (\citeauthor{Achinstein:2001ub}) who channels~\textcite{Carnap:1962ue}:

  \begin{quote}
    \emph{e} is evidence that \emph{h} if and only if \(p(h \mid e) > p(h)\).%
    \mbox{ }\hfill\mbox{(\citeauthor[45]{Achinstein:2001ub})}
  \end{quote}

  Then, impossible for \(e\) to be evidence for both \(h\) and \(q\) such that \(h\) and \(q\) cannot jointly occur.
  \footnote{
    Suppose \(h\) and \(q\) are incompatible.
    \(h \cap q = \emptyset\).

    \(p(h \mid e) + p(\lnot h \mid e) = 1\).
    \(p(h \mid e) > p(h)\).
    So, \(p(h) = x\) and \(p(h \mid e) = x + y\), where \(x, y > 0\).
    \(p(\lnot h \mid e) = (1 - x - y)\).
    \(p(\lnot h) = (1 - x)\).
    \((1 - x - y) \not> (1 - x)\).

    Now, \(p(q \mid e) \leq p(\lnot h \mid e)\).
    But, expand.
    \(h \cap q = \emptyset\).
    \(q \subseteq \lnot h\).
    \((q \cap e) \subseteq (\lnot h \subseteq e)\).
    \(p(q \land e) \leq p(\lnot h \land e)\).
    \(p(q \mid e) \leq p(\lnot h \mid e)\).
  }
\end{note}

\paragraph{Pryor}

\begin{note}
  \citeauthor{Pryor:2004ws}'s argument that type 4 over-generates is somewhat interesting.
  Details are in the following footnote.\footnote{
  Compatible with \citeauthor{Pryor:2004ws}'s objection to type 4 dependence.

  % \begin{illustration}
    % \mbox{}
    % \vspace{-\baselineskip}
    \begin{quote}
      Suppose you're watching a cat stalk a mouse. Your visual experiences justify you in believing:

      \begin{enumerate}[label=(\arabic*), ref=(\arabic*)]
        \setcounter{enumi}{10}
      \item
        \label{illu:Pryor:cat:1}
        The cat sees the mouse.
      \end{enumerate}

      You reason:

      \begin{enumerate}[label=(\arabic*), ref=(\arabic*), resume]
      \item
        \label{illu:Pryor:cat:2}
        If the cat sees the mouse, then there are some cases of seeing.
      \item
        \label{illu:Pryor:cat:3}
        So there are some cases of seeing.\nolinebreak
        \mbox{}\hfill\mbox{(\citeyear[361]{Pryor:2004ws})}
      \end{enumerate}
    \end{quote}
  % \end{illustration}

  Setting aside whether this is fine.

  Following \citeauthor{Pryor:2004ws}:

  Bad, given proposal, as if no cases of seeing, then the cat is not seeing. (\citeyear[361]{Pryor:2004ws})

  \citeauthor{Pryor:2004ws}'s position is as follows:

  \begin{quote}
    I don't think you need antecedent justification to believe \ref{illu:Pryor:cat:3}, before your experiences can give you justification to believe \ref{illu:Pryor:cat:1}.
    I also think it's plausible that your perceptual justification to believe \ref{illu:Pryor:cat:1} contributes to the credibility of \ref{illu:Pryor:cat:3}.\nolinebreak
    \mbox{}\hfill\mbox{(\citeyear[361]{Pryor:2004ws})}
  \end{quote}

  This fine when seen from the perspective of the conditional being a \requ{}.
  }
\end{note}


\paragraph{Knowing whether and knowing how}

\begin{note}
  Read the above examples in terms of knowing whether.

  More-or-less direct link between knowing whether and knowing how:

  Know whether \(?\{a,b,c,\dots\}\) know how to answer whether it is the case that \(?\{a,b,c,\dots\}\)
\end{note}

\begin{note}
  Ideas regarding \citeauthor{Ryle:1946tu}'s distinction between knowing \emph{how} and knowing \emph{that}~(Cf.~\citeyear{Ryle:1946tu}).

  Now, I confess my understanding of \citeauthor{Ryle:1946tu}'s distinction is limited --- I have not taken whatever opportunities I have had to read through \citeauthor{Ryle:1946tu}'s work.%
  \footnote{
    Though, I understand enough from passing commentary to note that the idea \emph{I} am perusing here does not, strictly, require that knowledge how and knowledge that are distinct kinds of knowledge.

    Intellectualist and anti-intellectualist views.

    For, granting that knowledge how reduces to knowledge that, it will remain the case that there is an event\dots
    (See~\textcite{Pavese:2022up} for more!)
  }

  Following analogy from~\textcite{Ryle:2009us}:

  \begin{quote}
    Knowing `\emph{if p, then q}' is, \dots rather like being in possession of a railway ticket.
    It is having a licence or warrant to make a journey from London to Oxford.
    (Knowing a variable hypothetical or `law' is like having a season ticket.)
    As a person can have a ticket without actually travelling with it and without ever being in London or getting to Oxford, so a person can have an inference warrant without actually making any inferences and even without ever acquiring the premisses from which to make them.%
    \mbox{ }\hfill\mbox{(\citeyear[250]{Ryle:2009us})}
  \end{quote}

  Continuing~\citeauthor{Ryle:2009us}'s analogy, in the case of \fc{1}:
  What matters is that the agent is currently in possession of the (season) ticket.
\end{note}

\begin{note}
  The relationship between \fc{1} and knowing whether is interesting.
  For, as stated, \fc{1}, just about potential event.
  Plausible paraphrase.

  Whether this goes in both directions.

  So, from knowing whether, generally understood, there's got to be an event.
  However, it is not clear to me that there is a `potential' constraint on the event.

  \cite{Bengson:2011th}.
  \begin{quote}
    \emph{Pi}.
    Louis, a competent mathematician, knows how to find the n\(^{\text{th}}\) numeral, for any numeral \(n\), in the decimal expansion of \(\pi\).
    He knows the algorithm and knows how to apply it in a given case.
    However, because of principled computational limitations, Louis (like all ordinary human beings) is unable to find the \(10^{46}\) numeral in the decimal expansion of \(\pi\).%
    \mbox{ }\hfill\mbox{(\citeyear[170]{Bengson:2011th})}
  \end{quote}

  Conversely, not clear \fc{0} leads to knowing whether.

  Issues here.

  First, factive constraint on knowledge.
  And, concluding need not be factive.

  Though, whether this matters in practice is not clear.
  For, from agent's perspective.

  Second, knowledge is not the right attitude from concluding.
  For example, focus is on deductive cases, but abductive conclusion.
  Or, some doubts about premises.
\end{note}


\section{Landman possible continuations}
\label{sec:landm-poss-cont}

\subparagraph{Possible Continuations}

\begin{note}
  Additional function.

  \begin{algorithm}[H]
    \SetAlgoLined
    \DontPrintSemicolon
    \Input{\(\langle \langle e,w \rangle_{i-1}, \langle g,u \rangle_{i} \rangle\)}
    \KwResult{Possible continuations of \(\langle g,u \rangle_{i} \rangle\)}
    \Begin{
      \(\Diamond\text{Continuations} \longleftarrow \emptyset\)\;
      \(\text{CloseWorlds} \longleftarrow \{u' \mid u' \text{ is among the closest world to } u\}\)\;
      \For{\(u' \in \text{CloseWorlds}\)}
      {
        \For{\(g'\) in \(u'\)}
        {
          \If{\(g' \text{ is a stage of } h \text{ in } u'\)}
          {
           \(\SetPC{} \longleftarrow \AlgAC{})(\langle \langle g,u \rangle_{i}, \langle g',u' \rangle_{i} \rangle)\)
          }
        }
      }
      \Return{\(\{\langle \langle g,u \rangle_{i+1}, \langle h,u \rangle_{i+1} \rangle \rangle \mid \langle h,u \rangle_{i+1} \rangle \in \Diamond\text{Continuations}\}\)}
    }
    \caption{\AlgPC{}\label{PrAl:g-p-c}}
  \end{algorithm}

  Important.
  \citeauthor{Landman:1992wh}'s original formulation, where it actually stops.
  However, here, we include possible stops.
  This covers darts case.

  Now, what matters is when evaluated.
  If moment of starting the throw, then progressive is false.
  However, if moment of release, then progressive is true.

  No distinction if just focus on how things develop.
\end{note}

\subparagraph{Continuations}

\begin{note}
  \begin{algorithm}[H]
    \SetAlgoLined
    \DontPrintSemicolon
    \Input{\(\langle \langle f,v \rangle_{i-1}, \langle g,u \rangle_{i}  \rangle\)}
    \KwResult{Continuations of \(\langle g,u \rangle_{i}\)}
    \Begin{
      \(\text{@-Continuations }\longleftarrow \AlgAC{}(\langle \langle f,v \rangle_{i - 1}, \langle g,u \rangle_{i}\rangle)\)\;
      \For{everything in actual}
      {
        \(\text{\(\Diamond\)-Continuations }\longleftarrow \AlgPC{}(\langle \langle f,v \rangle_{i - 1}, \langle g,u \rangle_{i}\rangle)\)\;
        \(\text{Continuations} \longleftarrow \SetAC{} \cup \SetPC{}\)\;
      }
      \Return{Continuations}
    }
    \caption{\AlgGetCs{}\label{PrAl:g-c}}
  \end{algorithm}
\end{note}

\section{Carrol and regress}
\label{sec:carrol-regress}

\begin{note}
  Similar to \citeauthor{Carroll:1895uj}.
  \begin{quote}
    Logic would take you by the throat, and \emph{force} you to do it!%
    \mbox{ }\hfill\mbox{(\citeyear[280]{Carroll:1895uj})}
  \end{quote}
  Looking at something static.
  Achilles fails to convey this to the Tortoise, arguably through some fault of Achilles' own.

  In parallel, we could stack up additional passives in the same way, but there's little interest in doing so.
  The point is the base \requ{} is not satisfied.
\end{note}

\begin{note}
  So, with \citeauthor{Carroll:1895uj}, we get a rule of inference, great.

  \citeauthor{Wieland:2013vf} characterises the general understanding of \textcite{Carroll:1895uj} in terms of two lessons:
  \begin{quote}
    [T]he negative lesson is that if you add ever more premises to an argument \dots, then you will never demonstrate that its conclusion follows logically.%
    \mbox{ }\hfill\mbox{(\citeyear[984]{Wieland:2013vf})}
  \end{quote}

  Parallel, static answers, still option for concluding otherwise.

  \begin{quote}
    [T]he positive lesson is that rules of inference, rather than premises of the form `if premises such and such are true, then the conclusion is true', will do the job.%
    \mbox{ }\hfill\mbox{(\citeyear[984]{Wieland:2013vf})}
  \end{quote}

  Parallel, the dynamic status of a rule.
\end{note}

\begin{note}
  No regress.

  Following \citeauthor{Wieland:2013vf}:

  \begin{quote}
    \begin{itemize}[noitemsep]
    \item[IR]
      For any item x of a certain type, S \(\varphi\)-s x only if
      \begin{enumerate}[label=(\roman*),noitemsep]
      \item
        there is a new item y of that same type, and
      \item
        S \(\varphi\)-s y.%
        \mbox{ }\hfill\mbox{(\citeyear[996]{Wieland:2013vf})}
      \end{enumerate}
    \end{itemize}
  \end{quote}
\end{note}

\section{Wright}
\label{sec:wright}

\begin{note}
  \color{red}
  Some of the \citeauthor{Wright:2011wn} cases are interesting.
  Especially the twin cases.
  In fact, especially this idea that situations are identical.
  For, one way of understanding this is that the agent makes a choice between two disjuncts, and it is possible for the agent to make the other choice, and then come to a different conclusion.
\end{note}

\paragraph[Actuality entailments]{Actuality entailments \hfill (Optional)}


\begin{note}
  
  If actuality entailment, then possibility of reducing to modal.

  However, clear failure here.
  For, \support{}, but same modal applies to concluding.

  Here, intuitively a significant issue if actuality entailment held.%
  \footnote{
    Inclined to allow concluding without witnessing, but for use, conclude only if witnessed.
  }
\end{note}


\begin{note}
  Clear example of this from Igal Kvart, via~\textcite{Landman:1992wh}:

  \begin{quote}
    Look at example~\ref{Landman:FRomans}:
    \begin{enumerate}[label=(\arabic*), ref=(\arabic*)]
      \setcounter{enumi}{19}
    \item
      \label{Landman:FRomans}
      Mary was wiping out the Roman army.
    \end{enumerate}

    The situation is that Mary, a person of moderate physical capacities, is battling the Roman army.
    She manages to kill a couple of soldiers before she gets killed.
    \ref{Landman:FRomans} is clearly false in this situation.\newline
    \mbox{ }\hfill\mbox{(\citeyear[18]{Landman:1992wh})}
  \end{quote}

  Mary may have survived, and more Roman soldiers may have perished by Mary's hand.
  However, relation of Mary to the Roman army means no sense in which event is Mary wiping out the Roman army.
\end{note}

\section{\citeauthor{Boylan:2020aa}}

\begin{note}
  To get clear on how \citeauthor{Boylan:2020aa} understands control intuition, develop the formal core of \citeauthor{Boylan:2020aa}'s theory.
  Difficulty will not depend on particular details, but general understanding of control given by details.

  Need a handful of things.
  \begin{itemize}
  \item
    Unsettled world
  \item
    Selection function
  \item
    Semantics for will.
  \end{itemize}

  First, unsettled world.
  `while the past is settled, the future is open' (\citeyear[1]{Boylan:2020aa})

  \begin{quote}
    \emph{Unsettled World}. \(\mathcal{I}_{c} = \{w \mid w\text{ is identical to }w_{c}\text{ up until }t_{c}\}\)%
    \mbox{ }\hfill\mbox{(\citeyear[11]{Boylan:2020aa})}
  \end{quote}
  Relative to some context \(c\), which includes world \(w_{c}\) and a time \(t_{c}\), set of worlds which are identical to to \(w_{c}\) up until \(t_{c}\).

  Note, unique with respect to context.

  So, with when considering the actual world at some point in time, corresponding unsettled world is just set containing all worlds identical to the past.

  Caution, an unsettled world is a set of worlds.

  Selection function:

  \begin{quote}
    \(s(\mathcal{I}, \mathbf{A})\) picks out the closest (possibly unsettled) world to \(\mathcal{I}\) which settles that \(\mathbf{A}\) is true.%
    \mbox{ }\hfill\mbox{(\citeyear[11]{Boylan:2020aa})}
  \end{quote}
  So, selection function takes unsettled world and action as input and returns set containing a world or collection of worlds such that action is settled to be true.

  Key thing here is \(f(w,t)\).
  Intuitively, restrict which worlds are part of the unsettled world.
  Strictly, closest (unsettled) world.
  However, closeness is not of interest.

  \citeauthor{Boylan:2020aa} does not define settled.
  However, \(\mathbf{A}\) is true at every world in set.%
  \footnote{
    `If \(\phi\) is true at \(s(\mathcal{I}, f(w,t))\) (i.e. true throughout \(\mathcal{I}\)),\dots'. \mbox{(\citeyear[12]{Boylan:2020aa})}
  }

  \(\mathcal{W}\) `will'.

  \begin{quote}
    \begin{enumerate}
      \setcounter{enumi}{33}
    \item
      \begin{enumerate}
      \item
        \(\sem[w,t,f,\mathcal{I}]{\mathcal{W}\phi}\) is determinate only if either
        \begin{enumerate}
        \item
          \(s(\mathcal{I}, f(w,t)) \subseteq \sem[t,f,\mathcal{I}]{\phi}\) or
        \item
          \(s(\mathcal{I}, f(w,t)) \subseteq \sem[t,f,\mathcal{I}]{\lnot\phi}\)
        \end{enumerate}
      \item
        If determinate \(\sem[w,t,f,\mathcal{I}]{\mathcal{W}\phi} = 1\) iff \(s(\mathcal{I}, f(w,t)) \subseteq \sem[t,f,\mathcal{I}]{\phi}\)
      \end{enumerate}
    \end{enumerate}
  \end{quote}

  Here, \(w\) world, \(t\) time, \(f\) a function from a world and a time to a set of worlds, and \(\mathcal{I}\) an unsettled world.

  \(f(w,t) = \top\), then, abstracting from closest, all future possibilities.
  \(f(w,t) = \mathbf{A}\), then, all future possibilities in which \(\mathbf{A}\) happens.

  Mechanism is unclear.
  Intuitively, think of \(f\) as returning a collection of propositions where to evaluate, context is the same, but shifted time, and world is excluded if does not match with some proposition.
  So, truth of proposition at some (restricted) time in world.

  Key idea, restrict future worlds of interest so that the agent has performed some action.

  Now turn to `can'.

  \begin{quote}
    \begin{enumerate}
      \setcounter{enumi}{41}
    \item
      \(\sem[w,t,f,\mathcal{I}]{\text{S can }\phi} = 1\) iff for some \(\alpha \in \mathcal{A}\colon \sem[w,t,f^{\alpha},\mathcal{I}]{\mathcal{W}(\text{S }\phi\text{'s})} = 1\)\newline
      i.e.\ iff for some \(\alpha \in \mathcal{A}(w,t)\colon s(\mathcal{I}, f^{\alpha}(w,t)) \subseteq \sem[t,f^{\alpha},\mathcal{I}]{\text{S }\phi\text{'s}} = 1\)\newline
      \mbox{ }\hfill\mbox{(\citeyear[16]{Boylan:2020aa})}
    \end{enumerate}
  \end{quote}
  Here, key thing is \(f^{\alpha}\).
  We're making sure that the agent performs the action in the unsettled world.

  In short, we chose some action, and figure out whether it's a phing action.
\end{note}


%%% Local Variables:
%%% mode: latex
%%% TeX-master: "master"
%%% End:
