\chapter{Positive answers and \qWhyV{}}
\label{cha:zSpAwhy}

\begin{note}
  We are almost done.

  \qzS{}.
  Answers to \qzS{}.

  In \autoref{cha:zS}, introduced and looked at positive answers to \qzS{}.
  Argued potential event.
  In other words, potential event without qualification, from the \agpe{}.

  Of course, from the \agpe{}.
  And, the \agpe{} is a qualifier.
  In particular, non-factive.
  So, does not follow that there is a potential event.
\end{note}

\begin{note}
  So, we've got how potential events are linked to answers to \qzS{}.

  Second part of the argument.
  Answers to \qzS{} and \qWhyV{}.
\end{note}

\begin{note}
  Recall, the condition.
\end{note}

\begin{note}
  In broad outline, need to get \support{}.
  And, then need to get a link to concluding.

  Link to concluding via argument from {\color{red} \dots}.
\end{note}

\section{Propositions}
\label{cha:zSpAwhy:sec:proposition}

\begin{note}
  Proposition for this chapter.

  \begin{proposition}
    \label{prop:fc-answers-why}
    For an agent \vAgent{}, \(\pv{\phi}{v}\), \(\pv{\psi}{v'}\):

    \begin{enumerate}
    \item[\emph{If}:]
      \begin{enumerate}[label=\alph*., ref=(\alph*)]
      \item
        \vAgent{} has concluded \(\pv{\phi}{v}\) from \(\Phi\).
      \end{enumerate}
    \item[\emph{And}:]
      \begin{enumerate}[label=\alph*., ref=(\alph*), resume]
      \item
        \qzS{} has positive answer when concluding \(\pv{\phi}{v}\) from \(\Phi\).
      \end{enumerate}
    \item[\emph{And}:]
      \begin{enumerate}[label=\alph*., ref=(\alph*), resume]
      \item \(\pvp{\psi}{v'}{\Psi}\) was a \requ{} of concluding \(\pv{\phi}{v}\) from \(\Phi\).
      \end{enumerate}
    \item[\emph{Then}:]
      \begin{enumerate}[label=\alph*., ref=(\alph*), resume]
      \item \support{2} between \(\pv{\psi}{v'}\) and \(\Psi\) (from \agpe{\vAgent{}'}), in part, answers \qWhyV{}.
      \end{enumerate}
    \end{enumerate}
  \end{proposition}
  In other words, the potential event in which the agent concludes \(\pv{\psi}{v'}\) from \(\Psi\), in part, answers \qWhyV{} the agent concluded \(\pv{\phi}{v}\) from \(\Phi\).
\end{note}

\begin{note}
  \autoref{prop:fc-answers-why} is \emph{almost} immediate.
  \(\pvp{\psi}{v'}{\Psi}\) is a \requ{0}.
  But positive answer to \qzS{}.
  So, \(\pvp{\psi}{v'}{\Psi}\) is a \fc{0}.
  Therefore, by \autoref{prop:fcs-only-if-support}:
  \support{2} between \(\pv{\psi}{v'}\) and \(\Psi\).

  However, `answers \qWhy{}'.
  Here is where the difficulty is.

  Well, would not conclude if not \fc{}.

  So, here, then, need to ensure no deviance.

  However, further problem.
  Deny by `embedding' in relation of \support{}.
\end{note}

\begin{note}
  Key is \autoref{prop:fc-answers-why:REDUX}:
  \begin{proposition}
    \label{prop:fc-answers-why:REDUX}
    If \fc{} is embedded in \support{}, then (plausibly) a further instance of \qzS{}.
  \end{proposition}
  Qualification, however, this is fine as we're just looking for the existence of \scen{1}.
\end{note}

\begin{note}
  Would not conclude without support.

  Answer this in two steps.
  \begin{enumerate}
  \item
    Would not conclude unless \fc{}.
  \item
    \fc{} only if \support{}.
  \end{enumerate}
\end{note}

\section{\autoref{prop:fc-answers-why:REDUX}}
\label{sec:simple-argument}

\begin{note}
  Intuitively, \autoref{prop:fc-answers-why} follows from how \qzS{} relates to concluding \(\pv{\phi}{v}\) from \(\Phi\).

  For, the way things are set up.
  \qzS{} is about the act of concluding.
  Therefore, answer \qzS{}, is tied to concluding, and so \support{} answers \qWhyV{}.
\end{note}

\paragraph{What's going on}

\begin{note}
  Now, key observation is that \requ{} of concluding \(\pv{\phi}{v}\) from \(\Phi\).
  \qzS{} is about this instance of concluding.

  Regardless of whether embedded, whether conclude from \(\Phi\).

  Think back to \requ{}.
  Would not conclude otherwise.
  If not \fc{}, then may conclude otherwise.
  Hence, may \deadEnd{} \support{}.
\end{note}

\begin{note}
  \begin{observation}
    Relation of support holds between pool of premises and a conclusion.
  \end{observation}
\end{note}

\begin{note}
  Relation of support is an interesting thing.
  It holds between pool of premises and a conclusion.
  This answers \qWhy{}.

  Initial observation, then, is that some care should be given to what we get.
  There's a difference between a relation between pool and a proposition, and relation between pool of propositions and an event.

  Understanding is that relation between pool and proposition-value answers, in part, why an event took place.
\end{note}

\begin{note}
  Now, \qzS{} is about the event taking place.
  \qzS{} is not (directly) about whether \(\phi\) has value \(v\), but about the event of concluding \(\pv{\phi}{v}\) from \(\Phi\).
\end{note}

\begin{note}
  At issue with \qzS{} is \emph{not}:

  If agent were to fail to hold that you would conclude \(\pv{\psi}{v'}\) from \(\Psi\) as \emph{a premise}.
\end{note}

\begin{note}
  Note, the tension is not about whether \(\phi\) has value \(v\).
  Instead, the tension is about whether the agent would have a certain property if they were to conclude \(\phi\) has value \(v\).
  \zS{}.
  Expanded, property of holding that any independent check is satisfied.
  For any \requ{}, would not fail to conclude.
\end{note}

\begin{note}
  This is the plain reading.

  I think the plain reading is appropriate.
\end{note}

\begin{note}[Aside]
  Note, this is part of what I find appealing about a positive resolution to \issueConstraint{} and concluding, etc.

  When concluding, we're interested in how to go from premises to conclusion.
  The relevant content of the premises is distinguished from relation which holds between premises and conclusion.

  The way in which content relates, rather than content itself.

  To elaborate.
  Conclusion is \(\pv{\phi}{v}\).
  Now, \(\pv{\chi}{v''}\) is in part an answer to why \(\pv{\phi}{v}\) is the case from the \agpe{}.
  However, not, in part, an answer to why agent concluded, from \agpe{}.
  For, concluding, what matters is relation of support between pool of premises containing \(\pv{\chi}{v''}\) and \(\pv{\phi}{v}\).
\end{note}

\begin{note}
  Task is to show that \qzS{} is not answered.
\end{note}

\section{Second proposition}
\label{sec:flexible-argument}

\begin{note}
  The argument presented rests on \qzS{}.

  Key link is how \qzS{} tied whether or not \fc{} the event of concluding.

  On the one hand, valid argument.
  On the other hand, still a question.

  So, is it possible to avoid this.
  Grant \fc{} is important, but allow the possibility of embedding.

  So, whether the relevant issue may be raised in a way such that don't get an instance of \qzS{}.

  Idea is, agent concludes \(\pv{\phi}{v}\) in part from positive answer to weaker variant of \qzS{}.
  If so, then it looks as though what should matter is \support{} from positive answer to \(\pv{\phi}{v}\).

  Recall, from \autoref{cha:var:ros:Emb}.
  Embedding.
  If \support{} from positive answer then \support{} is embedded.
\end{note}

\begin{note}
  Well, look:

  There are two ways to view this.
  First, whether the question is relevant.
  Second, any particular instance of the question.

  Deny the question, then things fall apart.
  Connexion to concluding is too strong.

  However, allow the possibility and deny that it ever needs a positive answer.

  This is good methodology.

  Proposal, then, is any instance, embed.

  \begin{idea}
    \label{idea:possible-fc-matters}
    Possible that if not \fc{} would not conclude.
  \end{idea}

  Here, the abstract possibility.
  Reasoning.
  Various different conclusions.
  In some cases, matters that the conclusion is determined, so to speak.

  Abstract property, motivated by some cases.

  Lost keys, what matters is whether the agent may reach a different conclusion.
  \autoref{idea:possible-fc-matters}, different conclusion matters.
  Conversely, no other conclusion matters.

  However, for any case, deal with this in a different way.

  \begin{idea}
    Possible for whether or not \fc{} to be embedded in \support{}.
  \end{idea}

  Grant these, then we can get moving.
\end{note}

\begin{note}
  \begin{proposition}
    If \fc{} is embedded in \support{}, then plausible \requ{}
  \end{proposition}

  \begin{proposition}
    If \requ{} then instance of \qzS{} which applies to present instance of concluding.
  \end{proposition}
\end{note}

\paragraph{Puzzling}

\begin{note}
  On the one hand, aligns with intuition.
  Noting \fc{} is how the agent concludes.
  Agent has not got \(\pv{\psi}{v'}\) from \(\Psi\).

  On the other hand, this is puzzling.
  For, \support{}, relation between a pool of premises and a proposition-value pair.
  But, issue of \qzS{} is \support{}.

  So, not immediate how this proposal works.

  However, sketch a way.

  Reason, then no conclude.
  At this point, \qzS{}.
  Then, from positive answer, combined with reasoning, conclude.

  Here, the agent doesn't directly conclude \(\pv{\phi}{v}\) from \(\Phi\).

  Here, \qzS{} with respect to reasoning.
  Perhaps.

  So, the result is we get a decent account of why the agent is sensitive to \(\pv{\psi}{v'}\) from \(\Psi\) being a \fc{}, while retaining \issueConstraint{}.
\end{note}

\begin{note}
  Suggestion.
  
\end{note}

\begin{note}
  Now, the argument, it's less about \deadEnd{1}.
  More about the act of concluding.
  For, if sensitive, then still going to be an issue for concluding.
  If it's possible that this makes a difference for the agent, then if embedded, the \fc{} raises to the level of concluding.
\end{note}





\paragraph{Notes}

\begin{note}
  Now, \deadEnd{1} allow easy construction of a \requ{}.
  Why is it the case that this matters from the \agpe{}?
\end{note}

\begin{note}
  Type of \requ{1} generated are different.
  In the basic motivating cases, ability.
  It's the same reasoning.
  Here, it may be distinct way of reasoning.
  However, remains a \requ{}.
\end{note}

\begin{note}
  The problem arising from recursion is that there's no positive answer.
  And, as soon as you grant a positive answer, there's no need for the recursion.
\end{note}

\newpage

\subsection{Abstract}
\label{cha:zSpAwhy:sec:argument:abstract}

\begin{note}
  \color{red}
  Question is not about whether \(\phi\) has value \(v\).
  Question is about whether the agent would perform the act.

  So, embedding, fails to answer why performed the act.

  Assume embedding.

  Relation of support between pool of premises and conclusion.

  So, in this case, the relation is between the \fc{} and the conclusion.

  However, this can't be right.
  What is at issue is not the conclusion, but the act of concluding.

  This is the basic idea.

  Though, this is not immediate, because it's possible to embed answers.
  So, then it turns to getting the \requ{} again.
  This doesn't depend on a recursion principle, only the observation that an agent may reasonably view the \requ{} reapplying.

  What is of some interest here is that it's the same \requ{}.
  What changes is the concluding  that it applies to.
\end{note}

\begin{note}
  \begin{itemize}
  \item By assumption, positive answer to \qzS{} matters.
  \item And, by assumption, no witnessing.
  \item So, \support{} between \(\pv{\psi}{v'}\) and \(\Psi\) isn't direct.
  \item Therefore, \support{} is embedded.
  \item No assumptions about the `depth' of embedding.
  \item However, it remains the case that \qzS{} holds.
  \item The distinction used here is the problem.
  \item We've just expanded the pool.
  \item Make this a little clearer by isolating.
    \begin{itemize}
    \item \(\pvp{\psi}{v'}{\Psi}\) is a \fc{} and \(\Psi\), therefore \(\pv{\psi}{v'}\).
    \item \(\pvp{\psi}{v'}{\Psi}\) as a \requ{}.
    \item At issue is not whether \(\pv{\psi}{v'}\) follows from \(\pv{\psi}{v'}\) being a \fc{}.
    \item Rather, at issue is whether \(\pv{\psi}{v'}{\Phi}\) \emph{is} a \fc{}.
    \item Note, in this case, if you define \(\Psi = \Psi \cup{ \fc{} }\) then the relevant \requ{} is without the \fc{}.
    \end{itemize}
  \end{itemize}
\end{note}

\paragraph{As a premise\dots}

\begin{note}
  Now, how to get to \(\pv{\phi}{v}\) from this is a further issue.

  Though, I don't think it is implausible.

  For example.

  Witnessed reasoning.
  Pair with conclusion.

  In this case, conclude \(\pv{\phi}{v}\) from \(\Phi^{+}\), where \(\Phi^{+}\) includes the option.

  So, in this respect, intermediate conclusion.
  Concluding \(\pv{\phi}{v}\) from \(\Phi\) in part involves concluding that \qzS{} has a positive answer.
  I mean, further, then, the relation of support is embedded twice.
\end{note}


\paragraph{Ah!}

\begin{note}
  So, here we have the problem.
  Still have issue of \requ{}.

  Really, it's about the reasoning.

  If you grant there are cases in which \qzS{} matters, and hold to inclusion, then an additional \requ{}.
\end{note}

\subsection{Summarising}
\label{sec:summarising}

\paragraph{Recursive?}
\nocite{Besson:2018wz} % TODO: CHECK THIS FOR MORE!!!
\nocite{Wieland:2013vf} % But the passages for this aren't really helpful.

\begin{note}
  Well, to the extent \citeauthor{Carroll:1895uj} seems to be.

  For, break out of the loop at any time.
  There's nothing to prevent the Tortoise from accepting.

  However, something on the Tortoise's mind.
  Generates further hypotheticals.

  \begin{generator}[\citeauthor{Carroll:1895uj}]
    Instance of rule of inference is sound only if hypothetical is true and hypothetical without rule of inference.
  \end{generator}

  So, every time Achilles appeals to instance of rule of inference, generator a new hypothetical, and get this without the rule of inference.
  Look, the Tortoise is really clear that you can get the hypothetical without the rule of inference.

  So, hypothetical is necessary, independent, and (possibly) antecedent.%
  \footnote{
    Here might be the trouble.
    Understanding conditional just is getting rule of inference.
    But, this is not obvious.
    For, other connectives.
    It's not obvious that I get conjunction i/e from understanding conjunction.
    For, could be the case that always translate conjunction to some other connective when applying rules of inference.

    Now, still \emph{modus ponens}.
    But, in principle no need for this.

    Consider a Gentzen system for propositional logic termed \textbf{G3cp} (\cite[\S3.5]{Troelstra:2000ue},\cite[\S3.1]{Negri:2008wy}).

    In \textbf{G3c}(\textbf{p}) the two rules concerning material implication are:

    \begin{quote}
      \mbox{ }\hfill%
      \(
      \AxiomC{\(\Gamma \Rightarrow \Delta, A\)}
      \AxiomC{\(B,\Gamma \Rightarrow \Delta\)}
      \LeftLabel{L\(\rightarrow\)}
      \BinaryInfC{\(A \rightarrow B, \Gamma \Rightarrow \Delta\)}
      \DisplayProof
      \)%
      \hfill
      \(
      \AxiomC{\(A,\Gamma \Rightarrow \Delta,B\)}
      \LeftLabel{R\(\rightarrow\)}
      \UnaryInfC{\(\Gamma \Rightarrow \Delta, A \rightarrow B\)}
      \DisplayProof
      \)%
      \hfill\mbox{ }\newline
      \mbox{ }\hfill\mbox{(\citeyear[77]{Troelstra:2000ue})}
    \end{quote}
    By inspection, neither rule corresponds to \emph{modus ponens} in any direct way.
  }\(^{,}\)%
  \footnote{
    Puzzle about the general lesson from \citeauthor{Carroll:1895uj}.

    \citeauthor{Wieland:2013vf} (\citeyear{Wieland:2013vf}) characterises the general understanding of \textcite{Carroll:1895uj} in terms of two lessons:
  \begin{quote}
    [T]he negative lesson is that if you add ever more premises to an argument \dots, then you will never demonstrate that its conclusion follows logically.\newline
    \mbox{ }\hfill\mbox{(\citeyear[984]{Wieland:2013vf})}
  \end{quote}

  Parallel, static answers, still option for concluding otherwise.

  \begin{quote}
    [T]he positive lesson is that rules of inference, rather than premises of the form `if premises such and such are true, then the conclusion is true', will do the job.\newline
    \mbox{ }\hfill\mbox{(\citeyear[984]{Wieland:2013vf})}
  \end{quote}
  Positive lesson \emph{is} the puzzle.
}

\begin{quote}
  My paradox \dots turns on the fact that, in a Hypothetical, the \emph{truth} of the Protasis, the \emph{truth} of the Apodosis, \& the \emph{validity of the sequence}, are 3 distinct Propositions.
  \begin{quote}
    For instance, if I grant

    \begin{enumerate}[label=(\arabic*)]
    \item
      All men are mortal, \& Socrates is a man, but not
    \item
      The sequence “If all men are mortal and if Socrates is a man, then Socrates is mortal” is valid,
    \end{enumerate}

    then I do not grant

    \begin{enumerate}[label=(\arabic*), resume]
    \item Socrates is mortal.
    \end{enumerate}
    Again, if I grant (2), but not (1), I still fail to grant (3).

    Hence, before granting (3), I must grant (1) \& (2)\newline
    \mbox{ }\hfill\mbox{(\citeyear[10--11]{Carroll:2016wl})}
  \end{quote}
\end{quote}

  \begin{generator}[\requ{3}]
    Positive answer to only if \requ{} restated with respect to current instance of concluding.
  \end{generator}

  Every time embed in support, un-embed.
\end{note}

%%% Local Variables:
%%% mode: latex
%%% TeX-master: "master"
%%% End:
