\chapter{Positive answers and \qWhy{}}
\label{cha:zSpAwhy}

\begin{note}
  \color{red}
  Okay, to keep things straightforward.
  The point of putting this after \autoref{cha:zSpA} is so that there's no confusion with relations of support.
  For, the way in which relations of support answer \qWhy{} is under-defined.
\end{note}

\begin{note}
  We're almost done.

  \qzS{}.
  Answers to \qzS{}.

  In \autoref{cha:zSpA}, looked at positive answers to \qzS{}.
  Argued potential event.
  In particular, that \ptivity{} must hold for any positive answer.
  In other words, potential event without qualification, from the agent's perspective.

  Of course, from the agent's perspective.
  And, the agent's perspective is a qualifier.
  In particular, non-factive.
  So, does not follow that there is a potential event.
\end{note}

\begin{note}
  So, we've got how potential events are linked to answers to \qzS{}.

  Second part of the argument.
  Answers to \qzS{} and \qWhy{}.

  As with previous proposition, simple argument.
  Which, we then develop an objection to and highlight the difficulty.

  Though, here, the objection is less about the argument, but whether \qzS{} can be raised in a way which does not require `direct' answers.
\end{note}

\section{Proposition}
\label{cha:zSpAwhy:sec:proposition}

\begin{note}
  Proposition for this chapter.

  \begin{proposition}
    \label{prop:fc-answers-why}
    If:
    \begin{enumerate}
    \item \(\pvp{\psi}{v'}{\Psi}\) is a \requ{} of concluding \(\pv{\phi}{v}\) from \(\Phi\).
    \end{enumerate}
    And:
    \begin{enumerate}[resume]
    \item
      \qzS{} has positive answer.
    \end{enumerate}
    And:
    \begin{enumerate}[resume]
    \item
      Agent concludes \(\pv{\phi}{v}\) from \(\Phi\).
    \end{enumerate}
    Then:
    \begin{enumerate}[resume]
    \item
      \(\pvp{\psi}{v'}{\Psi}\) being a \fc{} (directly) answers \qWhy{}.
    \end{enumerate}
  \end{proposition}
  In other words, the potential event in which the agent concludes \(\pv{\psi}{v'}\) from \(\Psi\), in part, answers \qWhy{} the agent concluded \(\pv{\phi}{v}\) from \(\Phi\).
\end{note}

\begin{note}
  With \autoref{prop:fc-answers-why} in hand, our final task will be to observe a final proposition holds:

  \begin{proposition}
    \label{prop:fc-RoS}
    \(\pvp{\psi}{v'}{\Psi}\) is a \fc{} \emph{only if} a relation of support holds between \(\pv{\psi}{v'}\) and \(\Psi\), from the agent's perspective.
  \end{proposition}

  The argument for \autoref{prop:fc-RoS} is not so interesting.
  Detail {\color{red}???}.

  But, really, way in which relations of support are related 
\end{note}

\section{Simple argument}
\label{sec:simple-argument}

\begin{note}
  Intuitively, \autoref{prop:fc-answers-why} follows from how \qzS{} relates to concluding \(\pv{\phi}{v}\) from \(\Phi\).

  For, the way things are set up.
  \qzS{} is about the act of concluding.
  Therefore, answer \qzS{}, and in doing so, potential event answers \qWhy{}.
\end{note}

\begin{note}
  The problem with this argument is that it assumes direct relation between answering and answers to \qWhy{}.

  However, instead of direct relation, possible, instead, to get a relation of support between \fc{} and \(\pv{\phi}{v}\).
\end{note}

\begin{note}
  Task is to show that \qzS{} is not answered.
\end{note}

\section{Difficulty}
\label{cha:zSpAwhy:sec:difficulty}

\begin{note}
  However, not quite.

  Our goal is to provide an argument against\dots

  This is still not complete.
  Need to show the relation of support between \(\pv{\psi}{v'}\) and \(\Psi\) answers \qWhy{}.

  To highlight the difficulty.

  Consider knowledge.

  Here, we've argued that \(\pv{\chi}{v''}\) holds from agent's point of view.
  However, what matters, when concluding, is the relation of support between \(\pv{\chi}{v''}\) and \(\pv{\phi}{v}\).

  Whether or not \(\pv{\chi}{v''}\) in part, answers \qWhy{}.
\end{note}

\begin{note}
  Now, this is fairly delicate.

  For this objection, need to deny content also answers why.

  So, here, the question is why the agent concluded.
  This specific act.
  And, I think there's a plausible case to be made that \(\pv{\chi}{v''}\) isn't right, but rather the relation of support between \(\pv{\chi}{v''}\) and other premises, and conclusion.

  To elaborate.
  Conclusion is \(\pv{\phi}{v}\).
  Now, \(\pv{\chi}{v''}\) is in part an answer to why \(\pv{\phi}{v}\) is the case from the agent's perspective.
  However, not, in part, an answer to why agent concluded, from agent's perspective.
  For, concluding, what matters is relation of support between pool of premises containing \(\pv{\chi}{v''}\) and \(\pv{\phi}{v}\).

  Whether or not you agree, this is the worry.

  So, functions as a premise.
\end{note}

\begin{note}
  So, here's the idea.
  In cases of concluding:
  For any proposition-value pair which is answers from agent's perspective, strictly,
  if it relation of support between pool of premises containing and conclusion.
\end{note}

\begin{note}
  Applied to the results so far, relation of support between \fc{} and \(\pv{\phi}{v}\).
\end{note}

\section{Argument}
\label{cha:zSpAwhy:sec:argument}

\begin{note}
  Argument for~\autoref{prop:fc-answers-why} is abstract.

  To clarify, provide a concrete example.
\end{note}

\subsection{Abstract}
\label{cha:zSpAwhy:sec:argument:abstract}

\begin{note}
  \color{red}
  Question is not about whether \(\phi\) has value \(v\).
  Question is about whether the agent would perform the act.

  So, embedding, answers why performed the act.

  Relation of support between pool of premises and conclusion.

  So, in this case, the relation is between the \fc{} and the conclusion.

  However, this can't be right.
  What is at issue is not the conclusion, but the act of concluding.

  This is the basic idea.
\end{note}

\paragraph{\qzS{} is about the act}

\begin{note}
  First observation is that \qzS{} is about the act of concluding.

  So, in order for \fc{} to function as a premise, the relevant conclusion is something to do with the act.
\end{note}

\begin{note}
  At issue with \qzS{} is \emph{not}:

  If agent were to fail to hold that you would conclude \(\pv{\psi}{v'}\) from \(\Psi\) as \emph{a premise}.
\end{note}

\begin{note}
  Highlight this with an example.

  Ability.
  General ability.
  So, get something.
  All of this may be the case, but \requ{} is about concluding.
  Again, observation of a \deadEnd{}.
  So, something which gets potential event doesn't really matter.
\end{note}

\begin{note}
  Note, the tension is not about whether \(\phi\) has value \(v\).
  Instead, the tension is about whether the agent would have a certain property if they were to conclude \(\phi\) has value \(v\).
  Property of having claimed \support{}.
  Expanded, property of holding that any independent check is satisfied.
  Any other reasoning about whether \(\phi\) has value \(v\) would conclude \(\phi\) has value \(v\).
\end{note}

\begin{note}
  So, for the abstract argument, the basic idea is that \fc{} matters somewhere.
  This we get from link between \qzS{} and \qWhy{}.
  Note, this link doesn't tell us anything direct.

  However, by assumption, doesn't answer \qWhy{}.

  However, if this is the case, then restate the \requ{} with respect to the main question.
\end{note}

\section{Examples}
\label{sec:examples}

\subsection{Concrete}
\label{sec:concrete}

\begin{note}
  Here, instance of general ability.
  General ability.
  But, no use just as premise.
\end{note}

\subsection{Embeddings}
\label{sec:embeddings}

\paragraph{Relation of support between pool of premises and a conclusion}

\begin{note}
  \begin{observation}
    Relation of support holds between pool of premises and a conclusion.
  \end{observation}
\end{note}

\begin{note}
  Relation of support is an interesting thing.
  It holds between pool of premises and a conclusion.
  This answers \qWhy{}.

  Initial observation, then, is that some care should be given to what we get.
  There's a difference between a relation between pool and a proposition, and relation between pool of propositions and an event.

  Understanding is that relation between pool and proposition-value answers, in part, why an event took place.
\end{note}

\begin{note}
  Now, \qzS{} is about the event taking place.
  \qzS{} is not (directly) about whether \(\phi\) has value \(v\), but about the event of concluding \(\pv{\phi}{v}\) from \(\Phi\).
\end{note}

\begin{note}
  So, in a naive way, we would only get a relation of support between \fc{} and \(\pvp{\phi}{v}{\Phi}\).
\end{note}

\begin{note}
  Two different routes, depending on how relation of support works.

  As conclusion is natural, this is route A.

  Route B is act.
\end{note}

\subsubsection{Route A}
\label{sec:route-A}

\paragraph{As a premise\dots}

\begin{note}
  In order for the potential event to function as a premise, the relevant conclusion involves concluding.
  So, conclude is that the agent has the option to conclude.

  Tentative way of understanding this.
  \begin{enumerate}
  \item
    \label{tentative:option}
    Option to conclude \(\pv{\phi}{v}\) from \(\Phi^{+}\).
  \end{enumerate}
  Note, \(\Phi^{+}\) to allow for additional premises, such as \ref{tentative:option}.
\end{note}

\begin{note}
  Now, how to get to \(\pv{\phi}{v}\) from this is a further issue.

  Though, I don't think it is implausible.

  For example.

  Witnessed reasoning.
  Pair with conclusion.

  In this case, conclude \(\pv{\phi}{v}\) from \(\Phi^{+}\), where \(\Phi^{+}\) includes the option.

  So, in this respect, intermediate conclusion.
  Concluding \(\pv{\phi}{v}\) from \(\Phi\) in part involves concluding that \qzS{} has a positive answer.
  I mean, further, then, the relation of support is embedded twice.
\end{note}

\begin{note}
  Focus is on~\ref{tentative:option}.
\end{note}

\begin{note}[Splitting up reasoning]
  Now, consider the \requ{}, but here it's whether conclude \(\pv{\phi}{v}\) from \(\Phi\).

  So, then, it's clear that the agent must have witnessed reasoning, I think.

  But, what is the status of this reasoning?

  I mean, here, we've got an act.
  Suitably related to concluding.
  So, \requ{} holds for this.

  {
    \color{red}
    This complicates things too much.
    It's much simpler to observe that the \requ{} applies to the option to conclude in the same way as it applies to the conclusion.

    This is the way in which \requ{1} are sort of recursive.
  }

  (Okay, this is kind of cool.)

  If concluded, and so \requ{} still holds.

  So, weaker.
  Well, weaker, as only require witnessed reasoning.
  However, I think just repeat the observation.
  Look, technically, but hollow.
  If grant \requ{}, then a variant is going to hold, as the witnessed reasoning needs to be good enough to stand in for concluding.
  And, at issue with a \requ{} is whether such reasoning is good enough.
\end{note}

\paragraph{Ah!}

\begin{note}
  So, here we have the problem.
  Still have issue of \requ{}.

  Really, it's about the reasoning.

  If you grant there are cases in which \qzS{} matters, and hold to inclusion, then an additional \requ{}.
\end{note}

\begin{note}
  Viewed all of this from the perspective of positive answers.
  And, as we're looking for a `counterexample', positive answers are of interest.

  Still, negative answers.
  Whether this makes sense.
\end{note}

\subsubsection{Route B}
\label{sec:route-B}

\begin{note}
  Route B, then relation of support holds between pool of premises and act of concluding.

  Now, same observation from route A.
  In this case, \fc{} holds in support.
  However, \qzS{} remains.
\end{note}

\begin{note}
  So, this route B is very similar to route A.
  The distinction is that we don't need to include link to \(\pv{\phi}{v}\).
  And, in this respect, it is cleaner.

  However, in order to avoid collapsing, it is not at all clear.
\end{note}

\subsection{Summarising}
\label{sec:summarising}

\paragraph{Content}

\begin{note}
  With initial cases, focus of \requ{} was on checks.
  This remains the case.
  Still, difference.

  For, earlier cases, the other reasoning was typically something else.
  Recall, for example, quadratic formula and factorisation({\color{red} ?}).

  Here, the relevant \qzS{} is reflexive.
  It's about the content.

  If embed, then restate \requ{} with respect to main event (of concluding).

  Due to the nature of \qzS{}, what we're asking about is an act.
\end{note}

\paragraph{A route B worry}

\begin{note}
  Here, the argument progresses, and if the proposal is right, then plausible that there's going to be a relation of support.

  This is fine.
  Though it is somewhat redundant.
\end{note}

\paragraph{Recursive?}

\begin{note}
  Well, to the extent \citeauthor{Carroll:1895uj} seems to be.

  For, break out of the loop at any time.
  There's nothing to prevent the Tortoise from accepting.
\end{note}





%%% Local Variables:
%%% mode: latex
%%% TeX-master: "master"
%%% End:
