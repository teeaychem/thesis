\chapter{Concluding}
\label{chapter:concluding}

\paragraph{Safety and sensitivity}

\begin{note}[Safety and sensitivity]
  {
    \color{red}
    Perhaps good to mention here that \zS{} doesn't have this property either.
  }
  We borrow the following definitions from~\citeauthor{Zalabardo:2017td}:

  \begin{quote}
    S's belief that p is \emph{safe} just in case, if S believed p, p would be true.\newline
    \mbox{}\hfill\mbox{(\citeyear[1]{Zalabardo:2017td})}
  \end{quote}

  \begin{quote}
    S's belief that p is \emph{sensitive} just in case, if p were false, S wouldn't believe p.\nolinebreak
    \mbox{}\hfill\mbox{(\citeyear[2]{Zalabardo:2017td})}
  \end{quote}

  In both definitions, `S's \support{} for p (being true)' may be substituted for `S's belief that p'.

  Further, both definitions consist of a subjunctive conditional which concerns the value that p has.
  In the case of safety, p has the value true, and in the case of sensitivity, p is false.
  And, indeed, it follows from either definition that if S believes that p then p is true.\nolinebreak
  \footnote{
    In the case of safety, this is immediate.
    In the case of sensitivity, if S believes that p and p is false, then it follows by sensitivity that S doesn't believe p.
    Hence, either S believes that p and p is true, or S does not believe that p.

    Note, however, that while both safety and sensitivity ensure that S believes that p only if p is true, the two conditions are distinguished by which possibilities the relevant subjunctive antecedent quantifiers over.
  }
\end{note}

\paragraph{Conclusions are determined}

\begin{note}[Conclusions are determined]
  \begin{assumption}[Conclusions are determined]
    \label{assu:conc:determined}
    For any value type \(v_{\tau}\):

    If an agent has the option of concluding \(\pv{\phi}{v_{\tau}}\), then the agent does not have the option of concluding \(\pv{\phi}{\overline{v_{\tau}}}\).
  \end{assumption}

  In other words, there are no cases where an agent may choose between concluding \(\phi\) has value \(v\) and concluding \(\phi\) has some value other than \(v\) of the same time.

  For example, if an agent has the option of concluding \(\phi\) is true, then the agent does not have the option of concluding \(\phi\) is false.
  Likewise, if the agent has the option of concluding \(\phi\) is undesirable, then the agent does not have the option of concluding \(\phi\) is desirable.
  However, as truth and desirability are distinct value types, an agent may conclude, e.g., \(\phi\) is false but desirable.
\end{note}

\begin{note}
  \Autoref{assu:conc:determined} is delicate.
  If the agent has the option, then it's determined.
  So, if the agent has concluded, then has the option.
  However, this does not guarantee that prior to reasoning the agent has the option.
  May be that in concluding, change in epistemic state, such that either \(\pv{\phi}{v}\) or \(\pv{\phi}{\overline{v}}\) is determined.

  For example, favourite flavour.
  The agent does not have the option of concluding they have a preference.
  However, conclude that X is favourite flavour.
  Now the agent has the option, and the agent does not have the option of concluding Y.
\end{note}

\begin{note}
  Though \autoref{assu:conc:determined} is delicate, will be important.
  However, key present epistemic state in mind.
\end{note}

\begin{note}
  We do not make any explicit assumptions about relations between value types, though we take for granted sensible constraints.
  For example, if an agent has the option of concluding \(\phi\) is true, the agent does not have the option of concluding \(\phi\) is impossible, though whether some proposition is possible or impossible is distinct from whether the proposition is true or false.
\end{note}

\begin{note}
  This is weaker than voluntarism.
  An agent may choose whether or not to conclude.
  And, may choose to whether to retract some previous conclusion.
\end{note}

%%% Local Variables:
%%% mode: latex
%%% TeX-master: "master"
%%% End: