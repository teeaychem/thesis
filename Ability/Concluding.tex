\chapter{Concluding}
\label{chapter:concluding}

\begin{note}
  Concluding has a central role in the main body of this document.

  Key assumption made is relation of support.

  This chapter details various other assumptions about concluding that have either an implicit or explicit role in the argument presented.
\end{note}

\section{Overview}
\label{sec:concluding:overview}

\begin{note}[Overview]
  Concluding.
  This is the most basic thing.
  Lay out assumptions as clearly as possible.

  Generally speaking, assumption does two things.
  What concluding is, what concluding is not.
  Assumptions will not provide an exhaustive account of what concluding is and what concluding is not, but sufficient for a main goal.

  Indeed, goal is a substantial result on what some instances of what concluding is.
  And, in turn, a substantial result on what concluding, in general, is not.

  And, as concluding instance of reasoning, the previous two points expand to reasoning in general.
\end{note}

\begin{note}[List of assumptions]
  Ideally, find some way to generate a toc for this section\dots
\end{note}


\paragraph*{Values and propositions}

\begin{note}[Value proposition]
  Reasoning and claims to support focus.
  Briefly introduce a pair of propositions to clarify claim to support and reasoning.

  \begin{restatable}[Claimed support is for a proposition having some value]{assumption}{assuCSVP}
    \label{assu:CSVP}
    When an agent concludes \(\phi\) has value \(v\), the agent assigns value \(v\) to the \world{1} described by \(\phi\).
    (Where propositions individuate \world{1} from the perspective of the agent.)
  \end{restatable}

  \autoref{assu:CSVP} fixes terminology.
  To illustrate, when stating the conclusion of the reasoning sketched above we used the proposition that \emph{the area of the rectangle is \(133\text{cm}^{2}\)}.
  The proposition refers to the \world{} in which the area of the rectangle is \(133\text{cm}^{2}\), and speaking a little more precisely, the agent claimed that the proposition has the value `true' --- though it may be the value turns out to be `false'.
  Or, perhaps if the agent was a little unsure about the accuracy of the ruler, that the proposition has the value `likely', `probable', or some quantitative credence.
  And, some other instance of reasoning may have concluded that the proposition has the value `desirable' --- e.g.\ if the agent was searching for a rectangle of some approximate size.
\end{note}

\begin{note}[Notation]
  \begin{notation}[Proposition-value pair]
    Proposition-value pair, abbreviate \(\pv{\phi}{v}\).
  \end{notation}
\end{note}

\begin{note}
  Core idea is that claim of support is that the \world{} is a certain way.
  Proposition, what the thing is.
  Value, the way it is.

  A handful of instances:
  \begin{itemize}
  \item \(p\) is assigned the value `true'. \hfill (\emph{p} is true.)
  \item \(p\) is assigned the value `ought to be'. \hfill (\emph{p} ought to be the case.)
  \item \(p\) is assigned the value `desirable'. \hfill (\emph{p} is desirable.)
  \item \(p\) is assigned the value `improbable'. \hfill (\emph{p} is improbable.)
  \end{itemize}
\end{note}

\begin{note}
In most cases the value will be clear (i.e. that the proposition is true, though sometimes that the proposition is desirable), and so we will talk of claiming support for the proposition.
  A handful of additional examples will be provided when illustrating the next proposition.
\end{note}

\begin{note}
  Nothing hangs on distinction between values.
  Reduce everything to truth and falsity.
  However, we do not assume this, and if you do not think this is the case either, then I would like to not suggest that the assumptions and arguments to follow concern only those propositions which may be evaluated as true or false.
\end{note}

\paragraph*{Premises are not assumptions}

\begin{note}
  By premises, we mean inputs of reasoning.
  These are not necessarily assumptions, though the may be.
  If conclude \(x\) from \(y\), then I am not necessarily concluding \(x\) given \(y\).
  Rather, by appealing to \(y\) I get \(x\).

  Possible to go for meal this evening.
  Need to pay.
  Sufficient bank balance.

  Without bank balance, would not conclude.
  However, not assuming sufficient bank balance.
  Rather, observing that I do have sufficient bank balance.
\end{note}

\paragraph*{Premises}

\begin{note}
  \begin{restatable}[Concluding from premises]{assumption}{assuCP}
    \label{assu:C-E-premises}
    An agent concludes \(\phi\) has value \(v\) just when \(\phi\) having value \(v\) is supported by some pool of premises \(\Phi\).
  \end{restatable}
\end{note}

\begin{note}[Notation]
  \begin{notation}[Proposition-value-premise pairing]
    Proposition-value-premise pairing, abbreviate \(\pvp{\phi}{v}{\Phi}\).
  \end{notation}
  Have seen \(\Phi\) supports \(\pv{\phi}{v}\).
  Relation of support applies to \(\pvp{\phi}{v}{\Phi}\).

  However, more general use.
\end{note}

\begin{note}
  The role of~\autoref{assu:C-E-premises} is primarily to ensure that concluding always guarantee the existence of premises and steps.
  With the exception of some broad constraints to be outlined in further assumptions below, we (will) have little to say about the specifics of what the reasoning consists of.
\end{note}

\begin{note}[Quick examples]
  \begin{itemize}[noitemsep]
  \item \(S\) testified that \(p\), so \(p\) is true.
  \item \(p\) would satisfy every member of the group, so \(p\) ought to be the case.
  \item The song is produced by \(S\), so it is desirable that I listen to it.
  \item The device reads \(p\) and is reliable, so \emph{not}-\(p\) is improbable.
  \end{itemize}
\end{note}

\begin{note}
  Instances of reasoning may culminate in other ways, so we are only interested in a specific type of reasoning.
\end{note}

\begin{note}[Understanding `having value \(v\)']
  In a deductive case, if the premises are true, then the conclusion is true.
  Means-end reasoning for desire.
  The value is important.
  If it is true that it past 6pm, then it is true the shop is closed.
  Provides value of shop being closed.

  However, if agent desires that it is past 6pm, then it doesn't follow that the agent desires that the shop is closed.
  Question an agent as to why they think their desires conform to truth --- is-ought problem.

  Means-end reasoning.
  It is true that there is cheese at the centre of the maze.
  And, it is desirable that I obtain the cheese at the centre of the maze.
  Further, it is true that I may only obtain the cheese at the centre of the maze by solving the maze.
  Therefore, it is desirable that I solve the maze.
\end{note}

\begin{note}[`Concluding']
  Rather, \(\phi\) \emph{has} value \(v\).

  So, from the perspective of an agent, `I conclude \(\phi\) has value \(v\)' and \(\phi\) has value \(v\) are interchangeable.
\end{note}

\begin{note}[Tied to premises]
  What does matter is role of premises.
  From some conclude \(\phi\) has value \(v\), from others conclude \(\phi\) does not have value \(v\).
\end{note}

\begin{note}[Generally]
  Concluding \(\phi\) has value \(v\).
  \(\phi\) having value \(v\) follows from premises.
  These premises, and therefore \(\phi\) has value \(v\).

  The `follow' or `therefore' may be deductive or non-deductive.
\end{note}

\paragraph*{Three distinct options}

\begin{assumption}
  Fix \(\phi\), \(v\), and \(\Phi\).
  Three distinct options:
  \begin{enumerate}
  \item Conclude \(\phi\) has value \(v\) from \(\Phi\).
    \(\pvp{\phi}{v}{\Phi}\).
  \item
    Conclude \(\phi\) has some value other than \(v\) from \(\Phi\).
    \(\pvp{\phi}{\overline{v}}{\Phi}\).
  \item
    Fail to conclude whether \(\phi\) has value \(v\) from \(\Phi\).
    \(\pvp{\phi}{?}{\Phi}\).
  \end{enumerate}
  The second may be further distinguished.
  That \(\phi\) has some other value, and some specific value.
  We have no need for such a finer grained distinction.
\end{assumption}



\paragraph*{Conclusing is not (necessarily) factive}
\label{concluding:not-factive}

\begin{note}
  Page~\pageref{mention:concluding-non-factive}, mentioned concluding \(\phi\) has value \(v\) doesn't matter if \(\phi\) has value \(v\).

  Brief \illu{0}.
\end{note}

\begin{note}
  \begin{illustration}
    Agent concludes \(0.999\dots \ne 1\), where the agent holds themselves to have a conventional understanding of real numbers.
  \end{illustration}
  The qualification is important, there are various interpretations under which \(0.999\dots \ne 1\), but it is convention that the Archimedean property holds for real numbers.

  Still, an agent may reason that even if \(0.999\dots = 1\), there must be \emph{some} difference between \(0.999\dots\) and \(1\) --- no matter how small --- and some difference between to things is sufficient to establish that they are not equal.

  Expanding: \(0.9 = (1 - 0.1)\), \(0.99 = (1 - 0.01)\), and so \(0.999\dots = (1 - 0.000\dots 1)\), hence \(1 = (0.999\dots + 0.000\dots 1)\), and because \(0.999\dots\) refers to some quantity, \(0.000\dots 1\) likewise refers to some quantity.

  In other worlds, an agent need not consider it \epPAd{} that the Archimedean property does not hold for real numbers.
  Of course, including either safety or sensitivity as additional constraints on \support{} would rule out this illustration, we do not require such constraints.
\end{note}

\begin{note}[Safety and sensitivity]
  {
    \color{red}
    Perhaps good to mention here that \zS{} doesn't have this property either.
  }
  We borrow the following definitions from~\citeauthor{Zalabardo:2017td}:

  \begin{quote}
    S's belief that p is \emph{safe} just in case, if S believed p, p would be true.\newline
    \mbox{}\hfill\mbox{(\citeyear[1]{Zalabardo:2017td})}
  \end{quote}

  \begin{quote}
    S's belief that p is \emph{sensitive} just in case, if p were false, S wouldn't believe p.\nolinebreak
    \mbox{}\hfill\mbox{(\citeyear[2]{Zalabardo:2017td})}
  \end{quote}

  In both definitions, `S's \support{} for p (being true)' may be substituted for `S's belief that p'.

  Further, both definitions consist of a subjunctive conditional which concerns the value that p has.
  In the case of safety, p has the value true, and in the case of sensitivity, p is false.
  And, indeed, it follows from either definition that if S believes that p then p is true.\nolinebreak
  \footnote{
    In the case of safety, this is immediate.
    In the case of sensitivity, if S believes that p and p is false, then it follows by sensitivity that S doesn't believe p.
    Hence, either S believes that p and p is true, or S does not believe that p.

    Note, however, that while both safety and sensitivity ensure that S believes that p only if p is true, the two conditions are distinguished by which possibilities the relevant subjunctive antecedent quantifiers over.
  }
\end{note}

\paragraph*{Conclusions are determined}

\begin{note}[Conclusions are determined]
  \begin{assumption}[Conclusions are determined]
    \label{assu:conc:determined}
    For any value type \(v_{\tau}\):

    If an agent has the option of concluding \(\pv{\phi}{v_{\tau}}\), then the agent does not have the option of concluding \(\pv{\phi}{\overline{v_{\tau}}}\).
  \end{assumption}

  In other words, there are no cases where an agent may choose between concluding \(\phi\) has value \(v\) and concluding \(\phi\) has some value other than \(v\) of the same time.

  For example, if an agent has the option of concluding \(\phi\) is true, then the agent does not have the option of concluding \(\phi\) is false.
  Likewise, if the agent has the option of concluding \(\phi\) is undesirable, then the agent does not have the option of concluding \(\phi\) is desirable.
  However, as truth and desirability are distinct value types, an agent may conclude, e.g., \(\phi\) is false but desirable.
\end{note}

\begin{note}
  \Autoref{assu:conc:determined} is delicate.
  If the agent has the option, then it's determined.
  So, if the agent has concluded, then has the option.
  However, this does not guarantee that prior to reasoning the agent has the option.
  May be that in concluding, change in epistemic state, such that either \(\pv{\phi}{v}\) or \(\pv{\phi}{\overline{v}}\) is determined.

  For example, favourite flavour.
  The agent does not have the option of concluding they have a preference.
  However, conclude that X is favourite flavour.
  Now the agent has the option, and the agent does not have the option of concluding Y.
\end{note}

\begin{note}
  Though \autoref{assu:conc:determined} is delicate, will be important.
  However, key present epistemic state in mind.
\end{note}

\begin{note}
  We do not make any explicit assumptions about relations between value types, though we take for granted sensible constraints.
  For example, if an agent has the option of concluding \(\phi\) is true, the agent does not have the option of concluding \(\phi\) is impossible, though whether some proposition is possible or impossible is distinct from whether the proposition is true or false.
\end{note}

\begin{note}
  This is weaker than voluntarism.
  An agent may choose whether or not to conclude.
  And, may choose to whether to retract some previous conclusion.
\end{note}

\paragraph*{Concluding is `description-free'}

\begin{note}[Descriptions]
  An important assumption is that an agent need not recognise that the culmination of some instance of reasoning is that some proposition has some value.

  \begin{assumption}[Concluding is `description-free']
    \label{assu:conc:d-free}
    It is not the case that an agent concludes \(\phi\) has value \(v\) only if the agent concludes \(\phi\) has value \(v\) under some description \emph{d}
  \end{assumption}

  In particular,~\autoref{assu:conc:d-free} holds for any description \emph{d} which includes an intensional reading of `\(\phi\) has value \(v\)'.
  More generally, it is possible for an agent to conclude \(\pv{\phi}{v}\) without consciously or otherwise entertaining either `\(\phi\)' or `\(v\)'.%
  \footnote{
    Compare with, for example, \citeauthor{Anscombe:1957aa} on intention action (\citeyear[\S19]{Anscombe:1957aa}) and \citeauthor{Davidson:1963aa} on primary reasons (\citeyear[5]{Davidson:1963aa}).
  }

  For the moment we will focus on intensionality.
  We will briefly explain the distinction between intensional and non-intensional readings, and motivate~\autoref{assu:conc:d-free} with respect to a handful of examples.

  Note, however, we are not providing an analysis of what it is for an agent to conclude \(\pv{\phi}{v}\).
  Rather,~\autoref{assu:conc:d-free} narrows down the particular sense of `concluding' of interest to us.
  There may be, and plausibly is, a sense of `concluding' for which~\autoref{assu:conc:d-free} does not hold (and in particular where the proposition-value pair of the conclusion is always intensional).
  However, our interest is with a sense of `concluding' for which~\autoref{assu:conc:d-free} holds.
\end{note}

\begin{note}[Looking ahead]
  Looking ahead, briefly,~\autoref{assu:conc:d-free} is a key assumption for developing tension.
  Tension will not follow from any reading of `concluding' in which in concluding \(\pv{\phi}{v}\) an agent concludes \(\pv{\phi}{v}\), and \(\pv{\phi}{v}\) alone.
  Rather, the tension we develop will involve an agent concluding \(\pv{\psi}{v'}\) when concluding \(\pv{\phi}{v}\).

  Now, our discussion of intensionality will suggest an agent may conclude \(\pv{\phi}{v}\) when concluding \(\pv{\varphi}{v}\) when there is significant overlap with what \(\phi\) and \(\varphi\) refer to.
  Still,~\autoref{assu:conc:d-free} allows that in concluding \(\pv{\phi}{v}\) an agent may also conclude \(\pv{\psi}{v}\), where there is no overlap between the reference of \(\phi\) and \(\psi\).
  When developing tension we will have interest with sufficient overlap with what \(\phi\) and \(\varphi\) refer to.
  So, we will not require~\autoref{assu:conc:d-free} in full generality, but equally it is only after developing the tension of interest that we will see in how~\autoref{assu:conc:d-free} may be restricted.
\end{note}

\begin{note}[Intensionality]
  Consider the following observation from~\citeauthor{Quine:1943vf}:

  \begin{quote}
    \begin{enumerate}[label=(\arabic*)]
    \item
      Giorgione = Barbarelli,
    \item
      Giorgione was so-called because of his size
    \end{enumerate}
    are true; however, replacement of the name 'Giorgione' by the name 'Barbarelli' turns (2) into the falsehood:

    \begin{center}
      Barbarelli was so-called because of his size.
    \end{center}
    \vspace{-\baselineskip}
    \mbox{ }%
    \mbox{}\hfill\mbox{(\citeyear[113]{Quine:1943vf})}
  \end{quote}

  Both `Giorgione was so-called because of his size' and `Barbarelli was so-called because of his size' are intensional in the sense that the truth value of each expression does not reduce to the reference of the expression's' components.
  Else, as `Giorgionee' and `Barbarelli' are co-referential, the predicate `was so-called because of his size' would apply equally to both `Giorgionee' and `Barbarelli'.

  Now, it is also not the case that in concluding `Giorgione was so-called because of his size', one also concludes `Barbarelli was so-called because of his size'.

  However, this observation is \emph{not} immediate from observing that the agent concluded `Giorgione was so-called because of his size' and did not conclude `Barbarelli was so-called because of his size'.

  The conclusion `Giorgione was so-called because of his size' may be intensional, but being intensional is not a property granted to some proposition-value pair by virtue of the proposition-value pair being a conclusion.

  For example, in concluding `Giorgione is large' the agent may also conclude `Barbarelli is large'.
  Indeed, the expression `Giorgione is large' may be read non-intensionally, and hence is true if and only if `Barbarelli is large', given that `Giorgionee' and `Barbarelli' are co-referential.

  Likewise, in concluding `\(2 + 2 = 4\)', an agent may also conclude `\(4 = 2 + 2\)'.
  Or, in concluding `\nagent{1} is shorter than \nagent{2}', an agent may also conclude `\nagent{2} is taller than \nagent{1}'.
  Though, in concluding `\nagent{1} is shorter than \nagent{2}' an agent may also fail to conclude `\nagent{2} is taller than \nagent{1}'.
  For example, if the relevant agent is not aware of the relationship between `shorter' and `taller'.
\end{note}

\begin{note}
  More broadly, I consider it intuitive that concluding any one of the following includes concluding any other:
  \begin{itemize}
  \item \(\phi\) has value \(v\).
  \item It is true that \(\phi\) has value \(v\).
  \item It is not the case that \(\phi\) does not have value \(v\).
  \item It is true that it is not the case that \(\phi\) does not have value \(v\).
    \begin{center}
      \(\vdots\)
    \end{center}
  \end{itemize}

  There are an infinite number of distinct proposition-value pairs that may be generated along these lines, but these distinct proposition-value pairs do not amount to distinction conclusions.
  A conclusion for one is a conclusion for all.
\end{note}

\begin{note}
  Indeed, we may form an explicit assumption governing certain proposition-value pairs.
  We start with the definition of \indicateN{}.
\end{note}

\begin{note}
  \begin{restatable}[\indicateN{2}]{definition}{defIndicate}
    \label{def:indication}
    \(\phi\) having value \(v\) \emph{\indicateV{1}} \(\psi\) has value \(v'\) if and only if:
    \begin{itemize}
    \item
      It is not \epPAd{}, from \vAgent{}' epistemic state, that \(\psi\) has value \(v'\) while \(\psi\) does not have value \(v'\).
    \end{itemize}
    \vspace{-\baselineskip}
  \end{restatable}
\end{note}

\begin{note}
  The assumption now holds that an agent concludes \(\pv{\psi}{v'}\) when concluding \(\pv{\phi}{v}\) just in case \(\pv{\psi}{v'}\) and \(\pv{\phi}{v}\) co-\indicateV{}.

  \begin{restatable}[\indicateN{2}]{assumption}{assuIndicate}
    \label{assu:indication}
    If \(\pv{\phi}{v}\) \indicateV{1} \(\pv{\psi}{v'}\), and \(\pv{\psi}{v'}\) \indicateV{1} \(\pv{\phi}{v}\), then:

    \begin{itemize}
    \item
      \vAgent{} concludes \(\pv{\phi}{v}\) just in case \vAgent{} concludes \(\pv{\psi}{v'}\).
    \end{itemize}
    \vspace{-\baselineskip}
  \end{restatable}
\end{note}

\begin{note}
  \Autoref{assu:indication} is mild closure condition on claiming support.
  Still, \autoref{assu:indication} is only a closure condition with respect to an agent's epistemic state.
  To illustrate:
  Suppose an agent has concluded that The Scarlet Pimpernel rescued Marquis de Lafayette.
  As `The Scarlet Pimpernel' and `Sir Percy Blakeney' are co-referential, it may be that the agent's conclusion \indicateV{1} that Sir Percy Blakeney' rescued Marquis de Lafayette.
  However, the conclusion will \indicateN{0} \emph{only if} there are no \epPW{1} in which `The Scarlet Pimpernel' and `Sir Percy Blakeney' refer to different individuals.

  Likewise, an agent may conclude that helping The Scarlet Pimpernel is desirable, without the conclusion \indicatePr{} that helping Sir Percy Blakeney is desirable.

  Indeed, an agent's conclusion that it is raining, while standing in Gower Street, may fail to \indicateN{} that it is raining in London if the agent considers it \epPAd{} that they are not in London.
\end{note}

\begin{note}
  Generalising,~\autoref{assu:indication} may be strengthened by weakening the restriction to `If \(\pv{\phi}{v}\) \indicateV{1} \(\pv{\psi}{v'}\)'.%
  \footnote{
    I.e.\ only the first conjunct of the restriction given.
  }

  Hence, in concluding \(\pv{\phi}{v}\) an agent would also conclude any proposition-value pair \(\pv{\psi}{v'}\) \emph{weaker} than \(\pv{\phi}{v}\), from the agent's epistemic state.
  Indeed, this leads to a much stronger closure condition on concluding.%
  \footnote{
    Consider by parallel closure of knowledge under known entailment:
    \begin{itemize}
    \item If an agent knows that \(\phi\) has value \(v\) only when \(\psi\) has value \(v'\), then if the agent knows \(\phi\) has value \(v\), then the agent knows \(\psi\) has value \(v'\).
    \end{itemize}
    This closure condition differs in forms, as it concerns knowledge as a state, by may be reformulated to a closer parallel:
        \begin{itemize}
    \item If an agent knows that \(\phi\) has value \(v\) only when \(\psi\) has value \(v'\), then in coming to know \(\phi\) has value \(v\) the agent comes to know \(\psi\) has value \(v'\).
    \end{itemize}
  }

  We will not assume this stronger variant of~\autoref{assu:indication} holds for the sense of `concluding' we are interested in.
  Rather, we have seen how~\autoref{assu:conc:d-free} allows for the possibility of an agent concluding \(\pv{\psi}{v'}\) when concluding \(\pv{\phi}{v}\), and with the exception of~\autoref{assu:indication} which we take to be sufficiently intuitive, we only advance argument in cases of interest.
\end{note}

\begin{note}
  Indeed, though the stronger variant of~\autoref{assu:indication} may be intuitive, there are certain issues we would like to avoid taking a stance on.
  In particular, whether concluding some statement which quantifiers over various objects includes concluding for each object the quantifier applies to.

  For example, suppose I have conclude that there are infinitely many primes.
  Reflecting a little on the natural numbers, I observe that if there are infinitely many primes, then for every natural number \(n\) there is some prime larger than \(n\) (for, the natural numbers are not dense).
  Hence, for every natural number \(n\) there is some prime larger than \(n\).

  On the stronger variant of~\autoref{assu:indication}, concluding `for every natural number \(n\) there is some prime larger than \(n\)' would also include concluding there is some prime larger than \(n\) for each \(n\).
  E.g.\ there is some prime larger than \(1\), \(16\), \(5^{43}\), \(53!^{793}\), \(54!^{794!}\), and so on\dots

  Specifically, looking ahead to tension, concluding that one has the general ability to witness some kind of reasoning would involve concluding that one has each specific instance of the general ability.
\end{note}

\begin{note}[Witnessing]
    While~\autoref{assu:conc:d-free} focuses on concluding, we take~\autoref{assu:conc:d-free} to apply equally to witnessing reasoning in which an agent concludes.
  Indeed, if a negative resolution to {\color{red} issue:Main} then any instance of concluding is also an instance of witnessing reasoning to the relevant conclusion, and hence~\autoref{assu:conc:d-free} would be in conflict with an assumption which states that an agent only witnesses reasoning which concludes \(\pv{\phi}{v}\) under some description.
  On the other hand, a positive resolution to {\color{red} issue:Main} does not ensure any instance of concluding is also an instance of witnessing reasoning to the relevant conclusion.
  However, we arrive at the same conflict with respect to any instance of concluding which is the result of witnessing reasoning to the relevant conclusion.

  Indeed, if \(\pv{\phi}{v}\) \indicatePr{} \(\pv{\psi}{v'}\), then \(\phi\) has value \(v\) \emph{only if} \(\psi\) has value \(v'\) (from the perspective of the agent's epistemic state).
  Hence, \emph{if} in concluding \(\pv{\phi}{v}\) an agent also concludes some \indicateVed{} \(\pv{\psi}{v'}\), then the relevant premises which allow the agent to also conclude \(\pv{\psi}{v'}\).
\end{note}

\begin{note}[Perspective on issue]
  Looking ahead, perspective on issue.
  In some cases, concluding one would conclude \(\pv{\phi}{v}\) from some premises \(\Phi\) is equivalent to concluding \(\pv{\phi}{v}\) from \(\Phi'\), where \(\Phi\) and \(\Phi'\) are distinct.
\end{note}

\paragraph*{Summary}

\begin{note}[Summary]
  Handful of assumptions regarding concluding.

  For the most part, I take these to be straightforward.

  Suspect,~\autoref{assu:conc:d-free} is of interest.
  Again, while there may be a sense of `concluding' for which~\autoref{assu:conc:d-free} does not apply, there is a sense of `concluding' for which~\autoref{assu:conc:d-free} does apply.
  This sense of `concluding'.

  However, I suspect that this does not impact interest.
  For, if recognise conclude in latter sense, then no issue concluding in former sense.
  Still, this is additional argument.
  And, would rather avoid issues regarding phenomenology of concluding.
  Well, do avoid such issues.
\end{note}


%%% Local Variables:
%%% mode: latex
%%% TeX-master: "master"
%%% End: