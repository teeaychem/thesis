\chapter{\requ{3}}
\label{cha:requs}

\begin{note}
  \autoref{cha:fcs} introduced \fc{1}.
  The present chapter links \fc{1} to concluding.

  State what \requ{1} are.
  Provide \illu{1}.
\end{note}

\begin{note}
  With the definition of a \requ{} in hand, the remainder of the chapter mostly consists of \illu{1} of \requ{1}.

  \requ{3} link \fc{1} to concluding, and whether or not an agent is concluding is involved in whether or not an agent concludes.
  And, important function in generating counterexamples to \issueConstraint{}.

  However, the existence of \requ{1} does not presuppose counterexamples to \issueConstraint{}.
  In particular, though all of the \illu{1} involve \fc{1}, the \illu{1} are constructed so that the agent has a \wit{1} for any \ros{} which follows from the agent's knowledge.

  We will turn to counterexamples to \issueConstraint{} in \autoref{cha:ces}, after explicitly linking \requ{1} to \qWhyV{} in \autoref{cha:binding}.
\end{note}

\section{\pinf{2} and \ninf{0}}

\begin{note}
  There are instances in which an agent has \emph{\ninf{0}} over whether or not they are concluding \(\pv{\phi}{v}\) from \(\Phi\).
  For, concluding is an activity performed by an agent, and in certain cases the agent may choose to stop the activity.

  For example, may decide that a result, even if obtained, is not worth the effort.%
  \footnote{
    Ex.\ bonus problem on a homework.
  }
  Indeed, this observation is not different to the observation that an agent has \ninf{0} over whether or not they are going to buy some eggs.
  For, the agent may choose to turn back home.

  To say an agent has \ninf{0} is not to say the agent has \emph{\pinf{}}.
  For, an agent may not have choice over whether an event develops into an event in which the agent concludes \(\pv{\phi}{v}\) from \(\Phi\).

  For example, take \(\chi\) to be the proposition `the area of a unit square is equal to the area of a unit circle'.
  An agent has \ninf{0} over whether or not they concluding \(\pv{\chi}{\text{True}}\), as the agent may choose not to make any attempt.
  However, assuming the agent would only conclude \(\pv{\chi}{\text{True}}\) from principles consistent with Euclidean geometry, then agent does not have \pinf{} over whether or not they are concluding \(\pv{\chi}{\text{True}}\).
  For, the area of a unit square is not equal to the area of a unit circle.
  Hence, it is not possible for the agent to ensure that some event may develop into an event in which they conclude \(\pv{\chi}{\text{True}}\).

  In parallel, an agent need not have \pinf{0} over whether or not they are going to buy some eggs.
  For, it may be there are no eggs for sale.
  Hence, it may not be possible for the agent to ensure that the event develops into an event in which the agent buys some eggs.
\end{note}


\section{\requ{3}}
\label{cha:requs:sec:definition}

\begin{note}
  \begin{definition}[A \requ{0}]
    \label{def:requ}
    \begin{itemize*}[noitemsep, label=\(\circ\)]
    \item
      An agent: \vAgent{}
    \item
      Propositions: \(\phi\), \(\psi\)
    \item
      Values: \(v\), \(v'\)
    \item
      \poP{3}: \(\Phi\), \(\Psi\)
    \item
      An event: \(e\)
    \item
      \mbox{ }
    \end{itemize*}

    \begin{itemize}
    \item
      \(\pvp{\psi}{v'}{\Psi}\) is a \emph{\requ{}} of \(e\), with respect to \(\pvp{\phi}{v}{\Phi}\).
    \end{itemize}

    \emph{If and only if}

    \begin{itemize}
    \item
      The following conditional is true:
      \begin{itemize}
      \item[\emph{If}:]
        \begin{enumerate}[label=\alph*., ref=(\alph*), series=requDefSeries]
        \item
          \label{def:requ:nK}
          \vAgent{} does not know \(\pv{\psi}{v'}\) from \(\Psi\) is a \fc{} throughout \(e\).
        \end{enumerate}
      \item[\emph{Then}:]
        \begin{enumerate}[label=\alph*., ref=(\alph*), resume*=requDefSeries]
        \item
          \label{def:requ:nC}
          \(e\) is not an event in which \vAgent{} is concluding \(\pv{\phi}{v}\) from \(\Phi\), due to \ref{def:requ:nK}.
        \end{enumerate}
      \end{itemize}
    \end{itemize}
    \vspace{-\baselineskip}
  \end{definition}
\end{note}

\begin{note}
  If an agent doubts that \(\pv{\psi}{v'}\) is a \fc{}, then the agent is not concluding \(\pv{\phi}{v}\) from \(\Phi\).
\end{note}

\begin{note}
  Our interest with \requ{1} is with regard to \ninf{0}.
  Intuitively, if an agent does not know that \(\pv{\psi}{v'}\) is a \fc{} from \(\Psi\), then the agent will stop the current event from developing any further

  If the agent stops the event, then it is not the case that the event is such that the agent is concluding \(\pv{\phi}{v}\) from \(\Phi\).

  Indeed, even if the event could have developed into an event in which the agent concluded \(\pv{\phi}{v}\) from \(\Phi\).

  Initial example, result was not worth the effort, even if obtained.
  The agent may well have obtained, but as the agent decided the result was not worth the effort, their decision ensures they are not concluding.
\end{note}

\begin{note}
  Key thing is that whether or not the agent knows that \(\pv{\psi}{v'}\) from \(\Psi\) is a \fc{} \influence{1} whether or not the agent is concluding.
\end{note}

\begin{note}
  The qualifier `due to' is appended to \ref{def:requ:nC} to ensure that the conditional captures the agent's \ninf{}, rather than being trivially true because the agent is not concluding \(\pv{\phi}{v}\) from \(\Phi\).%
  \footnote{
    See, for example, \citeauthor{Lewis:1997wg}'s (\citeyear{Lewis:1997wg}) discussion of finkish dispositions.
  }
\end{note}

\subsection{Two initial illustrations}
\label{cha:requs:sec:init-illustr}

\begin{note}
  Two initial \illu{1}.
  In each \illu{1}, we identify a \requ{} and indicate whether or not the agent exerts \ninf{}.
\end{note}

\subsubsection{Lost keys}

\begin{note}
  \begin{illustration}[Lost keys]
    \label{illu:lost-key}
    I think I might have lost my keys.
    I usually leave place my keys on the right side of my desk, next to a copy of~\citeauthor{Vickers:1989tr}'s~\citetitle{Vickers:1989tr} which I've been saving for a rainy day.
    And, my keys aren't there.

    I've searched on the desk, under the desk, and beside the desk.
    And, I haven't found my keys.

    Still, I haven't (yet, at least) \emph{concluded} that I've lost my keys.

    For, there might still be some place I haven't looked.
    If I think a little harder a figure out where that place is, I would conclude my keys might be in that place.
    And, my keys aren't lost if they are in that place.
    So, I might conclude that my keys aren't lost, which would conflict with concluding that my keys are lost.
  \end{illustration}

  You may disagree with the tension I see in~\autoref{illu:lost-key}.
  Perhaps it's fine to conclude my keys are lost while allowing for the possibility that they're some place I haven't yet thought of.
  However, there's tension for me.
  `I've lost my keys, but they might be under that book' feels bad to me, and to me the badness extends to `I've lost my keys, but they might in that place I haven't yet considered'.

  Though, my goal is only to convince you that my refusal to conclude I've lost my keys makes sense.
  The way in which you think about the truth conditions for the sentence `I've lost my keys' may be different, but I expect my thoughts are intelligible.
\end{note}

\begin{note}
  Filling in the details of the abstract sketch:
  \begin{itemize}[noitemsep]
  \item
    I am the agent.
  \item
    \(\phi\) is the proposition: `I've lost my keys'.
  \item
    \(\psi\) is a some proposition: `My keys are not in location \(l\)'
  \item
    Both \(v\) and \(v'\) are the value: `True'.
    And,
  \item
    The pools of premises \(\Phi\) and \(\Psi\) are left unspecified.
  \end{itemize}
\end{note}

\subsubsection{Sound rules}

\begin{note}
  Here, I want to state that the agent has done the proof a number of times.
  So, the agent knows that there is a \pevent{}.
\end{note}

\begin{note}
  Return to~\autoref{scen:squish}:

  \scenarioPLSquish*

  Non-standard `Squish'-elimination rule of inference.
  Uncommon, but enough to memorise the rule.

  Apply a rule of inference if and only if it is sound.

  Further, only if \fc{}.
  For, I consider my general understanding of propositional logic more important than memory.
  And, if failed, then would not consider sound.
\end{note}

\begin{note}
  Filling in the details of the second abstract sketch:
  \begin{itemize}[noitemsep]
  \item
    I am the agent.
  \item
    \(\phi\) is the proposition: `\((P \rightarrow Q) \rightarrow P, Q \vdash P \land Q\)'.
  \item
    \(\psi\) is a some proposition: `Squish elimination is sound'
  \item
    Both \(v\) and \(v'\) are the value: `True'.
    And,
  \item
    The pools of premises \(\Phi\) and \(\Psi\) are left unspecified.%
    \footnote{
      Note, premises of reasoning.
      Distinct from premises of deduction.
    }
  \end{itemize}
\end{note}

\begin{note}
  As with \autoref{illu:lost-key}, you may differ with respect to whether or not the soundness of squish is a \requ{}.
  It may be sufficient for you that you know squish is a sound rule of inference, regardless of whether or not it is a \fc{}.
  However, if not a \fc{}, then I have a more general concern about the result of any reasoning performed.
  For, the proof is quick, and if fail, then suggests to me that \emph{at present} I am not of right mind to be constructing syntactic proofs.
\end{note}


\subsection{Intuition}
\label{sec:intuition-1}

\begin{note}
  Why is the conditional true.
  Intended interpretation is that failure to know \fc{} amounts to an something like an undercutting defeater.%
  \footnote{
    To my understanding, undercutting defeaters were introduced by \citeauthor{Pollock:1987un} (\citeyear{Pollock:1987un}).
    And, \citeauthor{Pollock:1987un} defines an undercutting defeater as follows:
    \begin{quote}
    R is an \emph{undercutting defeater} for P as a prima facie reason for S to believe Q if and only if
    \begin{enumerate}[label=(UD\arabic*), ref=(UD\arabic*)]
    \item
      \label{pollock:ud:1}
      P is a reason for S to believe Q and R is logically consistent with P but (P and R) is not a reason for S to believe Q, and
    \item
      \label{pollock:ud:2}
      R is a reason for denying that P wouldn't be true unless Q were true.%
      \mbox{}\hfill\mbox{(\citeyear[485]{Pollock:1987un})}
    \end{enumerate}
  \end{quote}
  This definition is hard to square with a \requ{}.
  In particular, \ref{pollock:ud:1}.

  Issue: P is a reason.
  By parallel, the reasoning that the agent has done is sufficient for the agent to conclude.
  However, at issue is precisely whether this is the case.

  }

  We borrow the following sketch from \textcite{Worsnip:2018aa}:
  \begin{quote}
    Undercutting defeaters, which are easiest to think of in the context of the attitude of belief, are supposed to be considerations that undermine the justification of a belief in a proposition p not necessarily by providing (sufficient) positive evidence to think that p is false, but rather merely by suggesting (perhaps misleadingly) that one’s reasons for believing p are no good, in a way that neutralizes or mitigates their justificatory or evidential force.%
    \mbox{}\hfill\mbox{(\citeyear[29]{Worsnip:2018aa})}
  \end{quote}

  In particular, concluding.
  At issue is not whether the \(\phi\) has value \(v\), but whether the agent's reasoning from \(\Phi\) to \(\pv{\phi}{v}\) is any good.

  Justification, and this is one way to go.
  However, not tied to justification.
\end{note}

\paragraph{What a \requ{} isn't}

\begin{note}
  \requ{} isn't saying that if try and fail then no conclusion.
  \requ{} is stronger, if no guarantee at present, then no conclusion.
\end{note}

\begin{note}
  Similarly, at issue is not a guarantee.
  \pevent{1} are understood in terms of the progressive.
  If \(\pv{\psi}{v'}\) from \(\Psi\) fails then there's no reasonable sense in which the event develops, even if things had gone a little differently.
\end{note}

\section{Additional \illu{1}}
\label{cha:requs:sec:expanding}

\begin{note}
  Combine with the definition of a \fc{}.

    \begin{definition}[A \requ{0}]
    \label{def:requ}
    \begin{itemize*}[noitemsep, label=\(\circ\)]
    \item
      An agent: \vAgent{}
    \item
      Propositions: \(\phi\), \(\psi\)
    \item
      Values: \(v\), \(v'\)
    \item
      \poP{3}: \(\Phi\), \(\Psi\)
    \item
      An event: \(e\)
    \item
      \mbox{ }
    \end{itemize*}

    \begin{itemize}
    \item
      \(\pvp{\phi}{v'}{\Psi}\) is a \emph{\requ{}} of \(e\) just in case:
      \begin{itemize}
      \item[\emph{If}:]
        \begin{enumerate}[label=\alph*., ref=(\alph*), series=requDefSeries]
        \item
          \(e\) is an event in which \vAgent{} is concluding \(\pv{\phi}{v}\) from \(\Phi\).
        \end{enumerate}
      \item[\emph{Then}:]
        \begin{enumerate}[label=\alph*., ref=(\alph*), resume*=requDefSeries]
        \item
          \vAgent{} knows:
          \begin{itemize}
          \item
            There is a \pevent{} in which \vAgent{} concludes \(\pv{\psi}{v'}\) from \(\Psi\).
          \item
            There is no \pevent{} in which \vAgent{} concludes something incompatible with \(\pv{\psi}{v'}\).
          \end{itemize}
        \end{enumerate}
      \end{itemize}
    \end{itemize}
    \vspace{-\baselineskip}
  \end{definition}
\end{note}

\begin{note}
  So, interactions with either of these two things.
\end{note}

\begin{note}
  \(\pv{\phi}{v}\) from \(\Phi\) entails \(\pv{\psi}{v'}\) from \(\Psi\).

  Now, two things.

  If no \pevent{}, something has gone wrong.
  Likewise, if something conflicting.
\end{note}

\subsection{A}
\label{sec:concludes}

\begin{note}
  \(\pv{\phi}{v}\) from \(\Phi\) entails \(\pv{\psi}{v'}\) from \(\Psi\).
  Therefore, if no \pevent{}, something has gone wrong.
\end{note}

\subsection{B}
\label{sec:b}

\begin{note}[Simple \requ{}]
  Variant on lost keys, where agent considers plausible they may reason from some other premise.
  {
    \color{red}
    \begin{illustration}
      A search for `Measurement Theory' via the LCC `H61 .R593' returned no results.

      Consider the possibility that library does not use DDC indexing.

      Hence, do not conclude the library does not have a copy of `Measurement Theory'.
    \end{illustration}
  }

  A more direct variant is spot the difference without a clear statement of how many differences are in the picture.
  Continue to search.
\end{note}

\begin{note}[Wally]
  \begin{illustration}[Where's Wally]
    \label{illu:CS:wheres-wally}
    \nagent{15} has a book containing numerous drawings of bustling scenes in which various characters are doing a variety of things.
    And, somewhere in each scene is a character called `Wally', identifiable by a collection of individually necessary and jointly sufficient distinguishing features.
    These features include a red and white striped jumper, blue trousers, short brown wavy hair, and so on.

    \nagent{15} has searched through one particular scene, and has identified a character with a variety of the features.
    Before concluding that the character is Wally, \nagent{15} remembers that there is a picture of Wally On the cover of the book, with all the identifying features present.

    Wally is always wearing a pair of round glasses, but this was not a feature \nagent{15} kept in mind when searching for Wally.
    So, perhaps the character \nagent{15} identified is not wearing round glasses  --- \nagent{15} only recalls the features they identified.
  \end{illustration}

  Interest is with whether \nagent{15} may conclude from the variety of features identified that the character is Wally.

  The difficulty for \nagent{15} is that if \nagent{15} were to check whether the character is wearing a pair of round glasses, and the character is not wearing a pair of round glasses, then \nagent{15} would conclude that the character is not Wally.
  Hence, a \requ{}.
  And, not a \fc{}.

  I'm going to ask whether Wally is usually carrying a cane.

  Here, keep in mind premises.
  Most plausible thing is that go back and check.
  However, this plausibly results in an additional premise.
  There is some information that is missing, and once you add it, you will conclude.
  However, not from present information.
\end{note}

\begin{note}[Spot the difference]
  \begin{illustration}[Spot the difference]
    \label{illu:CS:spot-the-diff}
    The agent has been working through a spot-the-difference to pass some time.

    Though the time is not completely passed, the agent examined the two images with what seems sufficient care to claim support that they have found all the differences.
    However, the agent did not keep track of the number of differences.

    The agent announces `I have found all the differences' and their companion responds `All fifteen?'

    \begin{enumerate}[label=\arabic*., ref=(I\ref{illu:CS:spot-the-diff}.\arabic*)]
      \setcounter{enumi}{-1}
    \item
      \label{illu:CS:spot-the-diff:info}
      If I have found all the differences, I have found fifteen differences.
    \end{enumerate}

    The agent then reasons as follows:

    \begin{enumerate}[label=\arabic*., ref=(I\ref{illu:CS:spot-the-diff}.\arabic*), resume]
    \item Exhaustive search.
    \item
      \label{illu:CS:spot-the-diff:all}
      I found all the differences.
    \item
      \label{illu:CS:spot-the-diff:fif}
      So, I have found fifteen differences. \hfill (From \ref{illu:CS:spot-the-diff:info} and \ref{illu:CS:spot-the-diff:all})
    \end{enumerate}
  \end{illustration}
\end{note}

\begin{note}
  Wason selection task.

  Same principle.

  However, this is not a general rule.
  For, it may be the case that the agent is lazy.
  Choose some cards at random.
\end{note}

\subsection{Intermediate}
\label{sec:intermediate}

\begin{note}[Prior to concluding\dots]
  Not particularly marked.
  Allow agent to have built up a bunch of stuff while reasoning.

  Example.

  \begin{scenario}[Velocity]
    \label{ill:velocity}
    Agent is provided with information about how far a car has travelled north as a function of time travelled.

    From this, take the derivative of the function to obtain the (instantaneous) velocity of the car at a handful of points in time.

    And, from the (instantaneous) velocity of the car, the agent calculates the (instantaneous) acceleration of the car at each of the points in time.

    The agent also has information about the speed of the car as a function of time travelled, and the agent may calculate speed by the taking magnitude of the (instantaneous) velocity of the car.
  \end{scenario}

  \autoref{ill:velocity}, two step calculation.
  Velocity, acceleration.
  After the first step, check by taking the magnitude.
  Calculation of velocity is correct only if taking the magnitude matches speed.

  So, two events for which the agent is concluding.
  Distinct \requ{1} associated with each event.
\end{note}


\subsection{Failures}
\label{sec:failures-1}

\begin{note}
  Typically, \(\pv{\phi}{v}\) from \(\Phi\) is not a \requ{}.
  You don't need to know that you are concluding in order to be concluding.
\end{note}

\begin{note}[A copper kettle]
  A further \illu{0} builds on a story as told by~\citeauthor{Freud:1960wx}.
  \begin{illustration}[A copper kettle]
    \label{illu:kettle}
    \mbox{ }
    \vspace{-\baselineskip}
    \begin{quote}
      `A.\ borrowed a copper kettle from B.\ and after he had returned it was sued by B.\ because the kettle now had a big hole in it which made it unusable.
      His defence was:
      ``%
      First, I never borrowed a kettle from B.\ at all;
      secondly, the kettle had a hole in it already when I got it from him;
      and thirdly, I gave him back the kettle undamaged.%
      '''\newline
      \mbox{ }\hfill\mbox{(\citeyear[62]{Freud:1960wx})}
    \end{quote}
    An agent listens to A.'s defence, but does not conclude A.\ has provided testimony.
  \end{illustration}

  The agent's failure to conclude A.\ has provided testimony may be understood in terms of a \requ{}.
  For, A.\ has provided testimony only if what A.\ has said is true.
  And, what A.\ has said is true only if the three points of A.'s defence are jointly consistent.
  Putting these observations together, we have the following conditional:

  \begin{itemize}
  \item
    A.\ has provided testimony \emph{only if} if the three points of A.'s defence are jointly consistent.
  \end{itemize}

  Failure for the agent to conclude the consequent would prevent the agent from concluding the antecedent.

  Likewise, there are only three points, and checking for consistency does not require the agent to establish whether the points are (actually) true.

  \requ{} but not \fc{}.

  Before the agent concludes A.\ has provided testimony, the agent reasons about whether the three points of A.'s defence are jointly consistent.
  After, the agent does not conclude A.\ has provided testimony.%
  \footnote{
    Any pair of points are jointly inconsistent.
    For example, consider the first and third:
    If A.\ never borrowed the kettle from B, then it is not possible for A.\ to have returned the kettle to B.
  }

  The story as told by \citeauthor{Freud:1960wx} is comical, but the \requ{0} identified is fairly general.
  In many cases one may only accept a story if the details are in harmony, and dissonance leads to rejection.
\end{note}

\begin{note}
  Anything completely unrelated.
\end{note}

\begin{note}
  Testimony, but too much to check.

  \begin{illustration}[Testimony as a layperson]
    \label{illu:testimony-layperson}
    An agent is informed that there are exactly five intermediate logics that have the interpolation property.\nolinebreak
    \footnote{Cf.\ \textcite{Maksimova:1977un}}

    The agent does not have the means to query the proof.

    The agent concludes there are exactly five intermediate logics that have the interpolation property.
  \end{illustration}

  Here, also, logic with syntax and semantics.
\end{note}

\begin{note}
  Foggy day
\end{note}

\begin{note}[Problems of induction]
  Hence, the sketch does not apply to black ravens.
  I wouldn't conclude all ravens are black if I saw a white raven.

  I may worry about shortly seeing a white raven when concluding all ravens are black, and I may refuse to entertain the possibility that the sun will rise tomorrow when planning to mow the grass.

  However, it's not possible to reason to seeing a white raven, nor is it possible to reason to the sun not rising tomorrow.

  Abstractly, at issue in~\autoref{illu:lost-key} is the possibility of failing to a reason to some proposition-value pair given \emph{present} information, rather than the possibility of failing to a reason to some proposition-value pair given \emph{new} information.

  To the extent that problems of induction arise from receiving new information, what is at issue is distinct.%
  \footnote{
    See~\textcite{Henderson:2020wb} for more on the problem of induction.
  }

  Similar points for external world scepticism.
  Would not conclude that I have hand if disembodied brain in a vat.

  However, conclusion is out of reach.
\end{note}

\section{Details}
\label{sec:details}

\subsection{Knowledge}
\label{sec:knowledge}

\begin{note}
  Whether the agent \emph{knows} that \(\pv{\psi}{v'}\) from \(\Psi\) is a \fc{}.
\end{note}

\begin{note}
  With the exception of more-or-less instantaneous actions, future may develop in surprising ways.

  For example, plausible that an agent knows when they strike the cue ball in a certain way, a particular red ball will land in a pocket.
  However, not plausible that the agent knows where the cue ball will come to rest after the red ball lands in the pocket.
  Hence, agent does not know their following more, and so on.

  In parallel, an agent may have no guarantee that they will not be interrupted, etc.
  Hence, in most cases it seems implausible that an agent knows they will concluded.
  Yet, to be concluding does not require completion.

  With respect to \fc{}, whether event in which the agent concludes would be in progress.
\end{note}

\begin{note}
  It may be the case that that \emph{if-then} conditional holds from the \agpe{} but does not hold independently of the \agpe{}.
  Hence, the sense of dependence captured by \qWhyV{} is not equivalent with the intuitive sense of dependence captured by considering whether or not the \emph{if-then} conditional holds independently of the \agpe{}.

  The observation that the \emph{if-then} conditional may hold from the \agpe{} while failing to hold independently of the \agpe{} is clearest when considering conditionals more general.

  For example, suppose an agent has taken a gamble on a coin landing heads.
  The coin lands heads, and the agent receives a prize.
  From the \agpe{}, if the coin failed to lands heads, then the agent would not have received the prize.
  However, the agent was set to receive the prize for participating in the gamble, regardless of whether the coin landed heads.%
  \footnote{
    The present point is similar to issues raised by \citeauthor{Harman:1973ww} (\citeyear{Harman:1973ww}) regarding the proposed equivalence between reasons for which an agent believes something with reasons the agent would offer if asked to justify their belief.
  As \citeauthor{Harman:1973ww} notes, an agent may offer reasons because they think they will convince their audience, not because the agent is compelled by the reasons, etc.
  (\citeyear[Ch.2]{Harman:1973ww})

  To the extent that \citeauthor{Harman:1973ww}'s point is that what holds from an \agpe{} need not actually be the case, the point in the same.
  However, to the extent that \citeauthor{Harman:1973ww} relies on an under-specification of what holds from an \agpe{} --- i.e.\ the distinction between whether \(\phi\) has value \(v\) from the \agpe{} or whether the agent evaluates as true the proposition that their audience is responsive to \(\phi\) having value \(v\), the point is distinct.
  }

  So, it may be the case that, though from the \agpe{} they would not have concluded \(\pv{\phi}{v}\) from \(\Phi\) if \support{} failed to hold between \(\pv{\psi}{v'}\) and \(\Psi\), the agent would have concluded \(\pv{\phi}{v}\) from \(\Phi\) regardless.
\end{note}

\begin{note}
  This does not provide a complete solution to problem of factivity.
  For, what distinguishes one case from the other?

  However, this is nothing unique to cases under consideration, so long as relevant instances of \fc{} are plausibly knowledge.

  Though, this still differs from attitudes.
\end{note}




%%% Local Variables:
%%% mode: latex
%%% TeX-master: "master"
%%% End:
