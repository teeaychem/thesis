\chapter{\requ{3}}
\label{cha:requs}

\begin{note}
  \autoref{cha:typical} introduced \sR{0}.

  Some generality.
\end{note}

\begin{note}
  \autoref{cha:fcs} introduced \fc{1}.

  \fc{1} are such that:
  \begin{itemize}
  \item
    \ros{} holds between \(\pv{\psi}{v'}\) and \(\Psi\) (for the agent).
  \item
    The agent does not have a \wit{} for the \ros{} between \(\pv{\psi}{v'}\) and \(\Psi\).
  \end{itemize}
\end{note}

\begin{note}
  Present chapter, \requ{1}.

  The role of a \requ{} is to bind \fc{} to concluding, and in cases of interest, \sR{0} will motivate bind.
\end{note}

\begin{note}
  The present chapter established any relation between \(\pv{\psi}{v'}\) from \(\Psi\) being a \fc{} and the agent \emph{concluding} \(\pv{\phi}{v}\) from \(\Phi\).

  Idea which we term a \requ{}.

  Roughly, agent is concluding \(\pv{\phi}{v}\) from \(\Phi\) only if \(\pv{\psi}{v'}\) is a \fc{}.
\end{note}

\begin{note}
  \requ{3} link \fc{1} to concluding.

  And, important function in generating tension to \issueConstraint{}.

  However, whether an agent concludes.
  Still, this additional link is not so difficult.
  For, if agent concludes, agent was concluding.

  Link \requ{1} to conclusions and \issueConstraint{} in \autoref{cha:binding}.

  \requ{3} are designed to \emph{not} presuppose tension with \issueConstraint{}.%
  \footnote{
    I.e.\ one may hold that there are \requ{1} and that answers to \qWhyV{} are constrained by answers to \qHowV{} via \issueConstraint{}.
    We substantiate this claim with the argument to \autoref{prop:requ-not-n-ce} on \autopageref{prop:requ-not-n-ce}, below.
    }
\end{note}

\begin{note}
  The chapter is divided into three sections:
  \begin{itemize}
  \item
    \TOCLine{cha:requs:sec:infl}

    \ninf{}.

    Preliminary, general, idea to understand \requ{1}
  \item
    \TOCLine{cha:requs:sec:definition}

    Introduction, definition, intuition, and \illu{1}
  \item
    \TOCLine{cha:requs:sec:add-illu}

    \sR{}, in additional detail.
  \end{itemize}
\end{note}

\section{\ninf{2}}
\label{cha:requs:sec:infl}

\begin{note}
  Consider the following \scen{0}:

  \begin{scenario}[Apples]
    \label{scen:apples}
    Grey is walking to town to buy some apples.
    Grey notices a bicycle for sale.
    Grey purchases the bicycle, gets on the bicycle, and starts cycling to town.
  \end{scenario}

  Grey is performing some action.
  Walking.
  Progressive, incomplete action walks to the shops.
  Grey then performs an action, purchasing a bicycle.
  Following the purchase of the bicycle Grey is no longer walking to town.
  Instead, Grey is cycling to town.

  Grey does something so that it is not true that Grey is walking to town.

  Depending on \agpe{your}, you may think that Grey was not walking to town.%
  \footnote{
    For, perfection, bound to see the bicycle for sale, has been wanting a bicycle, etc.
  }
  However, our interest is not with whether or not Grey was walking and then cycling.
  Our interest is with the basic two-part observation:
  \begin{itemize}[noitemsep]
  \item
    Shortly after purchasing the bicycle, Grey is not walking to town.
  \item
    Grey is not walking to town as a result of an action Grey performed.
  \end{itemize}

  In this respect, when Grey started to ride their bicycle, Grey exerted `\ninf{}' over whether or not they were walking to town.

  For, regardless of whether or not Grey was walking to town, there is no longer a development of the event such that developed event is an event in which Grey walks into town --- some significant sub-event, Grey was on their bicycle.
\end{note}

\begin{note}
  We define \ninf{} as follows:
  \begin{restatable}[\ninf{2}]{definition}{defNinf}
    \label{def:ninf}
    \cenLine{
      \begin{itemize*}[noitemsep, label=\(\circ\)]
      \item
        Agent: \vAgent{}
      \item
        Event: \(e\)
      \item
        Action descriptions: \(\alpha, \beta\)
      \item
        \mbox{ }
      \end{itemize*}
    }

    \begin{itemize}
    \item
      \vAgent{} has \ninf{} over whether or not \(\text{Prog}(e, \alpha)\) is true.
    \end{itemize}

    \emph{If and only if}

    \begin{itemize}
    \item
      Both~\ref{def:ninf:action} and~\ref{def:ninf:prog} are true:
      \begin{enumerate}[label=\alph*., ref=(\alph*)]
      \item
        \label{def:ninf:action}
        There is some action \(\beta\) that \vAgent{} may (immediately) do.
      \item
        \label{def:ninf:prog}
        \(\text{Prog}(e', \alpha)\) would be not be true in the event \(e'\) in which \vAgent{} does \(\beta\).
      \end{enumerate}
    \end{itemize}
    \vspace{-\baselineskip}
  \end{restatable}

  With respect to \autoref{scen:apples}:

  \begin{itemize}[noitemsep]
  \item
    The agent is Grey.
  \item
    The event \(e\) spans some period of time which starts when or after Grey set out for town, and includes the Purchase of the bicycle.
  \item
    \(\alpha\) is the action `Grey walks to town'.
  \item
    \(\beta\) is the action `Grey rides some distance on their bicycle'.
  \end{itemize}
\end{note}

\begin{note}
  The purchase of a bicycle by Grey was not necessary for Grey to have \ninf{} over whether or not Grey was walking to town.
  For, Grey may have chosen to turn back home or raid the local orchard.

  Other examples of \ninf{0}:

  \begin{itemize}[noitemsep]
  \item
    Finishing a book late at night \hfill Turn off the lights.
  \item
    Listening to a speech \hfill Leave before the speech is over.
  \item
    Playing a game of chess \hfill Flip the board in frustration.
  \end{itemize}

  No \ninf{}:

  \begin{itemize}
  \item
    Taking a shower \hfill After having been in the shower for a few minutes.

    For, still true of event that taking a shower.
    Simply ended.
  \item
    Hearing a part song on the radio \hfill After recognising the song.

    For, enough to recognise.
    Hence, have heard part of the song.
  \item
    Playing chequers \hfill After having made a few moves.

    For, played.
    Difference to playing \emph{a game} of chequers (or chess) is that have not played a game after having made a few moves.
  \end{itemize}

  In these examples, absence of \ninf{} is not due to the absence of some action that the agent may immediately perform.
  Rather, absence of \ninf{} is due to the understanding of the event.
\end{note}

% \begin{note}
%   Noted that it does not matter whether or not \(e\) is such that \(\text{Prog}(e, \alpha)\) is true.
%   Hence, \ninf{} over finding oranges on the moon.
%   For, begin Cartesian meditation.
% \end{note}

\begin{note}
  \nocite{Peacocke:2021aa}
  \autoref{scen:apples} and the examples used to illustrate an agent having or not having \ninf{} were all non-mental actions.
  With the exception of one:
  Grey choosing to turn back home.
  In this case, the relevant \(\beta\) action.

  However, our interest is \(\alpha\).
  Specifically, an agent has \ninf{} over whether or not they are \emph{concluding} \(\pv{\phi}{v}\) from \(\Phi\).

  \begin{proposition}[\ninf{2} over concluding]
    \label{prop:ninfConcl}
    There are instances in which an agent exerts \ninf{0} over whether or not they are concluding \(\pv{\phi}{v}\) from \(\Phi\).
  \end{proposition}

  \begin{argument}{prop:ninfConcl}
    Consider an agent \vAgent{}, some proposition-value pair \(\pv{\phi}{v}\), \poP{} \(\Phi\), and event \(e\).

    The claim of \autoref{prop:ninfConcl} follows from two observations:

    \begin{itemize}[noitemsep]
    \item
      By \assuPP{2}, in order for \(e\) to be an event in which \vAgent{} is concluding \(\pv{\phi}{v}\) from \(\Phi\), there must be some \progAdj{0} development \(e^{\sharp}\) of \(e\) such that \(e^{\sharp}\) is an event in which \vAgent{} concludes \(\pv{\phi}{v}\) from \(\Phi\).
    \end{itemize}

    \begin{itemize}
    \item
      There are often actions which if performed by \vAgent{} would result in an event \(e^{+}\), where \(e^{+}\) is a development of \(e\), and there is no \progAdj{0} development of \(e^{+}\) in which \vAgent{} concludes \(\pv{\phi}{v}\) from \(\Phi\).
    \end{itemize}

    Now, two ways in which \(e^{+}\) does not develop into an event where \vAgent{} concludes \(\pv{\phi}{v}\) from \(\Phi\).
    \begin{itemize}
    \item
      \(e^{+}\) does not develop into an event in which \vAgent{} \emph{\underline{concludes \(\pv{\phi}{v}\) from \(\Phi\)}}.
    \item
      \(e^{+}\) does develop into an event in which  into an event in which \vAgent{} concludes \(\pv{\phi}{v}\) \emph{\underline{from \(\Phi\)}}.
    \end{itemize}

    To illustrate each way in turn:

    \begin{itemize}
    \item
      Suppose \vAgent{} is playing a game of chess against an opponent.
      The opponent claims checkmate, and \vAgent{} beings to determine whether or not they are in checkmate.

      Still, \vAgent{} may flip the board, scattering pieces everywhere.
      And, with pieces scattered everywhere, \vAgent{} will not have sufficient resources to determine whether or not they were in checkmate.

      Hence, while determining whether or not they are in checkmate, \vAgent{} has \ninf{} over whether they are concluding they are in checkmate.
    \end{itemize}

    \begin{itemize}
    \item
      Suppose \vAgent{} is working on a homework problem which asks for solutions to the quadratic equation \(2x^{2} - x - 1 = 0\).
      \vAgent{} starts to work on the problem via factorisation, and is making slow, but steady, progress.

      Still, \vAgent{} understands the quadratic formula.
      And, \vAgent{} may use the quadratic formula to solve the quadratic equation.
      However, if \vAgent{} uses the quadratic formula, the relevant \poP{} \(\Phi\) must expand to include the details regarding the quadratic formula.

      Hence, while working on the quadratic equation, \vAgent{} has \ninf{} over whether they are concluding \(x = 1\) or \(x = -\sfrac{1}{2}\) from \(\Phi\).
      For, the agent may turn to concluding \(x = 1\) or \(x = -\sfrac{1}{2}\) from some expanded or variant \poP{} \(\Phi'\).
    \end{itemize}
  \end{argument}

  Abstractly, there are cases in which there is nothing which prevents an agent from making unavailable some proposition-value pair necessary for the agent to reach some conclusion.
  And, there are cases in which an agent may chose to appeal to additional proposition-value pairs in order to reach a conclusion.
\end{note}

\begin{note}
  \autoref{prop:ninfConcl} does not entail agent has \ninf{} over whether they are \emph{reasoning}.
  In parallel to examples used to illustrate when an agent does not have \ninf{}, sufficiently developed so that the agent has reasoned.
\end{note}

\begin{note}
  Note:
  It does not follow from \autoref{prop:ninfConcl} that the agent had a choice over how event develops, if the agent does not exert \ninf{}.
  It may be that if the agent continued the event, they would have concluded \(\pv{\phi}{v}\) from \(\Phi\).
  And, may remain the case that \(\pv{\phi}{v}\) from \(\Phi\) is a \fc{}.
\end{note}

\section{\requ{3}}
\label{cha:requs:sec:definition}

\begin{note}
  \autoref{cha:requs:sec:infl} introduces the idea of an agent having \ninf{} over whether or not some action in the progressive is true of an event.
  Our attention now turn to a phenomenon in which an agent will (or would) exert \ninf{} over whether or not an event is an event in which the agent is concluding \(\pv{\phi}{v}\) from \(\Phi\) if some proposition-value-\poP{} pairing is not a \fc{}.
\end{note}


\subsection{Definition}
\label{cha:requs:sec:definition}

\begin{note}
  Our attention now turn to a phenomenon in which an agent will (or would) exert \ninf{} over whether or not an event is an event in which the agent is concluding \(\pv{\phi}{v}\) from \(\Phi\) if some proposition-value-\poP{} pairing is not a \fc{}.

  In such cases, we say the proposition-value-\poP{} is a \requ{} of the event being an event in which the agent is concluding \(\pv{\phi}{v}\) from \(\Phi\):%
  \footnote{
    \autoref{def:requ} parallels the definition of a \pevent{} (\autoref{def:potenital-event}, \autopageref{def:potenital-event}).
    The key difference is with respect to whether the progressive is true as a result of the agent doing the relevant action.
    With respect to \ninf{}, the progressive is false, while with respect to a \pevent{0} the progressive is true.

    However, circumstances are different.
    \pevent{} when progressive is not true of event.
    \ninf{} when progressive is true.

    Note, exertion of \ninf{} is compatible with progressive being true.
  }

  \begin{restatable}[A \requ{0}]{definition}{defRequ}
    \label{def:requ}
    \cenLine{
      \begin{itemize*}[noitemsep, label=\(\circ\)]
      \item
        Agent: \vAgent{}
      \item
        Propositions: \(\phi\), \(\psi\)
      \item
        Values: \(v\), \(v'\)
      \item
        \poP{3}: \(\Phi\), \(\Psi\)
      \item
        Event: \(e\)
      \item
        \mbox{ }
      \end{itemize*}
    }

    \begin{itemize}
    \item
      \(\pvp{\psi}{v'}{\Psi}\) is a \emph{\requ{}} of \(e\) being an event in which \vAgent{} is concluding \(\pv{\phi}{v}\) from \(\Phi\).
    \end{itemize}

    \emph{If and only if}

    \begin{itemize}
    \item
      The following subjunctive conditional is true, throughout \(e\):
      \begin{itemize}
      \item[\emph{If}:]
        \begin{enumerate}[label=\alph*., ref=(\alph*), series=requDefSeries]
        \item
          \label{def:requ:nK}
          \(\pv{\psi}{v'}\) from \(\Psi\) is not a \fc{} for \vAgent{}.
        \end{enumerate}
      \item[\emph{Then}:]
        \begin{enumerate}[label=\alph*., ref=(\alph*), resume*=requDefSeries]
        \item
          \label{def:requ:nC}
          \(e\) is not an event in which \vAgent{} is concluding \(\pv{\phi}{v}\) from \(\Phi\).
        \end{enumerate}
      \end{itemize}
    \end{itemize}
    \vspace{-\baselineskip}
  \end{restatable}

  Subjunctive conditional.
  Intuitively, if not \fc{}, then not concluding.
  And, if concluding, then \fc{}.

  The thing to pay attention to is the specificity of the event.
  We have the event fixed, and there is no way to alter whether or not \fc{}.

  In other words, suppose we have concluding and a \fc{}.
  Then, there is no way to alter the event so that not \fc{}.

  That's all I want.

  Why is this the case?
  Well, there are two different ideas.

  One, because of something that leads to \ninf{}.
  Suppose not \fc{}, then this something leads to \ninf{}.

  Two, because of \sR{}.
  So, absence of \ninf{}.
\end{note}

\begin{note}
  {
    \color{red}
    Here, I want, laws and relevance.

    Ah, so, think of subjunctive as due to the material conditional being a law.

    Do not include relevance, in part, as implicitly captured, but also as it is not strictly needed.
    Compatible with relevance, and the work done will implicitly assume relevance.
    But, no argumentative role.
  }

  {
    \color{blue}

    Laws.%
    \footnote{
      More general point made by (\cite{Tichy:1976tp}), see also (\cite{Veltman:2005tj}).
    }

    Consider:
    If heads, if tails.

    Leads to two subjunctives which hold.

    If heads, would've
    If tails, would've

    Conditional remains true, and in this respect, law.
    Indeed, even if material conditional.

    This is important.
    For, suppose concluding and \fc{}.
    Then, could be the case that very different.

    However, the appropriate interpretation.
    The event does not occur.
    \ninf{}.
  }

  \autoref{def:requ} is basic.
  Conditional is true.

  Intuitively, if \(\pv{\psi}{v'}\) is not a \fc{} from \(\Psi\), then the agent stops the event from developing into an event in which the agent concludes \(\pv{\phi}{v}\) from \(\Phi\).%
  \footnote{
    Indeed, even if the event could have developed into an event in which the agent concluded \(\pv{\phi}{v}\) from \(\Phi\).
  }

  That is, \ninf{} over whether or not the agent is concluding is the result of whether or not \fc{}.

  However, this is not part of the definition.

  For, too difficult.
  Instead, preservation under counterfactual assumption.
\end{note}

\begin{note}
  \requ{3} entails conditional holds, we state in both direct and contrapositive forms:

  \begin{itemize}
  \item
    Suppose \(\pv{\psi}{v'}\) is a \requ{} of \(e\) for an agent with respect to \(\pvp{\phi}{v}{\Phi}\).
    Then:
    \begin{enumerate}
    \item
      If \(\pv{\psi}{v'}\) from \(\Psi\) is not a \fc{} for the agent, then the agent is not concluding \(\pv{\phi}{v}\) from \(\Phi\).
    \item
      If the agent is concluding \(\pv{\phi}{v}\) from \(\Phi\), then \(\pv{\psi}{v'}\) from \(\Psi\) is a \fc{} for the agent.
    \end{enumerate}

    For, suppose not \fc{}, then actual situation applies.
  \end{itemize}

  First tells us when stop.
  Second, what must be the case when agent is concluding.

  In this respect, influenced by whether or not \(\pv{\psi}{v'}\) from \(\Psi\) is a \fc{0}.
\end{note}

\begin{note}
  The qualifier `due to' is appended to \ref{def:requ:nC} to ensure that the conditional captures the agent's \ninf{}, rather than being trivially true because the agent is not concluding \(\pv{\phi}{v}\) from \(\Phi\).%
  \footnote{
    See, for example, \citeauthor{Lewis:1997wg}'s (\citeyear{Lewis:1997wg}) discussion of finkish dispositions.
  }
\end{note}

\subsection{\illu{3}}
\label{cha:requs:sec:illu3}

\begin{note}
  First, \scen{1} in which \requ{} and agent is not concluding due to \requ{}.
  This \illu{0} without detailed discussion to motivate idea.
  Further, \requ{1} such that not concluding will not be of significant interest.

  Second, \scen{1} in which \requ{} and agent is concluding.
  This \illu{0}, detailed discussion.
  However, our initial discussion will be intuitive.
\end{note}

\subsubsection{\requ{3} such that the agent is not concluding}

\begin{note}
  Only if \fc{}.
  Two ways in which fail to be \fc{}.
  Consider both ways.
\end{note}

\paragraph{Lost keys}

\begin{note}
  \scen{0} in which an agent is not concluding due to \requ{}.
\end{note}

\begin{note}
  \begin{illustration}[Lost keys]
    \label{illu:lost-key}
    I think I might have lost my keys.
    I usually leave place my keys on the right side of my desk, near a copy of~\citeauthor{Vickers:1989tr}'s~\citetitle{Vickers:1989tr} which I've been saving for a rainy day.
    And, my keys aren't there.

    I've searched over the desk, under the desk, and beside the desk.
    And, I haven't found my keys.

    Still, I haven't (yet, at least) \emph{concluded} that I've lost my keys.

    For, there might still be some place I haven't looked.
    If I think a little harder a figure out where that place is, I would conclude my keys might be in that place.
    And, my keys aren't lost if they are in that place.
    So, I might conclude that my keys aren't lost, which conflicts with concluding my keys are lost.

    I do not go on to conclude I have lost my keys.
  \end{illustration}
\end{note}

\begin{note}
  Filling in the details of \autoref{illu:lost-key}:
  \begin{itemize}[noitemsep]
  \item
    I am the agent.
  \item
    \(\phi\) is the proposition: `I have lost my keys'.
  \item
    \(\psi\) is a some proposition: `My keys are not in location \(l\)'
  \item
    Both \(v\) and \(v'\) are the value: `True'.
  \item
    The pools of premises \(\Phi\) and \(\Psi\) are left unspecified.
  \end{itemize}

  Hence, the relevant instance of the conditional by which a \requ{} is defined is:

  \begin{enumerate}[label=]
  \item
    \begin{itemize}
    \item[\emph{If}:]
      If \(\pv{\text{My keys are not in location }l}{\text{True}}\) from \(\Psi\) is a not a \fc{}.
    \item[\emph{Then}:]
      I am not concluding \(\pv{\text{I have lost my keys}}{\text{True}}\) from \(\Phi\).
    \end{itemize}
  \end{enumerate}

  The antecedent is true, and hence the consequent is true due to \ninf{} I exert over whether or not I am concluding my keys are lost.
\end{note}

\begin{note}
  At issue is only whether \(\pvp{\psi}{v'}{\Psi}\) is a \requ{} for me in a particular event in which I am reasoning about whether I have lost my keys.%
  \footnote{
    Indeed, you may disagree with the tension I see in~\autoref{illu:lost-key}.
  Perhaps it's fine to conclude my keys are lost while allowing for the possibility that the keys are some place I haven't yet thought of.
  }

  Hence, if one does not worry about the possibility that the keys are some place they haven't yet thought of, \(\pvp{\psi}{v'}{\Psi}\) would not be a \requ{1} of concluding they have lost their keys.
\end{note}

\begin{note}
  \fc{1} focus on event.
  However, the existence of an event is secondary.
  Present concern that applies to any reasoning about where keys are.

  This is what secures that there is no event.
\end{note}

\paragraph{A conjecture}

\begin{note}
  Lost keys, failing to reach some conclusion.
  For, idea that is applied at any attempt to conclude.

  In this respect, conflict.
  From this, no \pevent{} in which agent concludes.
  However, conflict.

  Second \illu{0}, focus on no \pevent{0} in which agent concludes.
  From this, no \pevent{} in which the agent gets conflict.
\end{note}

\begin{note}
  \begin{illustration}[Goldbach Conjecture]
    The (Binary) Goldbach Conjecture states:

    \begin{quote}
      Every even number is a sum of two primes.
    \end{quote}

    I am travelling on a train with a some paper, a pencil.

    The Conjecture is only a conjecture, and so there may (for all I know), be an even number which is not the sum of two primes.

    I write down \(573402\) on the piece of paper, and attempt to find a pair of primes that it is equal to.
    Though, I am careful to ensure I do not make any arithmetical mistakes.

    I am not concluding the Goldbach Conjecture is false.
  \end{illustration}

  The conjecture has been verified up to \(4 \times 10^{14}\) (\cite[cf.][]{Richstein:2001aa}).
  Therefore, need to do a number greater.
  Careful about arithmetic, and limited time.
  Therefore, not a \fc{0}.

  Of course, certainly not proving false.
  For, number chosen lower than verified.
  Of interest is whether I am concluding.
  Answer is no.
  Not because finding two primes.
  There is nothing to suggest develops.
  For, will stop when arrive at destination.
  Rather, because not a \fc{0}.
\end{note}

\begin{note}
  More abstract that the first.
  For, with lost keys, clear idea that conflicts with any conclusion.
  All that's needed is to repeat the idea.

  In this \scen{0}, adherence to arithmetic.
  This ensures that I never get to the conclusion.
\end{note}

\subsubsection{\requ{3} such that the agent is concluding}

\begin{note}
  In contrast to \scen{1} of previous section, need to ensure both aspects of a \fc{0} hold.
  Is a  \pevent{} in which conclude \(\pv{\psi}{v'}\) from \(\Psi\).
  No \pevent{} in which conclude anything which conflicts with concluding \(\pv{\psi}{v'}\) from \(\Psi\).

  However, vary which of these things focus.
\end{note}

\begin{note}
  Two \illu{1}.
  First parallels~\autoref{illu:lost-key}.

  Second is closer to the kind of \scen{0} we will be interested in when we link \requ{1} to \issueConstraint{}.
\end{note}

\paragraph{Locked doors}

\begin{note}
  \begin{restatable}[Locked doors]{illustration}{scenarioKeycard}
    Cycling along.
    Going to work.
    Look at the time.

    Slightly early.
    Doors locked.

    Conclusion, use key card to unlock doors.

    However, to get to this, need key card.
    Feel wallet in pocket.

    I conclude that I will use the key card to unlock the doors.
  \end{restatable}

  If conclude key card is not in wallet, stop.
  Time it takes to find, no longer early.
  Continue, then coffee while waiting.

  However, no chance of failing to conclude.
  Feel wallet in pocket.
  Need some motivation for thinking key card is not in wallet.
  But, no such motivation.
\end{note}

\begin{note}
  Main point.
  Don't need to think about key card.

  Just, if not, then problem.

  And, this happens.
  Do recall failing to pack item for a trip, etc.

  Point is, without something to suggest reasoning would fail, the reasoning will go through.
  And, I'm not going to find anything.

  Much like commonsense argument.
\end{note}

\begin{note}
  So, here, I have done the positive part.
  I've gone from some premises to key card.
  At issue is finding something which conflicts.

  And, in contrast to lost keys, will use commonsense against anything that comes up.
\end{note}

\paragraph{Sound rules}

\begin{note}
  \phantlabel{squish-elimination-proof}

  \begin{restatable}[Squish elimination]{illustration}{scenarioPLSquish}
    \label{scen:squish}
    It is late morning on a sunny day.
    I ate a good breakfast, and drank some nice coffee.
    I have completed a handful of syntactic proofs for entailments of propositional logic using the basic rules of inference in a Fitch-style system.

    I create the following syntactic proof:
    \begin{center}
      \begin{fitch}
        \phantlabel{illu:sP:1}\fa (P \rightarrow Q) \rightarrow P \\
        \phantlabel{illu:sP:2}\fj Q \\
        \phantlabel{illu:sP:3}\fa P & \sqE{}:\hyperref[illu:sP:1]{1} \\
        \phantlabel{illu:sP:4}\fa P \land Q & \(\land\)\textbf{Intro:} \hyperref[illu:sP:2]{2},\hyperref[illu:sP:3]{3}
      \end{fitch}
    \end{center}

    Still, I haven't yet concluded \((P \rightarrow Q) \rightarrow P, Q \vdash P \land Q\).

    For, if \sqE{} is not a sound rule of inference, then \((P \rightarrow Q) \rightarrow P, Q\) may not entail \(P \land Q\).

    I go on to conclude \((P \rightarrow Q) \rightarrow P, Q \vdash P \land Q\).
  \end{restatable}

  The relevant propositions, values, and \poP{1} are as follows:
  \begin{itemize}[noitemsep]
  \item
    I am the agent.
  \item
    \(\phi\) is the proposition: `\((P \rightarrow Q) \rightarrow P, Q \vdash P \land Q\)'.
  \item
    \(\psi\) is a some proposition: `\sqE{} is sound'
  \item
    Both \(v\) and \(v'\) are the value: `True'.
    And,
  \item
    The pools of premises \(\Phi\) and \(\Psi\) are left unspecified.%
    \footnote{
      Note, \(P \rightarrow Q\) and \(Q\) are premises of the deduction, but are not elements of \(\Phi\).
      For, I am concluding \((P \rightarrow Q) \rightarrow P, Q \vdash P \land Q\), rather than concluding \(P \land Q\) from \(P \rightarrow Q,Q\).
    }
  \end{itemize}

  Hence, the relevant instance of the conditional by which a \requ{} is defined is:

  \begin{enumerate}[label=]
  \item
    \begin{itemize}
    \item[\emph{If}:]
      If \(\pv{\text{\sqE{} is sound}}{\text{True}}\) from \(\Psi\) is not a \fc{}.
    \item[\emph{Then}:]
      I am not concluding \(\pv{(P \rightarrow Q) \rightarrow P, Q \vdash P \land Q}{\text{True}}\) from \(\Phi\).
    \end{itemize}
  \end{enumerate}

  In contrast to \autoref{illu:lost-key}, \autoref{scen:squish} leads to a conclusion.
  Is this really a \requ{}?
\end{note}

\paragraph{Motivation}

\begin{note}
  Start with understanding from \agpe{my}, expand.

  However, briefly expand on \sqE{}.
\end{note}

\begin{note}
  \begin{definition}[\sqE{}]
    \label{def:sque}
    \sqE{} is the following rule:
    \begin{center}
      \begin{fitch}
        \ftag{\text{\scriptsize \emph{i}}}{\fa (\phi \rightarrow \psi) \rightarrow \phi} \\
        \ftag{\text{\scriptsize }}{\fa \vdots } \\
        \ftag{\text{\scriptsize \emph{j}}}{\fa \phi } & \sqE{}:\emph{i} \\
      \end{fitch}
    \end{center}
  \end{definition}

  \begin{proposition}[Soundness of \sqE{}]
    \label{prop:sqE-sound}
    \sqE{} is sound.
  \end{proposition}

  \begin{argument}{prop:sqE-sound}
    Rather than prove \sqE{} is sound (which would require a detailed statement of the proof system in question), we prove that the corresponding semantic entailment holds:

    Let \(v\) be an arbitrary (truth-functional) valuation, and assume \(v((\phi \rightarrow \psi) \rightarrow \phi) = \text{True}\).
    Further, assume for contradiction \(v(\phi) = \text{False}\).

    As \(v(\phi) = \text{False}\), it immediately follows that \(v(\phi \rightarrow \psi) = \text{True}\).
    Therefore, by the first assumption, it must be the case that \(v(\phi) = \text{True}\).
    This contradictions the second assumption.
    Hence, \((\phi \rightarrow \psi) \rightarrow \phi \vDash \phi\).
  \end{argument}
\end{note}


\subparagraph{From \agpe{my}}

\begin{note}
  \sR{2}.
  Only conclude if reasoning amounts to proving.

  Throughout \autoref{scen:squish} I \emph{know} \sqE{} is sound.
  Prior to \autoref{scen:squish} I have proved \sqE{} is sound on various occasions using the same basic observations made in the argument for \autoref{prop:sqE-sound}.

  However, there is a distinction between \emph{knowing} \sqE{} is sound and \emph{proving} \sqE{} is sound.

  For example, I (trivially) know that any sound rule is sound.
  Yet, it is not the case that I may prove any sound rule is sound.
  For, sound rules may have an arbitrary (finite) number of premises, and I may cease to be before even reading all of the premises.

  Of course, \sqE{} is a simple rule, and typically does not take more than a few moments to prove.
  Yet, it remains the case that I may fail to prove \sqE{} is sound.
  For example, if I have just drunk a considerable amount of wine, or woken from a night of tormented sleep.
  Generally said, I may not be thinking straight.

  To give a concrete example.
  Suppose I were to pause and reason to:
  \(P \rightarrow (Q \rightarrow P) \vdash P\).%
  \footnote{
    None of the following entailments hold:
    \begin{enumerate*}[noitemsep, label=]
    \item
      \(\phi \rightarrow (\psi \rightarrow \phi) \vDash \phi\),
    \item
      \(\psi \rightarrow (\phi \rightarrow \psi) \vDash \phi\),
    \item
      \((\psi \rightarrow \phi) \rightarrow \psi \vDash \phi\).
    \end{enumerate*}
    Constructing a counterexample to each is straightforward.
  }
  Then, apparent proof of \((P \rightarrow Q) \rightarrow P, Q \vdash P \land Q\) is in significant doubt.
  For, I know \(P \rightarrow (Q \rightarrow P) \vdash P\) does not hold.

  Whatever my current reasoning amounts to, it is not \sR{}.
\end{note}

\begin{note}
  Now, bottle of wine or bad sleep, then there may be no \requ{1}.
  I may conclude \(\pv{\phi}{v}\) from \(\Phi\) regardless of whether \(\pv{\psi}{v'}\) from \(\Psi\) is a \fc{}.

  Indeed, given some particularly good wine I may not only conclude \(P \rightarrow (Q \rightarrow P) \vdash P\), but also conclude \(P \land C \vdash O\).%
  \footnote{
    For, \(P \land C\) reads `Pac', \(O\) looks like a pellet, and pacman likes to eat pellets.
  }

  However, \requ{1} concern particular event.

  And, during \autoref{scen:squish} I am thinking straight.
  So, I am \requSens{0} to whether or not reasoning is \sR{0}.
  Hence, any doubt during \autoref{scen:squish} that reasoning is \sR{0} would lead to preventing the event from developing further.
\end{note}

\begin{note}
  Now, \agpe{my}, and \agpe{my} may be wrong.

  I concluded \((P \rightarrow Q) \rightarrow P, Q \vdash P \land Q\), but it need not be the case that \requSens{0} to whether or not \(\pv{\psi}{v'}\) from \(\Psi\) is a \fc{}.

  Perhaps, I would fail to conclude \(\pv{\psi}{v'}\) from \(\Psi\), or would conclude something which conflicts with concluding \(\pv{\psi}{v'}\) from \(\Psi\).
\end{note}

\subparagraph{Apart from \agpe{my}}

\begin{note}
  Regardless of \agpe{my}, then if \requ{} fails to hold, it is unclear whether reasoning is \sR{}.

  If no \requ{}, then what distinguishes my reasoning from arbitrarily reasoning to \((P \rightarrow Q) \rightarrow P, Q \vdash P \land Q\)?

  \agpe{My} is correct in so far as the reasoning does not amount to a syntactic proof.
  For, without \fc{1}, then is nothing to suggest that my reasoning is \sR{}.
\end{note}

\begin{note}
  Only to highlight a causal connexion.
  And, causal connexion is insufficient.

  % Concern about justification, implicit in \scen{0} as described from \agpe{my}.
  % However, specific worry is that there is no generality to reasoning.

  % If \agpe{my} is wrong, then it is the case that reasoning is not \sR{}.
  % Conclude only if \sR{}.
\end{note}

\begin{note}
  The motivation for the conditional rests on whether reasoning is \sR{}.

  A \requ{} is seems to be to be a necessary condition for conclusions which result from \sR{}.%
  \footnote{
    This is not sufficient, though.
    As with lost keys, \requ{} need not be tied to \sR{}.
    With lost keys, \sR{}, but whether that is sufficient for concluding.
  }
  For, if there are no \requ{1}, then there is nothing to indicate the agent would reasoning in accordance with other instances of the type of reasoning.
\end{note}

\begin{note}
  Distinction.
  Concluding, versus concluding via \sR{0}.
  It need not be the case that if not \sR[concluding]{0} then not concluding.
  \requ{} is doing substantial work to tie concluding to a type of reasoning.
\end{note}


% \subsection{What a \requ{} is not}

% \begin{note}
%   \requ{} isn't saying that if try and fail then no conclusion.
%   \requ{} is stronger, if no guarantee at present, then no conclusion.
% \end{note}

\subsection{Propositions}
\label{sec:propsoitions}

\begin{note}
  We briefly state two propositions which have no significant role in the arguments to follow, but which may be helpful.
\end{note}

\paragraph*{First}

\begin{note}
  The existence of \requ{1} are compatible with \issueConstraint{}.
  \begin{proposition}
    \label{prop:requ-not-n-ce}
    The existence of \requ{1} is compatible with \issueConstraint{}.
  \end{proposition}
  \begin{argument}{prop:requ-not-n-ce}
    In order for counterexample, case in which \ros{} answers \qWhyV{}, but the agent does not have a \wit{}.

    We have not yet consider the way in which \requ{1} connect to \qWhyV{}.
    However, any explicit connexion is not required.
    For, we only need to observe that an agent may have a \wit{} for any \requ{}.
    Then, the \wit{} is always a candidate for \qHowV{}.

    Consider \autoref{scen:squish}.
    At issue is whether the agent would repeat the repeat conclusion of \sqE{} as sound.

    Needs to be the case that \ros{} without \wit{}.

    However, the agent has previously concluded \sqE{} is sound.
    Therefore, the agent has a \wit{} for \ros{}.
  \end{argument}

  Note, however, that \autoref{prop:requ-not-n-ce} is weak.
  Compatible in terms of existential.
  Hence we only needed to show a single case in which \wit{} for \requ{}.
  This does not suggest that \requ{1} are not in tension with \issueConstraint{}.
\end{note}

\paragraph*{Second, third}

\begin{note}
  \begin{proposition}
    It is the case that \(\pvp{\phi}{v}{\Phi}\) is a \requ{} of concluding \(\pv{\phi}{v}\) from \(\Phi\).
  \end{proposition}

  So, suppose \(\pv{\phi}{v}\) from \(\Phi\) is not a \fc{0} for agent.
  Well, then, it is certainly the case that the agent is not concluding \(\pv{\phi}{v}\) from \(\Phi\).
  For, as not a \fc{}, then there is no \pevent{} in which agent concludes.
\end{note}

\begin{note}
  The second proposition may help clarify what the existence of \requ{1} amounts to:

  \begin{proposition}
    \label{prop:requ-not-refl}
    \cenLine{
      \begin{itemize*}[noitemsep, label=\(\circ\)]
      \item
        Agent: \vAgent{}
      \item
        Proposition: \(\phi\)
      \item
        Value: \(v\)
      \item
        \poP{2}: \(\Phi\)
      \item
        Event: \(e\)
      \item
        \mbox{ }
      \end{itemize*}
    }

    \begin{itemize}
    \item
      It is not \emph{necessarily} the case that if the agent concludes \(\pv{\phi}{v}\) from \(\Phi\), then \(\pvp{\phi}{v}{\Phi}\) \requ{} of some sub-event.
    \end{itemize}
    \vspace{-\baselineskip}
  \end{proposition}

  In other words, \(\pv{\phi}{v}\) from \(\Phi\) need not be a \fc{} in order for an agent to be concluding \(\pv{\phi}{v}\) from \(\Phi\).

  \begin{argument}{prop:requ-not-refl}
    {
      \color{red}
      The key observation is that it need not be the case that the agent is concluding.

      So, there is an quick and less quick argument.

      Quick, no constraints on what it is for an agent to conclude.

      Less quick, \fc{} for the agent.
    }


    It need not be the case that the agent knows that \(\pv{\phi}{v}\) from \(\Phi\) is a \fc{}.

    The simplest cases are at the outset.
    It need not be the case that \citeauthor{Maksimova:1977un} knew there are exactly five intermediate logics that have the interpolation property was a \fc{} before proving such (\cite[cf.][]{Maksimova:1977un}).

    However, the same hold for when the agent pairs \(\phi\) with \(v\).
    For, the agent need not know they would repeat the reasoning.
    To \illu{0}, consider being guided through a complex argument.
    Follow along, and conclude.
    At each step, do the reasoning, the guide highlights which sub-conclusions to draw.
    However, guide goes away.
    And, given complexity, no repetition without guide.
  \end{argument}

  In general, \autoref{prop:requ-not-refl} highlights that there are no trivial \requ{1}.

  Well, in terms of \sR{}.
  Don't need \sR{}.
  Novel results, and so on.

  And, with guidance, removed the need for \sR{} with respect to overall.
\end{note}

\section{\requ{3} and \sR{0}}
\label{cha:requs:sec:add-illu}

% {
%   \color{red}
%   This section is very sketchy at the moment.
%   In short, the idea is to put pressure on the idea that there is any generality to an agent's reasoning if there are no cases of \requ{1}.

%   In short, pick any case and deny a \requ{} exists.
%   Then, it is not the case that the agent is \requSens{0} to whether or not \(\pv{\psi}{v'}\) from \(\Psi\) is a \fc{}.
%   Therefore, I claim, there is no way to guarantee that the agent's reasoning is \sR{} reasoning.
%   For, if not \requSens{0}, then there is nothing which regulates whether or not the agent is performing any particular type of reasoning.
%   Hence, there is no way, it seems, to guarantee that the agent would have repeated the type of reasoning with a different collection of premises.
% }

\begin{note}
  Though \requ{1} are compatible with \issueConstraint{} (see \autoref{prop:requ-not-n-ce} on \autopageref{prop:requ-not-n-ce}), \requ{1} have a key role in constructing counterexamples to \issueConstraint{}.

  Therefore, it is important that there are cases in which \(\pvp{\psi}{v'}{\Psi}\) is a \requ{1} of concluding \(\pv{\phi}{v}\) from \(\Phi\).
  So, at issue is whether the \illu{1} of \autoref{cha:requs:sec:illu3} illustrate instances of \requ{1}.

  Still, any given \scen{0} is open to interpretation.
  Hence, general motivation for the existence of \requ{1}.

  The goal of this section is to motivate the existence of a certain kind of \requ{}.
  Specifically, the kind of \requ{0} present in \autoref{scen:squish}.

  This kind of \requ{}, \sR{}.
  And, agent is concluding.%
  \footnote{
    We set \autoref{illu:lost-key}, etc., aside.

    For, the importance of \requ{1} is limited to cases in which \(\pvp{\psi}{v'}{\Psi}\) is a \requ{0} of concluding \(\pv{\phi}{v}\) from \(\Phi\) and the agent goes on to conclude \(\pv{\phi}{v}\) from \(\Phi\).

    Hence, while cases such as \autoref{illu:lost-key} in which an agent does not conclude due to a \requ{} motivate the general idea of a \requ{}, nothing in particular hangs on such cases.
  }
\end{note}

\begin{note}
  Broad strokes, \sR{}.
  Without \requ{}, not \sR{}.
\end{note}

\subsection{Points}

\begin{note}
  General worry.

  No constraints on when an agent concludes.

  Recall, conclusion, paring of proposition and value.
  So, it is possible for the agent to conclude \(\pv{\phi}{v}\) from \(\Phi\) without satisfying these constraints.
\end{note}

\begin{note}
  \requSens{0}.

  Gist, conclusions arbitrary.%
  \footnote{
    The present point is similar to issues raised by \citeauthor{Harman:1973ww} (\citeyear{Harman:1973ww}) regarding the proposed equivalence between reasons for which an agent believes something with reasons the agent would offer if asked to justify their belief.
  As \citeauthor{Harman:1973ww} notes, an agent may offer reasons because they think they will convince their audience, not because the agent is compelled by the reasons, etc.
  (\citeyear[Ch.2]{Harman:1973ww})

  To the extent that \citeauthor{Harman:1973ww}'s point is that what holds from an \agpe{} need not actually be the case, the point in the same.
  However, to the extent that \citeauthor{Harman:1973ww} relies on an under-specification of what holds from an \agpe{} --- i.e.\ the distinction between whether \(\phi\) has value \(v\) from the \agpe{} or whether the agent evaluates as true the proposition that their audience is responsive to \(\phi\) having value \(v\), the point is distinct.
}\(,\)%
  \footnote{
    Also, \citeauthor{Schaffer:2010vq}'s (\citeyear{Schaffer:2010vq}) Debasing demon.

    The debasing demon \textquote{throws her victims into the belief state on an improper basis, while leaving them with the impression as if they had proceeded properly} (\citeyear[231]{Schaffer:2010vq})

    \citeauthor{Schaffer:2010vq} really is the way to think about things.
    For, \citeauthor{Schaffer:2010vq}'s entire point is that one may get arbitrary conclusions.
    There's nothing too important regarding basing here.
    And, the other paper is really puzzling in terms of what it offers.
    Parallel, knowledge, truth, therefore any account of knowledge is incompatible with scepticism.

    However, as \citeauthor{Bondy:2018tk} highlight, it is not clear that this raises problems for accounts of the basing relation.
    In short, beliefs may be formed arbitrarily, but it is not necessarily the case (and is indeed, not the case for significant accounts) that accounts of the basing relation consider arbitrary.

    Though, this is not to say basing relation avoids worries.
    Still, at best, basing is distinct from the way in which belief comes about.
  }

  However, tension.
  Whether or not the agent's reasoning is \sR{}.
\end{note}

\subsection{A Sudoku puzzle}

\begin{note}
  Additional \illu{0} to highlight \sR{}.
\end{note}

\begin{note}
  \begin{illustration}[Sudoku]
    \label{illu:gist:sudoku}
    % https://tex.stackexchange.com/questions/91422/tikz-sudoku-circle-and-connect-with-lines-some-cells
    Consider the following Sudoku puzzle:%
    \footnote{
      From~\textcite[84]{Coussement:2007up}.
    }
    \vspace{\baselineskip}

    \mbox{ }\hfill%
    \begin{adjustbox}{minipage=0.45\linewidth,scale=1}
      \centering
      \begin{tikzpicture}[scale=.5]
        \begin{scope}
          \draw (0, 0) grid (9, 9);
          \draw[very thick, scale=3] (0, 0) grid (3, 3);
          \setcounter{row}{1}
          % Single entries
          \setrow { }{ }{ }  { }{ }{ }  {1}{ }{ }
          \setrow { }{ }{ }  { }{ }{ }  { }{5}{ }
          \setrow {9}{ }{ }  { }{ }{ }  { }{ }{2}
          \setrow { }{ }{3}  { }{2}{ }  { }{ }{ }
          \setrow { }{ }{ }  {8}{ }{ }  {4}{6}{5}
          \setrow { }{4}{ }  { }{5}{9}  { }{ }{8}
          \setrow { }{8}{7}  {2}{3}{1}  { }{4}{6}
          \setrow {2}{1}{ }  {5}{ }{ }  { }{ }{3}
          \setrow {3}{ }{6}  {4}{ }{8}  { }{ }{ }
        \end{scope}
      \end{tikzpicture}
    \end{adjustbox}%
    \hfill\mbox{ }

  \end{illustration}

  Interactive.
  Fill in the grid.
  Difference between filling in the grid and concluding that solution to the puzzle.
  So, before conclude for any particular square, or for the grid as a whole.
  Is it the case that you would fill in the grid the same way?

  Key intuition, stop before committing to any mistake.

  Two aspects.

  First, may give up completely.
  Second, catch any mistakes and fix before moving on.
\end{note}

\begin{note}
  \autoref{illu:gist:sudoku} parallels \autoref{scen:squish}.

  In both \illu{1}, \(\pv{\phi}{v}\) follows from \(\Phi\) via a rules.

  However, rules are not of direct interest.
  \autoref{scen:squish} is a syntactic proof, but variant \scen{0} in which semantic proof.
  Relevant reasoning may be rule governed, but semantic proofs are not constrained.

  Rather, familiarity.

  The type of reasoning is general.
  Syntactic and semantic proofs, Sudoku puzzles, simple instances of chess problems, all seem to involve general reasoning.
  Likewise, counting, adding, subtracting, and so on.
  Competence established through various proofs, puzzles, problems, and practice.

  In this respect, there is no reasonable doubt of \sR{}.
  At issue is whether specific performance.
\end{note}

\begin{note}
  Same as \autoref{prop:requ-not-n-ce} on \autopageref{prop:requ-not-n-ce}.
  Attention only wrt.\ to reasoning.
  Hence, \wit{}.
\end{note}

\begin{note}
  Intuition for each of the points.

  \begin{itemize}
  \item
    \fc{1}, understand how to solve Sudoku puzzles, repeat any instances.
  \item
    \requSens{}.
    Catch mistakes.
  \end{itemize}
\end{note}

\subsection{The conditional}

\begin{note}
  Definition of a \requ{}:

  \defRequ*

  Need the conditional to be true.
  However, interest is in cases in which the agent is concluding \(\pv{\phi}{v}\) from \(\Phi\).

  Therefore:
  \begin{itemize}
  \item
    \(\pv{\psi}{v'}\) from \(\Psi\) is a \fc{0}.
  \item
    Agent would be negatively influenced by whether or not \(\pv{\psi}{v'}\) from \(\Psi\) is a \fc{}.
  \end{itemize}
\end{note}

\begin{note}
  Break this down further.

  \begin{itemize}
  \item
    Whether there's something which says yes/no \fc{}.
  \item
    Then, whether influence.
  \end{itemize}

  First, there's no way for the conditional to be true.
  Second, whether there really are instances in which the conditional is true.
\end{note}

\paragraph{\requSensAdj{2}}

\begin{note}
  \begin{itemize}
  \item
    Event in which concluding, but no telling whether or not \fc{}.
  \end{itemize}

  This happens.

  Whether or not the agent is concluding need not be apparent to the agent.

  So, guidance is one example of this.
  Here, guidance helps see the connexions, though the connexions are made by the agent.
  Hence, agent concludes, but is not concluding.
\end{note}

\begin{note}
  \begin{proposition}
    \label{prop:hinge}
    If \requ{} is representative of same type, then, \sR{} only if \requ{}.
  \end{proposition}

  \begin{argument}{prop:hinge}
    Suppose same type.

    Now, suppose not \requ{}.
    This means, concluding, though not a \fc{}.
    This, then, means that change so not \fc{}.
    But, then, there is something in the agent's reasoning which does not generalise.
    So, this means that the reasoning is not of type.
  \end{argument}

  The thing about this argument is that the counterfactual allows us to distinguish between types of reasoning.

  With additional complexity to definition of \requ{}, avoid the qualification.
  However, restricts some interesting cases of \requ{} and merely avoids this qualification.

  The gist of this argument, if not \requSens{0}, then it is not the case that certain things are \fc{1}.
  For, there's something about the reasoning which the agent will perform whenever.
  And, as such, there's plausibly going to have something at issue for a \fc{}.
\end{note}

\begin{note}
  Now, here, I do need to be a little more careful.
  For, it's the case that filling in square, or something.
\end{note}

\begin{note}
  This is a brief argument which (seems, at least) to do a lot.
  So, let us consider it in some more.

  Type of reasoning.
  There's some generality.
  So, then, well, there's no really much more to be said.
\end{note}

\begin{note}
  Is this really the case?
  Problems with understanding \sR{}.

  If there are problems, rests on subjunctive conditionals.
  Liberal, true.

  So, for example, \citeauthor{Boghossian:2014aa}.
  Taking condition.
  Here, it seems, avoid subjunctive conditionals.

  However, this doesn't help much.

  (Okay, this really should go in the other chapter.)

  Here, as we have seen, subjunctive understanding of \sR{} seems to follow from granting subjunctive conditionals.

  The difficulty is representatives.
  For, not the case that all of same type.
  However, in the cases used to motivate, representatives seem clear.
  For, the agent has seen the reasoning/concluding through to completion.
\end{note}

\begin{note}
  If problems, seems in expression.
  Subjunctives are `vague'.
  Used this vagueness.

  But, the motivating idea is simple.
  If agent is going to go off and conclude something wild, then there's a real problem in saying that whatever is happening works.

  Perhaps one point to mention.
  This doesn't show that the agent's reasoning is not of any type.
  Rather, it shows that the agent's reasoning is not of the specified type.
\end{note}

\newpage

\paragraph{Agent's perspective}

\begin{note}
  The arguments given avoid \agpe{}.

  Still, if you are inclined to think \agpe{} is correct, then\dots
\end{note}

\subsection{Intuition}
\label{sec:intuition-1}

\begin{note}
  Granting interpretation is correct, failure to know \fc{} amounts to an something like an undercutting defeater.%
  \footnote{
    To my understanding, undercutting defeaters were introduced by \citeauthor{Pollock:1987un} (\citeyear{Pollock:1987un}).
    And, \citeauthor{Pollock:1987un} defines an undercutting defeater as follows:
    \begin{quote}
    R is an \emph{undercutting defeater} for P as a prima facie reason for S to believe Q if and only if
    \begin{enumerate}[label=(UD\arabic*), ref=(UD\arabic*)]
    \item
      \label{pollock:ud:1}
      P is a reason for S to believe Q and R is logically consistent with P but (P and R) is not a reason for S to believe Q, and
    \item
      \label{pollock:ud:2}
      R is a reason for denying that P wouldn't be true unless Q were true.%
      \mbox{}\hfill\mbox{(\citeyear[485]{Pollock:1987un})}
    \end{enumerate}
  \end{quote}
  This definition is hard to square with a \requ{}.
  In particular, \ref{pollock:ud:1}.

  Issue: P is a reason.
  By parallel, the reasoning that the agent has done is sufficient for the agent to conclude.
  However, at issue is precisely whether this is the case.
  }

  We borrow the following sketch from \textcite{Worsnip:2018aa}:
  \begin{quote}
    Undercutting defeaters, which are easiest to think of in the context of the attitude of belief, are supposed to be considerations that undermine the justification of a belief in a proposition p not necessarily by providing (sufficient) positive evidence to think that p is false, but rather merely by suggesting (perhaps misleadingly) that one’s reasons for believing p are no good, in a way that neutralizes or mitigates their justificatory or evidential force.%
    \mbox{}\hfill\mbox{(\citeyear[29]{Worsnip:2018aa})}
  \end{quote}

  In particular, concluding.
  At issue is not whether the \(\phi\) has value \(v\), but whether the agent's reasoning from \(\Phi\) to \(\pv{\phi}{v}\) is sufficient for the agent to conclude \(\phi\) has value \(v\) (from \(\Phi\)).

  Justification, and this is one way to go.
  However, not tied to justification.
\end{note}

\section{Summary}
\label{sec:summary-1}

\begin{note}
  Introduced \requ{1}.
\end{note}

%%% Local Variables:
%%% mode: latex
%%% TeX-master: "master"
%%% End:


% \begin{note}[Problems of induction]
%   Hence, the sketch does not apply to black ravens.
%   I wouldn't conclude all ravens are black if I saw a white raven.

%   I may worry about shortly seeing a white raven when concluding all ravens are black, and I may refuse to entertain the possibility that the sun will rise tomorrow when planning to mow the grass.

%   However, it's not possible to reason to seeing a white raven, nor is it possible to reason to the sun not rising tomorrow.

%   Abstractly, at issue in~\autoref{illu:lost-key} is the possibility of failing to a reason to some proposition-value pair given \emph{present} information, rather than the possibility of failing to a reason to some proposition-value pair given \emph{new} information.

%   To the extent that problems of induction arise from receiving new information, what is at issue is distinct.%
%   \footnote{
%     See~\textcite{Henderson:2020wb} for more on the problem of induction.
%   }

%   Similar points for external world scepticism.
%   Would not conclude that I have hand if disembodied brain in a vat.

%   However, conclusion is out of reach.
% \end{note}