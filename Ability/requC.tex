\chapter{\requ{3}}
\label{cha:requs}

\begin{note}
  \autoref{cha:typical} introduced \tR{0}.

  Some generality.
\end{note}

\begin{note}
  \autoref{cha:fcs} introduced \fc{1}.

  \fc{1} are such that:
  \begin{itemize}
  \item
    \ros{} holds between \(\pv{\psi}{v'}\) and \(\Psi\) (for the agent).
  \item
    The agent does not have a \wit{} for the \ros{} between \(\pv{\psi}{v'}\) and \(\Psi\).
  \end{itemize}
\end{note}

\begin{note}
  Present chapter, \requ{1}.

  The role of a \requ{} is to bind \fc{} to concluding, and in cases of interest, \tR{0} will motivate bind.
\end{note}

\begin{note}
  The present chapter established any relation between \(\pv{\psi}{v'}\) from \(\Psi\) being a \fc{} and the agent \emph{concluding} \(\pv{\phi}{v}\) from \(\Phi\).

  Idea which we term a \requ{}.

  Roughly, agent is concluding \(\pv{\phi}{v}\) from \(\Phi\) only if \(\pv{\psi}{v'}\) is a \fc{}.
\end{note}

\begin{note}
  \requ{3} link \fc{1} to concluding.

  And, important function in generating tension to \issueConstraint{}.

  However, whether an agent concludes.
  Still, this additional link is not so difficult.
  For, if agent concludes, agent was concluding.

  Link \requ{1} to conclusions and \issueConstraint{} in \autoref{cha:binding}.

  \requ{3} are designed to \emph{not} presuppose tension with \issueConstraint{}.%
  \footnote{
    I.e.\ one may hold that there are \requ{1} and that answers to \qWhyV{} are constrained by answers to \qHowV{} via \issueConstraint{}.
    We substantiate this claim with the argument to \autoref{prop:requ-not-n-ce} on \autopageref{prop:requ-not-n-ce}, below.
    }
\end{note}

\begin{note}
  The chapter is divided into three sections:
  \begin{itemize}
  \item
    \TOCLine{cha:requs:sec:infl}

    \ninf{}.

    Preliminary, general, idea to understand \requ{1}
  \item
    \TOCLine{cha:requs:sec:definition}

    Introduction, definition, intuition, and \illu{1}
  \item
    \TOCLine{cha:requs:sec:add-illu}

    \tR{}, in additional detail.
  \end{itemize}
\end{note}

\section{\ninf{2}}
\label{cha:requs:sec:infl}

\begin{note}
  Consider the following \scen{0}:

  \begin{scenario}[Apples]
    \label{scen:apples}
    Grey is walking to town to buy some apples.
    Grey notices a bicycle for sale.
    Grey purchases the bicycle, gets on the bicycle, and starts cycling to town.
  \end{scenario}

  Grey is performing some action.
  Walking.
  Progressive, incomplete action walks to the shops.
  Grey then performs an action, purchasing a bicycle.
  Following the purchase of the bicycle Grey is no longer walking to town.
  Instead, Grey is cycling to town.

  Grey does something so that it is not true that Grey is walking to town.

  Depending on \agpe{your}, you may think that Grey was not walking to town.%
  \footnote{
    For, perfection, bound to see the bicycle for sale, has been wanting a bicycle, etc.
  }
  However, our interest is not with whether or not Grey was walking and then cycling.
  Our interest is with the basic two-part observation:
  \begin{itemize}[noitemsep]
  \item
    Shortly after purchasing the bicycle, Grey is not walking to town.
  \item
    Grey is not walking to town as a result of an action Grey performed.
  \end{itemize}

  In this respect, when Grey started to ride their bicycle, Grey exerted `\ninf{}' over whether or not they were walking to town.

  For, regardless of whether or not Grey was walking to town, there is no longer a development of the event such that developed event is an event in which Grey walks into town --- some significant sub-event, Grey was on their bicycle.
\end{note}

\begin{note}
  We define \ninf{} as follows:
  \begin{definition}[\ninf{2}]
    \label{def:ninf}
    \cenLine{
      \begin{itemize*}[noitemsep, label=\(\circ\)]
      \item
        Agent: \vAgent{}
      \item
        Event: \(e\)
      \item
        Action descriptions: \(\alpha, \beta\)
      \item
        \mbox{ }
      \end{itemize*}
    }

    \begin{itemize}
    \item
      \vAgent{} has \ninf{} over whether or not \(\text{Prog}(e, \alpha)\) is true.
    \end{itemize}

    \emph{If and only if}

    \begin{itemize}
    \item
      Both~\ref{def:ninf:action} and~\ref{def:ninf:prog} are true:
      \begin{enumerate}[label=\alph*., ref=(\alph*)]
      \item
        \label{def:ninf:action}
        There is some action \(\beta\) that \vAgent{} may (immediately) do.
      \item
        \label{def:ninf:prog}
        \(\text{Prog}(e', \alpha)\) would be not be true in the event \(e'\) in which \vAgent{} does \(\beta\).
      \end{enumerate}
    \end{itemize}
    \vspace{-\baselineskip}
  \end{definition}

  With respect to \autoref{scen:apples}:

  \begin{itemize}[noitemsep]
  \item
    The agent is Grey.
  \item
    The event \(e\) spans some period of time which starts when or after Grey set out for town, and includes the Purchase of the bicycle.
  \item
    \(\alpha\) is the action `Grey walks to town'.
  \item
    \(\beta\) is the action `Grey rides some distance on their bicycle'.
  \end{itemize}
\end{note}

\begin{note}
  The purchase of a bicycle by Grey was not necessary for Grey to have \ninf{} over whether or not Grey was walking to town.
  For, Grey may have chosen to turn back home or raid the local orchard.

  Other examples of \ninf{0}:

  \begin{itemize}[noitemsep]
  \item
    Finishing a book late at night \hfill Turn off the lights.
  \item
    Listening to a speech \hfill Leave before the speech is over.
  \item
    Playing a game of chess \hfill Flip the board in frustration.
  \end{itemize}

  No \ninf{}:

  \begin{itemize}
  \item
    Taking a shower \hfill After having been in the shower for a few minutes.

    For, still true of event that taking a shower.
    Simply ended.
  \item
    Hearing a part song on the radio \hfill After recognising the song.

    For, enough to recognise.
    Hence, have heard part of the song.
  \item
    Playing chequers \hfill After having made a few moves.

    For, played.
    Difference to playing \emph{a game} of chequers (or chess) is that have not played a game after having made a few moves.
  \end{itemize}

  In these examples, absence of \ninf{} is not due to the absence of some action that the agent may immediately perform.
  Rather, absence of \ninf{} is due to the understanding of the event.
\end{note}

% \begin{note}
%   Noted that it does not matter whether or not \(e\) is such that \(\text{Prog}(e, \alpha)\) is true.
%   Hence, \ninf{} over finding oranges on the moon.
%   For, begin Cartesian meditation.
% \end{note}

\begin{note}
  \nocite{Peacocke:2021aa}
  \autoref{scen:apples} and the examples used to illustrate an agent having or not having \ninf{} were all non-mental actions.
  With the exception of one:
  Grey choosing to turn back home.
  In this case, the relevant \(\beta\) action.

  However, our interest is \(\alpha\).
  Specifically, an agent has \ninf{} over whether or not they are \emph{concluding} \(\pv{\phi}{v}\) from \(\Phi\).

  \begin{proposition}[\ninf{2} over concluding]
    \label{prop:ninfConcl}
    There are instances in which an agent exerts \ninf{0} over whether or not they are concluding \(\pv{\phi}{v}\) from \(\Phi\).
  \end{proposition}

  \begin{argument}{prop:ninfConcl}
    Consider an agent \vAgent{}, some proposition-value pair \(\pv{\phi}{v}\), \pool{} \(\Phi\), and event \(e\).

    The claim of \autoref{prop:ninfConcl} follows from two observations:

    \begin{itemize}[noitemsep]
    \item
      By \assuPP{2}, in order for \(e\) to be an event in which \vAgent{} is concluding \(\pv{\phi}{v}\) from \(\Phi\), there must be some \progAdj{0} development \(e^{\sharp}\) of \(e\) such that \(e^{\sharp}\) is an event in which \vAgent{} concludes \(\pv{\phi}{v}\) from \(\Phi\).
    \end{itemize}

    \begin{itemize}
    \item
      There are often actions which if performed by \vAgent{} would result in an event \(e^{+}\), where \(e^{+}\) is a development of \(e\), and there is no \progAdj{0} development of \(e^{+}\) in which \vAgent{} concludes \(\pv{\phi}{v}\) from \(\Phi\).
    \end{itemize}

    Now, two ways in which \(e^{+}\) does not develop into an event where \vAgent{} concludes \(\pv{\phi}{v}\) from \(\Phi\).
    \begin{itemize}
    \item
      \(e^{+}\) does not develop into an event in which \vAgent{} \emph{\underline{concludes \(\pv{\phi}{v}\) from \(\Phi\)}}.
    \item
      \(e^{+}\) does develop into an event in which  into an event in which \vAgent{} concludes \(\pv{\phi}{v}\) \emph{\underline{from \(\Phi\)}}.
    \end{itemize}

    To illustrate each way in turn:

    \begin{itemize}
    \item
      Suppose \vAgent{} is playing a game of chess against an opponent.
      The opponent claims checkmate, and \vAgent{} beings to determine whether or not they are in checkmate.

      Still, \vAgent{} may flip the board, scattering pieces everywhere.
      And, with pieces scattered everywhere, \vAgent{} will not have sufficient resources to determine whether or not they were in checkmate.

      Hence, while determining whether or not they are in checkmate, \vAgent{} has \ninf{} over whether they are concluding they are in checkmate.
    \end{itemize}

    \begin{itemize}
    \item
      Suppose \vAgent{} is working on a homework problem which asks for solutions to the quadratic equation \(2x^{2} - x - 1 = 0\).
      \vAgent{} starts to work on the problem via factorisation, and is making slow, but steady, progress.

      Still, \vAgent{} understands the quadratic formula.
      And, \vAgent{} may use the quadratic formula to solve the quadratic equation.
      However, if \vAgent{} uses the quadratic formula, the relevant \pool{} \(\Phi\) must expand to include the details regarding the quadratic formula.

      Hence, while working on the quadratic equation, \vAgent{} has \ninf{} over whether they are concluding \(x = 1\) or \(x = -\sfrac{1}{2}\) from \(\Phi\).
      For, the agent may turn to concluding \(x = 1\) or \(x = -\sfrac{1}{2}\) from some expanded or variant \pool{} \(\Phi'\).
    \end{itemize}
  \end{argument}

  Abstractly, there are cases in which there is nothing which prevents an agent from making unavailable some proposition-value pair necessary for the agent to reach some conclusion.
  And, there are cases in which an agent may chose to appeal to additional proposition-value pairs in order to reach a conclusion.
\end{note}

\begin{note}
  \autoref{prop:ninfConcl} does not entail agent has \ninf{} over whether they are \emph{reasoning}.
  In parallel to examples used to illustrate when an agent does not have \ninf{}, sufficiently developed so that the agent has reasoned.
\end{note}

\begin{note}
  Note:
  It does not follow from \autoref{prop:ninfConcl} that the agent had a choice over how event develops, if the agent does not exert \ninf{}.
  It may be that if the agent continued the event, they would have concluded \(\pv{\phi}{v}\) from \(\Phi\).
  And, may remain the case that \(\pv{\phi}{v}\) from \(\Phi\) is a \fc{}.
\end{note}

\section{\requ{3}}
\label{cha:requs:sec:definition}

\begin{note}
  \autoref{cha:requs:sec:infl} introduced the idea of an agent having \ninf{} over whether or not some action in the progressive is true of an event.

  The key observation of \autoref{cha:requs:sec:infl} was \autoref{prop:ninfConcl}.
  An agent may exert \ninf{} over whether or the agent is concluding some proposition-value pair from some \pool{} is true of an event.

  Our attention now turn to a phenomenon in which an agent does or would exert \ninf{} over whether or not the agent is concluding `due to' some proposition-value-\pool{} pairing not being a \fc{}.
\end{note}

\begin{note}
  This section is split into three further subsections:

  \begin{enumerate}[label=]
  \item
    \TOCLine{cha:requs:sec:definition}

    The definition.
  \item
    \TOCLine{cha:requs:sec:illu3}

    A handful of \scen{1} to illustrate \requ{1}.
  \item
    \TOCLine{sec:propsoitions}

    A few propositions.
  \end{enumerate}

  The final section of this chapter, \autoref{cha:requs:sec:add-illu}, will motivate the existence of \requ{1} via \tR{}.
\end{note}


\subsection{The definition of a \requ{}}
\label{cha:requs:sec:definition}

\begin{note}
  \requ{3} capture a phenomenon in which an agent does or would exert \ninf{} over whether or not the agent is concluding \(\pv{\phi}{v}\) from \(\Phi\) `due to' \(\pvp{\psi}{v'}{\Psi}\) pairing not being a \fc{}.

  We state the definition of a \requ{} in full, and then clarify the way in which the definition works.

  \begin{definition}[A \requ{0}]
    \label{def:requ}
    \cenLine{
      \begin{itemize*}[noitemsep, label=\(\circ\)]
      \item
        Agent: \vAgent{}
      \item
        Propositions: \(\phi\), \(\psi\)
      \item
        Values: \(v\), \(v'\)
      \item
        \pool{3}: \(\Phi\), \(\Psi\)
      \item
        \mbox{ }
      \end{itemize*}
    }

    \cenLine{
      \begin{itemize*}[noitemsep, label=\(\circ\)]
      \item
        Event: \(e\)
      \item
        Event description: \(d\)
      \item
        \mbox{ }
      \end{itemize*}
    }

    \begin{itemize}
    \item
      \(\pvp{\psi}{v'}{\Psi}\) is a \emph{\requ{}} of \(e\) under description \(d\) being an event in which \vAgent{} is concluding \(\pv{\phi}{v}\) from \(\Phi\).
    \end{itemize}

    \emph{If and only if}

    \begin{itemize}
    \item
      For any alternative event \(e'\) to event \(e\) such that \(d\) is true of \(e'\):
      \begin{itemize}
      \item[\emph{If}:]
        \begin{enumerate}[label=\alph*., ref=(\alph*), series=requDefSeries]
        \item
          \label{def:requ:nK}
          \(\pv{\psi}{v'}\) from \(\Psi\) is not a \fc{} for \vAgent{}.
        \end{enumerate}
      \item[\emph{Then}:]
        \begin{enumerate}[label=\alph*., ref=(\alph*), resume*=requDefSeries]
        \item
          \label{def:requ:nC}
          \(e'\) is not an event in which \vAgent{} is concluding \(\pv{\phi}{v}\) from \(\Phi\).
        \end{enumerate}
      \end{itemize}
    \end{itemize}
    \vspace{-\baselineskip}
  \end{definition}

  As we will usually have an event under some description, shorten to:
  \begin{itemize}
  \item
    \(\pvp{\psi}{v'}{\Psi}\) is a \requ{} of an agent concluding \(\pv{\phi}{v}\) from \(\Phi\).
  \end{itemize}
\end{note}

\begin{note}
  \requ{3} are defined with respect to an event \(e\) under description \(d\).

  Given some event \(e\) and description \(d\), \(\pvp{\psi}{v'}{\Psi}\) being a \requ{} of an agent concluding \(\pv{\phi}{v}\) from \(\Phi\) intuitively corresponds to the following idea:
  \begin{itemize}
  \item
    The agent is concluding \(\pv{\phi}{v}\) from \(\Phi\) \emph{only if} \(\pv{\psi}{v'}\) from \(\Psi\) is a \fc{}, for the agent.
  \end{itemize}

  Or, as expressed in definition, fails to be a \fc{}, then \(e\) is not an event in which the agent is concluding \(\pv{\phi}{v}\) from \(\Phi\).

  Natural language, relevance.
  Not the material conditional.

  Suppose concluding.
  Then, counterfactual, if hadn't, then wouldn't be.

  Though, the converse fails.
  If not concluding, it doesn't need to be the case that would be if were \fc{}.

  Descriptions.
  The function of descriptions is to fix some things about the event so that consider way in which the event may have been.

  So, then, conditional is not true due to something contingent, but from something broader.

  Abstractly, \requ{3} are defined by a conditional of the form:
  \begin{itemize}
  \item
    If \(d\) is true for some event \(e'\), then description \(d'\) is (also) true of \(e'\).
  \end{itemize}
  Hence, though \(d'\) may happen be true of \(e\), what matters is that \(d'\) follows from \(d\), and hence \(d'\) would continue to be true given variations to \(e\) compatible with \(d\).

  Though, additional qualification of `alternative' events.
  In short, \(d'\) need not be a logical consequence of \(d\).
\end{note}

\begin{note}
  Simple example.

  \begin{illustration}[Coin flip]
    \label{illu:coinT}
    Agent \(B\) has accepted gable on coin toss from agent \(A\), captured by the following pair of conditionals:
    \begin{itemize}[noitemsep]
    \item
      If the coin lands heads, then agent \(B\) gives \texteuro{}5 to agent \(A\).
    \item
      If the coin lands tails, then agent \(A\) gives \texteuro{}5 to agent \(B\).
    \end{itemize}

    Agent \(A\) has tossed the coin, but the result of the toss is hidden between agent \(A\)'s hands.
  \end{illustration}

  \autoref{illu:coinT} is a description of an event.
  And, abstracting from whatever else is true of the event described in \autoref{illu:coinT}, the following conditionals follow:

  \begin{itemize}[noitemsep]
  \item
    If the coin has landed heads, then agent \(B\) is \texteuro{}5 poorer.
  \item
    If the coin has landed tails, then agent \(A\) is \texteuro{}5 poorer.
  \end{itemize}

  Indeed, these conditionals follow even if they are interpreted as material conditionals, and it just so happens to be that both \(A\) and \(B\) have been visited by a pickpocket.
  For, we may vary the presence of the pickpocket while maintaining the description of the \scen{0}, and given the gamble, the relevant transfer of wealth occurs.
  More broadly, so long as the description of the event holds, then the conditionals follow no matter what else is true of the event.

  Likewise, extend to event of \autoref{illu:coinT} so that the result of the toss has been revealed, then additional conditionals, depending on what the result of the toss was:%
  \footnote{
    Assuming the coin landed tails and heads, respectively.
  }

  \begin{itemize}[noitemsep]
  \item
    If the coin had landed heads, agent \(B\) would be \texteuro{}5 poorer.
  \item
    If the coin had landed tails, agent \(B\) would be \texteuro{}5 richer.
  \end{itemize}

  Here, the counterfactual conditionals in part follow from the gamble as specified in the initial description of \autoref{illu:coinT}.%
  \footnote{%
    \nocite{Tichy:1976tp}%
    Indeed, with respect to subjunctive conditionals, say the conditionals which specify the gamble are `laws' (\cite{Chisholm:1955aa,Lewis:1979vm,Veltman:2005tj}) or `lump together' the facts regarding the coin and relative wealth of agents \(A\) and \(B\) (\cite{Kratzer:1981aa,Kratzer:1989aa}).
  }

  Now, with respect to the definition of a \requ{}, the \emph{if}-\emph{then} conditional follows from the specified description \(d\) in parallel to the way the observed conditionals follow from the description of the event as given in~\autoref{illu:coinT}.

  Though, rather than providing details about wealth, the \emph{if}-\emph{then} conditional places a limit on whether the relevant event is an event in which the agent is concluding \(\pv{\phi}{v}\) from \(\Phi\).
\end{note}

\begin{note}
  Our analysis of \autoref{illu:coinT} is almost okay, but is flawed.
  For, we suggested the relevant conditionals follow from the description of~\autoref{illu:coinT}.
  This is not immediate.
  For, consider adding to the description of~\autoref{illu:coinT} that agent \(A\) is dishonest, and will run before handing anything to agent \(B\), that neither agent really has \texteuro{}5 at hand and are hoping they win, or \dots
  The point is, unless description \(d\) (logically) entails description \(d'\), there is some what for description \(d\) to be true and description \(d'\) to be false.
  Hence, it may be argued that \(d'\) does not follow from \(d\).

  The role of the qualifier `alternative' is to allow description \(d'\) to follow from description \(d\) without \(d'\) being a logical consequence of \(d\).%
  \footnote{
    In this respect, \requ{1} may be thought of in line with the safety condition on knowledge (\cite{Sosa:1999aa}), relevant alternatives (\cite{Dretske:1970to}), or anything which fits into a restrictor analysis of conditionals (\cite{Lewis:1975aa,Partee:1991aa}).
  }
  Hence, we need not consider all events compatible with description \(d\) when evaluating whether description \(d'\) follows from \(d\).

  Still, what `alternative' amounts to is not a choice.

  The definition of a \requ{} is given in terms of descriptions, but our interest is with the event described.
  With respect to~\autoref{illu:coinT}, our interest is with the limitations on action present in the relevant event given the details of the gamble.
  And, with respect to \requ{1}, our interest is with the limitations on action present in a relevant event given a \requ{}.

  Hence, the `alternative' qualifier functions to fill in whatever gap remains between description and event with respect to whether the description \(d'\) really does follow from description \(d\).
\end{note}

\begin{note}
  This is all, admittedly, quite complex.
  I hope a handful of \illu{1} to follow in \autoref{cha:requs:sec:illu3} will help clarify what \requ{1} amount to.
\end{note}

\subsection{\illu{3} of \requ{1}}
\label{cha:requs:sec:illu3}

\begin{note}
  In this section we illustrate \requ{1} through four \scen{1}.

  In each \scen{0} \(\pvp{\psi}{v'}{\Psi}\) is a \requ{} of concluding \(\pv{\phi}{v}\) from \(\Phi\).

  The first pair of \scen{1} are \scen{1} in which \(\pv{\psi}{v'}\) from \(\Psi\) is \emph{not} a \fc{} for the agent, and hence the agent is \emph{not} concluding \(\pv{\phi}{v}\) from \(\Psi\).
  The second pair of \scen{1} are \scen{1} in which the agent \emph{is} concluding \(\pv{\phi}{v}\) from \(\Psi\) and hence \(\pv{\psi}{v'}\) from \(\Psi\) \emph{is} a \fc{} for the agent.

  Each pair of \scen{1} emphasises different aspects of what \(\pv{\psi}{v'}\) from \(\Psi\) being a \fc{} amounts to.
  I.e.\ whether or not there is a \pevent{} in which the agent concludes \(\pv{\psi}{v'}\) from \(\Psi\), or whether there is no \pevent{} in which the agent concludes something incompatible with concluding \(\pv{\psi}{v'}\) from \(\Psi\).%
  \footnote{
    Cf.\ the definition of a \fc{}: \autoref{def:fc},~\autopageref{def:fc}.
  }
\end{note}

\subsubsection{\requ{3} such that the agent is not concluding}

\begin{note}
  Only if \fc{}.
  Two ways in which fail to be \fc{}.
  Consider both ways.
\end{note}

\paragraph{Lost keys}

\begin{note}
  \scen{0} in which an agent is not concluding due to \requ{}.
\end{note}

\begin{note}
  \begin{illustration}[Lost keys]
    \label{illu:lost-key}
    I think I might have lost my keys.
    I usually leave place my keys on the right side of my desk, near a copy of~\citeauthor{Vickers:1989tr}'s~\citetitle{Vickers:1989tr} which I've been saving for a rainy day.
    And, my keys aren't there.

    I've searched over the desk, under the desk, and beside the desk.
    And, I haven't found my keys.

    Still, I haven't (yet, at least) \emph{concluded} that I've lost my keys.

    For, there might still be some place I haven't looked.
    If I think a little harder a figure out where that place is, I would conclude my keys might be in that place.
    And, my keys aren't lost if they are in that place.
    So, I might conclude that my keys aren't lost, which conflicts with concluding my keys are lost.

    I do not go on to conclude I have lost my keys.
  \end{illustration}
\end{note}

\begin{note}
  Filling in the details of \autoref{illu:lost-key}:
  \begin{itemize}[noitemsep]
  \item
    I am the agent.
  \item
    \(\phi\) is the proposition: `I have lost my keys'.
  \item
    \(\psi\) is a some proposition: `My keys are not in location \(l\)'
  \item
    Both \(v\) and \(v'\) are the value: `True'.
  \item
    The pools of premises \(\Phi\) and \(\Psi\) are left unspecified.
  \item
    Key parts of description:
    \begin{itemize}
    \item
      If possible somewhere I haven't considered, not lost.
    \item
      Possible.
    \end{itemize}
  \end{itemize}

  Hence, the relevant instance of the conditional by which a \requ{} is defined is:

  \begin{enumerate}[label=]
  \item
    \begin{itemize}
    \item[\emph{If}:]
      If \(\pv{\text{My keys are not in location }l}{\text{True}}\) from \(\Psi\) is a not a \fc{}.
    \item[\emph{Then}:]
      I am not concluding \(\pv{\text{I have lost my keys}}{\text{True}}\) from \(\Phi\).
    \end{itemize}
  \end{enumerate}

  The antecedent is true, and hence the consequent is true due to \ninf{} I exert over whether or not I am concluding my keys are lost.
\end{note}

\begin{note}
  At issue is only whether \(\pvp{\psi}{v'}{\Psi}\) is a \requ{} for me in a particular event in which I am reasoning about whether I have lost my keys.%
  \footnote{
    Indeed, you may disagree with the tension I see in~\autoref{illu:lost-key}.
  Perhaps it's fine to conclude my keys are lost while allowing for the possibility that the keys are some place I haven't yet thought of.
  }

  Hence, if one does not worry about the possibility that the keys are some place they haven't yet thought of, \(\pvp{\psi}{v'}{\Psi}\) would not be a \requ{1} of concluding they have lost their keys.
\end{note}

\begin{note}
  \fc{1} focus on event.
  However, the existence of an event is secondary.
  Present concern that applies to any reasoning about where keys are.

  This is what secures that there is no event.
\end{note}

\paragraph{A conjecture}

\begin{note}
  Lost keys, failing to reach some conclusion.
  For, idea that is applied at any attempt to conclude.

  In this respect, conflict.
  From this, no \pevent{} in which agent concludes.
  However, conflict.

  Second \illu{0}, focus on no \pevent{0} in which agent concludes.
  From this, no \pevent{} in which the agent gets conflict.
\end{note}

\begin{note}
  \begin{illustration}[Goldbach Conjecture]
    The (Binary) Goldbach Conjecture states:

    \begin{quote}
      Every even number is a sum of two primes.
    \end{quote}

    I am travelling on a train with a some paper, a pencil.

    The Conjecture is only a conjecture, and so there may (for all I know), be an even number which is not the sum of two primes.

    I write down \(573402\) on the piece of paper, and attempt to find a pair of primes that it is equal to.
    Though, I am careful to ensure I do not make any arithmetical mistakes.

    I am not concluding the Goldbach Conjecture is false.
  \end{illustration}

  Filling in the details of \autoref{illu:lost-key}:

  \begin{itemize}[noitemsep]
  \item
    I am the agent.
  \item
    \(\phi\) is the proposition: `The (Binary) Goldbach Conjecture'.
  \item
    \(\psi\) is a some proposition: `573402 is equal to a the sum of two primes'
  \item
    Both \(v\) and \(v'\) are the value: `False'.
  \item
    The pools of premises \(\Phi\) and \(\Psi\) are left unspecified.
  \item
    Key parts of description:
    \begin{itemize}
    \item
      Not to making any arithmetical mistakes.
    \end{itemize}
  \end{itemize}

  The conjecture has been verified up to \(4 \times 10^{14}\) (\cite[cf.][]{Richstein:2001aa}).
  Therefore, need to do a number greater.
  Careful about arithmetic, and limited time.
  Therefore, not a \fc{0}.

  Of course, certainly not proving false.
  For, number chosen lower than verified.
  Of interest is whether I am concluding.
  Answer is no.
  Not because finding two primes.
  There is nothing to suggest develops.
  For, will stop when arrive at destination.
  Rather, because not a \fc{0}.
\end{note}

\begin{note}
  More abstract that the first.
  For, with lost keys, clear idea that conflicts with any conclusion.
  All that's needed is to repeat the idea.

  In this \scen{0}, adherence to arithmetic.
  This ensures that I never get to the conclusion.
\end{note}

\subsubsection{\requ{3} such that the agent is concluding}

\begin{note}
  In contrast to \scen{1} of previous section, need to ensure both aspects of a \fc{0} hold.
  Is a  \pevent{} in which conclude \(\pv{\psi}{v'}\) from \(\Psi\).
  No \pevent{} in which conclude anything which conflicts with concluding \(\pv{\psi}{v'}\) from \(\Psi\).

  However, vary which of these things focus.
\end{note}

\begin{note}
  Two \illu{1}.
  First parallels~\autoref{illu:lost-key}.

  Second is closer to the kind of \scen{0} we will be interested in when we link \requ{1} to \issueConstraint{}.
\end{note}

\paragraph{Locked doors}

\begin{note}
  \begin{illustration}[Locked doors]
    Cycling along.
    Going to work.
    Look at the time.

    Slightly early.
    Doors locked.

    Conclusion, use key card to unlock doors.

    However, to get to this, need key card.
    Feel wallet in pocket.

    I conclude that I will use the key card to unlock the doors.
  \end{illustration}

    \begin{itemize}[noitemsep]
  \item
    I am the agent.
  \item
    \(\phi\) is the proposition: `Unlock'.
  \item
    \(\psi\) is a some proposition: `Key card'
  \item
    Both \(v\) and \(v'\) are the value: `True'.
  \item
    The pools of premises \(\Phi\) and \(\Psi\) are left unspecified.
  \item
    Key parts of description:
    \begin{itemize}
    \item
      Wallet.
    \item
      Dismiss not in wallet.
    \end{itemize}
  \end{itemize}

  If conclude key card is not in wallet, stop.
  Time it takes to find, no longer early.
  Continue, then coffee while waiting.

  However, no chance of failing to conclude.
  Feel wallet in pocket.
  Need some motivation for thinking key card is not in wallet.
  But, no such motivation.
\end{note}

\begin{note}
  Main point.
  Don't need to think about key card.

  Just, if not, then problem.

  And, this happens.
  Do recall failing to pack item for a trip, etc.

  Point is, without something to suggest reasoning would fail, the reasoning will go through.
  And, I'm not going to find anything.

  Much like commonsense argument.
\end{note}

\begin{note}
  So, here, I have done the positive part.
  I've gone from some premises to key card.
  At issue is finding something which conflicts.

  And, in contrast to lost keys, will use commonsense against anything that comes up.
\end{note}

\paragraph{Sound rules}

\begin{note}
  \phantlabel{squish-elimination-proof}

  \begin{illustration}[Squish elimination]
    \label{scen:squish}
    It is late morning on a sunny day.
    I ate a good breakfast, and drank some nice coffee.
    I have completed a handful of syntactic proofs for entailments of propositional logic using the basic rules of inference in a Fitch-style system.

    I create the following syntactic proof:
    \begin{center}
      \begin{fitch}
        \phantlabel{illu:sP:1}\fa (P \rightarrow Q) \rightarrow P \\
        \phantlabel{illu:sP:2}\fj Q \\
        \phantlabel{illu:sP:3}\fa P & \sqE{}:\hyperref[illu:sP:1]{1} \\
        \phantlabel{illu:sP:4}\fa P \land Q & \(\land\)\textbf{Intro:} \hyperref[illu:sP:2]{2},\hyperref[illu:sP:3]{3}
      \end{fitch}
    \end{center}

    Still, I haven't yet concluded \((P \rightarrow Q) \rightarrow P, Q \vdash P \land Q\).

    For, if \sqE{} is not a sound rule of inference, then \((P \rightarrow Q) \rightarrow P, Q\) may not entail \(P \land Q\).

    I go on to conclude \((P \rightarrow Q) \rightarrow P, Q \vdash P \land Q\).
  \end{illustration}

  The relevant propositions, values, and \pool{1} are as follows:
  \begin{itemize}[noitemsep]
  \item
    I am the agent.
  \item
    \(\phi\) is the proposition: `\((P \rightarrow Q) \rightarrow P, Q \vdash P \land Q\)'.
  \item
    \(\psi\) is a some proposition: `\sqE{} is sound'
  \item
    Both \(v\) and \(v'\) are the value: `True'.
    And,
  \item
    The pools of premises \(\Phi\) and \(\Psi\) are left unspecified.%
    \footnote{
      Note, \(P \rightarrow Q\) and \(Q\) are premises of the deduction, but are not elements of \(\Phi\).
      For, I am concluding \((P \rightarrow Q) \rightarrow P, Q \vdash P \land Q\), rather than concluding \(P \land Q\) from \(P \rightarrow Q,Q\).
    }
  \item
    Key parts of description:
    \begin{itemize}[noitemsep]
    \item
      sunny day, good breakfast, nice coffee.
      I.e.\ ideal situations for syntactic proofs.
    \item
      Proved \sqE{} numerous times before.
    \end{itemize}
  \end{itemize}

  Hence, the relevant instance of the conditional by which a \requ{} is defined is:

  \begin{enumerate}[label=]
  \item
    \begin{itemize}
    \item[\emph{If}:]
      If \(\pv{\text{\sqE{} is sound}}{\text{True}}\) from \(\Psi\) is not a \fc{}.
    \item[\emph{Then}:]
      I am not concluding \(\pv{(P \rightarrow Q) \rightarrow P, Q \vdash P \land Q}{\text{True}}\) from \(\Phi\).
    \end{itemize}
  \end{enumerate}

  In contrast to \autoref{illu:lost-key}, \autoref{scen:squish} leads to a conclusion.
  Is this really a \requ{}?
\end{note}

\subparagraph{Motivation}

\begin{note}
  Start with understanding from \agpe{my}, expand.

  However, briefly expand on \sqE{}.
\end{note}

\begin{note}
  \begin{definition}[\sqE{}]
    \label{def:sque}
    \sqE{} is the following rule:
    \begin{center}
      \begin{fitch}
        \ftag{\text{\scriptsize \emph{i}}}{\fa (\phi \rightarrow \psi) \rightarrow \phi} \\
        \ftag{\text{\scriptsize }}{\fa \vdots } \\
        \ftag{\text{\scriptsize \emph{j}}}{\fa \phi } & \sqE{}:\emph{i} \\
      \end{fitch}
    \end{center}
  \end{definition}

  \begin{proposition}[Soundness of \sqE{}]
    \label{prop:sqE-sound}
    \sqE{} is sound.
  \end{proposition}

  \begin{argument}{prop:sqE-sound}
    Rather than prove \sqE{} is sound (which would require a detailed statement of the proof system in question), we prove that the corresponding semantic entailment holds:

    Let \(v\) be an arbitrary (truth-functional) valuation, and assume \(v((\phi \rightarrow \psi) \rightarrow \phi) = \text{True}\).
    Further, assume for contradiction \(v(\phi) = \text{False}\).

    As \(v(\phi) = \text{False}\), it immediately follows that \(v(\phi \rightarrow \psi) = \text{True}\).
    Therefore, by the first assumption, it must be the case that \(v(\phi) = \text{True}\).
    This contradictions the second assumption.
    Hence, \((\phi \rightarrow \psi) \rightarrow \phi \vDash \phi\).
  \end{argument}
\end{note}


\subparagraph{From \agpe{my}}

\begin{note}
  \tR{2}.
  Only conclude if reasoning amounts to proving.

  Throughout \autoref{scen:squish} I \emph{know} \sqE{} is sound.
  Prior to \autoref{scen:squish} I have proved \sqE{} is sound on various occasions using the same basic observations made in the argument for \autoref{prop:sqE-sound}.

  However, there is a distinction between \emph{knowing} \sqE{} is sound and \emph{proving} \sqE{} is sound.

  For example, I (trivially) know that any sound rule is sound.
  Yet, it is not the case that I may prove any sound rule is sound.
  For, sound rules may have an arbitrary (finite) number of premises, and I may cease to be before even reading all of the premises.

  Of course, \sqE{} is a simple rule, and typically does not take more than a few moments to prove.
  Yet, it remains the case that I may fail to prove \sqE{} is sound.
  For example, if I have just drunk a considerable amount of wine, or woken from a night of tormented sleep.
  Generally said, I may not be thinking straight.

  To give a concrete example.
  Suppose I were to pause and reason to:
  \(P \rightarrow (Q \rightarrow P) \vdash P\).%
  \footnote{
    None of the following entailments hold:
    \begin{enumerate*}[noitemsep, label=]
    \item
      \(\phi \rightarrow (\psi \rightarrow \phi) \vDash \phi\),
    \item
      \(\psi \rightarrow (\phi \rightarrow \psi) \vDash \phi\),
    \item
      \((\psi \rightarrow \phi) \rightarrow \psi \vDash \phi\).
    \end{enumerate*}
    Constructing a counterexample to each is straightforward.
  }
  Then, apparent proof of \((P \rightarrow Q) \rightarrow P, Q \vdash P \land Q\) is in significant doubt.
  For, I know \(P \rightarrow (Q \rightarrow P) \vdash P\) does not hold.

  Whatever my current reasoning amounts to, it is not \tR{}.
\end{note}

\begin{note}
  Now, bottle of wine or bad sleep, then there may be no \requ{1}.
  I may conclude \(\pv{\phi}{v}\) from \(\Phi\) regardless of whether \(\pv{\psi}{v'}\) from \(\Psi\) is a \fc{}.

  Indeed, given some particularly good wine I may not only conclude \(P \rightarrow (Q \rightarrow P) \vdash P\), but also conclude \(P \land C \vdash O\).%
  \footnote{
    For, \(P \land C\) reads `Pac', \(O\) looks like a pellet, and Pacman likes to eat pellets.
  }

  However, \requ{1} concern particular event.

  And, during \autoref{scen:squish} I am thinking straight.
  So, I am {\color{blue} keen} to whether or not reasoning is \tR{0}.
  Hence, any doubt during \autoref{scen:squish} that reasoning is \tR{0} would lead to preventing the event from developing further.
\end{note}

\begin{note}
  Now, \agpe{my}, and \agpe{my} may be wrong.

  I concluded \((P \rightarrow Q) \rightarrow P, Q \vdash P \land Q\), but it need not be the case that {\color{blue} keen} to whether or not \(\pv{\psi}{v'}\) from \(\Psi\) is a \fc{}.

  Perhaps, I would fail to conclude \(\pv{\psi}{v'}\) from \(\Psi\), or would conclude something which conflicts with concluding \(\pv{\psi}{v'}\) from \(\Psi\).
\end{note}

\subparagraph{Apart from \agpe{my}}

\begin{note}
  Regardless of \agpe{my}, then if \requ{} fails to hold, it is unclear whether reasoning is \tR{}.

  If no \requ{}, then what distinguishes whatever happened with arbitrarily pairing \((P \rightarrow Q) \rightarrow P, Q \vdash P \land Q\) with the value \(\text{True}\)?

  Indeed, conclusion just amounts to pairing.
  Hence, conclude.

  Without \fc{1}, then is nothing to suggest that my reasoning is \tR{}.
\end{note}

\begin{note}
  The motivation for the conditional rests on whether reasoning is \tR{}.

  A \requ{} \emph{seems} to be to be a necessary condition for conclusions which result from \tR{}.%
  \footnote{
    This is not sufficient, though.
    As with lost keys, \requ{} need not be tied to \tR{}.
    With lost keys, \tR{}, but whether that is sufficient for concluding.
  }
  For, if there are no \requ{1}, then there is nothing to indicate the agent would reasoning in accordance with other instances of the type of reasoning.
\end{note}

\subsection{A few propositions about \requ{1}}
\label{sec:propsoitions}

\begin{note}
  We briefly state two propositions which have no significant role in the arguments to follow, but which may be helpful.
\end{note}

\paragraph*{\requ{3} and \issueConstraint{}}

\begin{note}
  The existence of \requ{1} are compatible with \issueConstraint{}.
  \begin{proposition}
    \label{prop:requ-not-n-ce}
    The existence of \requ{1} is compatible with \issueConstraint{}.
  \end{proposition}
  \begin{argument}{prop:requ-not-n-ce}
    In order for counterexample, case in which \ros{} answers \qWhyV{}, but the agent does not have a \wit{}.

    We have not yet consider the way in which \requ{1} connect to \qWhyV{}.
    However, any explicit connexion is not required.
    For, we only need to observe that an agent may have a \wit{} for any \requ{}.
    Then, the \wit{} is always a candidate for \qHowV{}.

    Consider \autoref{scen:squish}.
    At issue is whether the agent would repeat the repeat conclusion of \sqE{} as sound.

    Needs to be the case that \ros{} without \wit{}.

    However, the agent has previously concluded \sqE{} is sound.
    Therefore, the agent has a \wit{} for \ros{}.
  \end{argument}

  Note, however, that \autoref{prop:requ-not-n-ce} is weak.
  Compatible in terms of existential.
  Hence we only needed to show a single case in which \wit{} for \requ{}.
  This does not suggest that \requ{1} are not in tension with \issueConstraint{}.
\end{note}

\paragraph*{No guaranteed \requ{1}}

\begin{note}
  \begin{proposition}
    \label{prop:requ-not-refl}
    \cenLine{
      \begin{itemize*}[noitemsep, label=\(\circ\)]
      \item
        Agent: \vAgent{}
      \item
        Proposition: \(\phi\)
      \item
        Value: \(v\)
      \item
        \pool{2}: \(\Phi\)
      \item
        Event: \(e\)
      \item
        \mbox{ }
      \end{itemize*}
    }

    \begin{itemize}
    \item
      It is not \emph{necessarily} the case that if the agent concludes \(\pv{\phi}{v}\) from \(\Phi\), then \(\pvp{\phi}{v}{\Phi}\) \requ{} of some sub-event.
    \end{itemize}
    \vspace{-\baselineskip}
  \end{proposition}

  In other words, \(\pv{\phi}{v}\) from \(\Phi\) need not be a \fc{} in order for an agent to be concluding \(\pv{\phi}{v}\) from \(\Phi\).

  Suppose event is described.
    Then, it follows that the agent is concluding \(\pv{\phi}{v}\) from \(\Phi\).
    So, first part of \fc{}.
    However, second part of \fc{} does not follow.

    \begin{argument}{prop:requ-not-refl}
      By observing that \(\pv{\phi}{v}\) from \(\Phi\) need not be a \fc{}.

      Second, in this case, some failure.
      Uninteresting case, absence of action.
      Still, weaken the definition.
      More significant, some problem with \(\Phi\).
      But, this does not prevent the agent from failing to notice and hence pairing \(\phi\) with \(v\).
  \end{argument}

  In general, \autoref{prop:requ-not-refl} highlights that there are no trivial \requ{1}.

  Well, in terms of \tR{}.
  Don't need \tR{}.
  Novel results, and so on.

  And, with guidance, removed the need for \tR{} with respect to overall.
\end{note}

\section{\requ{3} and \tR{0}}
\label{cha:requs:sec:add-illu}

\begin{note}
  Though \requ{1} are compatible with \issueConstraint{},%
  \footnote{
    Cf.\ \autoref{prop:requ-not-n-ce}, \autopageref{prop:requ-not-n-ce}.
  }
  \requ{1} have a key role in constructing counterexamples to \issueConstraint{}.

  The goal of this section is to provide general motivation for the existence of a certain kind of \requ{}.

  The central proposition of this section is \autoref{prop:hinge} which, in short, states that certain (non-trivial) proposition-value-\pool{} pairings are entailed by an agent concluding \(\pv{\phi}{v}\) from \(\Phi\) by {\color{red} sTR{}}.

  In turn, we observe this puts pressure on \emph{maintaining} that an agent is concluding \(\pv{\phi}{v}\) from \(\Phi\) by \tR{} while \emph{denying} that other instances of the type of reasoning are \requ{1}.

  Though, before turning to these positive arguments, we briefly clarify what is at issue.
\end{note}

\subsection{Arbitrariness}

\begin{note}
  \(\pvp{\psi}{v'}{\Psi}\) being a \requ{0} an agent concluding \(\pv{\phi}{v}\) from \(\Phi\) amounts, roughly, to a necessary condition:
  \begin{itemize}
  \item
    The agent is concluding \(\pv{\phi}{v}\) from \(\Phi\) \emph{only if} \(\pv{\psi}{v'}\) from \(\Psi\) is a \fc{}.
  \end{itemize}

  Hence, if an agent is concluding \(\pv{\phi}{v}\) from \(\Phi\), it must be the case that \(\pv{\psi}{v'}\) from \(\Psi\) is a \fc{}.

  However, we have granted that any instance of a \emph{conclusion} of \(\pv{\phi}{v}\) from \(\Phi\) may be arbitrary.
  Two observations follow:

  \begin{itemize}
  \item
    It need not be the case that there is any significant event in which the agent is concluding \(\pv{\phi}{v}\) from \(\Phi\).

  \item
    Even if there is some significant event in which the agent is concluding \(\pv{\phi}{v}\) from \(\Phi\) before the agent concludes \(\pv{\phi}{v}\) from \(\Phi\), it is consistent with our general understanding of concluding that \(\pvp{\psi}{v'}{\Psi}\) fails to be a \requ{}.%
    \footnote{
      The present point is similar to issues raised by \citeauthor{Harman:1973ww} (\citeyear{Harman:1973ww}) regarding the proposed equivalence between reasons for which an agent believes something with reasons the agent would offer if asked to justify their belief.
      As \citeauthor{Harman:1973ww} notes, an agent may offer reasons because they think they will convince their audience, not because the agent is compelled by the reasons, etc.
      (\citeyear[Ch.2]{Harman:1973ww})

      To the extent that \citeauthor{Harman:1973ww}'s point is that what holds from an \agpe{} need not actually be the case, the point in the same.
      However, to the extent that \citeauthor{Harman:1973ww} relies on an under-specification of what holds from an \agpe{} --- i.e.\ the distinction between whether \(\phi\) has value \(v\) from the \agpe{} or whether the agent evaluates as true the proposition that their audience is responsive to \(\phi\) having value \(v\), the point is distinct.
    }

    In other words, granting the agent is concluding \(\pv{\phi}{v}\) from \(\Phi\), it may be the agent goes on to conclude \(\pv{\phi}{v}\) from \(\Phi\) regardless of whether \(\pv{\psi}{v'}\) from \(\Psi\) is a \fc{}.
  \end{itemize}

  First point is familiar from the argument given for \autoref{prop:requ-not-refl} (\autopageref{prop:requ-not-refl}).

  No event in which concluding, for relies on the aid of an uninterested third party.
  More broadly, progressive and luck.

  Second point, various examples of proposition-value-\pool{} pairings which fail to be \requ{1}.
  For example, consider \autoref{scen:squish}.
  \sqE{}, issue.
  Concern is indifference.

  Perhaps I really am indifferent.
\end{note}

\begin{note}
  Additional concern, whether \fc{}.
  For, then there must be an event.
  However, as argued, this is not of particular interest.
  Hence, we set aside.
\end{note}

\subsection{\vtC{2}}
\label{sec:tr}

\begin{note}
  \tR{} addresses both points.

  Significant event.
  Agent, of that type.

  \requ{}, for else, difficult to say whether or not reasoning present in instance of concluding is of the type.
\end{note}

\begin{note}
  \begin{proposition}[\vtC{2} and \requ{1}]
    \label{prop:hinge}
    \cenLine{
      \begin{itemize*}[noitemsep, label=\(\circ\)]
      \item
        Agent: \vAgent{}
      \item
        Propositions: \(\phi\), \(\psi\)
      \item
        Values: \(v\), \(v'\)
      \item
        \pool{3}: \(\Phi\), \(\Psi\)
      \item
        \mbox{ }
      \end{itemize*}
    }

    \cenLine{
      \begin{itemize*}[noitemsep, label=\(\circ\)]
      \item
        Event: \(e\)
      \item
        Type of reasoning: \(T\)
      \item
        \mbox{ }
      \end{itemize*}
    }

    \begin{itemize}
    \item[\emph{If}:]
      \begin{enumerate}[label=\alph*., ref=(\alph*), series=propHingeSer]
      \item
        \label{prop:hinge:typical}
        \(e\) is an event such that under description \(d\), \vAgent{} is concluding \(\pv{\phi}{v}\) from \(\Phi\) \emph{only if} \vAgent{} is \vtCV{} \(\pv{\phi}{v}\) from \(\Phi\).
      \end{enumerate}
    \item[\emph{And}:]
      \begin{enumerate}[label=\alph*., ref=(\alph*), resume*=propHingeSer]
      \item
        \label{prop:hinge:rep}
        \(\pvp{\psi}{v'}{\Psi}\) is \tRep{} \(e\).
      \end{enumerate}
    \item[\emph{Then}:]
      \begin{enumerate}[label=\alph*., ref=(\alph*), resume*=propHingeSer]
      \item
        \label{prop:hinge:requ}
        \(\pvp{\psi}{v'}{\Psi}\) is a \requ{} of \(e\) being an event in which \vAgent{} is concluding \(\pv{\phi}{v}\) from \(\Phi\).
      \end{enumerate}
    \end{itemize}
    \vspace{-\baselineskip}
  \end{proposition}

  The key to \autoref{prop:hinge} is \fc{1}.
  \vtC{} extends \tC{} to \fc{1}.
  By assumption, concluding.
  Hence, it must be the case that we get the \fc{}.

  If this seems surprising, keep in mind that all of these definitions are in terms of the material conditional.

  \begin{argument}{prop:hinge}
    We assume both~\ref{prop:hinge:rep} and~\ref{prop:hinge:typical} hold, and suppose for a contradiction~\ref{prop:hinge:requ} does not hold.

    From~\ref{prop:hinge:typical}, \(e\) is an event.

    From~\ref{prop:hinge:rep}, \(\pvp{\psi}{v'}{\Psi}\) is representative of some type \(T\).

    Now, to show \requ{}.
    Needs to be the case that, if not \fc{}, then not event.
    Well, suppose not \fc{}.
    Then, cannot be event.
    For, conflicts with description.
  \end{argument}
\end{note}

\section{Summary}
\label{sec:summary-1}

\begin{note}
  Introduced \requ{1}.

  Motivation in terms of \tC{}.
\end{note}

%%% Local Variables:
%%% mode: latex
%%% TeX-master: "master"
%%% End:


% \begin{note}[Problems of induction]
%   Hence, the sketch does not apply to black ravens.
%   I wouldn't conclude all ravens are black if I saw a white raven.

%   I may worry about shortly seeing a white raven when concluding all ravens are black, and I may refuse to entertain the possibility that the sun will rise tomorrow when planning to mow the grass.

%   However, it's not possible to reason to seeing a white raven, nor is it possible to reason to the sun not rising tomorrow.

%   Abstractly, at issue in~\autoref{illu:lost-key} is the possibility of failing to a reason to some proposition-value pair given \emph{present} information, rather than the possibility of failing to a reason to some proposition-value pair given \emph{new} information.

%   To the extent that problems of induction arise from receiving new information, what is at issue is distinct.%
%   \footnote{
%     See~\textcite{Henderson:2020wb} for more on the problem of induction.
%   }

%   Similar points for external world scepticism.
%   Would not conclude that I have hand if disembodied brain in a vat.

%   However, conclusion is out of reach.
% \end{note}


% \subsection{Intuition}
% \label{sec:intuition-1}

% \begin{note}
%   Granting interpretation is correct, failure to know \fc{} amounts to an something like an undercutting defeater.%
%   \footnote{
%     To my understanding, undercutting defeaters were introduced by \citeauthor{Pollock:1987un} (\citeyear{Pollock:1987un}).
%     And, \citeauthor{Pollock:1987un} defines an undercutting defeater as follows:
%     \begin{quote}
%     R is an \emph{undercutting defeater} for P as a prima facie reason for S to believe Q if and only if
%     \begin{enumerate}[label=(UD\arabic*), ref=(UD\arabic*)]
%     \item
%       \label{pollock:ud:1}
%       P is a reason for S to believe Q and R is logically consistent with P but (P and R) is not a reason for S to believe Q, and
%     \item
%       \label{pollock:ud:2}
%       R is a reason for denying that P wouldn't be true unless Q were true.%
%       \mbox{}\hfill\mbox{(\citeyear[485]{Pollock:1987un})}
%     \end{enumerate}
%   \end{quote}
%   This definition is hard to square with a \requ{}.
%   In particular, \ref{pollock:ud:1}.

%   Issue: P is a reason.
%   By parallel, the reasoning that the agent has done is sufficient for the agent to conclude.
%   However, at issue is precisely whether this is the case.
%   }

%   We borrow the following sketch from \textcite{Worsnip:2018aa}:
%   \begin{quote}
%     Undercutting defeaters, which are easiest to think of in the context of the attitude of belief, are supposed to be considerations that undermine the justification of a belief in a proposition p not necessarily by providing (sufficient) positive evidence to think that p is false, but rather merely by suggesting (perhaps misleadingly) that one's reasons for believing p are no good, in a way that neutralizes or mitigates their justificatory or evidential force.%
%     \mbox{}\hfill\mbox{(\citeyear[29]{Worsnip:2018aa})}
%   \end{quote}

%   In particular, concluding.
%   At issue is not whether the \(\phi\) has value \(v\), but whether the agent's reasoning from \(\Phi\) to \(\pv{\phi}{v}\) is sufficient for the agent to conclude \(\phi\) has value \(v\) (from \(\Phi\)).

%   Justification, and this is one way to go.
%   However, not tied to justification.
% \end{note}

% \begin{note}
%   This serves dual purpose.

%   First, why \requ{}.
%   For, undercut reasoning.
% \end{note}
