\chapter{\requ{3}}
\label{cha:requs}

\begin{note}
  \autoref{cha:typical} introduced \sR{0}.

  Some generality.
\end{note}

\begin{note}
  \autoref{cha:fcs} introduced \fc{1}.

  \fc{1} are such that:
  \begin{itemize}
  \item
    \ros{} holds between \(\pv{\psi}{v'}\) and \(\Psi\) (for the agent).
  \item
    The agent does not have a \wit{} for the \ros{} between \(\pv{\psi}{v'}\) and \(\Psi\).
  \end{itemize}
\end{note}

\begin{note}
  Present chapter, \requ{1}.

  The role of a \requ{} is to bind \fc{} to concluding, and in cases of interest, \sR{0} will motivate bind.
\end{note}

\begin{note}
  The present chapter established any relation between \(\pv{\psi}{v'}\) from \(\Psi\) being a \fc{} and the agent \emph{concluding} \(\pv{\phi}{v}\) from \(\Phi\).

  Idea which we term a \requ{}.

  Roughly, agent is concluding \(\pv{\phi}{v}\) from \(\Phi\) only if \(\pv{\psi}{v'}\) is a \fc{}.
\end{note}

\begin{note}
  \requ{3} link \fc{1} to concluding.

  And, important function in generating tension to \issueConstraint{}.

  However, whether an agent concludes.
  Still, this additional link is not so difficult.
  For, if agent concludes, agent was concluding.

  Link \requ{1} to conclusions and \issueConstraint{} in \autoref{cha:binding}.

  \requ{3} are not designed to presuppose tension with \issueConstraint{}.
\end{note}

\begin{note}
  The chapter is divided into three sections:
  \begin{itemize}
  \item
    \TOCLine{cha:requs:sec:infl}

    \ninf{}.

    Preliminary, general, idea to understand \requ{1}
  \item
    \TOCLine{cha:requs:sec:definition}

    Introduction, definition, intuition, and \illu{1}
  \item
    \TOCLine{cha:requs:sec:add-illu}

    \sR{}, in additional detail.
  \end{itemize}
\end{note}

\section{\ninf{2}}
\label{cha:requs:sec:infl}

\begin{note}
  Consider the following \scen{0}:

  \begin{scenario}[Apples]
    \label{scen:apples}
    Grey is walking to town to buy some apples.
    Grey notices a bicycle for sale.
    Grey purchases the bicycle, gets on the bicycle, and starts cycling to town.
  \end{scenario}

  Grey is performing some action.
  Walking.
  Progressive, incomplete action walks to the shops.
  Grey then performs an action, purchasing a bicycle.
  Following the purchase of the bicycle Grey is no longer walking to town.
  Instead, Grey is cycling to town.

  Grey does something so that it is not true that Grey is walking to town.

  Depending on \agpe{your}, you may think that Grey was not walking to town.%
  \footnote{
    For, perfection, bound to see the bicycle for sale, has been wanting a bicycle, etc.
  }
  However, our interest is not with whether or not Grey was walking and then cycling.
  Our interest is with the basic two-part observation:
  \begin{itemize}[noitemsep]
  \item
    Shortly after purchasing the bicycle, Grey is not walking to town.
  \item
    Grey is not walking to town as a result of an action Grey performed.
  \end{itemize}

  In this respect, when Grey started to ride their bicycle, Grey exerted `\ninf{}' over whether or not they were walking to town.

  For, regardless of whether or not Grey was walking to town, there is no longer a development of the event such that developed event is an event in which Grey walks into town --- some significant sub-event, Grey was on their bicycle.
\end{note}

\begin{note}
  We define \ninf{} as follows:
  \begin{definition}[\ninf{2}]
    \label{def:ninf}
    \cenLine{
      \begin{itemize*}[noitemsep, label=\(\circ\)]
      \item
        Agent: \vAgent{}
      \item
        Event: \(e\)
      \item
        Action descriptions: \(\alpha, \beta\)
      \item
        \mbox{ }
      \end{itemize*}
    }

    \begin{itemize}
    \item
      \vAgent{} has \ninf{} over whether or not \(\text{Prog}(e, \alpha)\) is true.
    \end{itemize}

    \emph{If and only if}

    \begin{itemize}
    \item
      Both~\ref{def:ninf:action} and~\ref{def:ninf:prog} are true:
      \begin{enumerate}[label=\alph*., ref=(\alph*)]
      \item
        \label{def:ninf:action}
        There is some action \(\beta\) that \vAgent{} may immediately perform.
      \item
        \label{def:ninf:prog}
        \(\text{Prog}(e', \alpha)\) would be not be true in the event \(e'\) of \vAgent{} doing \(\beta\).
      \end{enumerate}
    \end{itemize}
    \vspace{-\baselineskip}
  \end{definition}

  With respect to \autoref{scen:apples}:

  \begin{itemize}[noitemsep]
  \item
    The agent is Grey.
  \item
    The event \(e\) spans some period of time which starts when or after Grey set out for town, and includes the Purchase of the bicycle.
  \item
    \(\alpha\) is the action `Grey walks to town'.
  \item
    \(\beta\) is the action `Grey rides some distance on their bicycle'.
  \end{itemize}
\end{note}

\begin{note}
  The purchase of a bicycle by Grey was not necessary for Grey to have \ninf{} over whether or not Grey was walking to town.
  For, Grey may have chosen to turn back home or raid the local orchard.

  Other examples of \ninf{0}:

  \begin{itemize}[noitemsep]
  \item
    Finishing a book late at night \hfill Turn off the lights.
  \item
    Listening to a speech \hfill Leave before the speech is over.
  \item
    Playing a game of chess \hfill Leaving in frustration.
  \end{itemize}

  No \ninf{}:

  \begin{itemize}
  \item
    Taking a shower \hfill After having been in the shower for a few minutes.

    For, still true of event that taking a shower.
    Simply ended.
  \item
    Hearing a part song on the radio \hfill After recognising the song.

    For, enough to recognise.
  \item
    Playing chequers \hfill After having made a few moves.

    For, played.
    Difference to above is that have no played a game.
  \end{itemize}

  Noted that it does not matter whether or not \(e\) is such that \(\text{Prog}(e, \alpha)\) is true.
  Hence, \ninf{} over finding oranges on the moon.
  For, begin Cartesian meditation.
\end{note}

\begin{note}
  \nocite{Peacocke:2021aa}
  \autoref{scen:apples} and the examples used to illustrate an agent having or not having \ninf{} were all non-mental actions.
  With the exception of one:
  Grey choosing to turn back home.
  In this case, the relevant \(\beta\) action.

  However, our interest is \(\alpha\).
  Specifically, an agent has \ninf{} over whether or not they are \emph{concluding} \(\pv{\phi}{v}\) from \(\Phi\).

  \begin{proposition}
    \label{prop:ninfConcl}
    There are instances in which an agent exerts \ninf{0} over whether or not they are concluding \(\pv{\phi}{v}\) from \(\Phi\).
  \end{proposition}

  \begin{argument}{prop:ninfConcl}
    Two points.
    \begin{itemize}[noitemsep]
    \item
      In order to be concluding, need to develop such that conclude.
    \item
      The agent may stop what they are doing.
    \end{itemize}
  \end{argument}

  \begin{note}
    \begin{itemize}
    \item
      Game of chess is lost. | Flip the table.
    \item
      Apparent testimony is true. | Read confession.
    \item
      \(math\) | Get bored.
    \end{itemize}
  \end{note}
\end{note}

\begin{note}
  Note, \autoref{prop:ninfConcl} does not entail agent has \ninf{} over whether they are \emph{reasoning} about whether \(\pv{\phi}{v}\) follows from \(\Phi\).
  In parallel to examples used to illustrate when an agent does not have \ninf{}, sufficiently developed so that the agent has reasoned.

  Note:
  It does not follow from \autoref{prop:ninfConcl} that the agent had a choice over how event develops, if the agent does not exert \ninf{}.
  It may be that if the agent continued the event, they would have concluded \(\pv{\phi}{v}\) from \(\Phi\).
  And, may remain the case that \(\pv{\phi}{v}\) from \(\Phi\) is a \fc{}.
\end{note}

\section{\requ{3}}
\label{cha:requs:sec:definition}

\begin{note}
  \autoref{cha:requs:sec:infl}, \ninf{}.
  General.

  Interest is with cases in which:
  \begin{itemize}
  \item
    \ninf{} relate to whether or not an event is an instance of an agent concluding \(\pv{\phi}{v}\) from \(\Phi\).
  \item
    \ninf{} if \(\pvp{\psi}{v'}{\Psi}\) is not a \fc{}.
  \end{itemize}

  In these cases, \(\pvp{\psi}{v'}{\Psi}\) is a \requ{}:

  \begin{definition}[A \requ{0}]
    \label{def:requ}
    \cenLine{
      \begin{itemize*}[noitemsep, label=\(\circ\)]
      \item
        Agent: \vAgent{}
      \item
        Propositions: \(\phi\), \(\psi\)
      \item
        Values: \(v\), \(v'\)
      \item
        \poP{3}: \(\Phi\), \(\Psi\)
      \item
        Event: \(e\)
      \item
        \mbox{ }
      \end{itemize*}
    }

    \begin{itemize}
    \item
      \(\pvp{\psi}{v'}{\Psi}\) is a \emph{\requ{}} of \(e\) for \vAgent{}, with respect to \(\pvp{\phi}{v}{\Phi}\).
    \end{itemize}

    \emph{If and only if}

    \begin{itemize}
    \item
      The following conditional is true:
      \begin{itemize}
      \item[\emph{If}:]
        \begin{enumerate}[label=\alph*., ref=(\alph*), series=requDefSeries]
        \item
          \label{def:requ:nK}
          \(\pv{\psi}{v'}\) from \(\Psi\) is not a \fc{} for \vAgent{} throughout \(e\).
        \end{enumerate}
      \item[\emph{Then}:]
        \begin{enumerate}[label=\alph*., ref=(\alph*), resume*=requDefSeries]
        \item
          \label{def:requ:nC}
          \(e\) is not an event in which \vAgent{} is concluding \(\pv{\phi}{v}\) from \(\Phi\).\newline
          \mbox{ }\hfill(\emph{Due, at least in part, to \ninf{} as a result of \ref{def:requ:nK}}.)
        \end{enumerate}
      \end{itemize}
    \end{itemize}
    \vspace{-\baselineskip}
  \end{definition}
\end{note}

\begin{note}
  If an agent doubts that \(\pv{\psi}{v'}\) is a \fc{}, then the agent is not concluding \(\pv{\phi}{v}\) from \(\Phi\).%
  \footnote{
    \autoref{def:requ} parallels the definition of a \pevent{} (\autoref{def:potenital-event}, \autopageref{def:potenital-event}).
    The key difference is with respect to whether the progressive is true as a result of the agent doing the relevant action.
    With respect to \ninf{}, the progressive is false, while with respect to a \pevent{0} the progressive is true.

    However, circumstances are different.
    \pevent{} when progressive is not true of event.
    \ninf{} when progressive is true.

    Note, exertion of \ninf{} is compatible with progressive being true.
  }

  Antecedent is true, some action, \ninf{}, not concluding.

  Intuitively, if \(\pv{\psi}{v'}\) is not a \fc{} from \(\Psi\), then the agent stops the event from developing into an event in which the agent concludes \(\pv{\phi}{v}\) from \(\Phi\).

  Indeed, even if the event could have developed into an event in which the agent concluded \(\pv{\phi}{v}\) from \(\Phi\).

  In part, it does not need to be the case that the agent would be concluding if not for performing the action.
\end{note}

\begin{note}
  Our interest with \requ{1} is with regard to \ninf{0}.

  Initial example, result was not worth the effort, even if obtained.
  The agent may well have obtained, but as the agent decided the result was not worth the effort, their decision ensures they are not concluding.
\end{note}

\begin{note}
  The qualifier `due to' is appended to \ref{def:requ:nC} to ensure that the conditional captures the agent's \ninf{}, rather than being trivially true because the agent is not concluding \(\pv{\phi}{v}\) from \(\Phi\).%
  \footnote{
    See, for example, \citeauthor{Lewis:1997wg}'s (\citeyear{Lewis:1997wg}) discussion of finkish dispositions.
  }
\end{note}

\begin{note}
  \requ{1} are substantial in a way general \ninf{} is not.

  Absence of a \fc{} influences present action.

  It is not the case that from the \agpe{}.
  Rather, it is the case that if no \fc{} then not concluding.

  Rather than abstract motivation, pair of \illu{1}.
\end{note}

\begin{note}
  Before \illu{1}, \autoref{prop:requ-fc} expands \autoref{def:requ} via the definition of a \fc{}.

  \begin{proposition}[A \requ{0}, expanded]
    \label{prop:requ-fc}
    \cenLine{
      \begin{itemize*}[noitemsep, label=\(\circ\)]
      \item
        Agent: \vAgent{}
      \item
        Propositions: \(\phi\), \(\psi\)
      \item
        Values: \(v\), \(v'\)
      \item
        \poP{3}: \(\Phi\), \(\Psi\)
      \item
        Event: \(e\)
      \item
        \mbox{ }
      \end{itemize*}
    }

    \begin{itemize}
    \item
      \(\pvp{\phi}{v'}{\Psi}\) is a \emph{\requ{}} of \(e\) for \vAgent{}, \emph{if and only if}:
      \begin{itemize}
      \item[\emph{If}:]
        \begin{enumerate}[label=\alph*., ref=(\alph*), series=requDefSeries]
        \item
          \label{prop:requ-fc:nk}
          Throughout \(e\), either~\ref{prop:requ-fc:nk:psi} or~\ref{prop:requ-fc:nk:no-conf} are true:
          \begin{enumerate}[label=\roman*., ref=(\roman*)]
          \item
            \label{prop:requ-fc:nk:psi}
            There is no \pevent{} in which \vAgent{} concludes \(\pv{\psi}{v'}\) from \(\Psi\).
          \item
            \label{prop:requ-fc:nk:no-conf}
            There is a \pevent{} in which \vAgent{} concludes something incompatible with a conclusion of \(\pv{\psi}{v'}\) from \(\Psi\).
          \end{enumerate}
        \end{enumerate}
      \item[\emph{Then}:]
        \begin{enumerate}[label=\alph*., ref=(\alph*), resume*=requDefSeries]
        \item
          \label{prop:requ-fc:ne}
          \(e\) is not an event in which \vAgent{} is concluding \(\pv{\phi}{v}\) from \(\Phi\).
        \end{enumerate}
      \end{itemize}
    \end{itemize}
    \vspace{-\baselineskip}
  \end{proposition}

  \begin{argument}{prop:requ-fc}
    From the definition of a \requ{} (\autoref{def:requ}, \autopageref{def:requ}) and the definition of a \fc{} (\autoref{def:fc}, \autopageref{def:fc}).
  \end{argument}
\end{note}

\subsection{\illu{3}}
\label{sec:illu3}

\begin{note}
  First, \scen{0} in which \requ{} and agent is not concluding due to \requ{}.
  This \illu{0} without detailed discussion to motivate idea.
  Further, \requ{1} such that not concluding will not be of significant interest.

  Second, \scen{0} in which \requ{} and agent is concluding.
  This \illu{0}, detailed discussion.
  However, our initial discussion will be intuitive.
\end{note}

\subsubsection{\requ{2} and not concluding}

\begin{note}
  \scen{0} in which an agent is not concluding due to \requ{}.
\end{note}

\begin{note}
  \begin{illustration}[Lost keys]
    \label{illu:lost-key}
    I think I might have lost my keys.
    I usually leave place my keys on the right side of my desk, near a copy of~\citeauthor{Vickers:1989tr}'s~\citetitle{Vickers:1989tr} which I've been saving for a rainy day.
    And, my keys aren't there.

    I've searched over the desk, under the desk, and beside the desk.
    And, I haven't found my keys.

    Still, I haven't (yet, at least) \emph{concluded} that I've lost my keys.

    For, there might still be some place I haven't looked.
    If I think a little harder a figure out where that place is, I would conclude my keys might be in that place.
    And, my keys aren't lost if they are in that place.
    So, I might conclude that my keys aren't lost, which conflicts with concluding my keys are lost.

    I do not go on to conclude I have lost my keys.
  \end{illustration}
\end{note}

\begin{note}
  Filling in the details of \autoref{illu:lost-key}:
  \begin{itemize}[noitemsep]
  \item
    I am the agent.
  \item
    \(\phi\) is the proposition: `I have lost my keys'.
  \item
    \(\psi\) is a some proposition: `My keys are not in location \(l\)'
  \item
    Both \(v\) and \(v'\) are the value: `True'.
  \item
    The pools of premises \(\Phi\) and \(\Psi\) are left unspecified.
  \end{itemize}

  Hence, the relevant instance of the conditional by which a \requ{} is defined is:

  \begin{enumerate}[label=]
  \item
    \begin{itemize}
    \item[\emph{If}:]
      If \(\pv{\text{My keys are not in location }l}{\text{True}}\) from \(\Psi\) is a not a \fc{}.
    \item[\emph{Then}:]
      I am not concluding \(\pv{\text{I have lost my keys}}{\text{True}}\) from \(\Phi\).
    \end{itemize}
  \end{enumerate}

  The antecedent is true, and hence the consequent is true due to \ninf{} I exert over the event.

  Antecedent is true.
  Have no checked.
  Hence, exert \ninf{}.

  The event of reasoning about lost keys is not an event in which I am concluding I have lost my keys, because it will not develop.
\end{note}

\begin{note}
  You may disagree with the tension I see in~\autoref{illu:lost-key}.
  Perhaps it's fine to conclude my keys are lost while allowing for the possibility that the keys are some place I haven't yet thought of.
  My goal is only to convince you that my refusal to conclude I've lost my keys is intelligible.

  At issue is only whether \(\pvp{\psi}{v'}{\Psi}\) is a \requ{} for me in a particular event in which I am reasoning about whether I have lost my keys.

  Hence, if one does not worry about the possibility that the keys are some place I haven't yet thought of, \(\pvp{\psi}{v'}{\Psi}\) would not be a \requ{1} of concluding I have lost my keys.
\end{note}

\begin{note}
  \fc{1} focus on event.
  However, the existence of an event is secondary.
  Present concern that applies to any reasoning about where keys are.

  This is what secures that there is no event.
\end{note}

\subsubsection{Sound rules}

\begin{note}
  Here, I want to state that the agent has done the proof a number of times.
  So, the agent knows that there is a \pevent{}.
\end{note}

\begin{note}
  \phantlabel{squish-elimination-proof}

  \begin{restatable}[Squish elimination]{illustration}{scenarioPLSquish}
    \label{scen:squish}
    It is late morning on a sunny day.
    I ate a good breakfast, and drank some nice coffee.
    I have completed a handful of syntactic proofs for entailments of propositional logic using the basic rules of inference in a Fitch-style system.

    I create the following syntactic proof:
    \begin{center}
      \begin{fitch}
        \phantlabel{illu:sP:1}\fa (P \rightarrow Q) \rightarrow P \\
        \phantlabel{illu:sP:2}\fj Q \\
        \phantlabel{illu:sP:3}\fa P & \sqE{}:\hyperref[illu:sP:1]{1} \\
        \phantlabel{illu:sP:4}\fa P \land Q & \(\land\)\textbf{Intro:} \hyperref[illu:sP:2]{2},\hyperref[illu:sP:3]{3}
      \end{fitch}
    \end{center}

    Still, I haven't yet concluded \((P \rightarrow Q) \rightarrow P, Q \vdash P \land Q\).

    For, if \sqE{} is not a sound rule of inference, then \((P \rightarrow Q) \rightarrow P, Q\) may not entail \(P \land Q\).

    I go on to conclude \((P \rightarrow Q) \rightarrow P, Q \vdash P \land Q\).
  \end{restatable}

  The relevant propositions, values, and \poP{1} are as follows:
  \begin{itemize}[noitemsep]
  \item
    I am the agent.
  \item
    \(\phi\) is the proposition: `\((P \rightarrow Q) \rightarrow P, Q \vdash P \land Q\)'.
  \item
    \(\psi\) is a some proposition: `\sqE{} is sound'
  \item
    Both \(v\) and \(v'\) are the value: `True'.
    And,
  \item
    The pools of premises \(\Phi\) and \(\Psi\) are left unspecified.%
    \footnote{
      Note, premises of reasoning.
      Distinct from premises of deduction.
    }
  \end{itemize}

  Hence, the relevant instance of the conditional by which a \requ{} is defined is:

  \begin{enumerate}[label=]
  \item
    \begin{itemize}
    \item[\emph{If}:]
      If \(\pv{\text{\sqE{} is sound}}{\text{True}}\) from \(\Psi\) is not a \fc{}.
    \item[\emph{Then}:]
      I am not concluding \(\pv{(P \rightarrow Q) \rightarrow P, Q \vdash P \land Q}{\text{True}}\) from \(\Phi\).
    \end{itemize}
  \end{enumerate}

  In contrast to \autoref{illu:lost-key}, \autoref{scen:squish} leads to a conclusion.
  Is this really a \requ{}?
\end{note}

\paragraph{Motivation}

\begin{note}
  Start with understanding from \agpe{my}, expand.

  However, briefly expand on \sqE{}.
\end{note}

\begin{note}
  \begin{definition}[\sqE{}]
    \label{def:sque}
    \sqE{} is the following rule:
    \begin{center}
      \begin{fitch}
        \ftag{\text{\scriptsize \emph{i}}}{\fa (\phi \rightarrow \psi) \rightarrow \phi} \\
        \ftag{\text{\scriptsize }}{\fa \vdots } \\
        \ftag{\text{\scriptsize \emph{j}}}{\fa \phi } & \sqE{}:\emph{i} \\
      \end{fitch}
    \end{center}
  \end{definition}

  \begin{proposition}[Soundness of \sqE{}]
    \label{prop:sqE-sound}
    \sqE{} is sound.
  \end{proposition}

  \begin{argument}{prop:sqE-sound}
    Rather than prove \sqE{} is sound (which would require a detailed statement of the proof system in question), we prove that the corresponding semantic entailment holds:

    Let \(v\) be an arbitrary (truth-functional) valuation, and assume \(v((\phi \rightarrow \psi) \rightarrow \phi) = \text{True}\).
    Further, assume for contradiction \(v(\phi) = \text{False}\).

    As \(v(\phi) = \text{False}\), it immediately follows that \(v(\phi \rightarrow \psi) = \text{True}\).
    Therefore, by the first assumption, it must be the case that \(v(\phi) = \text{True}\).
    This contradictions the second assumption.
    Hence, \((\phi \rightarrow \psi) \rightarrow \phi \vDash \phi\).
  \end{argument}
\end{note}


\subparagraph{From \agpe{my}}

\begin{note}
  \sR{2}.
  Only conclude if reasoning amounts to proving.

  Throughout \autoref{scen:squish} I \emph{know} \sqE{} is sound.
  Prior to \autoref{scen:squish} I have proved \sqE{} is sound on various occasions using the same basic observations made in the argument for \autoref{prop:sqE-sound}.

  However, there is a distinction between \emph{knowing} \sqE{} is sound and \emph{proving} \sqE{} is sound.

  For example, I (trivially) know that any sound rule is sound.
  Yet, it is not the case that I may prove any sound rule is sound.
  For, sound rules may have an arbitrary (finite) number of premises, and I may cease to be before even reading all of the premises.

  Of course, \sqE{} is a simple rule, and typically does not take more than a few moments to prove.
  Yet, it remains the case that I may fail to prove \sqE{} is sound.
  For example, if I have just drunk a considerable amount of wine, or woken from a night of tormented sleep.
  Generally said, I may not be thinking straight.

  To give a concrete example.
  Suppose conclude:
  \(P \rightarrow (Q \rightarrow P), Q \vdash P \land Q\).%
  \footnote{
    None of the following entailments hold:
    \begin{enumerate*}[noitemsep, label=]
    \item
      \(\phi \rightarrow (\psi \rightarrow \phi) \vDash \phi\),
    \item
      \(\psi \rightarrow (\phi \rightarrow \psi) \vDash \phi\),
    \item
      \((\psi \rightarrow \phi) \rightarrow \psi \vDash \phi\).
    \end{enumerate*}
    Constructing a counterexample to each is straightforward.
  }
  Doubt that I have proved \((P \rightarrow Q) \rightarrow P, Q \vdash P \land Q\).
  I have a what might be a syntactic proof.
  However, the apparent proof may just be an association.
  And, plausibly is if willing to `prove' syntactic entailments which do not hold.
\end{note}

\begin{note}
  Now, bottle of wine or bad sleep, may conclude \(P \rightarrow (Q \rightarrow P), Q \vdash P \land Q\).
  However, \requ{} concerns current event.

  I am thinking straight, which amounts to sensitivity regarding whether or not reasoning is \sR{0}.

  Hence, any doubt during \autoref{scen:squish} that reasoning is \sR{0} would lead to preventing the event from developing further.
  Further, I would refrain from concluding any syntactic entailment holds.
\end{note}

\begin{note}
  Now, \agpe{my}, and \agpe{my} may be wrong.
  I concluded \((P \rightarrow Q) \rightarrow P, Q \vdash P \land Q\), but it need not be the case that proof.
  Indeed, bottle of wine.
  Given enough wine, conclude arbitrary syntactic entailments.

  What is to say I'm not, at present, wasted.
\end{note}

\subparagraph{Apart from \agpe{my}}

\begin{note}
  Regardless of \agpe{my}, then if \requ{} fails to hold, it is unclear whether reasoning is \sR{}.

  Only to highlight a causal connexion.
  And, causal connexion is insufficient.

  Concern about justification, implicit in \scen{0} as described from \agpe{my}.
  However, specific worry is that there is no generality to reasoning.

  If \agpe{my} is wrong, then it is the case that reasoning is not \sR{}.
  Conclude only if \sR{}.
\end{note}

\begin{note}
  The motivation for the conditional rests on whether reasoning is \sR{}.

  A \requ{} is seems to be to be a necessary condition for conclusions which result from \sR{}.

  This is not necessary, though.
  As with lost keys, \requ{} need not be tied to \sR{}.
  With lost keys, \sR{}, but whether that is sufficient for concluding.

  Motivation for \requ{1} and concluding rests on \sR{}, but is plausibly broader.
\end{note}

\begin{note}
  Keep in mind distinction.
  Concluding, versus concluding via \sR{0}.
  It need not be the case that if not \sR{0} then not concluding.
  \requ{} is doing substantial work.
\end{note}

\subsection{\requ{} and concludes}
\label{sec:requ-concludes}

\begin{note}
  So, interactions with either of these two things.

  \begin{restatable}{proposition}{propRequConcludesFC}
    If \(e\) event in which \vAgent{} concludes, \(e^{\flat}\) sub-event in which the agent is concluding, and \requ{} of \(e\), then throughout \(e^{\flat}\):
    \begin{itemize}
    \item
      \pevent{} in which \vAgent{}
    \item
      No \pevent{} in which incompatible.
    \end{itemize}
  \end{restatable}
\end{note}

\subsection{Intuition}
\label{sec:intuition-1}

\begin{note}
  Granting interpretation is correct, failure to know \fc{} amounts to an something like an undercutting defeater.%
  \footnote{
    To my understanding, undercutting defeaters were introduced by \citeauthor{Pollock:1987un} (\citeyear{Pollock:1987un}).
    And, \citeauthor{Pollock:1987un} defines an undercutting defeater as follows:
    \begin{quote}
    R is an \emph{undercutting defeater} for P as a prima facie reason for S to believe Q if and only if
    \begin{enumerate}[label=(UD\arabic*), ref=(UD\arabic*)]
    \item
      \label{pollock:ud:1}
      P is a reason for S to believe Q and R is logically consistent with P but (P and R) is not a reason for S to believe Q, and
    \item
      \label{pollock:ud:2}
      R is a reason for denying that P wouldn't be true unless Q were true.%
      \mbox{}\hfill\mbox{(\citeyear[485]{Pollock:1987un})}
    \end{enumerate}
  \end{quote}
  This definition is hard to square with a \requ{}.
  In particular, \ref{pollock:ud:1}.

  Issue: P is a reason.
  By parallel, the reasoning that the agent has done is sufficient for the agent to conclude.
  However, at issue is precisely whether this is the case.
  }

  We borrow the following sketch from \textcite{Worsnip:2018aa}:
  \begin{quote}
    Undercutting defeaters, which are easiest to think of in the context of the attitude of belief, are supposed to be considerations that undermine the justification of a belief in a proposition p not necessarily by providing (sufficient) positive evidence to think that p is false, but rather merely by suggesting (perhaps misleadingly) that one’s reasons for believing p are no good, in a way that neutralizes or mitigates their justificatory or evidential force.%
    \mbox{}\hfill\mbox{(\citeyear[29]{Worsnip:2018aa})}
  \end{quote}

  In particular, concluding.
  At issue is not whether the \(\phi\) has value \(v\), but whether the agent's reasoning from \(\Phi\) to \(\pv{\phi}{v}\) is sufficient for the agent to conclude \(\phi\) has value \(v\) (from \(\Phi\)).

  Justification, and this is one way to go.
  However, not tied to justification.
\end{note}

\subsection{What a \requ{} is not}

\begin{note}
  \requ{} isn't saying that if try and fail then no conclusion.
  \requ{} is stronger, if no guarantee at present, then no conclusion.
\end{note}

\subsection{Propositions}
\label{sec:propsoitions}

\begin{note}
  We briefly state two propositions which have no significant role in the arguments to follow, but which may be helpful.
\end{note}

\begin{note}
  The existence of \requ{1} are compatible with \issueConstraint{}.
  \begin{proposition}
    \label{prop:requ-not-n-ce}
    The existence of \requ{1} is compatible with \issueConstraint{}.
  \end{proposition}
  \begin{argument}{prop:requ-not-n-ce}
    In order for counterexample, case in which \ros{} answers \qWhyV{}, but the agent does not have a \wit{}.

    We have not yet consider the way in which \requ{1} connect to \qWhyV{}.
    However, any explicit connexion is not required.
    For, we only need to observe that an agent may have a \wit{} for any \requ{}.
    Then, the \wit{} is always a candidate for \qHowV{}.

    Consider \autoref{scen:squish}.
    At issue is whether the agent would repeat the repeat conclusion of \sqE{} as sound.

    Needs to be the case that \ros{} without \wit{}.

    However, the agent has previously concluded \sqE{} is sound.
    Therefore, the agent has a \wit{} for \ros{}.
  \end{argument}

  Note, however, that \autoref{prop:requ-not-n-ce} is weak.
  Compatible in terms of existential.
  Hence we only needed to show a single case in which \wit{} for \requ{}.
  This does not suggest that \requ{1} are not in tension with \issueConstraint{}.
\end{note}

\begin{note}
  The second proposition may help clarify what the existence of \requ{1} amounts to:

  \begin{proposition}
    \label{prop:requ-not-refl}
    \cenLine{
      \begin{itemize*}[noitemsep, label=\(\circ\)]
      \item
        Agent: \vAgent{}
      \item
        Proposition: \(\phi\)
      \item
        Value: \(v\)
      \item
        \poP{2}: \(\Phi\)
      \item
        Event: \(e\)
      \item
        \mbox{ }
      \end{itemize*}
    }

    \begin{itemize}
    \item
      \(\pvp{\phi}{v}{\Phi}\) is not necessarily a \emph{\requ{}} of \(e\) for \vAgent{}, with respect to \(\pvp{\phi}{v}{\Phi}\).
    \end{itemize}

    (%
    I.e.\ need not be the case that \(\pv{\phi}{v}\) from \(\Phi\) is a \fc{} in order for \vAgent{} to be concluding \(\pv{\phi}{v}\) from \(\Phi\).%
    )
  \end{proposition}
  \begin{argument}{prop:requ-not-refl}
    It need not be the case that the agent knows that \(\pv{\phi}{v}\) from \(\Phi\) is a \fc{}.

    The simplest cases are at the outset.
    It need not be the case that \citeauthor{Maksimova:1977un} knew there are exactly five intermediate logics that have the interpolation property was a \fc{} before proving such.

    However, the same hold for when the agent pairs \(\phi\) with \(v\).
    For, the agent need not know they would repeat the reasoning.
    To \illu{0}, consider being guided through a complex argument.
    Follow along, and conclude.
    At each step, do the reasoning, the guide highlights which sub-conclusions to draw.
    However, guide goes away.
    And, given complexity, no repetition without guide.
  \end{argument}

  In general, \autoref{prop:requ-not-refl} highlights that there are no trivial \requ{1}.

  Well, in terms of \sR{}.
  Don't need \sR{}.
  Novel results, and so on.

  And, with guidance, removed the need for \sR{} with respect to overall.
\end{note}

\section{\requ{3}, competence}
\label{cha:requs:sec:add-illu}

\begin{note}
  \requ{3} have a key role in identifying counterexamples to \issueConstraint{}.

  Important that there are cases in which \requ{1} exist.
  At issue is whether \illu{3}~\ref{illu:lost-key}~and~\ref{scen:squish} illustrate instances of \requ{1}.
  At issue is \emph{not} whether \requ{1} are the correct interpretation of \scen{1} that parallel \illu{3}~\ref{illu:lost-key}~and~\ref{scen:squish}.

  These two issues are related.
  If there is some alternative interpretation of \illu{3}~\ref{illu:lost-key}~and~\ref{scen:squish}, then deny that \requ{1} exist.

  Hence, general motivation for the existence of \requ{}.
  Specifically, the goal of this section is to extend discussion of the way competence and performance motivated \autoref{scen:squish}.

  We will set \autoref{illu:lost-key} aside.
  For, the importance of \requ{1} is limited to cases in which \(\pvp{\psi}{v'}{\Psi}\) is a \requ{0} of concluding \(\pv{\phi}{v}\) from \(\Phi\) and the agent goes on to conclude \(\pv{\phi}{v}\) from \(\Phi\).

  Hence, while cases such as \autoref{illu:lost-key} in which an agent does not conclude due to a \requ{} motivate the general idea of a \requ{}, nothing in particular hangs on such cases.
\end{note}

\begin{note}
  \illu{3} of \requ{1}.
  Interest is in cases where

  \begin{itemize}
  \item
    \(\pvp{\psi}{v'}{\Psi}\) is a \requ{} of concluding \(\pv{\phi}{v}\) from \(\Phi\).
  \item
    The agent is concluding \(\pv{\phi}{v}\) from \(\Phi\).
  \end{itemize}
  For, in such cases:
  \begin{itemize}
  \item
    \(\pv{\psi}{v'}\) from \(\Psi\) is a \fc{}.
  \item
    Agent is sufficiently `sensitive' to whether or not \(\pv{\psi}{v'}\) from \(\Psi\) is a \fc{}.
  \item
    \ninf{} due to sensitivity.
  \end{itemize}
  In these cases, these three points may be objected to.

  However, interest is limited to second and third.

  \begin{itemize}
  \item
    Whether an agent is sensitive to whether or not the agent knows \(\pv{\psi}{v'}\) from \(\Psi\) is a \fc{}.
  \item
    \ninf{2} due to sensitivity.
  \end{itemize}
\end{note}

\begin{note}
  \begin{illustration}[Sudoku]
    \label{illu:gist:sudoku}
    % https://tex.stackexchange.com/questions/91422/tikz-sudoku-circle-and-connect-with-lines-some-cells
    Consider the following Sudoku puzzle:%
    \footnote{
      From~\textcite[84]{Coussement:2007up}.
    }
    \vspace{\baselineskip}

    \mbox{ }\hfill%
    \begin{adjustbox}{minipage=0.45\linewidth,scale=1}
      \centering
      \begin{tikzpicture}[scale=.5]
        \begin{scope}
          \draw (0, 0) grid (9, 9);
          \draw[very thick, scale=3] (0, 0) grid (3, 3);
          \setcounter{row}{1}
          % Single entries
          \setrow { }{ }{ }  { }{ }{ }  {1}{ }{ }
          \setrow { }{ }{ }  { }{ }{ }  { }{5}{ }
          \setrow {9}{ }{ }  { }{ }{ }  { }{ }{2}
          \setrow { }{ }{3}  { }{2}{ }  { }{ }{ }
          \setrow { }{ }{ }  {8}{ }{ }  {4}{6}{5}
          \setrow { }{4}{ }  { }{5}{9}  { }{ }{8}
          \setrow { }{8}{7}  {2}{3}{1}  { }{4}{6}
          \setrow {2}{1}{ }  {5}{ }{ }  { }{ }{3}
          \setrow {3}{ }{6}  {4}{ }{8}  { }{ }{ }
        \end{scope}
      \end{tikzpicture}
    \end{adjustbox}%
    \hfill\mbox{ }

  \end{illustration}

  Interactive.
  Fill in the grid.
  Difference between filling in the grid and concluding that solution to the puzzle.
  So, before conclude for any particular square, or for the grid as a whole.
  Is it the case that you would fill in the grid the same way?

  Key intuition, no mistakes.
\end{note}

\begin{note}
  \autoref{illu:gist:sudoku} parallels \autoref{scen:squish}.

  In both \illu{1}, \(\pv{\phi}{v}\) follows from \(\Phi\) via a rules.

  However, rules are not of direct interest.
  \autoref{scen:squish} is a syntactic proof, but variant \scen{0} in which semantic proof.
  Relevant reasoning may be rule governed, but semantic proofs are not constrained.

  Rather, familiarity.

  The type of reasoning is general.
  Syntactic and semantic proofs, Sudoku puzzles, simple instances of chess problems, all seem to involve general reasoning.
  Likewise, counting, adding, subtracting, and so on.
  Competence established through various proofs, puzzles, problems, and practice.

  In this respect, there is no reasonable doubt the agent is competent.
  At issue is performance.
  Whether reasoning is a specific instance of the general type.
\end{note}

\paragraph{\sR{2}}

\begin{note}
  So far, link between competence and performance.
  Performance is an instance of competence.
  This is the key idea.

  No argue that from this, \requ{1} follow.
\end{note}

\begin{note}
  By contradiction.
  Suppose it is never the case that sensitive.

  That is, never the case in which stop if performance is not instance of competence.

  Then, there is nothing which states whether performance is an instance of competence.
  This applies to reasoning the agent is doing.

  Then, this extends to \fc{1}.
  For, if nothing about next action, then not competence.
\end{note}

\begin{note}
  So, know whether or not present reasoning is \sR{}.

  At issue is whether \fc{}.
  Consider midpoint.
  If don't know \fc{}, then don't know performance is instance of competence.
  For, progressive.
  With the exception of sceptical scenarios, know that actions to perform.
  If competence, then following action is constrained.
\end{note}

\begin{note}
  Puzzle is whether there is any sense to competence if deny knowledge of \fc{}.
\end{note}

\section{Summary}
\label{sec:summary-1}

\begin{note}
  Introduced \requ{1}.
\end{note}

%%% Local Variables:
%%% mode: latex
%%% TeX-master: "master"
%%% End:


% \begin{note}[Problems of induction]
%   Hence, the sketch does not apply to black ravens.
%   I wouldn't conclude all ravens are black if I saw a white raven.

%   I may worry about shortly seeing a white raven when concluding all ravens are black, and I may refuse to entertain the possibility that the sun will rise tomorrow when planning to mow the grass.

%   However, it's not possible to reason to seeing a white raven, nor is it possible to reason to the sun not rising tomorrow.

%   Abstractly, at issue in~\autoref{illu:lost-key} is the possibility of failing to a reason to some proposition-value pair given \emph{present} information, rather than the possibility of failing to a reason to some proposition-value pair given \emph{new} information.

%   To the extent that problems of induction arise from receiving new information, what is at issue is distinct.%
%   \footnote{
%     See~\textcite{Henderson:2020wb} for more on the problem of induction.
%   }

%   Similar points for external world scepticism.
%   Would not conclude that I have hand if disembodied brain in a vat.

%   However, conclusion is out of reach.
% \end{note}