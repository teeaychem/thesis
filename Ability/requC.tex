\chapter{\requ{3}}
\label{cha:requs}

\begin{note}
  \autoref{cha:fcs} introduced \fc{1}.

  \fc{1} are such that:
  \begin{itemize}
  \item
    \ros{} holds between \(\pv{\psi}{v'}\) and \(\Psi\) (for the agent).
  \item
    The agent doesn't have a \wit{} for the \ros{} between \(\pv{\psi}{v'}\) and \(\Psi\).
  \end{itemize}

  A \requ{} is a relation between \(\pv{\psi}{v'}\) from \(\Psi\) being a \fc{} and the agent \emph{concluding} \(\pv{\phi}{v}\) from \(\Phi\).
  Roughly, agent is concluding \(\pv{\phi}{v}\) from \(\Phi\) only if \(\pv{\psi}{v'}\) is a \fc{}.
\end{note}

\begin{note}
  Important to generate counterexamples to \issueConstraint{}.
  Still, only certain \requ{1} generate counterexamples to \issueConstraint{}.%
  \footnote{
    I.e.\ one may jointly hold:
    \begin{enumerate*}[label=(\alph*)]
    \item
      various \requ{1} exist, and
    \item
      answers to \qWhyV{} are constrained by answers to \qHowV{} via \issueConstraint{}
    \end{enumerate*}%
    .
    For more, see \autoref{prop:requ-not-n-ce} on \autopageref{prop:requ-not-n-ce}, below.
  }
  Hence, we introduce and motivate the idea of a \requ{0} with \requ{1} which are compatible with \issueConstraint{}.
  \autoref{cha:binding} explicitly links \requ{1} and \issueConstraint{}.
\end{note}

\begin{note}
  The chapter has two sections:
  \begin{TOCEnum}
  \item
    \TOCLine{cha:requs:requs}

    Introduction, definition, intuition, and \illu{1}
  \item
    \TOCLine{sec:typicalRequs}

    \autoref{cha:typical}, \tC{}.
    Motivate existence of \requ{1}.
  \end{TOCEnum}
\end{note}


\section{\requ{3}}
\label{cha:requs:requs}

\begin{note}
  Idea of a \requ{} is something that must be the case in order for the agent to be concluding.

  Various things need to be the case.
  Alive, conscious, 

  In particular, this thing is a \fc{}.
\end{note}

\subsection{Definition}
\label{cha:requs:def}

\begin{note}
  Event.
  Event in which agent is concluding.
  However, various aspects of event independent of whether the agent is concluding.
  Familiar idea from event semantics.
  So, build up description of event, without concluding.
  A \requ{} then is a necessarily condition on concluding.

  Same as \qWhyV{}.


  Defined as follows:

  \begin{definition}[A \requ{0}]
    \label{def:requ}
    \cenLine{
      \begin{VAREnum}
      \item
        Agent: \vAgent{}
      \item
        \prop{3}: \(\phi\), \(\psi\)
      \item
        \val{3}: \(v\), \(v'\)
      \item
        \pool{3}: \(\Phi\), \(\Psi\)
      \item
        \mbox{ }
      \end{VAREnum}
    }

    \cenLine{
      \begin{VAREnum}
      \item
        Event: \(e\)
      \item
        Event description: \(d\)
      \item
        \mbox{ }
      \end{VAREnum}
    }

    \begin{itemize}
    \item
      \(\pvp{\psi}{v'}{\Psi}\) is a \emph{\requ{}} of \(e\) under description \(d\) being an event in which \vAgent{} is concluding \(\pv{\phi}{v}\) from \(\Phi\).
    \end{itemize}

    \emph{If and only if}

    \begin{itemize}
    \item
      For any event \(e'\) such that \(d\) is true of \(e'\):
      \begin{itenum}
      \item[\emph{If}:]
        \(\pv{\psi}{v'}\) from \(\Psi\) is not a \fc{} for \vAgent{}.
      \item[\emph{Then}:]
        \(e'\) is not an event in which \vAgent{} is concluding \(\pv{\phi}{v}\) from \(\Phi\).
      \end{itenum}
    \end{itemize}
    \vspace{-\baselineskip}
  \end{definition}

  \noindent%
  Shorten the defined term to:
  \begin{notation}
  \item
    \(\pvp{\psi}{v'}{\Psi}\) is a \requ{} of an agent concluding \(\pv{\phi}{v}\) from \(\Phi\).
  \end{notation}
\end{note}

\begin{note}
  \requ{3} are defined with respect to an event \(e\) under description \(d\).

  Given some event \(e\) and description \(d\), \(\pvp{\psi}{v'}{\Psi}\) being a \requ{} of an agent concluding \(\pv{\phi}{v}\) from \(\Phi\) intuitively corresponds to the following idea:
  \begin{itemize}
  \item
    The agent is concluding \(\pv{\phi}{v}\) from \(\Phi\) \emph{only if} \(\pv{\psi}{v'}\) from \(\Psi\) is a \fc{}, for the agent.
  \end{itemize}

  Or, as expressed in definition, fails to be a \fc{}, then \(e\) is not an event in which the agent is concluding \(\pv{\phi}{v}\) from \(\Phi\).
\end{note}

\subsection{Illustrations}
\label{cha:requs:firstIllu}

\begin{note}
  Start with an easy \scen{0}.
  Then, more complex to highlight issues.
\end{note}

\paragraph{Easy}

\begin{note}
  So, scenario is puzzle where independent sub-puzzles combine for solution.
  If don't get sub-puzzle, then don't get puzzle.
  So, if concluding answer to puzzle, then need sub-puzzles as \fc{1}.

  The way this works is that if concluding in this way, then main thing needs to be a \fc{}.
  And, anything necessary for main thing is also a \fc{}.
\end{note}

\begin{note}
  Key part of the argument is concluding, then this is a \fc{}.

  \begin{scenario}[Goldbach Conjecture]%
    The (Binary) Goldbach Conjecture states:

    \begin{quote}
      Every even number is a sum of two primes.
    \end{quote}

    An agent is travelling on a train with a some paper, a pencil.

    The Conjecture is only a conjecture, and so there may (for all the agent knows), be an even number which is not the sum of two primes.

    The agent is very bored.
    The attempt to show false.
    Methodology.
    Write down even number.
    Write down smaller primes.
    Check sums.

    The agent writes down an even number on the piece of paper, and attempts to find a pair of primes an even number is equal to.
    Though, the agent is careful to ensure they do not make any mistakes.
  \end{scenario}

  The event, train journey.
  Is the agent concluding the Goldbach Conjecture is false?

  Suppose agent is concluding.
  Then, conclude that some even number is not the sum of two primes.

  On train, with no new information.
  Chosen number.
  It must be the case that this is a \fc{}.
  But this doesn't really work neatly, as I really want a set.
  

  Now, in this case we need that it's a \fc{} because there's no new information.

  And, various things for not concluding.
  First, The conjecture has been verified up to \(4 \times 10^{14}\) (\cite[cf.][]{Richstein:2001aa}).
  Second, enumerating primes whether possible.
  Third, enumerating primes whether complete.

  And so on.

  Conversely, isolated on a train then simple cases where concluding.
  For, it's got to be the case that concluding.


  Failures of this conditional get failures of a \requ{}.
  Two key ways.

  New information.

  Conclude regardless.
\end{note}


\paragraph{Difficult}

\begin{note}
  \begin{scenario}[Lost keys]%
    \label{illu:lost-key}%
    An agent thinks they may have lost their keys.
    They usually leave place my keys on the right side of their desk, near a copy of~\citeauthor{Vickers:1989tr}'s~\citetitle{Vickers:1989tr} they've been saving for a rainy day.
    And, their keys aren't there.

    They've searched over the desk, under the desk, and beside the desk.
    They haven't found their keys.

    Still, the agent holds the following principle:
    \begin{quote}
      If the agent thinks of a place to look for an object, the agent does not conclude the object is lost without searching the place.
    \end{quote}
    % And, the agent may still think of a place to look.

    % They retrace earlier steps and recall they went to get coffee after arriving in the office.
    % Could the keys be near the coffee machine?
  \end{scenario}

  \noindent%
  At issue is whether the agent concludes they have lost their keys without any further search.
  Fix the event to begin when the \agents{} completes their initial search of the desk and extend the event until the agent either
  \begin{enumerate*}[label=(\alph*), ref=(\alph*)]
  \item
    concludes they have lost their keys without any further search, or
  \item
    sets of on a further search.
  \end{enumerate*}

  Given the description of \label{illu:lost-key}, we argue:%
  \footnote{
    The pools of premises \(\Phi\) and \(\Psi\) are left unspecified.
  }
  %
  \begin{quote}
    \pv{\propI{The agent has no further idea of where to look}}{\valI{True}} is a \requ{} of the agent concluding \pv{\propI{The agent has lost their keys}}{\valI{True}}
  \end{quote}
  %
  To do so, then the following conditional is true:

  \begin{quote}
    \begin{itenum}
    \item[\emph{If}:]
      If \pv{\propI{The agent has no further idea of where to look}}{\valI{True}} is not a \fc{}.
    \item[\emph{Then}:]
      The agent is not concluding \pv{\propI{The agent has lost their keys}}{\valI{True}}.
    \end{itenum}
  \end{quote}

  Argue for the contrapositive.

  Suppose the agent is concluding \pv{\propI{The agent has lost their keys}}{\valI{True}}.
  Then, by principle, it must be the case that no further place.
  For, else go look at that place.%
  \footnote{
    Assumes the agent has negative influence.

    I think this is clear.
    Various things that influence.

    And, though possible that there are various demons that change things, exclude the existence of such demons by the relevant description.

    Note, this does not entail that the agent has positive influence, nor choice.
  }
  And, if go look, then the relevant event is over.

  Now, with conditional, is it the case that concluding?
  Is the agent going to think of something?

  If the agent does, then they go and search.
  If the agent does not, then they conclude lost.

  There is not immediate answer.
  However, either or.

  To illustrate further.
  Suppose coffee.
  May think to retrace steps.
  If so, no conclusion.
  However, the agent may fail to think of anything.
  If so, the agent concludes.

  Still, it is not the case that conclusion is in progress, for the agent may think of a place to search.

  It may be the case that \fc{} and not concluding.
  But, this is no problem.
  For, \fc{1} are not exclusive.
  But, this is consistent with \requ{}.
\end{note}



\subsection{Observation}
\label{cha:requs:def:obs}

\begin{note}
  \begin{observation}[\requ{3} may be irrelevant]%
    \label{obs:requ-exp}%
    \cenLine{
      \begin{VAREnum}
      \item
        Agent: \vAgent{}
      \item
        \prop{3}: \(\phi\), \(\psi\)
      \item
        \val{3}: \(v\), \(v'\)
      \item
        \pool{3}: \(\Phi\), \(\Psi\)
      \item
        \mbox{ }
      \end{VAREnum}
    }

    \cenLine{
      \begin{VAREnum}
      \item
        Event: \(e\)
      \item
        Event description: \(d\)
      \item
        \mbox{ }
      \end{VAREnum}
    }

    \(\pvp{\psi}{v'}{\Psi}\) being a \requ{} of \(e\) under description \(d\) being an event in which \vAgent{} is concluding \(\pv{\phi}{v}\) from \(\Phi\) does necessarily capture something relevant for understanding an event under description \(d\).
  \end{observation}

  Lose proposition.
  The point is, just something that's true or false.
  Though, discussion of \autoref{illu:lost-key} may suggest otherwise.
  Here, briefly outline alternative, and example of this kind of observation.

  \begin{motivation}{obs:requ-exp}
    Have something which blocks conclusion.

    Suggests it may be the case that at issue is not the absence of a \fc{}, but the presence of something which prevents \fc{}.

    I think this is right.

    In turn, provide an account of the phenomenon which does not rely on \fc{}.
    However, this is not straightforward.
    Intuitively, there's obvious conflict.
    Possible location, so keys are not lost, by principle.

    Plausibly the case that agent gets by concluding the obvious.
  \end{motivation}

  It's not the case that \requ{} explains.
  Only thing for explanation is \qWhyV{}.
  And, \qWhyV{} is about when an agent concludes.
\end{note}

\begin{note}
  \fc{1} focus on event.
  However, the existence of an event is secondary.
  Present concern that applies to any reasoning about where keys are.

  This is what secures that there is no event.
\end{note}


\begin{note}
  \(\pvp{\psi}{v'}{\Psi}\) being a \requ{0} an agent concluding \(\pv{\phi}{v}\) from \(\Phi\) amounts, roughly, to a necessary condition:
  \begin{itemize}
  \item
    The agent is concluding \(\pv{\phi}{v}\) from \(\Phi\) \emph{only if} \(\pv{\psi}{v'}\) from \(\Psi\) is a \fc{}.
  \end{itemize}

  Hence, if an agent is concluding \(\pv{\phi}{v}\) from \(\Phi\), it must be the case that \(\pv{\psi}{v'}\) from \(\Psi\) is a \fc{}.

  However, we have granted that any instance of a \emph{conclusion} of \(\pv{\phi}{v}\) from \(\Phi\) may be arbitrary.
  Two observations follow:

  \begin{observation}
    \label{prop:requ-not-refl}
    \cenLine{
      \begin{VAREnum}
      \item
        Agent: \vAgent{}
      \item
        \prop{2}: \(\phi\)
      \item
        \val{2}: \(v\)
      \item
        \pool{2}: \(\Phi\)
      \item
        Event: \(e\)
      \item
        \mbox{ }
      \end{VAREnum}
    }

    \begin{itemize}
    \item
      The following conditional is not necessarily the case:
      \begin{itemize}
      \item
        If \vAgent{} concludes \(\pv{\phi}{v}\) from \(\Phi\), then \(\pvp{\phi}{v}{\Phi}\) \requ{} of some sub-event.
      \end{itemize}
    \end{itemize}
    \vspace{-\baselineskip}
  \end{observation}

  In other words, \(\pv{\phi}{v}\) from \(\Phi\) need not be a \fc{} in order for an agent to be concluding \(\pv{\phi}{v}\) from \(\Phi\).

  \begin{motivation}{prop:requ-not-refl}
    Consider the partnered case from \autoref{obs:cds-arb} (\autopageref{obs:cds-arb}).%
  \end{motivation}
\end{note}



\section{\requ{3} and \tC{}}
\label{sec:typicalRequs}

\begin{note}
  \begin{proposition}[\tC{2} and \requ{1}]
    \label{prop:hinge}
    \cenLine{
      \begin{VAREnum}
      \item
        Agent: \vAgent{}
      \item
        \prop{3}: \(\phi\), \(\psi\)
      \item
        \val{3}: \(v\), \(v'\)
      \item
        \pool{3}: \(\Phi\), \(\Psi\)
      \item
        \mbox{ }
      \end{VAREnum}
    }

    \cenLine{
      \begin{VAREnum}
      \item
        Event: \(e\)
      \item
        Type of reasoning: \(T\)
      \item
        \mbox{ }
      \end{VAREnum}
    }

    \begin{itenum}
    \item[\emph{If}:]
      \ref{prop:hinge:typical}, \ref{prop:hinge:rep}, and \ref{prop:hinge:tI} are true:

      \begin{enumerate}[label=\alph*., ref=(\alph*), series=propHingeSer]
      \item
        \label{prop:hinge:typical}
        \(e\) is an event in which \vAgent{} is \tCV{0} \(\pv{\phi}{v}\) from \(\Phi\) by~type~\(T\).
      \item
        \label{prop:hinge:rep}
        \(T'\) is a \tRep{} of \vAgent{} \tCV{} \(\pv{\phi}{v}\) from \(\Phi\) by type \(T\) in \(e\).
      \item
        \label{prop:hinge:tI}
        \(\pvp{\psi}{v'}{\Psi}\) is a \tI{} of \(T'\).
      \end{enumerate}

    \item[\emph{Then}:]
      \ref{prop:hinge:requ} is true:
      \begin{enumerate}[label=\alph*., ref=(\alph*), resume*=propHingeSer]
      \item
        \label{prop:hinge:requ}
        \(\pvp{\psi}{v'}{\Psi}\) is a \requ{} of \(e\) being an event in which \vAgent{} is concluding \(\pv{\phi}{v}\) from \(\Phi\).
      \end{enumerate}
    \end{itenum}
    \vspace{-\baselineskip}
  \end{proposition}

  The key to \autoref{prop:hinge} is \fc{1}.
  \tC{} extends \tC{} to \fc{1}.
  By assumption, concluding.
  Hence, it must be the case that we get the \fc{}.

  If this seems surprising, keep in mind that all of these definitions are in terms of the material conditional.

  \begin{argument}{prop:hinge}
    Assume both~\ref{prop:hinge:typical}~and~\ref{prop:hinge:tI} hold.
    And, suppose for a contradiction~\ref{prop:hinge:requ} does not hold.

    From~\ref{prop:hinge:typical}, \(e\) is an event.

    From~\ref{prop:hinge:tI}, \(\pvp{\psi}{v'}{\Psi}\) is representative of some type \(T\).

    Now, to show \requ{}.
    Needs to be the case that, if not \fc{}, then not event.
    Well, suppose not \fc{}.
    Then, cannot be event.
    For, conflicts with description.
  \end{argument}
\end{note}

\paragraph*{Sound rules}

\begin{note}
  \phantlabel{squish-elimination-proof}

  \begin{illustration}[Squish elimination]%
    \label{scen:squish}%
    It is late morning on a sunny day.
    I ate a good breakfast, and drank some nice coffee.
    I have completed a handful of syntactic proofs for entailments of propositional logic using the basic rules of inference in a Fitch-style system.

    I create the following syntactic proof:
    \begin{center}
      \begin{fitch}
        \phantlabel{illu:sP:1}\fa (P \rightarrow Q) \rightarrow P \\
        \phantlabel{illu:sP:2}\fj R \\
        \phantlabel{illu:sP:3}\fa P & \sqE{}:\hyperref[illu:sP:1]{1} \\
        \phantlabel{illu:sP:4}\fa P \land R & \(\land\)\textbf{Intro:} \hyperref[illu:sP:2]{2},\hyperref[illu:sP:3]{3}
      \end{fitch}
    \end{center}

    Still, I haven't yet concluded \((P \rightarrow Q) \rightarrow P, R \vdash P \land R\).

    For, \sqE{} may not be a sound rule of inference.

    I go on to conclude \((P \rightarrow Q) \rightarrow P, R \vdash P \land R\).
  \end{illustration}

  \begin{definition}[\sqE{}]%
    \label{def:sque}%
    \sqE{} is the following rule:
    \begin{center}
      \begin{fitch}
        \ftag{\text{\scriptsize \emph{i}}}{\fa (\phi \rightarrow \psi) \rightarrow \phi} \\
        \ftag{\text{\scriptsize }}{\fa \vdots } \\
        \ftag{\text{\scriptsize \emph{j}}}{\fa \phi } & \sqE{}:\emph{i} \\
      \end{fitch}
    \end{center}
  \end{definition}

  \sqE{} is sound.%
  \footnote{
    \label{prop:sqE-sound}
    Rather than prove \sqE{} is sound (which would require a detailed statement of the proof system in question), we prove that the corresponding semantic entailment holds:

    Let \(v\) be an arbitrary (truth-functional) valuation, and assume \(v((\phi \rightarrow \psi) \rightarrow \phi) = \valI{True}\).
    Further, assume for contradiction \(v(\phi) = \valI{False}\).

    As \(v(\phi) = \valI{False}\), it immediately follows that \(v(\phi \rightarrow \psi) = \valI{True}\).
    Therefore, by the first assumption, it must be the case that \(v(\phi) = \valI{True}\).
    This contradictions the second assumption.
    % Hence, \((\phi \rightarrow \psi) \rightarrow \phi \vDash \phi\).
  }

  The relevant propositions, values, and \pool{1} are as follows:
  \begin{itemize}[noitemsep]
  \item
    I am the agent.
  \item
    \(\phi\) is the proposition: \(\mathsf{(P \rightarrow Q) \rightarrow P, R \vdash P \land R}\).
  \item
    \(\psi\) is the proposition: \propI{\sqE{} is sound}
  \item
    Both \(v\) and \(v'\) are the value: \valI{True}.
    And,
  \item
    The pools of premises \(\Phi\) and \(\Psi\) are left unspecified.%
    \footnote{
      \((P \rightarrow Q) \rightarrow P\) and \(R\) are premises of the deduction, and not elements of \(\Phi\).

      For, my conclusion is \pv{\mathsf{(P \rightarrow Q) \rightarrow P, R \vdash P \land R}}{\valI{True}}.
      My conclusion is not \pv{\mathsf{P \land R}}{\valI{True}}.
    }
  \item
    Key parts of description:
    \begin{itemize}[noitemsep]
    \item
      sunny day, good breakfast, nice coffee.
      I.e.\ as ideal as situations may be for syntactic proofs.
    \item
      Proved \sqE{} numerous times before.
    \end{itemize}
  \end{itemize}

  Hence, the relevant instance of the conditional by which a \requ{} is defined is:

  \begin{quote}
    \begin{itenum}
    \item[\emph{If}:]
      If \(\pv{\propI{\sqE{} is sound}}{\valI{True}}\) from \(\Psi\) is not a \fc{}.
    \item[\emph{Then}:]
      I am not concluding \(\pv{\mathsf{(P \rightarrow Q) \rightarrow P, R \vdash P \land R}}{\valI{True}}\) from \(\Phi\).
    \end{itenum}
  \end{quote}

  In contrast to \autoref{illu:lost-key}, \autoref{scen:squish} leads to a conclusion.

  At issue is whether \requ{}.
  By \autoref{prop:hinge}, three things.
  \begin{itemize}
  \item
    \tCV{}.
  \item
    \tRep{}
  \item
    \sqE{} is \tI{} of \(T'\).
  \end{itemize}

  Middle two, for sure.

  Though, in short, would fail to be reasoning if failed?
  Answer, yes.

  Plausibly, four \prop{0}-\val{0}-\pool{0} pairings in type:

  \begin{center}
    \begin{tabular}{R{.45\textwidth} L{.45\textwidth}}
      \prop{2}-\val{0} pair & \pool{2} \\
      \hline
      \pv{(\phi \rightarrow \phi) \rightarrow \psi \space/\space \phi}{\textover[c]{\valI{Sound}}{\valI{Unsound}}} & \dots \\
      \pv{\phi \rightarrow (\psi \rightarrow \phi) \space/\space \phi}{\valI{Unsound}} & \dots \\
      \pv{\psi \rightarrow (\phi \rightarrow \psi) \space/\space \phi}{\valI{Unsound}} & \dots \\
      \pv{(\psi \rightarrow \phi) \rightarrow \psi \space/\space \phi}{\valI{Unsound}} & \dots \\
    \end{tabular}
  \end{center}

  Throughout \autoref{scen:squish} I \emph{know} \sqE{} is sound.
  Prior to \autoref{scen:squish} I have proved \sqE{} is sound on various occasions using the same basic observations made in the argument for \autoref{prop:sqE-sound}.

  However, there is a distinction between \emph{knowing} \sqE{} is sound and \emph{proving} \sqE{} is sound.
  For example, if I have just drunk a considerable amount of wine, or woken from a night of tormented sleep.
  Generally said, I may not be thinking straight.

  Of course, I may conclude \(\pv{\phi}{v}\) from \(\Phi\) regardless of whether \(\pv{\psi}{v'}\) from \(\Psi\) is a \fc{}.
  Indeed, given some particularly good wine I may conclude \(P \land C \vdash O\).%
  \footnote{
    For, \(P \land C\) reads `Pac', \(O\) looks like a pellet, and Pacman likes to eat pellets.
  }
  However, then not \tCV{}.
\end{note}

\begin{note}
  Note, does not depend on \agpe{my}.
\end{note}


\paragraph*{\requ{3} and \issueConstraint{}}

\begin{note}
  The existence of \requ{1} are compatible with \issueConstraint{}.

  \begin{observation}
    \label{prop:requ-not-n-ce}
    The existence of \requ{1} is compatible with \issueConstraint{}.
  \end{observation}

  \begin{motivation}{prop:requ-not-n-ce}
    In order for counterexample, case in which \ros{} answers \qWhyV{}, but the agent does not have a \wit{}.

    We have not yet consider the way in which \requ{1} connect to \qWhyV{}.
    However, any explicit connexion is not required.
    For, we only need to observe that an agent may have a \wit{} for any \requ{}.
    Then, the \wit{} is always a candidate for \qHowV{}.

    Consider \autoref{scen:squish}.
    At issue is whether the agent would repeat the repeat conclusion of \sqE{} as sound.

    Needs to be the case that \ros{} without \wit{}.

    However, the agent has previously concluded \sqE{} is sound.
    Therefore, the agent has a \wit{} for \ros{}.
  \end{motivation}

  Note, however, that \autoref{prop:requ-not-n-ce} is weak.
  Compatible in terms of existential.
  Hence we only needed to show a single case in which \wit{} for \requ{}.
  This does not suggest that \requ{1} are not in tension with \issueConstraint{}.
\end{note}

\section*{Summary}

\begin{note}
  Introduced \requ{1}.

  Defined in terms of \tC{}.

  Motivated by \tC{}.
\end{note}



% \begin{note}[Problems of induction]
%   Hence, the sketch does not apply to black ravens.
%   I wouldn't conclude all ravens are black if I saw a white raven.

%   I may worry about shortly seeing a white raven when concluding all ravens are black, and I may refuse to entertain the possibility that the sun will rise tomorrow when planning to mow the grass.

%   However, it's not possible to reason to seeing a white raven, nor is it possible to reason to the sun not rising tomorrow.

%   Abstractly, at issue in~\autoref{illu:lost-key} is the possibility of failing to a reason to some proposition-value pair given \emph{present} information, rather than the possibility of failing to a reason to some proposition-value pair given \emph{new} information.

%   To the extent that problems of induction arise from receiving new information, what is at issue is distinct.%
%   \footnote{
%     See~\textcite{Henderson:2020wb} for more on the problem of induction.
%   }

%   Similar points for external world scepticism.
%   Would not conclude that I have hand if disembodied brain in a vat.

%   However, conclusion is out of reach.
% \end{note}

    %   \footnote{
    %     The present point is similar to issues raised by \citeauthor{Harman:1973ww} (\citeyear{Harman:1973ww}) regarding the proposed equivalence between reasons for which an agent believes something with reasons the agent would offer if asked to justify their belief.
    %     As \citeauthor{Harman:1973ww} notes, an agent may offer reasons because they think they will convince their audience, not because the agent is compelled by the reasons, etc.
    %     (\citeyear[Ch.2]{Harman:1973ww})

    %     To the extent that \citeauthor{Harman:1973ww}'s point is that what holds from an \agpe{} need not actually be the case, the point in the same.
    %     However, to the extent that \citeauthor{Harman:1973ww} relies on an under-specification of what holds from an \agpe{} --- i.e.\ the distinction between whether \(\phi\) has value \(v\) from the \agpe{} or whether the agent evaluates as true the proposition that their audience is responsive to \(\phi\) having value \(v\), the point is distinct.
    %   }


%%% Local Variables:
%%% mode: latex
%%% TeX-master: "master"
%%% TeX-engine: luatex
%%% End:

