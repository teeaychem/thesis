\chapter{\requ{3}}
\label{cha:requs}

\begin{note}
  \autoref{cha:fcs} introduced \fc{1}.
  The present chapter links \fc{1} to concluding.

  State what \requ{1} are.
  Provide \illu{1}.
\end{note}

\begin{note}
  With the definition of a \requ{} in hand, the remainder of the chapter mostly consists of \illu{1} of \requ{1}.

  \requ{3} link \fc{1} to concluding, and whether or not an agent is concluding is involved in whether or not an agent concludes.
  And, important function in generating counterexamples to \issueConstraint{}.

  However, the existence of \requ{1} does not presuppose counterexamples to \issueConstraint{}.
  In particular, though all of the \illu{1} involve \fc{1}, the \illu{1} are constructed so that the agent has a \wit{1} for any \ros{} which follows from the agent's knowledge.

  We will turn to counterexamples to \issueConstraint{} in \autoref{cha:ces}, after explicitly linking \requ{1} to \qWhyV{} in \autoref{cha:binding}.
\end{note}

\begin{note}
  This chapter is divided into three sections:
  \begin{itemize}
  \item
    \TOCLine{cha:requs:sec:infl}

    Preliminary distinction used to understand \requ{1}
  \item
    \TOCLine{cha:requs:sec:definition}

    Introduction, definition, intuition, and \illu{1}
  \item
    \TOCLine{cha:requs:sec:add-illu}

    Additional \illu{1}.
  \end{itemize}
\end{note}

\section{\ninf{2}}
\label{cha:requs:sec:infl}

\begin{note}
  \begin{definition}[\ninf{2}]
    \label{def:ninf}
    For an agent \vAgent{} and action description \(\alpha\)

    \begin{itemize}
    \item
      \vAgent{} has \ninf{} over whether or not \(\text{Prog}(e, \alpha)\) is true.
    \end{itemize}

    \emph{If and only if}

    \begin{itemize}
    \item
      Both~\ref{def:ninf:action} and~\ref{def:ninf:prog} are true:
      \begin{enumerate}[label=\alph*., ref=(\alph*)]
      \item
        \label{def:ninf:action}
        There is some action \(a\) that \vAgent{} may immediately perform.
      \item
        \label{def:ninf:prog}
        \(\text{Prog}(e', \alpha)\) would be not be true in the event \(e'\) of \vAgent{} doing \(a\).
      \end{enumerate}
    \end{itemize}
    \vspace{-\baselineskip}
  \end{definition}

  In short, the as a result of performing some action, it is not the case that the event develops.
\end{note}

\begin{note}
  For example, an agent has \ninf{0} over whether or not they are going to buy some eggs.
  For, the agent may choose to turn back home.

  Likewise, \ninf{} over delivering a lecture.
  For, walk out of the room.
\end{note}

\begin{note}
  \begin{observation}
    \label{ob:ninf}
    There are instances in which an agent exerts \ninf{0} over whether or not they are concluding \(\pv{\phi}{v}\) from \(\Phi\).
  \end{observation}

  For, the agent may stop what they are doing.

  For example, may decide that a result, even if obtained, is not worth the effort.
  Ex.\ bonus problem on a homework.
\end{note}

\begin{note}
  Note:
  It does not follow from \autoref{ob:ninf} that the agent had a choice over how event develops.
  It may be that if the agent continued the event, they would have concluded \(\pv{\phi}{v}\) from \(\Phi\).
\end{note}

\begin{note}
  \autoref{ob:ninf} is general.
  Our interest is with a particular type of \ninf{}.
\end{note}

% \begin{note}
%   To say an agent has \ninf{0} is not to say the agent has \emph{\pinf{}}.
%   For, an agent may not have choice over whether an event develops into an event in which the agent concludes \(\pv{\phi}{v}\) from \(\Phi\).

%   For example, take \(\chi\) to be the proposition `the area of a unit square is equal to the area of a unit circle'.
%   An agent has \ninf{0} over whether or not they concluding \(\pv{\chi}{\text{True}}\), as the agent may choose not to make any attempt.
%   However, assuming the agent would only conclude \(\pv{\chi}{\text{True}}\) from principles consistent with Euclidean geometry, then agent does not have \pinf{} over whether or not they are concluding \(\pv{\chi}{\text{True}}\).
%   For, the area of a unit square is not equal to the area of a unit circle.
%   Hence, it is not possible for the agent to ensure that some event may develop into an event in which they conclude \(\pv{\chi}{\text{True}}\).

%   In parallel, an agent need not have \pinf{0} over whether or not they are going to buy some eggs.
%   For, it may be there are no eggs for sale.
%   Hence, it may not be possible for the agent to ensure that the event develops into an event in which the agent buys some eggs.
% \end{note}


\section{\requ{3}}
\label{cha:requs:sec:definition}

\begin{note}
  \begin{definition}[A \requ{0}]
    \label{def:requ}
    \begin{itemize*}[noitemsep, label=\(\circ\)]
    \item
      An agent: \vAgent{}
    \item
      Propositions: \(\phi\), \(\psi\)
    \item
      Values: \(v\), \(v'\)
    \item
      \poP{3}: \(\Phi\), \(\Psi\)
    \item
      An event: \(e\)
    \item
      \mbox{ }
    \end{itemize*}

    \begin{itemize}
    \item
      \(\pvp{\psi}{v'}{\Psi}\) is a \emph{\requ{}} of \(e\), with respect to \(\pvp{\phi}{v}{\Phi}\).
    \end{itemize}

    \emph{If and only if}

    \begin{itemize}
    \item
      The following conditional is true:
      \begin{itemize}
      \item[\emph{If}:]
        \begin{enumerate}[label=\alph*., ref=(\alph*), series=requDefSeries]
        \item
          \label{def:requ:nK}
          \(\pv{\psi}{v'}\) from \(\Psi\) is not a \fc{} for \vAgent{} throughout \(e\).
        \end{enumerate}
      \item[\emph{Then}:]
        \begin{enumerate}[label=\alph*., ref=(\alph*), resume*=requDefSeries]
        \item
          \label{def:requ:nC}
          \(e\) is not an event in which \vAgent{} is concluding \(\pv{\phi}{v}\) from \(\Phi\).\newline
          \mbox{ }\hfill(\emph{Due to \ref{def:requ:nK}})
        \end{enumerate}
      \end{itemize}
    \end{itemize}
    \vspace{-\baselineskip}
  \end{definition}
\end{note}

\begin{note}
  If an agent doubts that \(\pv{\psi}{v'}\) is a \fc{}, then the agent is not concluding \(\pv{\phi}{v}\) from \(\Phi\).%
  \footnote{
    \autoref{def:requ} parallels the definition of a \pevent{} (\autoref{def:potenital-event}, \autopageref{def:potenital-event}), and hence the definition of a \fc{} (\autoref{def:fc}, \autopageref{def:fc}).
    The key difference is with respect to whether the progressive is true as a result of the agent doing the relevant action.
    With respect to \ninf{}, the progressive is false, while with respect to a \pevent{0} the progressive is true.

    However, circumstances are different.
    \pevent{} when progressive is not true of event.
    \ninf{} when progressive is true.

    Note, exertion of \ninf{} is compatible with progressive being true.
  }
\end{note}

\begin{note}
  \autoref{prop:requ-fc} expands \autoref{def:requ} via the definition of a \fc{}.

  \begin{proposition}[A \requ{0}, expanded]
    \label{prop:requ-fc}
    \begin{itemize*}[noitemsep, label=\(\circ\)]
    \item
      An agent: \vAgent{}
    \item
      Propositions: \(\phi\), \(\psi\)
    \item
      Values: \(v\), \(v'\)
    \item
      \poP{3}: \(\Phi\), \(\Psi\)
    \item
      An event: \(e\)
    \item
      \mbox{ }
    \end{itemize*}

    \begin{itemize}
    \item
      \(\pvp{\phi}{v'}{\Psi}\) is a \emph{\requ{}} of \(e\) just in case:
      \begin{itemize}
      \item[\emph{If}:]
        \begin{enumerate}[label=\alph*., ref=(\alph*), series=requDefSeries]
        \item
          \label{prop:requ-fc:nk}
          Throughout \(e\), either~\ref{prop:requ-fc:nk:psi} or~\ref{prop:requ-fc:nk:no-conf} are true:
          \begin{enumerate}[label=\roman*., ref=(\roman*)]
          \item
            \label{prop:requ-fc:nk:psi}
            There is no \pevent{} in which \vAgent{} concludes \(\pv{\psi}{v'}\) from \(\Psi\).
          \item
            \label{prop:requ-fc:nk:no-conf}
            There is a \pevent{} in which \vAgent{} concludes something incompatible with a conclusion of \(\pv{\psi}{v'}\) from \(\Psi\).
          \end{enumerate}
        \end{enumerate}
      \item[\emph{Then}:]
        \begin{enumerate}[label=\alph*., ref=(\alph*), resume*=requDefSeries]
        \item
          \label{prop:requ-fc:ne}
          \(e\) is not an event in which \vAgent{} is concluding \(\pv{\phi}{v}\) from \(\Phi\).
        \end{enumerate}
      \end{itemize}
    \end{itemize}
    \vspace{-\baselineskip}
  \end{proposition}

  \begin{argument}{prop:requ-fc}
    From the definition of a \requ{} (\autoref{def:requ}, \autopageref{def:requ}) and the definition of a \fc{} (\autoref{def:fc}, \autopageref{def:fc}).
  \end{argument}
  So, interactions with either of these two things.
\end{note}


\begin{note}
  Our interest with \requ{1} is with regard to \ninf{0}.
  Intuitively, if \(\pv{\psi}{v'}\) is not a \fc{} from \(\Psi\), then the agent stops the event from developing into an event in which the agent concludes \(\pv{\phi}{v}\) from \(\Phi\).

  Indeed, even if the event could have developed into an event in which the agent concluded \(\pv{\phi}{v}\) from \(\Phi\).

  Initial example, result was not worth the effort, even if obtained.
  The agent may well have obtained, but as the agent decided the result was not worth the effort, their decision ensures they are not concluding.
\end{note}

\begin{note}
  The qualifier `due to' is appended to \ref{def:requ:nC} to ensure that the conditional captures the agent's \ninf{}, rather than being trivially true because the agent is not concluding \(\pv{\phi}{v}\) from \(\Phi\).%
  \footnote{
    See, for example, \citeauthor{Lewis:1997wg}'s (\citeyear{Lewis:1997wg}) discussion of finkish dispositions.
  }
\end{note}


\begin{note}
  \requ{1} are substantial.

  Absence of a \fc{} influences present action.

  It is not the case that from the \agpe{}.
  Rather, it is the case that if no \fc{} then not concluding.

  Rather than abstract motivation, pair of \illu{1}.
\end{note}

\subsection{\illu{3}}
\label{sec:illu3}

\begin{note}
  First, \scen{0} in which \requ{} and agent is not concluding due to \requ{}.
  This \illu{0} without detailed discussion to motivate idea.
  Further, \requ{1} such that not concluding will not be of significant interest.

  Second, \scen{0} in which \requ{} and agent is concluding.
  This \illu{0}, detailed discussion.
  However, our initial discussion will be intuitive.
\end{note}

\subsubsection{\requ{2} and not concluding}

\begin{note}
  \scen{0} in which an agent is not concluding due to \requ{}.
\end{note}

\begin{note}
  \begin{illustration}[Lost keys]
    \label{illu:lost-key}
    I think I might have lost my keys.
    I usually leave place my keys on the right side of my desk, next to a copy of~\citeauthor{Vickers:1989tr}'s~\citetitle{Vickers:1989tr} which I've been saving for a rainy day.
    And, my keys aren't there.

    I've searched on the desk, under the desk, and beside the desk.
    And, I haven't found my keys.

    Still, I haven't (yet, at least) \emph{concluded} that I've lost my keys.

    For, there might still be some place I haven't looked.
    If I think a little harder a figure out where that place is, I would conclude my keys might be in that place.
    And, my keys aren't lost if they are in that place.
    So, I might conclude that my keys aren't lost, which conflicts with concluding my keys are lost.
  \end{illustration}
\end{note}

\begin{note}
  Filling in the details of \autoref{illu:lost-key}:
  \begin{itemize}[noitemsep]
  \item
    I am the agent.
  \item
    \(\phi\) is the proposition: `I have lost my keys'.
  \item
    \(\psi\) is a some proposition: `My keys are not in location \(l\)'
  \item
    Both \(v\) and \(v'\) are the value: `True'.
  \item
    The pools of premises \(\Phi\) and \(\Psi\) are left unspecified.
  \end{itemize}

  Hence, the relevant instance of the conditional by which a \requ{} is defined is:

  \begin{enumerate}[label=]
  \item
    \begin{itemize}
    \item[\emph{If}:]
      If \(\pv{\text{My keys are not in location }l}{\text{True}}\) from \(\Psi\) is a not a \fc{}.
    \item[\emph{Then}:]
      I am not concluding \(\pv{\text{I have lost my keys}}{\text{True}}\) from \(\Phi\).
    \end{itemize}
  \end{enumerate}

  The antecedent is true, and hence the consequent is true due to \ninf{} I exert over the event.

  Antecedent is true.
  Have no checked.
  Hence, exert \ninf{}.

  The event of reasoning about lost keys is not an event in which I am concluding I have lost my keys, because it will not develop.
\end{note}

\begin{note}
  You may disagree with the tension I see in~\autoref{illu:lost-key}.
  Perhaps it's fine to conclude my keys are lost while allowing for the possibility that the keys are some place I haven't yet thought of.
  My goal is only to convince you that my refusal to conclude I've lost my keys is intelligible.

  At issue is only whether \(\pvp{\psi}{v'}{\Psi}\) is a \requ{} for me in a particular event in which I am reasoning about whether I have lost my keys.

  Hence, if one does not worry about the possibility that the keys are some place I haven't yet thought of, \(\pvp{\psi}{v'}{\Psi}\) would not be a \requ{1} of concluding I have lost my keys.
\end{note}

\subsubsection{Sound rules}

\begin{note}
  Here, I want to state that the agent has done the proof a number of times.
  So, the agent knows that there is a \pevent{}.
\end{note}

\begin{note}
  \phantlabel{squish-elimination-proof}

  \begin{restatable}[Squish elimination]{illustration}{scenarioPLSquish}
    \label{scen:squish}
    I conclude \((P \rightarrow Q) \rightarrow P, Q \vdash P \land Q\) from the both the following syntactic proof and the soundness of the rules of inference:
    \begin{center}
      \begin{fitch}
        \phantlabel{illu:sP:1}\fa (P \rightarrow Q) \rightarrow P \\
        \phantlabel{illu:sP:2}\fj Q \\
        \phantlabel{illu:sP:3}\fa P & \sqE{}:\hyperref[illu:sP:1]{1} \\
        \phantlabel{illu:sP:4}\fa P \land Q & \(\land\)\textbf{Intro:} \hyperref[illu:sP:2]{2},\hyperref[illu:sP:3]{3}
      \end{fitch}
    \end{center}
  \end{restatable}

  The proof consists of two premises and two rules of inference.
  The two rules of inference are of interest.

  The second rule of inference used is standard `\(\land\)' introduction, and applies to lines \hyperref[illu:sP:2]{2} and \hyperref[illu:sP:3]{3}.
  Where the conditional holds is unclear.
  On the one hand, troubled if failed to show that `\(\land\)' introduction is sound.
  However, testimony\dots

  The first rule of inference is a derived rule termed `\sqE{}', shorthand for `Squish elimination' applied to line \hyperref[illu:sP:1]{1}.%
  \footnote{
    For a quick proof, suppose \((\phi \rightarrow \psi) \rightarrow \phi\) is true.
  And for contradiction assume \(\phi\) is false.
  As \(\phi\) is false, it immediately follows that \(\phi \rightarrow \psi\) is true.
  Therefore, by the initial supposition, \(\psi\) is true.
  Hence, we have obtained the desired contradiction.
  }

  \begin{center}
    \begin{fitch}
      \ftag{\scriptsize i}{\fa (\phi \rightarrow \psi) \rightarrow \phi} \\
      \ftag{\scriptsize }{\fa \vdots } \\
      \ftag{\scriptsize j}{\fa \phi } & \sqE{}:\emph{i} \\
    \end{fitch}
  \end{center}
\end{note}

\begin{note}
  The relevant propositions, values, and \poP{1} are as follows:
  \begin{itemize}[noitemsep]
  \item
    I am the agent.
  \item
    \(\phi\) is the proposition: `\((P \rightarrow Q) \rightarrow P, Q \vdash P \land Q\)'.
  \item
    \(\psi\) is a some proposition: `\sqE{} is sound'
  \item
    Both \(v\) and \(v'\) are the value: `True'.
    And,
  \item
    The pools of premises \(\Phi\) and \(\Psi\) are left unspecified.%
    \footnote{
      Note, premises of reasoning.
      Distinct from premises of deduction.
    }
  \end{itemize}

  Hence, the relevant instance of the conditional by which a \requ{} is defined is:

  \begin{enumerate}[label=]
  \item
    \begin{itemize}
    \item[\emph{If}:]
      If \(\pv{\text{\sqE{} is sound}}{\text{True}}\) from \(\Psi\) is not a \fc{}.
    \item[\emph{Then}:]
      I am not concluding \(\pv{(P \rightarrow Q) \rightarrow P, Q \vdash P \land Q}{\text{True}}\) from \(\Phi\).
    \end{itemize}
  \end{enumerate}
\end{note}

\paragraph{Motivation}

\begin{note}
  Start with understanding from \agpe{my}, expand.
\end{note}

\begin{note}
  The motivation for the conditional rests on \gR{}, and whether reasoning is \sR{}.

  I know \sqE{} is sound.
  For, I have proved \sqE{} is sound on many different occasions.
  More generally I am competent with respect to deriving basic syntactic theorems.

  However, is it the case that \sR{}?
  For, I may not be thinking straight.
  Typically, because I am exhausted.

  If not \sR{}, then it is not clear to me what reasoning amounts to.
  I have a what might be a syntactic proof.
  However, the apparent proof may just be an association of thoughts.

  Hence, any doubt of performance aligns with competence would lead to abandoning reasoning.

  Indeed, I would refrain from concluding any syntactic entailment holds.
  For, while I know \sqE{} is sound, I would not know that my application of \sqE{} amounts to a syntactic proof.

  For example, consistent that, different premises, same conclusion.

  E.g. \(\alpha \rightarrow (\beta \rightarrow \alpha)\), \(\beta \rightarrow (\alpha \rightarrow \beta)\), and \((\beta \rightarrow \alpha) \rightarrow \beta\).
  It is not the case that \(\alpha\) follows from any of these.
  If would not stop, then not \sR{}.

  Do not conclude if reasoning is not \sR{}.

  Where, \sR{} then some general type of reasoning of which token.
  Here, type of reasoning is the general type of reasoning involved in constructing syntactic proofs.
\end{note}

\begin{note}
  Now, \agpe{my}, and \agpe{my} may be wrong.

  However, regardless of \agpe{my}, then if \requ{} fails to hold, it is unclear whether reasoning is \sR{}.

  Only to highlight a causal connexion.
  And, causal connexion is insufficient.

  Concern about justification, implicit in \scen{0} as described from \agpe{my}.
  However, specific worry is that there is no generality to reasoning.

  If \agpe{my} is wrong, then it is the case that reasoning is not \sR{}.
  Conclude only if \sR{}.
\end{note}

\begin{note}
  A \requ{} is seems to be to be a necessary condition for conclusions which result from \sR{} reasoning.

  This is not necessary, though.
  As with lost keys, \requ{} need not be tied to \sR{}.
  With lost keys, \sR{}, but whether that is sufficient for concluding.

  Motivation for \requ{1} and concluding rests on \sR{}, but is plausibly broader.
\end{note}

\subsection{Intuition}
\label{sec:intuition-1}

\begin{note}
  Granting interpretation is correct, failure to know \fc{} amounts to an something like an undercutting defeater.%
  \footnote{
    To my understanding, undercutting defeaters were introduced by \citeauthor{Pollock:1987un} (\citeyear{Pollock:1987un}).
    And, \citeauthor{Pollock:1987un} defines an undercutting defeater as follows:
    \begin{quote}
    R is an \emph{undercutting defeater} for P as a prima facie reason for S to believe Q if and only if
    \begin{enumerate}[label=(UD\arabic*), ref=(UD\arabic*)]
    \item
      \label{pollock:ud:1}
      P is a reason for S to believe Q and R is logically consistent with P but (P and R) is not a reason for S to believe Q, and
    \item
      \label{pollock:ud:2}
      R is a reason for denying that P wouldn't be true unless Q were true.%
      \mbox{}\hfill\mbox{(\citeyear[485]{Pollock:1987un})}
    \end{enumerate}
  \end{quote}
  This definition is hard to square with a \requ{}.
  In particular, \ref{pollock:ud:1}.

  Issue: P is a reason.
  By parallel, the reasoning that the agent has done is sufficient for the agent to conclude.
  However, at issue is precisely whether this is the case.
  }

  We borrow the following sketch from \textcite{Worsnip:2018aa}:
  \begin{quote}
    Undercutting defeaters, which are easiest to think of in the context of the attitude of belief, are supposed to be considerations that undermine the justification of a belief in a proposition p not necessarily by providing (sufficient) positive evidence to think that p is false, but rather merely by suggesting (perhaps misleadingly) that one’s reasons for believing p are no good, in a way that neutralizes or mitigates their justificatory or evidential force.%
    \mbox{}\hfill\mbox{(\citeyear[29]{Worsnip:2018aa})}
  \end{quote}

  In particular, concluding.
  At issue is not whether the \(\phi\) has value \(v\), but whether the agent's reasoning from \(\Phi\) to \(\pv{\phi}{v}\) is sufficient for the agent to conclude \(\phi\) has value \(v\) (from \(\Phi\)).

  Justification, and this is one way to go.
  However, not tied to justification.
\end{note}

\subsection{What a \requ{} is not}

\begin{note}
  \requ{} isn't saying that if try and fail then no conclusion.
  \requ{} is stronger, if no guarantee at present, then no conclusion.
\end{note}

\subsection{Propositions}
\label{sec:propsoitions}

\begin{note}
  We briefly state two propositions which have no significant role in the arguments to follow, but which may be helpful.
\end{note}

\begin{note}
  The existence of \requ{1} are compatible with \issueConstraint{}.
  \begin{proposition}
    \label{prop:requ-not-n-ce}
    The existence of \requ{1} is compatible with \issueConstraint{}.
  \end{proposition}
  \begin{argument}{prop:requ-not-n-ce}
    In order for counterexample, case in which \ros{} answers \qWhyV{}, but the agent does not have a \wit{}.

    We have not yet consider the way in which \requ{1} connect to \qWhyV{}.
    However, any explicit connexion is not required.
    For, we only need to observe that an agent may have a \wit{} for any \requ{}.
    Then, the \wit{} is always a candidate for \qHowV{}.

    Consider \autoref{scen:squish}.
    At issue is whether the agent would repeat the repeat conclusion of \sqE{} as sound.

    Needs to be the case that \ros{} without \wit{}.

    However, the agent has previously concluded \sqE{} is sound.
    Therefore, the agent has a \wit{} for \ros{}.
  \end{argument}

  Note, however, that \autoref{prop:requ-not-n-ce} is weak.
  Compatible in terms of existential.
  Hence we only needed to show a single case in which \wit{} for \requ{}.
  This does not suggest that \requ{1} are not in tension with \issueConstraint{}.
\end{note}

\begin{note}
  The second proposition may help clarify what the existence of \requ{1} amounts to:

  \begin{proposition}
    \label{prop:requ-not-refl}
    \(\pvp{\phi}{v}{\Phi}\) need not be a \requ{} of concluding \(\pv{\phi}{v}\) from \(\Phi\).
  \end{proposition}
  \begin{argument}{prop:requ-not-refl}
    It need not be the case that the agent knows that \(\pv{\phi}{v}\) from \(\Phi\) is a \fc{}.

    The simplest cases are at the outset.
    It need not be the case that \citeauthor{Maksimova:1977un} knew there are exactly five intermediate logics that have the interpolation property was a \fc{} before proving such.

    However, the same hold for when the agent pairs \(\phi\) with \(v\).
    For, the agent need not know they would repeat the reasoning.
    To \illu{0}, consider being guided through a complex argument.
    Follow along, and conclude.
    At each step, do the reasoning, the guide highlights which sub-conclusions to draw.
    However, guide goes away.
    And, given complexity, no repetition without guide.
  \end{argument}

  In general, \autoref{prop:requ-not-refl} highlights that there are no trivial \requ{1}.

  Well, in terms of \sR{}.
  Don't need \sR{}.
  Novel results, and so on.

  And, with guidance, removed the need for \sR{} with respect to overall.
\end{note}

\section{\requ{3}, competence}
\label{cha:requs:sec:add-illu}

\begin{note}
  \requ{3} have a key role in identifying counterexamples to \issueConstraint{}.

  Important that there are cases in which \requ{1} exist.
  At issue is whether \illu{1} \ref{illu:lost-key} and \ref{scen:squish} illustrate instances of \requ{1}.
  At issue is \emph{not} whether \requ{1} are the correct interpretation of \scen{1} that parallel \illu{1} \ref{illu:lost-key} and \ref{scen:squish}.

  These two issues are related.
  If there is some alternative interpretation of \illu{1} \ref{illu:lost-key} and \ref{scen:squish}, then deny that \requ{1} exist.

  Hence, general motivation for the existence of \requ{}.
  Specifically, the goal of this section is to extend discussion of the way competence and performance motivated \autoref{scen:squish}.

  We will set \autoref{illu:lost-key} aside.
  For, the importance of \requ{1} is limited to cases in which \(\pvp{\psi}{v'}{\Psi}\) is a \requ{0} of concluding \(\pv{\phi}{v}\) from \(\Phi\) and the agent goes on to conclude \(\pv{\phi}{v}\) from \(\Phi\).

  Hence, while cases such as \autoref{illu:lost-key} in which an agent does not conclude due to a \requ{} motivate the general idea of a \requ{}, nothing in particular hangs on such cases.
\end{note}

\begin{note}
  \illu{3} of \requ{1}.
  Interest is in cases where

  \begin{itemize}
  \item
    \(\pvp{\psi}{v'}{\Psi}\) is a \requ{} of concluding \(\pv{\phi}{v}\) from \(\Phi\).
  \item
    The agent is concluding \(\pv{\phi}{v}\) from \(\Phi\).
  \end{itemize}
  For, in such cases:
  \begin{itemize}
  \item
    \(\pv{\psi}{v'}\) from \(\Psi\) is a \fc{}.
  \item
    Agent is sufficiently `sensitive' to whether or not \(\pv{\psi}{v'}\) from \(\Psi\) is a \fc{}.
  \item
    \ninf{} due to sensitivity.
  \end{itemize}
  In these cases, these three points may be objected to.

  However, interest is limited to second and third.

  \begin{itemize}
  \item
    Whether an agent is sensitive to whether or not the agent knows \(\pv{\psi}{v'}\) from \(\Psi\) is a \fc{}.
  \item
    \ninf{2} due to sensitivity.
  \end{itemize}
\end{note}

\begin{note}
  \begin{illustration}[Sudoku]
    \label{illu:gist:sudoku}
    % https://tex.stackexchange.com/questions/91422/tikz-sudoku-circle-and-connect-with-lines-some-cells
    Consider the following Sudoku puzzle:%
    \footnote{
      From~\textcite[84]{Coussement:2007up}.
    }
    \vspace{\baselineskip}

    \mbox{ }\hfill%
    \begin{adjustbox}{minipage=0.45\linewidth,scale=1}
      \centering
      \begin{tikzpicture}[scale=.5]
        \begin{scope}
          \draw (0, 0) grid (9, 9);
          \draw[very thick, scale=3] (0, 0) grid (3, 3);
          \setcounter{row}{1}
          % Single entries
          \setrow { }{ }{ }  { }{ }{ }  {1}{ }{ }
          \setrow { }{ }{ }  { }{ }{ }  { }{5}{ }
          \setrow {9}{ }{ }  { }{ }{ }  { }{ }{2}
          \setrow { }{ }{3}  { }{2}{ }  { }{ }{ }
          \setrow { }{ }{ }  {8}{ }{ }  {4}{6}{5}
          \setrow { }{4}{ }  { }{5}{9}  { }{ }{8}
          \setrow { }{8}{7}  {2}{3}{1}  { }{4}{6}
          \setrow {2}{1}{ }  {5}{ }{ }  { }{ }{3}
          \setrow {3}{ }{6}  {4}{ }{8}  { }{ }{ }
        \end{scope}
      \end{tikzpicture}
    \end{adjustbox}%
    \hfill\mbox{ }

  \end{illustration}

  Interactive.
  Fill in the grid.
  Difference between filling in the grid and concluding that solution to the puzzle.
  So, before conclude.
  Is it the case that you would fill in the grid the same way?
\end{note}

\begin{note}
  \autoref{illu:gist:sudoku} parallels \autoref{scen:squish}.

  In both \illu{1}, \(\pv{\phi}{v}\) follows from \(\Phi\) via a rules.

  However, rules are not of direct interest.
  \autoref{scen:squish} is a syntactic proof, but variant \scen{0} in which semantic proof.
  Relevant reasoning may be rule governed, but semantic proofs are not constrained.

  Rather, familiarity.

  The type of reasoning is general.
  Syntactic and semantic proofs, Sudoku puzzles, simple instances of chess problems, all seem to involve general reasoning.
  Likewise, counting, adding, subtracting, and so on.
  Competence established through various proofs, puzzles, problems, and practice.

  In this respect, there is no reasonable doubt the agent is competent.
  At issue is performance.
  Whether reasoning is a specific instance of the general type.
\end{note}

\paragraph{\gR{2}}

\begin{note}
  So far, link between competence and performance.
  Performance is an instance of competence.
  This is the key idea.

  No argue that from this, \requ{1} follow.
\end{note}

\begin{note}
  By contradiction.
  Suppose it is never the case that sensitive.

  That is, never the case in which stop if performance is not instance of competence.

  Then, there is nothing which states whether performance is an instance of competence.
  This applies to reasoning the agent is doing.

  Then, this extends to \fc{1}.
  For, if nothing about next action, then not competence.
\end{note}

\begin{note}
  So, know whether or not present reasoning is \sR{}.

  At issue is whether \fc{}.
  Consider midpoint.
  If don't know \fc{}, then don't know performance is instance of competence.
  For, progressive.
  With the exception of sceptical scenarios, know that actions to perform.
  If competence, then following action is constrained.
\end{note}

\begin{note}
  Puzzle is whether there is any sense to competence if deny knowledge of \fc{}.
\end{note}






\section{Summary}
\label{sec:summary-1}

\begin{note}
  Introduced \requ{1}.
\end{note}

%%% Local Variables:
%%% mode: latex
%%% TeX-master: "master"
%%% End:


% \begin{note}[Problems of induction]
%   Hence, the sketch does not apply to black ravens.
%   I wouldn't conclude all ravens are black if I saw a white raven.

%   I may worry about shortly seeing a white raven when concluding all ravens are black, and I may refuse to entertain the possibility that the sun will rise tomorrow when planning to mow the grass.

%   However, it's not possible to reason to seeing a white raven, nor is it possible to reason to the sun not rising tomorrow.

%   Abstractly, at issue in~\autoref{illu:lost-key} is the possibility of failing to a reason to some proposition-value pair given \emph{present} information, rather than the possibility of failing to a reason to some proposition-value pair given \emph{new} information.

%   To the extent that problems of induction arise from receiving new information, what is at issue is distinct.%
%   \footnote{
%     See~\textcite{Henderson:2020wb} for more on the problem of induction.
%   }

%   Similar points for external world scepticism.
%   Would not conclude that I have hand if disembodied brain in a vat.

%   However, conclusion is out of reach.
% \end{note}

% \subsubsection{Kettle logic}
% \label{sec:failures-1}

% \begin{note}
%   Typically, \(\pv{\phi}{v}\) from \(\Phi\) is not a \requ{}.
%   You don't need to know that you are concluding in order to be concluding.
% \end{note}

% \begin{note}[A copper kettle]
%   A further \illu{0} builds on a story as told by~\citeauthor{Freud:1960wx}.
%   \begin{illustration}[A copper kettle]
%     \label{illu:kettle}
%     \mbox{ }
%     \vspace{-\baselineskip}
%     \begin{quote}
%       `A.\ borrowed a copper kettle from B.\ and after he had returned it was sued by B.\ because the kettle now had a big hole in it which made it unusable.
%       His defence was:
%       ``First, I never borrowed a kettle from B.\ at all;
%       secondly, the kettle had a hole in it already when I got it from him;
%       and thirdly, I gave him back the kettle undamaged.%
%       '''\newline
%       \mbox{ }\hfill\mbox{(\citeyear[62]{Freud:1960wx})}
%     \end{quote}
%     An agent listens to A.'s defence, but does not conclude A.\ has provided testimony.
%   \end{illustration}

%   The agent's failure to conclude A.\ has provided testimony may be understood in terms of a \requ{}.
%   For, A.\ has provided testimony only if what A.\ has said is true.
%   And, what A.\ has said is true only if the three points of A.'s defence are jointly consistent.
%   Putting these observations together, we have the following conditional:

%   \begin{itemize}
%   \item
%     A.\ has provided testimony \emph{only if} if the three points of A.'s defence are jointly consistent.
%   \end{itemize}

%   Failure for the agent to conclude the consequent would prevent the agent from concluding the antecedent.

%   Likewise, there are only three points, and checking for consistency does not require the agent to establish whether the points are (actually) true.

%   \requ{} but not \fc{}.

%   Before the agent concludes A.\ has provided testimony, the agent reasons about whether the three points of A.'s defence are jointly consistent.
%   After, the agent does not conclude A.\ has provided testimony.%
%   \footnote{
%     Any pair of points are jointly inconsistent.
%     For example, consider the first and third:
%     If A.\ never borrowed the kettle from B, then it is not possible for A.\ to have returned the kettle to B.
%   }

%   The story as told by \citeauthor{Freud:1960wx} is comical, but the \requ{0} identified is fairly general.
%   In many cases one may only accept a story if the details are in harmony, and dissonance leads to rejection.
% \end{note}
