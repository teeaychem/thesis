\chapter{\requ{3}}
\label{cha:requs}

\begin{note}
  \autoref{cha:fcs} introduced \fc{1}.

  \fc{1} are such that:
  \begin{itemize}
  \item
    \ros{} holds between \(\pv{\psi}{v'}\) and \(\Psi\) (for the agent).
  \item
    The agent doesn't have a \wit{} for the \ros{} between \(\pv{\psi}{v'}\) and \(\Psi\).
  \end{itemize}

  A \requ{} is a relation between \(\pv{\psi}{v'}\) from \(\Psi\) being a \fc{} and the agent \emph{concluding} \(\pv{\phi}{v}\) from \(\Phi\).
\end{note}

\begin{note}
  Important to generate counterexamples to \issueConstraint{}.
  Still, only certain \requ{1} generate counterexamples to \issueConstraint{}.%
  \footnote{
    I.e.\ one may jointly hold:
    \begin{enumerate*}[label=(\alph*)]
    \item
      various \requ{1} exist, and
    \item
      answers to \qWhyV{} are constrained by answers to \qHowV{} via \issueConstraint{}
    \end{enumerate*}%
    .
    For more, see \autoref{prop:requ-not-n-ce} on \autopageref{prop:requ-not-n-ce}, below.
  }
  Hence, we introduce and motivate the idea of a \requ{0} with \requ{1} which are compatible with \issueConstraint{}.
  \autoref{cha:binding} explicitly links \requ{1} and \issueConstraint{}.
\end{note}

\begin{note}
  The chapter has two sections:
  \begin{TOCEnum}
  \item
    \TOCLine{cha:requs:requs}

    Introduction, definition, intuition, and \illu{1}
  \item
    \TOCLine{sec:typicalRequs}

    \autoref{cha:typical}, \tC{}.
    Motivate existence of \requ{1}.
  \end{TOCEnum}
\end{note}

\section{\requ{3}}
\label{cha:requs:requs}

\begin{note}
  Idea of a \requ{} is something that must be the case in order for the agent to be concluding.

  Various things need to be the case.
  In general, the agent must be alive and conscious.
  Specific instances, further constraints.
  For example, the coffee is too hot.
  Cup of coffee and a preference for the temperature of coffee.

  A \requ{0} is a \prop{0}-\val{0} pair which must be a \fc{0} for the agent in order for the agent to be concluding some \prop{0}-\val{0} pair.

  In particular, this thing is a \fc{}.
  We define \requ{1} as follows:

  \begin{definition}[A \requ{0}]%
    \label{def:requ}%
    % \cenLine{
    %   \begin{VAREnum}
    %   \item
    %     Agent: \vAgent{}
    %   \item
    %     \prop{3}: \(\phi\), \(\psi\)
    %   \item
    %     \val{3}: \(v\), \(v'\)
    %   \item
    %     \pool{3}: \(\Phi\), \(\Psi\)
    %   \item
    %     \mbox{ }
    %   \end{VAREnum}
    % }\newline
    % \cenLine{
    %   \begin{VAREnum}
    %   \item
    %     Event: \(e\)
    %   \item
    %     \mbox{ }
    %   \end{VAREnum}
    % }
    \vspace{-\baselineskip}
    \begin{itemize}
    \item
      \(\pvp{\psi}{v'}{\Psi}\) is a \emph{\requ{}} of \(e\) being an event in which \vAgent{} is concluding \(\pv{\phi}{v}\) from \(\Phi\).
    \end{itemize}

    \emph{If and only if}

    \begin{itemize}
    \item
      \begin{itenum}
      \item[\emph{If}:]
        \(e\) is an event in which \vAgent{} is concluding \(\pv{\phi}{v}\) from \(\Phi\).
      \item[\emph{Then}:]
        \(\pv{\psi}{v'}\) from \(\Psi\) is a \fc{} for \vAgent{}.
      \end{itenum}
    \end{itemize}
    \vspace{-\baselineskip}
  \end{definition}

  \noindent%
  Shorten the defined term to:
  \begin{notationList}
  \item
    \(\pvp{\psi}{v'}{\Psi}\) is a \requ{} of an agent concluding \(\pv{\phi}{v}\) from \(\Phi\).
  \end{notationList}
\end{note}

\begin{note}
  Similar to \qWhyV{} on \autopageref{questionWhyV}.
  Something about event which gets this constraint.

  \itc{2}, so either not concluding or \fc{}.

  Something about the event secures requirement.

  Agent is or is not concluding \(\pv{\phi}{v}\) from \(\Phi\).
  However, requirement if agent is concluding.
  And, possible block if agent is not.

  However, it does not follow from \(\pvp{\phi}{v}{\Phi}\) being a \requ{} of an agent concluding \(\pv{\phi}{v}\) from \(\Phi\) that either:
  \begin{itemize}
  \item
    \(\pv{\phi}{v}\) being a \fc{} from \(\Psi\) is relevant to the agent concluding \(\pv{\phi}{v}\) from \(\Phi\), if the agent is concluding \(\pv{\phi}{v}\) from \(\Phi\).
  \item
    \(\pv{\phi}{v}\) not being a \fc{} from \(\Psi\) is relevant to the agent not concluding \(\pv{\phi}{v}\) from \(\Phi\), if the agent is not concluding \(\pv{\phi}{v}\) from \(\Phi\).
  \end{itemize}

  For \requ{} to be of interest, additional considerations.
  Following section.

  First, we two \illu{1} of \requ{1}.
\end{note}

\paragraph*{\illu{3}}

\begin{note}
  \begin{scenario}[Lucas numbers]%
    \label{scen:LucasNums}%
    The Lucas numbers are recursively defined as follows:%
    \footnote{
      Starting at \(2\), see \hyperlink{cite.OEIS.:aa}{OEIS sequence A000032} for more details.
    }
    %
    \[
      L_{n} = \left\{
        \begin{array}{ll}
          2 & \text{if } n = 0 \\
          1 & \text{if } n = 1 \\
          L_{n-1} + L_{n-2} & \text{if } n > 1 \\
        \end{array}
      \right.
    \]

    An agent is alone and quite bored.
    The agent sets out to calculate the first ten Lucas numbers by applying the recursive definition.

    The agent begins:

    \[
    \begin{array}{cccccc}
      L_{0} & L_{1} & L_{2} & L_{3} & L_{4} & \cdots \\
      \hline
      2 & 1 & 3 & 4 & 7 & \cdots \\
    \end{array}
    \]
  \end{scenario}

  \begin{observation}[\requ{3} of \autoref{scen:LucasNums}]%
    \label{obs:LucasRequ}%
    For the event \(e\) as described by \autoref{scen:LucasNums}:
    %
    \begin{itemize}
    \item
      \(\pv{\propI{L\textsubscript{8} = 47}}{\valI{True}}\) from \(\Phi_{8}\) and \(\pv{\propI{L\textsubscript{7} = 29}}{\valI{True}}\) from \(\Phi_{7}\) are both \requ{1} of the agent concluding \pv{\propI{L\textsubscript{[:9]} = [2, 1, 3, 4, 7, 11, 18, 29, 47 76]}}{\valI{True}} from \(\Phi\).
    \end{itemize}
    %
    Where \propI{L\textsubscript{[:9]} = [2, 1, 3, 4, 7, 11, 18, 29, 47 76]} abbreviates \propI{The first ten Lucas numbers are 2, 1, 3, 4, 7, 11, 18, 29, 47, and 76}, and the \pool{1} \(\Phi_{7}\), \(\Phi_{8}\), and \(\Phi\) are left unspecified.
  \end{observation}

  \begin{motivation}{obs:LucasRequ}
    Applying \autoref{def:requ} it is sufficient to establish:

    \begin{itenum}
    \item[\emph{If}:]
      \(e\) is an event in which the agent is concluding\newline \pv{\propI{L\textsubscript{[:9]} = [2, 1, 3, 4, 7, 11, 18, 29, 47 76]}}{\valI{True}} from \(\Phi\).
    \item[\emph{Then}:]
      \pv{\propI{L\textsubscript{8} = 47}}{\valI{True}} from \(\Phi_{8}\) is a \fc{} for the agent.
    \end{itenum}
    %
    And, likewise for \pv{\propI{L\textsubscript{7} = 29}}{\valI{True}} from \(\Phi_{8}\).

    Suppose the agent is concluding \pv{\propI{L\textsubscript{[:9]} = [2, 1, 3, 4, 7, 11, 18, 29, 47 76]}}{\valI{True}} from \(\Phi\).

    By \assuPP{}, there is an event in which the agent concludes\newline \pv{\propI{L\textsubscript{[:9]} = [2, 1, 3, 4, 7, 11, 18, 29, 47 76]}}{\valI{True}} from \(\Phi\).
    For, the proposition contains the tenth Lucas number, and so it seems the agent must have concluded \pv{\propI{L\textsubscript{9} = 76}}{\valI{True}} from some \pool{} \(\Phi_{9}\).

    And, as the agent is reasoning by the recursive definition, the agent obtains \(L_{9}\) by adding \(L_{8}\) and \(L_{7}\).
    Hence, prior to the conclusion \pv{\propI{L\textsubscript{9} = 76}}{\valI{True}}, it seems the agent must conclude \pv{\propI{L\textsubscript{8} = 47}}{\valI{True}} and \pv{\propI{L\textsubscript{7} = 29}}{\valI{True}}.

    And, as the agent is alone, it seems it must be the case the agent concludes \pv{\propI{L\textsubscript{8} = 47}}{\valI{True}} and \pv{\propI{L\textsubscript{7} = 29}}{\valI{True}} without any novel information.

    So, events.
    Hence, \fc{1}.
  \end{motivation}

  Important parts of the motivation for \autoref{obs:LucasRequ} are need in order to conclude, and no other option.

  By the same reasoning, \(L_{6}\) and \(L_{5}\).

  Indeed, \requ{1} of the event which secures \fc{}.

  And, \(L_{9}\) itself.%
  \footnote{
    However, it is not always the case that \(\pvp{\phi}{v}{\Phi}\) is a \requ{} of an agent concluding \(\pv{\phi}{v}\) from \(\Phi\).
    For example, consider the partnered case from \autoref{obs:cds-arb}
  }
\end{note}

\begin{note}
  Is the agent concluding?
  This depends on the event.
  Are the relevant \prop{0}-\val{0} pairs \fc{1}?

  For example, if the agent makes a mistake, then not concluding.
\end{note}

\begin{note}
  As defined, there is little to a \requ{1}.
  The role of a \requ{} is to link an agent concluding to \fc{1}.
\end{note}

\begin{note}
  \begin{scenario}[Lost keys]%
    \label{illu:lost-key}%
    An agent thinks they may have lost their keys.
    They usually leave place my keys on the right side of their desk, near a copy of~\citeauthor{Vickers:1989tr}'s~\citetitle{Vickers:1989tr} they've been saving for a rainy day.
    Their keys aren't there.

    They've searched over, under, and beside the desk.
    They haven't found their keys.

    Still, the agent holds the following principle:
    \begin{quote}
      If the agent thinks of a place to look for an object, the agent does not conclude the object is lost without searching the place.
    \end{quote}
  \end{scenario}

  \noindent%
  % At issue is whether the agent concludes they have lost their keys without any further search.
  % Fix the event to begin when the \agents{} completes their initial search of the desk and extend the event until the agent either
  % \begin{enumerate*}[label=(\alph*), ref=(\alph*)]
  % \item
  %   concludes they have lost their keys without any further search, or
  % \item
  %   sets of on a further search.
  % \end{enumerate*}
  Given the presentation of \autoref{illu:lost-key} it seems:
  \begin{itemize}
  \item
    \pvp{\propI{The agent has no further idea of where to look}}{\valI{True}}{\Psi} is a \requ{} of the agent concluding \pv{\propI{The agent has lost their keys}}{\valI{True}} from \(\Phi\).
  \end{itemize}
  The principle the agent holds requires the agent to conclude the have no further idea of where to look before concluding they have lost their keys.%
  \footnote{
    Assumes the agent has negative influence.

    I think this is clear.
    Various things that influence.

    And, though possible that there are various demons that change things, do not consider these relevant.

    Note, this does not entail that the agent has positive influence, nor choice.
  }
  Then, by principle, it must be the case that no further place.
  For, else go look at that place.
  And, if go look, then the relevant event is over.

  \requ{2}.
  However, only given the principle.
  It may be the case that the agent abandons principle.
  However, if so then description is not true of relevant event.
  Subtle point.
  Need something about the way things are which constrains the way things may develop.

  If this is not the case, then no \requ{1}.
  However, I assume this is the case.%
  \footnote{
    Various demons, rule out by description.
  }
\end{note}

\begin{note}
  Interesting features:

  \begin{observation}
    It need may be the case that both:
    \begin{itemize}
    \item
      \pvp{\propI{The agent has no further idea of where to look}}{\valI{True}}{\Psi}
    \item
      \pvp{\propI{The agent's may be by the coffee machine}}{\valI{True}}{\Psi}
    \end{itemize}
    are \fc{1}.
  \end{observation}

  This is fine.
  If the latter, then the agent does not conclude they have lost their keys.

  Now, with conditional, is it the case that concluding?
  Is the agent going to think of something?

  If the agent does, then they go and search.
  If the agent does not, then they conclude lost.

  There is not immediate answer.
  However, either or.

  To illustrate further.
  Suppose coffee.
  May think to retrace steps.
  If so, no conclusion.
  However, the agent may fail to think of anything.
  If so, the agent concludes.

  Still, it is not the case that conclusion is in progress, for the agent may think of a place to search.

  It may be the case that \fc{} and not concluding.
  But, this is no problem.
  For, \fc{1} are not exclusive.
  But, this is consistent with \requ{}.
\end{note}


\section{\requ{3} and \tC{}}
\label{sec:typicalRequs}

\begin{note}
  \begin{proposition}[\tC{2} and \requ{1}]
    \label{prop:hinge}
    % \cenLine{
    %   \begin{VAREnum}
    %   \item
    %     Agent: \vAgent{}
    %   \item
    %     \prop{3}: \(\phi\), \(\psi\)
    %   \item
    %     \val{3}: \(v\), \(v'\)
    %   \item
    %     \pool{3}: \(\Phi\), \(\Psi\)
    %   \item
    %     \mbox{ }
    %   \end{VAREnum}
    % }\newline
    % \cenLine{
    %   \begin{VAREnum}
    %   \item
    %     Event: \(e\)
    %   \item
    %     Type of reasoning: \(T\)
    %   \item
    %     \mbox{ }
    %   \end{VAREnum}
    % }
    \vspace{-\baselineskip}
    \begin{itenum}
    \item[\emph{If}:]
      Clauses~\ref{prop:hinge:A:rep},~\ref{prop:hinge:A:tI},~and~\ref{prop:hinge:A:novel} jointly hold for \(e\):
      \begin{enumerate}[label=\arabic*., ref=(\arabic*)]
      \item
        \label{prop:hinge:A:rep}
        \(T'\) is a \tRep{} of \vAgent{} \tCV{} \(\pv{\phi}{v}\) from \(\Phi\) by type \(T\) in \(e\).
      \item
        \label{prop:hinge:A:tI}
        \(\pvp{\psi}{v'}{\Psi}\) is a \tI{} of \(T'\)
      \item
        \label{prop:hinge:A:novel}
        The following conditional is true:
        \begin{itemize}
        \item[\emph{If}:]
          There is some available action \(a\) such that \vAgent{} is concluding \(\pv{\psi}{v'}\) from \(\Psi\), when \vAgent{} does \(a\).
        \item[\emph{Then}:]
          There is some available action \(a'\) such that \vAgent{} is concluding \(\pv{\psi}{v'}\) from \(\Psi\) without use of any novel information obtained by doing \(a'\), when \vAgent{} does \(a'\).
        \end{itemize}
      \end{enumerate}
    \item[\emph{Then}:]
      \(\pvp{\psi}{v'}{\Psi}\) is a \requ{} of \(e\) being an event in which \vAgent{} is \tCV{} \(\pv{\phi}{v}\) from \(\Phi\).
    \end{itenum}
    \vspace{-\baselineskip}
  \end{proposition}

  \noindent%
  \autoref{prop:hinge}, sufficient conditions for \requ{} when \tCV{}.
  Useful proposition with respect to counterexamples to \issueConstraint{}.

  Intuitive motivation for \autoref{prop:hinge} is simple.
  \autoref{prop:tC-and-fc} (\autopageref{prop:tC-and-fc}).

  If clauses hold and agent is \tCV{}, then \fc{}.
  This is exactly what is required for \requ{}.

  In detail:

  \begin{argument}{prop:hinge}
    Assume~ clauses~\ref{prop:hinge:A:rep},~\ref{prop:hinge:A:tI},~and~\ref{prop:hinge:A:novel} jointly hold.
    Suppose \tCV{}.
    Then, by \autoref{prop:tC-and-fc}, \fc{}.

    Then by \autoref{def:requ}, \requ{}.
  \end{argument}

  Recap.

  \tCV{}.
  Intuitively, various other things conclude.
  \tRep{}, requires events.
  Conditional of \autoref{prop:hinge:A:novel} strengthens so no novel information.

  As get \fc{}, then \requ{}.

  Importance of \autoref{prop:hinge} is existence of \requ{}.
  \tCV{} without novel information.
\end{note}

\paragraph*{Sound rules}

\begin{note}
  \phantlabel{squish-elimination-proof}

  \begin{illustration}[Squish elimination]%
    \label{scen:squish}%
    It is late morning on a sunny day.
    I ate a good breakfast, and drank some nice coffee.
    I have completed a handful of syntactic proofs for entailments of propositional logic using the basic rules of inference in a Fitch-style system.

    I create the following syntactic proof:
    \begin{center}
      \begin{fitch}
        \phantlabel{illu:sP:1}\fa (P \rightarrow Q) \rightarrow P \\
        \phantlabel{illu:sP:2}\fj R \\
        \phantlabel{illu:sP:3}\fa P & \sqE{}:\hyperref[illu:sP:1]{1} \\
        \phantlabel{illu:sP:4}\fa P \land R & \(\land\)\textbf{Intro:} \hyperref[illu:sP:2]{2},\hyperref[illu:sP:3]{3}
      \end{fitch}
    \end{center}

    Still, I haven't yet concluded \((P \rightarrow Q) \rightarrow P, R \vdash P \land R\).

    For, \sqE{} may not be a sound rule of inference.

    I go on to conclude \((P \rightarrow Q) \rightarrow P, R \vdash P \land R\).
  \end{illustration}

  \begin{definition}[\sqE{}]%
    \label{def:sque}%
    \sqE{} is the following rule:
    \begin{center}
      \begin{fitch}
        \ftag{\text{\scriptsize \emph{i}}}{\fa (\phi \rightarrow \psi) \rightarrow \phi} \\
        \ftag{\text{\scriptsize }}{\fa \vdots } \\
        \ftag{\text{\scriptsize \emph{j}}}{\fa \phi } & \sqE{}:\emph{i} \\
      \end{fitch}
    \end{center}
  \end{definition}

  \sqE{} is sound.%
  \footnote{
    \label{prop:sqE-sound}
    Rather than prove \sqE{} is sound (which would require a detailed statement of the proof system in question), we prove that the corresponding semantic entailment holds:

    Let \(v\) be an arbitrary (truth-functional) valuation, and assume \(v((\phi \rightarrow \psi) \rightarrow \phi) = \valI{True}\).
    Further, assume for contradiction \(v(\phi) = \valI{False}\).

    As \(v(\phi) = \valI{False}\), it immediately follows that \(v(\phi \rightarrow \psi) = \valI{True}\).
    Therefore, by the first assumption, it must be the case that \(v(\phi) = \valI{True}\).
    This contradictions the second assumption.
    % Hence, \((\phi \rightarrow \psi) \rightarrow \phi \vDash \phi\).
  }

  The relevant propositions, values, and \pool{1} are as follows:
  \begin{itemize}[noitemsep]
  \item
    I am the agent.
  \item
    \(\phi\) is the proposition: \(\mathsf{(P \rightarrow Q) \rightarrow P, R \vdash P \land R}\).
  \item
    \(\psi\) is the proposition: \propI{\sqE{} is sound}
  \item
    Both \(v\) and \(v'\) are the value: \valI{True}.
    And,
  \item
    The pools of premises \(\Phi\) and \(\Psi\) are left unspecified.%
    \footnote{
      \((P \rightarrow Q) \rightarrow P\) and \(R\) are premises of the deduction, and not elements of \(\Phi\).

      For, my conclusion is \pv{\mathsf{(P \rightarrow Q) \rightarrow P, R \vdash P \land R}}{\valI{True}}.
      My conclusion is not \pv{\mathsf{P \land R}}{\valI{True}}.
    }
  \item
    Key observations:
    \begin{itemize}[noitemsep]
    \item
      sunny day, good breakfast, nice coffee.
      I.e.\ as ideal as situations may be for syntactic proofs.
    \item
      Proved \sqE{} numerous times before.
    \end{itemize}
  \end{itemize}

  Hence, the relevant instance of the conditional by which a \requ{} is defined is:

  \begin{quote}
    \begin{itenum}
    \item[\emph{If}:]
      If \pv{\propI{\sqE{} is sound}}{\valI{True}} from \(\Psi\) is not a \fc{}.
    \item[\emph{Then}:]
      I am not concluding \(\pv{\mathsf{(P \rightarrow Q) \rightarrow P, R \vdash P \land R}}{\valI{True}}\) from \(\Phi\).
    \end{itenum}
  \end{quote}

  In contrast to \autoref{illu:lost-key}, \autoref{scen:squish} leads to a conclusion.

  At issue is whether \requ{}.
  By \autoref{prop:hinge}, three things.
  \begin{itemize}
  \item
    \tRep{}
  \item
    \sqE{} is \tI{} of \(T'\).
  \item
    Conditional.
  \end{itemize}

  Plausibly, four \prop{0}-\val{0}-\pool{0} pairs in type:

  \begin{center}
    \begin{tabular}{R{.45\textwidth} L{.45\textwidth}}
      \prop{2}-\val{0} pair & \pool{2} \\
      \hline
      \pv{(\phi \rightarrow \phi) \rightarrow \psi \space/\space \phi}{\textover[c]{\valI{Sound}}{\valI{Unsound}}} & \dots \\
      \pv{\phi \rightarrow (\psi \rightarrow \phi) \space/\space \phi}{\valI{Unsound}} & \dots \\
      \pv{\psi \rightarrow (\phi \rightarrow \psi) \space/\space \phi}{\valI{Unsound}} & \dots \\
      \pv{(\psi \rightarrow \phi) \rightarrow \psi \space/\space \phi}{\valI{Unsound}} & \dots \\
    \end{tabular}
  \end{center}

  Throughout \autoref{scen:squish} I \emph{know} \sqE{} is sound.
  Prior to \autoref{scen:squish} I have proved \sqE{} is sound on various occasions using the same basic observations made in the argument for \autoref{prop:sqE-sound}.

  However, there is a distinction between \emph{knowing} \sqE{} is sound and \emph{proving} \sqE{} is sound.
  For example, if I have just drunk a considerable amount of wine, or woken from a night of tormented sleep.
  Generally said, I may not be thinking straight.

  Of course, I may conclude \(\pv{\phi}{v}\) from \(\Phi\) regardless of whether \(\pv{\psi}{v'}\) from \(\Psi\) is a \fc{}.
  Indeed, given some particularly good wine I may conclude \(P \land C \vdash O\).%
  \footnote{
    For, \(P \land C\) reads `Pac', \(O\) looks like a pellet, and Pacman likes to eat pellets.
  }
  However, then not \tCV{}.
\end{note}

\begin{note}
  Note, does not depend on \agpe{my}.
\end{note}

\section*{Summary}

\begin{note}
  Introduced \requ{1}.

  Defined in terms of \tC{}.

  Motivated by \tC{}.
\end{note}



% \begin{note}[Problems of induction]
%   Hence, the sketch does not apply to black ravens.
%   I wouldn't conclude all ravens are black if I saw a white raven.

%   I may worry about shortly seeing a white raven when concluding all ravens are black, and I may refuse to entertain the possibility that the sun will rise tomorrow when planning to mow the grass.

%   However, it's not possible to reason to seeing a white raven, nor is it possible to reason to the sun not rising tomorrow.

%   Abstractly, at issue in~\autoref{illu:lost-key} is the possibility of failing to a reason to some proposition-value pair given \emph{present} information, rather than the possibility of failing to a reason to some proposition-value pair given \emph{new} information.

%   To the extent that problems of induction arise from receiving new information, what is at issue is distinct.%
%   \footnote{
%     See~\textcite{Henderson:2020wb} for more on the problem of induction.
%   }

%   Similar points for external world scepticism.
%   Would not conclude that I have hand if disembodied brain in a vat.

%   However, conclusion is out of reach.
% \end{note}

    %   \footnote{
    %     The present point is similar to issues raised by \citeauthor{Harman:1973ww} (\citeyear{Harman:1973ww}) regarding the proposed equivalence between reasons for which an agent believes something with reasons the agent would offer if asked to justify their belief.
    %     As \citeauthor{Harman:1973ww} notes, an agent may offer reasons because they think they will convince their audience, not because the agent is compelled by the reasons, etc.
    %     (\citeyear[Ch.2]{Harman:1973ww})

    %     To the extent that \citeauthor{Harman:1973ww}'s point is that what holds from an \agpe{} need not actually be the case, the point in the same.
    %     However, to the extent that \citeauthor{Harman:1973ww} relies on an under-specification of what holds from an \agpe{} --- i.e.\ the distinction between whether \(\phi\) has value \(v\) from the \agpe{} or whether the agent evaluates as true the proposition that their audience is responsive to \(\phi\) having value \(v\), the point is distinct.
    %   }


%%% Local Variables:
%%% mode: latex
%%% TeX-master: "master"
%%% TeX-engine: luatex
%%% End:

