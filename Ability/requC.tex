\chapter{\requ{3}}
\label{cha:requs}

\begin{note}
  \autoref{cha:fcs} introduced \fc{1}.
  The present chapter links \fc{1} to concluding.

  State what \requ{1} are.
  Provide \illu{1}.
\end{note}

\begin{note}
  With the definition of a \requ{} in hand, the remainder of the chapter mostly consists of \illu{1} of \requ{1}.

  \requ{3} link \fc{1} to concluding, and whether or not an agent is concluding is involved in whether or not an agent concludes.
  And, important function in generating counterexamples to \issueConstraint{}.

  However, the existence of \requ{1} does not presuppose counterexamples to \issueConstraint{}.
  In particular, though all of the \illu{1} involve \fc{1}, the \illu{1} are constructed so that the agent has a \wit{1} for any \ros{} which follows from the agent's knowledge.

  We will turn to counterexamples to \issueConstraint{} in \autoref{cha:ces}, after explicitly linking \requ{1} to \qWhyV{} in \autoref{cha:binding}.
\end{note}

\begin{note}
  \begin{itemize}
  \item
    \TOCLine{cha:requs:sec:infl}

    Preliminary distinction used to understand \requ{1}
  \item
    \TOCLine{cha:requs:sec:definition}

    Introduction, definition, intuition, and \illu{1}
  \item
    \TOCLine{cha:requs:sec:add-illu}

    Additional \illu{1}.
  \end{itemize}
\end{note}

\section{\pinf{2} and \ninf{0}}
\label{cha:requs:sec:infl}

\begin{note}
  There are instances in which an agent has \emph{\ninf{0}} over whether or not they are concluding \(\pv{\phi}{v}\) from \(\Phi\).
  For, concluding is an activity performed by an agent, and in certain cases the agent may choose to stop the activity.

  For example, may decide that a result, even if obtained, is not worth the effort.%
  \footnote{
    Ex.\ bonus problem on a homework.
  }
  Indeed, this observation is not different to the observation that an agent has \ninf{0} over whether or not they are going to buy some eggs.
  For, the agent may choose to turn back home.

  To say an agent has \ninf{0} is not to say the agent has \emph{\pinf{}}.
  For, an agent may not have choice over whether an event develops into an event in which the agent concludes \(\pv{\phi}{v}\) from \(\Phi\).

  For example, take \(\chi\) to be the proposition `the area of a unit square is equal to the area of a unit circle'.
  An agent has \ninf{0} over whether or not they concluding \(\pv{\chi}{\text{True}}\), as the agent may choose not to make any attempt.
  However, assuming the agent would only conclude \(\pv{\chi}{\text{True}}\) from principles consistent with Euclidean geometry, then agent does not have \pinf{} over whether or not they are concluding \(\pv{\chi}{\text{True}}\).
  For, the area of a unit square is not equal to the area of a unit circle.
  Hence, it is not possible for the agent to ensure that some event may develop into an event in which they conclude \(\pv{\chi}{\text{True}}\).

  In parallel, an agent need not have \pinf{0} over whether or not they are going to buy some eggs.
  For, it may be there are no eggs for sale.
  Hence, it may not be possible for the agent to ensure that the event develops into an event in which the agent buys some eggs.
\end{note}


\section{\requ{3}}
\label{cha:requs:sec:definition}

\begin{note}
  \begin{definition}[A \requ{0}]
    \label{def:requ}
    \begin{itemize*}[noitemsep, label=\(\circ\)]
    \item
      An agent: \vAgent{}
    \item
      Propositions: \(\phi\), \(\psi\)
    \item
      Values: \(v\), \(v'\)
    \item
      \poP{3}: \(\Phi\), \(\Psi\)
    \item
      An event: \(e\)
    \item
      \mbox{ }
    \end{itemize*}

    \begin{itemize}
    \item
      \(\pvp{\psi}{v'}{\Psi}\) is a \emph{\requ{}} of \(e\), with respect to \(\pvp{\phi}{v}{\Phi}\).
    \end{itemize}

    \emph{If and only if}

    \begin{itemize}
    \item
      The following conditional is true:
      \begin{itemize}
      \item[\emph{If}:]
        \begin{enumerate}[label=\alph*., ref=(\alph*), series=requDefSeries]
        \item
          \label{def:requ:nK}
          \vAgent{} does not (recognise that they) know \(\pv{\psi}{v'}\) from \(\Psi\) is a \fc{} throughout \(e\).
        \end{enumerate}
      \item[\emph{Then}:]
        \begin{enumerate}[label=\alph*., ref=(\alph*), resume*=requDefSeries]
        \item
          \label{def:requ:nC}
          \(e\) is not an event in which \vAgent{} is concluding \(\pv{\phi}{v}\) from \(\Phi\).\newline
          \mbox{ }\hfill(\emph{Due to \ref{def:requ:nK}})
        \end{enumerate}
      \end{itemize}
    \end{itemize}
    \vspace{-\baselineskip}
  \end{definition}
\end{note}

\begin{note}
  If an agent doubts that \(\pv{\psi}{v'}\) is a \fc{}, then the agent is not concluding \(\pv{\phi}{v}\) from \(\Phi\).
\end{note}

\begin{note}
  Our interest with \requ{1} is with regard to \ninf{0}.
  Intuitively, if an agent does not know that \(\pv{\psi}{v'}\) is a \fc{} from \(\Psi\), then the agent will stop the current event from developing any further

  If the agent stops the event, then it is not the case that the event is such that the agent is concluding \(\pv{\phi}{v}\) from \(\Phi\).

  Indeed, even if the event could have developed into an event in which the agent concluded \(\pv{\phi}{v}\) from \(\Phi\).

  Initial example, result was not worth the effort, even if obtained.
  The agent may well have obtained, but as the agent decided the result was not worth the effort, their decision ensures they are not concluding.
\end{note}

\begin{note}
  Key thing is that whether or not the agent knows that \(\pv{\psi}{v'}\) from \(\Psi\) is a \fc{} \influence{1} whether or not the agent is concluding.
\end{note}

\begin{note}
  The qualifier `due to' is appended to \ref{def:requ:nC} to ensure that the conditional captures the agent's \ninf{}, rather than being trivially true because the agent is not concluding \(\pv{\phi}{v}\) from \(\Phi\).%
  \footnote{
    See, for example, \citeauthor{Lewis:1997wg}'s (\citeyear{Lewis:1997wg}) discussion of finkish dispositions.
  }
\end{note}

\begin{note}
  \color{red}
  Not a fluke.
\end{note}


\begin{note}
  Combine with the definition of a \fc{}.

  \begin{proposition}[A \requ{0}, expanded]
    \label{prop:requ-fc}
    \begin{itemize*}[noitemsep, label=\(\circ\)]
    \item
      An agent: \vAgent{}
    \item
      Propositions: \(\phi\), \(\psi\)
    \item
      Values: \(v\), \(v'\)
    \item
      \poP{3}: \(\Phi\), \(\Psi\)
    \item
      An event: \(e\)
    \item
      \mbox{ }
    \end{itemize*}

    \begin{itemize}
    \item
      \(\pvp{\phi}{v'}{\Psi}\) is a \emph{\requ{}} of \(e\) just in case:
      \begin{itemize}
      \item[\emph{If}:]
        \begin{enumerate}[label=\alph*., ref=(\alph*), series=requDefSeries]
        \item
          \label{prop:requ-fc:nk}
          Throughout \(e\) \vAgent{} does not (recognise that they) know either:
          \begin{enumerate}[label=\roman*., ref=(\roman*)]
          \item
            \label{prop:requ-fc:nk:psi}
            There is a \pevent{} in which \vAgent{} concludes \(\pv{\psi}{v'}\) from \(\Psi\).
          \item
            \label{prop:requ-fc:nk:no-conf}
            There is no \pevent{} in which \vAgent{} concludes something incompatible with \(\pv{\psi}{v'}\).
          \end{enumerate}
        \end{enumerate}
      \item[\emph{Then}:]
        \begin{enumerate}[label=\alph*., ref=(\alph*), resume*=requDefSeries]
        \item
          \label{prop:requ-fc:ne}
          \(e\) is not an event in which \vAgent{} is concluding \(\pv{\phi}{v}\) from \(\Phi\).
        \end{enumerate}
      \end{itemize}
    \end{itemize}
    \vspace{-\baselineskip}
  \end{proposition}

  \begin{argument}{prop:requ-fc}
    Immediate from the definition of a \requ{} and the definition of a \fc{}.%
    \footnote{
      The definition of a \fc{} is \autoref{def:requ}, stated on~\autopageref{def:requ}.
    }
  \end{argument}
\end{note}

\begin{note}
  So, interactions with either of these two things.
\end{note}

\begin{note}
  \(\pv{\phi}{v}\) from \(\Phi\) entails \(\pv{\psi}{v'}\) from \(\Psi\).

  Now, two things.

  If no \pevent{}, something has gone wrong.
  Likewise, if something conflicting.
\end{note}

\subsection{Lost keys and sound rules}
\label{cha:requs:sec:init-illustr}

\begin{note}
  Two \illu{1}.
  In each \illu{1}, we identify a \requ{} and indicate whether or not the agent exerts \ninf{}.
\end{note}

\subsubsection{Lost keys}

\begin{note}
  \begin{illustration}[Lost keys]
    \label{illu:lost-key}
    I think I might have lost my keys.
    I usually leave place my keys on the right side of my desk, next to a copy of~\citeauthor{Vickers:1989tr}'s~\citetitle{Vickers:1989tr} which I've been saving for a rainy day.
    And, my keys aren't there.

    I've searched on the desk, under the desk, and beside the desk.
    And, I haven't found my keys.

    Still, I haven't (yet, at least) \emph{concluded} that I've lost my keys.

    For, there might still be some place I haven't looked.
    If I think a little harder a figure out where that place is, I would conclude my keys might be in that place.
    And, my keys aren't lost if they are in that place.
    So, I might conclude that my keys aren't lost, which would conflict with concluding that my keys are lost.
  \end{illustration}

  You may disagree with the tension I see in~\autoref{illu:lost-key}.
  Perhaps it's fine to conclude my keys are lost while allowing for the possibility that they're some place I haven't yet thought of.
  However, there's tension for me.
  `I've lost my keys, but they might be under that book' feels bad to me, and to me the badness extends to `I've lost my keys, but they might in that place I haven't yet considered'.

  Though, my goal is only to convince you that my refusal to conclude I've lost my keys makes sense.
  The way in which you think about the truth conditions for the sentence `I've lost my keys' may be different, but I expect my thoughts are intelligible.
\end{note}

\begin{note}
  Filling in the details of \autoref{illu:lost-key}:
  \begin{itemize}[noitemsep]
  \item
    I am the agent.
  \item
    \(\phi\) is the proposition: `I have lost my keys'.
  \item
    \(\psi\) is a some proposition: `My keys are not in location \(l\)'
  \item
    Both \(v\) and \(v'\) are the value: `True'.
    And,
  \item
    The pools of premises \(\Phi\) and \(\Psi\) are left unspecified.
  \end{itemize}

  Hence, the relevant instance of the conditional by which a \requ{} is defined is:

  \begin{quote}
    \begin{itemize}
    \item[\emph{If}:]
      If I do not (recognise that I) know \(\pv{\text{My keys are not in location }l}{\text{True}}\) from \(\Psi\) is a \fc{}.
    \item[\emph{Then}:]
      I am not concluding \(\pv{\text{I have lost my keys}}{\text{True}}\) from \(\Phi\).
    \end{itemize}
  \end{quote}

  The antecedent is true, and hence the consequent is true due to \ninf{} I exert over the event.
\end{note}

\subsubsection{Sound rules}

\begin{note}
  Here, I want to state that the agent has done the proof a number of times.
  So, the agent knows that there is a \pevent{}.
\end{note}

\begin{note}
  \phantlabel{squish-elimination-proof}

  \begin{restatable}[Squish elimination]{illustration}{scenarioPLSquish}
    \label{scen:squish}
    I conclude \((P \rightarrow Q) \rightarrow P, Q \vdash P \land Q\) from the both the following syntactic proof and the soundness of the rules of inference:
    \begin{center}
      \begin{fitch}
        \phantlabel{illu:sP:1}\fa (P \rightarrow Q) \rightarrow P \\
        \phantlabel{illu:sP:2}\fj Q \\
        \phantlabel{illu:sP:3}\fa P & Squish\textbf{Elim:} \hyperref[illu:sP:1]{1} \\
        \phantlabel{illu:sP:4}\fa P \land Q & \(\land\)\textbf{Intro:} \hyperref[illu:sP:2]{2},\hyperref[illu:sP:3]{3}
      \end{fitch}
    \end{center}
  \end{restatable}

  The proof consists of two premises and two rules of inference.
  The two rules of inference are of interest.

  The second rule of inference used is standard `\(\land\)' introduction, and applies to lines \hyperref[illu:sP:2]{2} and \hyperref[illu:sP:3]{3}.
  Where the conditional holds is unclear.
  On the one hand, troubled if failed to show that `\(\land\)' introduction is sound.
  However, testimony\dots

  The first rule of inference is non-standard `Squish' elimination applied to line \hyperref[illu:sP:1]{1}.

  \begin{center}
    \begin{fitch}
      \ftag{\scriptsize i}{\fa (\alpha \rightarrow \beta) \rightarrow \alpha} \\
      \ftag{\scriptsize }{\fa \vdots } \\
      \ftag{\scriptsize j}{\fa \alpha } & Squish\textbf{Elim:}\emph{i} \\
    \end{fitch}
  \end{center}

  \(\alpha\) `squishes' \(\beta\).

  Uncommon, but enough to memorise the rule.

  Apply a rule of inference if and only if it follows from basic rules.

  Further, only if \fc{}.
  For, I consider my general understanding of propositional logic more important than memory.
  And, if failed, then would not consider sound.

  Further, is \fc{}, as I have proved, and would prove again.%
  \footnote{
    For a quick proof, suppose \((P \rightarrow Q) \rightarrow P\) is true.
  And for contradiction assume \(P\) is false.
  As \(P\) is false, it immediately follows that \(P \rightarrow Q\) is true.
  Therefore, by the initial supposition, \(P\) is true.
  Hence, we have obtained the desired contradiction.
  }
\end{note}

\begin{note}
  Filling in the details of the second abstract sketch:
  \begin{itemize}[noitemsep]
  \item
    I am the agent.
  \item
    \(\phi\) is the proposition: `\((P \rightarrow Q) \rightarrow P, Q \vdash P \land Q\)'.
  \item
    \(\psi\) is a some proposition: `Squish elimination is sound'
  \item
    Both \(v\) and \(v'\) are the value: `True'.
    And,
  \item
    The pools of premises \(\Phi\) and \(\Psi\) are left unspecified.%
    \footnote{
      Note, premises of reasoning.
      Distinct from premises of deduction.
    }
  \end{itemize}

  Hence, the relevant instance of the conditional by which a \requ{} is defined is:

  \begin{quote}
    \begin{itemize}
    \item[\emph{If}:]
      If I do not (recognise that I) know \(\pv{\text{Squish elimination is sound}}{\text{True}}\) from \(\Psi\) is a \fc{}.
    \item[\emph{Then}:]
      I am not concluding \(\pv{(P \rightarrow Q) \rightarrow P, Q \vdash P \land Q}{\text{True}}\) from \(\Phi\).
    \end{itemize}
  \end{quote}

  The antecedent is true, and hence the consequent is true due to \ninf{} I exert over the event.
\end{note}

\begin{note}
  As with \autoref{illu:lost-key}, you may differ with respect to whether or not the soundness of squish is a \requ{}.
  It may be sufficient for you that you know squish is a sound rule of inference, regardless of whether or not it is a \fc{}.
  However, if not a \fc{}, then I have a more general concern about the result of any reasoning performed.
  For, the proof is quick, and if fail, then suggests to me that \emph{at present} I am not of right mind to be constructing syntactic proofs.
\end{note}


\subsection{Intuition}
\label{sec:intuition-1}

\begin{note}
  Why is the conditional true.
  Intended interpretation is that failure to know \fc{} amounts to an something like an undercutting defeater.%
  \footnote{
    To my understanding, undercutting defeaters were introduced by \citeauthor{Pollock:1987un} (\citeyear{Pollock:1987un}).
    And, \citeauthor{Pollock:1987un} defines an undercutting defeater as follows:
    \begin{quote}
    R is an \emph{undercutting defeater} for P as a prima facie reason for S to believe Q if and only if
    \begin{enumerate}[label=(UD\arabic*), ref=(UD\arabic*)]
    \item
      \label{pollock:ud:1}
      P is a reason for S to believe Q and R is logically consistent with P but (P and R) is not a reason for S to believe Q, and
    \item
      \label{pollock:ud:2}
      R is a reason for denying that P wouldn't be true unless Q were true.%
      \mbox{}\hfill\mbox{(\citeyear[485]{Pollock:1987un})}
    \end{enumerate}
  \end{quote}
  This definition is hard to square with a \requ{}.
  In particular, \ref{pollock:ud:1}.

  Issue: P is a reason.
  By parallel, the reasoning that the agent has done is sufficient for the agent to conclude.
  However, at issue is precisely whether this is the case.

  }

  We borrow the following sketch from \textcite{Worsnip:2018aa}:
  \begin{quote}
    Undercutting defeaters, which are easiest to think of in the context of the attitude of belief, are supposed to be considerations that undermine the justification of a belief in a proposition p not necessarily by providing (sufficient) positive evidence to think that p is false, but rather merely by suggesting (perhaps misleadingly) that one’s reasons for believing p are no good, in a way that neutralizes or mitigates their justificatory or evidential force.%
    \mbox{}\hfill\mbox{(\citeyear[29]{Worsnip:2018aa})}
  \end{quote}

  In particular, concluding.
  At issue is not whether the \(\phi\) has value \(v\), but whether the agent's reasoning from \(\Phi\) to \(\pv{\phi}{v}\) is any good.

  Justification, and this is one way to go.
  However, not tied to justification.
\end{note}

\subsection{(Recognition of) knowledge}
\label{sec:knowledge}

\begin{note}
  Whether the agent \emph{knows} that \(\pv{\psi}{v'}\) from \(\Psi\) is a \fc{}.
\end{note}

\begin{note}
  With the exception of more-or-less instantaneous actions, future may develop in surprising ways.

  For example, plausible that an agent knows when they strike the cue ball in a certain way, a particular red ball will land in a pocket.
  However, not plausible that the agent knows where the cue ball will come to rest after the red ball lands in the pocket.
  Hence, agent does not know their following move, and so on.

  In parallel, an agent may have no guarantee that they will not be interrupted, etc.
  Hence, in most cases it seems implausible that an agent knows they will concluded.

  Yet, to be \fc{} does not require completion.
  Only the case that concluding, and not concluding any proposition-value which is incompatible.
\end{note}

\begin{note}
  However, press further.
  Conditional fails because it is not the case that sensitive to failures of knowledge.
\end{note}

\begin{note}
  It may be the case that that \requ{0} holds from the \agpe{} but does not hold independently of the \agpe{}.

  At issue is not whether the agent knows, but whether the agent recognises that they know.
  In this respect, the due to clause does not link absence of knowledge to whether or not the agent is concluding, but instead mediated via the agent.

  However, at issue is \requ{} and the agent is concluding.
  For, then, the agent does recognise that they know \(\pv{\psi}{v'}\) from \(\Psi\) is a \fc{}.

  It may be the case that, though from the \agpe{} they would not have concluded \(\pv{\phi}{v}\) from \(\Phi\) if \ros{} failed to hold between \(\pv{\psi}{v'}\) and \(\Psi\), the agent would have concluded \(\pv{\phi}{v}\) from \(\Phi\) regardless.

  For example, suppose an agent has taken a gamble on a coin landing heads.
  The coin lands heads, and the agent receives a prize.
  From the \agpe{}, if the coin failed to lands heads, then the agent would not have received the prize.
  However, the agent was set to receive the prize for participating in the gamble, regardless of whether the coin landed heads.%
  \footnote{
    The present point is similar to issues raised by \citeauthor{Harman:1973ww} (\citeyear{Harman:1973ww}) regarding the proposed equivalence between reasons for which an agent believes something with reasons the agent would offer if asked to justify their belief.
  As \citeauthor{Harman:1973ww} notes, an agent may offer reasons because they think they will convince their audience, not because the agent is compelled by the reasons, etc.
  (\citeyear[Ch.2]{Harman:1973ww})

  To the extent that \citeauthor{Harman:1973ww}'s point is that what holds from an \agpe{} need not actually be the case, the point in the same.
  However, to the extent that \citeauthor{Harman:1973ww} relies on an under-specification of what holds from an \agpe{} --- i.e.\ the distinction between whether \(\phi\) has value \(v\) from the \agpe{} or whether the agent evaluates as true the proposition that their audience is responsive to \(\phi\) having value \(v\), the point is distinct.
  }
\end{note}

\begin{note}
  So, as the agent recognises that they know, the agent does not exert \ninf{}.

  So long as there are cases in which an agent recognises that they know, this should be fine.
  And, I think there are.

  Soundness of a rule is one such example.
  Below, Sudoku.
  Indeed, more generally, when the conclusion follows from some collection of well understood rules and where the relevant conclusion is easy.
  So, chess, arithmetic, and so on.

  In these cases, know.
\end{note}

\begin{note}
  This does not provide a complete solution.
  For any particular instance, possible to construct context so that the agent is insensitive.

  Familiar with external world skepticism.

  \requ{1} are distinct, in that it does not matter whether \(\phi\) has value \(v\).
  However, similar in that it does matter that there is a \pevent{}.

  If agent is a brain in a vat, then this does not prevent the agent from concluding.
  However, this may prevent the existence of a \pevent{}.

  Indeed, a closer parallel is \citeauthor{Schaffer:2010vq}'s (\citeyear{Schaffer:2010vq}) debasing demon.
  The demon \textquote{throws her victims into the belief state on an improper basis, while leaving them with the impression as if they had proceeded properly} (\citeyear[231]{Schaffer:2010vq})
\end{note}

\begin{note}
  These are genuine concerns, but I do not see a clear way of maintaining both skepticism with regard to \requ{1} and instances in which \ros{} answer \qWhyV{}.

  For, if entertaining such \scen{1}, then control over events.
  Hence, control over event in which agent concludes.
  Agent reasoned from \(\Phi\) to \(\pv{\phi}{v}\), but agent's conclusion was distinct.
  Here, in a manner parallel to deviant causal chains.
\end{note}

\begin{note}
  In short, is seems that if deny \requ{1}, then  there is no plausible sense in which \qWhy{} and \qHow{} may be asked, and in turn, no plausible sense in which \qWhyV{} and \qHowV{} may be asked.

  Taking for granted world works in a way that makes sense.
\end{note}

\subsection{What a \requ{} is not}

\begin{note}
  \requ{} isn't saying that if try and fail then no conclusion.
  \requ{} is stronger, if no guarantee at present, then no conclusion.
\end{note}

\begin{note}
  Similarly, at issue is not a guarantee.
  \pevent{1} are understood in terms of the progressive.
  If \(\pv{\psi}{v'}\) from \(\Psi\) fails then there's no reasonable sense in which the event develops, even if things had gone a little differently.
\end{note}

\section{Additional \illu{1}}
\label{cha:requs:sec:add-illu}

\subsubsection{Sudoku}
\label{sec:sudoku}

\begin{note}
  \begin{illustration}[Sudoku]
    \label{illu:gist:sudoku}
    % https://tex.stackexchange.com/questions/91422/tikz-sudoku-circle-and-connect-with-lines-some-cells
    Consider the following Sudoku puzzle:%
    \footnote{
      From~\textcite[84]{Coussement:2007up}.
    }
    \vspace{\baselineskip}

    \mbox{ }\hfill%
    \begin{adjustbox}{minipage=0.45\linewidth,scale=1}
      \centering
      \begin{tikzpicture}[scale=.5]
        \begin{scope}
          \draw (0, 0) grid (9, 9);
          \draw[very thick, scale=3] (0, 0) grid (3, 3);
          \setcounter{row}{1}
          % Single entries
          \setrow { }{ }{ }  { }{ }{ }  {1}{ }{ }
          \setrow { }{ }{ }  { }{ }{ }  { }{5}{ }
          \setrow {9}{ }{ }  { }{ }{ }  { }{ }{2}
          \setrow { }{ }{3}  { }{2}{ }  { }{ }{ }
          \setrow { }{ }{ }  {8}{ }{ }  {4}{6}{5}
          \setrow { }{4}{ }  { }{5}{9}  { }{ }{8}
          \setrow { }{8}{7}  {2}{3}{1}  { }{4}{6}
          \setrow {2}{1}{ }  {5}{ }{ }  { }{ }{3}
          \setrow {3}{ }{6}  {4}{ }{8}  { }{ }{ }
        \end{scope}
      \end{tikzpicture}
    \end{adjustbox}%
    \hfill\mbox{ }

  \end{illustration}

  {
    \color{red}
    Interactive.
    Fill in the grid.
    Difference between filling in the grid and concluding that solution to the puzzle.
    So, before conclude.
    Is it the case that you would fill in the grid the same way?
  }

  In contrast to~\autoref{illu:fc:chess:I}, \autoref{illu:gist:sudoku} involves a number of salient \fc{}.

  The relevant \fc{1} follow from a basic understanding the rules of Sudoku, in same way that the possibility for White to checkmate follows from understanding of the rules of chess in \autoref{illu:fc:chess:I}.

  Still, in contrast to \autoref{illu:fc:chess:I}, there is a interesting chance of error.
  For example, accidentally placing 9 above the 8 in the bottom centre sub-grid before observing the 9 in the same column.
  Or, 4 in the top-right of the top-left sub-grid before realising already placed 4 in top-left of the top-left sub-grid.

  However, errors do not (necessarily) amount to conclusions.
  One may make an error while completing the Sudoku puzzle, but refrain from concluding that the mistaken number is the correct number to place in the cell.
  Indeed, given the possibility of error one may only conclude \(i\) is the correct number to place in cell \(c\) only when all cells have been filled and they have ensured there are no errors.
\end{note}

\subsubsection{Kettle logic}
\label{sec:failures-1}

\begin{note}
  Typically, \(\pv{\phi}{v}\) from \(\Phi\) is not a \requ{}.
  You don't need to know that you are concluding in order to be concluding.
\end{note}

\begin{note}[A copper kettle]
  A further \illu{0} builds on a story as told by~\citeauthor{Freud:1960wx}.
  \begin{illustration}[A copper kettle]
    \label{illu:kettle}
    \mbox{ }
    \vspace{-\baselineskip}
    \begin{quote}
      `A.\ borrowed a copper kettle from B.\ and after he had returned it was sued by B.\ because the kettle now had a big hole in it which made it unusable.
      His defence was:
      ``First, I never borrowed a kettle from B.\ at all;
      secondly, the kettle had a hole in it already when I got it from him;
      and thirdly, I gave him back the kettle undamaged.%
      '''\newline
      \mbox{ }\hfill\mbox{(\citeyear[62]{Freud:1960wx})}
    \end{quote}
    An agent listens to A.'s defence, but does not conclude A.\ has provided testimony.
  \end{illustration}

  The agent's failure to conclude A.\ has provided testimony may be understood in terms of a \requ{}.
  For, A.\ has provided testimony only if what A.\ has said is true.
  And, what A.\ has said is true only if the three points of A.'s defence are jointly consistent.
  Putting these observations together, we have the following conditional:

  \begin{itemize}
  \item
    A.\ has provided testimony \emph{only if} if the three points of A.'s defence are jointly consistent.
  \end{itemize}

  Failure for the agent to conclude the consequent would prevent the agent from concluding the antecedent.

  Likewise, there are only three points, and checking for consistency does not require the agent to establish whether the points are (actually) true.

  \requ{} but not \fc{}.

  Before the agent concludes A.\ has provided testimony, the agent reasons about whether the three points of A.'s defence are jointly consistent.
  After, the agent does not conclude A.\ has provided testimony.%
  \footnote{
    Any pair of points are jointly inconsistent.
    For example, consider the first and third:
    If A.\ never borrowed the kettle from B, then it is not possible for A.\ to have returned the kettle to B.
  }

  The story as told by \citeauthor{Freud:1960wx} is comical, but the \requ{0} identified is fairly general.
  In many cases one may only accept a story if the details are in harmony, and dissonance leads to rejection.
\end{note}

\subsection{Not a \requ{}}
\label{sec:not-requ}

\begin{note}
  Testimony, but too much to check.

  \begin{illustration}[Testimony as a layperson]
    \label{illu:testimony-layperson}
    An agent is informed that there are exactly five intermediate logics that have the interpolation property.\nolinebreak
    \footnote{Cf.\ \textcite{Maksimova:1977un}}

    The agent does not have the means to query the proof.

    The agent concludes there are exactly five intermediate logics that have the interpolation property.
  \end{illustration}

  Here, also, logic with syntax and semantics.
\end{note}

\begin{note}[Problems of induction]
  Hence, the sketch does not apply to black ravens.
  I wouldn't conclude all ravens are black if I saw a white raven.

  I may worry about shortly seeing a white raven when concluding all ravens are black, and I may refuse to entertain the possibility that the sun will rise tomorrow when planning to mow the grass.

  However, it's not possible to reason to seeing a white raven, nor is it possible to reason to the sun not rising tomorrow.

  Abstractly, at issue in~\autoref{illu:lost-key} is the possibility of failing to a reason to some proposition-value pair given \emph{present} information, rather than the possibility of failing to a reason to some proposition-value pair given \emph{new} information.

  To the extent that problems of induction arise from receiving new information, what is at issue is distinct.%
  \footnote{
    See~\textcite{Henderson:2020wb} for more on the problem of induction.
  }

  Similar points for external world scepticism.
  Would not conclude that I have hand if disembodied brain in a vat.

  However, conclusion is out of reach.
\end{note}

\section{Summary}
\label{sec:summary-1}

\begin{note}
  Introduced \requ{1}.
\end{note}

%%% Local Variables:
%%% mode: latex
%%% TeX-master: "master"
%%% End:
