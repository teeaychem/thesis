\chapter{Two ways of concluding \(\phi\) has value \(v\)}
\label{cha:reasoning-two-ways}

\section{Introduction}
\label{sec:reasoning-two-ways:intro}

\subsection{Definitions}

\begin{note}
  Broad distinction, agent may claim support by appeal to some thing, but it is also possible to break that thing down in to parts or elements such that the agent may claim support by appeal to those parts or elements (and how they compose).

  `Break down' is metaphorical.

  In some cases, the thing itself, in other cases, more basic stuff that must be the case in order for the thing to be the case.

  Break down does the work.
  Agent will typically recognise.
  Break down is not required.

  In this sense, break down is more fundamental.

  `Because\dots'

  Unifying feature is that \adA{} allows claim support for \adB{}, so not clear that need to go via \adB{}.
  Indeed, unclear, given \ESU{}, that may claim support by \adB{}.
  We will only push this question with respect to ability, though.
\end{note}

\subsection{Additional illustrations}

\begin{note}

  \begin{illustration}
    \mbox{}
    \vspace{-\baselineskip}
    \begin{itemize}
    \item If bag are overweight then they can't be taken on the flight.
    \item Machine reads\dots
    \item Bag can't be taken on the flight.
    \end{itemize}
  \end{illustration}
  Contents of the bag are overweight.

  Combined weight of the items versus the combination of the individual weights.

  Compare, filling the bag and weighing it, versus summing the weight of the items as you fill the bag.

  Now, seems possible to fill the bag and weight it, then appeal to the sum of the items.

  So, this is a little more subtle.
  The bag has been weighed, and the distinction is between the weight of the contents of the bag, and the combined weight of the items that make up the contents of the bag.

  This is particularly interesting.
  Because, it seems clear that something is strange if someone talks about the weight of the contents of the bag without recognising that this is a function of the combined weight of all the individual elements of the bag.
  However, no idea what the contents of the bag are.

  So, claiming support from what is has been observed, the combined weight, rather than what must be the case in order to have made the observation.
\end{note}

\begin{note}[Fire alarm]
  \begin{illustration}
    \mbox{}
    \vspace{-\baselineskip}
    \begin{itemize}
    \item Fire alarm is ringing.
    \item Fire in the building.
    \item Should leave by the nearest exit.
    \end{itemize}
  \end{illustration}
  So, claiming for getting out of the building.
  Fire alarm.
  Or, fire, fire alarm has picked this up.

  So, difference between that there is a fire in the building, and \emph{the} fire in the building.

  The point here is that, okay, you need to go from alarm to fire, that's all fine, but fire itself is sufficient to claim support.
\end{note}

\begin{note}

  \begin{illustration}\label{ill:ad:factorial}
    \mbox{}
    \vspace{-\baselineskip}
    \begin{itemize}
    \item It is possible to write recursive functions in C.
    \item It is possible to write a recursive implementation of the factorial function in C.
    \end{itemize}
  \end{illustration}
  With proofs, abstract objects.

  Consider programming.

  Recursive implementation of factorial in C (chosen to make the implication clear).

  So, \adA{} is just the fact, so to speak.
  But, \adB{} points to the key step of calling function.
  Of course, this is just recursion, but appeal here is to the concept, so to speak, rather than the truth of the statement.

  Don't need to understand details.
  Go by form, so to speak.

  Claiming support by logical relation, rather than the states of affairs that ensure those logical relations hold up.

  Or, the definition is such that\dots
\end{note}

%%% Local Variables:
%%% mode: latex
%%% TeX-master: "master"
%%% End: