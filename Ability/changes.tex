\chapter*{Change log}
\label{cha:changes}


\begin{itemize}
\item
  The variants to \qWhy{} and \qHow{} are now termed `\qWhyV{}' and `\qHowV{}'.
\item
  The intuitive constraint on answers to \qWhy{} by answers to \qHow{} is referred to via `\issueInclusion{}', likewise `\issueConstraint{}' for the variants.
  \begin{itemize}
  \item
    The goal with this notation is to keep the questions present when stating the constraints in order for easier recall of what the constraints amount to.
  \end{itemize}
\item
  Slight revision to \qWhyV{}.
  \begin{itemize}
  \item
    It is now the case that \qWhyV{} asks whether there is some sub-event of the event in which the agent concludes such that, if a \ros{} failed to hold at the sub-event, the agent would not have concluded.

    The purpose of this is to ease the construction of counterexamples to \issueInclusion{}.

    However, I also think this condition is better suited, as it apples to cases of complex concluding, and avoids worries over \ros{1} which fall outside the event in which the agent concludes.
  \end{itemize}
\item
  Drop `from the \agpe{}' in favour of `for the agent'.
  \begin{itemize}
  \item
    `the \agpe{}' shifts evaluation to how things are for the agent.
    `for the agent' captures \agpe{our} on the agent.
    Strictly, I only need the latter, and shifting perspectives seems to add --- rather than reduce --- complexity.
  \end{itemize}
\item
  Understand concluding as a sub-instance of reasoning.
  \begin{itemize}
  \item
    This allow me to talk in terms of whether or not reasoning amounts to concluding.
  \end{itemize}
% \item
%   Introduce the idea of reasoning/concluding being \sR{}.
%   \begin{itemize}
%   \item
%     The role of \sR{} is to help capture the idea that conclusions are general.
%     E.g.\ conclusion via arithmetic competence, understanding the rules of chess, knowing how to prove syntactic theorems, and so on.
%   \item
%     In turn, this motivates the existence of \requ{1}.
  % \end{itemize}
\end{itemize}

%%% Local Variables:
%%% mode: latex
%%% TeX-master: "master"
%%% End:
