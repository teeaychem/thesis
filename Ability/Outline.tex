\chapter{Introduction}
\label{cha:introduction}

\begin{note}
  \begin{itemize}
  \item
    Initial \scen{0}, intuitions, questions, relation between answers, motivation for positive answer.
  \end{itemize}
\end{note}

\section{Concluding: Why? and How?}
\label{sec:overview:issue}

\subsection{Initial \scen{0}}

\begin{note}
  \begin{scenario}[Multiplication]
    \label{illu:gist:calc}
    An agent enters `\(23 \times 15\)' into a calculator and presses the button marked `\(=\)'.
    The calculator displays `\(345\)'.

    \mbox{ }

    The agent observes they have the option to calculate \(23 \times 15\) via their understanding of arithmetic.
    And, \emph{if} the calculator is trustworthy, then they would not fail to conclude \(23 \times 15 = 345\) via their understanding of arithmetic.

    \mbox{ }

    The agent concludes \(23 \times 15 = 345\).
  \end{scenario}

  Intuitively, the agent concluded \(23 \times 15 = 345\) from the calculator.
  Personifying a little, we may say the agent has concluded \(23 \times 15 = 345\) from the testimony of the calculator.%
  \footnote{
    Indeed, for our purposes \autoref{illu:gist:calc} may be recast in terms of an agent asking another agent to solve `\(23 \times 15\)', but using a calculator is more natural.
  }

  Still, as noted by the agent, they had the option of concluding \(23 \times 15 = 345\) through their own understanding of arithmetic, and hence without the use of the calculator.

  Still, it is intuitive that an agent concludes some proposition has some value%
  \footnote{In \scen{0} the proposition is `\(23 \times 15 = 345\)' and the value is `true'.
    In isolation, `\(23 \times 15 = 345\)' describes some possible state of affairs, and assigning the value `true' indicates the possible state of affairs is the actual state of affairs.
    Still, the agent may have concluded `\(23 \times 15 = 345\)' is `desired', `impossible', `probable', and so on.
    When speaking generally, we will keep explicit which value a proposition is paired with, though when describing specific \scen{0} or examples, we will leave the associated value implicit.
  }
  from some pool of premises only if the agent reasoned from those premises to the proposition-value pair.

  In other words, the agent may have concluded \(23 \times 15 = 345\) via their understanding of arithmetic, but as the agent did not calculate \(23 \times 15\) themselves, they did not conclude \(23 \times 15 = 345\) from whatever premises would be involved when reasoning via their understanding of arithmetic.

  Indeed, given the agent's understanding of arithmetic, it seems clear that prior to using the calculator the agent knew \emph{whether} \(23 \times 15 = 345\).
  Though, knowing whether \(23 \times 15 = 345\) is not knowing \(23 \times 15 = 345\).
  For example, I expect --- though one of the following equalities does not hold --- you know whether \(345 \times 11 = 3,795\), whether \(3,795 \div 5 = 760\), and whether \(760 \div 8 = 95\).

  Rephrasing things a little, and keeping track of truth, we may say \(23 \times 15 = 345\) was a \fc{} for the agent in \autoref{illu:gist:calc}.
  And, for you, \(345 \times 11 = 3,795\), \(3,795 \div 5 \ne 760\), and \(760 \div 8 = 95\) are \fc{1}.

  Instead, it seems the agent concluded \(23 \times 15 = 345\) by use of the calculator, regardless of whether \(23 \times 15 = 345\) was a \fc{1}.
  And, likewise, if you have concluded \(3,795 \div 5 \ne 760\) from the paragraph above, it was due to my testimony, and not your understanding of arithmetic.
\end{note}

\begin{note}[Summary of basic intuitions]
  So, intuitively, in \autoref{illu:gist:calc} the agent concluded \(23 \times 15 = 345\) from the testimony of the calculator.
  And, intuitively, \(23 \times 15 = 345\) being a \fc{0} had no significant role in the agent concluding \(23 \times 15 = 345\) from the testimony of the calculator.
\end{note}

\subsection{Two questions: Why? and How?}

\begin{note}[Not just concluding]
  So far we have spoken about intuitions of the basic form:

  \begin{itemize}
  \item
    Agent \(A\) concluded some proposition \(\phi\) has some value \(v\) from some pool of premises \(\Phi\).
  \end{itemize}

  I take intuitions of this basic form to be readily available in a variety of \scen{1}, and I also take those intuitions expressed in regards to \autoref{illu:gist:calc} to be fairly clear.

  \phantlabel{how-and-why-first-mention}
  Still, concluding is something an agent does, and in this respect there are (at least) two distinct questions intuitions of the above form may answer:%

  \begin{restatable}[\qWhy{}]{question}{questionWhyBasic}
    \label{q:why}
    Which proposition-value-premises pairings an involved in explaining \emph{why} the agent concluded \(\phi\) has value \(v\)?
  \end{restatable}

  \begin{restatable}[\qHow{}]{question}{questionHowBasic}
    \label{q:how}
    Which proposition-value-premises pairings an involved in explaining \emph{how} the agent concluded \(\phi\) has value \(v\), from the agent's perspective, from the agent's perspective?
  \end{restatable}

  In basic form, focus is on what the agent did, or alternatively whether some action way performed.
  Did the agent conclude \(\phi\) has value \(v\)?
  Or, did the agent conclude \(\phi\) has value \(v\) from the pool of premises \(\Phi\)?
  Or, perhaps, did the agent fail to conclude \(\phi\) has value \(v\)?

  `How?' and `why?' by contrast, take for granted the agent concluded \(\phi\) has value \(v\) and consider, respectively, how and why this action was performed.

  How do intuitions regarding whether or not an agent performed an action relation to how and why the agent performed the action?
\end{note}

\begin{note}
  \color{red}
  Designed to allow intermediate conclusions.
\end{note}

\begin{note}
  \qWhy{} and \qHow{} are distinct questions.
  Particular instances of broader `why?' and `how?'.
  And, generally speaking, `why?' and `how?' may have distinct answers.%
  \footnote{
    From the agent's perspective.
    Avoid explanations which are distinct from the agent's point of view.

    For example, it may be that what `really' explains why the agent arrived at the post-office was to take their dog for a walk.
    {
      \color{red}
      Here, there's an example from Russell\dots
    }
    Likewise, may have actually been a slight jog.
  }

  For example, consider asking why and how some agent arrived at some location.
  `By walking answers how the agent arrived at the post office, but does not answer why the agent arrived at the post office.
  Likewise, `to post a letter' intuitively answers why the agent arrived at the post office, but does not answer how the agent arrived at the post office.%
  \footnote{
    Of course, things get complex.
    Action, motivating reason, belief-desire pair.
    Hence, desire to post a letter is part of how.

    Preface with `intuitively'.
    The point is that may refine intuition regarding the agent concluding \(23 \times 15 = 345\).
  }

  So, the intuitions expressed with respect to \autoref{illu:gist:calc} may answer `how?' but not `why?', or `why?' but not `how?'.

  Still, I suspect this is not the case.
  In the case of \autoref{illu:gist:calc} both have the same rough answer:

  \begin{itemize}
  \item
    The pairing of testimony of the calculator with \(23 \times 15 = 345\), is, in part, \emph{both} how \emph{and} why the agent concluded \(23 \times 15 = 345\).
  \end{itemize}
  That premises associated with the agent's understanding of arithmetic do not answer either `how?' or `why?' is implicit by omission.

  Again, the agent used the testimony of the calculator to conclude \(23 \times 15 = 345\), and the agent appealed to the testimony of the calculator to conclude \(23 \times 15 = 345\).
  The agent did not use their understanding of arithmetic to conclude \(23 \times 15 = 345\), and the agent did not appeal to their understanding of arithmetic to conclude \(23 \times 15 = 345\).
\end{note}

\subsection{Relationship between \qWhy{} and \qHow{}}

\begin{note}
  Our observation that the testimony of the calculator seems to answer both `why?' and `how?' the agent concluded \(23 \times 15 = 345\) suggests, even if --- as a single \scen{0} --- only slightly, the following basic idea:%
  \footnote{
    The observation also suggests the converse holds --- an answer, in part, to `how?' is also, in part, an answer to `why?' --- though I think the converse faces some immediate difficulties.
    For, it seems answers to `how?' may include details that are irrelevant to `why?'.
    For example, typing digits and operators into the calculator answers, in part, how the agent concluded \(23 \times 15 = 345\) but these actions seems irrelevant to why the agent concluded \(23 \times 15 = 345\).
    Rather, an answer to `why?' seems to be limited to the calculator providing testimony that \(23 \times 15 = 345\), regardless of whether it was the agent who used the calculator, or whether the agent observed someone else using the calculator.
  }

  \phantlabel{how-and-why-relation-first-mention}
  \begin{restatable}[\issueInclusion{}]{issue}{issueInclusionFirst}
    \label{issue:why-inc-in-how}
    Some proposition-value-premises pairing is, in part, an answer to \qWhy{} only if that proposition-value-premises pairing is (also), in part, an answer to \qHow{}.
  \end{restatable}

  In other words, in order for premise to, in part, answer \emph{why} an agent concluded \(\phi\) has value \(v\), that premise must also, in part, answer \emph{how} the agent concluded \(\phi\) has value \(v\).

  With respect to \autoref{illu:gist:calc}, the testimony of the calculator satisfies the constraint imposed by the idea, while the agent's understanding of arithmetic does not.
  Specifically, the testimony of the calculator was part of how the agent concluded \(23 \times 15 = 345\), and so the testimony of the calculator may be, in part, an answer to why the agent concluded \(23 \times 15 = 345\).
  However, the agent's understanding of arithmetic was \emph{not} part of how the agent concluded \(23 \times 15 = 345\), and so the agent's understanding of arithmetic \emph{may not}, in part, an answer to why the agent concluded \(23 \times 15 = 345\).

  In addition, the basic idea may be taken to capture some explanatory significance and we may even say:
  The agent's understanding of arithmetic is not, in part, an answer to why the agent concluded \(23 \times 15 = 345\) \emph{because} the agent's understanding of arithmetic was \emph{not} part of how the agent concluded \(23 \times 15 = 345\).
\end{note}

\subsection{Motivation from \citeauthor{Davidson:1963aa}}

\begin{note}
  The basic idea, rather than intuitions regarding specific \scen{1} is our interest.
  Roughly, at least.%
  \footnote{
    We will shortly refine our understanding of `why?' and `how?' to focus on support between premises and conclusions, and will motivate a slightly weaker idea with respect to support.
  }

  Additional \scen{1} may provide additional motivation for the basic idea.
  Though, I think the basic idea is sufficiently intuitive independently of individual \scen{1}.
  Instead, observe the basic idea may be motivated not only by \scen{1}, but also by theories.
  Perhaps the most prominent is \citeauthor{Davidson:1963aa}' causal theory of action.

  \citeauthor{Davidson:1963aa} opens \textcite{Davidson:1963aa} with the following question:

  \begin{quote}
    What is the relation between a reason and an action when the reason explains the action by giving the agent's reason for doing what he did?
    We may call such explanations \emph{rationalizations}, and say that the reason \emph{rationalizes} the action.%
    \mbox{}\hfill\mbox{(\citeyear[685]{Davidson:1963aa})}
  \end{quote}

  As noted, concluding is an action, and hence \qWhy{} is a particular instance of \citeauthor{Davidson:1963aa}' question.
  And, answers to \qWhy{} will be reasons that rationalise the agent concluding \(\phi\) has value \(v\).

  \citeauthor{Davidson:1963aa} argues, in short, for the following answer to the relation between a reason and the rationalisation of an action:

  \begin{quote}
    \begin{enumerate}[label=\arabic*]
      [R]ationalization is a species of ordinary causal explanation.\newline
      \mbox{ }\hfill\mbox{(\citeyear[685]{Davidson:1963aa})}
    \end{enumerate}
  \end{quote}

  Following \citeauthor{Davidson:1963aa}, an answer to \qWhy{} is a rationalisation, rationalisation is an instance of ordinary causal explanation.
  So, the answer to why an agent concluded \(\phi\) has value \(v\) will, in part, by a cause of the agent concluding \(\phi\) has value \(v\).
  Therefore, an answer to why an agent concluded \(\phi\) has value \(v\) is, in part, an answer to how the agent concluded \(\phi\) has value \(v\).

  Implicit in this quick argument is the idea that a causal explanation answers `how?'.
  Note, however, that we did not appeal to the converse.
  Causal theories of action seem to motivate the basic idea, though I do not think the basic idea (directly, at least) motivates causal theories of action (or, specifically, concluding).
  In other words, for our purposes, answers to `how?' need not be causal explanations, though they may be.

  More broadly, I take the basic idea to capture a pre-theoretical constraint on classes of theories.
  There are theories that agree with the basic idea, such as \citeauthor{Davidson:1963aa}' causal theory of action (when the action is concluding) and there \emph{may be} theories which do not agree with the basic idea --- though I do not know of any specific theories that are explicitly of this kind.
\end{note}

\subsection{Questioning the intuitive relationship}

\begin{note}
  The basic idea is more-or-less the basic issue of this document.

  Both intuitions, such as those regarding \autoref{illu:gist:calc}, and theories, such as \citeauthor{Davidson:1963aa} causal theory of action, provide motivation for the basic issue.

  Our goal is to motivate the following, basic, contrary idea:

  \begin{itemize}
  \item
    There are cases in which something is, in part, an answer to `why?' and that something is \emph{not} (also), in part, an answer to `how?'.
  \end{itemize}

  The basic contrary idea is the negation of the basic idea.
  For, the basic idea states, roughly, answers to `why?' are always included in answers to `how?' while the basic contrary idea states that there are cases in which something that answers `why?' does not also answer `how?'.

  The basic contrary idea, then, has the form of an existential.
  We will not motivate the idea that there is always something which answers `why?' but does not also answer `how?'.
  And, in particular, it may be the case that the intuitions observed with respect to \autoref{illu:gist:calc} are correct.
\end{note}

\begin{note}
  Now, all this has been said without giving attention to the conditional observed by the agent in \autoref{illu:gist:calc}:

  \begin{itemize}
  \item
    If the calculator is trustworthy, then the agent would not fail to conclude \(23 \times 15 = 345\) via their understanding of arithmetic.
  \end{itemize}

  Both natural, and somewhat surprising.

  Consider the contraposition.

  \begin{itemize}
  \item
    If the agent were to fail to conclude \(23 \times 15 = 345\) via their understanding of arithmetic, then the calculator is not trustworthy.
  \end{itemize}

  \begin{itemize}
  \item
    Is it possible, from the agent's perspective, fail to conclude \(23 \times 15 = 345\) via their understanding of arithmetic?
  \end{itemize}

  If possible, then difficulty.
  For, testimony, so must be, but at the same time, possible that it is not.
  If not possible, then it doesn't seem that observing \(23 \times 15 = 345\) via the testimony of the calculator is sufficient.
  For, by the previous observation, difficulty with the testimony of the calculator.

  \begin{itemize}
  \item
    Testimony of the calculator \emph{only if} \(23 \times 15 = 345\) is a \fc{0} given the agent's understanding of arithmetic.
  \end{itemize}
\end{note}

\begin{note}
  The only if, interesting.

  Calculator provides information about what is a \fc{}.
  Agent's understanding of arithmetic is why it is a \fc{}.

  So, if being a \fc{0} is involved in concluding, then it may be that understanding of arithmetic is, in part, an answer to `why?'.
  And, as the agent has not concluded, not, in part, an answer to `how?'.
\end{note}

\begin{note}
  Important:

  Multiple ways to conclude.
  So, have a check.

  Differs to, for example, concluding {\color{red} ???} from a scientific calculator.
  {\color{red} ???} goes beyond typical understanding of arithmetic.
  Parallel pair of conditionals does not hold.

  Or, alternatively, testimony that {\color{red} ???}.
  Beyond understanding.
\end{note}

\begin{note}
  I am not sure what to make of \ref{illu:gist:calc}.
  Understanding of arithmetic is a partial check.
  However, testimony.

  Unsure because status of a premise.

  Basic contrary idea only requires some instances.

  Argument against this intuition.
  Type of cases, premises are fixed.
  Check on own reasoning.

  First, expand on intuition.
  Then, introduce type of \scen{0} of interest.
\end{note}

\subsection{Key things going forward}

\begin{note}
  Three things of interest.

  \begin{enumerate}
  \item
    \scen{1} like \autoref{illu:gist:calc} in which an agent concludes some proposition has some value.
  \item
    Relation between Why and how and agent concludes.
    Basic idea, and basic contrary idea.
  \item
    Idea of a \fc{} and how \fc{1} may be involved in concluding.
  \end{enumerate}


  Chapter will be split into three parts
  In \autoref{cha:clarification}, we will clarify what is of interest.
  In particular, \autoref{cha:clarification} will be split into two subsections.

  In~\autoref{sec:clarification:support} we will clarify why we are interested in intuitions regarding \scen{1} such as \autoref{illu:gist:calc}.


  First, clarify what is of interest.
  Two things in particular.
  1. What it is about concluding.
  2. Scenarios of interest.
  \autoref{illu:gist:calc} is similar to the type of \scen{0} that will be the focus on this document.
  However, the argument we provide will not directly apply to \scen{1} like \autoref{illu:gist:calc}.

  In \autoref{sec:clar:type-of-scen} we present the type of \scen{0} we will present an instance of the type of \scen{0} interested in, provide a general description of the \scen{0} type.
  Further, we will provide a detailed contrast between the type of \scen{0} we are interested in and \autoref{illu:gist:calc}.%
  \footnote{
    Roughly, if it were the agent failed to conclude \(23 \times 15 = 345\) in \autoref{illu:gist:calc}, then there would be conflict between the agent's understanding of arithmetic and the testimony of the calculator.
    Expressed differently, there would be conflict between the agent's failure to conclude \(23 \times 15 = 345\) by their understanding of arithmetic, and a premises involved in concluding \(23 \times 15 = 345\) via the calculator.
    I.e. supposing the agent concludes \(23 \times 15 \ne 345\), then from the agent's perspective the calculator is not a source of testimony.
    In the \scen{1} of interest, this hypothetical --- or in some cases possible --- conflict will strictly be between the agent's reasoning from pools of premises to conclusions.
  }

  Generally speaking, I am unsure about the intuitions expressed with respect to~\autoref{illu:gist:calc}.
  \Autoref{illu:gist:calc} shares an important feature with the type of \scen{0} that we will explore in detail.
  Yet, \dots

  I am inclined to think intuitions are fine.
  variation of scenario, what the difference is, and why focus.
  Second, sketch general argument of the paper, and role of \fc{1}.
\end{note}



%%% Local Variables:
%%% mode: latex
%%% TeX-master: "master"
%%% End: