\chapter{Overview}
\label{cha:overview}

\section{The issue}
\label{sec:issue}

\begin{note}[Main issue, positive resolution, quick argument for negative]
  Our interest is with the following issue:

  \begin{restatable}[Main]{issue}{issueMain}
    \label{issue:Main}
    Are the cases in which an agent may conclude \(\phi\) has value \(v\) from certain premises without witnessing reasoning that concludes \(\phi\) has value \(v\) from those premises?
  \end{restatable}

  The goal is to motivate a positive resolution:

  \begin{restatable}[Main: tick]{issue}{issueMainPositive}
    \label{issue:Main:R:p}
    \emph{There are cases in which} agent may conclude \(\phi\) has value \(v\) from certain premises without witnessing reasoning that concludes \(\phi\) has value \(v\) from those premises.
  \end{restatable}

  This positive resolution is, by my estimation, not straightforward.
\end{note}

\begin{note}[The quick argument]
  At least, the positive resolution is not straightforward without placing some constraints on \emph{which} premises an agent may appeal to when reasoning, and we will motivate the positive resolution without such constraints.

  For, without constraints on which premises an agent may appeal to when reasoning, one may argue as follows:
  \begin{enumerate}
  \item Any instance of reasoning is some process with start and end points and intermediary steps.
  \item If an agent has concluded \(\phi\) has value \(v\) by some reasoning, then the reasoning has start points and intermediary steps.
  \item Hence, the agent has concluded \(\phi\) has value \(v\) by witnessing some reasoning from some start points via some intermediary steps.
  \item In other words, the agent has concluded \(\phi\) has value \(v\) by witnessing some reasoning from some premises.
  \end{enumerate}

  In short, so long as an agent has concluded \(\phi\) has value \(v\), the agent has always witnessed reasoning from some premises.

  \begin{enumerate}[resume]
  \item So, either the start points are the premises of interest mentioned in the issue, or the agent has concluded \(\phi\) has value \(v\) from a distinct set of premises.
  \end{enumerate}

  In other words, either the agent has witnessed reasoning from the premises of interest, or the premises of interest (and any reasoning from them) are not required to conclude \(\phi\) has value \(v\).
\end{note}

\begin{note}[More on the quick argument]
  The quick argument does not directly lead to a negative resolution to the issue.
  Still, the quick argument does suggest that any appeal to premises \emph{without} witnessing reasoning from those premises is redundant.

  Now, perhaps redundancy isn't so bad.
  I only need a single key to ensure I have the option of unlocking a door, but a second key is useful if the first is lost.

  Still, I take it to be the case that redundancy provides leverage for a wide range of arguments motivating a negative resolution in the case of reasoning.

  For, if appeal to some premises is redundant, then any argument that requires witnessing need only observe that a counterargument must find some role for something which is not needed.

  Reasoning is an event, and distinct way of concluding \(\phi\) has value \(v\) may be useful, it is unclear why the distinct way of concluding \(\phi\) has value \(v\) is of use when concluding \(\phi\) has value \(v\) from present premises.
  To push the analogy, a second key may have various uses, but the second key is irrelevant in the event of unlocking the door with the first key.
  That the second key is would unlock the door if the first was lost has no role in the event of unlocking the door with the first key.

  From a different perspective, if appeal to certain premises without witnessing reasoning from those premises is redundant, then it seems any positive role given to appeal to those premises may be redistributed to the premises of the reasoning the agent did witness.

  More concretely, even if I were to show that there was some benefit for concluding \(\phi\) has value \(v\) via unwitnessed reasoning with respect to some particular account of reasoning, it seems at least plausible that the account of reasoning may be reformulated to derive the same benefit from the premises of the reasoning the agent witnessed.

  More generally, it may seem (and I suspect it does seem) intuitive that the issue should be resolved negatively.
  Reasoning just is obtaining a conclusion by witnessing reasoning from premises.
  And, if the quick argument succeeds, then there surely is some way to preserve the intuition.
\end{note}

\begin{note}
  So, part of the task is to show that the quick argument fails.
  Concluding \(\phi\) has value \(v\) from certain premises without witnessing reasoning that concludes \(\phi\) has value \(v\) from those premises has some role.
  Or, rather, that the quick argument is not without cost.
  Perhaps the issue really should be resolved in the negative, but this will require giving up some at least equally (I think) intuitive ideas.

  The result will be motivation for a positive resolution to the issue.
  However, the motivation will be somewhat narrow.
  To escape the tension, the positive resolution need only hold for a restricted pattern of reasoning.
  Still, with the existential motivated, I hope future work may expand the positive resolution to other patterns of reasoning.
  And, while such expansions may still need to argue that concluding \(\phi\) has value \(v\) from unwitnessed reasoning is makes sense with respect to the specific topic at hand, observing that the broad idea of concluding \(\phi\) has value \(v\) from unwitnessed reasoning may not be dismissed without cost may be an option.
\end{note}

\section{Ideas}
\label{sec:ideas-1}

\subsection{Concluding}
\label{sec:outline:concluding}

\begin{note}
  Given the positive resolution, concluding.
  Two place relation.
  Premises and conclusion.

  Result of some event.
  No special significance is attached to the term.

  \begin{assumption}[Concluding is the culmination of reasoning]
    An agent concludes \(\phi\) has value \(v\) just when \(\phi\) having value \(v\) is the culmination of reasoning from some premises.
  \end{assumption}

  Just the result of finishing a process.
  No special significance is attached to the term.

  In this respect, positive resolution, reasoning, but premises of reasoning are not the premises from which agent concludes \(\phi\) has value \(v\).
\end{note}

\begin{note}[`Concluding']
  Rather, \(\phi\) \emph{has} value \(v\).

  So, from the perspective of an agent, `I conclude \(\phi\) has value \(v\)' and \(\phi\) has value \(v\) are interchangeable.
\end{note}

\begin{note}[Concluding \(\phi\) has value \(v\)]
  Reasoning, concluding \(\phi\) has value \(v\).

  Clock, late for meeting.
  Poster, I want to see that film.
  Die, It might roll a 6.

  Not, I think they have no respect for other people's time.
  I hope I won't be disappointed when I watch the film.
  I am uncertain whether the die will land 6.
\end{note}

\begin{note}[Descriptions]
  An important assumption regarding `concluding' is that an agent need not recognise that they have concluded \(\phi\) has value \(v\).\nolinebreak
  \footnote{
    Contrast, for example, \citeauthor{Anscombe:1957aa} on intention action (\citeyear[\S19]{Anscombe:1957aa}) and \citeauthor{Davidson:1963aa} on primary reasons (\citeyear[5]{Davidson:1963aa}).
  }

  Specifically:
  \begin{assumption}[Concluding is `description-free']
    \label{assu:d-free}
    It is not the case that an agent concludes \(\phi\) has value \(v\) only if the agent concludes \(\phi\) has value \(v\) under some description \emph{d}, where \emph{d} includes ``\(\phi\) has value \(v\)''.
  \end{assumption}

  X, it is true that X has value v, it is not the case that X does not have value v, it is true that it is not the case that X does not have value v, and so on.
  The descriptions are infinite, but do not amount to distinction conclusions.

  Likewise, \(2 + 2 = 4\), also \(4 = 2 + 2\).
  X is taller than Y, Y is shorter than X.

  Now, could say that in concluding \(\phi\) has value \(v\) an agent concludes that \(\psi\) has value \(v'\) only if \(\phi\) having value \(v\) and \(\psi\) having value \(v'\) are equivalent from the perspective of the agent.

  As observed, there is an intuitive sense in which an agent concludes \(\phi\) has value \(v\) only if the agent recognises that \(\phi\) has value \(v\).
  And, a restriction to equivalent proposition-value pairs provides is also intuitive.
  However, we leave~\autoref{assu:d-free} as is.
  While some specifics of argument to follow will rely on~\autoref{assu:d-free} denying the requirement of a description in general, the spirit is independent.\nolinebreak
  \footnote{
    In particular, if~\autoref{assu:d-free} were rewritten to hold that `it is the case' rather than `it is not the case', our attention would turn to instances in which an agent recognises that \(\phi\) has value \(v\) only if they have the option of concluding \(\psi\) has value \(v'\).
  }

  Still, it is perhaps important to note that~\autoref{assu:d-free} is motivated only by proposition-value pairs which are equivalent from the perspective of the agent.
  In particular, we not assume a general `closure' condition%
  \footnote{
    Consider by parallel closure of knowledge under known entailment:
    \begin{itemize}
    \item If an agent knows that \(\phi\) has value \(v\) only when \(\psi\) has value \(v'\), then if the agent knows \(\phi\) has value \(v\), then the agent knows \(\psi\) has value \(v'\).
    \end{itemize}
    This closure condition differs in forms, as it concerns knowledge as a state, by may be reformulated to a closer parallel:
        \begin{itemize}
    \item If an agent knows that \(\phi\) has value \(v\) only when \(\psi\) has value \(v'\), then in coming to know \(\phi\) has value \(v\) the agent comes to know \(\psi\) has value \(v'\).
    \end{itemize}
  }
  of the form:
  \begin{itemize}
  \item If an agent has previously concluded that \(\phi\) has value \(v\) only when \(\psi\) has value \(v'\), then the agent concluding \(\phi\) has value \(v\) involves concluding that \(\psi\) has value \(v'\).
  \end{itemize}

  Though, as we will see, something like a closure principle will follow in certain cases of concluding.
\end{note}

\begin{note}[Tied to premises]
  What does matter is role of premises.
  From some conclude \(\phi\) has value \(v\), from others conclude \(\phi\) does not have value \(v\).
\end{note}

\begin{note}[Generally]
  Concluding \(\phi\) has value \(v\).
  \(\phi\) having value \(v\) follows from premises.
  These premises, and therefore \(\phi\) has value \(v\).

  The `follow' or `therefore' may be deductive or non-deductive.
\end{note}

\subsection{\csN{2}}
\label{sec:csn2}

\begin{note}[Introduction]
  Following~\autoref{sec:outline:concluding}, our broad interest is with reasoning by some agent that concludes \(\phi\) has value \(v\) (shorthand: reasoning that concludes \(\langle \phi,v \rangle\)).
  However, with respect to arguing for~\autoref{issue:Main:R:p}, we narrow our attention to when concluding \(\langle \phi,v \rangle\) has a particular property.
  We term the property `\csN{}'.

  The sole question for whether an agent \csV{} for \(\phi\) having vale \(v\) when concluding \(\langle \phi, v \rangle\) is:
  \begin{quote}
    Does \(\phi\) having value \(v\) ensure that there is some other proposition-value pair \(\langle \psi,v' \rangle\) such that (from the agent's perspective) the agent may have concluded that \(\phi\) \emph{does not} have value \(v\) if they were to have first reasoned about whether \(\psi\) has value \(v'\)?
    \begin{itemize}
    \item If there is no such \(\langle \psi,v' \rangle\), then the agent \csN{}.
    \item If there is some \(\langle \psi,v' \rangle\), then the agent fails to \csN{}.
    \end{itemize}
  \end{quote}

  In other words, an agent \csV{} for \(\phi\) having value \(v\) only if \(\phi\) having value \(v\) does not raise the possibility that the agent may have concluded otherwise.
\end{note}

\begin{note}[Simple examples of \csN{}]
  Simple examples of \csV{} for \(\phi\) having value \(v\) is concluding \(\phi\) has value \(v\) from the testimony of experts as a layperson.
  For, as a layperson one has no way of querying whether \(\phi\) has value \(v\).
  Hence, \(\phi\) having value \(v\) does not introduce the possibility of reasoning about some other proposition-value pair and concluding \(\phi\) does not have value \(v\).
  Fermat's last theorem is true, I am told, and I do not have the means to query the proof.

  Other examples involve unique sources of information.
  I conclude from the position of the hands on my watch that it is midday.
  The sky is cloudy, and without a second time piece I have no hope of reaching a different conclusion.

  And, more commonplace examples involve the gradual accumulation of proposition-value pairs.
  \nagent{16} \emph{said} they're coming to the party, but you know from \nagent{17} that \nagent{16} is coming to the party only if \nagent{18} is coming to the party.
  Without further information, reasoning about whether \nagent{18} is coming to the party might prevent you from taking \nagent{16} at the word.
  However, you have already have conformation from \nagent{18} that they are coming to the party.

  Likewise, suppose there are clear skies.
  Then, it is around midday only if the sun is (roughly) at the highest point of the sky.
  Though, I already concluded that the sun is (roughly) at the highest point of the sky prior to checking my watch for a more accurate read of the time.
\end{note}

\begin{note}[Simple failures]
  My preferred example for \emph{failure} to claim support is lost keys.
  Tempting as it may be to conclude that a pair of keys are lost after some searching, if the keys really are lost then there aren't in a handful of places you haven't yet thought to look.
  And, until you have concluded that the keys really aren't in those places, and that there is no-where else to look, the keys aren't really lost.

  Likewise, a friend's story may be entertaining, but one doesn't \csN{} that it actually happened without first checking that the details add up.
\end{note}

\begin{note}
  \begin{figure}[h]
    \centering
    \begin{tikzpicture}
      \node (origin) at (1,0) {};
      \node (psiSplit) at (3,0) {};
      \node (psiV) at  (5,-1) {\(\langle \psi,v' \rangle\)};
      \node (psiNv) at (5,-2) {\(\langle \psi,\overline{v'} \rangle\)};
      \node (psiQ) at (5,-3) {\(\langle \psi,? \rangle\)};
      \node (psiVPhiV) at (8,-1) {\(\langle \psi,v \rangle\)};
      \node (psiNvPhiU) at (8,-2) {\(\langle \psi,\{v,\overline{v},?\} \rangle\)};
      \node (psiQPhiU) at (8,-3) {\(\langle \psi,\{v,\overline{v},?\} \rangle\)};
      \node (phiSplit) at (5,0) {};
      \node (phiQ) at (9,1) {\(\langle \phi,? \rangle\)};
      \node (phiNv) at (9,2) {\(\langle \phi,\overline{v} \rangle\)};;
      \node (phiV) at (11,0) {\(\langle \phi,v \rangle\)};

      \draw[-]  (origin) -- (phiV);

      \draw[-]  (psiSplit) -- (psiV);
      \draw[-]  (psiSplit) -- (psiNv);
      \draw[-]  (psiSplit) -- (psiQ);

      \draw[-]  (phiSplit) -- (phiNv);
      \draw[-]  (phiSplit) -- (phiQ);

      \draw[-] (psiV) -- (psiVPhiV);
      \draw[-] (psiNv) -- (psiNvPhiU);
      \draw[-] (psiQ) -- (psiQPhiU);
    \end{tikzpicture}
    \caption{\csN{} \illu{}}
    \label{fig:csN:illu:overview}
  \end{figure}
\end{note}

\begin{note}

  \autoref{cha:claiming-support} will explore \csN{} in some detail.
  For the moment, we work with the following sketch which captures they key components:
  \begin{idea}[\csN{2}]
    \label{assu:CS:overview}
    An agent \csV{} for \(\phi\) having value \(v\) when concluding \(\langle \phi,v \rangle\) \emph{only if}:
    \begin{itemize}
    \item
      There is no proposition-value pair \(\langle \psi, v' \rangle\) such that the following jointly hold:
      \begin{enumerate}
      \item \(\phi\) has value \(v\) only if \(\psi\) has value \(v'\),
      \item \(\phi\) has value \(v\) only if the agent has the option of concluding \(\langle \psi, v' \rangle\) from some pool of premises.
      \item The agent has not concluded whether they would conclude \(\langle \psi, v' \rangle\) from the pool of premises \emph{without concluding \(\langle \phi, v \rangle\)}.
      \end{enumerate}
    \end{itemize}
  \end{idea}
  
\end{note}

\begin{note}[\csN{2}]
  In other words, an agent concluding \(\phi\) has value \(v\) is an instance of \csN{} just in case there is no proposition-value pair \(\langle \psi, v' \rangle\) such that if the agent may fail to conclude that \(\phi\) has value \(v\) if the agent were to reason about whether \(\psi\) has value \(v'\) \emph{independently} of whether \(\phi\) has value \(v\).
\end{note}

\begin{note}[\requ{3}]
  \begin{notion}[A \requ{2} of a conclusion]
    \(\langle \psi, v' \rangle\) such that \(\phi\) has value \(v\) only if the agent has the option of concluding \(\langle \psi, v' \rangle\) from some independent pool of premises \emph{without concluding \(\langle \phi, v \rangle\)}.
  \end{notion}

  \begin{notion}[Satisfied and unsatisfied \requ{1}]
    Two definitions.

    \begin{enumerate}
    \item
      A \requ{} is \emph{unsatisfied} if the agent not concluded whether they would conclude \(\langle \psi, v' \rangle\) from the pool of premises.
    \item
      A \requ{} is \emph{satisfied} if the agent has concluded they would conclude \(\langle \psi, v' \rangle\) from the pool of premises.
    \end{enumerate}
    \vspace{-\baselineskip}
  \end{notion}

  So,~\autoref{assu:CS:overview}, no unsatisfied \requ{1}.

  Keys.
  Lost keys, but until I've checked everywhere, might find them.
  So, conclude lost, then not in coat pocket.
  Well, not going to conclude lost at least until I've checked the pocket.
\end{note}

\begin{note}[Importance of \csN{}]
  \csN{} is key.
  Argument for positive resolution to~\autoref{issue:Main} largely rests on \csN{}.
  Our immediate goal is to motivate interest in \csN{}.

  However, briefly note that a few things.

  First, agent's reasoning.
  At issue is whether the agent may reason to a different conclusion.
  There's nothing that would lead me elsewhere.

  Second, agent's reasoning.
  Independent of whether \(\phi\) has value \(v\), \(\psi\) has value \(v'\), or any of the premises.
  Need not be the case that satisfaction amounts to anything substantial.
  No clause for justification, etc.

  Third, way.
  Getting \(\langle \phi, v \rangle\) from premises.
  Inclined to think that \csN{} is more general.
  From \(\langle \phi, v \rangle\).
  However, some concerns.
  See this after.

  Fourth, `would'.
  Competence, rather than performance.
\end{note}

\begin{note}[Building to \illu{1}]
  Consider a handful of cases.
\end{note}

\begin{note}[Trivial]
  No independent checks.
\end{note}

\begin{note}[Pruning]
  NTP server.
  Not from clock, but from NTP.
  Here, no way to check this.
\end{note}

\begin{note}[\illu{3}]
  Begin with a collection of \illu{1} with respect to a simple instance of reasoning: concluding it is midday from the positions of the hands on a clock.


  For example, with clocks.

  No option.
  One authoritative timepiece.

  Also, testimony that some mathematical theorem is true.

  No distinct premises.
  Reason from position of hands on clock to conclude that it is midday.
  All other clocks are kept in sync via an NTP time server, so no independent check.

  Also, two calculators.
  Same internals.

  Multiple checks.
  Two clocks.
  Wristwatch and clock on the library wall.

  Also, milk.
  Safe to drink from sell by date.
  Smell if it's off.

  Coming to the party.
  Well, then decided not to go to the beach.
  So, check this before.
\end{note}

\begin{note}
  Finally, ability.
  Ability enters indirectly.
  First, certain cases of reasoning.
  These, then, have some generality.
  Concluding \(\phi\) has value \(v\) involves general reasoning.
  Chess, sudoku, etc.
  It's no accident.
  Still, \csVed{}.

  Difficulty.
  Independent check.
  Would not conclude \(\phi\) has value \(v\) if were to fail to conclude \(\psi\) has value \(v'\).
  Some problem in the general reasoning.

  Now, applying previous observation.
  Either \csVed{} or witness.
  In these cases, no witnessing.
  So, \csVed{}.

  However, always a check on whether one has the general ability.
\end{note}

\begin{note}
  Reasoning, \support{}, would not reason to a different conclusion.

  Specifically, \requ{} of some conclusion.
  So long as conclusion, then it is possible to reason about whether \(\psi\) has value \(v'\), and unless conclude \(\psi\) has value \(v'\), would not conclude \(\phi\) has value \(v\).

  Intuitively, \requ{} as an independent check on the reasoning.
  If don't hold \(\psi\) from premises, then question about whether \(\phi\).

  Claiming support, necessary condition is satisfying all \requ{1}.
  Claiming support, then, is weaker than having support.
  Restricted to whether conclusion of reasoning would introduce a \requ{}.
  And, may be further restricted without impact to the tension we will develop to whether the conclusion would `clearly' introduce a \requ{}.
\end{note}

\begin{note}
  Now, consequence from \ideaCS{} and \ESU{},
  If potential witnessing event, then either concluded previously, or witness reasoning.
\end{note}

\begin{note}[Kettle logic]
  Well, it's true that the person must be acquitted, but at the same time, the person is going to have a hard time explaining how, for example, he brewed coffee for a week.

  Still, highlights what the neighbour needs to conclude.
  Did borrow.
  Was not lent damaged
  And, was returned damaged.

  Here, burden of argument.
  However, we're not interested in whether the neighbour would convince, but whether the neighbour would reach a different conclusion if they were first to reason about one of the three before concluding that the kettle was returned damaged.
\end{note}

\subsection{`Use'}
\label{sec:issue-refined}

\begin{note}
  Clarified concluding \(\phi\) has value \(v\).
  And, broad idea of \csN{0}.
  Briefly restate main issue.

  We're just looking for an existential.
\end{note}

\begin{note}
  \begin{restatable}[\ESU{0} --- \ESU{}]{target}{targetESU}
    \label{denied-claim}
    An agent concluding \(\phi\) has value \(v\) is an instance of claiming support \emph{only if}:

    For any pool of proposition-value pairs \(\langle \chi_{1},v_{1} \rangle,\dots,\langle \chi_{k},v_{k} \rangle\) appealed to as premises, the agent has witnessed reasoning which concludes \(\phi\) has value \(v\) from \(\chi_{i}\) having values \(v_{i}\).\nolinebreak
    \footnote{More generally, the agent has witnessed reasoning whose conclusion \emph{indicates} \(\phi\) has value \(v\).}
  \end{restatable}


  Some instance of reasoning from premises which indicates conclusion.
  Reasoning is compatible with the agent claiming support for the conclusion only if some agent has witnessed the reasoning.

  Necessary condition on claiming support.
  Key point is that, whether or not there are genuine cases of claiming support, \autoref{denied-claim} provides a clear condition for rejecting the possibility that some instance of reasoning is an instance of claiming support.
\end{note}

\begin{note}
  Important that \ESU{} is fairly weak.

  Has witnessed reasoning.

  Indeed, \ESU{} may be weakened further, \emph{some} agent.\nolinebreak
  \footnote{
    Simple example, know that some logic is decidable.
    Take a relatively simple fragment.
    Then, either proof or a countermodel.
    Same idea holds for tension, but there's no guarantee that anyone has concluded\dots
  }
  However, to keep things straightforward, focus on a single agent.
  Indeed, the tension arises from focus on the agent's own reasoning.
\end{note}

\begin{note}
  \begin{restatable}[\EAS{0} --- \EAS{}]{goal}{goalEAS}
    \label{prop:EAS}
    There are instances of reasoning in which an agent concludes \(\phi\) has value \(v\) by appeal to some pool of proposition-value pairs \(\langle \chi_{1},v_{1} \rangle,\dots,\langle \chi_{k},v_{k} \rangle\) as premises without witnessing reasoning from \(\langle \chi_{1},v_{1} \rangle,\dots,\langle \chi_{k},v_{k} \rangle\) to \(\langle \phi,v \rangle\).

    And, on some occasions, the reasoning is an instance of claiming support.
  \end{restatable}
\end{note}

\begin{note}
  The distinction here is whether some instance of reasoning needs to be witnessed in order for the reasoning to be compatible with the agent claiming support.

  \EAS{} contains two existentials.
  Some instances of reasoning, and some of these are instances of claiming support.
  Of course, interest is with the intersection.
  Instances of reasoning which are instances of claiming support.

  Key is ability.
  Specifically, ability to reason.
  This provides information to the agent.
  Given premises, then conclusion.
  Ability, in these cases, functions as an `interpolant'.

  Expand after outlining argument.
\end{note}

\begin{note}
  While \EAS{} is our `goal', the argument will be indirect.
  Motivate \EAS{} by observing that tension follows from combining \USE{} with two additional ideas.
\end{note}

\subsubsection{Two types of reasoning}
\label{sec:two-types-reasoning}

\begin{note}
  \begin{restatable}[\adA{}]{definition}{defADA}
    \label{AR:adA}
    \label{def:adA}
    \vAgent{} concludes \(\phi\) has value \(v\) by `\adA{}' if:
    \begin{enumerate}[label=\textsf{S:\arabic*}., ref=(\textsf{S}:\arabic*)]%, resume*=adA_counter]
    \item
      \label{def:adA:psi}
      \vAgent{} concludes \(\phi\) has value \(v\) by witnessing reasoning from some  pool of premises \(\chi_{1},\dots,\chi_{k}\) with values \(v_{1},\dots,v_{k}\).
    \end{enumerate}
    \vspace{-\baselineskip}
  \end{restatable}
\end{note}

\begin{note}
  \begin{restatable}[\adB{}]{definition}{defADB}
    \label{AR:adB}
    \label{def:adB}
    Suppose:
    \begin{enumerate}[label=\textsf{I:\arabic*}., ref=(\textsf{I}:\arabic*), series=adB_counter]
    \item
      \label{def:adB:poss}
      \(\mu\) having value \(v\) ensures:
      \begin{enumerate}
      \item
        There is some pool of proposition-value pairs \(\langle \chi_{1},v_{1} \rangle,\dots,\langle \chi_{k},v_{k} \rangle\) (where \(\mu\) is not equivalent to any \(\chi_{i}\)), such that:
      \item
        It is possible for \vAgent{} to conclude \(\phi\) has value \(v\) by witnessing reasoning from \(\langle \chi_{1},v_{1} \rangle,\dots,\langle \chi_{k},v_{k} \rangle\) to \(\langle \phi,v \rangle\).
      \end{enumerate}
    \end{enumerate}
    \vAgent{} concludes \(\phi\) has value \(v\) by `\adB{}' if:
    \begin{enumerate}[label=\textsf{I}:\arabic*., ref=(\textsf{I}:\arabic*), resume*=adB_counter]
    \item
      \label{def:adB:inter}
      \vAgent{} concludes \(\phi\) has value \(v\) by appeal to \(\chi_{1},\dots,\chi_{k}\) with respective values \(v_{1},\dots,v_{k}\) via the possibility of witnessing the relevant reasoning from \(\mu\) having value \(v\).
    \end{enumerate}
    \vspace{-\baselineskip}
  \end{restatable}

  Here, \(\mu\) is not a premise.
  The agent is concluding \(\phi\) has value \(v\) from \(\chi_{1},\dots,\chi_{k}\) having values \(v_{1},\dots,v_{k}\).
\end{note}


\subsection{Ability}
\label{sec:ability}

\begin{note}
  Finally, ability.

  General ability, specific ability.

  Claim support for having some general ability.

  Now, here, simple cases.
  Basic arithmetic.
  Sudoku puzzles.
  Chess problems with winning strategies.

  Roughly the same.
  More broadly:

  Logic problems.

  Crossword.

  Reading novels up to a certain level.
  Here, if you can't read, then the writing is bad.

  Fluency.

  So, specific instances of the general ability.
\end{note}

\begin{note}
  Well, conclude that you have the general ability, but also claim support.
  You don't need to go through specific instances.
  In these cases, fail to be an independent check.
  You not fail to reach the relevant conclusion.

  Would not reason to some incorrect summation.
  Would not fill out the Sudoku incorrectly.
  Would not fail to find a winning strategy.

  Of course, failures of performance, but not failures of competence.

  Intuitively, satisfied all \requ{1}.
\end{note}

\section{Tension}
\label{sec:tension}

\begin{note}
  Key argument, then, is that only satisfy a \requ{} by concluding \(\psi\) from \(\rho\).

  For, some other premise, get \(\psi\) from \(\rho\).
  Well, getting \(\psi\) from \(\rho\) is still a \requ{} for this.
\end{note}

\begin{note}
  Briefly stated.

  Specific instances, these introduce \requ{}.
  However, some reasoning.
  Can't jump to general to get rid of \requ{}, as this is forbidden.
  Further, if reason from some distinct set of premises, then still a \requ{}.

  If independent reasoning gets that specific instance of general ability, then doing the reasoning is still an independent check on this.

  So, the problem here is that need to ensure that would conclude \(\psi\) has value \(v'\) from certain premises.
  If appeal to any distinct premises, then failure to claim support.

  Hence, \ESU{} and \ideaCS{}, then no getting general ability without witnessing reasoning for specific instances.

  Core of the tension.
  Always some independent check with distinct premises with specific instance of general ability.

  So, either, allow to bypass independent check.
  Or, do not require witnessing reasoning from premises to conclusion.
\end{note}

\begin{note}
  Now, possible to resolve tension various ways.
  Reject \ideaCS{}.
  Reject \ESU{}.
  Reject claiming support for general ability.
  Or, any combination of the above.

  Interest is in rejecting \ESU{}.
  So long as \(\psi\) having value \(v'\) follows from some premises, then the reasoning doesn't matter.
  Claim support for \(\psi\) having value \(v'\) from those premises, given the possibility of reasoning.
\end{note}

\begin{note}
  Note, the tension is not about whether \(\phi\) has value \(v\).
  Instead, the tension is about whether the agent would have a certain property if they were to conclude \(\phi\) has value \(v\).
  Property of having claimed \support{}.
  Expanded, property of holding that any independent check is satisfied.
  Any other reasoning about whether \(\phi\) has value \(v\) would conclude \(\phi\) has value \(v\).
\end{note}

\begin{note}
  Returning to \EAS{}.
  Specific instances of the general ability.
  In this sense, the instances of \EAS{} we argue for are narrow.
  Need strong sign that the agent has the general ability.

  Further.
  It does not state that an agent having claimed support that they have the ability to reason to some conclusion is \emph{always permissible} to claim support for the conclusion by appealing to some premises that do not form part of the agent's reasoning.
  Instead, it states that \emph{may be permissible} for the agent claim support in a certain way.

  In various respects, these aren't particularly interesting cases.
  However, the goal is to argue that such cases exist.
  Whether these are constrained to the type of cases we consider for the argument is a further question.

  There may be more interesting cases, but given that \ESU{} is incompatible with all such cases, I see no compelling reason to explore such cases without \emph{first} motivating a rejection of \ESU{}.
\end{note}

\begin{note}
  Also suggests that the content of general ability is somewhat interesting.
  For, the content is itself general.
  It is a conclusion that ranges over all specific instances of the general ability.
\end{note}

\begin{note}[Terminology]
  So, the upshot of this is that an agent concludes various things in certain cases.
  In concluding \(\phi\), also conclude \(\psi,\dots\).
  And, in cases of interest, because of generality of the reasoning.

  This is somewhat puzzling.
  Though, I think less puzzling than first appears.
  Concluding \(\phi\) has value \(v\) is nothing special.
  Of course, the agent only explicitly concludes a handful of things, but allowing the generality is nothing that different from equivalences.

  It also doesn't follow that any of the additional properties of the reasoning, if any, are carried over to any \requ{1}.
  Is just about concluding.
  Here, then, various ways to keep the intuition for the negative resolution.
  There may be various things that are exclusive to witnessing reasoning from premises to conclusion.
  However, distinct from concluding.

  Still, stronger than being committed.
  Ranges over any implication.
  Conclude no winning strategy, then also conclude various other chess things.
  However, committed, but do not necessarily conclude that X is going to lose the game.

  Concluding is still of interest.
  Or, as noted, `reason', in the weak term.
\end{note}

\subsection{Matrix}

\begin{note}
  In cases of interest, \aben{the}.
  Only for this type of ability do we get a conclusion that \(\phi\) has value \(v\).
  \autoref{fig:saMtxInterpreted:outline} lays out different interpretations.
  Restricted to \adB{}
\end{note}

\begin{note}
  \begin{figure}[H]
    \centering
    \saMtxInterpreted{}
    \caption{Distinction matrix with \aben{the}}
    \label{fig:saMtxInterpreted:outline}
  \end{figure}
\end{note}

\begin{note}
  Recap.

  Claiming support.
  Constraint.

  Ability.
  In order to be compatible, satisfy constraint.
  Either of three options.
  Basic, ignore this.
  Property. Incompatible with constraint.
  Witness. Compatible.

  Here, display the matrix.
  I think this is the easiest way to visualise what is going on.
\end{note}


%%% Local Variables:
%%% mode: latex
%%% TeX-master: "master"
%%% End: