\chapter{Overview}
\label{cha:overview}

\section{The issue}
\label{sec:issue}

\begin{note}[Main issue, positive resolution, quick argument for negative]
  Our interest is with the following issue:

  \begin{restatable}[Main]{issue}{issueMain}
    \label{issue:Main}
    Are the cases in which an agent may conclude \(\phi\) has value \(v\) from certain premises without witnessing reasoning that concludes \(\phi\) has value \(v\) from those premises?
  \end{restatable}

  The goal is to motivate a positive resolution:

  \begin{restatable}[Main: tick]{issue}{issueMainPositive}
    \label{issue:Main:R:p}
    \emph{There are cases in which} agent may conclude \(\phi\) has value \(v\) from certain premises without witnessing reasoning that concludes \(\phi\) has value \(v\) from those premises.
  \end{restatable}

  This positive resolution is, by my estimation, not straightforward.
\end{note}

\begin{note}[The quick argument]
  At least, the positive resolution is not straightforward without placing some constraints on \emph{which} premises an agent may appeal to when reasoning, and we will motivate the positive resolution without such constraints.

  For, without constraints on which premises an agent may appeal to when reasoning, one may argue as follows:
  \begin{enumerate}
  \item Any instance of reasoning is some process with start and end points and intermediary steps.
  \item If an agent has concluded \(\phi\) has value \(v\) by some reasoning, then the reasoning has start points and intermediary steps.
  \item Hence, the agent has concluded \(\phi\) has value \(v\) by witnessing some reasoning from some start points via some intermediary steps.
  \item In other words, the agent has concluded \(\phi\) has value \(v\) by witnessing some reasoning from some premises.
  \end{enumerate}

  In short, so long as an agent has concluded \(\phi\) has value \(v\), the agent has always witnessed reasoning from some premises.

  \begin{enumerate}[resume]
  \item So, either the start points are the premises of interest mentioned in the issue, or the agent has concluded \(\phi\) has value \(v\) from a distinct set of premises.
  \end{enumerate}

  In other words, either the agent has witnessed reasoning from the premises of interest, or the premises of interest (and any reasoning from them) are not required to conclude \(\phi\) has value \(v\).
\end{note}

\begin{note}[More on the quick argument]
  The quick argument does not directly lead to a negative resolution to the issue.
  Still, the quick argument does suggest that any appeal to premises \emph{without} witnessing reasoning from those premises is redundant.

  Now, perhaps redundancy isn't so bad.
  I only need a single key to ensure I have the option of unlocking a door, but a second key is useful if the first is lost.

  Still, I take it to be the case that redundancy provides leverage for a wide range of arguments motivating a negative resolution in the case of reasoning.

  For, if appeal to some premises is redundant, then any argument that requires witnessing need only observe that a counterargument must find some role for something which is not needed.

  Reasoning is an event, and distinct way of concluding \(\phi\) has value \(v\) may be useful, it is unclear why the distinct way of concluding \(\phi\) has value \(v\) is of use when concluding \(\phi\) has value \(v\) from present premises.
  To push the analogy, a second key may have various uses, but the second key is irrelevant in the event of unlocking the door with the first key.
  That the second key is would unlock the door if the first was lost has no role in the event of unlocking the door with the first key.

  From a different perspective, if appeal to certain premises without witnessing reasoning from those premises is redundant, then it seems any positive role given to appeal to those premises may be redistributed to the premises of the reasoning the agent did witness.

  More concretely, even if I were to show that there was some benefit for concluding \(\phi\) has value \(v\) via unwitnessed reasoning with respect to some particular account of reasoning, it seems at least plausible that the account of reasoning may be reformulated to derive the same benefit from the premises of the reasoning the agent witnessed.

  More generally, it may seem (and I suspect it does seem) intuitive that the issue should be resolved negatively.
  Reasoning just is obtaining a conclusion by witnessing reasoning from premises.
  And, if the quick argument succeeds, then there surely is some way to preserve the intuition.
\end{note}

\begin{note}
  So, part of the task is to show that the quick argument fails.
  Concluding \(\phi\) has value \(v\) from certain premises without witnessing reasoning that concludes \(\phi\) has value \(v\) from those premises has some role.
  Or, rather, that the quick argument is not without cost.
  Perhaps the issue really should be resolved in the negative, but this will require giving up some at least equally (I think) intuitive ideas.

  The result will be motivation for a positive resolution to the issue.
  However, the motivation will be somewhat narrow.
  To escape the tension, the positive resolution need only hold for a restricted pattern of reasoning.
  Still, with the existential motivated, I hope future work may expand the positive resolution to other patterns of reasoning.
  And, while such expansions may still need to argue that concluding \(\phi\) has value \(v\) from unwitnessed reasoning is makes sense with respect to the specific topic at hand, observing that the broad idea of concluding \(\phi\) has value \(v\) from unwitnessed reasoning may not be dismissed without cost may be an option.
\end{note}

\section{Ideas}
\label{sec:ideas-1}

\subsection{The agent's epistemic state}
\label{sec:agents-epist-state}

\begin{note}
  Idea.
\end{note}

\begin{note}
  Distinguish agent's epistemic state from \stance{} of the agent.

  Various things we hold to be the case, and these may be relativised to the agent's perspective on how things actually are.
  Other things, hold regardless of the agent's perspective on how things actually are.
  And, things that hold regardless of whether the agent recognises.

  Example.
  Sam shorter than Taylor.
  Then, from epistemic state, Taylor shorter than Sam.
  Doesn't matter whether Sam is shorter than Taylor.
  From perspective of agent's epistemic state.
  Likewise, doesn't matter whether agent recognises shorter.

  Similarly, classical and intuitionistic.
  From int.\ perspective not a proof.
  From classical, also a proof of \(\phi\).

  And, if left the oven on, should check.
  Regardless of whether really did, and regardless of agent's morals.

  No clean divide.
  For present purposes, concluding and \csN{} will take epistemic state as input, but ignore the agent's \stance{}.
  What this means in practice will be seen through the following discussion.
  Roughly, what an agent concludes will depend on epistemic state, and whether \csN{} in concluding will likewise depend.
  Neither will depend on whether the agent recognises, or takes themselves, to have concluded or \csVed{}.
\end{note}

\subsection{Concluding}
\label{sec:outline:concluding}

\begin{note}[Overview]
  Concluding.
  This is the most basic thing.
  Lay out assumptions as clearly as possible.

  Generally speaking, assumption does two things.
  What concluding is, what concluding is not.
  Assumptions will not provide an exhaustive account of what concluding is and what concluding is not, but sufficient for a main goal.

  Indeed, goal is a substantial result on what some instances of what concluding is.
  And, in turn, a substantial result on what concluding, in general, is not.

  And, as concluding instance of reasoning, the previous two points expand to reasoning in general.
\end{note}


\paragraph{Values and propositions}

\begin{note}[Value proposition]
  Reasoning and claims to support focus.
  Briefly introduce a pair of propositions to clarify claim to support and reasoning.

  \begin{restatable}[Claimed support is for a proposition having some value]{assumption}{assuCSVP}
    \label{assu:CSVP}
    When an agent concludes \(\phi\) has value \(v\), the agent assigns value \(v\) to the \world{1} described by \(\phi\).
    (Where propositions individuate \world{1} from the perspective of the agent.)
  \end{restatable}

  \autoref{assu:CSVP} fixes terminology.
  To illustrate, when stating the conclusion of the reasoning sketched above we used the proposition that \emph{the area of the rectangle is \(133\text{cm}^{2}\)}.
  The proposition refers to the \world{} in which the area of the rectangle is \(133\text{cm}^{2}\), and speaking a little more precisely, the agent claimed that the proposition has the value `true' --- though it may be the value turns out to be `false'.
  Or, perhaps if the agent was a little unsure about the accuracy of the ruler, that the proposition has the value `likely', `probable', or some quantitative credence.
  And, some other instance of reasoning may have concluded that the proposition has the value `desirable' --- e.g.\ if the agent was searching for a rectangle of some approximate size.\nolinebreak
  \footnote{
    Nothing in particular hangs on the distinction between different values.
    If you prefer, you may expand the proposition (\world{}) to include additional factors, and consider only the values `true' and `false'.
    For example, the proposition that \emph{I desire the bath to be warm} is false, as opposed me assigned the proposition that \emph{the bath is warm} the value `undesirable'.
  }
\end{note}

\begin{note}
  Core idea is that claim of support is that the \world{} is a certain way.
  Proposition, what the thing is.
  Value, the way it is.

  A handful of instances:
  \begin{itemize}
  \item \(p\) is assigned the value `true'. \hfill (\emph{p} is true.)
  \item \(p\) is assigned the value `ought to be'. \hfill (\emph{p} ought to be the case.)
  \item \(p\) is assigned the value `desirable'. \hfill (\emph{p} is desirable.)
  \item \(p\) is assigned the value `improbable'. \hfill (\emph{p} is improbable.)
  \end{itemize}
\end{note}

\begin{note}
In most cases the value will be clear (i.e. that the proposition is true, though sometimes that the proposition is desirable), and so we will talk of claiming support for the proposition.
  A handful of additional examples will be provided when illustrating the next proposition.
\end{note}

\begin{note}
  Nothing hangs on distinction between values.
  Reduce everything to truth and falsity.
  However, we do not assume this, and if you do not think this is the case either, then I would like to not suggest that the assumptions and arguments to follow concern only those propositions which may be evaluated as true or false.
\end{note}

\paragraph{Premises are not assumptions}

\begin{note}
  By premises, we mean inputs of reasoning.
  These are not necessarily assumptions, though the may be.
  If conclude \(x\) from \(y\), then I am not necessarily concluding \(x\) given \(y\).
  Rather, by appealing to \(y\) I get \(x\).

  Possible to go for meal this evening.
  Need to pay.
  Sufficient bank balance.

  Without bank balance, would not conclude.
  However, not assuming sufficient bank balance.
  Rather, observing that I do have sufficient bank balance.
\end{note}

\paragraph{Culmination of reasoning}

\begin{note}
  Given the positive resolution, concluding.
  Two place relation.
  Premises and conclusion.

  Result of some event.
  No special significance is attached to the term.

  \begin{restatable}[Claiming support is the result of reasoning]{assumption}{assuCSRR}
    \label{assu:CS-culmination-of-R}
    An agent concludes \(\phi\) has value \(v\) just when \(\phi\) having value \(v\) is the culmination of reasoning from some premises \(\Phi\).
  \end{restatable}

  \begin{notation}
    Proposition-value-premise pairing, abbreviate \(\langle \phi,v,\Phi \rangle\) .
  \end{notation}

  Just the result of finishing a process.
  No special significance is attached to the term.

  In this respect, positive resolution, reasoning, but premises of reasoning are not the premises from which agent concludes \(\phi\) has value \(v\).
\end{note}

\begin{note}
  The role of~\autoref{assu:CS-culmination-of-R} is primarily to ensure that claiming support always guarantee the existence of premises and steps.
  With the exception of some broad constraints to be outlined in further assumptions below, we (will) have little to say about the specifics of what the reasoning consists of.
\end{note}

\begin{note}[Quick examples]
  \begin{itemize}
  \item \(S\) testified that \(p\), so \(p\) is true.
  \item \(p\) would satisfy every member of the group, so \(p\) ought to be the case.
  \item The song is produced by \(S\), so it is desirable that I listen to it.
  \item The device reads \(p\) and is reliable, so \emph{not}-\(p\) is improbable.
  \end{itemize}
\end{note}

\begin{note}
  Instances of reasoning may culminate in other ways, so we are only interested in a specific type of reasoning.
\end{note}

\begin{note}[Claiming support]
  Expand on this below.
  Briefly mention that this falls short of \emph{establishing} \(\phi\) has value \(v\).
\end{note}

\begin{note}[Understanding `having value \(v\)']
  In a deductive case, if the premises are true, then the conclusion is true.
  Means-end reasoning for desire.
  The value is important.
  If it is true that it past 6pm, then it is true the shop is closed.
  Provides value of shop being closed.

  However, if agent desires that it is past 6pm, then it doesn't follow that the agent desires that the shop is closed.
  Question an agent as to why they think their desires conform to truth --- is-ought problem.

  Means-end reasoning.
  It is true that there is cheese at the centre of the maze.
  And, it is desirable that I obtain the cheese at the centre of the maze.
  Further, it is true that I may only obtain the cheese at the centre of the maze by solving the maze.
  Therefore, it is desirable that I solve the maze.
\end{note}

\begin{note}[`Concluding']
  Rather, \(\phi\) \emph{has} value \(v\).

  So, from the perspective of an agent, `I conclude \(\phi\) has value \(v\)' and \(\phi\) has value \(v\) are interchangeable.
\end{note}

\begin{note}[Tied to premises]
  What does matter is role of premises.
  From some conclude \(\phi\) has value \(v\), from others conclude \(\phi\) does not have value \(v\).
\end{note}

\begin{note}[Generally]
  Concluding \(\phi\) has value \(v\).
  \(\phi\) having value \(v\) follows from premises.
  These premises, and therefore \(\phi\) has value \(v\).

  The `follow' or `therefore' may be deductive or non-deductive.
\end{note}

\paragraph{Three distinct options}

\begin{assumption}
  Fix \(\phi\), \(v\), and \(\Phi\).
  Three distinct options:
  \begin{enumerate}
  \item Conclude \(\phi\) has value \(v\) from \(\Phi\).
    \(\langle \phi,v,\Phi \rangle\).
  \item
    Conclude \(\phi\) has some value other than \(v\) from \(\Phi\).
    \(\langle \phi,\overline{v},\Phi \rangle\).
  \item
    Fail to conclude whether \(\phi\) has value \(v\) from \(\Phi\).
    \(\langle \phi,?,\Phi \rangle\).
  \end{enumerate}
  The second may be further distinguished.
  That \(\phi\) has some other value, and some specific value.
  We have no need for such a finer grained distinction.
\end{assumption}


\paragraph{Delicacy}

\begin{note}
  Reasoning that concludes \(\phi\) has value \(v\) is distinct from reasoning:
  \begin{enumerate}[label=\Alph*., ref=(\Alph*)]
  \item
    \label{CS:delicacy:O}
    Whose conclusion is (merely) \emph{about} \(\phi\) having value \(v\), and does not `require' \(\phi\) has value \(v\).
  \item
    \label{CS:delicacy:A}
    That concludes \(\phi\) has value \(v\) \emph{assuming} \dots\space --- where `\emph{assuming} \dots\space' is expanded to include some proposition-value pair \(\langle \chi,v \rangle\) such that they agent has \emph{not} concluded \(\langle \chi,v \rangle\).
  \end{enumerate}
\end{note}

\begin{note}[Epistemic operator]
  The statement of \ref{CS:delicacy:O} is loose, but the underlying idea is straightforward.

  we typically express the conclusion of the agent's reasoning which is (merely) about \(\phi\) having value \(v\) with some adjective.
  For example:

  \begin{quote}
    \vAgent{} \(\{ \text{hopes}, \text{imagines}, \text{desires}, \text{thinks}, \dots \}\) \(\phi\) is true.%
    \footnote{
      More generally, has value \(v\).
    }
  \end{quote}
  Each candidate adjective may be read in a way which is compatible with past, present, or future reasoning which concludes \(\phi\) is not true.
  In this sense, the reasoning is (merely) \emph{about} \(\phi\) being true, and does not require \(\phi\) is true.

  However, the relevant adjective may also be read as an evaluation of the relevant situation.
  It may be \(\phi\) \emph{is} hoped for, imagined, desired, thought, and so on.
  In this sense, the reasoning concludes \(\phi\) has some value \(v\), where the value is some value distinct from `true'.

  For the most part, `true' will be the relevant value \(v\) when discussing instances of reasoning.
  And, you may (unless there is clear tension) substitute mention or use of `\(\phi\) having value \(v\)' for `\(\phi\) being true'.
  Still, we do not require that `true' is the unique value of interest, and on occasion we will observe how various different valuations interact with observations made.
\end{note}

\begin{note}[Suppositions]
  \ref{CS:delicacy:A} is more straightforward.
  Contrast the instances of reasoning in~\autoref{fig:Rover}.
  \begin{figure}[h!]
    \mbox{}\hfill
    \begin{subfigure}{0.45\linewidth}
      \begin{enumerate}
      \item
        \label{fig:Rover:CS:1}
        Rover is tired.
      \item
        \label{fig:Rover:CS:2}
        Rover will fall asleep soon.
      \end{enumerate}
      \caption{}
      \label{fig:Rover:CS}
    \end{subfigure}
    \hfill
    \begin{subfigure}{0.45\linewidth}
      \begin{enumerate}[label=\arabic*\('\).,ref=(\arabic*\('\))]
      \item
        \label{fig:Rover:nCS:1}
        \emph{Supposing} Rover is tired.
      \item
        \label{fig:Rover:nCS:2}
        Rover will fall asleep soon.
      \end{enumerate}
      \caption{}
      \label{fig:Rover:nCS}
    \end{subfigure}
    \hfill\mbox{}
    \caption{Two instance of reasoning}
    \label{fig:Rover}
  \end{figure}

  \ref{fig:Rover:CS} and \ref{fig:Rover:nCS} are distinguishing by whether or not the premise that Rover is tired is a supposition (\ref{fig:Rover:nCS}) or not (\ref{fig:Rover:CS}).

  It may be the case that Rover would fall asleep soon if Rover were tired, but as \ref{fig:Rover:nCS:1} does not concern the actual state of Rover, \ref{fig:Rover:nCS:2} need not concern the actual state of Rover.

  By contrast,~\ref{fig:Rover:CS} is an instance of reasoning about how things are.
  The agent is holding that it is the case that Rover is tired, and therefore it is the case that Rover will fall asleep soon.

  I take this distinction to be straightforward.
  Though, it is not always immediately clear how the distinction applies to some conclusion of reasoning when stated independently of the preceding reasoning.
  Consider the conclusion:

  \begin{quote}
    That Rover is asleep is likely.
  \end{quote}

  Following the distinction, there are two broad ways of interpreting the conclusion:
  \begin{enumerate}
  \item
    The agent has made some plausible assumptions and granting those assumptions, Rover is asleep.
  \item
    For every proposition-value pair \(\chi_{i}\) having value \(v_{i}\) the agent has appealed to, the agent hold it to be the case \(\chi_{i}\) has value \(v_{i}\).
    However, the agent only concludes from \(\chi_{i}\) having values \(v_{i}\) that there is some (objective or subjective) chance that Rover is asleep.
  \end{enumerate}
  So, though claiming support is for some proposition \(\phi\) having some value \(v\), it is not clearly the case that claiming support is restricted to certain proposition-value combinations.
\end{note}

\begin{note}[Analogue]
  Analogue of concluding for conditional.
  Here, bundle into \(\phi\).
  More generally, not much need for conditional stuff.
\end{note}

\paragraph*{Conclusions are `unique'}

\begin{note}[Conclusions are unique]
  \begin{assumption}[Conclusions are `unique']
    \label{assu:conc:unique}
    For any value type \(v_{\tau}\):

    If an agent has the option of concluding \(\langle \phi,v_{\tau} \rangle\), then the agent does not have the option of concluding \(\langle \phi,\overline{v_{\tau}} \rangle\).
  \end{assumption}

  In other words, there are no cases where an agent may choose between concluding \(\phi\) has value \(v\) and concluding \(\phi\) has some value other than \(v\) of the same time.

  For example, if an agent has the option of concluding \(\phi\) is true, then the agent does not have the option of concluding \(\phi\) is false.
  Likewise, if the agent has the option of concluding \(\phi\) is undesirable, then the agent does not have the option of concluding \(\phi\) is desirable.
  However, as truth and desirability are distinct value types, an agent may conclude \(\phi\) is false but desirable.

  We do not make any explicit assumptions about relations between value types, though we take for granted sensible constraints.
  For example, if an agent has the option of concluding \(\phi\) is true, the agent does not have the option of concluding \(\phi\) is impossible, though whether some proposition is possible or impossible is distinct from whether the proposition is true or false.
\end{note}

\begin{note}
  This is weaker than voluntarism.
  An agent may choose whether or not to conclude.
  And, may choose to whether to retract some previous conclusion.
\end{note}

\paragraph*{Concluding is `description-free'}

\begin{note}[Descriptions]
  An important assumption is that an agent need not recognise that the culmination of some instance of reasoning is that some proposition has some value.

  \begin{assumption}[Concluding is `description-free']
    \label{assu:conc:d-free}
    It is not the case that an agent concludes \(\phi\) has value \(v\) only if the agent concludes \(\phi\) has value \(v\) under some description \emph{d}
  \end{assumption}

  In particular,~\autoref{assu:conc:d-free} holds for any description \emph{d} which includes an intensional reading of `\(\phi\) has value \(v\)'.
  More generally, it is possible for an agent to conclude \(\langle \phi,v \rangle\) without consciously or otherwise entertaining either `\(\phi\)' or `\(v\)'.%
  \footnote{
    Compare with, for example, \citeauthor{Anscombe:1957aa} on intention action (\citeyear[\S19]{Anscombe:1957aa}) and \citeauthor{Davidson:1963aa} on primary reasons (\citeyear[5]{Davidson:1963aa}).
  }

  For the moment we will focus on intensionality.
  We will briefly explain the distinction between intensional and non-intensional readings, and motivate~\autoref{assu:conc:d-free} with respect to a handful of examples.

  Note, however, we are not providing an analysis of what it is for an agent to conclude \(\langle \phi,v \rangle\).
  Rather,~\autoref{assu:conc:d-free} narrows down the particular sense of `concluding' of interest to us.
  There may be, and plausibly is, a sense of `concluding' for which~\autoref{assu:conc:d-free} does not hold (and in particular where the proposition-value pair of the conclusion is always intensional).
  However, our interest is with a sense of `concluding' for which~\autoref{assu:conc:d-free} holds.
\end{note}

\begin{note}[Looking ahead]
  Looking ahead, briefly,~\autoref{assu:conc:d-free} is a key assumption for developing tension.
  Tension will not follow from any reading of `concluding' in which in concluding \(\langle \phi,v \rangle\) an agent concludes \(\langle \phi,v \rangle\), and \(\langle \phi,v \rangle\) alone.
  Rather, the tension we develop will involve an agent concluding \(\langle \psi,v' \rangle\) when concluding \(\langle \phi,v \rangle\).

  Now, our discussion of intensionality will suggest an agent may conclude \(\langle \phi,v \rangle\) when concluding \(\langle \varphi, v \rangle\) when there is significant overlap with what \(\phi\) and \(\varphi\) refer to.
  Still,~\autoref{assu:conc:d-free} allows that in concluding \(\langle \phi,v \rangle\) an agent may also conclude \(\langle \psi,v \rangle\), where there is no overlap between the reference of \(\phi\) and \(\psi\).
  When developing tension we will have interest with sufficient overlap with what \(\phi\) and \(\varphi\) refer to.
  So, we will not require~\autoref{assu:conc:d-free} in full generality, but equally it is only after developing the tension of interest that we will see in how~\autoref{assu:conc:d-free} may be restricted.
\end{note}

\begin{note}[Intensionality]
  Consider the following observation from~\citeauthor{Quine:1943vf}:

  \begin{quote}
    \begin{enumerate}[label=(\arabic*)]
    \item
      Giorgione = Barbarelli,
    \item
      Giorgione was so-called because of his size
    \end{enumerate}
    are true; however, replacement of the name 'Giorgione' by the name 'Barbarelli' turns (2) into the falsehood:

    \begin{center}
      Barbarelli was so-called because of his size.
    \end{center}
    \vspace{-\baselineskip}
    \mbox{ }\nolinebreak
    \mbox{}\hfill\mbox{(\Citeyear[113]{Quine:1943vf})}
  \end{quote}

  Both `Giorgione was so-called because of his size' and `Barbarelli was so-called because of his size' are intensional in the sense that the truth value of each expression does not reduce to the reference of the expression's' components.
  Else, as `Giorgionee' and `Barbarelli' are co-referential, the predicate `was so-called because of his size' would apply equally to both `Giorgionee' and `Barbarelli'.

  Now, it is also not the case that in concluding `Giorgione was so-called because of his size', one also concludes `Barbarelli was so-called because of his size'.

  However, this observation is \emph{not} immediate from observing that the agent concluded `Giorgione was so-called because of his size' and did not conclude `Barbarelli was so-called because of his size'.

  The conclusion `Giorgione was so-called because of his size' may be intensional, but being intensional is not a property granted to some proposition-value pair by virtue of the proposition-value pair being a conclusion.

  For example, in concluding `Giorgione is large' the agent may also conclude `Barbarelli is large'.
  Indeed, the expression `Giorgione is large' may be read non-intensionally, and hence is true if and only if `Barbarelli is large', given that `Giorgionee' and `Barbarelli' are co-referential.

  Likewise, in concluding `\(2 + 2 = 4\)', an agent may also conclude `\(4 = 2 + 2\)'.
  Or, in concluding `\nagent{1} is shorter than \nagent{2}', an agent may also conclude `\nagent{2} is taller than \nagent{1}'.
  Though, in concluding `\nagent{1} is shorter than \nagent{2}' an agent may also fail to conclude `\nagent{2} is taller than \nagent{1}'.
  For example, if the relevant agent is not aware of the relationship between `shorter' and `taller'.
\end{note}

\begin{note}
  More broadly, I consider it intuitive that concluding any one of the following includes concluding any other:
  \begin{itemize}
  \item \(\phi\) has value \(v\).
  \item It is true that \(\phi\) has value \(v\).
  \item It is not the case that \(\phi\) does not have value \(v\).
  \item It is true that it is not the case that \(\phi\) does not have value \(v\).
    \begin{center}
      \(\vdots\)
    \end{center}
  \end{itemize}

  There are an infinite number of distinct proposition-value pairs that may be generated along these lines, but these distinct proposition-value pairs do not amount to distinction conclusions.
  A conclusion for one is a conclusion for all.
\end{note}

\begin{note}
  Indeed, we may form an explicit assumption governing certain proposition-value pairs.
  We start with the definition of \indicateN{}.
\end{note}

\begin{note}
  \begin{restatable}[\indicateN{2}]{definition}{defIndicate}
    \label{def:indication}
    \(\phi\) having value \(v\) \emph{\indicateV{1}} \(\psi\) has value \(v'\) if and only if:
    \begin{itemize}
    \item
      It is not \epPAd{}, from \vAgent{}' epistemic state, that \(\psi\) has value \(v'\) while \(\psi\) does not have value \(v'\).
    \end{itemize}
    \vspace{-\baselineskip}
  \end{restatable}
\end{note}

\begin{note}
  The assumption now holds that an agent concludes \(\langle \psi,v' \rangle\) when concluding \(\langle \phi,v \rangle\) just in case \(\langle \psi,v' \rangle\) and \(\langle \phi,v \rangle\) co-\indicateV{}.

  \begin{restatable}[\indicateN{2}]{assumption}{assuIndicate}
    \label{assu:indication}
    If \(\langle \phi,v \rangle\) \indicateV{1} \(\langle \psi,v' \rangle\), and \(\langle \psi,v' \rangle\) \indicateV{1} \(\langle \phi,v \rangle\), then:

    \begin{itemize}
    \item
      \vAgent{} concludes \(\langle \phi,v \rangle\) just in case \vAgent{} concludes \(\langle \psi,v' \rangle\).
    \end{itemize}
    \vspace{-\baselineskip}
  \end{restatable}
\end{note}

\begin{note}
  \Autoref{assu:indication} is mild closure condition on claiming support.
  Still, \autoref{assu:indication} is only a closure condition with respect to an agent's epistemic state.
  To illustrate:
  Suppose an agent has concluded that The Scarlet Pimpernel rescued Marquis de Lafayette.
  As `The Scarlet Pimpernel' and `Sir Percy Blakeney' are co-referential, it may be that the agent's conclusion \indicateV{1} that Sir Percy Blakeney' rescued Marquis de Lafayette.
  However, the conclusion will \indicateN{0} \emph{only if} there are no \epPW{1} in which `The Scarlet Pimpernel' and `Sir Percy Blakeney' refer to different individuals.

  Likewise, an agent may conclude that helping The Scarlet Pimpernel is desirable, without the conclusion \indicatePr{} that helping Sir Percy Blakeney is desirable.

  Indeed, an agent's conclusion that it is raining, while standing in Gower Street, may fail to \indicateN{} that it is raining in London if the agent considers it \epPAd{} that they are not in London.
\end{note}

\begin{note}
  Generalising,~\autoref{assu:indication} may be strengthened by weakening the restriction to `If \(\langle \phi,v \rangle\) \indicateV{1} \(\langle \psi,v' \rangle\)'.%
  \footnote{
    I.e.\ only the first conjunct of the restriction given.
  }

  Hence, in concluding \(\langle \phi,v \rangle\) an agent would also conclude any proposition-value pair \(\langle \psi,v' \rangle\) \emph{weaker} than \(\langle \phi,v \rangle\), from the agent's epistemic state.
  Indeed, this leads to a much stronger closure condition on concluding.%
  \footnote{
    Consider by parallel closure of knowledge under known entailment:
    \begin{itemize}
    \item If an agent knows that \(\phi\) has value \(v\) only when \(\psi\) has value \(v'\), then if the agent knows \(\phi\) has value \(v\), then the agent knows \(\psi\) has value \(v'\).
    \end{itemize}
    This closure condition differs in forms, as it concerns knowledge as a state, by may be reformulated to a closer parallel:
        \begin{itemize}
    \item If an agent knows that \(\phi\) has value \(v\) only when \(\psi\) has value \(v'\), then in coming to know \(\phi\) has value \(v\) the agent comes to know \(\psi\) has value \(v'\).
    \end{itemize}
  }

  We will not assume this stronger variant of~\autoref{assu:indication} holds for the sense of `concluding' we are interested in.
  Rather, we have seen how~\autoref{assu:conc:d-free} allows for the possibility of an agent concluding \(\langle \psi,v' \rangle\) when concluding \(\langle \phi,v \rangle\), and with the exception of~\autoref{assu:indication} which we take to be sufficiently intuitive, we only advance argument in cases of interest.
\end{note}

\begin{note}
  Indeed, though the stronger variant of~\autoref{assu:indication} may be intuitive, there are certain issues we would like to avoid taking a stance on.
  In particular, whether concluding some statement which quantifiers over various objects includes concluding for each object the quantifier applies to.

  For example, suppose I have conclude that there are infinitely many primes.
  Reflecting a little on the natural numbers, I observe that if there are infinitely many primes, then for every natural number \(n\) there is some prime larger than \(n\) (for, the natural numbers are not dense).
  Hence, for every natural number \(n\) there is some prime larger than \(n\).

  On the stronger variant of~\autoref{assu:indication}, concluding `for every natural number \(n\) there is some prime larger than \(n\)' would also include concluding there is some prime larger than \(n\) for each \(n\).
  E.g.\ there is some prime larger than \(1\), \(16\), \(5^{43}\), \(53!^{793}\), \(54!^{794!}\), and so on\dots

  Specifically, looking ahead to tension, concluding that one has the general ability to witness some kind of reasoning would involve concluding that one has each specific instance of the general ability.
\end{note}

\begin{note}[Witnessing]
    While~\autoref{assu:conc:d-free} focuses on concluding, we take~\autoref{assu:conc:d-free} to apply equally to witnessing reasoning in which an agent concludes.
  Indeed, if a negative resolution to~\autoref{issue:Main} then any instance of concluding is also an instance of witnessing reasoning to the relevant conclusion, and hence~\autoref{assu:conc:d-free} would be in conflict with an assumption which states that an agent only witnesses reasoning which concludes \(\langle \phi,v \rangle\) under some description.
  On the other hand, a positive resolution to~\autoref{issue:Main} does not ensure any instance of concluding is also an instance of witnessing reasoning to the relevant conclusion.
  However, we arrive at the same conflict with respect to any instance of concluding which is the result of witnessing reasoning to the relevant conclusion.

  Indeed, if \(\langle \phi,v \rangle\) \indicatePr{} \(\langle \psi,v' \rangle\), then \(\phi\) has value \(v\) \emph{only if} \(\psi\) has value \(v'\) (from the perspective of the agent's epistemic state).
  Hence, \emph{if} in concluding \(\langle \phi,v \rangle\) an agent also concludes some \indicateVed{} \(\langle \psi,v' \rangle\), then the relevant premises which allow the agent to also conclude \(\langle \psi,v' \rangle\).
\end{note}

\begin{note}[Perspective on issue]
  Looking ahead, perspective on issue.
  In some cases, concluding one would conclude \(\langle \phi,v \rangle\) from some premises \(\Phi\) is equivalent to concluding \(\langle \phi,v \rangle\) from \(\Phi'\), where \(\Phi\) and \(\Phi'\) are distinct.
\end{note}

\paragraph{Summary}

\begin{note}[Summary]
  Handful of assumptions regarding concluding.

  For the most part, I take these to be straightforward.

  Suspect,~\autoref{assu:conc:d-free} is of interest.
  Again, while there may be a sense of `concluding' for which~\autoref{assu:conc:d-free} does not apply, there is a sense of `concluding' for which~\autoref{assu:conc:d-free} does apply.
  This sense of `concluding'.

  However, I suspect that this does not impact interest.
  For, if recognise conclude in latter sense, then no issue concluding in former sense.
  Still, this is additional argument.
  And, would rather avoid issues regarding phenomenology of concluding.
  Well, do avoid such issues.
\end{note}

\section{\csN{2}}
\label{sec:overview:csn2}

\begin{note}
  \begin{quote}
    \issueMain*
  \end{quote}
  In~\autoref{sec:outline:concluding} we expanded on our understanding of concluding.
  {
    \color{red} summary?
  }

  Our goal is to establish tension, and thereby motivate a positive resolution to~\autoref{issue:Main}.
  In order to establish tension we narrow our attention to when concluding \(\langle \phi,v \rangle\) concluding \(\langle \phi,v \rangle\) involves the agent establishing a particular property with respect to \(\langle \phi,v \rangle\).
  We term the property `\csN{}'.

  Positive resolution only requires existence of cases.
  Hence, existence of cases with this property.
  This will be sufficient.
  Any case of concluding which involves \csVImp{} will also be an instance of concluding.

  As sketched, tension by {\color{red} \dots}.

  For the moment, however, we focus on providing a clear account of \csN{}.
  Tension delayed until \dots
  Indeed, following \csN{}, revise resolutions to ~\autoref{issue:Main}.
  And, additional building blocks for tension via two types of concluding.
\end{note}

\begin{note}
  The sole question for whether an agent \csV{} for \(\phi\) having vale \(v\) when concluding \(\langle \phi, v \rangle\) is:
  \begin{quote}
    Does \(\phi\) having value \(v\) ensure that there is some proposition-value-premise pairing \(\langle \psi,v',\Psi \rangle\) such that (from the agent's perspective) the agent \emph{may have failed} to conclude \(\langle \phi,v \rangle\) if they were to have first reasoned attempted to conclude \(\langle \psi,v' \rangle\) from \(\Psi\)?
    \begin{itemize}
    \item If there is no such \(\langle \psi,v',\Psi \rangle\), the agent \csV{}.
    \item If there is some \(\langle \psi,v',\Psi \rangle\), the agent fails to \csV{}.
    \end{itemize}
  \end{quote}
\end{note}

\begin{note}
  Two ways in which an agent may \csN{}.
  \begin{itemize}
  \item
    Absence of proposition-value pairs.
  \item
    Holding there is no such proposition-value pair.
  \end{itemize}
  First may be subsumed under the latter.
  Indeed, rephrase.
  For all such proposition-value pairs, would not fail to conclude.

  Certain cases, the universal is trivially satisfied.
\end{note}

\begin{note}[Simple examples of \csN{}]
  Simple examples of \csV{} for \(\phi\) having value \(v\) is concluding \(\phi\) has value \(v\) from the testimony of experts as a layperson.
  For, as a layperson one has no way of querying whether \(\phi\) has value \(v\).
  Hence, \(\phi\) having value \(v\) does not introduce the possibility of reasoning about some other proposition-value pair and concluding \(\phi\) does not have value \(v\).
  Fermat's last theorem is true, I am told, and I do not have the means to query the proof.

  Other examples involve unique sources of information.
  I conclude from the position of the hands on my watch that it is midday.
  The sky is cloudy, and without a second time piece I have no hope of reaching a different conclusion.

  And, more commonplace examples involve the gradual accumulation of proposition-value pairs.
  \nagent{16} \emph{said} they're coming to the party, but you know from \nagent{17} that \nagent{16} is coming to the party only if \nagent{18} is coming to the party.
  Without further information, reasoning about whether \nagent{18} is coming to the party might prevent you from taking \nagent{16} at the word.
  However, you have already have conformation from \nagent{18} that they are coming to the party.

  Likewise, suppose there are clear skies.
  Then, it is around midday only if the sun is (roughly) at the highest point of the sky.
  Though, I already concluded that the sun is (roughly) at the highest point of the sky prior to checking my watch for a more accurate read of the time.
\end{note}

\begin{note}[Simple failures]
  My preferred example for \emph{failure} to claim support is lost keys.
  Tempting as it may be to conclude that a pair of keys are lost after some searching, if the keys really are lost then there aren't in a handful of places you haven't yet thought to look.
  And, until you have concluded that the keys really aren't in those places, and that there is no-where else to look, the keys aren't really lost.

  Likewise, a friend's story may be entertaining, but one doesn't \csN{} that it actually happened without first checking that the details add up.
\end{note}

\begin{figure}[h]
  \centering
  \begin{tikzpicture}
    \node (origin) at (0,0) {};
    \node (psiSplit) at (1,0) {};
    \node (phiSplit) at (4,0) {};
    %
    \node[anchor=west] (psiV) at  (6,-1)  {\(\langle \psi,v' \rangle\)};
    \node[anchor=west] (psiNv) at (6,-2) {\(\langle \psi,\overline{v'} \rangle\)};
    \node[anchor=west] (psiQ) at (6,-3) {\(\langle \psi,? \rangle\)};
    %
    \node[anchor=west] (psiVPhiV) at (9,-1) {\(\langle \phi,v \rangle\)};
    \node[anchor=west] (psiNvPhiU) at (9,-2) {\(\langle \phi,\{\overline{v},?\} \rangle\)};
    \node[anchor=west] (psiQPhiU) at (9,-3) {\(\langle \phi,\{\overline{v},?\} \rangle\)};
    %
    \node[anchor=west] (phiQ) at (10,1) {\(\langle \phi,? \rangle\)};
    \node[anchor=west] (phiNv) at (10,2) {\(\langle \phi,\overline{v} \rangle\)};;
    \node[anchor=west] (phiV) at (10,0) {\(\langle \phi,v \rangle\)};
    %
    \draw[-]  (origin) -- (phiV);
    %
    \path[-,dashed] (phiSplit) edge [out=0, in=180] (phiNv);
    \path[-,dashed] (phiSplit) edge [out=0, in=180] (phiQ);
    %
    \path[-.] (psiSplit) edge [out=0, in=180] (psiV);
    \path[-, dashed] (psiSplit) edge [out=0, in=180] (psiNv);
    \path[-, dashed] (psiSplit) edge [out=0, in=180] (psiQ);
    %
    \draw[<-,dotted] (psiV) edge (psiVPhiV);
    \draw[->, dotted] (psiNv) edge (psiNvPhiU);
    \draw[->, dotted] (psiQ) edge (psiQPhiU);
    \end{tikzpicture}
    \caption{Sketch of when an agent has \csVed{} for \(\langle \phi,v \rangle\).}
    \label{fig:csN:illu:overview}
  \end{figure}

\begin{note}[Figure]
  \autoref{fig:csN:illu:overview} provides a rough visualisation of the constraint~\ref{idea:CS:overview} imposes on an instance of reasoning for the reasoning to be an instance of \csN{}.

  The flat line captures the agent's reasoning, which concludes with \(\langle \phi, v \rangle\).
  In concluding \(\langle \phi,v \rangle\) the agent rules out two possibilities with respect to \(\phi\).
  First, that \(\phi\) does not have value \(v\), indicated by \(\langle \phi,\overline{v} \rangle\).
  Second, that the agent does not assign any value to \(v\), indicated by \(\langle \phi,? \rangle\).
  Prior to concluding \(\langle \phi,v \rangle\), the agent's reasoning may have branched to either alternative path, but as the agent has concluded \(\langle \phi,v \rangle\), neither path is viable, and hence both paths are represented with a dashed line.

  So far, we have seen only that the agent has concluded \(\langle \phi,v \rangle\).

  With respect to \csN{}, observe that prior to ruling out alternative branches with respect to \(\phi\), the agent may have reasoned about whether \(\psi\) has value \(v\).
  And, from the agent's perspective, \(\phi\) has value \(v\) only if \(\psi\) has value \(v'\).
  If \(\psi\) does not have value \(v'\), then either \(\phi\) does not have value \(v\), or the agent's reasoning would not conclude with a value for \(\phi\), indicated by \(\langle \phi,\{\overline{v},?\} \rangle\).

  Hence, prior to concluding \(\langle \phi,v \rangle\), the agent has concluded \(\langle \psi,v' \rangle\).
\end{note}

\begin{note}[Following visual]
  Broadly, then, we may say that an agent has \csVed{} for \(\langle \phi,v \rangle\) just in case when concluding \(\langle \phi,v \rangle\) it is not the case that the agent's reasoning could have branched to a different conclusion with respect to \(\phi\).

  However, the visualisation of~\autoref{fig:csN:illu:overview} and this broad statement of \csN{} are a little too broad.
  For, we are only interested in proposition-value pairs guaranteed by \(\phi\) having value \(v\).
  \csN{} is not global with respect to all proposition-value pairs that the agent may have reasoned about, but local to those guaranteed by the proposition.

  This restriction may seem arbitrary, and to some degree I think it is.
  Ideally, an agent concluding \(\langle \phi,v \rangle\) is an instance of \csN{} just in case the agent would not have reasoned to a different conclusion if they were ti reason first about any other proposition-value pair.
  However, the advantage of focusing on some proposition-value pair `required' by \(\phi\) having value \(v\) is a significant constraint on the range of proposition-value pairs an agent needs to consider in order to \csN{}.

  In general, it may not be clear which proposition-value pairs may lead an agent to fail to conclude \(\phi\) has value \(v\), but so long the proposition-value pair of interest is given by \(\phi\) having value \(v\), an exhaustive search over all other proposition-value pairs may be avoided.

  Indeed, we will say that an agent has \emph{\support{}} for \(\phi\) having value \(v\) just in case they would not have reasoned otherwise, and reserve \emph{\claiming{}} \support{} for the weaker notion.

  We save discussion of \support{} for~\autoref{cha:claiming-support}, which will explore both \support{} and \csN{} in some detail.
\end{note}

\begin{note}[\csN{} idea]
  With the general notion of \csN{} in hand, and a few \illu{1} considered, we refine the notion of \csN{} into an idea.

  First, we introduce the help notion of some proposition-value pair being a \requ{} of concluding some (other) proposition-value pair.

  \begin{notion}[\requ{3}]
    \label{notion:overview:requ}
    \(\langle \psi,v' \rangle\) is a \requ{} of concluding \(\langle \phi,v \rangle\), with respect to an agent \vAgent{}'s epistemic state if:
    \begin{enumerate}
    \item
      \label{notion:overview:requ:main}
      From the perspective of \vAgent{}'s epistemic state, \(\phi\) has value \(v\) only if:
      \begin{enumerate}[label=\alph*., ref=\named{R:\alph*}]
      \item
        \label{notion:overview:requ:pool}
        \vAgent{} has the option of concluding \(\langle \psi, v' \rangle\) from some pool of premises \(\Psi\) where:
        \begin{enumerate}[label=\roman*., ref=\named{R:a.\roman*}, series=csIdeaCounter]
        \item
          \label{notion:overview:requ:pool:incl}
          \(\langle \phi,v \rangle\) is not a member of \(\Psi\).
        \item
          \label{notion:overview:requ:pool:int}
          \vAgent{} may conclude \(\langle \psi,v' \rangle\) from \(\Psi\) without concluding \(\langle \phi,v \rangle\) as an intermediary step.
        \item
          \label{notion:overview:requ:pool:ind}
          For any proposition-value pair \(\langle \psi_{i}, v_{i}' \rangle\) in \(\Psi\), \vAgent{} either has concluded or may conclude \(\langle \psi_{i}, v_{i} \rangle\) without concluding \(\langle \phi,v \rangle\).
        \end{enumerate}
      \item
        \label{notion:overview:requ:nPsi-nPhi}
        If \vAgent{} were to fail to conclude \(\langle \psi,v' \rangle\) from \(\Psi\) prior to reasoning about whether \(\phi\) has value \(v\), \vAgent{} would not conclude \(\langle \phi,v \rangle\) from \(\Phi\).
      \end{enumerate}
    \end{enumerate}
    \vspace{-\baselineskip}
  \end{notion}
  The notion of \(\langle \psi,v' \rangle\) being a \requ{} of concluding \(\langle \phi,v \rangle\) from \(\Phi\) consists of two key ideas.
\end{note}

\begin{note}[\requ{2}: First idea]
  First, if \(\phi\) has value \(v\), then the agent may conclude \(\langle \psi,v' \rangle\) from some pool of premises \(\Psi\) without first concluding \(\phi\) has value \(v\), captured by~\ref{notion:overview:requ:pool}.
  For, if \ref{notion:overview:requ:pool:incl} -- \ref{notion:overview:requ:pool:ind} hold then:
  \begin{itemize}
  \item
    By~\ref{notion:overview:requ:pool:incl}, the agent need not reason from \(\langle \phi,v \rangle\) as a premise to conclude \(\langle \psi,v' \rangle\).
  \item
    By~\ref{notion:overview:requ:pool:int}, the agent need to establish \(\langle \phi,v \rangle\) as a sub-conclusion when reasoning from the relevant pool of premises \(\Psi\).
  \item
    And, by~\ref{notion:overview:requ:pool:ind}, the agent need not have already concluded \(\langle \phi,v \rangle\) in order to appeal to any of the proposition-value pairs in the relevant pool of premises \(\Psi\).
  \end{itemize}
  In short, the agent may conclude \(\langle \psi,v \rangle\) from \(\Psi\) independently of concluding \(\langle \phi,v \rangle\) from \(\Phi\).
\end{note}

\begin{note}[\requ{2}: Second idea]
  Second, concluding \(\langle \psi,v' \rangle\) from \(\Psi\) would serve as a partial check on whether the agent may reason to a conclusion other than \(\langle \phi,v \rangle\), captured by~\ref{notion:overview:requ:nPsi-nPhi}.

  Concluding \(\langle \psi,v' \rangle\) from \(\Psi\) is a check.
  For, if the agent were to fail to conclude \(\langle \psi,v' \rangle\) from \(\Psi\) then, from the perspective of the agent's epistemic state, the agent would not conclude \(\langle \phi,v \rangle\) from \(\Phi\).
  Hence, contraposing, the agent would conclude \(\langle \phi,v \rangle\) from \(\Phi\) only if the agent would conclude \(\langle \psi,v' \rangle\) from \(\Psi\).
  However, the check is partial, as it need not be the case that the agent would conclude \(\langle \psi,v' \rangle\) from \(\Psi\) only if the agent \(\langle \phi,v \rangle\) from \(\Phi\).
  Therefore, failing to conclude \(\langle \psi,v' \rangle\) from \(\Psi\) may block concluding \(\langle \phi,v \rangle\) (from the perspective of the agent's epistemic state) though concluding \(\langle \psi,v' \rangle\) from \(\Psi\) need not ensure that the agent would conclude \(\langle \phi,v \rangle\).

  Now, \ref{notion:overview:requ:nPsi-nPhi} contains a slight subtlety.
  For, from~\autoref{assu:conc:d-free}, an agent may conclude various proposition-value pairs from some instance of reasoning without explicit recognition.
  Therefore,~\ref{notion:overview:requ:nPsi-nPhi} does not state that the agent may fail to conclude \(\langle \phi,v \rangle\) from \(\Phi\).
  Rather, \ref{notion:overview:requ:nPsi-nPhi} holds that from the perspective of the agent's epistemic state, the agent may fail to conclude \(\langle \phi,v \rangle\) from \(\Phi\).
  Again, we tread a fine line between the role of an agent's epistemic state and the role of the agent's \stance{}.
  The role of an agent's epistemic state determines whether \(\langle \psi,v \rangle\) is a \requ{} of concluding \(\langle \phi,v \rangle\) from \(\Phi\).
  And, an agent's epistemic state may determine whether an agent concludes \(\langle \chi,v'' \rangle\) when concluding \(\langle \phi,v \rangle\).%
  \footnote{
    Recall the above discussion of \(\langle \phi,v \rangle\) \indicatePr{} \(\langle \chi,v' \rangle\) in relation to~\autoref{assu:conc:d-free}.
  }
  Therefore, the agent's epistemic state --- the agent's perspective on how things are --- is key.
  However, the agent's \stance{} is unimportant.
  Whether an agent has concluded \(\langle \phi,v \rangle\), or whether \(\langle \psi,v' \rangle\) is a \requ{} is not a question of whether the agent recognises the have concluded \(\langle \phi,v \rangle\) or recognises \(\langle \psi,v' \rangle\) is a \requ{}.

  Combining these two ideas, intuitively, \(\langle \psi,v' \rangle\) is a \requ{} of concluding \(\langle \phi,v \rangle\) just in case there is some pool of premises \(\Psi\) such that determining whether the agent would conclude \(\langle \psi,v \rangle\) is an independent partial check on whether the agent may reason to a conclusion other than \(\langle \phi,v \rangle\).
\end{note}

\begin{note}
  We will expand on the \requ{1}, and provide a number additional \illu{1} below.
  First, we refine our intuitive understanding of \csN{} into an idea.
\end{note}

\begin{note}
  \begin{idea}[\csN{2}]
    \label{idea:CS:overview}
    An agent \csV{} for \(\langle \phi,v \rangle\) when concluding \(\langle \phi,v \rangle\) from \(\Phi\) \emph{only if}:
    \begin{enumerate}
    \item
      \label{idea:CS:overview:requ}
      For any proposition-value pair \(\langle \psi, v' \rangle\) which is a \requ{} of concluding \(\langle \phi,v \rangle\) from \(\Phi\) either:
      \begin{enumerate}
      \item
        \label{idea:CS:overview:requ-sat:Past}
        The agent has concluded that they would conclude \(\langle \psi, v' \rangle\) from the relevant pool of premises \(\Psi\).
      \item
        \label{idea:CS:overview:requ-sat:Pres}
        In concluding \(\langle \phi,v \rangle\) the agent \emph{also} concludes that they would conclude \(\langle \psi, v' \rangle\) from the relevant pool of premises \(\Psi\).
      \end{enumerate}
    \end{enumerate}
    \vspace{-\baselineskip}
  \end{idea}
\end{note}

\begin{note}
  \Autoref{idea:CS:overview} states that any \requ{} of concluding \(\langle \phi,v \rangle\) has been satisfied in one of two ways.

  First:
  The agent having concluded that they would conclude \(\langle \psi,v' \rangle\) from the relevant pool of premises prior to concluding \(\langle \phi,v \rangle\).

  Second:
  The agent simultaneously concluding that they would conclude \(\langle \psi, v' \rangle\) from the relevant pool of premises \(\Phi\) while concluding \(\langle \phi,v \rangle\).

  Observe, neither option (immediately, at least) requires the agent concludes \(\langle \psi,v' \rangle\) from the relevant pool of premises.
  What~\autoref{idea:CS:overview} states is weaker, concerning only whether or not they (the agent) would conclude \(\langle \psi,v' \rangle\) from the relevant pool of premises.

  For example, counterexample for some formula of propositional logic.
  Constructed a truth table.
  Identified a line.
  If counterexample, then line makes any tautology of propositional logic true.
  And, do not need to appeal to the line being a counterexample to the relevant formula to do so.
  However, do not need to reason from line to any recognised tautology.
  Conclude that if were to reason, tautology would be true.\nolinebreak
  \footnote{
    Though, it is also plausible that conclude the provides a model, and in doing so conclude that all tautologies hold.
    Any particular tautology need not be a distinct conclusion.
    Recall~\autoref{assu:conc:d-free}.
  }
\end{note}

\begin{note}
  Still, the second option may be of some concern.
  Note, however, this is only an option.
  If you go for negative resolution, then ruled out, but then first option alone.
  However, this is not obviously required.

  Propositional logic.
  These premises allow to conclude two things.
  Then, conclusion that \(\phi \land \psi\) is simultaneously a conclusion that \(\phi\) and that \(\psi\).

  Or, apples in a bag.
  Five.
  Well, could do at least four, three, etc.
  Conclude at the same time.
\end{note}

\begin{note}[\requ{1}]
  Now, the primary feature of \(\langle \psi,v' \rangle\) being a \requ{} of concluding \(\langle \phi,v \rangle\) is the option of concluding \(\langle \psi,v' \rangle\) from some pool of premises.
  However, the majority of the content of \label{idea:CS:overview:requ} concerns the relationship between \(\langle \phi,v \rangle\) and the pool of premises.
  In short, \autoref{notion:overview:requ:pool:incl},~\autoref{notion:overview:requ:pool:int}, and~\autoref{notion:overview:requ:pool:ind} combine to ensure that the agent has the option of concluding \(\langle \psi,v' \rangle\) independently of whether they have concluded \(\langle \phi,v \rangle\).
\end{note}


\begin{note}[Expanding pool constraints]
  To~\autoref{notion:overview:requ:pool} of~\autoref{notion:overview:requ} the following clause may also be added:
  \begin{enumerate}[label=]
  \item
    \begin{enumerate}[label=]
    \item
      \begin{enumerate}[label=\roman*., ref=(\roman*), resume*=csIdeaCounter]
        \setcounter{enumiii}{3}
      \item
        \label{notion:overview:requ:pool:method}
        Concluding \(\langle  \psi,v' \rangle\) from \(\Phi\) involves the same general method the agent would use to conclude \(\langle \phi,v \rangle\).
      \end{enumerate}
    \end{enumerate}
  \end{enumerate}
  We omit~\autoref{notion:overview:requ:pool:method} from the idea of \csN{} for two (related) reasons.
  First, it is not clear what `the same general method' amounts to in details.
  Second, avoiding questions about method affords flexibility when providing \illu{1} of \csN{}.
  However,~\autoref{notion:overview:requ:pool:method} may be imposed with no loss to the role of \csN{} in the overall argument.
\end{note}

\begin{note}[\requ{3}]
  So,~\autoref{idea:CS:overview}, no unsatisfied \requ{1}.

  Keys.
  Lost keys, but until I've checked everywhere, might find them.
  So, conclude lost, then not in coat pocket.
  Well, not going to conclude lost at least until I've checked the pocket.
\end{note}

\begin{note}[Importance of \csN{}]
  \csN{} is key.
  Argument for positive resolution to~\autoref{issue:Main} largely rests on \csN{}.
  Our immediate goal is to motivate interest in \csN{}.

  However, briefly note that a few things.

  First, agent's reasoning.
  At issue is whether the agent may reason to a different conclusion.
  There's nothing that would lead me elsewhere.

  Second, agent's reasoning.
  Independent of whether \(\phi\) has value \(v\), \(\psi\) has value \(v'\), or any of the premises.
  Need not be the case that satisfaction amounts to anything substantial.
  No clause for justification, etc.

  Third, way.
  Getting \(\langle \phi, v \rangle\) from premises.
  Inclined to think that \csN{} is more general.
  From \(\langle \phi, v \rangle\).
  However, some concerns.
  See this after.

  Fourth, `would'.
  Competence, rather than performance.
\end{note}

\begin{note}[Building to \illu{1}]
  Consider a handful of cases.
\end{note}

\begin{note}[Trivial]
  No independent checks.
\end{note}

\begin{note}[Pruning]
  NTP server.
  Not from clock, but from NTP.
  Here, no way to check this.
\end{note}


\begin{note}[Keys case]
  Without \(\langle \phi,v \rangle\) is important.
  Keys.
  In hand.
  Well, then conclude not in drawer, which haven't yet looked.

  There are two options.
  First, \requ{}.
  Hence, conclude that if look in drawer then would not find.
  This seems plausible.

  Second, not a \requ{}.
  Indeed, despite intuitiveness of first option, it is the second that is the case.
  \autoref{notion:overview:requ:nPsi-nPhi} is important.
  If looked in drawer and found a pair of keys.
  Well, this would not prevent you from concluding you're found your keys.
  Rather, confusion over a having a pair of keys.
  This would not show a mistake in reasoning, but rather a mistake your understanding of how things are.
  And, revised, there is no telling what conclusions you may reach.

  Now, there may be some analogue to \csN{} which entertains revised information.
  Indeed, conditional belief.
  However, \csN{} is about reasoning.

  The original keys example, then, is limited to reasoning about a possible location.
  If location, then would not clearly conclude lost.
  At issue was not finding and then concluding not lost.

  In this respect, reasoning after finding keys is significantly different.
\end{note}

\begin{note}[\illu{3}]
  Begin with a collection of \illu{1} with respect to a simple instance of reasoning: concluding it is midday from the positions of the hands on a clock.


  For example, with clocks.

  No option.
  One authoritative timepiece.

  Also, testimony that some mathematical theorem is true.

  No distinct premises.
  Reason from position of hands on clock to conclude that it is midday.
  All other clocks are kept in sync via an NTP time server, so no independent check.

  Also, two calculators.
  Same internals.

  Multiple checks.
  Two clocks.
  Wristwatch and clock on the library wall.

  Also, milk.
  Safe to drink from sell by date.
  Smell if it's off.

  Coming to the party.
  Well, then decided not to go to the beach.
  So, check this before.
\end{note}

\begin{note}
  Finally, ability.
  Ability enters indirectly.
  First, certain cases of reasoning.
  These, then, have some generality.
  Concluding \(\phi\) has value \(v\) involves general reasoning.
  Chess, sudoku, etc.
  It's no accident.
  Still, \csVed{}.

  Difficulty.
  Independent check.
  Would not conclude \(\phi\) has value \(v\) if were to fail to conclude \(\psi\) has value \(v'\).
  Some problem in the general reasoning.

  Now, applying previous observation.
  Either \csVed{} or witness.
  In these cases, no witnessing.
  So, \csVed{}.

  However, always a check on whether one has the general ability.
\end{note}

\begin{note}
  Reasoning, \support{}, would not reason to a different conclusion.

  Specifically, \requ{} of some conclusion.
  So long as conclusion, then it is possible to reason about whether \(\psi\) has value \(v'\), and unless conclude \(\psi\) has value \(v'\), would not conclude \(\phi\) has value \(v\).

  Intuitively, \requ{} as an independent check on the reasoning.
  If don't hold \(\psi\) from premises, then question about whether \(\phi\).

  Claiming support, necessary condition is satisfying all \requ{1}.
  Claiming support, then, is weaker than having support.
  Restricted to whether conclusion of reasoning would introduce a \requ{}.
  And, may be further restricted without impact to the tension we will develop to whether the conclusion would `clearly' introduce a \requ{}.
\end{note}

\begin{note}
  Now, consequence from \ideaCS{} and \ESU{},
  If potential witnessing event, then either concluded previously, or witness reasoning.
\end{note}

\begin{note}[Kettle logic]
  Well, it's true that the person must be acquitted, but at the same time, the person is going to have a hard time explaining how, for example, he brewed coffee for a week.

  Still, highlights what the neighbour needs to conclude.
  Did borrow.
  Was not lent damaged
  And, was returned damaged.

  Here, burden of argument.
  However, we're not interested in whether the neighbour would convince, but whether the neighbour would reach a different conclusion if they were first to reason about one of the three before concluding that the kettle was returned damaged.
\end{note}

\section{Resolutions to~\autoref{issue:Main}, refined}
\label{sec:issue-refined}

\begin{note}[Restating the issue]
  We began by expressing interest in the following issue:
  \vspace{-\baselineskip}
  \begin{quote}
    \issueMain*
  \end{quote}
  In \autoref{sec:outline:concluding} we clarified our understanding of concluding, and in~\autoref{sec:overview:csn2} we introduced \csN{0}.

  With a sufficient understanding of concluding and \csN{0} in hand, we provide a restatement of the two resolutions to~\autoref{issue:Main} of interest.

  The negative resolution denies there are such cases while the positive holds there are such cases.
  We term these resolutions \ESU{} and \EAS{}, respectively.

  We being with a statement of the two resolutions, before turning to discuss both \ESU{} and \EAS{} in some detail.
\end{note}

\subsection{The resolutions refined}
\label{sec:two-resolutions}

\begin{note}
  The two resolutions to~\autoref{issue:Main}, refined with respect to \csN{0} are as follows:
\end{note}

\begin{note}
  \begin{restatable}[\ESU{0} --- \ESU{}]{target}{targetESU}
    \label{denied-claim}
    An agent concluding \(\langle \phi,v \rangle\) is an instance of claiming support \emph{only if}:

    For any pool of proposition-value pairs \(\langle \phi_{1},v_{1} \rangle,\dots,\langle \phi_{k},v_{k} \rangle\) appealed to as premises, the agent \emph{has} witnessed reasoning which concludes \(\phi\) has value \(v\) from \(\phi_{i}\) having values \(v_{i}\).\nolinebreak
    \footnote{More generally, the agent has witnessed reasoning whose conclusion \emph{indicates} \(\phi\) has value \(v\).}
  \end{restatable}
\end{note}

\begin{note}
  \begin{restatable}[\EAS{0} --- \EAS{}]{goal}{goalEAS}
    \label{prop:EAS}
    There are instances of reasoning in which an agent concludes \(\langle \phi,v \rangle\) from some pool of proposition-value pairs \(\langle \phi_{1},v_{1} \rangle,\dots,\langle \phi_{k},v_{k} \rangle\) as premises \emph{without} witnessing reasoning from \(\langle \phi_{1},v_{1} \rangle,\dots,\langle \phi_{k},v_{k} \rangle\) to \(\langle \phi,v \rangle\) is an instance of claiming support.
  \end{restatable}
\end{note}

\begin{note}
  \ESU{} and \EAS{} are, then, fairly straightforward.
  The key difference is that \ESU{} does quantifies only over instances of concluding which are also instances of \csVImp{0}.
  And, likewise, \EAS{} only requires instances of reasoning in which concluding coincides with \csVImp{0}.
\end{note}

\begin{note}
  The relationship between \ESU{} and \EAS{} and~\autoref{issue:Main} is simple.

  Both \EAS{} and \ESU{} focus on instances of concluding which are also instances of \csVImp{0}.
  Therefore:
  \begin{itemize}
  \item
    \EAS{} entails a positive resolution to~\autoref{issue:Main}.
  \item
    A negative resolution to~\autoref{issue:Main} entails \ESU{}.
  \end{itemize}
  Naturally, we are interested in instances of concluding which are also instances of \csVImp{0} because we believe this is where tension arises.
  And, so long as there is sufficient tension to motivate \EAS{}, then by the entailment noted, we will also motivate the positive resolution to~\autoref{issue:Main}.

  However, from a dialectical perspective, the refinement may also ease rejection of the positive resolution we motivate.
  For,~\autoref{issue:Main} concerns all instances of concluding, while \ESU{} only concerns instances of concluding which are also instances of \csVImp{}.
  Hence, rejection of the positive resolution we aim to motivate need not commit one to a negative resolution to~\autoref{issue:Main} in general.
  Indeed, as we have placed few assumptions on concluding in general one may wary of extending any intuition for a negative resolution to~\autoref{issue:Main} to all instances of concluding.
  Still, I hope the discussion of \csN{0} has provided sufficient information to evaluate~\autoref{issue:Main} when restricted to \csN{0}.
  And I hope the additional discussion will help to clarify both \ESU{} and \EAS{}.

  {
    \color{red}
    Initially, just on what follows or does not follow from statement of \ESU{} and \EAS{}.
    Then, generalise to two ways of concluding.
    There, focus on the mechanics of \ESU{} and, in particular, \EAS{}.
  }
\end{note}

\paragraph*{\ESU{}}

\begin{note}
  \ESU{} holds that whenever an agent concluding \(\langle \phi,v \rangle\) is an instance of \csN{0}, the agent has always witnessed reasoning from some pool of premises.

  Our present goal is provide a handful of clarifying remarks.
\end{note}

\begin{note}
  The main clarifying remark concerns the interpretation of `has' in the statement of \ESU{}.
  We intend the relevant instance of `has' to cover all past reasoning.
  Therefore, \ESU{} does not require the agent to conclude \(\langle \phi,v \rangle\) from inputs to the present instance of reasoning which culminates with \(\langle \phi,v \rangle\).

  We will develop this remark in detail when we consider two types of reasoning in~\autoref{sec:overview:two-types-reasoning}.
  For the moment the gist is simple to state:

  Suppose at some point in the past an agent concluded \(\langle \phi,v \rangle\) from some premises \(\Phi\) by witnessing reasoning from \(\Phi\) to \(\langle \phi,v \rangle\) (and in doing so \csVed{} for \(\langle \phi,v \rangle\)).
  Then, at the present moment an agent may (re-)conclude \(\langle \phi,v \rangle\) from \(\Phi\) without witnessing reasoning \emph{in the present} from \(\Phi\) to \(\langle \phi,v \rangle\).
  For, the constraint imposed by \ESU{} has already been satisfied by witnessing reasoning from \(\Phi\) to \(\langle \phi,v \rangle\) in the past.

  For example, suppose I concluded that Riga is the capital of Latvia by studying a map.
  Now, at present, I am asked by a fried what the capital of Latvia is.
  I takes me a moment, but I recall from memory studying the map and concluding that Riga is the capital of Latvia.
  Hence, I (re-)conclude that Riga is the capital of Latvia, and I do so from the relevant premises involved when studying the map.

  Of course, \ESU{} is a necessary, rather than sufficient condition, and so further constraints may be added to rule out the present type of case.
  However, I take this observation to highlight that \ESU{} is a relatively weak assumption, focused solely on whether an agent has witnessed the relevant reasoning, and independent of when the agent witnessed the relevant reasoning.%
  \footnote{
    \label{ft:weak-esu}
    Indeed, \ESU{} may be weakened further to allow the agent to witness the relevant reasoning at some point in the future, and to quantify over all agents.
    Roughly, this weakened variant of \ESU{} states:
    \begin{quote}
      An agent concluding \(\langle \phi,v \rangle\) is an instance of claiming support \emph{only if}:

      For any pool of proposition-value pairs \(\langle \chi_{1},v_{1} \rangle,\dots,\langle \chi_{k},v_{k} \rangle\) appealed to as premises, \textbf{some} agent \emph{has} witnessed \textbf{or will witness} reasoning which concludes \(\phi\) has value \(v\) from \(\chi_{i}\) having values \(v_{i}\).
    \end{quote}
    Where the two adjustments are placed in bold.

    We favour \ESU{} over this weakened version of \ESU{} in order to keep the complexity of the issue and resolutions low.
    Indeed, though my intuitions regarding \ESU{} are clear (I consider \ESU{} to be intuitive --- though in tension with other ideas I find more intuitive), I am unsure on how \ESU{} generalises to other agent and future instances of reasoning.
    {
      \color{red}
      Simple example, know that some logic is decidable.
      Take a relatively simple fragment.
      Then, either proof or a countermodel.
      Same idea holds for tension, but there's no guarantee that anyone has concluded\dots
    }
  }
\end{note}

\begin{note}
  Finally, recall that~\autoref{assu:conc:d-free} implies that in witnessing reasoning which concludes \(\langle \phi,v \rangle\), an agent need not recognise they are concluding \(\langle \phi,v \rangle\).

  We have already provided an argument for this implication following the introduction of~\autoref{assu:conc:d-free} in~\autoref{sec:outline:concluding}.
  Still, with \ESU{} in hand this observation may be worth re-examining.
  For, again, the observation highlights the relative weakness of \ESU{}.
  If an agent has concluding \(\langle \phi,v \rangle\), then a proponent of \ESU{} is not committed to the agent explicitly concluding \(\langle \phi,v \rangle\) from the relevant instance of witnessing.
  Rather, the proponent of \ESU{} is committed only to the agent witnessing some reasoning and for that reasoning to involve concluding \(\langle \phi,v \rangle\).

  To \illu{}, recall that we say \(\langle \phi,v \rangle\) \indicatePr{} \(\langle \psi,v' \rangle\), just in case \(\phi\) has value \(v\) \emph{only if} \(\psi\) has value \(v'\) from the perspective of the agent's epistemic state.
  Hence, so long as an agent has witnessed reasoning which concludes \(\langle \phi,v \rangle\), the same reasoning may involve concluding \(\langle \psi,v' \rangle\).
  And, the inclusion may be straightforwardly explained by the observation that the premises from which the agent concluded \(\langle \phi,v \rangle\) are also sufficient to establish \(\langle \psi,v' \rangle\).

  Of course, as~\ESU{} is only a necessary condition, it need not be the case that by witnessing reasoning which concludes \(\langle \phi,v \rangle\) an agent also concludes \(\langle \psi,v' \rangle\).
  Still, as \ESU{} is consistent with endorsing the above, we do not have the option of establishing tension via reflection on how the relevant agent perceives any instance of witnessing.
\end{note}

\paragraph*{\EAS{}}

\begin{note}
  Following the main clarifying remark regarding \ESU{}, the interpretation of `without' in \EAS{} also covers past reasoning.%
  \footnote{
    And, indeed, if the weakened variant of \ESU{} from~\autoref{ft:weak-esu} is adopted, any future reasoning by any agent.
  }
  Hence, the instances of concluding that are of interest with respect to \EAS{} may be termed `novel' conclusions.
\end{note}

\begin{note}
  Some reasoning.
  Some inputs to that reasoning.
  Following above, also need to ensure that relevant instances may not be analysed as being \indicateVed{}.

  Again, do not need to endorse, but \ESU{} is left as weak as possible.
  Hence, concern.

  So, not the case that \(\langle \phi,v \rangle\) is \indicateVed{} by some previous reasoning or some intermediary conclusion of present reasoning.

  Abstract, but consequences of the quick argument.
  No constraints on premises.
  And, now we have developed both concluding and the negative resolution to~\autoref{issue:Main} --- i.e.~\ESU{} --- in detail, we see the flexibility for one interested in defending \ESU{}.
  To motivate \EAS{}, need instances in which concluding \(\langle \phi,v \rangle\) by witnessing reasoning from some pool of premises is viable.
\end{note}

\begin{note}[Uninformative]
  Just an existential.
  This does not tell us anything about how the agent concludes \(\langle \phi,v \rangle\).

  This is for the following section, to which we now turn.
\end{note}

\subsection{Two types of reasoning}
\label{sec:overview:two-types-reasoning}

\begin{note}[Two types]
  Here, expand on \ESU{} and \EAS{}.
  \ESU{} is, I expect, fairly clear.
  \EAS{}, does not provide detail regarding the reasoning involved.

  From assumptions regarding concluding, some premises.
  But, this does not tell us much.
  In part, question of when.
  Turn to this when developing tension.
  Initial question, though, of general structure.

  In this section, broad idea.
  \adA{} and \adB{}.
  \adA{} pairs with \USE{}.
  \adB{} the type of reasoning we're interested in for \EAS{}.

  Of course, if \ESU{}, then no instances of \adB{}.
  However, pairs of upshots.
  A view of the horizon, where argument for \EAS{} may expand.
  Insight into how tension will be developed, \csN{} and information about possibility of reasoning from premises to some conclusion.
\end{note}

\paragraph*{The first type of reasoning: \adA{}}

\begin{note}
  \begin{restatable}[\adA{}]{definition}{defADA}
    \label{AR:adA}
    \label{def:adA}
    \vAgent{} concludes \(\langle \phi,v \rangle\) from \(\Phi\) by `\adA{}' if:
    \begin{enumerate}[label=\textsf{S:\arabic*}., ref=(\textsf{S}:\arabic*)]
    \item
      \label{def:adA:psi}
      \vAgent{} concludes \(\langle \phi,v \rangle\) by witnessing reasoning from some pool of premises \(\langle \phi_{1},v_{1} \rangle, \dots, \langle \phi_{k},v_{k} \rangle, \dots\)
    \item
      \(\Phi\) is the collection of all and only the premises \(\langle \phi_{i},v_{i} \rangle\).
    \end{enumerate}
    \vspace{-\baselineskip}
  \end{restatable}

  \adA{} is straightforward.

  As before, concern may be raised about what the relevant premises are, and whether the relevant agent identifies those premises as premises.
  However, granting that an agent always concludes from some collection of premises, the relevant collection exists.

  The restriction to the \emph{exact} collection of premises the agent reasons from is for convenience.
  Nothing in particular hangs on this distinction, but equally nothing much is gained by allowing the inclusion of redundant proposition-value pairs.
\end{note}

\begin{note}[\illu{1}]
  Time from positions of hands on a clock and understanding of how time is represented by such a clock.

  Whether to make a bet from tolerance for risk, distribution of cards in a pack, cards in hand, and rules of the game.
\end{note}

\begin{note}
  \adA{} does not outline a specific way of reasoning.
  Deductive, inductive, etc.\
\end{note}

\begin{note}
  \phantlabel{abstract-adA}
  Basic (abstract) instance of \adA{}:

  {
    \small
    \begin{enumerate}[label=\arabic*., ref=\arabic*]
    \item\label{def:adA:ex:C:Cp} I have \csVed{0} \(\phi\) has value \(v\).
    \item\label{def:adA:ex:C:p} So, \(\phi\) has value \(v\). \hfill(From~\ref{def:adA:ex:C:Cp})
    \item\label{def:adA:ex:C:Cps} Likewise, I have \csVed{0} \(\psi\) has value \(v'\) when \(\phi\) has value \(v\).
    \item\label{def:adA:ex:C:ps} So, \(\psi\) has value \(v'\) when \(\phi\) has value \(v\). \hfill(From~\ref{def:adA:ex:C:Cps})
    \item\label{def:adA:ex:C:T} If \(\psi\) has value \(v'\) when \(\phi\) has value \(v\) and \(\phi\) has value \(v\), then it must be the case that \(\psi\) has value \(v'\). \hfill (From understanding of `if\dots then\dots')
    \item\label{def:adA:ex:C:s} Hence, \(\psi\) has value \(v'\).\newline
      \mbox{}\hfill (From \ref{def:adA:ex:C:p},~\ref{def:adA:ex:C:ps}~and~\ref{def:adA:ex:C:T})
    \item Therefore, I conclude \(\psi\) has value \(v'\). \hfill (From \ref{def:adA:ex:C:Cp} -- \ref{def:adA:ex:C:s})
    \end{enumerate}
  }
  From this reasoning, two clear premises.
  \(\langle CS(\langle \phi,v \rangle), \top \rangle\) and \(\langle CS(\langle \langle \phi,v \rangle \Rightarrow \langle \psi,v' \rangle, \top \rangle), \top \rangle\).
  Witness reasoning from these premises.
  The reasoning is verbose, premises are that the agent has \csVed{0}.
  Grant availability of factive inference from \csVed{0}.

  Indeed, plausibly an instance of \csVImp{}.
  For, \csVed{} for premises, an no other conclusion.
  Of course, in practice, check from \requ{}.

  % (Consider parallel reasoning with knowledge.%
  % \footnote{The parallel reasoning in full:
  %   \begin{enumerate}[label=\arabic*., ref=\arabic*]
  %   \item\label{def:adA:ex:K:Kp} I know \(\phi\) has value \(v\).
  %   \item\label{def:adA:ex:K:p} So, \(\phi\) has value \(v\). \hfill (From~\ref{def:adA:ex:K:Kp})
  %   \item\label{def:adA:ex:K:Kps} I know \(\psi\) has value \(v'\) when \(\phi\) has value \(v\).
  %   \item\label{def:adA:ex:K:ps} So, \(\psi\) has value \(v'\) when \(\phi\) has value \(v\). \hfill(From~\ref{def:adA:ex:K:Kps})
  %   \item\label{def:adA:ex:K:T} If \(\psi\) has value \(v'\) when \(\phi\) has value \(v\) and \(\phi\) has value \(v\), then it must be the case that \(\psi\) has value \(v'\). \hfill (From understanding of `if\dots then\dots')
  %   \item\label{def:adA:ex:K:s} Hence, \(\psi\) has value \(v'\). \hfill (From \ref{def:adA:ex:C:p},~\ref{def:adA:ex:C:ps}~and~\ref{def:adA:ex:C:T})
  %   \item So, I know \(\psi\) has value \(v'\) as \(\psi\) having value \(v'\) follows from~(\ref{def:adA:ex:K:Kp}) and~(\ref{def:adA:ex:K:Kps}).
  %     \mbox{}\hfill (From \ref{def:adA:ex:K:Kp} -- \ref{def:adA:ex:K:s})
  %   \end{enumerate}
  % }%
  % )
\end{note}


\paragraph*{The second type of reasoning: \adB{}}

\begin{note}[Turning to \adB{}]
  We now turn to the second type of reasoning: `\adB{}'.

  We begin with a definition of \adB{}.
  However, our attention will quickly turn to a pair of helper definitions which relate some proposition-value pair we identify as an `\itp{}' to some other proposition-value pair and pool of premises.
\end{note}

\begin{note}
  \begin{restatable}[\adB{}]{definition}{defADB}
    \label{def:adB}
    \vAgent{} concludes \(\langle \phi,v \rangle\) from \(\Phi\) by `\adB{}' if:
    \begin{enumerate}[label=\textsf{I}:\arabic*., ref=(\textsf{I}:\arabic*)]
    \item
      \label{def:adB:itp}
      \vAgent{} has concluded \(\langle \mu,v \rangle\) and  \(\langle \mu,v \rangle\) is either:
      \begin{enumerate}
      \item
        \label{def:adB:itp:between}
        An \itp{} \emph{between} \(\langle \phi,v \rangle\) and \(\Phi\), or:
      \item
        \label{def:adB:itp:for}
        An \itp{} \emph{for} \(\langle \phi,v \rangle\), with \(\Phi\) as the relevant pool of premises.
      \end{enumerate}
    \item
      \label{def:adB:conclude}
      \vAgent{} concludes \(\langle \phi,v \rangle\) by appeal to the premises \(\langle \phi_{1},v_{1} \rangle, \dots, \langle \phi_{k},v \rangle\) from the pool of premises \(\Phi\) via the possibility of witnessing the relevant reasoning from \(\langle \mu,v \rangle\).
    \end{enumerate}
    \vspace{-\baselineskip}
  \end{restatable}
\end{note}

\begin{note}
  Without definitions of what an \itp{0} between \(\langle \phi,v \rangle\) and \(\Phi\) is, or what an \itp{0} for \(\langle \phi,v \rangle\) is,~\autoref{def:adB} is incomplete.
  We will shortly turn to the relevant helper definitions.

  Though, working backwards from~\autoref{def:adB} gives a hint.
  An \itp{} should contain information that it is possible for the agent to witness reasoning that conclude \(\langle \phi,v \rangle\) from some pool of premises \(\Phi\).
\end{note}

\begin{note}
  Still, before turning to the pair of helper definitions, let me stress a key aspect of~\autoref{def:adB}:
  From~\ref{def:adB:conclude}, the agent concludes \(\langle \phi,v \rangle\) from \(\Phi\) and \(\Phi\) alone (without witnessing the relevant reasoning).
  The agent does not conclude \(\langle \phi,v \rangle\) from \(\Phi\) and \(\langle \mu,v \rangle\).
  From the perspective of defining \adB{}, the latter option may be included, but the possibility of the former will be important when we turn to tension.
\end{note}

\paragraph{Contrast}

\begin{note}
  The key difference between \adA{} and \adB{}:
  \begin{itemize}
  \item \adA{} involves the agent appealing to \(\phi\) in order to claim support for \(\psi\), while
  \item \adB{} does not involve the agent appealing to \(\psi\) to claim support for \(\psi\).
    Instead, the role of \(\phi\) is to highlight \(\rho_{1},\dots,\rho_{k}\) and the agent appeals to propositions \(\rho_{1},\dots,\rho_{k}\) to claim support for \(\psi\).
  \end{itemize}

  For the definition to be satisfied, \(\phi\) needs only be involved to the extent that it provides the link.
  Hence, \(\phi\) is not irrelevant.
  Still, the agent does not appeal to \(\phi\).
\end{note}

\paragraph*{\adB{}: Helper definitions}

\begin{note}[]
  We briefly noted, working backwards from~\autoref{def:adB}, that an \itp{} should contain information that it is possible for the agent to witness reasoning that conclude \(\langle \phi,v \rangle\) from some pool of premises \(\Phi\).
  We now detail what an \itp{0}.
  Or, rather, what \itp{1} are.

  There are two cases.
  First, an \itp{0} \emph{between} \(\langle \phi,v \rangle\) and \(\Phi\).
  Second, an \itp{0} \emph{for} \(\langle \phi,v \rangle\).
  The distinction between these cases is whether the relevant \itp{0} identifies a particular pool of premises.

  In practice, we will blur the distinction, but from a definitional perspective the latter is best seen as a generalisation of the former.
\end{note}

\subparagraph*{An \itp{0} between \(\langle \phi,v \rangle\) and \(\Phi\)}

\begin{note}[\itp{} between]
  \begin{definition}[An \itp{0} between \(\langle \phi,v \rangle\) and \(\Phi\) \hfill \named{I.b}]
    \label{def:itp:b}
    \(\langle \mu,v \rangle\) is an \itp{} \emph{between} \(\langle \phi,v \rangle\) and \(\Phi\) if and only if:

    \begin{enumerate}[label=\arabic*., ref=\named{\textsf{I.b}:\arabic*}]
    \item
      \label{def:itp:b:pR}
      \(\mu\) having value \(v\) ensures:
      \begin{itemize}
      \item
        It is possible for \vAgent{} to conclude \(\phi\) has value \(v\) by witnessing reasoning from \(\Phi\) to \(\langle \phi,v \rangle\), given \vAgent{}'s present epistemic state.
      \end{itemize}
    \item
      \label{def:itp:b:distinct}
      \(\langle \mu,v \rangle\) is not equivalent to any \(\langle \phi_{i},v_{i} \rangle\), given \vAgent{}'s present epistemic state.
    \end{enumerate}
    \vspace{-\baselineskip}
  \end{definition}
\end{note}

\begin{note}[Plan]
  The definition of an \itp{} between \(\langle \phi,v \rangle\) and \(\Phi\) consists of two components:~\ref{def:itp:b:pR} and~\ref{def:itp:b:distinct}.

  \ref{def:itp:b:pR} is the core of the definition, while~\ref{def:itp:b:distinct} narrows the definition to cases of interest.

  We begin by expanding on~\ref{def:itp:b:pR}, and then motivate the restriction given by~\ref{def:itp:b:distinct}.
\end{note}

\begin{note}[Expanding on~\ref{def:itp:b:pR}]
  Intuitively, think of an \itp{} between \(\langle \phi,v \rangle\) and \(\Phi\) as a particular kind of conditional of the (rough) form `if \(\Phi\) then \(\langle \phi,v \rangle\)'.

  Indeed, an `\emph{if} \dots \emph{then} \dots' statement between \(\Phi\) and \(\langle \phi,v \rangle\) may be constructed from any \itp{} between \(\langle \phi,v \rangle\) and \(\Phi\).

  For, if it is possible for an agent to conclude \(\phi\) has value \(v\) by witnessing reasoning from \(\Phi\) to \(\langle \phi,v \rangle\), then (from the agent's perspective at least), \(\langle \phi,v \rangle\) whenever \(\langle \phi_{i},v_{i} \rangle\) for each \(\langle \phi_{i},v_{i} \rangle\) in \(\Phi\).
  So, if every proposition \(\phi_{i}\) in \(\Phi\) has it's respective value \(v_{i}\), then \(\phi\) also has value \(v\).
  Or, more colloquially, if \(\Phi\) then \(\langle \phi,v \rangle\).

  However, an \itp{} between \(\langle \phi,v \rangle\) and \(\Phi\) is stronger than `if \(\Phi\) then \(\langle \phi,v \rangle\)'.
  For, not only is it the case that \(\langle \phi,v \rangle\) whenever \(\langle \phi_{i},v_{i} \rangle\) for each \(\langle \phi_{i},v_{i} \rangle\) in \(\Phi\), but in addition it is possible for the agent to conclude \(\langle \phi,v \rangle\) from \(\Phi\) given the agent's present epistemic state.

  Naturally, the possibility for the agent to conclude \(\langle \phi,v \rangle\) from \(\Phi\) goes beyond a plain conditional between \(\langle \phi,v \rangle\) and \(\Phi\).

  Breaking down \autoref{def:itp:b:pR}, observe we have an `inner' statement:
  \begin{quote}
     It is possible for \vAgent{} to conclude \(\phi\) has value \(v\) by witnessing reasoning from \(\Phi\) to \(\langle \phi,v \rangle\).
  \end{quote}
  And, a qualifier:
  \begin{quote}
    [G]iven \vAgent{}'s present epistemic state.
  \end{quote}

  The statement is simple.
  The relevant possibility is just for the agent to conclude \(\langle \phi,v \rangle\) from \(\Phi\) by an instance of \adA{}.
  Indeed, the relevant instances of \EAS{} we motivate by developing tension will always involve an \itp{}, and hence will always involve the possibility of witnessing reasoning to the relevant conclusion from some pool of premises.

  Turn now to the qualifier:
  \begin{quote}
    [G]iven \vAgent{}'s present epistemic state.
  \end{quote}
  This is a qualifier on possible witnessing.

  Generally speaking, it may be possible for an agent to conclude \(\langle \phi,v \rangle\) from \(\Phi\) by an instance of \adA{} from a distinct epistemic state.
  For example, if the agent were to learn some \(\langle \phi_{i},v_{i} \rangle\) in \(\Phi\) is the case, or if the agent were to improve their reasoning skills.
  However, the innermost qualifier ensures that it possible for an agent to conclude \(\langle \phi,v \rangle\) from \(\Phi\) without any revision to the agent's epistemic state.

  An important consequence of this qualifier is that the agent must already hold that for each \(\langle \phi_{i},v_{i} \rangle\) in \(\Phi\), \(\phi_{i}\) has value \(v_{i}\).
  For, if not, then \(\langle \phi_{i},v_{i} \rangle\) would not be available as a premise.
  Of course, the definition of an \itp{} between \(\langle \phi,v \rangle\) and \(\Phi\) may be given without this assumption, but we have no use for any more general definition.
\end{note}

\begin{note}[\illu{2}]
  \color{red}
    For example, consider being informed that the first player in a game of tic-tac-toe may always guarantee a draw.
  No premises are specified, but on reflection it is clear to see that one may reason through all the possible games to identify the strategy.
  The relevant \itp{0}, then, is the combination of the novel information and one's understanding of tic-tac-toe, the premises, some general results about tic-tac-toe, and the conclusion the required strategy.

  What guarantees the possibility of concluding --- general properties of tic-tac-toe which follow from the rules --- is intuitively distinct from the relevant pool of premises, which likely be limited to the rules themselves combined with the \dots

\end{note}

\begin{note}
  A quick observation.

  \(\langle \mu,v \rangle\) being an \itp{} between \(\langle \phi,v \rangle\) and \(\Phi\) depend on whether or not it is actually possible for the agent to conclude \(\langle \phi,v \rangle\) from \(\Phi\), given their epistemic state.
  If it is not possible for the agent to witness reasoning from \(\Phi\) to \(\langle \phi,v \rangle\), then no such \itp{} will exist.

  However, whether or not an agent \emph{concludes} \(\langle \mu,v \rangle\) is an \itp{} between \(\langle \phi,v \rangle\) and \(\Phi\) does not depend on whether or not it is actually possible for the agent to conclude \(\langle \phi,v \rangle\) from \(\Phi\), given their epistemic state.
  We do not assume that an agent concludes \(\langle \phi,v \rangle\) only if \(\phi\) actually has value \(v\).
  And, our interest with \itp{1} will typically be from the perspective of the agent's present epistemic state.
\end{note}

\begin{note}[Expanding on~\ref{def:itp:b:distinct}]
  The above has expanded on~\ref{def:itp:b:pR}.
  Finally, we turn to~\ref{def:itp:b:distinct}.

  In short,~\ref{def:itp:b:distinct} ensures that if \(\langle \mu,v \rangle\) is an \itp{0} between \(\langle \phi,v \rangle\) and \(\Phi\), then \(\langle \mu,v \rangle\) is not a premise that the agent would appeal to when witnessing the reasoning from \(\Phi\) to \(\langle \phi,v \rangle\) captured by~\ref{def:itp:b:pR}.

  More strictly, not only is \(\langle \mu,v \rangle\) not a premise, but is not equivalent to any \(\langle \phi_{i},v_{i} \rangle\) in \(\Phi\).
  Where, again, equivalence is evaluated from the perspective of the agent.

  From an abstract perspective, if \(\langle \mu,v \rangle\) is an \itp{0} between \(\langle \phi,v \rangle\) and \(\Phi\), then \(\langle \mu,v \rangle\) is purely descriptive of the relationship between \(\langle \phi,v \rangle\) and \(\Phi\).

  In other words, \(\langle \mu,v \rangle\) is not required to conclude \(\langle \phi,v \rangle\) from \(\Phi\).

  Now,~\ref{def:itp:b:distinct} is a somewhat arbitrary restriction.

  In general, it is plausible that some \(\langle \mu,v \rangle\) may both inform an agent that they may conclude \(\langle \phi,v \rangle\) from \(\Phi\), but is also a member of \(\Phi\).

  Indeed, consider the conditional `if \(\langle \alpha,v \rangle\) then \(\langle \beta,v' \rangle\)'.
  Granting the conditional allows detachment, then it is surely possible for an agent to reason from the pool of premises \(\{\langle \alpha,v \rangle, \text{if} \langle \alpha,v \rangle \text{ then } \langle \beta,v' \rangle\}\) to \(\langle \beta,v' \rangle\).%
  \footnote{
    So long as the agent has already concluded \(\langle \alpha,v \rangle\).
  }

  Indeed, without~\ref{def:itp:b:distinct}, \itp{1} would be abundant.
  Hence,~\ref{def:itp:b:distinct} narrows our attention to cases of interest, cases where \(\langle \mu,v \rangle\) merely describes --- and does not partake --- in the reasoning of interest.%
  \footnote{
    Our course, ruling out an abundance of potential \itp{1} via~\ref{def:itp:b:distinct} carries risk of ruling out interesting \itp{1}.
    I encourage further investigation.
    Still, as \itp{1} are only of indirect interest, simplicity via arbitrary restrictions is favoured over complexity from general definitions.
  }
\end{note}

\begin{note}[Summary of \itp{0} between]
  \dots
\end{note}

\subparagraph*{An \itp{} for \(\langle \phi,v \rangle\)}

\begin{note}[\itp{2} for]
  We now turn to the second helper definition, that of an \itp{0} for some proposition-value pair.
  In short, an \itp{0} \emph{between} \(\langle \phi,v \rangle\) and \(\Phi\) ensures the possibility of the agent concluding \(\langle \phi,v \rangle\) from \(\Phi\).
  And, by contrast, an \itp{0} \emph{for} \(\langle \phi,v \rangle\) ensures there is some \(\Phi\) such that it is possible for the agent to conclude \(\langle \phi,v \rangle\) from \(\Phi\).
\end{note}

\begin{note}[def: \itp{2} for]
  \begin{definition}[An \itp{0} for \(\langle \phi,v \rangle\) \dots --- \named{I.f}]
    \label{def:itp:f}
    \(\langle \mu,v \rangle\) is an \itp{} \emph{for} \(\langle \phi,v \rangle\) if and only if:
    \begin{enumerate}[label=\arabic*., ref=\named{I.f:\arabic*}]
    \item
      \label{def:itp:f:pR}
      \(\mu\) having value \(v\) ensures there is some pool of proposition-value pairs \(\Phi\) and proposition-value pair \(\langle \mu',v' \rangle\) such that:
      \begin{itemize}
      \item
        \(\langle \mu',v' \rangle\) is an \itp{} between \(\langle \phi,v \rangle\) and \(\Phi\).
      \end{itemize}
    \item
      \label{def:itp:f:distinct}
      \(\langle \mu,v \rangle\) is not equivalent to any \(\langle \phi_{i},v_{i} \rangle\), given \vAgent{}'s present epistemic state.
    \end{enumerate}
    \vspace{-\baselineskip}
  \end{definition}
\end{note}

\begin{note}
  As with~\autoref{def:itp:b},~\autoref{def:itp:f} contains two components; a core and a restriction.
\end{note}

\begin{note}
  The core is straightforward.
  \ref{def:itp:f:pR} requires \(\langle \mu,v \rangle\) to do ensure two things:
  \begin{enumerate}
  \item The existence of some pool of proposition-value pairs \(\Phi\), and
  \item The existence of an \itp{0} between \(\langle \phi,v \rangle\) and \(\Phi\).
  \end{enumerate}
  In other words, then, an \itp{0} \emph{for} \(\langle \phi,v \rangle\) just is the guarantee of an \itp{0} \emph{between} \(\langle \phi,v \rangle\) and some pool of premises \(\Phi\).

  Simply as it may be, the definition of a \itp{0} for will prove quite useful, as we avoid the need to specify any particular pool of premises.

  Further, any \itp{0} between \(\langle \phi,v \rangle\) and \(\Phi\) is always an \itp{0} for \(\langle \phi,v \rangle\).
  For, suppose \(\langle \mu,v \rangle\) is an \itp{0} between \(\langle \phi,v \rangle\) and \(\Phi\).
  Then, \(\Phi\) is the relevant pool of premises and \(\langle \mu,v \rangle\) itself is the relevant \(\langle \mu'v' \rangle\), and by definition \(\langle \mu,v \rangle\) is not equivalent to any \(\langle \phi_{i},v_{i} \rangle\) in \(\Phi\), given the agent's present epistemic state.

  Note, however, that in general, if \(\langle \mu,v \rangle\) is an \itp{0} for \(\langle \phi,v \rangle\) then \(\langle \mu,v \rangle\) need not be an \itp{0} between \(\langle \phi,v \rangle\) and some pool of premises.
  Simply, though any \itp{0} between is also an \itp{0} for, the converse does not hold.
  For, an \itp{0} requires a pool of premises to be specified.
  Therefore, \(\langle \mu,v \rangle\) and \(\langle \mu',v' \rangle\) must, in general, be distinct.

  Of course, an \itp{0} for \(\langle \phi,v \rangle\) may have the same general statement as an \itp{0} between \(\langle \phi,\Phi \rangle\).

  Consider again tic-tac-toe.
  Above, we considered the \itp{0} between the existence of a strategy for first player in a game to guarantee a draw from the rules of tic-tac-toe.
  Though, with a moments reflection the statement that there existence of a strategy for first player in a game to guarantee a draw will typically lead to an \itp{0} for the existence of the relevant strategy.
  For, some premises must exist, and given the simplicity of tic-tac-toe these are surely within the grasp of an agent who understands the rules of tic-tac-toe.
\end{note}

\begin{note}
  Finally, the restriction~\ref{def:itp:f:distinct} functions in parallel to the restriction \ref{def:itp:b:distinct} of~\autoref{def:itp:b}.
  An \itp{0} for \(\langle \phi,v \rangle\) is purely descriptive, and does not participate in concluding \(\langle \phi,v \rangle\) from the relevant pool of premises.
  A more general definition may be given without this restriction, but such a definition is beyond present interest.
\end{note}

\paragraph*{\adB{}}

\begin{note}
  With the two helper definitions in hand, let us return to the definition of \adB{}.

  Recall two components:~\ref{def:adB:itp} and~\ref{def:adB:conclude}.

  \ref{def:adB:itp} stated that the agent has concluded \(\langle \mu,v \rangle\) where \(\langle \mu,v \rangle\) is either:
  \begin{enumerate}[label=(\alph*)]
  \item An \itp{} \emph{between} \(\langle \phi,v \rangle\) and \(\Phi\), or
  \item An \itp{} \emph{for} \(\langle \phi,v \rangle\), with \(\Phi\) as the relevant pool of premises
  \end{enumerate}
  We have seen the relevant definitions.

  So, given the agent has concluded \(\langle \mu,v \rangle\), for some \itp{0} \(\langle \mu,v \rangle\) then it is possible for the agent to conclude \(\langle \phi,v \rangle\) from some pool of premises \(\Phi\).

  \ref{def:adB:conclude} is key.

  The agent concludes \(\langle \phi,v \rangle\) by appeal to the pool of premises \(\Phi\) \emph{via} the possibility of witnessing the relevant reasoning from \(\Phi\) to \(\langle \phi,v \rangle\) given by the \itp{0} \(\langle \mu,v \rangle\).

  Hence, the agent does \emph{not} conclude \(\langle \phi,v \rangle\) from \(\langle \mu,v \rangle\) or indeed from some pool of premises for which \(\langle \mu,v  \rangle\) is a member.
  Indeed, the latter point follows from~\ref{def:itp:b:distinct} and~\ref{def:itp:f:distinct} --- an \itp{0} is always purely descriptive of some possible reasoning.

  Instead, the pool of premises the agent concludes \(\langle \phi,v \rangle\) from is the collection of those premises that they agent would appeal to if they were to witness the relevant instance of reasoning given by the \itp{0}.

  Of course, the presence of an \itp{0} is, intuitively, crucial.
  For, without an \itp{0} the agent would lack information that it is possible to conclude \(\langle \phi,v \rangle\) from some pool of premises.
\end{note}

\subparagraph*{Role in argument}

\begin{note}
  \color{red}
  This expands on the quick argument a fair bit, and comes down to the same general issue.
\end{note}

\begin{note}
  The general idea of \adB{} is perhaps surprising.

  Grant an instance of \adB{}.
  Then, an agent has concluded \(\langle \phi,v \rangle\) from \(\Phi\) \emph{via} having concluded \(\langle \mu,v \rangle\) for some \itp{0} which states that it is possible for the agent to conclude \(\langle \phi,v \rangle\) from \(\Phi\).

  However, if the agent has concluded \(\langle \mu,v \rangle\) for some \itp{0} which states that it is possible for the agent to conclude \(\langle \phi,v \rangle\) from \(\Phi\), then \(\langle \mu,v \rangle\) alone is seem sufficient to conclude \(\langle \phi,v \rangle\).
  Indeed, at issue is principle which may be stated quite generally:

  \begin{idea}
    \label{idea:c-from-pc}
    It is permissible for an agent to conclude \(\langle \phi,v \rangle\) from the premise that it is possible for them to witness reasoning which concludes \(\langle \phi,v \rangle\), given their present epistemic state.
  \end{idea}

  Arguing in detail for~\autoref{idea:c-from-pc} goes beyond present interest.
  I take the idea to be sufficiently intuitive.

  Of course, concluding is not factive, so the possibility of witnessing reasoning which concludes \(\langle \phi,v \rangle\) does not guarantee \(\phi\) has value \(v\).
  However, by the same token,~\autoref{idea:c-from-pc} only concerns the agent concluding \(\langle \phi,v \rangle\), so it seems one only needs to argue that \(\phi\) must have value \(v\) from the agent's present epistemic state.
  And, if from the agent's present epistemic state it is possible for them to witness reasoning which concludes \(\langle \phi,v \rangle\), it seems the agent may not question whether \(\phi\) has value \(v\) without questioning their conclusion that it is possible for them to witness reasoning which concludes \(\langle \phi,v \rangle\).

  Still, set motivation aside.
  Our interest with~\autoref{idea:c-from-pc} is with the distinction between \adA{} and \adB{}.
  And, in particular, it seems that if~\autoref{idea:c-from-pc} is granted, then the resources present for any instance of \adB{} will always allow for an instance of \adA{}.

  For, an instance of \adB{} requires an agent has concluded \(\langle \mu,v \rangle\), where \(\langle \mu,v \rangle\) is an \itp{0}.
  Suppose \(\langle \mu,v \rangle\) is an \itp{0} for \(\langle \phi,v \rangle\).

  Now, by~\autoref{idea:c-from-pc}, it is permissible for the agent to conclude \(\langle \phi,v \rangle\) from \(\langle \mu,v \rangle\).
  The agent has concluded it is possible for them to witness reasoning which concludes \(\langle \phi,v \rangle\), given their present epistemic state, and therefore the agent may conclude \(\langle \phi,v \rangle\).

  Indeed, without any further thought, it seems we have established that any proposed instance of \adB{} must contain sufficient detail to allow the instance to be (at least) reinterpreted as an instance of \adA{}.

  I think this is broadly correct.
  Granting~\autoref{idea:c-from-pc}, then the relevant \itp{0} from any instance of \adB{} will allow the agent to conclude \(\langle \phi,v \rangle\) from the \itp{0}.
  And, I grant~\autoref{idea:c-from-pc}.

  Note, however, I have refrained from stating that the agent concluding \(\langle \phi,v \rangle\) from the \itp{0} is an instance of \adA{}.
  For, appeal to~\autoref{idea:c-from-pc} does not state that the agent concluding \(\langle \phi,v \rangle\) from the \itp{0} is a complete account of the instance of reasoning.
  Expanding, we have not show that the agent concluding \(\langle \phi,v \rangle\) from the \itp{0} does not also include the agent concluding some other proposition-value pair \(\langle \psi,v' \rangle\) from some other premises.
  And, if the instance does involve the agent concluding \(\langle \psi,v' \rangle\) from some other premises, the instance of reasoning will only be an instance of \adA{} if the agent witnesses reasoning from those premises.

  In other words,~\autoref{idea:c-from-pc} does not immediately show that have the option of analysing away any proposed instance of \adB{}.
  For, we have not shown that any analysis under \adA{} provides a complete account of the proposed instance of \adB{}.
\end{note}

\begin{note}
  For the moment we leave this observation as a promissory note.
  Tension will be developed in~\autoref{sec:tension}.
  Still, we have already seen one way of introducing additional structure into some conclusion by focusing on instances of concluding which are also instances of \csN{0}.
  And, an important part of the argument for tension in~\autoref{sec:tension} will be showing that certain instances of \csN{} require \adB{} as no analysis under \adA{} will provide a complete account of the relevant instance of concluding/\csVImp{0}.
\end{note}


\paragraph{A pair of \illu{3}}

\begin{note}
  To \illu{0} \adA{} and \adB{} we work through two \illu{1} in some detail.
  Both \illu{1} share two components:
  \begin{enumerate}[label=\alph*., ref=(\alph*)]
  \item
    \label{adX:illu:struc:mem}
    Memory of creating a syntactic proof for some first order formula.
  \item
    \label{adX:illu:struc:concl}
    Concluding that the relevant formula is a theorem of first-order logic.
  \end{enumerate}

  The key difference between the two \illu{1} is whether the memory~\ref{adX:illu:struc:mem} serves as a premise or an \itp{0} for the conclusion~\ref{adX:illu:struc:concl}.

\end{note}
\begin{note}[Two premises]
  \begin{quote}
    \begin{enumerate}[%
      label={(Mem)},%
      ref={(Mem)}%
      ]
    \item
      \label{ill:Eproof:mem}
      I remember having created a syntactic proof of \formula{\forall x Px \rightarrow \lnot \exists x \lnot P x} (using a sound first-order system).%
      \footnote{
        We use the phrasing `having created' rather than `creating' to imply completion.
      }
    \end{enumerate}
  \end{quote}
  And:
  \begin{quote}
    \begin{enumerate}[%
      label={(\(\exists\mathord{\vdash}{,}\top\))},%
      ref={(\(\exists\mathord{\vdash}{,}\top\))}%
      ]
    \item
      \label{ill:Eproof:def}
      The existence of a syntactic proof of a formula (using a sound first-order system) is sufficient to establish the formula is a (syntactic) theorem of first-order logic.
    \end{enumerate}
  \end{quote}
\end{note}

\paragraph{First \illu{0} (\adA{})}

\begin{note}
  \begin{illustration}[\adA{}]
    \label{ill:ad:proof:mem}
    \mbox{}
    \vspace{-\baselineskip}
    \begin{enumerate}[%
      label=\arabic*,%
      ref=({I}.{\ref{ill:ad:proof:mem}}:\arabic*)%
      ]
    \item \illEproofMem{} \hfill \ref{ill:Eproof:mem}
    \item
      \label{ill:Eproof:exP}
      So, there exists a syntactic proof of \formula{\forall x Px \rightarrow \lnot \exists x \lnot P x} (using a sound first-order system)
    \item
      \label{ill:Eproof:thm}
      Hence, by \ref{ill:Eproof:def}, \formula{\forall x Px \rightarrow \lnot \exists x \lnot P x} is a theorem of first-order logic.
    \end{enumerate}
    \vspace{-\baselineskip}
  \end{illustration}
\end{note}

\begin{note}[Discussion of \autoref{ill:ad:proof:mem}]
  I take \autoref{ill:ad:proof:mem} to be a straightforward case of concluding.
  \ref{ill:Eproof:mem}, memory,


  and \ref{ill:Eproof:def}, to recast the existence of a proof from \ref{ill:Eproof:mem} in terms of the formula being a theorem.

  Neither premise from anything more basic, and without either premise the conclusion would not be obtained.
  For, without \ref{ill:Eproof:mem} no proof, and without \ref{ill:Eproof:def} no recasting.
\end{note}

\begin{note}
  It seems sufficient, generally speaking, to conclude some proposition has value \(v\) by appeal to memory, hence the agent claims support that there was some event which culminated in a syntactic proof of the formula.

  Of course, the agent may have misremembered.
  Still, we do not require that any agent concludes \(\langle \phi,v \rangle\) from \(\Phi\) only if \(\phi\) \emph{actually} has value \(v\) and each \(\langle \phi_{i},v_{i} \rangle\) in \(\Phi\), \(\phi_{i}\) \emph{actually} has value \(v_{i}\).

  Following, this allows the agent to conclude a syntactic proof of the formula exists.
  As before, the agent may have failed to \emph{actually} create a syntactic proof of the formula
  Still, from the same perspective this does not prevent the agent from concluding they did (actually) create such a proof.

  Hence, finally, the agent claims support that the formula is a (syntactic) theorem of first-order logic.
\end{note}

\begin{note}
  To concisely summarise, we may say that the agent conclude the is a (syntactic) theorem of first-order logic \emph{because} of their understanding of syntactic theorem-hood and their memory of proving the formula.

  For sure,~\autoref{ill:ad:proof:mem} is designed to be as straightforward as possible.
  Of interest is not whether the agent claims support, but how the role the agent gives to their memory in claiming support.

  The agent appeals to their memory to establish that there exists a syntactic proof of the formula, and then combines the existence of a syntactic proof with~\ref{ill:Eproof:def} to claim support that the formula is a theorem.
  Hence, the agent's memory is directly involved in their claimed support for the formula being a theorem.\nolinebreak
    \footnote{
      \color{red}
      Whether proving is an unsatisfied \requ{}.
      However, recall that allowed a \requ{} to be satisfied by some instance of concluding.
      And, memory of concluding.
      Still question about original proof, but no problem with memory.
      Also, method.
    }
\end{note}

\paragraph{Second \illu{0} (\adB{})}

\begin{note}
  \begin{illustration}[\adB{}]
    \label{ill:ad:proof:eve}
    \mbox{}
    \vspace{-\baselineskip}
    \begin{enumerate}
    \item \illEproofMem{} \hfill \ref{ill:Eproof:mem}
    \item
      \label{ill:ad:proof:eve:app}
      In creating the syntactic proof I appealed to various aspects of some sound first-order system.
    \item
      \label{ill:ad:proof:eve:pos}
      As I created a proof, those various aspects of the sound first-order system are sufficient to ensure there exists a proof.
    \item
      Hence, by \ref{ill:Eproof:def}, \formula{\forall x Px \rightarrow \lnot \exists x \lnot P x} is a theorem of first-order logic.
    \end{enumerate}
    \vspace{-\baselineskip}
  \end{illustration}

  {
    \color{red}
    Premises are rules are part of a sound system, may be combined in this way.%
    \footnote{
      Indeed, relative simplicity is why we chose a syntactic rather than semantic proof.
      With semantic, need an argument which covers all models, and while premises plausibly exist, these have no common codification.
    }
  }

  As with~\autoref{ill:ad:proof:mem}, the agent's memory has a role in~\autoref{ill:ad:proof:eve}, but the role is quite different.
  Above, the agent claimed support for the formula being a theorem primarily \emph{because} they remembered creating a proof.
  By contrast, here the agent claims support for the formula being a theorem primarily because of the properties of some sound first-order system.

  Step~\ref{ill:ad:proof:eve:app} appeals to various aspects of some sound first-order system and, in turn, step~\ref{ill:ad:proof:eve:app} observes that those aspects are sufficient to ensure a proof exists.
  The agent claims support for the existence of a proof by appeal to the various aspects of some first-order system they appealed to when constructing the proof, rather than their memory of constructing the proof.
\end{note}

\begin{note}
  To help clarify, let's fix a particular syntactic proof using the Fitch-style proof system of~\textcite[557--560]{Barwise:1999tu}:

  \begin{figure}[H]
    \centering
    \begin{quote}
      \fitchprf{}{
        \subproof{\pline[1.]{\forall x P x}}{
          \subproof{\pline[2.]{\exists x \lnot Px}}{
            \boxedsubproof[3.]{a}{\lnot Pa}{
              \pline[4.]{Pa}[\lalle{1}] \\
              \pline[5.]{\bot}[\lfalsei{3}{4}]
            }
            \pline[6.]{\bot}[\lexie{2}{3--5}]
          }
          \pline[7.]{\lnot \exists x \lnot Px}[\lnoti{2--6}]
        }
        \pline[8.]{\forall x Px \rightarrow \lnot \exists x \lnot Px}[\lifi{1--7}]
      }
    \end{quote}
    \caption{A syntactic proof}\label{fig:syntx-prf}
  \end{figure}

  The proof consists of single instances of five introduction or elimination rules.
  Each rule is part of the Fitch-style proof system, and the specific application of the rules constitute the proof.
\end{note}


\begin{note}[Before\dots]
  Before returning to~\autoref{ill:ad:proof:eve}, let us observe that with the proof in hand one may claim support that a proof of the formula exists via the contents of~\autoref{fig:syntx-prf}.

  Broadly stated:

  \begin{enumerate}
  \item The proof is constructed from a sound first-order proof system.
  \item And, the particular application of some rules of the system to formulae is such that the proof begins with no assumptions and the last line of the proof is not part of any assumption made during the course of the proof.
  \end{enumerate}
\end{note}

\begin{note}
  Note, appeal to creation of the proof involves appeal to various aspects of the Fitch-style proof system.

  The object itself is mute to whether or not it is a proof.

  For example, adding `\formula{Ba}' as an assumption would void the proof, but you would need to observe that the appeal to existential elimination on line 6 requires that `\formula{a}' does not appear in the proof prior to its introduction on line 3 in order to claim support that the proof is void.

  Indeed, the proof consists of eight steps, each step is permitted by the first-order system, the proof begins with no assumptions, the last line of the proof is not part of any assumption made during the course of the proof and the proof, and so on.

  Sparing the details, claimed support that~\autoref{fig:syntx-prf} is a syntactic proof of \formula{\forall x Px \rightarrow \lnot \exists x \lnot P x} from the creation of~\autoref{fig:syntx-prf} is a matter of claiming support for each step of the creation.

  Indeed, to spare the details in general, let us instead talk of some collection of propositions and steps of reasoning.
  Claiming support that a proof exists from the some creation in the way under discussion is an instance of reasoning from details of the creation to the conclusion that a proof exists.
  Hence, as an instance of reasoning involves certain premises and steps of reasoning.
  And, whatever these turn out to be, the proceed from the creation of the proof rather than from some other source such as memory, testimony, and so on.
\end{note}

\begin{note}
  In other words, one may claim support that a proof of \formula{\forall x Px \rightarrow \lnot \exists x \lnot P x} exists (primarily) \emph{because} of their reasoning from some collection of premises and steps of reasoning concerning the creation to the existence of a proof of \formula{\forall x Px \rightarrow \lnot \exists x \lnot P x}.
\end{note}

\begin{note}[Return to \ref{ill:ad:proof:eve}]
  Now let us return to the reasoning of~\autoref{ill:ad:proof:eve}, and in particular steps~\ref{ill:ad:proof:eve:app} and~\ref{ill:ad:proof:eve:pos}:
  \begin{quote}
    \begin{enumerate}
      \setcounter{enumi}{1}
    \item In creating the syntactic proof I appealed to various aspects of some sound first-order system.
    \item As I created a proof, those various aspects of the sound first-order system are sufficient to ensure there exists a proof.
    \end{enumerate}
  \end{quote}
  Given that the agent remembers having created a syntactic proof, the `various aspects of some sound first-order system' of step~\ref{ill:ad:proof:eve} may be taken as those aspects of the first-order system that were appealed to in the premises and steps of reasoning when the agent created the proof.
  And step \ref{ill:ad:proof:eve}, in turn, appeals to how those various aspects of some sound first-order system were sufficient for the agent to claim support that a proof exists by the reasoning that occurred.

  In short, the agent remembers creating a syntactic proof and claiming support that a proof exists from the creation.
  The instance of claiming support involved reasoning from premises via steps to the relevant conclusion.
  Hence, it is possible to claim support for the conclusion by those premises and steps of reasoning.
  So, in~\ref{ill:ad:proof:eve} the agent observes that those premises and steps of reasoning are sufficient to claim support by way of their memory, and in turn appeals to those premises and steps of reasoning to claim support for the relevant conclusion.
\end{note}

\begin{note}
  {
    \color{red}
    Propositional support.
    (If I talk about this, it should be after the definitions.)
  }
\end{note}

\begin{note}
  Generalising, the way in which the agent claims support in~\autoref{ill:ad:proof:eve} is of interest because the agent appeals to premises and steps of reasoning that are not `part' of their present reasoning.
  The role of memory in the \illu{0} is (merely) a way for the agent to recognise that there are such premises and steps of reasoning.
  And, in the definitions that follow, we will abstract from any particular way in which the recognises that relevant premises and steps of reasoning are available.

  Still, even though memory is contingent, we may briefly observe that the way in which the agent claim support in~\autoref{ill:ad:proof:eve} is compatible with \ESU{}.
  For, \ESU{} requires that an agent may claim support for some conclusion from premises and steps of reasoning only if the agent has witnessed reasoning to the conclusion from those premises via those steps of reasoning.
  So, if the initial instance of claiming support conformed to \ESU{} then the agent will have witnessed reasoning from those steps and premises to the conclusion --- the instance of claiming support in~\autoref{ill:ad:proof:eve} does not involve such witnessing, but the agent's memory would be about how the relevant premises and steps were used to claim support.

  Of course, the way in which the agent claim support in~\autoref{ill:ad:proof:eve} is incompatible with a strengthened variant of \ESU{} which requires the agent to use any premises and steps they appeal to in the \emph{present} instance of reasoning, but the point for the moment is that the way in which the agent claims support in~\autoref{ill:ad:proof:eve} does not already require what we are arguing against: \ESU{}.
\end{note}

\begin{note}
  \color{red}
  In this case, \adB{} is not incompatible with \ESU{}.
  For, the agent has witnessed reasoning (granting memory).
  So, \ESU{} does not lead to an immediate rejection of \adB{}.

  Oh, this is noted.
\end{note}


\subsubsection{Looking to tension: \adB{} and ability}
\label{sec:looking-tension}


\paragraph{General and specific ability}

\begin{note}[General and specific ability]
  Suppose general ability.
  Then, various specific instances of general ability.
  These provide \itp{1}.
\end{note}

\begin{note}[Examples]
  Some examples are clear.
\end{note}


\paragraph{\aben{the}}


\begin{note}[\aben{the}]
  Here, not only do we conclude specific instance of general ability, but also the result of witnessing the ability.

  Here is where we get the relevant \itp{}.

  For, this means there are premises available, and from these obtain conclusion.
\end{note}

\paragraph{\adB{} and ability}

\begin{note}[Generate many \itp{1}]
  General ability to solve a particular type of problem.
  Specific instances.
  Then, generate \itp{1} from specific instances of type of problem.
  Result, concluding from specified premises.
  Of course, may be unclear on what the premises are, but still, existence is presumed.%
  \footnote{
    Presumed, but not guaranteed.
    Agent may not have the ability, and indeed no agent may have ability.
  }
\end{note}

\begin{note}[No clear tension yet]
  Of course, no clear tension yet.
  General ability.
  I have the general ability as a premise.
  Here, then, instances of \adA{}.
\end{note}

\subparagraph{Limitation}

\begin{note}
  From perspective of \EAS{}, limited.
  For, it seems always possible to include \itp{} as a premise.
\end{note}


\subsection{The horizon}

\begin{note}
  Seen \adB{}.
  Initial \illu{0}, and cases of interest.
  Before continuing,  final thing.
  The horizon.
  Cases where \adB{} applies, but outside of our scope.
\end{note}

\subparagraph{Other \illu{1}}

\begin{note}[Seen memory]
  Seen memory and specific case.
  Getting an \itp{}.

  We will expand on \itp{1} of interest below.
  Briefly, a handful of \illu{1} which fall outside immediate scope of argument.
\end{note}

\section{Tension}
\label{sec:tension}

\begin{note}[Intro]
  We have discussed concluding.
  Introduced \csN{}.
  Revised~\autoref{issue:Main} to \ESU{} and \EAS{}.
  And, two types of reasoning, \adA{} and \adB{}.
  Hints regarding ability.

  Task is now to bring these together develop tension.
\end{note}

\begin{note}[Basic idea of tension]
  Strategy.
  Identify some abstract phenomenon.
  Observe how leads to tension.
  Motivate instances of this phenomenon.
  Observe tension.

  Here, phenomenon is consequence of \csN{}.
  Hence, indirectly relies on assumptions concerning concluding.
  Split, either \ESU{} or \EAS{}.
  Construction involve \itp{} introduced with respect to \adB{}.
  Hence, discussion of \adB{} will both clarify and inform resolution to tension.

  Indeed, \adB{} will offer a more general perspective on the consequences of \csN{}.
\end{note}

\subsection{Sketch of tension}
\label{sec:overview-tension}

\begin{note}[Goal]
  To establish tension between \ESU{} and \csN{} we have the following goal:

  \begin{goal}
    \label{goal:tension}
    There are instances in which concluding \(\langle \phi,v \rangle\) from \(\Phi\) seems to be an instance of \csN{} for \(\langle \phi,v \rangle\) where:
    \begin{enumerate}[label=\arabic*., ref=\named{G\ref{goal:tension}:\arabic*}]
    \item
      \label{goal:tension:requ}
      There is some \(\langle \psi,v',\Psi \rangle\) which is a \requ{} of \(\langle \phi,v,\Phi \rangle\) such that:
      \begin{enumerate}[label=\alph*., ref=\named{G\ref{goal:tension}:1\alph*}]
      \item
        \label{goal:tension:requ:conclude}
        It is not possible for the agent to conclude that they would conclude \(\langle \psi,v' \rangle\) from \(\Psi\) \emph{without} concluding \(\langle \psi,v' \rangle\) from \(\Psi\).
      \item
        \label{goal:tension:requ:no-reason}
        The agent does not witness any reasoning from premises \(\Psi\).
      \end{enumerate}
    \end{enumerate}
    \vspace{-\baselineskip}
  \end{goal}

  If there are instance of the type described by \autoref{goal:tension}, then \ESU{} and \csN{} are in tension.

  For, we will have some proposition-value-premise pairing \(\langle \phi,v,\Phi \rangle\) such that an agent conclude \(\langle \phi,v \rangle\) from \(\Phi\), and in doing so \csV{} for \(\langle \phi,v \rangle\).
  And, by \autoref{goal:tension:requ} we have some \requ{} \(\langle \psi,v',\Psi \rangle\).

  Now, as \(\langle \psi,v',\Psi \rangle\) is a \requ{0} of concluding \(\langle \phi,v \rangle\) from \(\Phi\), it must be the case that the agent either has concluded or simultaneously concludes that they would conclude \(\langle \psi,v' \rangle\) from \(\Psi\) when concluding \(\langle \phi,v \rangle\) from \(\Phi\).

  Further, the \requ{} \(\langle \psi,v',\Psi \rangle\) has two key properties.

  First, from~\autoref{goal:tension:requ:conclude}, the agent may only conclude that they would conclude \(\langle \psi,v' \rangle\) from \(\Psi\) by concluding \(\langle \psi,v' \rangle\) from \(\Psi\).
  Hence, by \ESU{} it must be the case that the agent witnesses reasoning from \(\Psi\) which concludes with \(\langle \psi,v' \rangle\).

  Second, from~\autoref{goal:tension:requ:no-reason} the agent does not witness any reasoning from \(\Psi\).
  Hence, the agent does not witness reasoning from \(\Psi\) which concludes with \(\langle \psi,v' \rangle\).

  From the above, tension.
  For, if concluding \(\langle \phi,v \rangle\) from \(\Phi\) really is an instance of \csN{}, then the agent must have concluded \(\langle \psi,v' \rangle\) from \(\Psi\) without witnessing reasoning from \(\Psi\) to \(\langle \psi,v' \rangle\).
  However, if \ESU{} holds then it is not possible for the agent to conclude \(\langle \psi,v' \rangle\) from \(\Psi\) without witnessing reasoning from \(\Psi\) to \(\langle \psi,v' \rangle\).
  So, either the agent has not \csN{} for \(\langle \phi,v \rangle\) or \ESU{} does not hold.
  Hence, either \csN{} or \ESU{} must be restricted in some way.

  Of course, the exact nature of the tension, and how it should be resolved, depends in part on the proposition-value-premise pairings which satisfy \autoref{goal:tension}.

  Indeed,~\autoref{goal:tension:requ:conclude} and~\autoref{goal:tension:requ:no-reason} are significant restrictions, and it is not clear from the abstract statement that these correspond to sufficiently interesting phenomenon.
\end{note}

\begin{note}
  Whether or not~\autoref{goal:tension:requ:no-reason} is satisfied will depend on the specifics of any given instance.
  For, we have placed no constraints on which premises an agent appeals to, nor what the contents of those premises may be.

  \autoref{goal:tension:requ:conclude}, by contrast, will follows from a collection of \requ{1} forming a `\cluster{}'.

  In short, a \cluster{} is a collection of proposition-value-premise pairings such that every proposition-value-premise pairing is a \requ{} of every other proposition-value-premise pairing in the collection.

  Intuitively, every proposition-value-premise pairing of the \cluster{0} is an independent check on every other proposition-value-premise pairing of the \cluster{0}.

  Indeed, we will argue that \autoref{goal:tension:requ:conclude} is satisfied whenever \(\langle \phi,v,\Phi \rangle\) is a member of a \cluster{}.

  Finally, then, we seek some concrete instances.
\end{note}

\begin{note}[Ability]
  Here, our interest turns to ability.
  I take it as given that there are various instances of concluding \(\langle \phi,v \rangle\) from \(\Phi\) for which the reasoning from \(\Phi\) to \(\langle \phi,v \rangle\) is an instance of a general ability.

  For example, I conclude \(31 + 53 = 84\) from some premises, and the reasoning falls under my general ability to perform (simple) arithmetic.
  The specifics may differ, but there is sufficient overlap with concluding \(43 + 81 = 123\) and \(91 + 54 = 145\) to consider the reasoning of the same type.
  Indeed, \(532 - 91 = 441\), \(19 * 32 = 608\), and \(126/36 = 3.5\) may also fall under the same (general) ability.
  In other words, in concluding each equation I witness a specific instance of the general ability.

  Likewise, one may have the (general) ability to solve chess problems, complete \(\{ \text{Sudoku}, \text{KenKen}, \text{Nonogram}, \dots\}\) puzzles, or parse sentences in a given language.

  So long as you have the (general) ability, I expect each instance of witness the ability is intuitively an instance of \csN{}.
  I would not have concluded \(31 + 53 \ne 84\) and you would not have failed to identify the relevant winning strategy of some chess problem.

  However, in each of the examples of ability noted, every other specific instance of the general ability functions as an independent check on whether one has the relevant ability.
  I should make no mistake about \(85 + 21\) and you should make no mistake with the next chess problem.
  Granting, of course, that we do have the relevant abilities.

  Hence, it seems clear that if an agent \csV{} for \(\langle \phi,v \rangle\) when concluding \(\langle \phi,v \rangle\) from \(\Phi\) and the agent's reasoning is a specific instance of a general ability, then there at least various \requ{1} associated with concluding \(\langle \phi,v \rangle\) from \(\Phi\).

  More generally, concluding \(\langle \phi,v \rangle\) from \(\Phi\) from reasoning which is the specific instance of a general ability leads to a \cluster{}.
  And, it is perhaps already intuitive that one does not witness reasoning from the premises of at least some proposition-value-premise pairing in the \cluster{}.

  For example, it seems plausible that the configuration the chess board for any given problem forms of a premise of the agent's reasoning, but so long as one has not seen the relevant problem, it seems implausible that one has witnessed reasoning that includes the relevant configuration.

  Of course, it may seem equally implausible that an agent concludes there is some winning strategy for some configuration of a chess board they have not yet seen.
  However, this intuition should be carefully examined.

  With an unopened chess book by a reputable author before you, I expect you have no problem concluding that each of the solutions are correct.
  {
    \color{red}
    Or, that granting ability, conclude you would (also) conclude winning strategy or not for each chess piece.
  }
  Yet, you have not yet seen any of the solutions.
  So, in general, there seems no issue with an agent concluding there is some winning strategy for some configuration of a chess board they have not yet seen.

  Of course, in the case of the book there is a key premise:
  The book is written by a reputable author.
  However, if you wish to conclude there is a winning strategy via your own reasoning, then the above considerations take effect.%
  \footnote{
    And, indeed, may already hold with respect to the key premises.
    For, so long as you hold you have the general ability to reason about chess problems of the relevant kind, you have an independent check on whether the author really is reputable.
  }
\end{note}

\begin{note}[Moving on]
  So much for the rough outline of how~\autoref{goal:tension} leads tension between \csN{} and \ESU{}.
  Let us turn to the details.

  {
    \color{red}
    \begin{itemize}
    \item
      \autoref{sec:cluster3}, \cluster{1}.
      Abstract.
      Show how clusters lead to \ref{goal:tension:requ}.
    \item
      \autoref{sec:overview:ability}.
      More details on ability.
    \item
      \autoref{sec:overview:tension:subsection}.
      Tension in more detail.
    \item
      \autoref{sec:overview:resolving-tension}.
      Resolving tension.
    \item
      \autoref{sec:overview:observations}.
      So observations.
    \end{itemize}
  }
\end{note}

\subsection{Tension in the abstract}
\label{sec:tension-abstract}

\subsubsection{\cluster{3} of \requ{1}}
\label{sec:cluster3}

\begin{note}[Main idea]
  A \cluster{0} is a collection of proposition-value-premise pairings such that every proposition-value-premise pairing is a \requ{} of every other proposition-value-premise pairing in the collection.
\end{note}

\begin{note}[\requCluster{3}]
  \begin{definition}[A \requCluster{1}]
    \label{def:requCluster}
    Some collection of proposition-value-premise pairings \(\mathcal{C} = \{\langle \phi_{i}, v_{i}, \Phi_{i} \rangle\}_{i}\) is a \emph{\cluster{}} just in case:
    \begin{itemize}
    \item
      For any \(\langle \phi_{i}, v_{i}, \Phi_{i} \rangle\) in \(\mathcal{C}\), each \(\langle \phi_{j}, v_{j}, \Phi_{j} \rangle\) in \(\mathcal{C}\) (such that \(j \ne i\)) is a \requ{} of \(\langle \phi_{i}, v_{i}, \Phi_{i} \rangle\).
    \end{itemize}
    \vspace{-\baselineskip}
  \end{definition}

  Key link:

  \begin{proposition}[\cluster{3} and \csN{0}]
    \label{prop:cluster:csN}
    Suppose we have some \cluster{0} \(\mathcal{C}\).
    Consider some proposition-value-premise pairing \(\langle \phi,v,\Phi \rangle\) in \(\mathcal{C}\).
    Now, for concluding \(\langle \phi,v \rangle\) from \(\Sigma\) to be an instance of \csN{}, need to have had concluded, or to simultaneously conclude, that would conclude \(\langle \psi,v' \rangle\) from \(\Psi\) for any \(\langle \psi,v',\Psi \rangle\) in \(\mathcal{C}\).
  \end{proposition}

  \begin{argument}
    Follows from~\autoref{idea:CS:overview} and~\autoref{def:requCluster}
  \end{argument}

  \begin{proposition}
    \label{prop:cluster:simul}
    Need to do everything in a cluster at the same time.
  \end{proposition}

  \begin{argument}
    Straightforward.
    For, anything is a \requ{} for any other.
    So, only \csV{} at the same time.
  \end{argument}
\end{note}

\begin{note}[No \(\gamma\)]
  Consequence:

  \begin{corollary}
    \label{prop:cluster:no-general}
    No general \(\langle \gamma,v,\Gamma \rangle\) within cluster.
  \end{corollary}

  \begin{argument}
    Quickly, because of~\ref{prop:cluster:simul}.
    Only \(\gamma\) at same time as others.

    In more detail.
    For, by assumption, \requ{} means that it's possible for the agent to conclude.
    By the each other \requ{} functions as a check on \(\gamma\).
    If haven't figured out each individual, then the general is in question.
  \end{argument}
\end{note}

\begin{note}
  \autoref{prop:cluster:simul} is kind of wild.
  Though, this doesn't prevent reasoning, and then only getting \csN{} after the fact.
  However, this does prevent ruling out conflict when concluding.

  Of course, concluding each and then \csVImp{}, plausible.
  As, have not found an issue.

  Also, may conclude all from premises.
  For, special cases of \cluster{} in which all premises are the same.
  If so, then no clear tension.
  For, agent reasons from premises, well all the same premises.
  So, conclude simultaneously.
\end{note}

\begin{note}
  Indeed, narrow interest to \ragCluster{1}.

  \begin{definition}[\ragCluster{3}]
    For any cluster \(\mathcal{C}\), \(\mathcal{C}\) is a \emph{\ragCluster{}} if and only if:
    \begin{enumerate}
    \item
      There is some \(\langle \phi_{i},v_{i},\Sigma_{i} \rangle\) and \(\langle \phi_{j},v_{j},\Sigma_{j} \rangle\) such that \(\Sigma_{i}\) and \(\Sigma_{j}\) do not overlap.
    \end{enumerate}
    \vspace{-\baselineskip}
  \end{definition}

  A \ragCluster{}, then, is just a cluster where at least some distinct premises.
  Hence, avoid issue where same premises allow simultaneous conclusion, and fail to establish tension with \ESU{}.

  \begin{proposition}
    With \ragCluster{} concluded previous or violate \ESU{}.
  \end{proposition}

  Now, some caution.
  It may be the case that reason from some non-minimal collection of premises.
  Hence, some care when establishing \ragged{}.
  This means, argue that \cluster{} \emph{and} argue \cluster{} is \ragged{}.
  Again, without any clear bounds on premises, this argument is non-deductive.
  However, plausible in various cases.
\end{note}

\begin{note}[Relative \jag{1}]
  Given importance of \ragged{} and specific proposition-value-premise pairings of \ragged{}, terminology:

  \begin{definition}[Relative \jag{1} of a \ragCluster{}]
    \(\mathcal{C}\) some \ragCluster{}.
    \(\langle \psi,v',\Psi \rangle\) is a \emph{\jag{0}} relative to \(\langle \phi,v,\Phi \rangle\) if \(\Psi\) differs from \(\Phi\).
  \end{definition}

  \csN{} while presence of some \jag{} when premises are no general.
\end{note}

\begin{note}
  Suppose \ragged{}, no prior conclusion for some \jag{}.
  Then, if \csN{}, violation of \ESU{}.

  For, only \csN{} simultaneously.
  \jag{}, some not from current premises.
  And, no previous, so not from previous premises.
  Hence, \csN{} from premises of the \jag{}.

  Hence, need instances of \ragged{} with no prior conclusion for some \jag{}.
\end{note}

\begin{note}
  \begin{proposition}
    If \ragCluster{} and no prior conclusion for some \jag{}.
    Either:
    \begin{itemize}
    \item
      \ESU{} does not hold in general.
    \item
      No \csVImp{} for any proposition-value-premise pairing in cluster.
    \end{itemize}
      \begin{argument}
    More-or-less immediate from previous.
  \end{argument}
  \end{proposition}
\end{note}

\begin{note}
  Abstract tension, then, follows if there are instances of \ragCluster{1} with no prior conclusion for some \jag{}.
\end{note}

\subsection{Concrete tension}
\label{sec:concrete-tension}

\begin{note}
  Seen how to develop tension in the abstract.
  Now, concrete.
  Ability, as we have seen.
\end{note}

\subsubsection{Ability}
\label{sec:overview:ability}

\begin{note}[Ability]
  Ability to reason in certain ways leads to \cluster{1}.
  However, not interested in ability in general, but rather relatively simple instances of ability concerning specific problem types.

  Arithmetic.
  Sudoku.
  Chess.

  Indeed, the latter pair for \requCluster{1}.
  For, different starting positions.
\end{note}

\begin{note}
  Finally, ability.

  General ability, specific ability.

  Claim support for having some general ability.

  Now, here, simple cases.
  Basic arithmetic.
  Sudoku puzzles.
  Chess problems with winning strategies.

  Roughly the same.
  More broadly:

  Logic problems.

  Crossword.

  Reading novels up to a certain level.
  Here, if you can't read, then the writing is bad.

  Fluency.

  So, specific instances of the general ability.
\end{note}

\begin{note}
  Well, conclude that you have the general ability, but also claim support.
  You don't need to go through specific instances.
  In these cases, fail to be an independent check.
  You not fail to reach the relevant conclusion.

  Would not reason to some incorrect summation.
  Would not fill out the Sudoku incorrectly.
  Would not fail to find a winning strategy.

  Of course, failures of performance, but not failures of competence.

  Intuitively, satisfied all \requ{1}.
\end{note}


\begin{note}
  Key argument, then, is that only satisfy a \requ{} by concluding \(\psi\) from \(\rho\).

  For, some other premise, get \(\psi\) from \(\rho\).
  Well, getting \(\psi\) from \(\rho\) is still a \requ{} for this.
\end{note}

\begin{note}
  Briefly stated.

  Specific instances, these introduce \requ{}.
  However, some reasoning.
  Can't jump to general to get rid of \requ{}, as this is forbidden.
  Further, if reason from some distinct set of premises, then still a \requ{}.

  If independent reasoning gets that specific instance of general ability, then doing the reasoning is still an independent check on this.

  So, the problem here is that need to ensure that would conclude \(\psi\) has value \(v'\) from certain premises.
  If appeal to any distinct premises, then failure to claim support.

  Hence, \ESU{} and \ideaCS{}, then no getting general ability without witnessing reasoning for specific instances.

  Core of the tension.
  Always some independent check with distinct premises with specific instance of general ability.

  So, either, allow to bypass independent check.
  Or, do not require witnessing reasoning from premises to conclusion.
\end{note}

\subsubsection{\adB{}}
\label{sec:adb}

\begin{note}
  Now, these cases of ability.
  Work with \adB{}.
\end{note}

\begin{note}
  Here, it turns out that obtaining the \itp{} is just concluding.
\end{note}

\begin{note}[Conditionals, a point of interest]
  More generally, we have the following result.\nolinebreak
  \footnote{
    So long as we do not add~\autoref{notion:overview:requ:pool:method} to~\autoref{notion:overview:requ:pool} of the notion of a \requ{}.
    If so, then result will be constrained accordingly.
  }
  If appeal to some conditional which links premises to a conclusion, such that agent may reason from premises to conclusion, then the agent has always concluded premises from conclusion.

  This is interesting.
  If agent has concluded from conditional in this way, then in effect the conditional drops out as a premise.

  If \(\Sigma, \phi \rightarrow \psi \vdash \psi\) then \(\Sigma, \phi \vdash \psi\).

  If \(\Sigma, \phi \vdash \phi \rightarrow \psi\) then \(\Sigma, \phi \vdash \psi\).

  The second is close to a restricted form of the deduction theorem.

  Note, this is only when the conditional has a special \requ{}.

  In cases where no checking the conditional, then the elimination does not hold.

  Whether anything of independent interest follows from this, unclear.
  One example, responsibility.
  Then, to another person, not only reasoning with conditional, but also full reasoning.
  No way to distinguish between the two cases.
  Or, from the converse perspective, no need to add any additional clauses to account of responsibility.
  However, issues of this kind are far beyond the scope of this document.
\end{note}

\begin{note}
  Indeed, given the constraints of \itp{1}, conditional will satisfy just in case it is an \itp{}.
  Here, need that \(\langle \mu,v \rangle\) is not in \(\Phi\) to ensure that \itp{0} is redundant.
\end{note}

\subsubsection{Thoughts}
\label{sec:thoughts}

\begin{note}[Difficulty]
  Here, concrete instantiation of abstract structure.
  Question is, does the agent really \csN{}?
  Intuitively, it seems agent does.
  Adopt \stance{}, I am confident I would not reason otherwise.
  However, this is from adopting a \stance{}.
  And, whether or not concluded or \csVed{} is independent of \stance{}.
  Instead, question is, granting intuition, does this \stance{} reflect on our theory.

  How close these abstractions map to more commonsense reasoning.
  Motivated in part.
  Whether \csN{} seems natural.

  Seems to me, key question is the conditional link.
  It's identifying these two things.
  Abstracts from specific cases, but gives the general principle.
  So long as \csN{}, then these two things coincide.

  Of course, senses of `concluding'.
  However, difficulty in running `concluding' too close to the agent's \stance{} on what the have concluded.
\end{note}

\subsection{Tension}
\label{sec:overview:tension:subsection}


\subsubsection{A different path to tension}
\label{sec:diff-path-tens}

\begin{note}
  We have developed tension with respect to the definition of a \requ{}.
  However, noted that an additional constraint may be placed on the notion of a \requ{}.
  Same type of reasoning.

  If this is the case, then general ability will not be part of \cluster{}.
  Hence, the argument as given will no go through.

  May seem a compelling alternative.
  There are two issues.

  First, consequence.
  General ability functions as a premise in every case of witnessing a specific instance general ability.
  For, if just specific instance, then part of a \cluster{}.

  One way to resist this is to fine grain specific instances.
  However, this is very puzzling.

  Still, when paired against \EAS{} this may not be so bad.

  Hence, we turn to the more substantial issue.
  \csN{2} for the general ability.

  Need to have \csVed{0}.
  For, else, different conclusion about general ability, and hence no \csVImp{} for specific cases.

  So, \csVImp{} for general ability.
  Here, each specific instance is a \requ{}.
  And, we are back to the original problem.

  So, even if you grating general ability as a premise, still a question as to how this premise is obtained.

  Some wiggle room is left.
  Here, informed that one has the general ability.
  But, then, any attempt to witness the specific ability would not show mistake.
  And, no isolated conclusion.

  These, quite puzzling to defend.
\end{note}


\subsection{Resolving the tension}
\label{sec:overview:resolving-tension}

\begin{note}
  Note, the tension is not about whether \(\phi\) has value \(v\).
  Instead, the tension is about whether the agent would have a certain property if they were to conclude \(\phi\) has value \(v\).
  Property of having claimed \support{}.
  Expanded, property of holding that any independent check is satisfied.
  Any other reasoning about whether \(\phi\) has value \(v\) would conclude \(\phi\) has value \(v\).
\end{note}

\begin{note}
  Returning to \EAS{}.
  Specific instances of the general ability.
  In this sense, the instances of \EAS{} we argue for are narrow.
  Need strong sign that the agent has the general ability.

  Further.
  It does not state that an agent having claimed support that they have the ability to reason to some conclusion is \emph{always permissible} to claim support for the conclusion by appealing to some premises that do not form part of the agent's reasoning.
  Instead, it states that \emph{may be permissible} for the agent claim support in a certain way.

  In various respects, these aren't particularly interesting cases.
  However, the goal is to argue that such cases exist.
  Whether these are constrained to the type of cases we consider for the argument is a further question.

  There may be more interesting cases, but given that \ESU{} is incompatible with all such cases, I see no compelling reason to explore such cases without \emph{first} motivating a rejection of \ESU{}.
\end{note}

\begin{note}
  Also suggests that the content of general ability is somewhat interesting.
  For, the content is itself general.
  It is a conclusion that ranges over all specific instances of the general ability.
\end{note}

\begin{note}[Terminology]
  So, the upshot of this is that an agent concludes various things in certain cases.
  In concluding \(\phi\), also conclude \(\psi,\dots\).
  And, in cases of interest, because of generality of the reasoning.

  This is somewhat puzzling.
  Though, I think less puzzling than first appears.
  Concluding \(\phi\) has value \(v\) is nothing special.
  Of course, the agent only explicitly concludes a handful of things, but allowing the generality is nothing that different from equivalences.

  It also doesn't follow that any of the additional properties of the reasoning, if any, are carried over to any \requ{1}.
  Is just about concluding.
  Here, then, various ways to keep the intuition for the negative resolution.
  There may be various things that are exclusive to witnessing reasoning from premises to conclusion.
  However, distinct from concluding.

  Still, stronger than being committed.
  Ranges over any implication.
  Conclude no winning strategy, then also conclude various other chess things.
  However, committed, but do not necessarily conclude that X is going to lose the game.

  Concluding is still of interest.
  Or, as noted, `reason', in the weak term.
\end{note}

\subsubsection{Resolving tension by rejecting ideas}
\label{sec:resolv-tens-reject}

\begin{note}
  Now, possible to resolve tension various ways.
  Reject \ideaCS{}, \csN{} is of no interest.
  Reject \ESU{}, not witnessing is ok.
  Reject claiming support for general ability.

  Or, any combination of the above.

  Interest is in rejecting \ESU{}.
  So long as \(\psi\) having value \(v'\) follows from some premises, then the reasoning doesn't matter.
  \csV{2} for \(\psi\) having value \(v'\) from those premises, given the possibility of witnessing reasoning.
\end{note}

\subsubsection{Resolving tension by additional ideas}
\label{sec:resolv-tens-addit}

\begin{note}[Strong closure]
  So, we have weak constraints on concluding.
  Is there a way to keep \ESU{} by strengthening closure?

  The idea is that \csN{} relies on the possibility of an independent check.
  However, strong closure leaves open the option for denying independence.

  In certain cases, this seems viable.
  Arithmetic.
  Perhaps this does give everything.

  However, the other cases are more challenging.
  For, in these cases, specific premises.

  Chess.
  Winning strategy from board.
  So, then need to conclude would conclude winning strategy from board.
  Now, idea is that understanding basics already give you this.
  Therefore, as the reasoning from the board requires understanding rules, it also follows that before concluding from board, have already concluded winning strategy from board.

  Stepping back, relevant instance of reasoning requires certain premises.
  Possible obtaining those premises already involves concluding various things.
  If some of those conclusions are that one would not fail to conclude \dots
  Then, there is no space for \requ{1} of the relevant kind.

  For, there will be no `gap' between premises and conclusions of interest.
  Hence, no independent check on whether one would get conclusion from premises.

  This is really strong closure though.
  Intuitively, there is some gap between introduction and understanding.
  This is in part why \csN{} is of interest.
  A sufficiently strong closure principle would need to rule out possibility of failing to \csN{} while having the possibility to reason to the correct answer.
  And this, I don't see as genuinely viable.

  And, if not viable, then tension arises when an agent goes from `merely' concluding to also \csN{}.
\end{note}

\subsubsection{Resolving tension by terminology}
\label{sec:resolv-tens-term}

\begin{note}[`Concluding']
  Reject some of the assumptions regarding `concluding'.
  Or, more generally, recast \csN{} in terms of something like commitment.

  Argument has been developed with the terminology of `concluding' as this seems natural.

  Even if not concluding, take the result to be sufficiently interesting.
\end{note}


\subsection{Observations}
\label{sec:overview:observations}

\subsubsection{Ability}

\begin{note}
  \begin{figure}[H]
    \centering
    \saMtxInterpreted{}
    \caption{Distinction matrix with \aben{the}}
    \label{fig:saMtxInterpreted:outline}
  \end{figure}
\end{note}

\begin{note}
  Recap.

  Claiming support.
  Constraint.

  Ability.
  In order to be compatible, satisfy constraint.
  Either of three options.
  Basic, ignore this.
  Property. Incompatible with constraint.
  Witness. Compatible.

  Here, display the matrix.
  I think this is the easiest way to visualise what is going on.
\end{note}

\paragraph{Deviant causal chains}

\begin{note}
  Do these really matter in the case of reasoning?
\end{note}

\paragraph{Closure principles}

\begin{note}
  No doubt, already observed.
  This does lead to a closure principle, constrained by what reasoning it is possible for the agent to witness.

  There are two perspectives.
  First, leading to tension.
  Second, no need to reason.
\end{note}

%%% Local Variables:
%%% mode: latex
%%% TeX-master: "master"
%%% End: