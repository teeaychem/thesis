\chapter{Overview}
\label{cha:overview}

\section{The topic}
\label{sec:issue}

\begin{note}
  \begin{scenario}[Multiplication]
    \label{illu:gist:calc}
    An agent enters `\(23 \times 15\)' into a calculator and presses the button marked `\(=\)'.
    The calculator displays `\(345\)'.

    \mbox{ }

    The agent observes that they have the option to calculate \(23 \times 15\) via their understanding of arithmetic.
    And, \emph{if} the calculator is trustworthy, then they would not fail to conclude \(23 \times 15 = 345\) via their understanding of arithmetic.

    \mbox{ }

    The agent concludes \(23 \times 15 = 345\).
  \end{scenario}

  Intuitively, the agent concluded \(23 \times 15 = 345\) from the calculator.
  Personifying a little, we may say the agent has concluded \(23 \times 15 = 345\) from the testimony of the calculator.%
  \footnote{
    Indeed, for our purposes \autoref{illu:gist:calc} may be recast in terms of an agent asking another agent to solve `\(23 \times 15\)', but using a calculator is more natural.
  }

  Still, as noted by the agent, they had the option of concluding \(23 \times 15 = 345\) through their own understanding of arithmetic, and hence without the use of the calculator.

  Still, it is intuitive that an agent concludes some proposition has some value%
  \footnote{In \scen{0} the proposition is `\(23 \times 15 = 345\)' and the value is `true'.
    In isolation, `\(23 \times 15 = 345\)' describes some possible state of affairs, and assigning the value `true' indicates the possible state of affairs is the actual state of affairs.
    Still, the agent may have concluded `\(23 \times 15 = 345\)' is `desired', `impossible', `probable', and so on.
    When speaking generally, we will keep explicit which value a proposition is paired with, though when describing specific \scen{0} or examples, we will leave the associated value implicit.
  }
  from some pool of premises only if the agent reasoned from those premises to the proposition-value pair.

  In other words, the agent may have concluded \(23 \times 15 = 345\) via their understanding of arithmetic, but as the agent did not calculate \(23 \times 15\) themselves, they did not conclude \(23 \times 15 = 345\) from whatever premises would be involved when reasoning via their understanding of arithmetic.

  Indeed, given the agent's understanding of arithmetic, it seems clear that prior to using the calculator the agent knew \emph{whether} \(23 \times 15 = 345\).
  Though, knowing whether \(23 \times 15 = 345\) is not knowing \(23 \times 15 = 345\).
  For example, I expect --- though one of the following equalities does not hold --- you know whether \(345 \times 11 = 3,795\), whether \(3,795 \div 5 = 760\), and whether \(760 \div 8 = 95\).

  Rephrasing things a little, and keeping track of truth, we may say \(23 \times 15 = 345\) was a \fc{} for the agent in \autoref{illu:gist:calc}.
  And, for you, \(345 \times 11 = 3,795\), \(3,795 \div 5 \ne 760\), and \(760 \div 8 = 95\) are \fc{1}.

  Instead, it seems the agent concluded \(23 \times 15 = 345\) by use of the calculator, regardless of whether \(23 \times 15 = 345\) was a \fc{1}.
  And, likewise, if you have concluded \(3,795 \div 5 \ne 760\) from the paragraph above, it was due to my testimony, and not your understanding of arithmetic.
\end{note}

\begin{note}[Summary of basic intuitions]
  So, intuitively, in \autoref{illu:gist:calc} the agent concluded \(23 \times 15 = 345\) from the testimony of the calculator.
  And, intuitively, \(23 \times 15 = 345\) being a \fc{0} had no significant role in the agent concluding \(23 \times 15 = 345\) from the testimony of the calculator.
\end{note}

\begin{note}[Not just concluding]
  So far we have spoken about intuitions of the basic form:

  \begin{itemize}
  \item
    Agent \(A\) concluded some proposition \(\phi\) has some value \(v\) from some pool of premises \(\Phi\).
  \end{itemize}

  I take intuitions of this basic form to be readily available in a variety of \scen{1}, and I also take those intuitions expressed in regards to \autoref{illu:gist:calc} to be fairly clear.

  Still, concluding is something an agent does, and in this respect there are (at least) two distinct questions intuitions of the above form may answer:%

  \begin{itemize}
  \item
    \emph{How} did the agent conclude \(\phi\) has value \(v\)?
  \item
    \emph{Why} did the agent conclude \(\phi\) has value \(v\)?
  \end{itemize}

  In basic form, focus is on what the agent did, or alternatively whether some action way performed.
  Did the agent conclude \(\phi\) has value \(v\)?
  Or, did the agent conclude \(\phi\) has value \(v\) from the pool of premises \(\Phi\)?
  Or, perhaps, did the agent fail to conclude \(\phi\) has value \(v\)?

  `How?' and `why?' by contrast, take for granted the agent concluded \(\phi\) has value \(v\) and consider, respectively, how and why this action was performed.

  How do intuitions regarding whether or not an agent performed an action relation to how and why the agent performed the action?
\end{note}

\begin{note}
  `How?' and `why?' are distinct questions.
  So, generally speaking, `how?' and `why?' may have distinct answers.

  For example, consider replacing `conclude \(\phi\) has value \(v\)'  with `arrive at location \(l\)'.
  `By bicycle' answers how the agent arrived at the post office, but does not answer why the agent arrived at the post office.
  likewise, `to post a letter' intuitively answers why the agent arrived at the post office, but does not answer how the agent arrived at the post office.%
  \footnote{
    Of course, things get complex.
    Action, motivating reason, belief-desire pair.
    Hence, desire to post a letter is part of how.

    Preface with `intuitively'.
    The point is that may refine intuition regarding the agent concluding \(23 \times 15 = 345\).
  }

  So, the intuitions expressed with respect to \autoref{illu:gist:calc} may answer `how?' but not `why?', or `why?' but not `how?'.

  Still, I suspect this is not the case.
  In the case of \autoref{illu:gist:calc} both have the same rough answer:

  \begin{itemize}
  \item
    The testimony of the calculator, is, in part, \emph{both} how \emph{and} why the agent concluded \(23 \times 15 = 345\).
  \end{itemize}
  That premises associated with the agent's understanding of arithmetic do not answer either `how?' or `why?' is implicit by omission.

  Again, the agent used the testimony of the calculator to conclude \(23 \times 15 = 345\), and the agent appealed to the testimony of the calculator to conclude \(23 \times 15 = 345\).
  The agent did not use their understanding of arithmetic to conclude \(23 \times 15 = 345\), and the agent did not appeal to their understanding of arithmetic to conclude \(23 \times 15 = 345\).
\end{note}

\begin{note}
  Our observation that the testimony of the calculator seems to answer both `how?' and `why?' the agent concluded \(23 \times 15 = 345\) suggests, even if --- as a single \scen{0} --- only slightly, the following basic idea:%
  \footnote{
    The observation also suggests the converse holds --- an answer, in part, to `how?' is also, in part, an answer to `why?' --- though I think the converse faces some immediate difficulties.
    For, it seems answers to `how?' may include details that are irrelevant to `why?'.
    For example, typing digits and operators into the calculator answers, in part, how the agent concluded \(23 \times 15 = 345\) but these actions seems irrelevant to why the agent concluded \(23 \times 15 = 345\).
    Rather, an answer to `why?' seems to be limited to the calculator providing testimony that \(23 \times 15 = 345\), regardless of whether it was the agent who used the calculator, or whether the agent observed someone else using the calculator.
  }

  \begin{itemize}
  \item
    Something is, in part, an answer to `why?' only if that something is (also), in part, an answer to `how?'.
  \end{itemize}

  In other words, in order for something to, in part, account for \emph{why} an agent concluded \(\phi\) has value \(v\), that something must also be part of \emph{how} the agent concluded \(\phi\) has value \(v\).

  With respect to \autoref{illu:gist:calc}, the testimony of the calculator satisfies the constraint imposed by the idea, while the agent's understanding of arithmetic does not.
  Specifically, the testimony of the calculator was part of how the agent concluded \(23 \times 15 = 345\), and so the testimony of the calculator may be, in part, an answer to why the agent concluded \(23 \times 15 = 345\).
  However, the agent's understanding of arithmetic was \emph{not} part of how the agent concluded \(23 \times 15 = 345\), and so the agent's understanding of arithmetic \emph{may not}, in part, an answer to why the agent concluded \(23 \times 15 = 345\).

  In addition, the basic idea may be taken to capture some explanatory significance and we may even say:
  The agent's understanding of arithmetic is not, in part, an answer to why the agent concluded \(23 \times 15 = 345\) \emph{because} the agent's understanding of arithmetic was \emph{not} part of how the agent concluded \(23 \times 15 = 345\).
\end{note}

\begin{note}
  The basic idea, rather than intuitions regarding specific \scen{1} is our interest.
  Roughly, at least.%
  \footnote{
    We will shortly refine our understanding of `how?' and `why?' to focus on support between premises and conclusions, and will motivate a slightly weaker idea with respect to support.
  }

  Additional \scen{1} may provide additional motivation for the basic idea.
  Though, I think the basic idea is sufficiently intuitive independently of individual \scen{1}.
  Instead, observe the basic idea may be motivated not only by \scen{1}, but also by theories.
  Perhaps the most prominent is \citeauthor{Davidson:1963aa}' causal theory of action.

  \citeauthor{Davidson:1963aa} opens \textcite{Davidson:1963aa} with the following question:

  \begin{quote}
    What is the relation between a reason and an action when the reason explains the action by giving the agent's reason for doing what he did?
    We may call such explanations \emph{rationalizations}, and say that the reason \emph{rationalizes} the action.%
    \mbox{}\hfill\mbox{(\citeyear[685]{Davidson:1963aa})}
  \end{quote}

  As noted, concluding is an action, and hence our question `why?' is a particular instance of \citeauthor{Davidson:1963aa}' question.
  And, answers to `why?' will be reasons that rationalise the agent concluding \(\phi\) has value \(v\).

  \citeauthor{Davidson:1963aa} argues, in short, for the following answer to the relation between a reason and the rationalisation of an action:

  \begin{quote}
    \begin{enumerate}[label=\arabic*]
      [R]ationalization is a species of ordinary causal explanation.\newline
      \mbox{ }\hfill\mbox{(\citeyear[685]{Davidson:1963aa})}
    \end{enumerate}
  \end{quote}

  Following \citeauthor{Davidson:1963aa}, an answer to `why?' is a rationalisation, rationalisation is an instance of ordinary causal explanation.
  So, the answer to why an agent concluded \(\phi\) has value \(v\) will, in part, by a cause of the agent concluding \(\phi\) has value \(v\).
  Therefore, an answer to why an agent concluded \(\phi\) has value \(v\) is, in part, an answer to how the agent concluded \(\phi\) has value \(v\).

  Implicit in this quick argument is the idea that a causal explanation answers `how?'.
  Note, however, that we did not appeal to the converse.
  Causal theories of action seem to motivate the basic idea, though I do not think the basic idea (directly, at least) motivates causal theories of action (or, specifically, concluding).
  In other words, for our purposes, answers to `how?' need not be causal explanations, though they may be.

  More broadly, I take the basic idea to capture a pre-theoretical constraint on classes of theories.
  There are theories that agree with the basic idea, such as \citeauthor{Davidson:1963aa}' causal theory of action (when the action is concluding) and there \emph{may be} theories which do not agree with the basic idea --- though I do not know of any specific theories that are explicitly of this kind.
\end{note}

\begin{note}
  The basic idea is more-or-less the basic issue of this document.

  Both intuitions, such as those regarding \autoref{illu:gist:calc}, and theories, such as \citeauthor{Davidson:1963aa} causal theory of action, provide motivation for the basic issue.

  Our goal is to motivate the following, basic, contrary idea:

  \begin{itemize}
  \item
    There are cases in which something is, in part, an answer to `why?' and that something is \emph{not} (also), in part, an answer to `how?'.
  \end{itemize}

  The basic contrary idea is the negation of the basic idea.
  For, the basic idea states, roughly, answers to `why?' are always included in answers to `how?' while the basic contrary idea states that there are cases in which something that answers `why?' does not also answer `how?'.

  The basic contrary idea, then, has the form of an existential.
  We will not motivate the idea that there is always something which answers `why?' but does not also answer `how?'.
  And, in particular, it may be the case that the intuitions observed with respect to \autoref{illu:gist:calc} are correct.
\end{note}

\begin{note}
  Now, all this has been said without giving attention to the conditional observed by the agent in \autoref{illu:gist:calc}:

  \begin{itemize}
  \item
    If the calculator is trustworthy, then the agent would not fail to conclude \(23 \times 15 = 345\) via their understanding of arithmetic.
  \end{itemize}

  Both natural, and somewhat surprising.

  Consider the contraposition.

  \begin{itemize}
  \item
    If the agent were to fail to conclude \(23 \times 15 = 345\) via their understanding of arithmetic, then the calculator is not trustworthy.
  \end{itemize}

  \begin{itemize}
  \item
    Is it possible, from agent's perspective, fail to conclude \(23 \times 15 = 345\) via their understanding of arithmetic?
  \end{itemize}

  If possible, then difficulty.
  For, testimony, so must be, but at the same time, possible that it is not.
  If not possible, then it doesn't seem that observing \(23 \times 15 = 345\) via the testimony of the calculator is sufficient.
  For, by the previous observation, difficulty with the testimony of the calculator.

  \begin{itemize}
  \item
    Testimony of the calculator \emph{only if} \(23 \times 15 = 345\) is a \fc{0} given the agent's understanding of arithmetic.
  \end{itemize}
\end{note}

\begin{note}
  The only if, interesting.

  Calculator provides information about what is a \fc{}.
  Agent's understanding of arithmetic is why it is a \fc{}.

  So, if being a \fc{0} is involved in concluding, then it may be that understanding of arithmetic is, in part, an answer to `why?'.
  And, as the agent has not concluded, not, in part, an answer to `how?'.
\end{note}

\begin{note}
  Important:

  Multiple ways to conclude.
  So, have a check.

  Differs to, for example, concluding {\color{red} ???} from a scientific calculator.
  {\color{red} ???} goes beyond typical understanding of arithmetic.
  Parallel pair of conditionals does not hold.

  Or, alternatively, testimony that {\color{red} ???}.
  Beyond understanding.
\end{note}

\begin{note}
  I am not sure what to make of \ref{illu:gist:calc}.
  Understanding of arithmetic is a partial check.
  However, testimony.

  Unsure because status of a premise.

  Basic contrary idea only requires some instances.

  Argument against this intuition.
  Type of cases, premises are fixed.
  Check on own reasoning.

  First, expand on intuition.
  Then, introduce type of \scen{0} of interest.
\end{note}

\begin{note}
  Three things of interest.

  \begin{enumerate}
  \item
    \scen{1} like \autoref{illu:gist:calc} in which an agent concludes some proposition has some value.
  \item
    Relation between Why and how and agent concludes.
    Basic idea, and basic contrary idea.
  \item
    Idea of a \fc{} and how \fc{1} may be involved in concluding.
  \end{enumerate}

  These three 
\end{note}

\begin{note}
  This chapter will be split into three parts.

  In \autoref{overview:sec:interest}, we will clarify what is of interest.
  In particular, \autoref{overview:sec:interest} will be split into two subsections.

  In~\autoref{overview:sec:support} we will clarify why we are interested in intuitions regarding \scen{1} such as \autoref{illu:gist:calc}.
  And, more generally, 
  
  

  First, clarify what is of interest.
  Two things in particular.
  1. What it is about concluding.
  2. Scenarios of interest.
  \autoref{illu:gist:calc} is similar to the type of \scen{0} that will be the focus on this document.
  However, the argument we provide will not directly apply to \scen{1} like \autoref{illu:gist:calc}.

  In \autoref{overview:sec:type-of-scen} we present the type of \scen{0} we will present an instance of the type of \scen{0} interested in, provide a general description of the \scen{0} type.
  Further, we will provide a detailed contrast between the type of \scen{0} we are interested in and \autoref{illu:gist:calc}.%
  \footnote{
    Roughly, if it were the agent failed to conclude \(23 \times 15 = 345\) in \autoref{illu:gist:calc}, then there would be conflict between the agent's understanding of arithmetic and the testimony of the calculator.
    Expressed differently, there would be conflict between the agent's failure to conclude \(23 \times 15 = 345\) by their understanding of arithmetic, and a premises involved in concluding \(23 \times 15 = 345\) via the calculator.
    I.e. supposing the agent concludes \(23 \times 15 \ne 345\), then from the agent's perspective the calculator is not a source of testimony.
    In the \scen{1} of interest, this hypothetical --- or in some cases possible --- conflict will strictly be between the agent's reasoning from pools of premises to conclusions.
  }

  Generally speaking, I am unsure about the intuitions expressed with respect to~\autoref{illu:gist:calc}.
  \Autoref{illu:gist:calc} shares an important feature with the type of \scen{0} that we will explore in detail.
  Yet, 


  
  I am inclined to think intuitions are fine.
  variation of scenario, what the difference is, and why focus.
  Second, sketch general argument of the paper, and role of \fc{1}.
\end{note}

\section{Interest}
\label{overview:sec:interest}

\subsection{Intuitions and support}
\label{overview:sec:support}


\subparagraph*{Support}

\begin{note}
  Our interest is with the process of an agent concluding some proposition-value pair from some pool of premises.

  An agent concluding some proposition-value pair from some pool of premises is an event --- concluding is something that happens --- and, in particular, concluding is an act --- the agent is involved in that something happening.%
  \footnote{
    This passive construction is to avoid characterising an act as something brought about by an agent.
    Of course, agent being involved doesn't make an act.
    Involvement of agent, exist, but not an act.
    On the other hand, waking up.
    {
      \color{red}
      Also, Freud's example with adjourning a meeting.
      No intention, but perhaps conclusion.
    }
    {
      \color{red} This line of inquiry doesn't really matter.
    }
  }
  Concluding also involves some proposition-value pair, the conclusion, and some pool of premises.
  Concluding is directed.
  Agent concluded the proposition-value pair from the pool of premises.

  When I think of prefectures, I think of Gifu.

  Distinguish.
  Relation of support.
  Rather than provide an independent characterisation of support, we instead hold this as a necessary condition.

  Assumption:

  \begin{assumption}[Support]
    \label{assumption:support}
    If:
    \begin{itemize}
    \item
      An agent has concluded some proposition \(\phi\) has some value \(v\) from some pool of premises \(\Phi\).
    \end{itemize}
    Then:
    \begin{itemize}
    \item
      From the perspective of the agent, a \emph{relation of support} holds between \(\Phi\) and \(\phi\) having value \(v\).
    \end{itemize}
    \vspace{-\baselineskip}
  \end{assumption}
\end{note}

\begin{note}
  As we will shortly see, links to reasons.
  However, support is distinct from reasons.

  Relation of support holds between conclusion and pool of premises.
  Between some proposition-value pair and a collection of proposition-value pairs.

  Explanatory reason holds between something which explains and an action.

  Support involved in providing explanatory reason.
\end{note}

\begin{note}
  For example, \citeauthor{Boghossian:2014aa}'s Taking Condition:%
  \footnote{
    There are various objections to the taking condition.

    See, for example,~\textcite{Hlobil:2014tq}, \textcite{McHugh:2016vp}, and~\textcite{Wright:2014tt}.

    \citeauthor{Hlobil:2014tq} argues against the Taking Condition as it distracts from what accounts of reasoning out to explain, rather than arguing against the Taking Condition directly.

    \citeauthor{McHugh:2016vp} summarise various objects to the taking condition, and present district arguments against against (distinct) ideas in favour of the taking condition.
    In particular,~\autoref{assumption:support} is closer to what \citeauthor{McHugh:2016vp} term the `Consequence Condition' (\citeyear[cf.][316]{McHugh:2016vp}), which \citeauthor{McHugh:2016vp} also (indirectly) argue against.
    However, \citeauthor{McHugh:2016vp} does not consider an alternative account of what distinguishes concluding from any other action, and as~\autoref{assumption:support} is designed to capture this distinction, it is unclear to me whether \citeauthor{McHugh:2016vp}'s arguments apply to~\autoref{assumption:support} (if, indeed, they are sound).

    \citeauthor{Wright:2014tt} denies that reasoning must involve a state which connects premises to conclusions. (\citeyear[Cf.][33-34]{Wright:2014tt})
    Note,~\autoref{assumption:support} is compatible with \citeauthor{Wright:2014tt}'s objection.
  }

  \begin{quote}
    (Taking Condition):
    Inferring necessarily involves the thinker \emph{taking} his premises to support his conclusion and drawing his conclusion because of that fact.%
    \mbox{}\hfill\mbox{(\Citeyear[5]{Boghossian:2014aa})}
  \end{quote}

  \begin{quote}
    The intuition behind the Taking Condition is that no causal process counts as inference, unless it consists in an attempt to arrive at a belief by figuring out what, in some suitably broad sense, is supported by other things one believes.%
    \mbox{}\hfill\mbox{(\Citeyear[5]{Boghossian:2014aa})}
  \end{quote}

  Inference --- and hence reasoning with beliefs --- rather than reasoning more broadly (\Citeyear[cf][2]{Boghossian:2014aa}).

  \citeauthor{Boghossian:2014aa}, `taking'.
  Taking is required for support to explain.
  No explanation without taking.
\end{note}

\begin{note}
  \citeauthor{Broome:2002aa}'s `jogging account' of reasoning:

  \begin{note}
    [I]n reasoning you call to mind some of the premises, and doing so jogs into operation an automatic process that causes you to acquire a conclusion-attitude.%
    \mbox{}\hfill\mbox{(\citeyear[226]{Broome:2002aa})}
  \end{note}

  \citeauthor{Broome:2002aa} argues some things which satisfy jogging are clearly not reasoning.

  \citeauthor{Broome:2002aa} endorses rule following.

  \begin{quote}
    Active reasoning is a particular sort of process by which conscious premiseattitudes cause you to acquire a conclusion-attitude. The process is that you operate on the contents of your premise-attitudes following a rule, to construct the conclusion, which is the content of a new attitude of yours that you acquire in the process.\newline
    \mbox{ }\hfill\mbox{(\citeyear[234]{Broome:2002aa})}
  \end{quote}

  Understand support of \autoref{assumption:support} in terms of having followed a rule.
\end{note}

\begin{note}
  Present purposes, assume relation of support is involved in providing explanatory reasons for concluding.
\end{note}

\begin{note}
  Concluding, involves: agent, proposition-value pair as conclusion, pool of premises, and relation of support between pool of premises and conclusion.
\end{note}

\begin{note}
  Key assumption regarding concluding.
  Further assumptions detailed in~\autoref{chapter:concluding}.
\end{note}

\begin{note}
  Understand relation of support to, at least in part, capture why.

  Simple example.

  \begin{enumerate}
  \item
    \(a\) testified \(\phi\) has value \(v\).
  \item
    \(\phi\) has value \(v\)
  \end{enumerate}

  Support, for the agent, only possible to testify if is the case.

  Break this down further, but, unopinionated on what counts as a premise and how pools of premises may support proposition-value pairs.
\end{note}


\begin{note}[Pool of premises]
  Pool of premises, difficult to specify.
  Clear sense in which \(p\) and \(p \rightarrow q\) are premises from which an agent concludes \(q\) via modus pones.

  Understanding of arithmetic is not so clear.
  Robinson or Peano arithmetic together with the two numbers joined by an operator.
  Though, I doubt this.
  Still, something.

  Conclusion doesn't come for free, and pool of premises identifies whatever.

  Empty set, why not.
\end{note}

\begin{note}
  Understanding of support is generic.

  From agent's perspective.
  Perhaps the calculator is faulty, or the agent has an a unsteady grip on arithmetic.
  Still, conclusion from somewhere.
  Something account for why the agent holds.
\end{note}

\begin{note}[Support \emph{from the agent's perspective}]
    \begin{illustration}[A box of flan(nels)]
    \label{illu:flan-nels}
    Suppose `flan' is written on the side of a container.
    I may claim support that the container contains flan.
    And, it may be that the writing on the side of the container is support for the box containing flan.
    However, the straps ensuring the container remains closed is unfortunately placed, and if moved would reveal the side of the container reads `flannels'.
  \end{illustration}

  The unfortunate placing of the straps does not seem to prevent concluding, but I'm not sure whether it is right to say that the writing on the side of the box (straps in place) hence support.

  However, from an employee of the factory, no concluding, no support.
\end{note}

\begin{note}
  \phantlabel{mention:concluding-non-factive}
  In this respect, doesn't matter whether premises have values, or whether conclusion has value.
  Provide an \illu{0} of this in~\autoref{chapter:concluding}.%
  \footnote{
    \color{red}
    On page~\pageref{concluding:not-factive}.
  }
\end{note}

\begin{note}
  {
    \color{red}
    Not assuming that concluding involves belief.
  }
\end{note}

\subsection{Support: Why? and How?}

\begin{note}[Role of support]
  Identify key thing about concluding.
  \autoref{assumption:support}, only if.
  In other words, necessary.
\end{note}


\begin{note}[Why and how]
  State our interest in terms of the relationship between answers to two questions.

  Begin with the two questions.
  We state each question broadly, and then provide an alternative refinement in terms of relations of support.
\end{note}

\begin{note}[\qWhy{} and \qHow{}]
  \begin{question}[\qWhy{}?]
    \label{q:why}
    \mbox{ }
    \vspace{-\baselineskip}
    \begin{itemize}
    \item
      \emph{Why} has the agent concluded \(\phi\) has value \(v\)?
    \end{itemize}

    Alternatively:

    \begin{itemize}
    \item
      \emph{Which} relations of support are involved in the agent concluding \(\phi\) has value \(v\)?
    \end{itemize}
    \vspace{-\baselineskip}
    \vspace{-\baselineskip}
  \end{question}

  \Autoref{q:why}, why.
  Given a conclusion, which relations of support matter.
  From perspective of agent.
\end{note}

\begin{note}
  Our interest is with answers to \qWhy{}, rather than providing an answer to \qWhy{}.
  For, answers to \emph{why} an agent concluded may be constrained by \emph{how} an agent concluded.
  In other words, answers to \qWhy{} may be constrained by answers to \qHow{}:

  \begin{question}[\qHow{}?]
    \label{q:how}
    \mbox{ }
    \vspace{-\baselineskip}
    \begin{itemize}
    \item
      \emph{How} did the agent conclude \(\phi\) has value \(v\)?
    \end{itemize}

    Alternatively:

    \begin{itemize}
    \item
      For all the relations of support involved in the agent concluding \(\phi\) has value \(v\), \emph{in what way} did those relations of support come to be involved in the agent concluding \(\phi\) has value \(v\)?
    \end{itemize}
    \vspace{-\baselineskip}
    \vspace{-\baselineskip}
  \end{question}

  What information was appealed to by the agent to conclude \(\phi\) has value \(v\)?

  For example, \autoref{illu:gist:calc}, agent concluded \(23 \times 15 = 345\) from the testimony of the calculator.
\end{note}

\begin{note}
  \qWhy{} and \qHow{} are distinct questions.
  Quick answers to \qWhy{} and \qHow{} coincide.
  However, may add additional features to answer to \qHow{} that are irrelevant to \qWhy{}.
  For example, agent typed\dots pressed enter\dots

  Describes how the agent obtained testimony from the calculator, but see irrelevant to \qWhy{}.
  Watched a different agent type in.

  Still, included.
\end{note}

\begin{note}
  Consider the following issue:

  \begin{restatable}[Inclusion]{issue}{issueInclusion}
    \label{issue:why-inc-in-how}
    Is an answer to \qWhy{} always included in an answer to \qHow{}?
  \end{restatable}

  Concluding, reasoning, proposition-pairs.

  \Autoref{issue:why-inc-in-how} distinguishes classes of theories.
  A positive resolution to~\autoref{issue:why-inc-in-how} will not directly provide general answer to \qWhy{}.
  Though, a positive answer will rule out certain answers.

  For example, wrt.~\autoref{illu:gist:calc}.

  Indeed, accounts for the intuitions noted.
  Testimony of the calculator, but not agent's understanding of arithmetic.

  As we will see below, many accounts of concluding seem to fall in line with this.
\end{note}

\begin{note}[Analogy]
  By analogy, whether or not my mug of (once cold) coffee overheats in the microwave is the result of some process involving electromagnetic radiation.
  My desire that the mug of coffee does not overheat is not used as part of the process of heating the coffee, and so is irrelevant to the structure of the process.

  My desire may explain why the mug of coffee is taking part in a certain process, and an unused premise or step may explain why an agent performed so reasoning.
  Still, a premise or step must be used as part of the process of reasoning to stand in explanation for the result of reasoning.

  Press the analogy further: Reasoning is a causal process.
  And, any property of reasoning reduces to cause and effect.
  If premises or steps are not used, then those premises or steps stands outside the relevant causal trace, and may not be appealed to when accounting for some structural property of the conclusion of the instance of reasoning (here, that the agent claims support for the conclusion).
\end{note}

\begin{note}
  Inclined to think intuitions expressed may be correct.
  Though, not because the resolution to \autoref{issue:why-inc-in-how} is positive.

  For, general type of scenario \autoref{illu:gist:calc} conforms to falls just outside the scope of the core argument of this document.

  Hence, no clear reason to resist.
  Negative resolution to \autoref{issue:why-inc-in-how} is existential rather than universal.
\end{note}

\begin{note}
  Still, motivating a negative resolution to \autoref{issue:why-inc-in-how} is a goal of this document.

  \begin{goal}
    Motivate a negative resolution to~\autoref{issue:why-inc-in-how}.
  \end{goal}

  Different type of \scen{0}.
  Though, I expect initial intuitions will be similar.
  If skip ahead \dots

  First, loosen \autoref{issue:why-inc-in-how} slightly.
\end{note}

\paragraph*{Broader issue/Witnessing}

\begin{note}
  \begin{restatable}[Witnessing]{issue}{issueWitnessing}
    \label{issue:has-witnessed}
    Is it the case that:

    If a relation of support between \(\psi\) having value \(v'\) and some pool of premises \(\Psi\) answers, or is part of the answer to, \qWhy{} an agent concluded \(\phi\) has value \(v\), then the agent has witnessed reasoning from \(\Psi\) to \(\psi\) having value \(v'\).
  \end{restatable}

  {
    \color{red}
    \Autoref{issue:has-witnessed} generalises~\autoref{issue:why-inc-in-how} in two ways.
    \begin{itemize}
    \item
      Current reasoning.
    \item
      Other support relations.
    \end{itemize}
  }

  In contrast to~\autoref{issue:why-inc-in-how},~\autoref{issue:has-witnessed} does not require a relation of support which answers \qWhy{} to involve a pool of premises drawn from the agent's present reasoning.
  Instead, so long as the agent has --- either through their present reasoning or in some past reasoning --- witnessed reasoning from the pool of premises to the conclusion, the relevant relation of support may answer \qWhy{}.
\end{note}

\begin{note}
  \begin{restatable}[Witnessing]{idea}{ideaWitnessing}
    \label{idea:Witnessing}
    A relation of support holds between a pool of premises \(\Phi\) and a conclusion that \(\phi\) has value \(v\) is in part an answer to \qWhy{} \emph{only if} the agent has witnessed reasoning from \(\Phi\) to \(\phi\) having value \(v\).
  \end{restatable}

  When concluding, witnesses reasoning from pool of premises to conclusion.

  Relation of support between pool of premises and conclusion determines why the agent concluded \(\phi\) has value \(v\).

  Concluding, result of reasoning.
  Some relation of support between conclusion and pool of premises.
  From some inputs to some output.
  Process information.
  Relation of support from some of this information.

  May seem natural.
  Indeed, always have the resources.
  Above example, calculator is both how and why.
  If had reasoned from understanding of arithmetic, then that would be both how and why.
\end{note}

\begin{note}
  Before working through~\autoref{issue:has-witnessed} in a little more detail, let us observe

  \begin{restatable}%
    [Relations between resolutions to issues~\ref{issue:why-inc-in-how} and~\ref{issue:has-witnessed}]%
    {proposition}{propIssueRelation}
    \label{prop:IssueRelation}
    Issues~\ref{issue:why-inc-in-how} and~\ref{issue:has-witnessed} are related in the following two ways:
    \begin{itemize}
    \item
      A negative resolution to~\autoref{issue:has-witnessed} ensures a negative resolution to~\autoref{issue:why-inc-in-how}.
    \item
      A negative resolution to~\autoref{issue:why-inc-in-how} does not ensure a negative resolution to~\autoref{issue:has-witnessed}.
    \end{itemize}
    \begin{argument}
      Grant a negative resolution to~\autoref{issue:has-witnessed}.
      Then, there are cases in which a relation of support \(\phi\) having value \(v\) and some pool of premises \(\Phi\) answers, or is part of the answer to \qWhy{} an agent concluded \(\phi\) has value \(v\), such that the agent has not witnessed reasoning from \(\Phi\) to \(\phi\) having value \(v\).
      Therefore, it cannot be the case that in concluding \(\phi\) has value \(v\) the agent witnessed reasoning from \(\Phi\) to \(\phi\) having value \(v\).
      So, the cases which ensure a negative resolution to~\autoref{issue:has-witnessed} also ensure a negative resolution to~\autoref{issue:why-inc-in-how}.

      Conversely, a positive resolution to~\autoref{issue:has-witnessed} will ensure the agent has witnessed reasoning from \(\Phi\) to \(\phi\) having value \(v\).
      Still, if the witnessed reasoning was strictly in the past, the reasoning which secures a positive resolution to~\autoref{issue:has-witnessed} will not secure a positive resolution to~\autoref{issue:why-inc-in-how}.
    \end{argument}
  \end{restatable}
\end{note}

\begin{note}
  \autoref{prop:IssueRelation} as it shows an argument for a negative resolution to~\autoref{issue:has-witnessed} is also an argument for a negative resolution to~\autoref{issue:why-inc-in-how}.
  Hence, as our goal is to motivate a negative resolution to~\autoref{issue:has-witnessed}, \dots
\end{note}

\begin{note}
  \begin{goal}
    Motivate a negative resolution to~\autoref{issue:has-witnessed}.
  \end{goal}
\end{note}

\begin{note}
  For example, look through this document.
  Read~\autoref{prop:IssueRelation}.
  Recall providing the argument.
  Whatever pool of premises are extracted from the argument, rather than testimony of my current self.

  Though, to motivate the idea, something a little more emotional.

  Note from a loved one.
  Tear up.
  Not the note, it's them saying the words.

  Of course, might go the other way.
  Well, you can't see something that's not there, but on the other hand piece of paper.

  Likewise, scared of spider, but the spider left sight a long time ago.
\end{note}

\begin{note}
  So, may take~\autoref{issue:has-witnessed} to be weaker than~\autoref{issue:why-inc-in-how}.
  If not, then recast.

  Negative resolution to~\autoref{issue:has-witnessed}.
  Hence, also, negative resolution to~\autoref{issue:why-inc-in-how}.

  Still, \iWitness{} as our main point of interest.
\end{note}

\subsection{The type of \scen{0} we will focus on}
\label{overview:sec:type-of-scen}

\begin{note}
  \begin{scenario}[Quadratic roots]
    \label{illu:gist:roots}
    An agent is given the following statement:

    \begin{enumerate}[label=\arabic*., ref=(\arabic*)]
    \item
      \label{illu:gist:roots:eq}
      For some \(x \in \mathbb{R}\), \(2x^{2} - x - 1 = 0\).
    \end{enumerate}

    The agent reasons as follows:

    \begin{enumerate}[label=\arabic*., ref=(\arabic*), resume, itemsep=.125em]
    \item
      \label{illu:gist:roots:qf}
      The quadratic formula is \(x = \frac{-b \pm \sqrt{b^{2} - 4ac}}{2a}\) \hfill Memory
    \item
      \label{illu:gist:roots:subs}
      \(a = 2\), \(b = -1\), \(c = -1\) \hfill \ref{illu:gist:roots:eq}, How to use \ref{illu:gist:roots:qf}%
      \footnote{
        \(a\) is the coefficient of the \(x^{2}\) term, \(b\) is the coefficient of the x term, and \(c\) is the constant.
      }
    \item
      \label{illu:gist:roots:qf-subs}
      \(x = \frac{-(-1) \pm \sqrt{(-1)^{2} - 4(2)(-1)}}{2(2)}\) \hfill \ref{illu:gist:roots:qf}, \ref{illu:gist:roots:subs}, Substitution
    \item
      \label{illu:gist:roots:qf:1}
      \(x = \sfrac{(1 \pm 3)}{4}\) \hfill \ref{illu:gist:roots:qf-subs}, Simplification
    % \item
    %   \label{illu:gist:roots:qf:3}
    %   \(x = \sfrac{(1 + 3)}{4}\) or \(x = \sfrac{(1 - 3)}{4}\) \hfill \ref{illu:gist:roots:qf:1}, Expansion
    \item
      \label{illu:gist:roots:qf:done}
      \(x = 1\) or \(x = -\sfrac{1}{2}\) \hfill \ref{illu:gist:roots:qf:1}, Expansion, Simplification
    \end{enumerate}
    Hence, the agent concludes: if \(2x^{2} - x - 1 = 0\), \(x = 1\) or \(x = -\sfrac{1}{2}\).

    \mbox{ }

    Still, prior to concluding \(x = 1\) or \(x = -\sfrac{1}{2}\), the agent observed that \emph{if} \(x = 1\) or \(x = -\sfrac{1}{2}\), then they would also be able to observe this via factorisation.
  \end{scenario}

  The particular details of \autoref{illu:gist:roots} are present primarily to present a clear instance of reasoning.
  For present purposes, our interest is with steps \ref{illu:gist:roots:qf}, \ref{illu:gist:roots:subs}, and \ref{illu:gist:roots:qf-subs}.

  Intuitively, the agent concludes `\(x = 1\) or \(x = -\sfrac{1}{2}\) if \(2x^{2} - x - 1 = 0\)' in part from their understanding of arithmetic.
  And, in particular, from their understanding of the quadratic formula and how to use it.

  Further, as the agent does not derive the quadratic formula from more basic principles, the quadratic formula is, intuitively, a premise of the agent's reasoning.
\end{note}

\begin{note}
  Now, in the same way the agent may have calculated \(23 \times 15 = 345\) without the aid of a calculator in \autoref{illu:gist:calc}, the agent may have concluded \(x = 1\) or \(x = -\sfrac{1}{2}\) if \(2x^{2} - x - 1 = 0\) without the aid of the quadratic formula in~\autoref{illu:gist:roots}.

  For example, consider the following variant steps:
  \begin{quote}
    \begin{enumerate}[label=\arabic*\('\)., ref=(\arabic*\('\)), itemsep=.125em]
      \setcounter{enumi}{1}
    \item
      \label{illu:gist:roots:factor}
      \((2x + 1)(x - 1) = 0\) \hfill \ref{illu:gist:roots:eq}, Factoring
    \item
      \label{illu:gist:roots:zero}
      Either \((2x + 1) = 0\) or \((x - 1) = 0\) \hfill \ref{illu:gist:roots:factor}, Arithmetic
    \item
      \label{illu:gist:roots:case:a}
      If \((x - 1) = 0\), then \(x = 1\) \hfill \ref{illu:gist:roots:factor}, \ref{illu:gist:roots:zero}, Arithmetic
    \item
      \label{illu:gist:roots:case:b}
      If \((2x + 1) = 0\), then \(x = -\sfrac{1}{2}\) \hfill \ref{illu:gist:roots:factor}, \ref{illu:gist:roots:zero}, Arithmetic
    \item
      \label{illu:gist:roots:factor:done}
      Either \(x = 1\) or \(x = -\sfrac{1}{2}\). \hfill \ref{illu:gist:roots:zero}, \ref{illu:gist:roots:case:a}, \ref{illu:gist:roots:case:b}, Replacement
    \end{enumerate}
  \end{quote}

  Steps~\ref{illu:gist:roots:factor} to~\ref{illu:gist:roots:factor:done} yield the same results as steps~\ref{illu:gist:roots:qf} to~\ref{illu:gist:roots:qf:done}, but do not involve the quadratic formula.
\end{note}

\begin{note}
  Intuitively, did not conclude from a pool of premises which does not include the quadratic formula.
\end{note}

\begin{note}
  \scen{3}~\ref{illu:gist:calc} and~\ref{illu:gist:roots} are similar.
  Broadly, both involve an agent concluding \(\phi\) has value \(v\) from some pool of premises \(\Phi\) and an option for the agent to conclude \(\phi\) has value \(v\) from some distinct set of premises \(\Phi'\).

  The key difference between \scen{1}~\ref{illu:gist:calc} and~\ref{illu:gist:roots} is how concluding \(\phi\) has value \(v\) from \(\Phi'\) relates to the agent's conclusion of \(\phi\) having value \(v\) from \(\Phi\).
  Roughly, in \scen{0}~\ref{illu:gist:calc} the alternative reasoning relates to the premises \(\Phi\), while in \scen{0}~\ref{illu:gist:roots} the alternative reasoning relates to the reasoning from \(\Phi\) to \(\phi\) having value \(v\).

  This, noted.

  Argue this type of \scen{0} motivates negative resolution to~\autoref{issue:has-witnessed}, and, by extension,~\autoref{issue:why-inc-in-how} (given~\autoref{prop:IssueRelation}).

  Intuitions are difficult.
  Motivation will focus on:
  \begin{itemize}
  \item
    Conditional
  \item
    \fc{0}
  \end{itemize}

  Turn to sketching the general argument.

  Still, while we are here, abstract form of \scen{1}, a few additional examples, and a handful of contrasting examples.
\end{note}

\paragraph{A general characterisation of the type of \scen{0}}

\begin{note}
  Strictly, the key properties of the type of \scen{0} we are interested in concern an agent's epistemic state just prior to the agent concluding some proposition-value pair from some pool of premises.
\end{note}

\begin{note}
  \begin{scenarioType}[(Partial) check on reasoning-scenarios \hfill \cScen{1}]\mbox{ }

    An agent \vAgent{} has concluded some proposition \(\phi\) has some value \(v\) from some pool of premises \(\Phi\).

    And, \emph{prior} to concluding \(\phi\) has value \(v\) from \(\Phi\), it was the case that:
    \begin{itemize}
    \item
      From \vAgent{}' perspective, if \(\phi\) has value \(v\) then:
      \begin{itemize}
      \item
        There is some proposition-value pair \(\psi\) having value \(v'\) and pool of premises \(\Psi\) such that:
        \begin{itemize}
        \item
          If the \vAgent{} were to fail to conclude that \(\psi\) has value \(v'\) from \(\Psi\) prior to concluding, \vAgent{} would not conclude \(\phi\) has value \(v\) from \(\Phi\).
        \end{itemize}
      \end{itemize}
    \end{itemize}
    \vspace{-\baselineskip}
  \end{scenarioType}

  For ease of reference, we term scenarios of this type `\cScen{1}', where `CoRe' stands for `check on reasoning', or more carefully `partial check on reasoning'.

  The key feature of \cScen{1} is that prior to concluding \(\phi\) has value \(v\) from \(\Phi\), the agent has the option of reasoning about some other proposition-value-premise pairing \(\pvp{\psi}{v'}{\Psi}\), such that, from the agent's perspective, failure to conclude \(\psi\) has value \(v'\) from \(\Psi\), the agent would not conclude \(\phi\) has value \(v\) from \(\Phi\).

  In this respect, the agent reasoning about whether \(\psi\) has value \(v'\) is a partial check on whether it makes sense for the agent, from their present perspective, to conclude \(\phi\) has value \(v\) from \(\Phi\) given the reasoning they have performed.

  In other words, the proposition-value-premise pairing \(\pvp{\psi}{v'}{\Psi}\) is a partial check on the agent's reasoning from \(\Phi\) to \(\phi\) having value \(v\).

  We expand on this in three steps.
  First, why \(\pvp{\psi}{v'}{\Psi}\) is a \emph{check}.
  Second, why \(\pvp{\psi}{v'}{\Psi}\) is a check \emph{on reasoning}.
  And, finally, why \(\pvp{\psi}{v'}{\Psi}\) is a \emph{partial} check on reasoning.
\end{note}

\begin{note}[Check]
  Agent has the option.
  Something went wrong with the reasoning from premises to conclusion.
\end{note}

\begin{note}
  On the agent's reasoning.

  The key contrast here is between \scen{1}~\ref{illu:gist:roots} and~\ref{illu:gist:calc}.

  Difference is how failure relates to the premise.

  With \autoref{illu:gist:calc}, testimony.
  It seems, if testimony, then it is not possible to satisfy the conditional.
  In other words, if the conditional holds, then it seems the agent doesn't have \(23 \times 15 = 345\) by testimony.

  With \autoref{illu:gist:roots}, quadratic formula.
  Fail to conclude from factorising, but this would not involve revising premise.
  Rather, reasoning from premises.
  For example, application of quadratic formula to the quadratic equation.
  Or, reduction of the quadratic equation.

  Reasoning from premises, rather than premises.

  Way this is captured, concluding \(\phi\) has value \(v\) from \(\Phi\).
  Agent would not conclude, because revise either testimony or understanding of arithmetic.
  However, remains the case that \emph{given} testimony, agent would conclude.

  The conditional focuses on the relation between premises and conclusion.
\end{note}

\begin{note}
  In addition, relative to \(\phi\) having value \(v\).
  Checking that the conclusion does indeed follow from the premises.

  This property is shared in both examples.
  However, narrows things down.

  Case in which agent could reason about premises further.
\end{note}

\begin{note}
  Partial.
  Failure to conclude \(\psi\) has value \(v'\) from \(\Psi\).
  Nothing about successfully concluding \(\psi\) has value \(v'\) from \(\Psi\).
  In general, other considerations against concluding \(\phi\) has value \(v\) from \(\Phi\).
  With respect to \cScen{1}, other proposition-value-premise pairings for which the same holds.
\end{note}

\begin{note}[Similarities and differences]
  \color{red}
  The basic similarity between \scen{1}~\ref{illu:gist:roots} and~\ref{illu:gist:calc} is something we may intuitively identify as a premise.
  In~\autoref{illu:gist:calc}, the premise was the testimony of a calculator.
  And, in~\autoref{illu:gist:roots} we have the quadratic formula.

  The slight contrast between \scen{1}~\ref{illu:gist:roots} and~\ref{illu:gist:calc} is the role of the relevant premise in the agent's reasoning.

  In~\autoref{illu:gist:calc}, \(23 \times 15 = 345\) follows more-or-less immediately from the testimony of the calculator.
  Roughly, in order to receive testimony that \(\phi\) has value \(v\), \(\phi\) must have value \(v\).

  Yet, in~\autoref{illu:gist:roots}, nothing follows more-or-less immediately from the quadratic formula.
  The agent concluded `\(x = 1\) or \(x = -\sfrac{1}{2}\) if \(2x^{2} - x - 1 = 0\)' from pool of premises which includes the quadratic formula.
  However, variables needed to be substituted for constants, and the resulting expression needed to be evaluated.

  The contrast is slight, but --- again, intuitively --- from the perspective of the agent, \(23 \times 15 = 345\) follows immediately from the premises given in~\autoref{illu:gist:calc} while in~\autoref{illu:gist:roots} the possible values of \(x\) take some work.
  And, perhaps, there is room for the agent to make a mistake when performing this work.
\end{note}

\begin{note}
  \color{red}
  Though we observed a slight contrast between \scen{1}~\ref{illu:gist:roots} and~\ref{illu:gist:calc} given the differing roles of the testimony of the calculator and the quadratic formula in the agent's reasoning, these variant steps highlight a key parallel.

  For, in~\autoref{illu:gist:calc} we noted the agent may have concluded \(23 \times 15 = 345\) without the testimony of the calculator.
  And, so long as the agent understands factorisation, a parallel statement holds for~\ref{illu:gist:roots}:
  The agent may have concluded \(x = 1\) or \(x = -\sfrac{1}{2}\) if \(2x^{2} - x - 1 = 0\) without the aid of the quadratic formula in~\autoref{illu:gist:roots}.
\end{note}

\begin{note}
  \scen{3} \ref{illu:gist:calc} and \ref{illu:gist:roots} both fit this pattern of a partial check.

  \autoref{illu:gist:calc}, agent's understanding of arithmetic, whether to trust the calculator.

  \autoref{illu:gist:roots}, factorising instead of applying the quadratic formula.

  In both these cases, \(\phi\) and \(\psi\) are the same proposition, and \(v\) and \(v'\) are the same value.
  Difference is the pools of premises.
\end{note}

\begin{note}

\end{note}

\begin{note}
  Conditional holds from the agent's perspective.

  Support.

  Support is relative to agent.

  \begin{idea}
    If agent has option and fails to conclude, no support.
  \end{idea}

  \begin{idea}
    If support between \(\Phi\) and \(\phi\) having value \(v\) only if relation of support between \(\Psi\) and \(\psi\) having value \(v'\).
  \end{idea}
\end{note}

\begin{note}
  Of course, interest in this property has only just been made explicit.
  Perhaps revise intuition.
  Though, I expect not.

  Though it is true that the agent would not conclude, did not do the reasoning.

  Stated in other terms, \(\psi\) from \(v'\) is a \fc{} if \(\phi\) has value \(v\).
  Still, no relevance.
\end{note}

\paragraph{Additional examples}

\begin{note}[Logician]
  \begin{illustration}\label{illu:CS:tfc}
    Novice logician.
    \begin{enumerate}
    \item Claimed support that \(\{\land,\lnot\}\) are truth functionally complete.
    \item If \(\{\land,\lnot\}\) are truth functionally complete then \(\{\lor,\lnot\}\) are truth functionally complete.
    \item So, \(\{\land,\lnot\}\) are truth functionally complete.
    \item Hence, \(\{\lor,\lnot\}\) are truth functionally complete.
    \end{enumerate}
  \end{illustration}

  First, reasoning for \(\{\land,\lnot\}\) did not depend on \(\{\lor,\lnot\}\).
  But this is quite complex.
  Not explicit assumption, but perhaps implicit.
  Still, going back through reasoning, it seems this is fair.

  Second, interdefinability.
  Hence, good account of why deals with possibility, as in general what holds for \(\{\land,\lnot\}\) will hold form \(\{\lor,\lnot\}\).

  Indeed, this suggests an alternative way of getting to the conclusion.
\end{note}

\begin{note}[Programming]
  \begin{illustration}
    \label{illu:programming}
    Writing a program to automate some reasoning/processing of data.
  \end{illustration}
  Various test cases.
  In these, possible to do the reasoning oneself.
  Therefore, no appeal to program for these simple cases, at least.
  This is quite similar to the logic illustration in this sense.

  However, interest here as interdependence breaks down in interesting ways.
  For, may break down due to resource constraints.
  E.g.\ available time or complexity of inputs.

  And, after enough time with the program, failure to obtain the same result is not clearly going to indicate a problem with the program.
  Rather, one's reasoning.
  Though, in turn, this may be reversed after enough checking of the reasoning.
\end{note}

\begin{note}
  Three examples which follow basic pattern.
  All these involve reasoning which is both deductive and formal.
  Contingent feature.
  Straightforward to identify related proposition-value-premise pairings and motivate the option of reasoning with the related pairing.%
  \footnote{
    Note, in particular, non-deductive reasoning is fine.
    For, present epistemic state.
    May be the case that some novel information would overturn conclusion.
    However, novel information would require a revised epistemic State.
  }

  Favour these features in \scen{1} and examples as fairly general.
  Particular constants chosen are easily replaced.
  And, easy to find similar.%
  \footnote{
    For example, consider calculating exponents.
    One may calculated \(x^{n} \times x^{m}\) directly, or appeal to \(x^{n} \times x^{m} = x^{n + m}\).
 
    Also relatively simple, but there are a wealth of more complex examples.

    For example:
    See~\textcite{Fine:1997vc} for three distinct proofs of the fundamental theorem of algebra (i.e.\ any complex polynomial must have a complex root).
    Likewise, \textcite{Ribenboim:2012ts} contains eight proofs that their exist infinitely many prime numbers.
    And, \textcite{Wagon:1987vm} details fourteen proofs of a result about tiling a rectangle (specifically, whenever a rectangle is tiled by rectangles each of which has at least one integer side, then the tiled rectangle has at least one integer side).
  }
\end{note}

\subparagraph{Different to}

\begin{note}[Serial number]
  \begin{illustration}
    \label{illu:number-check}
    Handed a credit card.
    Conclude, will be able to bill after repairs are done.
  \end{illustration}

  Fairly straightforward partial check on whether the credit card is usable by applying the Luhn algorithm to check whether the credit card number is valid.
  Applying the Luhn algorithm requires only basic addition and multiplication.%
  \footnote{
    To illustrate.
    Credit card number is:
    \(4676\) \(2250\) \(1000\) \(0626\).
    Array:
    \([4,6,7,6,2,2,5,0,1,0,0,0,0,6,2]\)
    For every even index of the array (index from \(0\)), multiply the number by two, and add the resulting digits together if the results is greater than ten.
    \([8,6,5,6,4,2,1,0,1,0,0,0,0,6,4]\)
    Add elements of the array together with the check digit.
    Valid only if the result is equal to a multiple of ten.
  }
  However, if the agent is not aware of that the Lund algorithm may be applied to check whether the credit card number is valid, the agent will not have the option of checking given their present epistemic state.
\end{note}

\begin{note}[Fire alarm]
  \begin{illustration}
    \mbox{}
    \vspace{-\baselineskip}
    \begin{itemize}
    \item Fire alarm is ringing.
    \item Fire in the building.
    \item Should leave by the nearest exit.
    \end{itemize}
  \end{illustration}

  Agent could check whether there is a fire.
  Defeasible.
  If test, then no problem.

  Problem, no by reasoning alone.
\end{note}

\paragraph*{Interest}

\begin{note}
  As seen from \illu{1}~\ref{illu:gist:calc} and \ref{illu:gist:roots}, there is an intuitive sense in which an agent concluding some proposition \(\phi\) has some value \(v\) involves certain pools of premises and does not involve other pools of premises.

  Calculator, understanding of arithmetic.
  Quadratic formula, factorisation.

  I am somewhat unsure about the intuitions expressed with respect to \scen{1} like~\autoref{illu:gist:calc}.
  However, I quite unsure about the intuitions expressed with respect to \scen{1} like~\autoref{illu:gist:roots}.
\end{note}

\subsection{Motivation}

\subsubsection{Reasons \hfill (Optional)}

\begin{note}

\end{note}

\begin{note}
  Variant of the more general issue with which \citeauthor{Davidson:1963aa} opens \citetitle{Davidson:1963aa}:

  \begin{quote}
    What is the relation between a reason and an action when the reason explains the action by giving the agent’s reason for doing what he did?%
    \mbox{}\hfill\mbox{(\citeyear[685]{Davidson:1963aa})}
  \end{quote}

  \citeauthor{Davidson:1963aa}'s question is more general as covers all action.
  Our interest is specifically with concluding.

  Possible answers to \qWhy{}, lots.
\end{note}

\begin{note}
  Alternative may be to talk in terms of reasons.
  Concluding is an action.
  Reasons for action.
  Concludes from premises.
  Hence, premises are part of agent's reasons for action.

  With a little more care, we make further distinguish between normative and explanatory reasons.%
  \footnote{
    Normative: `reasons which show a given action, attitude, activity or outcome good, right, appropriate or called for'
    Explanatory: `the reasons why things happen, or why things are the way they are'
    \citeyear[410]{Hieronymi:2011aa}

    \citeauthor{Hieronymi:2011aa} also distinguishes `motivating' reasons,

    Motivating: `psychological facts which explain action'
    \citeyear[411--412]{Hieronymi:2011aa}

    This way of dividing reasons is difficult.
    \citeauthor{Hieronymi:2011aa}'s account of motivating reasons follows~\textcite{Smith:1994wo}, but~\citeauthor{Smith:1994wo} distinguishes motivating reasons from normative reasons.

    \begin{quote}
      The distinctive feature of a motivating reason to \(\phi\) is that, in virtue of having such a reason, an agent is in a state that is \emph{explanatory} of her \(\phi\)-ing, at least other things being equal --- other things must be equal because an agent may have a motivating reason to \(\phi\) without that reason's being overriding.%
      \mbox{}\hfill\mbox{(\citeyear{Smith:1994wo})}
    \end{quote}

    See also \citeauthor{Broome:2013aa}:
    \begin{quote}
      Sometimes the explanation of why a person does something has a particular character:
      roughly, it involves the person's rationality in a distinctive way that I shall not try to describe.
      Then we say the person does what she does for a reason.
      We might say ‘The reason for which Hannibal used elephants was to terrorize the Romans'.
      The reason for which a person does something is called a ‘motivating reason'.
      In general, a motivating reason is whatever explains or helps to explain what a person does in the distinctive way that involves her rationality.

      \mbox{}\hfill\(\vdots\)\hfill\mbox{}

      Whereas motivating reasons explain or help to explain why a person does something, normative reasons explain or help to explain why a person ought to do something, or to believe something, or to hope for something, or to like something, or in general to F, where ‘F' stands for a verb phrase.%
      \mbox{}\hfill\mbox{(\citeyear[46--47]{Broome:2013aa})}
    \end{quote}
  }

  In this way of looking at things, relation of support concerns explanatory reasons.
\end{note}

\begin{note}
  \citeauthor{Davidson:1963aa} resolution, in the \citeyear{Davidson:1963aa} paper at least, causal.
  Identifying \emph{something}%
  \footnote{
    \textcite{Hieronymi:2011aa} for a general overview.
  }
\end{note}


\paragraph*{Quick argument}

\begin{note}
  More generally, a quick argument for witnessing is that conclusion needs to come from somewhere.
\end{note}

\begin{note}[The general pre-theoretic argument]
  At least, the positive resolution is not straightforward without placing some constraints on \emph{which} premises an agent may appeal to when reasoning, and we will motivate the positive resolution without such constraints.

  For, without constraints on which premises an agent may appeal to when reasoning, one may argue as follows:
  \begin{enumerate}
  \item
    Any instance of reasoning is some process with start and end points and intermediary steps.
  \item
    If an agent has concluded \(\phi\) has value \(v\) by some reasoning, then the reasoning has start points and intermediary steps.
  \item
    Hence, the agent has concluded \(\phi\) has value \(v\) by witnessing some reasoning from some start points via some intermediary steps.
  \item
    In other words, the agent has concluded \(\phi\) has value \(v\) by witnessing some reasoning from some premises.
  \end{enumerate}

  In short, so long as an agent has concluded \(\phi\) has value \(v\), the agent has always witnessed reasoning from some premises.

  \begin{enumerate}[resume]
  \item
    So, either the start points are the premises of interest mentioned in the issue, or the agent has concluded \(\phi\) has value \(v\) from a distinct set of premises.
  \end{enumerate}

  In other words, either the agent has witnessed reasoning from the premises of interest, or the premises of interest (and any reasoning from them) are not required to conclude \(\phi\) has value \(v\).
\end{note}

\begin{note}[More on the quick argument]
  The quick argument does not directly lead to a negative resolution to the issue.
  Still, the quick argument does suggest that any appeal to premises \emph{without} witnessing reasoning from those premises is redundant.

  Now, perhaps redundancy isn't so bad.
  I only need a single key to ensure I have the option of unlocking a door, but a second key is useful if the first is lost.

  Still, I take it to be the case that redundancy provides leverage for a wide range of arguments motivating a negative resolution in the case of reasoning.

  For, if appeal to some premises is redundant, then any argument that requires witnessing need only observe that a counterargument must find some role for something which is not needed.

  Reasoning is an event, and distinct way of concluding \(\phi\) has value \(v\) may be useful, it is unclear why the distinct way of concluding \(\phi\) has value \(v\) is of use when concluding \(\phi\) has value \(v\) from present premises.
  To push the analogy, a second key may have various uses, but the second key is irrelevant in the event of unlocking the door with the first key.
  That the second key is would unlock the door if the first was lost has no role in the event of unlocking the door with the first key.

  From a different perspective, if appeal to certain premises without witnessing reasoning from those premises is redundant, then it seems any positive role given to appeal to those premises may be redistributed to the premises of the reasoning the agent did witness.

  More concretely, even if I were to show that there was some benefit for concluding \(\phi\) has value \(v\) via unwitnessed reasoning with respect to some particular account of reasoning, it seems at least plausible that the account of reasoning may be reformulated to derive the same benefit from the premises of the reasoning the agent witnessed.

  More generally, it may seem (and I suspect it does seem) intuitive that the issue should be resolved negatively.
  Reasoning just is obtaining a conclusion by witnessing reasoning from premises.
  And, if the quick argument succeeds, then there surely is some way to preserve the intuition.
\end{note}

\begin{note}
  So, part of the task is to show that the quick argument fails.
  Concluding \(\phi\) has value \(v\) from certain premises without witnessing reasoning that concludes \(\phi\) has value \(v\) from those premises has some role.
  Or, rather, that the quick argument is not without cost.
  Perhaps the issue really should be resolved in the negative, but this will require giving up some at least equally (I think) intuitive ideas.

  The result will be motivation for a positive resolution to the issue.
  However, the motivation will be somewhat narrow.
  To escape the tension, the positive resolution need only hold for a restricted pattern of reasoning.
  Still, with the existential motivated, I hope future work may expand the positive resolution to other patterns of reasoning.
  And, while such expansions may still need to argue that concluding \(\phi\) has value \(v\) from unwitnessed reasoning is makes sense with respect to the specific topic at hand, observing that the broad idea of concluding \(\phi\) has value \(v\) from unwitnessed reasoning may not be dismissed without cost may be an option.
\end{note}

\paragraph*{Summarising}

\begin{note}
  So, motivated positive resolutions to issues~\ref{issue:why-inc-in-how} and~\ref{issue:has-witnessed}.
  And, in particular, motivated~\autoref{issue:has-witnessed} over~\autoref{issue:why-inc-in-how}.
\end{note}

\begin{note}
  Focus on role of \iWitness{}.

  \begin{quote}
    \ideaWitnessing*
  \end{quote}

  This is the important thing, and in this respect it doesn't matter whether past or present.
  Whether a relation of support holds only if witnessed.
  Whether resolution to \qWhy{} only if the agent has witnessed.
\end{note}

\begin{note}[Pre-theoretical constraint]
  \iWitness{} captures intuitive constraint of no account of why without an account of how.
  Witnessing a pre-theoretical constraint.

  Broader than causation.
\end{note}

\begin{note}
  `Use'.

  Negative resolution, some pool of premises, why.
  Yet, agent does not have a specific account of how.
\end{note}

\begin{note}
  Now turn to sketching out argument.
\end{note}

\subsection{Strategy}

\begin{note}
  In this section we provide an outline the argument we will make for a negative resolution to~\autoref{issue:has-witnessed}.

  At the highest level, the strategy is fairly straightforward:
  Provide \scen{0} which motivate a negative resolution to~\autoref{issue:has-witnessed}.

  We being with a very high level overview, detailing the key ideas and how they fit together.
  Following the very high level overview we then provide a slightly more detailed pass.
\end{note}

\begin{note}
  Though, this may be misleading.
  Considered various \scen{0}, and intuitions with respect to these \scen{0}.
  Our strategy is not to provide equally intuitive \scen{0} which are incompatible with a positive resolution to~\autoref{issue:has-witnessed}.

  Rather, our strategy is to provide a detailed account of how a handful of ideas --- which are motivated independently of either a positive or negative resolution to ~\autoref{issue:has-witnessed} --- combine to provide an abstract type of \scen{0}.

  Tension between instances of abstract type of \scen{0} and positive resolution to~\autoref{issue:has-witnessed}.

  In this respect, I doubt the instantiations of the abstract type of \scen{0} will intuitively motivate a negative resolution to~\autoref{issue:has-witnessed}.
  Rather, motivation \emph{given} background ideas.%
  \footnote{
    Indeed, I continue to find positive resolution to~\autoref{issue:has-witnessed} intuitive.
    And, things are constructed very carefully to keep the range of cases as clear as possible.
    As we have seen, no revision to premises.
    Still puzzled about \scen{1} such as \autoref{illu:gist:calc}.
  }
  Hence, motivation.
\end{note}

\paragraph{Very high level overview}

\begin{note}
  \begin{enumerate}
  \item
    Support
  \item
    \cScen{1}
  \item
    If \fc{0}, then support.
  \item
    Relate support with \fc{0} to reasoning.
  \item
    \zS{}
  \item
    \zS{} compatible with either, so independent.
  \item
    Either concluded or \fc{0} is part of why.
  \item
    \cScen{0}, both disjuncts.
  \item
    Problem with restricting to concluded.
  \item
    So, \fc{0}
  \item
    \fc{0} is in part why.
  \item
    Negative resolution.
  \end{enumerate}
\end{note}

\begin{note}
  Core idea:
  \begin{idea}[\fc{1} and support]
    \label{idea:fc-and-support}
    If \(\phi\) having value \(v\) is a \fc{0}, then relation of support between \(\phi\) having value \(v\) and some pool of premises \(\Phi\).
  \end{idea}

  Following suggestion made with respect to \autoref{illu:gist:calc}.

  With this idea, possible for relation of support to be in part why.

  Given this, still a question of how \fc{0} are involved.
\end{note}

\begin{note}
  As with \autoref{illu:gist:calc}, possibility of failure to conclude.

  Particular type of support.

  Agent would not conclude otherwise.

  Fairly natural.
\end{note}

\begin{note}
  Motivation for \fc{0} from these kinds of cases.

  In these cases, if \zS{} then either concluded or \fc{0}.
\end{note}

\begin{note}
  Here, sort of interesting.

  \zS{}, compatible with positive resolutions to both issues.
  In these cases, only if concluded.
  So, then, positive resolution rules out \zS{} in various cases.
\end{note}

\begin{note}
  Press against this.
  Positive resolution rules out \zS{} in a wide variety of scenarios.

  Indeed, \cScen{1}.
  Show that \scen{1} of this type are common.

  In particular, whenever reasoning is a specific instance of a general ability.
  However, this is provide a more general schema.
\end{note}

\begin{note}
  Focusing on ability, though.
  Basic thing is that in the context of \zS{}, general ability reduces down to specific abilities.
  Hence, relation of support with specific abilities matters.
\end{note}

\begin{note}
  In one respect, somewhat surprising, and perhaps unintuitive.
  Have intuition and motivation for positive resolution to issues.

  And, intuitively, general ability, nothing more.

  On the other hand, general ability hides complexity.
\end{note}

\begin{note}
  So, scope of a fairly natural type of support.%
  \footnote{
    \color{red}
    This should be moved somewhere else, but it would be useful to emphasise that \zS{} really is a type of support, in the sense that it is why the agent concludes in various cases.
    For, without support for each \requ{}, the agent would not conclude.
  }
\end{note}

\paragraph{Lower pass}

\subparagraph{\fc{0}}

\begin{note}
  \autoref{idea:fc-and-support} is fundamental.

  Distinction between:
  \begin{enumerate}
  \item
    Support between premises and conclusion of a \fc{0}.
  \item
    Support between \fc{0} and conclusion of \fc{0}.
  \end{enumerate}

  To illustrate.
  Grant for a moment, understanding of arithmetic is involved in why.

  Not understanding of arithmetic, but that I, or you, understand arithmetic.
  Not understanding, but \emph{that} understand.
\end{note}

\begin{note}
  Argument is split into two parts.

  First, possibility.
  Second, importance.

  Possibility, from \itp{1}.
  \itp{3} serve two roles.
  \autoref{idea:fc-and-support} over \autoref{issue:why-inc-in-how}, and, as such, plausible instances of \fc{1}.
  Past, but also future by paring with ability.

  This is partial, though.
  Stronger argument for support.
  This comes from role.
\end{note}


\begin{note}[Outline]
  Argue for a negative resolution to~\autoref{issue:has-witnessed}, and by extension~\autoref{issue:why-inc-in-how}.

  Idea of a \fc{0}.
  \(\phi\) having value \(v\) is a \fc{0} with respect to an agent's epistemic state just in case the agent would not fail to conclude \(\phi\) having value \(v\).

  If \fc{0}, then some pool of premises \(\Phi\) available to the agent such that it is possible for the agent to witness reasoning from \(\Phi\) to \(\phi\) having value \(v\).

  Future variant of~\autoref{assumption:support}.
  If agent has concluded, then support.
  Agent has not concluded, but sufficient.

  Nothing too surprising.
  Recall, knowing whether.
  There is nothing apart from witnessing.

  \fc{0}, alone, then, not too interesting.
  Information that \(\phi\) having value \(v\) is a \fc{0}.
  A little more interesting.

  Now, here, \emph{that} \fc{0}.

  Information about the possibility of establishing some conclusion.
  Nothing too surprising.
  Questions in textbooks.

  Pair these two things together.
  Well, okay.
  Hint.
  Here, then, open to go from \emph{that} \fc{0} to \(\Phi\).

  Given \(\Phi\), everything else is redundant.

  With a \fc{0}, the pool of premises, these do the work of establishing the possibility of how.

\end{note}

\paragraph*{Major and minor problems}

\begin{note}
  Two problems.

  Minor and major.

  Both comes from information required.

  Minor.
  \emph{that} \fc{0}.
  Distinction between \(\Phi\) and that conclude from \(\Phi\).
  Need \emph{that} \fc{0}.

  Major.

  \itp{} gives information.
  So, conclude from that.
\end{note}

\paragraph*{Minor}

\begin{note}[Minor]
  Minor problem.
  Thing is, no clear leverage.
  Certain nice things about \(\Phi\), and \emph{that} \fc{0} is much weaker.
  Only thing missing is witnessing.

  Basically, only really get \emph{that} \fc{0} doing any work if support between \(\Phi\) and \(\pv{\phi}{v}\).

  In part, this is why focus attention on support.

  As we have seen, independent motivation for relation of support.
  Tied to why.

  Part of what it is for something to be a \fc{0} is a relation of support.
\end{note}

\paragraph*{Major}

\begin{note}[Major]
  Redundant.
  Possible answer to \qWhy{}.
  However, this doesn't show that is an answer to \qWhy{}.

  In general, various possible answers to \qWhy{}.
  This is key.
  Issues~\ref{issue:why-inc-in-how} and~\ref{issue:has-witnessed} narrow down.

  \citeauthor{Boghossian:2014aa}, recall.
  Of course, much different that just saying, or being response dependent.
  \fc{0}.
  So, answers \qWhy{}, at least in principle.

  Still, relevant information, \itp{}, etc.\ must do enough to get \(\pv{\phi}{v}\).
  For, if not then it seems can't use that \(\pv{\phi}{v}\) is a \fc{0}.

  Squeeze out role for \fc{0}.

  Indeed, \fc{0} from understanding of arithmetic.
  But, it is not obvious that this is why.
  Calculator does it all.

  Find role for \fc{0}.
  Tied to why an agent concludes.
  This is our goal.
  Show how \fc{0} is, in part, why an agent concludes.

  \begin{itemize}
  \item
    From the agent's perspective.
  \item
    Support holds between \(\Phi'\) and \(\pv{\phi}{v}\), and in part this is why agent concludes \(\pv{\phi}{v}\) from \(\Phi\).
    Where, \(\Phi' \ne \Phi\).
  \end{itemize}

  Still, get to \(\pv{\phi}{v}\).
  So, \(\Phi'\) is not required to conclude \(\pv{\phi}{v}\), in general.
\end{note}

\paragraph*{Role for \fc{1}}

\begin{note}
  Role for \fc{1}.

  Concluding from some pool of premises.
  More complex.
  Key thing, check.
  Prior to concluding \(\pv{\phi}{v}\) from some pool of premises \(\Phi\), check on whether it makes sense, from agent's perspective, to conclude \(\pv{\phi}{v}\) from \(\Phi\).

  In particular, cases where, if \(\phi\) has value \(v\) then it is possible for the agent to conclude \(\psi\) has value \(v'\) from some pool of premises \(\Psi\).
  Here, permutations.
  Of interest for the moment:
  \(\phi = \psi\) but \(\Phi \ne \Psi\).
  However, also \(\phi \ne \psi\) and \(\Phi \ne \Psi\).

  Keys.

  This prevents.

  Parcel delivered, check the address before opening.

  Cases go the other way.
  Check is satisfied.

  Wason selection task.
  Here, it is clear what needs to be checked.
  If this proposition is true, then these things are possible.

  Here, consistent with witnessing.
  Turned over cards, then fine.
  Haven't reasoned from place to absence of keys.

  Many instances are.

  Relative, so long as conclusion is true.
\end{note}

\begin{note}[Not all concluding is like this]
  Not all concluding is like this.
  Testimony.
  No way to check.
  Of course, may be various ways to check the sources.
  However, focus on novel conclusions.
\end{note}

\paragraph*{\zS{}}

\begin{note}[Second interest of document]
  This is the second thing of interest.
  What this amounts to, and how it relates to concluding.
\end{note}

\begin{note}[`\zS{}']
  Particular kind of support.
  \zS{}.

  Is it the case that agent may fail to conclude?

  Additional property of relation of \bS{}.
  If agent concludes, not only support, but \zS{}.
  \begin{itemize}
  \item
    \bS{2}: premises and conclusion.
  \item
    \zS{}: checks.
  \end{itemize}

  Key observation?
  In various cases, \bS{} only if \zS{}.
  `Moorean'?
  \(\pv{\phi}{v}\) and may conclude \(\pv{\phi}{\overline{v}}\)?

  Well, distinct pools of premises.
  Still, these pools of premises are `present'.

  This is a very important observation.
  Interested in \zS{} cases.
  One option is to get rid of this.
  However, this seems quite difficult.
  If thought of place, then wouldn't conclude keys are missing.
  If problem with cards, wouldn't conclude conditional holds.

  Insight, checks involve \fc{0}.
\end{note}

\begin{note}[Limitation and closure condition]
  Important observation.
\end{note}

\paragraph*{Cases}

\begin{note}[Shifting]
  Started with \illu{1}~\ref{illu:gist:calc} and \ref{illu:gist:roots}.
  Here, calculator and check.
  More generally, testimony.

  However, slight difficulty.
  Isn't about own reasoning.

  Motivating \zS{}, focus is on own reasoning.

  To keep parallel, variant case.
  Return to cases similar \illu{1}~\ref{illu:gist:calc} and \ref{illu:gist:roots} to later, and highlight connexion.
\end{note}

\begin{note}[New \illu{0}]
  \begin{illustration}
    \label{illu:sketch:prop-logic}
    Suppose an agent has a good grasp of propositional logic.
    In particular:
    \begin{itemize}
    \item
      The agent has a good understanding of some formal proof system.
      For example, some Fitch-style system.
    \item
      The agent has a good understanding of some method to construct semantic proofs.
      For example, by constructing truth tables, or reasoning about valuation functions.
    \item
      The agent understands the proof system is sound and complete.
      That is to say, the agent understands there exists a proof of some sentence \(A\) \emph{if and only if} \(A\) is true given an arbitrary valuation.
    \end{itemize}
    The agent constructs a proof of \(A\).

    Given the agent's understanding of propositional logic, the agent observes:
    \begin{quote}
      The construction is a proof of \(A\) \emph{if and only if} \(A\) is true given an arbitrary valuation.
    \end{quote}
  \end{illustration}

  Intuitively, if the agent were to reason about whether \(A\) is true given an arbitrary valuation and failed to conclude \(A\) is true given an arbitrary valuation, then the agent would not conclude the construction is a proof of \(A\).

  If were to go for semantic, then by soundness, works out.
\end{note}


\begin{note}
  Discussion of \autoref{illu:sketch:prop-logic} relied on some background familiarity with propositional logic.

  Slight variation on \citeauthor{Tolliver:1982us}'s Pendulum Case.%
  \footnote{
    Understood \citeauthor{Tolliver:1982us}'s case only after.
  }

  \begin{illustration}
    \label{illu:sketch:math}
    An agent calculates \(23 \times 15 = 345\).

    Given the agent's understanding of arithmetic, the agent recognises:
    \begin{quote}
      \(23 \times 15 = 345\) \emph{if and only if} \(\sfrac{345}{15} = 23\).
    \end{quote}
  \end{illustration}

  Intuitively, if the agent were to \emph{conclude} \(23 \times 15 = 345\), the agent would also, at the same time, conclude \(\sfrac{345}{15} = 23\).

  Of course, this intuition may be resisted.
  For example, \(345 = (3 \times 5 \times 23)\) and \(15 = 3 \times 5\).
  Intuitively, it is not the case that the agent would conclude \(23 \times (3 \times 5) = (3 \times 5 \times 23)\).

  Perhaps, with \citeauthor{Tolliver:1982us}, sufficient familiarity with pendulums, then \dots

  However, this is a more complex transformation.

  The equivalence between multiplication and division is sufficiently familiar.
  
\end{note}


\begin{note}[Interest]
  Propositional logic, soundness and completeness.
  Here, distinct premises, it seems.
  \(A\) is a theorem.
  Either syntax or semantic, strictly this makes a difference, context disambiguate.

  Similar to \citeauthor{Tolliver:1982us}, though a difference.
  For, here, not lacking information.
  With the pendulum, didn't have the option of concluding from period, as didn't have information about period.
  Here, understanding of both proof system and semantics.

  Delicate, the equivalence result is not in question.
  What's at issue is one's reasoning.
  The result tells me that if reasoning is fine for syntax, then also semantics and vice-versa.
\end{note}

\begin{note}
  Question.
  \zS{}?

  Suppose syntax.
  Well, check with semantics.
  If fail semantics, then something wrong with syntactic reasoning.

  Conversely, semantics.
  Then, syntax.

  Intuitively, \zS{0}.
  Not the case that agent would reach a different conclusion.
  Strength of reasoning for either syntax or semantics.

  Observation.
  Can't go from syntax to semantics while preserving \zS{}.
  Possible to check semantics ignoring proof.

  Conversely, semantics is no good for \zS{} without syntax.

  Key observation is, \zS{}, if either syntax or semantics is involved, then this doesn't answer the question.
  At issue is whether the agent may fail, and in doing so unwind the main conclusion.
\end{note}

\begin{note}[Options]
  \begin{enumerate}
  \item
    Give up on \zS{}.
    {
      \color{red}
      (Seems unintuitive, and, link to concluding.)
    }
  \item
    Conclude independently, and then get \zS{}.
    {
      \color{red}
      (Seems redundant.)
    }
  \item
    Get \fc{0} prior.
    {
      \color{red}
      (But, no information about what the result is.)
    }
  \item
    Same time.
    \begin{enumerate}
    \item
      Witness.
      {
        \color{red}
        (But, distinct premises.)
      }
    \item
      \fc{0}.
    \end{enumerate}
  \end{enumerate}
\end{note}

\begin{note}[Give up on \zS{}]
  Intuition, and \zS{} is quite weak.
  I mean, look, there's no doubt.
  This is clear with multiplication and division.
  There's no problem with reasoning via either multiplication or division.
  Observation that I would not conclude via X if I fail to conclude via Y.
  However, this does no suggest that there is anything problematic with reasoning.

  Compare to examples of failure.
  Not like keys.
  Not like\dots? (letter requires novel information.)

  The difficulty is accounting for why.
  Why would not fail, independent of multiplication.
\end{note}

\begin{note}[Do the reasoning]
  One option, do the reasoning.
  This seems excessive.
  For, mostly, the same reasons as above.
  Multiplication and division are straightforward.
\end{note}

\begin{note}[\fc{0} prior]
  Well, intuitively, \fc{0}.
  However, no information about \emph{which}.
\end{note}

\begin{note}[Simultaneous]
  Simultaneous conclusion.

  Reasoned from one, the other is a \fc{0}.
  Information from reasoning, \emph{which} \fc{0}.

  \fc{0}, \emph{why} \zS{}.

  Key point, premises for information are not premises for \fc{0}.
  If premise, then \zS{} is voided.
  Witnessing reasoning is in part \emph{how}, but is not tied to \emph{why}.

  Here, negative answers.
  No witnessed reasoning from division.
  Reasoning from division is not included in how.
\end{note}

\begin{note}
  Not clear to me how puzzling this is.

  On some days, quite puzzling.
  Agent has not witnessed reasoning.

  On other days, \fc{0}.
  There's really no need for the agent to witness.
  Though, it is clear how important this is for multiplication.
\end{note}

\begin{note}
  One issue, agent may still reason from division and fail.
  But, it is not clear to me how important this really is.
  Witness, then there is no guarantee that the reasoning is not faulty.
  This gets covered later.
\end{note}

\begin{note}[Variant with logic]
  Here, second go with syntax and semantics.
  This has the benefit of clarity over conclusion.
  Downside is that you may not be a logician.

  Syntactic reasoning is fine, and equally, semantic reasoning is fine.
\end{note}

\paragraph*{Back to testimony (and a variant of \zS{})}

\begin{note}
  The key difference between these two cases is:
  With multiplication and division, or syntax and semantics, question is about the agent's reasoning from premises to conclusion.

  However, with testimony, the question is with whether the premise of testimony is a good premise.

  The thing is, whether leave the premises fixed.

  Variant of \zS{}.
  Here, query premises, rather than the conclusion.
  Set this aside.

  Part of what makes \zS{} interesting is no revisions.
  In this respect, quite different to something like \citeauthor{Wright:2011wn} on transmission of warrant.

  Still, this is not too much of a stretch.
  For, would still prevent concluding.

  However, this gets tricky.
  Because, testimony is involved as a premise.
  Well, no clear recursion problem, assuming well-structured premise-conclusion.
  For, \zS{} still relative to premise-conclusion pairings.

  The real difficulty is conclusion being enough.

  I mean, it's just a different condition.
\end{note}

\begin{note}[Other cases are puzzling]
  The introductory \illu{0}, with the calculator, this is more puzzling.
  Seems, \fc{0} from calculator.

   Without \fc{0}, no concluding.
  Without calculator, no \fc{0}.

  No concluding, as would not conclude if different.
  \fc{0}, given understanding of arithmetic, but no account of which.

  However, difference.
  Calculator is independent of agent's own reasoning.
  Though, this gets somewhat tricky.
  Testimony, other agents.
\end{note}

\begin{note}
  Calculator.
  Check to see whether or not it's functioning correctly.
  Understanding of arithmetic.
  If fail to conclude sum, then calculator is no good.

  \zS{}.
  Again, key point here is that if something else, then wouldn't conclude.

  Conclude sum from calculator only if conclude sum from understanding of arithmetic.
\end{note}

\begin{note}
  {
    \color{blue}
    First thing, can't use calculator.

    More broadly, ordering problem.

    Without \fc{0}, no conclusion from testimony.
    Without testimony, no which for \fc{0}.
  }

  {
    \color{red}
    Solution.

    Well, key thing is that calculator is fine.
    And, if calc, then given understanding of arithmetic, would conclude.
    No suggestion that calculator is faulty in any way.
    However, this doesn't mean that we don't have a check.

    At the same time.
    So, calculator \emph{and} \fc{}.

    \fc{} is not coming from calculator.
    Only getting information from calculator.
    Not worries about calculator, because would conclude.

    This is a somewhat puzzling conclusion.
    However, it's fine.

    See, the best we can do here is press the idea that the agent may still conclude otherwise.
    But, not from the agent's present epistemic state.
    For, \fc{0}: Would not fail from present epistemic state.

    Of course, could still press this, but then no different from any other conclusion.

    Hence, would need to hold that the agent had to conclude from understanding of arithmetic strictly prior.
    This, the agent hasn't done.

    So, then, the problem with this is that in general, have cases of such `simultaneous conclusions'.
    This requirement, paired with idea of \zS{}, problem for ability.

    So, reject \zS{}.
    Yet, intuitive constraint on concluding.
    }

  In other words, calculator only if \fc{0}.
  Understanding of arithmetic also supports.

  Testimony.
  So, receive testimony.
  Also, understanding of arithmetic.
  No witnessing, so if testimony is right, then \fc{0}.
  But, if no support, then would not conclude.

  Testimony only to the extent something I already know (whether).
  So, part of why is that know whether.
  Only if already a \fc{0}.

  Hence, \qWhy{} not included in \qHow{}.
  And, more generally, no witnessing.

  Testimony\dots property of being testimony\dots
  Telling me things I already know (whether).
  No granting testimony if not already supported.


  Why is testimony fine?
  Because \fc{0}!
  Well, maybe.
  {
    \color{red}
    Ugh, this is difficult.
    The cases I have where \fc{0} is clearest involve other problems.
    I.e.\ sudoku, so don't need to test reasoning against any other problems.

    So, testimony is irrelevant to whether, it's not a \fc{0} because testimony.
    Already have \fc{0}, already `know' whether.
    What testimony does is inform \emph{which}.
  }

  {
    \color{red}
    This is kind of the puzzle.
    Without \fc{0}, don't get testimony (because, independent test).
    And,
    Without testimony, don't get \fc{0} (because, need info which).

    So, these two things come in at the same time.
    When go from testimony, also get that \fc{0}.

    But, why is only tied to understanding of arithmetic, because else an ordering problem.

    So, the core idea is that the relationship between testimony and understanding of arithmetic is already in place, prior to getting information about which from the calculator.

    Right, the key thing is that there are two sources of information.
    \begin{itemize}
    \item
      Testimony
    \item
      Understanding of arithmetic
    \end{itemize}
    What matters is that these align.
    So long as you have this, then when you get a new piece of information from either, expectation that both sources will provide the same information.
    If break equivalence between the two sources, then neither works.
    Here, each much function separately.

    Keeping parity.

    So, how really isn't all that important.
  }

  Going from testimony to \fc{0} doesn't work.
  Because, \fc{0} is independent check.
  Right, because then testimony is still pending on whether \fc{0}.

  This is delicate.
  It seems as though, check on whether testimony.
  So, no testimony.

  However, this is not quite right.
  It is a check on testimony.
  However, two things.
  First, understanding and second testimony.
  Testimony is fine, understanding of arithmetic only presents the option of checking.
  Understanding does not suggest that testimony is bad/statement does not amount to testimony.
  However, understanding of arithmetic is in part why because this dismisses the possibility.

  Testimony, pressure on understanding of arithmetic.

  Testimony, but if testimony and \fc{0}, then determine whether really is testimony.
  If go via testimony, then either limited support, or \fc{0} does work.

  Structurally similar to ability.
  If specific from general, then only so good as specific.

  Of course, only when possible to check.
  If not possible, then this restriction isn't in place.

  In other words, attention shifts from cases of concluding in general, to specific cases.

  If sufficient to conclude, then what role has alternative option?
\end{note}

\begin{note}
  Of course, \emph{that} \fc{0} is key with respect to how.
  If the agent has not witnessed reasoning, then need some source of information.
  However, not for \emph{why}.

  Note, possibility of witnessing is given by X, as is that X supports Y.
  X is sufficient.
  Nothing more is needed for X to support Y.
\end{note}

\begin{note}
  Main this is resolving issue.
  Two other upshots.

  \begin{enumerate}
  \item
    Reduction of concluding.
    In various cases, abundance of \fc{0} suggests witnessing is not explanatory.
    Limited to how.
  \item
    Concluding without witnessing.
    {
      \color{red}
      I think I get close to this regardless, by looking at cases where an agent develops some general ability.
    }
    Stronger variation.
    If \fc{0}, then don't need to witness.
    Note, this is strictly stronger.
    For, main issue only tells us that some pool of premises is involved in why.
  \end{enumerate}
\end{note}

{
  Goal is to argue that witnessing is the only distinguishing feature.
  There are cases, such as the one described above, in which equal role.

  Core idea, supports to more general points of interest.

  More broadly, separate witnessing from concluding.
  Suggest there are cases in which conclude without witnessing.
}

\paragraph*{Aside: Method}

\begin{note}
  Main points from above suggest including method.

  For example, independent set and vertex cover problem.
  Here, reduction.
  However, same premises and same conclusion (given understanding of equivalence).

  So, same issue.

  Here, to keep things relatively simple, avoid method.

  Problem here is that things are mostly the same in these cases, in contrast to calculator versus understanding of arithmetic. 
\end{note}

\paragraph*{More detail}

\begin{note}
  Does this relation hold only if witnessing?
  Indeed, granting that witnessing is sufficient, this question is an if and only if:
  Does this relation hold if and only if witnessing?

  Separate concluding to district components.
  Witnessing a foregone-conclusion.
  There is nothing that witnessing adds which is not already determined by the agent's present epistemic state.

  Nothing of interest to be gain by witnessing, hence, in some sense of the term concluding, conclude conclusion from premises without witnessing reasoning from premises to conclusion.

  Conclusion, as a term up for question.
  Whether witnessing reasoning from premises to conclusion is part of what the phenomenon of concluding some conclusion from some premises is.

  So, interest is with the following issue:

  \begin{restatable}[Concluding and witnessing]{issue}{issueMain}
    \label{issue:Main}
    Are the cases in which an agent may conclude \(\phi\) has value \(v\) from a pool of premises \(\Phi\) without witnessing reasoning from \(\Phi\) to \(\phi\) having value \(v\)?
  \end{restatable}

  Witnessing, some mental event with inputs as premises and proposition as output.
  In particular, causation implies witnessing.
\end{note}

\begin{note}[Resolutions]
  Our goal is to motivate a positive resolution to~\autoref{issue:Main}.
  Specifically:

  \begin{restatable}[Positive]{resolution}{issueMainPositive}
    \label{issue:Main:R:p}
    \emph{There are cases in which} agent may conclude \(\phi\) has value \(v\) from a pool of premises \(\Phi\) without witnessing reasoning from \(\Phi\) to \(\phi\) having value \(v\).
  \end{restatable}

  Contrast to the negative resolution:

  \begin{restatable}[Negative]{resolution}{issueMainNegative}
    \label{issue:Main:R:n}
    For every case in which an agent concludes \(\phi\) has value \(v\) from a pool of premises \(\Phi\), the agent does so by witnessing reasoning from \(\Phi\) to \(\phi\) having value \(v\).
  \end{restatable}
\end{note}

\section{Concluding}
\label{sec:ideas-1}

\begin{note}[Outline]
  Here, clarify our use of the terms `concludes', `concluding', `concluded', and so on.
\end{note}

\subsection{The agent's epistemic state}
\label{sec:agents-epist-state}

\begin{note}
  Idea.
\end{note}

\begin{note}
  Distinguish agent's epistemic state from \stance{} of the agent.

  Various things we hold to be the case, and these may be relativised to the agent's perspective on how things actually are.
  Other things, hold regardless of the agent's perspective on how things actually are.
  And, things that hold regardless of whether the agent recognises.

  Example.
  Sam shorter than Taylor.
  Then, from epistemic state, Taylor shorter than Sam.
  Doesn't matter whether Sam is shorter than Taylor.
  From perspective of agent's epistemic state.
  Likewise, doesn't matter whether agent recognises shorter.

  Similarly, classical and intuitionistic.
  From int.\ perspective not a proof.
  From classical, also a proof of \(\phi\).

  And, if left the oven on, should check.
  Regardless of whether really did, and regardless of agent's morals.

  No clean divide.
  For present purposes, concluding and \csN{} will take epistemic state as input, but ignore the agent's \stance{}.
  What this means in practice will be seen through the following discussion.
  Roughly, what an agent concludes will depend on epistemic state, and whether \csN{} in concluding will likewise depend.
  Neither will depend on whether the agent recognises, or takes themselves, to have concluded or \csVed{}.
\end{note}


\section{Two ways of concluding}
\label{sec:overview:two-types-reasoning}

\begin{note}[Goal]
  The goal here is to motivate a distinction between how and why.

  Introduce idea of an \itp{}.

  Suggest the possibility of negative resolution to~\autoref{issue:why-inc-in-how}, though not to~\autoref{issue:has-witnessed}.

  In part, motivation for focusing on~\autoref{issue:has-witnessed}.

  However, sufficient motivation as negative to~\autoref{issue:has-witnessed} entails negative to~\autoref{issue:why-inc-in-how}.
\end{note}

\begin{note}[Two ways of concluding]
  In this section, broad idea.
  \adA{} and \adB{}.
\end{note}

\paragraph*{The first type of reasoning: \adA{}}

\begin{note}
  \begin{restatable}[\adA{}]{definition}{defADA}
    \label{AR:adA}
    \label{def:adA}
    \vAgent{} concludes \(\pv{\phi}{v}\) from \(\Phi\) by `\adA{}' if:
    \begin{enumerate}[label=\textsf{S:\arabic*}., ref=(\textsf{S}:\arabic*)]
    \item
      \label{def:adA:psi}
      \vAgent{} concludes \(\pv{\phi}{v}\) by witnessing reasoning from some pool of premises \(\pv{\phi_{1}}{v_{1}},\dots,\pv{\phi_{k}}{v_{k}}\)
    \item
      \(\Phi\) is the collection of all and only the premises \(\pv{\phi_{i}}{v_{i}}\).
    \end{enumerate}
    \vspace{-\baselineskip}
  \end{restatable}

  \adA{} is straightforward.

  As before, concern may be raised about what the relevant premises are, and whether the relevant agent identifies those premises as premises.
  However, granting that an agent always concludes from some collection of premises, the relevant collection exists.

  The restriction to the \emph{exact} collection of premises the agent reasons from is for convenience.
  Nothing in particular hangs on this distinction, but equally nothing much is gained by allowing the inclusion of redundant proposition-value pairs.
\end{note}

\begin{note}[\illu{1}]
  Time from positions of hands on a clock and understanding of how time is represented by such a clock.

  Whether to make a bet from tolerance for risk, distribution of cards in a pack, cards in hand, and rules of the game.
\end{note}

\begin{note}
  \adA{} does not outline a specific way of reasoning.
  Deductive, inductive, etc.\
\end{note}

\begin{note}
  \phantlabel{abstract-adA}
  Basic (abstract) instance of \adA{}:

  {
    \small
    \begin{enumerate}[label=\arabic*., ref=\arabic*]
    \item\label{def:adA:ex:C:Cp} I have concluded \(\phi\) has value \(v\).
    \item\label{def:adA:ex:C:p} So, \(\phi\) has value \(v\). \hfill(From~\ref{def:adA:ex:C:Cp})
    \item\label{def:adA:ex:C:Cps} Likewise, I have concluded \(\psi\) has value \(v'\) when \(\phi\) has value \(v\).
    \item\label{def:adA:ex:C:ps} So, \(\psi\) has value \(v'\) when \(\phi\) has value \(v\). \hfill(From~\ref{def:adA:ex:C:Cps})
    \item\label{def:adA:ex:C:T} If \(\psi\) has value \(v'\) when \(\phi\) has value \(v\) and \(\phi\) has value \(v\), then it must be the case that \(\psi\) has value \(v'\). \hfill (From understanding of `if\dots then\dots')
    \item\label{def:adA:ex:C:s} Hence, \(\psi\) has value \(v'\).\newline
      \mbox{}\hfill (From \ref{def:adA:ex:C:p},~\ref{def:adA:ex:C:ps}~and~\ref{def:adA:ex:C:T})
    \item Therefore, I conclude \(\psi\) has value \(v'\). \hfill (From \ref{def:adA:ex:C:Cp} -- \ref{def:adA:ex:C:s})
    \end{enumerate}
  }
  From this reasoning, two clear premises.
  \(\pv{CS(\pv{\phi}{v})}{\top}\) and \(\pv{CS(\pv{\pv{\phi}{v} \Rightarrow \pv{\psi}{v'}}{\top})}{\top}\).
  Witness reasoning from these premises.
  The reasoning is verbose, premises are that the agent has concluded.
  Here, concluding is not factive.

  % (Consider parallel reasoning with knowledge.%
  % \footnote{The parallel reasoning in full:
  %   \begin{enumerate}[label=\arabic*., ref=\arabic*]
  %   \item\label{def:adA:ex:K:Kp} I know \(\phi\) has value \(v\).
  %   \item\label{def:adA:ex:K:p} So, \(\phi\) has value \(v\). \hfill (From~\ref{def:adA:ex:K:Kp})
  %   \item\label{def:adA:ex:K:Kps} I know \(\psi\) has value \(v'\) when \(\phi\) has value \(v\).
  %   \item\label{def:adA:ex:K:ps} So, \(\psi\) has value \(v'\) when \(\phi\) has value \(v\). \hfill(From~\ref{def:adA:ex:K:Kps})
  %   \item\label{def:adA:ex:K:T} If \(\psi\) has value \(v'\) when \(\phi\) has value \(v\) and \(\phi\) has value \(v\), then it must be the case that \(\psi\) has value \(v'\). \hfill (From understanding of `if\dots then\dots')
  %   \item\label{def:adA:ex:K:s} Hence, \(\psi\) has value \(v'\). \hfill (From \ref{def:adA:ex:C:p},~\ref{def:adA:ex:C:ps}~and~\ref{def:adA:ex:C:T})
  %   \item So, I know \(\psi\) has value \(v'\) as \(\psi\) having value \(v'\) follows from~(\ref{def:adA:ex:K:Kp}) and~(\ref{def:adA:ex:K:Kps}).
  %     \mbox{}\hfill (From \ref{def:adA:ex:K:Kp} -- \ref{def:adA:ex:K:s})
  %   \end{enumerate}
  % }%
  % )
\end{note}


\paragraph*{The second type of reasoning: \adB{}}

\begin{note}[Turning to \adB{}]
  We now turn to the second type of reasoning: `\adB{}'.

  We begin with a definition of \adB{}.
  However, our attention will quickly turn to a pair of helper definitions which relate some proposition-value pair we identify as an `\itp{}' to some other proposition-value pair and pool of premises.
\end{note}

\begin{note}
  \begin{restatable}[\adB{}]{definition}{defADB}
    \label{def:adB}
    \vAgent{} concludes \(\pv{\phi}{v}\) from \(\Phi\) by `\adB{}' if:
    \begin{enumerate}[label=\textsf{I}:\arabic*., ref=(\textsf{I}:\arabic*)]
    \item
      \label{def:adB:itp}
      \vAgent{} has concluded \(\pv{\mu}{v}\) and  \(\pv{\mu}{v}\) is either:
      \begin{enumerate}
      \item
        \label{def:adB:itp:between}
        An \itp{} \emph{between} \(\pv{\phi}{v}\) and \(\Phi\), or:
      \item
        \label{def:adB:itp:for}
        An \itp{} \emph{for} \(\pv{\phi}{v}\), with \(\Phi\) as the relevant pool of premises.
      \end{enumerate}
    \item
      \label{def:adB:conclude}
      \vAgent{} concludes \(\pv{\phi}{v}\) by appeal to the premises \(\pv{\phi_{1}}{v_{1}},\dots,\pv{\phi_{k}}{v_{k}}\) from the pool of premises \(\Phi\) via the possibility of witnessing the relevant reasoning from \(\pv{\mu}{v}\).
    \end{enumerate}
    \vspace{-\baselineskip}
  \end{restatable}
\end{note}

\begin{note}
  Without definitions of what an \itp{0} between \(\pv{\phi}{v}\) and \(\Phi\) is, or what an \itp{0} for \(\pv{\phi}{v}\) is,~\autoref{def:adB} is incomplete.
  We will shortly turn to the relevant helper definitions.

  Though, working backwards from~\autoref{def:adB} gives a hint.
  An \itp{} should contain information that it is possible for the agent to witness reasoning that conclude \(\pv{\phi}{v}\) from some pool of premises \(\Phi\).
\end{note}

\begin{note}
  Still, before turning to the pair of helper definitions, let me stress a key aspect of~\autoref{def:adB}:
  From~\ref{def:adB:conclude}, the agent concludes \(\pv{\phi}{v}\) from \(\Phi\) and \(\Phi\) alone (without witnessing the relevant reasoning).
  The agent does not conclude \(\pv{\phi}{v}\) from \(\Phi\) and \(\pv{\mu}{v}\).
  From the perspective of defining \adB{}, the latter option may be included, but the possibility of the former will be important when we turn to tension.
\end{note}

\paragraph*{Contrast}

\begin{note}
  The key difference between \adA{} and \adB{}:
  \begin{itemize}
  \item \adA{} involves the agent appealing to \(\pv{\mu}{v}\) in order to conclude \(\pv{\phi}{v}\), while
  \item \adB{} does not involve the agent appealing to \(\pv{\mu}{v}\) to conclude \(\pv{\phi}{v}\).
    Instead, the role of \(\phi\) is to highlight \(\rho_{1},\dots,\rho_{k}\) and the agent appeals to propositions \(\rho_{1},\dots,\rho_{k}\) to claim support for \(\psi\).
  \end{itemize}

  For the definition to be satisfied, \(\phi\) needs only be involved to the extent that it provides the link.
  Hence, \(\phi\) is not irrelevant.
  Still, the agent does not appeal to \(\phi\).
\end{note}

\paragraph*{\adB{}: Helper definitions}

\begin{note}[]
  We briefly noted, working backwards from~\autoref{def:adB}, that an \itp{} should contain information that it is possible for the agent to witness reasoning that conclude \(\pv{\phi}{v}\) from some pool of premises \(\Phi\).
  We now detail what an \itp{0}.
  Or, rather, what \itp{1} are.

  There are two cases.
  First, an \itp{0} \emph{between} \(\pv{\phi}{v}\) and \(\Phi\).
  Second, an \itp{0} \emph{for} \(\pv{\phi}{v}\).
  The distinction between these cases is whether the relevant \itp{0} identifies a particular pool of premises.

  In practice, we will blur the distinction, but from a definitional perspective the latter is best seen as a generalisation of the former.
\end{note}

\subparagraph*{An \itp{0} between \(\pv{\phi}{v}\) and \(\Phi\)}

\begin{note}[\itp{} between]
  \begin{definition}[An \itp{0} between \(\pv{\phi}{v}\) and \(\Phi\) \hfill \named{I.b}]
    \label{def:itp:b}
    \(\pv{\mu}{v}\) is an \itp{} \emph{between} \(\pv{\phi}{v}\) and \(\Phi\) if and only if:

    \begin{enumerate}[label=\arabic*., ref=\named{\textsf{I.b}:\arabic*}]
    \item
      \label{def:itp:b:pR}
      \(\mu\) having value \(v\) ensures:
      \begin{itemize}
      \item
        It is possible for \vAgent{} to conclude \(\phi\) has value \(v\) by witnessing reasoning from \(\Phi\) to \(\pv{\phi}{v}\), given \vAgent{}'s present epistemic state.
      \end{itemize}
    \item
      \label{def:itp:b:distinct}
      \(\pv{\mu}{v}\) is not equivalent to any \(\pv{\phi_{i}}{v_{i}}\), given \vAgent{}'s present epistemic state.
    \end{enumerate}
    \vspace{-\baselineskip}
  \end{definition}
\end{note}

\begin{note}[Plan]
  The definition of an \itp{} between \(\pv{\phi}{v}\) and \(\Phi\) consists of two components:~\ref{def:itp:b:pR} and~\ref{def:itp:b:distinct}.

  \ref{def:itp:b:pR} is the core of the definition, while~\ref{def:itp:b:distinct} narrows the definition to cases of interest.

  We begin by expanding on~\ref{def:itp:b:pR}, and then motivate the restriction given by~\ref{def:itp:b:distinct}.
\end{note}

\begin{note}[Expanding on~\ref{def:itp:b:pR}]
  Intuitively, think of an \itp{} between \(\pv{\phi}{v}\) and \(\Phi\) as a particular kind of conditional of the (rough) form `if \(\Phi\) then \(\pv{\phi}{v}\)'.

  Indeed, an `\emph{if} \dots \emph{then} \dots' statement between \(\Phi\) and \(\pv{\phi}{v}\) may be constructed from any \itp{} between \(\pv{\phi}{v}\) and \(\Phi\).

  For, if it is possible for an agent to conclude \(\phi\) has value \(v\) by witnessing reasoning from \(\Phi\) to \(\pv{\phi}{v}\), then (from the agent's perspective at least), \(\pv{\phi}{v}\) whenever \(\pv{\phi_{i}}{v_{i}}\) for each \(\pv{\phi_{i}}{v_{i}}\) in \(\Phi\).
  So, if every proposition \(\phi_{i}\) in \(\Phi\) has it's respective value \(v_{i}\), then \(\phi\) also has value \(v\).
  Or, more colloquially, if \(\Phi\) then \(\pv{\phi}{v}\).

  However, an \itp{} between \(\pv{\phi}{v}\) and \(\Phi\) is stronger than `if \(\Phi\) then \(\pv{\phi}{v}\)'.
  For, not only is it the case that \(\pv{\phi}{v}\) whenever \(\pv{\phi_{i}}{v_{i}}\) for each \(\pv{\phi_{i}}{v_{i}}\) in \(\Phi\), but in addition it is possible for the agent to conclude \(\pv{\phi}{v}\) from \(\Phi\) given the agent's present epistemic state.

  Naturally, the possibility for the agent to conclude \(\pv{\phi}{v}\) from \(\Phi\) goes beyond a plain conditional between \(\pv{\phi}{v}\) and \(\Phi\).

  Breaking down \autoref{def:itp:b:pR}, observe we have an `inner' statement:
  \begin{quote}
     It is possible for \vAgent{} to conclude \(\phi\) has value \(v\) by witnessing reasoning from \(\Phi\) to \(\pv{\phi}{v}\).
  \end{quote}
  And, a qualifier:
  \begin{quote}
    [G]iven \vAgent{}'s present epistemic state.
  \end{quote}

  The statement is simple.
  The relevant possibility is just for the agent to conclude \(\pv{\phi}{v}\) from \(\Phi\) by an instance of \adA{}.
  Indeed, the relevant instances of \EAS{} we motivate by developing tension will always involve an \itp{}, and hence will always involve the possibility of witnessing reasoning to the relevant conclusion from some pool of premises.

  Turn now to the qualifier:
  \begin{quote}
    [G]iven \vAgent{}'s present epistemic state.
  \end{quote}
  This is a qualifier on possible witnessing.

  Generally speaking, it may be possible for an agent to conclude \(\pv{\phi}{v}\) from \(\Phi\) by an instance of \adA{} from a distinct epistemic state.
  For example, if the agent were to learn some \(\pv{\phi_{i}}{v_{i}}\) in \(\Phi\) is the case, or if the agent were to improve their reasoning skills.
  However, the innermost qualifier ensures that it possible for an agent to conclude \(\pv{\phi}{v}\) from \(\Phi\) without any revision to the agent's epistemic state.

  An important consequence of this qualifier is that the agent must already hold that for each \(\pv{\phi_{i}}{v_{i}}\) in \(\Phi\), \(\phi_{i}\) has value \(v_{i}\).
  For, if not, then \(\pv{\phi_{i}}{v_{i}}\) would not be available as a premise.
  Of course, the definition of an \itp{} between \(\pv{\phi}{v}\) and \(\Phi\) may be given without this assumption, but we have no use for any more general definition.
\end{note}

\begin{note}[\illu{2}]
  \color{red}
    For example, consider being informed that the first player in a game of tic-tac-toe may always guarantee a draw.
  No premises are specified, but on reflection it is clear to see that one may reason through all the possible games to identify the strategy.
  The relevant \itp{0}, then, is the combination of the novel information and one's understanding of tic-tac-toe, the premises, some general results about tic-tac-toe, and the conclusion the required strategy.

  What guarantees the possibility of concluding --- general properties of tic-tac-toe which follow from the rules --- is intuitively distinct from the relevant pool of premises, which likely be limited to the rules themselves combined with the \dots

\end{note}

\begin{note}
  A quick observation.

  \(\pv{\mu}{v}\) being an \itp{} between \(\pv{\phi}{v}\) and \(\Phi\) depend on whether or not it is actually possible for the agent to conclude \(\pv{\phi}{v}\) from \(\Phi\), given their epistemic state.
  If it is not possible for the agent to witness reasoning from \(\Phi\) to \(\pv{\phi}{v}\), then no such \itp{} will exist.

  However, whether or not an agent \emph{concludes} \(\pv{\mu}{v}\) is an \itp{} between \(\pv{\phi}{v}\) and \(\Phi\) does not depend on whether or not it is actually possible for the agent to conclude \(\pv{\phi}{v}\) from \(\Phi\), given their epistemic state.
  We do not assume that an agent concludes \(\pv{\phi}{v}\) only if \(\phi\) actually has value \(v\).
  And, our interest with \itp{1} will typically be from the perspective of the agent's present epistemic state.
\end{note}

\begin{note}[Expanding on~\ref{def:itp:b:distinct}]
  The above has expanded on~\ref{def:itp:b:pR}.
  Finally, we turn to~\ref{def:itp:b:distinct}.

  In short,~\ref{def:itp:b:distinct} ensures that if \(\pv{\mu}{v}\) is an \itp{0} between \(\pv{\phi}{v}\) and \(\Phi\), then \(\pv{\mu}{v}\) is not a premise that the agent would appeal to when witnessing the reasoning from \(\Phi\) to \(\pv{\phi}{v}\) captured by~\ref{def:itp:b:pR}.

  More strictly, not only is \(\pv{\mu}{v}\) not a premise, but is not equivalent to any \(\pv{\phi_{i}}{v_{i}}\) in \(\Phi\).
  Where, again, equivalence is evaluated from the perspective of the agent.

  From an abstract perspective, if \(\pv{\mu}{v}\) is an \itp{0} between \(\pv{\phi}{v}\) and \(\Phi\), then \(\pv{\mu}{v}\) is purely descriptive of the relationship between \(\pv{\phi}{v}\) and \(\Phi\).

  In other words, \(\pv{\mu}{v}\) is not required to conclude \(\pv{\phi}{v}\) from \(\Phi\).

  Now,~\ref{def:itp:b:distinct} is a somewhat arbitrary restriction.

  In general, it is plausible that some \(\pv{\mu}{v}\) may both inform an agent that they may conclude \(\pv{\phi}{v}\) from \(\Phi\), but is also a member of \(\Phi\).

  Indeed, consider the conditional `if \(\pv{\alpha}{v}\) then \(\pv{\beta}{v'}\)'.
  Granting the conditional allows detachment, then it is surely possible for an agent to reason from the pool of premises \(\{\pv{\alpha}{v}, \text{if} \pv{\alpha}{v} \text{ then } \pv{\beta}{v'}\}\) to \(\pv{\beta}{v'}\).%
  \footnote{
    So long as the agent has already concluded \(\pv{\alpha}{v}\).
  }

  Indeed, without~\ref{def:itp:b:distinct}, \itp{1} would be abundant.
  Hence,~\ref{def:itp:b:distinct} narrows our attention to cases of interest, cases where \(\pv{\mu}{v}\) merely describes --- and does not partake --- in the reasoning of interest.%
  \footnote{
    Our course, ruling out an abundance of potential \itp{1} via~\ref{def:itp:b:distinct} carries risk of ruling out interesting \itp{1}.
    I encourage further investigation.
    Still, as \itp{1} are only of indirect interest, simplicity via arbitrary restrictions is favoured over complexity from general definitions.
  }
\end{note}

\begin{note}[Summary of \itp{0} between]
  \dots
\end{note}

\subparagraph*{An \itp{} for \(\pv{\phi}{v}\)}

\begin{note}[\itp{2} for]
  We now turn to the second helper definition, that of an \itp{0} for some proposition-value pair.
  In short, an \itp{0} \emph{between} \(\pv{\phi}{v}\) and \(\Phi\) ensures the possibility of the agent concluding \(\pv{\phi}{v}\) from \(\Phi\).
  And, by contrast, an \itp{0} \emph{for} \(\pv{\phi}{v}\) ensures there is some \(\Phi\) such that it is possible for the agent to conclude \(\pv{\phi}{v}\) from \(\Phi\).
\end{note}

\begin{note}[def: \itp{2} for]
  \begin{definition}[An \itp{0} for \(\pv{\phi}{v}\) \dots --- \named{I.f}]
    \label{def:itp:f}
    \(\pv{\mu}{v}\) is an \itp{} \emph{for} \(\pv{\phi}{v}\) if and only if:
    \begin{enumerate}[label=\arabic*., ref=\named{I.f:\arabic*}]
    \item
      \label{def:itp:f:pR}
      \(\mu\) having value \(v\) ensures there is some pool of proposition-value pairs \(\Phi\) and proposition-value pair \(\pv{\mu'}{v'}\) such that:
      \begin{itemize}
      \item
        \(\pv{\mu'}{v'}\) is an \itp{} between \(\pv{\phi}{v}\) and \(\Phi\).
      \end{itemize}
    \item
      \label{def:itp:f:distinct}
      \(\pv{\mu}{v}\) is not equivalent to any \(\pv{\phi_{i}}{v_{i}}\), given \vAgent{}'s present epistemic state.
    \end{enumerate}
    \vspace{-\baselineskip}
  \end{definition}
\end{note}

\begin{note}
  As with~\autoref{def:itp:b},~\autoref{def:itp:f} contains two components; a core and a restriction.
\end{note}

\begin{note}
  The core is straightforward.
  \ref{def:itp:f:pR} requires \(\pv{\mu}{v}\) to do ensure two things:
  \begin{enumerate}
  \item The existence of some pool of proposition-value pairs \(\Phi\), and
  \item The existence of an \itp{0} between \(\pv{\phi}{v}\) and \(\Phi\).
  \end{enumerate}
  In other words, then, an \itp{0} \emph{for} \(\pv{\phi}{v}\) just is the guarantee of an \itp{0} \emph{between} \(\pv{\phi}{v}\) and some pool of premises \(\Phi\).

  Simply as it may be, the definition of a \itp{0} for will prove quite useful, as we avoid the need to specify any particular pool of premises.

  Further, any \itp{0} between \(\pv{\phi}{v}\) and \(\Phi\) is always an \itp{0} for \(\pv{\phi}{v}\).
  For, suppose \(\pv{\mu}{v}\) is an \itp{0} between \(\pv{\phi}{v}\) and \(\Phi\).
  Then, \(\Phi\) is the relevant pool of premises and \(\pv{\mu}{v}\) itself is the relevant \(\pv{\mu'}{v'}\), and by definition \(\pv{\mu}{v}\) is not equivalent to any \(\pv{\phi_{i}}{v_{i}}\) in \(\Phi\), given the agent's present epistemic state.

  Note, however, that in general, if \(\pv{\mu}{v}\) is an \itp{0} for \(\pv{\phi}{v}\) then \(\pv{\mu}{v}\) need not be an \itp{0} between \(\pv{\phi}{v}\) and some pool of premises.
  Simply, though any \itp{0} between is also an \itp{0} for, the converse does not hold.
  For, an \itp{0} requires a pool of premises to be specified.
  Therefore, \(\pv{\mu}{v}\) and \(\pv{\mu'}{v'}\) must, in general, be distinct.

  Of course, an \itp{0} for \(\pv{\phi}{v}\) may have the same general statement as an \itp{0} between \(\langle \phi,\Phi \rangle\).

  Consider again tic-tac-toe.
  Above, we considered the \itp{0} between the existence of a strategy for first player in a game to guarantee a draw from the rules of tic-tac-toe.
  Though, with a moments reflection the statement that there existence of a strategy for first player in a game to guarantee a draw will typically lead to an \itp{0} for the existence of the relevant strategy.
  For, some premises must exist, and given the simplicity of tic-tac-toe these are surely within the grasp of an agent who understands the rules of tic-tac-toe.
\end{note}

\begin{note}
  Finally, the restriction~\ref{def:itp:f:distinct} functions in parallel to the restriction \ref{def:itp:b:distinct} of~\autoref{def:itp:b}.
  An \itp{0} for \(\pv{\phi}{v}\) is purely descriptive, and does not participate in concluding \(\pv{\phi}{v}\) from the relevant pool of premises.
  A more general definition may be given without this restriction, but such a definition is beyond present interest.
\end{note}

\paragraph*{\adB{}}

\begin{note}
  With the two helper definitions in hand, let us return to the definition of \adB{}.

  Recall two components:~\ref{def:adB:itp} and~\ref{def:adB:conclude}.

  \ref{def:adB:itp} stated that the agent has concluded \(\pv{\mu}{v}\) where \(\pv{\mu}{v}\) is either:
  \begin{enumerate}[label=(\alph*)]
  \item An \itp{} \emph{between} \(\pv{\phi}{v}\) and \(\Phi\), or
  \item An \itp{} \emph{for} \(\pv{\phi}{v}\), with \(\Phi\) as the relevant pool of premises
  \end{enumerate}
  We have seen the relevant definitions.

  So, given the agent has concluded \(\pv{\mu}{v}\), for some \itp{0} \(\pv{\mu}{v}\) then it is possible for the agent to conclude \(\pv{\phi}{v}\) from some pool of premises \(\Phi\).

  \ref{def:adB:conclude} is key.

  The agent concludes \(\pv{\phi}{v}\) by appeal to the pool of premises \(\Phi\) \emph{via} the possibility of witnessing the relevant reasoning from \(\Phi\) to \(\pv{\phi}{v}\) given by the \itp{0} \(\pv{\mu}{v}\).

  Hence, the agent does \emph{not} conclude \(\pv{\phi}{v}\) from \(\pv{\mu}{v}\) or indeed from some pool of premises for which \(\pv{\mu}{v}\) is a member.
  Indeed, the latter point follows from~\ref{def:itp:b:distinct} and~\ref{def:itp:f:distinct} --- an \itp{0} is always purely descriptive of some possible reasoning.

  Instead, the pool of premises the agent concludes \(\pv{\phi}{v}\) from is the collection of those premises that they agent would appeal to if they were to witness the relevant instance of reasoning given by the \itp{0}.

  Of course, the presence of an \itp{0} is, intuitively, crucial.
  For, without an \itp{0} the agent would lack information that it is possible to conclude \(\pv{\phi}{v}\) from some pool of premises.
\end{note}

\paragraph{A pair of \illu{3}}

\begin{note}
  To \illu{0} \adA{} and \adB{} we work through two \illu{1} in some detail.
  Both \illu{1} share two components:
  \begin{enumerate}[label=\alph*., ref=(\alph*)]
  \item
    \label{adX:illu:struc:mem}
    Memory of creating a syntactic proof for some first order formula.
  \item
    \label{adX:illu:struc:concl}
    Concluding that the relevant formula is a theorem of first-order logic.
  \end{enumerate}

  The key difference between the two \illu{1} is whether the memory~\ref{adX:illu:struc:mem} serves as a premise or an \itp{0} for the conclusion~\ref{adX:illu:struc:concl}.

\end{note}
\begin{note}[Two premises]
  \begin{quote}
    \begin{enumerate}[%
      label={(Mem)},%
      ref={(Mem)}%
      ]
    \item
      \label{ill:Eproof:mem}
      I remember having created a syntactic proof of \formula{\forall x Px \rightarrow \lnot \exists x \lnot P x} (using a sound first-order system).%
      \footnote{
        We use the phrasing `having created' rather than `creating' to imply completion.
      }
    \end{enumerate}
  \end{quote}
  And:
  \begin{quote}
    \begin{enumerate}[%
      label={(\(\exists\mathord{\vdash}{,}\top\))},%
      ref={(\(\exists\mathord{\vdash}{,}\top\))}%
      ]
    \item
      \label{ill:Eproof:def}
      The existence of a syntactic proof of a formula (using a sound first-order system) is sufficient to establish the formula is a (syntactic) theorem of first-order logic.
    \end{enumerate}
  \end{quote}
\end{note}

\paragraph{First \illu{0} (\adA{})}

\begin{note}
  \begin{illustration}[\adA{}]
    \label{ill:ad:proof:mem}
    \mbox{}
    \vspace{-\baselineskip}
    \begin{enumerate}[%
      label=\arabic*,%
      ref=({I}.{\ref{ill:ad:proof:mem}}:\arabic*)%
      ]
    \item \illEproofMem{} \hfill \ref{ill:Eproof:mem}
    \item
      \label{ill:Eproof:exP}
      So, there exists a syntactic proof of \formula{\forall x Px \rightarrow \lnot \exists x \lnot P x} (using a sound first-order system)
    \item
      \label{ill:Eproof:thm}
      Hence, by \ref{ill:Eproof:def}, \formula{\forall x Px \rightarrow \lnot \exists x \lnot P x} is a theorem of first-order logic.
    \end{enumerate}
    \vspace{-\baselineskip}
  \end{illustration}
\end{note}

\begin{note}[Discussion of \autoref{ill:ad:proof:mem}]
  I take \autoref{ill:ad:proof:mem} to be a straightforward case of concluding.
  \ref{ill:Eproof:mem}, memory,


  and \ref{ill:Eproof:def}, to recast the existence of a proof from \ref{ill:Eproof:mem} in terms of the formula being a theorem.

  Neither premise from anything more basic, and without either premise the conclusion would not be obtained.
  For, without \ref{ill:Eproof:mem} no proof, and without \ref{ill:Eproof:def} no recasting.
\end{note}

\begin{note}
  It seems sufficient, generally speaking, to conclude some proposition has value \(v\) by appeal to memory, hence the agent claims support that there was some event which culminated in a syntactic proof of the formula.

  Of course, the agent may have misremembered.
  Still, we do not require that any agent concludes \(\pv{\phi}{v}\) from \(\Phi\) only if \(\phi\) \emph{actually} has value \(v\) and each \(\pv{\phi_{i}}{v_{i}}\) in \(\Phi\), \(\phi_{i}\) \emph{actually} has value \(v_{i}\).

  Following, this allows the agent to conclude a syntactic proof of the formula exists.
  As before, the agent may have failed to \emph{actually} create a syntactic proof of the formula
  Still, from the same perspective this does not prevent the agent from concluding they did (actually) create such a proof.

  Hence, finally, the agent claims support that the formula is a (syntactic) theorem of first-order logic.
\end{note}

\begin{note}
  To concisely summarise, we may say that the agent conclude the is a (syntactic) theorem of first-order logic \emph{because} of their understanding of syntactic theorem-hood and their memory of proving the formula.

  For sure,~\autoref{ill:ad:proof:mem} is designed to be as straightforward as possible.
  Of interest is not whether the agent claims support, but how the role the agent gives to their memory in claiming support.

  The agent appeals to their memory to establish that there exists a syntactic proof of the formula, and then combines the existence of a syntactic proof with~\ref{ill:Eproof:def} to claim support that the formula is a theorem.
  Hence, the agent's memory is directly involved in their claimed support for the formula being a theorem.%
    \footnote{
      \color{red}
      Whether proving is an unsatisfied \requ{}.
      However, recall that allowed a \requ{} to be satisfied by some instance of concluding.
      And, memory of concluding.
      Still question about original proof, but no problem with memory.
      Also, method.
    }
\end{note}

\paragraph{Second \illu{0} (\adB{})}

\begin{note}
  \begin{illustration}[\adB{}]
    \label{ill:ad:proof:eve}
    \mbox{}
    \vspace{-\baselineskip}
    \begin{enumerate}
    \item \illEproofMem{} \hfill \ref{ill:Eproof:mem}
    \item
      \label{ill:ad:proof:eve:app}
      In creating the syntactic proof I appealed to various aspects of some sound first-order system.
    \item
      \label{ill:ad:proof:eve:pos}
      As I created a proof, those various aspects of the sound first-order system are sufficient to ensure there exists a proof.
    \item
      Hence, by \ref{ill:Eproof:def}, \formula{\forall x Px \rightarrow \lnot \exists x \lnot P x} is a theorem of first-order logic.
    \end{enumerate}
    \vspace{-\baselineskip}
  \end{illustration}

  {
    \color{red}
    Premises are rules are part of a sound system, may be combined in this way.%
    \footnote{
      Indeed, relative simplicity is why we chose a syntactic rather than semantic proof.
      With semantic, need an argument which covers all models, and while premises plausibly exist, these have no common codification.
    }
  }

  As with~\autoref{ill:ad:proof:mem}, the agent's memory has a role in~\autoref{ill:ad:proof:eve}, but the role is quite different.
  Above, the agent claimed support for the formula being a theorem primarily \emph{because} they remembered creating a proof.
  By contrast, here the agent claims support for the formula being a theorem primarily because of the properties of some sound first-order system.

  Step~\ref{ill:ad:proof:eve:app} appeals to various aspects of some sound first-order system and, in turn, step~\ref{ill:ad:proof:eve:app} observes that those aspects are sufficient to ensure a proof exists.
  The agent claims support for the existence of a proof by appeal to the various aspects of some first-order system they appealed to when constructing the proof, rather than their memory of constructing the proof.
\end{note}

\begin{note}
  To help clarify, let's fix a particular syntactic proof using the Fitch-style proof system of~\textcite[557--560]{Barwise:1999tu}:

  \begin{figure}[H]
    \centering
    \begin{quote}
      \fitchprf{}{
        \subproof{\pline[1.]{\forall x P x}}{
          \subproof{\pline[2.]{\exists x \lnot Px}}{
            \boxedsubproof[3.]{a}{\lnot Pa}{
              \pline[4.]{Pa}[\lalle{1}] \\
              \pline[5.]{\bot}[\lfalsei{3}{4}]
            }
            \pline[6.]{\bot}[\lexie{2}{3--5}]
          }
          \pline[7.]{\lnot \exists x \lnot Px}[\lnoti{2--6}]
        }
        \pline[8.]{\forall x Px \rightarrow \lnot \exists x \lnot Px}[\lifi{1--7}]
      }
    \end{quote}
    \caption{A syntactic proof}\label{fig:syntx-prf}
  \end{figure}

  The proof consists of single instances of five introduction or elimination rules.
  Each rule is part of the Fitch-style proof system, and the specific application of the rules constitute the proof.
\end{note}


\begin{note}[Before\dots]
  Before returning to~\autoref{ill:ad:proof:eve}, let us observe that with the proof in hand one may claim support that a proof of the formula exists via the contents of~\autoref{fig:syntx-prf}.

  Broadly stated:

  \begin{enumerate}
  \item The proof is constructed from a sound first-order proof system.
  \item And, the particular application of some rules of the system to formulae is such that the proof begins with no assumptions and the last line of the proof is not part of any assumption made during the course of the proof.
  \end{enumerate}
\end{note}

\begin{note}
  Note, appeal to creation of the proof involves appeal to various aspects of the Fitch-style proof system.

  The object itself is mute to whether or not it is a proof.

  For example, adding `\formula{Ba}' as an assumption would void the proof, but you would need to observe that the appeal to existential elimination on line 6 requires that `\formula{a}' does not appear in the proof prior to its introduction on line 3 in order to claim support that the proof is void.

  Indeed, the proof consists of eight steps, each step is permitted by the first-order system, the proof begins with no assumptions, the last line of the proof is not part of any assumption made during the course of the proof and the proof, and so on.

  Sparing the details, claimed support that~\autoref{fig:syntx-prf} is a syntactic proof of \formula{\forall x Px \rightarrow \lnot \exists x \lnot P x} from the creation of~\autoref{fig:syntx-prf} is a matter of claiming support for each step of the creation.

  Indeed, to spare the details in general, let us instead talk of some collection of propositions and steps of reasoning.
  Claiming support that a proof exists from the some creation in the way under discussion is an instance of reasoning from details of the creation to the conclusion that a proof exists.
  Hence, as an instance of reasoning involves certain premises and steps of reasoning.
  And, whatever these turn out to be, the proceed from the creation of the proof rather than from some other source such as memory, testimony, and so on.
\end{note}

\begin{note}
  In other words, one may claim support that a proof of \formula{\forall x Px \rightarrow \lnot \exists x \lnot P x} exists (primarily) \emph{because} of their reasoning from some collection of premises and steps of reasoning concerning the creation to the existence of a proof of \formula{\forall x Px \rightarrow \lnot \exists x \lnot P x}.
\end{note}

\begin{note}[Return to \ref{ill:ad:proof:eve}]
  Now let us return to the reasoning of~\autoref{ill:ad:proof:eve}, and in particular steps~\ref{ill:ad:proof:eve:app} and~\ref{ill:ad:proof:eve:pos}:
  \begin{quote}
    \begin{enumerate}
      \setcounter{enumi}{1}
    \item In creating the syntactic proof I appealed to various aspects of some sound first-order system.
    \item As I created a proof, those various aspects of the sound first-order system are sufficient to ensure there exists a proof.
    \end{enumerate}
  \end{quote}
  Given that the agent remembers having created a syntactic proof, the `various aspects of some sound first-order system' of step~\ref{ill:ad:proof:eve} may be taken as those aspects of the first-order system that were appealed to in the premises and steps of reasoning when the agent created the proof.
  And step \ref{ill:ad:proof:eve}, in turn, appeals to how those various aspects of some sound first-order system were sufficient for the agent to claim support that a proof exists by the reasoning that occurred.

  In short, the agent remembers creating a syntactic proof and claiming support that a proof exists from the creation.
  The instance of claiming support involved reasoning from premises via steps to the relevant conclusion.
  Hence, it is possible to claim support for the conclusion by those premises and steps of reasoning.
  So, in~\ref{ill:ad:proof:eve} the agent observes that those premises and steps of reasoning are sufficient to claim support by way of their memory, and in turn appeals to those premises and steps of reasoning to claim support for the relevant conclusion.
\end{note}

\begin{note}
  {
    \color{red}
    Propositional support.
    (If I talk about this, it should be after the definitions.)
  }
\end{note}

\begin{note}
  Generalising, the way in which the agent claims support in~\autoref{ill:ad:proof:eve} is of interest because the agent appeals to premises and steps of reasoning that are not `part' of their present reasoning.
  The role of memory in the \illu{0} is (merely) a way for the agent to recognise that there are such premises and steps of reasoning.
  And, in the definitions that follow, we will abstract from any particular way in which the recognises that relevant premises and steps of reasoning are available.

  Still, even though memory is contingent, we may briefly observe that the way in which the agent claim support in~\autoref{ill:ad:proof:eve} is compatible with \ESU{}.
  For, \ESU{} requires that an agent may claim support for some conclusion from premises and steps of reasoning only if the agent has witnessed reasoning to the conclusion from those premises via those steps of reasoning.
  So, if the initial instance of claiming support conformed to \ESU{} then the agent will have witnessed reasoning from those steps and premises to the conclusion --- the instance of claiming support in~\autoref{ill:ad:proof:eve} does not involve such witnessing, but the agent's memory would be about how the relevant premises and steps were used to claim support.

  Of course, the way in which the agent claim support in~\autoref{ill:ad:proof:eve} is incompatible with a strengthened variant of \ESU{} which requires the agent to use any premises and steps they appeal to in the \emph{present} instance of reasoning, but the point for the moment is that the way in which the agent claims support in~\autoref{ill:ad:proof:eve} does not already require what we are arguing against: \ESU{}.
\end{note}

\begin{note}
  \color{red}
  In this case, \adB{} is not incompatible with \ESU{}.
  For, the agent has witnessed reasoning (granting memory).
  So, \ESU{} does not lead to an immediate rejection of \adB{}.

  Oh, this is noted.
\end{note}

\subsubsection{Additional illustrations}

\begin{note}

  \begin{illustration}
    \mbox{}
    \vspace{-\baselineskip}
    \begin{itemize}
    \item If bag are overweight then they can't be taken on the flight.
    \item Machine reads\dots
    \item Bag can't be taken on the flight.
    \end{itemize}
  \end{illustration}
  Contents of the bag are overweight.

  Combined weight of the items versus the combination of the individual weights.

  Compare, filling the bag and weighing it, versus summing the weight of the items as you fill the bag.

  Now, seems possible to fill the bag and weight it, then appeal to the sum of the items.

  So, this is a little more subtle.
  The bag has been weighed, and the distinction is between the weight of the contents of the bag, and the combined weight of the items that make up the contents of the bag.

  This is particularly interesting.
  Because, it seems clear that something is strange if someone talks about the weight of the contents of the bag without recognising that this is a function of the combined weight of all the individual elements of the bag.
  However, no idea what the contents of the bag are.

  So, claiming support from what is has been observed, the combined weight, rather than what must be the case in order to have made the observation.
\end{note}

\section{\fc{3}}
\label{sec:foregone-conclusions}

\begin{note}[Foregone-conclusions]
  Basic idea of a foregone-conclusion.

  \begin{restatable}[Foregone-conclusions]{definition}{definitionForegoneC}
    \(\pv{\phi}{v}\) is a foregone-conclusion from some pool of premises \(\Phi\) just in case, given the agent's present epistemic, the agent would not fail to conclude \(\pv{\phi}{v}\) from \(\Phi\) were the agent to reason.
  \end{restatable}

  Whether foregone-conclusion takes agent's present epistemic state as a function.
  However, does not need to be the case that the agent recognises foregone-conclusion.

  At most, witnessing provides information about method.

  For any property \(P\) which would follow from any instance of witnessing reasoning \(\pv{\phi}{v}\) from \(\Phi\), the agent's present epistemic state is sufficient to determine \(P\) without witnessing reasoning from \(\Phi\) to \(\pv{\phi}{v}\).

  Suppose \(P\) follows from concluding.
  Forgegone-conclusion.
  So, agent's present epistemic state, agent would not fail.
  However, it then follows that \(P\).

  Here, restricted \(P\) to follow from any.
  Hence, if there are multiple methods, \(P\) may be restricted.

  However, broaden.
\end{note}

\begin{note}[Intuitive cases]
  Knowing whether and knowing how to.
  More or less interchangeable.

  Know whether \(x + y = z\).
  Know how to calculate \(x + y\).
  Indeed, for any \(z\), know whether \(x + y = z\).

  Sudoku puzzles.
  Know how to figure out.
  So, know whether any solution is valid.

  Of course, in certain cases, there are shortcuts.
  Two even numbers, then know whether by checking whether the last digit is even or odd.
  And, other cases, contingent shortcut, such as two of the same number in a square for Sudoku.

  So, really, knowing how to.
\end{note}

\begin{note}[Weaken]
  \fc{2} is weaker.
  Knowing, factive.
  Though, plausible that these amount to the same thing in various cases.
  Either because \fc{} is determined by knowing how to.
  Or, because knowing is weakened to the agent's perspective.
\end{note}

\begin{note}
  \begin{proposition}
    For any path, present epistemic state determines availability of path.
  \end{proposition}

  Start.
  Then, continue.
  Started from \(\Phi\), so will conclude.
  Hence, no matter choice made, must have taken the possibility of this choice into account.
  So, it must be the case that determined.

  Hence, if witness, then via some path.

  So, witnessing predetermined path.
  Any instances of concluding by witnessing reduces to witnessing predetermined path.

  Witnessing may provide information about path, but witnessing doesn't 


  For any X from W,
  present determines whether or not X from agent's point of view, then forgone conclusion.

  In other words, agent's present epistemic state determines.
  Agent may need to witness to figure out how determined, but witnessing does not influence.
\end{note}

\begin{note}[Non-cases]
  Now, \(p, p \rightarrow q \vdash q\) case.
  Well, determines \(q\), if we ignore possibility of revision.
  However, this doesn't tell us about all X.

  This is much stronger.
  There is nothing witnessing adds which is not already determined.

  2 + 2, peano arithmetic.

  More intuitive.
  Though, still question about deriving 2 + 2 = 4 from peano arithmetic.

  Understanding of arithmetic.
  Then, add two numbers.
  Forgone-conclusion.

  % Sudoku puzzle.
  % Solution is a foregone-conclusion.
\end{note}

\begin{note}
  To \illu{0}, questions and answers.

  Do you know whether \(83\) is prime?

  Not off the top of my head.

  Do you know whether \(28 + 55 = 83\)?

  Sure, but give me a moment.

  Do you know whether \dots

  No.

  Of course, might hold that the agent needs to have figured things out.
  But, then we have a plausible reduction.
  Knowing whether, and witnessing whether.
  Common component.

  Now, idea is a little different, as knowledge implies factivity.
  Interest with concluding is that not necessarily factive.
  From the agent's perspective.

  `Determining whether'.
  Or, rather `\fc{0}'.
\end{note}

\begin{note}
  Similar to Goldman, etc.?
  The idea is justification\dots
\end{note}

\begin{note}[Trimming]
  \begin{proposition}
    Basically, there's no role for anything beyond \(\Phi\) in the case of a foregone-conclusion.

    \(\Phi\), in the context of the agent's present epistemic state is sufficient to secure the conclusion, and the possibility of witnessing reasoning.
  \end{proposition}

  \begin{proposition}
    Foregone-conclusion just in case \(\Phi\) supports \(\pv{\phi}{v}\).
    \begin{argument}
      In short, given agent's present epistemic state, there's a guaranteed path from \(\Phi\) to \(\pv{\phi}{v}\).
    \end{argument}
    In other words, if \(\Phi\) does not support \(\pv{\phi}{v}\), then \(\pvp{\phi}{v}{\Phi}\) is not a foregone-conclusion.
  \end{proposition}

  Now, as the agent has not witnessed reasoning, need information that \(\pvp{\phi}{v}{\Phi}\) is a foregone-conclusion in order to recognise this.
  However, with information that \(\pvp{\phi}{v}{\Phi}\) is a foregone-conclusion, the information has no role in supporting \(\pv{\phi}{v}\).
\end{note}

\begin{note}
  So, that \(\pvp{\phi}{v}{\Phi}\) is a \fc{0} provides information, and explains, in part or whole, \emph{how} the agent concludes \(\pv{\phi}{v}\).
  However, \emph{why} is accounted for by \(\Phi\).
\end{note}

\begin{note}
  Why does it matter which?

  \begin{idea}[Reduction]
    If foregone-conclusion, then witness relation established by being a foregone-conclusion.
    Hence, reduction.
  \end{idea}
  Here, reduction.
  Restriction to `some'.
  It is not the case that every conclusion is a foregone-conclusion.

  However, if foregone-conclusions are of interest, then some motivation.

  Still, why of interest?
  What role do foregone-conclusion have?

  In particular.
  Need to get that some conclusion is a foregone-conclusion.
  From some reasoning.
  So, some pool of premises.
  And, that pool of premises is sufficient for conclusion.

  So, source for 2 + 2 = 4 and general ability.
  Already concluded that 2 + 2 = 4.
  Not quite, still going from general ability to 2 + 2 = 4.

  So, it's not clear that reduction is irrelevant.

  However, it is also not clear that this reduction is general.
  Want to show that foundation for reduction is not limited.
\end{note}

\begin{note}[Two worries]
  Two worries.

  First, that even though \fc{0}, the agent would not conclude.
  Either because \(\Phi\) is unavailable, or because no potential witnessing event.
  So, can't remove \fc{0} from account of why.

  However, then \fc{0} does not support.

  If grant that \fc{0} supports, then this seems to work out.
  Further, if require existence, then things that support get very messy.
  Dopeganger cases.
  Reason is I saw A, but it wasn't A, appealing to something that doesn't exist.
  Various other cases like this.

  Difference.
  In these cases, have premise, thing is that the truth value is distinct.
  Here, possibly no premise.

  Well, this is different.
  However, I don't think this is sufficient to reject the idea.
  Just because this distinction doesn't arise in the case of witnessing doesn't really do much.

  Look, a `bad' premise offers no more support for the agent than no premise.

  Second, need \emph{that} \fc{0}.
  However, the point is that this is about the agent's present epistemic state.
  \emph{Without} \fc{0}, the agent would reason.
  This is just the key point reiterated.
  Know whether, \fc{0} just adds information about which.
\end{note}


\paragraph{Foregone-concluding}

\begin{note}[Foregone-concluding]
  Pair this with a key idea.

  \begin{restatable}[Foregone-concluding]{idea}{ideaForegoneCing}
    \label{idea:reassignment}
    If foregone-conclusion, then may conclude.
    %\vspace{-\baselineskip}
  \end{restatable}

  Cases where concluding by witnessing reduces to witnessing forgone conclusion.
  \emph{Concluding \(\pv{\psi}{v'}\) from \(\Psi\) is just witnessing foregone-conclusion.}
  So, reduction, in certain cases.
  Further, if forgone conclusion, then conclude.
  At least, in certain cases.
\end{note}



\begin{note}[???]
  Only argue for a positive resolution to~\autoref{issue:Main} given~\autoref{idea:reassignment}.

  And, leave~\autoref{idea:reassignment} as an idea.
  Insight into adopting this idea, or something like this.
\end{note}

\begin{note}
  Positive resolution may read easier if something like `in committing'.
  Commit to location from map, sum from arithmetic.
  Indeed, perhaps intuitive sense is just commitment via witnessing reasoning.

  But, we then have a reduction.
  Question is, what work does commitment do, and what work does witnessing do?

  Still, why think this?
  Why not think that concluding leads to commitments.
  Independent consequence.
  In same way that knowledge as basic entails justification, same for commitments.
  Indeed, in same way that relevant justification may be distinctive in the case of knowledge, same for commitment with concluding.
\end{note}

\begin{note}
  Assume motivate.
  What exactly is concluding?
  This will be beyond the scope of this document.
  I hope motivating a difference in extension motivates further questions about what the relation is, and the importance of witnessing in certain cases.
  As we will see, making this argument is by no means straightforward.

  I think there needs to be some instance of witnessing --- concluding does not arise from nowhere.
  Still, if witnessing leads to additional conclusions, then what do we get from concluding?
  What we will get is a general closure condition.

  To do so we narrow things down a little.
  Focus on particular types of concluding, and when concluding is accompanied by an additional property.
  Still, if in these instances no witnessing, then not more generally.
  Additional property, concluding some proposition from some premises.

  Argument, won't directly rely on intuitions about whether agent has concluded.
\end{note}



\section{\zSN{2}}
\label{sec:overview:zS}

\begin{note}
  \begin{quote}
    \issueMain*
  \end{quote}
  In~\autoref{chapter:concluding} we expanded on our understanding of concluding.
  {
    \color{red} summary?
  }

  Our goal is to establish tension, and thereby motivate a positive resolution to~\autoref{issue:Main}.
  In order to establish tension we narrow our attention to when concluding \(\pv{\phi}{v}\) concluding \(\pv{\phi}{v}\) involves the agent establishing a particular property with respect to \(\pv{\phi}{v}\).
  We term the property `\zSN{0}', or `\zS{}' for short.

  Positive resolution only requires existence of cases.
  Hence, existence of cases with this property.
  This will be sufficient.
  Any case of concluding which involves \csVImp{} will also be an instance of concluding.

  As sketched, tension by {\color{red} \dots}.

  For the moment, however, we focus on providing a clear account of \csN{}.
  Tension delayed until \dots
  Indeed, following \csN{}, revise resolutions to ~\autoref{issue:Main}.
  And, additional building blocks for tension via two types of concluding.
\end{note}

\begin{note}
  Our choice of the term `\zgb{0}' is metaphorical.
  \zgb{2} is a family of flower plants which, typically, have the appearance of a single stem with no branches.
  If one starts just before the flower and works back down the stem, one will not find a branch which, if taken, would lead to a different flower.
  In comparison, if one starts with an agent's epistemic state prior to the agent concluding \(\pv{\phi}{v}\) from \(\Phi\) and~\autoref{question:zs} has a negative answer with respect to \(\pvp{\phi}{v}{\Phi}\), then one will not find a branch which leads to a different conclusion.

  I have some doubts as to whether or not this metaphor really works, but some term is required.
  `Palm-tree-support', or `Arecaceae-support' would also work.
\end{note}

\subsection{The question}
\label{sec:ZS:question}

\begin{note}
  This section is considers the following question:

  \begin{restatable}[\zSN{2}]{question}{questionZS}
    \label{question:zs}
    For an agent \vAgent{}, when concluding \(\pv{\phi}{v}\) from \(\Phi\):

    \begin{enumerate}[label=\arabic*., ref=(\arabic*)]
    \item
      For any proposition-value-premise pairing \(\pvp{\psi}{v'}{\Psi}\) such that:
    \begin{enumerate}[label=\alph*., ref=(\alph*)]
    \item
      From \vAgent{}'s perspective:
      \begin{enumerate}[label=\roman*., ref=(\roman*)]
      \item
        \label{question:zs:subjunctive}
        \vAgent{} \emph{would not} conclude \(\pv{\phi}{v}\) from \(\Phi\) if \vAgent{} were to either:
        \begin{itemize}
        \item Conclude \(\pv{\psi}{\overline{v'}}\) from \(\Psi\).
        \item Fail to conclude either \(\pv{\psi}{v'}\) or \(\pv{\psi}{\overline{v'}}\) from \(\Psi\).
        \end{itemize}
        \footnote{
          Clause~\ref{question:zs:subjunctive} is expressed by a subjunctive conditional because we have no guarantee that an agent will attempt to conclude \(\pv{\psi}{v'}\) from \(\Psi\).

          \color{red}
          As this alternative expression makes clear,~\autoref{question:zs} focuses on the agent (and their epistemic state).
          At no point do we consider any variation of the agent's epistemic state.
          Likewise,~\autoref{question:zs} concerns only the agent's perspective on concluding \(\pv{\psi}{v'}\) from \(\Psi\).
          Whether or not the agent would conclude \(\pv{\psi}{v'}\) from \(\Psi\) is irrelevant.
          What matters is whether, from the agent's perspective, there is potential for reasoning about whether \(\pv{\psi}{v'}\) follows from \(\Psi\) to block concluding \(\pv{\phi}{v}\) from \(\Phi\).
        }
      \item
        \label{question:zs:option}
        \(\phi\) having value \(v\) ensures \vAgent{} has the option of concluding \(\pv{\psi}{v'}\) from \(\Psi\), given their present epistemic state.
      \end{enumerate}
    % \item
    %   \label{question:zs:not-concluded}
    %   \vAgent{} has not concluded \(\pv{\psi}{v'}\) from \(\Psi\)
    \end{enumerate}
      \label{question:zs:may-fail}
      Is it the case that, from \vAgent{}' perspective, \vAgent{} may fail to conclude \(\pv{\psi}{v'}\) from \(\Psi\)?
    \end{enumerate}
    \vspace{-\baselineskip}
  \end{restatable}

  Speaking of~\autoref{question:zs} as a question is somewhat inexact.
  Strictly,~\autoref{question:zs} requires an agent, an understanding of the agent's epistemic state, some pool of premises \(\Phi\), and a proposition-value pair \(\pv{\phi}{v}\) such that the agent has not (yet) concluded \(\pv{\phi}{v}\) from \(\Phi\) and some candidate proposition-value-premise pairing \(\pvp{\psi}{v'}{\Psi}\) which meets a number of conditions.
  When the above conditions have been met, \emph{then}~\autoref{question:zs} raises a question.
  Specifically, whether, from the agent's perspective, the agent may fail to conclude \(\pv{\phi}{v}\) from \(\Phi\).

  Still, to keep our things simple, we will treat~\autoref{question:zs} as a question.%
  \footnote{
    More generally, both~\autoref{question:zs} and~\autoref{issue:Main} are interrogatives.
    However, we speak of~\autoref{question:zs} as a `question' and~\autoref{issue:Main} as an `issue' to help distinguish the two interrogatives.
    Further, we speak of answers to questions, and resolutions to issues.
    And, finally, our interest is primarily with negative answers to~\autoref{question:zs} and positive resolutions to~\autoref{issue:Main}.
  }
  In particular, we will say that:
  \begin{itemize}
  \item
    \Autoref{question:zs} has a \emph{positive answer} just in case there is some proposition-value-premise pairing \(\pvp{\psi}{v'}{\Psi}\) such that, from the agent's perspective, the agent may fail to conclude \(\pv{\psi}{v'}\) from \(\Psi\).
  \item
    \Autoref{question:zs} has a \emph{negative answer} just in case either:
    \begin{itemize}
    \item
      There is no proposition-value-premise pairing \(\pvp{\psi}{v'}{\Psi}\) for which the relevant conditions are met.
    \item
      There is some proposition-value-premise pairing \(\pvp{\psi}{v'}{\Psi}\) but, from the agent's perspective, the agent would not fail to conclude \(\pv{\psi}{v'}\) from \(\Psi\).
    \end{itemize}
  \end{itemize}
\end{note}

\begin{note}
  The primary clauses of interest are clauses~\ref{question:zs:subjunctive} and \ref{question:zs:option}.

  Intuitively, clause~\ref{question:zs:subjunctive} expresses that concluding \(\pv{\psi}{v'}\) from \(\Psi\) is a check on whether it makes sense for the agent, from their perspective, to conclude \(\pv{\phi}{v}\) from \(\Phi\).
  And,~\ref{question:zs:option} means that, so long as \(\phi\) has value \(v\), the agent has the option of checking whether it makes sense for the agent to conclude \(\pv{\phi}{v}\) from \(\Phi\).

  Note, however, clause~\ref{question:zs:option} means that the agent may only have the option of checking the relevant \(\pvp{\psi}{v'}{\Psi}\)-pairing if \(\phi\) has value \(v\).
  Hence, it need not be the case that the agent has the option of concluding \(\pv{\psi}{v'}\) from their current epistemic state (as the agent has not yet concluded that \(\phi\) has value \(v\)).
  This is subtle, but important point.

  For, not \(\psi\), then not \(\phi\).
  If \(\psi\), then there is no clear issue.

  Or, rather, if interrupt, then this would reduce to an issue about conflict between whether the agent has the option of concluding \(\pv{\psi}{v'}\) from their current epistemic state, rather than concluding \(\pv{\phi}{v}\).

  If positive answer, then two possibilities.

  On the one hand, the agent may continue their present reasoning, and conclude \(\pv{\phi}{v}\) from \(\Phi\).
  On the other hand, the agent may interrupt their reasoning to attempt to (first) conclude \(\pv{\psi}{v'}\) from \(\Psi\).
  And, if the agent fails to conclude \(\pv{\psi}{v'}\) from \(\Psi\), the agent would not conclude \(\pv{\phi}{v}\) from \(\Phi\).

  For an abstract example, suppose an agent's epistemic state consists of three formulas, \(p\), \(p \rightarrow q\), and \(r \rightarrow \lnot q\).
  This is a consistent set, if we assume the agent only reasons in accordance with sound rules of inference, the agent will never draw any contradictory conclusions from the set.
  It is only if the agent considers reasoning by some unsound rule that the agent would be presented with distinct options.
  In particular, consider the possibilities when concluding \(r\).
  From \(r\), the agent may conclude \(\lnot q\), and hence from \(p \rightarrow q\) conclude \(\lnot p\).
  Concluding \(r\), that's the problem, and if did not allow the possibility to be introduced, then would not capture this.

  This is the type of phenomenon of interest.

  \footnote{
    First, the combination of~{\color{red} ???} and~{\color{red} ???} keeps the complexity of resolving~\autoref{question:zs} relatively low.
  We only need to consider what \(\phi\) having value \(v\) would commit the agent to, so to speak.
  For example, we do not need to consider the consequences of the agent reasoning about some arbitrary proposition-value pair \(\pv{\chi}{v''}\) prior to concluding \(\pv{\phi}{v}\) from \(\Phi\).

  Of course, if the agent would conclude both \(\pv{\chi}{v''}\) and \(\pv{\chi}{\overline{v''}}\) for some \(\chi\), then it seems the agent's epistemic state is in bad shape.
  Still, given that an agent will typically revise their epistemic state upon concluding both \(\pv{\chi}{v''}\) and \(\pv{\chi}{\overline{v''}}\) for some \(\chi\), such concerns may be isolated to a distinct question.

  Second, we avoid --- to some extent --- concerns about over-generating.
  Our overall argument will put~\autoref{question:zs} to work in motivating a negative resolution to~\autoref{issue:Main}.
  Primarily by drawing consequences from what we will argue is an equivalent characterisation of negative resolutions to~\autoref{question:zs}.
  A concern is that this overall argument may over-generate.
  Given that a negative resolution to~\autoref{issue:Main} seems by no means clear, unintended consequences of our interest in~\autoref{question:zs} may diminish interest in the tension we hope to motivate, and hence tip favour to a positive answer to~\autoref{issue:Main}.
  Whether or not there are unintended consequences of broadening the scope of~\autoref{question:zs} is unclear to me.
  Still, without need to investigate, we may ignore any such consequences that may arise.
  }
\end{note}

\begin{note}
  Stated main interest with clauses~\ref{question:zs:subjunctive} and \ref{question:zs:option}.

  Three possibilities.
  \begin{enumerate}
  \item Conclude.
  \item Conclude different.
  \item Fail.
  \end{enumerate}

  Understanding of fail is that the agent does not reach either conclusion.

  However, in an intuitive sense, failure may not be clear to the agent.
  First, exhausted all resources.
  Second, things didn't go to plan.

  What matters is that if the agent reasons, then no concluding \(\pv{\phi}{v}\) without concluding \(\pv{\psi}{v'}\).
  Hence, failure until conclusion either way.
  And, in particular, no perpetual reasoning.

  Alternative expression:

  \begin{enumerate}
  \item
    \label{question:zs:subjunctive-non-trivial}
    It is not the case that: \vAgent{} would conclude \(\pv{\phi}{v}\) from \(\Phi\) if \vAgent{} were not to conclude \(\pv{\psi}{v'}\) from \(\Psi\).
  \end{enumerate}

  This expression is a little clearer, but leads to a distinct issue.
  On plausible reading, requires the agent to conclude \(\pv{\psi}{v'}\) from \(\Psi\).

  Clause~\ref{question:zs:subjunctive} is more subtle.
  Do not require the agent to reason in order to conclude.
  Instead, only expresses what would be the case if the agent took the option.

  If desired, the condition may be rephrased by the more verbose:

  \begin{enumerate}
  \item
    If \vAgent{} attempts to conclude \(\pv{\psi}{v'}\) from \(\Psi\) and fails to conclude \(\pv{\psi}{v'}\) from \(\Psi\), then (given the failure to conclude \(\pv{\psi}{v'}\) from \(\Psi\)) \vAgent{} will not conclude \(\pv{\phi}{v}\) from \(\Phi\) (without some revision to \vAgent{}' epistemic state).
  \end{enumerate}

  






  Note, however, this does not prevent the agent from concluding.
  The agent still needs to reason about whether \(\pv{\psi}{v'}\) follows from \(\Psi\).

  Perhaps weaker still, would not consider it appropriate, for there may be insufficient control.
  However, to keep things simple, conclude.m
\end{note}

{
  \color{red}
  Note, \ref{question:zs:may-fail} is delicate.
  For, the combination of \ref{question:zs:subjunctive} and \ref{question:zs:option} suggest there is a way of concluding \(\pv{\psi}{v'}\) from \(\Psi\).
  Hence, \ref{question:zs:may-fail} may be read in reference to this.
  However, \ref{question:zs:may-fail} is intended to allow other ways of concluding \(\pv{\psi}{v'}\) from \(\Psi\).
  What matters is that the agent has not concluded \(\pv{\psi}{v'}\) from \(\Psi\), the agent has the option, and the agent may fail.%
  \footnote{
    This is important for witnessing, but also motivated by different methods.
    A different way to putting this is that concluding is two place relation.
    Between premises and conclusion.
    Concluding is not a three place relation between premises, conclusion, and method.
    I should really have this stated as an assumption.
    
    Still, there is a variant where method comes into play, as I have this via ability.
  }
}

\paragraph{\illu{3}}

\begin{note}[Simple positive answers]
  My preferred example for a positive answer to~\ref{question:zs} is lost keys.
  Tempting as it may be to conclude that a pair of keys are lost after some searching, if the keys really are lost then there aren't in a handful of places you haven't yet thought to look.
  And, until you have concluded that the keys really aren't in those places, and that there is no-where else to look, the keys aren't really lost.

  Likewise, a friend's story regarding an event may be entertaining, but there is a distinction between concluding the event actually happened without checking that the details add up, and concluding the event actually happened after checking that the details add up.

  Subtle, theory, observation, possible to account for observation.
  Prior to concluding theory is sound, possible.
  And, if not, would not conclude the theory is sound.
\end{note}

\begin{note}[Success]
  Simple examples of a negative answer to~\ref{question:zs} is concluding \(\phi\) has value \(v\) from the testimony of experts as a layperson.
  For, as a layperson one has no way of querying whether \(\phi\) has value \(v\).
  Hence, \(\phi\) having value \(v\) does not introduce the possibility of reasoning about some other proposition-value pair and concluding \(\phi\) does not have value \(v\).
  Fermat's last theorem is true, I am told, and I do not have the means to query the proof.

  Other examples involve unique sources of information.
  I conclude from the position of the hands on my watch that it is midday.
  The sky is cloudy, and without a second time piece I have no hope of reaching a different conclusion.

  And, more commonplace examples involve the gradual accumulation of proposition-value pairs.
  \nagent{16} \emph{said} they're coming to the party, but you know from \nagent{17} that \nagent{16} is coming to the party only if \nagent{18} is coming to the party.
  Without further information, reasoning about whether \nagent{18} is coming to the party might prevent you from taking \nagent{16} at the word.
  However, you have already have conformation from \nagent{18} that they are coming to the party.

  Likewise, suppose there are clear skies.
  Then, it is around midday only if the sun is (roughly) at the highest point of the sky.
  Though, I already concluded that the sun is (roughly) at the highest point of the sky prior to checking my watch for a more accurate read of the time.
\end{note}

\begin{note}
  Intuitively, if an agent has the option to conclude \(\pv{\phi}{v}\) from \(\Phi\) and there is some \(\pvp{\psi}{v'}{\Psi}\) as described by~\autoref{question:zs}, then check.
\end{note}

\paragraph{More \illu{3}}

\begin{note}[Wally]
  \begin{illustration}[Where's Wally]
    \label{illu:CS:wheres-wally}
    \nagent{15} has a book containing numerous drawings of bustling scenes in which various characters are doing a variety of things.
    And, somewhere in each scene is a character called `Wally', identifiable by a collection of individually necessary and jointly sufficient distinguishing features.
    These features include a red and white striped jumper, blue trousers, short brown wavy hair, and so on.

    \nagent{15} has searched through one particular scene, and has identified a character with a variety of the features.
    Before concluding that the character is Wally, \nagent{15} remembers that there is a picture of Wally On the cover of the book, with all the identifying features present.

    Wally is always wearing a pair of round glasses, but this was not a feature \nagent{15} kept in mind when searching for Wally, and it is \epVAd{} for \nagent{15} that the character they identified is not wearing round glasses  --- \nagent{15} only recalls the features they identified.
  \end{illustration}

  Our interest is, generally, in whether \nagent{15} may conclude from the variety of features identified that the character is Wally.

  Descriptively, of course, there seems no barrier.
  An agent may reason to an arbitrary conclusion to arbitrary premises.

  So, specifically, our interest is in whether \nagent{15} would claim support that the character is Wally by concluding that the character is Wally from the variety of features identified.

  The difficulty for \nagent{15} is that so long as they consider it \epVAd{} that the character is not wearing round glasses, then there a clear check on whether \nagent{15} may reason to a different conclusion.
  For, if \nagent{15} were to check whether the character is wearing a pair of round glasses, and the character is not wearing a pair of round glasses, then \nagent{15} would conclude that the character is not Wally.
\end{note}

\begin{note}
  {
    \color{red}
    Here, revise the Wally scenario to involve reasoning about the characteristic features of Wally from memory.

    Part of the interest here, then, is that in some cases there is an `internal parallel' to doing something that doesn't involve reasoning.
  }
\end{note}

\begin{note}
  \color{red}
  Observation:
  \zS{} is about the agent's present epistemic state.
  So, we're not interested in novel information.
  However, to the extent that the same idea applies when there is the possibility of novel information, there may be some suggestion.

  The problem, in turn, then, is inductive reasoning.

  So, here, I'd like to:
  \begin{enumerate}
  \item
    Give an example which seems good, such as the library example.
  \item
    Observe that this leads to difficulties with inductive reasoning.
  \item
    And, then finally observe why \zS{} as stated does not lead to such problems.
  \item
    A different option would be to restrict to deductive reasoning.
    However, even though many examples of this, and the main examples all have this feature, I think this is too strong.
  \end{enumerate}
\end{note}

\begin{note}
  \begin{illustration}
    Computer, not turning on.
    Broken.
    Check that it's plugged in.
  \end{illustration}
\end{note}

\begin{note}
  \begin{illustration}
    For the agent it is \epVAd{} that the library uses one of a number of different indexing systems.
    E.g.\ LCC or DDC indexing.
    \begin{enumerate}
    \item A search for `H61 .R593' returned no results.
    \item The library does not have a copy of `Measurement Theory'.
    \end{enumerate}
  \end{illustration}

  Holding that a library does not have a copy of a book because a search for the book under a particular indexing system would be a mistake.
  For, if the library does not use the particular indexing system then a search using that indexing system will always fail, regardless of whether or not the library has a copy of the book.

  In turn, a failed search for an LCC index in the library's database does not seems sufficient for an agent to claim that the library does not have a copy of the book unless the agent is in a position to claim support that the library uses LCC indexing.
  Following, it seems the failed response to the `Even if\dots' test may be supplemented by noting that the library is a research library, and therefore likely uses LCC indexing, etc.\
\end{note}

\begin{note}
  Similar issue, though so long as background, then it's not clear why this would need to be explicit premise.

  Hence, broader idea of a \requ{}.
\end{note}

\begin{note}
  Of interest.
  The collection of properties might be very strong.
  It may be that there is really little chance that the agent is not Wally.
  However, of interest is that there is some reasoning that the agent may witness.
\end{note}

\begin{note}[Milk]
  \begin{illustration}
    Suppose \nagent{13} is interested in concluding that they're set to have coffee before travelling to work from the premises that they have the time and resources to make a cup of coffee.
    And, the additional premise that they have milk.
    (For, the milk will cool the coffee quickly enough for \nagent{13} to drink before leaving.)

    Still, \nagent{13} is unsure about whether the milk is safe to drink.
    The milk smells okay-ish and \nagent{13} is fairly sure that they bought it fairly recently.

    \nagent{13} holds that if the milk is past it's expiry date, then it is not safe to drink.
    But, if the milk is within it's expiry date, then as it smells okay-ish, \nagent{13} would conclude that the milk is safe to drink.
  \end{illustration}
  \epVAd{} that the milk is not safe to drink.
  It is not \epVAd{} for \nagent{13} to conclude conclusion from premises if the milk is not safe to drink.
  \nagent{13} is holding the milk, so trivial to check the expiry date.

  It seems intuitive, at least to me, that \nagent{13} should check the expiry date.

  For present purposes, this intuition is accounted for by there being an antecedent check on whether it makes sense for the agent to conclude.
  The expiry date.

  {
    \color{red}
    Here, revise to:
    Bad if the milk is over a week old.
    So, remember last time went shopping, and reason from there.

    Then, observe that this also extends more straightforwardly to checking the expiration date.
  }
\end{note}

\begin{note}
  Here, close to \citeauthor{Dretske:1970to}'s zebra case?
\end{note}

\begin{note}[Spot the difference]
  \begin{illustration}[Spot the difference]
    \label{illu:CS:spot-the-diff}
    The agent has been working through a spot-the-difference to pass some time.

    Though the time is not completely passed, the agent examined the two images with what seems sufficient care to claim support that they have found all the differences.
    However, the agent did not keep track of the number of differences.

    The agent announces `I have found all the differences' and their companion responds `All fifteen?'.

    \begin{enumerate}[label=\arabic*., ref=(I\ref{illu:CS:spot-the-diff}.\arabic*)]
      \setcounter{enumi}{-1}
    \item
      \label{illu:CS:spot-the-diff:info}
      If I have found all the differences, I have found fifteen differences.
    \end{enumerate}

    The agent then reasons as follows:

    \begin{enumerate}[label=\arabic*., ref=(I\ref{illu:CS:spot-the-diff}.\arabic*), resume]
    \item Exhaustive search.
    \item
      \label{illu:CS:spot-the-diff:all}
      I found all the differences.
    % \item\label{illu:CS:spot-the-diff:info} My companion has testified that there are fifteen differences.
    % \item\label{illu:CS:spot-the-diff:cond} If I have found all the differences, I have found fifteen differences.
    \item
      \label{illu:CS:spot-the-diff:fif}
      So, I have found fifteen differences. \hfill (From \ref{illu:CS:spot-the-diff:info} and \ref{illu:CS:spot-the-diff:all})
    \end{enumerate}
  \end{illustration}

  Before going further, structure of this.

  The agent performed some reasoning, and concluded that they found all the differences.
  However, that reasoning is mentioned but not stated in the \illu{0}.
  Rather, present is distinct instance of reasoning after being provided with information.
  ``If not 15, then problem''.
  Present reasoning appeals to past reasoning, and draws out consequence of this given new information.
  Important: the present reasoning does not consider possibility that the agent did not find all 15 differences.
  Instead, consequence of conclusion of previous instance of reasoning.
  Still, epistemically possible that the agent did not find 15 differences.
\end{note}

\begin{note}
    Providing additional information about what the agent has claimed support for.
  Recall, \autoref{assu:CSVP}, information rather than \world{}.
  \nolinebreak
  \footnote{
    Still slight issue.
    Offering a redescription.
    You met Clark Kent, so you met Superman.
    In this case, rather than claiming support for meeting Superman, provided information is seen as an equivalent formulation.
    It is possible to read \autoref{illu:CS:spot-the-diff} in this way, and this might be the most natural interpretation.
    However, it is not the interpretation under which see the problem.
    Rather, problem is where the conditional is explicit.
    Unlike Superman case, proper conditional.
  }
\end{note}

\begin{note}
  Information leads to \requ{}.

  Possibility of not fifteen.
  And, not merely that the agent performed the reasoning, but that the reasoning identified all.
  If not fifteen, then not all, so would involve appeal to something that is not the case.

  And, present reasoning does not include reasoning about \requ{}.
\end{note}

\begin{note}
  \color{red}
  Though, this is interesting.
  For, the agent may have found fifteen.
  This, then, helps stress the point that it's not just reasoning to the conclusion.
\end{note}


\paragraph{Visualisation of what is at issue when asking~\autoref{question:zs}}

\begin{figure}[h]
  \centering
  \begin{tikzpicture}
    \node (origin) at (0,0) {};
    \node (psiSplit) at (1,0) {};
    \node (phiSplit) at (4,0) {};
    %
    \node[anchor=west] (psiV) at  (6,-1)  {\(\pvp{\psi}{v'}{\Psi}\)};
    \node[anchor=west] (psiNv) at (6,-2) {\(\pvp{\psi}{\overline{v'}}{\Psi}\)};
    \node[anchor=west] (psiQ) at (6,-3) {\(\pvp{\psi}{?}{\Psi}\)};
    %
    % \node[anchor=west] (psiVPhiV) at (9,-1) {\(\pv{\phi}{v}\)};
    \node[anchor=west] (psiNvPhiU) at (9,-2) {\(\pv{\phi}{\{\overline{v},?\}}\)};
    \node[anchor=west] (psiQPhiU) at (9,-3) {\(\pv{\phi}{\{\overline{v},?\}}\)};
    %
    \node[anchor=west] (phiQ) at (10,1) {\(\pv{\phi}{?}\)};
    \node[anchor=west] (phiNv) at (10,2) {\(\pv{\phi}{\overline{v}}\)};;
    \node[anchor=west] (phiV) at (10,0) {\(\pv{\phi}{v}\)};
    %
    \draw[-]  (origin) -- (phiV);
    %
    \path[-,dashed] (phiSplit) edge [out=0, in=180] (phiNv);
    \path[-,dashed] (phiSplit) edge [out=0, in=180] (phiQ);
    %
    \path[-.] (psiSplit) edge [out=0, in=180] (psiV);
    \path[-, dashed] (psiSplit) edge [out=0, in=180] (psiNv);
    \path[-, dashed] (psiSplit) edge [out=0, in=180] (psiQ);
    %
    \draw[<-,dotted] (psiV) edge [out=0, in=180] (phiV);
    \draw[->, dotted] (psiNv) edge (psiNvPhiU);
    \draw[->, dotted] (psiQ) edge (psiQPhiU);
    \end{tikzpicture}
    \caption{Sketch of when an agent has a negative resolution for~\autoref{question:zs}.}
    \label{fig:csN:illu:overview}
  \end{figure}

\begin{note}[Figure]
  \autoref{fig:csN:illu:overview} provides a rough visualisation of~\autoref{question:zs}

  The flat line captures the agent's reasoning, which concludes with \(\pv{\phi}{v}\).
  In concluding \(\pv{\phi}{v}\) the agent rules out two possibilities with respect to \(\phi\).
  First, that \(\phi\) does not have value \(v\), indicated by \(\pv{\phi}{\overline{v}}\).
  Second, that the agent does not assign any value to \(v\), indicated by \(\pv{\phi}{?}\).
  Prior to concluding \(\pv{\phi}{v}\), the agent's reasoning may have branched to either alternative path, but as the agent has concluded \(\pv{\phi}{v}\), neither path is viable, and hence both paths are represented with a dashed line.

  So far, we have seen only that the agent has concluded \(\pv{\phi}{v}\).

  We now consider some proposition-value-premise pairing \(\pv{\psi}{v'}{\Psi}\) such that if the agent were to fail to conclude \(\pv{\psi}{v'}\) from \(\Phi\), the agent would not conclude \(\pv{\phi}{v}\) from \(\Phi\).

  Intuitively, the dotted arrows from the various combinations of \(\psi\) and \(\{v',\overline{v'},?\}\) read, from top to bottom:
  \begin{itemize}
  \item If \(\phi\) has value \(v\) then the agent may conclude \(\pv{\psi}{v'}\) from \(\Psi\), and:
  \item If the agent concludes \(\psi\) has some value \(\overline{v'}\) from \(\Psi\), then the agent either concludes \(\phi\) has some value other than \(v\), or the agent fails to reach a conclusion regarding \(\phi\) from \(\Phi\).
    Both options are combined via the shorthand \(\pv{\phi}{\{\overline{v},?\}}\).
  \item
    And, likewise if the agent fails to conclude \(\pv{\psi}{v'}\) from \(\Psi\).
  \end{itemize}

  With respect to \csN{}, observe that prior to ruling out alternative branches with respect to \(\phi\), the agent may have reasoned about whether \(\psi\) has value \(v\).
  And, from the agent's perspective, \(\phi\) has value \(v\) only if \(\psi\) has value \(v'\).
  If \(\psi\) does not have value \(v'\), then either \(\phi\) does not have value \(v\), or the agent's reasoning would not conclude with a value for \(\phi\), indicated by \(\pv{\phi}{\{\overline{v},?\}}\).

  Hence, prior to concluding \(\pv{\phi}{v}\), the agent has concluded \(\pv{\psi}{v'}\).
\end{note}

\begin{note}
  Broadly, then, we may say that an agent has \csVed{} for \(\pv{\phi}{v}\) just in case when concluding \(\pv{\phi}{v}\) it is not the case that the agent's reasoning could have branched to a different conclusion with respect to \(\phi\).

  However, the visualisation of~\autoref{fig:csN:illu:overview} and this broad statement of \csN{} are a little too broad.
  For, we are only interested in proposition-value pairs guaranteed by \(\phi\) having value \(v\).
  \csN{} is not global with respect to all proposition-value pairs that the agent may have reasoned about, but local to those guaranteed by the proposition.
\end{note}

\paragraph*{\zSN{2}}

\begin{note}
  We opened this section with a question~---~\autoref{question:zs}.
  We have expanded, in part, on what this question amounts to, and have seen a handful of \illu{1}.

  To ease discussion going forward, we reformulate negative and positive answers to~\autoref{question:zs} in terms of whether an agent satisfies a certain property.

  In longform we term this property `\zSN{2}', though with few exceptions we will use the term `\zSN{0}'.

  We also take this opportunity to refine our understanding of the relevant \(\pvp{\psi}{v'}{\Psi}\) proposition-value-premise pairings from~\autoref{question:zs}.
\end{note}

\paragraph*{\requ{3}}

\begin{note}
  We begin by refining the relevant \(\pvp{\psi}{v'}{\Psi}\) proposition-value-premise pairings of interest from~\autoref{question:zs}.
  We term such proposition-value-premise pairings `\requ{1}' of concluding \(\pv{\phi}{v}\) from \(\Phi\).
\end{note}

\begin{note}[Notion of a \requ{}]
  \begin{notion}[\requ{3}]
    \label{notion:overview:requ}
    \(\pvp{\phi}{v'}{\Psi}\) is a \requ{} of concluding \(\pv{\phi}{v}\) from \(\Phi\), with respect to an agent \vAgent{}'s epistemic state if:
    \begin{enumerate}
    \item
      \label{notion:overview:requ:main}
      From the perspective of \vAgent{}'s epistemic state, \(\phi\) has value \(v\) only if:
      \begin{enumerate}[label=\alph*., ref=\named{R:\alph*}]
      \item
        \label{notion:overview:requ:pool}
        \vAgent{} has the option of concluding \(\pv{\psi}{v'}\) from \(\Psi\) where:
        \begin{enumerate}[label=\roman*., ref=\named{R:a.\roman*}, series=csIdeaCounter]
        \item
          \label{notion:overview:requ:pool:int}
          \vAgent{} may conclude \(\pv{\psi}{v'}\) from \(\Psi\) without concluding \(\pv{\phi}{v}\) from \(\Phi\) as an intermediary step.
        \item
          \label{notion:overview:requ:pool:ind}
          For any proposition-value pair \(\pv{\psi_{i}}{v_{i}}\) in \(\Psi\), \vAgent{} either has concluded or may conclude \(\pv{\psi_{i}}{v_{i}}\) without concluding \(\pv{\phi}{v}\) from \(\Phi\).
        \end{enumerate}
      \item
        \label{notion:overview:requ:nPsi-nPhi}
        If \vAgent{} were to fail to conclude \(\pv{\psi}{v'}\) from \(\Psi\) prior to reasoning about whether \(\phi\) has value \(v\) given \(\Phi\), \vAgent{} would not conclude \(\pv{\phi}{v}\) from \(\Phi\).
      \end{enumerate}
    \end{enumerate}
    \vspace{-\baselineskip}
  \end{notion}

  With the key clause linking~\autoref{notion:overview:requ} to \qzS{} is clause~\ref{notion:overview:requ:nPsi-nPhi}.
  Indeed, clause~\ref{notion:overview:requ:nPsi-nPhi} captures the core idea of failure to conclude \(\pv{\psi}{v'}\) from \(\Psi\) leading to failure to conclude \(\pv{\phi}{v}\) from \(\Phi\).

  The role of clause~\ref{notion:overview:requ:pool} is explicitly state various properties \(\pv{\psi}{v'}{\Psi}\) must have in order for any failure to conclude \(\pv{\psi}{v'}\) from \(\Psi\) is relevant to concluding \(\pv{\phi}{v}\) from \(\Phi\).%
  \footnote{
    Indeed, we take \ref{notion:overview:requ:pool:int} and~\ref{notion:overview:requ:pool:ind} to be more-or-less implicit constraints on \(\pvp{\psi}{v'}{\Psi}\) in the statement of \qzS{}.
  }
  In particular, \ref{notion:overview:requ:pool:int} and~\ref{notion:overview:requ:pool:ind} are required to ensure the agent may conclude \(\pv{\psi}{v'}\) from \(\Psi\) independently of concluding \(\pv{\phi}{v}\) from \(\Phi\).

    For, if \ref{notion:overview:requ:pool:int} and \ref{notion:overview:requ:pool:ind} were to fail to hold then:
  \begin{itemize}
  \item
    By~\ref{notion:overview:requ:pool:int}, the agent would need to conclude \(\pv{\phi}{v}\) from \(\Phi\) as a sub-conclusion when reasoning from the relevant pool of premises \(\Psi\).
    Hence, it would not be possible to conclude \(\pv{\psi}{v'}\) from \(\Psi\) without first concluding \(\pv{\phi}{v}\) from \(\Phi\).
  \item
    And, likewise, by~\ref{notion:overview:requ:pool:ind}, the agent need to have already concluded \(\pv{\phi}{v}\) from \(\Phi\) in order to appeal to some of the proposition-value pairs in the relevant pool of premises \(\Psi\).
  \end{itemize}

  Conversely, if both~\ref{notion:overview:requ:pool:int} and~\ref{notion:overview:requ:pool:ind} hold, the agent may conclude \(\pv{\psi}{v'}\) from \(\Psi\) independently of concluding \(\pv{\phi}{v}\) from \(\Phi\).

  Note, however, neither~\ref{notion:overview:requ:pool:int} nor~\ref{notion:overview:requ:pool:ind} rule out the possibility of the agent concluding \(\pv{\phi}{v}\) from \(\Phi\) when concluding \(\pv{\psi}{v'}\) from \(\Psi\) or, conversely, concluding \(\pv{\psi}{v'}\) from \(\Psi\) when concluding \(\pv{\phi}{v'}\) from \(\Phi\).
  There may be an interesting variant of the notion of a \requ{} with such a constraint in place, but such a constraint is not of interest with respect to \qzS{}.
  For, at issue is only whether the agent may at interest is only failure to conclude \(\pv{\psi}{v'}\) from \(\Psi\), and both~\ref{notion:overview:requ:pool:int} and~\ref{notion:overview:requ:pool:ind} ensure lack of concluding \(\pv{\phi}{v}\) from \(\Phi\) will not prevent the agent from reaching a conclusion regarding whether \(\psi\) has value \(v\) given \(\Psi\).
\end{note}

\begin{note}
  \color{red}
  Has the option.
\end{note}

\begin{note}[\requ{2}: Partial check]
  Intuitively, concluding \(\pv{\psi}{v'}\) from \(\Psi\) would serve as a partial check on whether the agent may reason to a conclusion other than \(\pv{\phi}{v}\), captured by~\ref{notion:overview:requ:nPsi-nPhi}.

  Concluding \(\pv{\psi}{v'}\) from \(\Psi\) is a check.
  For, if the agent were to fail to conclude \(\pv{\psi}{v'}\) from \(\Psi\) then, from the perspective of the agent's epistemic state, the agent would not conclude \(\pv{\phi}{v}\) from \(\Phi\).
  Hence, contraposing, the agent would conclude \(\pv{\phi}{v}\) from \(\Phi\) only if the agent would conclude \(\pv{\psi}{v'}\) from \(\Psi\).
  However, the check is partial, as it need not be the case that the agent would conclude \(\pv{\psi}{v'}\) from \(\Psi\) only if the agent \(\pv{\phi}{v}\) from \(\Phi\).
  Therefore, failing to conclude \(\pv{\psi}{v'}\) from \(\Psi\) may block concluding \(\pv{\phi}{v}\) (from the perspective of the agent's epistemic state) though concluding \(\pv{\psi}{v'}\) from \(\Psi\) need not ensure that the agent would conclude \(\pv{\phi}{v}\).

  Now, \ref{notion:overview:requ:nPsi-nPhi} contains a slight subtlety.
  For, from~\autoref{assu:conc:d-free}, an agent may conclude various proposition-value pairs from some instance of reasoning without explicit recognition.
  Therefore,~\ref{notion:overview:requ:nPsi-nPhi} does not state that the agent may fail to conclude \(\pv{\phi}{v}\) from \(\Phi\).
  Rather, \ref{notion:overview:requ:nPsi-nPhi} holds that from the perspective of the agent's epistemic state, the agent may fail to conclude \(\pv{\phi}{v}\) from \(\Phi\).
  Again, we tread a fine line between the role of an agent's epistemic state and the role of the agent's \stance{}.
  The role of an agent's epistemic state determines whether \(\pv{\psi}{v'}\) is a \requ{} of concluding \(\pv{\phi}{v}\) from \(\Phi\).
  And, an agent's epistemic state may determine whether an agent concludes \(\pv{\chi}{v''}\) when concluding \(\pv{\phi}{v}\).%
  \footnote{
    Recall the above discussion of \(\pv{\phi}{v}\) \indicatePr{} \(\pv{\chi}{v'}\) in relation to~\autoref{assu:conc:d-free}.
  }
  Therefore, the agent's epistemic state --- the agent's perspective on how things are --- is key.
  However, the agent's \stance{} is unimportant.
  Whether an agent has concluded \(\pv{\phi}{v}\), or whether \(\pv{\psi}{v'}\) is a \requ{} is not a question of whether the agent recognises the have concluded \(\pv{\phi}{v}\) or recognises \(\pv{\psi}{v'}\) is a \requ{}.

  Combining these two ideas, intuitively, \(\pv{\psi}{v'}\) is a \requ{} of concluding \(\pv{\phi}{v}\) just in case there is some pool of premises \(\Psi\) such that determining whether the agent would conclude \(\pv{\psi}{v'}\) is an independent partial check on whether the agent may reason to a conclusion other than \(\pv{\phi}{v}\).
\end{note}

\begin{note}
  With the notion of a \requ{} in hand, we now state when an agent has \zSN{0} for some proposition-value pair:

  % \begin{restatable}[\zSN{2} --- \zS{}]{idea}{ideazs}
  %   \label{idea:zs}
  %   An agent \vAgent{} has \emph{\zSAb{-}} for a proposition-value pair \(\pv{\phi}{v}\) with respect to some pool of premises \(\Phi\) when concluding \(\pv{\phi}{v}\) from \(\Phi\) just in case:
  %   \begin{itemize}
  %   \item
  %     When concluding \(\pv{\phi}{v}\) from \(\Phi\):
  %     \begin{enumerate}[label=\arabic*., ref=\named{\zSAb{}:\arabic*}]
  %     \item
  %       For any proposition-value-premise pairing \(\pvp{\psi}{v'}{\Psi}\) which is a \requ{} of concluding \(\pv{\phi}{v}\) from \(\Phi\):
  %       \begin{enumerate}[label=\alph*., ref=\named{\zSAb{-}:1.\alph*}]
  %       \item
  %         \vAgent{} has `settled' that they would conclude \(\pv{\psi}{v'}\) from \(\Psi\).
  %       \item
  %         \vAgent{} simultaneously `settles' that they would conclude \(\pv{\psi}{v'}\) from \(\Psi\) when concluding \(\pv{\phi}{v}\) from \(\Phi\).
  %       \end{enumerate}
  %     \end{enumerate}
  %   \end{itemize}
  %   \vspace{-\baselineskip}
  % \end{restatable}
    \begin{idea}[\iZS{-} --- \iZS{}]
      \label{idea:Zs:overview}
      \label{idea:zs}
    An agent \vAgent{} has \emph{\ZS{-}} for a proposition-value pair \(\pv{\phi}{v}\) when concluding \(\pv{\phi}{v}\) from some pool of premises \(\Phi\) just in case:
    \begin{itemize}
    \item When concluding \(\pv{\phi}{v}\) from \(\Phi\):
    \begin{enumerate}[label=\arabic*., ref=\named{CS:\arabic*}]
    \item
      \label{idea:Zs:overview:requ}
      For any proposition-value-premise pairing \(\pvp{\psi}{v'}{\Psi}\) which is a \requ{} of concluding \(\pv{\phi}{v}\) from \(\Phi\) either:
      \begin{enumerate}[label=\alph*., ref=\named{CS:1.\alph*}]
      \item
        \label{idea:Zs:overview:requ-sat:Past}
        \vAgent{} has concluded \(\pv{\psi}{v'}\) from \(\Psi\).
      \item
        \label{idea:Zs:overview:requ-sat:Pres}
        In concluding \(\pv{\phi}{v}\), \vAgent{} simultaneously concludes \(\pv{\psi}{v'}\) from \(\Psi\).
      \item
        \label{idea:Zs:overview:requ-sat:Forgone}
        \(\pvp{\psi}{v'}{\Psi}\) is a \fc{0}.
      \end{enumerate}
    \end{enumerate}
  \end{itemize}
  \vspace{-\baselineskip}
  \end{idea}
\end{note}

\begin{note}
    \begin{proposition}[Equivalence between \qzS{} and \izS{}]
    \label{prop:qzs-tick-equals-iCS}
    For any agent, the following are equivalent:
    \begin{enumerate}[label=\arabic*., ref=(\arabic*)]
    \item
      \label{prop:qzs-tick-equals-iCS:qzS}
      \qzS{} has a negative resolution when concluding \(\pv{\phi}{v}\) from \(\Phi\).
    \item
      \label{prop:qzs-tick-equals-iCS:ZS}
      The agent has \ZS{} for \(\pv{\phi}{v}\) when concluding \(\pv{\phi}{v}\) from \(\Phi\).
    \end{enumerate}
    \vspace{-\baselineskip}
  \end{proposition}
  In other words, we hold that an agent ruling out failure to conclude \(\pv{\phi}{v}\) from \(\Phi\) for any \requ{} \(\pvp{\psi}{v'}{\Psi}\) is \emph{equivalent}%
  \footnote{
    In the context of a negative resolution to \qzS{}.
  }
  to the agent either
  \begin{enumerate*}[label=(\alph*)]
  \item
    having had concluded \(\pv{\psi}{v'}\) from \(\Psi\), or
  \item
    the agent simultaneously concluding \(\pv{\psi}{v'}\) from \(\Psi\) when concluding \(\pv{\phi}{v}\) from \(\Phi\).
  \item
    \(\pvp{\psi}{v'}{\Psi}\) being a \fc{0}.
  \end{enumerate*}
\end{note}


\paragraph*{Observations}

\paragraph*{Contraposition}

\begin{note}[Contraposition]
  An argument for~\autoref{prop:qzs-tick-equals-iCS} is important from the perspective of the overall argument of this document.

  Suppose we have~\autoref{prop:qzs-tick-equals-iCS}.
  Then, view \iZS{} as a clarification of \qzS{}.

  Specifically, the left-to-right direction.

  Only negative resolution if \iZS{}.
  I.e.\ only if concluded, for any \requ{}.

  \begin{itemize}
  \item
    If an agent has concluded \(\pv{\phi}{v}\) from \(\Phi\) \emph{without} having concluded \(\pv{\psi}{v'}\) from some \(\Psi\), where \(\pv{\psi}{v'}\) is a \requ{} of concluding \(\pv{\phi}{v}\) from \(\Phi\), then the agent has not \csVed{} for \(\pv{\phi}{v}\) from \(\Phi\).
  \end{itemize}

  Taking the contraposition, if not \iZS{}, then no negative resolution.
  If intuitions are unclear, then \iZS{} offers a way to clarify those intuitions.

  Conversely, fix a negative resolution to \qzS{}, then from left-to-right direction, draw out what must also be the case.%
  \footnote{
    Consider, by analogy, knowledge, and the idea that knowledge is closed under known entailment.
    \begin{quote}
      If \vAgent{} knows both
      \begin{enumerate*}[label=(\roman*)]
      \item \(\phi\), and
      \item \(\phi\) entails \(\psi\),
      \end{enumerate*}
      then \vAgent{} knows \(\psi\).
    \end{quote}
    Observe the same dynamics are present.

    If whether an agent knows both \(\phi\) and \(\phi\) entails \(\psi\) is at issue, then establishing the agent does not know \(\psi\) establishes that either the agent does not know \(\phi\) or the agent does not know \(\psi\).

    Conversely, if an agent knows both \(\phi\) and \(\phi\) entails \(\psi\), then by closure of knowledge under known entailment, the agent must also know \(\psi\).

    Of course, whether knowledge \emph{is} closed under known entailment is unclear, but the same dynamics hold for any similar condition.
    In general, these observations amount to little more than both the closure of knowledge under known entailment and \iZS{} both having the general form of a conditional \(A \Rightarrow B\), such that:
    \begin{itemize}
    \item
      From \(A \Rightarrow B\) and \(A\), one may infer \(B\), and
    \item
      From \(A \Rightarrow B\) and \emph{not}-\(B\), one may infer \emph{not}-\(A\).
    \end{itemize}
  }

  \begin{itemize}
  \item
    If an agent has \ZS{} for \(\pv{\phi}{v}\) from \(\Phi\), by concluding \(\pv{\phi}{v}\) from \(\Phi\) then, has concluded \(\pv{\psi}{v'}\) from \(\Psi\), for any \(\pv{\psi}{v'}{\Psi}\) which is a \requ{} of concluding \(\pv{\phi}{v}\) from \(\Phi\).
  \end{itemize}

  In outline, path to tension.
  Cases in which \qzS{} has negative resolution.
  Draw out as a consequence of~\autoref{prop:qzs-tick-equals-iCS} that the agent has concluded.
  So long as cases in which no witnessing, we will have tension.
  On the one hand, negative resolution, and on the other hand, has not witnessed reasoning.

  Now, if some weaker, that does not require concluding, then lack a way to generate tension.

  So,~\autoref{prop:qzs-tick-equals-iCS} should be treated with some caution.

  Of course, this does not guarantee anything interesting.
  Tension will still depend on such cases.
\end{note}

\begin{note}
  With the importance of~\autoref{prop:qzs-tick-equals-iCS} motivated, we now turn to arguing for~\autoref{prop:qzs-tick-equals-iCS}.

  The argument we provide for~\autoref{prop:qzs-tick-equals-iCS} is somewhat involved, and will go via an intermediary lemma.
  We being by outlining the structure of the argument, before turning to the details.
\end{note}

\paragraph*{Same time}

\begin{note}[Importance of at same time]
  Propositional logic.
  These premises allow to conclude two things.
  Then, conclusion that \(\phi \land \psi\) is simultaneously a conclusion that \(\phi\) and that \(\psi\).

  Or, apples in a bag.
  Five.
  Well, could do at least four, three, etc.
  Conclude at the same time.
\end{note}

\begin{note}
  For example, counterexample for some formula of propositional logic.
  Constructed a truth table.
  Identified a line.
  If counterexample, then line makes any tautology of propositional logic true.
  And, do not need to appeal to the line being a counterexample to the relevant formula to do so.
  reason from line to any recognised tautology.
  Conclude, tautology would be true.
\end{note}

\paragraph*{Inductive, abductive, etc.\ reasoning}

\begin{note}
  Narrow, but not too narrow.
\end{note}

\begin{note}
  This doesn't rule out inductive or abductive reasoning.
  Consider standard induction.
  Here, there may be novel information, but this is not available from the agent's present epistemic state, and \qzS{} only concerns the agent's present epistemic state.
  Perhaps the possibility alone would prevent conclusion.
  However, it seems most conclude in recognition of such possibility.
  Instead, what one would need is considerations against uniformity principle.

  Same for any bridge between probabilistic and full.
  Toss a coin \(n\) times, conclude it is fair.
  Possible to toss \(m\) more times, not fair.
  However, \(n\) is sufficient, then no problem.
  It is true that there is something more you could do, but this would require acquiring new information.
\end{note}

\paragraph*{Fragility}

\begin{note}
  Kind of fragility.
  If concludes \(\pv{\phi}{v}\) from \(\Phi\), and does not have \zS{}, then from agent's perspective, possibility of revision.

  Indeed, may break down into two components.

  First, possibility of different conclusion.
  Agent's epistemic state is potentially unstable.

  Second, isolation of potential instability to \(\pv{\phi}{v}\) from \(\Phi\).
\end{note}

\paragraph*{Whether the agent may conclude \(\phi\) has value \(v\), regardless of \(\Phi\)}

\begin{note}
  Not about the proposition-value pair.
  Rather, it is about the concluding.
  At interest is not whether \(\phi\) has value \(v\), but whether it makes sense to conclude \(\pv{\phi}{v}\) from \(\Phi\).
  Of course, if the agent has no information about whether \(\phi\) has value \(v\), then this is also part of the picture, but that is a consequence of the base concern.
\end{note}

\paragraph*{Normative?}

\begin{note}[Just a property]
  There's no kind of normative evaluation here.
  We do not hold any conclusion for which this fails is bad.
  Nor do we hold that any conclusion for which this holds is good.
  Indeed, \zS{} is narrow, far too narrow for a general evaluation.

  Indeed, whether or not \zS{} doesn't tell us anything about the relationship between \(\pv{\phi}{v}\) and \(\Phi\) in general, as relative to an agent's epistemic state.
  May be that there is some \(\pvp{\psi}{v'}{\Psi}\), but only due to some quirk of the agent.
\end{note}

\paragraph*{The agent concluding \(\pv{\psi}{v'}\) from \(\Psi\)}

\begin{note}
  From the perspective of the agent.
  It doesn't matter whether the agent really has the option.
  Indeed, this perspective is important for fragility.
\end{note}

\subparagraph*{Almost-transitive}

\begin{note}
  Key idea that may be obvious is that \csN{} is almost-transitive.
  If it needs to be the case that I'd conclude \(\pv{\psi}{v'}\) from \(\Psi\) to get \(\pv{\phi}{v}\) from \(\Phi\), and \(\pv{\chi}{v''}\) from \(X\) to get \(\pv{\psi}{v'}\) from \(\Psi\) then, need to get \(\pv{\chi}{v''}\) from \(X\) to get \(\Phi\), and \(\pv{\chi}{v''}\).

  However, this is not quite right.
  For, it may be the case that the agent has the option of getting to \(\pv{\psi}{v'}\) from \(\Psi\) from the different branch.
\end{note}

\begin{note}
  Only about the option of concluding.
  There are various other properties.
  In this respect, \qzS{} is narrow.
\end{note}

\begin{note}
  Most reasoning is short, and composes.
  Further, it needs to be the case that novel proposition-value pair introduces this concern.
  So, some extraneous proposition.
  This is irrelevant.
\end{note}

\begin{note}[Recursion]
   {
    \color{red}
    Here, it is important that we don't go fully recursive.
    For, we're only interested in \requ{1} arising from concluding \(\pv{\phi}{v}\) from \(\Phi\).
    This means that it is no immediate.
    In particular, it may be the case that some of \(\Phi\) remove what would otherwise be a \requ{} of some \requ{}.
    But, then, this is still a \requ{} with respect to the current epistemic state.

    So, I think I actually get the result that this is recursive.
    However, with a slight change.
    For, \ref{idea:Zs:overview:requ-sat:Past} looks to the past, and instead of \csVImp{} in the past, it only need to be the case that the agent has \csVed{} from present epistemic state.

    The point is, that there may be some pruning without concluding.
    For, something fails to be a \requ{}.
    Likewise, it may be the case that the tree grows, as the agent's epistemic state develops.

    So, we don't get a clear recursive clause.

    This leads to an observation.
    Reasoning about a \requ{} may lead to a revision of the agent's epistemic state.

    This is a somewhat interesting consequence.
    We began with the idea of failing to conclude \(\pv{\phi}{v}\) from \(\Phi\).
    This said nothing about revision.
    However, we see that this raises the possibility of revision.

    From a different perspective, this should come as no surprise.
    If the agent concluded without \csN{}, then this is going to remain a problem for any further conclusions, unless the agent figures out they were mistaken regarding the proposition-value-premise pairing being a \requ{}.
  }
\end{note}

\paragraph*{Introduced by \(\pv{\phi}{v}\) from \(\Phi\)}

\begin{note}[Proposition-value-premise pairing introduced by \(\pv{\phi}{v}\) from \(\Phi\)]
  This restriction may seem arbitrary, and to some degree I think it is.
  Ideally, an agent concluding \(\pv{\phi}{v}\) is an instance of \csN{} just in case the agent would not have reasoned to a different conclusion if they were to reason first about any other proposition-value pair.
  However, the advantage of focusing on some proposition-value pair `required' by \(\phi\) having value \(v\) is a significant constraint on the range of proposition-value pairs an agent needs to consider in order to \csN{}.

  In general, it may not be clear which proposition-value pairs may lead an agent to fail to conclude \(\phi\) has value \(v\), but so long the proposition-value pair of interest is given by \(\phi\) having value \(v\), an exhaustive search over all other proposition-value pairs may be avoided.

  Indeed, we will say that an agent has \emph{\support{}} for \(\phi\) having value \(v\) just in case they would not have reasoned otherwise, and reserve \emph{\claiming{}} \support{} for the weaker notion.
\end{note}

\paragraph*{When}

\begin{note}
  \emph{When} concluding \(\pv{\phi}{v}\) from \(\Phi\) in order to keep things simple.
  A variant of the question may be asked if the agent has (already) concluded \(\pv{\phi}{v}\) from \(\Phi\).
  Here, rather than asking whether the agent would not conclude \(\pv{\phi}{v}\) from \(\Phi\), we may ask whether the agent would revise their conclusion of \(\pv{\phi}{v}\) from \(\Phi\).
\end{note}

\subsection{An equivalent statement of \zS{}, \ZS{}}
\label{sec:overview:an-equiv-stat-of-zs}

\begin{note}
  We began this section with the statement of a question ---~\autoref{question:zs}, or \qzS{}  --- concerning whether something is the case when an agent concludes some proposition-value pair \(\pv{\phi}{v}\) from some pool of premises \(\Phi\).
  We then reformulated the question in terms of a property, \zS{}.

  The terminology of \zS{}, or \zSN{0}, is certainly artificial.
  Still, I take the question, and the content of \zS{} to be fairly intuitive.

  Roughly, at issue is whether there is some proposition-value-premise pairing which would prevent an agent from concluding some proposition-value pair from some pool of premises.

  Some technical concepts have been appealed to in order to clarify the question --- in particular with respect to our assumptions regarding concluding, and our focus on a fixed epistemic state --- but the question itself is fairly natural.

  Consider again the \illu{1} regarding a lost pair of keys.
  There does seem to be a difference between an exhaustive consideration of all the places the keys may be before concluding the keys are lost on the one hand, and concluding the keys are lost while expecting some place on hasn't looked will come to mind on the other.
  Or, concluding that something happened based on a friends story by passive acceptance on the one hand, and concluding the thing happened after checking for consistency on the other.

  So, we take \qzS{} to be an intuitive question, and \zS{} to be an intuitive property.
  Non-standard, perhaps, but still intuitive.
\end{note}

\begin{note}
  Our attention now turns to providing an equivalent statement of \zS{}, or in other words necessary and sufficient conditions for negative resolution to \qzS{}.%
  \footnote{
    Indirectly, necessary and sufficient conditions for a positive answer, by negating either side.
  }
  We term this equivalent statement of \zS{}, `\ZS{-}', or `\ZS{}' for short.

  \ZS{} will have a key role in our main argument for a negative resolution to~\autoref{issue:Main}.

  Given the prominent role \ZS{} will have in our main argument, we will take some additional care in stating \ZS{}.
  In particular, we will provide an separate characterisation of the relevant \(\pvp{\psi}{v'}{\Psi}\) proposition-value-premise pairings of interest.
  These we will term `\requ{1}' of concluding \(\pv{\phi}{v}\) from \(\Phi\).
  And we will include additional discussion of some subtleties regarding \requ{1} given our assumption regarding concluding from~\autoref{chapter:concluding}.

  The following account of \ZS{} will then be relatively straightforward.
  We hold that an agent has \ZS{} for some proposition-value pair \(\pv{\phi}{v}\) with respect to some pool of premises \(\Phi\) just in case the agent has concluded \(\pv{\psi}{v'}\) from \(\Psi\) for any \requ{} \(\pvp{\psi}{v'}{\Psi}\) of concluding \(\pv{\phi}{v}\) from \(\Phi\).

  As noted in the introduction, this leads to a closure condition.
  If an agent has \ZS{} for some proposition-value pair \(\pv{\phi}{v}\) with respect to some pool of premises \(\Phi\), then it will follow that an agent has concluded \(\pv{\psi}{v'}\) from \(\Psi\) for any \requ{} \(\pvp{\psi}{v'}{\Psi}\) of concluding \(\pv{\phi}{v}\) from \(\Phi\).
  Before arguing that \ZS{} is equivalent to \zS{}, we will walk through this observation in some detail.
\end{note}

\begin{note}
  Still, it is important to note the (proposed) link between \qzS{}, \zS{}, and \ZS{}.

  \ZS{} has the potential to be a strong condition, \emph{if} we show that \ZS{} applies to some instance of concluding \(\pv{\phi}{v}\) from \(\Phi\) where the agent has not witnessed reasoning from \(\Psi\) to \(\pv{\psi}{v'}\) for some \requ{} \(\pvp{\psi}{v'}{\Psi}\).%
  \footnote{
    I take it this, by itself, is not immediate.
  }
  However, even if \ZS{} turn out to be a strong condition, it remains motivated by, and --- granting the arguments to follow --- equivalent to a fairly intuitive question.
\end{note}

\begin{note}[Intuition]
  This may seem a trivial point, but I think it is important to keep in mind.
  We have introduced \zS{} via \qzS{}, and it may be easy to grant us authority over what \zS{} is, or what \qzS{} asks.
  However, the interrogative core of \qzS{} is stated in neutral terms.
  So long as you have some intuitive understanding of `concluding', then, granting sufficient information about an agent's epistemic state, you are in position to determine whether \qzS{} has a negative or positive answer, and hence whether and agent has \zS{} for the relevant proposition-value-premise pairing.

  Our argument for the equivalence between \zS{} and \ZS{} will not further specify the content of \qzS{}.
  Instead, \ZS{} will provide an alternative characterisation of \zS{} from which we will draw further consequences.
  In short, it will be up to you to evaluate whether the \ZS{} really is an alternative characterisation of \zS{}.
\end{note}

\paragraph*{The argument for \autoref{prop:qzs-tick-equals-iCS}}

\begin{note}
  We now turn to providing an argument for~\autoref{prop:qzs-tick-equals-iCS}.
  To do so, we argue for an equivalent proposition focusing on \qzS{} and \ZS{}:

  \begin{proposition}
    \label{prop:qzs-tick-equals-iCS:var}
    For any property \(\chi\):
    \begin{enumerate}[label=\Alph*.]
    \item
      \label{squish:A}
      \squishA{An}{the}
    \end{enumerate}
    \emph{if and only if}:
    \begin{enumerate}[label=\Alph*.,resume]
    \item
      \label{squish:B}
      \squishB{the}.
    \end{enumerate}
    \vspace{-\baselineskip}
  \end{proposition}

  \Autoref{prop:qzs-tick-equals-iCS:var} states that an agent satisfying \iZS{} is a necessary condition of the agent satisfying an property \(\chi\) which is sufficient to ensure a negative resolution to~\autoref{question:zs}.

  The purpose of arguing for~\autoref{prop:qzs-tick-equals-iCS} via~\autoref{prop:qzs-tick-equals-iCS:var} is option to contrast some property \(\chi\) with \ZS{}.

  Specifically, if we contrapose the left-to-right direction we have some property \(\chi\) which does not entail satisfaction of \iZS{}.
  And, we will argue that any such property \(\chi\) is insufficient for a negative resolution to \qzS{}.

  Indeed, the left-to-right direction is the only direction of significant interest with respect to~\autoref{prop:qzs-tick-equals-iCS:var}.
  The right-to-left direction requires little effort.

  This is not to say the argument for the left-to-right direction will be watertight.
  Rather, we will focus on localised tension.
  Still, we are not ready to provide the argument for~\autoref{prop:qzs-tick-equals-iCS:var} just yet.

  First we need to note the equivalence between~\autoref{prop:qzs-tick-equals-iCS} and~\autoref{prop:qzs-tick-equals-iCS:var}.

  Then, with the equivalence in hand, we may turn to arguing for~\autoref{prop:qzs-tick-equals-iCS:var}.
\end{note}

\paragraph*{The equivalence between~\autoref{prop:qzs-tick-equals-iCS} and~\autoref{prop:qzs-tick-equals-iCS:var}}

\begin{note}
  \begin{proposition}
    Equivalence between~\autoref{prop:qzs-tick-equals-iCS} and~\autoref{prop:qzs-tick-equals-iCS:var}.
  \end{proposition}
\end{note}

\begin{note}
  To see the equivalence between~\autoref{prop:qzs-tick-equals-iCS} and~\autoref{prop:qzs-tick-equals-iCS:var} observe that ~\autoref{prop:qzs-tick-equals-iCS} and~\autoref{prop:qzs-tick-equals-iCS:var} entail one another.

  For, assume~\autoref{prop:qzs-tick-equals-iCS} holds.
  Then, an agent satisfying \iZS{} is both necessary and sufficient for a positive resolution to~\qzS{}.

  Now, take some property \(\chi\) and assume~\ref{squish:A} holds.
  Given~\ref{squish:A} holds, \(\chi\) is sufficient to positively resolve~\qzS{}.
  And, by~\autoref{prop:qzs-tick-equals-iCS}, a positive resolution to \qzS{} entail the agent satisfies \iZS{}.
  Therefore, whenever the agent satisfies \(\chi\), the agent also satisfies \iZS{}.
  So,~\ref{squish:B} holds.

  Conversely, take some property \(\chi\) and assume~\ref{squish:B} holds.
  Then, if the agent satisfies \(\chi\), the agent also satisfies \iZS{}.
  Given~\autoref{prop:qzs-tick-equals-iCS}, satisfying \iZS{} is sufficient to resolve~\autoref{question:zs}.
  So,~\ref{squish:A} holds.

  Now assume~\autoref{prop:qzs-tick-equals-iCS:var} holds.

  First, suppose the agent has resolved \qzS{}.
  Then, the agent satisfies some property \(\chi\), sufficient to resolve \qzS{}.
  Therefore, by~\autoref{prop:qzs-tick-equals-iCS:var}, we have that the agent also satisfies \iZS{}.
  Hence, satisfying \iZS{} is a necessary condition of resolving \qzS{}.

  Second, suppose the agent satisfies \iZS{}.
  It is immediate that satisfaction of \iZS{} entails satisfaction of \iZS{}.
  Therefore, satisfaction of \iZS{} is sufficient for a positive resolution to~\qzS{}.

  This tells us \iZS{} is necessary.
  And, as an immediate consequence, \iZS{} is sufficient.
  For, holds for all sufficient conditions.
  Though, the most straightforward argument for the right-to-left direction involves establishing this directly.
\end{note}

% \begin{note}
%   \footnote{
%     Symbolically, represent this as follows.

%     First, shorthand for \(\pvp{\psi}{v'}{\Psi}\) being a \requ{} of \(\pvp{\phi}{v}{\Phi}\):
%     \begin{itemize}
%     \item \(\pvp{\psi}{v'}{\Psi} \rleadsto \pvp{\phi}{v}{\Phi}\)
%     \end{itemize}
%     Now, quantify:
%     \begin{itemize}
%     \item \(\forall \pvp{\psi}{v'}{\Psi}\colon (\pvp{\psi}{v'}{\Psi} \rleadsto \pvp{\phi}{v}{\Phi})\)
%     \end{itemize}
%     Concludes
%     \begin{itemize}
%     \item \(\mathcal{C}(\pv{\psi}{v'},\psi)\)
%     \end{itemize}
%     Statement that the agent concludes or has concluded \(\pv{\psi}{v'}\) from \(\Psi\) for all \(\pvp{\psi}{v'}{\Psi}\) such that \(\pvp{\psi}{v'}{\Psi}\) is a \requ{} of \(\pvp{\phi}{v}{\Phi}\).
%     \begin{itemize}
%     \item \(\forall \pvp{\psi}{v'}{\Psi}\colon (\pvp{\psi}{v'}{\Psi} \rleadsto \pvp{\phi}{v}{\Phi}) \rightarrow \mathcal{C}(\pv{\psi}{v'},\Psi)\)
%     \end{itemize}
%     Let this be \(\mathcal{c}\).

%     Now, \csN{}.
%     \begin{itemize}
%     \item \(\mathsf{CS}(\pvp{\phi}{v}{\Phi})\)
%     \end{itemize}
%     And, sufficient to resolve question:
%     \begin{itemize}
%     \item \(R_{S}(\chi,?\mathsf{CS}(\pvp{\phi}{v}{\Phi}))\)
%     \end{itemize}
%     So, getting \(\chi\) is sufficient to resolve the question of whether the agent has \csVed{} for \(\pv{\phi}{v}\) from \(\Phi\).

%     First, we get
%     \begin{itemize}
%     \item \(R_{S}(\mathcal{c},?\mathsf{CS}(\pvp{\phi}{v}{\Phi}))\)
%     \end{itemize}
%     Concluding is sufficient.
%     From this, anything that entails \(\mathcal{c}\) is sufficient.

%     \begin{itemize}
%     \item \(\forall \chi((\chi \rightarrow c) \rightarrow R_{S}(\chi,?\mathsf{CS}(\pvp{\phi}{v}{\Phi})))\)
%     \end{itemize}

%     Conversely, if some \(\chi\) is sufficient, \(\chi\) entails \(\mathcal{c}\)

%     \begin{itemize}
%     \item \(\forall \chi(R_{S}(\chi,?\mathsf{CS}(\pvp{\phi}{v}{\Phi})) \rightarrow (\chi \rightarrow c))\)
%     \end{itemize}

%     From the two above:
%     \begin{itemize}
%     \item \(\forall \chi(R(\chi,?\mathsf{CS}(\pvp{\phi}{v}{\Phi})) \leftrightarrow (\chi \rightarrow c))\)
%     \end{itemize}
%   }
%   \end{note}

\paragraph*{The argument for~\autoref{prop:qzs-tick-equals-iCS:var}}

\begin{note}
  Given the equivalence between~\autoref{prop:qzs-tick-equals-iCS} and~\autoref{prop:qzs-tick-equals-iCS:var}, we now turn to arguing for~\autoref{prop:qzs-tick-equals-iCS:var}.

  As suggested above, we split the argument into two parts.
  The right-to-left direction and the left-to-right direction.

  We begin with the right-to-left direction as it is mostly straightforward.
  We then turn to the left-to-right direction, which is significantly more involved.
\end{note}

\paragraph*{Right-to-left}

\begin{note}
  For the right-to-left direction our goal is to establish:

  For any property \(\chi\):
  \begin{quote}
  \begin{enumerate}
    \item[B.]
      \squishB{an}
    \end{enumerate}
    \emph{implies}
    \begin{enumerate}
    \item[A.]
      \squishA{The}{the}.
    \end{enumerate}
  \end{quote}
\end{note}

\begin{note}
  Fix an agent, take some property \(\chi\), and assume \squishB{the}.

  Now, \(\chi\) entails the agent satisfies \iZS{}, but \(\chi\) may also be \ZS{}.
  Indeed, \ZS{} is the minimal property for which satisfaction of \(\chi\) entails \ZS{}.
  So, though \(\chi\) is arbitrary, we must show that satisfaction of \ZS{} alone is sufficient to resolve whether the agent has \zS{} for \(\pv{\phi}{v}\) when concluding \(\pv{\phi}{v}\) from \(\Phi\).

  Still, this is relatively straightforward.
  \qzS{} asks whether there is some \requ{} \(\pvp{\psi}{v'}{\Psi}\) such that the agent has not settled whether \(\pv{\psi}{v'}\) follows from \(\Psi\).

  Now, for any such \(\pvp{\psi}{v'}{\Psi}\), satisfying \iZS{} requires one of the following to conditions holds:
  \begin{itemize}
  \item
    The agent has concluded \(\pv{\psi}{v'}\) from \(\Psi\).
  \item
    When concluding \(\pv{\phi}{v}\) from \(\Phi\) the agent also concludes \(\pv{\psi}{v'}\) from \(\Psi\).
  \end{itemize}
  In both cases, when concluding \(\pv{\psi}{v'}\) from \(\Psi\) for any such \(\pvp{\psi}{v'}{\Psi}\) the agent will have concluded \(\pv{\psi}{v'}\) from \(\Psi\).
  And, given a conclusion of \(\pv{\psi}{v'}\) from \(\Psi\), it seems clear the agent has settled whether \(\pv{\psi}{v'}\) follows from \(\Psi\).
  For, the agent has concluded \(\pv{\psi}{v'}\) from \(\Psi\)!
  Indeed, at question is whether the agent would conclude \(\pv{\psi}{v'}\) from \(\Psi\), and so there is no clearer answer to this question than concluding \(\pv{\psi}{v'}\) from \(\Psi\).

  So, satisfying clauses of \iZS{} is sufficient for a negative resolve~\qzS{}.
  For, grant that notion of a \requ{0} captures the relevant cases, \iZS{} requires concluding.
\end{note}

\subsection{Left-to-right}

\begin{note}
  For the left-to-right direction our goal is to establish:

  For any property \(\chi\):
  \begin{quote}
  \begin{enumerate}
    \item[A.]
      \squishA{An}{the}
    \end{enumerate}
    \emph{implies}
    \begin{enumerate}
    \item[B.]
      \squishB{the}.
    \end{enumerate}
  \end{quote}

  To do so, we take some arbitrary property \(\chi\), and then contrapose the conditional.
  In other words, assume \(\chi\) does not entail the agent has satisfied \iZS{} with the sub-goal of showing that an agent satisfying \(\chi\) is sufficient to resolve whether the agent has \(\zS{}\) for \(\pv{\phi}{v}\) when concluding \(\pv{\phi}{v}\) from \(\Phi\).

  To aid the clarify of the argument we begin by introducing a plausible candidate for \(\chi\).
  We will then use the candidate to develop the core of the argument, and to conclude we will observe how the argument generalises.
\end{note}


\subsubsection{\iZSm{}, a candidate for \(\chi\)}
\label{overview:sec:iCS-iCSm-limitation-closure}

\begin{note}
  \begin{idea}[\iZSm{2} --- \iZSm{}]
    \label{idea:Zsm:overview}
    An agent \vAgent{} has \ZSm{} for \(\pv{\phi}{v}\) with respect to some pool of premises \(\Phi\) \emph{only if}:
    \begin{enumerate}[label=\arabic*., ref=\named{Z\(^{-}\)S:\arabic*}]
    \item
      \label{idea:Zsm:overview:requ}
      For any proposition-value pair \(\pv{\psi}{v'}\) which is a \requ{} of concluding \(\pv{\phi}{v}\) from \(\Phi\) either:
      \begin{enumerate}[label=\alph*., ref=\named{Z\(^{-}\)S:1.\alph*}]
      \item
        \label{idea:Zsm:overview:requ-sat:Past}
        \vAgent{} holds \emph{\vAgent{} would conclude} \(\pv{\psi}{v'}\) from the relevant pool of premises \(\Psi\).
      \item
        \label{idea:Zsm:overview:requ-sat:Pres}
        In concluding \(\pv{\phi}{v}\) \vAgent{} \emph{also} holds \emph{\vAgent{} would conclude} \(\pv{\psi}{v'}\) from the relevant pool of premises \(\Psi\).
      \end{enumerate}
    \end{enumerate}
    \vspace{-\baselineskip}
  \end{idea}
\end{note}

\begin{note}[Difference between \iZS{} and \iZSm{}]
  The difference between \iZS{} and \iZSm{} is straightforward.

  \iZS{} requires an agent concludes \(\pv{\psi}{v'}\) from \(\Psi\) for any \requ{} \(\pvp{\psi}{v'}{\Psi}\).
  By contrast, \iZSm{} requires an agent holds \emph{they would conclude \(\pv{\psi}{v'}\) from \(\Psi\)}.%
  \footnote{
    Expressed differently, the agent concluding \(pv{\psi}{v'}\) from \(\Psi\) is replaced with the agent concluding that they would (conclude \(\pv{\psi}{v'}\) from \(\Psi\)).
    Here, the parentheses indicate that in both~\ref{idea:Zsm:overview:requ-sat:Past} and~\ref{idea:Zsm:overview:requ-sat:Pres} the agent is not required to conclude anything from \(\Psi\) directly.
  }

  The expression of an agent concluding that they would conclude may be somewhat stilted, but expresses a simple idea.
  Rather than concluding \(\pv{\psi}{v'}\) from \(\Psi\), the agent concludes that if they were to reason about whether \(\pv{\psi}{v'}\) follows from \(\Psi\), they would conclude \(\pv{\psi}{v'}\) from \(\Psi\).
  Note, \iZSm{} does not detail the relevant pool of premises that the agent draw the conclusion from.
\end{note}

\begin{note}[\iZS{} is (intuitively) stronger than \iZSm{}]
  In particular, \iZS{} is intuitively stronger than \iZSm{}.

  First, observe that~\ref{idea:Zs:overview:requ-sat:Past} and~\ref{idea:Zs:overview:requ-sat:Pres} (plausibly) entail ~\ref{idea:Zsm:overview:requ-sat:Past} and~\ref{idea:Zsm:overview:requ-sat:Pres}, respectively.
  For, if an agent has concluded \(\pv{\psi}{v'}\) from \(\Psi\), then \emph{in so doing} the agent has shown that they would conclude \(\pv{\psi}{v'}\) from \(\Psi\).

  Second, observe neither~\ref{idea:Zsm:overview:requ-sat:Past} nor~\ref{idea:Zsm:overview:requ-sat:Pres} (plausibly) entail~\ref{idea:Zs:overview:requ-sat:Past} nor \ref{idea:Zs:overview:requ-sat:Pres}, respectively, so long as there are plausible cases in which an agent concludes that they would conclude \(\pv{\phi}{v}\) from \(\Phi\) without concluding \(\pv{\phi}{v}\) from \(\Phi\).

  Indeed, it seems there are plausible cases.

  For, it seems an agent may conclude that they would conclude \(\pv{\phi}{v}\) from \(\Phi\) without concluding \(\pv{\phi}{v}\) from \(\Phi\).
  For example, one may be informed that one would conclude \(\pv{\phi}{v}\) from \(\Phi\) via testimony.
  Hence, the only relevant premises one plausibly requires is that they have been informed they would conclude \(\pv{\phi}{v}\) from \(\Phi\) via testimony, and \(\Phi\) may be arbitrary.
  E.g.\ I tell you that if you looked at the map you would conclude that East Palo Alto is directly north of Palo Alto (shock!) and if you trust the map you may even conclude that East Palo Alto \emph{is} directly north of Palo Alto.
  Still, you do not (obviously) conclude East Palo Alto is directly north of Palo Alto from any premises associated with details of the map.
\end{note}

\begin{note}
  \iZSm{} as a candidate \(\chi\).
  {
    \color{red}
    Following, arbitrary \(\chi\).
    However, substitute in \iZSm{} is desired.
  }
\end{note}

\begin{note}
  Now, we have seen how~\ref{idea:Zsm:overview:requ-sat:Past} and~\ref{idea:Zsm:overview:requ-sat:Pres} are (plausibly) \emph{strictly} weaker than~\ref{idea:Zs:overview:requ-sat:Past} and~\ref{idea:Zs:overview:requ-sat:Pres}, respectively.
  And, we have noted that both~\ref{idea:Zsm:overview:requ-sat:Past} and~\ref{idea:Zsm:overview:requ-sat:Pres} seem to align with the motivation provided from \csN{}.
  We briefly highlight why the distinction between ~\ref{idea:Zsm:overview:requ-sat:Past} and~\ref{idea:Zsm:overview:requ-sat:Pres} and ~\ref{idea:Zs:overview:requ-sat:Past} and~\ref{idea:Zs:overview:requ-sat:Pres}, respectively, will matter for our overall argument.

  Observe, \csN{}%
  \footnote{
    As stated, with~\ref{idea:Zs:overview:requ-sat:Past} and~\ref{idea:Zs:overview:requ-sat:Pres} over ~\ref{idea:Zsm:overview:requ-sat:Past} and~\ref{idea:Zsm:overview:requ-sat:Pres}, respectively.
  }
  will lead to tension with a positive resolution to~\autoref{issue:Main} just in case we manage to find an instance in which an agent \csN{} for \(\pv{\phi}{v}\) from \(\Phi\) without witnessing reasoning from \(\Psi\) to \(\pv{\psi}{v'}\) for some \requ{} \(\pvp{\psi}{v'}{\Psi}\) of concluding \(\pv{\phi}{v}\) from \(\Phi\).
  However, this tension will not follow if the weakened variants of~\ref{idea:Zs:overview:requ-sat:Past} and~\ref{idea:Zs:overview:requ-sat:Pres} are adopted.
  For, neither~\ref{idea:Zsm:overview:requ-sat:Past} nor~\ref{idea:Zsm:overview:requ-sat:Pres} would require the agent to conclude \(\pv{\psi}{v'}\) from \(\Psi\).

  Indeed, we will argue for the existence of cases of exactly the kind described.
  Hence, the role of~\ref{idea:Zs:overview:requ-sat:Past} and~\ref{idea:Zs:overview:requ-sat:Pres} over~\ref{idea:Zsm:overview:requ-sat:Past} and~\ref{idea:Zsm:overview:requ-sat:Pres} is not merely an issue of motivation, but also crucial to establishing tension.
\end{note}

\paragraph*{Arguing}

\begin{note}
  Fix an agent.
  Take arbitrary \(\chi\), such that \(\chi\) does not imply satisfaction of \iZS{}.
  (E.g.\ \iZSm{}, as seen above.)
\end{note}

\begin{note}
  Now, assume the agent is concluding \(\pv{\phi}{v}\) from \(\Phi\).
  There are two cases to consider.
  We state the each case for \(\chi\) generally, and then below state the case with respect to \iZSm{}.
  \begin{enumerate}[label=\Roman*., ref=\Roman*]
  \item
    \label{iZm:arg:case:I}
    The agent satisfies \(\chi\) when concluding \(\pv{\phi}{v}\) from \(\Phi\).
    \begin{itemize}
    \item
      The agent concludes they would conclude \(\pv{\psi}{v'}\) from \(\Psi\) when concluding \(\pv{\phi}{v}\) from \(\Phi\), for any \(\pvp{\psi}{v'}{\Psi}\) which is a \requ{} of concluding \(\pv{\phi}{v}\) from \(\Phi\).
    \end{itemize}
  \item
    \label{iZm:arg:case:II}
    The agent has already satisfied \(\chi\) prior to concluding \(\pv{\phi}{v}\) from \(\Phi\).
    \begin{itemize}
    \item
      The agent has (already) concluded they would conclude \(\pv{\psi}{v'}\) from \(\Psi\) when concluding \(\pv{\phi}{v}\) from \(\Phi\), for any \(\pvp{\psi}{v'}{\Psi}\) which is a \requ{} of concluding \(\pv{\phi}{v}\) from \(\Phi\).
    \end{itemize}
  \end{enumerate}

  We take each case in turn.
  Further, as \(\chi\) does not imply satisfaction of \iZS{}, and \iZS{} is not trivially satisfied, for both cases we will assume the agent does not satisfy \iZS{}.
\end{note}

\subparagraph*{Case~\ref{iZm:arg:case:I}}

\begin{note}[With \(\chi\)]
  \(\chi\) is sufficient for a negative answer to \qzS{}.

  Key observation.
  In order for \(\chi\) to be sufficient, agent's perspective.

  \begin{proposition}
    \label{prop:chiProp:no-may-fail}
    \(\chi\) must ensure that from the agent's perspective, the agent may not fail to conclude \(\pv{\psi}{v'}\) from \(\Psi\).
    \begin{argument}
      Direct from \qzS{}.
      For, negative answer.
      However, negative answer only if it is not the case that the agent may fail to conclude \(\pv{\psi}{v'}\) from \(\Psi\).
    \end{argument}
  \end{proposition}

  However, from~\ref{question:zs:option}, the agent has the option of concluding \(\pv{\psi}{v'}\) from \(\Psi\).

  By assumption the agent has not yet satisfied \(\chi\), as the agent has not yet concluded \(\pv{\phi}{v}\) from \(\Phi\).
  Hence, the agent has a check on whether they would come to satisfy \(\chi\) when concluding \(\pv{\phi}{v}\) from \(\Phi\).

  For, if the agent reasons about whether \(\pv{\psi}{v'}\) follows from \(\Psi\) and fails to conclude \(\pv{\psi}{v'}\) from \(\Psi\), then it would not be the case that the agent satisfies \(\chi\).

  Rephrase.
  \(\chi\) is sufficient for a negative answer to \qzS{}.
  In order for a negative answer, it may not be the case that the agent may fail to conclude \(\pv{\psi}{v'}\) from \(\Psi\).
  Therefore, the agent not failing, from their perspective, is required to satisfy \(\chi\).
  However, this means that before concluding \(\pv{\phi}{v}\) from \(\Phi\), the agent may establish whether they would come to satisfy \(\chi\).

  So, this means that there is a possibility of branching.
  If the agent reasons about whether \(\pv{\psi}{v'}\) follows from \(\Psi\), then the agent may fail to conclude \(\pv{\phi}{v}\) from \(\Phi\).

  Whether the agent would conclude \(\pv{\phi}{v}\) from \(\Phi\) is at issue.
  And, the agent only gets \(\chi\) when concluding.
  So, whether or not \(\chi\) cannot be sufficient for a negative answer.

  If failing is sufficient for positive answer, then failing is also sufficient to show that the agent does not satisfy \(\chi\).
  Therefore, \(\chi\) cannot be sufficient.
\end{note}

\begin{note}[With \iZSm{}]
  From \iZSm{}.
  \(\pvp{\psi}{v'}{\Psi}\) is a \requ{} of concluding concluding.
  Hence, so long as the agent has not already\dots it follows that concluding concluding is not sufficient for a negative answer to \qzS{}.
\end{note}

\begin{note}[Summary]
  Summary.
  \qzS{}, what would happen if the agent first reasoned about whether \(\pv{\psi}{v'}\) follows from \(\Psi\).
  This is the question.
  In order for negative answer, from agent's point of view, would not fail.
  However, question still holds if this is not yet the agent's point of view.
\end{note}

\begin{note}[Observation]
  Observe, as not yet \(\chi\), re-expressed \qzS{} as a question about whether \(\chi\).
  Therefore, the above argument only applies given we are in case~\ref{iZm:arg:case:I}.
  In case~\ref{iZm:arg:case:II}, we assume the agent already satisfies \(\chi\), and hence the argument will be distinct.
\end{note}

\subparagraph*{Case~\ref{iZm:arg:case:II}}

\begin{note}
  {
    \color{red}
    This should be revised, as with the idea of a \fc{0}, I no longer need to worry about the possibility of revision.

    Instead, the basic argument is that if a \requ{0}, then this still applies to whatever \(\chi\) is.
    For anything weaker, check.
  }
\end{note}

\begin{note}
  Now turn to case~\ref{iZm:arg:case:II}.

  With case~\ref{iZm:arg:case:I}, argued that \(\chi\) fails to be sufficient, because \qzS{} applies equally to whether the agent satisfies \(\chi\).
  Possibility of present reasoning branching so the agent does not satisfy \(\chi\).

  Case~\ref{iZm:arg:case:II} requires distinct argument, as by assumption the agent satisfies \(\chi\).
  Again, we have the assumption that the agent has not concluded \(\pv{\psi}{v'}\) from \(\Psi\).
\end{note}

\begin{note}
  Observe, the strategy applied to case~\ref{iZm:arg:case:I} does not extend to case~\ref{iZm:arg:case:II}.
  For, \requ{} of concluding \(\pv{\phi}{v}\) from \(\Phi\).
  Hence, it need not be the case that the agent had the option of concluding \(\pv{\psi}{v'}\) from \(\Psi\) when satisfying \(\chi\).
  In particular, when establishing from their perspective they would not fail to conclude.

  From the perspective of \iZSm{}, concluded would conclude when the agent did not have the option of concluding.

  For example, consider a case of (apparent) testimony.
  If follow strategy, win game.
  Did not have an understanding of the rules.
  Hence, did not have the option to reason from premises and reach a different conclusion.
  Only after coming to understand rules (or more strictly, adopting the perspective of understanding rules) does the option of evaluating the conditional become available.
\end{note}

\begin{note}
  Our strategy is to split case~\ref{iZm:arg:case:II} into two sub-cases, depending on whether the agent may revise whether or not the agent may revise their satisfaction of \(\chi\).
  Specifically:

  \begin{enumerate}[label=\roman*., ref=\roman*]
  \item
    \label{iZm:arg:case:II:sub:i}
    From the agent's perspective:
    The agent may revise their epistemic state so that the agent does not satisfy \(\chi\), given the agent's current epistemic state.
  \item
    \label{iZm:arg:case:II:sub:ii}
    From the agent's perspective:
    The agent may not revise their epistemic state so that the agent does not satisfy \(\chi\), given the agent's current epistemic state.
  \end{enumerate}

  It may seem only sub-case~\ref{iZm:arg:case:II:sub:ii} is compatible with satisfaction of \(\chi\).

  For, as we have observed in~\autoref{prop:chiProp:no-may-fail}, not the case that the agent may fail.

  Rewriting:

  \begin{enumerate}[label=\roman*\('\)., ref=\roman*\('\)]
  \item
    \label{iZm:arg:case:II:sub:i:var}
    From the agent's perspective:
    The agent may fail to conclude \(\pv{\psi}{v'}\) from \(\Psi\), if the agent were to attempt to conclude \(\pv{\psi}{v'}\) from \(\Psi\).
  \item
    \label{iZm:arg:case:II:sub:ii:var}
    From the agent's perspective:
    The agent may not fail to conclude \(\pv{\psi}{v'}\) from \(\Psi\), if the agent were to attempt to conclude \(\pv{\psi}{v'}\) from \(\Psi\).
  \end{enumerate}

  So, it may appear only sub-case~\ref{iZm:arg:case:II:sub:ii:var} is compatible with the agent satisfying \(\chi\).

  However, it is important to keep in mind the scope of the relevant instance of `may'.
  Given \(\chi\), failure to conclude \(\pv{\psi}{v'}\) from \(\Psi\) may be ruled out from the agent's perspective.
  However, it may also be the case that the agent entertains the possibility of failing to satisfy \(\chi\).
  Hence, as the agent may not satisfy \(\chi\), the agent may fail to conclude \(\pv{\psi}{v'}\) from \(\Psi\).%
  \footnote{
    Of course, if you interpreted \qzS{} in line with sub-case~\ref{iZm:arg:case:II:sub:ii}, you may ignore sub-case~\ref{iZm:arg:case:II:sub:i}.

    Still, I take \qzS{} to be compatible with both sub-cases~\ref{iZm:arg:case:II:sub:i} and~\ref{iZm:arg:case:II:sub:ii}.
  }
\end{note}

\begin{note}[What we will argue]
  Respectively, we will argue:
  \begin{itemize}
  \item
    For sub-case~\ref{iZm:arg:case:II:sub:i}, \(\chi\) is insufficient for a negative answer to \qzS{}.
  \item
    For sub-case~\ref{iZm:arg:case:II:sub:ii}, re-assignment of concluding, hence \(\chi\) (trivially) entails concluding. Hence, conflict with our assumption that \(\chi\) does not entail concluding.
  \end{itemize}

  In other words:
  \begin{itemize}
  \item
    \qzS{} scopes over certain revisions to an agent's epistemic state.
  \item
    If no revision, then the relation between \(\pv{\psi}{v'}\) and \(\Psi\) is sufficient to reduce the relevant instance of concluding to \(R\) and witnessing.
  \end{itemize}

  \(R\) the agent would conclude \(\pv{\psi}{v'}\) from \(\Psi\) if the agent were to reason from \(\Psi\) to \(\pv{\psi}{v'}\), and from the agent's current epistemic state, \(R\) may not fail to hold.

  Here, importance of~\ref{idea:reassignment}.
  Re-assignment.
  The only thing for the agent to do is witness \(R\).
\end{note}

\subparagraph*{\(\pvp{\psi}{v'}{\Psi}\) remains a \requ{0} of concluding \(\pv{\phi}{v}\) from \(\Phi\), given \(\chi\)}

\begin{note}
  We begin with a minor, but important observation.

  \begin{proposition}
    \(\pvp{\psi}{v'}{\Psi}\) remains a \requ{0} of concluding \(\pv{\phi}{v}\) from \(\Phi\), given \(\chi\)
  \end{proposition}
\end{note}

\begin{note}[Still a \requ{}]
  Observe that \(\pvp{\psi}{v'}{\Psi}\) is a \requ{} with respect to concluding \(\pv{\phi}{v}\) from \(\Phi\).

  Even if agent has satisfied \(\chi\), and so from perspective, still holds up.
  Indeed, whether \(\pvp{\psi}{v'}{\Psi}\) is \requ{} is independent of whether the agent satisfies \(\chi\) or has concluded \(\pv{\psi}{v'}\) from \(\Psi\).

  For a \requ{}, what matters is option and failure.
  And, for a negative resolution to \qzS{}, no failure.

  Of course, from agent's perspective they would not fail.
  However, if reason and did fail, problem.

  So, from the present point of view, a \requ{}.
\end{note}

\paragraph{The sub-cases}

\subparagraph*{Sub-case~\ref{iZm:arg:case:II:sub:i}.}

\begin{note}[Distinction between \(\chi\) and concluding]
  There is an important difference between the two cases.
  By assumption, \(\chi\), has not concluded.

  So with \(\chi\), two types of possible epistemic states.
  Concluded, not concluded.
  Further, possibility of either of these epistemic states.

  Point is, with \(\chi\), because no entailment, and assumption, these two types of epistemic state are open.

  From present, the agent only expect to go to one.
  However, this is from the perspective of current epistemic state.

  By contrast, with concluding, the agent already in the relevant epistemic state.
  The agent has concluded \(\pv{\psi}{v'}\) from \(\Psi\).
\end{note}

\begin{note}
  As the agent is not in the relevant type of epistemic state, and has the option to go to the right epistemic state, \qzS{}.
\end{note}

\begin{note}
  At issue is whether the agent may do some reasoning any end up not concluding \(\pv{\phi}{v}\) from \(\Phi\) granted they have not already concluded \(\pv{\psi}{v'}\) from \(\Psi\).
  And, as the agent has not concluded \(\pv{\psi}{v'}\) from \(\Psi\), then regardless of perspective on how reasoning would go, the option is present, and as \(\chi\) does not entail, the agent does not have the result of taking the option.
  Taking the option would give something distinct.
  Option, so figure out whether.
\end{note}

\begin{note}
  Counterpoint.

  Do not need to drop the pen in order to know that it will fall to the ground.

  Or, perhaps, do not need to go and check my car is parked outside in order to know that it is parked outside.

  Difference here is these cases involve acquiring novel information, while \qzS{} involves reasoning with respect to the agent's current epistemic state.
\end{note}

\begin{note}
  Although we have introduced the possibility of revision, it is very narrow.
  Our interest is not with revision in general.
  The idea that an agent would conclude \(\pv{\phi}{v}\) from \(\Phi\) given arbitrary revision is incredibly strong.

  Rather, the revision is quite specific.
  It is because the agent has concluded \emph{that}, and because this is now a \requ{}.
  This does not scope over querying arbitrary premises, nor adopting addition premises.
  Only because this question remains open.
\end{note}


\subparagraph*{Sub-case~\ref{iZm:arg:case:II:sub:ii}.}

\begin{note}
  No revision.

  So, from the agent's perspective, if reason from \(\Psi\), would conclude \(\pv{\psi}{v'}\).
\end{note}

\begin{note}
  Now, by assumption witnessing can't do anything.
  There can be nothing added by witnessing that would matter to concluding \(\pv{\psi}{v'}\) from \(\Psi\).
  If there were, then possibility of revision.

  Okay, so, this means that witnessing doesn't add anything.

  Hence, whether on not X is determined by the agent's present epistemic state.
  This does not mean X, as X may be the result of witnessing.
  However, must have enough.
  So, reduce whether to something independent of witnessing.




  Some X, but then, agent's present epistemic state secures X.







  So, have something in the agent's present epistemic state.
  
\end{note}

\begin{note}
  So, sufficient resources given present epistemic state.

  For, if insufficient, then fail to conclude.

  So, witnessing, putting those resources into action.

  So, take any X.
  If X makes a difference to concluding, then, availability of X + witnessing.

  So, something, X.
  Get this by witnessing.
  However, if this makes a difference, then split into pre-X and witnessing X.
  Given sub-case, no failure.
  So, only pre-X makes a difference.
  Yet, have pre-X.
  Adding in witnessing won't do anything.

  So, then, split.

  Not possible to find some X, such from the agent's point of view, such that whether X is unique to witnessing reasoning from \(\Psi\) to \(\pv{\psi}{v'}\) such that X matters to whether conclude, and X is not composite.

  This is quite strong, but does not generalise easily.
  For, if possibility of branching, revision, etc.\ then there may be some such X.
  Indeed, agent has not concluded, might fail, so conclusion may add something unique.

  But, then, re-assignment.

  Hence, concluded.

  To \illu{1}, deterministic causation.
  Some causal model, a bunch of equations.
  Wouldn't say X has caused Y given application of equations.
  However, nothing more than putting those equations in motion.

  Note, though, that given re-assignment, this is what we get.
  What we're after is this reduction.
  At issue is not the intuitive sense of `concluding', but whether there is some reduction of this kind.

  Observe, this kind of thing does give a reduction.
  However, no clear cases of this happening.
  Too much information required.
  And, deterministic, at least with respect to description at level of premises and conclusions.
  Strong assumptions.

  Further, we have no eliminated role of witnessing.
  As with causation, we have no shown that this relation is specified independently.

  However, we do have a static reduction.
  Nothing of relevance is introduced by the dynamics.
  We might not get a specification without reference to the dynamics, but we don't get anything of relevance from the dynamics themselves.

  This is basically a redescription of the assumption made.
  If there is no possibility of failing, then the dynamics are pre-determined, at least from the agent's perspective.





  Sufficient resources plus witnessing.
  These resources, no possibility of failure.
  So, for anything that does not involve witnessing, we have this.

  In addition, for anything witnessing would get, the agent has a guarantee that they would obtain.

  Not perfect information.
  Maybe exact premises, steps of reasoning, etc.
  Still, all of these are available. 
\end{note}



\begin{note}[Edge case]
  \(\chi\), the agent reasons, but does not conclude \(\pv{\psi}{v'}\) from \(\Psi\).
  Nor does the agent fail to conclude \(\pv{\psi}{\overline{v'}}\) from \(\Psi\).

  However, no revision.
  Well, then the agent won't conclude \(\pv{\phi}{v}\) from \(\Phi\).
  If the agent does, then revised epistemic state so that \ref{question:zs:subjunctive} does not hold.
  Hence, \qzS{} would no longer apply.
\end{note}

\begin{note}[Key idea]
  \begin{itemize}
  \item
    \(\chi\) does not entail concluded.
  \item
    Option to conclude.
  \item
    If agent were to reason, to types resulting states.
  \item
    Concluded, failed to conclude.
    Really, various possible resulting states, as reasoning may not terminate either way, and in general there may be various values other than \(v'\) that the agent may conclude.
  \item
    The agent isn't in either type of epistemic state.
  \item
    Now, \(\chi\) states from current perspective, what the result would be.
  \item
    However, it remains the case that these two types of epistemic states are open.
  \item
    Hence, it remains the case that reasoning could lead to either type.
  \item
    So, it remains the case that the agent may fail to conclude \(\pv{\psi}{v'}\) from \(\Psi\).
  \end{itemize}

  The core of the idea is that \qzS{} concerns the dynamics.
  If reason, then what would the result be with respect to concluding \(\pv{\phi}{v}\) from \(\Phi\).
  So, concluding to dynamics.
  Hence, \(\chi\) of the appropriate kind does not settle.
\end{note}

\subparagraph*{Notes}

\begin{note}[Witnessing?]
  Intuitively, this suggests a negative resolution to~\autoref{issue:Main}.
  Agent needs to witness reasoning, else, the agent does not have sufficient information about the relevant dynamics.

  Indeed, if negative resolution to~\autoref{issue:Main}, then I suspect this brief sketch is intuitive.
  Of course, negative answer to \qzS{} is not required for concluding, as \qzS{} requires that this holds for all \requ{1}.
  Still, given what is intuitively required for negative answer to \qzS{}, witnessing.
\end{note}

\begin{note}
  Now, important to keep in mind is that we have no assumed that concluding involves witnessing.
  By assumption, there is some difference, else we are done.
  But, what exactly this distinction amounts to is unclear.
  An account of concluding, beyond scope.
  Further, still have the issue to resolve.
  However, have a clue.
  Concluding, between \(\pv{\phi}{v}\) and \(\Phi\).
  Concluding that, existence of a relation.
  Whatever the relation of concluding is, it does not yet hold, but the agent has a guarantee that it will hold.
\end{note}

\begin{note}
  The point, in brief, is that \qzS{} is about concluding.
  And, so long as the agent has not concluded \(\pv{\psi}{v'}\) from \(\Psi\) when concluding \(\pv{\phi}{v}\), that failure to conclude would lead to failure.
  So, \(\pvp{\psi}{v'}{\Psi}\) remains unsettled.
  If the agent does the reasoning, then would not conclude.
  So, the agent may fail to conclude, because on reasoning about whether, failure to conclude \(\pv{\psi}{v'}\) from \(\Psi\) would lead to a recognised failure of \(\chi\).

  \qzS{} is about whether the agent would conclude.
  If agent hasn't concluded, then question remains.
  Still, I do not think this is immediate.
\end{note}


\begin{note}
  Important to note, that still all relative to the agent's present epistemic state, and what the agent is interested in concluding.
  For, it may be that the agent revises their epistemic state so that some relation does not hold.
  For example, the agent has concluded that some reduction holds.
  Therefore, if they prove this, then they prove something else, and, also prove the something else by other means.
  However, learn this reduction does not hold.
  Now, no longer a \requ{}.
  Indeed, relation fails because the reduction does not hold, or because the agent does not have the means to prove.
\end{note}

\paragraph*{Closing}

\begin{note}[Closure]
  So, this is our motivation.
  What we have is an intuitive idea which leads to this kind of limitation, and hence conclusre condition.
  So, if there are cases of interest, motivation that an agent concludes.
  But, conversely, the condition is strong.
  So, these cases are harder.
\end{note}

\paragraph{More details on \iZS{}}

\begin{note}
  We've only focused on failure to conclude.
  However, the agent may also conclude something else.
  Possible that there are premises for the agent \(\Psi\) and \(\overline{\Psi}\) such that from \(\Psi\) get \(\pv{\psi}{v'}\) and from \(\overline{\Psi}\), get \(\pv{\psi}{\overline{v'}}\).

  For sure, but this is a different condition.
  You may also want to impose this given the intuitive motivation for \csN{}.
  Indeed, from the perspective of no branching.
  Distinct condition, and we will not impose this.

  Note, also, that so long as distinction between concluding \emph{that} and concluding, then this is also going to be insufficient in isolation.
  For, though the agent may have exhausted other possibilities, this won't get a conclusion.
  And, if not distinct, then a plausible path to negative resolution.
\end{note}

\begin{note}
  \iZS{} does not require the conclusion to be any good.
  If you want to build this in, sure.
  However, not for us.
  It is a strong assumption, and would have no function in the arguments to follow.
\end{note}

\paragraph*{Minor clarifications}

\begin{note}[Importance of \csN{}]
  \iZS{} is key.
  Argument for negative resolution to~\autoref{issue:Main} largely rests on \iZS{}.
  As we have seen, closure.

  However, briefly note that a few things.

  First, agent's reasoning.
  At issue is whether the agent may reason to a different conclusion.
  There's nothing that would lead me elsewhere.

  Second, agent's reasoning.
  Independent of whether \(\phi\) has value \(v\), \(\psi\) has value \(v'\), or any of the premises.
  Need not be the case that satisfaction amounts to anything substantial.
  No clause for justification, etc.

  Third, competence, rather than performance.
\end{note}

\paragraph*{\emph{Concluding}}

\begin{note}[Key feature]
  We now turn to the key feature of \iZS{}:
  The requirement that an agent \emph{concludes} \(\pv{\psi}{v'}\) from \(\Psi\) when \(\pvp{\phi}{v'}{\Psi}\) is a \requ{} of concluding \(\pv{\phi}{v}\) from \(\Phi\).

  We begin by noting why this requirement should be treated with caution from a general perspective.
  We then further motivate caution by first relating the requirement to the motivation we have provided for \csN{}, and second by highlighting how the requirement will have a role in developing tension.
  This discussion will involve considering a weaker requirement, and following our motivation of caution we will argue that no weaker requirement will suffice to capture the motivation we have provided for \csN{}.
\end{note}

\paragraph*{Narrowing \requ{1}}

\begin{note}[Expanding pool constraints]
  To~\ref{notion:overview:requ:pool} of~\autoref{notion:overview:requ} the following clause may also be added:
  \begin{enumerate}[label=]
  \item
    \begin{enumerate}[label=]
    \item
      \begin{enumerate}[label=\roman*., ref=(\roman*), resume*=csIdeaCounter]
        \setcounter{enumiii}{3}
      \item
        \label{notion:overview:requ:pool:method}
        Concluding \(\pv{\psi}{v'}\) from \(\Psi\) involves the same general method the agent would use to conclude \(\pv{\phi}{v}\) from \(\Phi\).
      \end{enumerate}
    \end{enumerate}
  \end{enumerate}
  We omit~\autoref{notion:overview:requ:pool:method} from the idea of \csN{} for two (related) reasons.
  First, it is not clear what `the same general method' amounts to in detail.
  Second, avoiding questions about method affords flexibility when providing \illu{1} of \zS{}.
  However,~\autoref{notion:overview:requ:pool:method} may be imposed with no loss to the role of \zS{} in the overall argument.
  However, always a check on whether one has the general ability.
\end{note}

\begin{note}
  Reasoning, \support{}, would not reason to a different conclusion.

  Specifically, \requ{} of some conclusion.
  So long as conclusion, then it is possible to reason about whether \(\psi\) has value \(v'\), and unless conclude \(\psi\) has value \(v'\), would not conclude \(\phi\) has value \(v\).

  Intuitively, \requ{} as an independent check on the reasoning.
  If don't hold \(\psi\) from premises, then question about whether \(\phi\).

  Claiming support, necessary condition is satisfying all \requ{1}.
  Claiming support, then, is weaker than having support.
  Restricted to whether conclusion of reasoning would introduce a \requ{}.
  And, may be further restricted without impact to the tension we will develop to whether the conclusion would `clearly' introduce a \requ{}.
\end{note}

\begin{note}
  Now, consequence from \ideaCS{} and \ESU{},
  If potential witnessing event, then either concluded previously, or witness reasoning.
\end{note}

\begin{note}[Kettle logic]
  Well, it's true that the person must be acquitted, but at the same time, the person is going to have a hard time explaining how, for example, he brewed coffee for a week.

  Still, highlights what the neighbour needs to conclude.
  Did borrow.
  Was not lent damaged
  And, was returned damaged.

  Here, burden of argument.
  However, we're not interested in whether the neighbour would convince, but whether the neighbour would reach a different conclusion if they were first to reason about one of the three before concluding that the kettle was returned damaged.
\end{note}

\subsection{Literature}
\label{sec:zS:literature}

\paragraph{Circularity}

\begin{note}
  \ideaCS{} is not about circular reasoning in the sense that the term `circularity' suggests that the reasoner has taken the conclusion of the reasoning for granted.

  There's nothing in \ideaCS{} that appeals to getting \(\psi\) having value \(v'\) from \(\phi\) having value \(v\).

  However, does identify a problem in the sense that would prevent the agent from getting \(\psi\) having value \(v'\) from \(\phi\) having value \(v\).
\end{note}

\begin{note}[Testimony 1]
  \begin{illustration}[Testimony 1]
    \label{illu:CS:test:basic}
    \mbox{}
    \begin{enumerate}[label=\arabic*., ref=(\arabic*)]
    \item
      \label{ex:eiS:t:basic:test}
      \nagent{11} stated that they are trustworthy when speaking on matters regarding their personal character.
    \item
      \label{ex:eiS:t:basic:ok}
      \nagent{11} is trustworthy when speaking on matters regarding their personal character.
    \end{enumerate}
  \end{illustration}
  This kind of case is intuitively problematic.
  It seems that already need trustworthy.
  However, in order for \csN{} to apply, need for it to be the case that one has some check on whether \nagent{11} is trustworthy.
  And, by reasoning.

  This need not be the case.
  Of course, this does not mean that an agent need \csN{}.
  May be other necessary conditions.
\end{note}

\paragraph{Sgaravatti}

\begin{note}
  For example, consider what \citeauthor{Sgaravatti:2013wu} terms the `Justification Account' of circularity.\nolinebreak
  \footnote{
    As \citeauthor{Sgaravatti:2013wu} notes, the Justification Account of circularity is a rewriting of the third type of `epistemic dependence' considered by \citeauthor{Pryor:2004ws}~(\citeyear[359]{Pryor:2004ws}).
    Neither \citeauthor{Pryor:2004ws} nor \citeauthor{Sgaravatti:2013wu} endorse the Justification Account, but I take the spirit of the account to sufficient for interest.
    Still, the considerations which follow also apply to distinguish the {\color{red} problem identified} from \citeauthor{Sgaravatti:2013wu}'s favoured account (\Citeyear[\S3]{Sgaravatti:2013wu}) and the fifth type of `epistemic dependence' considered by \citeauthor{Pryor:2004ws}~(\citeyear[359]{Pryor:2004ws}).
  }

  \begin{quote}
    \begin{enumerate}[label=(JA), ref=(JA)]
    \item\label{sg:JA} An argument is circular if and only if for you to have justification to believe the premisses, it is necessary that you have justification to believe the conclusion.\nolinebreak
      \mbox{}\hfill\mbox{(\Citeyear[754]{Sgaravatti:2013wu})}
    \end{enumerate}
  \end{quote}
  Where `justification to believe' is to be read as in terms of having formed the belief in an epistemically appropriate way as opposed to (merely) possessing sufficient resources to form formed the belief in an epistemically appropriate way.\nolinebreak
  \footnote{
    Or, however you prefer to characterise \citeauthor{Firth:1978vi}'s (\Citeyear{Firth:1978vi}) distinction between doxastic and propositional justification (or warrant).
    See also \citeauthor{Silva:2020aa} (\Citeyear{Silva:2020aa}) --- esp.\ fn.\ 1.
  }
  (\citeauthor[Cf.][754--755]{Sgaravatti:2013wu})
\end{note}

\begin{note}
  First, reliance on something like justification.

  With \support{}, we arguably have something distinct.
  Have not placed constraints on reasoning.
  Hence, \ideaCS{} applies even when no justification (or any other epistemic attribute) is found.

  Indeed, to the extent that the value \(v\) need not be truth, \ideaS{} and \ideaCS{} are broader.

  Point extends to relation between the premises and the conclusion of a step of reasoning.
  There's some issue with whether there's a clear reduction to premises.

  Now, both these points may be addressed by linking justification to steps of reasoning.
  However, it still remains that get this kind of circularity by placing a constraint on permissible steps of reasoning.
\end{note}

\begin{note}
  Second, having something.
  Contrasts to reasoning in an interesting way.
\end{note}

\begin{note}[\citeauthor{Sgaravatti:2013wu} on necessity]

  \begin{quote}
    For my present purposes it will suffice to say that a good test of A’s being necessary for B (and thus of B’s being sufficient for A) is the satisfaction of two subjunctive conditionals. First, if A did not hold, B would not hold; secondly, if B were to hold, A would hold.%
    \mbox{}\hfill\mbox{(\citeyear[761]{Sgaravatti:2013wu})}
  \end{quote}
  This is very similar to what is captured by a \requ{}.

  Also, points out only a test due to implications.
  For us, \requ{} is not a test.
  And, the implications are embraced.
  Though, differences limit these somewhat.

  For the moment, point with the implications is that this makes \zS{} fairly strong.
\end{note}

\paragraph{Pryor}

\begin{note}[\citeauthor{Pryor:2004ws}'s Type 4]
  An instance of a limitation arising from assuming that the possibility obtains is the fourth type of dependence between premise and conclusion considered by \citeauthor{Pryor:2004ws}.

  \begin{quote}
    [Type 4] dependence between premise and conclusion is that the conclusion be such that evidence \emph{against it} would (to at least some degree) undermine the kind of justification you purport to have for the premises.\nolinebreak
    \mbox{}\hfill\mbox{(\citeyear[359]{Pryor:2004ws})}
  \end{quote}

  Again, plausible.\nolinebreak
  \footnote{
    A variant of \citeauthor{Pryor:2004ws}'s Type 4 dependence is~\citeauthor{Jackson:1984vk}'s account of circularity.
    \begin{quote}
      [I]t may be that a given argument to a given conclusion is such that anyone --- or anyone sane --- who doubted the conclusion would have background beliefs relative to which the evidence for the premises would be no evidence.\space \dots

      Such an argument could be of no use in convincing doubters, and is most properly said to beg the question.\nolinebreak
      \mbox{}\hfill\mbox{(\Citeyear[111-12]{Jackson:1984vk})}
    \end{quote}
    Still, in contrast to \citeauthor{Pryor:2004ws}'s Type 4, \citeauthor{Jackson:1984vk}'s account of circularity is dialectical.
    Indeed, on \citeauthor{Jackson:1984vk}'s account (without additional constraints on when an agent has justification or evidence) it need not be the case that the agent's own justification would be undermined by someone doubting the conclusion.
    In this respect, \ideaCS{} is further distinguished from a proposal such as \citeauthor{Jackson:1984vk}'s as \ideaCS{} makes mention only of the relevant agent's epistemic state and reasoning.
  }
  Further, weaken from justification to any reasoning.
  In this respect, motivated by \ideaS{}, plausibly.
  However, much stronger.
  \ideaS{} is just about entertaining.
  Subjunctive with stronger is less clear.

  Issue:
  \begin{enumerate}
  \item Evidence undermines the kind of justification the agent purports to have for the premises.
  \end{enumerate}

  And, as \citeauthor{Pryor:2004ws} notes, \emph{kind} is important.
  However, it seems kind is not the only problem.
\end{note}

\begin{note}
  \citeauthor{Pryor:2004ws}'s argument that type 4 over-generates is somewhat interesting.
  Details are in the following footnote.\footnote{
  Compatible with \citeauthor{Pryor:2004ws}'s objection to type 4 dependence.

  % \begin{illustration}
    % \mbox{}
    % \vspace{-\baselineskip}
    \begin{quote}
      Suppose you're watching a cat stalk a mouse. Your visual experiences justify you in believing:

      \begin{enumerate}[label=(\arabic*), ref=(\arabic*)]
        \setcounter{enumi}{10}
      \item
        \label{illu:Pryor:cat:1}
        The cat sees the mouse.
      \end{enumerate}

      You reason:

      \begin{enumerate}[label=(\arabic*), ref=(\arabic*), resume]
      \item
        \label{illu:Pryor:cat:2}
        If the cat sees the mouse, then there are some cases of seeing.
      \item
        \label{illu:Pryor:cat:3}
        So there are some cases of seeing.\nolinebreak
        \mbox{}\hfill\mbox{(\citeyear[361]{Pryor:2004ws})}
      \end{enumerate}
    \end{quote}
  % \end{illustration}

  Setting aside whether this is fine.

  Following \citeauthor{Pryor:2004ws}:

  Bad, given proposal, as if no cases of seeing, then the cat is not seeing. (\citeyear[361]{Pryor:2004ws})

  \citeauthor{Pryor:2004ws}'s position is as follows:

  \begin{quote}
    I don't think you need antecedent justification to believe \ref{illu:Pryor:cat:3}, before your experiences can give you justification to believe \ref{illu:Pryor:cat:1}.
    I also think it's plausible that your perceptual justification to believe \ref{illu:Pryor:cat:1} contributes to the credibility of \ref{illu:Pryor:cat:3}.\nolinebreak
    \mbox{}\hfill\mbox{(\citeyear[361]{Pryor:2004ws})}
  \end{quote}

  This may be compatible with \ideaS{} and \ideaCS{}.
  With \ideaCS{}, somewhat trivial, if \ref{illu:Pryor:cat:3} holds throughout \epVW{1}.

  More generally, weaker proposition.
  Hence, it seems \indicateV{1}.
  So there's no issue with the reasoning.
  However, `contributes to the credibility\dots'.
  }
\end{note}

\begin{note}[Issue]
  Somewhat similar to above.
  Here, however, role of novel information is of interest.
  Hence, dynamic.
  And, \csN{} is, in this respect, static.
\end{note}


\paragraph{Others}

\begin{note}
  This also extends to \citeauthor{Wright:2011wn}.
  For, \citeauthor{Wright:2011wn} relies on the idea of doubt.

  The issue here is what is required in order to doubt.
  One may need to revise one's epistemic state.

  Of course, if idea of claiming support is taken generally, then it should be the case that for any \epPW{}, it is possible for the agent to conclude from reasoning that \(\phi\) having value \(v\) holds for any \epVAd{} \world{}.

  So, if satisfy claiming support, then may satisfy doubt idea.
  However, ideal.
  Pointing out the issue here does not require such a general thing as doubt.
\end{note}

\begin{note}
  Instead, as \(\psi\) not having value \(v'\) is an \ep{}, it is possible that \(\psi\) does not have value \(v'\).
  And, if \(\psi\) does not have value \(v'\), then step \(\delta'\) does not apply to how things are.
  Hence, observing that \(\psi\) having value \(v'\) follows in turn from the conclusion of step \(\delta'\) (together with other premises) is uninformative about how things are.
\end{note}

\begin{note}
  \color{red}
  Some of the \citeauthor{Wright:2011wn} cases are interesting.
  Especially the twin cases.
  In fact, especially this idea that situations are identical.
  For, one way of understanding this is that the agent makes a choice between two disjuncts, and it is possible for the agent to make the other choice, and then come to a different conclusion.
\end{note}

\subsection{Summarising}
\label{sec:overview:CS:summarising}



\section{Resolutions to~\autoref{issue:Main}, refined}
\label{sec:issue-refined}

\begin{note}[Restating the issue]
  We began by expressing interest in the following issue:
  \vspace{-\baselineskip}
  \begin{quote}
    \issueMain*
  \end{quote}
  In \autoref{chapter:concluding} we clarified our understanding of concluding, and in~\autoref{sec:overview:zS} we introduced \csN{0}.

  With a sufficient understanding of concluding and \csN{0} in hand, we provide a restatement of the two resolutions to~\autoref{issue:Main} of interest.

  The positive answer denies there are such cases while the positive holds there are such cases.
  We term these resolutions \ESU{} and \EAS{}, respectively.

  We being with a statement of the two resolutions, before turning to discuss both \ESU{} and \EAS{} in some detail.
\end{note}

\subsection{The resolutions refined}
\label{sec:two-resolutions}

\begin{note}
  The two resolutions to~\autoref{issue:Main}, refined with respect to \csN{0} are as follows:
\end{note}

\begin{note}
  \begin{restatable}[\ESU{0} --- \ESU{}]{target}{targetESU}
    \label{denied-claim}
    An agent concluding \(\pv{\phi}{v}\) is an instance of \zS{} \emph{only if}:

    For any pool of proposition-value pairs \(\pv{\phi_{1}}{v_{1}},\dots,\pv{\phi_{k}}{v_{k}}\) appealed to as premises, the agent \emph{has} witnessed reasoning which concludes \(\phi\) has value \(v\) from \(\phi_{i}\) having values \(v_{i}\).%
    \footnote{More generally, the agent has witnessed reasoning whose conclusion \emph{indicates} \(\phi\) has value \(v\).}
  \end{restatable}
\end{note}

\begin{note}
  \begin{restatable}[\EAS{0} --- \EAS{}]{goal}{goalEAS}
    \label{prop:EAS}
    There are instances of reasoning in which an agent concludes \(\pv{\phi}{v}\) from some pool of proposition-value pairs \(\pv{\phi_{1}}{v_{1}},\dots,\pv{\phi_{k}}{v_{k}}\) as premises \emph{without} witnessing reasoning from \(\pv{\phi_{1}}{v_{1}},\dots,\pv{\phi_{k}}{v_{k}}\) to \(\pv{\phi}{v}\) is an instance of \zS{}.
  \end{restatable}
\end{note}

\begin{note}
  \ESU{} and \EAS{} are, then, fairly straightforward.
  The key difference is that \ESU{} does quantifies only over instances of concluding which are also instances of \csVImp{0}.
  And, likewise, \EAS{} only requires instances of reasoning in which concluding coincides with \csVImp{0}.
\end{note}

\begin{note}
  The relationship between \ESU{} and \EAS{} and~\autoref{issue:Main} is simple.

  Both \EAS{} and \ESU{} focus on instances of concluding which are also instances of \csVImp{0}.
  Therefore:
  \begin{itemize}
  \item
    \EAS{} entails a positive resolution to~\autoref{issue:Main}.
  \item
    A positive answer to~\autoref{issue:Main} entails \ESU{}.
  \end{itemize}
  Naturally, we are interested in instances of concluding which are also instances of \csVImp{0} because we believe this is where tension arises.
  And, so long as there is sufficient tension to motivate \EAS{}, then by the entailment noted, we will also motivate the positive resolution to~\autoref{issue:Main}.

  However, from a dialectical perspective, the refinement may also ease rejection of the positive resolution we motivate.
  For,~\autoref{issue:Main} concerns all instances of concluding, while \ESU{} only concerns instances of concluding which are also instances of \csVImp{}.
  Hence, rejection of the positive resolution we aim to motivate need not commit one to a positive answer to~\autoref{issue:Main} in general.
  Indeed, as we have placed few assumptions on concluding in general one may wary of extending any intuition for a positive answer to~\autoref{issue:Main} to all instances of concluding.
  Still, I hope the discussion of \csN{0} has provided sufficient information to evaluate~\autoref{issue:Main} when restricted to \csN{0}.
  And I hope the additional discussion will help to clarify both \ESU{} and \EAS{}.

  {
    \color{red}
    Initially, just on what follows or does not follow from statement of \ESU{} and \EAS{}.
    Then, generalise to two ways of concluding.
    There, focus on the mechanics of \ESU{} and, in particular, \EAS{}.
  }
\end{note}

\paragraph*{\ESU{}}

\begin{note}
  \ESU{} holds that whenever an agent concluding \(\pv{\phi}{v}\) is an instance of \csN{0}, the agent has always witnessed reasoning from some pool of premises.

  Our present goal is provide a handful of clarifying remarks.
\end{note}

\begin{note}
  The main clarifying remark concerns the interpretation of `has' in the statement of \ESU{}.
  We intend the relevant instance of `has' to cover all past reasoning.
  Therefore, \ESU{} does not require the agent to conclude \(\pv{\phi}{v}\) from inputs to the present instance of reasoning which culminates with \(\pv{\phi}{v}\).

  We will develop this remark in detail when we consider two types of reasoning in~\autoref{sec:overview:two-types-reasoning}.
  For the moment the gist is simple to state:

  Suppose at some point in the past an agent concluded \(\pv{\phi}{v}\) from some premises \(\Phi\) by witnessing reasoning from \(\Phi\) to \(\pv{\phi}{v}\) (and in doing so \csVed{} for \(\pv{\phi}{v}\)).
  Then, at the present moment an agent may (re-)conclude \(\pv{\phi}{v}\) from \(\Phi\) without witnessing reasoning \emph{in the present} from \(\Phi\) to \(\pv{\phi}{v}\).
  For, the constraint imposed by \ESU{} has already been satisfied by witnessing reasoning from \(\Phi\) to \(\pv{\phi}{v}\) in the past.

  For example, suppose I concluded that Riga is the capital of Latvia by studying a map.
  Now, at present, I am asked by a fried what the capital of Latvia is.
  I takes me a moment, but I recall from memory studying the map and concluding that Riga is the capital of Latvia.
  Hence, I (re-)conclude that Riga is the capital of Latvia, and I do so from the relevant premises involved when studying the map.

  Of course, \ESU{} is a necessary, rather than sufficient condition, and so further constraints may be added to rule out the present type of case.
  However, I take this observation to highlight that \ESU{} is a relatively weak assumption, focused solely on whether an agent has witnessed the relevant reasoning, and independent of when the agent witnessed the relevant reasoning.%
  \footnote{
    \label{ft:weak-esu}
    Indeed, \ESU{} may be weakened further to allow the agent to witness the relevant reasoning at some point in the future, and to quantify over all agents.
    Roughly, this weakened variant of \ESU{} states:
    \begin{quote}
      An agent concluding \(\pv{\phi}{v}\) is an instance of claiming support \emph{only if}:

      For any pool of proposition-value pairs \(\pv{\chi_{1}}{v_{1}},\dots,\pv{\chi_{k}}{v_{k}}\) appealed to as premises, \textbf{some} agent \emph{has} witnessed \textbf{or will witness} reasoning which concludes \(\phi\) has value \(v\) from \(\chi_{i}\) having values \(v_{i}\).
    \end{quote}
    Where the two adjustments are placed in bold.

    We favour \ESU{} over this weakened version of \ESU{} in order to keep the complexity of the issue and resolutions low.
    Indeed, though my intuitions regarding \ESU{} are clear (I consider \ESU{} to be intuitive --- though in tension with other ideas I find more intuitive), I am unsure on how \ESU{} generalises to other agent and future instances of reasoning.
    {
      \color{red}
      Simple example, know that some logic is decidable.
      Take a relatively simple fragment.
      Then, either proof or a countermodel.
      Same idea holds for tension, but there's no guarantee that anyone has concluded\dots
    }
  }
\end{note}

\begin{note}
  Finally, recall that~\autoref{assu:conc:d-free} implies that in witnessing reasoning which concludes \(\pv{\phi}{v}\), an agent need not recognise they are concluding \(\pv{\phi}{v}\).

  We have already provided an argument for this implication following the introduction of~\autoref{assu:conc:d-free} in~\autoref{chapter:concluding}.
  Still, with \ESU{} in hand this observation may be worth re-examining.
  For, again, the observation highlights the relative weakness of \ESU{}.
  If an agent has concluding \(\pv{\phi}{v}\), then a proponent of \ESU{} is not committed to the agent explicitly concluding \(\pv{\phi}{v}\) from the relevant instance of witnessing.
  Rather, the proponent of \ESU{} is committed only to the agent witnessing some reasoning and for that reasoning to involve concluding \(\pv{\phi}{v}\).

  To \illu{}, recall that we say \(\pv{\phi}{v}\) \indicatePr{} \(\pv{\psi}{v'}\), just in case \(\phi\) has value \(v\) \emph{only if} \(\psi\) has value \(v'\) from the perspective of the agent's epistemic state.
  Hence, so long as an agent has witnessed reasoning which concludes \(\pv{\phi}{v}\), the same reasoning may involve concluding \(\pv{\psi}{v'}\).
  And, the inclusion may be straightforwardly explained by the observation that the premises from which the agent concluded \(\pv{\phi}{v}\) are also sufficient to establish \(\pv{\psi}{v'}\).

  Of course, as~\ESU{} is only a necessary condition, it need not be the case that by witnessing reasoning which concludes \(\pv{\phi}{v}\) an agent also concludes \(\pv{\psi}{v'}\).
  Still, as \ESU{} is consistent with endorsing the above, we do not have the option of establishing tension via reflection on how the relevant agent perceives any instance of witnessing.
\end{note}

\paragraph*{\EAS{}}

\begin{note}
  Following the main clarifying remark regarding \ESU{}, the interpretation of `without' in \EAS{} also covers past reasoning.%
  \footnote{
    And, indeed, if the weakened variant of \ESU{} from~\autoref{ft:weak-esu} is adopted, any future reasoning by any agent.
  }
  Hence, the instances of concluding that are of interest with respect to \EAS{} may be termed `novel' conclusions.
\end{note}

\begin{note}
  Some reasoning.
  Some inputs to that reasoning.
  Following above, also need to ensure that relevant instances may not be analysed as being \indicateVed{}.

  Again, do not need to endorse, but \ESU{} is left as weak as possible.
  Hence, concern.

  So, not the case that \(\pv{\phi}{v}\) is \indicateVed{} by some previous reasoning or some intermediary conclusion of present reasoning.

  Abstract, but consequences of the quick argument.
  No constraints on premises.
  And, now we have developed both concluding and the positive answer to~\autoref{issue:Main} --- i.e.~\ESU{} --- in detail, we see the flexibility for one interested in defending \ESU{}.
  To motivate \EAS{}, need instances in which concluding \(\pv{\phi}{v}\) by witnessing reasoning from some pool of premises is viable.
\end{note}

\begin{note}[Uninformative]
  Just an existential.
  This does not tell us anything about how the agent concludes \(\pv{\phi}{v}\).

  This is for the following section, to which we now turn.
\end{note}


\subparagraph*{Role in argument}

\begin{note}
  \color{red}
  This expands on the quick argument a fair bit, and comes down to the same general issue.
\end{note}

\begin{note}
  The general idea of \adB{} is perhaps surprising.

  Grant an instance of \adB{}.
  Then, an agent has concluded \(\pv{\phi}{v}\) from \(\Phi\) \emph{via} having concluded \(\pv{\mu}{v}\) for some \itp{0} which states that it is possible for the agent to conclude \(\pv{\phi}{v}\) from \(\Phi\).

  However, if the agent has concluded \(\pv{\mu}{v}\) for some \itp{0} which states that it is possible for the agent to conclude \(\pv{\phi}{v}\) from \(\Phi\), then \(\pv{\mu}{v}\) alone is seem sufficient to conclude \(\pv{\phi}{v}\).
  Indeed, at issue is principle which may be stated quite generally:

  \begin{idea}
    \label{idea:c-from-pc}
    It is permissible for an agent to conclude \(\pv{\phi}{v}\) from the premise that it is possible for them to witness reasoning which concludes \(\pv{\phi}{v}\), given their present epistemic state.
  \end{idea}

  Arguing in detail for~\autoref{idea:c-from-pc} goes beyond present interest.
  I take the idea to be sufficiently intuitive.

  Of course, concluding is not factive, so the possibility of witnessing reasoning which concludes \(\pv{\phi}{v}\) does not guarantee \(\phi\) has value \(v\).
  However, by the same token,~\autoref{idea:c-from-pc} only concerns the agent concluding \(\pv{\phi}{v}\), so it seems one only needs to argue that \(\phi\) must have value \(v\) from the agent's present epistemic state.
  And, if from the agent's present epistemic state it is possible for them to witness reasoning which concludes \(\pv{\phi}{v}\), it seems the agent may not question whether \(\phi\) has value \(v\) without questioning their conclusion that it is possible for them to witness reasoning which concludes \(\pv{\phi}{v}\).

  Still, set motivation aside.
  Our interest with~\autoref{idea:c-from-pc} is with the distinction between \adA{} and \adB{}.
  And, in particular, it seems that if~\autoref{idea:c-from-pc} is granted, then the resources present for any instance of \adB{} will always allow for an instance of \adA{}.

  For, an instance of \adB{} requires an agent has concluded \(\pv{\mu}{v}\), where \(\pv{\mu}{v}\) is an \itp{0}.
  Suppose \(\pv{\mu}{v}\) is an \itp{0} for \(\pv{\phi}{v}\).

  Now, by~\autoref{idea:c-from-pc}, it is permissible for the agent to conclude \(\pv{\phi}{v}\) from \(\pv{\mu}{v}\).
  The agent has concluded it is possible for them to witness reasoning which concludes \(\pv{\phi}{v}\), given their present epistemic state, and therefore the agent may conclude \(\pv{\phi}{v}\).

  Indeed, without any further thought, it seems we have established that any proposed instance of \adB{} must contain sufficient detail to allow the instance to be (at least) reinterpreted as an instance of \adA{}.

  I think this is broadly correct.
  Granting~\autoref{idea:c-from-pc}, then the relevant \itp{0} from any instance of \adB{} will allow the agent to conclude \(\pv{\phi}{v}\) from the \itp{0}.
  And, I grant~\autoref{idea:c-from-pc}.

  Note, however, I have refrained from stating that the agent concluding \(\pv{\phi}{v}\) from the \itp{0} is an instance of \adA{}.
  For, appeal to~\autoref{idea:c-from-pc} does not state that the agent concluding \(\pv{\phi}{v}\) from the \itp{0} is a complete account of the instance of reasoning.
  Expanding, we have not show that the agent concluding \(\pv{\phi}{v}\) from the \itp{0} does not also include the agent concluding some other proposition-value pair \(\pv{\psi}{v'}\) from some other premises.
  And, if the instance does involve the agent concluding \(\pv{\psi}{v'}\) from some other premises, the instance of reasoning will only be an instance of \adA{} if the agent witnesses reasoning from those premises.

  In other words,~\autoref{idea:c-from-pc} does not immediately show that have the option of analysing away any proposed instance of \adB{}.
  For, we have not shown that any analysis under \adA{} provides a complete account of the proposed instance of \adB{}.
\end{note}

\begin{note}
  For the moment we leave this observation as a promissory note.
  Tension will be developed in~\autoref{sec:tension}.
  Still, we have already seen one way of introducing additional structure into some conclusion by focusing on instances of concluding which are also instances of \csN{0}.
  And, an important part of the argument for tension in~\autoref{sec:tension} will be showing that certain instances of \csN{} require \adB{} as no analysis under \adA{} will provide a complete account of the relevant instance of concluding/\csVImp{0}.
\end{note}

\subsubsection{Looking to tension: \adB{} and ability}
\label{sec:looking-tension}

\paragraph{General and specific ability}

\begin{note}[General and specific ability]
  Suppose general ability.
  Then, various specific instances of general ability.
  These provide \itp{1}.
\end{note}

\begin{note}[Examples]
  Some examples are clear.
\end{note}

\paragraph{\aben{the}}

\begin{note}[\aben{the}]
  Here, not only do we conclude specific instance of general ability, but also the result of witnessing the ability.

  Here is where we get the relevant \itp{}.

  For, this means there are premises available, and from these obtain conclusion.
\end{note}

\paragraph{\adB{} and ability}

\begin{note}[Generate many \itp{1}]
  General ability to solve a particular type of problem.
  Specific instances.
  Then, generate \itp{1} from specific instances of type of problem.
  Result, concluding from specified premises.
  Of course, may be unclear on what the premises are, but still, existence is presumed.%
  \footnote{
    Presumed, but not guaranteed.
    Agent may not have the ability, and indeed no agent may have ability.
  }
\end{note}

\begin{note}[No clear tension yet]
  Of course, no clear tension yet.
  General ability.
  I have the general ability as a premise.
  Here, then, instances of \adA{}.
\end{note}

\subparagraph{Limitation}

\begin{note}
  From perspective of \EAS{}, limited.
  For, it seems always possible to include \itp{} as a premise.
\end{note}

\subsection{The horizon}

\begin{note}
  Seen \adB{}.
  Initial \illu{0}, and cases of interest.
  Before continuing,  final thing.
  The horizon.
  Cases where \adB{} applies, but outside of our scope.
\end{note}

\subparagraph{Other \illu{1}}

\begin{note}[Seen memory]
  Seen memory and specific case.
  Getting an \itp{}.

  We will expand on \itp{1} of interest below.
  Briefly, a handful of \illu{1} which fall outside immediate scope of argument.
\end{note}

\section{Tension}
\label{sec:tension}

\begin{note}[Intro]
  We have discussed concluding.
  Introduced \csN{}.
  Revised~\autoref{issue:Main} to \ESU{} and \EAS{}.
  And, two types of reasoning, \adA{} and \adB{}.
  Hints regarding ability.

  Task is now to bring these together develop tension.
\end{note}

\begin{note}[Basic idea of tension]
  Strategy.
  Identify some abstract phenomenon.
  Observe how leads to tension.
  Motivate instances of this phenomenon.
  Observe tension.

  Here, phenomenon is consequence of \csN{}.
  Hence, indirectly relies on assumptions concerning concluding.
  Split, either \ESU{} or \EAS{}.
  Construction involve \itp{} introduced with respect to \adB{}.
  Hence, discussion of \adB{} will both clarify and inform resolution to tension.

  Indeed, \adB{} will offer a more general perspective on the consequences of \csN{}.
\end{note}

\subsection{Sketch of tension}
\label{sec:overview-tension}

\begin{note}[Goal]
  To establish tension between \ESU{} and \csN{} we have the following goal:

  \begin{goal}
    \label{goal:tension}
    There are instances in which concluding \(\pv{\phi}{v}\) from \(\Phi\) seems to be an instance of \csN{} for \(\pv{\phi}{v}\) where:
    \begin{enumerate}[label=\arabic*., ref=\named{G\ref{goal:tension}:\arabic*}]
    \item
      \label{goal:tension:requ}
      There is some \(\pvp{\psi}{v'}{\Psi}\) which is a \requ{} of \(\pvp{\phi}{v}{\Phi}\) such that:
      \begin{enumerate}[label=\alph*., ref=\named{G\ref{goal:tension}:1\alph*}]
      \item
        \label{goal:tension:requ:conclude}
        It is not possible for the agent to conclude that they would conclude \(\pv{\psi}{v'}\) from \(\Psi\) \emph{without} concluding \(\pv{\psi}{v'}\) from \(\Psi\).
      \item
        \label{goal:tension:requ:no-reason}
        The agent does not witness any reasoning from \(\Psi\).
      \end{enumerate}
    \end{enumerate}
    \vspace{-\baselineskip}
  \end{goal}

  If there are instance of the type described by \autoref{goal:tension}, then \ESU{} and \csN{} are in tension.

  For, we will have some proposition-value-premise pairing \(\pvp{\phi}{v}{\Phi}\) such that an agent conclude \(\pv{\phi}{v}\) from \(\Phi\), and in doing so \csV{} for \(\pv{\phi}{v}\).
  And, by \autoref{goal:tension:requ} we have some \requ{} \(\pvp{\psi}{v'}{\Psi}\).

  Now, as \(\pvp{\psi}{v'}{\Psi}\) is a \requ{0} of concluding \(\pv{\phi}{v}\) from \(\Phi\), it must be the case that the agent either has concluded or simultaneously concludes that they would conclude \(\pv{\psi}{v'}\) from \(\Psi\) when concluding \(\pv{\phi}{v}\) from \(\Phi\).

  Further, the \requ{} \(\pvp{\psi}{v'}{\Psi}\) has two key properties.

  First, from~\autoref{goal:tension:requ:conclude}, the agent may only conclude that they would conclude \(\pv{\psi}{v'}\) from \(\Psi\) by concluding \(\pv{\psi}{v'}\) from \(\Psi\).
  Hence, by \ESU{} it must be the case that the agent witnesses reasoning from \(\Psi\) which concludes with \(\pv{\psi}{v'}\).

  Second, from~\autoref{goal:tension:requ:no-reason} the agent does not witness any reasoning from \(\Psi\).
  Hence, the agent does not witness reasoning from \(\Psi\) which concludes with \(\pv{\psi}{v'}\).

  From the above, tension.
  For, if concluding \(\pv{\phi}{v}\) from \(\Phi\) really is an instance of \csN{}, then the agent must have concluded \(\pv{\psi}{v'}\) from \(\Psi\) without witnessing reasoning from \(\Psi\) to \(\pv{\psi}{v'}\).
  However, if \ESU{} holds then it is not possible for the agent to conclude \(\pv{\psi}{v'}\) from \(\Psi\) without witnessing reasoning from \(\Psi\) to \(\pv{\psi}{v'}\).
  So, either the agent has not \csN{} for \(\pv{\phi}{v}\) or \ESU{} does not hold.
  Hence, either \csN{} or \ESU{} must be restricted in some way.

  Of course, the exact nature of the tension, and how it should be resolved, depends in part on the proposition-value-premise pairings which satisfy \autoref{goal:tension}.

  Indeed,~\autoref{goal:tension:requ:conclude} and~\autoref{goal:tension:requ:no-reason} are significant restrictions, and it is not clear from the abstract statement that these correspond to sufficiently interesting phenomenon.
\end{note}

\begin{note}
  Whether or not~\autoref{goal:tension:requ:no-reason} is satisfied will depend on the specifics of any given instance.
  For, we have placed no constraints on which premises an agent appeals to, nor what the contents of those premises may be.

  \autoref{goal:tension:requ:conclude}, by contrast, will follows from a collection of \requ{1} forming a `\cluster{}'.

  In short, a \cluster{} is a collection of proposition-value-premise pairings such that every proposition-value-premise pairing is a \requ{} of every other proposition-value-premise pairing in the collection.

  Intuitively, every proposition-value-premise pairing of the \cluster{0} is an independent check on every other proposition-value-premise pairing of the \cluster{0}.

  Indeed, we will argue that \autoref{goal:tension:requ:conclude} is satisfied whenever \(\pvp{\phi}{v}{\Phi}\) is a member of a \cluster{}.

  Finally, then, we seek some concrete instances.
\end{note}

\begin{note}[Ability]
  Here, our interest turns to ability.
  I take it as given that there are various instances of concluding \(\pv{\phi}{v}\) from \(\Phi\) for which the reasoning from \(\Phi\) to \(\pv{\phi}{v}\) is an instance of a general ability.

  For example, I conclude \(31 + 53 = 84\) from some premises, and the reasoning falls under my general ability to perform (simple) arithmetic.
  The specifics may differ, but there is sufficient overlap with concluding \(43 + 81 = 123\) and \(91 + 54 = 145\) to consider the reasoning of the same type.
  Indeed, \(532 - 91 = 441\), \(19 * 32 = 608\), and \(126/36 = 3.5\) may also fall under the same (general) ability.
  In other words, in concluding each equation I witness a specific instance of the general ability.

  Likewise, one may have the (general) ability to solve chess problems, complete \(\{ \text{Sudoku}, \text{KenKen}, \text{Nonogram}, \dots\}\) puzzles, or parse sentences in a given language.

  So long as you have the (general) ability, I expect each instance of witness the ability is intuitively an instance of \csN{}.
  I would not have concluded \(31 + 53 \ne 84\) and you would not have failed to identify the relevant winning strategy of some chess problem.

  However, in each of the examples of ability noted, every other specific instance of the general ability functions as an independent check on whether one has the relevant ability.
  I should make no mistake about \(85 + 21\) and you should make no mistake with the next chess problem.
  Granting, of course, that we do have the relevant abilities.

  Hence, it seems clear that if an agent \csV{} for \(\pv{\phi}{v}\) when concluding \(\pv{\phi}{v}\) from \(\Phi\) and the agent's reasoning is a specific instance of a general ability, then there at least various \requ{1} associated with concluding \(\pv{\phi}{v}\) from \(\Phi\).

  More generally, concluding \(\pv{\phi}{v}\) from \(\Phi\) from reasoning which is the specific instance of a general ability leads to a \cluster{}.
  And, it is perhaps already intuitive that one does not witness reasoning from the premises of at least some proposition-value-premise pairing in the \cluster{}.

  For example, it seems plausible that the configuration the chess board for any given problem forms of a premise of the agent's reasoning, but so long as one has not seen the relevant problem, it seems implausible that one has witnessed reasoning that includes the relevant configuration.

  Of course, it may seem equally implausible that an agent concludes there is some winning strategy for some configuration of a chess board they have not yet seen.
  However, this intuition should be carefully examined.

  With an unopened chess book by a reputable author before you, I expect you have no problem concluding that each of the solutions are correct.
  {
    \color{red}
    Or, that granting ability, conclude you would (also) conclude winning strategy or not for each chess piece.
  }
  Yet, you have not yet seen any of the solutions.
  So, in general, there seems no issue with an agent concluding there is some winning strategy for some configuration of a chess board they have not yet seen.

  Of course, in the case of the book there is a key premise:
  The book is written by a reputable author.
  However, if you wish to conclude there is a winning strategy via your own reasoning, then the above considerations take effect.%
  \footnote{
    And, indeed, may already hold with respect to the key premises.
    For, so long as you hold you have the general ability to reason about chess problems of the relevant kind, you have an independent check on whether the author really is reputable.
  }
\end{note}

\begin{note}[Moving on]
  So much for the rough outline of how~\autoref{goal:tension} leads tension between \csN{} and \ESU{}.
  Let us turn to the details.

  {
    \color{red}
    \begin{itemize}
    \item
      In~{\color{red} ???} we develop tension in the abstract.
      Our attention will be focused solely on drawing out how \csN{} and \ESU{} are in tension if cases of a certain type exist.
      In particular, we develop the notion of a \cluster{} introduced above through a handful of definitions, and state relevant consequences in a number of propositions.
    \item
      With an abstract understanding of how \csN{} is in conflict with \ESU{} given the existence of certain cases, we then turn to arguing for the existence of such cases in~\autoref{sec:concrete-tension}.
      Follow the sketch given, these cases will center around the relationship between the general ability to reason about certain types of problems and specific instances of the general ability.
    \item
      In~\autoref{sec:tension:adb} abstract a little.
      Relationship between general ability and \itp{1}.
      Broader understanding of what the tension amounts to, and the relationship between \EAS{} and \adB{}.
    \item
      Finally, in~\autoref{sec:overview:resolving-tension} we turn to resolving the tension and a number of latent issues.
    \end{itemize}
  }
\end{note}

\subsection{Tension in the abstract}
\label{sec:tension-abstract}

\begin{note}[Plan]
  In this section we establish how \csN{} and \ESU{} are in tension from an abstract perspective.
  Our immediate goal is to provide a clear characterisation of the kind of cases which will lead to tension between \csN{} and \ESU{}, if concrete instances of those cases exist.
  In the following section (\ref{sec:concrete-tension}) we will then argue for the existence of such cases.

  We begin by defining a \cluster{1} of \requ{1}, and will then refine the definition of a \cluster{1} to that of a \ragCluster{} to further clarify the kind of cases of interest.
  The two-step definition will allow us to observe exactly what is required of cases for tension.
\end{note}

\begin{note}[\requCluster{3}]
  We begin with the definition of a \cluster{}.
  \begin{definition}[A \requCluster{1}]
    \label{def:requCluster}
    Some collection of proposition-value-premise pairings \(\mathcal{C} = \{\pvp{\phi_{i}}{v_{i}}{\Phi_{i}}\}_{i}\) is a \emph{\cluster{}} with respect to some agent \vAgent{}('s epistemic state) just in case:
    \begin{itemize}
    \item
      For any \(\pvp{\phi_{i}}{v_{i}}{\Phi_{i}}\) in \(\mathcal{C}\), each \(\pvp{\phi_{j}}{v_{j}}{\Phi_{j}}\) in \(\mathcal{C}\) (such that \(j \ne i\)) is a \requ{} of \(\pvp{\phi_{i}}{v_{i}}{\Phi_{i}}\).
    \end{itemize}
    \vspace{-\baselineskip}
  \end{definition}

  Intuitively, a \cluster{0} is a collection of proposition-value-premise pairings such that every proposition-value-premise pairing is a \requ{} of every other proposition-value-premise pairing in the collection.
\end{note}

\begin{note}
  With the definition of a \cluster{} in hand, we are ready to state our first proposition.

  \begin{proposition}[\cluster{3} and \csN{0}]
    \label{prop:cluster:csN}
    Suppose \(\mathcal{C}\) is a \requCluster{0} with respect to an agent \vAgent{}('s epistemic state).
    And, let \(\pvp{\phi}{v}{\Phi}\) be some proposition-value-premise pairing in \(\mathcal{C}\).

    \begin{itemize}
    \item
      \vAgent{} \csV{} for \(\pv{\phi}{v}\) when concluding \(\pv{\phi}{v}\) from \(\Phi\) only if \emph{either}:
      \begin{itemize}
      \item
        For any \(\pvp{\psi}{v'}{\Psi}\) in \(\mathcal{C}\):
        \begin{itemize}
        \item \vAgent{} has at some point in the past concluded that \vAgent{} would conclude \(\pv{\psi}{v'}\) from \(\Psi\).
        \item
          When concluding \(\pv{\phi}{v}\) from \(\Phi\), \vAgent{} simultaneously concludes that \vAgent{} would conclude \(\pv{\psi}{v'}\) from \(\Psi\).
        \end{itemize}
      \end{itemize}
    \end{itemize}
  \end{proposition}

  \Autoref{prop:cluster:csN} follows directly from~\iZS{} and~\autoref{def:requCluster}.

  \begin{argument}
    Suppose \(\mathcal{C}\) is a \requCluster{0} with respect to an agent \vAgent{}('s epistemic state).
    Let \(\pvp{\phi}{v}{\Phi}\) be some proposition-value-premise pairing in \(\mathcal{C}\).
    And, let \(\pvp{\psi}{v'}{\Psi}\) be some proposition-value-premise pairing in \(\mathcal{C}\).

    By~\autoref{def:requCluster} we have that \vAgent{} concluding that \vAgent{} would conclude \(\pv{\psi}{v'}\) from \(\Psi\) is a \requ{} of concluding \(\pv{\phi}{v}\) from \(\Phi\).
    And, by~\iZS{}, it must be the case that either:
    \begin{itemize}
    \item
      \vAgent{} has concluded that they would conclude \(\pv{\psi}{v'}\) from \(\Psi\), ref{idea:Zs:overview:requ-sat:Past}.
      Or,
    \item
      In concluding \(\pv{\phi}{v}\) \vAgent{} simultaneously concludes \(\pv{\psi}{v'}\) from \(\Psi\), ref{idea:Zs:overview:requ-sat:Pres}.
    \end{itemize}
    Hence, we have established~\ref{prop:cluster:csN}.
  \end{argument}
\end{note}

\begin{note}
  \begin{proposition}
    \label{prop:cluster:simul}
    Need to do everything in a cluster at the same time.
  \end{proposition}

  \begin{argument}
    Straightforward.
    For, anything is a \requ{} for any other.
    So, only \csV{} at the same time.
  \end{argument}
\end{note}

\begin{note}[No \(\gamma\)]
  Consequence:

  \begin{corollary}
    \label{prop:cluster:no-general}
    No general \(\pvp{\gamma}{v}{\Gamma}\) within cluster.
  \end{corollary}

  \begin{argument}
    Quickly, because of~\ref{prop:cluster:simul}.
    Only \(\gamma\) at same time as others.

    In more detail.
    For, by assumption, \requ{} means that it's possible for the agent to conclude.
    By the each other \requ{} functions as a check on \(\gamma\).
    If haven't figured out each individual, then the general is in question.
  \end{argument}
\end{note}

\begin{note}
  \autoref{prop:cluster:simul} is kind of wild.
  Though, this doesn't prevent reasoning, and then only getting \csN{} after the fact.
  However, this does prevent ruling out conflict when concluding.

  Of course, concluding each and then \csVImp{}, plausible.
  As, have not found an issue.

  Also, may conclude all from premises.
  For, special cases of \cluster{} in which all premises are the same.
  If so, then no clear tension.
  For, agent reasons from premises, well all the same premises.
  So, conclude simultaneously.
\end{note}

\begin{note}
  Indeed, narrow interest to \ragCluster{1}.

  \begin{definition}[\ragCluster{3}]
    For any cluster \(\mathcal{C}\), \(\mathcal{C}\) is a \emph{\ragCluster{}} if and only if:
    \begin{enumerate}
    \item
      There is some \(\pvp{\phi_{i}}{v_{i}}{\Phi_{i}}\) and \(\pvp{\phi_{j}}{v_{j}}{\Phi_{j}}\) such that \(\Phi_{i}\) and \(\Phi_{j}\) do not overlap.
    \end{enumerate}
    \vspace{-\baselineskip}
  \end{definition}

  A \ragCluster{}, then, is just a cluster where at least some distinct premises.
  Hence, avoid issue where same premises allow simultaneous conclusion, and fail to establish tension with \ESU{}.

  \begin{proposition}
    With \ragCluster{} concluded previous or violate \ESU{}.
  \end{proposition}

  Now, some caution.
  It may be the case that reason from some non-minimal collection of premises.
  Hence, some care when establishing \ragged{}.
  This means, argue that \cluster{} \emph{and} argue \cluster{} is \ragged{}.
  Again, without any clear bounds on premises, this argument is non-deductive.
  However, plausible in various cases.
\end{note}

\begin{note}[Relative \jag{1}]
  Given importance of \ragged{} and specific proposition-value-premise pairings of \ragged{}, terminology:

  \begin{definition}[Relative \jag{1} of a \ragCluster{}]
    \(\mathcal{C}\) some \ragCluster{}.
    \(\pvp{\psi}{v'}{\Psi}\) is a \emph{\jag{0}} relative to \(\pvp{\phi}{v}{\Phi}\) if \(\Psi\) differs from \(\Phi\).
  \end{definition}

  \csN{} while presence of some \jag{} when premises are no general.
\end{note}

\begin{note}
  Suppose \ragged{}, no prior conclusion for some \jag{}.
  Then, if \csN{}, violation of \ESU{}.

  For, only \csN{} simultaneously.
  \jag{}, some not from current premises.
  And, no previous, so not from previous premises.
  Hence, \csN{} from premises of the \jag{}.

  Hence, need instances of \ragged{} with no prior conclusion for some \jag{}.
\end{note}

\begin{note}
  \begin{proposition}
    If \ragCluster{} and no prior conclusion for some \jag{}.
    Either:
    \begin{itemize}
    \item
      \ESU{} does not hold in general.
    \item
      No \csVImp{} for any proposition-value-premise pairing in cluster.
    \end{itemize}
      \begin{argument}
    More-or-less immediate from previous.
  \end{argument}
  \end{proposition}
\end{note}

\begin{note}
  Abstract tension, then, follows if there are instances of \ragCluster{1} with no prior conclusion for some \jag{}.
\end{note}

\begin{note}[Pointed cluster]
    \begin{definition}[Pointed cluster]
    For any cluster \(\mathcal{C}\), \(\mathcal{C}\) is a \emph{pointed cluster} if and only if:
    \begin{enumerate}
    \item
      Some conclusions are the same.
    \end{enumerate}
    \vspace{-\baselineskip}
  \end{definition}
\end{note}

\subsection{Concrete tension}
\label{sec:concrete-tension}

\begin{note}
  \color{red}
  The key observation is that ability gives rise to \ragCluster{1}.
  Hence, no \csN{}.

  Why does this lead to tension?
  Well, in certain cases, it seems clear that an agent has the ability, and the ability ensures that the agent would not have failed to conclude.
\end{note}

\begin{note}
  Seen how to develop tension in the abstract.
  Now, concrete.
  Ability, as we have seen.
\end{note}

\begin{note}[Ability]
  Ability to reason in certain ways leads to \cluster{1}.
  However, not interested in ability in general, but rather relatively simple instances of ability concerning specific problem types.

  Arithmetic.
  Sudoku.
  Chess.

  Indeed, the latter pair for \requCluster{1}.
  For, different starting positions.
\end{note}

\begin{note}
  Finally, ability.

  General ability, specific ability.

  Claim support for having some general ability.

  Now, here, simple cases.
  Basic arithmetic.
  Sudoku puzzles.
  Chess problems with winning strategies.

  Roughly the same.
  More broadly:

  Logic problems.

  Crossword.

  Reading novels up to a certain level.
  Here, if you can't read, then the writing is bad.

  Fluency.

  So, specific instances of the general ability.
\end{note}

\begin{note}
  Well, conclude that you have the general ability, but also claim support.
  You don't need to go through specific instances.
  In these cases, fail to be an independent check.
  You not fail to reach the relevant conclusion.

  Would not reason to some incorrect summation.
  Would not fill out the Sudoku incorrectly.
  Would not fail to find a winning strategy.

  Of course, failures of performance, but not failures of competence.

  Intuitively, satisfied all \requ{1}.
\end{note}


\begin{note}
  Key argument, then, is that only satisfy a \requ{} by concluding \(\psi\) from \(\rho\).

  For, some other premise, get \(\psi\) from \(\rho\).
  Well, getting \(\psi\) from \(\rho\) is still a \requ{} for this.
\end{note}

\begin{note}
  Briefly stated.

  Specific instances, these introduce \requ{}.
  However, some reasoning.
  Can't jump to general to get rid of \requ{}, as this is forbidden.
  Further, if reason from some distinct set of premises, then still a \requ{}.

  If independent reasoning gets that specific instance of general ability, then doing the reasoning is still an independent check on this.

  So, the problem here is that need to ensure that would conclude \(\psi\) has value \(v'\) from certain premises.
  If appeal to any distinct premises, then failure to claim support.

  Hence, \ESU{} and \ideaCS{}, then no getting general ability without witnessing reasoning for specific instances.

  Core of the tension.
  Always some independent check with distinct premises with specific instance of general ability.

  So, either, allow to bypass independent check.
  Or, do not require witnessing reasoning from premises to conclusion.
\end{note}

\subsubsection{\adB{}}
\label{sec:tension:adb}

\begin{note}
  Now, these cases of ability.
  Work with \adB{}.
\end{note}

\begin{note}
  Here, it turns out that obtaining the \itp{} is just concluding.
\end{note}

\begin{note}[Conditionals, a point of interest]
  More generally, we have the following result.%
  \footnote{
    So long as we do not add~\autoref{notion:overview:requ:pool:method} to~\autoref{notion:overview:requ:pool} of the notion of a \requ{}.
    If so, then result will be constrained accordingly.
  }
  If appeal to some conditional which links premises to a conclusion, such that agent may reason from premises to conclusion, then the agent has always concluded premises from conclusion.

  This is interesting.
  If agent has concluded from conditional in this way, then in effect the conditional drops out as a premise.

  If \(\Sigma, \phi \rightarrow \psi \vdash \psi\) then \(\Sigma, \phi \vdash \psi\).

  If \(\Sigma, \phi \vdash \phi \rightarrow \psi\) then \(\Sigma, \phi \vdash \psi\).

  The second is close to a restricted form of the deduction theorem.

  Note, this is only when the conditional has a special \requ{}.

  In cases where no checking the conditional, then the elimination does not hold.

  Whether anything of independent interest follows from this, unclear.
  One example, responsibility.
  Then, to another person, not only reasoning with conditional, but also full reasoning.
  No way to distinguish between the two cases.
  Or, from the converse perspective, no need to add any additional clauses to account of responsibility.
  However, issues of this kind are far beyond the scope of this document.
\end{note}

\begin{note}
  Indeed, given the constraints of \itp{1}, conditional will satisfy just in case it is an \itp{}.
  Here, need that \(\pv{\mu}{v}\) is not in \(\Phi\) to ensure that \itp{0} is redundant.
\end{note}

\subsubsection{Thoughts}
\label{sec:thoughts}

\begin{note}[Difficulty]
  Here, concrete instantiation of abstract structure.
  Question is, does the agent really \csN{}?
  Intuitively, it seems agent does.
  Adopt \stance{}, I am confident I would not reason otherwise.
  However, this is from adopting a \stance{}.
  And, whether or not concluded or \csVed{} is independent of \stance{}.
  Instead, question is, granting intuition, does this \stance{} reflect on our theory.

  How close these abstractions map to more commonsense reasoning.
  Motivated in part.
  Whether \csN{} seems natural.

  Seems to me, key question is the conditional link.
  It's identifying these two things.
  Abstracts from specific cases, but gives the general principle.
  So long as \csN{}, then these two things coincide.

  Of course, senses of `concluding'.
  However, difficulty in running `concluding' too close to the agent's \stance{} on what the have concluded.
\end{note}

\subsection{Tension}
\label{sec:overview:tension:subsection}

\subsubsection{A different path to tension}
\label{sec:diff-path-tens}

\begin{note}
  We have developed tension with respect to the definition of a \requ{}.
  However, noted that an additional constraint may be placed on the notion of a \requ{}.
  Same type of reasoning.

  If this is the case, then general ability will not be part of \cluster{}.
  Hence, the argument as given will no go through.

  May seem a compelling alternative.
  There are two issues.

  First, consequence.
  General ability functions as a premise in every case of witnessing a specific instance general ability.
  For, if just specific instance, then part of a \cluster{}.

  One way to resist this is to fine grain specific instances.
  However, this is very puzzling.

  Still, when paired against \EAS{} this may not be so bad.

  Hence, we turn to the more substantial issue.
  \csN{2} for the general ability.

  Need to have \csVed{0}.
  For, else, different conclusion about general ability, and hence no \csVImp{} for specific cases.

  So, \csVImp{} for general ability.
  Here, each specific instance is a \requ{}.
  And, we are back to the original problem.

  So, even if you grating general ability as a premise, still a question as to how this premise is obtained.

  Some wiggle room is left.
  Here, informed that one has the general ability.
  But, then, any attempt to witness the specific ability would not show mistake.
  And, no isolated conclusion.

  These, quite puzzling to defend.
\end{note}


\subsection{Resolving the tension}
\label{sec:overview:resolving-tension}

\begin{note}
  Note, the tension is not about whether \(\phi\) has value \(v\).
  Instead, the tension is about whether the agent would have a certain property if they were to conclude \(\phi\) has value \(v\).
  Property of having claimed \support{}.
  Expanded, property of holding that any independent check is satisfied.
  Any other reasoning about whether \(\phi\) has value \(v\) would conclude \(\phi\) has value \(v\).
\end{note}

\begin{note}
  Returning to \EAS{}.
  Specific instances of the general ability.
  In this sense, the instances of \EAS{} we argue for are narrow.
  Need strong sign that the agent has the general ability.

  Further.
  It does not state that an agent having claimed support that they have the ability to reason to some conclusion is \emph{always permissible} to claim support for the conclusion by appealing to some premises that do not form part of the agent's reasoning.
  Instead, it states that \emph{may be permissible} for the agent claim support in a certain way.

  In various respects, these aren't particularly interesting cases.
  However, the goal is to argue that such cases exist.
  Whether these are constrained to the type of cases we consider for the argument is a further question.

  There may be more interesting cases, but given that \ESU{} is incompatible with all such cases, I see no compelling reason to explore such cases without \emph{first} motivating a rejection of \ESU{}.
\end{note}

\begin{note}
  Also suggests that the content of general ability is somewhat interesting.
  For, the content is itself general.
  It is a conclusion that ranges over all specific instances of the general ability.
\end{note}

\begin{note}[Terminology]
  So, the upshot of this is that an agent concludes various things in certain cases.
  In concluding \(\phi\), also conclude \(\psi,\dots\).
  And, in cases of interest, because of generality of the reasoning.

  This is somewhat puzzling.
  Though, I think less puzzling than first appears.
  Concluding \(\phi\) has value \(v\) is nothing special.
  Of course, the agent only explicitly concludes a handful of things, but allowing the generality is nothing that different from equivalences.

  It also doesn't follow that any of the additional properties of the reasoning, if any, are carried over to any \requ{1}.
  Is just about concluding.
  Here, then, various ways to keep the intuition for the positive answer.
  There may be various things that are exclusive to witnessing reasoning from premises to conclusion.
  However, distinct from concluding.

  Still, stronger than being committed.
  Ranges over any implication.
  Conclude no winning strategy, then also conclude various other chess things.
  However, committed, but do not necessarily conclude that X is going to lose the game.

  Concluding is still of interest.
  Or, as noted, `reason', in the weak term.
\end{note}

\subsubsection{Resolving tension by rejecting ideas}
\label{sec:resolv-tens-reject}

\begin{note}
  Now, possible to resolve tension various ways.
  Reject \ideaCS{}, \csN{} is of no interest.
  Reject \ESU{}, not witnessing is ok.
  Reject claiming support for general ability.

  Or, any combination of the above.

  Interest is in rejecting \ESU{}.
  So long as \(\psi\) having value \(v'\) follows from some premises, then the reasoning doesn't matter.
  \csV{2} for \(\psi\) having value \(v'\) from those premises, given the possibility of witnessing reasoning.
\end{note}

\subsubsection{Resolving tension by additional ideas}
\label{sec:resolv-tens-addit}

\begin{note}[Strong closure]
  So, we have weak constraints on concluding.
  Is there a way to keep \ESU{} by strengthening closure?

  The idea is that \csN{} relies on the possibility of an independent check.
  However, strong closure leaves open the option for denying independence.

  In certain cases, this seems viable.
  Arithmetic.
  Perhaps this does give everything.

  However, the other cases are more challenging.
  For, in these cases, specific premises.

  Chess.
  Winning strategy from board.
  So, then need to conclude would conclude winning strategy from board.
  Now, idea is that understanding basics already give you this.
  Therefore, as the reasoning from the board requires understanding rules, it also follows that before concluding from board, have already concluded winning strategy from board.

  Stepping back, relevant instance of reasoning requires certain premises.
  Possible obtaining those premises already involves concluding various things.
  If some of those conclusions are that one would not fail to conclude \dots
  Then, there is no space for \requ{1} of the relevant kind.

  For, there will be no `gap' between premises and conclusions of interest.
  Hence, no independent check on whether one would get conclusion from premises.

  This is really strong closure though.
  Intuitively, there is some gap between introduction and understanding.
  This is in part why \csN{} is of interest.
  A sufficiently strong closure principle would need to rule out possibility of failing to \csN{} while having the possibility to reason to the correct answer.
  And this, I don't see as genuinely viable.

  And, if not viable, then tension arises when an agent goes from `merely' concluding to also \csN{}.
\end{note}

\subsubsection{Resolving tension by terminology}
\label{sec:resolv-tens-term}

\begin{note}[`Concluding']
  Reject some of the assumptions regarding `concluding'.
  Or, more generally, recast \csN{} in terms of something like commitment.

  Argument has been developed with the terminology of `concluding' as this seems natural.

  Even if not concluding, take the result to be sufficiently interesting.
\end{note}


\subsection{Observations}
\label{sec:overview:observations}

\subsubsection{Ability}

\begin{note}
  \begin{figure}[H]
    \centering
    \saMtxInterpreted{}
    \caption{Distinction matrix with \aben{the}}
    \label{fig:saMtxInterpreted:outline}
  \end{figure}
\end{note}

\begin{note}
  Recap.

  Claiming support.
  Constraint.

  Ability.
  In order to be compatible, satisfy constraint.
  Either of three options.
  Basic, ignore this.
  Property. Incompatible with constraint.
  Witness. Compatible.

  Here, display the matrix.
  I think this is the easiest way to visualise what is going on.
\end{note}

\paragraph{Deviant causal chains}

\begin{note}
  Do these really matter in the case of reasoning?
\end{note}

\paragraph{Closure principles}

\begin{note}
  No doubt, already observed.
  This does lead to a closure principle, constrained by what reasoning it is possible for the agent to witness.

  There are two perspectives.
  First, leading to tension.
  Second, no need to reason.
\end{note}

%%% Local Variables:
%%% mode: latex
%%% TeX-master: "master"
%%% End: