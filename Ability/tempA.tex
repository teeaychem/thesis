\chapter{Temp}
\label{cha:temp}


\section{\pevent{3}}
\label{sec:assupp2}

\begin{note}
  we noted the progressive allows us to capture conclusions which are available to the agent, but not yet made.
  Though, to identify such conclusions via the progressive it must be the conclusion is in progress.
  Still, with the progressive in hand we may define broader modalities in terms of the progressive.
  In particular, we define a \pevent{} as follows:

  \begin{definition}[\pevent{3}]%
    \label{def:potenital-event}%
    \begin{itemize}
    \item
      There is a \emph{\pevent{}} in which \vAgent{} does \(X\).
    \end{itemize}

    \emph{If and only if}:


    \begin{itemize}
    \item
      There is some action \(a\) such that both~\ref{def:PE:action} and~\ref{def:PE:prog} are true:

      \begin{enumerate}[label=\alph*., ref=(\alph*)]
      \item
        \label{def:PE:action}
        \(a\) is an action available to \vAgent{}.
      \item
        \label{def:PE:prog}
        An event in which \vAgent{} does \(X\) is in progress when \vAgent{} does \(a\).
      \end{enumerate}
    \end{itemize}
    \vspace{-\baselineskip}
  \end{definition}

  The role of \pevent{1} is to capture an event which is `possible' in the sense of the progressive without the truth of the progressive.
  Of course, the added modality to move to an event in which the progressive is true, and this modality may be scrutinised.
  Still, I take an `available action' to be sufficiently intuitive.%
  \footnote{
    For background reading, \autoref{def:potenital-event} parallels \citeauthor{Mandelkern:2017aa}'s act conditional analysis of ability, where `practically available' parallels `available' (\citeyear[\S5]{Mandelkern:2017aa}).
    Likewise, consider `options' in \citeauthor{Boylan:2020aa}'s `determinacy' analysis (\citeyear[\S4]{Boylan:2020aa}).
  }
\end{note}

\begin{note}
  To close this section, let's put \pevent{1} to work.

  \begin{illustration}[Darts]
    There is a \pevent{} in which agent wins at darts just in case there is some action available to the agent, such that if the agent were to perform the action they would be winning at darts.

    Winning is a complex action.
  An agent has three dart throws to lower their score from 501 to 0 before play switches to the other player, and play continues until neither player may lower their score further on their next turn (without going past 0).
  Playing a game is a complex action, as the region of a dartboard an agent wishes to hit changes according to previous throws.
  For example, if the agent's score is 51 with three throws remaining, the agent will not wish to hit bullseye, as there is no way to reduce their score by a single point using two darts.
  If the agent goes on to hit 20, then the score of the remaining to darts should equal 31, and so on.
  \end{illustration}

  Note, the initial sequence of actions may be more or less arbitrary.
  It is not possible to score 501 in three or six dart throws, so an agent \emph{could} start by throwing a few darts blindly, so long as they have sufficient skill to recover on subsequent throws.

  Of course, throwing darts is quite different from concluding, but this note extends.
  An agent may be concluding a theorem is true even though their first line of enquiry turns out to be a dead end, etc.

  This is the appeal of the progressive.
  Speaking of \pevent{1} allows us to talk about the way in which an agent may bring about an event without the agent being engaged in the process of bringing the event about.
\end{note}


%%% Local Variables:
%%% mode: latex
%%% TeX-master: "master"
%%% TeX-engine: luatex
%%% End:
