\section{The progressive (Part II)}
\label{cha:sec:fcs-def:progressive}

\begin{note}
  \autoref{cha:sec:fcs-def} we defined \fc{1} in terms of \pevent{1}:

  \vspace{-\baselineskip}
  \begin{quote}
    \definitionForegoneC*
  \end{quote}

  In \autoref{cha:fcs:sec:progressive} introduced the idea of understanding \pevent{1} in terms of the progressive:

  \vspace{-\baselineskip}
  \begin{quote}
    \definitionPEvent*
  \end{quote}

  \autoref{cha:sec:fcs-def:ability} took a slight detour to observe difficulty with linking \fc{1} to ability.
  \begin{itemize}
  \item
    Specific.
  \item
    Ability and the past.
  \item
    Understanding of \AbControl{}.
  \end{itemize}
  \AbControl{} is the key issue.

  Choice of action, similar to act conditional analysis.

  Together with progressive captures specific.
  Just need to be concluding after performing action.
  This doesn't rely on general performance.

  Key difference is with respect to \AbControl{}.
  Consequence of the action is that \(S\) is \(\varphi\)\emph{ing}.
  That \(S\) \(\varphi\)s is not a consequence of the action, only that there is some possible developments (\assuPP{2}).

  In this section, task is to develop understanding of the progressive.
  Consider and revise \citeauthor{Landman:1992wh}'s (\citeyear{Landman:1992wh}) account of the progressive.

  Interest with \citeauthor{Landman:1992wh}'s account is somewhat arbitrary.
  Two key features:
  Goes through the reasoning.
  Minimal background, events and closeness between possible worlds.

  Further, develop understanding by modifying \citeauthor{Landman:1992wh}'s analysis.
\end{note}

\begin{note}
  Layout.
  Three sections:
  \begin{itemize}[noitemsep]
  \item
    Intuition governing \citeauthor{Landman:1992wh}'s account.%
    \hfill\autoref{cha:sec:fcs-def:progressive-landman}
  \item
    Background for and details of \citeauthor{Landman:1992wh}'s account.%
    \hfill \autoref{cha:fcs:sec:Landman:details}
  \item
    Algorithmic (re)construction of \citeauthor{Landman:1992wh}'s account.%
    \hfill \autoref{cha:fcs:sec:Prog:L:Alg}
  \end{itemize}
\end{note}



\subsection[\citeauthor{Landman:1992wh}'s account of the progressive (modified)]{\citeauthor{Landman:1992wh}'s (\citeyear{Landman:1992wh}) account of the progressive (intuition)}
\label{cha:sec:fcs-def:progressive-landman}
\nocite{Portner:1998um}
\nocite{Engelberg:1999vi}

\begin{note}
  In broad summary:%
  \footnote{
    \textcite{Szabo:2004ul}:
  \begin{quote}
    [A] progressive sentence is true at some time just in case some event occurs at that time, and if we follow the development of the event (within our world as long as it goes, then jumping into a nearby world, and iterating the process within the limits of reasonability) we will reach a possible world where the perfective correlate is true of the continuation.%
    \mbox{ }\hfill\mbox{(\citeyear[34]{Szabo:2004ul})}
  \end{quote}
  }
  \citeauthor{Landman:1992wh} holds that an action in the progressive holds of some event just in case the event, if allowed to develop, would develop into an event in which the action is performed.

  As we have seen with the perfective paradox, some action in the progressive need not continue in the actual world, and hence the core of \citeauthor{Landman:1992wh}'s account of the progressive is an account of allowing an event to continue.

  Roughly, on \citeauthor{Landman:1992wh}'s account the way in which an event is allowed to develop is indirectly captured by considering different ways in which the actual world may have been.

  In short, we follow an event through it's development in the actual world until it does not continue any further in the actual world.
  Then, just before the event stops, we jump to the closest `reasonable' world in which the same event happened and continues a little further (if such a world exists) and follow the development of the event in the close world until it does not continue any further.
  This process continues until it is not possible to jump to a `reasonable' world.
\end{note}

\begin{note}
  To see how the above sketch functions in practice, we follow~\citeauthor{Portner:1998um}'s (\citeyear[764--766]{Portner:1998um}) illustration of \citeauthor{Landman:1992wh}'s account.
\end{note}

\begin{note}
  Our interest is with the following sentence:
  \begin{enumerate}
  \item
    \label{prog:max:bad}
    Max is crossing the street.
  \end{enumerate}

  Hence, the relevant action of interest is the action of Max crossing the street.
  Let us fix the event \ref{prog:max:bad} is (assumed to be) true of as \(e\) and fix \(w\) for the world \(e\) happens in.

  Following \citeauthor{Landman:1992wh} (and in line with \assuPP{}), Max is crossing the street is true of \(e\) just in case, if allowed to develop, \(e\) would develop into an event in which Max crosses the street.

  Now, suppose that in \(w\) the event \(e\) does not develop any further.
  Instead (and quite unfortunately), Max is hit by a bus cruising at thirty miles per hour.
  Somewhat ominously, let us identify this bus as `bus \#1'.

  Still, \(e\) does not include Max being hit bus \#1, and had things been a little different, Max may have continued a little further across the street.
  Hence, there is some world \(v\) which is close to \(w\) in which \(e\) develop a little further.

  So far so good, but in \(w\) Max was hit by bus \#1 in \(w\).
  Hence, it seems that \(v\) also involves Max being hit by a bus.
  For, \(v\) is a possible world close to \(w\), and if Max avoids being hit by a bus altogether in \(v\) then there is surely some possible world \(v'\) closer to \(w\) than \(v\).
  So, although Max makes it a little further across the road in \(v\), Max is still hit by a bus.
  We identify the bus in \(v\) as `bus \#2'.

  However, as with bus \#1 in \(w\), it seems Max may have walked a little further across the street in possible worlds close to \(v\).
  Hence, by the same reasoning we may consider some possible world \(u\) close to \(v\) and so on\dots

  \autoref{fig:max-bus} is a modification of \citeauthor{Portner:1998um}'s figure 1. (\citeyear[767]{Portner:1998um})
  \begin{figure}[!h]
    \centering
    \begin{tikzpicture}
      \tikzmath{
        % x positions
        \x1 = 11;
        \xb1 = 2/9*\x1; \xb2 = 4/9*\x1; \xb3 = 6/9*\x1;
        % y positions
        \y1 = 2/5*\x1; \ymid = 1/2*\y1;
        \yw1 = \y1; \yw2 = 1/2*\y1; \yw3 = 0*\y1; \yb2 = 1/5*\y1;
        % event e
        \xe = 1/2*\xb1; \yediff = \yw2 - \yb2;
        \ye = \yw2 - 1/2*\yediff;
        \enudge = .1;
        \xel = 0; \xer = \xb1; \yen = \yw2 - \enudge;
        % bus 1 description location
        \xbx = 1.5/9*\x1; \xby = 4/5*\y1;
        % bus 2 description location
        \xb9 = 2.5/9*\x1;
      }
      % Paths
      \draw[line width=0.25mm, line cap=round] (\xb1,\ymid) -- (\xb3,\yw1); % world 1
      \draw[line width=1mm, line cap=round, dash pattern=on 175pt off 5pt on 5pt off 5pt on 5pt off 5pt on 5pt off 5pt on 5pt off 5pt on 5pt off 5pt on 5pt off 5pt on 5pt off 5pt on 5pt off 5pt on 5pt off 5pt on 5pt off 5pt on 5pt off 5pt on 5pt off 5pt on 5pt off 5pt on 5pt off 5pt on 5pt off 5pt on 5pt off 5pt on 5pt off 5pt on 5pt off 5pt on 5pt off 5pt] (0,\ymid) -- (\xb1,\ymid) -- (\xb2,\yb2) -- (\xb3,\yw2); % world 2
      \draw[line width=0.25mm, line cap=round] (\xb2,\yb2) -- (\xb3,\yw3); % world 3
      % World descriptions
      \filldraw[black] (\xb3,\yw1) circle (0pt) node[anchor=west, align=left]{world 1: Max hit by \\ bus \# 1};
      \filldraw[black] (\xb3,\yw2) circle (0pt) node[anchor=west, align=left]{world \(i\): Max \\ crosses street};
      \filldraw[black] (\xb3,\yw3) circle (0pt) node[anchor=west, align=left]{world 2: Max hit by \\ bus \# 2};
      % Event
      \draw[] (\xe,\ye) -- (\xel,\yen); % event l
      \draw[] (\xe,\ye) -- (\xer,\yen); % event r
      % Event description
      \filldraw[black] (\xe,\ye) circle (0pt) node[anchor=north, align=left]{event e};
      % Splits
      \filldraw[black, dashed] (\xbx,\xby) circle (0pt) node[anchor=south, align=left]{bus \#1 hits Max};
      \filldraw[black, dashed] (\xb9,\yw3) circle (0pt) node[anchor=north, align=left]{bus \#2 hits Max};
      % Split descriptions
      \draw[-Stealth, dashed] (\xbx,\xby) -- (\xb1,\ymid + \enudge); % bus 2 arrow
      \draw[-Stealth, dashed] (\xb9,\yw3) -- (\xb2 - \enudge,\yb2 - \enudge); % bus 2 arrow
    \end{tikzpicture}
    \caption{
      Continuation path of `Max was crossing the street'. \\
    }
    \label{fig:max-bus}
  \end{figure}

  The thick black line captures \(e\) as \(e\) is allowed to develop, and the changes in angle reflect shifts to alternative possible worlds.

  The dashed line indicates that Max may need to avoids being hit by additional busses in order for \(e\) to develop into an event in which Max crosses the road.

  Still, does \(e\) develop into an event in which Max crosses the road?
  If Max is hit by a bus in \(w\), then surely Max is hit by a bus in all the possible worlds close to \(w\).

  Here we turn to the key property of \citeauthor{Landman:1992wh}'s account which we have so far made without explicit comment:
  Prior to bus \#2 hitting Max in \(v\), we shift to \(u\) where \(u\) is a world which is close to \(v\) rather than \(w\).
  Hence, as the development of \(e\) develops, closeness is understood relative to the development of \(e\), rather than \(e\) itself as \(e\) happened in \(w\).
  So, as Max progresses a little further each time a possible world in which Max crosses the street gets a little closer until, eventually, Max crosses the street.

  This is the core idea of \citeauthor{Landman:1992wh}'s account.
  To borrow a piece of terminology from \textcite{Dowty:1979vq}, \(e\) has sufficient \emph{inertia} to develop in some possible world \(v\), and as \(e\) develops in \(v\), inertia continues to build until Max crosses the road.

  However, an important restriction is placed on shifts to possible worlds.
  Intuitively, Max does not avoid being hit by bus \#\(j\) because Max has the strength to stop as moving bus.
  Yet, a possible world in which Max has the strength to stop a moving bus may be close to the world in which Max is not hit by bus \#\(j - 1\).
  In \citeauthor{Landman:1992wh}'s terminology, the relevant possible worlds in which \(e\) develops must be `reasonable'.
  We will return to `reasonableness' in \autoref{cha:fcs:sec:Prog:L:Alg:branches}.
\end{note}

\subsection[\citeauthor{Landman:1992wh}~(\citeyear{Landman:1992wh})]{\citeauthor{Landman:1992wh}'s (\citeyear{Landman:1992wh}) account in detail}
\label{cha:fcs:sec:Landman:details}

\begin{note}
  \citeauthor{Landman:1992wh}'s account of the progressive:

  \begin{quote}
    \(\sem{\text{PROG}(e, P)}_{w,g} = 1\) iff \(\exists f \exists v\colon \langle f,v \rangle \in \text{CON}(g(e), w)\)\newline
    \phantom{an} and \(\sem{P}_{v,g}(f) = 1\)\par

    where \(\text{CON}(g(e), w)\) is the continuation branch of \(g(e)\) in \(w\).\newline
    \mbox{ }\hfill\mbox{(\citeyear[27]{Landman:1992wh})}
  \end{quote}
  Account of continuation branch.
\end{note}


\begin{note}
  Immediate goal is to present \citeauthor{Landman:1992wh}'s account of a continuation branch.
  In turn, this will require expansion on three further points.
  Stages of an event.
  Continuations and stops with respect to events.
  Reasonable options.

  With understanding in hand algorithmic reconstruction of continuation branch.
  Expand in two ways.
  First, tree, to allow for forks.
  Second, a different account of how to identify branches.
\end{note}

\subsubsection{Continuation branch}

\begin{note}
  \citeauthor{Landman:1992wh}'s account of a continuation branch is as follows:
  \begin{quote}
    The \emph{continuation branch} for \(e\) in \(w\) is the smallest set of pairs of events and worlds such that
    \begin{enumerate}
    \item
      \label{Landman:CB:continues}
      for every event \(f\) in \(w\) such that \(e\) is a stage of \(\langle f,w \rangle \in C(e,w)\);
      the continuation stretch of \(e\) in \(w\);
    \item
      \label{Landman:CB:stops}
      if the continuation stretch of \(e\) in \(w\) stops in \(w\), it has a maximal element \(f\) and \(f\) stops in \(w\).
      Consider the closest world \(v\) where \(f\) does not stop:
      \begin{enumerate}[label=--]
      \item
        if \(v\) is not in \(\lRwe{w}{e}\), the continuation branch stops.
      \item
        if \(v\) is in \(\lRwe{w}{e}\), then \(\langle f,v \rangle \in C(e,w)\).
        In this case, we repeat the construction:
      \end{enumerate}
    \item
      \label{Landman:CB:continues:again}
      for every \(g\) in \(v\) such that \(f\) is a stage of \(g\), \(\langle g,v \rangle \in C(e,w)\), the continuation stretch of \(e\) in \(v\);
    \item
      \label{Landman:CB:stops:again}
      if the continuation stretch of \(e\) in \(v\) stops, we look at the closest world \(z\) where its maximal element \(g\) does not stop:
      \begin{enumerate}[label=--]
      \item
        if \(z\) is not in \(\lRwe{w}{e}\), the continuation branch stops.
      \item
        if \(z\) is in \(\lRwe{w}{e}\), then \(\langle g,z \rangle \in C(e,w)\) and we continue as above, etc.%
        \mbox{ }\hfill\mbox{(\citeyear[26--27]{Landman:1992wh})}
      \end{enumerate}
    \end{enumerate}
  \end{quote}

  Describes the process of following an event \(e\) and a world \(w\) and jumping to nearby reasonable worlds when \(e\) stops in \(w\) (or \(w'\), etc.).%
  \footnote{
    More general point.

    Whether it is possible to capture by event alone is unclear.
    As \textcite[1256]{Portner:2011vi} observes:
    \begin{enumerate}[label=\arabic*., ref=(\arabic*)]
    \item
      \label{Portner:reD:a}
      Max is crossing the street.
    \end{enumerate}
    May be re-described
    \begin{enumerate}[label=\arabic*\('\)., ref=(\arabic*\('\))]
    \item
      \label{Portner:reD:b}
      Max is walking into the path of an oncoming bus.
    \end{enumerate}
    However, it is not possible for these two sentences to both be true of the same event given \citeauthor{Landman:1992wh}'s account of the progressive.
    For,~\ref{Portner:reD:a} requires a continuation branch in which Max crosses the street while~\ref{Portner:reD:b} requires a continuation branch in which Max walks into the path of an oncoming bus --- it seems Max crosses the street if and only if Max does not walk into the path of an oncoming bus.

    \citeauthor{Landman:1992wh} considers introducing the notion of a perspective (\citeyear[30--31]{Landman:1992wh}) to capture how an event is singled out.
    Strictly, then, the event identified such that~\ref{Portner:reD:a} is true need not be the same event identified such that~\ref{Portner:reD:b} is true.
    For example, \ref{Portner:reD:a} may be true of the event of Max is walking across the road that does not include the bus, while~\ref{Portner:reD:a} is true of the larger event which includes the bus.

    Still, how events are singled out is external to the construction of a continuation branch, which takes an event as an argument.
    Therefore, we set this worry aside.
  }

  Observe, the construction of a continuation branch is iterative.
  Clauses~\ref{Landman:CB:continues:again} and~\ref{Landman:CB:stops:again} and duplicates of Clauses~\ref{Landman:CB:continues} and~\ref{Landman:CB:stops} shifted to \(g\) --- some development of \(e\) --- in some possible world \(v\).

  Reconstruction via a recursive algorithm.
  For the moment we leave \citeauthor{Landman:1992wh}'s account of a continuation branch without commentary.
\end{note}

\subsubsection{Stages}

\begin{note}
  An event being a stage of some other event.
  Clause~\ref{Landman:CB:continues} (and~\ref{Landman:CB:continues:again}).

  \citeauthor{Landman:1992wh}'s definition is light:
  \begin{quote}
    An event is a stage of another event if the second can be regarded as a more developed version of the first, that is, if we can point at it and say, ``It's the same event in a further stage of development.''\newline
    \mbox{ }\hfill\mbox{(\citeyear[23]{Landman:1992wh})}
  \end{quote}
  \citeauthor{Landman:1992wh}'s definition is difficult.

  We may ignore the qualification `can be regarded as' to avoid defining a stage of an event from an agent's perspective.
  However, this reduces the question of what as stage is to what it is for some event to be a more developed version of some other event.
  Indeed, \citeauthor{Landman:1992wh}'s paraphrase is particularly troubling because our statements about the development of an event may be (and likely are) informed by whether we think various instances of the progressive are true of an event.
  And, if so we will require an independent account of the progressive to develop \citeauthor{Landman:1992wh}'s account of the progressive.

  Unfortunately I have no positive contribution to this problem.
  However, some contrasts may help:%
  \footnote{
    Unfortunately \citeauthor{Landman:1992wh} does not provide any further discussion.
  }

  \begin{quote}
    Hanny Schaft's acts of resistence stopped when she was murdered before the end of the Second World War; they are part of the Second World War but it extends beyond them.
    \dots
    This is where stages come in: we cannot say that when an event stops in a world, there is no bigger event of which it is part in that world, but we can say that when it stops, there is no bigger event in the world of which it is a \emph{stage}[.]%
    \mbox{ }\hfill\mbox{(\citeyear[23]{Landman:1992wh})}
  \end{quote}

  \begin{itemize}
  \item
    The event in which I make breakfast.
    \begin{itemize}[noitemsep]
    \item
      The event in which I place some butter on a slice toast is a stage.
    \item
      The event in which I hear the rumble of the refrigerator is not a stage.
    \end{itemize}
  \end{itemize}

  \begin{itemize}
  \item
    The event in which I arrive at work.
    \begin{itemize}[noitemsep]
    \item
      The event in which I get on my bike is stage.
    \item
      The event in which I eat breakfast is not a stage.
    \end{itemize}
  \end{itemize}

  \begin{itemize}
  \item
    The event in which I order a meal.
    \begin{itemize}[noitemsep]
    \item
      The event in which I open a menu is a stage.
    \item
      The event in which I return the menu is not a stage.
    \end{itemize}
  \end{itemize}

  We will treat the notion of one event being a stage of another as a primitive.
  Further insights into this primitive will be obtained from how it is put to use.
\end{note}

\subsubsection{Continuations and Stops}

\begin{note}[Continuations and Stops]
  The primitive of a stage is important for defining both the continuation and when an event stops.
  We present \citeauthor{Landman:1992wh}'s definition of both terms, and then provide restated definitions.
\end{note}

\begin{note}
  \citeauthor{Landman:1992wh} combines the definition of a continuation and when an event stops:
  \begin{quote}
    This is where stages come in: we cannot say that when an event stops in a world, there is no bigger event of which it is part in that world, but we can say that when it stops, there is no bigger event in the world of which it is a \emph{stage}:
    \begin{enumerate}[label=, noitemsep]
    \item
      Let \(e\) be an event that goes on at \(i\) in \(w\).
      Let \(f\) be an event that goes on at \(j\) in where \(i\) is a subinterval of \(j\).
    \item
      \(j\) is a continuation of \(e\) iff \(e\) is a stage of \(f\).
    \item
      Let \(j\) be a non-final interval.
    \item
      \(f\) stops at \(j\) in \(w\) iff no event of which \(f\) is a stage goes on beyond \(i\) in \(w\) (i.e., at a later ending interval).\newline
      \mbox{ }\hfill\mbox{(\citeyear[23--24]{Landman:1992wh})}
    \end{enumerate}
  \end{quote}

  To clarify the definitions, we borrow the relevant definitions regarding intervals from \textcite{Dowty:1979vq}:

  \begin{quote}
    \(I\) is a subinterval of \(J\) iff \(I \subseteq J\), where \(I\) and \(J\) are intervals.
    \(I\) is a proper subinterval of \(J\) iff \(I \subset J\).
    \(I\) is an \emph{initial subinterval} of \(J\) iff \(I\) is a subinterval of \(J\) and there is no \(t \in (J - I)\) for which there is \(t' \in I\) such that \(t \leq t'\).
    \emph{Final subinterval} is defined similarly\dots\newline
    \mbox{ }\hfill\mbox{(\citeyear[140]{Dowty:1979vq})}
  \end{quote}
  So, following \citeauthor{Dowty:1979vq}:
  \begin{quote}
    \(I\) is a \emph{final subinterval} of \(J\) iff \(I\) is a subinterval of \(J\) and there is no \(t \in (J - I)\) for which there is \(t' \in I\) such that \(t' \leq t\).
  \end{quote}
  And, from the definition of a non-final subinterval follows similarly:
  \begin{quote}
    \(I\) is a \emph{non-final subinterval} of \(J\) iff \(I\) is a subinterval of \(J\) and there is \emph{some} \(t \in (J - I)\) for which there is \(t' \in I\) such that \(t' \leq t\).
  \end{quote}
  Intuitively, then, \(I\) is a \emph{non}-final interval of \(J\), just in case \(I \subseteq J\) and \(J\) progresses further in time than \(I\).

  Let us now turn to restating the definitions of a continuation and a stop.
  We take each in turn.
\end{note}

\begin{note}
  The definition of \(f\) being a continuation of \(e\) requires two things:
  First, it must be the case the event at which \(e\) takes place is a subinterval of the event at which \(f\) takes place, and it must be the case that \(e\) is a stage of \(f\).

  In full:

  \begin{definition}[Continuations]
    \label{def:Landman:conts}
    For events \(e\) and \(f\):
    \begin{itemize}
    \item \(f\) is a \emph{continuation} of \(e\)
    \end{itemize}
    \emph{if and only if}
    \begin{itemize}
    \item
      The following jointly hold:
      \begin{enumerate}[label=\alph*., ref=(\alph*)]
      \item
        \label{def:Landman:conts:interval}
        \begin{enumerate}
        \item[\emph{If}:]
          \begin{enumerate}[label=\roman*.]
          \item
            \(i\) is the interval at which \(e\) takes place in \(w\).
          \end{enumerate}
        \item[\emph{And}:]
          \begin{enumerate}[label=\roman*., resume]
          \item
            \(j\) is the interval at which \(f\) takes place.
          \end{enumerate}
        \item[\emph{Then}:]
          \begin{enumerate}[label=\roman*., resume]
          \item
            \(i\) is a subinterval of \(j\).
          \end{enumerate}
        \end{enumerate}
      \item
        \label{def:Landman:conts:stage}
        \(e\) is a stage of \(f\).
      \end{enumerate}
    \end{itemize}
    \vspace{-\baselineskip}
  \end{definition}

  Inclusion is important.
  For, consider `finishing the book'.
  What we need is an event in which `finishes the book'.
  Whether or not \(e\) itself is part of the event in which X finishes the book matters.
  Likewise, Max is reaching the other side of the street.
  False when Max starts walking.
  However, true when Max is almost at the other side of the street.
\end{note}

\begin{note}
  The definition of \(f\) stopping at \(j\) in \(w\) takes a little more work.

  There are two immediate issues with \citeauthor{Landman:1992wh}'s definition.

  First, the definition requires that \(j\) is a non-final interval.
  But, of what?
  By assumption \(i\) is a subinterval of \(j\), so \(j\) cannot be a non-final interval of \(i\).%
  \footnote{
    The term `non-final interval' only appears in the above quote from \citeauthor{Landman:1992wh}.
  }

  Second, there must be no event of which \(f\) is a stage that goes beyond \(i\) in \(w\).
  However, by assumption \(f\) is an event that goes on at \(j\), and \(i\) is a subinterval of \(j\).
  Yet, why is \(f\) bound by some arbitrary interval \(i\)?

  I propose to resolve both issues with the following definition:
  \begin{definition}[Stops]
    \label{def:Landman:Stops}
    For an event \(e\), interval \(i\), and world \(w\):
    \begin{itemize}
    \item
      \(e\) \emph{stops} at \(i\) in \(w\)
    \end{itemize}
    \emph{If and only if:}
    \begin{itemize}
    \item
      There no is interval \(j\) nor event \(f\) such that the following jointly hold:
      \begin{enumerate}[label=\alph*., noitemsep]
      \item
        \(i\) is non-final subinterval of \(j\).
      \item
        \(f\) goes on at \(j\).
      \item
        \(f\) is a stage of \(e\).
      \end{enumerate}
    \end{itemize}
    \vspace{-\baselineskip}
  \end{definition}
  In short, there is no expansion from \(i\) \emph{forward} in time to obtain an interval \(j\) such that an event which is a stage of \(e\) got on at \(j\).

  Hence, \autoref{def:Landman:Stops} captures \citeauthor{Landman:1992wh}'s initial gloss: `[W]hen [\(e\)] stops, there is no bigger event in the world of which [\(e\)] is a \emph{stage}' (\citeyear[23]{Landman:1992wh})
\end{note}

\begin{note}[Example of \autoref{def:Landman:Stops}]
  Interesting case to consider.
  Time limits.
  Running the race in record time.
  But, falls short of the record.

  Now, it may be false.
  Or, it may be true.
  Stops, no longer possible to complete race in record time.
  Shift to possible world.
  However, stops when event is still in progress.
  So, shift two possible world at time when stop will be completing race in record time.
\end{note}

\subsubsection{\(\lRwe{w}{e}\)}

\begin{note}
  Finally, we turn to \citeauthor{Landman:1992wh}'s account of the set `\(\lRwe{w}{e}\)'.
  From role in definition, constrain possible worlds in some way and closest world must be a member of the set.
  The exact details are fuzzy.

  \citeauthor{Landman:1992wh} defines the set as follows:

  \begin{quote}
    \label{def:LandRwe}
    \(v \in \lRwe{w}{e}\) iff there is a reasonable chance on the basis of what is internal to \(e\) in \(e\) that \(e\) continues in \(w\) as far as it does in \(v\).%
    \mbox{ }\hfill\mbox{(\citeyear[26]{Landman:1992wh})}
  \end{quote}

  Look at possible worlds.
  Consider event \(e\) in possible world \(v\).
  Consider \(e\) in actual world.
  If there is a reasonable chance that \(e\) continues in \(w\) as in \(v\), then \(v\) is reasonable, on the basis of what is `internal' to \(e\).

  Not at all clear.
  However, significant insight from role in \citeauthor{Landman:1992wh}'s account of the progressive.
  \autoref{cha:fcs:sec:Prog:L:Alg:branches} considers in detail.

  Briefly, however, `reasonable' is intuitively epistemic, but need not be.
\end{note}

\subsection{An algorithmic variant}
\label{cha:fcs:sec:Prog:L:Alg}
\nocite{Cormen:2009uw}

\begin{note}
  However, following proposal by \citeauthor{Szabo:2004ul}, variant account of the progressive which allows for continuation trees (as opposed to branches).

  \begin{quote}
    \begin{enumerate}[label=(\Roman*), ref=(\Roman*)]
      \setcounter{enumi}{5}
    \item
      \emph{Prog}[\(\varphi\)] is true at \(t\) in \(w\) iff there is an \(e\) at \(t\) in \(w\) and for every \(\langle e^{\ast}, w^{\ast} \rangle\) on the continuation tree for \(e\) in \(w\) if \(\varphi\) is not true of \(e^{\ast}\) at \(w^{\ast}\) then there is an \(\langle e', w' \rangle\) on the continuation tree for \(e\) in \(w\) such that \(e'\) is a continuation of \(e^{\ast}\) in \(w'\) and \(\varphi\) is true of \(e'\) at \(w'\).%
      \mbox{ }\hfill\mbox{(\citeyear[37]{Szabo:2004ul})}
    \end{enumerate}
  \end{quote}

  Continuation tree, as there may be no unique continuation branch for \(e\).
  \citeauthor{Szabo:2004ul}'s (proposed) definition covers possibility, and requires that for every way event develops, always part of some branch or has been true.
  I.e.\ that while false, there is some (possible) future where event is true.

  Note, continuation tree is generalisation of continuation branch.
  Hence, \autoref{fig:max-bus} does not represent a continuation tree.

  Indeed, \citeauthor{Szabo:2004ul}'s (proposed) definition is equivalent to \citeauthor{Landman:1992wh}'s definition if `continuation branch' is substituted for `continuation tree', for the relevant continuations will be limited to a single branch.

  Still, not simply adopting \citeauthor{Szabo:2004ul}'s (proposed) definition.
  Significant changes to continuation tree.
\end{note}

\begin{note}
  Breaking this down into a recursive algorithm.
  Goal is to create a tree, which will be a set of ordered event-world-pairing pairs indexed according to depth.
  For example:
  \[\text{Tree} = \{\langle \langle e,w \rangle_{1}, \langle f,w \rangle_{1} \rangle, \langle \langle f,w \rangle_{1}, \langle g,v \rangle_{2} \rangle, \langle \langle f,w \rangle_{1}, \langle g,v' \rangle_{2} \rangle, \dots \}\]

  Root, initial event world pair, and then branch from event world pair to some distinct event world pair when index changes.%
  \footnote{
    Indexing is required construct the tree as allow events to develop in different ways.
    However, also important for keeping track of the stage of construction.
  }

  We start by (re)constructing three basic algorithms.
  Following, we introduce a variation to one of these algorithms.
  And, finally, we (re)construct a recursive algorithm to build a continuation branch of some event-world pairing.

  \begin{itemize}[noitemsep]
  \item
    \AlgAC{(\(\langle g,u \rangle_{i}\))}%
    \hfill%
    \autoref{cha:fcs:sec:Prog:L:Alg:conts}
    \begin{itemize}
    \item
      An algorithm to obtain the continuation stretch of some event \(g\) in world \(u\).

      Builds on Clause~\ref{Landman:CB:continues} (and~\ref{Landman:CB:continues:again}).

      Expanded to help following algorithms.
    \end{itemize}
  \item
    \AlgGetStops{(\(\text{Continuations}\))}%
    \hfill%
    \autoref{cha:fcs:sec:Prog:L:Alg:stops}
    \begin{itemize}
    \item
      An algorithm to find the stopping points in some set of continuation stretches.

      Builds on the antecedent of Clause~\ref{Landman:CB:stops} (and~\ref{Landman:CB:stops:again})

      Follows \citeauthor{Landman:1992wh}, though will be revised.
    \end{itemize}
  \item
    \AlgFindBranches{(\(\langle \langle f,v \rangle_{i-1}, \langle g,u \rangle_{i}\rangle, e, w\))}%
    \hfill%
    \autoref{cha:fcs:sec:Prog:L:Alg:branches}
    \begin{itemize}
    \item
      An algorithm to identify alternative worlds in which \(g\) happens.

      Builds on the \emph{consequent} of Clause~\ref{Landman:CB:stops} (and~\ref{Landman:CB:stops:again})

      Adjusted from \citeauthor{Landman:1992wh}'s suggestions in two significant ways:
      \begin{itemize}[noitemsep]
      \item
        First, to allow multiple branches.
      \item
        Second, to resolve some issues with \citeauthor{Landman:1992wh}'s account.
      \end{itemize}
    \end{itemize}
  \end{itemize}
  Combined:
  \begin{itemize}
  \item
    \AlgDevelopTree{(\(\text{Tree},e,w,n\))}%
    \hfill%
    \autoref{cha:fcs:sec:Prog:L:Alg:tree}
    \begin{itemize}
    \item
      Recursive algorithm to build tree.

      Unifies \AlgAC{}, \AlgGetStops{}, and \AlgFindBranches{}.

      Explicitly recasts the iteration implicit in \citeauthor{Landman:1992wh}'s definition of a continuation branch in terms of recursion.
    \end{itemize}
  \end{itemize}
  Alternative to \AlgGetStops{}:
  \begin{itemize}
  \item
    \AlgGetPStops{(\(\text{Continuations}, e, w\))}%
    \hfill%
    \autoref{cha:fcs:sec:Prog:L:Alg:R-stops}
    \begin{itemize}
    \item
      Variation of \AlgGetStops{}.

      Considers `reasonable' ways in which \(g\) may have stopped.

      Ensures the continuation branch of a progressive does entail the progressive is true because the agent will \(\alpha\) in the actual world.
    \end{itemize}
  \end{itemize}
\end{note}

\subsubsection{Continuations}
\label{cha:fcs:sec:Prog:L:Alg:conts}

\begin{note}[\AlgAC{}]
  Clause~\ref{Landman:CB:continues} (and~\ref{Landman:CB:continues:again}) of \citeauthor{Landman:1992wh}'s definition of a continuation branch, which characterises the idea of a `continuation stretch' of some event \(g\) in world \(u\).
  We term the algorithm `\AlgAC{}':

  \begin{algorithm}[H]
    \label{PrAl:g-a-c}
    \caption{\AlgAC{}}
    \SetAlgoLined
    \DontPrintSemicolon
    \Input{\(\langle g,u \rangle_{i}\) \hfill An (indexed) event-world pairing}
    \KwResult{Continuation \hfill The continuation of \(g\) in \(u\)}
    \Begin{
      \(\text{Continuation} \longleftarrow \emptyset\)\;
      \label{PrAl:g-a-c:CSetInt}
      \(t_{s} \longleftarrow \text{start time of }g\text{ in }u\)\;
      \label{PrAl:g-a-c:ts}
      \(t_{e} \longleftarrow \text{end time of }g\text{ in }u\)\;
      \label{PrAl:g-a-c:te}
      \(g_{x} \longleftarrow g\)\;
      \label{PrAl:g-a-c:gx}
      \While{\(g_{x}\) is an event in \(u\)}
      {
        \label{PrAl:g-a-c:while:s}
        \(g_{x} \longleftarrow \emptyset\)\;
        \label{PrAl:g-a-c:gx:discard}
        \(t_{e} \longleftarrow t_{e} + 1\)\;
        \label{PrAl:g-a-c:te:plus}
        \(I \longleftarrow [t_{s},t_{e}]\)\;
        \label{PrAl:g-a-c:te:I}
        \For{\(g_{y} \in \{g_{y} \mid g_{y} \text{ is an event in } u\}\)}{
          \label{PrAl:g-a-c:for:s}
          \If{\(g_{y}\) is a stage of \(g\)}
          {
            \label{PrAl:g-a-c:for:test}
            \(\text{Continuation} \longleftarrow \text{Continuation} \cup \langle \langle g_{x},u \rangle_{i}, \langle g_{y},u \rangle_{i} \rangle\)\;
            \label{PrAl:g-a-c:C:new}
            \(g_{x} \longleftarrow g_{y}\)\;
            \label{PrAl:g-a-c:gx:new}
          }
        }
      }
      \label{PrAl:g-a-c:while:e}
      \Return{\(\text{Continuation}\)}
      \label{PrAl:g-a-c:return}
    }
  \end{algorithm}

  \AlgAC{} takes an event-world pairing \(\langle g,u \rangle_{i}\) and returns a set containing a (non-branching tree) which captures the continuation stretch of \(g\) in \(u\).

  Intuitively, \AlgAC{} starts with \(\langle g,u \rangle_{i}\).
  Then, \AlgAC{} finds the smallest event \(g^{+}\) such that \(g^{+}\) is a stage of \(g\) in \(u\), and adds \(\langle g,u \rangle_{i}, \langle g^{+},u \rangle_{i} \rangle\) as a continuation.
  Now, \(g\) may develop further in \(u\).
  So, \AlgAC{} continues to the smallest \(g^{++}\) such that \(g^{++}\) is a stage of \(g\) in \(u\) at a later time than \(g^{+}\), and adds \(\langle g^{+},u \rangle_{i}, \langle g^{++},u \rangle_{i} \rangle\) as a continuation.
  The hypothetical set \(\text{Continuation}\) is now \(\{\langle g,u \rangle_{i}, \langle g^{+},u \rangle_{i} \rangle, \langle g^{+},u \rangle_{i}, \langle g^{++},u \rangle_{i} \rangle\}\).

  This process repeats until there are no further stages of \(g\) in \(u\).
\end{note}

\begin{note}[Motivation for \AlgAC{}]
  % There are two pieces of motivation for the construction of \AlgAC{}.

  The construction of \AlgAC{} is follows \citeauthor{Landman:1992wh}'s definition of a continuation.
  % and \citeauthor{Landman:1992wh}'s truth condition for the progressive.

  \citeauthor{Landman:1992wh}'s definition of a continuation requires that for any continuation \(g^{+}\) of \(g\), there is a subinterval \(J\) of the interval \(I\) over which \(g^{+}\) happens such that \(g\) happens at \(J\).

  % Likewise, \(\sem{\text{PROG}(e, P)}_{w,g}\) checks to see if there exists \emph{some} event-world pairing \(\langle f,v \rangle\) in the continuation branch of \(e\) (in \(w\)) such that \(\sem{(f, P)}_{v,g}\) is true.
  % And, in general it may be the case that \(g\) fails to be the relevant \(f\) either because \(g\) does not continue sufficiently far or \(g\) exceeds \(f\).
  % To illustrate:
  % If Max has not yet crossed the road, then \(g\) is not an event in which Max crosses the road.
  % And, if Max has crossed the road and gone to get a coffee, then \(g\) is not (clearly) an event in which Max crosses the road.
\end{note}

\begin{note}[Construction of \AlgAC{}]
  Finally, then, we have the way in which \(\text{Continuation}\) is constructed.

  We start by initialising \(\text{Continuation}\) as an empty set (\autoref{PrAl:g-a-c:CSetInt}).
  Then, we identify the start and end times of the interval in which \(g\) takes place (Lines~\ref{PrAl:g-a-c:ts} and~\ref{PrAl:g-a-c:te}).

  The task is then to continue to expand \(\text{Continuation}\) so long as there are further stages of \(g\) in \(u\).
  We achieve this a while loop %
  (Lines~\ref{PrAl:g-a-c:while:s}--\ref{PrAl:g-a-c:while:e}) %
  that will fail when we are no longer considering an event in \(u\).
  The variable event \(g_{x}\) is initially set to \(g\), to guaranteed at least one pass through the loop (\autoref{PrAl:g-a-c:gx}).

  In a instance of the while loop we first discard \(g_{x}\) to ensure the loop will fail if there are no further stages of \(g\) (\autoref{PrAl:g-a-c:gx:discard}).
  Then, we construct an interval \(I\) by shifting forward one step in time (Lines~\ref{PrAl:g-a-c:te:plus}~and~\ref{PrAl:g-a-c:te:I}).

  At this point, we have a new interval \(I\) to consider, but no immediate event.
  So, we consider every event \(g_{y}\) which occurs in \(u\) and test to see if \(g_{y}\) happens in \(I\) \emph{and} is a stage of \(g\) (Lines~\ref{PrAl:g-a-c:for:s}~and~\ref{PrAl:g-a-c:for:test}).
  If successful, we update \(\text{Continuation}\) and \(g_{x}\) (Lines~\ref{PrAl:g-a-c:C:new}~and~\ref{PrAl:g-a-c:gx:new}).

  Here single assumption:
  \begin{itemize}[noitemsep]
  \item
    If there is some further stage \(g_{z}\) of \(g_{x}\), then there is stage \(g_{y}\) of \(g_{x}\) at an interval obtained by stepping one tick forward in time.%
  \footnote{
    This may be avoided by separating the test on~\autoref{PrAl:g-a-c:for:test} into two separate tests, and shifting the reassignment on~\autoref{PrAl:g-a-c:gx:new} outside the scope of the if-clause.
    Though the while-loop will then only terminate when there are no further events in \(u\).
  }
  \end{itemize}

  After the stages of \(g\) in \(u\) have been exhausted, \AlgAC{} returns \(\text{Continuation}\) (\autoref{PrAl:g-a-c:return}) and terminates.
\end{note}

\subsubsection{Stops}
\label{cha:fcs:sec:Prog:L:Alg:stops}

\begin{note}[\AlgGetStops{}]
  \AlgAC{} captures Clause~\ref{Landman:CB:continues} (and~\ref{Landman:CB:continues:again}) of \citeauthor{Landman:1992wh}'s definition of a continuation branch.
  Given some event in the actual world, the continuation branch of the event captures the event as it develops in the actual world.
  The \emph{antecedent} of Clause~\ref{Landman:CB:stops} (and~\ref{Landman:CB:stops:again}) of \citeauthor{Landman:1992wh}'s definition queries whether the continuation stretch of \(g\) stops in \(u\), and considers the `maximal' event \(g'\) in \(u\) (if \(g\) stops).
  We capture \emph{antecedent} of Clause~\ref{Landman:CB:stops} (and~\ref{Landman:CB:stops:again}) via the algorithm `\AlgGetStops{}'.

  Intuitively, \AlgGetStops{} searches through some continuation stretch of \(g\) in \(i\) (i.e.\ the result of \AlgAC{\((\langle g,u \rangle_{i})\)}) to identify the stopping point of \(g\) in \(u\).
  Hence, if \(g\) stops in \(u\) \AlgGetStops{} returns the relevant `maximal' event.
  Otherwise, \AlgGetStops{} does not return an event-world pairing.
  So, by inspecting the result of \AlgGetStops{} we may determine whether the antecedent of Clause~\ref{Landman:CB:stops} (and~\ref{Landman:CB:stops:again}) is fulfilled.

  Strictly, the construction of \AlgGetStops{} is generalised to cover finding the stopping points of multiple continuation stretches, but the same intuition extends to the more general case.

  After identifying the relevant stopping points of an event \(g\) in world \(u\), the following task will be to find continuations of \(g\) in other worlds.

  \AlgGetStops{} is as follows:

  \begin{algorithm}[H]
    \label{PrAl:g-s}
    \caption{\AlgGetStops{}}
    \SetAlgoLined
    \DontPrintSemicolon
    \Input{\(\text{Continuations}\) \hfill I.e.\ results of \AlgAC{} --- in general, a tree}
    \KwResult{Stops \hfill Stopping points for each event in \(\text{Continuations}\)}
    \Begin{
      \(\text{Stops} \longleftarrow \emptyset\)\;
      \label{PrAl:g-s:mk-st-Stops}
      \For{\(\langle \langle f,v \rangle_{i-1}, \langle g,u \rangle_{i} \rangle \in \text{Continuation}\)}
      {
        \label{PrAl:g-s:for:start}
        \If{\(g\text{ stops in }u\)}
        {
          \label{PrAl:g-s:if:start}
          \(\text{Stops} \longleftarrow \text{Stops} \cup \{\langle \langle f,v \rangle_{i-1}, \langle g,u \rangle_{i} \rangle\}\)\;
          \label{PrAl:g-s:make}
        }
      }
      \Return{\(\text{Stops}\)}
      \label{PrAl:g-s:return}
    }
  \end{algorithm}
\end{note}

\begin{note}[Construction of \AlgGetStops{}]
  The construction of \AlgGetStops{} is simple:%
  \footnote{
    \AlgGetStops{} differs in presentation from \citeauthor{Landman:1992wh}.
    Recall:
    \begin{quote}
      \begin{enumerate}[label=\arabic*., ref=(\arabic*)]
        \setcounter{enumi}{1}
      \item
        if the continuation stretch of \(e\) in \(w\) stops in \(w\), it has a maximal element \(f\) and \(f\) stops in \(w\).
        Consider the closest world \(v\) where \(f\) does not stop:\dots
      \end{enumerate}
    \end{quote}
    Clause~\ref{Landman:CB:stops} reads as an imperative to find the relevant maximal event.
    Still, so long as \AlgGetStops{} is applied to the result of an instance of \AlgAC{}, the result is equivalent --- the set of continuation given by \AlgAC{} will include the relevant maximal event.
    For, an event does not continue if it stops, hence any stopping event is maximal.
  }

  \AlgGetStops{} initialises \(\text{Stops}\) as an empty set (\autoref{PrAl:g-s:mk-st-Stops}), and for every element of \(\text{Continuations}\), \AlgGetStops{} takes the right-event-world pairing and queries whether the event stops in the world (Lines~\ref{PrAl:g-s:for:start}--\ref{PrAl:g-s:if:start}).
  If the event stops, \AlgGetStops{} takes the point at which the event stops and adds the element of \(\text{Continuations}\)%
  % with the maximal continuation substituted for the event
  to \(\text{Stops}\) (\autoref{PrAl:g-s:make}).
  \AlgGetStops{} returns \(\text{Stops}\) (\autoref{PrAl:g-s:return}) and terminates.
\end{note}

\begin{note}
  Applied to a single instance of \AlgAC{}, \AlgGetStops{} will either return an empty set or a singleton set containing a pairing in which the event of the right element stops.
  And, applied to a set containing multiple results of \AlgAC{}, \AlgGetStops{} will either return an empty set or a set of size bounded by the instances of \AlgAC{}.
\end{note}

\begin{note}
  Role of \AlgGetStops{} is somewhat artificial.
  It would be far more effective to identify a stopping event in the construction of \AlgAC{}.

  However, after introducing \AlgFindBranches{} to find the continuations of a stopping event in some possible world (\autoref{cha:fcs:sec:Prog:L:Alg:branches}) we will introduce variant of \AlgGetStops{} (\autoref{cha:fcs:sec:Prog:L:Alg:R-stops}).
\end{note}

\subsubsection{Finding branches}
\label{cha:fcs:sec:Prog:L:Alg:branches}

\begin{note}
  The combination of \AlgAC{} and \AlgGetStops{} allows us to capture Clauses ~\ref{Landman:CB:continues} and~\ref{Landman:CB:stops} (also~\ref{Landman:CB:continues:again} and~\ref{Landman:CB:stops:again}) of \citeauthor{Landman:1992wh}'s definition of a continuation branch.
  Following \citeauthor{Landman:1992wh}'s definition, the \emph{consequent} of Clause~\ref{Landman:CB:stops} (and~\ref{Landman:CB:stops:again}) requires identifying the continuation of an event \(g\) which stops in \(u\) in some possible world \(u'\), if the continuation exists.

  The algorithm `\AlgFindBranches{}' is designed to identify the possible worlds in which an event exits.
  The continuation of the event may then be obtained via \AlgAC{}.
  Hence, the primary use-case is applying \AlgFindBranches{} to the result of \AlgGetStops{}.
  However, we will also use \AlgFindBranches{} to provide a variant of \AlgGetStops{}.
\end{note}

\begin{note}[Omen]
  In contrast to \AlgAC{} and \AlgGetStops{}, our discussion of \AlgFindBranches{} will be somewhat long.
  There are two primary reasons:

  First, the way in which a continuation tree (or branch) branches is the core of the account of the progressive favoured by \citeauthor{Landman:1992wh} and ourselves.
  The mechanisms involved are delicate, and some care will be given to motivating how branches are identified.

  Second, our goal is to develop a sufficient understanding of progressive rather than to provide an account of the progressive and part of \citeauthor{Landman:1992wh}'s account (the idea of a reasonable option) is unfortunately underdeveloped.
  Hence, we will provide an alternative with the primary goal of clarifying the relevant considerations.
\end{note}

\begin{note}[Division]
  Given the detail, divide the present section into further sections.

  \begin{itemize}[noitemsep]
  \item
    The algorithm \AlgFindBranches{}.%
    \hfill\autoref{cha:fcs:sec:Prog:L:Alg:branches:alg}
    \begin{itemize}
    \item
      We begin by stating \AlgFindBranches{} for reference in the discussion to follow.
    \end{itemize}
  \item
    \drift{2}~I.%
    \hfill\autoref{cha:fcs:sec:Prog:L:Alg:branches:role}
    \begin{itemize}
    \item
      An overview of the (primary) role of \AlgFindBranches{}.

      In short, help identify the continuation \(g\) of some initial event in \(e\) in world \(w\) such that, though \(e\) stops in \(w\), \(g\) is an event in some possible world \(u\).
      Term this process `\drift{}'.
      Hence, role of \AlgFindBranches{} is to capture \drift{0}.

      An \emph{instance} of \AlgFindBranches{} helps by taking some event-world pairing \(\langle f,v \rangle\) and returning event-world pairings \(\langle f,v' \rangle\) such that \(f\) happens in \(v'\) (and is a continuation of \(e\) in \(v'\)).

      Hence, multiple instances of \AlgFindBranches{} paired with \AlgAC{} (to develop \(f\) in \(v'\)) and \AlgGetStops{} (to determine which events to branch from) to achieve primary role.

      How branches are identified is key to understanding progressive.
      For, how we obtain event in which the perfective correlate of the progressive is true.
    \end{itemize}
  \item Growing a tree.%
    \hfill\autoref{cha:fcs:sec:Prog:L:Alg:branches:tree}
    \begin{itemize}
    \item
      A minor change to \citeauthor{Landman:1992wh}' definition to allow for a continuation \emph{tree} (as opposed to branch).
    \end{itemize}
  \item
    \drift{2}~II.%
    \hfill\autoref{cha:fcs:sec:Prog:L:Alg:branches:drift}
    \begin{itemize}
    \item
      The final section, and the most involved.
      Ensuring that branching results in \drAdj{0} continuations of \(e\) in \(w\).
      \begin{itemize}
      \item
        Close worlds%
        \hfill\autoref{cha:fcs:sec:Prog:L:Alg:branches:drift:close}
        \begin{itemize}
        \item
          Follows \citeauthor{Landman:1992wh}.
        \end{itemize}
      \item
        Problem of \ndrAdj{} \drift{0}.
      \item
        \(\lRweSym{}\) and \(\mRweSym{}\)%
        \hfill\autoref{cha:fcs:sec:Prog:L:Alg:branches:drift:reasonable}
        \begin{itemize}
        \item
          \(\lRweSym{}\) is \citeauthor{Landman:1992wh}.
          \(\mRweSym{}\) is alternative we develop.
        \end{itemize}
      \end{itemize}
    \end{itemize}
  \end{itemize}
\end{note}

\paragraph{The algorithm}
\label{cha:fcs:sec:Prog:L:Alg:branches:alg}

\begin{note}[\AlgFindBranches{} alg]
  \AlgFindBranches{} is as follows:

  \begin{algorithm}[H]
    \label{PrAl:find-branches}
    \caption{\AlgFindBranches{}}
    \SetAlgoLined
    \DontPrintSemicolon
    \Input{\(\langle \langle f,v \rangle_{i-1}, \langle g,u \rangle_{i}\rangle\) \hfill An element of a tree\\
      \(e\) \hfill An event \\
      \(w\) \hfill A world \\
    }
    \KwResult{Branches \hfill A set containing event-world pairs \\
      \hfill such that \(g\) is a stage of \(e\) in \(u'\)}
    \Begin{
      \textcolor{comment}{\texttt{//} \(\text{CloseWorlds} \longleftarrow \{u' \mid u' \text{ is the closest world to } u\}\)}\;
      \textcolor{comment}{\texttt{//} \(\text{CloseWorlds} \longleftarrow \text{CloseWorlds} \cap \lRwe{w}{e}\)}\;
      \label{PrAl:find-branches:R}
      \label{PrAl:find-branches:close:Landman}
      \(\text{CloseWorlds} \longleftarrow \{u' \mid u' \text{ is among the closest worlds to } u\}\)\;
      \label{PrAl:find-branches:close}
      \(\text{Branches} \longleftarrow \emptyset\)\;
      \For{\(u' \in \text{CloseWorlds}\)}{
        \label{PrAl:find-branches:loop:start}
        \If{\(g\) is an event in \(u'\)}{
          % Get the continuation of \(g\)\;
          % Loop through, ensuring that branch looks good\;
          % this is implicit in revised R\;
          \If{\(\mRwe{\langle e,w \rangle}{\langle g,u' \rangle}\)}
          {
            \label{PrAl:find-branches:Rprime:check}
            \(\text{Branches} \longleftarrow \text{Branches} \cup \{ \langle \langle f,v \rangle_{i-1}, \langle g,u' \rangle_{i+1}\rangle \}\)\;
          }
          \label{PrAl:find-branches:Rprime:check:end}
        }
      }
      \label{PrAl:find-branches:loop:end}
      \Return{\(\text{Branches}\)}
    }
  \end{algorithm}
\end{note}

\begin{note}
  In broad outline, the function of \AlgFindBranches{} is to consider an event-world pair \(\langle g,u \rangle\) such that \(g\) is the continuation of some initial event \(e\) in world \(w\) and return event-world pairings \(\langle g,u' \rangle\) such that the same event \(g\) is a continuation of \(e\) in \(u'\) (rather than \(u\)).
  In particular, the primary use case for \AlgFindBranches{} is when the event \(g\) stops in \(u\).
  For, it may be the case that, while \(g\) stops in \(u\), \(g\) does not stop in some \closeW{1} to \(u\).%
  \footnote{
    Recall, if an event \(g\) stops in some world \(u\), then \(g\) happens in \(u\) but \(g\) does not continue further in \(u\).
  }
  In particular, if the result of \AlgFindBranches{} is given as an argument to \AlgAC{} then the result will be a collection of ways in which \(g\) continues in \closeW{1} to \(u\).
\end{note}

\begin{note}
  \AlgFindBranches{} parallels \citeauthor{Landman:1992wh}'s general approach to branching.
  However, there is a significant difference in detail:
  \citeauthor{Landman:1992wh} captures branching by considering possible worlds which are both close to \(u\) and elements of \(\lRwe{w}{e}\).
  \AlgFindBranches{}, by contrast, captures branching by considering possible worlds that are close to \(u\) and events which are elements of \(\mRweSym{}\).%
  \footnote{
    \label{fn:Alg:branches:getLandman}
    The following changes to \AlgFindBranches{} may be made obtain \citeauthor{Landman:1992wh}'s account:
    \begin{itemize}[noitemsep]
    \item
      Uncomment \autoref{PrAl:find-branches:close:Landman} and comment \autoref{PrAl:find-branches:close} to retain \citeauthor{Landman:1992wh}'s assumption of a (unique) closest world to \(u\).
    \item
      Uncomment Line~\ref{PrAl:find-branches:R}, to restrict the possible worlds via \citeauthor{Landman:1992wh}'s \(\lRwe{w}{e}\).
    \item
      Comment Lines~\ref{PrAl:find-branches:Rprime:check} and~\ref{PrAl:find-branches:Rprime:check:end} to remove the restriction of events via our \(\mRweSym{}\).
    \end{itemize}
    \vspace{-\baselineskip}
  }
\end{note}

\begin{note}
  Observe, element of tree, event \(e\) and world \(w\) as arguments.
  Intended, \(e\) is event for which determining whether progressive is true, and \(w\) is the world in which \(e\) takes place.
  In general, \(e\) and \(w\) may be arbitrary.
  However, we have no interest in calling instances of \AlgFindBranches{} with unintended arguments.
  Hence, in the following discussion we will use \(e\) and \(w\) to refer to initial event-world pairing.
  Still, \AlgFindBranches{} applied after continuation tree has been constructed.
  Hence, \(g\) and \(u\).
\end{note}

\begin{note}
  Here, shift the index.
  When run recursive algorithm, each call of the algorithm will explore branches.
  Increase index for next call.
\end{note}

\paragraph{\drift{2} I}
\label{cha:fcs:sec:Prog:L:Alg:branches:role}

\begin{note}
  The primary role of \AlgFindBranches{} is to enable us to consider how some initial event \(e\) in world \(w\) develops even though \(e\) stops in \(w\).

  For, account is premised on \assuPP{}.
  If progressive is true of some event, then there is some possible event in which perfective correlate is true.
  And, possible event need not take place in the world in which \(e\) happens.

  Delicacy is that \AlgFindBranches{} only considers a single event-world pairing, and so repeated calls to \AlgFindBranches{} need to be kept in mind.

  Term the process `\emph{\drift{0}}'.
  There are two important parts to \drift{0}:
   \begin{enumerate}[label=\Roman*., ref=(\Roman*), noitemsep]
  \item
    \label{drift:part:I}
    Identify possible worlds \(u'\) close to \(u\) so that some development \(g\) of \(e\) in some possible world \(u\) may continue in \(u'\).
  \item
    \label{drift:part:II}
    Ensure that the way in which \(g\) comes about in \(u'\) is \drAdj{} with respect to the initial event \(e\) in the world of origin \(w\).
  \end{enumerate}

  In this section our focus will be on Part~\ref{drift:part:I}.
  In short, our goal is to provide a basic overview of how \drift{0} is achieved via repeated calls to \AlgFindBranches{}.
  We will return to Part~\ref{drift:part:II} in \autoref{cha:fcs:sec:Prog:L:Alg:branches:drift}.
\end{note}

\begin{note}
  Balance, \drift{} so that move away from the initial world, but retain sufficient proximity that way event develops is \drAdj{}.
\end{note}

\begin{note}
  Let us recall the observation from \autoref{cha:fcs:sec:progressive} and Max.

  The general structure of the following scenario will be familiar from our initial discussion of the progressive in \autoref{cha:fcs:sec:progressive} and \citeauthor{Portner:1998um}'s scenario of Max walking across the road given on~\autopageref{cha:sec:fcs-def:progressive-landman} to illustrate \citeauthor{Landman:1992wh}'s account of the progressive:%
  \footnote{
    The \scen{0} is also a variation on \citeauthor{Schwarz:2020aa}'s (\citeyear{Schwarz:2020aa}) scenario on \autopageref{Schwarz:pi} used to motivate \AbControl{}.
  }

  \begin{scenario}[Saying \(\pi\)]
    \label{scen:prog:Cyril:know}
    Cyril has been asked to state the first ten digits (of the decimal expansion) of \(\pi\).
    Cyril knows the first ten digits of \(\pi\).
    Cyril says `\(3\)' followed by `\(1\)'.
    However, interrupted a bird flying into the classroom through the open window.
  \end{scenario}
  Intuitively, as Cyril says `\(1\)' it is true that:
  \begin{enumerate}[label=\arabic*., ref=(\arabic*), series=CyrilProg]
  \item
    \label{Cyril:pi:progressive}
    Cyril is saying the first ten digits of \(\pi\).
  \end{enumerate}

  Consider \(e\) and \(w\) such that Cyril has said `\(1\)' and has not yet been interrupted by the bird.

  Given our understanding of the progressive, if \ref{Cyril:pi:progressive} is true, then Cyril would have gone on to say `\(4\)' followed by `\(2\)', etc.\ -- if Cyril has not been interrupted by the bird flying into the classroom.
  And, as Cyril knows the first ten digits of \(\pi\) in \autoref{scen:prog:Cyril:know}, let us assume \ref{Cyril:pi:progressive} is true.

  First call to \AlgFindBranches{}.
  \closeW{3} to \(w\) such that \(e\) happens in \(w\).
  I.e.\ collection of \(\langle e,v \rangle\) where Cyril has said `\(1\)' in \(v\).
  Consider \(\langle e,v \rangle\).
  \AlgAC{} to obtain \(f,v\).
  Cyril has gone on to say `\(4\)' in \(v\).

  Assume second bird just after Cyril says `\(4\)'.
  Hence, second call to \AlgFindBranches{}, this time \(\langle f,v \rangle\).
  Consider \(\langle f,u \rangle\).
  \AlgAC{} to obtain \(g,u\).
  Cyril has gone on to say `\(2\)' in \(u\).

  Term this process `\emph{\drift{}}'.
\end{note}

\begin{note}
  Likewise, consider the following variation on \autoref{scen:prog:Cyril:know}:
  \begin{scenario}[Not saying \(\pi\)]
    \label{scen:prog:Cyril:know:not}
    Cyril has been asked to state the first ten digits (of the decimal expansion) of \(\pi\).
    Cyril does \emph{not} know the first ten digits of \(\pi\).
    However, Cyril picks digits at random and says `\(3\)' followed by `\(1\)'.
    After saying `\(1\)' a bird flies into the classroom through an open window and interrupts Cyril.
  \end{scenario}

  \autoref{scen:prog:Cyril:know:not} mirrors \autoref{scen:prog:Cyril:know} with the substitution that Cyril does \emph{not} know the first ten digits of \(\pi\) and instead guesses digits at random.

  \drift{}, however in this case leads to the progressive being false.

  There is no guarantee that Cyril states the first ten digits of \(\pi\).

  For, it seems the worlds in which Cyril says `\(n\)' for \(0 \leq n \leq 9\) are all equally close to \(w\).
  Therefore, there is a \closeW{0} in which Cyril says `\(4\)'.
  Likewise, \(u\).
  Therefore, \closeW{0} where Cyril says `\(2\)'.

  Hence, though we may \drift{0} through a sequence of \closeW{1} until Cyril says the first ten digits of \(\pi\) we may also \drift{0} through a sequence of \closeW{1} where Cyril does not say the first ten digits of \(\pi\).%
  \footnote{
    Started with random guess.
    What matters is that correct is at least as good as any other digit.

    And, indeed any finite sequence of digits from (the decimal representation of) \(\pi\).
  }

  Still, with some care in expression, state something true.
  \begin{enumerate}[label=\arabic*., ref=(\arabic*), resume*=CyrilProg]
  \item
    Cyril is randomly guessing what the first ten digits of \(\pi\) are.
  \end{enumerate}
  For this, do not need it to be the case that Cyril says the first ten digits.
  Any ten digits is a guess.
\end{note}

\begin{note}
  Note, as \drifting{}, it need not be the case that world \drift{0} to is close with respect to \(w\).
  Cyril is interrupted in \(w\), and so close worlds may only delay bird flying though the open window slightly.
  Though, by drifting extend delay so that Cyril completes saying first ten digits before bird flies though the open window.

  Given observation, introduce two pieces of shorthand.
  \begin{itemize}
  \item
    \(v\) is a \emph{\driftW{}} relative to an event \(e\) and world \(w\) if \(v\) is a possible world which results from the process of \drifting{} from \(e\) in \(w\).
  \item
    \(v\) is a \emph{\closeW{}} relative to \(u\) if \(v\) is a possible world which is close to \(u\).
  \end{itemize}
\end{note}

\paragraph{Growing a tree}
\label{cha:fcs:sec:Prog:L:Alg:branches:tree}

\begin{note}
  Before turning to how \AlgFindBranches{} helps capture \drift{} we observe a minor change to \citeauthor{Landman:1992wh}'s account of a continuation branch which allows the possibility of a continuation \emph{tree}.

  The change is minor:
  We consider the set of \closeW{1}, in contrast to following \citeauthor{Landman:1992wh} who assumes the existence of a unique \emph{closest}-world.%
  \footnote{
    This change occurs on \autoref{PrAl:find-branches:close}, and may be reverted by removing (or commenting) \autoref{PrAl:find-branches:close} and un-commenting \autoref{PrAl:find-branches:close:Landman}, which follows \citeauthor{Landman:1992wh}'s assumption.
  }\(^{,}\)%
  \footnote{
    \citeauthor{Landman:1992wh} states that the assumption of a closest-world is made for `ease' (\citeyear[26]{Landman:1992wh}).
    However, \citeauthor{Landman:1992wh} does not clarify what is simplified by the assumption.
  }

  Observe this minor change gives rise to the possibility of a continuation \emph{tree} as opposed to a continuation branch.
  For, suppose \(g\) happens in \(u\).
  Then, it may be the case that there are multiple \closeW{1} \(u'_{1},\dots,u_{j},\dots\) to \(u\) in which the \(g\) (also) happens.

  Indeed, \autoref{scen:prog:Cyril:know:not} is one such case.
  For, the worlds in which Cyril says `\(n\)' for \(0 \leq n \leq 9\) seem equally close given Cyril is randomly guessing digits of \(\pi\).

  Further, multiple \closeW{1} may have some role in understanding the progressive with respect to \autoref{scen:prog:Cyril:know}.
  For, while it is the case that Cyril knows the first ten digits of \(\pi\) and so will say the following digit of \(\pi\) in any \closeW{0}, Cyril may be interrupted in distinct --- but equally close --- ways.
  For example, two different birds may be competing to fly through the open window.
\end{note}

\begin{note}
  Further, whether or not an instance of the progressive is true turns on possibility of distinct branches.
  Consider the following:%
  \footnote{
    A variation on the kind of example developed by \citeauthor{Bonomi:1997uq} (\citeyear{Bonomi:1997uq}) in \citeauthor{Bonomi:1997uq}'s discussion of the `multiple-choice paradox' (\citeyear[\S4]{Bonomi:1997uq}).
    (See also \cite[37]{Szabo:2004ul})
  }
  \begin{enumerate}[label=\arabic*., ref=(\arabic*), resume*=CyrilProg]
  \item
    \label{sen:Cyril:pi-tau:choice}
    Cyril is deciding whether or not to say some initial digits \(\pi\) or \(\tau\).%
    \footnote{
      \(\pi\) is the ratio between the circumference and diameter of a circle.
      \(\tau\) is the ratio between a circumference and radius of a circle.
    }
  \end{enumerate}
  Intuitively, the truth of~\ref{sen:Cyril:pi-tau:choice} does not entail the truth of~\ref{sen:Cyril:pi-tau:choice:pi} nor the truth of~\ref{sen:Cyril:pi-tau:choice:tau}:

  \begin{enumerate}[label=\arabic*., ref=(\arabic*), resume*=CyrilProg]
  \item
    \label{sen:Cyril:pi-tau:choice:pi}
    Cyril is deciding to say some initial digits \(\pi\).
  \item
    \label{sen:Cyril:pi-tau:choice:tau}
    Cyril is deciding to say some initial digits \(\tau\).
  \end{enumerate}

  However, given progressive perfection, and the existence of a unique \closeW{0}.
  From progressive perfection, it follows that the event of Cyril \emph{deciding} develops into an event in which Cyril \emph{decides}.
  Further, as we have assumed the existence of a unique \closeW{0}, the event of Cyril deciding develops \emph{either} into a decision to say some initial digits \(\pi\) or to say some initial digits \(\tau\) (or both).
  Assume without loss of generality that Cyril decides to say some initial digits \(\tau\).
  Then, as the event of Cyril is deciding whether or not to say some initial digits \(\pi\) or \(\tau\) develops into an event in which Cyril decides to say some initial digits \(\tau\), it follows that the event of Cyril deciding whether or not to say some initial digits \(\pi\) or \(\tau\) is also an event of Cyril deciding to say some initial digits \(\tau\).

  Given a continuation tree (rather than branch), problematic entailments such as the above are avoided.%
  \footnote{
    Grating that Cyril is interrupted given \AlgGetStops{}, and even if Cyril is interrupted given the variation to \AlgGetStops{} we develop in \autoref{cha:fcs:sec:Prog:L:Alg:R-stops}.
  }
  For, so long as Cyril decides in all \driftW{1}, then \ref{sen:Cyril:pi-tau:choice} is true.
  And, so long as Cyril decides to say some initial digits \(\tau\) in some \driftW{1}, \ref{sen:Cyril:pi-tau:choice:pi} will be false.
  And, likewise, so long as Cyril decides to say some initial digits \(\pi\) in some \driftW{1}, \ref{sen:Cyril:pi-tau:choice:tau} will likewise be false.
\end{note}

\begin{note}[Moving on]
  With the possibility of a continuation tree in hand, we return to the process of \drifting{}.
\end{note}

\paragraph{\drift{2} II}
\label{cha:fcs:sec:Prog:L:Alg:branches:drift}

\begin{note}
  We now return to \drift{}.
  Recall, in \autoref{cha:fcs:sec:Prog:L:Alg:branches:role} we outline two parts to \drift{0}:
  \begin{enumerate}[label=\Roman*., ref=(\Roman*), noitemsep]
  \item
    Identify \closeW{1} \(u'\) so that some development \(g\) of \(e\) in some possible world \(u\) may continue in \(u'\).
  \item
    Ensure that the way in which \(g\) comes about in \(u'\) is \drAdj{} with respect to the initial event \(e\) in the world of origin \(w\).
  \end{enumerate}

  \autoref{cha:fcs:sec:Prog:L:Alg:branches:role} covered Part~\ref{drift:part:I}.
  In this section we focus on Part~\ref{drift:part:II}.

  First observe how \closeW{1} provide a partial solution.
  Then, consider \citeauthor{Landman:1992wh}'s set \(\lRweSym{}\) and our relation \(\mRweSym{}\).
\end{note}

\subparagraph{\closeW{3}}
\label{cha:fcs:sec:Prog:L:Alg:branches:drift:close}

\begin{note}
  As observed in \autoref{cha:fcs:sec:Prog:L:Alg:branches:role}.
  Consider \closeW{1}, but \closeW{1} relative to the world drifted to.
  Partially solves two difficulties:

  First, there may be no \closeW{0} \(u\) to \(w\) such that \(e\) continues in \(u\).
  To illustrate, consider \autoref{scen:prog:Cyril:know}.
  Cyril was interrupted after saying the first two digits of \(\pi\) by a bird flying into the classroom.
  However, we assumed that there was a \closeW{0} in which the bird did not interrupt Cyril prior to Cyril saying the first three digits of \(\pi\).
  However, it is not clear that there is a \closeW{0} in which Cyril says the first ten digits of \(\pi\).
  Indeed, Cyril was interrupted in \(w\), so it seems plausible that worlds in which Cyril is quickly interrupted are closer to \(w\) than worlds in which Cyril says the first ten digits of \(\pi\) without interruption.
  Hence, at best shifting to a \closeW{0} may have only delayed the interruption of the bird by a moment.
  However, granting that the interruption of the bird was delayed in the \closeW{0} \(u\) it seems there may a \closeW{0} \(u'\) to \(u\) (rather than \(w\)) in which the interruption of the bird is delayed further, and so on until Cyril says the first ten digits of \(\pi\).%
  \footnote{
    For a clear discussion of the difficulties closeness between worlds see \citeauthor{Fine:1975tj}'s (\citeyear{Fine:1975tj}) notice on \citeauthor{Lewis:1973th}'s (\citeyear{Lewis:1973th}) account of counterfactuals.

    See also \citeauthor{Veltman:2005tj}'s (\citeyear{Veltman:2005tj}) revision to~\citeauthor{Tichy:1976tp}'s (\citeyear{Tichy:1976tp}) counterexample to \citeauthor{Lewis:1979vm}' (\citeyear{Lewis:1973th,Lewis:1979vm}) theory of counterfactuals for additional concerns about how closeness between worlds functions.
    Note, however, that the construction of a continuation tree does not appeal to counterfactuals, hence it is unclear whether holding  fixed laws and certain matters of fact (as suggested by \citeauthor{Veltman:2005tj}) may be incorporated into construction.
  }

  To summarise.
  Cyril is interrupted in \(w\) before saying the first ten digits of \(\pi\).
  Hence, it seems that in all the \closeW{1} to \(w\) Cyril may be likewise interrupted before saying the first ten digits of \(\pi\), even if there is some \closeW{} where Cyril says an additional digit of \(\pi\).
  However, if we consider the evaluation of closeness relative to worlds in which Cyril continues, then as Cyril continues it need not be the case that Cyril is interrupted before saying the first ten digits of \(\pi\).
\end{note}

\begin{note}
  Second, whether development is \drAdj{}.
  Possible world where Max is not hit by the bus because birds do not exist.
  Start at \closeW{1} to \(w\), then progressive will not be true due to such possibilities.
  Further, intuitive that progressive could not be true due to such possibilities.
  For, not close such that event happens, but close simpliciter.
  And, event fails to happen is closer than any such possibility.
\end{note}

\begin{note}
  Closeness is difficult to specify, but sufficiently common tool that this does not impact sufficient understanding of the progressive.
  Intuitively, not possible to \drift{0} to non-existence of birds, as no \closeW{0} in which birds exist a little less than they do in the actual world to get the process of \drifting{0} stated.
\end{note}

\subparagraph{The problem of \ndrAdj{} \drift{0}}
\label{drift:problem}


\begin{note}
  Partial, because between worlds alone does not quite \drift{0}.
  For, \closeW{1}, doesn't matter how continuation happens in the world.
  Only that event happens in world.
  Hence, avoid non-existence of birds, but way in which \(e\) develops may still failure to align with intuitive instances of the progressive.
\end{note}

\begin{note}
  In this section, problem of \ndrAdj{} \drift{0}.
  Difficulty raised by detaching closeness from \(w\), world of origin.

  Strictly, two problems.
  false-positives and false-negatives.
  Progressive is false, but branches on a continuation tree.
  Progressive is true, but perfective correlate fails to hold on some branch of the continuation tree.

  False-positives follows \citeauthor{Landman:1992wh}.
  False-negatives is novel, and assumes continuation tree, rather than branch.
\end{note}

\subparagraph*{False-positives}

\begin{note}
  Instance of the problem follows \citeauthor{Landman:1992wh}.%
  \footnote{
    See \citeauthor{Landman:1992wh} (\citeyear[\S2.3]{Landman:1992wh}) for discussion.
  }

  \begin{quote}
    (20) Mary was wiping out the Roman army.

    The situation is that Mary, a person of moderate physical capacities, is battling the Roman army. She manages to kill a couple of soldiers before she gets killed.
    (20) is clearly false in this situation.%
    \mbox{ }\hfill\mbox{(\citeyear[18]{Landman:1992wh})}
  \end{quote}

  The difficulty with Mary is that it seems the continuation tree parallels Max crossing the road or Cyril saying the first ten digits of \(\pi\).

  In particular, always looking for some continuation.
\end{note}


\subparagraph*{False-negatives}

\begin{note}
  Given that Cyril knows the first ten digits of \(\pi\) in \autoref{scen:prog:Cyril:know} the result of \drifting{} through \closeW{1} returns a continuation of \(e\).

  For, in \(w\) Cyril knows the first ten digits of \(\pi\).
  So, as \drift{} so that Cyril says first ten digits of \(\pi\) all \driftW{1} are such that Cyril knows the first ten digits of \(\pi\).
  Hence, no compelling motivation to assume that Cyril says the first ten digits of \(\pi\) is not the result of Cyril applying their knowledge in the relevant \driftW{0}, just as Cyril saying `\(3\)' and then `\(1\)' is the result of Cyril applying their knowledge in the actual world \(w\) before Cyril is interrupted by a bird.
  In short, it seems any \driftW{0} in which Cyril says the first ten digits of \(\pi\) is a continuation of \(e\).
\end{note}

\begin{note}
  Consider once again:
  \begin{enumerate}[label=\arabic*., ref=(\arabic*), resume*=CyrilProg]
  \item
    \label{Cyril:pi:progressive:random}
    Cyril is randomly guessing what the first ten digits of \(\pi\) are.
  \end{enumerate}
  \ref{Cyril:pi:progressive:random} seems true with respect to~\autoref{scen:prog:Cyril:know:not}.

  Now, consider as Cyril gets digits correct.
  As progress, it seems \closeW{1} are such that Cyril knows the first ten digits of \(\pi\).
  Closest world in which Cyril states first ten digits, Cyril knows.

  Of course, in these \closeW{1}, Cyril did not start by guessing randomly.
  However, comes to know.
  So, while saying, Cyril receives information.
  Hence, after some digits, remaining digits are the result of knowledge rather than random guesses.

  So some world where Cyril has started guessing randomly for initial sequence but remainder of the sequence is not a guess.

  Whether or not this particular instance strikes you as compelling, the general issue should be clear.
  Possible for an event to develop in such a way that way in which event happens diverges from initial stages.
  Clearly happens in the actual world.
  Start walking a some particular route on a hike, but as the day progresses you find the weather is pleasant and you're feeling energetic and switch to a some other, longer or more challenging, route.

  Note, in particular, it does not seem viable to say that world in which Cyril is handed a note with the first ten digits of \(\pi\) is not a continuation of \(e\).
  For, although Cyril may stops guessing and this would serve as a suitable stopping event, it is by no means clear how to do so without already granting that Cyril is randomly guessing.

  Going to \closeW{1} compounds issue, by introducing additional changes to how things are which are not necessarily expected from the world in which the event originates.

  For \ref{Cyril:pi:progressive:random} to be true, it needs to be the case that in all the \closeW{1} to \(w_{9}\) in which Cyril (correctly) says the tenth digit (\(3\)), Cyril does so by randomly guessing.
  Now, Cyril started by guessing randomly.
  However, we have continued to \drift{0} a significant way from the actual world and Cyril has correctly said the first ten digits of \(\pi\).
  Hence, it is not clear that Cyril guesses randomly in all the \closeW{1} to \(w_{9}\).
  It seems plausible that Cyril did not guess randomly in at least one of the \closeW{1} to \(w_{9}\).%
  \footnote{
    If unclear, observe that extend the initial sequence of \(\pi\) further.
  }

  \autoref{scen:prog:Cyril:know:not} presents a difficulty.
  For, then, progressive isn't true, yet intuitively is.
  Stress, all branches, so only need one instance of failure.
  Not the case that Cyril randomly guesses.
  Rather, overall things have shifted so that no random guess.%
  \footnote{
    First, relevant event is not such that \(e\) develops such that birds do not appear because birds are scared of hearing Cyril state digits of \(\pi\).
    Likewise, second, not because Cyril forgets how to say `\(6\)'.
    Somewhat absurd, but really isn't clear.
  }
  The \drift{0} through possible worlds is \ndrAdj{}.

  Term this the `problem of \ndrAdj{} \drift{0}'.
\end{note}

\begin{note}
  How is the case in which Cyril knows a continuation of \(e\)?
  This is the issue.
  Start randomly, but Cyril receives information as they state.
  It is not clear how this happens, but equally it's not clear how \closeW{} is after tenth digit correct.
\end{note}

\subparagraph{\(\lRweSym{}\) and \(\mRweSym{}\)}
\label{cha:fcs:sec:Prog:L:Alg:branches:drift:reasonable}

\begin{note}
  \autoref{cha:fcs:sec:Prog:L:Alg:branches:drift:close} outlined the problem of \ndrAdj{} \drift{0}.
  We now turn to \citeauthor{Landman:1992wh}'s approach to ruling out \ndrAdj{} \drift{0} and our alternative approach.
\end{note}

\begin{note}[\citeauthor{Landman:1992wh}'s approach]
  \citeauthor{Landman:1992wh}'s approach to ruling out \ndrAdj{} \drift{} is to:
  \begin{enumerate}[label=\alph*., ref=(\alph*), noitemsep]
  \item
    Intersect possible worlds with `reasonable' worlds with respect to the initial event \(e\) and word \(w\) captured by the set \(\lRwe{w}{e}\), and
  \item
    Restrict \drift{} to the intersection worlds which are both `reasonable' and close.%
    \footnote{
      Compare the consequent of Clause~\ref{Landman:CB:stops} (and~\ref{Landman:CB:stops:again}) of \citeauthor{Landman:1992wh}'s definition to \autoref{PrAl:find-branches:R} of \AlgFindBranches{}.
      See also \autoref{fn:Alg:branches:getLandman} on \autopageref{fn:Alg:branches:getLandman} for the full edit to \AlgFindBranches{}.
    }
  \end{enumerate}

  The difficulty with \citeauthor{Landman:1992wh}'s approach is that it is unclear what it is for a world to be `reasonable' with respect to \(e\) and \(w\).

  Recall, \citeauthor{Landman:1992wh} provided the following account:
  \begin{quote}
    \(v \in \lRwe{w}{e}\) iff there is a reasonable chance on the basis of what is internal to \(e\) in \(w\) that \(e\) continues in \(w\) as far as it does in \(v\).%
    \mbox{ }\hfill\mbox{(\citeyear[26]{Landman:1992wh})}
  \end{quote}

  However, it is by no means clear that the chance is suitable if chance is understood probabilistically and `reasonable' sets some threshold.%
  \footnote{
    If reasonable chance (on the basis of what is internal to an event in a world) is not understood probabilistically, then it is not clear (to me) what \citeauthor{Landman:1992wh} has in mind.
  }
  To illustrate, the probability of Cyril saying any sequence of digits in \autoref{scen:prog:Cyril:know:not} (where Cyril does not know the digits of \(\pi\)) is low.%
  \footnote{
    If ten digits seems borderline, extend the sequence.
  }\(^{,}\)%
  \footnote{
    Following \citeauthor{Landman:1992wh}, also highlights no way to reduce to similarity between worlds.
    For, \citeauthor{Landman:1992wh} holds that unreasonable even though what happened.
    Suppose Cyril hadn't been interrupted and chooses correctly.
    Still not reasonable change when starting.
  }
  Still, above the relevant threshold.
  For, these are event in which Cyril guesses the first ten digits of \(\pi\).

  Now, consider the probability of the bird \emph{not} flying into the classroom in~\autoref{scen:prog:Cyril:know}.
  Given a sufficiently determined bird and large enough window, the probability of the bird not flying into the classroom is likewise low.
  Hence, Cyril continuing to state the first ten digits of \(\pi\) in~\autoref{scen:prog:Cyril:know} does not meet the threshold.%
  \footnote{
    This objection parallels the issues~\citeauthor{Foley:2009vl} (\citeyear{Foley:2009vl}) raises with respect to the Lockean thesis (i.e.~the thesis that qualitative belief may be identified with probabilistic belief above some threshold).
  }

  Strictly, \citeauthor{Landman:1992wh} does not appeal to reasonable chance alone but `reasonable' chance \emph{on the basis of} what is internal to \(e\) in \(w\).
  However, \citeauthor{Landman:1992wh} does not elaborate on the distinction between what is internal or external to an event.%
  \footnote{
    For additional criticism on this point see:
    (\cite[35]{Szabo:2004ul}), (\cite[203,fn.2]{Bonomi:1997uq}), (\cite[49--50]{Engelberg:1999vi}), and (\cite[767]{Portner:1998um}).
  }

  The difficulty with \citeauthor{Landman:1992wh}'s approach is not the \emph{distinction} between what is internal or external to an event.
  Rather, the difficulty is that we have no independent grasp on what the distinction amounts to.
  Indeed, in developing \citeauthor{Landman:1992wh}'s account of the progressive we have relied on the idea of the same event being present in distinct possible worlds.
  Hence, we have assumed there is some way of separating an event from the world in which it takes place.
  And, as a result we are committed to some implicit understanding of the distinction.
  %
  \footnote{
    This itself may be of some concern, but I do not see this as any more concerning that identifying the same individual, object, property, etc.\ across possible worlds.
  }
  Still, without a clear account of what the distinction amounts to, it is unclear how to evaluate what `reasonable chance' on the basis of the distinction is.

  Perhaps it is possible to suitably expand \citeauthor{Landman:1992wh}'s definition of \(\lRwe{w}{e}\).
  However, we will not pursue the definition further.
  Instead, we turn to a substitute which appeals to the idea of closeness that allowed for \drift{0}.
\end{note}

\begin{note}[Our approach]
  Our proposal roughly, is restrict development in the following way:
  \begin{quote}
    \begin{itemize}
    \item[\emph{If}:]
      \begin{enumerate}[label=\alph*., ref=(\alph*)]
      \item
        \label{myR:sketch:a}
        \(g\) in \(u'\) is a \drAdj{} development of an event \(e\) in world \(w\).
      \end{enumerate}
    \item[\emph{Then}:]
      \begin{enumerate}[label=\alph*., ref=(\alph*), resume]
      \item
        \label{myR:sketch:b}
        In some \closeW{0} \(v\) to \(w\) where \(g\) develops, the way in which \(g\) continues to develop in \(v\) is as a development of \(e\).
      \end{enumerate}
    \end{itemize}
  \end{quote}

  Stated otherwise:
  If more plausible initial conditions for the result of tracing the development of \(e\) through possible worlds, then \(g\) in \(u'\) is not part of the continuation tree.
  Or, conversely, \(g\) in \(u'\) is part of the continuation tree of \(e\) only if the way in which \(g\) in \(u'\) develops is among the most plausible ways for \(e\) to develop from \(w\).
\end{note}

\begin{note}
  To get a feel for how this proposal works, let us return to Cyril and the result of \drift{0} where Cyril does not choose the tenth digit of \(\pi\) randomly.

  Intuitively, it seems that from \(w\), if Cyril does not randomly choose the tenth digits of \(\pi\), then this is because Cyril knew the first ten digits of \(\pi\) when Cyril started saying the first ten digits of \(\pi\).
  Hence, Cyril does not choosing the tenth digit of \(\pi\) randomly would be the result of a distinct event.
  So, the relevant event \(g\) would not be a stage of \(e\) in any of the \closeW{1} to \(w\) in which \(g\) happens.
\end{note}

\begin{note}
  The proposal avoids relying on any particular account of the distinction between what is internal or external to an event by indirectly regulating the development of \(e\) as the event develops through possible worlds by closeness between worlds.

  Loosely, the proposal may be paraphrased in terms of subjunctive conditionals.
  For, the proposal assumes we have identified \(g\) in some world \(u\) as the continuation of \(e\) in \(w\).
  Hence, the following subjunctive is true:
  \begin{itemize}
  \item
    If \(e\) develops, then \(g\) happens.
  \end{itemize}
  This is what we get from the process of shifting through \closeW{1}.
  Hence, we consider the (rough) converse of this subjunctive:
  \begin{itemize}
  \item
    If \(g\) happens, then \(g\) happens as a development of \(e\).
  \end{itemize}
  First does not imply second.
  For, first is obtained by holding fixed that \(e\) develops, while the second allows for the possibility that \(e\) does not develop.
  Likewise, second does not imply the former, for obtained by fixing \(g\), though \(g\) may not happen by jumping between \closeW{1}.%
  \footnote{
    More carefully stated:
  \begin{quote}
    The subjunctive conditionals are \emph{false} when \(e\) happens in \(w\):
    \begin{itemize}
    \item
      If \(e\) were to continue, then \(g\) is \emph{not} a development of \(e\).
    \item
      If \(g\) were to happen then the way in which \(g\) happens would \emph{not} be as a development of \(e\).
    \end{itemize}
  \end{quote}
  }
\end{note}

\begin{note}[Two parts of the proposal]
  We clarify two key parts of the proposal:
  \begin{itemize}
  \item
    Focus on the \emph{development of} \(g\) rather than \(g\) itself.
  \item
    Focus on the `way' in which \(g\) develops.
  \end{itemize}
\end{note}

\begin{note}[Development]
  Our interest in the \emph{development of} \(g\) rather than \(g\) itself follows from our understanding of why an instance of the progressive is true.
  For, in line with \assuPP{}, we capture the truth of the an instance of the progressive by the perfective correlate of the perfective.
  \AlgFindBranches{} is key part of finding event in which perfective correlate is true.
  However, the perfective correlate need not be true of \(g\) in \(u\).
  Instead, the perfective correlate may be true of some continuation \(g^{+}\) of \(g\) in \(u\).
  Intuitively, the role of \AlgFindBranches{} is to find some other world \(u'\) such that, although \(g\) stops in \(u\), \(g\) continues in \(u'\).
  Hence, our concern is not with whether \(g\) in \(u'\) is \drAdj{}, but rather with whether continuation \(g^{+}\) is \drAdj{}.

  To illustrate consider again Cyril as used to illustrate the problem of \ndrAdj{} \drift{0} in \autoref{drift:problem}.
  Cyril started by guessing randomly, but when Cyril said the tenth digit of \(\pi\) Cyril did not make a random guess.
  Instead, it turned out that as \(g\) developed in \(u'\), Cyril came to know the first ten digits of \(\pi\).
  However, Cyril saying the ninth digit of \(\pi\) in \(u\) may still have been a random guess.

  By focusing on the development of \(g\) rather than \(g\) itself we ensure any event in which the perfective correlate of the progressive is true is a \drAdj{} event.%
  \footnote{
    Observe that this idea is implicit in \citeauthor{Landman:1992wh}'s account of \(\lRweSym{}\) as \citeauthor{Landman:1992wh} considers worlds rather than events, and hence considers how \(g\) develops in the relevant world.
  }
\end{note}

\begin{note}[Way]
  We now turn to the second part of the proposal; focus on the `way' in which \(g\) develops.

  The motivation for considering the way in which an event is that by \autoref{def:Landman:conts} \(e\) is included in the continuation of \(g\).
  Therefore, it be trivially true that the continuation of \(g\) is a continuation of \(e\).
  Though, this issue is not merely definitional.

  To illustrate, consider \ref{Cyril:pi:progressive:random}:
  \begin{enumerate}[label=\arabic*., ref=(\arabic*)]
  \item[\ref{Cyril:pi:progressive:random}]
    Cyril is randomly guessing what the first ten digits of \(\pi\) are.
  \end{enumerate}
  For the progressive to be true, the relevant event's in the continuation tree must be as captured by~\ref{R:event-problem:full} rather than~\ref{R:event-problem:part}:

  \begin{enumerate}[label=\alph*., ref=(\alph*)]
  \item
    \label{R:event-problem:full}
    The event in which Cyril says the \emph{first ten} digits of \(\pi\).
  \item
    \label{R:event-problem:part}
    The event in which Cyril says the \emph{tenth} digit of \(\pi\).
  \end{enumerate}

  Observe, it does not follow from \ref{R:event-problem:part} that Cyril says the first ten digits of \(\pi\) --- Cyril may have said `\(3\)' in response to how many digits of \(\pi\) they have guessed.
  Hence, \ref{R:event-problem:part} does not make the perfective correlate of Cyril saying the first ten digits of \(\pi\) true.
  On the contrary, only \ref{R:event-problem:full} make the perfective correlate of Cyril saying the first ten digits of \(\pi\) true.

  However, as the initial event \(e\) included Cyril saying `\(3\)' and `\(1\)', \(e\) must be part of any event captured by \ref{R:event-problem:full}, regardless of how continuations are defined.
  Therefore, it trivially true that in some \closeW{0} in which \ref{R:event-problem:full} happens, \ref{R:event-problem:full} is a continuation of \(e\).

  Still, as Cyril must say the tenth digit of \(\pi\) in order to say the first ten digits of \(\pi\) it follows that if~\ref{R:event-problem:full} happens then ~\ref{R:event-problem:part} is a sub-event of~\ref{R:event-problem:full}.
  In this respect, it seems that if~\ref{R:event-problem:full} happens then how ~\ref{R:event-problem:part} happens captures, in part, the way in which ~\ref{R:event-problem:full} happens.
  We state this idea as a general assumption:

  \begin{assumption}[The way in which an event happens]
    \label{assu:event:way}
    The way in which an event \(g\) happens is a function of how the sub-events of \(g\) happen.
  \end{assumption}

  The role of \autoref{assu:event:way} is to allow rewriting \ref{myR:sketch:b} in terms of the existence of events in \closeW{0}:

  \begin{enumerate}[label=\alph*\('\)., ref=(\alph*\('\))]
    \setcounter{enumi}{1}
  \item
    In some \closeW{0} \(v\) to \(w\) where a sub-event \(g^{\flat}\) of the development of \(g\) happens, \(g^{\flat}\) is a continuation of \(e\).
  \end{enumerate}

  Whether~\autoref{assu:event:way} holds in general is unclear to me.%
  \footnote{
    I have not succeeded in developing a compelling counterexample, but likewise I have not succeeded in developing a argument for~\autoref{assu:event:way}.
  }
  Still, hence~\autoref{assu:event:way} may be considered a definition rather than an assumption.
  Either way, the sub-event

  classed as \ndrAdj{0} as in all the \closeW{1} to \(w\) in which Cyril correctly says the tenth digit of \(\pi\), the event is not a continuation of Cyril randomly guessing.
\end{note}

\begin{note}
  Our full proposal for a \drAdj{0} continuation of an event is as follows:
  \begin{definition}[\drAdj{2} instance of event]
    \label{def:myRwe}
    For an event-world-pairings \(\langle e,w \rangle\) and \(\langle g,u' \rangle\), and an event description \(\alpha\):
    \begin{itemize}
    \item
      \(\mRwe{\langle e,w \rangle}{\langle g,u' \rangle}\) is true.\newline
      \mbox{ }\hfill (\(g\) in \(u'\) is a \emph{\drAdj{0} continuation} of event \(e\) in world \(w\).)
    \end{itemize}

    \emph{If and only if}

    \begin{itemize}
    \item
      \(\forall \langle g^{\sharp},u' \rangle \in \AlgAC{\langle g,u' \rangle}\):
      \begin{itemize}
      \item
        \(\forall g^{\flat} \in \{ g^{\flat} \mid g^{\flat} \text{ is the restriction of }g\text{ to some subinterval of }g^{\sharp} \}\):
      \begin{itemize}
      \item
        In some \closeW{0} to \(w\) such that \(g^{\flat}\) happens, \(g^{\flat}\) is a stage of \(e\).
      \end{itemize}
    \end{itemize}
  \end{itemize}
  \vspace{-\baselineskip}
  \end{definition}

  In line with our first observation regarding the need to consider continuations of \(g\), we consider the continuation of \(g\) in \(u'\) as given via \AlgAC{}.
  As noted, \AlgAC{} capture the continuations of \(g\), which by \autoref{def:Landman:conts} include \(e\).%
  \footnote{
    For (relative) simplicity we allow for the possibility that \(g\) in \(u'\) is not part of the continuation tree for an event \(e\) because the development of \(g\) in \(u\) is \ndrAdj{0} even if the relevant development is only \ndrAdj{0} due to some sub-event that occurs after the perfective correlate of a sentence in the progressive true.

    I have no clear intuition on whether or not this matters.

    Still, \(\mRweSym{}\) may be modified to ignore sub-events after the perfective correlate is true by taking the correlate as an argument, and ignoring sub-events that occur after the correlate is made true.
  }
  And, following~\autoref{assu:event:way} we consider the set of sub-events that capture the way in which \(g\) develops.

  In full, \(g\) in \(u'\) is a \emph{\drAdj{0} continuation} of event \(e\) in world \(w\) just in case for every sub-event \(g^{\flat}\) of the development of \(g^{\sharp}\) of \(g\) in \(u'\), there is a \closeW{0} to \(w\) such that \(g^{\flat}\) happens and in that \closeW{0} \(g^{\flat}\) is a stage of \(e\).
\end{note}

\begin{note}[Summary]
  Figuring out whether or not whether  \(\mRwe{\langle e,w \rangle}{\langle g,u' \rangle}\) is true for any instance of \(e,w,g\) and \(v\) is by no means straightforward.

  Still, the intuition seems sufficiently sound.

  Our interest is in how \(e\) may have developed other than the way \(e\) develops in the initial world \(w\).
  And, so long as the way in which \(e\) develops in some possible world is not closer to the development of some other event \(e\), the development of \(e\) in the possible world will be of interest.%
  \footnote{
    One difficulty is the idea that the way some event \(e\) develops in the actual world may be considered \ndrAdj{0}, though will always be evaluated as \drAdj{0} given \autoref{def:myRwe}.

    Indeed, \citeauthor{Landman:1992wh} explicitly holds that the actual world may not be a `reasonable' world.
    However, in \citeauthor{Landman:1992wh}'s analysis the idea actual world may not be a `reasonable' world is considered a difficulty which \citeauthor{Landman:1992wh} avoids by only considering `reasonable' worlds if the event fails to continue in the actual world.
    (\citeyear[Cf.][26]{Landman:1992wh})
  }
\end{note}

\paragraph{Summarising}

\begin{note}
  I hope to have provided a sufficiently detailed account of how \AlgFindBranches{} functions.
  The considerations are complex, though are at the core of understanding the progressive via continuation trees, and I suspect of capturing the progressive via possible worlds in general.

  In short, allow an event \(e\) to develop by \drifting{0} through possible worlds in which \(e\) happens, but restrict \drift{0} to `\drAdj{0}' events, events such that the way in which the event happens is plausibly as a continuation of \(e\) rather than some other event.

  Whether or not this proposal works (and I am hesitant to say it does), the relevant considers have hopefully helped clarify to some degree the considers relevant to evaluating whether an instance of the perfective is true.

  In the \autoref{cha:fcs:sec:Prog:L:Alg:R-stops} we will present one final modification to \citeauthor{Wieland:2013vf}'s account of continuation branches.
\end{note}

\subsubsection{The (re)construction so far}

\begin{note}
  Three algorithms.
  Key parts of \citeauthor{Landman:1992wh}.
  Continuation stretch, stopping points, and branches.

  Branches is the most involved, primarily due to lack of clarity.
  Incorporated alternative.
  As event develops in possible worlds, remains the case that subsequent events plausibly develop from initial event.
\end{note}

\begin{note}
  Let us consider as a \illu{0} the following \scen{0} from \citeauthor{Engelberg:1999vi} (\citeyear{Engelberg:1999vi})%
  \footnote{
    Modified by~\citeauthor{Engelberg:1999vi} from~\textcite[475]{Asher:1992ug}
  }%
  :
  \begin{quote}
    Rebecca stood in front of a huge minefield, started walking, and walked about 50 yards into the minefield.%
    \mbox{ }\hfill\mbox{(\citeyear[49]{Engelberg:1999vi})}
  \end{quote}
  Reports most speakers find it acceptable to say `Rebecca is walking to the other side of the minefield'.
  (\citeyear[49]{Engelberg:1999vi}).

  `Rebecca is walking to the other side of the minefield' will be true just in case there are repeated application of \AlgAC{}, \AlgGetStops{}, and \AlgFindBranches{} such that in every way the event \(e\) of which the sentence is said to be true develops into an event in which Rebecca walks to the other side of the minefield.%
  \footnote{
    If someone nearby, then stop.

    `Rebecca's walking across a minefield and I'm preventing Rebecca from walking across a minefield' seems impossible.
    Either Rebecca is walking, so failing to prevent or preventing and so Rebecca isn't walking.
    Likewise if will take some time.
    `Rebecca's is walking partway across a minefield because the helicopter will rescue Rebecca before they get to far'.
  }

  We break this task down into two steps.
  First, we show why there is \emph{a} development of event in which Rebecca walks to the other side of the minefield.
  Second, we show how the observations from the first step generalise to all branches of the continuation tree.

  The setting is an event \(e\) in \(w\) in which Rebecca has walked 50 yards into a minefield.
  Via \AlgAC{} we will extend \(e\) in \(w\) until \(e\) stops.
  We capture the stopping event via \AlgGetStops{}, term this event \(f\)
  Following, we then consider \(f\) in some \closeW{1} \(v\) to \(w\) and repeat the processes, by extending \(f\) to some stopping event \(g\) in \(v\) and consider \(g\) in \(u\), etc.

  Suppose Rebecca doesn't make it much further and steps on a mine.
  This is the event \(f\).
  Now, prior to Rebecca stepping on the mine, it is possible that Rebecca may have avoided the mine?
  It seems to me the answer is yes.
  Rebecca has walked 45.72 meters into a minefield, which is quite far.
  Hence, it seems there is a \closeW{} in which \(f\) happens and Rebecca continues a little further.%
  \footnote{
    Note, the same need not be true if Rebecca has only walked onto a mine after treading a centimeter into the minefield.
    That Rebecca has walked 150 feet suggests Rebecca has some luck on their side, and though the luck ran out when Rebecca stepped on a mine, it seems there is a \closeW{0} in which Rebecca's luck holds out a little longer.
  }

  Further, it seems clear that if Rebecca continues a little further, then the continuation is a continuation of \(e\).
  For absence of additional information, it seems Rebecca has no choice but to walk across the minefield.

  Repeating these ideas, through repeated applications of \AlgAC{}, \AlgGetStops{}, and \AlgFindBranches{}, it seems we may \drift{} to an event in which Rebecca walks across the minefield.
\end{note}

\begin{note}
  Now, for the progressive to be true it must be the case that Rebecca walks to the other side of the minefield in all continuation paths.
  And, it is by no means clear that Rebecca is destined to avoid walking into a mine.

  However, recall that branch at a stopping event, where a stopping event is such that the event has happened but does not continue further, to a world in which the event continues.
  Hence, branch \emph{prior} to Rebecca walking into a mine and so that Rebecca makes some further move.
  Therefore, so long as there is a \closeW{0} in which Rebecca does not walk into a mine, then it seems the relevant possibilities included in the continuation tree will be such that Rebecca walks to the other side.
\end{note}

\subsubsection{Alternative stops}
\label{cha:fcs:sec:Prog:L:Alg:R-stops}

\begin{note}
  In \autoref{cha:fcs:sec:Prog:L:Alg:R-stops} we gave a reconstruction of the antecedent of Clause~\ref{Landman:CB:stops} (and~\ref{Landman:CB:stops:again}) of \citeauthor{Landman:1992wh}'s definition of a continuation branch.

  In this section, we motivate and detail an alternative algorithm.

  The basic observations are
  \begin{enumerate}[label=\arabic*., ref=(\arabic*), noitemsep]
  \item
    \label{landman:alg:R-stops:ob:1}
    If an agent performs some action \(\alpha\), then so long as \(\alpha\) is not instantaneous, there is some prior event such that the agent is \(\alpha\)ing.
  \item
    \label{landman:alg:R-stops:ob:2}
    It is not the case that Observation~\ref{landman:alg:R-stops:ob:1} holds, in general.
  \end{enumerate}
  To resolve this tension, we consider a modification of \AlgGetStops{} which returns `reasonable' rather than actual stops of an event.
  We term the modification `\AlgGetPStops{}'.

  First, we substantiate Observations~\ref{landman:alg:R-stops:ob:1} and~\ref{landman:alg:R-stops:ob:2}.
\end{note}

\begin{note}
  Observation~\ref{landman:alg:R-stops:ob:1} follows from  Clause~\ref{Landman:CB:stops} (and~\ref{Landman:CB:stops:again}) of \citeauthor{Landman:1992wh}'s definition of a continuation branch.
  For, Clause~\ref{Landman:CB:stops} (and~\ref{Landman:CB:stops:again}) is a conditional:
  \begin{quote}
    \begin{enumerate}
      \setcounter{enumi}{1}
    \item
      if the continuation stretch of \(e\) in \(w\) stops in \(w\), it has a maximal element \(f\) and \(f\) stops in \(w\).
      Consider the closest world \(v\) where \(f\) does not stop: \dots
    \end{enumerate}
  \end{quote}
  Hence, for an event \(e\) and world \(w\), Clause~\ref{Landman:CB:stops} only includes the closest world where \(e\) does not stop \emph{if} \(e\) stops in \(w\).
  So, if \(e\) does not stop in the actual world, and \(e\) develops into an event \(e'\) such that the agent \(\alpha\)s in \(e'\), then it will be true that the agent is \(\alpha\)ing with respect to \(e\).
\end{note}

\begin{note}
  We now turn to Observation~\ref{landman:alg:R-stops:ob:2}.

  Observation~\ref{landman:alg:R-stops:ob:2} is the converse of the above%
  \footnote{
    See the discussion of the imperfective paradox on~\autopageref{imperfective-paradox:intro}.
  }
  observation that an event in which an agent \(\alpha\)s does not need to happen for it to be true that the agent is \(\alpha\)ing.
  For, restated, Observation~\ref{landman:alg:R-stops:ob:2} reads:%
  \footnote{
    \citeauthor{Landman:1992wh} considers a similar observation under the term `the problem of non-interruptions' (\citeyear[14--17,30--31]{Landman:1992wh}).
    \citeauthor{Landman:1992wh}'s suggestion is to distinguish events by the perspective of the agent. (\citeyear[31]{Landman:1992wh})
    The basic idea, as I understand \citeauthor{Landman:1992wh} is to deny that the event \(e'\) in which the agent \(\alpha\)s is sufficiently distinct from any prior event \(e\).
    Hence, it seems that strictly speaking \citeauthor{Landman:1992wh} does not endorse Observation~\ref{landman:alg:R-stops:ob:2}.
  }
  \begin{enumerate}[label=\arabic*\('\)., ref=(\arabic*\('\))]
    \setcounter{enumi}{1}
  \item
    There are cases in which \(e\) is not an instance of an agent \(\alpha\)ing, though \(e\) develops into \(e'\) and \(e'\) is an event in which the agent \(\alpha\)s.
  \end{enumerate}

    We present a pair of \illu{1} with both provide instances of Observation~\ref{landman:alg:R-stops:ob:2}.
    Following, we turn to theoretical motivation.
\end{note}

\begin{note}
  \begin{illustration}[Flipping coins]
    \label{illu:fc:coins}
    Agent is flipping a coin in the air and recording how it lands.%
    \footnote{
      Cf.~\textcite{Gelman:2002ww} and~\textcite{Keller:1986tz}.
    }
    The coin lands heads ten times in a row.

    The following seem true:
    \begin{itemize}
    \item
      The coin landed heads ten in a row.
    \item
      The agent landed the coin heads ten times in a row.
    \end{itemize}
    However, the following seem false:
    \begin{itemize}
    \item
      The coin was landing heads ten times in a row.
    \item
      The agent was landing a coin heads ten times in a row.
    \end{itemize}
    \vspace{-\baselineskip}
  \end{illustration}
  Intuition is clear.
  No time prior to the coin landing heads on the tenth flip was the coin landing heads ten times in a row.
  Consider prior.
  \begin{itemize}
  \item
    I am landing this coin heads ten times in a row.
  \end{itemize}
  Only reasonable interpretation is if the agent will continue to keep flipping the coin until the coin lands heads ten times in a row.
  However, if the agent is limited to ten flips, absurd.

  Though, the tenth flip is a clear continuation of the event containing the previous nine coin flips.

  Possible objection.
  Event was not the result of the agent.
  And, if agent then works out because prior to agent completing action, progressive.
  However, the action is flipping the coin so that it lands heads.
  No different from encrypting a message so that it will not be understood by the messenger.
  Once sent, the agent has no involvement in the messenger's understanding.
  Though, given a sufficiently strong algorithm, it seems true that the agent is encrypting the message when working through the algorithm.
\end{note}

\begin{note}[Second]
  \begin{illustration}[Chess III]
    \label{illu:fc:chess:III}
    Consider as the game state on the left transitions to the game state on the right.

    \noindent\mbox{ }\hspace{2em}%
    \begin{adjustbox}{minipage=.4\linewidth,scale=.75}%
      \centering%
      \newchessgame[%
      setwhite={ka6,pa5,pb6,pc7,rg2,bh1},%
      addblack={ka8,rg8,na3},%
      ]%
      \setchessboard{showmover=false}%
      \chessboard%
    \end{adjustbox}%
    \hfil%
    \begin{adjustbox}{minipage=.4\linewidth,scale=.75}%
      \centering%
      \newchessgame[%
      setwhite={ka6,pa5,pb6,pc7,rg2,bh1},%
      addblack={ka8,rc8,na3},%
      ]%
      \setchessboard{showmover=false}%
      \chessboard%
    \end{adjustbox}%
    \hspace{2em}\mbox{ }

    On the left, white has just moved their pawn from c6 to c7.
    Though, white is quite bad at chess.
    And, the strength of their pieces on the board is the result of black intentionally playing even worse.
    So, it is not true that:
    \begin{itemize}
    \item
      White is winning the game of chess.
    \end{itemize}
    For example, if black moves their rook to h8 then on the following turn white may move their pawn to c8 and exchange it for a rook which black may then capture, etc.\dots

    Still, on the following turn, black moves their rook from g8 to c8, as shown on the right.
    As a result:
    \begin{itemize}
    \item
      White wins the game of chess.
    \end{itemize}
  \end{illustration}

  May argue about the interpretations of `wins'.
  For, white still needs to make a move.
  However, any move made by white results in checkmate for black.
  Hence, true.

  In contrast to \autoref{illu:fc:coins}, \autoref{illu:fc:chess:III} does not involve luck.
  Instead, \autoref{illu:fc:chess:III} involves the action by some other agent which leads to desired outcome.
  %
  \footnote{
    More generally, one consider the structure of \autoref{illu:fc:chess:III} applied to any competitive activity in which a mistake by one participant hands victory to the other participant.
    It seems clear to me that a win does not entail that the participant was winning at any point throughout the game.
    E.g.\ a racing car driver may be handed a win due to their competitor having a bad pit stop, or a football team may be handed a win due to an unexpected injury on the other team, etc.
  }
\end{note}

\begin{note}[Theoretical motivation]
  On the one hand, intuition.
  On the other hand, motivated by \citeauthor{Landman:1992wh}'s analysis.
  Distinction between event and world.
  Interested in the event getting to completion regardless of what happens in the world.
  Hence, allows us to shift to possible worlds.
  Progressive is not false because of contingencies of the actual world.
  By parallel, progressive is not true because of contingencies of actual world.
\end{note}

\begin{note}
  Given this, understand both \illu{1}.
  Deviance.

  Once noted, plausible to distinguish where in event progressive comes true.
  Final \illu{}.
  \begin{illustration}[Choosing cards]
    An agent is presented with a shuffled deck of playing cards and is asked to choose a number \(1 < n \leq 52\).
    After the agent announces the number, the presenter of the deck of cards will remove \(n - 1\) cards from the top of the deck and reveal the \(n\)th card.

    The agent considers their options with gravitas.
    \begin{itemize}
    \item
      First, the agent decides they will choose and odd number
    \item
      Second, the agent decides they will choose a card from the top half of the deck.
    \item
      Third, the agent decides they will choose a prime number.
    \item
      Finally, the agent chooses \(17\).
    \end{itemize}
    The agent is unaware that the pack is shuffled so that every card corresponding to a prime number is a red card, though the \(9\)th card isn't a red card.
    Hence, after the third choice, but not before, it is true that:
    \begin{itemize}
    \item
      The agent is choosing a red card.
    \end{itemize}
    \vspace{-\baselineskip}
  \end{illustration}
  So, before third decision, agent may have chosen a non-prime number.
  In particular, \(9\) is not a prime number and the \(9\)th card is not a red card.
  However, after the decision, no turning back.
\end{note}

\begin{note}
  So, Observations~\ref{landman:alg:R-stops:ob:1} and~\ref{landman:alg:R-stops:ob:2}.

  The suggestion is simple.
  Rather than search the continuation of an event \(f\) in world \(v\) 

  Slight difficulty is closest world.
  Instead, closest worlds.
  Intuitively, obtained from counterfactuals against what happened in the actual world.
  However, reasonable, as these are worlds in which the event continues.
\end{note}

\begin{note}
  Modification builds on principles for \AlgFindBranches{}.
  For, \AlgFindBranches{} finds continuation of event in nearby world.


  \AlgGetPStops{}:

  \begin{algorithm}[H]
    \label{PrAl:g-s:poss}
    \caption{\AlgGetPStops{}}
    \SetAlgoLined
    \DontPrintSemicolon
    \Input{%
      \(\text{Continuation}\) \hfill I.e.\ the result of \AlgAC{} --- in general, a tree \\
      \(e\) \hfill An event \\
      \(w\) \hfill A world
    }
    \KwResult{ReasonableStops \hfill \emph{Reasonable} stopping points \\
      \mbox{ } \hfill for each event in \(\text{Continuation}\)}
    \Begin{
      \(\text{R-Stops} \longleftarrow \emptyset\)\;
      \For{\(\langle \langle f,v \rangle_{i-1}, \langle g,u \rangle_{i} \rangle \in \text{Continuation}\)}
      {
        \(\text{Branches} \longleftarrow \AlgFindBranches{(\langle \langle f,v \rangle_{i-1}, \langle g,u \rangle_{i} \rangle, e,w)}\)\;
        \For{\(\langle \langle f,v \rangle_{i-1}, \langle g,u' \rangle_{i} \rangle \in \text{Branches}\)}
        {
          \If{\(g\text{ stops in }u'\)}
          {
            \(\text{R-Stops} \longleftarrow \text{R-Stops} \cup \{\langle \langle f,v \rangle_{i-1}, \langle g,u' \rangle_{i} \rangle\}\)\;
          }
        }
      }
      \Return{\(\text{R-Stops}\)}
    }
  \end{algorithm}

  \AlgGetPStops{} is a variation on \AlgGetStops{}.

  The for-loop from \AlgGetStops{}, repeated, and check from \AlgGetStops{} is repeated.
  However, check within a for-loop for whether there is a `reasonable world' in which the event of some event-world pairing stops.
\end{note}

\begin{note}
  Note, branch.
  However, only if result of \AlgGetPStops{} is part of tree.
  And, antecedent of conditional.
  Branches will be included, but only when calling \AlgFindBranches{}.
  This we now turn to.
\end{note}


\subsubsection{Build tree}
\label{cha:fcs:sec:Prog:L:Alg:tree}

\begin{note}
  With \AlgAC{}, \AlgGetStops{}/\AlgGetPStops{}, and \AlgFindBranches{} in hand, we now turn to \AlgDevelopTree{}, a recursive algorithm which completes any tree.

  Focus here is on linking \AlgAC{}, \AlgGetStops{}/\AlgGetPStops{}, and \AlgFindBranches{} in the same way as \citeauthor{Landman:1992wh}, and clarifying the `\dots and we continue as above, etc.' from \ref{Landman:CB:stops:again}.

  \begin{algorithm}[H]
    \label{PrAl:dev-tree}
    \caption{\AlgDevelopTree{}}
    \SetAlgoLined
    \DontPrintSemicolon
    \Input{%
      \(\text{Tree}\) \hfill A partially completed tree\\
      \(e\) \hfill The initial event of the tree\\
      \(w\) \hfill The initial world of the tree\\
      \(n\) \hfill An index to keep track recently added branches\\
    }
    \KwResult{Tree expanded so that there are no further stopping events}
    \Begin{
      \label{PrAl:dev-tree:start}
      \(\text{Stems} \longleftarrow \{ \langle g,u \rangle_{n} \mid \exists f,v \colon \langle \langle f,v \rangle_{n - 1}, \langle g,u \rangle_{n}\rangle \in \text{Tree}\}\)\;
      \label{PrAl:dev-tree:Extend:start}
      \label{PrAl:dev-tree:Extend:Stems}
      \(\text{GrownStems} \longleftarrow \emptyset\)\;
      \label{PrAl:dev-tree:Extend:FreshContsVar}
      \For{\(\langle g,u \rangle_{n} \in \text{Stems}\)}
      {
        \label{PrAl:dev-tree:Extend:Loop:start}
        \(\text{Growth} \longleftarrow \AlgAC{}(\langle g,u \rangle_{n})\)\;
        \(\text{GrownStems} \longleftarrow \text{GrownStems} \cup \text{Growth}\)\;
      }
      \label{PrAl:dev-tree:Extend:end}
      \(\text{Tree} \longleftarrow \text{Tree} \cup \text{GrownStems}\)\;
      \label{PrAl:dev-tree:Extend:merge}
      \textcolor{comment}{\texttt{//} \(\text{Stops} \longleftarrow \AlgGetStops{}(\text{GrownStems})\)}\;
      \label{PrAl:dev-tree:Stops:Land}%
      \(\text{Stops} \longleftarrow \AlgGetPStops{}(\text{GrownStems})\)\;
      \label{PrAl:dev-tree:Stops:Me}
      \(\text{Branches} \longleftarrow \emptyset\)\;
      \label{PrAl:dev-tree:pro-bra:start}
      \For{\(\langle \langle g,u \rangle_{n}, \langle h,u \rangle_{n+1}\rangle \in \text{Stops}\)}
      {
        \label{PrAl:dev-tree:BranchLoop:Start}
        \(\text{Temp} \longleftarrow \AlgFindBranches{}(\langle \langle g,u \rangle_{n}, \langle h,u \rangle_{n+1}\rangle, e, w)\)\;
        \label{PrAl:dev-tree:getBranches}
        \(\text{Branches} \longleftarrow \text{Branches} \cup \text{Temp}\)\;
        \label{PrAl:dev-tree:gatherBranches}
      }
      \label{PrAl:dev-tree:pro-bra:end}
      \eIf{\(\text{Branches} = \emptyset\)}{
        \label{PrAl:dev-tree:StopsCond}
        \textbf{return} Tree\;
        \label{PrAl:dev-tree:NoStops:returnTree}
      }
      {
        \(\text{Tree} \longleftarrow \text{Tree} \cup \text{Branches}\)\;
        \label{PrAl:dev-tree:merge-branches}
        \AlgDevelopTree{}(\(\text{Tree}, e,w, n+1\))\;
        \label{PrAl:dev-tree:recurse}
      }
      \label{PrAl:dev-tree:BranchLoop:End}
    }
  \end{algorithm}

  \AlgDevelopTree{} processes single instance of branching on each run.
  So, given a tree, where terminal nodes are event world pairing \(\langle f,v \rangle_{i}\) such that \(f\) \emph{may} continue in \(v\).
  Term these terminal nodes `stems'.

  Tasks.
  \begin{enumerate}
  \item
    Identify and extend stems.%
    \hfill%
    Lines~\ref{PrAl:dev-tree:Extend:start}--\ref{PrAl:dev-tree:Extend:merge}.
  \item
    Determine whether the extended stems branch.%
    \hfill% Lines~\autoref{PrAl:dev-tree:Stops:Land}/\autoref{PrAl:dev-tree:Stops:Me}--\ref{PrAl:dev-tree:pro-bra:end}.
  \item
    Either return tree or continue.%
    \hfill%
    Lines~\ref{PrAl:dev-tree:StopsCond}--\ref{PrAl:dev-tree:BranchLoop:End}.
  \end{enumerate}

  Minor commentary.

  \begin{itemize}
  \item
    First task.
    \begin{itemize}
    \item
      Extending steps, and every for which we haven't yet considered branching.

      When extending steps, immediate goal is to add these to the tree.

      However, growth is important for determining stops.

      Therefore, keep the result in the set \(\text{GrownStems}\).

      The loop (\ref{PrAl:dev-tree:Extend:Loop:start}--\ref{PrAl:dev-tree:Extend:end}) is due to the possibility that there are multiple branches.
    \end{itemize}
  \end{itemize}

  Remainder of algorithm checks for branching, and for whether to continue.

  \begin{itemize}
  \item
    Second task
    \begin{itemize}
    \item
      account of branching depending on which stopping points are considered.

      \autoref{PrAl:dev-tree:Stops:Land} \citeauthor{Landman:1992wh} Branching only occurs if the (development of the) event stops in the relevant possible world.


      \autoref{PrAl:dev-tree:Stops:Me}, revised account of branching:
      Branching occurs if there is some `reasonable' world in which the event stopped.
    \end{itemize}
  \item
    With choice in hand, process of collecting branches is simple.
  \end{itemize}

  \autoref{PrAl:dev-tree:StopsCond} to \autoref{PrAl:dev-tree:BranchLoop:End} determine whether to continue growing the continuation tree.
  Fail to continue if either no stops to consider or no branches.

  Delicate point.
  Ignore stops if no continuation.
  May think that stopping point shows the progressive is false.
  However, if the present event stops is already part of the tree, as obtained from growing stems.
  Hence, no continuation, then progressive is false.
  If progressive is true, branches develop the event in some possible world, but then already true.

  Implicitly capture both via check on branches.
  For, if no stops then empty set passed to and hence alg returns empty set.

  \begin{itemize}
  \item
    No branches or stops then return tree \autoref{PrAl:dev-tree:StopsCond}
  \item
    Branches (and hence stops) then add branches to tree (\autoref{PrAl:dev-tree:merge-branches}) and make recursive call (\autoref{PrAl:dev-tree:recurse}).
  \end{itemize}

  Now, start with \(\langle e,w \rangle\).
  Slight thing, ordered pairs.
  So, pass a vacant event.
  Run, \AlgDevelopTree{}(\(\langle \langle -,- \rangle_{0}, \langle e,w \rangle_{1} \rangle, e, w, 1\))

  This gives us continuation tree.
\end{note}

\subsection{Summary}

\begin{note}
  \citeauthor{Landman:1992wh}'s account with two key changes.

  First, allow for multiple branching.
  Motivated by concerns from \citeauthor{Bonomi:1997uq} and \citeauthor{Szabo:2004ul}.
  Important for \fc{1}, as explore all paths in reasoning.

  Second, possible stops.
  Other ways things could have gone.
  Motivated by intuitive instances of the progressive.
  Important for \fc{1}, again, all parts in reasoning.

  So, basically, progressive is true just in case, some event in progress, and no matter how the event develops, there is always some possibility in which complete.
  Possibility is existential as shift to possible world.
\end{note}

\begin{note}
  \pevent{} just in case there is some action avaiable to the agent, and were the agent to perform the action, the agent would be \(\alpha\)-ing.
\end{note}

\begin{note}
  With respect to concluding, reasonable constraint is neat.
  How the agent is in the actual world.
  And, shifts to closest worlds avoid blunders.
\end{note}

\subsection[The progressive and ability]{The progressive and ability \hfill (Optional)}

\begin{note}
  As an aside, this plausibly also gets failure of \BoyVS{}.
  Surprising, this seems to capture all the entailments that \citeauthor{Boylan:2020aa} is interested in.
  Though, I think we've just reduced ability to choice of action.
\end{note}


\begin{note}
  Now, \BoyVS{}.
  Holds on \citeauthor{Landman:1992wh}'s account of the progressive.
  Holds on revised account, \emph{only} if disjoin all possible results.

  Two aspects.
  First, \AlgGetPStops{}, different from how things actually happen.
  Second, universal quantification over branches.

  So, if some point not considered, then disjunction without point fails to hold.

  This seems good.

  Then can't works, because \AlgGetPStops{}.

  Finally, \BoyPS{}.
  Complex.

  Holds, given the same method of evaluation.
  Which worlds are nearby worlds.

  Doesn't hold for the progressive in general.
  Though, I don't think it holds for ability.

  However, will hold when fully determined.
  Had the ability to win the race, after got into first.
  Had the ability to hit the bullseye, just before releasing the dart.

  Deals with wind.

  And, then this also gets joint abilities.
\end{note}

%%% Local Variables:
%%% mode: latex
%%% TeX-master: "master"
%%% End:
