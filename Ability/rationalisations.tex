\chapter{\issueInclusion{} and rationalisations}
\label{sec:reasons}


\begin{note}
  This chapter highlights some similarities and differences between \fofr{1} and rationalisations.

  In short, \fofr{1} and rationalisations are distinct abstractions which plausibly coincide in certain cases.
  Further, extant accounts of rationalisations plausibly motive \issueInclusion{} when \fofr{1} and rationalisations coincide.
\end{note}

\begin{note}
  We make three observations:

  \begin{enumerate}
  \item
    Similar to rationalisations.
  \item
    Variants of \issueInclusion{} have been proposed for \qWhyR{} and \qHow{}.
  \item
    There is no straightforward motivation for \issueInclusion{} by constraints on answers to \qWhyR{} in terms of answers to \qHow{}.
  \end{enumerate}
\end{note}


\begin{note}
  Distinction between two different abstractions that plausibly overlap in certain cases.
\end{note}


\section{Rationalisations and \qWhyR{}}

\begin{note}
  Two sub-sections.

  First we introduce rationalisations, a variant of \qWhy{}, and outline a plausible link between rationalisations.

  Then, we link answers to \qWhy{} to \qWhyR{}.
\end{note}


\subsection{\qWhy{} and \qWhyR{}}

\begin{note}
  The definition of a rationalisation we work with follows \citeauthor{Davidson:1963aa}:

  \begin{definition}[Rationalisations]%
    \label{def:rationalisation}
    Given \(e\) is an \eiw{0} an agent does action \(a\):
    \begin{itemize}
    \item
      A \emph{rationalisation} is a relation between a reason \(R\) and \(a\) such that:
      \begin{itemize}
      \item
        \(R\) explains why \(e\) happened by giving the \agents{} reasons for doing \(a\).
      \end{itemize}
    \end{itemize}
    \vspace{-\baselineskip}
  \end{definition}

  \noindent%
  \autoref{def:rationalisation} is a paraphrase the account given by \citeauthor{Davidson:1963aa} in \citetitle{Davidson:1963aa}.%
  \footnote{
    \begin{quote}
      What is the relation between a reason and an action when the reason explains the action by giving the agent's reason for doing what he did?
      We may call such explanations \emph{rationalizations}[.]%
      \mbox{}\hfill\mbox{(\citeyear[685]{Davidson:1963aa})}
    \end{quote}
    %
    The only significant difference is event and mention of why.
    I think event is implicit.
    And, why is optional.
    Indeed, we only include `why' to contrast with `how'.

    \begin{quote}
      A reason [is given by a rationalisation] only if it leads us to see something the agent saw, or thought he saw, in his action[.\space\dots]
      We cannot explain why someone did what he did simply by saying the particular action appealed to him; we must indicate what it was about the action that appealed.
    \end{quote}
  }
  Small difference.
  \citeauthor{Davidson:1963aa} \emph{the} reason.
  For our purposes, allow parts.%
  \footnote{
    Countersigning.
  }

  Note, the reason is stated from \agpe{our} and explains.
  The \agents{} reason is included in the reason.

  So, included in a rationalisation is a relation which holds from the \agpe{}.
  Distinction between a reason which is a rationalisation and an \agents{} reason which is given by a rationalisation.
  Loosely following the terminology of \citeauthor{Smith:1994wo} (\citeyear{Smith:1994wo}), we term the \agents{} reason a \emph{motivating} reason:%
    \footnote{
    \textquote{[T]he distinctive feature of a motivating reason to \(\phi\) is that, in virtue of having such a reason, an agent is in a state that is \emph{explanatory} of her \(\phi\)-ing, at least other things being equal --- other things must be equal because an agent may have a motivating reason to \(\phi\) without that reason's being overriding.}
    (\citeyear[96]{Smith:1994wo})

    The thing here is \textquote{\emph{in virtue of having such a reason}}.
    This is quite strong.
    \textquote{\emph{partly} in virtue of having such a reason}.
  }

  \begin{definition}[\motingr{3}]
    \motb[1]{A reason}{an action}:
    \begin{itemize}
    \item
      An \agents{} reason.
    \item
      Given as part of a rationalisation between some reason and the action.
    \end{itemize}
  \end{definition}

  With def.\ parallel question.

  \begin{question}{questionWhyR}{\qWhyR{}}
    Given \(e\) is an \eiw{0} \vAgent{} does \(a\):

    \begin{itemize}
    \item
      Which \motingr{1} (partially) explain \emph{why} \(e\) is an \eiw{0} \vAgent{} does \(a\)?
    \end{itemize}
    \vspace{-1\baselineskip}
  \end{question}

  Where, the sense of `why' in \qWhyR{} is the same as the sense of `why' in \qWhy{}.
\end{note}


\begin{note}
  \qWhy{} seeks an understanding of the way an event developed such that an agent concludes \(\pv{\phi}{v}\) from \(\Phi\) in terms of \fingfr{1}.
  And, similarly, \qWhyR{} seeks an understanding of the way an event developed such that an agent did some action in terms of rationalisations.

  Likewise, \qHowR{} seeks an understanding of the way an event developed such that an agent did some action in terms of what happened, and \qHow{} is a restriction of \qHowR{} to \eiw{1} the relevant action is `concludes \(\pv{\phi}{v}\) from \(\Phi\)'.
\end{note}


\begin{note}
  Now, an answer to \qWhyR{} is not a rationalisation.
  Instead, given by a rationalisation.

  \begin{idea}
    Answers to \qWhyR{} are given by rationalisations.
  \end{idea}

  Whether you agree with this idea depends on sense of explanation present in rationalisations.
  I think this is plausible.
  Still, this is not immediate.

  For the moment we set this aside.
  We will soon see both \citeauthor{Davidson:1963aa} and \citeauthor{Hieronymi:2011aa} get this.
\end{note}


\subsection{\motingr{3} and \fingfr{1}}


\begin{note}
  Our interest is with connexion between \motingr{1} and \fingfr{1}.

  Rationalisations are stated with respect to any action.
  So, the action may be a \evalion{} of \(\phi\) having \val{0} \(v\) from \(\Phi\).
  Though, conclusion of \(\pv{\phi}{v}\) from \(\Phi\).

  Interest is with \motingr{1} and \fofr{1}.

  A \fofr{} captures what led the agent to conclude \(\pv{\phi}{v}\) from \(\Phi\).
  In this respect, may say \motb{the \prop{0}-\val{0} pairs included in \(\Phi\)}{a conclusion of \(\pv{\phi}{v}\)}.

  For, intuitively, \fingfb{\(\pv{\phi}{v}\)}{\(\Phi\)} is part of what it is.
\end{note}

\begin{note}
  The following idea seems plausible:

  \begin{idea}[\fingfr{} and rationalisations --- basic]%
    \label{idea:fof-r}%
    Given \(e\) is an \eiw{0} an agent concludes \(\pv{\phi}{v}\) from \(\Phi\):
    \begin{itenum}
    \item[\emph{If}:]
      \fingfb{\(\pv{\phi}{v}\)}{\(\Phi\)} answers \qWhy{}.
    \item[\emph{Then}:]
      \motingb{\(\Phi\)}{\evaling{} \(\phi\) as having \val{0} \(v\)} answers \qWhyR{}.
    \end{itenum}
  \end{idea}

  Three instances, where we have argued \fingfb{\(\pv{\phi}{v}\)}{\(\Phi\)} answers \qWhy{}:

  \begin{itemize}
  \item
    \motingb{The \agents{} understanding of factorisation}{\evaling{} \propM{\rootsCon{}} as \valI{True}} answers \qWhyR{} the agent concluded \pv{\propM{\rootsCon{}}}{\valI{True}} from their understanding of factorisation.
  \item
    \motingb{Pritcher's opening `I come from Miran'}{\evaling{} \propI{\signConB{}} as \valI{True}} answers \qWhyR{} Fox concluded \pv{\propI{\signConA{}}}{\valI{True}} from the sequence of countersigning.
  \item
    \motingb{The \agents{} understanding of a recursive definition of the Lucas numbers}{\evaling{} \propI{The first ten Lucas numbers are 2, 1, 3, 4, 7, 11, 18, 29, 47, and 76} as \valI{True}} answers \qWhyR{} the agent concluded \pv{\propI{The first ten Lucas numbers are 2, 1, 3, 4, 7, 11, 18, 29, 47, and 76}}{\valI{True}} from their understanding of a recursive definition of the Lucas numbers.
  \end{itemize}

  Note, tied to answers.
  In general, \fingfr{} does not entail \motingr{}.
  For, action.
  May be \fof{} but no interest in \evalion{}.
\end{note}

\section{\issueInclusion{} and rationalisations}


\begin{note}
  \begin{observation}%
    \label{obs:whRconstraintMotivation}%
    Variants of \issueInclusion{} have been proposed for \qWhyR{} and \qHow{}.
  \end{observation}
  \begin{motivation}{obs:whRconstraintMotivation}
    We consider proposals by \citeauthor{Davidson:1963aa} and \citeauthor{Hieronymi:2011aa} in turn.
    \medskip

    \noindent%
    \citeauthor{Davidson:1963aa}'s argument of \citetitle{Davidson:1963aa} is:
    % 
    \begin{quote}
      [R]ationalization is a species of ordinary causal explanation.%
      \mbox{ }\hfill\mbox{(\citeyear[685]{Davidson:1963aa})}
    \end{quote}
    % 
    So, a reason explains the action by giving the agent's reasons for doing what they did when the agent's reasons are part of a \emph{causal} explanation of the action.%
    \footnote{
      For \citeauthor{Davidson:1963aa} this is strengthened.
      The \agents{} reason \emph{is} the cause.
      However, we allow a variety of reasons.
    }

    A causal relation captures how something happened, and \citeauthor{Davidson:1963aa} holds causal relation explains how event happens.
    Hence, \citeauthor{Davidson:1963aa} constrains answers to \qWhyR{} by answers to \qHowR{}.
    \medskip

    \noindent%
    In contrast to causal explanation, \citeauthor{Hieronymi:2011aa} (\citeyear{Hieronymi:2011aa}) appeals to a complex fact:%
    \footnote{
      Consider also \citeauthor{Harman:1973ww}'s account of reason explanation:
      \textquote{Reasons may or may not be causes; but explanation by reasons is not causal or deterministic explanation.
        It describes the sequence of considerations that led to belief in a conclusion without supposing that the sequence was determined} (\citeyear[52]{Harman:1973ww}).
    }
    % 
    \begin{quote}
      [W]henever an agent acts for reasons, the agent, in some sense, takes certain considerations to settle the question of whether so to act, therein intends so to act, and executes that intention in action.

      If this much is uncontroversial (and, under some interpretation, I believe it must be), we can use it as a form for filling out.
      I propose, then, that we explain an event that is an action done for reasons by appealing to the fact that the agent took certain considerations to settle the question of whether to act in some way, therein intended so to act, and successfully executed that intention in action.
      I suggest that \emph{this} complex fact [\dots] explains the action by giving the agent's reason for acting.%
      \mbox{ }\hfill\mbox{(\citeyear[421]{Hieronymi:2011aa})}
    \end{quote}
    % 
    In short, \citeauthor{Hieronymi:2011aa} observes a particular (complex) fact holds when an agent agents for reasons.
    If we grant that the fact \citeauthor{Hieronymi:2011aa} observes explains how an agent did what they did, then \citeauthor{Hieronymi:2011aa}'s proposal likewise constrains answers to \qWhyR{} in terms of answers to \qHowR{} --- in order for a rationalisation to explain why and agent did what they did there must be an event where the agent settles what to do via the (\agents{}) reasons present in the rationalisation.
  \end{motivation}

  \noindent%
  In short, the arguments/proposals of \citeauthor{Davidson:1963aa} and \citeauthor{Hieronymi:2011aa} entail a constraint which parallels \issueInclusion{}, with respect to the variant questions \qWhyR{} and \qHowR{}.
\end{note}


\section{Problem}
\label{sec:problem}


\begin{note}
  So, answers to \qWhyR{} may be constrained by answers to \qHowR{}.
  And, get a restricted variant of \issueInclusion{}.

  Still, we observed any counterexample to \issueInclusion{} must involve \fingfb{\(\pv{\psi}{v'}\)}{\(\Psi\)}, where either \(\psi\) is distinct from \(\phi\), \(v'\) is distinct from \(v\), or \(\Psi\) is distinct from \(\Phi\).
  Hence, the plausible conditional does not cover \fingfr{1} which \emph{may} be counterexamples to \issueInclusion{}.

  We argue, this does not generalise to any case of interest.

  \begin{proposition}
    \label{prop:no-r}
    It is not possible to capture the way \fingfb{\(\pv{\psi}{v'}\)}{\(\Psi\)} answers \qWhy{} via a rationalisation.
  \end{proposition}

  \begin{argument}{prop:no-r}
    A rationalisation explains an action by giving an agent's reason for doing the action.

    Hence, rationalisations have a single free variable for the \agents{} reason.

    Applied to conclusions, rationalisations are of the form:
    % 
    \begin{itemize}
    \item
      \(R\) is the \agents{} reason for their conclusion of \(\pv{\phi}{v}\) from \(\Phi\).
    \end{itemize}
    % 
    Where \(R\) stands for an \agents{} reason.

    Now, assume \fingfb{\(\pv{\psi}{v'}\)}{\(\Psi\)} answers \qWhy{}, where \(\psi\) is distinct from \(\phi\) and \(v'\) is distinct from \(v\).

    However, rationalisation only gets relation between \(R\) and action.
    It is not possible to obtain relation between \(\pv{\psi}{v'}\) and \(\Psi\).

    Now, any constraint placed on rationalisations does not apply.
  \end{argument}

  Maybe help with an example.

  I think the only plausible proposal given these constraints is to understand \fingfb{\(\pv{\psi}{v'}\)}{\(\Psi\)} as the \agents{} reason:
  % 
  \begin{itemize}
  \item
    \fingfb{\(\pv{\psi}{v'}\)}{\(\Psi\)} is the \agents{} reason for the conclusion of \(\pv{\phi}{v}\) from \(\Phi\).
  \end{itemize}
  % 
  Yet, we introduced \fingfr{} as a act-neutral account of what follows from what.
  Hence, the above suggests the conclusion of \(\pv{\phi}{v}\) from \(\Phi\) is \emph{from} \fingfb{\(\pv{\psi}{v'}\)}{\(\Psi\)}.
  So, the relevant \fingfr{} which answers \qWhy{} is between \(\pv{\phi}{v}\) and [\fingfb{\(\pv{\psi}{v'}\)}{\(\Psi\)}].
  Yet, our assumption was \fingfb{\(\pv{\psi}{v'}\)}{\(\Psi\)} answers \qWhy{}.
\end{note}


\begin{note}
  Basic point is that rationalisations are different things.
  Restricted.
  Various things may be entailed by a rationalisation, but both \citeauthor{Davidson:1963aa} and \citeauthor{Hieronymi:2011aa} argue for a constraint on rationalisations.
  Neither \citeauthor{Davidson:1963aa} and \citeauthor{Hieronymi:2011aa} argue for a constraint on any distinct relation.
\end{note}

\begin{note}
  Further, given results of main argument:

  \begin{proposition}
    It is not the case that:
    \begin{itenum}
    \item[\emph{If}:]
      \fingfb{\(\pv{\psi}{v'}\)}{\(\Phi\)} answers \qWhy{}
    \item[\emph{Then}:]
      \begin{itemize}
      \item
        \(\psi = \phi\), \(v = v'\), \(\Psi = \Phi\).
      \item
        The relation between \(\Phi\) and the act in which the agent concludes \(\pv{\phi}{v}\) from \(\Phi\) is a rationalisation.
      \end{itemize}
    \end{itenum}
  \end{proposition}%
  \footnote{
    Variants of \qWhy{} and \qHow{} may also be stated in terms of the (epistemic) basing relation.
    Where, the basing relation is understood as a \textquote{relation which holds between a reason and a belief if and only if the reason is a reason for which the belief is held} (\cite{Korcz:2021ue}).

    In contrast to rationalisations, instances of the basing relation concern a belief, rather than action.
    Further, instances of the basing relation are typically tied to justification rather than explanation.
    (See \cite{Korcz:2021ue} and \cite[35]{Pollock:1999tm})
    Still, as an abstract relation, accounts of the basing relation may suggest motivation for or against \issueInclusion{}.

    Now, the core of \autoref{obs:whRdifficult} concerns how rationalisations hold fixed an action.
    And, as basing relations hold fixed a belief, a variant of \autoref{obs:whRdifficult} extends to the basing relation.
    So, accounts of the basing relation do not provide direct motivation \emph{for} \issueInclusion{}.

    Still, given variants of \qWhy{} and \qHow{}, motivation \emph{against} \issueInclusion{}.

    For example, \citeauthor{Swain:1981wd}'s (\citeyear{Swain:1981wd}) `causal-counterfactual' account of the basing relation is promising in name.
    Perhaps there are instances of the basing relation which hold given the causal relations of counterfactual events!
    Yet, the relevant counterfactuals of \citeauthor{Swain:1981wd}'s concern events which happened that would have been a cause if the actual cause of an \agents{} belief had not occurred, and hence require a \wit{0}.

    Indeed, the \emph{causal} accounts of the basing relation I have worked through in detail (\cite{Moser:1989tv}, \cite{Ye:2020ux}, and \cite{Turri:2011aa}) seem to motivate \issueInclusion{}, at best.

    \emph{Doxastic} accounts of the basing relation are more permissive, and \citeauthor{Tolliver:1982us}'s (\citeyear{Tolliver:1982us}) account is compatible with failures of a variant of \issueInclusion{} as the account makes no reference to anything that happened, only what the agent believes at a given time.
    Still, the examples \citeauthor{Tolliver:1982us} gives to motivate their account are consistent with a variant of \issueInclusion{}.

    Further, some care must be taken in order to understand the way an account of the basing relation is connected to concluding.
    For example, taken at face value, \citeauthor{Evans:2013tw}'s disposition account of the basing relation holds:
    \textquote{S's belief that \emph{p} is based on \emph{m} iff S is disposed to revise her belief that \emph{p} when she loses \emph{m}} (\citeyear[2952]{Evans:2013tw})/
  This is suggestive of counterexamples to a variant of \issueInclusion{} as an agent may be disposed without witnessing the disposition.
  However, \citeauthor{Evans:2013tw}' theory is designed to capture what \emph{sustains} an agent's belief.%
  And, our interest with \issueInclusion{} is restricted to the \eiw{0} an agent concludes \(\pv{\phi}{v}\) from \(\Phi\).
  \issueInclusion{} is silent about whatever happens after an agent concludes.

  Similar observations extend to \citeauthor{Moretti:2019wx}'s (\citeyear{Moretti:2019wx}) account of basing via enthymematic inferences, and to  \citeauthor{Audi:1986to}'s suggestion of cases where `[b]elieving for a reason does not entail having \textbf{come} to believe for that reason, or for any reason' (\citeyear[32--33]{Audi:1986to}).
  }
\end{note}


\begin{note}
  The upshot of \autoref{prop:no-r} entails it is not possible for \issueInclusion{} to follow from either \citeauthor{Davidson:1963aa}'s or \citeauthor{Hieronymi:2011aa}'s account of rationalisations.
\end{note}


\section{Agency?}
\label{sec:agency}

\begin{note}
  Still, there is distinction between psychological facts and the exercise of agency.
  To illustrate, there a senses with which the following two questions are distinct:

  \begin{itemize}
  \item
    \emph{Why} is \(e\) is an \eiw{0} \vAgent{} concludes \(\pv{\phi}{v}\)?
  \item
    \emph{Why} is it the case \vAgent{} concludes \(\pv{\phi}{v}\) in \(e\)?
  \end{itemize}
  %
  Consider \citeauthor{Davidson:1973vd}'s climber:
  %
  \begin{quote}
    A climber might want to rid himself of the weight and danger of holding another man on a rope, and he might know that by loosening his hold on the rope he could rid himself of the weight and danger.
    This belief and want might so unnerve him as to cause him to loosen his hold, and yet it might be the case that he never chose to loosen his hold, nor did he do it intentionally.%
    \mbox{ }\hfill\mbox{(\citeyear[79]{Davidson:1973vd})}
  \end{quote}
  %
  The \agents{} belief and want are psychological facts about the agent, and answer why \emph{the event} is an \eiw{0} the agent loosens their hold on the rope.
  However, the \agents{} belief and want do not answer why \emph{the agent} loosens their hold on the rope.
  Indeed, there seems no answer to why the agent loosens their hold on the rope --- the act was not an exercise of agency.

  \citeauthor{Davidson:1973vd}'s climber does not conclude.
  Still, if an agent may loosen their grip without exercising their agency, it seems plausible \fingfb{\(\pv{\psi}{v'}\)}{\(\Psi\)} may fail to answer why \eiw[\(e\)]{0} an agent concludes \(\pv{\phi}{v}\).

  In short, it is not clear that any counterexample to \issueInclusion{} relates to an exercise of agency.
  Whether this matters is for you to decide.
  On some days I think this matters a great deal, and on other days I doubt there is any coherent idea of what an exercise of agency amounts to.
\end{note}


\begin{note}
  \color{blue}
  The main body of the document counterexample to \issueInclusion{}.

  This does not entail agency.

  Still, concluding.
  Concluding is an act of agency.
  Of interest is that the agent is concluding \(\pv{\phi}{v}\) from \(\Phi\).
  It is not clear to me whether it is possible to separate an act of concluding from an act of agency.
\end{note}


%   \begin{quote}
%     Sometimes the explanation of why a person does something has a particular character:
%     roughly, it involves the person's rationality in a distinctive way that I shall not try to describe.
%     Then we say the person does what she does for a reason.
%     We might say ‘The reason for which Hannibal used elephants was to terrorize the Romans'.
%     The reason for which a person does something is called a ‘motivating reason'.
%     In general, a motivating reason is whatever explains or helps to explain what a person does in the distinctive way that involves her rationality.
%     \mbox{}\hfill\mbox{(\citeyear[46--47]{Broome:2013aa})}
%   \end{quote}
% \end{note}

% \begin{note}
%   \color{red}
%   \begin{quote}
%     \emph{R} is a primary reason why an agent performed the action \emph{A} under the description \emph{d} only if \emph{R} consists of a pro attitude of the agent toward actions with a certain property, and a belief of the agent that \emph{A}, under the description \emph{d}, has that property.\newline
%     \mbox{ }\hfill\mbox{(\citeyear[687]{Davidson:1963aa})}
%   \end{quote}

%   We have distinguished \qWhy{} from pro-attitudes.
%   However, fill in whatever motivation.
%   What matters is the belief.
%   This is the relevant proposition-value pair.

%   If \citeauthor{Davidson:1963aa}, then granting restriction, seems we don't need to look beyond the proposition-value pair.
% \end{note}



%%% Local Variables:
%%% mode: latex
%%% TeX-master: "master"
%%% TeX-engine: luatex
%%% End:
