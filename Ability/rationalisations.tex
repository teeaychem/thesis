\chapter{\qWhy{}, \qHow{}, and rationalisations}
\label{sec:reasons}


\begin{note}
  \citeauthor{Davidson:1963aa} opens \citetitle{Davidson:1963aa} with the following question:
  % 
  \begin{quote}
    What is the relation between a reason and an action when the reason explains the action by giving the agent's reason for doing what he did?
    We may call such explanations \emph{rationalizations}, and say that the reason \emph{rationalizes} the action.%
    \mbox{}\hfill\mbox{(\citeyear[685]{Davidson:1963aa})}
  \end{quote}
  % 
  In short, a rationalisation is an explanatory reason for why an agent did some action which states the \agents{} reason for doing what they did.%
  \footnote{
    The \agents{} reason may be understood, in line with \citeauthor{Smith:1994wo} (\citeyear{Smith:1994wo}), as a `motivating reason' where
    \textquote{[t]he distinctive feature of a motivating reason to \(\phi\) is that, in virtue of having such a reason, an agent is in a state that is \emph{explanatory} of her \(\phi\)-ing, at least other things being equal --- other things must be equal because an agent may have a motivating reason to \(\phi\) without that reason's being overriding}
    (\citeyear[96]{Smith:1994wo}).
  }

  Variants of \qWhy{} and \qHow{} suited to rationalisations and actions follow:

  \begin{question}{questionWhyR}{\qWhyR{}}
    Given \(e\) is an event in which \vAgent{} does \(a\):

    \begin{itemize}
    \item
      Which rationalisations explain \emph{why} \(e\) is an event in which \vAgent{} does \(a\)?
    \end{itemize}
    \vspace{-1.5\baselineskip}
  \end{question}

  \begin{question}{questionHowR}{\qHowR{}}
    Given \(e\) is an event in which \vAgent{} does \(a\):

    \begin{itemize}
    \item
      Which past or present events explain \emph{how} \(e\) is an event in which \vAgent{} does \(a\)?
    \end{itemize}
    \vspace{-1.5\baselineskip}
  \end{question}

  \noindent%
  \qWhyR{} seeks an understanding of the way an event developed such that an agent did some action in terms of rationalisations.
  In contrast, \qWhy{} seeks an understanding of the way an event developed such that an agent concludes \(\phi\) has value \(v\) from \(\Phi\) in terms of \ros{}.
  Likewise, \qHowR{} seeks an understanding of the way an event developed such that an agent did some action in terms of what happened, while \qHow{} is a restriction of \qHowR{} to events in which the relevant action is `concludes \(\phi\) has value \(v\) from \(\Phi\)'.
\end{note}

\begin{note}
  \qWhyR{} is a variant of \qWhy{} and \qHowR{} is a variant of \qHow{}.
  We make two observations.

  \begin{enumerate}
  \item
    Variants of \issueInclusion{} have been proposed for \qWhyR{} and \qHow{}.
  \item
    There is no straightforward motivation for \issueInclusion{} by constraints on answers to \qWhyR{} in terms of answers to \qHow{}.
  \end{enumerate}
\end{note}

\begin{note}
  \begin{observation}%
    \label{obs:whRconstraintMotivation}%
    Variants of \issueInclusion{} have been proposed for \qWhyR{} and \qHow{}.
  \end{observation}
  \begin{motivation}{obs:whRconstraintMotivation}
    We consider proposals by \citeauthor{Davidson:1963aa} and \citeauthor{Hieronymi:2011aa} in turn.
    \medskip

    \noindent%
    \citeauthor{Davidson:1963aa}'s argument of \citetitle{Davidson:1963aa} is:
    % 
    \begin{quote}
      [R]ationalization is a species of ordinary causal explanation.%
      \mbox{ }\hfill\mbox{(\citeyear[685]{Davidson:1963aa})}
    \end{quote}
    % 
    So, a reason explains the action by giving the agent's reason for doing what he did when the agent's reason \emph{causes} the action.

    A causal relation captures how something happened, and \citeauthor{Davidson:1963aa} holds causal relation explains how event happens.
    Hence, \citeauthor{Davidson:1963aa} constrains answers to \qWhyR{} by answers to \qHowR{}.
    \medskip

    \noindent%
    In contrast to causal explanation, \citeauthor{Hieronymi:2011aa} (\citeyear{Hieronymi:2011aa}) appeals to a complex fact:%
    \footnote{
      Consider also \citeauthor{Harman:1973ww}'s account of reason explanation:
      \textquote{Reasons may or may not be causes; but explanation by reasons is not causal or deterministic explanation.
        It describes the sequence of considerations that led to belief in a conclusion without supposing that the sequence was determined} (\citeyear[52]{Harman:1973ww}).
    }
    % 
    \begin{quote}
      [W]henever an agent acts for reasons, the agent, in some sense, takes certain considerations to settle the question of whether so to act, therein intends so to act, and executes that intention in action.

      If this much is uncontroversial (and, under some interpretation, I believe it must be), we can use it as a form for filling out.
      I propose, then, that we explain an event that is an action done for reasons by appealing to the fact that the agent took certain considerations to settle the question of whether to act in some way, therein intended so to act, and successfully executed that intention in action.
      I suggest that \emph{this} complex fact [\dots] is the reason that rationalizes the action---that explains the action by giving the agent's reason for acting.%
      \mbox{ }\hfill\mbox{(\citeyear[421]{Hieronymi:2011aa})}
    \end{quote}
    % 
    In short, \citeauthor{Hieronymi:2011aa} observes a particular (complex) fact holds when an agent agents for reasons.
    If we grant that the fact \citeauthor{Hieronymi:2011aa} observes explains how an agent did what they did, then \citeauthor{Hieronymi:2011aa}'s proposal likewise constrains answers to \qWhyR{} in terms of answers to \qHowR{} --- in order for a rationalisation to explain why and agent did what they did there must be an event where the agent settles what to do via the (\agents{}) reasons present in the rationalisation.
  \end{motivation}

  \noindent%
  In short, the arguments/proposals of \citeauthor{Davidson:1963aa} and \citeauthor{Hieronymi:2011aa} entail a constraint which parallels \issueInclusion{}, with respect to the variant questions \qWhyR{} and \qHowR{}.

  Still, to argue \issueInclusion{} follows from either \citeauthor{Davidson:1963aa}'s or \citeauthor{Hieronymi:2011aa}'s account of rationalisations requires an argument for the following conditional:
  % 
  \begin{itemize}
  \item
    \emph{If} a \ros{} answers \qWhy{} \emph{then} there is a corresponding rationalisation.
  \end{itemize}
  % 
  Given this conditional, \ros{1} which answer \qWhy{} are constrained by answers to \qHowR{}.
  And, answers to \qHow{} are just a special case of answers to \qHowR{} where the action is the agent's conclusion of \(\phi\) having value \(v\) from \(\Phi\).
\end{note}

\begin{note}
  Intuitively, certain \ros{1} correspond to rationalisations.
  For example, we may casually say with relative ease:%
  \footnote{
    This intuition ignores details about what \citeauthor{Davidson:1963aa} and \citeauthor{Hieronymi:2011aa} think reasons are.
  }

  \begin{itemize}
  \item
    The reason for which the agent concluded \propM{\gistCalcEq{}} is \valI{True} was the calculator.
  \item
    The reason for which the birds concluded \propI{Four legs good, two legs bad} is \valI{True} was (the existence of) Snowball's explanation.
  \end{itemize}
  % 
  Abstractly, the following conditional seems plausible:
  \begin{itemize}
  \item
    \begin{itenum}
    \item[\emph{If}:]
      \(e\) is an event in which an agent concludes \(\phi\) has value \(v\) from \(\Phi\).
    \item[\emph{And}:]
      A \ros{} between \(\phi\), \(v\), and \(\Phi\) answers \qWhy{}.
    \item[\emph{Then}:]
      \(\Phi\) rationalises the act in which the agent concludes \(\phi\) has value \(v\).
    \end{itenum}
  \end{itemize}
  % 
  Still, we observed any counterexample to \issueInclusion{} must involve a \ros{} between \(\psi\), \(v'\), and \(\Psi\) (where either \(\psi\) is distinct from \(\phi\), \(v'\) is distinct from \(v\), or \(\Psi\) is distinct from \(\Phi\)).
  Hence, the plausible conditional does not cover \ros{1} which \emph{may} be counterexamples to \issueInclusion{}.

  \begin{observation}
    \label{obs:whRdifficult}%
    There is no straightforward motivation for \issueInclusion{} by constraints on answers to \qWhyR{} in terms of answers to \qHow{}.

    In particular, there is no way link a \ros{} between \(\psi\), \(v'\), and \(\Psi\) to a rationalisation if either \(\psi\) is distinct from \(\phi\) or \(v\) is distinct from \(v'\).
  \end{observation}

  \begin{motivation}{obs:whRdifficult}%
    A rationalisation explains an action by giving an agent's reason for doing the action.
    Hence, rationalisations have a single free variable for the \agents{} reason.

    Applied to conclusions, rationalisations are of the form:
    % 
    \begin{itemize}
    \item
      \(R\) is the \agents{} reason for the conclusion \(\phi\) has value \(v\) (from \(\Phi\)).
    \end{itemize}
    % 
    Where \(R\) stands for an \agents{} reason.

    Now, assume a \ros{} between \(\psi\), \(v'\), and \(\Psi\) answers \qWhy{}, where \(\psi\) is distinct from \(\phi\) and \(v'\) is distinct from \(v\).

    By assumption, the \ros{} answers \qWhy{}, and so to capture the explanatory relation between the \ros{} and the \ros{} and the \agents {} conclusion \(\phi\) has value \(v\) (from \(\Phi\)), \(\phi\) or \(v'\) must be referenced by \(R\).

    Further \(\psi\) is distinct from \(\phi\) and \(v'\) is distinct from \(v\), the relevant rationalisation is for the conclusion \(\psi\) has value \(v'\) (from \(\Phi\)).

    I think the only plausible proposal given these constraints is to understand the \ros{} between \(\psi\), \(v'\) and \(\Psi\) must be understood as the \agents{} reason:
    % 
    \begin{itemize}
    \item
      The \ros{} between \(\psi\), \(v'\), and \(\Psi\) is the \agents{} reason for the conclusion \(\phi\) has value \(v\) (from \(\Phi\)).
    \end{itemize}
    % 
    Yet, we introduced \ros{} as a act-neutral account of what follows from what.
    Hence, the above suggests conclusion \(\phi\) has value \(v\) (from \(\Phi\)) is \emph{from} the \ros{} between \(\psi\), \(v'\), and \(\Psi\).
    So, the relevant \ros{} which answers \qWhy{} is a \ros{} between \(\phi\) having value \(v\) and [the \ros{} between \(\psi\), \(v'\), and \(\Psi\)].
    Yet, our assumption was the \ros{} between \(\psi\), \(v'\), and \(\Psi\) answers \qWhy{}.
  \end{motivation}

  \noindent%
  A rationalisation implicitly may capture a specific kind of \ros{}.
  However, as a rationalisation is, at best, restricted to a specific \ros{1}, rationalisations are not suitable for capturing \ros{1} in general.

  So, while certain accounts of rationalisations constrain answers to a why question by answers to a how question in a similar way to \issueInclusion{}, I doubt it is possible to motivate \issueInclusion{} directly by those accounts of rationalisations.

  To progress further some additional premise is needed.
  For example, if you hold:
  % 
  \begin{itemize}
  \item
    A \ros{} answers \qWhy{} \emph{only if} there is a corresponding rationalisation that answers \qWhyR{} (in line with the plausible conditional above).
  \end{itemize}
  % 
  Then a more carefully developed version of \autoref{obs:whRdifficult} entails it is not possible for a \ros{} between \(\phi\), \(v'\), and \(\Phi\) to answer \qWhy{} (where \(\psi\) is distinct from \(\phi\) or \(v'\) is distinct from \(v\)).
  And, combined with \citeauthor{Davidson:1963aa}'s or \citeauthor{Hieronymi:2011aa}'s account of rationalisations, \issueInclusion{} follows.%
  \footnote{
    Variants of \qWhy{} and \qHow{} may also be stated in terms of the (epistemic) basing relation.
    Where, the basing relation is understood as a \textquote{relation which holds between a reason and a belief if and only if the reason is a reason for which the belief is held} (\cite{Korcz:2021ue}).

    In contrast to rationalisations, instances of the basing relation concern a belief, rather than action.
    Further, instances of the basing relation are typically tied to justification rather than explanation.
    (See \cite{Korcz:2021ue} and \cite[35]{Pollock:1999tm})
    Still, as an abstract relation, accounts of the basing relation may suggest motivation for or against \issueInclusion{}.

    Now, the core of \autoref{obs:whRdifficult} concerns how rationalisations hold fixed an action.
    And, as basing relations hold fixed a belief, a variant of \autoref{obs:whRdifficult} extends to the basing relation.
    So, accounts of the basing relation do not provide direct motivation \emph{for} \issueInclusion{}.

    Still, given variants of \qWhy{} and \qHow{}, motivation \emph{against} \issueInclusion{}.

    For example, \citeauthor{Swain:1981wd}'s (\citeyear{Swain:1981wd}) `causal-counterfactual' account of the basing relation is promising in name.
    Perhaps there are instances of the basing relation which hold given the causal relations of counterfactual events!
    Yet, the relevant counterfactuals of \citeauthor{Swain:1981wd}'s concern events which happened that would have been a cause if the actual cause of an \agents{} belief had not occurred, and hence require a \wit{0}.

    Indeed, the \emph{causal} accounts of the basing relation I have worked through in detail (\cite{Moser:1989tv}, \cite{Ye:2020ux}, and \cite{Turri:2011aa}) seem to motivate \issueInclusion{}, at best.

    \emph{Doxastic} accounts of the basing relation are more permissive, and \citeauthor{Tolliver:1982us}'s (\citeyear{Tolliver:1982us}) account is compatible with failures of a variant of \issueInclusion{} as the account makes no reference to anything that happened, only what the agent believes at a given time.
    Still, the examples \citeauthor{Tolliver:1982us} gives to motivate their account are consistent with a variant of \issueInclusion{}.

    Further, some care must be taken in order to understand the way in which an account of the basing relation is connected to concluding.
    For example, taken at face value, \citeauthor{Evans:2013tw}'s disposition account of the basing relation holds:
    \textquote{S's belief that \emph{p} is based on \emph{m} iff S is disposed to revise her belief that \emph{p} when she loses \emph{m}} (\citeyear[2952]{Evans:2013tw})/
  This is suggestive of counterexamples to a variant of \issueInclusion{} as an agent may be disposed without witnessing the disposition.
  However, \citeauthor{Evans:2013tw}' theory is designed to capture what \emph{sustains} an agent's belief.%
  And, our interest with \issueInclusion{} is restricted to the event in which an agent concludes \(\phi\) has value \(v\) from \(\Phi\).
  \issueInclusion{} is silent about whatever happens after an agent concludes.

  Similar observations extend to \citeauthor{Moretti:2019wx}'s (\citeyear{Moretti:2019wx}) account of basing via enthymematic inferences, and to  \citeauthor{Audi:1986to}'s suggestion of cases in which `[b]elieving for a reason does not entail having \textbf{come} to believe for that reason, or for any reason' (\citeyear[32--33]{Audi:1986to}).
  % Expansion of an enthymematic inferences may be basis for an \agents{} belief with respect to justification, but it is not clear the expansion of an enthymematic inference matters with respect to an \agents{} conclusion if the agent concludes via an enthymematic inference.
  % 
  % An agent may come to believe for one reason, and sustain the belief by some other reason, but unless the sustaining reason explains why the agent formed the initial belief, such cases are of no interest to us.
  }
\end{note}

% \begin{note}
%   Motivation.
%   Difficulty here is that both \citeauthor{Davidson:1963aa} and \citeauthor{Hieronymi:2011aa} are concerned with an agent's reason(s).

%   It is not clear that \ros{} which answers \qWhy{} captures agent's reasons.


%   In general, lots of this explain why.

%   I do not think \issueInclusion{} is too distinct.
%   \ros{} is from the \agpe{}.
%   Hence, answers to \qWhy{} are close.
%   But this is still distinct from a reason.
% \end{note}


\section{Agency?}
\label{sec:agency}

\begin{note}
  Still, there is distinction between psychological facts and the exercise of agency.
  To illustrate, there a senses in which the following two questions are distinct:

  \begin{itemize}
  \item
    \emph{Why} is \(e\) is an event in which \vAgent{} concludes \(\phi\) has value \(v\)?
  \item
    \emph{Why} is it the case \vAgent{} concludes \(\phi\) has value \(v\) in \(e\)?
  \end{itemize}
  %
  Consider \citeauthor{Davidson:1973vd}'s climber:
  %
  \begin{quote}
    A climber might want to rid himself of the weight and danger of holding another man on a rope, and he might know that by loosening his hold on the rope he could rid himself of the weight and danger.
    This belief and want might so unnerve him as to cause him to loosen his hold, and yet it might be the case that he never chose to loosen his hold, nor did he do it intentionally.%
    \mbox{ }\hfill\mbox{(\citeyear[79]{Davidson:1973vd})}
  \end{quote}
  %
  The \agents{} belief and want are psychological facts about the agent, and answer why \emph{the event} is an event in which the agent loosens their hold on the rope.
  However, the \agents{} belief and want do not answer why \emph{the agent} loosens their hold on the rope.
  Indeed, there seems no answer to why the agent loosens their hold on the rope --- the act was not an exercise of agency.

  \citeauthor{Davidson:1973vd}'s climber does not conclude.
  Still, if an agent may loosen their grip without exercising their agency, it is seems plausible that a \ros{} between \(\psi\), \(v'\) and \(\Psi\) which answers why is \emph{\(e\) is an event} in which \vAgent{} concludes \(\phi\) has value \(v\) may fail to answer why is it the case the agent concludes \(\phi\) has value \(v\) in \(e\).

  In short, it is not clear that any counterexample to \issueInclusion{} relates to an exercise of agency.
  Whether this matters is for you to decide.
  On some days I think this matters a great deal, and on other days I doubt there is any coherent idea of what an exercise of agency amounts to.

  We set concerns about agency aside for the argument against \issueInclusion{}, and then motivate that relevant counterexamples either involve agency or have nearby neighbours which involve agency.
\end{note}


%   \begin{quote}
%     Sometimes the explanation of why a person does something has a particular character:
%     roughly, it involves the person's rationality in a distinctive way that I shall not try to describe.
%     Then we say the person does what she does for a reason.
%     We might say ‘The reason for which Hannibal used elephants was to terrorize the Romans'.
%     The reason for which a person does something is called a ‘motivating reason'.
%     In general, a motivating reason is whatever explains or helps to explain what a person does in the distinctive way that involves her rationality.
%     \mbox{}\hfill\mbox{(\citeyear[46--47]{Broome:2013aa})}
%   \end{quote}
% \end{note}

% \begin{note}
%   \color{red}
%   \begin{quote}
%     \emph{R} is a primary reason why an agent performed the action \emph{A} under the description \emph{d} only if \emph{R} consists of a pro attitude of the agent toward actions with a certain property, and a belief of the agent that \emph{A}, under the description \emph{d}, has that property.\newline
%     \mbox{ }\hfill\mbox{(\citeyear[687]{Davidson:1963aa})}
%   \end{quote}

%   We have distinguished \qWhy{} from pro-attitudes.
%   However, fill in whatever motivation.
%   What matters is the belief.
%   This is the relevant proposition-value pair.

%   If \citeauthor{Davidson:1963aa}, then granting restriction, seems we don't need to look beyond the proposition-value pair.
% \end{note}



%%% Local Variables:
%%% mode: latex
%%% TeX-master: "master"
%%% TeX-engine: luatex
%%% End:
