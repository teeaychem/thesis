\chapter{\issueInclusion{} and \rationalisation{1}}
\label{sec:reasons}


\begin{note}
  This chapter highlights some similarities and differences between \fofr{1} and \rationalisation{1}.

  In short, \fofr{1} and \rationalisation{1} are distinct abstractions which plausibly coincide in certain cases.
  Further, extant accounts of \rationalisation{1} plausibly motive \issueInclusion{} when \fofr{1} and \rationalisation{1} coincide.
\end{note}

\begin{note}
  Distinction between two different abstractions that plausibly overlap in certain cases.
\end{note}


\subsection{\rationalisation{3} and \motingr{1}}

\begin{note}
  The definition of a \rationalisation{0} we work with follows (\cite{Davidson:1963aa}):%
  \footnote{
    \label{fn:DRat}
    For comparison, \citeauthor{Davidson:1963aa} writes:
    \begin{quote}
      What is the relation between a reason and an action when the reason explains the action by giving the agent's reason for doing what he did?
      We may call such explanations \emph{rationalizations}[.]%
      \mbox{}\hfill\mbox{(\citeyear[685]{Davidson:1963aa})}
    \end{quote}
    %
    To help clarify things a little we omit a \rationalisation{} being a reason.
    Sometimes it's enjoyable to sink into confusion over whether a reason being referred to is our reason which explains an action or the \agents{} reason for doing the action, but too bad.
  }

  \begin{definition}[\rationalisation{3}]%
    \label{def:rationalisation}%
    Given \(e\) is an \eiw{0} an agent does action \(a\):
    %
    \begin{itemize}
    \item
      A \emph{\rationalisation{0}} explains \(e\) by giving the \agents{} reason for doing \(a\).
    \end{itemize}
    %
    For some sense of `explanation' and `reason'.
  \end{definition}

  \noindent%
  \autoref{def:rationalisation} does not specify a sense of `explanation' and `reason'.
  In this respect, \autoref{def:rationalisation} captures the \emph{structure} of \rationalisation{1}.
  So, when we speak of a \rationalisation{0} we are talking about an explanation (in some sense of the term) of an \eiw{} an agent does some action \emph{such that} the reason gives the \agents{} for doing the action.%
  \footnote{
    This may follow \citeauthor{Davidson:1963aa}'s introduction of the term.
    For, the passage in \autoref{fn:DRat} is from the beginning of \citeauthor{Davidson:1963aa}'s paper.
    And, \citeauthor{Davidson:1963aa} goes on to argue for specific senses of `explanation' and `reason'.

    So, \citeauthor{Davidson:1963aa} may be understood as filling in a structural definition.
    On the other hand, \citeauthor{Davidson:1963aa} may be understood as clarifying the sense of `explanation' and `reason' already present in their account of \rationalisation{1}.

    Either way, our interest is with a structural definition.
  }

  Key features:

  \begin{itemize}
  \item
    A \rationalisation{} is stated from \agpe{our} and explains.
  \item
    The \agents{} reason is included in the reason.
  \item
    Constraint on rationalisations, constraint on \agents{} reasons.
  \end{itemize}

  \citeauthor{Davidson:1963aa}'s argument of \citetitle{Davidson:1963aa} is:
  %
  \begin{quote}
    [R]ationalization is a species of ordinary causal explanation.%
    \mbox{ }\hfill\mbox{(\citeyear[685]{Davidson:1963aa})}
  \end{quote}

  So, included in a \rationalisation{0} is a relation which holds from the \agpe{}.
  Distinction between a reason which is a \rationalisation{0} and an \agents{} reason which is given by a \rationalisation{0}.
\end{note}


\begin{note}
  Given \citeauthor{Davidson:1963aa}, plausible parallel to \issueInclusion{}.

  \begin{observation}
    \label{prop:IssIn-D}
    Granting \rationalisation{} is a species of ordinary causal explanation:
    \begin{itenum}
    \item[\emph{If}:]
      \motb[1]{A \pool{} \(\Phi\)}{\evaling{} \(\phi\) as having \val{} \(v\)} is part of a \rationalisation{}.
    \item[\emph{Then:}]
      Either:
      \begin{itemize}
      \item
        The agent concluded \(\pv{\phi}{v}\) from \(\Phi\).
      \item
        Deviant causal chain.
      \end{itemize}
    \end{itenum}
  \end{observation}

  \begin{motivation}{prop:IssIn-D}
    Has to be causation.
    So, causally involved.
    In the case of conclusion, if the agent does not conclude \(\pv{\phi}{v}\) from \(\Phi\) then must be deviant.

    So, a reason explains the action by giving the agent's reasons for doing what they did when the agent's reasons are part of a \emph{causal} explanation of the action.%

    A causal relation captures how something happened, and \citeauthor{Davidson:1963aa} holds causal relation explains how event happens.

    However, at issue is `from'.
    Climber suggests a counterexample.
    I don't think this is too problematic, though.
  \end{motivation}

  So, if the argument is refined, parallel to \issueInclusion{}.

  Indeed, alternative accounts of \rationalisation{1} which avoid deviant causal chains may strengthen this.%
  \footnote{
    In contrast to causal explanation, \citeauthor{Hieronymi:2011aa} (\citeyear{Hieronymi:2011aa}) appeals to a complex fact:%
    %
    \begin{quote}
      [W]henever an agent acts for reasons, the agent, in some sense, takes certain considerations to settle the question of whether so to act, therein intends so to act, and executes that intention in action.

      If this much is uncontroversial (and, under some interpretation, I believe it must be), we can use it as a form for filling out.
      I propose, then, that we explain an event that is an action done for reasons by appealing to the fact that the agent took certain considerations to settle the question of whether to act in some way, therein intended so to act, and successfully executed that intention in action.
      I suggest that \emph{this} complex fact [\dots] explains the action by giving the agent's reason for acting.%
      \mbox{ }\hfill\mbox{(\citeyear[421]{Hieronymi:2011aa})}
    \end{quote}
    %
    Now, is it possible for an agent to take their understanding of factorisation to settle without concluding from?
  }%
  \(^{,}\)%
  \footnote{
    Consider also \citeauthor{Harman:1973ww}'s account of reason explanation:
    \textquote{Reasons may or may not be causes; but explanation by reasons is not causal or deterministic explanation.
      It describes the sequence of considerations that led to belief in a conclusion without supposing that the sequence was determined} (\citeyear[52]{Harman:1973ww}).
  }
\end{note}

\paragraph*{}



\begin{note}
  Unfortunately, i.

  \begin{proposition}
    \label{prop:NotRats}
    It is not possible for any account of \rationalisation{1} to constrain answers to \qWhy{}
  \end{proposition}

  \begin{argument}{prop:NotRats}
    \rationalisation{3} are explanations of an \eiw{} an agent does an action by giving the \agents{} reason for doing the action.

    So, an account of \rationalisation{1} covers a relation between an \agents{} reason and an action.

    A \fingfr{} is a relation between some \prop{0}-\val{0} pair and some \pool{}.

    So, in order for an account of \rationalisation{1} to constraint answers to \qWhy{} there are two options:

    \begin{enumerate}
    \item
      \prop{0}-\val{0} pair action, \agents{} reason \pool{}.
    \item
      \prop{0}-\val{0} pair \agents{} reason, action \pool{}.
    \end{enumerate}

    So, consider an event in which an agent concludes \(\pv{\phi}{v}\) from \(\Phi\) and \fingfb{\(\pv{\psi}{v'}\)}{\(\Psi\)} answers \qWhy{}, where:
    \begin{itemize}
    \item
      \(\psi \ne \phi\), \(\v \ne v'\).
    \end{itemize}

    Event is conclusion.
    Specific action is \evalion{} of \(\phi\) to have \val{} \(v\).

    But, \(\psi \ne \phi\), \(\v \ne v'\).
    So, \(\pv{\psi}{v'}\) is 
  \end{argument}





  
  Now, this document focuses on \eiw{} an agent concludes some \prop{0}-\val{0} pair \(\pv{\phi}{v}\) from some \pool{} \(\Phi\).
  And, our interest with \rationalisation{1} is due to a possible connexion between \rationalisation{} and answers to \qWhy{}.

  Answers to \qWhy{}, these are parts of explanations.
  In this respect, the \fingfr{} obtained by an answer to \qWhy{} is part of something which explains an event.
  And, in this respect is `on par' with the \agents{} reason given by a \rationalisation{0}.


  
  To apply \rationalisation{1} to an \eiw{} the agent concludes \(\pv{\phi}{v}\) from \(\Phi\) we need to specify an action of interest.
  And, in our terminology, the relevant action is the \agents{} \evaling{} \(\phi\) to have \val{0} \(v\).

  \begin{itemize}
  \item
    \motingb{The \agents{} understanding of factorisation}{\evaling{} \propM{\rootsCon{}} as \valI{True}}.
  \item
    \motingb{The \agents{} understanding of a recursive definition of the Lucas numbers}{\evaling{} \propI{The first ten Lucas numbers are 2, 1, 3, 4, 7, 11, 18, 29, 47, and 76} as \valI{True}}.
  \end{itemize}
  
  And, an \eiw{} the agent concludes \(\pv{\phi}{v}\) from \(\Phi\) spans the \agents{} reasoning from \(\Phi\) to \(\pv{\phi}{v}\) involves a specific action of interest
  
\end{note}



\begin{note}
  
\end{note}



\section{Parallels}
\label{sec:parallel}

\begin{note}
  \begin{proposition}
    Given an event \(e\) in which the agent concluded \(\pv{\phi}{v}\) from \(\Phi\).

    \begin{itenum}
    \item[\emph{If}:]
      \motb[1]{A \pool{} \(\Psi\)}{\evaling{} \(\psi\) as having \val{} \(v'\)} is part of a \rationalisation{}.
    \item[\emph{Then:}]
      \(\phi = \psi\), \(v = v'\).
    \end{itenum}
  \end{proposition}
\end{note}


\begin{note}
  This sets up a limitation.
  For, suppose get rid of deviance.
  \rationalisation{1} do not cover cases of interest.

  Of course, there may be other ways to obtain some link which is interesting.
  However, none of this is of any direct interest.
  For, so long as focus is given to \rationalisation{1}, there is no clear path to a constraint on \fingfr{1}.
\end{note}


\begin{note}
  \footnote{
    Variants of \qWhy{} and \qHow{} may also be stated in terms of the (epistemic) basing relation.
    Where, the basing relation is understood as a \textquote{relation which holds between a reason and a belief if and only if the reason is a reason for which the belief is held} (\cite{Korcz:2021ue}).

    In contrast to \rationalisation{1}, instances of the basing relation concern a belief, rather than action.
    Further, instances of the basing relation are typically tied to justification rather than explanation.
    
    Still, as an abstract relation, accounts of the basing relation may suggest motivation for or against \issueInclusion{}.

    Now, the core of \autoref{obs:whRdifficult} concerns how \rationalisation{1} hold fixed an action.
    And, as basing relations hold fixed a belief, a variant of \autoref{obs:whRdifficult} extends to the basing relation.
    So, accounts of the basing relation do not provide direct motivation \emph{for} \issueInclusion{}.

    Still, given variants of \qWhy{} and \qHow{}, motivation \emph{against} \issueInclusion{}.

    For example, \citeauthor{Swain:1981wd}'s (\citeyear{Swain:1981wd}) `causal-counterfactual' account of the basing relation is promising in name.
    Perhaps there are instances of the basing relation which hold given the causal relations of counterfactual events!
    Yet, the relevant counterfactuals of \citeauthor{Swain:1981wd}'s concern events which happened that would have been a cause if the actual cause of an \agents{} belief had not occurred, and hence require a \wit{0}.

    Indeed, the \emph{causal} accounts of the basing relation I have worked through in detail (\cite{Moser:1989tv}, \cite{Ye:2020ux}, and \cite{Turri:2011aa}) seem to motivate \issueInclusion{}, at best.

    \emph{Doxastic} accounts of the basing relation are more permissive, and \citeauthor{Tolliver:1982us}'s (\citeyear{Tolliver:1982us}) account is compatible with failures of a variant of \issueInclusion{} as the account makes no reference to anything that happened, only what the agent believes at a given time.
    Still, the examples \citeauthor{Tolliver:1982us} gives to motivate their account are consistent with a variant of \issueInclusion{}.

    Further, some care must be taken in order to understand the way an account of the basing relation is connected to concluding.
    For example, taken at face value, \citeauthor{Evans:2013tw}'s disposition account of the basing relation holds:
    \textquote{S's belief that \emph{p} is based on \emph{m} iff S is disposed to revise her belief that \emph{p} when she loses \emph{m}} (\citeyear[2952]{Evans:2013tw})/
  This is suggestive of counterexamples to a variant of \issueInclusion{} as an agent may be disposed without witnessing the disposition.
  However, \citeauthor{Evans:2013tw}' theory is designed to capture what \emph{sustains} an agent's belief.%
  And, our interest with \issueInclusion{} is restricted to the \eiw{0} an agent concludes \(\pv{\phi}{v}\) from \(\Phi\).
  \issueInclusion{} is silent about whatever happens after an agent concludes.

  Similar observations extend to \citeauthor{Moretti:2019wx}'s (\citeyear{Moretti:2019wx}) account of basing via enthymematic inferences, and to  \citeauthor{Audi:1986to}'s suggestion of cases where `[b]elieving for a reason does not entail having \textbf{come} to believe for that reason, or for any reason' (\citeyear[32--33]{Audi:1986to}).
  }
\end{note}


\section{Agency?}
\label{sec:agency}

\begin{note}
  Still, there is distinction between psychological facts and the exercise of agency.
  To illustrate, there a senses with which the following two questions are distinct:

  \begin{itemize}
  \item
    \emph{Why} is \(e\) is an \eiw{0} \vAgent{} concludes \(\pv{\phi}{v}\)?
  \item
    \emph{Why} is it the case \vAgent{} concludes \(\pv{\phi}{v}\) in \(e\)?
  \end{itemize}
  %
  Consider \citeauthor{Davidson:1973vd}'s climber:
  %
  \begin{quote}
    A climber might want to rid himself of the weight and danger of holding another man on a rope, and he might know that by loosening his hold on the rope he could rid himself of the weight and danger.
    This belief and want might so unnerve him as to cause him to loosen his hold, and yet it might be the case that he never chose to loosen his hold, nor did he do it intentionally.%
    \mbox{ }\hfill\mbox{(\citeyear[79]{Davidson:1973vd})}
  \end{quote}
  %
  The \agents{} belief and want are psychological facts about the agent, and answer why \emph{the event} is an \eiw{0} the agent loosens their hold on the rope.
  However, the \agents{} belief and want do not answer why \emph{the agent} loosens their hold on the rope.
  Indeed, there seems no answer to why the agent loosens their hold on the rope --- the act was not an exercise of agency.

  \citeauthor{Davidson:1973vd}'s climber does not conclude.
  Still, if an agent may loosen their grip without exercising their agency, it seems plausible \fingfb{\(\pv{\psi}{v'}\)}{\(\Psi\)} may fail to answer why \eiw[\(e\)]{0} an agent concludes \(\pv{\phi}{v}\).

  In short, it is not clear that any counterexample to \issueInclusion{} relates to an exercise of agency.
  Whether this matters is for you to decide.
  On some days I think this matters a great deal, and on other days I doubt there is any coherent idea of what an exercise of agency amounts to.
\end{note}


\begin{note}
  \color{red}
  So, I don't get an answer to this.
  However, I think this is only a problem if there's good reason to think there is no agency involved in the relevant counterexamples.
  The way things are done is compatible with failure.
  But, whatever.
  I see no reason to deny agency, even if you could do this.
\end{note}


\begin{note}
  \color{blue}
  The main body of the document counterexample to \issueInclusion{}.

  This does not entail agency.

  Still, concluding.
  Concluding is an act of agency.
  Of interest is that the agent is concluding \(\pv{\phi}{v}\) from \(\Phi\).
  It is not clear to me whether it is possible to separate an act of concluding from an act of agency.
\end{note}


%   \begin{quote}
%     Sometimes the explanation of why a person does something has a particular character:
%     roughly, it involves the person's rationality in a distinctive way that I shall not try to describe.
%     Then we say the person does what she does for a reason.
%     We might say ‘The reason for which Hannibal used elephants was to terrorize the Romans'.
%     The reason for which a person does something is called a ‘motivating reason'.
%     In general, a motivating reason is whatever explains or helps to explain what a person does in the distinctive way that involves her rationality.
%     \mbox{}\hfill\mbox{(\citeyear[46--47]{Broome:2013aa})}
%   \end{quote}
% \end{note}

% \begin{note}
%   \color{red}
%   \begin{quote}
%     \emph{R} is a primary reason why an agent performed the action \emph{A} under the description \emph{d} only if \emph{R} consists of a pro attitude of the agent toward actions with a certain property, and a belief of the agent that \emph{A}, under the description \emph{d}, has that property.\newline
%     \mbox{ }\hfill\mbox{(\citeyear[687]{Davidson:1963aa})}
%   \end{quote}

%   We have distinguished \qWhy{} from pro-attitudes.
%   However, fill in whatever motivation.
%   What matters is the belief.
%   This is the relevant proposition-value pair.

%   If \citeauthor{Davidson:1963aa}, then granting restriction, seems we don't need to look beyond the proposition-value pair.
% \end{note}



%%% Local Variables:
%%% mode: latex
%%% TeX-master: "master"
%%% TeX-engine: luatex
%%% End:
