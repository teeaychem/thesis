\chapter{\requ{3}}
\label{cha:zS:sec:requs}

\begin{note}
  As observed, dependence in line with dependence identified by \qWhyV{}.
  However, no \ros{}.

  In this chapter, \requ{}.
  From the \agpe{}.
  Converse problem.
  The \agpe{} doesn't establish dependence.
\end{note}

\section{\requ{3}}
\label{sec:requ3}

\begin{note}
  \begin{definition}[A \requ{0}]
    \label{def:requ}
    For:

    \begin{itemize*}[noitemsep]
    \item
      An agent: \vAgent{}
    \item
      Propositions: \(\phi\), \(\psi\)
    \item
      Values: \(v\), \(v'\)
    \item
      \poP{3}: \(\Phi\), \(\Psi\)
    \item
      An event: \(e\)
    \end{itemize*}

    \begin{itemize}
    \item
    \(\pvp{\phi}{v'}{\Psi}\) is a \emph{\requ{}} of \(e\) just in case:

    \begin{enumerate}[label=\alph*., ref=(\alph*), series=requDefSeries]
      \item
        \(e\) is an event in which \vAgent{} is concluding \(\pv{\phi}{v}\) from \(\Phi\).
      \end{enumerate}

    \emph{Only if}

    \begin{enumerate}[label=\alph*., ref=(\alph*), resume*=requDefSeries]
    \item
      Both \ref{def:requ:curb-agpe} and \ref{def:requ:feedback} hold:
      \begin{enumerate}[label=\roman*., ref=(\roman*)]
      \item
        \label{def:requ:curb-agpe}
        \(\pvp{\psi}{v'}{\Psi}\) is a \curb{} on concluding \(\pv{\phi}{v}\) from \(\Phi\), from \agpe{\vAgent{}'}.
      \item
        \label{def:requ:feedback}
        \(\pvp{\psi}{v'}{\Psi}\) is a \curb{} on concluding \(\pv{\phi}{v}\) from \(\Phi\).%, due to \ref{def:requ:curb-agpe}.
      \end{enumerate}
    \end{enumerate}
  \end{itemize}
  \vspace{-\baselineskip}
  \end{definition}

  `\requ{2}' in the sense of `deemed necessary'.

  So, understand \requ{}.
  Here, from \agpe{}.
  Event is an event in which the agent is concluding \(\pv{\phi}{v}\) from \(\Phi\) just in case there is a \pevent{} in which the agent concludes \(\pv{\psi}{v'}\) from \(\Psi\).

  From \ref{def:requ:curb-agpe}, if there's no \pevent{} in which conclude \(\pv{\psi}{v'}\) from \(\Psi\) then it is not the case that agent is concluding \(\pv{\phi}{v}\) from \(\Phi\).

  So, cases where \ref{def:requ:curb-agpe} holds are straightforward.
  Consider various cases we have seen so far.

  And, from \ref{def:requ:feedback}, the agent's perspective influences whether or not the event develops such that the agent concludes \(\pv{\phi}{v}\) from \(\Phi\).

  We term this \feedback{}.
\end{note}


\section{\feedback{2}}

\begin{note}
  \begin{definition}[\feedback{2}]
    If \curb{} from \agpe{}, then \curb{}.
  \end{definition}

  This is what we're really interested in.
  Some self reflection, and concerns about the way in which the present event may develop influence the development of the event.
\end{note}

\begin{note}
  \feedback{} is a non-trivial.
  It is not only the case that block from the \agpe{}, but actual block.
\end{note}

\subsection{\feedback{2} and \fc{1}}
\label{sec:fc}

\begin{note}
  Noted that \curb{} does not entail \fc{}.
  The same holds for \requ{1}.

  For, don't need to ensure that no other conclusion.

  However, \feedback{0} does interesting things.
  Because, if know conclusion, in any non-trivial sense, then cannot be the case that conclude anything else.
\end{note}


\begin{note}
  \begin{proposition}[\requ{3} and \fc{1}]
    For an agent \vAgent{}, and proposition-value-premises pairings \(\pvp{\phi}{v}{\Phi}\), \(\pvp{\psi}{v'}{\Psi}\):

    \begin{itemize}
    \item
      \(\pvp{\phi}{v'}{\Psi}\) is a \requ{} of \vAgent{} concluding \(\pv{\phi}{v}\) from \(\Phi\)
    \item
      Know \pevent{}.
    \item
      Not arbitrary.
    \end{itemize}

    \emph{Only if}

    \begin{itemize}
    \item
        \begin{enumerate}
        \item[\emph{If}:]
          \(\pv{\psi}{v'}\) is not a \fc{} from \(\Psi\), from \agpe{\vAgent{}'}.
        \item[\emph{Then}:]
          \begin{enumerate}[label=\alph*., ref=(\alph*), resume]
          \item
            \label{def:curb:fail}
            \vAgent{} would not be concluding \(\pv{\phi}{v}\) from \(\Phi\).
          \end{enumerate}
      \end{enumerate}
    \end{itemize}
    \begin{argument}
      \requ{}, so \curb{} from perspective.
      So, then via \feedback{0}, get \curb{}.
    \end{argument}
  \end{proposition}
\end{note}

\begin{note}
  Consequence.

  \begin{proposition}
    If \requ{} and \ros{} fails to hold from \agpe{}, then not concluding.
    \begin{argument}
      Suppose \ros{} fails to hold.

      Therefore, not a \fc{}.

      If not a \fc{}, then not concluding.
    \end{argument}
  \end{proposition}

  So:

  \begin{proposition}
    If \requ{}, then if concluding, \ros{} holds.

    \begin{argument}
      Rewriting.
    \end{argument}
  \end{proposition}
\end{note}


\section{\curb{3} as checks on concluding}
\label{cha:zS:sec:curbs:checks}

\begin{note}[\autoref{illu:sketch:prop-logic}]
  Likewise, \autoref{illu:sketch:prop-logic} involves an agent concluding some sentence \(A\) is a syntactic theorem of propositional logic via a formula derivation.

  And, when concluding \(A\) is a syntactic theorem, the agent observes that \(A\) is a syntactic theorem only if \(A\) is also a semantic theorem (from soundness).

  In other words, if the agent attempt to show \(A\) is true under an arbitrary valuation and failed, the agent would not conclude \(A\) is a syntactic theorem.

  Only if semantic proof.
  Syntactic proofs, at least in my experience, may be out of reach.
  However, semantic proofs, often straightforward.
\end{note}

\begin{note}
  Here, highlight initial scenario.
  For, there, testimony.
  Overrides agent's own reasoning, plausibly.
\end{note}


%%% Local Variables:
%%% mode: latex
%%% TeX-master: "master"
%%% End:
