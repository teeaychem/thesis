\chapter{Lost keys and sound rules}
\label{cha:zS:sec:lost-keys}

\begin{note}
  Previous chapter, \fc{1}.
  Potential event in which agent concludes.

  \fc{3} are about what the agent may reason to.

  Now, turn to broader, whether there is something incompatible.
  Alternative conclusion.
\end{note}

\begin{note}
  Use the \illu{} to provide a general introduction.
\end{note}


\begin{note}
  Granting the intelligibility of~\autoref{illu:lost-key}, our interest is with the following sketch:

  \begin{sketch}
    \label{sketch:zS:fail}
    For some agent, the follow two conditions obtain:

    \begin{itemize}
    \item
      There is some \(\pvp{\psi}{v'}{\Psi}\) such that, from the \agpe{}:
      \begin{enumerate}
      \item
        Some proposition-value pair, so not concluding.
      \item
        Recognition by the agent.
      \end{enumerate}
    \end{itemize}
  \vspace{-\baselineskip}
  \end{sketch}

  In short, the agent, from their perspective, thinks the may reason to some other conclusion, and therefore (it seems) the agent does not conclude.

  
\end{note}

\begin{note}[I think about this\dots]
  \autoref{illu:lost-key} is introduces this phenomena as I seem to encounter the pattern every time I think I've lost something, whether keys, books, or files.
  After some searching I feel I should have accumulated enough evidence to conclude the item is lost.
  However, the item isn't lost (from my perspective, at least), while there remains a place to check, and experience shows I eventually think of a place to check and, more often than not, the item is there.
\end{note}

\begin{note}
  Recognition may fail.

  For example, suppose an agent has been informed of something by some other agent.
  Third party is well aware.
  And, third party will soon inform the agent that the thing is not the case.

  So, agent is soon to be interrupted.
  Given information, agent will not conclude.

  Trivially, no recognition.
\end{note}

\begin{note}
  Closely related.
  \begin{enumerate}[label=\alph*\('\)., ref=(\alph*\('\))]
  \item
    \label{sketch:zS:fail:curb:conditional:var}
    The agent would conclude \(\pv{\phi}{v}\) from \(\Phi\) \emph{only if} the agent would not conclude \(\pv{\psi}{v'}\) from \(\Psi\).
  \end{enumerate}
  And, replace \dots with:
  \begin{enumerate}[label=\arabic*\('\)., ref=(\arabic*\('\))]
    \setcounter{enumi}{1}
  \item
    \label{sketch:zS:fail:no-c:var}
    The agent entertains the possibility of concluding \(\pv{\psi}{v'}\) from \(\Psi\), and therefore (it seems) they do not conclude \(\pv{\phi}{v}\) from \(\Phi\).
  \end{enumerate}

  Given~\ref{sketch:zS:fail:curb:conditional:var}, the relevant instance of `\(\psi\)' would be: `My keys \emph{might be} in location \(l\)'.

  Given~\ref{sketch:zS:fail:curb:conditional:var} and~\ref{sketch:zS:fail:no-c:var}, focus shifts from failing to conclude something to concluding something.

  Plausible variation.
  Concerned about failing to conclude keys are not in location \(l\) or concerned about concluding keys might be in location \(l\).

  Benefit \ref{sketch:zS:fail}, failure to conclude \(\pv{\psi}{v'}\) is also a problem.
  Specific location.
  If keys are lost, then conclude not in location.
\end{note}

\begin{note}
  \autoref{sketch:zS:fail}, failing to conclude.
  Agent does not conclude \(\pv{\phi}{v}\) from \(\Phi\), in part, because the agent entertains the possibility of failing to conclude \(\pv{\psi}{v'}\) from \(\Psi\).

  Consider the following variation on~\autoref{sketch:zS:fail}:
  \begin{sketch}
    \label{sketch:zS:succeed}
    For some agent, the follow two conditions obtain:
    \begin{enumerate}
    \item
      \label{sketch:zS:succeed:curb}
      There is some \(\pvp{\psi}{v'}{\Psi}\) such that, from the \agpe{}:
      \begin{enumerate}[label=\alph*., ref=(\alph*)]
      \item
        \label{sketch:zS:succeed:curb:opportunity}
        Opportunity to reason about whether \(\pv{\psi}{v'}\) follows from \(\psi\).
      \item
        \label{sketch:zS:succeed:curb:conditional}
        If the agent took to opportunity, the agent would conclude \(\pv{\phi}{v}\) from \(\Phi\) \emph{only if} the agent would conclude \(\pv{\psi}{v'}\) from \(\Psi\).
      \end{enumerate}
    \item
      \label{sketch:zS:succeed:no-c}
      The agent \emph{does not} entertain the possibility of concluding \(\pv{\psi}{v'}\) from \(\Psi\), and so they conclude \(\pv{\phi}{v}\) from \(\Phi\).
    \end{enumerate}%
    \vspace{-\baselineskip}
  \end{sketch}

  The difference between \TNSketch{3}~\ref{sketch:zS:fail} and~\ref{sketch:zS:succeed} is with respect to {\color{red} Whether or not the agent is concluding}.
\end{note}

\begin{note}
  If \scen{1} satisfying~\autoref{sketch:zS:succeed}, then close to counterexample to \issueConstraint{}.
  For, does not entertain \emph{because} \fc{}.
  And, from  \fc{} relation of support.
  And, no deviance.

  Though, caution, also need not witnessed reasoning.

  Provide an initial \scen{} which does not fully work as a counterexample.
  Then, develop~\autoref{sketch:zS:succeed} with care.
\end{note}


\begin{note}
  So, as with lost keys.
  I wonder about derived rules of inference.
  Clearly a problem.
  Understanding of propositional logic is good, better than memory.
  But, same understanding, sound.
\end{note}

\section{Analysis}
\label{cha:zS:sec:lost-keys:analysis}

\begin{note}
  So, the way to understand these cases is in terms of whether the event is an event in which the agent is concluding.

  Then, the way \ros{1} from \fc{1} work is by explaining why the overall event is an event of concluding, rather than the agent simply pairing \(\pv{\phi}{v}\).
\end{note}

\begin{note}
  Two key components of these cases.
  \begin{enumerate}[label=\Roman*., ref=(\Roman*)]
  \item
    \label{zS:breakdown:check}
    \check{2}.

    Whether or not would concluding \(\pv{\psi}{v'}\) from \(\Psi\).
  \item
    \label{zS:breakdown:psat}
    Feedback.

    \check{} is recognised by agent, and determines, in part, whether or not the agent is concluding.
  \end{enumerate}

  From the sketches, {\color{red} \dots}
\end{note}

\begin{note}
  Build this out in two separate parts.

  To begin,~\autoref{cha:zS:sec:curbs} will partially develop~\ref{zS:breakdown:check} to form the idea of a `\curb{}'.
  To end,~\ref{cha:zS:sec:requs} will combine the idea of a \curb{} with~\ref{zS:breakdown:psat} in the form the idea of a \requ{}.
\end{note}

%%% Local Variables:
%%% mode: latex
%%% TeX-master: "master"
%%% End:
