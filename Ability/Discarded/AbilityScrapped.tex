\subsection{Reasoning \nr{} and reasoning \ur{}}
\label{sec:reas-dd-reas}

\begin{note}
  We now turn to the distinction between reasoning \emph{using reference} and reasoning \emph{not using reference}.
  Or, from now on, reasoning \ur{} and reasoning \nr{}.

  The use of (probably mistranslated) Latin is in part due to some similarity between the distinction under discussion and variety of distinctions made by using the terms `\dd{}' and `\dr{}'.
  In larger part, though, Latin is used because I had a hard time finding a combination of English terms which are not suggestive of some distinction.
  For example, you may already be considering whether `\emph{using reference}' and `\emph{not using reference}' correspond to `semantic' reasoning by way of some interpretation function and `syntactic' reasoning by rule governed syntactic manipulation.
  As with the \dd{}/\dr{} distinction, we will contrast \ur{} and \nr{} with `semantic' and `syntactic' reasoning below.
  For the moment I hope the use of Latin will allow for some flexibility.
\end{note}

\begin{note}[Plan]
  The plan for this section is as follows:
  \begin{itemize}
  \item First, situate the distinction.
  \item Background for distinction.
  \item Initial illustrations.
  \item Definitions.
  \item Illustrations for key type of reasoning.
    \begin{itemize}
    \item With logic
    \item Difference from syntax and semantics
    \item Existentials.
    \end{itemize}
  \item Contrast with \dd{}/\dr{}.
  \item Apply to ability/distinction matrix.
  \end{itemize}
\end{note}

\paragraph*{Situating the distinction}

\begin{note}
  To situate the distinction between reasoning \ur{} and \nr{} we will first outline how it applies to ideas that have already been introduced, and then note the purpose distinction in the argument ahead.
\end{note}

\subparagraph*{With respect to ideas already introduced}

\begin{note}
  In~\autoref{sec:abil-access-supp} we restricted our use of `reasoning' to involve an agent claiming support for some proposition \(\phi\) having some value \(v\) (\autoref{prop:RisTV}).
  For example, an instance of reasoning may culminate with an agent claiming support that it is true that the front door was locked, or that it is desirable that they take a walk.

  The distinction between reasoning \ur{} and \nr{} is targeted at the way in which an agent claims support that some proposition \(\phi\) has some value \(v\).
\end{note}

\begin{note}
  Further, in~\autoref{sec:cases-interest} we introduced particular instances of an agent claiming support for some conclusion which involved two key steps.
  The reasoning, in outline:
  \begin{enumerate}[label=\arabic*., ref=(\arabic*)]
  \item\label{NUR2:ro:i} I have some general ability \(\gamma\).
  \item\label{NUR2:ro:ii} If I have general ability \(\gamma\) then I have some specific ability \(\varsigma\) to \emph{V} that \(\phi\).
  \item\label{NUR2:ro:iii} I have the specific ability \(\varsigma\) to \emph{V} that \(\phi\). \hfill (From~\ref{NUR2:ro:i} and~\ref{NUR2:ro:ii})
  \item\label{NUR2:ro:iv} It is only possible to \emph{V} that \(\phi\) if \(\phi\) is already the case.
  \item\label{NUR2:ro:v} \(\phi\) is the case. \hfill (From~\ref{NUR2:ro:iii} and~\ref{NUR2:ro:iv})
  \end{enumerate}

  The two key steps are from~\ref{NUR2:ro:i} and~\ref{NUR2:ro:ii} to~\ref{NUR2:ro:iii} and from~\ref{NUR2:ro:iii} and~\ref{NUR2:ro:iv} to~\ref{NUR2:ro:v}.

  The first key step involves the conditional of~\ref{NUR2:ro:ii}, termed `\gsi{-}', and clarified in~\autoref{sec:type-scenario}.

  The second key step involves the conditional of~\ref{NUR2:ro:iv}, termed `\aben{an}', and clarified in~\autoref{sec:ability-entailment}.

  Both these steps involve conditionals, and hence using the conditional to claimed support for the consequent of the conditional given claimed support for the antecedent of the conditional.
  And, the distinction between reasoning \ur{} and \nr{} is of interest to use with respect to these two instances of claiming support in particular.

  Finally, in~\autoref{sec:wr-ar} we introduced two (schematic) interpretations of ability --- \AR{} and \WR{} --- and as both key steps appeal to ability, the distinction between reasoning \ur{} and \nr{} will further inform how these interpretations of ability function in an instance of reasoning.
\end{note}

\begin{note}
  To summarise, from what we have seen, then, the distinction between reasoning \ur{} and \nr{} is targeted at the way in which an agent claims support that some proposition \(\phi\) has some value \(v\).
  So, the distinction is often interest when applied to the instance of reasoning which is the focus of this paper, and, hence, different ways in which the instance of reasoning may be understood.
\end{note}

\subparagraph*{With respect to the argument ahead}

\begin{note}
  Now, with the application to previously discussed ideas in hand, the distinction has to key purposes looking ahead:

  In~\autoref{sec:first-conditional} the distinction will separate an interpretation of the instance of reasoning which is incompatible with \ESU{} (\ur{}) form an interpretation which is compatible (\nr{}).

  And, in~\autoref{sec:second-conditional} the distinction will separate an interpretation of the instance of reasoning which is incompatible with \nI{} (\nr{}) form an interpretation which is compatible (\ur{}).
\end{note}

\paragraph*{Basic distinction}

\begin{note}[Reasoning]
  As noted, reasoning involves claiming support that a proposition has some value.

  Some propositions and parts of proposition, when appealed to or used in reasoning, \emph{refer}.

  In the two quick examples given above, propositions are that the front door was locked and that the agent takes a walk.
  When part of some instance of reasoning, propositions are about what the status of some particular door was, and a type of action that some particular agent may take.
  In these examples, the agent performing the reasoning for which the propositions are a part is aware of what they refer to.
  `The front door' is the front door of the agent's house, and `they' is either the agent or some close acquaintance.

  This is not always the case.

\end{note}

\begin{note}[Quick clarification]
  The stronger claim that propositions, when part of reasoning, refer is not without issue.
  Let me illustrate with a quick argument.

  \begin{itemize}
  \item All propositions refer.
  \item A proposition is something that may be assigned value.
  \item One such value is truth.
  \item Some conditionals are true.
  \item Some conditionals refer.
  \end{itemize}

  Three intuitive statements about propositions which combined with strong assumption lead to a problem.

  Settling whether or not conditionals (for example) refer --- or what conditionals refer to if the do refer --- is of no real interest for present purposes.
  Argument should go through either way.
  Trouble is I would to be committed to something quite strong.

  Would be nice to narrow down the appropriate sense of proposition.
  However, task with little reward.

  Nothing depends on borderline cases.

  Conditionals in which antecedent and consequent refer are viewed in terms of information about reference.

  In sort, what we're interested in those propositions which do refer.

  I had to unlock the door, so the door was locked.
  I am feeling stressed, so it is desirable that I take a walk.
\end{note}

\paragraph*{Illustrations for intuition}

\begin{note}[Illustration]
  Before defining \ur{} and \nr{}, let us work up some intuition by re-examining an instance of claiming support.

  \begin{quote}
    From claimed support that a rectangle measures \(19\text{cm}\) by \(7\text{cm}\), an agent claims support that the area of the rectangle is \(133\text{cm}^{2}\).
  \end{quote}
  In \autoref{ill:rectangle:basic} measurement, understanding of how to calculate the area of a rectangle, and some grasp of mathematics.\nolinebreak
  \footnote{
    In \autoref{ill:rectangle:ability} applied to ability, but not interested in that here.
  }
  The following two illustrations detail two different ways in which the claim support.

  The purpose of the pairing is to help develop some intuition for the ways in which an agent may or may not appeal to or use the referent of a proposition when reasoning.
  Beyond this, the subject matter and steps of reasoning hold no (direct) interest.

  Both illustrations start with a the same premise --- the rectangle measures \(19\text{cm}\) by \(7\text{cm}\) --- and end with the same conclusion --- the area of the rectangle is \(133\text{cm}^{2}\).

  The key difference between the two illustrations is how the agent claims support for the conclusion given claimed support for the premises.

  In the first illustration the agent will refer to a particular rectangle throughout the intermediate reasoning.
  And, by contrast, in the second the agent will not refer to a particular rectangle throughout the intermediate reasoning.
\end{note}

\begin{note}[]
  \begin{illustration}\label{ill:rectangle:ur}
    \vspace{-\baselineskip}
    \begin{enumerate}[label=\(\protect\tBox\)\space\arabic*., ref=(\(\protect\tBox\)\space\arabic*), align=left, leftmargin=*]
    \item[\(\protect\tBox\)\space P.]\label{tB:measure} This rectangle measures \(19\text{cm}\) by \(7\text{cm}\).
    \item\label{tB:width} Width is \(19\text{cm}\), so divide into \(19\) columns containing some number of unit squares, where unit is a centimetre.
    \item\label{tB:height} Height is \(7\text{cm}\).
    \item\label{tB:counting} So, there are seven unit squares in each column.
    \item\label{tB:total} Therefore, total of \(133\) square centimetres.
    \item[\(\protect\tBox\)\space C.]\label{tB:conclusion} Hence, the area of this rectangle is \(133\text{cm}^{2}\).
    \end{enumerate}
    \vspace{-\baselineskip}
  \end{illustration}

  The reasoning of~\autoref{ill:rectangle:ur} somewhat stilted in style, but plausible.

  The agent understands how to calculate the area of a triangle and applies the calculation to the specific rectangle.
  Key is that at no point does the agent abstract from the particular rectangle to an arbitrary rectangle with the same dimensions --- steps~\ref{tB:width} to~\ref{tB:total} are about the specific rectangle.
  Of course, the same reasoning may be applied to any rectangle with the same dimensions, but steps~\ref{tB:measure} to \ref{tB:conclusion} refer to \emph{that} particular rectangle, and the agent claims support because of what they have established about that triangle.

  In this sense, the propositions concerning rectangles refer, and the agent appeals to or uses the referent of those propositions to claim support.

  The `\(\tBox\)' prefix for each step is designed to indicate that the agent is reasoning about a particular object throughout.

  \begin{illustration}\label{ill:rectangle:nr}
    \vspace{-\baselineskip}
    \begin{enumerate}[label=\(\protect\tBoxd\)\space\arabic*., ref=(\(\protect\tBoxd\)\space\arabic*), align=left, leftmargin=*]
    \item[\(\protect\tBoxd\)\space P.]\label{tBd:measure} This rectangle measures \(19\text{cm}\) by \(7\text{cm}\).
    \item\label{tBd:calculate} Calculate the area of any two-dimensional object, take length and width in a common unit and multiply together to get area in common unit squared.
    \item\label{tBd:abstract} So, if object with \(19\) and \(7\), then area is \(19 \times 7\) cm2.
    \item\label{tBd:instantiate} Put together.
    \item[\(\protect\tBoxd\)\space C.]\label{tBd:conclusion} Hence, the area of this rectangle is \(133\text{cm}^{2}\).
    \end{enumerate}
    \vspace{-\baselineskip}
  \end{illustration}

  As with~\autoref{ill:rectangle:ur}, the reasoning of~\autoref{ill:rectangle:nr} is also somewhat stilted in style, but plausible.

  The agent understands how to calculate the area of a rectangle and although they conclude by claiming support with respect the specific rectangle, the agent quickly abstracts to any rectangle with the same dimensions.

  Key is that the agent abstracts from the particular rectangle to an arbitrary rectangle with the same dimensions for the most part of their reasoning --- steps~\ref{tBd:calculate} to~\ref{tBd:abstract} are not about any specific rectangle.

  In this sense, the propositions may refer, as they apply to any rectangle with the respective dimensions, but the agent neither appeals to nor uses possible referents of those propositions to claim support at key steps of the reasoning.

  The `\(\tBoxd\)' prefix for each step is designed to indicate that the agent is only sometimes reasoning about a particular object.
\end{note}

\begin{note}[Propositions, recap]
  {\color{red} \autoref{prop:RisTV} has reasoning as establishing a value}
  Two ways of tracing value.
  With and without reasoning about what the constituents of the proposition refer to.\nolinebreak
  \footnote{
    Footnote on Russel, Frege, etc.
  }

  Here, truth conditions, but more general evaluation.
  I.e.\ truth isn't the only evaluation of interest.

  Here, truth conditions.
  However, value.
  So, truth conditions + value, reasoning about something.
  Distinction is with respect to how the agent goes about reasoning about the something.

  Have the idea of a proposition.
  Something which gets a value.
  Reason in terms of preservation of value.
\end{note}

\begin{note}[The distinction]
  {
    \color{red}
    This is leading to a distinct between `there is a' and `the'.
    Though, that's just a gloss.
  }
  Focus on deductive reasoning.
\end{note}

\paragraph*{Definitions}

\begin{note}
  Have basic distinction, and a pair of illustrations.

  Following, apply this distinction to additional illustrations for which reasoning involved which will resemble reasoning of interest with respect to  ability scenarios.

  First, though, summarise the ideas by stating definitions.
\end{note}

\begin{note}[A pair of definitions]
  \begin{definition}[\ur{}]
    \vspace{-\baselineskip}
    \begin{itemize}
    \item Let \(t\) be some thing, and
    \item Let \(S\) be some step of reasoning that involves appeal to claimed support for propositions \(\psi_{1},\dots,\psi_{n}\) to claim support for some proposition \(\phi\).
    \end{itemize}

    The step of reasoning \(S\) is \ur{} with respect to \(t\) if the agent \emph{appeals to} \(t\) when claiming support for \(\phi\) by \(S\).
  \end{definition}

  \begin{definition}[\nr{}]
    \vspace{-\baselineskip}
    \begin{itemize}
    \item Let \(t\) be some thing, and
    \item Let \(S\) be some step of reasoning that involves appeal to claimed support for propositions \(\psi_{1},\dots,\psi_{n}\) to claim support for some proposition \(\phi\).
    \end{itemize}

    The step of reasoning \(S\) is \nr{} with respect to \(t\) if the agent \emph{does not appeal to} \(t\) when claiming support for \(\phi\) by \(S\).
  \end{definition}
\end{note}

\begin{note}
  Respective definitions differ only with respect to whether appeals --- or does not appeal to --- the referent of the proposition, or part.

  So, initial discussion of definitions covers both.

  Start with conditions.

  \(t\) fix some thing.
  Any thing, really.
  Possible referent of a referential term.
  Reasoning is complex, plausible that any step may involve a complex of factors.
  Only interested in specific terms, so definition avoids difficulties with classifying steps of reasoning as a whole.

  Don't require that \(t\) occurs, to keep things simple.
  Implicit that \nr{} with respect to \(t\) if \(t\) does not occur.

  However, possible for \(t\) to be in either premise(s) or conclusion of step.

  \(S\)
  When claiming support for some proposition from some other proposition.
  So, definitions cover a single step of reasoning --- not interested in classifying anything broader.

  Possible to claim support for a proposition from something non-propositional.
  However, only interested in proposition-to-proposition case, so this restriction is fine.

  Existential.
  Key idea is that \ur{} applies whenever appeals to referent.
  The agent claim support in this way.

  Finally, broad point.
  Claiming support.
  Doesn't say that this is successful, that there is not some other way, in particular going by \nr{} or \ur{} instead.
\end{note}

\begin{note}[Simpliciter]
  Definitions are with respect to some thing.
  That's what we're interested in.
  Further, they don't characterise steps of reasoning in general.
  Hence, no immediate (at least) consequences for what is involved in a step of reasoning.
  For example, possible \ur{} with respect to some thing and \nr{} with respect to some other thing.
\end{note}

\begin{note}[The distinction and entailment]
  The purpose of these definitions is to capture when the agent claim support by appeal to some thing.

  Applied to rectangles.
  \autoref{ill:rectangle:ur} as the agent references the particular triangle throughout the reasoning.
  So, \ur{} holds for each step with respect to the rectangle.
  The way in which the agent claims support is such that the rectangle does work.


  \nr{} holds for steps~\ref{tBd:calculate}~to~\ref{tBd:instantiate} of~\autoref{ill:rectangle:nr}.


  \emph{Why} something follows from something else.

  With \ur{} get an argument where the agent ensure that the reference for the consequence works out.
  This is the one to start with.
  The argument is that things \emph{are} a certain way, so to speak.
  That is, reasoning works out because reference works out.


  \nr{} doesn't do this.
  No reference.
  So, transformation to the information that the agent already has.
\end{note}

\begin{note}[To keep in mind]
  The distinction between \ur{} and \nr{} is not about the presence (or absence) of referential terms.
  Nor that the agent may claim support by noting that a term refers.

  In rectangle, important that refers to any rectangle.

  Main interest with this distinction will be whether agent appeals to referent of a referential term, so to speak.
\end{note}

\paragraph*{Illustrations for interest}

\begin{note}
  We now turn to additional illustrations.

  Two goals.

  First, relate to familiar stuff.
  Difference from distinction from syntactic and semantic reasoning.

  Second, to consider reasoning with existentials.
  This, then expanded on when turn to application of the distinction.

  Start with simple case.
  Then, turn to existentials.

  Get
  More familiarity with distinction.
  Additional considerations.

  Focused on deductive when giving examples.
  However, same applies to other kinds of reasoning.
\end{note}

\begin{note}[Instance of reasoning]
  The scenarios of interest:
  \begin{quote}
    From claimed support that a dog is \RIPa{} and \RIPb{}, an agent claims support that the dog is \RIPb{}.
  \end{quote}

  In outline, the reasoning is straightforward:
  \begin{quote}
    \begin{itemize}
    \item[P.] \nagent{10} is \RIPa{} and \RIPb{} dog.
    \item[---.] So, \nagent{10} is \RIPa{} dog and a \RIPb{} dog.
    \item[C.] Hence, \nagent{10} is a \RIPb{} dog.
    \end{itemize}
  \end{quote}

  Though straightforward, the reasoning is not completely trivial.
  `\RIPa{} and \RIPb{}' is the combination of two adjectives --- `\RIPa{}' and `\RIPb{}', respectively.
  And, it is not always the case that the attribution of a combination of adjectives to an object allows the attribution of the separate adjective.

  For example, it does not follow from the picture being black and white that the picture is white.\nolinebreak
  \footnote{
    You may prefer `black-and-white'.
    If so, I suggest `\RIPa{}-and-\RIPb{}'.
  }
  Nor does it follow from the weather being cloudy and sunny that the weather is sunny.

  And, of course, this isn't unique to `and' and adjectives.
  It does not follow from the transit time being an hour and five minutes that the transit time is five minutes.
  Nor does it follow from book being written by A and B that the book was written by B.

  The point, though a minor one, is that `and' behaves in a variety of ways and so some reasoning is required to move from the premise to the conclusion.
  \nolinebreak
  \footnote{
    Compare to obedient and \RIPb{}.
    May think obedience restricts \RIPb{}.
    Obedient and \RIPb{}, but not \RIPb{}.
    Not to say that the meaning of `and' works the same in both constructions.
    However, enough to require that syntax should be respected.
  }
\end{note}

\begin{note}
  Outline of reasoning.

  Focus for the moment is on relation to syntactic and semantic.
\end{note}


\begin{note}[Example, \nr{}]
  Recall, \nr{}.

  Idea is to illustrate this kind of reasoning in terms of transformations.
  Following steps in the first-order language.

  \begin{illustration}\label{ill:dog:C:nr}
    \vspace{-\baselineskip}
    \begin{enumerate}[label=\(\protect\iDogd\)\space\arabic*., ref=(\(\protect\iDogd\)\space\arabic*), align=left, leftmargin=*]
      % \item[\(\protect\iDogd\)\space P.] The dog is \RIPa{} and \RIPb{}.
    \item\label{ill:iDogd:abd} \(\text{\RIPa{-} \& \RIPb{} dog}(w)\)
    \item\label{ill:iDogd:sep-gen} \(\forall x(\text{\RIPa{-} \& \RIPb{} dog}(x) \rightarrow (\text{\RIPa{-} dog}(x) \land \text{\RIPb{-} dog})(x))\)
    \item\label{ill:iDogd:sep-app} \(\text{\RIPa{-} \& \RIPb{} dog}(w) \rightarrow ((\text{\RIPa{-} dog}(x) \land \text{\RIPb{-} dog})(w))\)
    \item\label{ill:iDogd:sep-con} \(\text{\RIPa{-} dog}(w) \land \text{\RIPb{-} dog}(w)\)
    \item\label{ill:iDogd:done} \(\text{\RIPb{-} dog}(w)\)
      % \item[\(\protect\iDogd\)\space C.] Hence, the dog is \RIPb{}.
    \end{enumerate}
    \vspace{-\baselineskip}
  \end{illustration}
\end{note}

\begin{note}[Main point]
  Moving between the steps by rules applied that do not depend on interpreting predicates and constants.
\end{note}

\begin{note}[Background]
  Key point here is that we have some background.

  Understand the syntax, and understand the intended interpretation of the syntax.
  `\RIPa{-} and \RIPb{} is a predicate, and applied to some constant `\(w\)' (abbreviating `\nagent{10}').
  For the moment, all that matters is that these are predicates and constants.

  Form of the argument.
  Two assumptions.
  Three premises obtained by applying rules following main connective of previous premise.
  In order, universal quantifier, conditional, conjunction.
\end{note}

\begin{note}[Walk-through]
  Walk through these rules.
  Observation is that each instance conforms to reasoning \nr{}.

  First step, from \ref{ill:iDogd:sep-gen} to \ref{ill:iDogd:sep-app} instance of universal instantiation.
  Applied to \(w\) because \(w\) is a constant.
  In order to apply this rule, it doesn't matter what \(w\) refers to, nor what the predicates involved refer to.
  Universal instantiation allows any constant.

  From \ref{ill:iDogd:sep-app} to \ref{ill:iDogd:sep-con} instance of conditional elimination.

  And, from \ref{ill:iDogd:sep-con} to \ref{ill:iDogd:done} conjunction elimination.

  If I may, I encourage you to check that the argument is valid.
  The task is instructive because in doing so you too will abstract away from the intended interpretation of the terms.
\end{note}

\begin{note}[Two things]
  Two questions here.

  Question about how to obtain \ref{ill:iDogd:abd} from initial premise, and how to obtain conclusion from \ref{ill:iDogd:done}, but there doesn't seem to be mystery about what is going on.
  Same as with illustration above.
  Move from measurements regarding a particular rectangle to the manipulation of symbols which may be given an interpretation.
  Here, instead of mathematics, we have first order logic.

  \ref{ill:iDogd:sep-gen}.
  Granted two assumptions, valid.

  Key thing is choice of predicates.
  Designed to ensure that the argument is sound when applied given intended interpretation.

  It's also not clear that any dog which is \RIPa{} and \RIPb{} is so independently of it being a dog.
  A \RIPb{} dog need not be \RIPb{} in the same way a small elephant need not be small.

  These aren't distractions for the sake a pedantry.

  Reasoning proceeds by applying rules, regardless of reference.
  But, want soundness so that they may be applied without reference.
  \nr{} is about the reasoning that takes place, but it doesn't hold that reference is irrelevant to reasoning.


  Of course, it's fair to say that the equivalence only holds because of what `\RIPa{}' and `\RIPb{}' refer to.
  However, key is that once the rule has been obtained an agent may reason about the terms, rather than reasoning about \RIPa{}ness and \RIPb{}ness.


  Well,
\end{note}

\begin{note}[Summarise]
  Premise and conclusion about a particular dog, \nagent{10}.
  However, intermediary steps are instances of reasoning \nr{} as the agent does not go by referent.
  All the agent is concerned with is the logical form, so to speak.

  However, stress that although the steps of reasoning are independent of reference, it may still matter that parts refer.
  \nagent{10} may not feature, but that propositions are about \nagent{10} when given intended interpretation may matter.
\end{note}

\begin{note}
  Contrast to reasoning \ur{}.

  Following illustration traces follows the same general pattern as~\autoref{ill:dog:C:nr}.
  Difference, roughly stated, is that the agent will reason about \nagent{10}, \RIPa{} and \RIPb{} dogs, and so on.
\end{note}

\begin{note}[Example, \ur{}]

  \begin{illustration}
    \vspace{-\baselineskip}
    \begin{enumerate}[label=\(\protect\iDog\)\space\arabic*., ref=\arabic*, align=left, leftmargin=*]
    \item\label{ill:iDog:mixed} \nagent{10} is a big and playful dog.
    \item\label{ill:iDog:sep-gen} The thing of being a \RIPa{} and \RIPb{} dog is just the combination of being a \RIPa{} dog and being a \RIPb{} dog.
    \item\label{ill:iDog:sep-app} \nagent{10} being a \RIPa{} and \RIPb{} dog is the case when \nagent{10} is both a \RIPa{} dog and a \RIPb{} dog.
    \item\label{ill:iDog:sep-con} \nagent{10} is a \RIPa{} dog and \nagent{10} a \RIPb{} dog.
    \item\label{ill:iDog:sep-res} \nagent{10} is a \RIPb{} dog.
    \end{enumerate}
    \vspace{-\baselineskip}
  \end{illustration}
\end{note}

\begin{note}
  Natural language to ease discussion.
  Could go with semantic interpretation.\nolinebreak
  \footnote{
    Follow convention and use \(\sem{--}\) to capture the reference of some term `\(\text{--}\)'.
    \begin{illustration}
      \vspace{-\baselineskip}
      \begin{enumerate}[label=\(\protect\iDog\)\space\arabic*., ref=\arabic*, align=left, leftmargin=*]
      \item \(\sem{w} \in \sem{\RIPa{-} \& \RIPb{} dog}\)
      \item \(\sem{\RIPa{-} and \RIPb{} dog} \subseteq (\sem{\RIPa{-} dog} \cap \sem{\RIPb{-} dog})\)
      \item \emph{If} \(\sem{w} \in \sem{\RIPa{-} \& \RIPb{} dog}\), \emph{then} \(\sem{w} \in \sem{\RIPa{-} dog}\) \emph{and} \(\sem{w} \in \sem{\RIPb{-} dog}\)
      \item \(\sem{w} \in \sem{\RIPa{-} dog}\) and \(\sem{w} \in \sem{\RIPb{-} dog}\)
      \item \(\sem{w} \in \sem{\RIPb{-} dog}\)
      \end{enumerate}
      \vspace{-\baselineskip}
    \end{illustration}
    Imports set theoretical background from first order logic.
    Constants are individuals, predicates are treated extensionally.
    Read simply, \(\sem{The dog} \in \sem{\RIPa{-} and \RIPb{}}\) may suggest that the agent is reasoning that the reference of the term `the dog' is a member of the reference of `\RIPa{} and \RIPb{}'.

    \emph{\nagent{10}} is such that they are a \emph{\RIPa{} and \RIPb{}} dog.
  }
\end{note}

\begin{note}
  \ref{ill:iDog:mixed} meet \nagent{10} for the first time.
  \nagent{10} is playing with a ball.
  Person who takes care of \nagent{10} remarks that they're a \RIPa{} and \RIPb{} dog.
  You wonder about adjectives, what's being communicated here.
  Think about these two adjectives, \ref{ill:iDog:sep-gen}.
  Of course, apply to \nagent{10}, \ref{ill:iDog:sep-app}.
  And, observed, so \ref{ill:iDog:sep-con}.
  Hence, \ref{ill:iDog:sep-res}.
\end{note}

\begin{note}
  \ur{} with respect to \nagent{} and properties, roughly.
  So, four things.
  \nagent{10}, \RIPa{} \& \RIPb{}, \RIPa{}, and \RIPb{}.

  Throughout, properties.

  Variations.
  Focus on the adjectives and how these relate to properties, or abstract to concepts.

  Key point is that \ur{}.
  Agent is reasoning about some things, and claiming support by appeal to those things.
\end{note}

\begin{note}
  As with first pair of illustrations, claiming support by reference to something.
  This is all the distinction amounts to.
  Broader, applied to different kinds of reasoning, and differences that follow from these two kinds of reasoning.
\end{note}

\begin{note}[Difference]
  First illustration, logical structure of propositions.
  Don't need to appeal to reference.
  Logical structure for convenience.
  Claim support by transformations.

  Second illustration, surface presentation remains similar, but properties.

  \mom{}.
  First, \nagent{10} and transformations.
  World appears this way, and reasoning is independent.

  Second, \nagent{10} and understanding of properties involved.
  No reason to think that properties have been thought about badly, and testimony or observation seems good enough.
  World appears this way.
\end{note}

\begin{note}
  Overall similarity.
  Designed so that the explanation of the steps applies to both illustrations.
  Steps are missing from outline of reasoning present.
  Nothing about presentation.
  Indeed, second as semantic counterpart to syntactic reasoning of first.

  Alternatively, formal and non-formal.

  Something about this is right, but it's not quite right.
  Hopefully overlap will lend some clarity.
\end{note}

\paragraph*{Two related ideas}

\begin{note}
  Two related distinctions.
  Overlap and differences.
  Not an exhaustive discussion, just enough to identify similarities and differences.

  Similarity, then difference.
\end{note}

\begin{note}
  Should already be cautious.
  \ur{} and \nr{} is about whether some thing is appealed to when taking a step.
  It not obvious that anything else follows about the step, which is what the distinctions apply to.

  It's not obvious, but not immediate either.
  Still, difficult to work with.
  So, let's simplify for the moment.

  \begin{definition}
    Let \(S\) be some step of reasoning that involves appeal to claimed support for propositions \(\psi_{1},\dots,\psi_{n}\) to claim support for some proposition \(\phi\).

    The step of reasoning \(S\) is:
    \begin{itemize}
    \item \ur{} \emph{simpliciter}, if there is something thing for which \(S\) is \ur{} with respect to.
    \item \nr{} \emph{simpliciter}, otherwise.
    \end{itemize}
    \vspace{-\baselineskip}
  \end{definition}
  Now, apply to steps of reasoning as a whole.
\end{note}

\begin{note}[Issues]
  \nr{} for both sides of consequence.
  So, \nr{} does not imply syntactic reasoning.
  Still, seems formal then \nr{}.

  However, \nr{} does not imply formal reasoning.
\end{note}

\subparagraph*{Syntactic and semantic perspectives on logical consequence}

\begin{note}
  Similarity: well, syntactic transformation and something involving reference.

  Quick point is that role of reference in semantic perspective of logical consequence is different from \ur{}.

  Given some background, this is somewhat obvious.
  It's still logical consequence that we're talking about.
\end{note}

\begin{note}
  Using first order logic.
  Further, appeal to logical consequence.

  Presentations given are valid.

  Suggestion that the distinction I've been relying on to illustrate \ur{} and \nr{} just is the distinction at issue.

  On the one hand, syntactic reasoning and on the other semantic.

  Understood distinction between syntactic and semantic accounts of logical consequence.

  This is not to say that logical consequence is at issue.
  Problem with logical consequence is that it's always independent of content.
  This is ~\cite{Etchemendy:1990wo,Etchemendy:2008wz}.

  Instead, difference is the type of reasoning.
  E.g. the basics of the representational or interpretational approaches.

  Thing here is that when applied to logical consequence these approaches aren't really \ur{}.
  It's not about what the terms refer to, but possible referents of the terms.

  This is really important.
  \nr{} doesn't say that reference isn't relevant.
  And, get to the end of some semantic reasoning and it's possible to construct a counterexample.
  Reference is important.
  But, there's no need to appeal to that counterexample.

  Still, because semantics is about reference, get instances of \ur{} which follow semantic consequence.
  Every semantic consequence will lead to a valid instance of \ur{}, because here we're just fixing on one particular interpretation.

  Converse does not seem to hold.
  Consider illustration again.
  Middle steps are there to ensure that no matter the reference.
  But, as it's \nagent{10} it seems these aren't required.
  It's a logical consequence, but it's not at all clear that these steps are required.


  This is why syntax versus semantics as applied to logical consequence is tricky.
  Logical consequence is it's own thing.
  Two different interpretations of this.
  However, same consequences.
  And, the difference between illustrations is in part a difference in consequence.

  Really, we're looking at two 

  The point, really, of \ur{} is that the agent is not claiming support for the conclusion by appeal to logical consequence.
\end{note}

\begin{note}[Quick syntax vs semantics]
  Quick distinction is between syntactic and semantic reasoning.\nolinebreak
  \footnote{
    Avoid using term `logical form' as this doesn't distinguish between the two different things.
  }
  Familiar distinction.
  Consequence of reference.

  Two issues.
  First, background system.
  Second, and more important, suggests absence of reference.
  However, quite possible that the agent requires that some term refers.

  Note, also distinction doesn't rely on views of reference.
  Possible to have a view where reference is by the agent, or a view where reference is by language.
\end{note}

\begin{note}[Avoid syntax semantics terminology]
  Don't use syntax/semantics terminology because the focus of the contrast is reference, rather than two `systems' that falls from the contrast.
  Indeed, syntax/semantics requires a background system.
  Won't stray from first order logic, or at least reasoning that may be captured by first order logic.
  So, free to use this terminology is you prefer.
  However, caution, as will be seen in the following illustration.
\end{note}

\subparagraph*{Formal and non-formal reasoning}

 \cite{Beall:2019ty}

\begin{note}
  Above, different kinds of consequence.
  Issue was logical consequence.

  So, perhaps formal and non-formal.\nolinebreak
  \footnote{
    Distinct from informal logic~\textcite{Groarke:2021tk}.
  }

  Instances of \nr{} considered arithmetic and first order logic.
  Formal systems.
  Appeal to something beyond form(al system).

  Intuitive distinction, but beyond this I'm not sure how exactly to characterise the distinction.

  Well, seems that this distinction fits well.
  If logical is just no particular reference, then we've got a nice distinction.
\end{note}

\begin{note}[`Formal' and \nr{}, `non-formal' and \ur{}]
  Basic idea.
\end{note}

\begin{note}[Two cases]
  \begin{itemize}
  \item `Formal' and \ur{}: nonstandard models. Reasoning about formal properties of natural numbers. Just because these don't uniquely identify doesn't prevent this.
    However, still non-formal in the sense that the agent goes beyond formalism.
    So, this does end up being non-formal in a sense.
  \item `Non-formal' and \nr{}: time
  \end{itemize}
\end{note}

\begin{note}[`Formal' and \ur{}: nonstandard models]
  Example.

  For an illustration being put to use in a different context, consider intended and non-standard models.
  Specifically, with respect to arithmetic.
  Intended model example.

  The Peano Axioms are sound with respect to the natural numbers, addition, and multiplication.
  However, the Peano Axioms are also sound with respect to various non-standard structures.
  For example, by considering an object \(\omega\) which is larger than any natural number.

  So, there's a barrier to claiming that reasoning \nr{} is also reasoning \ur{}.
  If reasoning about the natural numbers, then reasoning \ur{}, because reasoning \nr{} does not do enough to fix on the natural numbers.

  More generally, incompleteness.
  (Nonstandard constructions based on incompleteness too.)

  From soundness, things still hold up.
  However, clear that there's a distinction between whether reference is being used.

  This is the thing that's important.

  Still, some caution.
  The way we've drawn the distinction requires some subtlety.
  For, it is a case of \nr{} for the agent to reason that `one' refers to one.
  Hence, nonstandard models don't follow from \nr{} reasoning alone.
  Still, follows that claiming support would not distinguish.
  And, that's where the key issue is.
\end{note}

\begin{note}[`Non-formal' and \nr{}: time]
  Well, rectangle.

  Though, this is a little tricky.

  So, reading the time from a clock.
\end{note}



\paragraph*{Existential illustrations}

\begin{note}[Example with an existential]
  The initial illustration was simple.
  Our interest is with something more complex.
  Existentials.

  \begin{quote}
    \begin{itemize}
    \item[P.] Some dog is \RIPa{} and \RIPb{}.
    \item[---] So, some dog is \RIPa{} and some dog is \RIPb{}.
    \item[C.] Hence, some dog is \RIPb{}.
    \end{itemize}
  \end{quote}
\end{note}

\begin{note}[Exists \nr{}]
  \nr{} is kind of straightforward.
  Not going to think about what this \(x\) refers to, just manipulate like any other individual.

    \begin{illustration}\label{ill:dog:E:nr}
    \vspace{-\baselineskip}
    \begin{enumerate}[label=\(\protect\iEDogd\)\space\arabic*., ref=(\(\protect\iEDogd\)\space\arabic*), align=left, leftmargin=*]
      % \item[\(\protect\iDogd\)\space P.] The dog is \RIPa{} and \RIPb{}.
    \item\label{ill:iDogd:E:abd} \(\exists x \text{\RIPa{-} \& \RIPb{} dog}(x)\)
    \item\label{ill:iDogd:E:sep-gen} \(\forall x(\text{\RIPa{-} \& \RIPb{} dog}(x) \rightarrow (\text{\RIPa{-} dog}(x) \land \text{\RIPb{-} dog}(x)))\)
    \item\label{ill:iDogd:E:abd-2} \(\text{\RIPa{-} \& \RIPb{} dog}(i)\)
    \item\label{ill:iDogd:E:sep-app} \(\text{\RIPa{-} \& \RIPb{} dog}(i) \rightarrow (\text{\RIPa{-} dog}(i) \land \text{\RIPb{-} dog}(i))\)
    \item\label{ill:iDogd:E:sep-con} \(\text{\RIPa{-} dog}(i) \land \text{\RIPb{-} dog}(i)\)
    \item\label{ill:iDogd:E:inst} \(\text{\RIPb{-} dog}(i)\)
    \item\label{ill:iDogd:E:done} \(\exists x\text{\RIPb{-} dog}(x)\)
      % \item[\(\protect\iDogd\)\space C.] Hence, the dog is \RIPb{}.
    \end{enumerate}
    \vspace{-\baselineskip}
  \end{illustration}
\end{note}

\begin{note}
  \autoref{ill:dog:E:nr} is \ref{ill:dog:C:nr}.

  Steps \ref{ill:iDogd:E:abd} to \ref{ill:iDogd:E:inst} mirror the reasoning with universal quantifier, conditionals, and conjunctions as before.

  Difference is these are with respect to some fresh constant.

  Standard.
  And, \nr{} with respect to any possible referent of \(i\).
  For, as before, agent doesn't require an interpretation of the non-logical vocabulary in order to apply these rules.
\end{note}

\begin{note}[Exists \ur{}]
  \begin{illustration}\label{ill:dog:E:ur}
    \vspace{-\baselineskip}
    \begin{enumerate}[label=\(\protect\iEDog\)\space\arabic*., ref=(\(\protect\iEDog\)\space\arabic*), align=left, leftmargin=*]
    \item Some dog is \RIPa{} \& \RIPb{}.
    \item The thing of being a \RIPa{} and \RIPb{} dog is just the combination of being a \RIPa{} dog and being a \RIPb{} dog.
    \item \emph{That dog} is a \RIPa{} and \RIPb{} dog
    \item \emph{That dog} being a \RIPa{} and \RIPb{} dog is the case when \emph{that dog} is both a \RIPa{} dog and a \RIPb{} dog.
    \item \emph{That dog} is a \RIPa{} dog and \emph{that dog} a \RIPb{} dog.
    \item \emph{That dog} is a \RIPb{} dog.
    \item Some dog is \RIPb{}.
    \end{enumerate}
    \vspace{-\baselineskip}
  \end{illustration}
\end{note}

\begin{note}
  \ur{} is far more interesting.
  First order logic, update to point to something.
  From \ur{} the agent is reasoning about that thing.
  About it's \RIPa{}ness and \RIPb{}ness.
  The conclusion doesn't require further information about the referent.
  So, remains at some level of generality.

  This may seem odd.
  Agent doesn't information about how reference is resolved.
  However, no different from common cases.
  For example, `Plato', `Grice', etc.
  Existential secures a reference, and that's all that's required.
\end{note}

\begin{note}[\emph{That dog} might not exist]
  Possible that \emph{that dog} does not exist, so the agent is not appealing to some existing thing.
  Well, sure, but same problem with `Plato'.
  Existential is sufficient to claim support.

  In other respects, the same.

  Parallel to \nagent{10}.
  Appealed to \nagent{10}'s \RIPa{} \& \RIPb{}-ness to claim support for \nagent{10}'s \RIPb{}-ness.
  In the same way, claiming support for \emph{that dog}'s \nagent{10}'s \RIPb{}-ness by appeal to \emph{that dog}'s \RIPa{} \& \RIPb{}-ness.

  Appeal to \emph{that dog} and those properties have same role.
\end{note}

\paragraph*{Similarity to \dd{}/\dr{} distinction.}

\begin{note}[Similarity to \dd{}/\dr{} distinction.]
  SEP
  Semantically de re/de dicto:
  A sentence is semantically de re just in case it permits substitution of co-designating terms salva veritate.
  Otherwise, it is semantically de dicto.

  Simple distinction in terms of co-designating terms.
  Here, issue is about how some referent contributes to truth conditions of proposition.
  However, comes down to the same idea.
  Whether or not the agent puts reference to use.
  Still, there's an important difference.
  \dd{} and \dr{} seems to talk about the status of the agent's relation to things.
  In particular, about the relation between reference applied to distinct terms.
  So, in cases with existentials, the point is that the agent doesn't have any candidate for co-reference.

  It's important to keep these two things separate.
  The agent's reasoning determines \ur{} and \nr{}.
  In contrast, it is not in general possible for the agent to determine \dd{} and \dr{}.

  The \dd{}/\dr{} distinction is more about the quality of reference.
  So, agent fails to fix a transparent reference relation.
  Yet, this doesn't uncover the way in which the agent reasons.
  So,~\citeauthor{Fitch:1981vg}'s example.
  Quite possible that the agent is reasoning about Cicero, not whatever satisfies the term.
  However, \dd{} because it's not possible to substitute.

  Some clear examples.
  Difference is with how the reference may be resolved from the perspective of the agent.
  \dr{}, object.
  \dd{}, satisfying attributions.
  So, whether the referent contributes to truth conditions.
  This makes sense.
  In the case of \dr{}, yes, in the case of \dd{}, no.

  And, in turn, this is different from how the agent proceeds to reason from proposition.

  Suppose \dr{}.
  Possible \nr{}, as the agent may not appeal to reference.
  And, possible \ur{} as agent might reason about the individual.

  Suppose \dd{}.
  Possible \nr{}, same as before.
  Also possible \ur{}.
  Still \dd{} as the referent is not \emph{directly} contributing to the truth conditions.
  However, the agent is still reasoning about that thing, whatever it is.

  So, easiest when thinking about the earlier example.
  Reasoning about some number, not something which satisfies the Peano axioms.

  Now, \dd{} and \nr{} are a natural pairing.
  If no fix on reference, then don't reason with reference.
  The important thing, however, is the way in which the proposition is evaluated.
\end{note}

\begin{note}[Metaphysical distinction.]
  Same issue.
  Details about predication don't seem to be up to the agent.
\end{note}

\begin{note}[rigid and non-rigid designation]
  Also different from rigid and non-rigid designation.
  This only applies when \ur{}.
  It may seem that non-rigid and \nr{} pair up.
  But this isn't right.
  Quite possible to fix that the reference relation is rigid, but also to never use the reference relation.
  Go through any example where one can use some kind of logical reasoning.
  It's not the case that the result could vary simply because one reasoned by logical form, so to speak.
\end{note}

\begin{note}[\ESU{}]
  Important to note that there's no issue with \ESU{} yet.
  Both types of reasoning are quite compatible.
  So, this distinction is important, but nothing follows from this distinction alone.

  I mean, \nr{} is clearly fine with both.
  The only difficulty is with \ur{}.
  However, this is also fine, as \ESU{} is only about what the agent appeals to when reasoning, and it's a different claim to hold that reference either is or is not important.

  This is a distinction with a difference, but the \emph{significance} of this difference will come later.
\end{note}

\begin{note}[Examples with properties and events]
  E.g.\ Brutus hugging Caesar.

  This is also the thing about Davidsonian event semantics.
  Arguably, \emph{de constructione} with respect to the event.
  I mean, motivated by logic.
  Still, doesn't prevent \emph{de materia}.
\end{note}

\subsection{Filling in the matrix}
\label{sec:filling-matrix}

\begin{note}
  Handful of distinctions.

  Fill in the matrix.

  First, recall \gsi{}.
\end{note}



\section{Major argument}
\label{sec:broad-argum-overv}

\begin{note}[Overview]
  Tension resulting from the unrestricted scope of~\ESU{}.
  We begin by introducing a particular type of scenario involving ability, and observe how~\ESU{} requires a unique interpretation of the scenario.
  We then introduce an additional principle regarding support, which conflicts with the interpretation of the type of scenario introduction required by~\ESU{}.
\end{note}

\begin{note}[Introducing key parts]
  Type of information and entailment.
  Two ways to understand entailment.
  Then, if information and entailment \dots
  Principles constrain understanding.
  \ESU{} and a second principle.
\end{note}

\section{Old ideas about \requ{1}}
\label{sec:old-ideas-about}

\subsection{\result{3} and \requ{1}}
\label{sec:claim-supp-requ}

\begin{note}
  To develop \ideaS{} and \ideaCS{} into assumptions we start by defining a `\requ{}' of some instance of reasoning.

  Both \ideaS{} and \ideaCS{} concern propositions having values appealed to in reasoning.
  In turn, the role of a \requ{} is to, on the one hand provide a sufficiently clear account of an important proposition-value pairs.

  Of course, one may hold that additional proposition-value pairs, or indeed other factors beyond \requ{1} are important for claiming support.
  Our interest, however, will be limited to \requ{1} (and an extension of \requ{1}).
  For, our interest is with providing sufficient conditions to highlight failures of claiming support.
\end{note}



\begin{note}
  To make this precise, we first define a `\result{}' of making a step of some instance of reasoning.
  This will then allow us to identify specific instances of \requ{1}.
  Leads to a general definition.
  Important distinction between (kinds of) \prequ{1} and \crequ{1}.

  Given \requ{} in general, \ideaCS{} is going to require some stuff.
  However, specifically with respect to (kinds of) \prequ{1}.
\end{note}

\subsubsection{\result{3}}
\label{sec:def-of-result}

\paragraph{Definition}

\begin{note}
  \begin{restatable}[\result{3} of (making a step of) some reasoning]{definition}{defResult}
    \label{def:result}
    Let \(R\) be some instance of reasoning which concludes that \(\phi\) has value \(v\).

    \(\psi\) has value \(v'\) is \emph{\result{}} of a step \(\delta\) of reasoning \(R\) if and only if:
    \begin{enumerate}[label=\arabic*., ref=\named{P:\arabic*}]
    % \item
      % \label{def:result:ep}
      % \(\psi\) not having value \(v'\) is an \ep{}.
    \item
      \label{def:result:conseq}
      Making step \(\delta\) while holding that \(\psi\) does not have value \(v'\), relative to other commitments of the agent (if any), would involve either:
      \begin{enumerate}[label=\alph*., ref=\named{P:2\alph*}]
      \item
        \label{def:result:conseq:a}
        A premise having some other value.
      \item
        \label{def:result:conseq:c}
        The conclusion having some other value.
      \end{enumerate}
    \item
      \label{def:requ:persists}
      No other part of the reasoning that concludes that \(\phi\) has value \(v\) involves a step \(\delta'\) such that \(\psi\) having value \(v'\) satisfies \ref{def:result:conseq} with respect to \(\delta'\).
    \end{enumerate}
    \vspace{-\baselineskip}
  \end{restatable}
\end{note}

\begin{note}
  A `\result{}' of a step of reasoning is a complex thing.
  First, at least one step.
  And, no step which avoids.
  Local and global.

  Key interest is following from reasoning, so global.
  But also important to pinpoint step, so local.
\end{note}

\paragraph*{Detailing \autoref{def:result} through an example with a variation}

\begin{note}
  Consider the following reasoning:
  \begin{example}\mbox{}
    The agent is working for some company, and is in a room of an office building which has a clock on the wall.
    \begin{enumerate}[label=\arabic*., ref=(\arabic*)]
    \item\label{result:ex:temp:w} The clock appears to be working
    \item\label{result:ex:temp:f} The clock is working. \hfill \CStepA{}: From \ref{result:ex:temp:w}
    \item\label{result:ex:temp:t} The clock reads 5:05pm.
    \item\label{result:ex:temp:c} It is 5:05pm. \hfill \CStepB{}: From \ref{result:ex:temp:f} \& \ref{result:ex:temp:t}
    \item\label{result:ex:temp:t-to-d} If it is after 5pm, no meetings may be started.
    \item\label{result:ex:temp:d} No meetings may be started.
      \hfill \CStepC{}: From~\ref{result:ex:temp:c} \&~\ref{result:ex:temp:t-to-d}
    \end{enumerate}
  \end{example}

  \ref{def:result:ep} requires that \(\psi\) not having value \(v'\) is an \ep{}.
  Our interest is only in proposition-value pairs `involved' in some instance of reasoning that may have some other value from the perspective of the agent.

  It is plausible that neither \ref{result:ex:temp:t} nor \ref{result:ex:temp:t-to-d} are \ep{1} for the agent.
  For, we may assume that~\ref{result:ex:temp:t} is obtained by visual inspection, and that \ref{result:ex:temp:t-to-d} is part of the company's rule-book.

  Still, we may consider the remaining proposition-value pairs as \ep{}.
  For, the clock may not be working, and, as the agent appeals to the clock working in order to reason to it being 5:05pm, it (also) may not be 5:05pm.
  In particular, it may be that it is 4:55pm, and so there is a few minutes in which someone may start a meeting.
  So,~\ref{result:ex:temp:w},~\ref{result:ex:temp:f},~\ref{result:ex:temp:c}, and~\ref{result:ex:temp:d} are all candidates for being \result{1}
\end{note}

\begin{note}
  Still, given \ref{def:result:ep}, any \ep{} proposition-value pair is a candidate for being a \result{}.
  For example, it being the case that it will rain tomorrow, or that one would enjoy what follows the 10o'clock news this evening.
  Hence, \ref{def:result:conseq} ensures that the proposition value pair is relevant to the instance reasoning.
\end{note}

\begin{note}
  Let us consider Steps B and C in detail:

  If the clock is not working, the \CStepB{} would involve appealing to something that is not the case, and hence \ref{result:ex:temp:f} satisfies \ref{def:result:conseq:a} with respect to \CStepB{}.
  Likewise, if the time is not 5:05pm, then \CStepB{} would involve moving to something that is not the case, and hence \ref{result:ex:temp:c} satisfies \ref{def:result:conseq:c}.

  Similarly, the time being 5:05pm satisfies \ref{def:result:conseq:a} with respect to \CStepC{}, and the that no meeting may be started satisfies \ref{def:result:conseq:c}.\nolinebreak
  \footnote{
    Note, however, that the clock working does not satisfy \ref{def:result:conseq:a} with respect to \CStepC{}, as the step moves from what the time is, not the report of what the time is by a working clock.
  }

  And, without any background constraints, no other proposition-value pairs seems to satisfy~\ref{def:result:conseq}.
\end{note}

\begin{note}
  Still,~\ref{def:result:conseq} mentions `other commitments' of the agent `(if any)'.
  Hence, the time being 5:05pm is not the only proposition-value pair which satisfies \ref{def:result:conseq:c}.
  For,~\ref{result:ex:temp:t-to-d} is a commitment of the agent:
  If the meeting may be started, then the time is not 5:05p.
  So, that no meetings may be started \emph{also} satisfies \ref{def:result:conseq:c} with respect to \CStepB{}.
\end{note}

\begin{note}[Summary]
  The preceding is summarised in~\autoref{fig:def:result:conseq:table}, with \CStepA{} also included.
  \begin{figure}[!h]
    \centering
    \begin{tabular}{  c | c | c | c  }
      & \CStepA{} & \CStepB{} & \CStepC{} \\
      \hline
      \ref{def:result:conseq:a} & \ref{result:ex:temp:w} & \ref{result:ex:temp:f} & \ref{result:ex:temp:c} \\
      \hline
      \ref{def:result:conseq:c} & \ref{result:ex:temp:f} & \ref{result:ex:temp:c}, \ref{result:ex:temp:d}  & \ref{result:ex:temp:d}
    \end{tabular}
    \caption{Proposition-value pairs satisfying \ref{def:result:conseq} with respect to steps.}
    \label{fig:def:result:conseq:table}
  \end{figure}
\end{note}

\begin{note}
  The final clause of \ref{def:result} to consider is \ref{def:requ:persists}.
  As noted, the basic idea behind \ref{def:requ:persists} is to ensure that the relevant proposition-value pair matters to the conclusion of the instance of reasoning for which the step is a part.

  The notion of a `part' of reasoning is left intuitive.
  However, that the relevant `parts' of reasoning provides sufficient information to observe that the relevant proposition-value pairs identified all satisfy \ref{def:requ:persists}.
  For, that no meetings may be started is the conclusion of the instance of reasoning, and both steps A and B are involved in obtaining this conclusion.

  Hence,~\ref{result:ex:temp:f},~\ref{result:ex:temp:c} and~\ref{result:ex:temp:d} are all \result{1} of the instance of reasoning with respect to \CStepB{}, and~\ref{result:ex:temp:c} and~\ref{result:ex:temp:d} are \result{1} of reasoning with respect to \CStepC{}.\nolinebreak
  \footnote{
    And both \ref{result:ex:temp:w} and \ref{result:ex:temp:f} are \result{1} with respect to \CStepA{}.
  }
  Intuitively, each proposition-value pair is involved in obtaining the conclusion.
\end{note}

\begin{note}
  Still, to illustrate why~\ref{def:requ:persists} is important, consider the agent receiving a memo prior to their reasoning, which states the following:
  \begin{enumerate}[label=\arabic*\('\), ref=(\arabic*\('\))]
  \item\label{result:ex:temp:memo} If, and only if, the clock is not working, no meetings shall be held.
  \end{enumerate}
  Let us assume the memo is stamped, and therefore that the information contained is not an \ep{}.

  The agent now has the option of reasoning by cases.
  If the clock is working, then the agent reasons to the conclusion that no meetings may be started as in~\ref{result:ex:temp:f} to~\ref{result:ex:temp:d}, above.
  However, if the clock is not working, then the agent may obtain the conclusion from \ref{result:ex:temp:memo}.
  In particular, the agent would not need to appeal to the clock working in order to appeal to the clock not working, so \ref{result:ex:temp:f} would not be a \result{} of \CStepB{}.
  And, likewise the agent would obtain the conclusion without appealing to the time being 5:05pm.
  Hence, \ref{result:ex:temp:c} would neither be a \result{} of \CStepB{} nor of \CStepC{}.

  So, only \ref{result:ex:temp:d} would be a \result{} of reasoning by cases.
\end{note}

\begin{note}
  The details here are secondary to the key idea:

  Not only does \(\psi\) having value \(v'\) follow as a result of making the step of reasoning, but making the step of reasoning is such that the conclusion of reasoning is obtained in part by making the relevant step of reasoning.

  If the agent has the option of reasoning by cases, then the agent's reasoning does not depend any proposition-value pair that is unique to a particular case.
  Hence, that proposition-value pair will not be a \result{}.
\end{note}

\begin{note}
  Determining whether a proposition-value pair is a \result{} of some step of reasoning is, in general, difficult.
  For, \ref{def:requ:persists} is a global property.
  And, in general, some instance of reasoning may contain multiple steps, numerous instances of reasoning by cases, and so on.

  Naturally, complex reasoning is difficult, and ideas applied are significant.
  We will not have any direct interest in a complex of step --- in particular, our focus will be on straightforward (sub-)instances of reasoning which involve a single step.
  However, in the case that the step is part of a larger complex, \ref{def:requ:persists} is important to highlight why some proposition-value pair associated with the step matters.
\end{note}

\subsubsection{\requ{3}}
\label{sec:def-of-requ}

\begin{note}
  With the definition of a \result{} in hand, we not turn to defining \requ{1}.
\end{note}

\begin{note}
  Intuitively, a \requ{} of a step of reasoning is a non-disposable consequence of some step of reasoning and the reasoning as a whole, whose failure is epistemic possibility for the agent, and is such that the failure of the \requ{} would mean that the step of reasoning appealed to something that is not the case.

  The definition of a \result{} clarified how a proposition-value pair is important in some instance of reasoning, but the definition of a \result{} alone does not account for why it is not possible for the agent to conclude the instance of reasoning without (indirectly) appealing to the proposition-value pair.
  To capture the mentioned role of proposition-value pair, the definition of a \requ{} is broken down into three distinct types of \requ{}, with the definition of a \result{} being used to clarify two basic \requ{}-types.
\end{note}

\begin{note}
  The definition of a \requ{} is as follows:
  \begin{restatable}[\requ{3} of a step of reasoning]{definition}{defRequisite}
    \label{def:requsite}
    \(\psi\) has value \(v'\) is a \emph{\requ{}} of a step of reasoning that concludes that \(\phi\) has value \(v\) (from an agent's perspective) if and only if:
    \begin{enumerate}[label=\arabic*., ref=\named{R:\arabic*}]
    \item[] \(\psi\) having value \(v'\) is either:
      \begin{enumerate}
      \item
        A \prequ{}.
      \item
        A \crequ{}.
      \item
        A \cprequ{}.
      \end{enumerate}
      \vspace{-\baselineskip}
    \end{enumerate}
    \vspace{-\baselineskip}
  \end{restatable}
\end{note}

\begin{note}
  Before proceeding to define the distinct types of \requ{}, a quick note on the definitions that follow.

  As with the definition of a \result{}, each definition of a \requ{}-type focuses on some step of reasoning.
  However, as with the definition of a \result{}, each definition of a \requ{}-type will include a `global' clause which ensures that the proposition-value pair is not only important for the step of reasoning, but for the instance of reasoning as a whole.
  The clause present in each of the definitions is as follows:
  \begin{itemize}
  \item
    \requGlobalClause{}
  \end{itemize}
  This clause means that the collection of definitions is circular.
  We a \requ{} is in part defined in terms of a \prequ{}, and in turn a \prequ{} is in part defined in terms of a \requ{}.
  However, the circularity is benign.
  In is simply a convenient way to quantify over all other \requ{}-types.
  Semantically, the result of expanding each definition is a check that no other proposition-value pair is one of the three \requ{}-types defined.\nolinebreak
  \footnote{
    Though, syntactically, expanding each definition results in a proposition of unbounded length.
  }
\end{note}

\paragraph{\pandcrequ{3}}

\begin{note}
  We begin by defining \prequ{1} and \crequ{1}.
  These two definitions build on the definition of a \result{}.\nolinebreak
  \footnote{
    The reason for separating these definitions is

    First, the separation of the definitions helps reduce the complexity of the respective definitions.
    Second, separation helps shift focus.
    With \result{}, involvement.
    With \requ{}, importance.
  }
\end{note}

\subparagraph*{\prequ{3}}

\begin{note}
  \begin{restatable}[\prequ{2} of a step of reasoning]{definition}{defRequisiteP}
    \label{def:prequ}
    \(\psi\) has value \(v'\) is a \emph{\prequ{0}} of some step \(\delta\) of reasoning if and only if

    \begin{enumerate}[label=\arabic*., ref=\named{pR:\arabic*}]
    \item
      \(\psi\) has value \(v'\) is a \result{0} of the \emph{premises} of Step \(\delta\).\nolinebreak
      \footnote{
        I.e.\ by satisfying~\ref{def:result:conseq:a} of~\autoref{def:result}.
      }
    \item
      \label{def:prequ:subjunctive}
      It is not \epVAd{0} for the agent to make Step \(\delta\) without either:
      \begin{enumerate}[label=\alph*., ref=\named{pR:2\alph*}]
      \item
        \label{def:prequ:subjunctive:psi}
        \(\psi\) having value \(v'\) prior to Step \(\delta\), or
      \item
        \label{def:prequ:subjunctive:other}
        the other commitments (if any) which lead to \(\psi\) having value \(v'\) being a \result{0} of Step \(\delta\) prior to making Step \(\delta\).
      \end{enumerate}
    \item
      \requGlobalClause{}
    \end{enumerate}
    \vspace{-\baselineskip}
  \end{restatable}
\end{note}

\begin{note}[Summary of \prequ{0}]
  In short, a \prequ{0} is simply a \result{0} which follows from the \emph{premises} of the step of reasoning with the added constraint regarding the relevant commitments of the agent, if there are any.
\end{note}

\begin{note}
  To illustrate the important of \ref{def:prequ:subjunctive}, consider the agent having the following commitments prior to the instance of reasoning:

  \begin{enumerate}[label=B\(^{+}\)\arabic*., ref=(B\(^{+}\)\arabic*)]
  \item
    \label{result:ex:temp:repair}
    If the clock is working, then it was repaired yesterday.
  \item
    \label{result:ex:temp:chime}
    If the clock is working, then it chimes on the hour.
  \end{enumerate}

  \CStepB{} involved appeal to the proposition-value pair that the clock is working.
  So, the consequents of both conditionals are \result{1} of \CStepB{} (of the reasoning from \ref{result:ex:temp:f} to \ref{result:ex:temp:d}).\nolinebreak
  \footnote{
    I.e.\ excluding the possibility instance of reasoning by cases introduced by \ref{result:ex:temp:memo}.
  }
  Hence, the consequents of both conditionals are candidates for being \requ{1}.

  However, the two consequents differ with respect to whether it would be \epVAd{} for the agent to make \CStepB{}.
  If the clock was not repaired yesterday, then it seems the agent would refrain from appealing to the clock functioning.
  By contrast, if the clock does not chime on the hour, the agent may hold that \ref{result:ex:temp:chime} is false, rather than being committed to the clock being broken.

  Of course, this evaluation assumes further information.
  First, that the clock was in need of repair, and second that the agent does not consider whether the clock chimes to be important for whether the clock is telling the correct time.

  In general, the possible \result{1}, and hence \prequ{1}, of a step of reasoning may be substantial, given the range of commitments an agent may have.
  Still, the key idea of \ref{def:prequ:subjunctive} is hopefully clear:
  \begin{itemize}
  \item
    A \prequ{} of a step of reasoning is a result of that step of reasoning that is not only involved in making the step, but is also important to maintaining that what the agent appeals to when making the step is appropriate.
  \end{itemize}
\end{note}

\begin{note}
  Note, also, that \ref{result:ex:temp:f} is clearly a \prequ{0} with respect to \CStepB{}.
  And, \ref{result:ex:temp:c} is clearly a \prequ{} with respect to \CStepC{}.
\end{note}

\subparagraph{\crequ{3}}

\begin{note}[Moving to \crequ{0}]
  The definition of a \crequ{0} is the same as the definition of a \prequ{0}, but for attention on the conclusion, rather than the premises of the step of reasoning.
\end{note}

\begin{note}
  \begin{restatable}[\crequ{2} of a step of reasoning]{definition}{defRequisiteC}
    \label{def:crequ}
    \(\psi\) has value \(v'\) is a \emph{\crequ{0}} of a step of reasoning if and only if
    \begin{enumerate}[label=\arabic*., ref=\named{cR:\arabic*}]
    \item
      \(\psi\) has value \(v'\) is a \result{0} of the \emph{conclusion} of Step \(\delta\).\nolinebreak
      \footnote{
        I.e.\ by satisfying~\ref{def:result:conseq:c} of~\autoref{def:result}.
      }
    \item
      \label{def:crequ:subjunctive}
      It is not \epVAd{} for the agent to make Step \(\delta\) without either:
      \begin{enumerate}[label=\alph*., ref=\named{cR:2\alph*}]
      \item \(\psi\) having value \(v'\), or
      \item the other commitments (if any) which lead to \(\psi\) having value \(v'\) being a \result{0} of Step \(\delta\).
      \end{enumerate}
    \item
      \requGlobalClause{}
    \end{enumerate}
    \vspace{-\baselineskip}
  \end{restatable}
\end{note}

\begin{note}
  Intuitively, moving to something that is not the case, such that no making the step.
\end{note}

\begin{note}
  For example, consider the following two conditionals with respect to \CStepC{}:
  \begin{enumerate}[label=C\(^{+}\)\arabic*., ref=(C\(^{+}\)\arabic*)]
  \item
    \label{result:ex:temp:B:summon}
    If no meetings may be started, then I may not summon my team for a meeting.
  \item
    \label{result:ex:temp:B:see}
    If no meetings may be started, then I won't see \nagent{12} until tomorrow, at the earliest.
  \end{enumerate}
  The consequents of both conditionals are \result{1} of making \CStepC{}, and as they follow from what \CStepC{} establishes, both consequents are candidates for being \crequ{1}.
  However, while the agent may not give up \ref{result:ex:temp:B:summon}, it seems the agent would easily give up \ref{result:ex:temp:B:see} if \nagent{12} were to (unexpectedly) visit the office.
  Hence, only the consequent of \ref{result:ex:temp:B:summon} is a \crequ{1} of \CStepC{}.
\end{note}

\begin{note}
  As before, the possible \result{1}, and hence \crequ{1}, of a step of reasoning may be substantial, given the range of commitments an agent may have.
  Still, the key idea of \ref{def:crequ:subjunctive} is hopefully clear:
  \begin{itemize}
  \item
    A \crequ{} of a step of reasoning is a result of that step of reasoning that is not only involved in making the step, but is also important for maintaining that what follows from making the step is appropriate.
  \end{itemize}
\end{note}

\begin{note}
  Note, also, that \ref{result:ex:temp:d} is clearly a \crequ{} of \CStepC{}.
  And, both \ref{result:ex:temp:c} and~\ref{result:ex:temp:d} are clearly \crequ{1} of \CStepB{}.
\end{note}

\subparagraph*{Summarising \prequ{1} and \crequ{1}}

\begin{note}
  From the perspective of their definition, \crequ{1} and \prequ{1} are distinguished only by how \(\psi\) having value \(v'\) is a \result{0}.
  In the case of \prequ{1}, \(\psi\) having value \(v'\) is a \result{0} of the premises, while in the case of \cprequ{1}, it is a \result{0} of the conclusion.
\end{note}

\begin{note}[No \crequ{1} wrt.\ deductive reasoning]
  \phantlabel{crequ-only-non-deductive}
  As noted, we distinguish \prequ{1} and \crequ{1} in part for the definition to follow, but also due to the kind of reasoning the definitions apply to.
  For, in contrast to \prequ{1}, there are no \crequ{1} in the case of purely deductive reasoning.

  For, in the case of deductive reasoning, any conclusion of the step of reasoning follows from the relevant premises, in the sense that it is not \epVAd{} for the agent to jointly hold that the premises of the step have their respective values while the conclusion of the step does not have some value.\nolinebreak
  \footnote{
    This is only to state that, with respect to \emph{conditional detachment}, it is not \epVAd{} for an agent to hold that \(\phi\) is true and \(\phi \rightarrow \psi\) is true and not to hold that \(\psi\) is false.
    And, it is compatible with this statement that the agent holds both the premises but does not entertain \(\psi\) as either true or false.
  }

  Hence, in order for \(\psi\) having value \(v'\) to be a \crequ{}, the step of reasoning must introduce something which does not directly follow from the premises.
  For example, moving from inductive considerations for \(\phi\) being true to \(\phi\) being true.
\end{note}

\begin{note}
  \phantlabel{step-a-non-deductive}
  Indeed, \CStepA{} is an example of a non-deductive step.

  It does not deductively follow from the clock appearing to be working that the clock is working.
\end{note}

\paragraph{\cprequ{3}}

\begin{note}
  We now turn to the third (and final) type of \requ{}.
  In contrast to \pandcrequ{1}, this type of \requ{} builds on parts of the definition of a \result{}, and on the definition \crequ{}.

  The key idea is that:
  \begin{itemize}
  \item
    In the case of non-deductive reasoning, a step of reasoning may introduce a proposition-value pair which interacts with the premises of the step to yield further proposition-value pairs that are important for maintaining that what follows from making the step is appropriate.
  \end{itemize}
  Hence, follow from the note \hyperref[crequ-only-non-deductive]{above}, \cprequ{1} are unique to non-deductive reasoning.
\end{note}

\begin{note}
  \begin{restatable}[\cprequ{2} of (a step of) reasoning]{definition}{defRequisiteCP}
    \label{def:cprequ}
    \(\psi\) has value \(v'\) is a \emph{\cprequ{0}} of a step of reasoning \(\delta\) iff
    \begin{enumerate}[label=\arabic*., ref=\named{cpR:\arabic*}]
    \item
      \label{def:cprequ:result}
      \(\psi\) has value \(v'\) is a \result{0} of the step of reasoning \(\delta\).
    \item It is the case that:
      \label{def:cprequ:subjunctive}
      \begin{enumerate}[label=\alph*., ref=\named{cpR:2\alph*}]
      \item
        \label{def:cprequ:subjunctive:chi}
        \(\chi_{i}\) having values \(v''_{i}\) are \crequ{1} of \(\delta\).
      \item
        \label{def:cprequ:subjunctive:unique}
        \(\psi \ne \chi_{i}\) for any \(i\).
      \item
        \label{def:cprequ:subjunctive:relation}
        It is not \epVAd{} for the agent to jointly hold:
          \begin{enumerate}
          \item
            The \prequ{1} of \(\delta\),
          \item
            that \(\chi_{i}\) have values \(v''_{i}\), and
          \item
            that \(\psi\) does not have value \(v'\).
          \end{enumerate}
      \end{enumerate}
    \item
      \requGlobalClause{}
    \end{enumerate}
    \vspace{-\baselineskip}
  \end{restatable}
\end{note}

\begin{note}
  As with the definitions of \pandcrequ{1}, the first clause of the definition of a \cprequ{} requires that \(\psi\) having value \(v'\) is a \result{0} of the step of reasoning \(\delta\).
  However, unlike the definitions of \pandcrequ{1}, \ref{def:cprequ:result} does not constrain the way in which \(\psi\) having value \(v'\) is a \result{0}.

  Instead, the relevant clarification is found in~\ref{def:cprequ:subjunctive}.
  In short,~\ref{def:cprequ:subjunctive} holds when:
  \begin{enumerate}[label=\alph*.]
  \item There are some proposition-value pairs which follow from the conclusion of the step,
  \item \(\psi\) having value \(v'\) is not (merely) a \crequ{}, and
  \item \(\psi\) having value \(v'\) follows from those proposition-value pairs.
  \end{enumerate}
  In other words, then, the importance of \(\psi\) having value \(v'\) to the step of reasoning is introduced by the proposition-values pairs \(\chi_{i}\) having values \(v''_{i}\) being a \crequ{}.

  In this respect, \ref{def:cprequ:subjunctive:unique} is important to ensure that \(\psi\) having value \(v'\) is not a \crequ{}.
  For, \(\chi\) having value \(v''\) trivially follows from \(\chi\) having value \(v''\), and hence without the restriction imposed by \ref{def:cprequ:subjunctive:unique}, any \crequ{} would also be a \cprequ{}.
\end{note}

\begin{note}
  As noted \hyperref[step-a-non-deductive]{above}, \CStepA{} is a non-deductive step.
  It is not the case that the clock is working follows deductive from the appearance that the clock is working.
  So, with respect to \CStepA{}, that it is the case that the clock is working is a relevant \(\chi_{i}\) having value \(v''\) instance of \ref{def:crequ:subjunctive}.

  And, as \CStepB{} involved the clock working as a premise, both conditionals we considered as commitments prior to the instance of reasoning may lead to \cprequ{1} with respect to \CStepA{}.
  Restated:

  \begin{enumerate}[label=A\(^{+}\)\arabic*., ref=(A\(^{+}\)\arabic*)]
  \item
    \label{result:ex:temp:repair}
    If the clock is working, then it was repaired yesterday.
  \item
    \label{result:ex:temp:chime}
    If the clock is working, then it chimes on the hour.
  \end{enumerate}

  In both cases, the consequents of the conditionals follow from a \crequ{} of \CStepA{} --- the conclusion of the step itself.
  Hence, the consequents of both conditionals are relevant instance of \(\psi\) having value \(v'\).

  As before, it seems that (granting a natural interpretation of the agent's epistemic state), that it is not \epVAd{} for the agent to hold that the clock is working \emph{and} that the clock was not repaired yesterday, hence the consequent of \ref{result:ex:temp:repair} is plausibly a \cprequ{} of \CStepA{}.

  And, as before, by contrast it seems \epVAd{} for the agent to abandon \ref{result:ex:temp:chime} if the clock does not chime, and so the consequent of \ref{result:ex:temp:chime} it plausibly not a \cprequ{} of \CStepA{}.
\end{note}

\hozline

\begin{note}[No further interactions]
  An important observation to make is that any proposition-value pair which follows from a \cprequ{} is also a \cprequ{}.
  In short, this is because something being a \cprequ{} of some step requires the step to have introduced a novel proposition-value pair, and assessing whether some proposition-value pair is a \cprequ{} does not involve steps which may introduce novel proposition-value pairs.\nolinebreak
  \footnote{
    In more detail:

    Suppose \(\psi\) having value \(v'\) is a \cprequ{} of some step of reasoning.
    And, suppose that \(\psi_{+}\) having value \(v'_{+}\) follows from the combination of some \crequ{1} of the step, the \prequ{1} of the step. and \(\psi\) having value \(v'\) (and is distinct from any \(\chi_{i}\)).

    If so, then \(\psi_{+}\) having value \(v'_{+}\) satisfies \ref{def:cprequ:subjunctive}.
    For, if \(\psi_{+}\) were not to have value \(v'_{+}\), then \(\psi\) would not have value \(v'\).
    Hence, holding that \(\psi_{+}\) were not to have value \(v'_{+}\) would not be \epVAd{}.
  }

  Hence, the combination of \prequ{1}, \crequ{1}, and \cprequ{1} exhaust the scope of the intuitive idea of a \requ{}.
\end{note}

\paragraph{Summary}

\begin{note}
  Key thing with \requ{1} is that the proposition-value pair is important for making the step.

  This is a difficult boundary to draw.
  Only interested in things which deductively follow and are important.
  Consequence of this is that in deductive steps, any \crequ{} is also a \prequ{}.
\end{note}


\begin{note}
  \autoref{def:requsite} is how we build on \ideaS{} and \ideaCS{}.
  For, the definition of a \requ{} applies to reasoning in general, claiming support is an instance of reasoning, and hence applies to instances of claiming support.
  In particular, in instances of claiming support:
  {
    \color{reword}
    Conclusion is a \requ{}, as epistemically possible that it does not hold, and non-disposable.
    Moving from sub-premises to sub-conclusions plausibly involves a \requ{}, given the epistemic possibility that a conclusion of an instance of claiming support does not hold.

    So, \ideaS{} for possibility.
    Dealing with \requ{1} for \ideaCS{}.
    Indeed, \ideaCS{} will focus on \requ{} for step, rather than reasoning in general.
  }

  Still, as \autoref{def:requsite} is quite complex, we will first work through the clauses and how they combine in some detail before providing a handful of \illu{1}.
\end{note}

\paragraph{\illu{3}}

\begin{note}
  Four \illu{1}.
  The first pair identify \requ{1}.
  Second pair concern consequences of steps of reasoning which fail to be \requ{1}.
\end{note}

\begin{note}
  \begin{illustration}
    \label{illu:requ:bank}
    `Where is the nearest bank?'
    Reason that `next to the post office' is an appropriate response.
  \end{illustration}
  However, friend is asking about where the nearest bank in the sense of financial establishment, rather than river bank.
  If appropriate disambiguation, then appealed to something that is not the case.
  Hence, that `river bank' is not the appropriate disambiguation is a \requ{}.

  Note, however, that if agent did not learn about ambiguity of `bank' then no \requ{}.
\end{note}

\begin{note}
  We will see many more examples of \requ{1} below.
  For the moment, we offer a second, abstract, instance of a \requ{} and then an instance of something which is not a \requ{}.
\end{note}

\begin{note}
  \begin{illustration}
    \label{illu:requ:import-export}
    Suppose we have some conditional `\(\rightarrow\)' that admits of import-export.\nolinebreak
    \footnote{
      For example, the material conditional, but not necessarily the natural language conditional expressed in certain `if \dots then \dots' constructions {\color{red} Vann}.
    }
    I.e.:
    \begin{quote}
      \(\phi \rightarrow (\psi \rightarrow \xi)\) if and only if \((\phi \text{ and } \psi) \rightarrow \xi\)
    \end{quote}
    And, suppose an agent's reasoning has the structure:
    \begin{enumerate}
    \item \(\phi\)
    \item \(\phi \rightarrow (\psi \rightarrow \xi)\)
    \item \(\psi \rightarrow \xi\)
    \end{enumerate}
    Such that possible \(\phi \rightarrow (\psi \rightarrow \xi)\) is not the case.
  \end{illustration}

  Here, \((\phi \text{ and } \psi) \rightarrow \xi\) is a \prequ{} of the step from 2 to 3.
  For, if \(\phi \rightarrow (\psi \rightarrow \xi)\) then also \((\phi \text{ and } \psi) \rightarrow \xi\).
  Hence, if \((\phi \text{ and } \psi) \rightarrow \xi\) is not the case, then \(\phi \rightarrow (\psi \rightarrow \xi)\) is also not the case.
\end{note}

\begin{note}
  In these two \illu{1}, something that would highlight a fault in reasoning.
  No all things that would highlight faults in reasoning are \requ{1}.
\end{note}


\begin{note}
  {
    \color{red}
    This is a \requ{}.
    Point is that should indicate that the textbook answer is the same!
    It doesn't follow that it is the same.
    But, should indicate at least.

    Or, if textbook having the wrong answer is not \epVAd{}, not a \requ{}, because not a \result{}.
    Agent doesn't need to indicate this is the case.
  }
  \begin{illustration}[Textbook answers]
    \label{illu:textbook-answers}
    Suppose an agent is working through a logic textbook with practice problems at the end of each chapter to test a student's understanding.

    For any given question, the following conditional holds with respect to the agent:
    \begin{itemize}
    \item If my answer to the problem is incorrect, then I {\color{red} appealed to something that is not the case or moved to something that is not the case}.
    \end{itemize}
  \end{illustration}
  Possible that the agent {\color{red} appealed to something that is not the case or moved to something that is not the case}.

  The proposition of interest here is:
  \begin{itemize}
  \item My answer is not incorrect.
  \item Or, the textbook has the same answer.
  \end{itemize}

  Not a \requ{} of any intermediate step of reasoning, as these don't get the answer.
  Not a \requ{} of final step, because there's no plausible consequence relation.
  Making the step doesn't imply anything about the textbook.

  Similarly, for if the textbook has not made a mistake, then same answer.

  Because, although it something not being the case would ensure the step of reasoning is bad, it does not follow that by making the step of reasoning, it is a consequence that there is no mistake.

  Key point is that a \requ{} follows from some part of the agent's reasoning, not from things that would hold if the agent's reasoning is successful.
\end{note}

\subsubsection{Being \mistaken{} or \misled{}}

\begin{note}
  \begin{restatable}[\mistaken{0} and \misled{}]{definition}{defMoM}\label{def:MoM}
    Some reasoning such that reasoning culminates with \(\phi\) having value \(v\).
    \begin{itemize}
    \item
      The reasoning is \emph{\mistaken{}} if \(\psi\) having value \(v'\) is a \prequ{} and \(\psi\) does not have value \(v'\).
    \item
      The reasoning is \emph{\misled{}} if \(\psi\) having value \(v'\) is either a \crequ{} or a \cprequ{} and \(\psi\) does not have value \(v'\).
    \end{itemize}
    \vspace{-\baselineskip}
  \end{restatable}

  Generalise to reasoning.
  \mom{} just in case there exists some step that is \mom{}, respectively.
\end{note}

\begin{note}[M\&M \illu{2}]
  To illustrate:

  \begin{illustration}[Clock]
    \label{illu:mom:clock}
    Suppose I glance at the clock on the wall.
    The clock reads 11:45a, so I reason that it is 11:45a.
  \end{illustration}

  Two possibilities:
  \begin{enumerate}
  \item Clock is not functioning.
  \item Clock is incorrectly set.
  \end{enumerate}

  If not functioning.
  By claiming support from the time expressed by the clock, I would have been \emph{\misled{}} about what the time actually is.
  For, not functioning.
  However, not necessarily \mistaken{}.
  For, might be that the time is 11:45a.

  If incorrectly set.
  By claiming support from the time expressed by the clock, I would have been \emph{\mistaken{}} about what the time actually is.

  However, not necessarily \misled{}.
  For, claimed support by appeal to a functioning clock.
  Though, despite the clock being broken, it is 11:45a and so the claim to support is not misleading.

  Combining, claimed support for the time from a broken clock expressing the wrong time would be both \misled{} \emph{and} \mistaken{}.\nolinebreak
  \footnote{
    A second \illu{0}:
    Consider a smoke detector, designed to sound an alarm if and only if sufficient levels of smoke are detected.
    Hence, if the alarm sounds, one may claim support there being smoke in the room where the alarm is installed.
    One may be misled; the alarm may have malfunctioned, so no fire.
    Or, one may be mistaken; the same type of alarm may be installed in a different room, wouldn't be a useful indicator.
  }

  {
    \color{red}
    Of course, clocks are typically glanced at, and a glance at a clock is often insufficient to determine whether the clock is incorrectly set or broken.
    Hence, the \emph{possibility} that a clock is incorrectly set or broken --- or more broadly the possibility that claimed support is misleading or mistaken --- does not prevent an agent from claiming support.
    So, ensuring that to-be-claimed support would be \mom{} is not a necessary condition for claiming support.
  }
\end{note}

\paragraph{Summary}

\begin{note}
  Not the case that any thing that is appealed to is a \requ{}.
  For, may add in certain additional things.
  In other words, careful to make sure that the subjunctive part is there.
\end{note}

\begin{note}
  Really important here is that being a \requ{} does not imply that the reasoning involves direct appeal.
  {
    \color{red}
    Indeed, this distinguishes from \citeauthor{Sgaravatti:2013wu}, which requires belief in each of premises.
  }
\end{note}

\section{Old assumption}
\label{sec:old-assumption}

\section{Refining \ideaCS{} into an assumption}
\label{sec:assumpt-from-ideas}

\begin{note}
  Two ideas, \ideaS{} and \ideaCS{}.
  Possibility, and requirement from this.
  As noted, claiming support is still quite general, and this is by design.
\end{note}

\begin{note}
  \ideaCS{}.
  Claiming support indicates regardless.

  Developing this.

  Key idea is reasoning.

  What it is to `appeal' to a proposition-value pair.
  Develop this through a series of definitions.
  These are, unfortunately, complex.

  The basic idea, however, is sort of okay.
  What we are interested in is proposition-value pairs that are important to the premises.

  Term this a \requ{}.

  Of course, in a broader sense of `appeal', reasoning involves appeal to proposition-value pairs which are not \ep{1}.
  However, no particular interest in these.
  The sense of appeal is stronger.
\end{note}

\subsubsection{The assumption}
\label{sec:two-assumpt-relat-to-requ}

\begin{note}
  Now turn to \ideaCS{}.
\end{note}

\begin{note}
  \begin{restatable}[\eiS{0} --- \eiS{}]{assumption}{assuCSRReq}
    \label{assu:supp:independence}
    Let \vAgent{} be an agent and suppose that:
    \begin{enumerate}[label=\Alph*., ref=(\Alph*)]
    \item
      \label{assu:supp:ind:step}
      \(\psi\) having value \(v'\) is a \requ{} of some step \(\delta\) of some instance of reasoning that concludes with \(\phi\) having value \(v\).
    \end{enumerate}

    The instance of reasoning that concludes with \(\phi\) having value \(v\) is an instance of claiming support \emph{only if}:

    \begin{enumerate}[label=\arabic*., ref=(\arabic*)]
    \item
      \label{assu:supp:ind:indicate}
      \vAgent{} may --- prior to making \(\delta\) --- may witness some reasoning which \indicateV{1} that \(\psi\) has value \(v'\).
    \end{enumerate}
    And:
    \begin{enumerate}[label=\arabic*., ref=(\arabic*),resume]
    \item
      \label{assu:supp:ind:circular}
      If \(\delta\) is a non-deductive step then:
      \begin{enumerate}[label=\alph*., ref=(\alph*)]
      \item
        \label{assu:supp:ind:circular:p}
        If \(\psi\) having value \(v'\) is a \prequ{} of \(\delta\) then:
        \begin{itemize}
        \item
          The reasoning of \ref{assu:supp:ind:indicate} does not involve appeal to a proposition-value pair that is only obtained from conclusion of \(\delta\).
        \end{itemize}
      \item
        \label{assu:supp:ind:circular:cp}
        If \(\psi\) having value \(v'\) is  a \cprequ{} of \(\delta\) then:
        \begin{itemize}
        \item
          The reasoning of \ref{assu:supp:ind:indicate} does not involve appeal to a proposition-value pair that is only obtained from the step of reasoning and \crequ{1} which leads to \(\psi\) having value \(v'\) being a \prequ{}.
        \end{itemize}
      \end{enumerate}
    \end{enumerate}
    \vspace{-\baselineskip}
  \end{restatable}
\end{note}

\begin{note}
  \autoref{assu:supp:independence} consists of two clauses, where the first clause requires the existence of some reasoning, and the second clause places constraints on that reasoning if a certain condition holds.

  So, to capture the intuition behind \autoref{assu:supp:independence}, let us focus on \ref{assu:supp:ind:indicate} which requires past or present reasoning about any \requ{} of the instance of reasoning.
  We will then turn to \ref{assu:supp:ind:circular} to clarify why restrictions are placed on the reasoning about certain \requ{1} with the core intuition in hand.
\end{note}

\paragraph{Clause 1}

\begin{note}
  If \(\psi\) having value \(v'\) is a \requ{} of some step \(\delta\), then three important things follow:
  \begin{itemize}
  \item
    \(\psi\) not having value \(v'\) is an \ep{} for the agent, \emph{and}
  \item
    If \(\psi\) does not have value \(v'\), the step would involve some premise or conclusion having some other value.
  \item
    \(\psi\) having value \(v'\) is important not only for the agent making the step, but also for the instance of reasoning as a whole.
  \end{itemize}

  In other words, if \(\psi\) were not to have value \(v'\), then not only the step of reasoning, but the instance of reasoning as whole would be undercut as the agent would appeal to proposition-value pair which is not the case.

  Hence, if the agent has not previously engaged in reasoning that indicates that \(\psi\) has value \(v'\), or does not engage in reasoning that indicates that \(\psi\) has value \(v'\) as part of the present reasoning, then the reasoning that the agent has performed does not indicate that \(\phi\) has value \(v\) regardless of whether \dots

  \emph{because} if it is the case that \(\psi\) does not have value \(v'\), then

  and, nothing that indicates that this is not the case.
\end{note}

\begin{note}
  \begin{enumerate}
  \item If clock is working, then it was repaired yesterday.
  \end{enumerate}
  So, without this being the case, time from a broken clock.

  \begin{enumerate}
  \item If may call a meeting, then bad time.
  \end{enumerate}

  If bank is disambiguated differently, then no an answer to the question.

  If conditional doesn't hold, then \mistaken{}.

  If textbook answer is different, then something.
\end{note}

\begin{note}
  Due to being a \requ{}, important for the step of reasoning that proposition-value pair is such.


  The agent's reasoning does not indicate that the relevant conclusion holds regardless of whether premises and conclusion have respective values, because there is some proposition-value pair that the agent's reasoning require to being the case.

  It is the case that if proposition-value pair holds, then indicates.
  However, does not indicate \emph{regardless}.
\end{note}

\subparagraph{???}

\begin{note}
  {
    \color{red}
    (Where 2 is motivated by possibility that it doesn't hold. If only with consequence, then only when does hold.)
  }

    Reasoning about how the considerations the agent has cited indicate \(\phi\) even when \requ{} does not have value.

    So, any \requ{}, compatible with that being false.
    Recognise the possibility, claiming support is limited in that way.
    However, that possibility alone does not prevent appeal to the premise.

    There is a subtly.
    Agent may grant that the conclusion would not be the case.
    That's not the worry.
    Instead, the worry is whether appealing to the premise is fine.

    Perhaps Moore's proof is useful here.
    It is true that no external world is possibility.
    However, this doesn't prevent me from appealing to perception of two hands.
    Would need that this is only okay if I actually have two hands.
    But, this is far too much.

  \autoref{assu:supp:independence} expresses a necessary condition from \ideaCS{}.
  The basic idea is that a \requ{} has the possibility to highlight a fault in agent's reasoning.
  So, if an agent has failed to reason about whether a \requ{} has {\color{red} value}, then the agent's reasoning fails to indicate that \(\phi\) has value \(v\) regardless of whether {\color{red} some problem with premise or conclusion}.
  {\color{red} In particular, with the \requ{} not having value.}

  \autoref{assu:supp:independence} does not include any particular constraint on what the reasoning about a \requ{} amounts to {\color{red} other than \requ{} not having value.}
\end{note}

\subparagraph{Reasoning}

\begin{note}
  Before turning to the details of \autoref{assu:supp:independence}, an immediate clarification is in order:
  \ref{assu:supp:ind:indicate} requires that the agent does or has reasoned about any \requ{1} of the relevant instance of reasoning.
  However, we do not require the relevant reasoning to be explicitly recognised by the agent.
  Holding that some instance of reasoning is an instance of claiming support is the attribution of a property to some activity of the agent.
  However, it is a property of interest from our perspective as observers of the agent, and from this is does not follow that the agent needs to reflexively recognise that their reasoning is an instance of claiming support.

  Of course, an agent may recognise that their reasoning would not satisfy \autoref{assu:supp:independence} and explicitly reason about a \requ{1} in light of such a recognition.
  However, the observation that an agent may adjust their reasoning in order to ensure that they do not fail to satisfy \autoref{assu:supp:independence} does not, in turn, require that it must always be possible for the agent to recognise that their reasoning does \autoref{assu:supp:independence}, nor that the recognition of such satisfaction is itself important for claiming support.
\end{note}

\begin{note}
  First, in case of \requ{} being the conclusion of the subinstance of reasoning, there is nothing to say that the premises appealed to aren't doing the work.
  Remember X, so X.
  Well, not X.
  However, saw X.
  That's all that's needed to indicate X.

  There is no suggestion in the definition of a \requ{} or in the assumption that the premises appealed to are not sufficient.
  Likewise for other \requ{}.

  Remember a dalmation with a red collar?
  Yes.
  Oh, then you remember spotty(???)
  Yes.
  Same X here, but now it's not the conclusion of the instance of claiming support.

  There is nothing about the definition or assumption that suggests there is any problem here.
\end{note}

\paragraph{The second clause}


\begin{note}
  {
    \color{red}
    I need separate clauses here because, there's nothing clearly wrong with taking some result of the step of inference that doesn't lead to the feedback.
    At issue is not the step itself, but the step paired with some result of the step.
  }

  {
    \color{red}
    Key observation here is that in the deductive case, the premises alone are sufficient to get the conclusion, so they will also be good to get anything else which deductively follows.
    The agent doesn't introduce anything new.
    So, the non-deductive qualification makes sense.
  }


  Intuitively\dots refined \ideaCS{} with the idea of a \requ{}.
  And, necessary conditions given certain types of \requ{}.
\end{note}

\begin{note}
  One may question whether the agent needs to reason about \crequ{1}.
  For, while {\color{red} XXX} does not constrain how the agent reasons, still some reasoning.

  The observation is simple.
  If no reasoning, then possible that agent would retract step in recognition of conclusion.
  This is a point observed by \citeauthor{Harman:1973ww}.

  In other words, reasoning about \crequ{} ensures that the step of reasoning, and hence the instance of reasoning is `stable' with respect to the agent's present epistemic state.
\end{note}

\hozline

\begin{note}
  Still, circularity in general is a bit of an issue.

  Rules out some cases of reasoning.
  Claiming support for memory, some like circular reasoning.

  One limitation is that memory is not clearly a \requ{}.
  In appeal to memory, supposing that memory is not correct does not establish that moving from something false or to something false.
  My memory is quite bad, but I remember X.
  Still leaves specific memory, but that's a different issue.
\end{note}


\paragraph{Other observations}

\begin{note}
  Various ways in which assumption is limited.
\end{note}

\begin{note}
  Claiming support?
\end{note}

\begin{note}
  Third, basic appeal.

  Pryor.\nolinebreak
  \footnote{
    Ried is a different issue.
  }
  \autoref{assu:supp:independence} does not rule this out.
  For, no limitations placed on what reasoning for \requ{} is.
\end{note}

\begin{note}
  You don't need to agree with these.
  Epistemic possibility and memory does introduce a \requ{} for general reliability, so there is an issue for claiming support without reasoning which indicates.
  The point is that for present purposes we do not need to, nor shall we, assume such a strong variant of \autoref{assu:supp:independence}.
\end{note}

\begin{note}[Strengthen?]
  Of course, one may wish to strengthen this necessary condition by specifying additional constraints on what is involved in reasoning about a \requ{}.

  For example, suppose \(\psi\) having value \(v'\) is a \requ{} of reasoning that concludes with \(\phi\) has value \(v\).
  If the reasoning that concludes with \(\phi\) having \(v\) is an instance of claiming support, then it seems reasoning that amounts to it being nice if \(\psi\) has value \(v'\), or that it is not obvious that \(\psi\) does not have value \(v'\) seems inadequate.

  In short, while \autoref{assu:supp:independence} may be a necessary condition, it does not seem the strongest necessary condition of its kind.

  Still, for our purposes \autoref{assu:supp:independence} will be sufficient.
  Either the lack of reasoning, or that it is not possible to do the required reasoning --- though not necessarily not possible in general (\nI{} for details).
\end{note}

\begin{note}
  \eiS{} does not deny that things may need to be a certain way for an agent to claim, or to be in a position to, claim support.
  It may be the case that no agent would be in a position to claim support that the speed of light is constant if the speed of light were not constant.
  Still, in claiming support an agent must expect that possible defeaters do not obtain, e.g.\ that the laws of nature are constant, and that no mistakes have been made when observing relevant phenomena.
\end{note}

\hozline

\paragraph{Meta-proposition-value pairs}

\subparagraph*{Meta-proposition-value pairs: Steps of reasoning}

\begin{note}
  Meta-proposition-value pairs are a distinct concern.
  For, though it may be that reasoning about some proposition-value pair applies to any meta-proposition-value pair concerning that source proposition-value pair, it is not clear that any combination of reasoning about the premises or conclusions of a step of reasoning will apply to the step of reasoning itself.
\end{note}

\begin{note}
  The important observation with respect to \requ{1} is that they are defined in terms of \result{1}, and hence for a meta-proposition-value pairs about a step of reasoning to be of concern, it may not reduce to issues concerning the \requ{1} of the step.

  For example, consider the following proposition-value pair.
  \begin{itemize}
  \item By making this step of reasoning I am not appealing to a proposition-value pair that is not the case.
  \end{itemize}
  As the proposition-value pair reduces to a question of whether the agent is appealing to proposition-value pairs that the agent is appealing to when making the step of reasoning, the combined reasoning about those proposition-value pairs may apply to the meta-proposition-value pair.
\end{note}

\begin{note}
  Hence, for the meta-proposition-value pair to be of interest as an objection, it must concern the step itself.
  For example.
  \begin{itemize}
  \item This step of reasoning is a \(\{\text{good},\text{suitable},\text{rational},\dots\}\) step of reasoning.
  \end{itemize}
  The relevant adjective is unimportant.
  Rather, the idea is that whatever the adjective is, it does not reduce to the proposition-value pairs appealed to in making the step.

  However, any meta-proposition-value pair of this kind will fail to be a \requ{1} of the step of reasoning \emph{because} it does reduce to issues concerning the \requ{1} of the step.
\end{note}

\begin{note}
  \color{red}
  Leaves open the possibility that you may hold the agent needs to have some reasoning to indicate the relevant proposition.
  However, this does not follow from the assumptions made.
  Therefore, we may take a neutral stance on such propositions.
\end{note}

\subparagraph*{Summary of meta-proposition-value pairs}

\begin{note}
  To summarise, if a meta-proposition-value pair follows from a \requ{0} of reasoning, then while it may be the case that the meta-proposition-value is also a \requ{0} of reasoning, the reasoning about the source-\requ{0} may be taken to apply to the meta-proposition-value.
  Hence, \autoref{assu:supp:independence} does not require that an agent explicitly reasons about such meta-proposition-value pairs.

  And, conversely, if a meta-proposition-value pair does not follow from a \requ{0} of reasoning, then \autoref{assu:supp:independence} does not require that an agent reasons about such meta-proposition-value pairs, neither explicitly nor implicitly.
\end{note}


\subsubsection{Proposition}
\label{sec:proposition}

\begin{note}
  Following is an immediate consequence of \ref{assu:supp:nfactive} and \autoref{assu:supp:independence}:

  \begin{restatable}[Reason about recognised \requ{1}]{proposition}{propRecogniseDefeaters}
    \label{prop:CS-only-if-reason-recognised-defeaters}
    \requ{}, and:
    For any such recognised \requ{} at time of reasoning and does not reason, not an instance of claiming support.
  \end{restatable}
\end{note}

  \begin{note}
  \begin{itemize}
  \item If result of reasoning to \(\phi\) having value \(v\) is such that agent considers that reasoning fails if \(\phi\) does not have value \(v\), then reasoning is not an instance of claiming support.
  \item Not possible that instance of reasoning to \(\phi\) having value \(v\) is claimed support only if \(\phi\) has value \(v\).
  \item Claimed support for \(\phi\) having value \(v\) never requires that \(\phi\) has value \(v\).
  \end{itemize}
\end{note}

\begin{note}
  \color{red}
    \begin{itemize}
    \item Always possibility of \mom{} from \nfcs{}.
    \item This means that the agent has no guarantee that \(\phi\) has value \(v\) --- or better put the agent considers it to be an (epistemic) possibility that their claimed support is \mom{}.
    \item However, if the agent requires that \(\phi\) is the case, then there is no possibility of the claimed support being \emph{mistaken}.
    \item Well, no reasoning against being mistaken with respect to claimed support for this \requ{}.
  \end{itemize}

  Again, it does not seem impossible for an agent to adopt an attitude that recognises the possibility but assumes regardless.
\end{note}

\begin{note}
  Important consequence of \autoref{sec:basic-assumptions:props-and-vals}, block:

  S did not reason about possibility that Q is false.
  If Q is false, then P must also be false.
  Hence, P may be false.
  S did not reason about the possibility that P is false.

  Examples:
  \begin{itemize}
  \item Possibility that the trains are on strike.
  \item No Indication of strike, so do not consider live possibility.
  \item Read newspaper.
  \item Newspaper reported strike.
  \item Consequence of possibility is that the newspaper misreported.
  \item Reasoning does not extend to newspaper.
  \end{itemize}

  \begin{itemize}
  \item Out of milk.
  \item Then come to hold that there is milk in the fridge.
  \item Hallucinating.
  \item Does not extend.
  \end{itemize}

  \begin{itemize}
  \item Turing machine reduction.
  \item If possible then also possible.
  \item So, give up.
  \end{itemize}
\end{note}


\subsection{Persistence}
\label{sec:persistence}

\begin{note}
  \autoref{assu:supp:nfactive} and \autoref{assu:supp:independence} are about the activity of claiming support.
  Key thing is a \requ{}.

  However, nothing about the role of claimed support in reasoning.
  Seems plausible that appeal to claimed support.
  And, many cases will concern appeal to claimed support.
  Hence, make this explicit.
\end{note}


%%% Local Variables:
%%% mode: latex
%%% TeX-master: "master"
%%% End:
