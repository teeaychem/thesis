\chapter{Temp}

  % \begin{restatable}[\zSN{2} --- \zS{}]{idea}{ideazs}
  %   \label{idea:zs}
  %   An agent \vAgent{} has \emph{\zSAb{-}} for a proposition-value pair \(\pv{\phi}{v}\) with respect to some pool of premises \(\Phi\) when concluding \(\pv{\phi}{v}\) from \(\Phi\) just in case:
  %   \begin{itemize}
  %   \item
  %     When concluding \(\pv{\phi}{v}\) from \(\Phi\):
  %     \begin{enumerate}[label=\arabic*., ref=\named{\zSAb{}:\arabic*}]
  %     \item
  %       For any proposition-value-premises pairing \(\pvp{\psi}{v'}{\Psi}\) which is a \requ{} of concluding \(\pv{\phi}{v}\) from \(\Phi\):
  %       \begin{enumerate}[label=\alph*., ref=\named{\zSAb{-}:1.\alph*}]
  %       \item
  %         \vAgent{} has `settled' that they would conclude \(\pv{\psi}{v'}\) from \(\Psi\).
  %       \item
  %         \vAgent{} simultaneously `settles' that they would conclude \(\pv{\psi}{v'}\) from \(\Psi\) when concluding \(\pv{\phi}{v}\) from \(\Phi\).
  %       \end{enumerate}
  %     \end{enumerate}
  %   \end{itemize}
  %   \vspace{-\baselineskip}
  % \end{restatable}

\subsection{Deviance}
\label{sec:deviance}

\begin{note}
  Here, causal deviance.
\end{note}

\begin{note}
  Problem is, there's no way to guarantee a link between positive answer to \qzS{} and the agent concluding or not refraining from concluding.
\end{note}

\begin{note}
  Argument relies on tying content to explanation.

  In this respect, there is room for an objection.
  Deviant causal chains.
  Point here is that there are cases where these come apart.

  This isn't only a problem for causal theories of reasoning.
  The point is, some instantiation, and so long as act may be caused by something else, then possibly caused by the instantiation.

  So, possible here.

  Well, hold on.
  What is need is the relevance of the content.
  For this objection to work, need to take a theoretical perspective.
  See, in Davidson's case, the idea is fusing these two things together.
  We answer two different questions with a common thing viewed in two ways.

  Still, I think the objection can be pressed!
  Only \emph{really} an explanation is no deviance.
  To the same extent that potential event matters, it matters to the agent that there is no deviance.

  {
    \color{red}
    Resolution is, if deviance, then no agency.
  }

  I think this makes sense, or at least makes enough sense.
  Answers to `why', on this understanding, are tentative.

  Or, rest on presupposition that agent performed the action.

  So, contingent on showing there is no causal deviance.

  This is different to error.
  With error, thing appealed to isn't the case, but appeal still did work.
  Here, it doesn't matter whether or not the case, no work is done.

  In contrast to more typical instances of the problem, don't need to rule out deviant causal chains.
  Instead, just need one instance to fail to hold.
  One instance of non-deviousness.

  Still a problem for a compatible account which avoids.
  For, here, there can't be any direct link from perspective to reason.

  For example, \citeauthor{Hieronymi:2011aa}

    \begin{quote}
      [W]e explain an event that is an action done for reasons by appealing to the fact that the agent took certain considerations to settle the question of whether to act in some way, therein intended so to act, and successfully executed that intention in action.
    [\emph{T}]\emph{his} complex fact, [\dots] is the reason that rationalizes the action---that explains the action by giving the agent's reason for acting.%
    \mbox{ }\hfill\mbox{(\citeyear[431]{Hieronymi:2011aa})}
  \end{quote}

  So, here, considerations which settle question, and in so settling question.
  Link between settling the question and acting.

  Following \citeauthor{Hieronymi:2011aa}, no room for deviance.
  Too tight.

  In other words, so long as this fact holds, there is no distinction between settling the question and acting.
  Therefore, no deviance.

  Compatible, I think.
  Question is whether in resolving \qzS{} is sufficiently tied to resolving the question \citeauthor{Hieronymi:2011aa} identifies.
  And, plausibly is.
  This is what the motivation for \qzS{} did.

  Trouble is, for our purposes, need at least sufficient conditions for when this complex fact obtains.
  And, no account of this.

  \citeauthor{Hieronymi:2011aa} notes the gaps.

  Some tension.
  These considerations aren't premises.
\end{note}

\begin{note}
  So, the other option is to embrace deviant causal chains.
  Have the content, but this doesn't work in the way the agent thinks it does.

  Example from Davidson.

  The trouble here is that the content and resulting action match.
  So, things make sense from the \agpe{}.

  Deviant, but maybe not so deviant here.

  Systematic deviance, where content is separated from role of mental state.

  But, I see no motivation for this.

  Solution to causal chains doesn't get round this, because the result is a restricted account.
  So, there's no guaranteed trade-off here.
  Trouble is, it seems hard to see a case where this wouldn't be the case.
\end{note}

\paragraph{When an agent has \zS{}}

\begin{note}
  Now, two basic propositions follow.

  \begin{proposition}[When a agent has \zS{}]
    Agent and proposition-value-premises pairing.

    \begin{itemize}
    \item
      Agent \emph{has} \zS{} for \(\pv{\phi}{v}\) after concluding \(\pv{\phi}{v}\) from \(\Phi\).
    \end{itemize}

    \emph{if and only if}

    \begin{itemize}
    \item
      Either:
      \begin{enumerate}[label=(\alph*), ref=\alph*]
      \item
        No \requ{1} of concluding \(\pv{\phi}{v}\) from \(\Phi\) from the \agpe{}.
      \item
        For any \requ{} \(\pvp{\psi}{v'}{\Psi}\) of concluding \(\pv{\phi}{v}\) from \(\Phi\), the agent would not fail to conclude \(\pv{\psi}{v'}\) from \(\Psi\), from the \agpe{}.
      \end{enumerate}
    \end{itemize}
  \end{proposition}

  Second, when an agent does not have \zS{}.

  \begin{proposition}[When a agent does not have \zS{}]
    Agent and proposition-value-premises pairing.
    \begin{itemize}
    \item
      Agent \emph{does not} have \zS{} for \(\pv{\phi}{v}\) after concluding \(\pv{\phi}{v}\) from \(\Phi\).
    \end{itemize}

    \emph{if and only if}

    \begin{itemize}
    \item
      There is some proposition-value-premises pairing \(\pvp{\psi}{v'}{\Psi}\) such that, from the \agpe{}, the agent may fail to conclude \(\pv{\psi}{v'}\) from \(\Psi\).
    \end{itemize}
  \end{proposition}
\end{note}

\paragraph{Lazy argument}

\begin{note}
  Three premises:
  \begin{enumerate}
  \item
    \label{lazy:evidence-constraint}
    If evidence for \(\pv{\phi}{v}\), then no evidence for conflicting \(\pv{\psi}{v'}\).
  \item
    \label{lazy:evidence}
    Support only if evidence.
  \item
    \label{lazy:reason}
    Support to \(\pv{\phi}{v}\) from \(\Phi\) only if possible to conclude from \(\pv{\phi}{v}\) to \(\Phi\) from present perspective.
  \end{enumerate}

  Second, reasons are evidence.
  General expression found in \hyperlink{cite.Hume:1748tp}{Hume}'s \hyperlink{cite.Hume:1748tp}{Enquiry}:
  \begin{quote}
    A wise man, \dots, proportions his belief to the evidence.
    In such conclusions as are founded on an infallible experience, he expects the event with the last degree of assurance, and regards his past experience as a full proof of the future existence of that event.%
    \mbox{ }\hfill\mbox{(\hyperlink{cite.Hume:1748tp}{E.10.1.4})}
  \end{quote}

  Though, stronger formulation termed `evidentialism'~\citeauthor{Feldman:1985wx}:

  \begin{quote}
    \begin{enumerate}
    \item[EJ] Doxastic attitude \emph{D} toward proposition \emph{p} is epistemically justified for \emph{S} at \emph{t} if and only if having \emph{D} toward \emph{p} fits the evidence \emph{S} has at \emph{r}.%
      \mbox{ }\hfill\mbox{(\citeyear[15]{Feldman:1985wx})}
    \end{enumerate}
  \end{quote}

  Read, reason only if evidence.

  Evidence is consistent.
  Take as given.

  Here we cite~\citeauthor{Achinstein:2001ub} (\citeauthor{Achinstein:2001ub}) who channels~\textcite{Carnap:1962ue}:

  \begin{quote}
    \emph{e} is evidence that \emph{h} if and only if \(p(h \mid e) > p(h)\).%
    \mbox{ }\hfill\mbox{(\citeauthor[45]{Achinstein:2001ub})}
  \end{quote}

  Then, impossible for \(e\) to be evidence for both \(h\) and \(q\) such that \(h\) and \(q\) cannot jointly occur.
  \footnote{
    Suppose \(h\) and \(q\) are incompatible.
    \(h \cap q = \emptyset\).

    \(p(h \mid e) + p(\lnot h \mid e) = 1\).
    \(p(h \mid e) > p(h)\).
    So, \(p(h) = x\) and \(p(h \mid e) = x + y\), where \(x, y > 0\).
    \(p(\lnot h \mid e) = (1 - x - y)\).
    \(p(\lnot h) = (1 - x)\).
    \((1 - x - y) \not> (1 - x)\).

    Now, \(p(q \mid e) \leq p(\lnot h \mid e)\).
    But, expand.
    \(h \cap q = \emptyset\).
    \(q \subseteq \lnot h\).
    \((q \cap e) \subseteq (\lnot h \subseteq e)\).
    \(p(q \land e) \leq p(\lnot h \land e)\).
    \(p(q \mid e) \leq p(\lnot h \mid e)\).
  }
\end{note}

\paragraph{Pryor}

\begin{note}
  \citeauthor{Pryor:2004ws}'s argument that type 4 over-generates is somewhat interesting.
  Details are in the following footnote.\footnote{
  Compatible with \citeauthor{Pryor:2004ws}'s objection to type 4 dependence.

  % \begin{illustration}
    % \mbox{}
    % \vspace{-\baselineskip}
    \begin{quote}
      Suppose you're watching a cat stalk a mouse. Your visual experiences justify you in believing:

      \begin{enumerate}[label=(\arabic*), ref=(\arabic*)]
        \setcounter{enumi}{10}
      \item
        \label{illu:Pryor:cat:1}
        The cat sees the mouse.
      \end{enumerate}

      You reason:

      \begin{enumerate}[label=(\arabic*), ref=(\arabic*), resume]
      \item
        \label{illu:Pryor:cat:2}
        If the cat sees the mouse, then there are some cases of seeing.
      \item
        \label{illu:Pryor:cat:3}
        So there are some cases of seeing.\nolinebreak
        \mbox{}\hfill\mbox{(\citeyear[361]{Pryor:2004ws})}
      \end{enumerate}
    \end{quote}
  % \end{illustration}

  Setting aside whether this is fine.

  Following \citeauthor{Pryor:2004ws}:

  Bad, given proposal, as if no cases of seeing, then the cat is not seeing. (\citeyear[361]{Pryor:2004ws})

  \citeauthor{Pryor:2004ws}'s position is as follows:

  \begin{quote}
    I don't think you need antecedent justification to believe \ref{illu:Pryor:cat:3}, before your experiences can give you justification to believe \ref{illu:Pryor:cat:1}.
    I also think it's plausible that your perceptual justification to believe \ref{illu:Pryor:cat:1} contributes to the credibility of \ref{illu:Pryor:cat:3}.\nolinebreak
    \mbox{}\hfill\mbox{(\citeyear[361]{Pryor:2004ws})}
  \end{quote}

  This fine when seen from the perspective of the conditional being a \requ{}.
  }
\end{note}


\paragraph{Knowing whether and knowing how}

\begin{note}
  Read the above examples in terms of knowing whether.

  More-or-less direct link between knowing whether and knowing how:

  Know whether \(?\{a,b,c,\dots\}\) know how to answer whether it is the case that \(?\{a,b,c,\dots\}\)
\end{note}

\begin{note}
  Ideas regarding \citeauthor{Ryle:1946tu}'s distinction between knowing \emph{how} and knowing \emph{that}~(Cf.~\citeyear{Ryle:1946tu}).

  Now, I confess my understanding of \citeauthor{Ryle:1946tu}'s distinction is limited --- I have not taken whatever opportunities I have had to read through \citeauthor{Ryle:1946tu}'s work.%
  \footnote{
    Though, I understand enough from passing commentary to note that the idea \emph{I} am perusing here does not, strictly, require that knowledge how and knowledge that are distinct kinds of knowledge.

    Intellectualist and anti-intellectualist views.

    For, granting that knowledge how reduces to knowledge that, it will remain the case that there is an event\dots
    (See~\textcite{Pavese:2022up} for more!)
  }

  Following analogy from~\textcite{Ryle:2009us}:

  \begin{quote}
    Knowing `\emph{if p, then q}' is, \dots rather like being in possession of a railway ticket.
    It is having a licence or warrant to make a journey from London to Oxford.
    (Knowing a variable hypothetical or `law' is like having a season ticket.)
    As a person can have a ticket without actually travelling with it and without ever being in London or getting to Oxford, so a person can have an inference warrant without actually making any inferences and even without ever acquiring the premisses from which to make them.%
    \mbox{ }\hfill\mbox{(\citeyear[250]{Ryle:2009us})}
  \end{quote}

  Continuing~\citeauthor{Ryle:2009us}'s analogy, in the case of \fc{1}:
  What matters is that the agent is currently in possession of the (season) ticket.
\end{note}

\begin{note}
  The relationship between \fc{1} and knowing whether is interesting.
  For, as stated, \fc{1}, just about potential event.
  Plausible paraphrase.

  Whether this goes in both directions.

  So, from knowing whether, generally understood, there's got to be an event.
  However, it is not clear to me that there is a `potential' constraint on the event.

  \cite{Bengson:2011th}.
  \begin{quote}
    \emph{Pi}.
    Louis, a competent mathematician, knows how to find the n\(^{\text{th}}\) numeral, for any numeral \(n\), in the decimal expansion of \(\pi\).
    He knows the algorithm and knows how to apply it in a given case.
    However, because of principled computational limitations, Louis (like all ordinary human beings) is unable to find the \(10^{46}\) numeral in the decimal expansion of \(\pi\).%
    \mbox{ }\hfill\mbox{(\citeyear[170]{Bengson:2011th})}
  \end{quote}

  Conversely, not clear \fc{0} leads to knowing whether.

  Issues here.

  First, factive constraint on knowledge.
  And, concluding need not be factive.

  Though, whether this matters in practice is not clear.
  For, from \agpe{}.

  Second, knowledge is not the right attitude from concluding.
  For example, focus is on deductive cases, but abductive conclusion.
  Or, some doubts about premises.
\end{note}


\section{Landman possible continuations}
\label{sec:landm-poss-cont}

\subparagraph{Possible Continuations}

\begin{note}
  Additional function.

  \begin{algorithm}[H]
    \SetAlgoLined
    \DontPrintSemicolon
    \Input{\(\langle \langle e,w \rangle_{i-1}, \langle g,u \rangle_{i} \rangle\)}
    \KwResult{Possible continuations of \(\langle g,u \rangle_{i} \rangle\)}
    \Begin{
      \(\Diamond\text{Continuations} \longleftarrow \emptyset\)\;
      \(\text{CloseWorlds} \longleftarrow \{u' \mid u' \text{ is among the closest world to } u\}\)\;
      \For{\(u' \in \text{CloseWorlds}\)}
      {
        \For{\(g'\) in \(u'\)}
        {
          \If{\(g' \text{ is a stage of } h \text{ in } u'\)}
          {
           \(\SetPC{} \longleftarrow \AlgAC{})(\langle \langle g,u \rangle_{i}, \langle g',u' \rangle_{i} \rangle)\)
          }
        }
      }
      \Return{\(\{\langle \langle g,u \rangle_{i+1}, \langle h,u \rangle_{i+1} \rangle \rangle \mid \langle h,u \rangle_{i+1} \rangle \in \Diamond\text{Continuations}\}\)}
    }
    \caption{\AlgPC{}\label{PrAl:g-p-c}}
  \end{algorithm}

  Important.
  \citeauthor{Landman:1992wh}'s original formulation, where it actually stops.
  However, here, we include possible stops.
  This covers darts case.

  Now, what matters is when evaluated.
  If moment of starting the throw, then progressive is false.
  However, if moment of release, then progressive is true.

  No distinction if just focus on how things develop.
\end{note}

\subparagraph{Continuations}

\begin{note}
  \begin{algorithm}[H]
    \SetAlgoLined
    \DontPrintSemicolon
    \Input{\(\langle \langle f,v \rangle_{i-1}, \langle g,u \rangle_{i}  \rangle\)}
    \KwResult{Continuations of \(\langle g,u \rangle_{i}\)}
    \Begin{
      \(\text{@-Continuations }\longleftarrow \AlgAC{}(\langle \langle f,v \rangle_{i - 1}, \langle g,u \rangle_{i}\rangle)\)\;
      \For{everything in actual}
      {
        \(\text{\(\Diamond\)-Continuations }\longleftarrow \AlgPC{}(\langle \langle f,v \rangle_{i - 1}, \langle g,u \rangle_{i}\rangle)\)\;
        \(\text{Continuations} \longleftarrow \SetAC{} \cup \SetPC{}\)\;
      }
      \Return{Continuations}
    }
    \caption{\AlgGetCs{}\label{PrAl:g-c}}
  \end{algorithm}
\end{note}

\section{Carrol and regress}
\label{sec:carrol-regress}

\begin{note}
  Similar to \citeauthor{Carroll:1895uj}.
  \begin{quote}
    Logic would take you by the throat, and \emph{force} you to do it!%
    \mbox{ }\hfill\mbox{(\citeyear[280]{Carroll:1895uj})}
  \end{quote}
  Looking at something static.
  Achilles fails to convey this to the Tortoise, arguably through some fault of Achilles' own.

  In parallel, we could stack up additional passives in the same way, but there's little interest in doing so.
  The point is the base \requ{} is not satisfied.
\end{note}

\begin{note}
  So, with \citeauthor{Carroll:1895uj}, we get a rule of inference, great.

  \citeauthor{Wieland:2013vf} characterises the general understanding of \textcite{Carroll:1895uj} in terms of two lessons:
  \begin{quote}
    [T]he negative lesson is that if you add ever more premises to an argument \dots, then you will never demonstrate that its conclusion follows logically.%
    \mbox{ }\hfill\mbox{(\citeyear[984]{Wieland:2013vf})}
  \end{quote}

  Parallel, static answers, still option for concluding otherwise.

  \begin{quote}
    [T]he positive lesson is that rules of inference, rather than premises of the form `if premises such and such are true, then the conclusion is true', will do the job.%
    \mbox{ }\hfill\mbox{(\citeyear[984]{Wieland:2013vf})}
  \end{quote}

  Parallel, the dynamic status of a rule.
\end{note}

\begin{note}
  No regress.

  Following \citeauthor{Wieland:2013vf}:

  \begin{quote}
    \begin{itemize}[noitemsep]
    \item[IR]
      For any item x of a certain type, S \(\varphi\)-s x only if
      \begin{enumerate}[label=(\roman*),noitemsep]
      \item
        there is a new item y of that same type, and
      \item
        S \(\varphi\)-s y.%
        \mbox{ }\hfill\mbox{(\citeyear[996]{Wieland:2013vf})}
      \end{enumerate}
    \end{itemize}
  \end{quote}
\end{note}

\section{Wright}
\label{sec:wright}

\begin{note}
  \color{red}
  Some of the \citeauthor{Wright:2011wn} cases are interesting.
  Especially the twin cases.
  In fact, especially this idea that situations are identical.
  For, one way of understanding this is that the agent makes a choice between two disjuncts, and it is possible for the agent to make the other choice, and then come to a different conclusion.
\end{note}

\paragraph[Actuality entailments]{Actuality entailments \hfill (Optional)}


\begin{note}
  
  If actuality entailment, then possibility of reducing to modal.

  However, clear failure here.
  For, \support{}, but same modal applies to concluding.

  Here, intuitively a significant issue if actuality entailment held.%
  \footnote{
    Inclined to allow concluding without witnessing, but for use, conclude only if witnessed.
  }
\end{note}


\begin{note}
  Clear example of this from Igal Kvart, via~\textcite{Landman:1992wh}:

  \begin{quote}
    Look at example~\ref{Landman:FRomans}:
    \begin{enumerate}[label=(\arabic*), ref=(\arabic*)]
      \setcounter{enumi}{19}
    \item
      \label{Landman:FRomans}
      Mary was wiping out the Roman army.
    \end{enumerate}

    The situation is that Mary, a person of moderate physical capacities, is battling the Roman army.
    She manages to kill a couple of soldiers before she gets killed.
    \ref{Landman:FRomans} is clearly false in this situation.\newline
    \mbox{ }\hfill\mbox{(\citeyear[18]{Landman:1992wh})}
  \end{quote}

  Mary may have survived, and more Roman soldiers may have perished by Mary's hand.
  However, relation of Mary to the Roman army means no sense in which event is Mary wiping out the Roman army.
\end{note}

\section{\citeauthor{Boylan:2020aa}}

\begin{note}
  To get clear on how \citeauthor{Boylan:2020aa} understands control intuition, develop the formal core of \citeauthor{Boylan:2020aa}'s theory.
  Difficulty will not depend on particular details, but general understanding of control given by details.

  Need a handful of things.
  \begin{itemize}
  \item
    Unsettled world
  \item
    Selection function
  \item
    Semantics for will.
  \end{itemize}

  First, unsettled world.
  `while the past is settled, the future is open' (\citeyear[1]{Boylan:2020aa})

  \begin{quote}
    \emph{Unsettled World}. \(\mathcal{I}_{c} = \{w \mid w\text{ is identical to }w_{c}\text{ up until }t_{c}\}\)%
    \mbox{ }\hfill\mbox{(\citeyear[11]{Boylan:2020aa})}
  \end{quote}
  Relative to some context \(c\), which includes world \(w_{c}\) and a time \(t_{c}\), set of worlds which are identical to to \(w_{c}\) up until \(t_{c}\).

  Note, unique with respect to context.

  So, with when considering the actual world at some point in time, corresponding unsettled world is just set containing all worlds identical to the past.

  Caution, an unsettled world is a set of worlds.

  Selection function:

  \begin{quote}
    \(s(\mathcal{I}, \mathbf{A})\) picks out the closest (possibly unsettled) world to \(\mathcal{I}\) which settles that \(\mathbf{A}\) is true.%
    \mbox{ }\hfill\mbox{(\citeyear[11]{Boylan:2020aa})}
  \end{quote}
  So, selection function takes unsettled world and action as input and returns set containing a world or collection of worlds such that action is settled to be true.

  Key thing here is \(f(w,t)\).
  Intuitively, restrict which worlds are part of the unsettled world.
  Strictly, closest (unsettled) world.
  However, closeness is not of interest.

  \citeauthor{Boylan:2020aa} does not define settled.
  However, \(\mathbf{A}\) is true at every world in set.%
  \footnote{
    `If \(\phi\) is true at \(s(\mathcal{I}, f(w,t))\) (i.e. true throughout \(\mathcal{I}\)),\dots'. \mbox{(\citeyear[12]{Boylan:2020aa})}
  }

  \(\mathcal{W}\) `will'.

  \begin{quote}
    \begin{enumerate}
      \setcounter{enumi}{33}
    \item
      \begin{enumerate}
      \item
        \(\sem[w,t,f,\mathcal{I}]{\mathcal{W}\phi}\) is determinate only if either
        \begin{enumerate}
        \item
          \(s(\mathcal{I}, f(w,t)) \subseteq \sem[t,f,\mathcal{I}]{\phi}\) or
        \item
          \(s(\mathcal{I}, f(w,t)) \subseteq \sem[t,f,\mathcal{I}]{\lnot\phi}\)
        \end{enumerate}
      \item
        If determinate \(\sem[w,t,f,\mathcal{I}]{\mathcal{W}\phi} = 1\) iff \(s(\mathcal{I}, f(w,t)) \subseteq \sem[t,f,\mathcal{I}]{\phi}\)
      \end{enumerate}
    \end{enumerate}
  \end{quote}

  Here, \(w\) world, \(t\) time, \(f\) a function from a world and a time to a set of worlds, and \(\mathcal{I}\) an unsettled world.

  \(f(w,t) = \top\), then, abstracting from closest, all future possibilities.
  \(f(w,t) = \mathbf{A}\), then, all future possibilities in which \(\mathbf{A}\) happens.

  Mechanism is unclear.
  Intuitively, think of \(f\) as returning a collection of propositions where to evaluate, context is the same, but shifted time, and world is excluded if does not match with some proposition.
  So, truth of proposition at some (restricted) time in world.

  Key idea, restrict future worlds of interest so that the agent has performed some action.

  Now turn to `can'.

  \begin{quote}
    \begin{enumerate}
      \setcounter{enumi}{41}
    \item
      \(\sem[w,t,f,\mathcal{I}]{\text{S can }\phi} = 1\) iff for some \(\alpha \in \mathcal{A}\colon \sem[w,t,f^{\alpha},\mathcal{I}]{\mathcal{W}(\text{S }\phi\text{'s})} = 1\)\newline
      i.e.\ iff for some \(\alpha \in \mathcal{A}(w,t)\colon s(\mathcal{I}, f^{\alpha}(w,t)) \subseteq \sem[t,f^{\alpha},\mathcal{I}]{\text{S }\phi\text{'s}} = 1\)\newline
      \mbox{ }\hfill\mbox{(\citeyear[16]{Boylan:2020aa})}
    \end{enumerate}
  \end{quote}
  Here, key thing is \(f^{\alpha}\).
  We're making sure that the agent performs the action in the unsettled world.

  In short, we chose some action, and figure out whether it's a phing action.
\end{note}

\section{Ability}
\label{sec:ability}

\footnote{
    \citeauthor{Kenny:1976vh} offers an example with cards.

    \begin{quote}
      Given a pack of cards, I have the ability to pick out on request a card which is either black or red; but I don't have the ability to pick out a red card on request nor the ability to pick out a black card on request.
      That is to say, the following \dots is true:

      \begin{enumerate}[label=]
      \item
        I can bring it about that either I am picking a red or I am picking a black
      \end{enumerate}

      but the following \dots is false:

      \begin{enumerate}[label=]
      \item
        Either I can bring it about that I am picking a red or I can bring it about that I am picking a black.%
        \mbox{ }\hfill\mbox{(\citeyear[215]{Kenny:1976vh})}
      \end{enumerate}
    \end{quote}

    Shift to the progressive.
    Intuitions, a little difficult.
    Once set on card, either red or black.
    However, prior to being set on a card, still picking out a card, but neither.
  }

\begin{note}
  \begin{illustration}
    I am throwing a dart at the board.

    Lands somewhere, suppose the upper-left-quadrant of the bullseye.

    Does not entail:

    I was throwing a dart at the upper-left-quadrant of the bullseye when I pulled back my arm.

    Does entail:

    I was throwing a dart at the upper-left-quadrant of the bullseye when I released the dart.
  \end{illustration}

  For, though when the dart leaves hand it may be determined, progressive true before this.
  Pull back my arm, I am throwing, but still need to push arm forward (and release).
  And, many different ways in which arms may be pushed forward.%
  \footnote{
    Related to \BoyVS{} and \BoyPS{}.
    Still, ability seems more difficult.
    Prior to performing the action.
    Though, I think this is roughly correct.
    \BoyPS{} holds, but due to choice of past point of evaluation.
  }

  Think about the bus, could have been travelling a little faster.
  If so, interrupted sooner.
  Well, really, different outcome?
\end{note}

\begin{note}
  Difficulty is parallel to \BoyPS{}.
  \begin{enumerate}[label=]
  \item
    \BoyPS{}: \(\text{Past}(S\text{ does }\phi) \Rightarrow \text{Past}(\text{Prog}[S\text{ does }\phi])\)
  \end{enumerate}
  For \citeauthor{Landman:1992wh}, shift only when stops ensures this is true.
\end{note}

\section{Landman}
\label{sec:landman}

\begin{note}
  Different perspective on~\citeauthor{Fine:1975tj}'s counterexample to~\citeauthor{Lewis:1973th}'s (\citeyear{Lewis:1973th}) theory of counterfactuals.

  In short, \citeauthor{Fine:1975tj} argues that the counterfactual `if Nixon had pressed the button there would have been a nuclear holocaust' is likely false on~\citeauthor{Lewis:1973th}'s theory.
  For, on~\citeauthor{Lewis:1973th}'s theory a counterfactual of the form `\(A \leadsto B\)' is true at a world just in case \(B\) is true at all the closest worlds in which \(A\) is true.

  And, as \citeauthor{Fine:1975tj} observes, it is plausible that in the closest worlds in which Nixon pressed the button, some malfunction occurs and a nuclear holocaust does not occur.
  For, generally speaking, the dissimilarity from our world due to a malfunction is small in contrast to the dissimilarity of a nuclear holocaust'.
  (\citeyear[452]{Fine:1975tj})

  Reasonable is constrained by the event, rather than similarity between worlds.
  So, general characteristics of world do not have a role in what is reasonable.

  Now, consider button pressing with no electrical fault.
  Reasonable that \(e\) continues the same way as it does in nuclear holocaust worlds.

  However, now suppose second check down the line.
  If just focus on event, then this has nothing to say about this.
\end{note}

\begin{note}[First problem]
  Let us begin by observing how close worlds addresses the first problem.

  First, observe that \(g\) is an event in \(u\).
  Hence, \AlgFindBranches{} return some \(\langle g,u' \rangle\)-pairing such that \(g\) happened in \(u'\) in parallel to \(g\) happening in \(u'\).
  Stressed, \AlgFindBranches{} \emph{does not} return \(\langle g,u' \rangle\)-pairing such that \(g\) happened in \(u'\) even though \(g\) didn't happen in \(u'\).
  This follows \citeauthor{Landman:1992wh}'s definition.
  For, while \(g\) stopping in \(u\) means only that \(g\) does not continue in \(u\) (cf.~\autoref{def:Landman:Stops} on \autopageref{def:Landman:Stops}).

  In turn, this means \(\langle g,u' \rangle\) may be passed as an argument to \AlgFindBranches{}.
  So, second, observe Line~\autoref{PrAl:find-branches:close} considers possible worlds close to the argument event-world pairing.
  Hence, repeated application of \AlgFindBranches{} may lead to `drift'.
  \(u'\) may be close to \(u\), and \(u''\) may be close to \(u'\), but \(u''\) need not be close to \(u\).
  This is an important feature of \citeauthor{Landman:1992wh}'s account of the progressive.
  Recall the instance of the progressive from \autopageref{prog:max:bad}:
  \begin{enumerate}
  \item
    Max is crossing the street.
  \end{enumerate}
  Also recall \ref{prog:max:bad} is true even though Max is hit by a bus.
  Now, in close worlds to the actual world, Max may be hit by a bus in various different ways, and hence there is no close world \emph{to the actual world} in which Max crosses the street.
  However, by the process of drifting through possible worlds, we expect to find a possible world in which Max is not hit by a bus (and crosses the street).
  {
    \color{red} not to counterfactual reduction?
  }
\end{note}


\begin{algorithm}[H]
    \label{PrAl:dev-tree}
    \caption{\AlgDevelopTree{}}
    \SetAlgoLined
    \DontPrintSemicolon
    \Input{%
      \(\text{Tree}\) \hfill A partially completed tree\\
      \(e\) \hfill The initial event of the tree\\
      \(w\) \hfill The initial world of the tree\\
      \(n\) \hfill An index to keep track recently added branches\\
    }
    \KwResult{Tree expanded so that there are no further stopping events}
    \Begin{
      \label{PrAl:dev-tree:start}
      \(\text{Stems} \longleftarrow \{ \langle g,u \rangle_{n} \mid \exists f,v \colon \langle \langle f,v \rangle_{n - 1}, \langle g,u \rangle_{n}\rangle \in \text{Tree}\}\)\;
      \label{PrAl:dev-tree:Extend:start}
      \label{PrAl:dev-tree:Extend:Stems}
      \(\text{GrownStems} \longleftarrow \emptyset\)\;
      \label{PrAl:dev-tree:Extend:FreshContsVar}
      \For{\(\langle g,u \rangle_{n} \in \text{Stems}\)}
      {
        \label{PrAl:dev-tree:Extend:Loop:start}
        \(\text{Growth} \longleftarrow \AlgAC{}(\langle g,u \rangle_{n})\)\;
        \(\text{GrownStems} \longleftarrow \text{GrownStems} \cup \text{Growth}\)\;
      }
      \label{PrAl:dev-tree:Extend:end}
      \(\text{Tree} \longleftarrow \text{Tree} \cup \text{GrownStems}\)\;
      \label{PrAl:dev-tree:Extend:merge}
      \textcolor{comment}{\texttt{//} \(\text{Stops} \longleftarrow \AlgGetStops{}(\text{GrownStems})\)}\;
      \label{PrAl:dev-tree:Stops:Land}%
      \(\text{Stops} \longleftarrow \AlgGetPStops{}(\text{GrownStems})\)\;
      \label{PrAl:dev-tree:Stops:Me}
      \eIf{\(\text{Stops} == \emptyset\)}
      {
        \label{PrAl:dev-tree:Stops:cond:start}
        \textbf{return} \(\text{Tree}\)\;
        \label{PrAl:dev-tree:Stops:cond:no-stops-finish}
      }
      {
        \label{PrAl:dev-tree:Stops:cond:else:start}
        \(\text{Tree} \longleftarrow \text{Tree} \cup \text{Stops}\)\;
        \label{PrAl:dev-tree:Stops:cond:tree-fix}
        \(\text{Branches} \longleftarrow \emptyset\)\;
        \label{PrAl:dev-tree:Stops:cond:else:futureB:start}
        \For{\(\langle \langle g,u \rangle_{n}, \langle h,u \rangle_{n+1}\rangle \in \text{Stops}\)}
        {
          \label{PrAl:dev-tree:Stops:cond:else:futureB:loop:start}
          \(\text{Temp} \longleftarrow \AlgFindBranches{}(\langle \langle g,u \rangle_{n}, \langle h,u \rangle_{n+1}\rangle, e, w)\)\;
          \label{PrAl:dev-tree:Stops:cond:else:futureB:loop:getBranches}
          \(\text{Branches} \longleftarrow \text{Branches} \cup \text{Temp}\)\;
          \label{PrAl:dev-tree:Stops:cond:else:futureB:loop:gather}
        }
        \label{PrAl:dev-tree:Stops:cond:else:futureB:end}
        \eIf{\(\text{Branches} == \emptyset\)}{
          \label{PrAl:dev-tree:Stops:cond:else:futureB:process:start}
          \textbf{return} Tree\;
          \label{PrAl:dev-tree:Stops:cond:else:futureB:process:cancel}
        }
        {
          \(\text{Tree} \longleftarrow \text{Tree} \cup \text{Branches}\)\;
          \label{PrAl:dev-tree:Stops:cond:else:futureB:process:expand}
          \AlgDevelopTree{}(\(\text{Tree}, e,w, n+1\))\;
          \label{PrAl:dev-tree:Stops:cond:else:futureB:process:end}
        }
        \label{PrAl:dev-tree:Stops:cond:else:end}
      }
      \label{PrAl:dev-tree:Stops:cond:end}
    }
  \end{algorithm}

  Contrast the instances of reasoning in~\autoref{fig:Rover}.
  \begin{figure}[h!]
    \mbox{}\hfill
    \begin{subfigure}{0.45\linewidth}
      \begin{enumerate}[label=\arabic*.,ref=(\arabic*)]
      \item
        \label{fig:Rover:CS:1}
        Rover is tired.
      \item
        \label{fig:Rover:CS:2}
        Rover will fall asleep soon.
      \end{enumerate}
      \caption{}
      \label{fig:Rover:CS}
    \end{subfigure}
    \hfill
    \begin{subfigure}{0.45\linewidth}
      \begin{enumerate}[label=\arabic*\('\).,ref=(\arabic*\('\))]
      \item
        \label{fig:Rover:nCS:1}
        \emph{Supposing} Rover is tired.
      \item
        \label{fig:Rover:nCS:2}
        Rover will fall asleep soon.
      \end{enumerate}
      \caption{}
      \label{fig:Rover:nCS}
    \end{subfigure}
    \hfill\mbox{}
    \caption{Two instance of reasoning}
    \label{fig:Rover}
  \end{figure}
  Have evaluation.
  However, the second is a conditional.

\begin{note}
  Illustrate in the idealised setting of proofs in propositional logic.

  Capture, \ros{} by \(\Phi \vdash \phi\)

\end{note}

\begin{note}
  \begin{center}
    \begin{fitch}
      \ftag{\scriptsize i}{\fh P \rightarrow Q} & \\
      \ftag{\scriptsize i}{\fa \fh \lnot Q} & \\
      \ftag{\scriptsize i}{\fa \fa \fh P} & \\
      \ftag{\scriptsize }{\fa \fa \fa Q} & \\
      \ftag{\scriptsize j}{\fa \fa \fa \bot} & \\
      \ftag{\scriptsize j}{\fa \fa  \lnot P} & \\
      \ftag{\scriptsize j}{\fa \lnot Q \rightarrow \lnot P} & \\
    \end{fitch}
  \end{center}

  In this idealised setting, the agent concludes \(\lnot P\) from two premises; \(\lnot Q\) and \(P \rightarrow Q\).

  Does not include the premise: \((\lnot Q \land (P \rightarrow Q)) \rightarrow \lnot P\).

    \begin{center}
    \begin{fitch}
      \ftag{\scriptsize i}{\fa \lnot Q} & \\
      \ftag{\scriptsize i}{\fa P \rightarrow Q} & \\
      \ftag{\scriptsize i}{\fj (P \rightarrow Q) \rightarrow (\lnot Q \rightarrow \lnot P)} & \\
      \ftag{\scriptsize j}{\fa \lnot Q \rightarrow \lnot P} & \\
      \ftag{\scriptsize j+1}{\fa \lnot P} & \(\rightarrow\)\textbf{Intro:} \emph{i}--\emph{j} \\
    \end{fitch}
  \end{center}

  Agent concludes \(\lnot P\) from three premises; \(\lnot Q\), \(P \rightarrow Q\), and \((P \rightarrow Q) \rightarrow (\lnot Q \rightarrow \lnot P)\).
\end{note}

\subsubsection{Additional motivation}

\begin{note}
  General account of why we get a \requ{}.
  Three ideas.
  Conclude, then support from \agpe{}.
  However, plausible accounts of support independent of \agpe{} that means support from concluding is problematic.
\end{note}

\begin{note}
  Turn to lazy argument for interest in \requ{}.
  Turns on reasoning to a different conclusion.
  Provide an argument that, in general, not permissible for an agent to reason to conflicting conclusion.

  This differs from motivation given.
  In \scen{1}, considered the \agpe{}, didn't matter why.
  Here, lazy motivation.
\end{note}

\begin{note}
  Sketch of a lazy argument.
  Three premises:
  \begin{enumerate}
  \item
    \label{lazy:evidence-constraint}
    If evidence for \(\pv{\phi}{v}\), then no evidence for conflicting \(\pv{\psi}{v'}\).
  \item
    \label{lazy:evidence}
    Support only if evidence.
  \item
    \label{lazy:reason}
    Support to \(\pv{\phi}{v}\) from \(\Phi\) only if possible to conclude from \(\pv{\phi}{v}\) to \(\Phi\) from present perspective.
  \end{enumerate}

  So, from first, evidence doesn't support conflicting.
  Second, evidence determines support.
  Third, evidence for agent means conclude.

  So, assume 

  So, from \requ{}.
  If not possible to reason to \(\pv{\psi}{v'}\) from \(\Psi\), then not possible to reason from \(\pv{\phi}{v}\) to \(\Phi\).
  Here, possibility, quantifies over opportunities.
  Then, via \ref{lazy:reason}, weaken the consequent.
  If not possible to reason to \(\pv{\psi}{v'}\) from \(\Psi\), then not \support{} from \(\pv{\phi}{v}\) to \(\Phi\).
  If support to \(\pv{\phi}{v}\) from \(\Phi\), then possible to conclude \(\pv{\psi}{v'}\) from \(\Psi\).
  But, agent also has the opportunity.
  So, agent would conclude.

  So, premise seems fine.

  \citeauthor{Way:2016vq} (\citeyear{Way:2016vq}) terms the `reasoning constraint':

  \begin{quote}
    \begin{enumerate}
    \item[RC] Reasons for you to \(\varphi\) must be considerations from which you could reason to \(\varphi\)-ing.%
      \mbox{ }\hfill\mbox{(\citeyear[806]{Way:2016vq})}
    \end{enumerate}
  \end{quote}
  As \citeauthor{Way:2016vq} notes, wide support. (\citeyear[806]{Way:2016vq})

  For example,%
  \footnote{Mentioned by \textcite{Way:2016vq}.}
  we have the following statement from~\citeauthor{Williams:1979wi} (\citeyear{Williams:1979wi}):
  \begin{quote}
    \begin{enumerate}[label=(\roman*)]
      \setcounter{enumi}{3}
    \item internal reason statements can be discovered in deliberative reasoning.%
      \mbox{ }\hfill\mbox{(\citeyear[19]{Williams:1979wi})}
    \end{enumerate}
  \end{quote}
\end{note}

\begin{note}
  The difficulty with this argument is premise.
  It's the almost-converse, of basic premise regarding support.
  Problem is, difficulty if the agent concludes.
  For, with first premise get support.
  So, this can't be right.
  Well, no coherent notion of support.

  Or, stick to reasons.
  But, well, responding to reasons, maybe.
\end{note}

\begin{note}
  So, granting the lazy argument, we get stronger instance of \zS{}.
  Now, \qzS{} internalises this.
  Hence, motivation for \qzS{} need not fall on \scen{}.

  Lazy argument is lazy.
  I have done nothing to persuade you other than cite a handful reference which provide isolated support for each of the premises.

  Though, the point is broader.
  Plausible that support only if reasoning is unique.
  Internalise.
  \qzS{}.
  Any argument for the core idea will work.
  And, internalisation step seems good.
\end{note}

\subsection{\requP{3}}
\label{sec:requ3-plus}

\begin{note}
  \begin{definition}[A \requP{0}]
    \label{idea:requP}
    For an agent \vAgent{}, and proposition-value-premises pairings \(\pvp{\phi}{v}{\Phi}\) and \(\pvp{\psi}{v'}{\Psi}\):

    \begin{itemize}
    \item
      \(\pvp{\phi}{v'}{\Psi}\) is a \emph{\requP{}} of concluding \(\pv{\phi}{v}\) from \(\Phi\) when:
      \begin{enumerate}[label=\arabic*., ref=\named{R:\arabic*}]
      \item
        \label{idea:requP:pool}
        \vAgent{} has the opportunity to (attempt to) conclude \(\pv{\psi}{v'}\) from \(\Psi\) such that:
        \begin{enumerate}[label=\roman*., ref=\named{R:1\roman*}]
        \item
          \label{idea:requP:pool:int}
          It is possible for \vAgent{} to conclude \(\pv{\psi}{v'}\) from \(\Psi\) without concluding \(\pv{\phi}{v}\) from \(\Phi\) as an intermediary step.
        \end{enumerate}
      \end{enumerate}

      \begin{enumerate}[label=\arabic*., ref=\named{R:\arabic*}, resume]
      \item
        \label{idea:requP:nPsi-nPhi}
        The following conditional is true:
        \begin{enumerate}
        \item[\emph{If}:]
          \begin{enumerate}[label=\alph*., ref=\named{R:2\alph*}]
          \item
            \label{idea:requP:nPsi-nPhi:opp}
            \vAgent{} were to take the opportunity to reason about whether \(\pv{\psi}{v'}\) follows from \(\Psi\).
          \end{enumerate}
        \item[\emph{And}:]
          \begin{enumerate}[label=\alph*., ref=\named{R:2\alph*}, resume]
          \item
            \label{idea:requP:nPsi-nPhi:link}
            \vAgent{} were to fail to conclude \(\pv{\psi}{v'}\) from \(\Psi\) prior to concluding \(\phi\) has value \(v\).
          \end{enumerate}
        \item[\emph{Then}:]
          \begin{enumerate}[label=\alph*., ref=\named{R:2\alph*}, resume]
            \label{idea:requP:nPsi-nPhi:fail}
          \item
            \vAgent{} would not conclude \(\pv{\phi}{v}\) from \(\Phi\) (\emph{due to}~\ref{idea:requP:nPsi-nPhi:opp} and~\ref{idea:requP:nPsi-nPhi:link}).
          \end{enumerate}
        \end{enumerate}
      \end{enumerate}
    \end{itemize}
    \vspace{-\baselineskip}
  \end{definition}
\end{note}

\begin{note}
  \autoref{idea:requ}, \fc{}.
  Need one instance of concluding \(\pv{\phi}{v}\) from \(\Psi\), and, general guarantee that wouldn't conclude something incompatible.
  Benefit is understanding of potential events.

  Tolerant of failure.
  \autoref{idea:requP} is on first pass much stronger.
  For each instance, would conclude.

  Switch from `or' to `and'.

  Parallel `due to' clause likewise switches.
  Question is what significance failure has for the agent.
\end{note}


% \begin{note}
%   Important to observe here that with dispositions and ability, the subjunctive analysis is an analysis.
%   So, in principle possible to provide a distinct analysis.
%   This is surely the case, and I can probably find some example.

%   By contrast, in the case of positive answers to \qzS{}, the subjunctive is `built in' to the question.
% \end{note}

% \subsubsection{Dispositions and ability}
% \label{sec:dispositions}

% \begin{note}[Parallel between dispositions and ability]
%   Consider \citeauthor{Choi:2021wg}'s characterisation of the Simple Conditional Analysis of dispositions:
%   \begin{quote}
%     An object is disposed to \emph{M} when \emph{C} iff it would \emph{M} if it were the case that \emph{C}.\nolinebreak
%     \mbox{}\hfill\mbox{(\citeyear{Choi:2021wg})}
%   \end{quote}
%   For example, an object is disposed to dissolve when it is placed in water iff the object would dissolve if it were the case that it is placed in water.

%   The Simple Conditional Analysis may be challenged, but for our purposes it is adequate.
%   We are interested in the broad form of the truth condition, and various more refined analyses share the same broad form.
%   Note, in particular, that it being the case that \emph{C} and \emph{M} happening describes an event.
%   Given appropriate conditions; salt dissolves, glass breaks, and I mumble when I am tired.
%   The key idea is that the property of being disposed to \emph{M} when \emph{C} is analysed in terms of the (possible) event of \emph{M} happening when \emph{C}.

%   The parallel to ability is established by noting that ability may also be analysed in terms of a (possible) event, as we have seen.
%   In particular, by incorporating volition in the analysans of the Simple Conditional Analysis.
%   To illustrate, \citeauthor{Mandelkern:2017aa} trace the Conditional Analysis of ability  to \textcite{Hume:1748tp} and \textcite{Moore:1912te}, among others:
%   \begin{quote}
%     S can \(\phi\) iff S would \(\phi\) if S tried to \(\phi\)\nolinebreak
%     \mbox{}\hfill\mbox{(\citeyear[Cf.][308]{Mandelkern:2017aa})}
%   \end{quote}
%   Compare to the Simple Conditional Analysis of dispositions:
%   The object is some agent \emph{S}, \emph{C} is `S tried to \(\phi\)' and \emph{M} is `S \(\phi\)s' --- it is volition alone which distinguishes the analyses.
% \end{note}

\subsection{Interpretation}
\label{cha:zSpA:sec:interpretation}

\begin{note}
  Doxastic justification.
\end{note}



\subsubsection{Doxastic justification}
\label{cha:fcs:sec:dox-just}

\begin{note}
  \citeauthor{Turri:2010aa}

  \begin{quote}
    Necessarily, for all S, \emph{p}, and \emph{t}, if \emph{p} is propositionally justified for S at \emph{t}, then \emph{p} is propositionally justified for S at \emph{t} because S currently possesses at least one means of coming to believe \emph{p} such that, were S to believe \emph{p} in one of those ways, S's belief would thereby be doxastically justified.%
    \mbox{ }\hfill\mbox{(\citeyear[316]{Turri:2010aa})}
  \end{quote}

  Key is that doxastic justification depends on what the agent does.

  \citeauthor{Turri:2010aa}'s focus is on how reasons are used.
  What the agent does.

  Seen with example.

  \begin{quote}
    \begin{enumerate}[label=(P\arabic*)]
      \setcounter{enumi}{4}
    \item
      The Spurs will win if they play the Pistons.
    \item
      The Spurs will play the Pistons.
    \end{enumerate}

    \hbox to \hsize{\hfil{\vdots}\hfil}

    \begin{enumerate}[label=(P\arabic*), resume]
    \item
      Therefore, the Spurs will win.%
    \mbox{ }\hfill\mbox{(\citeyear[317]{Turri:2010aa})}
    \end{enumerate}
  \end{quote}

  Rather than \emph{modus ponens}, `\emph{modus profusus}'.
  Conclude \(r\) from \(p\) and \(q\).
  (\citeyear[317]{Turri:2010aa})

  \begin{quote}
    The way in which the subject performs, the manner in which she makes use of her reasons, fundamentally determines whether her belief is doxastically justified.
    Poor utilization of even the best reasons for believing \emph{p} will prevent you from justifiedly believing or knowing that \emph{p}.%
    \mbox{ }\hfill\mbox{(\citeyear[316]{Turri:2010aa})}
  \end{quote}

  Variant of ~\cite{Prior:1960wh}'s `tonk' connective.
  Though, difference is between connective and rule.
  \(p\) tonk \(q\) would not be propositionally justified.
\end{note}

\begin{note}
  \citeauthor{Turri:2010aa} is similar to \citeauthor{Goldman:1979ui}

  Begin with justification.

  \begin{quote}
    \begin{enumerate}[label=(\arabic*)]
      \setcounter{enumi}{10}
    \item
      Person \emph{S} is \emph{ex ante} justified in believing \emph{p} at \emph{t} if and only if there is a reliable belief-forming operation available to \emph{S} which is such that if \emph{S} applied that operation to this total cognitive state at \emph{t}, \emph{S} would believe \emph{p} at \emph{t}-plus-delta (for a suitably small delta) and that belief would be \emph{ex post} justified.
    \end{enumerate}
  \end{quote}

  Where, sufficient condition for belief would be \emph{ex post} justified:
  \begin{quote}
    \begin{enumerate}[label=(\arabic*)]
      \setcounter{enumi}{4}
    \item
      If S's believing \emph{p} at \emph{t} results from a reliable cognitive belief-forming process (or set of processes), then S's belief in \emph{p} at \emph{t} is justified.%
      \mbox{ }\hfill\mbox{(\citeyear[13]{Goldman:1979ui})}
    \end{enumerate}
  \end{quote}
  Roughly, at least.
  \citeauthor{Goldman:1979ui} refines this a fair bit, but this isn't important.

  Availability of a reliable belief-forming operation!

  Relation here is brittle.
  Account of justification, apply to concluding.
  Well, then all we get is that before concluding, would make sense to conclude only if available.
  Running something like the \citeauthor{Carroll:1895uj} regress, not some state.
  But, this only tells us about suitability to conclude.

  Still, key point is process.

  Another useful thing to highlight is the suitably small delta.
  With \requ{}, this is captured in terms of the option.
\end{note}

\begin{note}
  Significant difference is in the case of justification, we're not interested in the \agpe{}.
  Hence, these accounts are understood in terms of the agent having the ability, roughly.
\end{note}


\begin{note}
  \citeauthor{Rapoport:1989ww}'s (\citeyear{Rapoport:1989ww}) overview of the iterated Prisoner's Dilemma:%
  \footnote{
    Strictly a little more general than what I want.
    However, it's a helpful \illu{0} as gets the core idea, and suggests how to interpret other instances of the term.
  }

  \begin{quote}
    However, this argument does not apply to the play known to be the last, because no retaliation can follow.
    Thus, \(D\) dominates \(C\) on the last play, and according to the Sure-thing Principle, \(D_{1}D_{2}\) is a foregone conclusion.%
    \mbox{ }\hfill\mbox{(\citeyear[202]{Rapoport:1989ww})}
  \end{quote}

  \citeauthor{Rapoport:1989ww}'s uses the term `foregone conclusion' to stress that \(D_{1}D_{2}\) will happen.
  Applies not to a conclusion in terms of a proposition-value pair, but a proposition, both players defect.
  However, foregone conclusion in the way we will understand the term.
  For, reasoning why foregone conclusion traces the reasoning of each player.

  Reconstruct.

  % \begin{tabular}{cr|c|c|}
  %   & \multicolumn{1}{c}{} & \multicolumn{2}{c}{\(2\)}\\
  %   & \multicolumn{1}{c}{} & \multicolumn{1}{c}{\(C_{2}\)}  & \multicolumn{1}{c}{\(D_{2}\)}\\
  %   \cline{3-4}
  %   \multirow{2}*{\(1\)}  & \(C_{1}\) & \(-1,-1\) & \(-3,\phantom{-}0\)\\
  %   \cline{3-4}
  %   & \(D_{1}\) & \(\phantom{-}0,-3\) & \(-2,-2\)\\
  %   \cline{3-4}
      %     \end{tabular}

  \begin{figure}[H]
    \centering
    \begin{tabular}{r|c|c|}
      \multicolumn{1}{c}{} & \multicolumn{1}{c}{\(C_{2}\)} & \multicolumn{1}{c}{\(D_{2}\)} \\
      \cline{2-3}
      \(C_{1}\) & \(-1,-1\) & \(-3,\phantom{-}0\)\\
      \cline{2-3}
      \(D_{1}\) & \(\phantom{-}0,-3\) & \(-2,-2\)\\
      \cline{2-3}
    \end{tabular}
    \caption{Test}
    \label{fig:PD:matrix}
  \end{figure}

  Assumption.
  Player will always chose the perform an action with the highest expected payoff.
  Consider player \(1\).

  Suppose player \(2\) chooses to cooperate.
  If player \(1\) also chooses to cooperate, payoff is \(-1\).
  If player \(1\) chooses to defect, payoff is \(0\).
  Hence, if player \(2\) chooses to cooperate, defect.

  By similar reasoning.
  Suppose player \(2\) chooses to defect.
  If player \(1\) chooses to cooperate, payoff is \(-3\).
  If player \(1\) chooses to defect, payoff is \(-2\).
  Hence, if player \(2\) chooses to defect, also defect.

  The reasoning is parallel when players are switched.

  Instance of PD.
  \(D\) dominates \(C\), best response is \(D\) regardless of what the agent does.
  Hence, \(D\) if \(D\) and \(D\) if \(C\).
  Sure-thing principle, same action in case of \(A\) and case of \emph{not}-\(A\), then same action if they do not know whether \(A\) (or \emph{not}-\(A\)).
  So, \(D\).

  Paired with parallel reasoning for player \(1\).

  Prior to the last play, both players will reason in this way, and hence foregone conclusion in the sense we have in mind.
\end{note}

\subsection{Positive answers and ability}
\label{sec:posit-answ-abil}

\begin{note}
  Presented \scen{1}.

  Broad argument for positive answers in certain cases.
\end{note}

\begin{note}
  An interesting observation here is that in certain this all arises, to a certain extent, because of \abgen{} abilities.
  \abgen{2} ability spans multiple different proposition-value-premises pairings.
  Hence, all of these function as \requ{1}, so long as the agent has the option.

  \abgen{2} ability spans multiple different proposition-value-premises pairings.
  Hence, all of these function as \requ{1}, so long as the agent has the option.

  \begin{itemize}
  \item
    \abgen{2} and \abspec{} abilities.
  \item
    Answers to why, then.
    Note, here, that opportunity is interesting.
    The whole conjunction of all instance of the \abgen{} ability is plausibly not a \requ{}.
    However, all that's needed is the \emph{individual} instances, and for these to raise a problem.
  \item
    The point is, \requ{1} for any \abgen{} ability, and these are also \requ{1} for main pairing.
    (%
    Note --- or perhaps emphasise --- here, that the problem is \emph{not} recursive.
    Instead, the problem is about the spread.%
    )
  \item
    Here, then, ability is both the problem and the answer.
    What's interesting is the way in which ability functions.
    It's not merely \emph{that} the agent has the ability.
    Instead, it \emph{is} the ability.
  \end{itemize}
\end{note}

\begin{note}
  So, the way in which past reasoning relates is by ensuring that the agent would reach the same conclusion.
  About the agent's reasoning.
  \emph{How} rather than \emph{that}.

  Look, what we are getting is that the agent would conclude.
  If something were to happen, then some action would be performed.
  There's no distinction between the answer and performing the act, roughly.
  Or, better put, the answer \emph{is about present reasoning}.
  Answer states that in present reasoning, would not fail.

  It is about the agent's present epistemic state, and in particular what the agent's present epistemic state is capable of.

  In other words, ability.
  What answers is ability, in the sense that ability iff would.

  This is very important to the understanding of \fc{}.

  And, I kind of want to have ability as a gloss, while focusing on \fc{} to avoid going into ability in too much detail.

  So, positive answer, then it's the pairing \emph{being} a \fc{}.
  (I should always use this instance of the copula.)
\end{note}

\subsection*{Narrowing \requ{1}}

\begin{note}[Expanding pool constraints]
  {
    \color{red} Check counter!
  }
  To~\ref{idea:requ:pool} of~\autoref{idea:requ} the following clause may also be added:
  \begin{enumerate}[label=]
  \item
    \begin{enumerate}[label=]
    \item
      \begin{enumerate}[label=\roman*., ref=(\roman*)]
        \setcounter{enumiii}{3}
      \item
        \label{idea:requ:pool:method}
        Concluding \(\pv{\psi}{v'}\) from \(\Psi\) involves the same general method the agent would use to conclude \(\pv{\phi}{v}\) from \(\Phi\).
      \end{enumerate}
    \end{enumerate}
  \end{enumerate}
  We omit~\autoref{idea:requ:pool:method} from the idea of a \requ{} for two (related) reasons.
  First, it is not clear what `the same general method' amounts to in detail.
  Second, avoiding questions about method affords flexibility when providing \illu{1} of \zS{}.
  However,~\autoref{idea:requ:pool:method} may be imposed with no loss to the role of \zS{} in the overall argument.
  However, always a check on whether one has the \abgen{} ability.
\end{note}

\begin{note}
    \begin{restatable}[\sCe{2}]{definition}{defZCe}
    \label{def:sCe}
    For an agent \vAgent{}, proposition-value pair \(\pv{\phi}{v}\), \poP{} \(\Phi\), and event \(e\):

    \begin{itemize}
    \item
      \(e\) is event in which \vAgent{} \emph{\sCe{}} \(\pv{\phi}{v}\) from \(\Phi\)
    \end{itemize}

    \emph{If and only if}

    \begin{itemize}
    \item
      For any sub-event \(e^{\flat}\) of \(e\) such that \(e^{\flat}\) is an event in which the agent is concluding \(\pv{\phi}{v}\) from \(\Phi\), \(e^{\flat}\) is an event in which \vAgent{} is \sCing{} \(\pv{\phi}{v}\) from \(\Phi\).
    \end{itemize}
    \vspace{-\baselineskip}
  \end{restatable}

    \begin{restatable}[\sCon{2}]{definition}{defZCon}
    \label{def:sCon}
    For an agent \vAgent{}, proposition-value pair \(\pv{\phi}{v}\), \poP{} \(\Phi\), and event \(e\):

    \begin{itemize}
    \item
      \(\pv{\phi}{v}\) is a \emph{\sCon{}} from \(\Phi\).
    \end{itemize}

    \emph{If and only if}

    \begin{itemize}
    \item
      The event \(e\) in which the \vAgent{} pairs \(\phi\) with \(v\) is a sub-event of an event \(e^{\sharp}\) in which \vAgent{} \sCe{1} \(\pv{\phi}{v}\) from \(\Phi\).
    \end{itemize}
    \vspace{-\baselineskip}
  \end{restatable}
\end{note}

\begin{note}
  Informally, if \(e\) is an event in which an agent \sCe{} \(\pv{\phi}{v}\) from \(\Phi\), then, as the event way developing, there was nothing\dots
\end{note}


  \footnote{
    Here might be the trouble.
    Understanding conditional just is getting rule of inference.
    But, this is not obvious.
    For, other connectives.
    It's not obvious that I get conjunction i/e from understanding conjunction.
    For, could be the case that always translate conjunction to some other connective when applying rules of inference.

    Now, still \emph{modus ponens}.
    But, in principle no need for this.

    Consider a Gentzen system for propositional logic termed \textbf{G3cp} (\cite[\S3.5]{Troelstra:2000ue},\cite[\S3.1]{Negri:2008wy}).

    In \textbf{G3c}(\textbf{p}) the two rules concerning material implication are:

    \begin{quote}
      \mbox{ }\hfill%
      \(
      \AxiomC{\(\Gamma \Rightarrow \Delta, A\)}
      \AxiomC{\(B,\Gamma \Rightarrow \Delta\)}
      \LeftLabel{L\(\rightarrow\)}
      \BinaryInfC{\(A \rightarrow B, \Gamma \Rightarrow \Delta\)}
      \DisplayProof
      \)%
      \hfill
      \(
      \AxiomC{\(A,\Gamma \Rightarrow \Delta,B\)}
      \LeftLabel{R\(\rightarrow\)}
      \UnaryInfC{\(\Gamma \Rightarrow \Delta, A \rightarrow B\)}
      \DisplayProof
      \)%
      \hfill\mbox{ }\newline
      \mbox{ }\hfill\mbox{(\citeyear[77]{Troelstra:2000ue})}
    \end{quote}
    By inspection, neither rule corresponds to \emph{modus ponens} in any direct way.
  }\(^{,}\)%

  \begin{quote}
    My paradox \dots turns on the fact that, in a Hypothetical, the \emph{truth} of the Protasis, the \emph{truth} of the Apodosis, \& the \emph{validity of the sequence}, are 3 distinct Propositions.
    \begin{quote}
      For instance, if I grant

      \begin{enumerate}[label=(\arabic*)]
      \item
        All men are mortal, \& Socrates is a man, but not
      \item
        The sequence “If all men are mortal and if Socrates is a man, then Socrates is mortal” is valid,
      \end{enumerate}

      then I do not grant

      \begin{enumerate}[label=(\arabic*), resume]
      \item
        Socrates is mortal.
      \end{enumerate}
      Again, if I grant (2), but not (1), I still fail to grant (3).

      Hence, before granting (3), I must grant (1) \& (2)\newline
      \mbox{ }\hfill\mbox{(\citeyear[10--11]{Carroll:2016wl})}
    \end{quote}
  \end{quote}

% \begin{note}
%   To say an agent has \ninf{0} is not to say the agent has \emph{\pinf{}}.
%   For, an agent may not have choice over whether an event develops into an event in which the agent concludes \(\pv{\phi}{v}\) from \(\Phi\).

%   For example, take \(\chi\) to be the proposition `the area of a unit square is equal to the area of a unit circle'.
%   An agent has \ninf{0} over whether or not they concluding \(\pv{\chi}{\text{True}}\), as the agent may choose not to make any attempt.
%   However, assuming the agent would only conclude \(\pv{\chi}{\text{True}}\) from principles consistent with Euclidean geometry, then agent does not have \pinf{} over whether or not they are concluding \(\pv{\chi}{\text{True}}\).
%   For, the area of a unit square is not equal to the area of a unit circle.
%   Hence, it is not possible for the agent to ensure that some event may develop into an event in which they conclude \(\pv{\chi}{\text{True}}\).

%   In parallel, an agent need not have \pinf{0} over whether or not they are going to buy some eggs.
%   For, it may be there are no eggs for sale.
%   Hence, it may not be possible for the agent to ensure that the event develops into an event in which the agent buys some eggs.
% \end{note}

\begin{note}
  \autoref{prop:requ-fc} expands \autoref{def:requ} via the definition of a \fc{}.

  \begin{proposition}[A \requ{0}, expanded]
    \label{prop:requ-fc}
    \cenLine{
      \begin{itemize*}[noitemsep, label=\(\circ\)]
      \item
        Agent: \vAgent{}
      \item
        Propositions: \(\phi\), \(\psi\)
      \item
        Values: \(v\), \(v'\)
      \item
        \poP{3}: \(\Phi\), \(\Psi\)
      \item
        Event: \(e\)
      \item
        \mbox{ }
      \end{itemize*}
    }

    \begin{itemize}
    \item
      \(\pvp{\phi}{v'}{\Psi}\) is a \emph{\requ{}} of \(e\) for \vAgent{}, \emph{if and only if}:
      \begin{itemize}
      \item[\emph{If}:]
        \begin{enumerate}[label=\alph*., ref=(\alph*), series=requDefSeries]
        \item
          \label{prop:requ-fc:nk}
          Throughout \(e\), either~\ref{prop:requ-fc:nk:psi} or~\ref{prop:requ-fc:nk:no-conf} are true:
          \begin{enumerate}[label=\roman*., ref=(\roman*)]
          \item
            \label{prop:requ-fc:nk:psi}
            There is no \pevent{} in which \vAgent{} concludes \(\pv{\psi}{v'}\) from \(\Psi\).
          \item
            \label{prop:requ-fc:nk:no-conf}
            There is a \pevent{} in which \vAgent{} concludes something incompatible with a conclusion of \(\pv{\psi}{v'}\) from \(\Psi\).
          \end{enumerate}
        \end{enumerate}
      \item[\emph{Then}:]
        \begin{enumerate}[label=\alph*., ref=(\alph*), resume*=requDefSeries]
        \item
          \label{prop:requ-fc:ne}
          \(e\) is not an event in which \vAgent{} is concluding \(\pv{\phi}{v}\) from \(\Phi\).
        \end{enumerate}
      \end{itemize}
    \end{itemize}
    \vspace{-\baselineskip}
  \end{proposition}

  \begin{argument}{prop:requ-fc}
    From the definition of a \requ{} (\autoref{def:requ}, \autopageref{def:requ}) and the definition of a \fc{} (\autoref{def:fc}, \autopageref{def:fc}).
  \end{argument}
\end{note}


% \subsection{\citeauthor{Owens:2006tw}}

% \begin{note}
%   For example, \citeauthor{Owens:2006tw} argues for a belief expression model of assertion in which the rationality of a belief formed by an agent via testimony is connected to justification of the testifier:

%   \begin{quote}
%     Trusting an expression of belief by accepting what a speaker says involves entering a state of mind which gets its rationality from the rationality of the belief expressed.
%     This state's rationality depends on the speaker's justification for the belief he expresses, not on his justification for the action of expressing it.
%     And to hear a speaker as making a sincere assertion, as expressing a belief, is \emph{ceteris paribus} to feel able to tap into \emph{that} justification (whether or not his assertion was directed at you) by accepting what he says.%
%     \mbox{}\hfill\mbox{(\citeyear[123]{Owens:2006tw})}
%   \end{quote}

%   On the view advanced by \citeauthor{Owens:2006tw}, justification.
%   View in terms of \support{}.

%   \support{} directly.
%   Rationality of agent is rationality of speaker.

%   However, `depends'.

%   Distinction between rationality of state, and relation between rationality of state and rationality of state.

%   Inclined to think \citeauthor{Owens:2006tw} is arguing for the former.%
%   \footnote{
%     \begin{quote}
%       If we are to believe what the speaker indicates he believes, either the speaker must justify this belief to us, or we must supply some justification of our own
%       [\dots]
%       Neither act can be part of a rationality preserving mechanism for belief.%
%       \mbox{ }\hfill\mbox{(\citeyear[123--124]{Owens:2006tw})}
%     \end{quote}
%   }
%   Though, it is not clear to me that embedded isn't a viable option.

%   Regardless, distinction that is important.
% \end{note}



\begin{note}
  builds on two ideas regarding events:
  \begin{enumerate}
  \item
    An event may or may not develop into some other event.
  \item
    Something about an event may influence which other events it may develop into.
  \end{enumerate}
  %
  To illustrate, consider the following \scen{0}:

  \begin{scenario}[Coin flip]%
    \label{illu:coinT}%
    Agent \(B\) has accepted gamble on coin toss from Agent \(A\).
    The gamble is captured by the following pair of conditionals:
    %
    \begin{itemize}
    \item
      If the coin lands heads, Agent \(B\) gives \texteuro{}5 to Agent \(A\).
    \item
      If the coin lands tails, Agent \(A\) gives \texteuro{}5 to Agent \(B\).
    \end{itemize}
    %
    Agent \(A\) has tossed the coin, and the result is hidden between Agent \(A\)'s hands.
  \end{scenario}

  \noindent%
  Intuitively, the gamble of \autoref{illu:coinT} influences any development of \autoref{illu:coinT}.
  In particular, a pair of conditional follow from the gamble:
  %
  \begin{itemize}
  \item
    If the coin has landed heads, then agent \(B\) will soon be \texteuro{}5 poorer.
  \item
    If the coin has landed tails, then agent \(A\) will soon be \texteuro{}5 poorer.
  \end{itemize}
  %
  Likewise, given the gamble, it is not the case that Agent \(A\) notices the coin has landed heads and receives \texteuro{}5 from Agent \(B\).%
  \footnote{
    \label{fn:desc-con-caveat}
    Note the distinction between our description of the event and the event.
    It is consistent with our description of the event that Agent \(A\) is dishonest and will run before handing anything to \(B\), or that neither agent really has \texteuro{}5 at hand and are hoping they win, or \dots
    However, at issue is the event.
  }

  The \itc{} of \qWhyV{} parallels to the way the observed conditionals hold of the event as given in~\autoref{illu:coinT}.

  For, consider an event \(e\) in which an agent concludes \(\pv{\phi}{v}\) from \(\Phi\).
  This event may be broken down into various sub-events, and for any sub-event may or may not develop into \(e\).
  And, an sub-event \(e^{\flat}\) of \(e\) may required a \ros{} to hold between \(\pv{\psi}{v'}\) and \(\Psi\) to developed into \(e\).

  {
    \color{red}
    WITHOUT CONDITIONALS, THEN `WHY' IS IT THE CASE THAT MONEY IS HANDED OVER?
  }
\end{note}



\subsection{\fc{3} and \issueConstraint{}}

\begin{note}
  \begin{proposition}%
    \label{prop:fc-wit}%
    \vspace{-\baselineskip}
    \begin{itenum}
    \item[\emph{If}:]
      Both~\ref{prop:fc-wit:fc} and~\ref{prop:fc-wit:noC} hold:
      \begin{enumerate}[label=\alph*., ref=(\alph*)]
      \item
        \label{prop:fc-wit:fc}
        \(\pv{\psi}{v'}\) is a \fc{0} from \(\Psi\), for \vAgent{} throughout \(e\).
      \item
        \label{prop:fc-wit:noC}
        \vAgent{} has not concluded \(\pv{\psi}{v'}\) from \(\Psi\).
      \end{enumerate}
    \item[\emph{Then}:]
      Both~\ref{prop:fc-wit:ros} and~\ref{prop:fc-wit:noW} hold:
      \begin{enumerate}[label=\alph*\('\)., ref=(\alph*\('\))]
      \item
        \label{prop:fc-wit:ros}
        A \ros{} between \(\pv{\psi}{v'}\) and \(\Psi\), for \vAgent{}.
      \item
        \label{prop:fc-wit:noW}
        \vAgent{} doesn't have a \wit{} for the \ros{} between \(\pv{\psi}{v'}\) and \(\Psi\).
      \end{enumerate}
    \end{itenum}
    \vspace{-\baselineskip}
  \end{proposition}

  \begin{argument}{prop:fc-wit}
    \ref{prop:fc-wit:fc} entails \ref{prop:fc-wit:ros} by \supportII{}.

    \noindent \ref{prop:fc-wit:noC} entails \ref{prop:fc-wit:noW} by the definition of a \wit{} (\witpage{}).
  \end{argument}

  For, in order to argue against \issueConstraint{}, need some \(\pvp{\psi}{v'}{\Psi}\) such that answers \qWhyV{}.
  Consider \autoref{scen:calc}.
  \fc{}.
\end{note}

\begin{note}
  Abstracting a little, \(\pv{\psi}{v'}\) being a \fc{} from \(\Psi\) is never required to understand why \(e\) is an event in which an agent concludes \(\pv{\phi}{v}\) from \(\Phi\).

  It may be the case that \(\pv{\psi}{v'}\) is a \fc{} from \(\Psi\), but irrelevant.
  In particular \(\pv{\phi}{v}\) may be a \fc{} from \(\Phi\), but what matters is the agent's conclusion of \(\pv{\phi}{v}\) from \(\Phi\) (--- \emph{not} \(\pv{\phi}{v}\) being a \fc{} from \(\Phi\)).

  In this respect, \fc{1} are compatible with \issueConstraint{}.
\end{note}



%%% Local Variables:
%%% mode: latex
%%% TeX-master: "master"
%%% TeX-engine: luatex
%%% End:
