\chapter{Additional examples of \fc{1}}
\label{cha:addit-exampl-fcs1}


\subsection{\scen{3} where some \prop{0}-\val{0} pair is a \fc{}}

\begin{note}[Propositional logic]
  \begin{scenario}[Propositional theorems]
    \label{illu:sketch:prop-logic}
    Suppose an agent has a good grasp of propositional logic.
    In particular:
    The agent has a good understanding of some method to construct semantic proofs.
    For example, by constructing truth tables, or reasoning about valuation functions.
  \end{scenario}

  Given the agent's understanding of propositional logic, the following is a \fc{}:
  \begin{itemize}
  \item
    For any valuation \(v\), \(v \vDash ((P \rightarrow Q) \rightarrow P) \rightarrow P \)
  \end{itemize}

  There's nothing particularly special about the formula.
  Given a good understanding, any formula of a reasonable length will do.

  Note, the agent is concluding.
  It's fine for the agent to work things through on a piece of paper.
\end{note}


\begin{note}[Poppies]
  To finish, we take something from literature:

  \begin{scenario}[Poppies]
    \label{illu:fc:poppies}
    \mbox{ }
    \vspace{-\baselineskip}
    \begin{quote}
      Was Tarquinius Superbus in seinem Garten mit den Mohnköpfen sprach, verstand der Sohn, aber nicht der Bote.

      [What Tarquinius Superbus said in the garden by means of the poppies, the son understood but the messenger did not].\newline
      \mbox{ }\hfill\mbox{(\cite[190]{Hamann:1822vp}/\cite[3]{Kierkegaard:1983ta})}
  \end{quote}
  \vspace{-\baselineskip}
  \end{scenario}

  \noindent The quote is from the epigraph to~\citeauthor{Kierkegaard:1983ta}'s \hyperlink{cite.Kierkegaard:1983ta}{Fear and Trembling}.
  \hyperlink{cite.Kierkegaard:1983ta}{H.\ Hong and E.\ Hong} detail the relevant background:

  \begin{quote}
    When the son of Tarquinius Superbus had craftily gotten Gabii in his power, he sent a messenger to his father asking what he should do with the city.
    Tarquinius, not trusting the messenger, gave no reply but took him into the garden, where with his cane he cut off the flowers of the tallest poppies.
    The son understood from this that he should eliminate the leading men of the city.%
    \mbox{ }\hfill\mbox{(\citeyear[339]{Kierkegaard:1983ta})}
  \end{quote}

  \noindent For Superbus' son, but not for the messenger the following was a \fc{} from some \pool{}:

  \begin{enumerate}[label=C\thescenarioCounter., ref=(C\thescenarioCounter)]
  \item
    \label{illu:fc:poppies:c}
    \pv{\propI{Eliminate the leading men of the city}}{\valI{Should}}
  \end{enumerate}

  \noindent Or, at least, Superbus \emph{expected}~\ref{illu:fc:poppies:c} be a \fc{} for his son {\color{blue} after the son sees the poppies}.
\end{note}

\subsection{\scen{3} where a \prop{0}-\val{0} pair is not a \fc{}}

\paragraph*{Novel information}

\begin{note}
  \begin{scenario}[Where's Wally]
    \label{illu:fc:wally}
    An agent has a book containing numerous drawings of scenes in which various characters are doing a variety of things.
    Somewhere in each scene is a character called `Wally', identifiable by a collection distinguishing features.
    These features include a red and white striped jumper, blue trousers, short brown wavy hair, round glasses, and so on.
  \end{scenario}

  Agent does not keep in mind 

  \begin{enumerate}[label=C\thescenarioCounter., ref=(C\thescenarioCounter)]
  \item
    \label{illu:fc:wally:c}
    \pv{\propI{The character identified is Wally}}{\valI{True}}
  \end{enumerate}

  Consider slight variation.
  Agent forgets whether Wally wears round glasses part way through.
  It remains the case that \ref{illu:fc:wally:c} is a \fc{}, so long as there was some 
\end{note}

\paragraph*{Conflict}

\begin{note}
  \begin{scenario}[Knowing whether and belief]%
    \label{ill:fcs:kw}%
    \citeauthor{Barker:1975un} suggests the following two principles hold with respect to knowing whether:

    \begin{enumerate}[label=(\Alph*), ref=(\Alph*)]
    \item
      \label{Barker:1975un:A}
      If \emph{S} knows whether \emph{p} and \emph{S} believes that \emph{p}, then \emph{p}.
    \item
      \label{Barker:1975un:B}
      If \emph{S} knows whether \emph{p} and \emph{S} believes that not-\emph{p}, then not-\emph{p}.%
      \mbox{ }\hfill\mbox{(\citeyear[281]{Barker:1975un})}
    \end{enumerate}
  \end{scenario}

  \noindent The conclusions of interest are:

  \begin{enumerate}[label=C\Alph*., ref=(C\Alph*)]
  \item
    \label{ill:fcs:kw:cA}
    \pv{\propI{If \emph{S} knows whether \emph{p} and \emph{S} believes that \emph{p}, then \emph{p}}}{\valI{True}}
  \item
    \label{ill:fcs:kw:cB}
    \pv{\propI{If \emph{S} knows whether \emph{p} and \emph{S} believes that not-\emph{p}, then not-\emph{p}}}{\valI{True}}
  \end{enumerate}

  \noindent I expect neither~\ref{ill:fcs:kw:cA} nor~\ref{ill:fcs:kw:cB} are \fc{1}.
  For, rather than conclude either~\ref{ill:fcs:kw:cA} or~\ref{ill:fcs:kw:cB} one will find a counterexample.%
  \footnote{
    For example, consider an agent \emph{A} playing speed chess.
    It's the end game, and \emph{A} believes they have a winning strategy.
    \emph{A} knows whether they have a winning strategy, but doesn't have time to work through the details.
    Given \ref{Barker:1975un:A}, \emph{A} has a winning strategy.
    \emph{A} does not have a winning strategy.
  }
  And, given a counterexample one will not conclude the principle is true.
\end{note}

% \begin{note}[Non-deductive \illu{1}]
%   \autoref{illu:fc:chess:I} and parallel examples are motivated by an agent's grasp on some effective method to solve a type of problem.
%   However, \(\pv{\psi}{v'}\) being a \fc{} from \(\Psi\) does not require an effective method.
%   It need only be the case that the agent is concluding \(\pv{\psi}{v'}\) from \(\Psi\) after an action is done.
%   Consider, the following \scen{0}:

%   \begin{scenario}[Sunny days]%
%     \label{illu:fc:sunny}%
%     It's mid summer day in the Bay Area.
%   \end{scenario}

%   \noindent For me, the following conclusion is a \fc{} from some \pool{}:

%   \begin{enumerate}[label=C\thescenarioCounter., ref=(C\thescenarioCounter)]
%   \item
%     \label{illu:fc:sunny:c}
%     \pv{\propI{It will rain tomorrow}}{\valI{False}}
%   \end{enumerate}

%   \noindent%
%   There is no effective method for me to determine whether it will rain tomorrow, and I recognise there may be rain tomorrow.
%   Still, I am sufficiently committed to some uniformity principle.%
%   \footnote{
%     Cf. (\cite[70]{Hempel:1965aa}),~(\cite{Henderson:2022aa}).
%   }
%   And, that the principle together with past experience, ensure that if I consider whether it will rain tomorrow, I conclude it will not rain.%
% \end{note}

%%% Local Variables:
%%% mode: latex
%%% TeX-master: "master"
%%% TeX-engine: luatex
%%% End:
