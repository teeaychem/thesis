
% \subsubsection{Concluding}
% \label{sec:concluding}

% \begin{note}
%   It is not clear the following suggestion holds.
%   \begin{suggestion}
%     If \(e\) is event which may develop, then \(e\) is event in which concluding.

%     {
%       \color{red}
%       This is where the problem is.
%       I don't think this is obvious.

%       We have assumed that if \(e^{sub}\) \emph{is} an event of concluding, then may develop.

%       However, this suggestion is the converse.
%     }
%   \end{suggestion}
% \end{note}

% \begin{note}
%   For, \(e\) is an event in which the agent throws a dart at the centre of the dartboard.
%   However, at the initial stage of \(e\), it is not the case that the agent is sure to throw the dart at the centre of the dartboard.

%   There is a difference between how the event developed, and how the may have developed.

%   So, this idea isn't right in general.

%   Problem.
%   For, if existential account of progressive, then get explain how things came to be simply by focusing on what happened.

%   I do have an answer to this, which is that the role of the \fc{} is what happened.

%   Okay, so hang on.
%   For an agent to be concluding, it does not need to be the case that the event is guaranteed to develop in a way such that things work out.

%   However, it does need to be the case that there is some way things work out.

%   And, the way in which \requ{1} work is that failure means there is no way in which things work out.
% \end{note}

% \begin{note}
%   There are ways to approach this:
%   \begin{itemize}
%   \item
%     Deny the premise.

%     If event in which an agent concludes, it need not be the case that the agent was concluding throughout all the relevant sub-events.

%     Then, no issue is resolved.
%   \item
%     Deny that the event was an event in which the agent concluded.

%     Specifically, narrow the relevant event.
%     For, we have some initial event, which develops in two a larger event, which finishes with an event in which the agent concludes \(\pv{\phi}{v}\).
%     So long as final sub-event, then the agent concludes.
%     The issue only arises from the apparent link to earlier (sub-)events.
%     But, if otherwise, then motivation to reject these as (sub-)events of an event in which the agent concludes.
%   \item
%     Shift perspective.

%     What it true after the fact need not be true as things are developing.
%     This is \citeauthor{Boylan:2020aa}'s approach.
%   \end{itemize}

%   I am inclined to consider the former.
%   When speak on concluded, just picking out an event.
%   Naturally event to some default event.
%   However, flexible.

%   So, here \citeauthor{Boylan:2020aa} and darts.
%   I was able to, sure, but only after things hand developed so far.

%   Additionally, consider picture.
%   Drew a dog.
%   However, not the case that drawing a dog throughout associated event.
%   Started out as a doodle.

%   Ran 10k.
%   However, started out, plan was to run 5k.
%   Interesting consequence here is that not running 5k.
%   For, supposing extension came about by choice, something changed, and hence no force to finish at 5k.

%   Running 10k includes running 5k.
%   But, in a more basic case, running.
%   Extends to drawing.

%   In the case of concluding, have reasoning.
% \end{note}


%%% Local Variables:
%%% mode: latex
%%% TeX-master: "master"
%%% TeX-engine: luatex
%%% End:
