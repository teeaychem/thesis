\begin{note}
  Express the idea using familiar ideas used to model propositional attitudes via modal logic.

  Set of possible worlds.
  Set of propositional atoms, \(\{P,Q,R,\dots\}\)
  Valuation function, which maps propositional atoms to possible worlds.
  For example \dots

  World \(w\).
  Say whether something is true or not \(w \vDash P\).
  And, build up complex expressions using clauses.
  \(w \vDash P \land Q\) iff \(w \vDash P\) and \(w \vDash Q\).

  Belief.
  Collection of worlds which are candidates for the actual world.

  Now, say whether or not an agent believes something \(w \vDash B\phi\) iff \(\phi\) at all most plausible worlds.

  So, this gives us an account of what the agent believes, but not from the agent's point of view.

  Now, semantics of belief.
  Some collection of epistemically plausible worlds.

  \(\mathcal{B}\).

  \(\mathcal{B} \vDash \phi\).

  So, here, on this understanding, agent's perspective from belief.

  Note, some propositions are neither true nor false.
\end{note}

\paragraph*{Quick argument}

\begin{note}
  More generally, a quick argument for witnessing is that conclusion needs to come from somewhere.
\end{note}

\begin{note}[The general pre-theoretic argument]
  At least, the positive resolution is not straightforward without placing some constraints on \emph{which} premises an agent may appeal to when reasoning, and we will motivate the positive resolution without such constraints.

  For, without constraints on which premises an agent may appeal to when reasoning, one may argue as follows:
  \begin{enumerate}
  \item
    Any instance of reasoning is some process with start and end points and intermediary steps.
  \item
    If an agent has concluded \(\phi\) has value \(v\) by some reasoning, then the reasoning has start points and intermediary steps.
  \item
    Hence, the agent has concluded \(\phi\) has value \(v\) by witnessing some reasoning from some start points via some intermediary steps.
  \item
    In other words, the agent has concluded \(\phi\) has value \(v\) by witnessing some reasoning from some premises.
  \end{enumerate}

  In short, so long as an agent has concluded \(\phi\) has value \(v\), the agent has always witnessed reasoning from some premises.

  \begin{enumerate}[resume]
  \item
    So, either the start points are the premises of interest mentioned in the issue, or the agent has concluded \(\phi\) has value \(v\) from a distinct set of premises.
  \end{enumerate}

  In other words, either the agent has witnessed reasoning from the premises of interest, or the premises of interest (and any reasoning from them) are not required to conclude \(\phi\) has value \(v\).
\end{note}

\begin{note}[More on the quick argument]
  The quick argument does not directly lead to a negative resolution to the issue.
  Still, the quick argument does suggest that any appeal to premises \emph{without} witnessing reasoning from those premises is redundant.

  Now, perhaps redundancy isn't so bad.
  I only need a single key to ensure I have the option of unlocking a door, but a second key is useful if the first is lost.

  Still, I take it to be the case that redundancy provides leverage for a wide range of arguments motivating a negative resolution in the case of reasoning.

  For, if appeal to some premises is redundant, then any argument that requires witnessing need only observe that a counterargument must find some role for something which is not needed.

  Reasoning is an event, and distinct way of concluding \(\phi\) has value \(v\) may be useful, it is unclear why the distinct way of concluding \(\phi\) has value \(v\) is of use when concluding \(\phi\) has value \(v\) from present premises.
  To push the analogy, a second key may have various uses, but the second key is irrelevant in the event of unlocking the door with the first key.
  That the second key is would unlock the door if the first was lost has no role in the event of unlocking the door with the first key.

  More generally, it may seem (and I suspect it does seem) intuitive that the issue should be resolved negatively.
  Reasoning just is obtaining a conclusion by witnessing reasoning from premises.
  And, if the quick argument succeeds, then there surely is some way to preserve the intuition.
\end{note}

\paragraph*{Summarising}

\begin{note}[Pre-theoretical constraint]
  Issue captures intuitive constraint of no account of why without an account of how.
  Witnessing a pre-theoretical constraint.

  Broader than causation.
\end{note}


%%% Local Variables:
%%% mode: latex
%%% TeX-master: "master"
%%% End:
