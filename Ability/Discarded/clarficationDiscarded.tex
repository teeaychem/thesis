\begin{note}
  Express the idea using familiar ideas used to model propositional attitudes via modal logic.

  Set of possible worlds.
  Set of propositional atoms, \(\{P,Q,R,\dots\}\)
  Valuation function, which maps propositional atoms to possible worlds.
  For example \dots

  World \(w\).
  Say whether something is true or not \(w \vDash P\).
  And, build up complex expressions using clauses.
  \(w \vDash P \land Q\) iff \(w \vDash P\) and \(w \vDash Q\).

  Belief.
  Collection of worlds which are candidates for the actual world.

  Now, say whether or not an agent believes something \(w \vDash B\phi\) iff \(\phi\) at all most plausible worlds.

  So, this gives us an account of what the agent believes, but not from the agent's point of view.

  Now, semantics of belief.
  Some collection of epistemically plausible worlds.

  \(\mathcal{B}\).

  \(\mathcal{B} \vDash \phi\).

  So, here, on this understanding, agent's perspective from belief.

  Note, some propositions are neither true nor false.
\end{note}

%%% Local Variables:
%%% mode: latex
%%% TeX-master: "master"
%%% End:
