\chapter{Notes}
\label{cha:notes}

\section{Specific accounts of basing}
\label{sec:spec-acco-basing}

\subparagraph{Basing and causation}

\begin{note}
  For a concrete instance, consider \citeauthor{Moser:1989tv}'s account of the basing relation:
  \begin{quote}
    \emph{S}'s believing or assenting to \emph{P} is based on his justifying propositional reason \emph{Q} \(=_{\text{df}}\) \emph{S}'s believing or assenting to \emph{P} is causally sustained in a nondeviant manner by his believing or assenting to \emph{Q}, and by his associating \emph{P} and \emph{Q}.\nolinebreak
    \mbox{}\hfill\mbox{(\cite*[157]{Moser:1989tv})}
  \end{quote}

  Suppose we have a conclusion from some premises and steps of reasoning.
  If the agent has not witnessed the relevant reasoning, then it seems the conclusion is not causally sustained in a nondeviant manner by his believing or assenting to the premises of the reasoning, nor has the agent associated the conclusion with the premises by witnessing the relevant steps of reasoning.

  As I've not witnessed, then no role for \emph{Q}, whatever that turns out to be.
\end{note}

\begin{note}
  This is a quick argument, and borders on conjecture, so let us examine the relevant association in greater detail.
  \citeauthor{Moser:1989tv} distinguishes between occurrent and non-occurrent satisfaction of association relations.

  We start with occurrent satisfaction of an association relation:
  \begin{quote}
    \emph{S} occurrently satisfies an association relation between \emph{E} and \emph{P} \(=_{\text{df}}\)
    \begin{enumerate*}[label=(\roman*), ref=(\roman*)]
    \item\label{moser:oar:i} S has a \emph{de re} awareness of \emph{E}'s supporting \emph{P}, and
    \item\label{moser:oar:ii} as a nondeviant result of this awareness, \emph{S} is in a dispositional state whereby if he were to focus his attention only on his evidence for \emph{P} (while all else remained the same), he would focus his attention on \emph{E}.\newline
    \mbox{}\hfill\mbox{(\Citeyear[141--142]{Moser:1989tv})}
    \end{enumerate*}
  \end{quote}

  \emph{de re} awareness is a non-propositional, direct awareness of \emph{E} supporting \emph{P}.
  (\Citeyear[142]{Moser:1989tv})
\end{note}

\begin{note}
  \ESU{} follows from~\ref{moser:oar:i}.
  \emph{de re} awareness, but this doesn't rule out use.
  \ESU{} does not require that the agent engages in propositional reasoning.

  In cases where the agent has not witnessed reasoning, there is no \emph{de re} awareness.
  Without the reasoning taking place, the agent is not directly aware of what the reasoning consists of.

  Following, the definition of non-occurrent satisfaction of an association relations is derived from occurrent satisfaction of an association relations by allowing~\ref{moser:oar:i} to be satisfied at some point in the past while requiring that~\ref{moser:oar:ii} continues to be satisfied in the present.
  As noted, \ESU{} is compatible with the agent having witnessed the reasoning at some point in the past.
  Therefore, \ESU{} is entailed given both occurrent and non-occurrent satisfaction of association relations
\end{note}

\begin{note}
  \citeauthor{Moser:1989tv} then, builds in quite a lot.
  More general point is that if you have intuitions about basing, then these also plausibly extend to \ESU{}.
\end{note}

\subparagraph{Basing and representation}

\begin{note}[Representationalism]
  \citeauthor{Neta:2019aa} generalises (purely) epistemic interest in basing relations to cover the explanatory relation between reasons and (rationally evaluable) states held, or actions performed, by an agent.

  On the way to a novel proposal, \citeauthor{Neta:2019aa} sketches a broad characterisation of representationalist theories of (generalised) basing:
  \begin{quote}
    \begin{enumerate}[label=(R\arabic*), ref=(R\arabic*)]
    \item\label{neta:RC:b} \emph{basing} C on R involves the agent's representing R as justifying C, and
    \item\label{neta:RC:jb} \emph{justifying basing} of C on R consists in the adroitness of this representation.\nolinebreak
      \mbox{}\hfill\mbox{(\Citeyear[192]{Neta:2019aa})}
    \end{enumerate}
  \end{quote}

  \begin{quote}
    \begin{enumerate}[label=(R\arabic*\('\)), ref=(R\arabic*\('\))]
    \item\label{neta:RC:jp} for an agent to C based on reason R involves not merely the agent's representing R as justifying C---it also involves \emph{this latter representation (or its content) being part of the reason why the agent C's}.\nolinebreak
      \mbox{}\hfill\mbox{(\Citeyear[197]{Neta:2019aa})}
    \end{enumerate}
  \end{quote}
  The added clause states that the relevant representation must explain why the agent formed a belief.
  Hence, given~\ref{neta:RC:jp} the agent would not be permitted to base their belief in the content of the speech given that they were swayed by emotion.
  Intuitively,~\ref{neta:RC:jp} expands on what it is for premises and steps of reasoning to be use when forming a belief.
  So, given that representation requires use, the expanded clause may be seen as focusing on \emph{how} the representation is used.
\end{note}


%%% Local Variables:
%%% mode: latex
%%% TeX-master: "master"
%%% End:
