\chapter{\fc{3} discard}
\label{cha:fc3-discard}

\begin{note}[\fc{2} definition]
  We define \(\pv{\phi}{v}\) being a \emph{\fc{0}} as follows:

  \begin{restatable}[\fc{3}]{definition}{definitionForegoneC}
    \label{def:fc}
    \begin{itemize*}[noitemsep, label=\(\circ\)]
    \item
      An agent: \vAgent{}
    \item
      A proposition: \(\phi\)
    \item
      A value: \(v\)
    \item
      A \poP{0}: \(\Phi\)
    \item
      \mbox{ }
    \end{itemize*}

    \begin{itemize}
    \item
      \(\pv{\phi}{v}\) from \(\Phi\) is a \emph{\fc{0}} for \vAgent{}.
      % \newline
      % \mbox{ }\hfill(equiv.\ \(\pvp{\phi}{v}{\Phi}\) is a \fc{0} for \vAgent{})
    \end{itemize}
    \emph{If and only if}
    \begin{enumerate}[label=]
    \item
      Both~\ref{def:fc:is-pe-good} and~\ref{def:fc:no-pe-bad} are true:
      \begin{enumerate}[label=\alph*., ref=(\alph*)]
      \item
        \label{def:fc:is-pe-good}
        There is \emph{some} \pevent{} \(p\) in which \vAgent{} concludes \(\pv{\phi}{v}\) from \(\Phi\).
      \item
        \label{def:fc:no-pe-bad}
        There is \emph{no} \pevent{} \(p\) in which \vAgent{} concludes some proposition-value-premises pairing which is incompatible with concluding \(\pv{\phi}{v}\) from \(\Phi\).%
        \footnote{
          Incompatible:
          \(\pv{\chi}{v''}\) from \(X\) where:
        If conclude \(\pv{\chi}{v''}\) from \(X\), then does not conclude \(\pv{\phi}{v}\) from \(\Phi\).
        }
      \end{enumerate}
    \end{enumerate}
    \vspace{-\baselineskip}
  \end{restatable}
\end{note}

\begin{note}[Intuition]
  Significant attention will be given to what is means for there to be a potential event in which an agent performs some action in \autoref{cha:sec:fcs-def:potential-events}, below.
  However, the basic features of \autoref{def:fc} follow from substituting `possible' for `\pevent{}'.
  Given this substitution:

  \begin{itemize}[noitemsep]
  \item
    Clause~\ref{def:fc:is-pe-good} ensures that there is some possibility in which the agent to conclude \(\pv{\phi}{v}\) from \(\Phi\).
  \item
    Clause~\ref{def:fc:no-pe-bad} ensures that there is no possibility in which the agent concludes something incompatible with concluding \(\pv{\phi}{v}\) from \(\Phi\).
    Hence, Clause~\ref{def:fc:no-pe-bad} rules out the agent failing to conclude \(\pv{\phi}{v}\) from \(\Phi\) because some incompatible proposition-value-premises pairing is (also) a \fc{0}.
  \end{itemize}

  Intuitively, and inevitable result of possible reasoning.
  For, there is some possibility in which the agent concludes \(\pv{\phi}{v}\) from \(\Phi\) via Clause~\ref{def:fc:is-pe-good}.
  And, at no point prior to concluding could the agent have concluded some other proposition-value-premises pairing which would prevent the agent from concluding \(\pv{\phi}{v}\) from \(\Phi\) via~\ref{def:fc:no-pe-bad}.

  Further, not merely that agent would not prevent on success, but that there is no way to block success.
\end{note}

\subsection{\pevent{3}}
\label{cha:sec:fcs-def:potential-events}

\begin{note}
  \autoref{def:fc} appeals to~\ref{def:fc:is-pe-good} the existence of some \pevent{} and~\ref{def:fc:no-pe-bad} the non-existence of some \pevent{}.

  However, \autoref{def:fc} does not rely on anything more than existential quantification.
  The choice is deliberate.
  We given necessary and sufficient conditions for the \emph{existence} of some \pevent{} in terms of
  \begin{enumerate*}[label=(\roman*)]
  \item
    actions available to the agent, and
  \item
    truth conditions for the progressive.
  \end{enumerate*}
\end{note}


%%% Local Variables:
%%% mode: latex
%%% TeX-master: "master"
%%% End:
