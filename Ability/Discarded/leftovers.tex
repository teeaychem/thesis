
\chapter{Leftovers}
\label{cha:leftovers}



\begin{note}
  Framework developed to do one thing.
  Counterexamples to \issueInclusion{}.

  \fc{3} are central.
  Function of \fc{1} is to get \ros{} without \wit{}.
  In turn, appeal to \fc{1} when defining \requ{1} and so on is to have something which gets \ros{}.

  Of course, abstraction.
  What matters is whatever it is that ensures \(\pv{\phi}{v}\) is a \fc{} from \(\Phi\), rather than \(\pv{\phi}{v}\) \emph{being} a \fc{} from \(\Phi\).
\end{note}

\begin{note}
  So, other sufficient conditions for a \ros{}.

  One way is to weaken the definition of a \fc{}.
  Here, we had immediate actions.
  Though, consider student taking an exam.
  No action, but still possible in some weaker sense.
\end{note}


\begin{note}
  Applications aside from counterexamples to \issueInclusion{}.

  For example, transferability.
  What's at issue here is that complete proof is a \fc{} for reader with sufficient background.
\end{note}


\begin{note}
  \agents{2} reasons.
\end{note}



%%% Local Variables:
%%% mode: latex
%%% TeX-master: "master"
%%% TeX-engine: luatex
%%% End:
