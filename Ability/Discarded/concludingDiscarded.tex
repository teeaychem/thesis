\paragraph{Concluding is `description-free'}

\begin{note}[Descriptions]
  An important assumption is that an agent need not recognise that the culmination of some instance of reasoning is that some proposition has some value.

  \begin{assumption}[Concluding is `description-free']
    \label{assu:conc:d-free}
    It is not the case that an agent concludes \(\phi\) has value \(v\) only if the agent concludes \(\phi\) has value \(v\) under some description \emph{d}
  \end{assumption}

  In particular,~\autoref{assu:conc:d-free} holds for any description \emph{d} which includes an intensional reading of `\(\phi\) has value \(v\)'.
  More generally, it is possible for an agent to conclude \(\pv{\phi}{v}\) without consciously or otherwise entertaining either `\(\phi\)' or `\(v\)'.%
  \footnote{
    Compare with, for example, \citeauthor{Anscombe:1957aa} on intention action (\citeyear[\S19]{Anscombe:1957aa}) and \citeauthor{Davidson:1963aa} on primary reasons (\citeyear[5]{Davidson:1963aa}).
  }

  For the moment we will focus on intensionality.
  We will briefly explain the distinction between intensional and non-intensional readings, and motivate~\autoref{assu:conc:d-free} with respect to a handful of examples.

  Note, however, we are not providing an analysis of what it is for an agent to conclude \(\pv{\phi}{v}\).
  Rather,~\autoref{assu:conc:d-free} narrows down the particular sense of `concluding' of interest to us.
  There may be, and plausibly is, a sense of `concluding' for which~\autoref{assu:conc:d-free} does not hold (and in particular where the proposition-value pair of the conclusion is always intensional).
  However, our interest is with a sense of `concluding' for which~\autoref{assu:conc:d-free} holds.
\end{note}

\begin{note}[Looking ahead]
  Looking ahead, briefly,~\autoref{assu:conc:d-free} is a key assumption for developing tension.
  Tension will not follow from any reading of `concluding' in which in concluding \(\pv{\phi}{v}\) an agent concludes \(\pv{\phi}{v}\), and \(\pv{\phi}{v}\) alone.
  Rather, the tension we develop will involve an agent concluding \(\pv{\psi}{v'}\) when concluding \(\pv{\phi}{v}\).

  Now, our discussion of intensionality will suggest an agent may conclude \(\pv{\phi}{v}\) when concluding \(\pv{\varphi}{v}\) when there is significant overlap with what \(\phi\) and \(\varphi\) refer to.
  Still,~\autoref{assu:conc:d-free} allows that in concluding \(\pv{\phi}{v}\) an agent may also conclude \(\pv{\psi}{v}\), where there is no overlap between the reference of \(\phi\) and \(\psi\).
  When developing tension we will have interest with sufficient overlap with what \(\phi\) and \(\varphi\) refer to.
  So, we will not require~\autoref{assu:conc:d-free} in full generality, but equally it is only after developing the tension of interest that we will see in how~\autoref{assu:conc:d-free} may be restricted.
\end{note}

\begin{note}[Intensionality]
  Consider the following observation from~\citeauthor{Quine:1943vf}:

  \begin{quote}
    \begin{enumerate}[label=(\arabic*)]
    \item
      Giorgione = Barbarelli,
    \item
      Giorgione was so-called because of his size
    \end{enumerate}
    are true; however, replacement of the name 'Giorgione' by the name 'Barbarelli' turns (2) into the falsehood:

    \begin{center}
      Barbarelli was so-called because of his size.
    \end{center}
    \vspace{-\baselineskip}
    \mbox{ }%
    \mbox{}\hfill\mbox{(\citeyear[113]{Quine:1943vf})}
  \end{quote}

  Both `Giorgione was so-called because of his size' and `Barbarelli was so-called because of his size' are intensional in the sense that the truth value of each expression does not reduce to the reference of the expression's' components.
  Else, as `Giorgionee' and `Barbarelli' are co-referential, the predicate `was so-called because of his size' would apply equally to both `Giorgionee' and `Barbarelli'.

  Now, it is also not the case that in concluding `Giorgione was so-called because of his size', one also concludes `Barbarelli was so-called because of his size'.

  However, this observation is \emph{not} immediate from observing that the agent concluded `Giorgione was so-called because of his size' and did not conclude `Barbarelli was so-called because of his size'.

  The conclusion `Giorgione was so-called because of his size' may be intensional, but being intensional is not a property granted to some proposition-value pair by virtue of the proposition-value pair being a conclusion.

  For example, in concluding `Giorgione is large' the agent may also conclude `Barbarelli is large'.
  Indeed, the expression `Giorgione is large' may be read non-intensionally, and hence is true if and only if `Barbarelli is large', given that `Giorgionee' and `Barbarelli' are co-referential.

  Likewise, in concluding `\(2 + 2 = 4\)', an agent may also conclude `\(4 = 2 + 2\)'.
  Or, in concluding `\nagent{1} is shorter than \nagent{2}', an agent may also conclude `\nagent{2} is taller than \nagent{1}'.
  Though, in concluding `\nagent{1} is shorter than \nagent{2}' an agent may also fail to conclude `\nagent{2} is taller than \nagent{1}'.
  For example, if the relevant agent is not aware of the relationship between `shorter' and `taller'.
\end{note}

\begin{note}
  More broadly, I consider it intuitive that concluding any one of the following includes concluding any other:
  \begin{itemize}
  \item \(\phi\) has value \(v\).
  \item It is true that \(\phi\) has value \(v\).
  \item It is not the case that \(\phi\) does not have value \(v\).
  \item It is true that it is not the case that \(\phi\) does not have value \(v\).
    \begin{center}
      \(\vdots\)
    \end{center}
  \end{itemize}

  There are an infinite number of distinct proposition-value pairs that may be generated along these lines, but these distinct proposition-value pairs do not amount to distinction conclusions.
  A conclusion for one is a conclusion for all.
\end{note}


\begin{note}
  Indeed, we may form an explicit assumption governing certain proposition-value pairs.
  We start with the definition of \indicateN{}.
\end{note}

\begin{note}
  \begin{restatable}[\indicateN{2}]{definition}{defIndicate}
    \label{def:indication}
    \(\phi\) having value \(v\) \emph{\indicateV{1}} \(\psi\) has value \(v'\) if and only if:
    \begin{itemize}
    \item
      It is not \epPAd{}, from \vAgent{}' epistemic state, that \(\psi\) has value \(v'\) while \(\psi\) does not have value \(v'\).
    \end{itemize}
    \vspace{-\baselineskip}
  \end{restatable}
\end{note}

\begin{note}
  The assumption now holds that an agent concludes \(\pv{\psi}{v'}\) when concluding \(\pv{\phi}{v}\) just in case \(\pv{\psi}{v'}\) and \(\pv{\phi}{v}\) co-\indicateV{}.

  \begin{restatable}[\indicateN{2}]{assumption}{assuIndicate}
    \label{assu:indication}
    If \(\pv{\phi}{v}\) \indicateV{1} \(\pv{\psi}{v'}\), and \(\pv{\psi}{v'}\) \indicateV{1} \(\pv{\phi}{v}\), then:

    \begin{itemize}
    \item
      \vAgent{} concludes \(\pv{\phi}{v}\) just in case \vAgent{} concludes \(\pv{\psi}{v'}\).
    \end{itemize}
    \vspace{-\baselineskip}
  \end{restatable}
\end{note}

\begin{note}
  \Autoref{assu:indication} is mild closure condition on claiming support.
  Still, \autoref{assu:indication} is only a closure condition with respect to an agent's epistemic state.
  To illustrate:
  Suppose an agent has concluded that The Scarlet Pimpernel rescued Marquis de Lafayette.
  As `The Scarlet Pimpernel' and `Sir Percy Blakeney' are co-referential, it may be that the agent's conclusion \indicateV{1} that Sir Percy Blakeney' rescued Marquis de Lafayette.
  However, the conclusion will \indicateN{0} \emph{only if} there are no \epPW{1} in which `The Scarlet Pimpernel' and `Sir Percy Blakeney' refer to different individuals.

  Likewise, an agent may conclude that helping The Scarlet Pimpernel is desirable, without the conclusion \indicatePr{} that helping Sir Percy Blakeney is desirable.

  Indeed, an agent's conclusion that it is raining, while standing in Gower Street, may fail to \indicateN{} that it is raining in London if the agent considers it \epPAd{} that they are not in London.
\end{note}

\begin{note}
  Generalising,~\autoref{assu:indication} may be strengthened by weakening the restriction to `If \(\pv{\phi}{v}\) \indicateV{1} \(\pv{\psi}{v'}\)'.%
  \footnote{
    I.e.\ only the first conjunct of the restriction given.
  }

  Hence, in concluding \(\pv{\phi}{v}\) an agent would also conclude any proposition-value pair \(\pv{\psi}{v'}\) \emph{weaker} than \(\pv{\phi}{v}\), from the agent's epistemic state.
  Indeed, this leads to a much stronger closure condition on concluding.%
  \footnote{
    Consider by parallel closure of knowledge under known entailment:
    \begin{itemize}
    \item If an agent knows that \(\phi\) has value \(v\) only when \(\psi\) has value \(v'\), then if the agent knows \(\phi\) has value \(v\), then the agent knows \(\psi\) has value \(v'\).
    \end{itemize}
    This closure condition differs in forms, as it concerns knowledge as a state, by may be reformulated to a closer parallel:
        \begin{itemize}
    \item If an agent knows that \(\phi\) has value \(v\) only when \(\psi\) has value \(v'\), then in coming to know \(\phi\) has value \(v\) the agent comes to know \(\psi\) has value \(v'\).
    \end{itemize}
  }

  We will not assume this stronger variant of~\autoref{assu:indication} holds for the sense of `concluding' we are interested in.
  Rather, we have seen how~\autoref{assu:conc:d-free} allows for the possibility of an agent concluding \(\pv{\psi}{v'}\) when concluding \(\pv{\phi}{v}\), and with the exception of~\autoref{assu:indication} which we take to be sufficiently intuitive, we only advance argument in cases of interest.
\end{note}

\begin{note}
  Indeed, though the stronger variant of~\autoref{assu:indication} may be intuitive, there are certain issues we would like to avoid taking a stance on.
  In particular, whether concluding some statement which quantifiers over various objects includes concluding for each object the quantifier applies to.

  For example, suppose I have conclude that there are infinitely many primes.
  Reflecting a little on the natural numbers, I observe that if there are infinitely many primes, then for every natural number \(n\) there is some prime larger than \(n\) (for, the natural numbers are not dense).
  Hence, for every natural number \(n\) there is some prime larger than \(n\).

  On the stronger variant of~\autoref{assu:indication}, concluding `for every natural number \(n\) there is some prime larger than \(n\)' would also include concluding there is some prime larger than \(n\) for each \(n\).
  E.g.\ there is some prime larger than \(1\), \(16\), \(5^{43}\), \(53!^{793}\), \(54!^{794!}\), and so on\dots

  Specifically, looking ahead to tension, concluding that one has the general ability to witness some kind of reasoning would involve concluding that one has each specific instance of the general ability.
\end{note}

\begin{note}[Witnessing]
    While~\autoref{assu:conc:d-free} focuses on concluding, we take~\autoref{assu:conc:d-free} to apply equally to witnessing reasoning in which an agent concludes.
  Indeed, if a negative resolution to {\color{red} issue:Main} then any instance of concluding is also an instance of witnessing reasoning to the relevant conclusion, and hence~\autoref{assu:conc:d-free} would be in conflict with an assumption which states that an agent only witnesses reasoning which concludes \(\pv{\phi}{v}\) under some description.
  On the other hand, a positive resolution to {\color{red} issue:Main} does not ensure any instance of concluding is also an instance of witnessing reasoning to the relevant conclusion.
  However, we arrive at the same conflict with respect to any instance of concluding which is the result of witnessing reasoning to the relevant conclusion.

  Indeed, if \(\pv{\phi}{v}\) \indicatePr{} \(\pv{\psi}{v'}\), then \(\phi\) has value \(v\) \emph{only if} \(\psi\) has value \(v'\) (from the perspective of the agent's epistemic state).
  Hence, \emph{if} in concluding \(\pv{\phi}{v}\) an agent also concludes some \indicateVed{} \(\pv{\psi}{v'}\), then the relevant premises which allow the agent to also conclude \(\pv{\psi}{v'}\).
\end{note}

\begin{note}[Perspective on issue]
  Looking ahead, perspective on issue.
  In some cases, concluding one would conclude \(\pv{\phi}{v}\) from some premises \(\Phi\) is equivalent to concluding \(\pv{\phi}{v}\) from \(\Phi'\), where \(\Phi\) and \(\Phi'\) are distinct.
\end{note}

\paragraph{Premises}

\begin{note}[Understanding `having value \(v\)']
  In a deductive case, if the premises are true, then the conclusion is true.
  Means-end reasoning for desire.
  The value is important.
  If it is true that it past 6pm, then it is true the shop is closed.
  Provides value of shop being closed.

  However, if agent desires that it is past 6pm, then it doesn't follow that the agent desires that the shop is closed.
  Question an agent as to why they think their desires conform to truth --- is-ought problem.

  Means-end reasoning.
  It is true that there is cheese at the centre of the maze.
  And, it is desirable that I obtain the cheese at the centre of the maze.
  Further, it is true that I may only obtain the cheese at the centre of the maze by solving the maze.
  Therefore, it is desirable that I solve the maze.
\end{note}

\paragraph{Values and propositions}

\begin{note}[Value proposition]
  Reasoning and claims to support focus.
  Briefly introduce a pair of propositions to clarify claim to support and reasoning.

  \begin{restatable}[Claimed support is for a proposition having some value]{assumption}{assuCSVP}
    \label{assu:CSVP}
    When an agent concludes \(\phi\) has value \(v\), the agent assigns value \(v\) to the \world{1} described by \(\phi\).
    (Where propositions individuate \world{1} from the perspective of the agent.)
  \end{restatable}
\end{note}

%%% Local Variables:
%%% mode: latex
%%% TeX-master: "master"
%%% End:
