\chapter{\curb{3}}
\label{cha:zS:sec:curbs}

\begin{note}
  Idea with a \curb{} is that this is a necessary condition for event to be develop such that agent concludes.
\end{note}

\begin{note}
  Turn our attention to concluding.

  When concluding, certain things may happen.
  Agent gets interrupted, distracted, etc.

  In these cases, the agent is concluding, it just so happens that the event is interrupted.

  However, the agent may fail to be concluding.

  Reasoning doesn't work out.

  Assuming some competency no agent is concluding that White has a checkmate.
  For, it is not possible.

  Likewise, may be the case that agent has the option of concluding something else.
  Start with the definition of a \bCurb{}.
  Expand to \curb{}, and link \curb{1} to \fc{1}.
\end{note}

\section{\curb{3}}
\label{sec:curb}

\begin{note}
  Being with the idea of a \curb{}.
  Relates whether or not concluding to proposition-value-premise pairings.
\end{note}

\begin{note}
  \begin{definition}[A \curb{0}]
    \label{def:curb}
    For:

    \begin{itemize*}[noitemsep]
    \item
      An agent: \vAgent{}
    \item
      Propositions: \(\phi\), \(\psi\)
    \item
      Values: \(v\), \(v'\)
    \item
      \poP{3}: \(\Phi\), \(\Psi\)
    \item
      An event: \(e\)
    \item
      \mbox{ }
    \end{itemize*}

    \begin{itemize}
    \item
      \(\pvp{\phi}{v'}{\Psi}\) is a \emph{\curb{}} on \(e\) just in case:

      \begin{enumerate}[label=\alph*., ref=(\alph*), series=curbDefSeries]
      \item
        \(e\) is an event in which \vAgent{} is concluding \(\pv{\phi}{v}\) from \(\Phi\).
      \end{enumerate}

      \emph{Only if}

      % \begin{enumerate}
      % \item [\emph{If}:]
      \begin{enumerate}[label=\alph*., ref=(\alph*), resume*=curbDefSeries]
      \item
        \label{def:curb:opp}
        There is a \pevent{} from \(e\) in which \vAgent{} concludes \(\pv{\psi}{v'}\) from \(\Psi\).
      \end{enumerate}
      % \item[\emph{Then}:]
      %   \begin{enumerate}[label=\alph*., ref=(\alph*), resume]
      %   \item
      %     \label{def:curb:fail}
      %     \(e\) does not develop in an event in which \vAgent{} concludes \(\pv{\phi}{v}\) from \(\Phi\).%
      %     \mbox{ }\hfill(\emph{Due to}~\ref{def:curb:opp})
      %   \end{enumerate}
      % \end{enumerate}
    \end{itemize}
    \vspace{-\baselineskip}
  \end{definition}

  {
    \color{red}
    Only if, because don't get entailment from \pevent{} to concluding, that's far too strong.
  }

  So, \curb{1} are with respect to \emph{concluding}.
  In order for the event in which the agent concludes to be in progress, then must be \pevent{} from \(e\).

  Clearest way to understand this is by considering how an event in which agent concludes develops.
  At the start, the only thing for sure is \pevent{} in which the agent concludes \(\pv{\phi}{v}\) from \(\Phi\).

  However, as the event develops, what is required becomes clearer.
  So, certain sub-conclusions are necessary.

  In general, no matter how things turn out in the details, what is common between all of the ways in which the event fully develops so that the agent concludes.
\end{note}

\begin{note}
  Now, in case of a \curb{}, it is not immediately clear that there are instances in which the relevant \pevent{} is not a sub-event of the event in which the agent concludes \(\pv{\phi}{v}\) from \(\Phi\).

  However, the definition makes sense.
  Indeed, it is very clear when applied to concluding.
\end{note}

\begin{note}
  This is a curb.
  Idea is simple, the agent may wonder about \(\pv{\phi}{v}\) from \(\Psi\).
  And, the agent may attempt to conclude.
  But, if doesn't conclude, then doesn't conclude \(\pv{\phi}{v}\) from \(\Phi\).
\end{note}

\begin{note}
  The parenthetical `due to' appended to~\ref{def:curb:fail} ensures that the agent not concluding \(\pv{\phi}{v}\) from \(\Phi\) is tied to failing to conclude \(\pv{\psi}{v'}\) from \(\Psi\) after taking the relevant opportunity.
  Don't have a specific account of `due to'.
  Move to the level of theories, and overall goal is to provide a theory independent motivation for rejecting \issueConstraint{}.
  So, some difficulty, may wonder whether the conditional holds.
\end{note}

\section{\illu{3}}
\label{sec:illu3}

\subsection{Instances of concluding}
\label{sec:instances-concluding}

\subsubsection{Positive}
\label{sec:positive}

\paragraph{Trivial}

\begin{note}
  \begin{proposition}
    For an agent \vAgent{}, proposition-value pair \(\pv{\phi}{v}\), and \poP{} \(\Phi\):

    \begin{itemize}
    \item
      \(\pvp{\phi}{v}{\Phi}\) is a \curb{} of \vAgent{} concluding \(\pv{\phi}{v}\) from \(\Phi\).
    \end{itemize}
    \begin{argument}
      Immediate by \autoref{def:curb}.
      For, assume there is no \pevent{} in which \vAgent{} concludes \(\pv{\phi}{v}\) from \(\Phi\).
      Then, it is not the case that \(e\) develops into an event in which \vAgent{} concludes \(\pv{\phi}{v}\) from \(\Phi\).
      Hence, \(e\) is not an event in which \vAgent{} is concluding \(\pv{\phi}{v}\) from \(\Phi\).

      So, the conditional by which \curb{1} are defined is always true for \(\pv{\phi}{v}{\Phi}\).
    \end{argument}
  \end{proposition}

  For a specific example, chess.
\end{note}

\paragraph{Difficult}

\subsubsection{Notes}
\label{sec:notes}


\section{\curb{3} and \fc{1}}

\begin{note}
  \begin{proposition}[\curb{3} and \fc{1}]
    For an agent \vAgent{}, and proposition-value-premises pairings \(\pvp{\phi}{v}{\Phi}\), \(\pvp{\psi}{v'}{\Psi}\):

    \begin{itemize}
    \item
      \(\pvp{\phi}{v'}{\Psi}\) is a \emph{\curb{}} of concluding \(\pv{\phi}{v}\) from \(\Phi\)
    \end{itemize}

    \emph{Not If and only if}

    \begin{itemize}
    \item
      \(\pv{\psi}{v'}\) is a \fc{} of \(\Psi\).
    \end{itemize}
    \begin{argument}
      It is not the case that the b condition of \fc{} holds.
    \end{argument}
  \end{proposition}
\end{note}

% \begin{note}
%   In some cases, it may be the case that \(\pv{\phi}{v}\) follows trivially after getting \(\pv{\psi}{v'}\), this doesn't matter.

%   It may also be the case that \(\pv{\phi}{v}\) entails \(\pvp{\psi}{v'}{\Psi}\), but this is also fine, so long as there is an alternative way.
% \end{note}

\section{A missing link}
\label{cha:zS:sec:missing-link}

\begin{note}
  \autoref{prop:sCing} is important.
  For, what matters for the conclusion.
  If agent concludes \(\pv{\phi}{v}\) from \(\Phi\), then concluding \(\pv{\phi}{v}\) from \(\Phi\).

  So, suppose, \curb{}.
  Then, need it to be the case that \(\pv{\psi}{v'}\) from \(\Psi\) is a \fc{}.

  In this respect, \fc{} explains, in part, and in some sense, why the agent concludes \(\pv{\phi}{v}\) from \(\Phi\).

  However, that \(\pv{\psi}{v'}\) from \(\Psi\) is a \fc{} does not answer, in part, \qWhyV{}.
  For, no \ros{}.
  \ros{} from \agpe{}.
  Only get this from \(\pv{\psi}{v'}\) from \(\Psi\) being a \fc{}, from \agpe{}.

  In this respect, \curb{1} are compatible with \issueConstraint{}!
\end{note}

\section{Summary}
\label{cha:zS:sec:curbs:summary}


%%% Local Variables:
%%% mode: latex
%%% TeX-master: "master"
%%% TeX-engine: luatex
%%% End:
