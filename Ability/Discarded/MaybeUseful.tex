\section{(Aside) Circularity}
\label{sec:aside-circularity}


\paragraph{\citeauthor{Sgaravatti:2013wu} on circularity}

{
  \color{red}
  The main interest of this is simultaneous conclusions.
  Here, pointing out that we don't get circularity.
}

\begin{note}
  \zS{} is not about circular reasoning in the sense that the term `circularity' suggests that the reasoner has taken the conclusion of the reasoning for granted.

  Indeed, \iRequ{} is constructed in such a way that rules out this possibility.
  Though, not \emph{to} rule out this possibility --- worry is rather\dots

  Of interest because some similarity.
\end{note}

\begin{note}
  Consider what \citeauthor{Sgaravatti:2013wu} terms the `Justification Account' of circularity.\nolinebreak
  \footnote{
    As \citeauthor{Sgaravatti:2013wu} notes, the Justification Account of circularity is a rewriting of the third type of `epistemic dependence' considered by \citeauthor{Pryor:2004ws}~(\citeyear[359]{Pryor:2004ws}).
    Neither \citeauthor{Pryor:2004ws} nor \citeauthor{Sgaravatti:2013wu} endorse the Justification Account, but I take the spirit of the account to sufficient for interest.
    Still, the considerations which follow also apply to distinguish the {\color{red} problem identified} from \citeauthor{Sgaravatti:2013wu}'s favoured account (\citeyear[\S3]{Sgaravatti:2013wu}) and the fifth type of `epistemic dependence' considered by \citeauthor{Pryor:2004ws}~(\citeyear[359]{Pryor:2004ws}).
  }

  \begin{quote}
    \begin{enumerate}[label=(JA), ref=(JA)]
    \item
      \label{sg:JA}
      An argument is circular if and only if for you to have justification to believe the premisses, it is necessary that you have justification to believe the conclusion.%
      \mbox{}\hfill\mbox{(\citeyear[754]{Sgaravatti:2013wu})}
    \end{enumerate}
  \end{quote}
  Where `justification to believe' is to be read as in terms of having formed the belief in an epistemically appropriate way as opposed to (merely) possessing sufficient resources to form formed the belief in an epistemically appropriate way.\nolinebreak
  \footnote{
    Or, however you prefer to characterise \citeauthor{Firth:1978vi}'s (\citeyear{Firth:1978vi}) distinction between doxastic and propositional justification (or warrant).
    See also \citeauthor{Silva:2020aa} (\citeyear{Silva:2020aa}) --- esp.\ fn.\ 1.
  }
  (\citeyear[754--755]{Sgaravatti:2013wu})
\end{note}

\begin{note}[\citeauthor{Sgaravatti:2013wu} on necessity]

  \begin{quote}
    For my present purposes it will suffice to say that a good test of A's being necessary for B (and thus of B's being sufficient for A) is the satisfaction of two subjunctive conditionals.
    First, if A did not hold, B would not hold; secondly, if B were to hold, A would hold.\newline
    \mbox{ }\hfill\mbox{(\citeyear[761]{Sgaravatti:2013wu})}
  \end{quote}
  This is similar to what is captured by a \requ{}.

  Important difference is the second subjunctive conditional.
  This does not need to hold.

  In the case of parallel \requ{}, we get if B did not hold, A would not hold.

  \emph{Still, in general this is not a premise conclusion relation}
  But, in certain cases it is?
  No, in general it can't be, because then would have a failing of a \requ{}.
  For, conclusion would need to be premise and vice-versa.
\end{note}
