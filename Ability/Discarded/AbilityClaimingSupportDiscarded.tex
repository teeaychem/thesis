\chapter{Discarded from claiming support chaper}

\paragraph{Internalism}

\begin{note}
  Agent's own state.
  Consistency.

  Does not think that some proposition-value pair appealed to has some other value.
  Conclusion, then reject premises.
  If premise, then something that the agent thinks is not the case.
\end{note}

\begin{note}
  \color{red}
  Internalist flavour, specifically `mentalism' in the sense of \citeauthor{Feldman:2001uy} where:
  \begin{quote}
    \dots a person's beliefs are justified only by things that are \dots internal to the person's mental life.\nolinebreak
    \mbox{}\hfill\mbox{(\citeyear[233]{Feldman:2001uy})}\nolinebreak
    \footnote{
      See also~\textcite[\S4,9]{Pappas:2017vi}.
    }
  \end{quote}
  Result of reasoning, and indication is part of that reasoning.
  Reasoning is internal to an individual's mental life, and hence internalist in this sense.\nolinebreak
  \footnote{
    \color{red}
    Intuitively, external circumstances may impact the \emph{support} the agent has.
    However, as these are external, it seems external circumstances does not impact \emph{claiming} support.

    This is how you get puzzles for externalism.
    In both cases, it's fine for the agent to claim support, but the external circumstances impact whether the agent \emph{has} support.
    The internalist/externalist divide would seem to affect the conditions on claiming.

    Way to expand on this is reconstructing bootstrapping examples with and without \eiS{}.
    If the agent would only get basic support if reliable, then it's not clear that bootstrapping is a problem.
  }

  However, the internal/external divide is somewhat difficult.
  For, what is appealed to in an instance of reasoning that amounts to claiming support need not be internal.
\end{note}

\begin{note}[Illustration of I/E]
  \color{red}
  To illustrate, post has arrived.
  Heard the something drop from the letterbox.
  Something dropped.
  Could be a flyer.
  However, sign on the window and penalty.

  Reasoning, but appeal to things `external' to one's mental life.
  First, to something having dropped.
  Concern is that it is not the post.
  Appeal to the sign in the window.
  Not claiming support because I have reasoned about these things.
  Rather, claiming support because of what these considerations suggest is the case independently of my reasoning.

  In other words, it may be that I claim support that the post has arrived because I have reasoned about these things.
  Hence, claiming support is internal to my mental life.
  However, if I did not reason about the flyer and the sign in the window (or similar things) then the reasoning would not be an instance of claiming support.
  Hence, what matters when claiming support may be external to my mental life.
\end{note}

\subsubsection{\ideaCSC{1}}
\label{sec:ideaCSC}

\begin{note}
  \ideaCS{}, necessary condition for an instance of reasoning to be an instance of claiming support.
  In particular, \ideaCS{} focuses on steps of reasoning.

  With \ideaCSC{} we observe a consequence of \ideaCS{}.
  This takes the form of a structural constraint on reasoning.
  However, \ideaCSC{} is not motivated by structural considerations.
  Rather, observe that substantive constraint of \ideaCS{} entails structural constraint.
\end{note}

\begin{note}
  \begin{itemize}
  \item State idea.
  \item Clarify.
  \item Why follows from \ideaCS{}.
  \item Circularity.
  \end{itemize}
\end{note}

\begin{note}
  Basic idea.
  If some step which required \(\psi\) to have value \(v'\), yet \(\psi\) not having value \(v'\) was \epVAd{} when making the step, then from consequence of step to \(\psi\) having value \(v'\) is no good.
\end{note}

\begin{note}
  \begin{restatable}[\ideaCSC{0} --- \ideaCSC{}]{idea}{iCSC}
    \label{idea:cs:imp-stp}
    An instance of reasoning is an instance of claiming support \emph{only if} the reasoning does not contain any \impotent{0} steps.
  \end{restatable}
\end{note}

\begin{note}
  Where:
  \begin{restatable}[\impotent{2} steps of reasoning]{notion}{defImpotence}
    A step of reasoning \(\delta\), made by agent \vAgent{}, from premises \(\chi_{1},\dots,\chi_{j}\) having values \(v_{1}, \dots, v_{j}\) to conclusion \(\psi\) having value \(v'\) is \emph{\impotent{}} just in case:

    \begin{enumerate}[label=\arabic*., ref=(\arabic*)]
    \item \(\psi\) not having vale \(v'\) is \epVAd{} prior to making \(\delta\).
    \item Some premise \(\chi_{i}\) having value \(v_{i}\) is obtained from a prior step of reasoning \(\delta'\) such that:
      \begin{enumerate}[label=\alph*., ref=(2\alph*)]
      \item
        \label{idea:CS:B:step:requ}
        \(\psi\) having value \(v\) is a \requ{} of \(\delta'\).
      \end{enumerate}
    \end{enumerate}
    \vspace{-\baselineskip}
  \end{restatable}
\end{note}

\begin{note}
  Focusing in on the structure of some instance of reasoning.
\end{note}

\begin{note}
  \impotent{} because, does not go anywhere.
\end{note}

\begin{note}
  \color{red}
  Note, it's not the case that making step \(\delta\) alone is sufficient for a problem.
  For, agent may go on to conclude that \(\psi\) has value \(v'\) from some independent instance of reasoning.
\end{note}

\begin{note}
  If \autoref{idea:CS:B} is satisfied, then reasoning.
  \autoref{idea:cs:imp-stp} is weaker.

  So, this is a structural constraint.
  Motivated by foundational constraint.
\end{note}

\subsubsection{A trip to the zoo}

\begin{note}
  \illu{3} \ref{illu:CS:spot-the-diff} and \ref{illu:CS:wheres-wally} looked at reasoning and \requ{1}.
  Final \illu{0} is no different, structurally similar to \autoref{illu:CS:wheres-wally}.
  However, surrounding discussion to clarify details.

  After discussion, a corollary and a conjecture.
\end{note}

\begin{note}
  Zebra.

  Dretske is about knowledge.
  Problem for knowledge as factive.

  Still, don't need factive move.
  Possible not zebra, but vision is sufficient to expect that such a possibility does not obtain.

  Key here is that claiming support is never going to be strong enough to establish knowledge, at least to the extent that knowledge is factive.
\end{note}

\begin{note}
  \begin{illustration}[A trip to the zoo]
    \label{illu:CS:dretske-zebra}
    \mbox{}
    \vspace{-\baselineskip}
  \begin{quote}
    You take your son to the zoo, see several zebras, and, when questioned by your son, tell him they are zebras.
    Do you know they are zebras?
    Well, most of us would have little hesitation in saying that we did know this.
    We know what zebras look like, and, besides, this is the city zoo and the animals are in a pen clearly marked ``Zebras.''
    Yet, something's being a zebra implies that it is not a mule and, in particular, not a mule cleverly disguised by the zoo authorities to look like a zebra.
    Do you know that these animals are not mules cleverly disguised by the zoo authorities to look like zebras?

    \mbox{ }\hfill \(\vdots\) \hfill\mbox{ }

    Did you examine the animals closely enough to detect such a fraud?\linebreak
    \mbox{}\hfill\mbox{(\citeyear[1015--1016]{Dretske:1970to})}
  \end{quote}
  \vspace{-\baselineskip}
  \end{illustration}
\end{note}

\begin{note}
  \citeauthor{Dretske:1970to}'s presentation focuses on knowledge, so let us briefly form a parallel with respect to claiming support:

  \begin{illustration}
    \label{illu:dretske-zebra-var}
    \mbox{}
    `What if those animals are mules cleverly disguised by the zoo authorities to look like zebras?'
    \begin{enumerate}[label=\arabic*., ref=(I\ref{illu:CS:wheres-wally}.\arabic*)]
      \setcounter{enumi}{-1}
    \item If zebra, then not cleverly disguised mule.
    \end{enumerate}
    Reasons as follows:
    \begin{enumerate}[label=\arabic*., ref=(I\ref{illu:CS:wheres-wally}.\arabic*)]
    \item That animal appears to be a zebra.
    \item That animal is a zebra.
    \item That animal is not a cleverly disguised mule.
    \end{enumerate}
  \end{illustration}

  As with \illu{3}~\ref{illu:CS:spot-the-diff} and~\ref{illu:CS:wheres-wally}, no reasoning about the possibility that the animal is a cleverly disguised mule.
  However, given conditional, \requ{} when moving from appearance to animal.

  Introduction of additional consequence, so same structure as~\autoref{illu:CS:wheres-wally}.
  And, plausible that the agent may reason about the possibility that the animal is a cleverly disguised mule in a way sufficient to claim support.\nolinebreak
  \footnote{
    \citeauthor{Dretske:1970to} has a number of suggestions.
    \begin{quote}
    You have some general uniformities on which you rely, regularities to which you give expression by such remarks as, ``That isn't very likely'' or ``Why should the zoo authorities do that?''
    Granted, the hypothesis (if we may call it that) is not very plausible, given what we know about people and zoos.
    But the question here is not whether this alternative is plausible, not whether it is more or less plausible than that there are real zebras in the pen, but whether you know that this alternative hypothesis is false.\nolinebreak
    \mbox{}\hfill\mbox{(\citeyear[1016]{Dretske:1970to})}
  \end{quote}
  }
\end{note}

\begin{note}
  if think that doesn't know, then not too much of an issue.
  However, does know?
  Indeed, \citeauthor{Dretske:1970to}.

  No clear tension.
  Knowledge and claiming support, different.

  However, problem identified.
\end{note}

\begin{note}
  \begin{enumerate}[label=K\Alph*., ref=K\Alph*]
  \item\label{Dretske:No-C:cond:no-k-then-ep} If an agent does not know that \(\phi\) is true, then \(\phi\) being false is an epistemic possibility for that agent.
  \item\label{Dretske:No-C:cond:ep-then-no-k} If \(\phi\) being false is an epistemic possibility for some agent, then that agent does not know that \(\phi\) is true.
  \end{enumerate}
  When taken together, (\ref{Dretske:No-C:cond:no-k-then-ep}) and (\ref{Dretske:No-C:cond:ep-then-no-k}) state that: An agent knows that \(\phi\) is true if and only if \(\phi\) being false is not an epistemic possibility for the agent.
  Still, our interest will primarily be with (\ref{Dretske:No-C:cond:ep-then-no-k}).\nolinebreak
  \footnote{
    \label{fn:factivity-two-readings}
    Contrast to `factivity':
    \begin{itemize}
    \item If \(S\) knows that \(\phi\), then \(\phi\).
    \end{itemize}
    This may be read in at least two different ways.
    \begin{itemize}
    \item First, relation between epistemic state of the agent and actual \world{}:\newline
      \mbox{} \qquad If \(S\) knows that \(\phi\), then the \world{} is such that \(\phi\) is the case.
    \item Second, how things appear from epistemic state:\newline
      \mbox{} \qquad If \(S\) knows that \(\phi\) then every way the \world{} may be for \(S\) includes \(\phi\).
    \end{itemize}

    The two reading are independent of one another.

    For example, suppose you walked to the shop but the only epistemic possibility entertained by your friend is that you drove to the shop.
    Here, it is not possible for your friend to know that you drove to the shop on the first reading of factivity, but the second reading is not ruled out.

    Conversely, suppose it is the case that you walked to the shop but your friend considers it epistemically possibly that you drove.
    Here, knowing on the second reading of factivity is ruled out, but the first reading is not ruled out.

    Of course, you may endorse both readings of factivity.
    Our focus is on the `weaker' reading as we have made no connexion between claiming support and the state of the word.
    (Perhaps it is of some interest to note that \citeauthor{Dretske:1970to} explicitly denies the second reading, but not the first.)
  }\(^{,}\)\nolinebreak
  \footnote{
    The dogmatism paradox (\cite[39,43--45]{Kripke:2011wv};\cite[148]{Harman:1973ww}) seems to concern the second reading of factivity from~\autoref{fn:factivity-two-readings}, and intuitions concerning evidence.

  Roughly stated, the paradox pairs the following two propositions:
  \begin{enumerate}[label=D\arabic*., ref=(D\arabic*)]
  \item\label{dog:1} If an agent is aware that they know that \(\phi\), then the agent may disregard any evidence against \(\phi\).
  \item\label{dog:2} Rational agents respect their evidence
    (\cite[Cf.][\S2]{Kelly:2016wk})
  \end{enumerate}
  Given~\ref{dog:2}, it seems no agent should not disregard any instance of evidence, even if the antecedent of~\ref{dog:1} is satisfied.

  And, it seems \ref{dog:1} is motivated by factivity.
  For, if the agent is aware that they know that \(\phi\) then the agent knows that \(\phi\).
  And, as knowledge is factive it follows (by second reading) that \(\phi\) is the case.
  In turn, if it is the case that \(\phi\) then any evidence against \(\phi\) is evidence for something that is not the case.
  Hence, the agent may disregard any evidence against \(\phi\).


  Indeed, the second reading of factivity seems required.
  For, it seems an agent is only (apparently) in a position disregard any evidence against \(\phi\) because there knowledge that \(\phi\) guarantees that \(\phi\) is the case.
  If \emph{not}-\(\phi\) is (merely) an epistemic impossibility, and it is not clear why evidence may require an agent to revise what they consider possible.

  Note:
  Neither \citeauthor{Kripke:2011wv} (nor \citeauthor{Harman:1973ww}) make explicit mention of the agent being aware that they know \(\phi\) when formulating the Dogmatism paradox.
      Still, the paradox is clearer with this stated, as it's require addition work to find issue with a permission (to disregard evidence) if an agent is not aware that they have such a permission.

      More generally, I agree with \citeauthor{Zhaoqing:2015vj}'s (\Citeyear{Zhaoqing:2015vj}) proposal to understand the paradox in terms of knowledge attribution rather than of knowledge proper.
  }
  \citeauthor{Dretske:1970to} observes that endorsing (\ref{Dretske:No-C:cond:no-k-then-ep}) and (\ref{Dretske:No-C:cond:ep-then-no-k}) leads to closure.
  The following is a reconstruction.\nolinebreak
  \footnote{
    Specifically, the following passage:
    \begin{quote}
      A slightly more elaborate form of the same argument goes like this:
      If \(S\) does not know whether or not \(Q\) is true, then for all he knows it might be false.
      If \(Q\) is false, however, then \(P\) must also be false.
      Hence, for all \(S\) knows, \(P\) may be false.
      Therefore, \(S\) does not know that \(P\) is true.\nolinebreak
      \mbox{}\hfill\mbox{(\citeyear[1011]{Dretske:1970to})}
    \end{quote}
    Note: (\ref{Dretske:No-C:cond:no-k-then-ep}) is a reformulation of the first conditional of the passage, while a formulation (\ref{Dretske:No-C:cond:ep-then-no-k}) seems required to move from `\(P\) may be false' to `\(S\) does not know that \(P\) is true'.
  }
\end{note}

\begin{note}[Closure argument]
  Let \(S\) be some agent and suppose:
  \begin{enumerate}[label=\arabic*., ref=\arabic*]
  \item
    \label{Dretske:No-C:k-entail}
    \(S\) knows that \(\phi\) entails \(\psi\).
  \item
    \label{Dretske:No-C:dunno-psi}
    \(S\) does not know that \(\psi\) is true.
  \end{enumerate}
  Consider the following argument:
  \begin{enumerate}[label=\arabic*., ref=\arabic*,resume]
  \item
    \label{Dretske:No-C:ep-not-psi}
    \(\psi\) being false is an epistemic possibility for \(S\).%
    \hfill (\ref{Dretske:No-C:cond:no-k-then-ep} \& \ref{Dretske:No-C:dunno-psi})
  \item
    \label{Dretske:No-C:no-ep-no-entail}
    \(\phi\) not entailing \(\psi\) is not an epistemic possibility for \(S\)%
    \hfill (\ref{Dretske:No-C:cond:ep-then-no-k} \& \ref{Dretske:No-C:k-entail})
  \item
    \label{Dretske:No-C:ep-not-psi-and-phi}
    \(\phi\) being true while \(\psi\) is false is not an epistemic possibility for \(S\).%
    \hfill (\ref{Dretske:No-C:no-ep-no-entail})
  \item
    \label{Dretske:No-C:ep-not-phi}
    \(\phi\) may be false.%
    \hfill (\ref{Dretske:No-C:ep-not-psi} \& \ref{Dretske:No-C:ep-not-psi-and-phi})
  \item
    \label{Dretske:No-C:not-k-phi}
    \(S\) does not know that \(\phi\) is true.%
    \hfill (\ref{Dretske:No-C:cond:ep-then-no-k} \& \ref{Dretske:No-C:ep-not-phi})
  \end{enumerate}

  Hence, we have shown that, given (\ref{Dretske:No-C:cond:no-k-then-ep}) and (\ref{Dretske:No-C:cond:ep-then-no-k}), (\ref{Dretske:No-C:k-entail}) and (\ref{Dretske:No-C:dunno-psi}) imply (\ref{Dretske:No-C:not-k-phi}).

  That is to say, we have shown:
  \begin{enumerate}[label=K\Alph*., ref=(K\Alph*)]
    \setcounter{enumi}{2}
  \item
    \label{K:closure:from-arg}
    If \(S\) knows that \(\phi\) entails \(\psi\) and \(S\) does not know that \(\psi\) is true, then \(S\) does not know that \(\phi\) is true.%
    \mbox{} \hfill \((K_{S}(\phi \rightarrow \psi) \land \lnot K_{S}\psi) \rightarrow \lnot K_{S}\phi\)
  \end{enumerate}
  And rewriting:\nolinebreak
  \footnote{
    \((\phi \land \lnot\psi) \rightarrow \lnot\xi\) iff \(\phi \rightarrow (\lnot\psi \rightarrow \lnot\xi)\) iff \(\phi \rightarrow (\xi \rightarrow \psi)\).
  }
  \begin{enumerate}[label=K\Alph*\('\)., ref=(K\Alph*\('\))]
    \setcounter{enumi}{2}
  \item
    \label{K:closure:standard}
    If \(S\) knows that \(\phi\) entails \(\psi\), then if \(S\) knows that \(\phi\) is true then \(S\) knows that \(\psi\) is true.%
    \mbox{} \hfill \(K_{S}(\phi \rightarrow \psi) \rightarrow (K_{S}\phi \rightarrow K_{S}\psi)\)
  \end{enumerate}
\end{note}

\begin{note}
  Return to \autoref{illu:CS:dretske-zebra}.

  Let us assume you know that:
  \begin{itemize}
  \item If the animals are zebras then the animals are not cleverly disguised mules. And,
  \item The animals are zebras.
  \end{itemize}

  If \ref{K:closure:standard} holds, then you also know that the animals are not cleverly disguised mules.
  However, following \citeauthor{Dretske:1970to}'s intuition, you do not know that the animals are cleverly disguised mules.

  Hence, to accommodate \citeauthor{Dretske:1970to}'s intuition, either (\ref{Dretske:No-C:cond:no-k-then-ep}) or (\ref{Dretske:No-C:cond:ep-then-no-k}) must be rejected.
  \citeauthor{Dretske:1970to} rejects (\ref{Dretske:No-C:cond:ep-then-no-k}).\nolinebreak
  \footnote{
    See below.
  }
\end{note}

\begin{note}
  We now return to claiming support.

  \autoref{assu:supp:nfactive} requires that claiming support for \(\phi\) is compatible with the (epistemic) possibility that the claimed support is \nmom{}.
  And, the claimed support is \misled{} just in case \(\phi\) is not the case.
  Hence~\autoref{assu:supp:nfactive} requires that claimed support for \(\phi\) is compatible with the (epistemic) possibility that \(\phi\) is not the case.

  Therefore, the result of substituting `claimed support' from `knowledge' in (\ref{Dretske:No-C:cond:ep-then-no-k}) conflicts with \autoref{assu:supp:nfactive}.
  And so~\autoref{assu:supp:nfactive} parallels \citeauthor{Dretske:1970to}'s rejection of (\ref{Dretske:No-C:cond:ep-then-no-k}).
  However, \citeauthor{Dretske:1970to}'s rejection of (\ref{Dretske:No-C:cond:ep-then-no-k}) is motivated by a rejection of~\ref{K:closure:standard}.
\end{note}

\begin{note}[Link]
  The refinement of~\ideaCS{} through \autoref{assu:supp:independence} has lead to a proposition close to~\ref{K:closure:standard}

  \begin{enumerate}[label=P\ref{prop:CS-only-if-reason-recognised-defeaters}\('\).]
  \item If \(\psi\) having value \(v'\) is a \requ{} of claiming support for \(\phi\), then if the agent has claimed support for \(\phi\) having value \(v\) then the agent has reasoned about whether \(\psi\) has value \(v'\).\newline
    \mbox{}\hfill \((\phi \leadsto \psi) \rightarrow (\text{CS}\phi \rightarrow \text{R}\psi)\)
  \end{enumerate}

  Both build on \autoref{def:requisite}.
  A \requ{}.
  This is a complex parallel to knowing that \(\phi\) entails \(\psi\).
  However, somewhat general, and applies to \(K\) also.
  {\color{red} (Strictly speaking, because of contraposition, make this more explicit below)}

  \autoref{assu:supp:independence} then requires reasoning.

  Requiring something with respect to \(\psi\) having value \(v'\) given some state of the agent an principle which relates \(\psi\) having value \(v'\) to that state.

  For sure, appealing, or having claimed support is distinct, so the closure is not that with respect to an \emph{an} epistemic operator such as knowledge.
  However, the tension here is in terms of viewing the rejection of \ref{K:closure:standard} as the endorsement of a `locality constraint'.
  {\color{red} What this means.}
  And, seem to violate such a constraint.
\end{note}

\begin{note}[`Locality constraint']
  \begin{quote}
    To know that \(x\) is \(A\) is to know that \(x\) is \(A\) within a framework of relevant alternatives, \(B\), \(C\), and \(D\).
    This set of contrasts, together with the fact that \(x\) is \(A\), serve to define what it is that is known when one knows that \(x\) is \(A\).
    One cannot change this set of contrasts without changing what a person is said to know when he is said to know that \(x\) is \(A\).\nolinebreak
    \mbox{}\hfill\mbox{(\citeyear[1022]{Dretske:1970to})}
  \end{quote}

  Where:
  \begin{quote}
    A relevant alternative is an alternative that might have been realized in the existing circumstances if the actual state of affairs had not materialized.\nolinebreak
    \footnote{
      \citeauthor{Dretske:1970to} adds:
  \begin{quote}
    \dots alternatives that \emph{might} have been realized in the existing circumstances if the actual state of affairs had not materialized.
    \dots are not relevant alternatives.\nolinebreak
    \mbox{}\hfill\mbox{(\citeyear[fn.6][1021]{Dretske:1970to})}
  \end{quote}
    }
    \nolinebreak
    \mbox{}\hfill\mbox{(\citeyear[1021]{Dretske:1970to})}
  \end{quote}
  Different from a \requ{}.
\end{note}

\begin{note}
  Point is not direct clash.\nolinebreak
  \footnote{Two problems.

    First, \citeauthor{Dretske:1970to} seems to go with no recognition at time, compatible with claiming support at time.
    So, would need instance of appealing to knowledge for some other purpose.
    Restating is not sufficient, assumptions made are compatible with persistence (as noted).

    Second, only get no need to rule out from \citeauthor{Dretske:1970to}, which does not require no reasoning.
  }
  At issue, rather, it the degree to which possibility is compatible with the absence of reasoning.
  That is, if reject closure, is this due to reasoning or due to requirements placed on such reasoning?\nolinebreak
  \footnote{
    (Problem here, but also extends to \nI{}.)
  }

  There is some subtlety, however.
  Kind of closure seems to break for many attitudes.
  This is not at issue, distinguishing feature of claiming support is closure, and the constraints this places on an agent.
  However, reject for stronger attitudes, then why for weaker?
\end{note}

\begin{note}
  Reasoning, but haven't placed constraints.
  So, this allows for something that may be quite weak.
  % I am here only restating what has gone before, but the added context may help.
  No reasoning, then leaves open possibility that does lead to a problem with the reasoning performed.

  Consider again relevant alternatives from internalist perspective.
  It seems, agent determines whether relevant or not will require reasoning.

  Consider:
  \begin{quote}
    `What if those reports about the zoo authorities cleverly disguising animals to look like other animals?
    If there are, could those animals be cleverly disguised mules?'
  \end{quote}
  Perhaps not a relevant alternative if the sense of `might' is non-epistemic, but if epistemic, then seems a problem.
\end{note}

\begin{note}
  To summarise.

  \citeauthor{Dretske:1970to}'s case.
  Rejection of closure.
  Question whether assumptions are okay.
  Argued that these are given understanding of claimed support, and that they plausibly extend to knowledge (and other attitudes intuitively stronger than that of having claimed support).
  For, rejection of closure plausibly amounts to rejection on strength of reasoning, rather than requirement to reason.

  Focus on this point for two reasons:
  First, distinguishing feature of claiming support and following will be about claiming support.
  Second, appeal to similar limitation later.
\end{note}


%%% Local Variables:
%%% mode: latex
%%% TeX-master: "masterD"
%%% End:
