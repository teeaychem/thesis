\chapter{Instances of \issueConstraint{}}
\label{cha:lit}

\begin{note}
  \color{red}
  What I say about \agpe{} and \ros{} is compatible with denying \ros{} is a premise.
  If I have this made explicit already, then the following arguments are a little stronger.
\end{note}

\begin{note}
  \autoref{cha:var} introduced \qWhyV{}, \qHowV{}, and \issueConstraint{}, variants to \qWhy{}, \qHow{}, and \issueInclusion{}, respectively.

  In short, having a \wit{0} for a \ros{} is necessary for the \ros{} to, in part, explain why an agent concluded.

  Motivation for \issueConstraint{} follows motivation for \issueInclusion{}.

  Primary motivation is intuition.
  \scen{1} such as \autoref{scen:calc} and \autoref{scen:animalism}.

  Theoretical motivation.
  In particular \citeauthor{Davidson:1963aa}'s (causal) theory of action.
\end{note}

\begin{note}
  In this section we collect a handful of extracts from the literature which suggest this intuition is not merely an intuition, but a common theoretical constraint.

  Provide extract.
  Observe how plausibly understood in terms of \wit{} and constraint.

  In each of these cases, there is no immediate entailment.
  However, suggestive, \dots
\end{note}

\begin{note}
  \begin{TOCEnum}
  \item
    ???
  \end{TOCEnum}
\end{note}

\begin{note}
  Additional `doxastic'.
  Consider in \autoref{cha:embed} with the aid of definition.
\end{note}

\newpage

\subsection*{Puzzling}

\begin{note}


  Distinct from \wit{} because only interest is belief.
  Distinct from embedding because constitutive.
  In other words, the agent does not reason from their belief.
  Rather, reasoning is the belief.

  Inclined to understand in terms of \ros{}.
  For our purposes, role of \ros{} is to capture relationship between premises and conclusion.
  However, understood in a particular way, may amount to a belief.

  Still, not so straightforward.
  How does on get the belief that \emph{p} follows from \emph{R}?
  This, to my mind, is what is at issue.
  However, it seems that this is not how things are for \citeauthor{Valaris:2014un}.

  

  Unless belief is process, then it seems reasoning understood in this way is instantaneous.

  I am not sure what to make of this.
\end{note}


%%% Local Variables:
%%% mode: latex
%%% TeX-master: "master"
%%% TeX-engine: luatex
%%% End:
