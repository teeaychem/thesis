\chapter{Temp}
\label{cha:temp}


% \section{\pevent{3}}
% \label{sec:assupp2}

% \begin{note}
%   we noted the progressive allows us to capture conclusions which are available to the agent, but not yet made.
%   Though, to identify such conclusions via the progressive it must be the conclusion is in progress.
%   Still, with the progressive in hand we may define broader modalities in terms of the progressive.
%   In particular, we define a \pevent{} as follows:

%   \begin{definition}[\pevent{3}]%
%     \label{def:potenital-event}%
%     \begin{itemize}
%     \item
%       There is a \emph{\pevent{}} in which \vAgent{} does \(X\).
%     \end{itemize}

%     \emph{If and only if}:


%     \begin{itemize}
%     \item
%       There is some action \(a\) such that both~\ref{def:PE:action} and~\ref{def:PE:prog} are true:

%       \begin{enumerate}[label=\alph*., ref=(\alph*)]
%       \item
%         \label{def:PE:action}
%         \(a\) is an action available to \vAgent{}.
%       \item
%         \label{def:PE:prog}
%         An event in which \vAgent{} does \(X\) is in progress when \vAgent{} does \(a\).
%       \end{enumerate}
%     \end{itemize}
%     \vspace{-\baselineskip}
%   \end{definition}

%   The role of \pevent{1} is to capture an event which is `possible' in the sense of the progressive without the truth of the progressive.
%   Of course, the added modality to move to an event in which the progressive is true, and this modality may be scrutinised.
%   Still, I take an `available action' to be sufficiently intuitive.%
%   \footnote{
%     For background reading, \autoref{def:potenital-event} parallels \citeauthor{Mandelkern:2017aa}'s act conditional analysis of ability, where `practically available' parallels `available' (\citeyear[\S5]{Mandelkern:2017aa}).
%     Likewise, consider `options' in \citeauthor{Boylan:2020aa}'s `determinacy' analysis (\citeyear[\S4]{Boylan:2020aa}).
%   }
% \end{note}



%%% Local Variables:
%%% mode: latex
%%% TeX-master: "master"
%%% TeX-engine: luatex
%%% End:
