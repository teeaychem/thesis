\chapter{Wrangling}
\label{cha:var:wrang}

\subsection{Dependence}

\begin{note}
  Difficulty.
  How is it the case that the agent concludes only due to \fc{}?

  Method is to consider dependence of \qWhyVnP{}, from the \agpe{}:

  \begin{restatable}[\qWhyV{}]{question}{questionWhyV}
    \label{q:why:v}
    Given an agent \vAgent{}, proposition-value pair \(\pv{\phi}{v}\), \poP{} \(\Phi\), and event \(e\) in which \vAgent{} concludes \(\pv{\phi}{v}\) from \(\Phi\):

    \begin{quote}
      Which proposition-value-premises pairings \(\pvp{\psi}{v'}{\Psi}\) are such that, when \vAgent{} pairs \(\phi\) with \(v\):

      \begin{enumerate}[label=]
      \item
        \begin{enumerate}[label=\alph*., ref=(\alph*), series=qWhyVDef]
        \item
          A \ros{0} between \(\pv{\psi}{v'}\) and \(\Psi\) holds, from \agpe{\vAgent{}'}.
        \end{enumerate}
      \end{enumerate}

      And, from \agpe{\vAgent{}'}:

      \begin{enumerate}
      \item[\emph{If}:]
        \begin{enumerate}[label=\alph*., ref=(\alph*), resume*=qWhyVDef]
        \item
          The \ros{0} between \(\pv{\psi}{v'}\) and \(\Psi\) when pairing \(\phi\) with \(v\) failed to hold, from \agpe{\vAgent{}'}.%
          \footnote{
            It seems more natural to omit `from \agpe{\vAgent{}'}':
            \begin{itemize}
            \item
              The \ros{0} between \(\pv{\psi}{v'}\) and \(\Psi\) when pairing \(\phi\) with \(v\) failed to hold.
            \end{itemize}
            Indeed, evaluating conditional from \agpe{}.
            However, distinction between whether \ros{} and whether \ros{} from \agpe{}.
            Latter, and keeping the qualifier helps clarify.
          }
        \end{enumerate}
      \item[\emph{Then}:]
        \begin{enumerate}[label=\alph*., ref=(\alph*), resume*=qWhyVDef]
        \item
          \(e\) would not have been an event in which \vAgent{} concluded \(\pv{\phi}{v}\) from \(\Phi\).
        \end{enumerate}
      \end{enumerate}
    \end{quote}
    \vspace{-\baselineskip}
  \end{restatable}
\end{note}

\begin{note}
  The idea is:

  Counterexamples.

  One option is to specify (apparent) counterexamples so that the truth of the \emph{if-then} conditional follows.
  Then, at issue is whether analysis is correct.
  It may be that the truth of the conditional does not follow.

  Other option, specify (apparent) counterexamples so that the \emph{if-then} conditional holds from the \agpe{}.
  Then, at issue is whether the \agpe{} is correct.

  Trade a issue about analysis of counterexamples for a issue about what the counterexample achieves.

  Preference for the second option is ease of specifying examples.
  Build up an understanding of how and why such examples arise, and then try to figure out whether they result in anything substantial rather than attempting defend position that analysis of an example counters.

  In short:

  On first, whether analysis is correct.

  On second, whether agent is correct.

  Preference for whether agent is correct.
  Though, equivalent.
\end{note}

\begin{note}
  In order for \qWhyV{} to be of interest, must be cases where the \agpe{} is correct.

  \begin{proposition}
    \label{prop:why-n-p-link}
    Instances where \(\pvp{\psi}{v'}{\Psi}\) answers \qWhyVnP{} in virtue of answering \qWhyV{}
  \end{proposition}

  Find instances which witness the truth of \autoref{prop:why-n-p-link}.

  Stress, \autoref{prop:why-n-p-link} does not amount to an entailment.
  They may be cases where \agpe{} returns an answer to \qWhyV{} which is not an answer to \qWhyVnP{}.
\end{note}

\subsubsection{Concerns}
\label{sec:pitfalls}

\begin{note}
  Subdivide into two concerns.

  \begin{enumerate}
  \item
    Is \agpe{} informative?
  \item
    Granting true from \agpe{}, is informative, may it still be the case that \ros{} fails to answer \qWhyVnP{}.
  \end{enumerate}
\end{note}

%%% Local Variables:
%%% mode: latex
%%% TeX-master: "master"
%%% End:
