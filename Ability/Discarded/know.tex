\chapter{\influence{2}}
\label{cha:know}

\subsection{(Recognition of) knowledge}
\label{sec:knowledge}

\begin{note}
  Whether the agent \emph{knows} that \(\pv{\psi}{v'}\) from \(\Psi\) is a \fc{}.
\end{note}

\begin{note}
  Seen the way in which this works.
  Both \ros{} and dependence.
\end{note}

\begin{note}
  Our interest with knowledge is with respect to two things.

  First, propositional attitude, and hence has a role.

  Second, factivity.

  First, this is what leads to \influence{} with respect to \requ{}.
  Second, this ensures that there is a \ros{}.

  Anything for which these two features hold may be substituted for knowledge.

  However, both of these features raise difficulties.
  We will discuss, and then set aside factivity.
  Focus is on \influence{}.
\end{note}

\begin{note}
  It may be the case that, though from the \agpe{} they would not have concluded \(\pv{\phi}{v}\) from \(\Phi\) if \ros{} failed to hold between \(\pv{\psi}{v'}\) and \(\Psi\), the agent would have concluded \(\pv{\phi}{v}\) from \(\Phi\) regardless.

  For example, suppose an agent has taken a gamble on a coin landing heads.
  The coin lands heads, and the agent receives a prize.
  From the \agpe{}, if the coin failed to lands heads, then the agent would not have received the prize.
  However, the agent was set to receive the prize for participating in the gamble, regardless of whether the coin landed heads.%
  \footnote{
    The present point is similar to issues raised by \citeauthor{Harman:1973ww} (\citeyear{Harman:1973ww}) regarding the proposed equivalence between reasons for which an agent believes something with reasons the agent would offer if asked to justify their belief.
  As \citeauthor{Harman:1973ww} notes, an agent may offer reasons because they think they will convince their audience, not because the agent is compelled by the reasons, etc.
  (\citeyear[Ch.2]{Harman:1973ww})

  To the extent that \citeauthor{Harman:1973ww}'s point is that what holds from an \agpe{} need not actually be the case, the point in the same.
  However, to the extent that \citeauthor{Harman:1973ww} relies on an under-specification of what holds from an \agpe{} --- i.e.\ the distinction between whether \(\phi\) has value \(v\) from the \agpe{} or whether the agent evaluates as true the proposition that their audience is responsive to \(\phi\) having value \(v\), the point is distinct.
  }
\end{note}

\subsubsection{Factivity}
\label{sec:factivity-1}

\begin{note}
  Factivity, it may not be the case that \(\pv{\psi}{v'}\) from \(\Psi\) is a \fc{}.
  Therefore, it is not the case that a \ros{} holds, from the \agpe{}.
\end{note}

\begin{note}
  Why does this matter?
  Well, \qWhyV{}.
  Explains why.
  If no \ros{}, then non-factive explanation.
\end{note}

\begin{note}
  Is it plausible that factivity fails?

  No.
\end{note}

\begin{note}
  Scope.

  \fc{}.
\end{note}

\begin{note}
  So, not interested in whether \(\psi\) has value \(v'\).

  Likewise, not interested in whether agent method by which the agent concludes corresponds to the rules of chess.
  What matters is that the agent would be concluding.
\end{note}

\begin{note}
  With the exception of more-or-less instantaneous actions, future may develop in surprising ways.

  For example, plausible that an agent knows when they strike the cue ball in a certain way, a particular red ball will land in a pocket.
  However, not plausible that the agent knows where the cue ball will come to rest after the red ball lands in the pocket.
  Hence, agent does not know their following move, and so on.

  In parallel, an agent may have no guarantee that they will not be interrupted, etc.
  Hence, in most cases it seems implausible that an agent knows they will concluded.

  Yet, to be \fc{} does not require completion.
  Only the case that concluding, and not concluding any proposition-value which is incompatible.
\end{note}

\begin{note}
  So, if failure of factivity, then it's with respect to action.
  However, if this is the case, then further implication.
  If action, then would be concluding.

  I don't see a significant difference between these two things.
  Formulation of \fc{} is to keep in line with \supportI{}.
  However, seems fine to go with resources.

  If \ros{} without \wit{} due to act, then I think the actual availability of the act is irrelevant, so long as it is true that if act, then get that concluding.

  And, I see no way for this to plausibly fail.
\end{note}

\begin{note}
  So, the trouble is, that if one tries to get non-factivity, then it seems there is no generality.

  It doesn't matter what the agent is doing, what matters is that the agent would repeat if given the opportunity.
  If this is not the case, in general, as needs to be, as we're only interested in there existing cases, then there's no generality.
\end{note}

\begin{note}
  In conclusion, generality.
  And, if this fails, then I see no interest in answers to \qWhy{}.
\end{note}

\begin{note}
  \color{red}
  I have no idea how this relates to \textquote{Kripkenstein}.
\end{note}

\paragraph{Generality}


\begin{note}
  I do not find failure of factivity compelling.
  However, it highlights the core puzzle.

  Generality.

  Plausible that conclusions are specific instance of general.

  And, if this is the case, no collection of \wit{1} captures full generality.
  In order for this to be the case, \fc{1} where agent does not have a \wit{1}.

  If you find this compelling, then apply to own reasoning.
  And, at issue is whether \influence{}.

  Though, I will have made this observation already.
\end{note}

\subsubsection{Sensitivity}
\label{sec:sensitivity}

\begin{note}
  Factivity, argued that rejecting this ends up turning on whether there is any generality to the agent's reasoning.

  Accept generality, but deny that the agent is ever sensitive to generality.

  In turn, deny that that the agent knows the way in which they concluded.
\end{note}

\begin{note}
  This is a problem.
  Because, it seems that in a variety of cases, knowledge is the default state.

  For, we have generality.
  Hence, at issue is whether instance of generality.
  Exception that do not perform instance of generality.

  This is not to deny may fail to perform.
  However, so routine, that exceptions are clear.

  The key point, it needs to be the case that no sensitive.
  And, just looking for cases, no sensitivity in general.
  So, no times when sensitive.

  Well, why think generality if insensitive.
\end{note}

\begin{note}
  Further, the point here is really about reasoning performed.
  If insensitive, then seems the agent does not know the way in which they concluded.
  This is very unintuitive.

  Soundness of a rule is one such example.
  Below, Sudoku.
  Indeed, more generally, when the conclusion follows from some collection of well understood rules and where the relevant conclusion is easy.
  So, chess, arithmetic, and so on.

  In these cases, know.
\end{note}

\begin{note}
  Further, is it more plausible that the agent fails to know, or that the agent is not concluding \(\pv{\phi}{v}\) from \(\Phi\).

  That is, interested in \qWhy{}.
  What was involved in the agent conclusion of \(\pv{\phi}{v}\) from \(\Phi\).
  If the agent did not conclude \(\pv{\phi}{v}\) from \(\Phi\), then \qWhy{} and \qWhyV{} are misapplied.

  Not only do I think it is plausible, but I think it is strictly more plausible than agent failing to know.
\end{note}

\begin{note}
  \begin{observation}
    \label{obs:factivity:know-knew}
    In positive cases with same rules, if know concluding then know \fc{}.
  \end{observation}
  \begin{argument}{obs:factivity:know-knew}
    Suppose agent knows concluding, but does not know \fc{}.
    Then, agent does not know that they would repeat same reasoning.

    Converesely, If don't know \fc{}, then seem don't know applied method.
  \end{argument}
  In short, the difficulty is maintaining that the agent knows.

  Then, this combines with intuition that know.
\end{note}

\begin{note}
  So, look, if not sensitive, then in what sense is this a specific instance of general.
\end{note}

\newpage

\begin{note}
  This does not provide a complete solution.
  For any particular instance, possible to construct context so that the agent is insensitive.

  Familiar with external world skepticism.

  \requ{1} are distinct, in that it does not matter whether \(\phi\) has value \(v\).
  However, similar in that it does matter that there is a \pevent{}.

  If agent is a brain in a vat, then this does not prevent the agent from concluding.
  However, this may prevent the existence of a \pevent{}.

  Indeed, a closer parallel is \citeauthor{Schaffer:2010vq}'s (\citeyear{Schaffer:2010vq}) debasing demon.
  The demon \textquote{throws her victims into the belief state on an improper basis, while leaving them with the impression as if they had proceeded properly} (\citeyear[231]{Schaffer:2010vq})
\end{note}

\section{Support}
\label{sec:support}

\begin{note}
  So, instead of focusing on way in which deny these premises, focus on the upshot.

  Here, we have a case in which does not answer \qWhyV{}.

  This means, then, that if \ros{} failed to hold, agent would conclude.

  In what sense is the agent's conclusion that of a general pattern.

  For, if ignore this, then what is there to explain the pattern?
\end{note}

%%% Local Variables:
%%% mode: latex
%%% TeX-master: "master"
%%% End:
