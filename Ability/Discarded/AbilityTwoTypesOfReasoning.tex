\chapter{Two ways of claiming support}
\label{sec:two-ways-of-claiming-support}

\section{\adA{} and \adB{}}
\label{sec:ability-ads-adc}

\begin{note}[Recall\dots]
  Let us briefly summarise our progress so far.

  In~\autoref{sec:cases-interest} we introduced particular instances of an agent claiming support for some conclusion which involved two key steps.
  The reasoning, in outline:
  \begin{enumerate}[label=\arabic*., ref=(\arabic*)]
  \item\label{NUR:ro:i} I have some general ability \(\gamma\).
  \item\label{NUR:ro:ii} If I have general ability \(\gamma\) then I have some specific ability \(\varsigma\) to \emph{V} that \(\phi\).
  \item\label{NUR:ro:iii} I have the specific ability \(\varsigma\) to \emph{V} that \(\phi\). \hfill (From~\ref{NUR:ro:i} and~\ref{NUR:ro:ii})
  \item\label{NUR:ro:iv} It is only possible to \emph{V} that \(\phi\) if \(\phi\) is already the case.
  \item\label{NUR:ro:v} \(\phi\) is the case. \hfill (From~\ref{NUR:ro:iii} and~\ref{NUR:ro:iv})
  \end{enumerate}

  The reasoning involves claiming support by two important steps.

  \begin{itemize}
  \item The first step, from~\ref{NUR:ro:i} and~\ref{NUR:ro:ii} to~\ref{NUR:ro:iii}, involves the conditional of~\ref{NUR:ro:ii}, termed `\gsi{0}', clarified in~\autoref{sec:type-scenario}. And,
  \item The second step, from~\ref{NUR:ro:iii} and~\ref{NUR:ro:iv} to~\ref{NUR:ro:v}, is an instance of `\aben{an}', clarified in~\autoref{sec:ability-entailment}.
  \end{itemize}

  Both steps involve reasoning, and in particular claiming support, by appeal to ability.
  The first step, claiming support from a specific ability from a general ability.
  The second step, claiming support for some proposition from specific ability.

  Issue is how the agent claims support.
  In turn, how an agent reasons with ability.
  The sketch captures key premises and steps, but does not provide an interpretation of those steps.

  In~\autoref{sec:wr-ar} we introduced two (schematic) interpretations of specific ability --- \AR{} and \WR{}.
  A few purposes for these (schematic) interpretations.
  First, some insight into how an agent may claim support.
  \AR{} some property, \WR{} some event.
  No stance on these.
  Distinction on one hand allows us to state in greater detail, and on the other hand ensures that the arguments to follow do not presuppose a particular (schematic) interpretation of ability.

  Now, two steps under either \AR{} or \WR{} may look straightforward.
  Both involve conditionals, so matter of something like \emph{modus ponens}.
  In this respect, distinction between \AR{} and \WR{} matters only for finer details of how the agent claims support, rather than the reasoning involved.
  `Something like \emph{modus ponens}' sufficiently similar so that it's the conditional that's doing the work.
  In the sense that the conditional form does most of the work.
  \AR{} and \WR{} why it's claiming support as opposed to some other kind, say subjunctive or suppositional, reasoning.

  However, this is not immediate.
  The sketch does not provide an interpretation of those steps.
  The purpose of this section is to outline an alternative way of claiming support that may be applied to the sketch.
  Key part in argument against \ESU{} (and for \EAS{}).

  Start with \illu{0}.
  Then, definition.
  Applied to further \illu{1}.
  Applied to \AR{} and \WR{}.


  Distinction between reasoning \adA{} and \adB{} is of interest to use with respect to these two instances of claiming support in particular.

  So, pair \AR{} and \WR{} with \adA{} and \adB{} and we have various ways of understanding how agent claim support.

  Keep in mind that here we're elaborating on what this sketch amounts to.
\end{note}

%%% Local Variables:
%%% mode: latex
%%% TeX-master: "master"
%%% End: