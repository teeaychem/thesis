\chapter{zSpA discarded}
\label{cha:zspa-discarded}

\subsection{Re-expression}
\label{sec:answers-which-are}

\begin{note}
  \begin{itemize}
  \item
    Somewhat simple.
  \item
    Is there a flaw?
  \item
    Well, point is there's something that isn't a proposition-value pair.
  \item
    Parallel?
  \item
    Knowledge how and knowledge that.
  \item
    Same argument applies to dispute, in a fairly straightforward way.
    Specifically, regarding knowledge that, regardless of knowledge how.
  \end{itemize}
\end{note}

\begin{note}
  Sketch of the argument.

  \begin{enumerate}
  \item
    Potential witnessing event in which agent concludes.
  \item
    \label{pwe-iff-kh}
    This is the case if and only if knows how to conclude (if not knowledge, then insufficient grasp on witnessing event).
  \item
    \label{kw-is-kt}
    Knowledge how is a species of knowledge that.
  \item
    Knowledge that \(\varphi\)
  \item
    Knowledge that \(\varphi\) does not involve event.
  \item
    Equivalent.
  \item
    So, at least possible to answer with something that does not involve event.
  \end{enumerate}

  So, replacement.

  Limitation is option to replace.

  Still, this is enough to highlight a flaw in \autoref{prop:PWEs}.

  In addition, given that agent is not literally answering the question, additional argument that this is how to understand.
\end{note}

\begin{note}
  \dots In outline, depends on how the details are filled in.

  \autoref{pwe-iff-kh} and \autoref{kw-is-kt} in particular.

  Grant \autoref{pwe-iff-kh}, explore \autoref{kw-is-kt}.
\end{note}

\begin{note}
  Not just strong intellectualism, but\dots

  Knowing that is not,~\cite{Stalnaker:2012tp} `justified true belief, painted over with a Gettier-proof coating of some kind.' (\citeyear[754]{Stalnaker:2012tp})

  Instead, ~\citeauthor{Stanley:2011ut} (and~\citeauthor{Stalnaker:2012tp}'s) views are compatible with knowing that involving, at least in part, a potential witnessing event.
  Sparing the details, characterisation by~\citeauthor{Weatherson:2017tb}:%
  \footnote{
    \textcite{Weatherson:2017tb} investigates kind of dispositions involved.
  }

  \nocite{Stanley:2012wg}
  \begin{quote}
    Knowing that \emph{p} is not just a matter of having \emph{p} written in a knowledge box somewhere in the brain; it can in part be constituted by active dispositions.%
    \mbox{ }\hfill\mbox{(\citeyear[8]{Weatherson:2017tb})}
  \end{quote}

  \citeauthor{Stalnaker:2012tp} highlights irony.

  Active disposition.
  Hence, potential witnessing event.%
  \footnote{
    Question whether views such as those of \cite{Stalnaker:2012tp} and \citeauthor{Stanley:2011ut} help with argument here.

    Non-committal.
    Roughly, need two things:
    It to be the case that what matters to the agent is viewing conclusion in terms of general disposition.
    And, that have the disposition, rather than obtaining the conclusion as an instance of the disposition.

    So, still a question about \qzS{} and \qWhy{}.
  }
\end{note}

\begin{note}
  So, finding a theory to make this argument work is not immediate.
  Hence, even if valid, not clear sound.
  Still, this won't deter.
\end{note}

\begin{note}
  Point is, that, even if you get this reduction, still need the potential event.
\end{note}

\begin{note}
  Or, still get a \requ{}.
  For, \deadEnd{}.
\end{note}


%%% Local Variables:
%%% mode: latex
%%% TeX-master: "master"
%%% End:
