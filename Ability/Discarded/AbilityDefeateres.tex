\chapter{Defeaters}
\label{cha:tension}

\subparagraph{Defeaters}

\begin{note}
  \color{red}
  The problem is that there's no sense in which \ideaCS{} makes sense with respect to defeaters.
  Rather, the \epVAd{} of \(\psi\) not having value \(v'\) straight up blocks claiming support.

  The only place that a discussion of defeaters would be relevant is with respect to \ideaS{}.
  However, I don't think I really need to say more of \ideaS{}.
\end{note}

\begin{note}
  \begin{proposition}
    No indication if the agent only considers their reasoning to indicate that \(\phi\) has value \(v\) only if \(\phi\) has value \(v\).
  \end{proposition}
\end{note}


\begin{note}
  \phantlabel{first-mention-undercutting-defeater} % first mention of undercutting defeaters
  Undercutting.

  Following~\citeauthor{Moretti:2018va}:
  \begin{quote}
    A \emph{rebutting} defeater for a belief that \(P\) of \(S\) is, roughly, a reason of \(S\) for believing the negation of \(P\) or for believing some proposition \(Q\) incompatible with \(P\).
    Whereas an \emph{undercutting} defeater for a belief that \(P\) of \(S\) is, roughly, a reason of \(S\) that attacks the connection between S's ground for believing \(P\) and \(P\)[.]\nolinebreak
    \mbox{}\hfill\mbox{(\citeyear{Moretti:2018va})}
  \end{quote}
\end{note}

\begin{note}
  An epistemic defeaters is something that is the case.
  As a plausible consequence, possibility of undercutting and rebutting defeaters.
  So, here, \epP{} of either.
\end{note}

\begin{note}
  However, does not entail that there are defeaters, nor that there are a defeater is a genuine (i.e.\ non-epistemic) possibility --- only that defeat is an \epP{}.
\end{note}

\begin{note}
  In this respect, the \epPAd{} of a rebutting or undercutting defeater may be viewed a kind of undercutting defeater\dots
\end{note}

\begin{note}[Quick intuition]
  \emph{Undercutting} defeaters.

  Type II defeaters.

  \begin{quote}
    The second kind of defeater attacks the connection between \(P\) and \(Q\) rather than attacking \(Q\) directly.

    \mbox{}\hfill\(\vdots\)\hfill\mbox{}

    A type II defeater is any reason for believing that \({\sim}(P => Q)\) which is not also a reason for believing that \({\sim}Q\).\nolinebreak
    \mbox{}\hfill\mbox{(\cite[43]{Pollock:1974uk})}
  \end{quote}
  Where `\(=>\)' is the subjunctive conditional. (\Citeyear[42]{Pollock:1974uk})

  With \citeauthor{Pollock:1974uk}'s original formulation, the defeater attacks the link.\nolinebreak
  \footnote{
    See \textcite[196,fn.166]{Pollock:1999tm} for a brief note of the history of undercutting defeaters.
    \textcite{Pollock:1974uk} is more-or-less a direct expansion of discussion in~\textcite{Pollock:1970un}.
  }

  However, generalise.
  That there's a relation of claimed support, evidence, or what have you between \(P\) and \(Q\).

  Generalisation by \citeauthor{Bergmann:2005ws}\nolinebreak
  \footnote{
    I am also inclined to \citeauthor{Worsnip:2018aa}'s sketch of undercutting defeaters, which builds on~\citeauthor{Bergmann:2005ws}'s.
    \begin{quote}
      Undercutting defeaters, which are easiest to think of in the context of the attitude of belief, are supposed to be considerations that undermine the justification of a belief in a proposition p not necessarily by providing (sufficient) positive evidence to think that p is false, but rather merely by suggesting (perhaps misleadingly) that one’s reasons for believing p are no good, in a way that neutralizes or mitigates their justificatory or evidential force.\linebreak
      \mbox{}\hfill\mbox{(\Citeyear[29]{Worsnip:2018aa})}
    \end{quote}
  }
  \begin{quote}
    \emph{d} is an \emph{undercutting defeater} for \emph{b} iff \emph{d} is a defeater for \emph{b} which is (or is an epistemically appropriate basis for) the belief that one's actual ground or reason for \emph{b} is not indicative of \emph{b}'s truth.\newline
    \mbox{}\hfill\mbox{(\citeyear[424]{Bergmann:2005ws})}
  \end{quote}

  Similar generalisation, but similar to \citeauthor{Pollock:1974uk} the relation is useful.
  So, generalise the subjunctive conditional.
\end{note}

\begin{note}
  \color{red}
  Idea is that the \epVAd{} \world{} acts as an undercutting defeater.
  Incorporating \ideaS{}, not getting a \sink{} is going to be an undercutting defeater.
\end{note}


%%% Local Variables:
%%% mode: latex
%%% TeX-master: "masterD"
%%% End:
