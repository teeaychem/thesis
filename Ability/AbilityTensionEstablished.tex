\chapter{Establishing tension/summary}
\label{sec:establishing-tension}

\begin{note}[Results of the distinctions]
  Recap.

  \begin{itemize}
  \item Kind of scenario involving ability.
  \item Distinction between \AR{} and \WR{}.
  \item Distinction between \adA{} and \adB{}.
  \item Distinction matrix.
  \item Point of this was to provide an exhaustive account of the ways in which the agent may claim support in the scenarios of interest.
  \end{itemize}

  Then, moved to figuring out whether the respective combinations of the distinction matrix are permissible.
  \begin{itemize}
  \item \ESU{} ruled out \adB{}, with the exception of \AR{} basic.
  \item \nI{} ruled out \adA{} no matter way in which ability was though about.
  \end{itemize}

  So, if this is correct, we've got three options.
  \begin{itemize}
  \item \AR{} basic, with \adB{}.
  \item Reject \nI{}.
  \item Reject \ESU{}.
  \end{itemize}
\end{note}

\begin{note}[Matrix]
  \begin{figure}[H]
    \centering
    \saMtxRuledOut{}
    \repeatCaption{fig:saMtxRuledOut}{Distinction matrix}
  \end{figure}
\end{note}

\section{\AR{} Basic with \adB{}}
\label{sec:ar-basic-with}

\begin{note}
  Main issue here is that it doesn't seem as though this is a basic step of reasoning.

  For, ability breaks down into various components.

  Response here is that there do seem to be basic steps of reasoning with are similar in this respect.
  For example, it seems as though many cases of moving from cause to effect will do this.

  I think going back to dispositions might help here.
\end{note}

\section{Reject reasoning}
\label{sec:reject-reasoning}

\begin{note}
  If not basic, then reject reasoning.
\end{note}

\begin{note}
  Problem here is that this seems too strong.
\end{note}

\begin{note}[Main problem]
  The main problem is that it seems fine for the agent to claim support for specific ability.
  And, that \aben{the} applies.

  This gives rise to some tension.

  Possible resolution here is that agent expects things are as they would be if agent witnessed, but does not get to claim support.
  Issues only follow from claiming support.

  However, then the agent doesn't get to do anything with the proposition.

  Flipside is that claiming support is minimal.
  Agent does this to use the proposition in further reasoning, and only constraint is \ideaCS{}.
\end{note}

\begin{note}
  Addition:
  There's some parallel here with reflection.
\end{note}

\begin{note}[Reflection]
\begin{quote}
    Reflection states that agents should treat their future selves as experts or, roughly, that an agent’s current credence in any proposition A should equal his or her expected future credence in A.\linebreak
    \mbox{}\hfill\mbox{(\Citeyear[59]{Briggs:2009up})}
  \end{quote}
\end{note}

\begin{note}[Difference to reflection]
  Key difference is that in these cases, there's no guarantee that the agent will go through with ability.
  So, it's not necessarily a future self of the agent.
  Though, that's only on a quick surface reading of reflection.

  This is somewhat delicate.
  For, reflection has some strong background assumptions.
  Problem with ability is that agent might witness ability.
  With reflection, we don't consider restrictions on the reasoning the agent would do.

  Now, weakening reflection is difficult.

  One the one hand, can consider all evidence that the agent would reason through.
  If so, then it looks as though ability is going to fall within the scope.
  Problem here, however, because the argument for reflection is in terms of coherence.
  And, it's not clear how to apply conditionalisation to boundedness.
  Dutch books are about coherent credence functions.

  So, it is not clear that there's a way to derive instances of \aben{the} from principles which motivate reflection.
\end{note}

\begin{note}
  Taking a step back, in these kinds of cases it's something like evidence of evidence.
  This is in \textcite[2]{Tal:2017uw}, linking to another paper.

  And, this is kind of similar to what's going on with ability.

  This only works easily from \AR{} perspective.
  Still\dots

  Things get complex here.
  For, if this is the case, then it's not clear that the agent needs to worry about \aben{the}.
  So, the issues arising from the matrix don't really apply.

  Point here is that the agent could go straight for general ability.

  Problem is that agent still needs to get specific ability.

  Same issues with \ESU{} and \nI{} apply here.
  Still get something appealed to but not used.
\end{note}

\begin{note}
  Had scenarios.
  Here, just added that there are close things in the literature which seem fine.
\end{note}


\section{Reject \nI{}}
\label{sec:reject-ni}

\begin{note}
  Follows from \ideaCS{}, mostly.

  And, \ideaCS{} seems quite plausible.
\end{note}

\begin{note}
  Inclined to discount similarities to \citeauthor{Wright:2011wn} and \citeauthor{Weisberg:2010to}.
  Enough of a different.
  And, these kind of things are very difficult.
\end{note}

\begin{note}
  Rather, motivation and \illu{0}.
\end{note}

\begin{note}
  Response here to focus on the difficult cases.
  But, as noted, because \nI{} is only about a way of claiming support, there are various ways in which one may deal with those cases.
\end{note}

\section{Reject \ESU{}}
\label{sec:reject-esu}

\begin{note}
  The literature.

  And, intuitive appeal.

  Maybe that's enough.
  Still, I want to see argumentation.
\end{note}

\section{Outlook}
\label{sec:outlook}

\begin{note}
  Reject \AR{} basic and \ESU{}.

  This leaves us with two options for understanding \aben{the}.
\end{note}

\section{\AR{}}
\label{sec:ar-2}

\begin{note}
  I don't have too much to say here.

  Outlined the idea of \AR{}.

  Considered this in terms of evidence of evidence.
  Ability then doing the work of making that evidence evidence for the agent.

  Role for ability here is securing that there is evidence.
  For, without ability, agent doesn't get to establish these background conditions.
\end{note}

\section{\WR{}}
\label{sec:wr-2}

\begin{note}
  Favoured
\end{note}


%%% Local Variables:
%%% mode: latex
%%% TeX-master: "master"
%%% End: