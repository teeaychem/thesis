\chapter{An equivalent statement of \zSN{2}: \zetaS{}}

\begin{note}
  In this chapter, an equivalent statement of \zSN{2}: \zetaS{}.

  \zSN{2} developed in \autoref{cha:zS}.
  And, {\color{link} as noted}, stated in plain terms.
  Whether an agent would conclude.
\end{note}

\section{\zetaS{}}
\label{cha:zS:sec:zetaS}

\begin{note}
  \autoref{cha:zS} introduced a question ---\qzS{}, and defined a property --- \zS{} in terms of positive answers to the question.

  Here, equivalent statement.
\end{note}


\subsection{\zetaS{}}
\label{sec:zs2}

\begin{note}
  With the notion of a \requ{} in hand, we now state when an agent has \zSN{0} for some proposition-value pair:

  \begin{idea}[\izetaS{}]
    \label{idea:zetaS}
    An agent \vAgent{} has \emph{\zetaS{}} for a proposition-value pair \(\pv{\phi}{v}\) when concluding \(\pv{\phi}{v}\) from some pool of premises \(\Phi\) just in case:
    \begin{itemize}
    \item When concluding \(\pv{\phi}{v}\) from \(\Phi\):
      \begin{enumerate}[label=\arabic*., ref=\named{CS:\arabic*}]
      \item
        \label{idea:zetaS::requ}
        For any proposition-value-premises pairing \(\pvp{\psi}{v'}{\Psi}\) which is a \requ{} of concluding \(\pv{\phi}{v}\) from \(\Phi\) either:
        \begin{enumerate}[label=\alph*., ref=\named{CS:1.\alph*}]
        \item
          \label{idea:zetaS::requ-sat:Past}
          \vAgent{} has concluded \(\pv{\psi}{v'}\) from \(\Psi\).
        \item
          \label{idea:zetaS::requ-sat:Pres}
          In concluding \(\pv{\phi}{v}\), \vAgent{} simultaneously concludes \(\pv{\psi}{v'}\) from \(\Psi\).
        \item
          \label{idea:zetaS::requ-sat:Forgone}
          \(\pvp{\psi}{v'}{\Psi}\) is a \fc{0}.
        \end{enumerate}
      \end{enumerate}
    \end{itemize}
    \vspace{-\baselineskip}
  \end{idea}
\end{note}

\begin{note}
  \begin{proposition}[Equivalence between \zS{} and \zetaS{}]
    \label{prop:qzs-tick-equals-iCS}
    For an agent \vAgent{}, the following are equivalent:
    \begin{enumerate}[label=\arabic*., ref=(\arabic*)]
    \item
      \label{prop:qzs-tick-equals-iCS:qzS}
      \vAgent{} has \zS{} for \(\pv{\phi}{v}\) after concluding \(\pv{\phi}{v}\) from \(\Phi\).
      (\qzS{} had a negative resolution when concluding \(\pv{\phi}{v}\) from \(\Phi\))
    \item
      \label{prop:qzs-tick-equals-iCS:ZS}
      \vAgent{} has \zetaS{} for \(\pv{\phi}{v}\) after concluding \(\pv{\phi}{v}\) from \(\Phi\).
    \end{enumerate}
    \vspace{-\baselineskip}
  \end{proposition}
  In other words, we hold that an agent ruling out failure to conclude \(\pv{\phi}{v}\) from \(\Phi\) for any \requ{} \(\pvp{\psi}{v'}{\Psi}\) is \emph{equivalent}%
  \footnote{
    In the context of a negative resolution to \qzS{}.
  }
  to the agent either
  \begin{enumerate*}[label=(\alph*)]
  \item
    having had concluded \(\pv{\psi}{v'}\) from \(\Psi\), or
  \item
    the agent simultaneously concluding \(\pv{\psi}{v'}\) from \(\Psi\) when concluding \(\pv{\phi}{v}\) from \(\Phi\).
  \item
    \(\pvp{\psi}{v'}{\Psi}\) being a \fc{0}.
  \end{enumerate*}
\end{note}

\paragraph*{Observations}

\paragraph*{Contraposition}

\begin{note}[Contraposition]
  An argument for~\autoref{prop:qzs-tick-equals-iCS} is important from the perspective of the overall argument of this document.

  Suppose we have~\autoref{prop:qzs-tick-equals-iCS}.
  Then, view \izetaS{} as a clarification of \qzS{}.

  Specifically, the left-to-right direction.

  Only negative resolution if \izetaS{}.
  I.e.\ only if concluded, for any \requ{}.

  \begin{itemize}
  \item
    If an agent has concluded \(\pv{\phi}{v}\) from \(\Phi\) \emph{without} having concluded \(\pv{\psi}{v'}\) from some \(\Psi\), where \(\pv{\psi}{v'}\) is a \requ{} of concluding \(\pv{\phi}{v}\) from \(\Phi\), then the agent has not \csVed{} for \(\pv{\phi}{v}\) from \(\Phi\).
  \end{itemize}

  Taking the contraposition, if not \izetaS{}, then no negative resolution.
  If intuitions are unclear, then \izetaS{} offers a way to clarify those intuitions.

  Conversely, fix a negative resolution to \qzS{}, then from left-to-right direction, draw out what must also be the case.%
  \footnote{
    Consider, by analogy, knowledge, and the idea that knowledge is closed under known entailment.
    \begin{quote}
      If \vAgent{} knows both
      \begin{enumerate*}[label=(\roman*)]
      \item \(\phi\), and
      \item \(\phi\) entails \(\psi\),
      \end{enumerate*}
      then \vAgent{} knows \(\psi\).
    \end{quote}
    Observe the same dynamics are present.

    If whether an agent knows both \(\phi\) and \(\phi\) entails \(\psi\) is at issue, then establishing the agent does not know \(\psi\) establishes that either the agent does not know \(\phi\) or the agent does not know \(\psi\).

    Conversely, if an agent knows both \(\phi\) and \(\phi\) entails \(\psi\), then by closure of knowledge under known entailment, the agent must also know \(\psi\).

    Of course, whether knowledge \emph{is} closed under known entailment is unclear, but the same dynamics hold for any similar condition.
    In general, these observations amount to little more than both the closure of knowledge under known entailment and \izetaS{} both having the general form of a conditional \(A \Rightarrow B\), such that:
    \begin{itemize}
    \item
      From \(A \Rightarrow B\) and \(A\), one may infer \(B\), and
    \item
      From \(A \Rightarrow B\) and \emph{not}-\(B\), one may infer \emph{not}-\(A\).
    \end{itemize}
  }

  \begin{itemize}
  \item
    If an agent has \zetaS{} for \(\pv{\phi}{v}\) from \(\Phi\), by concluding \(\pv{\phi}{v}\) from \(\Phi\) then, has concluded \(\pv{\psi}{v'}\) from \(\Psi\), for any \(\pvp{\psi}{v'}{\Psi}\) which is a \requ{} of concluding \(\pv{\phi}{v}\) from \(\Phi\).
  \end{itemize}

  In outline, path to tension.
  Cases in which \qzS{} has negative resolution.
  Draw out as a consequence of~\autoref{prop:qzs-tick-equals-iCS} that the agent has concluded.
  So long as cases in which no witnessing, we will have tension.
  On the one hand, negative resolution, and on the other hand, has not witnessed reasoning.

  Now, if some weaker, that does not require concluding, then lack a way to generate tension.

  So,~\autoref{prop:qzs-tick-equals-iCS} should be treated with some caution.

  Of course, this does not guarantee anything interesting.
  Tension will still depend on such cases.
\end{note}

\begin{note}
  With the importance of~\autoref{prop:qzs-tick-equals-iCS} motivated, we now turn to arguing for~\autoref{prop:qzs-tick-equals-iCS}.

  The argument we provide for~\autoref{prop:qzs-tick-equals-iCS} is somewhat involved, and will go via an intermediary lemma.
  We being by outlining the structure of the argument, before turning to the details.
\end{note}

\paragraph*{Same time}

\begin{note}[Importance of at same time]
  Propositional logic.
  These premises allow to conclude two things.
  Then, conclusion that \(\phi \land \psi\) is simultaneously a conclusion that \(\phi\) and that \(\psi\).

  Or, apples in a bag.
  Five.
  Well, could do at least four, three, etc.
  Conclude at the same time.
\end{note}

\begin{note}
  For example, counterexample for some formula of propositional logic.
  Constructed a truth table.
  Identified a line.
  If counterexample, then line makes any tautology of propositional logic true.
  And, do not need to appeal to the line being a counterexample to the relevant formula to do so.
  reason from line to any recognised tautology.
  Conclude, tautology would be true.
\end{note}




\subparagraph*{Almost-transitive}

\begin{note}
  Key idea that may be obvious is that \zetaS{} is almost-transitive.
  If it needs to be the case that I'd conclude \(\pv{\psi}{v'}\) from \(\Psi\) to get \(\pv{\phi}{v}\) from \(\Phi\), and \(\pv{\chi}{v''}\) from \(X\) to get \(\pv{\psi}{v'}\) from \(\Psi\) then, need to get \(\pv{\chi}{v''}\) from \(X\) to get \(\Phi\), and \(\pv{\chi}{v''}\).

  However, this is not quite right.
  For, it may be the case that the agent has the option of getting to \(\pv{\psi}{v'}\) from \(\Psi\) from the different branch.
\end{note}

\begin{note}
  Only about the option of concluding.
  There are various other properties.
  In this respect, \qzS{} is narrow.
\end{note}

\begin{note}
  Most reasoning is short, and composes.
  Further, it needs to be the case that novel proposition-value pair introduces this concern.
  So, some extraneous proposition.
  This is irrelevant.
\end{note}

\begin{note}[Recursion]
   {
    \color{red}
    Here, it is important that we don't go fully recursive.
    For, we're only interested in \requ{1} arising from concluding \(\pv{\phi}{v}\) from \(\Phi\).
    This means that it is no immediate.
    In particular, it may be the case that some of \(\Phi\) remove what would otherwise be a \requ{} of some \requ{}.
    But, then, this is still a \requ{} with respect to the current epistemic state.

    So, I think I actually get the result that this is recursive.
    However, with a slight change.
    For, \ref{idea:zetaS::requ-sat:Past} looks to the past, and instead of \csVImp{} in the past, it only need to be the case that the agent has \csVed{} from present epistemic state.

    The point is, that there may be some pruning without concluding.
    For, something fails to be a \requ{}.
    Likewise, it may be the case that the tree grows, as the agent's epistemic state develops.

    So, we don't get a clear recursive clause.

    This leads to an observation.
    Reasoning about a \requ{} may lead to a revision of the agent's epistemic state.

    This is a somewhat interesting consequence.
    We began with the idea of failing to conclude \(\pv{\phi}{v}\) from \(\Phi\).
    This said nothing about revision.
    However, we see that this raises the possibility of revision.

    From a different perspective, this should come as no surprise.
    If the agent concluded without \csN{}, then this is going to remain a problem for any further conclusions, unless the agent figures out they were mistaken regarding the proposition-value-premises pairing being a \requ{}.
  }
\end{note}

\section{Equivalence between \qzS{} and \izetaS{}}
\label{sec:overview:an-equiv-stat-of-zs}

\begin{note}
  We began this section with the statement of a question ---~\autoref{question:zs}, or \qzS{}  --- concerning whether something is the case when an agent concludes some proposition-value pair \(\pv{\phi}{v}\) from some pool of premises \(\Phi\).
  We then reformulated the question in terms of a property, \zS{}.

  The terminology of \zS{}, or \zSN{0}, is certainly artificial.
  Still, I take the question, and the content of \zS{} to be fairly intuitive.

  Roughly, at issue is whether there is some proposition-value-premises pairing which would prevent an agent from concluding some proposition-value pair from some pool of premises.

  Some technical concepts have been appealed to in order to clarify the question --- in particular with respect to our assumptions regarding concluding, and our focus on a fixed epistemic state --- but the question itself is fairly natural.

  Consider again the \illu{1} regarding a lost pair of keys.
  There does seem to be a difference between an exhaustive consideration of all the places the keys may be before concluding the keys are lost on the one hand, and concluding the keys are lost while expecting some place on hasn't looked will come to mind on the other.
  Or, concluding that something happened based on a friends story by passive acceptance on the one hand, and concluding the thing happened after checking for consistency on the other.

  So, we take \qzS{} to be an intuitive question, and \zS{} to be an intuitive property.
  Non-standard, perhaps, but still intuitive.
\end{note}

\begin{note}
  Our attention now turns to providing an equivalent statement of \zS{}, or in other words necessary and sufficient conditions for negative resolution to \qzS{}.%
  \footnote{
    Indirectly, necessary and sufficient conditions for a positive answer, by negating either side.
  }
  We term this equivalent statement of \zS{}, `\izetaS{}', or `\zetaS{}' for short.

  \zetaS{} will have a key role in our main argument for a negative resolution to~{\color{red} issue:Main}.

  Given the prominent role \zetaS{} will have in our main argument, we will take some additional care in stating \zetaS{}.
  In particular, we will provide an separate characterisation of the relevant \(\pvp{\psi}{v'}{\Psi}\) proposition-value-premises pairings of interest.
  These we will term `\requ{1}' of concluding \(\pv{\phi}{v}\) from \(\Phi\).
  And we will include additional discussion of some subtleties regarding \requ{1} given our assumption regarding concluding from~\autoref{chapter:concluding}.

  The following account of \zetaS{} will then be relatively straightforward.
  We hold that an agent has \zetaS{} for some proposition-value pair \(\pv{\phi}{v}\) with respect to some pool of premises \(\Phi\) just in case the agent has concluded \(\pv{\psi}{v'}\) from \(\Psi\) for any \requ{} \(\pvp{\psi}{v'}{\Psi}\) of concluding \(\pv{\phi}{v}\) from \(\Phi\).

  As noted in the introduction, this leads to a closure condition.
  If an agent has \zetaS{} for some proposition-value pair \(\pv{\phi}{v}\) with respect to some pool of premises \(\Phi\), then it will follow that an agent has concluded \(\pv{\psi}{v'}\) from \(\Psi\) for any \requ{} \(\pvp{\psi}{v'}{\Psi}\) of concluding \(\pv{\phi}{v}\) from \(\Phi\).
  Before arguing that \zetaS{} is equivalent to \zS{}, we will walk through this observation in some detail.
\end{note}

\begin{note}
  Still, it is important to note the (proposed) link between \qzS{}, \zS{}, and \zetaS{}.

  \zetaS{} has the potential to be a strong condition, \emph{if} we show that \zetaS{} applies to some instance of concluding \(\pv{\phi}{v}\) from \(\Phi\) where the agent has not witnessed reasoning from \(\Psi\) to \(\pv{\psi}{v'}\) for some \requ{} \(\pvp{\psi}{v'}{\Psi}\).%
  \footnote{
    I take it this, by itself, is not immediate.
  }
  However, even if \zetaS{} turn out to be a strong condition, it remains motivated by, and --- granting the arguments to follow --- equivalent to a fairly intuitive question.
\end{note}

\begin{note}[Intuition]
  This may seem a trivial point, but I think it is important to keep in mind.
  We have introduced \zS{} via \qzS{}, and it may be easy to grant us authority over what \zS{} is, or what \qzS{} asks.
  However, the interrogative core of \qzS{} is stated in neutral terms.
  So long as you have some intuitive understanding of `concluding', then, granting sufficient information about an agent's epistemic state, you are in position to determine whether \qzS{} has a negative or positive answer, and hence whether and agent has \zS{} for the relevant proposition-value-premises pairing.

  Our argument for the equivalence between \zS{} and \zetaS{} will not further specify the content of \qzS{}.
  Instead, \zetaS{} will provide an alternative characterisation of \zS{} from which we will draw further consequences.
  In short, it will be up to you to evaluate whether the \zetaS{} really is an alternative characterisation of \zS{}.
\end{note}

\paragraph*{The argument for \autoref{prop:qzs-tick-equals-iCS}}

\begin{note}
  We now turn to providing an argument for~\autoref{prop:qzs-tick-equals-iCS}.
  To do so, we argue for an equivalent proposition focusing on \qzS{} and \zetaS{}:

  \begin{proposition}
    \label{prop:qzs-tick-equals-iCS:var}
    For any property \(\chi\):
    \begin{enumerate}[label=\Alph*.]
    \item
      \label{squish:A}
      \squishA{An}{the}
    \end{enumerate}
    \emph{if and only if}:
    \begin{enumerate}[label=\Alph*.,resume]
    \item
      \label{squish:B}
      \squishB{the}.
    \end{enumerate}
    \vspace{-\baselineskip}
  \end{proposition}

  \Autoref{prop:qzs-tick-equals-iCS:var} states that an agent satisfying \izetaS{} is a necessary condition of the agent satisfying an property \(\chi\) which is sufficient to ensure a negative resolution to~\autoref{question:zs}.

  The purpose of arguing for~\autoref{prop:qzs-tick-equals-iCS} via~\autoref{prop:qzs-tick-equals-iCS:var} is option to contrast some property \(\chi\) with \zetaS{}.

  Specifically, if we contrapose the left-to-right direction we have some property \(\chi\) which does not entail satisfaction of \izetaS{}.
  And, we will argue that any such property \(\chi\) is insufficient for a negative resolution to \qzS{}.

  Indeed, the left-to-right direction is the only direction of significant interest with respect to~\autoref{prop:qzs-tick-equals-iCS:var}.
  The right-to-left direction requires little effort.

  This is not to say the argument for the left-to-right direction will be watertight.
  Rather, we will focus on localised tension.
  Still, we are not ready to provide the argument for~\autoref{prop:qzs-tick-equals-iCS:var} just yet.

  First we need to note the equivalence between~\autoref{prop:qzs-tick-equals-iCS} and~\autoref{prop:qzs-tick-equals-iCS:var}.

  Then, with the equivalence in hand, we may turn to arguing for~\autoref{prop:qzs-tick-equals-iCS:var}.
\end{note}

\paragraph*{The equivalence between~\autoref{prop:qzs-tick-equals-iCS} and~\autoref{prop:qzs-tick-equals-iCS:var}}

\begin{note}
  \begin{proposition}
    Equivalence between~\autoref{prop:qzs-tick-equals-iCS} and~\autoref{prop:qzs-tick-equals-iCS:var}.
  \end{proposition}
\end{note}

\begin{note}
  To see the equivalence between~\autoref{prop:qzs-tick-equals-iCS} and~\autoref{prop:qzs-tick-equals-iCS:var} observe that ~\autoref{prop:qzs-tick-equals-iCS} and~\autoref{prop:qzs-tick-equals-iCS:var} entail one another.

  For, assume~\autoref{prop:qzs-tick-equals-iCS} holds.
  Then, an agent satisfying \izetaS{} is both necessary and sufficient for a positive resolution to~\qzS{}.

  Now, take some property \(\chi\) and assume~\ref{squish:A} holds.
  Given~\ref{squish:A} holds, \(\chi\) is sufficient to positively resolve~\qzS{}.
  And, by~\autoref{prop:qzs-tick-equals-iCS}, a positive resolution to \qzS{} entail the agent satisfies \izetaS{}.
  Therefore, whenever the agent satisfies \(\chi\), the agent also satisfies \izetaS{}.
  So,~\ref{squish:B} holds.

  Conversely, take some property \(\chi\) and assume~\ref{squish:B} holds.
  Then, if the agent satisfies \(\chi\), the agent also satisfies \izetaS{}.
  Given~\autoref{prop:qzs-tick-equals-iCS}, satisfying \izetaS{} is sufficient to resolve~\autoref{question:zs}.
  So,~\ref{squish:A} holds.

  Now assume~\autoref{prop:qzs-tick-equals-iCS:var} holds.

  First, suppose the agent has resolved \qzS{}.
  Then, the agent satisfies some property \(\chi\), sufficient to resolve \qzS{}.
  Therefore, by~\autoref{prop:qzs-tick-equals-iCS:var}, we have that the agent also satisfies \izetaS{}.
  Hence, satisfying \izetaS{} is a necessary condition of resolving \qzS{}.

  Second, suppose the agent satisfies \izetaS{}.
  It is immediate that satisfaction of \izetaS{} entails satisfaction of \izetaS{}.
  Therefore, satisfaction of \izetaS{} is sufficient for a positive resolution to~\qzS{}.

  This tells us \izetaS{} is necessary.
  And, as an immediate consequence, \izetaS{} is sufficient.
  For, holds for all sufficient conditions.
  Though, the most straightforward argument for the right-to-left direction involves establishing this directly.
\end{note}

% \begin{note}
%   \footnote{
%     Symbolically, represent this as follows.

%     First, shorthand for \(\pvp{\psi}{v'}{\Psi}\) being a \requ{} of \(\pvp{\phi}{v}{\Phi}\):
%     \begin{itemize}
%     \item \(\pvp{\psi}{v'}{\Psi} \rleadsto \pvp{\phi}{v}{\Phi}\)
%     \end{itemize}
%     Now, quantify:
%     \begin{itemize}
%     \item \(\forall \pvp{\psi}{v'}{\Psi}\colon (\pvp{\psi}{v'}{\Psi} \rleadsto \pvp{\phi}{v}{\Phi})\)
%     \end{itemize}
%     Concludes
%     \begin{itemize}
%     \item \(\mathcal{C}(\pv{\psi}{v'},\psi)\)
%     \end{itemize}
%     Statement that the agent concludes or has concluded \(\pv{\psi}{v'}\) from \(\Psi\) for all \(\pvp{\psi}{v'}{\Psi}\) such that \(\pvp{\psi}{v'}{\Psi}\) is a \requ{} of \(\pvp{\phi}{v}{\Phi}\).
%     \begin{itemize}
%     \item \(\forall \pvp{\psi}{v'}{\Psi}\colon (\pvp{\psi}{v'}{\Psi} \rleadsto \pvp{\phi}{v}{\Phi}) \rightarrow \mathcal{C}(\pv{\psi}{v'},\Psi)\)
%     \end{itemize}
%     Let this be \(\mathcal{c}\).

%     Now, \csN{}.
%     \begin{itemize}
%     \item \(\mathsf{CS}(\pvp{\phi}{v}{\Phi})\)
%     \end{itemize}
%     And, sufficient to resolve question:
%     \begin{itemize}
%     \item \(R_{S}(\chi,?\mathsf{CS}(\pvp{\phi}{v}{\Phi}))\)
%     \end{itemize}
%     So, getting \(\chi\) is sufficient to resolve the question of whether the agent has \csVed{} for \(\pv{\phi}{v}\) from \(\Phi\).

%     First, we get
%     \begin{itemize}
%     \item \(R_{S}(\mathcal{c},?\mathsf{CS}(\pvp{\phi}{v}{\Phi}))\)
%     \end{itemize}
%     Concluding is sufficient.
%     From this, anything that entails \(\mathcal{c}\) is sufficient.

%     \begin{itemize}
%     \item \(\forall \chi((\chi \rightarrow c) \rightarrow R_{S}(\chi,?\mathsf{CS}(\pvp{\phi}{v}{\Phi})))\)
%     \end{itemize}

%     Conversely, if some \(\chi\) is sufficient, \(\chi\) entails \(\mathcal{c}\)

%     \begin{itemize}
%     \item \(\forall \chi(R_{S}(\chi,?\mathsf{CS}(\pvp{\phi}{v}{\Phi})) \rightarrow (\chi \rightarrow c))\)
%     \end{itemize}

%     From the two above:
%     \begin{itemize}
%     \item \(\forall \chi(R(\chi,?\mathsf{CS}(\pvp{\phi}{v}{\Phi})) \leftrightarrow (\chi \rightarrow c))\)
%     \end{itemize}
%   }
%   \end{note}

\paragraph*{The argument for~\autoref{prop:qzs-tick-equals-iCS:var}}

\begin{note}
  Given the equivalence between~\autoref{prop:qzs-tick-equals-iCS} and~\autoref{prop:qzs-tick-equals-iCS:var}, we now turn to arguing for~\autoref{prop:qzs-tick-equals-iCS:var}.

  As suggested above, we split the argument into two parts.
  The right-to-left direction and the left-to-right direction.

  We begin with the right-to-left direction as it is mostly straightforward.
  We then turn to the left-to-right direction, which is significantly more involved.
\end{note}

\paragraph*{Right-to-left}

\begin{note}
  For the right-to-left direction our goal is to establish:

  For any property \(\chi\):
  \begin{quote}
  \begin{enumerate}
    \item[B.]
      \squishB{an}
    \end{enumerate}
    \emph{implies}
    \begin{enumerate}
    \item[A.]
      \squishA{The}{the}.
    \end{enumerate}
  \end{quote}
\end{note}

\begin{note}
  Fix an agent, take some property \(\chi\), and assume \squishB{the}.

  Now, \(\chi\) entails the agent satisfies \izetaS{}, but \(\chi\) may also be \zetaS{}.
  Indeed, \zetaS{} is the minimal property for which satisfaction of \(\chi\) entails \zetaS{}.
  So, though \(\chi\) is arbitrary, we must show that satisfaction of \zetaS{} alone is sufficient to resolve whether the agent has \zS{} for \(\pv{\phi}{v}\) when concluding \(\pv{\phi}{v}\) from \(\Phi\).

  Still, this is relatively straightforward.
  \qzS{} asks whether there is some \requ{} \(\pvp{\psi}{v'}{\Psi}\) such that the agent has not settled whether \(\pv{\psi}{v'}\) follows from \(\Psi\).

  Now, for any such \(\pvp{\psi}{v'}{\Psi}\), satisfying \izetaS{} requires one of the following to conditions holds:
  \begin{itemize}
  \item
    The agent has concluded \(\pv{\psi}{v'}\) from \(\Psi\).
  \item
    When concluding \(\pv{\phi}{v}\) from \(\Phi\) the agent also concludes \(\pv{\psi}{v'}\) from \(\Psi\).
  \end{itemize}
  In both cases, when concluding \(\pv{\psi}{v'}\) from \(\Psi\) for any such \(\pvp{\psi}{v'}{\Psi}\) the agent will have concluded \(\pv{\psi}{v'}\) from \(\Psi\).
  And, given a conclusion of \(\pv{\psi}{v'}\) from \(\Psi\), it seems clear the agent has settled whether \(\pv{\psi}{v'}\) follows from \(\Psi\).
  For, the agent has concluded \(\pv{\psi}{v'}\) from \(\Psi\)!
  Indeed, at question is whether the agent would conclude \(\pv{\psi}{v'}\) from \(\Psi\), and so there is no clearer answer to this question than concluding \(\pv{\psi}{v'}\) from \(\Psi\).

  So, satisfying clauses of \izetaS{} is sufficient for a negative resolve~\qzS{}.
  For, grant that notion of a \requ{0} captures the relevant cases, \izetaS{} requires concluding.
\end{note}

\subsection{Left-to-right}

\begin{note}
  For the left-to-right direction our goal is to establish:

  For any property \(\chi\):
  \begin{quote}
  \begin{enumerate}
    \item[A.]
      \squishA{An}{the}
    \end{enumerate}
    \emph{implies}
    \begin{enumerate}
    \item[B.]
      \squishB{the}.
    \end{enumerate}
  \end{quote}

  To do so, we take some arbitrary property \(\chi\), and then contrapose the conditional.
  In other words, assume \(\chi\) does not entail the agent has satisfied \izetaS{} with the sub-goal of showing that an agent satisfying \(\chi\) is sufficient to resolve whether the agent has \(\zS{}\) for \(\pv{\phi}{v}\) when concluding \(\pv{\phi}{v}\) from \(\Phi\).

  To aid the clarify of the argument we begin by introducing a plausible candidate for \(\chi\).
  We will then use the candidate to develop the core of the argument, and to conclude we will observe how the argument generalises.
\end{note}


\subsubsection{\izetaSm{}, a candidate for \(\chi\)}
\label{overview:sec:iCS-iCSm-limitation-closure}

\begin{note}
  \begin{idea}[\izetaSm{2} --- \izetaSm{}]
    \label{idea:Zsm}
    An agent \vAgent{} has \izetaSm{2} for \(\pv{\phi}{v}\) with respect to some pool of premises \(\Phi\) \emph{only if}:
    \begin{enumerate}[label=\arabic*., ref=\named{\(\zeta^{-}\)S:\arabic*}]
    \item
      \label{idea:Zsm:requ}
      For any proposition-value pair \(\pv{\psi}{v'}\) which is a \requ{} of concluding \(\pv{\phi}{v}\) from \(\Phi\) either:
      \begin{enumerate}[label=\alph*., ref=\named{\(\zeta^{-}\)S:1.\alph*}]
      \item
        \label{idea:Zsm:requ-sat:Past}
        \vAgent{} holds \emph{\vAgent{} would conclude} \(\pv{\psi}{v'}\) from the relevant pool of premises \(\Psi\).
      \item
        \label{idea:Zsm:requ-sat:Pres}
        In concluding \(\pv{\phi}{v}\) \vAgent{} \emph{also} holds \emph{\vAgent{} would conclude} \(\pv{\psi}{v'}\) from the relevant pool of premises \(\Psi\).
      \end{enumerate}
    \end{enumerate}
    \vspace{-\baselineskip}
  \end{idea}
\end{note}

\begin{note}[Difference between \izetaS{} and \izetaSm{}]
  The difference between \izetaS{} and \izetaSm{} is straightforward.

  \izetaS{} requires an agent concludes \(\pv{\psi}{v'}\) from \(\Psi\) for any \requ{} \(\pvp{\psi}{v'}{\Psi}\).
  By contrast, \izetaSm{} requires an agent holds \emph{they would conclude \(\pv{\psi}{v'}\) from \(\Psi\)}.%
  \footnote{
    Expressed differently, the agent concluding \(pv{\psi}{v'}\) from \(\Psi\) is replaced with the agent concluding that they would (conclude \(\pv{\psi}{v'}\) from \(\Psi\)).
    Here, the parentheses indicate that in both~\ref{idea:Zsm:requ-sat:Past} and~\ref{idea:Zsm:requ-sat:Pres} the agent is not required to conclude anything from \(\Psi\) directly.
  }

  The expression of an agent concluding that they would conclude may be somewhat stilted, but expresses a simple idea.
  Rather than concluding \(\pv{\psi}{v'}\) from \(\Psi\), the agent concludes that if they were to reason about whether \(\pv{\psi}{v'}\) follows from \(\Psi\), they would conclude \(\pv{\psi}{v'}\) from \(\Psi\).
  Note, \izetaSm{} does not detail the relevant pool of premises that the agent draw the conclusion from.
\end{note}

\begin{note}[\izetaS{} is (intuitively) stronger than \izetaSm{}]
  In particular, \izetaS{} is intuitively stronger than \izetaSm{}.

  First, observe that~\ref{idea:zetaS::requ-sat:Past} and~\ref{idea:zetaS::requ-sat:Pres} (plausibly) entail ~\ref{idea:Zsm:requ-sat:Past} and~\ref{idea:Zsm:requ-sat:Pres}, respectively.
  For, if an agent has concluded \(\pv{\psi}{v'}\) from \(\Psi\), then \emph{in so doing} the agent has shown that they would conclude \(\pv{\psi}{v'}\) from \(\Psi\).

  Second, observe neither~\ref{idea:Zsm:requ-sat:Past} nor~\ref{idea:Zsm:requ-sat:Pres} (plausibly) entail~\ref{idea:zetaS::requ-sat:Past} nor \ref{idea:zetaS::requ-sat:Pres}, respectively, so long as there are plausible cases in which an agent concludes that they would conclude \(\pv{\phi}{v}\) from \(\Phi\) without concluding \(\pv{\phi}{v}\) from \(\Phi\).

  Indeed, it seems there are plausible cases.

  For, it seems an agent may conclude that they would conclude \(\pv{\phi}{v}\) from \(\Phi\) without concluding \(\pv{\phi}{v}\) from \(\Phi\).
  For example, one may be informed that one would conclude \(\pv{\phi}{v}\) from \(\Phi\) via testimony.
  Hence, the only relevant premises one plausibly requires is that they have been informed they would conclude \(\pv{\phi}{v}\) from \(\Phi\) via testimony, and \(\Phi\) may be arbitrary.
  E.g.\ I tell you that if you looked at the map you would conclude that East Palo Alto is directly north of Palo Alto (shock!) and if you trust the map you may even conclude that East Palo Alto \emph{is} directly north of Palo Alto.
  Still, you do not (obviously) conclude East Palo Alto is directly north of Palo Alto from any premises associated with details of the map.
\end{note}

\begin{note}
  \izetaSm{} as a candidate \(\chi\).
  {
    \color{red}
    Following, arbitrary \(\chi\).
    However, substitute in \izetaSm{} is desired.
  }
\end{note}

\begin{note}
  Now, we have seen how~\ref{idea:Zsm:requ-sat:Past} and~\ref{idea:Zsm:requ-sat:Pres} are (plausibly) \emph{strictly} weaker than~\ref{idea:zetaS::requ-sat:Past} and~\ref{idea:zetaS::requ-sat:Pres}, respectively.
  And, we have noted that both~\ref{idea:Zsm:requ-sat:Past} and~\ref{idea:Zsm:requ-sat:Pres} seem to align with the motivation provided from \csN{}.
  We briefly highlight why the distinction between ~\ref{idea:Zsm:requ-sat:Past} and~\ref{idea:Zsm:requ-sat:Pres} and ~\ref{idea:zetaS::requ-sat:Past} and~\ref{idea:zetaS::requ-sat:Pres}, respectively, will matter for our overall argument.

  Observe, \csN{}%
  \footnote{
    As stated, with~\ref{idea:zetaS::requ-sat:Past} and~\ref{idea:zetaS::requ-sat:Pres} over ~\ref{idea:Zsm:requ-sat:Past} and~\ref{idea:Zsm:requ-sat:Pres}, respectively.
  }
  will lead to tension with a positive resolution to~{\color{red} issue:Main} just in case we manage to find an instance in which an agent \csN{} for \(\pv{\phi}{v}\) from \(\Phi\) without witnessing reasoning from \(\Psi\) to \(\pv{\psi}{v'}\) for some \requ{} \(\pvp{\psi}{v'}{\Psi}\) of concluding \(\pv{\phi}{v}\) from \(\Phi\).
  However, this tension will not follow if the weakened variants of~\ref{idea:zetaS::requ-sat:Past} and~\ref{idea:zetaS::requ-sat:Pres} are adopted.
  For, neither~\ref{idea:Zsm:requ-sat:Past} nor~\ref{idea:Zsm:requ-sat:Pres} would require the agent to conclude \(\pv{\psi}{v'}\) from \(\Psi\).

  Indeed, we will argue for the existence of cases of exactly the kind described.
  Hence, the role of~\ref{idea:zetaS::requ-sat:Past} and~\ref{idea:zetaS::requ-sat:Pres} over~\ref{idea:Zsm:requ-sat:Past} and~\ref{idea:Zsm:requ-sat:Pres} is not merely an issue of motivation, but also crucial to establishing tension.
\end{note}

\paragraph*{Arguing}

\begin{note}
  Fix an agent.
  Take arbitrary \(\chi\), such that \(\chi\) does not imply satisfaction of \izetaS{}.
  (E.g.\ \izetaSm{}, as seen above.)
\end{note}

\begin{note}
  Now, assume the agent is concluding \(\pv{\phi}{v}\) from \(\Phi\).
  There are two cases to consider.
  We state the each case for \(\chi\) generally, and then below state the case with respect to \izetaSm{}.
  \begin{enumerate}[label=\Roman*., ref=\Roman*]
  \item
    \label{iZm:arg:case:I}
    The agent satisfies \(\chi\) when concluding \(\pv{\phi}{v}\) from \(\Phi\).
    \begin{itemize}
    \item
      The agent concludes they would conclude \(\pv{\psi}{v'}\) from \(\Psi\) when concluding \(\pv{\phi}{v}\) from \(\Phi\), for any \(\pvp{\psi}{v'}{\Psi}\) which is a \requ{} of concluding \(\pv{\phi}{v}\) from \(\Phi\).
    \end{itemize}
  \item
    \label{iZm:arg:case:II}
    The agent has already satisfied \(\chi\) prior to concluding \(\pv{\phi}{v}\) from \(\Phi\).
    \begin{itemize}
    \item
      The agent has (already) concluded they would conclude \(\pv{\psi}{v'}\) from \(\Psi\) when concluding \(\pv{\phi}{v}\) from \(\Phi\), for any \(\pvp{\psi}{v'}{\Psi}\) which is a \requ{} of concluding \(\pv{\phi}{v}\) from \(\Phi\).
    \end{itemize}
  \end{enumerate}

  We take each case in turn.
  Further, as \(\chi\) does not imply satisfaction of \izetaS{}, and \izetaS{} is not trivially satisfied, for both cases we will assume the agent does not satisfy \izetaS{}.
\end{note}

\subparagraph*{Case~\ref{iZm:arg:case:I}}

\begin{note}[With \(\chi\)]
  \(\chi\) is sufficient for a negative answer to \qzS{}.

  Key observation.
  In order for \(\chi\) to be sufficient, agent's perspective.

  \begin{proposition}
    \label{prop:chiProp:no-may-fail}
    \(\chi\) must ensure that from the agent's perspective, the agent may not fail to conclude \(\pv{\psi}{v'}\) from \(\Psi\).
    \begin{argument}
      Direct from \qzS{}.
      For, negative answer.
      However, negative answer only if it is not the case that the agent may fail to conclude \(\pv{\psi}{v'}\) from \(\Psi\).
    \end{argument}
  \end{proposition}

  However, from~\ref{question:zs:option}, the agent has the option of concluding \(\pv{\psi}{v'}\) from \(\Psi\).

  By assumption the agent has not yet satisfied \(\chi\), as the agent has not yet concluded \(\pv{\phi}{v}\) from \(\Phi\).
  Hence, the agent has a check on whether they would come to satisfy \(\chi\) when concluding \(\pv{\phi}{v}\) from \(\Phi\).

  For, if the agent reasons about whether \(\pv{\psi}{v'}\) follows from \(\Psi\) and fails to conclude \(\pv{\psi}{v'}\) from \(\Psi\), then it would not be the case that the agent satisfies \(\chi\).

  Rephrase.
  \(\chi\) is sufficient for a negative answer to \qzS{}.
  In order for a negative answer, it may not be the case that the agent may fail to conclude \(\pv{\psi}{v'}\) from \(\Psi\).
  Therefore, the agent not failing, from their perspective, is required to satisfy \(\chi\).
  However, this means that before concluding \(\pv{\phi}{v}\) from \(\Phi\), the agent may establish whether they would come to satisfy \(\chi\).

  So, this means that there is a possibility of branching.
  If the agent reasons about whether \(\pv{\psi}{v'}\) follows from \(\Psi\), then the agent may fail to conclude \(\pv{\phi}{v}\) from \(\Phi\).

  Whether the agent would conclude \(\pv{\phi}{v}\) from \(\Phi\) is at issue.
  And, the agent only gets \(\chi\) when concluding.
  So, whether or not \(\chi\) cannot be sufficient for a negative answer.

  If failing is sufficient for positive answer, then failing is also sufficient to show that the agent does not satisfy \(\chi\).
  Therefore, \(\chi\) cannot be sufficient.
\end{note}

\begin{note}[With \izetaSm{}]
  From \izetaSm{}.
  \(\pvp{\psi}{v'}{\Psi}\) is a \requ{} of concluding concluding.
  Hence, so long as the agent has not already\dots it follows that concluding concluding is not sufficient for a negative answer to \qzS{}.
\end{note}

\begin{note}[Summary]
  Summary.
  \qzS{}, what would happen if the agent first reasoned about whether \(\pv{\psi}{v'}\) follows from \(\Psi\).
  This is the question.
  In order for negative answer, from agent's point of view, would not fail.
  However, question still holds if this is not yet the agent's point of view.
\end{note}

\begin{note}[Observation]
  Observe, as not yet \(\chi\), re-expressed \qzS{} as a question about whether \(\chi\).
  Therefore, the above argument only applies given we are in case~\ref{iZm:arg:case:I}.
  In case~\ref{iZm:arg:case:II}, we assume the agent already satisfies \(\chi\), and hence the argument will be distinct.
\end{note}

\subparagraph*{Case~\ref{iZm:arg:case:II}}

\begin{note}
  {
    \color{red}
    This should be revised, as with the idea of a \fc{0}, I no longer need to worry about the possibility of revision.

    Instead, the basic argument is that if a \requ{0}, then this still applies to whatever \(\chi\) is.
    For anything weaker, check.
  }
\end{note}

\begin{note}
  Now turn to case~\ref{iZm:arg:case:II}.

  With case~\ref{iZm:arg:case:I}, argued that \(\chi\) fails to be sufficient, because \qzS{} applies equally to whether the agent satisfies \(\chi\).
  Possibility of present reasoning branching so the agent does not satisfy \(\chi\).

  Case~\ref{iZm:arg:case:II} requires distinct argument, as by assumption the agent satisfies \(\chi\).
  Again, we have the assumption that the agent has not concluded \(\pv{\psi}{v'}\) from \(\Psi\).
\end{note}

\begin{note}
  Observe, the strategy applied to case~\ref{iZm:arg:case:I} does not extend to case~\ref{iZm:arg:case:II}.
  For, \requ{} of concluding \(\pv{\phi}{v}\) from \(\Phi\).
  Hence, it need not be the case that the agent had the option of concluding \(\pv{\psi}{v'}\) from \(\Psi\) when satisfying \(\chi\).
  In particular, when establishing from their perspective they would not fail to conclude.

  From the perspective of \izetaSm{}, concluded would conclude when the agent did not have the option of concluding.

  For example, consider a case of (apparent) testimony.
  If follow strategy, win game.
  Did not have an understanding of the rules.
  Hence, did not have the option to reason from premises and reach a different conclusion.
  Only after coming to understand rules (or more strictly, adopting the perspective of understanding rules) does the option of evaluating the conditional become available.
\end{note}

\begin{note}
  Our strategy is to split case~\ref{iZm:arg:case:II} into two sub-cases, depending on whether the agent may revise whether or not the agent may revise their satisfaction of \(\chi\).
  Specifically:

  \begin{enumerate}[label=\roman*., ref=\roman*]
  \item
    \label{iZm:arg:case:II:sub:i}
    From the agent's perspective:
    The agent may revise their epistemic state so that the agent does not satisfy \(\chi\), given the agent's current epistemic state.
  \item
    \label{iZm:arg:case:II:sub:ii}
    From the agent's perspective:
    The agent may not revise their epistemic state so that the agent does not satisfy \(\chi\), given the agent's current epistemic state.
  \end{enumerate}

  It may seem only sub-case~\ref{iZm:arg:case:II:sub:ii} is compatible with satisfaction of \(\chi\).

  For, as we have observed in~\autoref{prop:chiProp:no-may-fail}, not the case that the agent may fail.

  Rewriting:

  \begin{enumerate}[label=\roman*\('\)., ref=\roman*\('\)]
  \item
    \label{iZm:arg:case:II:sub:i:var}
    From the agent's perspective:
    The agent may fail to conclude \(\pv{\psi}{v'}\) from \(\Psi\), if the agent were to attempt to conclude \(\pv{\psi}{v'}\) from \(\Psi\).
  \item
    \label{iZm:arg:case:II:sub:ii:var}
    From the agent's perspective:
    The agent may not fail to conclude \(\pv{\psi}{v'}\) from \(\Psi\), if the agent were to attempt to conclude \(\pv{\psi}{v'}\) from \(\Psi\).
  \end{enumerate}

  So, it may appear only sub-case~\ref{iZm:arg:case:II:sub:ii:var} is compatible with the agent satisfying \(\chi\).

  However, it is important to keep in mind the scope of the relevant instance of `may'.
  Given \(\chi\), failure to conclude \(\pv{\psi}{v'}\) from \(\Psi\) may be ruled out from the agent's perspective.
  However, it may also be the case that the agent entertains the possibility of failing to satisfy \(\chi\).
  Hence, as the agent may not satisfy \(\chi\), the agent may fail to conclude \(\pv{\psi}{v'}\) from \(\Psi\).%
  \footnote{
    Of course, if you interpreted \qzS{} in line with sub-case~\ref{iZm:arg:case:II:sub:ii}, you may ignore sub-case~\ref{iZm:arg:case:II:sub:i}.

    Still, I take \qzS{} to be compatible with both sub-cases~\ref{iZm:arg:case:II:sub:i} and~\ref{iZm:arg:case:II:sub:ii}.
  }
\end{note}

\begin{note}[What we will argue]
  Respectively, we will argue:
  \begin{itemize}
  \item
    For sub-case~\ref{iZm:arg:case:II:sub:i}, \(\chi\) is insufficient for a negative answer to \qzS{}.
  \item
    For sub-case~\ref{iZm:arg:case:II:sub:ii}, re-assignment of concluding, hence \(\chi\) (trivially) entails concluding. Hence, conflict with our assumption that \(\chi\) does not entail concluding.
  \end{itemize}

  In other words:
  \begin{itemize}
  \item
    \qzS{} scopes over certain revisions to an agent's epistemic state.
  \item
    If no revision, then the relation between \(\pv{\psi}{v'}\) and \(\Psi\) is sufficient to reduce the relevant instance of concluding to \(R\) and witnessing.
  \end{itemize}

  \(R\) the agent would conclude \(\pv{\psi}{v'}\) from \(\Psi\) if the agent were to reason from \(\Psi\) to \(\pv{\psi}{v'}\), and from the agent's current epistemic state, \(R\) may not fail to hold.

  Here, importance of~\ref{idea:reassignment}.
  Re-assignment.
  The only thing for the agent to do is witness \(R\).
\end{note}

\subparagraph*{\(\pvp{\psi}{v'}{\Psi}\) remains a \requ{0} of concluding \(\pv{\phi}{v}\) from \(\Phi\), given \(\chi\)}

\begin{note}
  We begin with a minor, but important observation.

  \begin{proposition}
    \(\pvp{\psi}{v'}{\Psi}\) remains a \requ{0} of concluding \(\pv{\phi}{v}\) from \(\Phi\), given \(\chi\)
  \end{proposition}
\end{note}

\begin{note}[Still a \requ{}]
  Observe that \(\pvp{\psi}{v'}{\Psi}\) is a \requ{} with respect to concluding \(\pv{\phi}{v}\) from \(\Phi\).

  Even if agent has satisfied \(\chi\), and so from perspective, still holds up.
  Indeed, whether \(\pvp{\psi}{v'}{\Psi}\) is \requ{} is independent of whether the agent satisfies \(\chi\) or has concluded \(\pv{\psi}{v'}\) from \(\Psi\).

  For a \requ{}, what matters is option and failure.
  And, for a negative resolution to \qzS{}, no failure.

  Of course, from agent's perspective they would not fail.
  However, if reason and did fail, problem.

  So, from the present point of view, a \requ{}.
\end{note}

\paragraph{The sub-cases}

\subparagraph*{Sub-case~\ref{iZm:arg:case:II:sub:i}.}

\begin{note}[Distinction between \(\chi\) and concluding]
  There is an important difference between the two cases.
  By assumption, \(\chi\), has not concluded.

  So with \(\chi\), two types of possible epistemic states.
  Concluded, not concluded.
  Further, possibility of either of these epistemic states.

  Point is, with \(\chi\), because no entailment, and assumption, these two types of epistemic state are open.

  From present, the agent only expect to go to one.
  However, this is from the perspective of current epistemic state.

  By contrast, with concluding, the agent already in the relevant epistemic state.
  The agent has concluded \(\pv{\psi}{v'}\) from \(\Psi\).
\end{note}

\begin{note}
  As the agent is not in the relevant type of epistemic state, and has the option to go to the right epistemic state, \qzS{}.
\end{note}

\begin{note}
  At issue is whether the agent may do some reasoning any end up not concluding \(\pv{\phi}{v}\) from \(\Phi\) granted they have not already concluded \(\pv{\psi}{v'}\) from \(\Psi\).
  And, as the agent has not concluded \(\pv{\psi}{v'}\) from \(\Psi\), then regardless of perspective on how reasoning would go, the option is present, and as \(\chi\) does not entail, the agent does not have the result of taking the option.
  Taking the option would give something distinct.
  Option, so figure out whether.
\end{note}

\begin{note}
  Counterpoint.

  Do not need to drop the pen in order to know that it will fall to the ground.

  Or, perhaps, do not need to go and check my car is parked outside in order to know that it is parked outside.

  Difference here is these cases involve acquiring novel information, while \qzS{} involves reasoning with respect to the agent's current epistemic state.
\end{note}

\begin{note}
  Although we have introduced the possibility of revision, it is very narrow.
  Our interest is not with revision in general.
  The idea that an agent would conclude \(\pv{\phi}{v}\) from \(\Phi\) given arbitrary revision is incredibly strong.

  Rather, the revision is quite specific.
  It is because the agent has concluded \emph{that}, and because this is now a \requ{}.
  This does not scope over querying arbitrary premises, nor adopting addition premises.
  Only because this question remains open.
\end{note}


\subparagraph*{Sub-case~\ref{iZm:arg:case:II:sub:ii}.}

\begin{note}
  No revision.

  So, from the agent's perspective, if reason from \(\Psi\), would conclude \(\pv{\psi}{v'}\).
\end{note}

\begin{note}
  Now, by assumption witnessing can't do anything.
  There can be nothing added by witnessing that would matter to concluding \(\pv{\psi}{v'}\) from \(\Psi\).
  If there were, then possibility of revision.

  Okay, so, this means that witnessing doesn't add anything.

  Hence, whether on not X is determined by the agent's present epistemic state.
  This does not mean X, as X may be the result of witnessing.
  However, must have enough.
  So, reduce whether to something independent of witnessing.




  Some X, but then, agent's present epistemic state secures X.







  So, have something in the agent's present epistemic state.
  
\end{note}

\begin{note}
  So, sufficient resources given present epistemic state.

  For, if insufficient, then fail to conclude.

  So, witnessing, putting those resources into action.

  So, take any X.
  If X makes a difference to concluding, then, availability of X + witnessing.

  So, something, X.
  Get this by witnessing.
  However, if this makes a difference, then split into pre-X and witnessing X.
  Given sub-case, no failure.
  So, only pre-X makes a difference.
  Yet, have pre-X.
  Adding in witnessing won't do anything.

  So, then, split.

  Not possible to find some X, such from the agent's point of view, such that whether X is unique to witnessing reasoning from \(\Psi\) to \(\pv{\psi}{v'}\) such that X matters to whether conclude, and X is not composite.

  This is quite strong, but does not generalise easily.
  For, if possibility of branching, revision, etc.\ then there may be some such X.
  Indeed, agent has not concluded, might fail, so conclusion may add something unique.

  But, then, re-assignment.

  Hence, concluded.

  To \illu{1}, deterministic causation.
  Some causal model, a bunch of equations.
  Wouldn't say X has caused Y given application of equations.
  However, nothing more than putting those equations in motion.

  Note, though, that given re-assignment, this is what we get.
  What we're after is this reduction.
  At issue is not the intuitive sense of `concluding', but whether there is some reduction of this kind.

  Observe, this kind of thing does give a reduction.
  However, no clear cases of this happening.
  Too much information required.
  And, deterministic, at least with respect to description at level of premises and conclusions.
  Strong assumptions.

  Further, we have no eliminated role of witnessing.
  As with causation, we have no shown that this relation is specified independently.

  However, we do have a static reduction.
  Nothing of relevance is introduced by the dynamics.
  We might not get a specification without reference to the dynamics, but we don't get anything of relevance from the dynamics themselves.

  This is basically a redescription of the assumption made.
  If there is no possibility of failing, then the dynamics are pre-determined, at least from the agent's perspective.





  Sufficient resources plus witnessing.
  These resources, no possibility of failure.
  So, for anything that does not involve witnessing, we have this.

  In addition, for anything witnessing would get, the agent has a guarantee that they would obtain.

  Not perfect information.
  Maybe exact premises, steps of reasoning, etc.
  Still, all of these are available.
\end{note}



\begin{note}[Edge case]
  \(\chi\), the agent reasons, but does not conclude \(\pv{\psi}{v'}\) from \(\Psi\).
  Nor does the agent fail to conclude \(\pv{\psi}{\overline{v'}}\) from \(\Psi\).

  However, no revision.
  Well, then the agent won't conclude \(\pv{\phi}{v}\) from \(\Phi\).
  If the agent does, then revised epistemic state so that \ref{question:zs:subjunctive} does not hold.
  Hence, \qzS{} would no longer apply.
\end{note}

\begin{note}[Key idea]
  \begin{itemize}
  \item
    \(\chi\) does not entail concluded.
  \item
    Option to conclude.
  \item
    If agent were to reason, to types resulting states.
  \item
    Concluded, failed to conclude.
    Really, various possible resulting states, as reasoning may not terminate either way, and in general there may be various values other than \(v'\) that the agent may conclude.
  \item
    The agent isn't in either type of epistemic state.
  \item
    Now, \(\chi\) states from current perspective, what the result would be.
  \item
    However, it remains the case that these two types of epistemic states are open.
  \item
    Hence, it remains the case that reasoning could lead to either type.
  \item
    So, it remains the case that the agent may fail to conclude \(\pv{\psi}{v'}\) from \(\Psi\).
  \end{itemize}

  The core of the idea is that \qzS{} concerns the dynamics.
  If reason, then what would the result be with respect to concluding \(\pv{\phi}{v}\) from \(\Phi\).
  So, concluding to dynamics.
  Hence, \(\chi\) of the appropriate kind does not settle.
\end{note}

\subsection{No revision to the agent's present epistemic state}
\label{sec:no-revision-agents}

\begin{note}
  Reflect on this constraint.

  This is what stops the argument from going all the way.

  But, there are good reasons for this.

  Difficulty with figuring out what would be the case under revisions to present epistemic state.

  In particular, the relevant conditionals.
\end{note}

\begin{note}
  Look, it's really really hard.
  For, the conditionals, or that the agent has the ability are key.
  Without these, there's no check on the agent's present reasoning.

  Hence, these can't be up for grabs when it comes to checks.

  If still possible to check whether it's really the case that\dots

  WAIT!

  There's a difference between the conditional and the consequent of the conditional.

  This should be part of the argument for getting to it being a \fc{}.

  Okay, this is interesting.
  Though, it gets tricky.
  Because, here, option to appeal to general ability.
  General ability, so conditional.
  This then would lead to the same problem.

  So an example here is testimony.
  Though, this is delicate.
  If testimony comes in just before concluding, then problem.
  However, general ability is testified.
  Then, get this as an instance, where there's really no question whether this is an instance.

  Though, here, this means prior to everything, expanded general to all specific.
  I don't see a problem, generally speaking, here.

  So, a \fc{}.
  From the agent's present epistemic state it's already been settled that the agent would conclude, if they were to reason.
  This is not something that the agent is concluding now.
  If it was, then there would be an issue, for check on whether this conclusion really makes sense.

  So, motivate a shift prior, to when the agent concluded this.
  Now, there's various cases.

  Though, as all I want is the move to general, all I need is that the agent concluded.

  Well, hang on, these are now just a special case.
  That's interesting.
  So, in general, \fc{}.
  But, given this, move back to when not a \fc{}.
  If you allow this move to happen, then there's difficulty.
  So, plausibly this doesn't actually do much.
\end{note}

\section{Notes}

\begin{note}
  Important to note, that still all relative to the agent's present epistemic state, and what the agent is interested in concluding.
  For, it may be that the agent revises their epistemic state so that some relation does not hold.
  For example, the agent has concluded that some reduction holds.
  Therefore, if they prove this, then they prove something else, and, also prove the something else by other means.
  However, learn this reduction does not hold.
  Now, no longer a \requ{}.
  Indeed, relation fails because the reduction does not hold, or because the agent does not have the means to prove.
\end{note}

\paragraph*{Closing}

\begin{note}[Closure]
  So, this is our motivation.
  What we have is an intuitive idea which leads to this kind of limitation, and hence conclusre condition.
  So, if there are cases of interest, motivation that an agent concludes.
  But, conversely, the condition is strong.
  So, these cases are harder.
\end{note}

\paragraph{More details on \zetaS{}}

\begin{note}
  We've only focused on failure to conclude.
  However, the agent may also conclude something else.
  Possible that there are premises for the agent \(\Psi\) and \(\overline{\Psi}\) such that from \(\Psi\) get \(\pv{\psi}{v'}\) and from \(\overline{\Psi}\), get \(\pv{\psi}{\overline{v'}}\).

  For sure, but this is a different condition.
  You may also want to impose this given the intuitive motivation for \csN{}.
  Indeed, from the perspective of no branching.
  Distinct condition, and we will not impose this.

  Note, also, that so long as distinction between concluding \emph{that} and concluding, then this is also going to be insufficient in isolation.
  For, though the agent may have exhausted other possibilities, this won't get a conclusion.
  And, if not distinct, then a plausible path to negative resolution.
\end{note}

\begin{note}
  \izetaS{} does not require the conclusion to be any good.
  If you want to build this in, sure.
  However, not for us.
  It is a strong assumption, and would have no function in the arguments to follow.
\end{note}

\subsection{Literature}
\label{sec:zS:literature}

\paragraph{Circularity}

\begin{note}
  \ideaCS{} is not about circular reasoning in the sense that the term `circularity' suggests that the reasoner has taken the conclusion of the reasoning for granted.

  There's nothing in \ideaCS{} that appeals to getting \(\psi\) having value \(v'\) from \(\phi\) having value \(v\).

  However, does identify a problem in the sense that would prevent the agent from getting \(\psi\) having value \(v'\) from \(\phi\) having value \(v\).
\end{note}

\begin{note}[Testimony 1]
  \begin{illustration}[Testimony 1]
    \label{illu:CS:test:basic}
    \mbox{}
    \begin{enumerate}[label=\arabic*., ref=(\arabic*)]
    \item
      \label{ex:eiS:t:basic:test}
      \nagent{11} stated that they are trustworthy when speaking on matters regarding their personal character.
    \item
      \label{ex:eiS:t:basic:ok}
      \nagent{11} is trustworthy when speaking on matters regarding their personal character.
    \end{enumerate}
  \end{illustration}
  This kind of case is intuitively problematic.
  It seems that already need trustworthy.
  However, in order for \csN{} to apply, need for it to be the case that one has some check on whether \nagent{11} is trustworthy.
  And, by reasoning.

  This need not be the case.
  Of course, this does not mean that an agent need \csN{}.
  May be other necessary conditions.
\end{note}

\paragraph{Sgaravatti}

\begin{note}
  For example, consider what \citeauthor{Sgaravatti:2013wu} terms the `Justification Account' of circularity.\nolinebreak
  \footnote{
    As \citeauthor{Sgaravatti:2013wu} notes, the Justification Account of circularity is a rewriting of the third type of `epistemic dependence' considered by \citeauthor{Pryor:2004ws}~(\citeyear[359]{Pryor:2004ws}).
    Neither \citeauthor{Pryor:2004ws} nor \citeauthor{Sgaravatti:2013wu} endorse the Justification Account, but I take the spirit of the account to sufficient for interest.
    Still, the considerations which follow also apply to distinguish the {\color{red} problem identified} from \citeauthor{Sgaravatti:2013wu}'s favoured account (\Citeyear[\S3]{Sgaravatti:2013wu}) and the fifth type of `epistemic dependence' considered by \citeauthor{Pryor:2004ws}~(\citeyear[359]{Pryor:2004ws}).
  }

  \begin{quote}
    \begin{enumerate}[label=(JA), ref=(JA)]
    \item\label{sg:JA} An argument is circular if and only if for you to have justification to believe the premisses, it is necessary that you have justification to believe the conclusion.\nolinebreak
      \mbox{}\hfill\mbox{(\Citeyear[754]{Sgaravatti:2013wu})}
    \end{enumerate}
  \end{quote}
  Where `justification to believe' is to be read as in terms of having formed the belief in an epistemically appropriate way as opposed to (merely) possessing sufficient resources to form formed the belief in an epistemically appropriate way.\nolinebreak
  \footnote{
    Or, however you prefer to characterise \citeauthor{Firth:1978vi}'s (\Citeyear{Firth:1978vi}) distinction between doxastic and propositional justification (or warrant).
    See also \citeauthor{Silva:2020aa} (\Citeyear{Silva:2020aa}) --- esp.\ fn.\ 1.
  }
  (\citeauthor[Cf.][754--755]{Sgaravatti:2013wu})
\end{note}

\begin{note}
  First, reliance on something like justification.

  With \support{}, we arguably have something distinct.
  Have not placed constraints on reasoning.
  Hence, \ideaCS{} applies even when no justification (or any other epistemic attribute) is found.

  Indeed, to the extent that the value \(v\) need not be truth, \ideaS{} and \ideaCS{} are broader.

  Point extends to relation between the premises and the conclusion of a step of reasoning.
  There's some issue with whether there's a clear reduction to premises.

  Now, both these points may be addressed by linking justification to steps of reasoning.
  However, it still remains that get this kind of circularity by placing a constraint on permissible steps of reasoning.
\end{note}

\begin{note}
  Second, having something.
  Contrasts to reasoning in an interesting way.
\end{note}

\begin{note}[\citeauthor{Sgaravatti:2013wu} on necessity]

  \begin{quote}
    For my present purposes it will suffice to say that a good test of A’s being necessary for B (and thus of B’s being sufficient for A) is the satisfaction of two subjunctive conditionals. First, if A did not hold, B would not hold; secondly, if B were to hold, A would hold.%
    \mbox{}\hfill\mbox{(\citeyear[761]{Sgaravatti:2013wu})}
  \end{quote}
  This is very similar to what is captured by a \requ{}.

  Also, points out only a test due to implications.
  For us, \requ{} is not a test.
  And, the implications are embraced.
  Though, differences limit these somewhat.

  For the moment, point with the implications is that this makes \zS{} fairly strong.
\end{note}

\paragraph{Pryor}

\begin{note}[\citeauthor{Pryor:2004ws}'s Type 4]
  An instance of a limitation arising from assuming that the possibility obtains is the fourth type of dependence between premise and conclusion considered by \citeauthor{Pryor:2004ws}.

  \begin{quote}
    [Type 4] dependence between premise and conclusion is that the conclusion be such that evidence \emph{against it} would (to at least some degree) undermine the kind of justification you purport to have for the premises.\nolinebreak
    \mbox{}\hfill\mbox{(\citeyear[359]{Pryor:2004ws})}
  \end{quote}

  Again, plausible.\nolinebreak
  \footnote{
    A variant of \citeauthor{Pryor:2004ws}'s Type 4 dependence is~\citeauthor{Jackson:1984vk}'s account of circularity.
    \begin{quote}
      [I]t may be that a given argument to a given conclusion is such that anyone --- or anyone sane --- who doubted the conclusion would have background beliefs relative to which the evidence for the premises would be no evidence.\space \dots

      Such an argument could be of no use in convincing doubters, and is most properly said to beg the question.\nolinebreak
      \mbox{}\hfill\mbox{(\Citeyear[111-12]{Jackson:1984vk})}
    \end{quote}
    Still, in contrast to \citeauthor{Pryor:2004ws}'s Type 4, \citeauthor{Jackson:1984vk}'s account of circularity is dialectical.
    Indeed, on \citeauthor{Jackson:1984vk}'s account (without additional constraints on when an agent has justification or evidence) it need not be the case that the agent's own justification would be undermined by someone doubting the conclusion.
    In this respect, \ideaCS{} is further distinguished from a proposal such as \citeauthor{Jackson:1984vk}'s as \ideaCS{} makes mention only of the relevant agent's epistemic state and reasoning.
  }
  Further, weaken from justification to any reasoning.
  In this respect, motivated by \ideaS{}, plausibly.
  However, much stronger.
  \ideaS{} is just about entertaining.
  Subjunctive with stronger is less clear.

  Issue:
  \begin{enumerate}
  \item Evidence undermines the kind of justification the agent purports to have for the premises.
  \end{enumerate}

  And, as \citeauthor{Pryor:2004ws} notes, \emph{kind} is important.
  However, it seems kind is not the only problem.
\end{note}

\begin{note}
  \citeauthor{Pryor:2004ws}'s argument that type 4 over-generates is somewhat interesting.
  Details are in the following footnote.\footnote{
  Compatible with \citeauthor{Pryor:2004ws}'s objection to type 4 dependence.

  % \begin{illustration}
    % \mbox{}
    % \vspace{-\baselineskip}
    \begin{quote}
      Suppose you're watching a cat stalk a mouse. Your visual experiences justify you in believing:

      \begin{enumerate}[label=(\arabic*), ref=(\arabic*)]
        \setcounter{enumi}{10}
      \item
        \label{illu:Pryor:cat:1}
        The cat sees the mouse.
      \end{enumerate}

      You reason:

      \begin{enumerate}[label=(\arabic*), ref=(\arabic*), resume]
      \item
        \label{illu:Pryor:cat:2}
        If the cat sees the mouse, then there are some cases of seeing.
      \item
        \label{illu:Pryor:cat:3}
        So there are some cases of seeing.\nolinebreak
        \mbox{}\hfill\mbox{(\citeyear[361]{Pryor:2004ws})}
      \end{enumerate}
    \end{quote}
  % \end{illustration}

  Setting aside whether this is fine.

  Following \citeauthor{Pryor:2004ws}:

  Bad, given proposal, as if no cases of seeing, then the cat is not seeing. (\citeyear[361]{Pryor:2004ws})

  \citeauthor{Pryor:2004ws}'s position is as follows:

  \begin{quote}
    I don't think you need antecedent justification to believe \ref{illu:Pryor:cat:3}, before your experiences can give you justification to believe \ref{illu:Pryor:cat:1}.
    I also think it's plausible that your perceptual justification to believe \ref{illu:Pryor:cat:1} contributes to the credibility of \ref{illu:Pryor:cat:3}.\nolinebreak
    \mbox{}\hfill\mbox{(\citeyear[361]{Pryor:2004ws})}
  \end{quote}

  This may be compatible with \ideaS{} and \ideaCS{}.
  With \ideaCS{}, somewhat trivial, if \ref{illu:Pryor:cat:3} holds throughout \epVW{1}.

  More generally, weaker proposition.
  Hence, it seems \indicateV{1}.
  So there's no issue with the reasoning.
  However, `contributes to the credibility\dots'.
  }
\end{note}

\begin{note}[Issue]
  Somewhat similar to above.
  Here, however, role of novel information is of interest.
  Hence, dynamic.
  And, \csN{} is, in this respect, static.
\end{note}


\paragraph{Others}

\begin{note}
  This also extends to \citeauthor{Wright:2011wn}.
  For, \citeauthor{Wright:2011wn} relies on the idea of doubt.

  The issue here is what is required in order to doubt.
  One may need to revise one's epistemic state.

  Of course, if idea of claiming support is taken generally, then it should be the case that for any \epPW{}, it is possible for the agent to conclude from reasoning that \(\phi\) having value \(v\) holds for any \epVAd{} \world{}.

  So, if satisfy claiming support, then may satisfy doubt idea.
  However, ideal.
  Pointing out the issue here does not require such a general thing as doubt.
\end{note}

\begin{note}
  Instead, as \(\psi\) not having value \(v'\) is an \ep{}, it is possible that \(\psi\) does not have value \(v'\).
  And, if \(\psi\) does not have value \(v'\), then step \(\delta'\) does not apply to how things are.
  Hence, observing that \(\psi\) having value \(v'\) follows in turn from the conclusion of step \(\delta'\) (together with other premises) is uninformative about how things are.
\end{note}

\begin{note}
  \color{red}
  Some of the \citeauthor{Wright:2011wn} cases are interesting.
  Especially the twin cases.
  In fact, especially this idea that situations are identical.
  For, one way of understanding this is that the agent makes a choice between two disjuncts, and it is possible for the agent to make the other choice, and then come to a different conclusion.
\end{note}

\subsection{Summarising}

%%% Local Variables:
%%% mode: latex
%%% TeX-master: "master"
%%% End:
