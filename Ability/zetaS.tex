\chapter{Scraps}

\begin{note}
  Counterpoint.

  Do not need to drop the pen in order to know that it will fall to the ground.

  Or, perhaps, do not need to go and check my car is parked outside in order to know that it is parked outside.

  Difference here is these cases involve acquiring novel information, while \qzS{} involves reasoning with respect to the agent's current epistemic state.
\end{note}

\subsection{Literature}
\label{sec:zS:literature}

\paragraph{Pryor}

\begin{note}[\citeauthor{Pryor:2004ws}'s Type 4]
  An instance of a limitation arising from assuming that the possibility obtains is the fourth type of dependence between premise and conclusion considered by \citeauthor{Pryor:2004ws}.

  \begin{quote}
    [Type 4] dependence between premise and conclusion is that the conclusion be such that evidence \emph{against it} would (to at least some degree) undermine the kind of justification you purport to have for the premises.%
    \mbox{}\hfill\mbox{(\citeyear[359]{Pryor:2004ws})}
  \end{quote}

  Again, plausible.

  Issue:
  \begin{enumerate}
  \item Evidence undermines the kind of justification the agent purports to have for the premises.
  \end{enumerate}

  And, as \citeauthor{Pryor:2004ws} notes, \emph{kind} is important.
  However, it seems kind is not the only problem.
\end{note}

\begin{note}
  \citeauthor{Pryor:2004ws}'s argument that type 4 over-generates is somewhat interesting.
  Details are in the following footnote.\footnote{
  Compatible with \citeauthor{Pryor:2004ws}'s objection to type 4 dependence.

  % \begin{illustration}
    % \mbox{}
    % \vspace{-\baselineskip}
    \begin{quote}
      Suppose you're watching a cat stalk a mouse. Your visual experiences justify you in believing:

      \begin{enumerate}[label=(\arabic*), ref=(\arabic*)]
        \setcounter{enumi}{10}
      \item
        \label{illu:Pryor:cat:1}
        The cat sees the mouse.
      \end{enumerate}

      You reason:

      \begin{enumerate}[label=(\arabic*), ref=(\arabic*), resume]
      \item
        \label{illu:Pryor:cat:2}
        If the cat sees the mouse, then there are some cases of seeing.
      \item
        \label{illu:Pryor:cat:3}
        So there are some cases of seeing.\nolinebreak
        \mbox{}\hfill\mbox{(\citeyear[361]{Pryor:2004ws})}
      \end{enumerate}
    \end{quote}
  % \end{illustration}

  Setting aside whether this is fine.

  Following \citeauthor{Pryor:2004ws}:

  Bad, given proposal, as if no cases of seeing, then the cat is not seeing. (\citeyear[361]{Pryor:2004ws})

  \citeauthor{Pryor:2004ws}'s position is as follows:

  \begin{quote}
    I don't think you need antecedent justification to believe \ref{illu:Pryor:cat:3}, before your experiences can give you justification to believe \ref{illu:Pryor:cat:1}.
    I also think it's plausible that your perceptual justification to believe \ref{illu:Pryor:cat:1} contributes to the credibility of \ref{illu:Pryor:cat:3}.\nolinebreak
    \mbox{}\hfill\mbox{(\citeyear[361]{Pryor:2004ws})}
  \end{quote}

  This fine when seen from the perspective of the conditional being a \requ{}.
  }
\end{note}

\paragraph{Others}

\begin{note}
  This also extends to \citeauthor{Wright:2011wn}.
  For, \citeauthor{Wright:2011wn} relies on the idea of doubt.

  The issue here is what is required in order to doubt.
  One may need to revise one's epistemic state.

  Of course, if idea of claiming support is taken generally, then it should be the case that for any \epPW{}, it is possible for the agent to conclude from reasoning that \(\phi\) having value \(v\) holds for any \epVAd{} \world{}.

  So, if satisfy claiming support, then may satisfy doubt idea.
  However, ideal.
  Pointing out the issue here does not require such a general thing as doubt.
\end{note}

\begin{note}
  Instead, as \(\psi\) not having value \(v'\) is an \ep{}, it is possible that \(\psi\) does not have value \(v'\).
  And, if \(\psi\) does not have value \(v'\), then step \(\delta'\) does not apply to how things are.
  Hence, observing that \(\psi\) having value \(v'\) follows in turn from the conclusion of step \(\delta'\) (together with other premises) is uninformative about how things are.
\end{note}

\begin{note}
  \color{red}
  Some of the \citeauthor{Wright:2011wn} cases are interesting.
  Especially the twin cases.
  In fact, especially this idea that situations are identical.
  For, one way of understanding this is that the agent makes a choice between two disjuncts, and it is possible for the agent to make the other choice, and then come to a different conclusion.
\end{note}

%%% Local Variables:
%%% mode: latex
%%% TeX-master: "master"
%%% End:
