\chapter{\curb{3}}
\label{cha:zS:sec:curbs}

\begin{note}
  Idea with a \curb{} is that this is a necessary condition for event to be develop such that agent concludes.
\end{note}

\begin{note}
  Turn our attention to concluding.

  When concluding, certain things may happen.
  Agent gets interrupted, distracted, etc.

  In these cases, the agent is concluding, it just so happens that the event is interrupted.

  However, the agent may fail to be concluding.

  Reasoning doesn't work out.

  Assuming some competency no agent is concluding that White has a checkmate.
  For, it is not possible.

  Likewise, may be the case that agent has the option of concluding something else.
  Start with the definition of a \bCurb{}.
  Expand to \curb{}, and link \curb{1} to \fc{1}.
\end{note}

\section{\curb{3}}
\label{sec:curb}

\begin{note}
  Being with the idea of a \curb{}.
  Relates whether or not concluding to proposition-value-premise pairings.
\end{note}

\begin{note}
  \begin{definition}[A \curb{0}]
    \label{def:curb}
    For:

    \begin{itemize*}[noitemsep]
    \item
      An agent: \vAgent{}
    \item
      Propositions: \(\phi\), \(\psi\)
    \item
      Values: \(v\), \(v'\)
    \item
      \poP{3}: \(\Phi\), \(\Psi\)
    \item
      An event: \(e\)
    \item
      \mbox{ }
    \end{itemize*}

    \begin{itemize}
    \item
      \(\pvp{\phi}{v'}{\Psi}\) is a \emph{\curb{}} on \(e\) just in case:

      \begin{enumerate}[label=\alph*., ref=(\alph*), series=curbDefSeries]
      \item
        \(e\) is an event in which \vAgent{} is concluding \(\pv{\phi}{v}\) from \(\Phi\).
      \end{enumerate}

      \emph{Only if}

      % \begin{enumerate}
      % \item [\emph{If}:]
      \begin{enumerate}[label=\alph*., ref=(\alph*), resume*=curbDefSeries]
      \item
        \label{def:curb:opp}
        There is a \pevent{} from \(e\) in which \vAgent{} concludes \(\pv{\psi}{v'}\) from \(\Psi\).
      \end{enumerate}
      % \item[\emph{Then}:]
      %   \begin{enumerate}[label=\alph*., ref=(\alph*), resume]
      %   \item
      %     \label{def:curb:fail}
      %     \(e\) does not develop in an event in which \vAgent{} concludes \(\pv{\phi}{v}\) from \(\Phi\).%
      %     \mbox{ }\hfill(\emph{Due to}~\ref{def:curb:opp})
      %   \end{enumerate}
      % \end{enumerate}
    \end{itemize}
    \vspace{-\baselineskip}
  \end{definition}

  {
    \color{red}
    Only if, because don't get entailment from \pevent{} to concluding, that's far too strong.
  }

  So, \curb{1} are with respect to \emph{concluding}.
  In order for the event in which the agent concludes to be in progress, then must be \pevent{} from \(e\).

  \emph{Forward looking}.

  Clearest way to understand this is by considering how an event in which agent concludes develops.
  At the start, the only thing for sure is \pevent{} in which the agent concludes \(\pv{\phi}{v}\) from \(\Phi\).

  However, as the event develops, what is required becomes clearer.
  So, certain sub-conclusions are necessary.

  In general, no matter how things turn out in the details, what is common between all of the ways in which the event fully develops so that the agent concludes.
\end{note}

\begin{note}
  Now, in case of a \curb{}, it is not immediately clear that there are instances in which the relevant \pevent{} is not a sub-event of the event in which the agent concludes \(\pv{\phi}{v}\) from \(\Phi\).

  However, the definition makes sense.
  Indeed, it is very clear when applied to concluding.
\end{note}

\begin{note}
  This is a curb.
  Idea is simple, the agent may wonder about \(\pv{\phi}{v}\) from \(\Psi\).
  And, the agent may attempt to conclude.
  But, if doesn't conclude, then doesn't conclude \(\pv{\phi}{v}\) from \(\Phi\).
\end{note}

\begin{note}
  The parenthetical `due to' appended to~\ref{def:curb:fail} ensures that the agent not concluding \(\pv{\phi}{v}\) from \(\Phi\) is tied to failing to conclude \(\pv{\psi}{v'}\) from \(\Psi\) after taking the relevant opportunity.%
  \footnote{
    General problem of deviance.
    Subjunctive conditional, without statement, allows for failure to conclude to be unrelated.
    Or, a finkish disposition (\cite[cf.][144]{Lewis:1997wg}).
    % ~\cite{Lewis:1997wg}
    % \begin{quote}
    %   Anything can cause anything; so stimulus \emph{s} itself might chance to be the very thing that would cause the disposition to give response \emph{r} to stimulus \emph{s} to go away.
    %   If it went away quickly enough, it would not be manifested.
    %   In this way it could be false that if \emph{x} were to undergo \emph{s}, \emph{x} would give response \emph{r}.
    %   And yet, so long as s does not come along, \emph{x} retains its disposition.
    %   Such a disposition, which would straight away vanish if put to the test, is called finkish.%
    %   \mbox{ }\hfill\mbox{(\citeyear[144]{Lewis:1997wg})}
    % \end{quote}
  }
  Don't have a specific account of `due to'.
  Move to the level of theories, and overall goal is to provide a theory independent motivation for rejecting \issueConstraint{}.
  So, some difficulty, may wonder whether the conditional holds.
\end{note}

\section{\illu{3}}
\label{sec:illu3}

\subsection{Instances of concluding}
\label{sec:instances-concluding}

\subsubsection{Positive}
\label{sec:positive}

\paragraph{Trivial}

\begin{note}
  \begin{proposition}
    For an agent \vAgent{}, proposition-value pair \(\pv{\phi}{v}\), and \poP{} \(\Phi\):

    \begin{itemize}
    \item
      \(\pvp{\phi}{v}{\Phi}\) is a \curb{} of \vAgent{} concluding \(\pv{\phi}{v}\) from \(\Phi\).
    \end{itemize}
    \begin{argument}
      Immediate by \autoref{def:curb}.
      For, assume there is no \pevent{} in which \vAgent{} concludes \(\pv{\phi}{v}\) from \(\Phi\).
      Then, it is not the case that \(e\) develops into an event in which \vAgent{} concludes \(\pv{\phi}{v}\) from \(\Phi\).
      Hence, \(e\) is not an event in which \vAgent{} is concluding \(\pv{\phi}{v}\) from \(\Phi\).

      So, the conditional by which \curb{1} are defined is always true for \(\pv{\phi}{v}{\Phi}\).
    \end{argument}
  \end{proposition}

  For a specific example, chess.
\end{note}

\paragraph{Non-trivial}

\begin{note}
  Wally example.

  I'm going to ask whether Wally is usually carrying a cane.
\end{note}

\begin{note}
  Lost keys
\end{note}

\paragraph{Difficult}

\begin{note}
  Wason selection task.

  With a lazy agent.

  They have a number of cards, and will choose a handful at random.

  This is difficult.
  On the one hand, it seems all of the cards, if chosen at random.
  However, it is also possible that the agent only checks confirming cards.
  Hence, the agent is concluding.
\end{note}

\begin{note}[Prior to concluding\dots]
  Not particularly marked.
  Allow agent to have built up a bunch of stuff while reasoning.

  Example.

  \begin{scenario}[Velocity]
    \label{ill:velocity}
    Agent is provided with information about how far a car has travelled north as a function of time travelled.

    From this, take the derivative of the function to obtain the (instantaneous) velocity of the car at a handful of points in time.

    And, from the (instantaneous) velocity of the car, the agent calculates the (instantaneous) acceleration of the car at each of the points in time.

    The agent also has information about the speed of the car as a function of time travelled, and the agent may calculate speed by the taking magnitude of the (instantaneous) velocity of the car.
  \end{scenario}

  \autoref{ill:velocity}, two step calculation.
  Velocity, acceleration.
  After the first step, check by taking the magnitude.
  Calculation of velocity is correct only if taking the magnitude matches speed.

  So, two events for which the agent is concluding.
  Distinct \curb{1} associated with each event.
\end{note}

\subsubsection{Negative}
\label{sec:negative}

\begin{note}
  Anything completely unrelated.
\end{note}

\begin{note}
  Testimony, but too much to check.
\end{note}

\begin{note}
  Foggy day
\end{note}

\begin{note}[Problems of induction]
  Hence, the sketch does not apply to black ravens.
  I wouldn't conclude all ravens are black if I saw a white raven.

  I may worry about shortly seeing a white raven when concluding all ravens are black, and I may refuse to entertain the possibility that the sun will rise tomorrow when planning to mow the grass.

  However, it's not possible to reason to seeing a white raven, nor is it possible to reason to the sun not rising tomorrow.

  Abstractly, at issue in~\autoref{illu:lost-key} is the possibility of failing to a reason to some proposition-value pair given \emph{present} information, rather than the possibility of failing to a reason to some proposition-value pair given \emph{new} information.

  To the extent that problems of induction arise from receiving new information, what is at issue is distinct.%
  \footnote{
    See~\textcite{Henderson:2020wb} for more on the problem of induction.
  }

  Similar points for external world scepticism.
  Would not conclude that I have hand if disembodied brain in a vat.

  However, conclusion is out of reach.
\end{note}

\subsubsection{Notes}
\label{sec:notes}

\begin{note}
  Important to keep in mind \(\pv{\phi}{v}\) and \(\Phi\).

  At issue is not whether there is a \pevent{} in which the agent concludes \(\pv{\phi}{v}\), but a \pevent{} in which the agent concludes \(\pv{\phi}{v}\) from \(\Phi\).
\end{note}

\begin{note}
  Supply testimony, leads to additional premise.

  So, doing the calculation, \pevent{} in which do the calculation.

  Get testimony.

  In both cases, same conclusion, but distinct \poP{1}, and distinct \curb{1}.
\end{note}

\section{\curb{3} and \fc{1}}

\begin{note}
  \begin{proposition}[\curb{3} and \fc{1}]
    For an agent \vAgent{}, and proposition-value-premises pairings \(\pvp{\phi}{v}{\Phi}\), \(\pvp{\psi}{v'}{\Psi}\):

    \begin{itemize}
    \item
      \(\pvp{\phi}{v'}{\Psi}\) is a \emph{\curb{}} of concluding \(\pv{\phi}{v}\) from \(\Phi\)
    \end{itemize}

    \emph{Not If and only if}

    \begin{itemize}
    \item
      \(\pv{\psi}{v'}\) is a \fc{} of \(\Psi\).
    \end{itemize}
    \begin{argument}
      It is not the case that the b condition of \fc{} holds.
    \end{argument}
  \end{proposition}
\end{note}

% \begin{note}
%   In some cases, it may be the case that \(\pv{\phi}{v}\) follows trivially after getting \(\pv{\psi}{v'}\), this doesn't matter.

%   It may also be the case that \(\pv{\phi}{v}\) entails \(\pvp{\psi}{v'}{\Psi}\), but this is also fine, so long as there is an alternative way.
% \end{note}

\section{A missing link}
\label{cha:zS:sec:missing-link}

\begin{note}
  \autoref{prop:sCing} is important.
  For, what matters for the conclusion.
  If agent concludes \(\pv{\phi}{v}\) from \(\Phi\), then concluding \(\pv{\phi}{v}\) from \(\Phi\).

  So, suppose, \curb{}.
  Then, need it to be the case that \(\pv{\psi}{v'}\) from \(\Psi\) is a \fc{}.

  In this respect, \fc{} explains, in part, and in some sense, why the agent concludes \(\pv{\phi}{v}\) from \(\Phi\).

  However, that \(\pv{\psi}{v'}\) from \(\Psi\) is a \fc{} does not answer, in part, \qWhyV{}.
  For, no \ros{}.
  \ros{} from \agpe{}.
  Only get this from \(\pv{\psi}{v'}\) from \(\Psi\) being a \fc{}, from \agpe{}.

  In this respect, \curb{1} are compatible with \issueConstraint{}!
\end{note}

\section{Summary}
\label{cha:zS:sec:curbs:summary}


%%% Local Variables:
%%% mode: latex
%%% TeX-master: "master"
%%% End:
