\chapter{\curb{3}}
\label{cha:zS:sec:curbs}

\begin{note}
  Turn our attention to concluding.

  When concluding, certain things may happen.
  Agent gets interrupted, distracted, etc.

  In these cases, the agent is concluding, it just so happens that the event is interrupted.

  However, the agent may fail to be concluding.

  Reasoning doesn't work out.

  Assuming some competency no agent is concluding that White has a checkmate.
  For, it is not possible.

  Likewise, may be the case that agent has the option of concluding something else.
  Start with the definition of a \bCurb{}.
  Expand to \curb{}, and link \curb{1} to \fc{1}.
\end{note}

\section{\bCurb{3}}
\label{sec:bCurb}

\begin{note}
  Being with the idea of a \bCurb{}.
  Relates whether or not concluding to proposition-value-premise pairings.
\end{note}

\begin{note}
  \begin{definition}[A \bCurb{0}]
    \label{def:bCurb}
    For an agent \vAgent{}, and proposition-value-premises pairings \(\pvp{\phi}{v}{\Phi}\), \(\pvp{\psi}{v'}{\Psi}\):

    \begin{itemize}
    \item
      \(\pvp{\phi}{v'}{\Psi}\) is a \emph{\bCurb{}} for \(e\) to be an event in which \vAgent{} is concluding \(\pv{\phi}{v}\) from \(\Phi\).
    \end{itemize}

    \emph{If and only if}

    \begin{itemize}
    \item
        \begin{enumerate}
        \item[\emph{If}:]
          \begin{enumerate}[label=\alph*., ref=(\alph*)]
          \item
            \label{def:bCurb:opp}
            There is no \pevent{} from \(e\) in which \vAgent{} concludes \(\pv{\psi}{v'}\) from \(\Psi\).
          \end{enumerate}
        \item[\emph{Then}:]
          \begin{enumerate}[label=\alph*., ref=(\alph*), resume]
          \item
            \label{def:bCurb:fail}
            \(e\) is not an event in which \vAgent{} is concluding \(\pv{\phi}{v}\) from \(\Phi\).\newline
            \mbox{ }\hfill(\emph{Due to}~\ref{def:bCurb:opp})
          \end{enumerate}
      \end{enumerate}
    \end{itemize}
    \vspace{-\baselineskip}
  \end{definition}

  So, \curb{1} are defined with respect to \emph{concluding}.
  In order for the event in which the agent concludes to be in progress, then event must not diverge.

  At issue is what may occur, which is captured in terms of \pevent{1}.
\end{note}

\begin{note}
  Example, chess from above.
\end{note}

\begin{note}
  Indeed, \(\pvp{\phi}{v}{\Phi}\) is a \bCurb{} of an agent concluding \(\pv{\phi}{v}\) from \(\Phi\).
  Intuitively, if there is no \pevent{} in which the agent concludes \(\pv{\phi}{v}\) from \(\Phi\), then it is not possible for the agent to be concluding \(\pv{\phi}{v}\) from \(\Phi\).

  We make this observation a little with the following proposition:

  \begin{proposition}
    For an agent \vAgent{}, proposition-value pair \(\pv{\phi}{v}\), and \poP{} \(\Phi\):

    \begin{itemize}
    \item
      \(\pvp{\phi}{v}{\Phi}\) is a \bCurb{} of \vAgent{} concluding \(\pv{\phi}{v}\) from \(\Phi\).
    \end{itemize}
    \begin{argument}
      Immediate by \autoref{def:bCurb}.
      For, assume there is no \pevent{} in which \vAgent{} concludes \(\pv{\phi}{v}\) from \(\Phi\).
      Then, it is not the case that \(e\) develops into an event in which \vAgent{} concludes \(\pv{\phi}{v}\) from \(\Phi\).
      Hence, \(e\) is not an event in which \vAgent{} is concluding \(\pv{\phi}{v}\) from \(\Phi\).

      So, the conditional by which \bCurb{1} are defined is always true for \(\pv{\phi}{v}{\Phi}\).
    \end{argument}
  \end{proposition}
\end{note}

\begin{note}
  However, not restricted to \(\pvp{\phi}{v}{\Phi}\).

  Somewhat uninteresting case is where we take entailments.
  Wally example.

  But, more interesting.
  Apparent testimony.
  You say, I might check.
  So, here, I'm checking a premise used in testimony.
\end{note}

\begin{note}
  This is a basic curb.
  Idea is simple, the agent may wonder about \(\pv{\phi}{v}\) from \(\Psi\).
  And, the agent may attempt to conclude.
  But, if doesn't conclude, then doesn't conclude \(\pv{\phi}{v}\) from \(\Phi\).
\end{note}

\section{\curb{3}}
\label{sec:curb}

\begin{note}
  \bCurb{1} capture the key phenomenon we are interested in.
  However, generalise slightly to include a further option.
\end{note}

\begin{note}[\curb{3}]
  \begin{definition}[A \curb{0}]
    \label{def:curb}
    For an agent \vAgent{}, and proposition-value-premises pairings \(\pvp{\phi}{v}{\Phi}\), \(\pvp{\psi}{v'}{\Psi}\):

    \begin{itemize}
    \item
      \(\pvp{\phi}{v'}{\Psi}\) is a \emph{\curb{}} of concluding \(\pv{\phi}{v}\) from \(\Phi\)
    \end{itemize}

    \emph{If and only if}

    \begin{itemize}
    \item
        \begin{enumerate}
        \item[\emph{If}:]
          Either~\ref{def:curb:opp} or~\ref{def:curb:link} is true:

          \begin{enumerate}[label=\alph*., ref=(\alph*)]
          \item
            \label{def:curb:opp}
            There is no \pevent{} in which \vAgent{} concludes \(\pv{\psi}{v'}\) from \(\Psi\).
          \item
            \label{def:curb:link}
            There is some \pevent{} in which \vAgent{} concludes \(\pv{\chi}{v''}\) from \(X\) such that \(\pv{\chi}{v''}\) is incompatible with concluding \(\pv{\psi}{v'}\) from \(\Psi\).
          \end{enumerate}
        \item[\emph{Then}:]
          \begin{enumerate}[label=\alph*., ref=(\alph*), resume]
          \item
            \label{def:curb:fail}
            \vAgent{} would not be concluding \(\pv{\phi}{v}\) from \(\Phi\).%
            \hfill(\emph{Due to}~\ref{def:curb:opp} or~\ref{def:curb:link})
          \end{enumerate}
      \end{enumerate}
    \end{itemize}
    \vspace{-\baselineskip}
  \end{definition}

  With respect to the breakdown of \TNSketch{3}~\ref{sketch:zS:fail} and~\ref{sketch:zS:succeed}, a \curb{} captures both the opportunity for the agent to reason about whether \(\pv{\psi}{v'}\) follows from \(\Psi\) and the idea of whether \(\pv{\psi}{v'}\) follows from \(\Psi\) being a \check{0} on concluding \(\pv{\psi}{v'}\) from \(\Phi\).
\end{note}

\begin{note}
  Still an issue of how the subjunctive relates to concluding \(\pv{\phi}{v}\) from \(\Phi\).
  This, \autoref{cha:zS:sec:question}.
  \curb{} is half, or two thirds of the story.
\end{note}

\begin{note}
  The parenthetical `due to' appended to~\ref{def:curb:fail} ensures that the agent not concluding \(\pv{\phi}{v}\) from \(\Phi\) is tied to failing to conclude \(\pv{\psi}{v'}\) from \(\Psi\) after taking the relevant opportunity.%
  \footnote{
    General problem of deviance.
    Subjunctive conditional, without statement, allows for failure to conclude to be unrelated.
    Or, a finkish disposition (\cite[cf.][144]{Lewis:1997wg}).
    % ~\cite{Lewis:1997wg}
    % \begin{quote}
    %   Anything can cause anything; so stimulus \emph{s} itself might chance to be the very thing that would cause the disposition to give response \emph{r} to stimulus \emph{s} to go away.
    %   If it went away quickly enough, it would not be manifested.
    %   In this way it could be false that if \emph{x} were to undergo \emph{s}, \emph{x} would give response \emph{r}.
    %   And yet, so long as s does not come along, \emph{x} retains its disposition.
    %   Such a disposition, which would straight away vanish if put to the test, is called finkish.%
    %   \mbox{ }\hfill\mbox{(\citeyear[144]{Lewis:1997wg})}
    % \end{quote}
  }
  Don't have a specific account of `due to'.
  Move to the level of theories, and overall goal is to provide a theory independent motivation for rejecting \issueConstraint{}.
  So, some difficulty, may wonder whether the conditional holds.
\end{note}

\begin{note}
  \begin{proposition}
    \(\pvp{\phi}{v}{\Phi}\) is not necessarily a \curb{0} of concluding \(\pv{\phi}{v}\) from \(\Phi\).
    \begin{argument}
      This is trivial.

      For:

      \begin{itemize}
      \item
        \ref{def:curb:opp}, then there is no development of the current event in which the agent concludes \(\pv{\phi}{v}\) from \(\Phi\). Hence, the event is not an instance of concluding.
      \item
        \ref{def:curb:link} is where the difficulty lies.
        For, suppose so \pevent{}.
        At issue is whether the event in progress is such that the agent concludes something incompatible.
        And, in general this may not be the case.
        For example, various cases where overlook something obvious.
      \end{itemize}
    \end{argument}
  \end{proposition}
\end{note}




\section{\curb{3} and \fc{1}}

\begin{note}
  \begin{proposition}[\curb{3} and \fc{1}]
        For an agent \vAgent{}, and proposition-value-premises pairings \(\pvp{\phi}{v}{\Phi}\), \(\pvp{\psi}{v'}{\Psi}\):

    \begin{itemize}
    \item
      \(\pvp{\phi}{v'}{\Psi}\) is a \emph{\curb{}} of concluding \(\pv{\phi}{v}\) from \(\Phi\)
    \end{itemize}

    \emph{If and only if}

    \begin{itemize}
    \item
        \begin{enumerate}
        \item[\emph{If}:]
          \(\pv{\psi}{v'}\) is not a \fc{} from \(\Psi\).
        \item[\emph{Then}:]
          \begin{enumerate}[label=\alph*., ref=(\alph*), resume]
          \item
            \label{def:curb:fail}
            \vAgent{} would not be concluding \(\pv{\phi}{v}\) from \(\Phi\).
          \end{enumerate}
      \end{enumerate}
    \end{itemize}
    \begin{argument}
      The conditional by which \curb{1} are defined is of the form `\emph{If}~(\ref{def:curb:opp} \emph{or} \ref{def:curb:link}) \emph{then}~\ref{def:curb:fail}'.
      And, `\ref{def:curb:opp}~\emph{or}~\ref{def:curb:link}' is equivalent to \emph{not}-(\emph{not}-\ref{def:curb:opp}~and~\emph{not}-\ref{def:curb:link}).
      Hence, the conditional is equivalent to `\emph{If}~\emph{not}-(\emph{not}-\ref{def:curb:opp}~\emph{and}~\emph{not}-\ref{def:curb:link})~\emph{then}~\ref{def:curb:fail}'.

    And, `(\emph{not}-\ref{def:curb:opp}~and~\emph{not}-\ref{def:curb:link})' is equivalent to pair of conditions for \(\pvp{\psi}{v'}{\Psi}\) being a \fc{}.
    \end{argument}
  \end{proposition}
\end{note}

\begin{note}
  The idea of \(\pvp{\psi}{v'}{\Psi}\) being a \curb{2} of concluding \(\pv{\phi}{v}\) form \(\Phi\) for some agent is stated without reference to the \agpe{}.
  However, our interest with \curb{1} will be from an \agpe{}.

  Easily embed within.%
  \footnote{
    Recall, same with respect to \fc{1} --- see page~\pageref{fcs-neutral-perspective}.
  }
\end{note}

% \begin{note}
%   In some cases, it may be the case that \(\pv{\phi}{v}\) follows trivially after getting \(\pv{\psi}{v'}\), this doesn't matter.

%   It may also be the case that \(\pv{\phi}{v}\) entails \(\pvp{\psi}{v'}{\Psi}\), but this is also fine, so long as there is an alternative way.
% \end{note}

\begin{note}
  \color{red}
  The definition of a \curb{} expresses the idea that concluding \(\pv{\psi}{v'}\) from \(\Psi\) is a check on whether it makes sense for the agent, from the \agpe{}, to conclude \(\pv{\phi}{v}\) from \(\Phi\).
\end{note}

\section{Examples}
\label{cha:zS:sec:curbs:examples}

\begin{note}
  Seen two examples of a \curb{}.

  \autoref{illu:lost-key} and~\autoref{scen:squish}.
\end{note}

\begin{note}[Calculator]
  Likewise, failure of \(\pv{\psi}{v'}\) being a \curb{}.
  Opening \scen{},~\autoref{illu:gist:calc}.

  Emphasis on testimony.

  Example here, failure for the conditional to hold, though plausibly have the option.

  With respect to the \scen{0}, testimony.

  However, more broadly, agent values reasoning over another.

  Abstract from the testimony of a calculator to settings where the dynamic is more intuitive.

  For example, student in a classroom.

  Though, with the broader idea, ways to make calculator easier.

  Under the weather.
  Things are a little foggy.
  Some act, which may be such that concluding.
  However, no guarantee the act is an act of concluding.
\end{note}

\begin{note}[Failure but no option]
  \citeauthor{Dretske:1970to}.
  \begin{scenario}[A trip to the zoo]\mbox{ }
    \label{scen:trip-to-zoo}
    \vspace{-\baselineskip}
    \begin{quote}
      You take your son to the zoo, see several zebras, and, when questioned by your son, tell him they are zebras.
      Do you know they are zebras?
      [\dots]
      We know what zebras look like, and, besides, this is the city zoo and the animals are in a pen clearly marked ``Zebras.''
      Yet, something's being a zebra implies that it is not a mule and, in particular, not a mule cleverly disguised by the zoo authorities to look like a zebra.
      Do you know that these animals are not mules cleverly disguised by the zoo authorities to look like zebras?\newline
      \mbox{ }\hfill\mbox{(\citeyear[1015--1016]{Dretske:1970to})}
    \end{quote}
    \vspace{-\baselineskip}
  \end{scenario}

  \autoref{scen:trip-to-zoo} is framed in terms of knowledge, and is designed to raise a problem for conclude of knowledge under known entailment.
  Intuitively, you know the animals in the pen are zebras.
  And, you know the following conditional is true:
  The animals in the pen are zebras \emph{only if} the animals in the pen are not cleverly disguised mules.
  However, you (intuitively) don't know the animals in the pen are not cleverly disguised mules.

  If knowledge is closed under known entailment, then you:
  \begin{enumerate}
  \item \(\phi\) has value \(v\) only if \(\psi\) has value \(v'\)
  \end{enumerate}
  then, if
  \begin{enumerate}
  \item
    \(\phi\) has value \(v\), then
  \end{enumerate}
  \begin{enumerate}
  \item \(\psi\) has value \(v'\)
  \end{enumerate}

  Framed in terms of knowledge, but relation is similar to \curb{}.

  There is no premises to distinguish.
  \scen{3} designed to test closure principles provide various examples of this kind.
  In particular, \citeauthor{Wright:2011wn}.
\end{note}

\begin{note}
  Returning to instances of \curb{1}, ability.
  Most interesting case, from my perspective.

  Simple case is Sudoku puzzles, or puzzles in general.
\end{note}

\section{\curb{3} as checks on concluding}
\label{cha:zS:sec:curbs:checks}

\begin{note}
  Strictly:

  \curb{3} as \emph{partial} checks on \emph{whether or not an agent is} concluding.

  Core is from \ref{def:curb:opp} and~\ref{def:curb:link}.
  The idea is, that we're given an event, in which the agent is doing some reasoning.
  And, if either of the conditions obtain, then the event is interrupted.

  \begin{itemize}
  \item
    \ref{def:curb:opp}.

    Here, given concluding, then also conclude.
    Hence, if no conclusion, then fail.

  \item
    \ref{def:curb:link}.

    Likewise, but the difficulty is strengthened to something which conflicts.
  \end{itemize}
\end{note}

\subsection{Check}

\begin{note}[Available action]
  As seen with the \citeauthor{Dretske:1970to} case.
  No opportunity, and therefore not a check in the relevant sense.

  Really, opportunity is easy to overlook, but very important.
  Conditional is subjunctive.
  Opportunity ensures that concluding from present state.
  Stops the subjunctive from wandering too far.

  Similar sense of check as checking date of birth.
  Fail to be of age, then no purchasing \dots

  Difference sense of check to label on the box.
  In a sense, determines whether or not make the purchase.
  Though, what really matters is whether the shop assistant asks for date of birth.

  Failure.
  This is stronger than should not, or might not.
\end{note}

\paragraph{Premises}

\begin{note}
  Stated independently of \agpe{}, whether check in this sense is unclear.

  However, some caution.
  Harman style cases where there's some change.
  \curb{} is tied to particular pools of premises.
  At issue is not whether the agent would conclude \(\pv{\phi}{v}\) after failing to conclude, but conclude \(\pv{\phi}{v}\) \emph{from \(\Phi\)}.
  Concluding \(\pv{\phi}{v}\) from \(\Phi'\) may be okay, but \(\Phi\) really is bad.

  Still, from the \agpe{}, fine.
  I was confident I'd stop if failed to get validity of squish.

  Worry about what actually happens relies on things that perspective doesn't take into account.
  But, from perspective, fine.

  Note, with Harman style cases, also the possibility that whether something is a \curb{} may change.
\end{note}

\subsection{Concluding}

\begin{note}
  Check on concluding in the sense that event in progress, and whether that event is an event of concluding.

  {
    \color{red}
    Two senses in which this is check on concluding.
    First, concerns reasoning to \(\pv{\phi}{v}\) from \(\Phi\).
    Second, given reasoning to \(\pv{\phi}{v}\) from \(\Phi\).
  }
\end{note}

\begin{note}[Problems of induction]
  Note, however, both~\autoref{illu:lost-key} and~\autoref{scen:squish} focus on concluding.

  In turn, \TNSketch{3}~\ref{sketch:zS:fail} and~\ref{sketch:zS:succeed} focus on the failure to conclude to some proposition-value pair which would follow from concluding some other proposition-value pair.

  Hence, the sketch does not apply to black ravens.
  I wouldn't conclude all ravens are black if I saw a white raven.

  I may worry about shortly seeing a white raven when concluding all ravens are black, and I may refuse to entertain the possibility that the sun will rise tomorrow when planning to mow the grass.

  However, it's not possible to reason to seeing a white raven, nor is it possible to reason to the sun not rising tomorrow.

  Abstractly, at issue in~\autoref{illu:lost-key} is the possibility of failing to a reason to some proposition-value pair given \emph{present} information, rather than the possibility of failing to a reason to some proposition-value pair given \emph{new} information.

  To the extent that problems of induction arise from receiving new information, what is at issue is distinct.%
  \footnote{
    See~\textcite{Henderson:2020wb} for more on the problem of induction.
  }

  Similar points for external world scepticism.
  Would not conclude that I have hand if disembodied brain in a vat.

  However, conclusion is out of reach.
\end{note}

\subsection{Partial}

\begin{note}
  The `converse' to the conditional by which \curb{1} are defined does not hold.
  Naturally, as \fc{}\dots

  Still, the more significant issue is that there may be no \emph{unique} \curb{0} of concluding \(\pv{\phi}{v}\) from \(\Phi\).
  For example, both \(\pvp{\psi}{v'}{\Psi}\) and \(\pvp{\chi}{v''}{X}\) may be \curb{0} of concluding \(\pv{\phi}{v}\) from \(\Phi\).

  {
    \color{red}
    Seen in \scen{0}???
  }
  And, there may be checks other than \(\pvp{\psi}{v'}{\Psi}\) being a \curb{}.

  A different, but related check, would consider whether the agent has concluded.
  However, less interesting.
  Consider the squish scenario.
  Have concluded.
  However, what's of interest is how things are.
\end{note}

\subsubsection{When concluding}

\begin{note}[Prior to concluding\dots]
  Not particularly marked.
  Allow agent to have built up a bunch of stuff while reasoning.

  Example.

  \begin{scenario}[Velocity]
    \label{ill:velocity}
    Agent is provided with information about how far a car has travelled north as a function of time travelled.

    From this, take the derivative of the function to obtain the (instantaneous) velocity of the car at a handful of points in time.

    And, from the (instantaneous) velocity of the car, the agent calculates the (instantaneous) acceleration of the car at each of the points in time.

    The agent also has information about the speed of the car as a function of time travelled, and the agent may calculate speed by the taking magnitude of the (instantaneous) velocity of the car.
  \end{scenario}

  \autoref{ill:velocity}, two step calculation.
  Velocity, acceleration.
  After the first step, check by taking the magnitude.
  Calculation of velocity is correct only if taking the magnitude matches speed.

  Just before concluding to include cases such as this.
\end{note}

\begin{note}
  Example highlights how `intermediate conclusions' relate.
  Further point of interest:
  Failure to conclude.

  Two ways to view agent's calculation of the velocity of the car.

  First, as a conclusion.
  Same status as the function.

  Or, as temporary.

  Difference in how we understand agent's present epistemic state.

  On first, the agent's present epistemic state is inconsistent.
  Two proposition-value pairs which conflict.
  Not possible for the car to have velocity the agent calculated and acceleration the agent has been informed of.

  May also be that the function and information about acceleration are inconsistent, but may also be that the agent made a mistake in calculating the velocity of the car.

  On second, the agent's present epistemic state may be consistent.

  For, made a mistake.
  But, proposition-value pair is not part of present epistemic state, so distinguished from function and information about acceleration, which are consistent.

  This is a distinction we have little interest in.
  What matters is failure to conclude speed.
  Result is either revising inconsistent epistemic state, or abandoning intermediary steps of reasoning.
\end{note}

\section{Summary}
\label{cha:zS:sec:curbs:summary}

\begin{note}
  Two conditions from breakdown of \TNSketch{3}~\ref{sketch:zS:fail} and~\ref{sketch:zS:succeed}.
  Opportunity and \check{}.

  \curb{}.
  Minor clarifications.
  From the \agpe{}.

  Examples.

  Idea that a \curb{} identifies a (partial) check on reasoning in detail.

  Undercutting defeaters.

  Minor details.
\end{note}

\section{\curb{3} and concluding}
\label{cha:zS:sec:question}

\begin{note}[The question]
  With the idea of a \curb{} in hand \dots

  \begin{restatable}[\curb{3} and concluding]{proposition}{propCurbCing}
    \label{prop:sCing}
    For an agent \vAgent{}, etc.\dots

    \begin{itemize}
    \item
      Event is instance of concluding \(\pv{\phi}{v}\) from \(\Phi\).
    \end{itemize}

    \emph{If and only if}

    \begin{itemize}
    \item
      For any proposition-value-premises pairing \(\pvp{\psi}{v'}{\Psi}\):
      \begin{itemize}
      \item[\emph{If}:]
        \begin{enumerate}[label=\alph*., ref=(\alph*)]
        \item
          \label{question:zs:option}
          \(\pvp{\psi}{v'}{\Psi}\) is a \curb{} of concluding \(\pv{\phi}{v}\) from \(\Phi\).
        \end{enumerate}
      \item[\emph{Then}:]
        \begin{enumerate}[label=\alph*., ref=(\alph*), resume]
        \item
          \label{question:zs:may-fail}
          \(\pv{\psi}{v'}\) is a \fc{} from \(\Psi\).
        \end{enumerate}
      \end{itemize}
    \end{itemize}
    \begin{argument}
      Trivial.
      For, \curb{}.
      If either condition obtains, then not concluding.
      Hence, neither condition obtains.
      Therefore, by the other proposition which links to \fc{}, this means \fc{}.
    \end{argument}
  \end{restatable}
\end{note}

\begin{figure}[h]
  \centering
  \begin{tikzpicture}
    \node (origin) at (0,0) {};
    \node (psiSplit) at (1,0) {};
    \node (phiSplit) at (4,0) {};
    %
    \node[anchor=west] (Phi) at  (0,0)  {\(\Phi\)};

    \node[anchor=west] (psiV) at  (6,-1)  {\(\pvp{\psi}{v'}{\Psi}\)};
    \node[anchor=west] (psiNv) at (6,-2) {\(\pvp{\psi}{\{\overline{v'}, ?\}}{\Psi}\)};
    % \node[anchor=west] (psiQ) at (6,-3) {\(\pvp{\psi}{?}{\Psi}\)};
    %
    % \node[anchor=west] (psiVPhiV) at (9,-1) {\(\pv{\phi}{v}\)};
    \node[anchor=west] (psiNvPhiU) at (10,-2) {\(\pv{\phi}{\{\overline{v},?\}}\)};
    % \node[anchor=west] (psiQPhiU) at (9,-3) {\(\pv{\phi}{\{\overline{v},?\}}\)};
    %
    \node[anchor=west] (phiQ) at (10,1) {\(\pv{\phi}{\{\overline{v}, ?\}}\)};
    % \node[anchor=west] (phiNv) at (10,2) {\(\pv{\phi}{\overline{v}}\)};;
    \node[anchor=west] (phiV) at (10,0) {\(\pv{\phi}{v}\)};
    %
    \draw[-]  (Phi) -- (phiV);
    %
    % \path[-,dotted] (phiSplit) edge [out=0, in=180] (phiNv);
    \path[-,dotted] (phiSplit) edge [out=0, in=180] (phiQ);
    %
    \path[-, dashed] (psiSplit) edge [out=0, in=180] (psiV);
    \path[-, dotted] (psiSplit) edge [out=0, in=180] (psiNv);
    % \path[-, dotted] (psiSplit) edge [out=0, in=180] (psiQ);
    %
    \draw[-,dashed] (psiV) edge [out=0, in=180] (phiV);
    \draw[-, dotted] (psiNv) edge (psiNvPhiU);
    % \draw[-, dotted] (psiQ) edge (psiQPhiU);
    \end{tikzpicture}
    \caption{Visualisation of a conclusion given a \curb{}.}
    \label{fig:csN:illu:overview}
  \end{figure}

\begin{note}[Figure]
  The flat line captures the agent's reasoning, which concludes with \(\pv{\phi}{v}\).
  In concluding \(\pv{\phi}{v}\) the agent rules out two possibilities with respect to \(\phi\).
  First, that \(\phi\) does not have value \(v\), indicated by \(\pv{\phi}{\overline{v}}\).
  Second, that the agent does not assign any value to \(v\), indicated by \(\pv{\phi}{?}\).
  Prior to concluding \(\pv{\phi}{v}\), the agent's reasoning may have branched to either alternative path, but as the agent has concluded \(\pv{\phi}{v}\), neither path is viable, and hence both paths are represented with a dashed line.

  So far, we have seen only that the agent has concluded \(\pv{\phi}{v}\).

  We now consider some proposition-value-premises pairing \(\pv{\psi}{v'}{\Psi}\) such that if the agent were to fail to conclude \(\pv{\psi}{v'}\) from \(\Phi\), the agent would not conclude \(\pv{\phi}{v}\) from \(\Phi\).

  Intuitively, the dotted arrows from the various combinations of \(\psi\) and \(\{v',\overline{v'},?\}\) read, from top to bottom:
  \begin{itemize}
  \item If \(\phi\) has value \(v\) then the agent may conclude \(\pv{\psi}{v'}\) from \(\Psi\), and:
  \item If the agent concludes \(\psi\) has some value \(\overline{v'}\) from \(\Psi\), then the agent either concludes \(\phi\) has some value other than \(v\), or the agent fails to reach a conclusion regarding \(\phi\) from \(\Phi\).
    Both options are combined via the shorthand \(\pv{\phi}{\{\overline{v},?\}}\).
  \item
    And, likewise if the agent fails to conclude \(\pv{\psi}{v'}\) from \(\Psi\).
  \end{itemize}

  With respect to concluding, observe that prior to ruling out alternative branches with respect to \(\pv{\phi}{\{\overline{v},?\}}\), the agent may have reasoned about whether \(\psi\) has value \(v\).
  And, from the \agpe{}, \(\phi\) has value \(v\) only if \(\psi\) has value \(v'\).
  If \(\psi\) does not have value \(v'\), then either \(\phi\) does not have value \(v\), or the agent's reasoning would not conclude with a value for \(\phi\), indicated by \(\pv{\phi}{\{\overline{v},?\}}\).

  Hence, prior to concluding \(\pv{\phi}{v}\), the agent has concluded \(\pv{\psi}{v'}\).
\end{note}

\subsection{Odd sands}

\begin{note}
  \autoref{prop:sCing} may seem odd.
  For, suppose an agent concludes \(\pv{\phi}{v}\) from \(\Phi\).
  Then, it seems that throughout the event, then agent was concluding.
  However, if there was some \curb{}, then would not have been concluding.
  Hence, it seems there must be no cases in which an agent concludes though the agent may have concluded otherwise.

  However, this seems unintuitive.
  For, there are various cases in which it seems an agent may have concluded otherwise.

  There are two ways to approach this:
  \begin{itemize}
  \item
    Deny that the event was an event in which the agent concluded.

    Specifically, narrow the relevant event.
    For, we have some initial event, which develops in two a larger event, which finishes with an event in which the agent concludes \(\pv{\phi}{v}\).
    So long as final sub-event, then the agent concludes.
    The issue only arises from the apparent link to earlier (sub-)events.
    But, if otherwise, then motivation to reject these as (sub-)events of an event in which the agent concludes.
  \item
    Deny the premise.

    If event in which an agent concludes, it need not be the case that the agent was concluding throughout all the relevant sub-events.

    Then, no issue is resolved.
  \item
    Shift perspective.

    What it true after the fact need not be true as things are developing.
    This is \citeauthor{Boylan:2020aa}'s approach.
  \end{itemize}

  I am inclined to consider the former.
  When speak on concluded, just picking out an event.
  Naturally event to some default event.
  However, flexible.

  So, here \citeauthor{Boylan:2020aa} and darts.
  I was able to, sure, but only after things hand developed so far.

  Additionally, consider picture.
  Drew a dog.
  However, not the case that drawing a dog throughout associated event.
  Started out as a doodle.

  Ran 10k.
  However, started out, plan was to run 5k.
  Interesting consequence here is that not running 5k.
  For, supposing extension came about by choice, something changed, and hence no force to finish at 5k.

  Running 10k includes running 5k.
  But, in a more basic case, running.
  Extends to drawing.

  In the case of concluding, have reasoning.
\end{note}

\begin{note}[\autoref{sketch:zS:succeed} is \(\exists\), but \curb{} is \(\forall\)]
  Result of this is distinct from~\autoref{sketch:zS:succeed}.

  \begin{quote}
    There is a \curb{} and would conclude.
  \end{quote}

  Weaker, in sense holds if no \curb{}.
  Stronger, in sense that quantifies over all \curb{}.
\end{note}


\begin{note}
  Limits.

  Is going to be the case that there's a limit on what may be a \curb{}.
  For, instance of concluding.
  Hence, must be such that, if \(\pvp{\psi}{v'}{\Psi}\) is a \curb{} then would conclude \(\pv{\psi}{v'}\) from \(\Psi\) and conclude \(\pv{\phi}{v}\) from \(\Phi\).

  Hence, it is not the case that \curb{} iff incompatible with conclusion of \(\pv{\phi}{v}\) from \(\Phi\).

  No, this is another instance where the due to is important.
  For, in this case, it's not due to the failure to conclude, it's due to the resources required.
\end{note}

\section{A missing link}
\label{cha:zS:sec:missing-link}

\begin{note}
  \autoref{prop:sCing} is important.
  For, what matters for the conclusion.
  If agent concludes \(\pv{\phi}{v}\) from \(\Phi\), then concluding \(\pv{\phi}{v}\) from \(\Phi\).

  So, suppose, \curb{}.
  Then, need it to be the case that \(\pv{\psi}{v'}\) from \(\Psi\) is a \fc{}.

  In this respect, \fc{} explains, in part, and in some sense, why the agent concludes \(\pv{\phi}{v}\) from \(\Phi\).

  However, that \(\pv{\psi}{v'}\) from \(\Psi\) is a \fc{} does not answer, in part, \qWhyV{}.
  For, no \ros{}.
  \ros{} from \agpe{}.
  Only get this from \(\pv{\psi}{v'}\) from \(\Psi\) being a \fc{}, from \agpe{}.

  In this respect, \curb{1} are compatible with \issueConstraint{}!
\end{note}


%%% Local Variables:
%%% mode: latex
%%% TeX-master: "master"
%%% End:
