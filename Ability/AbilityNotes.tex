\chapter{Notes}
\label{cha:notes}

\subsection{Motivation for \ESU{}}
\label{sec:motiv-main-prem}

Some motivation for the \ESU{}:

\begin{enumerate}
\item Davidson
\item Responding to reasons
\item Hieronymi
\item Intuitive
\end{enumerate}

Broadly, the \ESU{} is interesting because it constrains how an agent obtains some conclusion by reasoning.
Instances that conform seem good, and instances that do not conform seem bad.

[Examples]

May also see how this is applied when finding solutions to difficult cases.
For example, the \ESU{} is in the background with Bratman on temptation, where the central idea is that desires fail to count as reasons because the agent would then need to act in a certain way.


\subsection{Reasoning \nr{} and reasoning \ur{}}
\label{sec:reas-dd-reas}

\begin{note}
  We now turn to the distinction between reasoning \emph{using reference} and reasoning \emph{not using reference}.
  Or, from now on, reasoning \ur{} and reasoning \nr{}.

  The use of (probably mistranslated) Latin is in part due to some similarity between the distinction under discussion and variety of distinctions made by using the terms `\dd{}' and `\dr{}'.
  In larger part, though, Latin is used because I had a hard time finding a combination of English terms which are not suggestive of some distinction.
  For example, you may already be considering whether `\emph{using reference}' and `\emph{not using reference}' correspond to `semantic' reasoning by way of some interpretation function and `syntactic' reasoning by rule governed syntactic manipulation.
  As with the \dd{}/\dr{} distinction, we will contrast \ur{} and \nr{} with `semantic' and `syntactic' reasoning below.
  For the moment I hope the use of Latin will allow for some flexibility.
\end{note}

\begin{note}[Plan]
  The plan for this section is as follows:
  \begin{itemize}
  \item First, situate the distinction.
  \item Background for distinction.
  \item Initial illustrations.
  \item Definitions.
  \item Illustrations for key type of reasoning.
    \begin{itemize}
    \item With logic
    \item Difference from syntax and semantics
    \item Existentials.
    \end{itemize}
  \item Contrast with \dd{}/\dr{}.
  \item Apply to ability/distinction matrix.
  \end{itemize}
\end{note}

\paragraph*{Situating the distinction}

\begin{note}
  To situate the distinction between reasoning \ur{} and \nr{} we will first outline how it applies to ideas that have already been introduced, and then note the purpose distinction in the argument ahead.
\end{note}

\subparagraph*{With respect to ideas already introduced}

\begin{note}
  In~\autoref{sec:abil-access-supp} we restricted our use of `reasoning' to involve an agent claiming support for some proposition \(\phi\) having some value \(v\) (\autoref{prop:RisTV}).
  For example, an instance of reasoning may culminate with an agent claiming support that it is true that the front door was locked, or that it is desirable that they take a walk.

  The distinction between reasoning \ur{} and \nr{} is targeted at the way in which an agent claims support that some proposition \(\phi\) has some value \(v\).
\end{note}

\begin{note}
  Further, in~\autoref{sec:cases-interest} we introduced particular instances of an agent claiming support for some conclusion which involved two key steps.
  The reasoning, in outline:
  \begin{enumerate}[label=\arabic*., ref=(\arabic*)]
  \item\label{NUR:ro:i} I have some general ability \(\gamma\).
  \item\label{NUR:ro:ii} If I have general ability \(\gamma\) then I have some specific ability \(\varsigma\) to \emph{V} that \(\phi\).
  \item\label{NUR:ro:iii} I have the specific ability \(\varsigma\) to \emph{V} that \(\phi\). \hfill (From~\ref{NUR:ro:i} and~\ref{NUR:ro:ii})
  \item\label{NUR:ro:iv} It is only possible to \emph{V} that \(\phi\) if \(\phi\) is already the case.
  \item\label{NUR:ro:v} \(\phi\) is the case. \hfill (From~\ref{NUR:ro:iii} and~\ref{NUR:ro:iv})
  \end{enumerate}

  The two key steps are from~\ref{NUR:ro:i} and~\ref{NUR:ro:ii} to~\ref{NUR:ro:iii} and from~\ref{NUR:ro:iii} and~\ref{NUR:ro:iv} to~\ref{NUR:ro:v}.

  The first key step involves the conditional of~\ref{NUR:ro:ii}, termed `\gsi{-}', and clarified in~\autoref{sec:type-scenario}.

  The second key step involves the conditional of~\ref{NUR:ro:iv}, termed `\aben{an}', and clarified in~\autoref{sec:ability-entailment}.

  Both these steps involve conditionals, and hence using the conditional to claimed support for the consequent of the conditional given claimed support for the antecedent of the conditional.
  And, the distinction between reasoning \ur{} and \nr{} is of interest to use with respect to these two instances of claiming support in particular.

  Finally, in~\autoref{sec:wr-ar} we introduced two (schematic) interpretations of ability --- \AR{} and \WR{} --- and as both key steps appeal to ability, the distinction between reasoning \ur{} and \nr{} will further inform how these interpretations of ability function in an instance of reasoning.
\end{note}

\begin{note}
  To summarise, from what we have seen, then, the distinction between reasoning \ur{} and \nr{} is targeted at the way in which an agent claims support that some proposition \(\phi\) has some value \(v\).
  So, the distinction is often interest when applied to the instance of reasoning which is the focus of this paper, and, hence, different ways in which the instance of reasoning may be understood.
\end{note}

\subparagraph*{With respect to the argument ahead}

\begin{note}
  Now, with the application to previously discussed ideas in hand, the distinction has to key purposes looking ahead:

  In~\autoref{sec:first-conditional} the distinction will separate an interpretation of the instance of reasoning which is incompatible with \ESU{} (\ur{}) form an interpretation which is compatible (\nr{}).

  And, in~\autoref{sec:second-conditional} the distinction will separate an interpretation of the instance of reasoning which is incompatible with \nI{} (\nr{}) form an interpretation which is compatible (\ur{}).
\end{note}

\paragraph*{Basic distinction}

\begin{note}[Reasoning]
  As noted, reasoning involves claiming support that a proposition has some value.

  Some propositions and parts of proposition, when appealed to or used in reasoning, \emph{refer}.

  In the two quick examples given above, propositions are that the front door was locked and that the agent takes a walk.
  When part of some instance of reasoning, propositions are about what the status of some particular door was, and a type of action that some particular agent may take.
  In these examples, the agent performing the reasoning for which the propositions are a part is aware of what they refer to.
  `The front door' is the front door of the agent's house, and `they' is either the agent or some close acquaintance.

  This is not always the case.

\end{note}

\begin{note}[Quick clarification]
  The stronger claim that propositions, when part of reasoning, refer is not without issue.
  Let me illustrate with a quick argument.

  \begin{itemize}
  \item All propositions refer.
  \item A proposition is something that may be assigned value.
  \item One such value is truth.
  \item Some conditionals are true.
  \item Some conditionals refer.
  \end{itemize}

  Three intuitive statements about propositions which combined with strong assumption lead to a problem.

  Settling whether or not conditionals (for example) refer --- or what conditionals refer to if the do refer --- is of no real interest for present purposes.
  Argument should go through either way.
  Trouble is I would to be committed to something quite strong.

  Would be nice to narrow down the appropriate sense of proposition.
  However, task with little reward.

  Nothing depends on borderline cases.

  Conditionals in which antecedent and consequent refer are viewed in terms of information about reference.

  In sort, what we're interested in those propositions which do refer.

  I had to unlock the door, so the door was locked.
  I am feeling stressed, so it is desirable that I take a walk.
\end{note}

\paragraph*{Illustrations for intuition}

\begin{note}[Illustration]
  Before defining \ur{} and \nr{}, let us work up some intuition by re-examining an instance of claiming support.

  \begin{quote}
    From claimed support that a rectangle measures \(19\text{cm}\) by \(7\text{cm}\), an agent claims support that the area of the rectangle is \(133\text{cm}^{2}\).
  \end{quote}
  In \autoref{ill:rectangle:basic} measurement, understanding of how to calculate the area of a rectangle, and some grasp of mathematics.\nolinebreak
  \footnote{
    In \autoref{ill:rectangle:ability} applied to ability, but not interested in that here.
  }
  The following two illustrations detail two different ways in which the claim support.

  The purpose of the pairing is to help develop some intuition for the ways in which an agent may or may not appeal to or use the referent of a proposition when reasoning.
  Beyond this, the subject matter and steps of reasoning hold no (direct) interest.

  Both illustrations start with a the same premise --- the rectangle measures \(19\text{cm}\) by \(7\text{cm}\) --- and end with the same conclusion --- the area of the rectangle is \(133\text{cm}^{2}\).

  The key difference between the two illustrations is how the agent claims support for the conclusion given claimed support for the premises.

  In the first illustration the agent will refer to a particular rectangle throughout the intermediate reasoning.
  And, by contrast, in the second the agent will not refer to a particular rectangle throughout the intermediate reasoning.
\end{note}

\begin{note}[]
  \begin{illustration}\label{ill:rectangle:ur}
    \vspace{-\baselineskip}
    \begin{enumerate}[label=\(\protect\tBox\)\space\arabic*., ref=(\(\protect\tBox\)\space\arabic*), align=left, leftmargin=*]
    \item[\(\protect\tBox\)\space P.]\label{tB:measure} This rectangle measures \(19\text{cm}\) by \(7\text{cm}\).
    \item\label{tB:width} Width is \(19\text{cm}\), so divide into \(19\) columns containing some number of unit squares, where unit is a centimetre.
    \item\label{tB:height} Height is \(7\text{cm}\).
    \item\label{tB:counting} So, there are seven unit squares in each column.
    \item\label{tB:total} Therefore, total of \(133\) square centimetres.
    \item[\(\protect\tBox\)\space C.]\label{tB:conclusion} Hence, the area of this rectangle is \(133\text{cm}^{2}\).
    \end{enumerate}
    \vspace{-\baselineskip}
  \end{illustration}

  The reasoning of~\autoref{ill:rectangle:ur} somewhat stilted in style, but plausible.

  The agent understands how to calculate the area of a triangle and applies the calculation to the specific rectangle.
  Key is that at no point does the agent abstract from the particular rectangle to an arbitrary rectangle with the same dimensions --- steps~\ref{tB:width} to~\ref{tB:total} are about the specific rectangle.
  Of course, the same reasoning may be applied to any rectangle with the same dimensions, but steps~\ref{tB:measure} to \ref{tB:conclusion} refer to \emph{that} particular rectangle, and the agent claims support because of what they have established about that triangle.

  In this sense, the propositions concerning rectangles refer, and the agent appeals to or uses the referent of those propositions to claim support.

  The `\(\tBox\)' prefix for each step is designed to indicate that the agent is reasoning about a particular object throughout.

  \begin{illustration}\label{ill:rectangle:nr}
    \vspace{-\baselineskip}
    \begin{enumerate}[label=\(\protect\tBoxd\)\space\arabic*., ref=(\(\protect\tBoxd\)\space\arabic*), align=left, leftmargin=*]
    \item[\(\protect\tBoxd\)\space P.]\label{tBd:measure} This rectangle measures \(19\text{cm}\) by \(7\text{cm}\).
    \item\label{tBd:calculate} Calculate the area of any two-dimensional object, take length and width in a common unit and multiply together to get area in common unit squared.
    \item\label{tBd:abstract} So, if object with \(19\) and \(7\), then area is \(19 \times 7\) cm2.
    \item\label{tBd:instantiate} Put together.
    \item[\(\protect\tBoxd\)\space C.]\label{tBd:conclusion} Hence, the area of this rectangle is \(133\text{cm}^{2}\).
    \end{enumerate}
    \vspace{-\baselineskip}
  \end{illustration}

  As with~\autoref{ill:rectangle:ur}, the reasoning of~\autoref{ill:rectangle:nr} is also somewhat stilted in style, but plausible.

  The agent understands how to calculate the area of a rectangle and although they conclude by claiming support with respect the specific rectangle, the agent quickly abstracts to any rectangle with the same dimensions.

  Key is that the agent abstracts from the particular rectangle to an arbitrary rectangle with the same dimensions for the most part of their reasoning --- steps~\ref{tBd:calculate} to~\ref{tBd:abstract} are not about any specific rectangle.

  In this sense, the propositions may refer, as they apply to any rectangle with the respective dimensions, but the agent neither appeals to nor uses possible referents of those propositions to claim support at key steps of the reasoning.

  The `\(\tBoxd\)' prefix for each step is designed to indicate that the agent is only sometimes reasoning about a particular object.
\end{note}

\begin{note}[Propositions, recap]
  {\color{red} \autoref{prop:RisTV} has reasoning as establishing a value}
  Two ways of tracing value.
  With and without reasoning about what the constituents of the proposition refer to.\nolinebreak
  \footnote{
    Footnote on Russel, Frege, etc.
  }

  Here, truth conditions, but more general evaluation.
  I.e.\ truth isn't the only evaluation of interest.

  Here, truth conditions.
  However, value.
  So, truth conditions + value, reasoning about something.
  Distinction is with respect to how the agent goes about reasoning about the something.

  Have the idea of a proposition.
  Something which gets a value.
  Reason in terms of preservation of value.

  
\end{note}

\begin{note}[The distinction]
  {
    \color{red}
    This is leading to a distinct between `there is a' and `the'.
    Though, that's just a gloss.
  }
  Focus on deductive reasoning.
\end{note}

\paragraph*{Definitions}

\begin{note}
  Have basic distinction, and a pair of illustrations.

  Following, apply this distinction to additional illustrations for which reasoning involved which will resemble reasoning of interest with respect to  ability scenarios.

  First, though, summarise the ideas by stating definitions.
\end{note}

\begin{note}[A pair of definitions]
  \begin{definition}[\ur{}]
    \vspace{-\baselineskip}
    \begin{itemize}
    \item Let \(t\) be some thing, and
    \item Let \(S\) be some step of reasoning that involves appeal to claimed support for propositions \(\psi_{1},\dots,\psi_{n}\) to claim support for some proposition \(\phi\).
    \end{itemize}

    The step of reasoning \(S\) is \ur{} with respect to \(t\) if the agent \emph{appeals to} \(t\) when claiming support for \(\phi\) by \(S\).
  \end{definition}

  \begin{definition}[\nr{}]
    \vspace{-\baselineskip}
    \begin{itemize}
    \item Let \(t\) be some thing, and
    \item Let \(S\) be some step of reasoning that involves appeal to claimed support for propositions \(\psi_{1},\dots,\psi_{n}\) to claim support for some proposition \(\phi\).
    \end{itemize}

    The step of reasoning \(S\) is \nr{} with respect to \(t\) if the agent \emph{does not appeal to} \(t\) when claiming support for \(\phi\) by \(S\).
  \end{definition}
\end{note}

\begin{note}
  Respective definitions differ only with respect to whether appeals --- or does not appeal to --- the referent of the proposition, or part.

  So, initial discussion of definitions covers both.

  Start with conditions.

  \(t\) fix some thing.
  Any thing, really.
  Possible referent of a referential term.
  Reasoning is complex, plausible that any step may involve a complex of factors.
  Only interested in specific terms, so definition avoids difficulties with classifying steps of reasoning as a whole.

  Don't require that \(t\) occurs, to keep things simple.
  Implicit that \nr{} with respect to \(t\) if \(t\) does not occur.

  However, possible for \(t\) to be in either premise(s) or conclusion of step.

  \(S\)
  When claiming support for some proposition from some other proposition.
  So, definitions cover a single step of reasoning --- not interested in classifying anything broader.

  Possible to claim support for a proposition from something non-propositional.
  However, only interested in proposition-to-proposition case, so this restriction is fine.

  Existential.
  Key idea is that \ur{} applies whenever appeals to referent.
  The agent claim support in this way.

  Finally, broad point.
  Claiming support.
  Doesn't say that this is successful, that there is not some other way, in particular going by \nr{} or \ur{} instead.
\end{note}

\begin{note}[Simpliciter]
  Definitions are with respect to some thing.
  That's what we're interested in.
  Further, they don't characterise steps of reasoning in general.
  Hence, no immediate (at least) consequences for what is involved in a step of reasoning.
  For example, possible \ur{} with respect to some thing and \nr{} with respect to some other thing.
\end{note}

\begin{note}[The distinction and entailment]
  The purpose of these definitions is to capture when the agent claim support by appeal to some thing.

  Applied to rectangles.
  \autoref{ill:rectangle:ur} as the agent references the particular triangle throughout the reasoning.
  So, \ur{} holds for each step with respect to the rectangle.
  The way in which the agent claims support is such that the rectangle does work.


  \nr{} holds for steps~\ref{tBd:calculate}~to~\ref{tBd:instantiate} of~\autoref{ill:rectangle:nr}.


  \emph{Why} something follows from something else.

  With \ur{} get an argument where the agent ensure that the reference for the consequence works out.
  This is the one to start with.
  The argument is that things \emph{are} a certain way, so to speak.
  That is, reasoning works out because reference works out.


  \nr{} doesn't do this.
  No reference.
  So, transformation to the information that the agent already has.
\end{note}

\begin{note}[To keep in mind]
  The distinction between \ur{} and \nr{} is not about the presence (or absence) of referential terms.
  Nor that the agent may claim support by noting that a term refers.

  In rectangle, important that refers to any rectangle.

  Main interest with this distinction will be whether agent appeals to referent of a referential term, so to speak.
\end{note}

\paragraph*{Illustrations for interest}

\begin{note}
  We now turn to additional illustrations.

  Two goals.

  First, relate to familiar stuff.
  Difference from distinction from syntactic and semantic reasoning.

  Second, to consider reasoning with existentials.
  This, then expanded on when turn to application of the distinction.

  Start with simple case.
  Then, turn to existentials.

  Get
  More familiarity with distinction.
  Additional considerations.

  Focused on deductive when giving examples.
  However, same applies to other kinds of reasoning.
\end{note}

\begin{note}[Instance of reasoning]
  The scenarios of interest:
  \begin{quote}
    From claimed support that a dog is \RIPa{} and \RIPb{}, an agent claims support that the dog is \RIPb{}.
  \end{quote}

  In outline, the reasoning is straightforward:
  \begin{quote}
    \begin{itemize}
    \item[P.] \nagent{10} is \RIPa{} and \RIPb{} dog.
    \item[---.] So, \nagent{10} is \RIPa{} dog and a \RIPb{} dog.
    \item[C.] Hence, \nagent{10} is a \RIPb{} dog.
    \end{itemize}
  \end{quote}

  Though straightforward, the reasoning is not completely trivial.
  `\RIPa{} and \RIPb{}' is the combination of two adjectives --- `\RIPa{}' and `\RIPb{}', respectively.
  And, it is not always the case that the attribution of a combination of adjectives to an object allows the attribution of the separate adjective.

  For example, it does not follow from the picture being black and white that the picture is white.\nolinebreak
  \footnote{
    You may prefer `black-and-white'.
    If so, I suggest `\RIPa{}-and-\RIPb{}'.
  }
  Nor does it follow from the weather being cloudy and sunny that the weather is sunny.

  And, of course, this isn't unique to `and' and adjectives.
  It does not follow from the transit time being an hour and five minutes that the transit time is five minutes.
  Nor does it follow from book being written by A and B that the book was written by B.

  The point, though a minor one, is that `and' behaves in a variety of ways and so some reasoning is required to move from the premise to the conclusion.
  \nolinebreak
  \footnote{
    Compare to obedient and \RIPb{}.
    May think obedience restricts \RIPb{}.
    Obedient and \RIPb{}, but not \RIPb{}.
    Not to say that the meaning of `and' works the same in both constructions.
    However, enough to require that syntax should be respected.
  }
\end{note}

\begin{note}
  Outline of reasoning.

  Focus for the moment is on relation to syntactic and semantic.

  
\end{note}


\begin{note}[Example, \nr{}]
  Recall, \nr{}.

  Idea is to illustrate this kind of reasoning in terms of transformations.
  Following steps in the first-order language.

  \begin{illustration}\label{ill:dog:C:nr}
    \vspace{-\baselineskip}
    \begin{enumerate}[label=\(\protect\iDogd\)\space\arabic*., ref=(\(\protect\iDogd\)\space\arabic*), align=left, leftmargin=*]
      % \item[\(\protect\iDogd\)\space P.] The dog is \RIPa{} and \RIPb{}.
    \item\label{ill:iDogd:abd} \(\text{\RIPa{-} \& \RIPb{} dog}(w)\)
    \item\label{ill:iDogd:sep-gen} \(\forall x(\text{\RIPa{-} \& \RIPb{} dog}(x) \rightarrow (\text{\RIPa{-} dog}(x) \land \text{\RIPb{-} dog})(x))\)
    \item\label{ill:iDogd:sep-app} \(\text{\RIPa{-} \& \RIPb{} dog}(w) \rightarrow ((\text{\RIPa{-} dog}(x) \land \text{\RIPb{-} dog})(w))\)
    \item\label{ill:iDogd:sep-con} \(\text{\RIPa{-} dog}(w) \land \text{\RIPb{-} dog}(w)\)
    \item\label{ill:iDogd:done} \(\text{\RIPb{-} dog}(w)\)
      % \item[\(\protect\iDogd\)\space C.] Hence, the dog is \RIPb{}.
    \end{enumerate}
    \vspace{-\baselineskip}
  \end{illustration}
\end{note}

\begin{note}[Main point]
  Moving between the steps by rules applied that do not depend on interpreting predicates and constants.
\end{note}

\begin{note}[Background]
  Key point here is that we have some background.

  Understand the syntax, and understand the intended interpretation of the syntax.
  `\RIPa{-} and \RIPb{} is a predicate, and applied to some constant `\(w\)' (abbreviating `\nagent{10}').
  For the moment, all that matters is that these are predicates and constants.

  Form of the argument.
  Two assumptions.
  Three premises obtained by applying rules following main connective of previous premise.
  In order, universal quantifier, conditional, conjunction.
\end{note}

\begin{note}[Walk-through]
  Walk through these rules.
  Observation is that each instance conforms to reasoning \nr{}.

  First step, from \ref{ill:iDogd:sep-gen} to \ref{ill:iDogd:sep-app} instance of universal instantiation.
  Applied to \(w\) because \(w\) is a constant.
  In order to apply this rule, it doesn't matter what \(w\) refers to, nor what the predicates involved refer to.
  Universal instantiation allows any constant.

  From \ref{ill:iDogd:sep-app} to \ref{ill:iDogd:sep-con} instance of conditional elimination.

  And, from \ref{ill:iDogd:sep-con} to \ref{ill:iDogd:done} conjunction elimination.

  If I may, I encourage you to check that the argument is valid.
  The task is instructive because in doing so you too will abstract away from the intended interpretation of the terms.
\end{note}

\begin{note}[Two things]
  Two questions here.

  Question about how to obtain \ref{ill:iDogd:abd} from initial premise, and how to obtain conclusion from \ref{ill:iDogd:done}, but there doesn't seem to be mystery about what is going on.
  Same as with illustration above.
  Move from measurements regarding a particular rectangle to the manipulation of symbols which may be given an interpretation.
  Here, instead of mathematics, we have first order logic.

  \ref{ill:iDogd:sep-gen}.
  Granted two assumptions, valid.

  Key thing is choice of predicates.
  Designed to ensure that the argument is sound when applied given intended interpretation.

  It's also not clear that any dog which is \RIPa{} and \RIPb{} is so independently of it being a dog.
  A \RIPb{} dog need not be \RIPb{} in the same way a small elephant need not be small.

  These aren't distractions for the sake a pedantry.

  Reasoning proceeds by applying rules, regardless of reference.
  But, want soundness so that they may be applied without reference.
  \nr{} is about the reasoning that takes place, but it doesn't hold that reference is irrelevant to reasoning.


  Of course, it's fair to say that the equivalence only holds because of what `\RIPa{}' and `\RIPb{}' refer to.
  However, key is that once the rule has been obtained an agent may reason about the terms, rather than reasoning about \RIPa{}ness and \RIPb{}ness.


  Well, 
\end{note}

\begin{note}[Summarise]
  Premise and conclusion about a particular dog, \nagent{10}.
  However, intermediary steps are instances of reasoning \nr{} as the agent does not go by referent.
  All the agent is concerned with is the logical form, so to speak.

  However, stress that although the steps of reasoning are independent of reference, it may still matter that parts refer.
  \nagent{10} may not feature, but that propositions are about \nagent{10} when given intended interpretation may matter.
\end{note}

\begin{note}
  Contrast to reasoning \ur{}.

  Following illustration traces follows the same general pattern as~\autoref{ill:dog:C:nr}.
  Difference, roughly stated, is that the agent will reason about \nagent{10}, \RIPa{} and \RIPb{} dogs, and so on.
\end{note}

\begin{note}[Example, \ur{}]

  \begin{illustration}
    \vspace{-\baselineskip}
    \begin{enumerate}[label=\(\protect\iDog\)\space\arabic*., ref=\arabic*, align=left, leftmargin=*]
    \item\label{ill:iDog:mixed} \nagent{10} is a big and playful dog.
    \item\label{ill:iDog:sep-gen} The thing of being a \RIPa{} and \RIPb{} dog is just the combination of being a \RIPa{} dog and being a \RIPb{} dog.
    \item\label{ill:iDog:sep-app} \nagent{10} being a \RIPa{} and \RIPb{} dog is the case when \nagent{10} is both a \RIPa{} dog and a \RIPb{} dog.
    \item\label{ill:iDog:sep-con} \nagent{10} is a \RIPa{} dog and \nagent{10} a \RIPb{} dog.
    \item\label{ill:iDog:sep-res} \nagent{10} is a \RIPb{} dog.
    \end{enumerate}
    \vspace{-\baselineskip}
  \end{illustration}
\end{note}

\begin{note}
  Natural language to ease discussion.
  Could go with semantic interpretation.\nolinebreak
  \footnote{
    Follow convention and use \(\sem{--}\) to capture the reference of some term `\(\text{--}\)'.
    \begin{illustration}
      \vspace{-\baselineskip}
      \begin{enumerate}[label=\(\protect\iDog\)\space\arabic*., ref=\arabic*, align=left, leftmargin=*]
      \item \(\sem{w} \in \sem{\RIPa{-} \& \RIPb{} dog}\)
      \item \(\sem{\RIPa{-} and \RIPb{} dog} \subseteq (\sem{\RIPa{-} dog} \cap \sem{\RIPb{-} dog})\)
      \item \emph{If} \(\sem{w} \in \sem{\RIPa{-} \& \RIPb{} dog}\), \emph{then} \(\sem{w} \in \sem{\RIPa{-} dog}\) \emph{and} \(\sem{w} \in \sem{\RIPb{-} dog}\)
      \item \(\sem{w} \in \sem{\RIPa{-} dog}\) and \(\sem{w} \in \sem{\RIPb{-} dog}\)
      \item \(\sem{w} \in \sem{\RIPb{-} dog}\)
      \end{enumerate}
      \vspace{-\baselineskip}
    \end{illustration}
    Imports set theoretical background from first order logic.
    Constants are individuals, predicates are treated extensionally.
    Read simply, \(\sem{The dog} \in \sem{\RIPa{-} and \RIPb{}}\) may suggest that the agent is reasoning that the reference of the term `the dog' is a member of the reference of `\RIPa{} and \RIPb{}'.

    \emph{\nagent{10}} is such that they are a \emph{\RIPa{} and \RIPb{}} dog.
  }
\end{note}

\begin{note}
  \ref{ill:iDog:mixed} meet \nagent{10} for the first time.
  \nagent{10} is playing with a ball.
  Person who takes care of \nagent{10} remarks that they're a \RIPa{} and \RIPb{} dog.
  You wonder about adjectives, what's being communicated here.
  Think about these two adjectives, \ref{ill:iDog:sep-gen}.
  Of course, apply to \nagent{10}, \ref{ill:iDog:sep-app}.
  And, observed, so \ref{ill:iDog:sep-con}.
  Hence, \ref{ill:iDog:sep-res}.
\end{note}

\begin{note}
  \ur{} with respect to \nagent{} and properties, roughly.
  So, four things.
  \nagent{10}, \RIPa{} \& \RIPb{}, \RIPa{}, and \RIPb{}.

  Throughout, properties.

  Variations.
  Focus on the adjectives and how these relate to properties, or abstract to concepts.

  Key point is that \ur{}.
  Agent is reasoning about some things, and claiming support by appeal to those things.
\end{note}

\begin{note}
  As with first pair of illustrations, claiming support by reference to something.
  This is all the distinction amounts to.
  Broader, applied to different kinds of reasoning, and differences that follow from these two kinds of reasoning.
\end{note}

\begin{note}[Difference]
  First illustration, logical structure of propositions.
  Don't need to appeal to reference.
  Logical structure for convenience.
  Claim support by transformations.

  Second illustration, surface presentation remains similar, but properties.

  \mom{}.
  First, \nagent{10} and transformations.
  World appears this way, and reasoning is independent.

  Second, \nagent{10} and understanding of properties involved.
  No reason to think that properties have been thought about badly, and testimony or observation seems good enough.
  World appears this way.
\end{note}

\begin{note}
  Overall similarity.
  Designed so that the explanation of the steps applies to both illustrations.
  Steps are missing from outline of reasoning present.
  Nothing about presentation.
  Indeed, second as semantic counterpart to syntactic reasoning of first.

  Alternatively, formal and non-formal.

  Something about this is right, but it's not quite right.
  Hopefully overlap will lend some clarity.
\end{note}

\paragraph*{Two related ideas}

\begin{note}
  Two related distinctions.
  Overlap and differences.
  Not an exhaustive discussion, just enough to identify similarities and differences.

  Similarity, then difference.
\end{note}

\begin{note}
  Should already be cautious.
  \ur{} and \nr{} is about whether some thing is appealed to when taking a step.
  It not obvious that anything else follows about the step, which is what the distinctions apply to.

  It's not obvious, but not immediate either.
  Still, difficult to work with.
  So, let's simplify for the moment.

  \begin{definition}
    Let \(S\) be some step of reasoning that involves appeal to claimed support for propositions \(\psi_{1},\dots,\psi_{n}\) to claim support for some proposition \(\phi\).

    The step of reasoning \(S\) is:
    \begin{itemize}
    \item \ur{} \emph{simpliciter}, if there is something thing for which \(S\) is \ur{} with respect to.
    \item \nr{} \emph{simpliciter}, otherwise.
    \end{itemize}
    \vspace{-\baselineskip}
  \end{definition}
  Now, apply to steps of reasoning as a whole.
\end{note}

\begin{note}[Issues]
  \nr{} for both sides of consequence.
  So, \nr{} does not imply syntactic reasoning.
  Still, seems formal then \nr{}.

  However, \nr{} does not imply formal reasoning.
\end{note}

\subparagraph*{Syntactic and semantic perspectives on logical consequence}

\begin{note}
  Similarity: well, syntactic transformation and something involving reference.

  Quick point is that role of reference in semantic perspective of logical consequence is different from \ur{}.

  Given some background, this is somewhat obvious.
  It's still logical consequence that we're talking about.
\end{note}

\begin{note}
  Using first order logic.
  Further, appeal to logical consequence.

  Presentations given are valid.

  Suggestion that the distinction I've been relying on to illustrate \ur{} and \nr{} just is the distinction at issue.

  On the one hand, syntactic reasoning and on the other semantic.

  Understood distinction between syntactic and semantic accounts of logical consequence.

  This is not to say that logical consequence is at issue.
  Problem with logical consequence is that it's always independent of content.
  This is ~\cite{Etchemendy:1990wo,Etchemendy:2008wz}.

  Instead, difference is the type of reasoning.
  E.g. the basics of the representational or interpretational approaches.

  Thing here is that when applied to logical consequence these approaches aren't really \ur{}.
  It's not about what the terms refer to, but possible referents of the terms.

  This is really important.
  \nr{} doesn't say that reference isn't relevant.
  And, get to the end of some semantic reasoning and it's possible to construct a counterexample.
  Reference is important.
  But, there's no need to appeal to that counterexample.

  Still, because semantics is about reference, get instances of \ur{} which follow semantic consequence.
  Every semantic consequence will lead to a valid instance of \ur{}, because here we're just fixing on one particular interpretation.

  Converse does not seem to hold.
  Consider illustration again.
  Middle steps are there to ensure that no matter the reference.
  But, as it's \nagent{10} it seems these aren't required.
  It's a logical consequence, but it's not at all clear that these steps are required.


  This is why syntax versus semantics as applied to logical consequence is tricky.
  Logical consequence is it's own thing.
  Two different interpretations of this.
  However, same consequences.
  And, the difference between illustrations is in part a difference in consequence.

  Really, we're looking at two 

  The point, really, of \ur{} is that the agent is not claiming support for the conclusion by appeal to logical consequence.
\end{note}

\begin{note}[Quick syntax vs semantics]
  Quick distinction is between syntactic and semantic reasoning.\nolinebreak
  \footnote{
    Avoid using term `logical form' as this doesn't distinguish between the two different things.
  }
  Familiar distinction.
  Consequence of reference.

  Two issues.
  First, background system.
  Second, and more important, suggests absence of reference.
  However, quite possible that the agent requires that some term refers.

  Note, also distinction doesn't rely on views of reference.
  Possible to have a view where reference is by the agent, or a view where reference is by language.
\end{note}

\begin{note}[Avoid syntax semantics terminology]
  Don't use syntax/semantics terminology because the focus of the contrast is reference, rather than two `systems' that falls from the contrast.
  Indeed, syntax/semantics requires a background system.
  Won't stray from first order logic, or at least reasoning that may be captured by first order logic.
  So, free to use this terminology is you prefer.
  However, caution, as will be seen in the following illustration.
\end{note}

\subparagraph*{Formal and non-formal reasoning}

 \cite{Beall:2019ty}

\begin{note}
  Above, different kinds of consequence.
  Issue was logical consequence.

  So, perhaps formal and non-formal.\nolinebreak
  \footnote{
    Distinct from informal logic~\textcite{Groarke:2021tk}.
  }

  Instances of \nr{} considered arithmetic and first order logic.
  Formal systems.
  Appeal to something beyond form(al system).

  Intuitive distinction, but beyond this I'm not sure how exactly to characterise the distinction.

  Well, seems that this distinction fits well.
  If logical is just no particular reference, then we've got a nice distinction.
\end{note}

\begin{note}[`Formal' and \nr{}, `non-formal' and \ur{}]
  Basic idea.
\end{note}

\begin{note}[Two cases]
  \begin{itemize}
  \item `Formal' and \ur{}: nonstandard models. Reasoning about formal properties of natural numbers. Just because these don't uniquely identify doesn't prevent this.
    However, still non-formal in the sense that the agent goes beyond formalism.
    So, this does end up being non-formal in a sense.
  \item `Non-formal' and \nr{}: time
  \end{itemize}
\end{note}

\begin{note}[`Formal' and \ur{}: nonstandard models]
  Example.

  For an illustration being put to use in a different context, consider intended and non-standard models.
  Specifically, with respect to arithmetic.
  Intended model example.

  The Peano Axioms are sound with respect to the natural numbers, addition, and multiplication.
  However, the Peano Axioms are also sound with respect to various non-standard structures.
  For example, by considering an object \(\omega\) which is larger than any natural number.

  So, there's a barrier to claiming that reasoning \nr{} is also reasoning \ur{}.
  If reasoning about the natural numbers, then reasoning \ur{}, because reasoning \nr{} does not do enough to fix on the natural numbers.

  More generally, incompleteness.
  (Nonstandard constructions based on incompleteness too.)

  From soundness, things still hold up.
  However, clear that there's a distinction between whether reference is being used.

  This is the thing that's important.

  Still, some caution.
  The way we've drawn the distinction requires some subtlety.
  For, it is a case of \nr{} for the agent to reason that `one' refers to one.
  Hence, nonstandard models don't follow from \nr{} reasoning alone.
  Still, follows that claiming support would not distinguish.
  And, that's where the key issue is.
\end{note}

\begin{note}[`Non-formal' and \nr{}: time]
  Well, rectangle.

  Though, this is a little tricky.

  So, reading the time from a clock.
\end{note}



\paragraph*{Existential illustrations}

\begin{note}[Example with an existential]
  The initial illustration was simple.
  Our interest is with something more complex.
  Existentials.

  \begin{quote}
    \begin{itemize}
    \item[P.] Some dog is \RIPa{} and \RIPb{}.
    \item[---] So, some dog is \RIPa{} and some dog is \RIPb{}.
    \item[C.] Hence, some dog is \RIPb{}.
    \end{itemize}
  \end{quote}
\end{note}

\begin{note}[Exists \nr{}]
  \nr{} is kind of straightforward.
  Not going to think about what this \(x\) refers to, just manipulate like any other individual.

    \begin{illustration}\label{ill:dog:E:nr}
    \vspace{-\baselineskip}
    \begin{enumerate}[label=\(\protect\iEDogd\)\space\arabic*., ref=(\(\protect\iEDogd\)\space\arabic*), align=left, leftmargin=*]
      % \item[\(\protect\iDogd\)\space P.] The dog is \RIPa{} and \RIPb{}.
    \item\label{ill:iDogd:E:abd} \(\exists x \text{\RIPa{-} \& \RIPb{} dog}(x)\)
    \item\label{ill:iDogd:E:sep-gen} \(\forall x(\text{\RIPa{-} \& \RIPb{} dog}(x) \rightarrow (\text{\RIPa{-} dog}(x) \land \text{\RIPb{-} dog}(x)))\)
    \item\label{ill:iDogd:E:abd} \(\text{\RIPa{-} \& \RIPb{} dog}(i)\)
    \item\label{ill:iDogd:E:sep-app} \(\text{\RIPa{-} \& \RIPb{} dog}(i) \rightarrow (\text{\RIPa{-} dog}(i) \land \text{\RIPb{-} dog}(i))\)
    \item\label{ill:iDogd:E:sep-con} \(\text{\RIPa{-} dog}(i) \land \text{\RIPb{-} dog}(i)\)
    \item\label{ill:iDogd:E:inst} \(\text{\RIPb{-} dog}(i)\)
    \item\label{ill:iDogd:E:done} \(\exists x\text{\RIPb{-} dog}(x)\)
      % \item[\(\protect\iDogd\)\space C.] Hence, the dog is \RIPb{}.
    \end{enumerate}
    \vspace{-\baselineskip}
  \end{illustration}
\end{note}

\begin{note}
  \autoref{ill:dog:E:nr} is \ref{ill:dog:C:nr}.

  Steps \ref{ill:iDogd:E:abd} to \ref{ill:iDogd:E:inst} mirror the reasoning with universal quantifier, conditionals, and conjunctions as before.

  Difference is these are with respect to some fresh constant.

  Standard.
  And, \nr{} with respect to any possible referent of \(i\).
  For, as before, agent doesn't require an interpretation of the non-logical vocabulary in order to apply these rules.
\end{note}

\begin{note}[Exists \ur{}]
  \begin{illustration}\label{ill:dog:E:ur}
    \vspace{-\baselineskip}
    \begin{enumerate}[label=\(\protect\iEDog\)\space\arabic*., ref=(\(\protect\iEDog\)\space\arabic*), align=left, leftmargin=*]
    \item Some dog is \RIPa{} \& \RIPb{}.
    \item The thing of being a \RIPa{} and \RIPb{} dog is just the combination of being a \RIPa{} dog and being a \RIPb{} dog.
    \item \emph{That dog} is a \RIPa{} and \RIPb{} dog
    \item \emph{That dog} being a \RIPa{} and \RIPb{} dog is the case when \emph{that dog} is both a \RIPa{} dog and a \RIPb{} dog.
    \item \emph{That dog} is a \RIPa{} dog and \emph{that dog} a \RIPb{} dog.
    \item \emph{That dog} is a \RIPb{} dog.
    \item Some dog is \RIPb{}.
    \end{enumerate}
    \vspace{-\baselineskip}
  \end{illustration}
\end{note}

\begin{note}
  \ur{} is far more interesting.
  First order logic, update to point to something.
  From \ur{} the agent is reasoning about that thing.
  About it's \RIPa{}ness and \RIPb{}ness.
  The conclusion doesn't require further information about the referent.
  So, remains at some level of generality.

  This may seem odd.
  Agent doesn't information about how reference is resolved.
  However, no different from common cases.
  For example, `Plato', `Grice', etc.
  Existential secures a reference, and that's all that's required.
\end{note}

\begin{note}[\emph{That dog} might not exist]
  Possible that \emph{that dog} does not exist, so the agent is not appealing to some existing thing.
  Well, sure, but same problem with `Plato'.
  Existential is sufficient to claim support.

  In other respects, the same.

  Parallel to \nagent{10}.
  Appealed to \nagent{10}'s \RIPa{} \& \RIPb{}-ness to claim support for \nagent{10}'s \RIPb{}-ness.
  In the same way, claiming support for \emph{that dog}'s \nagent{10}'s \RIPb{}-ness by appeal to \emph{that dog}'s \RIPa{} \& \RIPb{}-ness.

  Appeal to \emph{that dog} and those properties have same role.
\end{note}

\paragraph*{Similarity to \dd{}/\dr{} distinction.}

\begin{note}[Similarity to \dd{}/\dr{} distinction.]
  SEP
  Semantically de re/de dicto:
  A sentence is semantically de re just in case it permits substitution of co-designating terms salva veritate.
  Otherwise, it is semantically de dicto.

  Simple distinction in terms of co-designating terms.
  Here, issue is about how some referent contributes to truth conditions of proposition.
  However, comes down to the same idea.
  Whether or not the agent puts reference to use.
  Still, there's an important difference.
  \dd{} and \dr{} seems to talk about the status of the agent's relation to things.
  In particular, about the relation between reference applied to distinct terms.
  So, in cases with existentials, the point is that the agent doesn't have any candidate for co-reference.

  It's important to keep these two things separate.
  The agent's reasoning determines \ur{} and \nr{}.
  In contrast, it is not in general possible for the agent to determine \dd{} and \dr{}.

  The \dd{}/\dr{} distinction is more about the quality of reference.
  So, agent fails to fix a transparent reference relation.
  Yet, this doesn't uncover the way in which the agent reasons.
  So,~\citeauthor{Fitch:1981vg}'s example.
  Quite possible that the agent is reasoning about Cicero, not whatever satisfies the term.
  However, \dd{} because it's not possible to substitute.

  Some clear examples.
  Difference is with how the reference may be resolved from the perspective of the agent.
  \dr{}, object.
  \dd{}, satisfying attributions.
  So, whether the referent contributes to truth conditions.
  This makes sense.
  In the case of \dr{}, yes, in the case of \dd{}, no.

  And, in turn, this is different from how the agent proceeds to reason from proposition.

  Suppose \dr{}.
  Possible \nr{}, as the agent may not appeal to reference.
  And, possible \ur{} as agent might reason about the individual.

  Suppose \dd{}.
  Possible \nr{}, same as before.
  Also possible \ur{}.
  Still \dd{} as the referent is not \emph{directly} contributing to the truth conditions.
  However, the agent is still reasoning about that thing, whatever it is.

  So, easiest when thinking about the earlier example.
  Reasoning about some number, not something which satisfies the Peano axioms.

  Now, \dd{} and \nr{} are a natural pairing.
  If no fix on reference, then don't reason with reference.
  The important thing, however, is the way in which the proposition is evaluated.
\end{note}

\begin{note}[Metaphysical distinction.]
  Same issue.
  Details about predication don't seem to be up to the agent.
\end{note}

\begin{note}[rigid and non-rigid designation]
  Also different from rigid and non-rigid designation.
  This only applies when \ur{}.
  It may seem that non-rigid and \nr{} pair up.
  But this isn't right.
  Quite possible to fix that the reference relation is rigid, but also to never use the reference relation.
  Go through any example where one can use some kind of logical reasoning.
  It's not the case that the result could vary simply because one reasoned by logical form, so to speak.
\end{note}

\begin{note}[\ESU{}]
  Important to note that there's no issue with \ESU{} yet.
  Both types of reasoning are quite compatible.
  So, this distinction is important, but nothing follows from this distinction alone.

  I mean, \nr{} is clearly fine with both.
  The only difficulty is with \ur{}.
  However, this is also fine, as \ESU{} is only about what the agent appeals to when reasoning, and it's a different claim to hold that reference either is or is not important.

  This is a distinction with a difference, but the \emph{significance} of this difference will come later.
\end{note}

\begin{note}[Examples with properties and events]
  E.g.\ Brutus hugging Caesar.

  This is also the thing about Davidsonian event semantics.
  Arguably, \emph{de constructione} with respect to the event.
  I mean, motivated by logic.
  Still, doesn't prevent \emph{de materia}.
\end{note}

\subsection{Filling in the matrix}
\label{sec:filling-matrix}

\begin{note}
  Handful of distinctions.

  Fill in the matrix.

  First, recall \gsi{}.
\end{note}


%%% Local Variables:
%%% mode: latex
%%% TeX-master: "master"
%%% End:
