%%% Local Variables:
%%% TeX-master: "master"
%%% End:

\chapter{Notes}
\label{cha:notes}

\section{Names}
\label{sec:names}

\begin{itemize}
\item[(uRp)] Use requires possession.
\item[(uRh)] Use requires having.
\item[(uRa)] \mp{-}.\newline
%  うら (裏)
\end{itemize}



\subsection{Motivation for \uRa{}}
\label{sec:motiv-main-prem}

Some motivation for the \uRa{}:

\begin{enumerate}
\item Davidson
\item Responding to reasons
\item Hieronymi
\item Intuitive
\end{enumerate}

Broadly, the \uRa{} is interesting because it constrains how an agent obtains some conclusion by reasoning.
Instances that conform seem good, and instances that do not conform seem bad.

[Examples]

May also see how this is applied when finding solutions to difficult cases.
For example, the \uRa{} is in the background with Bratman on temptation, where the central idea is that desires fail to count as reasons because the agent would then need to act in a certain way.