\chapter{Notes}
\label{cha:notes}

\subsection{Motivation for \ESU{}}
\label{sec:motiv-main-prem}

Some motivation for the \ESU{}:

\begin{enumerate}
\item Davidson
\item Responding to reasons
\item Hieronymi
\item Intuitive
\end{enumerate}

Broadly, the \ESU{} is interesting because it constrains how an agent obtains some conclusion by reasoning.
Instances that conform seem good, and instances that do not conform seem bad.

[Examples]

May also see how this is applied when finding solutions to difficult cases.
For example, the \ESU{} is in the background with Bratman on temptation, where the central idea is that desires fail to count as reasons because the agent would then need to act in a certain way.

%%% Local Variables:
%%% mode: latex
%%% TeX-master: "master"
%%% End:
