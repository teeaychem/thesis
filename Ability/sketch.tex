\chapter{Sketch}
\label{cha:sketch}

{
  \color{red}
  To be removed.
}

\begin{note}
  The key difference between these two cases is:
  With multiplication and division, or syntax and semantics, question is about the agent's reasoning from premises to conclusion.

  However, with testimony, the question is with whether the premise of testimony is a good premise.
  So, if you already think testimony, then as testimony is factive, there wouldn't be a problem in concluding from testimony.
  Key is statement of testimony.
  In some sense, there is something puzzling.
  For, different conclusion then either calculator or own reasoning is mistaken.
  But, in the scenario we have specified testimony.

  The thing is, whether leave the premises fixed.

  Variant of \zS{}.
  Here, query premises, rather than the conclusion.
  Set this aside.

  Part of what makes \zS{} interesting is no revisions.
  In this respect, quite different to something like \citeauthor{Wright:2011wn} on transmission of warrant.
\end{note}

%%% Local Variables:
%%% mode: latex
%%% TeX-master: "master"
%%% End:
