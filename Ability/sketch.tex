\chapter{Sketch}
\label{cha:sketch}

\begin{note}
  \color{red}
  The general plan.

  Next part:
  \begin{itemize}
  \item
    Type of scenario.
  \item
    Phenomena.
  \item
    Relation between phenomena and scenario.
  \end{itemize}
  Part after:
  \begin{itemize}
  \item
    Negative resolution to \issueConstraint{} from foundations of previous part.
  \end{itemize}
\end{note}

\begin{note}
  In this chapter we provide an outline the argument we will make for a negative resolution to \issueConstraint{}, and, by extension, \issueInclusion{}.

  Arguing for something negative, so counterexample.
  At the highest level, the strategy is fairly straightforward:
  \begin{quote}
    Provide \scen{0} which require a negative resolution to \issueConstraint{}.
  \end{quote}

  Though, really, way to generate scenarios.

  Split into two parts.

  First, the scenarios.
  Second, why the scenarios require a negative resolution.
\end{note}

\begin{note}
  Second part, argument fairly simple.
  Difficult parts are scenarios, and clarifying the argument.
\end{note}

\begin{note}[Goal]
  \begin{itemize}
  \item
    From the agent's perspective.
  \item
    Support holds between \(\Phi'\) and \(\pv{\phi}{v}\), and in part this is why agent concludes \(\pv{\phi}{v}\) from \(\Phi\).
    Where, \(\Phi' \ne \Phi\).
  \end{itemize}

  Still, get to \(\pv{\phi}{v}\).
  So, \(\Phi'\) is not required to conclude \(\pv{\phi}{v}\), in general.
\end{note}

\section{Gist}

\begin{note}
  \begin{enumerate}
  \item
    Support
  \item
    \cScen{1}
  \item
    \qzS{}
  \item
    If \qzS{} holds, then relation of support, in part, answers \qWhy{}.
  \item
    Negative resolution.
  \end{enumerate}
\end{note}

\subsection{\zS{}}
\label{sec:zs}

\begin{note}
  Basic observation.

  These conditionals.

  If were to fail, would not.

  Present in all \scen{1} considered, and in type of \scen{0} of interest, concern reasoning the agent has the option of doing.

  In some cases, seems a clear role for these type of conditionals.
\end{note}

\begin{note}
  Negative cases, where concluding is blocked due to the presence of some conditional, and doubt about whether it is the case.
\end{note}

\paragraph{\zSN{0}}

\begin{note}[`\zSN{0}']
  Particular kind of support.
  \zSN{0}, or, rather, \zS{}.

  Is it the case that agent may fail to conclude?

  Additional property of relation of \support{}.
  If agent concludes, not only support, but \zS{}.
  \begin{itemize}
  \item
    \support{2}: premises and conclusion.
  \item
    \zS{}: checks.
  \end{itemize}

  In various cases, \support{} only if \zS{}.
\end{note}

\begin{note}
  \begin{idea}
    In cases of \zS{}:

    If agent were to take up option and fail to conclude, no support.
  \end{idea}
\end{note}

\begin{note}[Limitation and closure condition]
  So, conversely.

  \begin{itemize}
  \item
    Support \emph{only if} not the case that if the agent were to take up the option, would fail to conclude.
  \end{itemize}

  Simply:

  \begin{itemize}
  \item
    Support \emph{only if} if the agent were to take up the option, would not fail to conclude.
  \end{itemize}

  Here, turned into a closure condition.
\end{note}

\begin{note}
  \color{red}

  Now, an important thing to note here is that we're not characterising a distinct instance of support.
  Rather, we have these additional constraints on the agent's epistemic state which, in turn, constrain support.

  Here, then, in part explanation why.

  For, support, only if conditional holds.
\end{note}

\paragraph*{Working with a case}

\begin{note}
  Return to~\autoref{illu:sketch:math}.

  \begin{quote}
    %\scenarioClacMulDiv*
  \end{quote}

  Intuitively, if the agent were to \emph{conclude} \(23 \times 15 = 345\), the agent would also, at the same time, conclude \(345 \div 15 = 23\).

  Of course, this intuition may be resisted.
  For example, \(345 = (3 \times 5 \times 23)\) and \(15 = 3 \times 5\).
  Intuitively, it is not the case that the agent would conclude \(23 \times (3 \times 5) = (3 \times 5 \times 23)\).

  The equivalence between multiplication and division is sufficiently familiar.
\end{note}

\begin{note}
  {
    \color{red}
    Possibly to go in a footnote, or move to later chapter.
  }

  Suppose syntax.
  Well, check with semantics.
  If fail semantics, then something wrong with syntactic reasoning.

  Conversely, semantics.
  Then, syntax.

  Intuitively, \zS{}.
  Not the case that agent would reach a different conclusion.
  Strength of reasoning for either syntax or semantics.

  Observation.
  Can't go from syntax to semantics while preserving \zS{}.
  Possible to check semantics ignoring proof.

  Conversely, semantics is no good for \zS{} without syntax.

  Key observation is, \zS{}, if either syntax or semantics is involved, then this doesn't answer the question.
  At issue is whether the agent may fail, and in doing so unwind the main conclusion.
\end{note}

\begin{note}[Give up on \zS{}]
  Intuition, and \zS{} is quite weak.
  I mean, look, there's no doubt.
  This is clear with multiplication and division.
  There's no problem with reasoning via either multiplication or division.
  Observation that I would not conclude via X if I fail to conclude via Y.
  However, this does no suggest that there is anything problematic with reasoning.

  Compare to examples of failure.
  Not like keys.
  Not like\dots? (letter requires novel information.)

  The difficulty is accounting for why.
  Why would not fail, independent of multiplication.
\end{note}

\begin{note}[Do the reasoning]
  One option, do the reasoning.
  This seems excessive.
  For, mostly, the same reasons as above.
  Multiplication and division are straightforward.
\end{note}

\begin{note}
  Not clear to me how puzzling this is.

  On some days, quite puzzling.
  Agent has not witnessed reasoning.

  On other days, \fc{0}.
  There's really no need for the agent to witness.

  What is surprising, at least, is how this relates to \qWhy{} and \qHow{}.
\end{note}

\begin{note}[Variant with logic]
  Here, second go with syntax and semantics.
  This has the benefit of clarity over conclusion.
  Downside is that you may not be a logician.

  Syntactic reasoning is fine, and equally, semantic reasoning is fine.
\end{note}

\paragraph{More broadly}

\begin{note}
  Key thing, check.
  Prior to concluding \(\pv{\phi}{v}\) from some pool of premises \(\Phi\), check on whether it makes sense, from agent's perspective, to conclude \(\pv{\phi}{v}\) from \(\Phi\).

  In particular, cases where, if \(\phi\) has value \(v\) then it is possible for the agent to conclude \(\psi\) has value \(v'\) from some pool of premises \(\Psi\).
  Here, permutations.
  Of interest for the moment:
  \(\phi = \psi\) but \(\Phi \ne \Psi\).
  However, also \(\phi \ne \psi\) and \(\Phi \ne \Psi\).

  Keys.

  This prevents.

  Parcel delivered, check the address before opening.

  Cases go the other way.
  Check is satisfied.

  Wason selection task.
  Here, it is clear what needs to be checked.
  If this proposition is true, then these things are possible.

  Here, consistent with witnessing.
  Turned over cards, then fine.
  Haven't reasoned from place to absence of keys.

  Many instances are.

  Relative, so long as conclusion is true.
\end{note}

\begin{note}[Not all concluding is like this]
  Not all concluding is like this.
  Testimony.
  No way to check.
  Of course, may be various ways to check the sources.
  However, focus on novel conclusions.
\end{note}

\begin{note}[Re-evaluate intuitions?]
  Of course, interest in this property has only just been made explicit.
  Perhaps revise intuition.

  If so, then great!

  Though, I expect not.
\end{note}

\section{Supplements}
\label{sec:overview:supplements}

\subsection{Back to testimony (and a variant of \zS{})}

\begin{note}
  The key difference between these two cases is:
  With multiplication and division, or syntax and semantics, question is about the agent's reasoning from premises to conclusion.

  However, with testimony, the question is with whether the premise of testimony is a good premise.
  So, if you already think testimony, then as testimony is factive, there wouldn't be a problem in concluding from testimony.
  Key is statement of testimony.
  In some sense, there is something puzzling.
  For, different conclusion then either calculator or own reasoning is mistaken.
  But, in the scenario we have specified testimony.

  The thing is, whether leave the premises fixed.

  Variant of \zS{}.
  Here, query premises, rather than the conclusion.
  Set this aside.

  Part of what makes \zS{} interesting is no revisions.
  In this respect, quite different to something like \citeauthor{Wright:2011wn} on transmission of warrant.
\end{note}

\begin{note}
  Calculator.
  Check to see whether or not it's functioning correctly.
  Understanding of arithmetic.
  If fail to conclude sum, then calculator is no good.

  \zS{}.
  Again, key point here is that if something else, then wouldn't conclude.

  Conclude sum from calculator only if conclude sum from understanding of arithmetic.

  But, may also go for calculator over own reasoning.
  Especially when tired, or stressed.
\end{note}

\begin{note}
  {
    Right, the key thing is that there are two sources of information.
    \begin{itemize}
    \item
      Testimony
    \item
      Understanding of arithmetic
    \end{itemize}
    What matters is that these align.
    So long as you have this, then when you get a new piece of information from either, expectation that both sources will provide the same information.
    If break equivalence between the two sources, then neither works.
    Here, each may function separately.

    Keeping parity.

    So, how really isn't all that important.
  }
\end{note}

\paragraph{Aside: Method}

\begin{note}
  Main points from above suggest including method.

  For example, independent set and vertex cover problem.
  Here, reduction.
  However, same premises and same conclusion (given understanding of equivalence).

  So, same issue.

  Here, to keep things relatively simple, avoid method.

  Problem here is that things are mostly the same in these cases, in contrast to calculator versus understanding of arithmetic.
\end{note}


%%% Local Variables:
%%% mode: latex
%%% TeX-master: "master"
%%% End:
