\chapter{Sketch}
\label{cha:sketch}

\begin{note}
  In this chapter we provide an outline the argument we will make for a negative resolution to \issueConstraint{}, and, by extension, \issueInclusion{}.

  Arguing for something negative, so counterexample.
  At the highest level, the strategy is fairly straightforward:
  \begin{quote}
    Provide \scen{0} which require a negative resolution to \issueConstraint{}.
  \end{quote}

  Though, really, way to generate scenarios.

  Split into two parts.

  First, the scenarios.
  Second, why the scenarios require a negative resolution.
\end{note}

\begin{note}
  Second part, argument fairly simple.
  Difficult parts are scenarios, and clarifying the argument.
\end{note}

\section{Gist}

\begin{note}
  \begin{enumerate}
  \item
    Support
  \item
    \cScen{1}
  \item
    If \fc{0}, then support.
  \item
    Relate support with \fc{0} to reasoning.
  \item
    \zS{}
  \item
    \zS{} compatible with either, so independent.
  \item
    Either concluded or \fc{0} is part of why.
  \item
    \cScen{0}, both disjuncts.
  \item
    Problem with restricting to concluded.
  \item
    So, \fc{0}
  \item
    \fc{0} is in part why.
  \item
    Negative resolution.
  \end{enumerate}
\end{note}

\subsection{\zS{}}
\label{sec:zs}

\begin{note}
  Basic observation.

  These conditionals.

  If were to fail, would not.

  Present in all \scen{1} considered, and in type of \scen{0} of interest, concern reasoning the agent has the option of doing.

  In some cases, seems a clear role for these type of conditionals.
\end{note}

\begin{note}
  Negative cases, where concluding is blocked due to the presence of some conditional, and doubt about whether it is the case.
\end{note}

\begin{note}
  Partial argument, by parity.
  Cases where blocks, then also cases where reason why.
\end{note}

\begin{note}
  Here, this is compatible with positive resolution to \issueConstraint{}.
  Agent has already witnessed reasoning for any such conditional.
\end{note}

\begin{note}
  Disjunction.
\end{note}

\begin{note}
  Motivation, range of cases where this kind of support would not hold.
\end{note}

\hozline{}

\paragraph{\zSN{0}}

\begin{note}[`\zSN{0}']
  Particular kind of support.
  \zSN{0}, or, rather, \zS{}.

  Is it the case that agent may fail to conclude?

  Additional property of relation of \support{}.
  If agent concludes, not only support, but \zS{}.
  \begin{itemize}
  \item
    \support{2}: premises and conclusion.
  \item
    \zS{}: checks.
  \end{itemize}

  In various cases, \support{} only if \zS{}.
\end{note}

\begin{note}
  \begin{idea}
    In cases of \zS{}:

    If agent were to take up option and fail to conclude, no support.
  \end{idea}
\end{note}

\begin{note}[Limitation and closure condition]
  So, conversely.

  \begin{itemize}
  \item
    Support \emph{only if} not the case that if the agent were to take up the option, would fail to conclude.
  \end{itemize}

  Simply:

  \begin{itemize}
  \item
    Support \emph{only if} if the agent were to take up the option, would not fail to conclude.
  \end{itemize}

  Here, turned into a closure condition.
\end{note}

\begin{note}
  \color{red}

  Now, an important thing to note here is that we're not characterising a distinct instance of support.
  Rather, we have these additional constraints on the agent's epistemic state which, in turn, constrain support.

  Here, then, in part explanation why.

  For, support, only if conditional holds.
\end{note}

\paragraph*{Working with a case}

\begin{note}
  Return to~\autoref{illu:sketch:math}.

  \begin{quote}
    \scenarioClacMulDiv*
  \end{quote}

  Intuitively, if the agent were to \emph{conclude} \(23 \times 15 = 345\), the agent would also, at the same time, conclude \(345 \div 15 = 23\).

  Of course, this intuition may be resisted.
  For example, \(345 = (3 \times 5 \times 23)\) and \(15 = 3 \times 5\).
  Intuitively, it is not the case that the agent would conclude \(23 \times (3 \times 5) = (3 \times 5 \times 23)\).

  The equivalence between multiplication and division is sufficiently familiar.
\end{note}

\begin{note}
  {
    \color{red}
    Possibly to go in a footnote, or move to later chapter.
  }

  Suppose syntax.
  Well, check with semantics.
  If fail semantics, then something wrong with syntactic reasoning.

  Conversely, semantics.
  Then, syntax.

  Intuitively, \zS{0}.
  Not the case that agent would reach a different conclusion.
  Strength of reasoning for either syntax or semantics.

  Observation.
  Can't go from syntax to semantics while preserving \zS{}.
  Possible to check semantics ignoring proof.

  Conversely, semantics is no good for \zS{} without syntax.

  Key observation is, \zS{}, if either syntax or semantics is involved, then this doesn't answer the question.
  At issue is whether the agent may fail, and in doing so unwind the main conclusion.
\end{note}

\begin{note}[Options]
  \begin{enumerate}
  \item
    Give up on \zS{}.
    {
      \color{red}
      (Seems unintuitive, and, link to concluding.)
    }
  \item
    Conclude independently, and then get \zS{}.
    {
      \color{red}
      (Seems redundant.)
    }
  \item
    Get \fc{0} prior.
    {
      \color{red}
      (But, no information about what the result is.)
    }
  \item
    Same time.
    \begin{enumerate}
    \item
      Witness.
      {
        \color{red}
        (But, distinct premises.)
      }
    \item
      \fc{0}.
    \end{enumerate}
  \end{enumerate}
\end{note}

\begin{note}[Give up on \zS{}]
  Intuition, and \zS{} is quite weak.
  I mean, look, there's no doubt.
  This is clear with multiplication and division.
  There's no problem with reasoning via either multiplication or division.
  Observation that I would not conclude via X if I fail to conclude via Y.
  However, this does no suggest that there is anything problematic with reasoning.

  Compare to examples of failure.
  Not like keys.
  Not like\dots? (letter requires novel information.)

  The difficulty is accounting for why.
  Why would not fail, independent of multiplication.
\end{note}

\begin{note}[Do the reasoning]
  One option, do the reasoning.
  This seems excessive.
  For, mostly, the same reasons as above.
  Multiplication and division are straightforward.
\end{note}

\begin{note}[\fc{0} prior]
  Well, intuitively, \fc{0}.
  However, no information about \emph{which}.
\end{note}

\begin{note}[Simultaneous]
  Simultaneous conclusion.

  Reasoned from one, the other is a \fc{0}.
  Information from reasoning, \emph{which} \fc{0}.

  \fc{0}, \emph{why} \zS{}.

  Key point, premises for information are not premises for \fc{0}.
  If premise, then \zS{} is voided.
  Witnessing reasoning is in part \emph{how}, but is not tied to \emph{why}.

  Here, negative answers.
  No witnessed reasoning from division.
  Reasoning from division is not included in how.
\end{note}

\begin{note}
  Not clear to me how puzzling this is.

  On some days, quite puzzling.
  Agent has not witnessed reasoning.

  On other days, \fc{0}.
  There's really no need for the agent to witness.
  Though, it is clear how important this is for multiplication.
\end{note}

\begin{note}
  One issue, agent may still reason from division and fail.
  But, it is not clear to me how important this really is.
  Witness, then there is no guarantee that the reasoning is not faulty.
  This gets covered later.
\end{note}

\begin{note}[Variant with logic]
  Here, second go with syntax and semantics.
  This has the benefit of clarity over conclusion.
  Downside is that you may not be a logician.

  Syntactic reasoning is fine, and equally, semantic reasoning is fine.
\end{note}


\subsection{\fc{3}}
\label{sec:fc3}

\begin{note}
  So, general type of \scen{0}.

  Here, we have some test.

  Relation of support between \(\pvp{\psi}{v'}{\Psi}\) is, in part, answer to why.
\end{note}



\begin{note}[Outline]
  Idea of a \fc{0}.
  \(\phi\) having value \(v\) is a \fc{0} with respect to an agent's epistemic state just in case the agent would not fail to conclude \(\phi\) having value \(v\).

  If \fc{0}, then some pool of premises \(\Phi\) available to the agent such that it is possible for the agent to witness reasoning from \(\Phi\) to \(\phi\) having value \(v\).

  Future variant of~\autoref{idea:support}.
  If agent has concluded, then support.
  Agent has not concluded, but sufficient.

  Nothing too surprising.
  Recall, knowing whether.
  There is nothing apart from witnessing.

  \fc{0}, alone, then, not too interesting.
  Information that \(\phi\) having value \(v\) is a \fc{0}.
  A little more interesting.

  Now, here, \emph{that} \fc{0}.

  Information about the possibility of establishing some conclusion.
  Nothing too surprising.
  Questions in textbooks.

  Pair these two things together.
  Well, okay.
  Hint.
  Here, then, open to go from \emph{that} \fc{0} to \(\Phi\).

  Given \(\Phi\), everything else is redundant.

  With a \fc{0}, the pool of premises, these do the work of establishing the possibility of how.

\end{note}


\paragraph{Key idea: \fc{1}}

\begin{note}
  Core idea:
  \begin{idea}[\fc{1} and support]
    \label{idea:fc-and-support}
    If \(\phi\) having value \(v\) is a \fc{0}, then relation of support between \(\phi\) having value \(v\) and some pool of premises \(\Phi\).
  \end{idea}

  Following suggestion made with respect to \autoref{illu:gist:calc}.

  With this idea, possible for relation of support to be in part why.

  Given this, still a question of how \fc{0} are involved.
\end{note}

\begin{note}
  As with \autoref{illu:gist:calc}, possibility of failure to conclude.

  Particular type of support.

  Agent would not conclude otherwise.

  Fairly natural.
\end{note}

\begin{note}
  So, scope of a fairly natural type of support.%
  \footnote{
    \color{red}
    This should be moved somewhere else, but it would be useful to emphasise that \zS{} really is a type of support, in the sense that it is why the agent concludes in various cases.
    For, without support for each \requ{}, the agent would not conclude.
  }
\end{note}

\begin{note}
  \color{red}
  Important to note here is that this doesn't entail the relation of support, in part, answers \qWhyV{}.

  By-product.

  Still, interesting.
  This will, more-or-less be the basis for our argument.
\end{note}

\begin{note}
  \color{blue}
  Perhaps here include some comments on general and specific ability.
  This is in part how get these kind of scenarios.
  In some cases, same ability, different premises, and in other cases links between ability.
\end{note}

\begin{note}
  \autoref{idea:fc-and-support} is fundamental.

  Distinction between:
  \begin{enumerate}
  \item
    Support between premises and conclusion of a \fc{0}.
  \item
    Support between \fc{0} and conclusion of \fc{0}.
  \end{enumerate}

  To illustrate.
  Grant for a moment, understanding of arithmetic is involved in why.

  Not understanding of arithmetic, but that I, or you, understand arithmetic.
  Not understanding, but \emph{that} understand.
\end{note}

\begin{note}[Re-evaluate intuitions?]
  Of course, interest in this property has only just been made explicit.
  Perhaps revise intuition.
  Though, I expect not.

  Though it is true that the agent would not conclude, did not do the reasoning.

  Stated in other terms, \(\psi\) from \(v'\) is a \fc{} if \(\phi\) has value \(v\).
  Still, no relevance.
\end{note}

\paragraph*{Major and minor problems}

\begin{note}
  Two problems.

  Minor and major.

  Both comes from information required.

  Minor.
  \emph{that} \fc{0}.
  Distinction between \(\Phi\) and that conclude from \(\Phi\).
  Need \emph{that} \fc{0}.

  Major.

  \itp{} gives information.
  So, conclude from that.
\end{note}

\paragraph*{Minor}

\begin{note}[Minor]
  Minor problem.
  Thing is, no clear leverage.
  Certain nice things about \(\Phi\), and \emph{that} \fc{0} is much weaker.
  Only thing missing is witnessing.

  Basically, only really get \emph{that} \fc{0} doing any work if support between \(\Phi\) and \(\pv{\phi}{v}\).

  In part, this is why focus attention on support.

  As we have seen, independent motivation for relation of support.
  Tied to why.

  Part of what it is for something to be a \fc{0} is a relation of support.
\end{note}

\paragraph*{Major}

\begin{note}[Major]
  Redundant.
  Possible answer to \qWhy{}.
  However, this doesn't show that is an answer to \qWhy{}.

  In general, various possible answers to \qWhy{}.
  This is key.
  Issues \issueInclusion{} and \issueConstraint{} narrow down.

  \citeauthor{Boghossian:2014aa}, recall.
  Of course, much different that just saying, or being response dependent.
  \fc{0}.
  So, answers \qWhy{}, at least in principle.

  Still, relevant information, \itp{}, etc.\ must do enough to get \(\pv{\phi}{v}\).
  For, if not then it seems can't use that \(\pv{\phi}{v}\) is a \fc{0}.

  Squeeze out role for \fc{0}.

  Indeed, \fc{0} from understanding of arithmetic.
  But, it is not obvious that this is why.
  Calculator does it all.

  Find role for \fc{0}.
  Tied to why an agent concludes.
  This is our goal.
  Show how \fc{0} is, in part, why an agent concludes.

  \begin{itemize}
  \item
    From the agent's perspective.
  \item
    Support holds between \(\Phi'\) and \(\pv{\phi}{v}\), and in part this is why agent concludes \(\pv{\phi}{v}\) from \(\Phi\).
    Where, \(\Phi' \ne \Phi\).
  \end{itemize}

  Still, get to \(\pv{\phi}{v}\).
  So, \(\Phi'\) is not required to conclude \(\pv{\phi}{v}\), in general.
\end{note}

\paragraph*{Role for \fc{1}}

\begin{note}
  Role for \fc{1}.

  Concluding from some pool of premises.
  More complex.
  Key thing, check.
  Prior to concluding \(\pv{\phi}{v}\) from some pool of premises \(\Phi\), check on whether it makes sense, from agent's perspective, to conclude \(\pv{\phi}{v}\) from \(\Phi\).

  In particular, cases where, if \(\phi\) has value \(v\) then it is possible for the agent to conclude \(\psi\) has value \(v'\) from some pool of premises \(\Psi\).
  Here, permutations.
  Of interest for the moment:
  \(\phi = \psi\) but \(\Phi \ne \Psi\).
  However, also \(\phi \ne \psi\) and \(\Phi \ne \Psi\).

  Keys.

  This prevents.

  Parcel delivered, check the address before opening.

  Cases go the other way.
  Check is satisfied.

  Wason selection task.
  Here, it is clear what needs to be checked.
  If this proposition is true, then these things are possible.

  Here, consistent with witnessing.
  Turned over cards, then fine.
  Haven't reasoned from place to absence of keys.

  Many instances are.

  Relative, so long as conclusion is true.
\end{note}

\begin{note}[Not all concluding is like this]
  Not all concluding is like this.
  Testimony.
  No way to check.
  Of course, may be various ways to check the sources.
  However, focus on novel conclusions.
\end{note}




\subsection{Tension}
\label{sec:tension-1}

\section{Supplements}
\label{sec:overview:supplements}

\subsection{Back to testimony (and a variant of \zS{})}

\begin{note}
  The key difference between these two cases is:
  With multiplication and division, or syntax and semantics, question is about the agent's reasoning from premises to conclusion.

  However, with testimony, the question is with whether the premise of testimony is a good premise.

  The thing is, whether leave the premises fixed.

  Variant of \zS{}.
  Here, query premises, rather than the conclusion.
  Set this aside.

  Part of what makes \zS{} interesting is no revisions.
  In this respect, quite different to something like \citeauthor{Wright:2011wn} on transmission of warrant.

  Still, this is not too much of a stretch.
  For, would still prevent concluding.

  However, this gets tricky.
  Because, testimony is involved as a premise.
  Well, no clear recursion problem, assuming well-structured premise-conclusion.
  For, \zS{} still relative to premise-conclusion pairings.

  The real difficulty is conclusion being enough.

  I mean, it's just a different condition.
\end{note}

\begin{note}[Other cases are puzzling]
  The introductory \illu{0}, with the calculator, this is more puzzling.
  Seems, \fc{0} from calculator.

   Without \fc{0}, no concluding.
  Without calculator, no \fc{0}.

  No concluding, as would not conclude if different.
  \fc{0}, given understanding of arithmetic, but no account of which.

  However, difference.
  Calculator is independent of agent's own reasoning.
  Though, this gets somewhat tricky.
  Testimony, other agents.
\end{note}

\begin{note}
  Calculator.
  Check to see whether or not it's functioning correctly.
  Understanding of arithmetic.
  If fail to conclude sum, then calculator is no good.

  \zS{}.
  Again, key point here is that if something else, then wouldn't conclude.

  Conclude sum from calculator only if conclude sum from understanding of arithmetic.
\end{note}

\begin{note}
  {
    \color{blue}
    First thing, can't use calculator.

    More broadly, ordering problem.

    Without \fc{0}, no conclusion from testimony.
    Without testimony, no which for \fc{0}.
  }

  {
    \color{red}
    Solution.

    Well, key thing is that calculator is fine.
    And, if calc, then given understanding of arithmetic, would conclude.
    No suggestion that calculator is faulty in any way.
    However, this doesn't mean that we don't have a check.

    At the same time.
    So, calculator \emph{and} \fc{}.

    \fc{} is not coming from calculator.
    Only getting information from calculator.
    Not worries about calculator, because would conclude.

    This is a somewhat puzzling conclusion.
    However, it's fine.

    See, the best we can do here is press the idea that the agent may still conclude otherwise.
    But, not from the agent's present epistemic state.
    For, \fc{0}: Would not fail from present epistemic state.

    Of course, could still press this, but then no different from any other conclusion.

    Hence, would need to hold that the agent had to conclude from understanding of arithmetic strictly prior.
    This, the agent hasn't done.

    So, then, the problem with this is that in general, have cases of such `simultaneous conclusions'.
    This requirement, paired with idea of \zS{}, problem for ability.

    So, reject \zS{}.
    Yet, intuitive constraint on concluding.
    }

  In other words, calculator only if \fc{0}.
  Understanding of arithmetic also supports.

  Testimony.
  So, receive testimony.
  Also, understanding of arithmetic.
  No witnessing, so if testimony is right, then \fc{0}.
  But, if no support, then would not conclude.

  Testimony only to the extent something I already know (whether).
  So, part of why is that know whether.
  Only if already a \fc{0}.

  Hence, \qWhy{} not included in \qHow{}.
  And, more generally, no witnessing.

  Testimony\dots property of being testimony\dots
  Telling me things I already know (whether).
  No granting testimony if not already supported.


  Why is testimony fine?
  Because \fc{0}!
  Well, maybe.
  {
    \color{red}
    Ugh, this is difficult.
    The cases I have where \fc{0} is clearest involve other problems.
    I.e.\ sudoku, so don't need to test reasoning against any other problems.

    So, testimony is irrelevant to whether, it's not a \fc{0} because testimony.
    Already have \fc{0}, already `know' whether.
    What testimony does is inform \emph{which}.
  }

  {
    \color{red}
    This is kind of the puzzle.
    Without \fc{0}, don't get testimony (because, independent test).
    And,
    Without testimony, don't get \fc{0} (because, need info which).

    So, these two things come in at the same time.
    When go from testimony, also get that \fc{0}.

    But, why is only tied to understanding of arithmetic, because else an ordering problem.

    So, the core idea is that the relationship between testimony and understanding of arithmetic is already in place, prior to getting information about which from the calculator.

    Right, the key thing is that there are two sources of information.
    \begin{itemize}
    \item
      Testimony
    \item
      Understanding of arithmetic
    \end{itemize}
    What matters is that these align.
    So long as you have this, then when you get a new piece of information from either, expectation that both sources will provide the same information.
    If break equivalence between the two sources, then neither works.
    Here, each much function separately.

    Keeping parity.

    So, how really isn't all that important.
  }

  Going from testimony to \fc{0} doesn't work.
  Because, \fc{0} is independent check.
  Right, because then testimony is still pending on whether \fc{0}.

  This is delicate.
  It seems as though, check on whether testimony.
  So, no testimony.

  However, this is not quite right.
  It is a check on testimony.
  However, two things.
  First, understanding and second testimony.
  Testimony is fine, understanding of arithmetic only presents the option of checking.
  Understanding does not suggest that testimony is bad/statement does not amount to testimony.
  However, understanding of arithmetic is in part why because this dismisses the possibility.

  Testimony, pressure on understanding of arithmetic.

  Testimony, but if testimony and \fc{0}, then determine whether really is testimony.
  If go via testimony, then either limited support, or \fc{0} does work.

  Structurally similar to ability.
  If specific from general, then only so good as specific.

  Of course, only when possible to check.
  If not possible, then this restriction isn't in place.

  In other words, attention shifts from cases of concluding in general, to specific cases.

  If sufficient to conclude, then what role has alternative option?
\end{note}

\begin{note}
  Of course, \emph{that} \fc{0} is key with respect to how.
  If the agent has not witnessed reasoning, then need some source of information.
  However, not for \emph{why}.

  Note, possibility of witnessing is given by X, as is that X supports Y.
  X is sufficient.
  Nothing more is needed for X to support Y.
\end{note}



\section{Concluding}
\label{sec:ideas-1}

\begin{note}[Outline]
  Here, clarify our use of the terms `concludes', `concluding', `concluded', and so on.
\end{note}

\subsection{The agent's epistemic state}
\label{sec:agents-epist-state}

\begin{note}
  Idea.
\end{note}

\begin{note}
  Distinguish agent's epistemic state from \stance{} of the agent.

  Various things we hold to be the case, and these may be relativised to the agent's perspective on how things actually are.
  Other things, hold regardless of the agent's perspective on how things actually are.
  And, things that hold regardless of whether the agent recognises.

  Example.
  Sam shorter than Taylor.
  Then, from epistemic state, Taylor shorter than Sam.
  Doesn't matter whether Sam is shorter than Taylor.
  From perspective of agent's epistemic state.
  Likewise, doesn't matter whether agent recognises shorter.

  Similarly, classical and intuitionistic.
  From int.\ perspective not a proof.
  From classical, also a proof of \(\phi\).

  And, if left the oven on, should check.
  Regardless of whether really did, and regardless of agent's morals.

  No clean divide.
  For present purposes, concluding and \csN{} will take epistemic state as input, but ignore the agent's \stance{}.
  What this means in practice will be seen through the following discussion.
  Roughly, what an agent concludes will depend on epistemic state, and whether \csN{} in concluding will likewise depend.
  Neither will depend on whether the agent recognises, or takes themselves, to have concluded.
\end{note}

\section{Broader implications}

\begin{note}
  Separate concluding to district components.
  Witnessing a foregone-conclusion.
  There is nothing that witnessing adds which is not already determined by the agent's present epistemic state.

  Nothing of interest to be gain by witnessing, hence, in some sense of the term concluding, conclude conclusion from premises without witnessing reasoning from premises to conclusion.

  Conclusion, as a term up for question.
  Whether witnessing reasoning from premises to conclusion is part of what the phenomenon of concluding some conclusion from some premises is.
\end{note}

\begin{note}
  Main this is resolving issue.
  Two other upshots.

  \begin{enumerate}
  \item
    Reduction of concluding.
    In various cases, abundance of \fc{0} suggests witnessing is not explanatory.
    Limited to how.
  \item
    Concluding without witnessing.
    {
      \color{red}
      I think I get close to this regardless, by looking at cases where an agent develops some general ability.
    }
    Stronger variation.
    If \fc{0}, then don't need to witness.
    Note, this is strictly stronger.
    For, main issue only tells us that some pool of premises is involved in why.
  \end{enumerate}
\end{note}

{
  Goal is to argue that witnessing is the only distinguishing feature.
  There are cases, such as the one described above, in which equal role.

  Core idea, supports to more general points of interest.

  More broadly, separate witnessing from concluding.
  Suggest there are cases in which conclude without witnessing.
}

\paragraph*{Aside: Method}

\begin{note}
  Main points from above suggest including method.

  For example, independent set and vertex cover problem.
  Here, reduction.
  However, same premises and same conclusion (given understanding of equivalence).

  So, same issue.

  Here, to keep things relatively simple, avoid method.

  Problem here is that things are mostly the same in these cases, in contrast to calculator versus understanding of arithmetic.
\end{note}


%%% Local Variables:
%%% mode: latex
%%% TeX-master: "master"
%%% End:
