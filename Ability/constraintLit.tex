\chapter{Instances of \issueConstraint{}}
\label{cha:lit}

\begin{note}
  \autoref{cha:var} introduced \qWhyV{}, \qHowV{}, and \issueConstraint{}, variants to \qWhy{}, \qHow{}, and \issueInclusion{}, respectively.

  In short, having a \wit{0} for a \ros{} is necessary for the \ros{} to, in part, explain why an agent concluded.

  Motivation for \issueConstraint{} follows motivation for \issueInclusion{}.

  Primary motivation is intuition.
  \scen{1} such as \autoref{illu:gist:calc} and \autoref{scen:animalism}.

  Theoretical motivation.
  In particular \citeauthor{Davidson:1963aa}'s (causal) theory of action.
\end{note}

\begin{note}
  In this section we collect a handful of extracts from the literature which suggest this intuition is not merely an intuition, but a common theoretical constraint.

  Provide extract.
  Observe how plausibly understood in terms of \wit{} and constraint.

  In each of these cases, there is no immediate entailment.
  However, suggestive, \dots
\end{note}

\begin{note}
  \begin{TOCEnum}
  \item
    \TOCLine{cha:lit:causal}
  \item
    \TOCLine{cha:lit:indeterminate}
  \item
    \TOCLine{cha:lit:non-causal}
  \item
    \TOCLine{cha:lit:normative}
  \item
    \TOCLine{cha:lit:misc}
  \end{TOCEnum}
\end{note}

\begin{note}
  Additional `doxastic'.
  Consider in \autoref{cha:embed} with the aid of definition.
\end{note}

\newpage


\section{Causal}
\label{cha:lit:causal}

\begin{note}
  Causal theories of reasoning.

  Broadly, premises stand in a causal relation to conclusion.

  Cause is something which happens, therefore \wit{}.
  What is relevant to the activity is given why causal process.
\end{note}

\subsection*{\textcite{Armstrong:1968vh}}

\begin{note}
  We being with~\cite{Armstrong:1968vh}'s (\citeyear{Armstrong:1968vh}) account of inferring.

  \begin{quote}
    We are not concerned here with logicians' questions about inference, but solely with the psychological process of inferring.
    The primary sense of the word is that in which it involves acquiring a belief on the basis of a belief already held.

    [\dots] to say that A infers \emph{p} from \emph{q} is simply to say that A's believing \emph{q} \emph{causes} him to acquire the belief \emph{p}.
    And the sense of `cause' employed here is the common or billiardball sense of `cause', whatever that sense is.%
    \mbox{}\hfill\mbox{(\citeyear[194]{Armstrong:1968vh})}
  \end{quote}

  \citeauthor{Armstrong:1968vh} tightens the account of inference a little in order to avoid including any belief \emph{r} in the causal chain leading to an agent acquiring the belief \emph{p} as a premise, but we set these details aside.
  (\citeyear[195--197]{Armstrong:1968vh})%
  \footnote{
    Indeed, any disagreement with \citeauthor{Armstrong:1968vh}'s restriction is of no real interest to us.
    For, if we grant that the relevant instance of causation provides a \wit{} for a \ros{}, then, re-expressed, \citeauthor{Armstrong:1968vh}'s revisions narrow down the \wit{1} of interest.
  }

  \citeauthor{Armstrong:1968vh} talks about `inference' rather than `conclusion'.
  However, plausibly the same thing.
  Belief \(\phi\) only if pair \(\phi\) with value `true'.
\end{note}

\begin{note}
  Of some interest.
  \citeauthor{Armstrong:1968vh} traces the causal account of inference to ~\citeauthor{Moore:1962up} and Hume.
\end{note}

\subsection*{\textcite{Boghossian:2014aa}}

\begin{note}
  Seen above.

  \citeauthor{Boghossian:2014aa}'s Taking Condition narrows the relevant causal processes.

  \begin{quote}
    The intuition behind the Taking Condition is that no causal process counts as inference, unless it consists in an attempt to arrive at a belief by figuring out what, in some suitably broad sense, is supported by other things one believes.
    %
    \mbox{ }\hfill\mbox{(\citeyear[5]{Boghossian:2014aa})}
  \end{quote}

  So, causal processes, and hence to matter, causally involved.
  If causally involved, then \wit{}.
\end{note}

\begin{note}
  Plausibly, at least.

  
\end{note}

\section{Indeterminate}
\label{cha:lit:indeterminate}

\subsection*{\textcite{Wright:2014tt}}

\begin{note}
  \citeauthor{Wright:2014tt}'s (\citeyear{Wright:2014tt}) `Simple Proposal'.
  Appealed to the Simple Proposal above on \autopageref{Wright-simple-supportI} to help clarify \supportI{}.

  Here, observation is on acceptance and move.
  Recall, the core of the Simple Proposal is the following idea:

  \begin{quote}
    [A] thinker infers q from p\(_{1}\) \(\cdots\) p\(_{\text{n}}\) when he accepts each of p\(_{1}\) \(\cdots\) p\(_{\text{n}}\), moves to accept q, and does so for the reason that he accepts p\(_{1}\) \(\cdots\) p\(_{\text{n}}\).%
    \mbox{}\hfill\mbox{(\citeyear[33]{Wright:2014tt})}
  \end{quote}

  Accepting and moving provides a \wit{}, and inferring is understood via the \wit{}.
\end{note}

\subsection*{\textcite{Broome:2002aa}}

\begin{note}
  The same observation extends to \citeauthor{Broome:2002aa}'s (\citeyear{Broome:2013aa}) rule following account of (active) reasoning.

  \begin{quote}
    Active reasoning is a particular sort of process by which conscious premise-attitudes cause you to acquire a conclusion-attitude.
    The process is that you operate on the contents of your premise-attitudes following a rule, to construct the conclusion, which is the content of a new attitude of yours that you acquire in the process.%
    \mbox{ }\hfill\mbox{(\citeyear[234]{Broome:2002aa})}
  \end{quote}

  Understand \ros{0} entailed by an event in which an agent concludes \(\pv{\phi}{v}\) from \(\Phi\) by~\supportI{} in terms of having followed a rule.
  However, if this entailment holds then the converse entailment, that the agent concluded by following the rules that gave rise to \ros{0} does not hold.
\end{note}

\section{Non-causal}
\label{cha:lit:non-causal}

\subsection*{\textcite{Harman:1973ww}}

\begin{note}
  \begin{quote}
    Reasons may or may not be causes; but explanation by reasons is not causal or deterministic explanation.
    It describes the sequence of considerations that led to belief in a conclusion without supposing that the sequence was determined.%
    \mbox{ }\hfill\mbox{(\citeyear[52]{Harman:1973ww})}
  \end{quote}

  Sequence of considerations provides a \wit{}.
\end{note}

\subsection*{\textcite{Hieronymi:2011aa}}

\begin{note}
  More broadly, though similar.
  \citeauthor{Hieronymi:2011aa}'s (\citeyear{Hieronymi:2011aa}) account of acting for reasons.

  \begin{quote}
    The proposal starts with this simple thought: whenever an agent acts for reasons, the agent, in some sense, takes certain considerations to settle the question of whether so to act, therein intends so to act, and executes that intention in action.

    If this much is uncontroversial (and, under some interpretation, I believe it must be), we can use it as a form for filling out.
    I propose, then, that we explain an event that is an action done for reasons by appealing to the fact that the agent took certain considerations to settle the question of whether to act in some way, therein intended so to act, and successfully executed that intention in action.
    I suggest that \emph{this} complex fact, [\dots] is the reason that rationalizes the action---that explains the action by giving the agent's reason for acting.%
    \mbox{ }\hfill\mbox{(\citeyear[431]{Hieronymi:2011aa})}
  \end{quote}

  So, reason is the complex fact.
  Complex fact gives the reason the agent acted, and so content of constituent considerations from \agpe{}.

  In particular, note here that everything is directed at the question.
  Premise-conclusion relationship.

  As with \citeauthor{Harman:1973ww}, capture the trace, which is given by a \wit{}.
\end{note}

\section{Normative}
\label{cha:lit:normative}

\subsection*{\textcite{Lord:2018aa}}

\begin{note}[Responding to reasons]
  Consider the proposal at the core of \citeauthor{Lord:2018aa}'s (\citeyear{Lord:2018aa}) thesis that being rational is to correctly respond to reasons.

  \begin{quote}
    \textbf{Correctly Responding:} What it is for A's \(\phi\)-ing to be ex post rational is for A to possess sufficient reason S to \(\phi\) and for A's \(\phi\)-ing to be a manifestation of knowledge about how to use S as sufficient reason to \(\phi\).%
    \mbox{}\hfill\mbox{(\citeyear[143]{Lord:2018aa})}
  \end{quote}

  An \agents{} action is rational only if the action is a manifestation of some know-how.
  \citeauthor{Lord:2018aa} summaries:

  \begin{quote}
    [\dots] when one manifests one's know-how, dispositions that are directly sensitive to normative facts are manifesting. Thus, the competences involved in the relevant know-how make one directly sensitive to the normative facts%
    \mbox{}\hfill\mbox{(\citeyear[16]{Lord:2018aa})}
  \end{quote}

  For our purposes, following example of manifesting know-how directly relates to reasoning:

  \begin{quote}
    The most salient disposition [when appealing to \emph{p} as a reason]%
    \footnote{
      Note, \citeauthor{Lord:2018aa} (explicitly) not talking about believing that \emph{p} is a reason, but argues that the cited disposition to present both when appealing to p as a reason and believing that \emph{p} is a reason.
    }
    is the disposition to (competently) use \emph{p} as a premise in reasoning.%
    \mbox{}\hfill\mbox{(\citeyear[25]{Lord:2018aa})}
  \end{quote}

  Hence, suppose an agent concludes.
  Then, if the agent does not \wit{} reasoning from \pool{}, it seems the agent does not manifest know-how, which is required for the appeal to meet \citeauthor{Lord:2018aa}'s account of rational action.

  Of course, that the noted disposition is the most salient does not rule out alternative, less noteworthy, dispositions.
  However, issues is \emph{manifesting} know-how without a \wit{}.
\end{note}

\begin{note}[Illustration]
  Clear that there is no manifestation of understanding of arithmetic.
\end{note}

\begin{note}
  Whether or not argument to be develop is of any difficulty turns on manifesting.
  In various cases, plausible that \ros{1} at issue would arise from the same disposition the agent manifests when the agent concludes.
\end{note}


\section{Miscellaneous}
\label{cha:lit:misc}


\subsection*{Puzzling}

\begin{note}
  A recent account of reasoning given by \cite{Valaris:2014un} (\citeyear{Valaris:2014un}) is separate from \wit{} and embedding.%
  \footnote{
    For simplicity we ignore \citeauthor{Valaris:2014un}'s distinction between basic and non-basic instances of reasoning.
    The excerpts concern non-basic reasoning.
  }

    \begin{quote}
    Suppose that one believes \emph{R} and that \emph{p} follows from \emph{R}.
    What else might it take for one to count as believing \emph{p} by reasoning from \emph{R}?
    The crucial point here is that, if one believes both \emph{R} and that \emph{p} follows from \emph{R}, then --- barring inattention or irrationality --- one thereby believes \emph{p}.
    [\dots]
    In general, the relation between believing that one has conclusive evidence for a proposition and believing that proposition is constitutive, not merely causal.%
    \mbox{ }\hfill\mbox{(\citeyear[110 ]{Valaris:2014un})}
  \end{quote}

  Distinct from \wit{} because only interest is belief.
  Distinct from embedding because constitutive.
  In other words, the agent does not reason from their belief.
  Rather, reasoning is the belief.

  Inclined to understand in terms of \ros{}.
  For our purposes, role of \ros{} is to capture relationship between premises and conclusion.
  However, understood in a particular way, may amount to a belief.

  Still, not so straightforward.
  How does on get the belief that \emph{p} follows from \emph{R}?
  This, to my mind, is what is at issue.
  However, it seems that this is not how things are for \citeauthor{Valaris:2014un}.

  \begin{quote}
    [R]easoning just is believing that one's conclusion follows from one's premisses, and thereby believing one's conclusion.%
    \mbox{ }\hfill\mbox{(\citeyear[112]{Valaris:2014un})}
  \end{quote}

  Unless belief is process, then it seems reasoning understood in this way is instantaneous.

  I am not sure what to make of this.
\end{note}

\subsection*{Basing}
\label{sec:basing}

\begin{note}
  I am not aware of any account of conclusion, or any sufficiently related phenomena, which either explicitly or implicitly denies \issueConstraint{}.
\end{note}

\begin{note}
  Though focus on events, basing relation.

  \citeauthor{Pollock:1999tm} introduce the basing relation with the following observation:
  \begin{quote}
    To be justified in believing something it is not sufficient merely to \emph{have} a good reason for believing it.
    One could have a good reason at one's disposal but never make the connection.
    \dots
    Surely, you are not justified in believing [something], despite the fact that you have impeccable reasons for it at your disposal.
    What is lacking is that you do not believe the conclusion on the basis of those reasons.%
    \mbox{}\hfill\mbox{(\citeyear[35]{Pollock:1999tm})}
  \end{quote}
  The observation falls short of being an account of the basing relation, but the intuition \citeauthor{Pollock:1999tm} appeal to is instructive.
  It seems that an agent must connect reasons and the content of a belief in order for the belief to be formed on the basis of those reasons, and hence be justified by those reasons.

  Still, causal and doxastic accounts of the basing relation parallel theories of reasoning expressed above.

  \citeauthor{Swain:1981wd}'s (\citeyear{Swain:1981wd}) `causal-counterfactual' account of the basing relation is promising in name, but the relevant counterfactuals concern events which happened that would have been a cause if the actual cause of an \agents{} belief had not occurred, and hence require a \wit{0}.

  Further, some care must be taken in order to understand the way in which an account of the basing relation is connected to concluding.
  For example, taken at face value, \citeauthor{Evans:2013tw}'s disposition account of the basing relation intuitively permits failure of \issueConstraint{}:

  \begin{quote}
    S's belief that \emph{p} is based on \emph{m} iff S is disposed to revise her belief that \emph{p} when she loses \emph{m}.%
    \mbox{}\hfill\mbox{(\citeyear[2952]{Evans:2013tw})}
  \end{quote}

  However, \citeauthor{Evans:2013tw}' dispositional theory is designed to capture why an agent sustains a belief, rather than why an agent forms a belief.%
  \footnote{
    See also \textcite{Audi:1986to} for a discussion of cases in which `[b]elieving for a reason does not entail having \textbf{come} to believe for that reason, or for any reason.' (\citeyear[32--33]{Audi:1986to})
  }
  A similar observation extends to \citeauthor{Moretti:2019wx}'s (\citeyear{Moretti:2019wx}) account of basing via enthymematic inferences.
\end{note}


%%% Local Variables:
%%% mode: latex
%%% TeX-master: "master"
%%% TeX-engine: luatex
%%% End:
