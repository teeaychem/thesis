\chapter{Ability}
\label{sec:major-argument}
\label{sec:broad-argum-overv}
\label{sec:all-about-ability}

\begin{itemize}
\item Type of case.
\item \gsi{-}.
  \begin{itemize}
  \item Two parts.
  \item Claiming support for specific ability
  \item Claiming support for result of specific ability.
  \end{itemize}
\item ability entailment.
\item Schematic interpretations of ability: \AR{} and \WR{}.
\item Exhaustive.
\item Relate to \ESU{} and \EAS{}.
\item Return to two parts of \gsi{}.
  \begin{itemize}
  \item \ESU{} requires that when agent reasons to specific ability, agent claims support for a property in line with \AR{}.
  \item However, \nI{} requires that when an agent reasons from specific ability, agent does not claim support from a property, in contrast to \AR{}.
  \end{itemize}
\item \ESU{} requires agent reasons with \AR{}.
\item Reasoning with \AR{} is incompatible with \nI{}.
\end{itemize}

So, the key thing is that we're looking at two different aspects of reasoning with ability.
Reasoning to and reasoning from.
And, it is not possible to give an account of \emph{both} reasoning to and from ability given \ESU{} and \nI{}.
So, the tension isn't, so to speak, `direct'.
Rather, tension arises due to some extended piece of reasoning.

So, one way to resolve tension is to deny extended reasoning.
First part argues that extended reasoning is plausible.


\begin{itemize}
\item Upshot:
  \begin{itemize}
  \item If scenarios, then either \AR{} or \WR{}.
  \item In turn, either not \nI{} or not \ESU{}.
  \item Alternatively, if scenarios then either \ESU{} or \nI{} (the scenarios give rise to this conflict).
  \item In turn, either \AR{} or \WR{}.
  \end{itemize}
\end{itemize}

We've already seen a decent amount of stuff regarding \ESU{} and \EAS{}.
These more or less correspond to \AR{} and \WR{}, kind of.

% \begin{note}[Before turning to the argument\dots]
%   Before turning to the argument, we conclude this introduction with a handful of notes regarding~\ESU{} and~\EAS{}.
% \end{note}

% \begin{note}[Scope of \ESU{}]
%   \ESU{} does not say anything in particular about what the agent may claim support for, only what must be the case in order for an agent to appeal to support for some conclusion on the basis of support for premises.

%   Talking in terms of (support for) premises and conclusions restricts attention to reasoning.
%   There may be broader use of `premise' and `conclusion' where an agent is not required to reason from premise to conclusion in order for the premise to support the conclusion.
%   For example, if visual perception is immediate.
%   Perhaps it may be said that an agent's visual experience is a premise to the conclusion that a dog is sleeping.
%   Still, for present purposes, `conclusion' refers to the output of some process of reasoning performed by an agent which is either actual or potential, and `premises' to the input of that process.

%   Note, also, that in both cases the relation between premises and conclusion is important.
%   If agent does not reason, then neither~\USE{} nor~\ESU{} apply.
%   If there are multiple ways to obtain a conclusion, then~\ESU{} does not require the agent to reason from a particular set of premises.

%   Likewise,~\ESU{} does not require that an agent is required to obtain support for a proposition by valid and subjectively sound reasoning from some premises.

%   Rather,~\ESU{} requires that an agent reason from premises to conclusion in order to establishes support between premises and conclusion
%   By contrast,~\USE{} holds that reasoning is sufficient to establish such a relation.
% \end{note}

% \begin{note}[\ESU{} is intuitive]
%   \ESU{} is intuitive, and is quite common, though not without exceptions.
% (For example, there's views on testimony in which the testifier provides agent access to support the testifier has.
% One may understand this as conflicting with~\ESU{}, or that the fact that these are accessible is the relevant piece of support.)
% \end{note}

% \begin{note}[Alternative]
%   \EAS{} restricts~\ESU{}.
%   This is not to say the agent obtains support equivalent to that which would be obtained were the agent to do, or have done, the reasoning.
%   Nor, that the agent is aware of the relevant premises.

%   Intuitively, \EAS{} states that the agent may appeal to the reasoning they are able to perform in support for the conclusion of that reasoning, and as that reasoning moves from premises to conclusion, it is on the basis of the support for those premises that the agent would identify by reasoning that the agent obtains (some) support for the conclusion.

%   Hence, \EAS{} is in line with the spirit of~\USE{}.
%   For the exception to~\ESU{} is granted by the agent appealing to a witnessing event in which the antecedent (and consequent) of~\USE{} are satisfied.
% \end{note}

% \begin{note}[Ability ensures propositional?]
%   Plausible that if the agent has the ability, then the agent already has propositional support for the relevant proposition.
% \end{note}

\section{Scenarios}
\label{sec:cases-interest}

Our goal is to argue for \EAS{} and against \ESU{}.
At the core of the argument is reasoning about ability.
Specifically, a certain type of scenario in which an agent reason to and from information that they have the ability to witness some specific act.
How the agent reasons with such (specific) ability information in the scenarios of interest will provide a type of counterexample to \ESU{} and in turn an argument for \EAS{}.

In this section we outline two key features of the scenarios we are interested in.
Subsection~\ref{sec:type-scenario} will introduce \gsi{-} to characterise how the agent reasons to the (specific) ability information.
Then, subsection~\ref{sec:ability-entailment} will introduce `\aben{the}' to characterise how the agent reason from the (specific) ability information.
Finally, subsection~\ref{sec:scenarios} will combine \gsi{-} and `\aben{the}' to provide an in-depth understanding of the type of scenarios we are interested in.

\subsection{\Gsi{}}
\label{sec:type-scenario}

\begin{note}[Tension, information]
    \begin{restatable}[\gsi{}]{definition}{defGSI}\label{def:gsi}
    \Gsi{-} is information that:\nolinebreak
    \footnote{
      Strictly speaking the formulation of \gsi{} as a conditional isn't important.
      What matters is that the agent is required to claim support for the general ability in order to claim support for the specific ability.
      For example, the conditional may be reformulated as:
      \begin{enumerate}[label=(\gsi{}\('\)), ref=(\gsi{}\('\))]
      \item Either \emph{S} does not have the general ability to \(\gamma\), or the agent has a specific ability to \(\varsigma\).
      \end{enumerate}
    }
    \begin{quote}
      If \emph{S} has a general ability to \(\gamma\), then \emph{S} has a specific ability to \(\varsigma\).
    \end{quote}
    Where \emph{S} is some agent, \(\gamma\) is some general ability, \(\varsigma\) is some specific ability, and it is either implicitly or explicitly stated that \(\varsigma\) is instance of \(\gamma\).
  \end{restatable}
  
  The following pair of examples are instances of \gsi{}.
  \begin{enumerate}[label=(\gsi{}:\arabic*), ref=(\gsi{}:\arabic*)]
  \item\label{qe:cond} If you have the ability to reason with the rules of chess, then you have the ability to demonstrate that, given the arrangement of the board, there is a sequences of moves that will ensure a win for one of the players (as an instance of the general ability to reason with the rules of chess).
  \end{enumerate}

  \begin{enumerate}[label=(\gsi{}:\arabic*), ref=(\gsi{}:\arabic*), resume]
  \item\label{qe:cond:french} If you have the ability to read French, then you have the ability to read The Count of Monte Cristo without a translation (as an instance of the general ability to read French).
  \end{enumerate}
  In both examples an agent is informed that they have the ability to perform a specific act --- demonstrating a strategy or reading a book --- so long as they have some general ability --- an understanding of chess or French literacy --- because the witnessing the specific ability act would be an instance of witnessing the agent's general ability.

  \gsi{} does not directly provide the agent with the information that they have the specific ability.\nolinebreak
  \footnote{Nor (looking ahead to section~\ref{sec:ability-entailment}) does \gsi{} directly provide the agent with information that the result of witnessing the specific ability is when \aben{the} holds with respect to the specific ability.}
  The agent is not informed that they have the general ability and that therefore they have a specific ability.
  To illustrate, I am confident I have the ability to reason with the rules of chess, and so given \ref{qe:cond} I may be confident that I am able to demonstrate the existence of such a strategy.
  By contrast, I do not have the ability to read French, and so I do not have the ability to read The Count of Monte Cristo without a translation.

  Still, I may also be mistaken.
  It may be that I am overconfident, that I do not have the ability to reason with the rules of chess, and hence it may be the case that I do not have the ability to demonstrate the existence of the relevant chess strategy.
  Likewise, I may have the ability to read French, and may have the ability to read The Count of Monte Cristo without a translation.
  However unlikely this may be, I haven't tried to read French in quite some time.
\end{note}

\begin{note}[Not direct]
  \Gsi{} contrasts with what we term `\dsi{-}' --- information that the agent has some ability.
    \begin{restatable}[\dsi{}]{definition}{defDSI}\label{def:dsi}
    \Dsi{-} is information that:
    \begin{quote}
      \emph{S} has the ability to \(\varsigma\).
    \end{quote}
    Where \emph{S} is some agent and \(\varsigma\) is some specific ability.
  \end{restatable}
  For example, the following is a `direct' recreation of~\ref{qe:cond}:

  \begin{enumerate}[label=(\dsi{}:\arabic*), ref=(\dsi{}:\arabic*), series=dsi_count]
  \item\label{qe:cons} You have the ability to demonstrate that there is a sequences of moves that will ensure a win for one of the players as an instance of your general ability to reason with the rules of chess.
  \end{enumerate}

  If~\ref{qe:cons} is true then the agent has the ability to demonstrate some strategy.
  And, in turn,~\ref{qe:cons} expands on why the agent has the relevant specific ability.
  By contrast,~\ref{qe:cond} may be true even if the agent does not have the ability to demonstrate some strategy.
  Hence, \dsi{} is not in general entailed by \gsi{}.\nolinebreak
  \footnote{
    However, if it is the case that an agent has the general ability mentioned in the antecedent of \gsi{}, then a corresponding instance of \dsi{} will be true.
    Note, this is ensured because the consequent of~\ref{qe:cond} ensures the relevant `instance of' relation obtains.
    % So, if I have the ability to reason with the rules of chess and~\ref{qe:cond} is true with respect to me, then \ref{qe:cons} will also be true with respect to me.
  }
\end{note}

\begin{note}[Important features of \gsi{}]
  \gsi{}, then, has two important features:
  \begin{enumerate}
  \item \gsi{} ensures that the agent is on the hook, so to speak, for claiming support they have the specific ability.
  \item If the agent may claim support for having the relevant general ability, then \gsi{} provides the agent with an account of why they may claim support for having some specific ability.
  \end{enumerate}
  Hence, \gsi{} ensure that an agent must themselves claim support that they have some specific ability while providing the agent with relevant information about why they may claim support for having the specific ability.
\end{note}

\begin{note}[Merging \gsi{} and \dsi{}]
  Finally, though we will focus on \gsi{}, there is a variant that merges \gsi{} and \dsi{} which could be substituted for \gsi{} in further discussion.
  This variant involves informing an agent that they have some general ability, and some specific ability as an instance of that general ability, but requires the agent to identify what the general ability is.

  Here is the variant applied to~\ref{qe:cond}.
  \begin{enumerate}[label=(\gsi{}\(^{'}\):\arabic*), ref=(\gsi{}\(^{'}\):\arabic*)]
  \item
    \begin{enumerate}
    \item You have some general ability \(\gamma\), and a specific ability \(\varsigma\) (as an instance of that general ability). And,
    \item If \(\gamma\) is the ability to reason with the rules of chess, then \(\varsigma\) is the ability to demonstrate that, given the arrangement of the board, there is a sequences of moves that will ensure a win for one of the players (as an instance of the general ability)
    \end{enumerate}
  \end{enumerate}
  The agent remains on the hook, so to speak, for claiming support that they have the relevant specific ability because it is up to the agent to identify the general ability \emph{as} the ability to reason with the rules of chess.
  And, likewise, if the agent may claim support for identifying the general ability in a particular way, then the variant allows the agent to claim support that they have a particular specific ability.

  We favour \gsi{} given it's comparative structural simplicity, but the variant highlights that that the agent claiming support for having some specific ability is not of interest.
  Rather, what is interest is that \gsi{} allows the agent to claim support for the particulars of some specific ability.

  In section~\ref{sec:ability-entailment} we will highlight why the particulars matter.
\end{note}

\subsection{An ability entailment}
\label{sec:ability-entailment}

\begin{note}[\aben{(The)}]
  The second component in scenarios of interest is the availability of an entailment from the specific ability.

  We term an instance of the entailment as an `\aben{}'.

  \begin{restatable}[Ability entailment]{definition}{defAE}\label{def:aben}
    \aben{The} is any entailment of the form:
    \begin{quote}
      \emph{S} has the (specific) ability to \emph{V} that \(\phi\) \emph{therefore} \(\phi\) is the case.
    \end{quote}
    Where \emph{S} is an agent, \emph{V} is some action, and \(\phi\) is some proposition.
  \end{restatable}

  The rough intuition behind instances of \aben{the} is that \(\phi\) being the case does not depend on \emph{S} witnessing the (specific) ability to \emph{V} that \(\phi\).
  So, \aben{the} links ability and something that must be the case in order to have ability and the result of witnessing ability must be the case in order for the agent to have the ability

  For example, \aben{the} holds with respect to the (specific) ability to demonstrate the existence of a chess strategy from \ref{qe:cond} as whether or not a given chess strategy exists depends on the moves permitted by the rules of chess --- a strategy that has not been demonstrated is a strategy.
  Likewise, \emph{S} has the (specific) ability to discover that their keys are in their jacket pocket only if it is the case that their keys are in their jacket pocket --- whether or not \emph{S}'s keys are in their jacket pocket does not depend on \emph{S} discovering that to be the case.

  By contrast, `to read The Count of Monte Cristo without a translation' is an action and so \aben{the} does not apply to the specific ability of~\ref{qe:cond:french}.
  Even so, \aben{the} apply to nearby variants and not others.
  \emph{S} may have the specific ability to read that Dantès was a merchant sailor, and it follows that Dantès was a merchant sailor.
  In contrast, while \emph{S} may have the ability to believe that certain passages cannot be adequately translated, it does not follow that those passages cannot be adequately translated.
  Similarly, \emph{S} may have the ability to hope that they will employ the chess strategy discovered in a competitive game, but it does not follow that \emph{S} will employ the strategy.

  More broadly, \aben{the} holds with respect to factive verbs, such as `see', `know', `understand', and so on.
  Though, I doubt factive verbs are an adequate explanation for \aben{the}.
  Consider `read'.
  I have the ability to read that Elvis Presley was born in 1935, but I also have the ability to read that Elvis is working undercover for the DEA.
  What matters, then, is not the verb used, but how the agent would witness the relevant ability.
  I have the ability to read that Elvis was born in 1935 from a reliable source, and hence \aben{the} applies.
  The same is not true for my ability to read that Elvis is working for the DEA.

  Indeed, \aben{the} merely identifies an entailment.
  It does not provide an account of when or why such entailments hold.
  We identify entailments of this type because our interest is in how (in certain cases) agent's reason with instances of \aben{the}.
\end{note}

\subsection{Details of scenarios}
\label{sec:scenarios}

\begin{note}[Both things are important]
  The scenarios we are interested in combine \gsi{} with \aben{the}.

  The role of \gsi{} is to ensure that the agent is not provided with direct information about specific ability.
  And the role of \aben{the} is to highlight that the agent is in a position to claim support for some further proposition if they claim support for specific ability.
  Hence, scenarios combine claiming support \emph{for} specific ability and claiming support \emph{from} specific ability.

  To illustrate, consider the following pattern of reasoning:
  \begin{enumerate}[label=\arabic*., ref=(\arabic*)]
  \item\label{scen:rp:1} I have the general ability to reason with the rules of chess.
  \item\label{scen:rp:2} I received \gsi{} information that if they have the general ability to reason with the rules of chess then they have the ability to demonstrate the existence of some strategy.
  \item\label{scen:rp:3} So, from~\ref{scen:rp:1} and~\ref{scen:rp:2} it follows that I have the ability to demonstrate the existence of some strategy.
  \item\label{scen:rp:4} And, as \aben{the} hold with respect to~\ref{scen:rp:3}, the relevant strategy exists.
  \end{enumerate}
  I reason to (\ref{scen:rp:1} --- \ref{scen:rp:3}) and from (\ref{scen:rp:3} --- \ref{scen:rp:4}) a specific ability.
  The reasoning pattern seems sound.
  And, at no point do I need to witness their general ability to reason with the rules of chess, or the specific application of the general ability to demonstrate the existence of the strategy.
\end{note}

\begin{note}
  Both components are important.
  Focus on \gsi{} will restrict the interpretation of what the agent claims support for.
  And, in turn, what the agent has claimed support for will determine what the agent appeals to when appealing to \aben{the} entailment.\nolinebreak
  \footnote{
    I suspect it may be possible to focus only on \gsi{}.
    As we will see, this is where the key step of the argument takes place.
    However, this is not trivial.
    Would require more focus on how the agent gets to specific from general.
    By splitting in this way, we avoid details.
    Instead, focus on what it is that the agent gets, and then \aben{the} is forced to work with this.
  }
  \gsi{} and \aben{the} combine to provide a (partial) functional characterisation of reasoning with specific ability.
\end{note}

\begin{note}
  Note, however, that there is a distinction between how an agent reasons about ability, and what ability is.
  We are interested in how agent's reason about (specific) ability, and not what makes it true that an agent has a (specific) ability.
  Our focus will shortly turn to how to interpret (specific) ability when appealed to in the type of scenario described.
  We will outline two general schematic interpretations of ability, argue that these are exhaustive, and note how general constraints such as \ESU{} constrain which interpretation is available.
\end{note}

\begin{note}[Scenario proposition]
  For ease of reference, we wrap scenarios involving the limited information as a proposition.
    \begin{restatable}[\eA{-} --- \eA{}]{proposition}{propScenariosExist}\label{prop:SE}
    There are scenarios in which an agent \emph{S} receives \gsi{} information of the form:
    % \mbox{ }\vspace{5pt}
    \begin{center}
      If \emph{S} has a general ability to \(\gamma\), then \emph{S} has a specific ability to \emph{V} that \(\phi\).
    \end{center}
    % \mbox{ }\vspace{5pt}

    \noindent Such that \aben{the} applies to the specific ability to \emph{V} that \(\phi\).

    In turn:
    \begin{enumerate}
    \item \emph{S} may reason from claimed support that they have the general ability to \(\gamma\) in order to claim support for having the specific ability to \emph{V} that \(\phi\). And,
    \item \emph{S} may reason from their claimed support that they have the ability to \emph{V} that \(\phi\) to claim support that \(\phi\) is the case by appealing to \aben{the}.
    \end{enumerate}
    \vspace{-\baselineskip}
  \end{restatable}
\end{note}

\begin{note}[Possible restrictions]
  First, \eA{} holds only that there are cases in which the agent may appeal to ability to obtain support.
  It is therefore consistent with~\eA{} that there are cases in which the details of the cases outlined are satisfied, but where kind of support is unsuitable for certain purposes.
  For example, some witness of ability may be demanded by a third-party.
  In this respect, the content of \eA{} is similar to an analogous claim with respect to memory.
  If an agent remembers proving that \(\phi\), then \(\phi\) is the case.
  Still, one may still request that an agent provides you with a proof of \(\phi\) in order to for you to be satisfied that \(\phi\) is the case --- many exams are like this.
  So, that an agent may not always and in any context claim support for \(\phi\) from claimed support for their ability to \emph{V} that \(\phi\) is not an objection to~\eA{}.
\end{note}

\begin{note}
  Second, \eA{} does not require that an agent reason in the way described given \gsi{} and availability of \aben{the}.

  For example, the following statement is an instance of \gsi{}:
  \begin{enumerate}
  \item Any person who has the (general) ability to reason with the rules of chess has the (specific) ability to identify Alekhine's Defense as a fine opening move.
  \end{enumerate}
  The universal quantifier implies that the statement is true with respect to me, among others.
  Still, I am confident that there is at least one other person who has the ability to reason with the rules of chess, and may therefore infer that Alekhine's Defense as a fine opening move without appealing to my own ability.
  Indeed, if I am inclined to doubt my own (general) ability in contrast to a Grandmaster, then I may be more confident that Alekhine's Defense as a fine opening move if I appeal to the existence of a Grandmaster.

  Again, it is consistent with \eA{} that an agent may reason in such a way.
  Still, in defence of \eA{} it is important to note that \gsi{} information may be limited to the agent in question.
  For example, I may have studied your notes on how to play chess and identified a strategy which follows from those notes.
  I have no doubt that you have the ability to identify the same strategy, so when I provide \gsi{} my emphasis is on whether you have the ability to reason with \emph{chess}, rather than some closely related game.

  There are many ways to build context so that an agents is required to reason with \gsi{} and \aben{the} if the agent is to reason with (specific) ability at all, but I doubt these are required.
  The reasoning described by \eA{} (and illustrated above) seems plain and permissible.
\end{note}

\begin{note}
  Finally, \gsi{} and \aben{the} are constraints which do not hold in all cases of reasoning with specific ability.

  For example, one may be told that a gift of a metal detector grants the ability to discover if there is buried treasure in the garden.
  The former does not entail that there is buried treasure in the garden, and testimony or the metal detector may be claimed as support for the ability.

  So, question about whether this really does anything for general understanding of ability.
  \gsi{} and \aben{the} combine to require a particular interpretation.
  However, interpretation with general applicability is not restricted to instances in which it is forced.
  The role of a counterexample is not (typically) to establish that every instance of a theory is mistaken, but to identify a gap.
  And, even if the original theory may be restricted to non-problematic cases, the alternative theory may compete with the original theory.
  So, given that the particular interpretation is required to hold given additional stipulations, interest is in whether it holds without additional stipulations.
\end{note}

\subsection{Reasoning in the scenarios}
\label{sec:reasoning-scenarios}

\begin{note}
  {
    \color{red}
    Two different ways of reasoning here.
    Either from specific to \(\phi\) or from general to \(\phi\).
  }
\end{note}

\begin{note}
  With general ability, this is a `derived ability entailment'.
\end{note}

\section{Two (schematic) interpretations of (specific) ability}
\label{sec:wr-ar}

\begin{note}
  In the previous section we introduced \gsi{-} and \aben{the}.
  In the present section we motivate two interpretations of (specific) ability in the context of reasoning to (specific) ability from \gsi{} and reasoning from (specific) ability with \aben{the}.

  The two interpretations are termed `\AR{}' and `\WR{}' in turn, and are schematic.
  Roughly:
  \AR{} holds that when appealing to (specific) ability an agent appeals to some property or attribute that they have.
  And, by contrast, \WR{} holds that when appealing to (specific) ability an agent appeals to the action that they would perform by witnessing the relevant ability.
  \AR{} and \WR{} are distinguished, then, by whether an agent reasons with a property (\AR{}) or an event (\WR{}).

  To illustrate by analogy, consider a mechanical clock.
  The clock has the property of displaying the correct time, by it is also involved in the event of changing it's configuration as time passes.
  The property that the clock is displaying the correct time is important for determining whether one is late for a meeting.
  By contrast, the event of changing it's configuration as time passes is important for determining when to remove a brewing teabag.
  A meeting starts at a certain point in time, while tea is brewed over a period of time.
  If the clock does not represent the correct time, then three minutes passing will not, in general, help determine whether one is late to the meeting.
  And, whether or not it is 3pm is not, in general, important with respect to whether or not the tea has finished brewing.
  The qualifier `in general' is important.
  Measuring the passage of is useful if I know the length of time before the meeting is due, and the correct time is useful if I know when I started brewing the tea.

  The distinction between \AR{} and \WR{} is similar.
  Both interpretations may be more or less useful in certain circumstances, and interchangeable in others.
  Still, the combination of \gsi{} and \aben{the} identify a pattern of reasoning in which we may elaborate how the relevant interpretation of (specific) ability is important, and in turn broader principles (\ESU{} and, to be introduced below, \nI{}) will constrain whether the interpretations are permissible.

  {
    \color{red}
    Outline of subsections.
  }
\end{note}

\begin{note}[Table]
  \begin{figure}[H]
    \centering
    \begin{tblr}{abovesep=8pt, belowsep=8pt, width=0.95\textwidth, colspec={Q[c,m]|Q[c,m]|Q[1.8,c,m]|Q[1.8,c,m]}}
      \multicolumn{2}{c}{} & \adA{} & \adB{} \\
      \hline
      \multicolumn{2}{c}{\WR{}} & ? & ? \\
      \hline
      \multirow{2}{*}{\AR{}} & Basic & ? & ? \\
      \cline[dashed]{2-4}
      & Derived & ?  & ? \\
    \end{tblr}
    \caption{Distinction matrix for interpretations of \aben{the}. \\ Rows are interpretations of ability, columns are type of reasoning regarding ability.}
  \end{figure}
\end{note}


\section{\AR{} and \WR{}}
\label{sec:ar-wr-1}

\begin{note}[\WR{} and \AR{}]
  We term the two schematic interpretations of \aben{the} `\AR{}' and `\WR{}', respectively.
  Brief descriptions from detached perspective.
  Given that the interpretations are schematic, they fall short of a full account of how an agent claims support by \aben{an}.
  However, the arguments to follow are of interest in part because they concern any way in which the schematic interpretations are filled out.
\end{note}

{
  \color{red}
  I should emphasise that here we're interested in reasoning.

  Also, the distinction is important to ensure that the argument's don't depend on a specific reading of ability.
}

\subsection{\AR{}}
\label{sec:ar-1}

\begin{note}
  \begin{restatable}[\AR{}]{definition}{defAttribution}\label{AR:def}
    An agent's reasoning with an instance of \aben{the} by claiming support for \(\phi\) from \emph{S} having ability to \emph{V} that \(\phi\) is an instance of \emph{\AR{}} when the agent holds that:

    \emph{S} has the ability to \emph{V} that \(\phi\)
    \begin{enumerate*}[label=(\textsf{A}\arabic*), ref=(\textsf{A}\arabic*)]
    \item\label{A:s:1} is or reduces to some (possibly complex) property \emph{P} of \emph{S}, and
    \item\label{A:s:2} \emph{P}, or some part of \emph{P}, entails that \(\phi\) is the case.\nolinebreak
      \footnote{Intuitively, because the agent could not have \emph{P} without \(\phi\) already being the case.
      The notion of entailment here does not require that \(\phi\) is true \emph{because} of \emph{P}.}
    \end{enumerate*}
  \end{restatable}
  
  {
    \color{red}
    \AR{} identifies instances of reasoning in which an agent applies \aben{the} by holding the ability to \emph{V} that \(\phi\) is a property of an agent.\nolinebreak
    \footnote{
      Note, this does not say anything about what the ability to \(\phi\) is.
      Rather, way in which the agent claims support.
    }
    Note, when appealing to \aben{the} an agent need not be aware of what the (potentially complex) property of \emph{S} is.
    Rather, claimed support that \emph{S} has the ability to \emph{V} that \(\phi\) allows the agent to claim support for the existence of some property of \emph{S} which in turn entails \(\phi\).
  }

  Now, generally speaking properties are things which may be predicated or attributed of other things.
  The coffee is hot, I am thirsty, my mouth is sensitive to heat, I am reckless, I am in pain, and so on\dots
  And, properties come cheap.
  For example, the participation of an agent in some event gives rise to a property that may be attributed to the agent.
  Specifically, the property of participating in the event.
  Moments ago I participated in the event of recklessly drinking hot coffee with a mouth that is sensitive to heat.
  Therefore, I have the property of participating in such an event.

  So,~\ref{A:s:1} is trivially true.
  When we speak of an agent having some ability we are predicating or attributing ability to an agent.
  However,~\ref{A:s:2} requires that the property entails that \(\phi\) is the case.
  And, it is not clear that an entailment which follows from an event is always reflected in the property of being a participant in the event.
  For example, it seems that I am in pain because I participated in the event of drinking hot coffee, \emph{not} because I have the property of having participated in the event of drinking hot coffee.
  By contrast, that I have the property of having participated in the event of drinking hot coffee entails that I have the property of having participated in the event of drinking something.

  % From~\ref{A:s:2} it must be the case that the relevant property entails \(\phi\).
  % And, from~\ref{A:s:3} the property must not analysed in terms of there being a potential event in which \emph{S} witnesses the act of \emph{V}ing that \(\phi\).
  % This is, from one perspective, an arbitrary restriction.
  % For example, if there is a potential event in which an agent witnesses the act of \emph{V}ing that \(\phi\), then the agent has the property of being a participant of that potential event.
  % From a different perspective,~\ref{A:s:3}

  Roughly, we may expect the property of interest is akin to having a heart, possessing ¥500, being of a certain age, and so on\dots

  {
    \color{red}
    Key idea with \AR{} is that the agent `directly' claims support for a property when using \aben{the}.
  }

  To illustrate \AR{} we focus on the idea of reducing the ability to \emph{V} that \(\phi\) to some (potentially complex) property of \emph{S}.
  Again, when appealing to \aben{the} an agent need not be aware of what the (potentially complex) property of \emph{S} is.
  Rather, these illustrations suggest that such properties exist.

  \begin{illustration}
    Consider the proposition that \emph{S} has the ability to hear that the birds are signing.
    Again, it seems \aben{the} holds, and one may infer that birds are singing.

    So, by \AR{} there is some (complex) property \emph{P} of \emph{S} such that \emph{P}, or some part of \emph{P}, entails the the birds are signing.

    Consider the complex property of a well-functioning auditory system and sufficient proximity to the birds singing.
    The property of having well-functioning auditory system ensures that \emph{S} has the ability to hear nearby noises.
    And, having well-functioning auditory system together sufficient proximity to the birds singing together ensure that \emph{S} has the ability to hear the nearby noise of the birds singing.

    \aben{the} follows from part of this complex property.
    If the agent has the property of being in sufficient proximity to the birds singing, then it follows that there are birds singing.
  \end{illustration}

  \begin{illustration}
    Consider the proposition that the prosecution has the ability to demonstrate that the defendant is guilty.
    Intuitively, \aben{the} holds, as it is not possible to demonstrate the guilt of an innocent person.\nolinebreak
    \footnote{
      It is a different matter to convince a jury of the guilt of an innocent person.
      And, \aben{the} does not seem to hold with respect to the ability to convince a jury that the defendant is guilty.
    }
    By \AR{} there is some (complex) property \emph{P} of the lawyer such that \emph{P}, or some part of \emph{P}, entails the guilt of the defendant.
    Say, the lawyer is in possession of evidence sufficient to establish guilt of the defendant.
    If so, it is a property of the lawyer that they are in possession of such evidence, and by assumption the evidence entails that the defendant is guilty.

    It seems possession of evidence alone may not be sufficient to establish that the lawyer has the ability to prove that the defendant is guilty.
    For, it is plausible that a lawyer may be in possession of evidence that they do not understand.
    However, as our interest is with \aben{the} it is sufficient to observe that the evidence alone entails the guilt of the defendant.
  \end{illustration}

  Again, these illustrations highlight ways in which \emph{S} having the ability to \emph{V} that \(\phi\) may be reduced to some (complex) property of \emph{S}.
  \AR{} does not hold that an agent identifies such a property when claimed support by an instance of \aben{the}.
  Rather, \AR{} holds that the agent reasons with ability as a property of the agent.
  Indeed, while these suggestions reduce ability to complex properties, \AR{} also admits of the possibility that the ability to \emph{V} that \(\phi\) is a basic property which does not admit of further analysis.
  If so, then it seems that \aben{the} must also be taken as basic.\nolinebreak
  \footnote{
    I lack any suggestion for how to understand \AR{} if the property is indeed basic, but there is no need to rule out this option ---  no part of the following arguments depend on whether or how these schemas may be substantiated.
  }
  So, to summarise.
  The distinguishing feature of \AR{} is that there are instances when an agent claims support for \(\phi\) from claimed support that \emph{S} has the ability to \emph{V} that \(\phi\) because the latter ensures that there is some property \emph{P} holds of \emph{S} and \emph{P} entails \(\phi\).
  If the agent has the ability to \emph{V} that \(\phi\), then there may also be some action, \emph{V}ing, that the agent may witness.
  However, as \AR{} appeals to some property, the witnessing event is irrelevant to the way in which the agent claims support for \(\phi\).
\end{note}

\begin{note}
  {
    \color{red}
    Some additional notes on \AR{} that haven't been merged with the above follow.
  }
\end{note}

\begin{note}[Compatibility]
  \AR{} suggests an alternative way to obtain support for the conclusion of reasoning the agent is able to do.
  Specifically, if order for the agent to \emph{have} the attribute of being able to reason to the conclusion, the conclusion of the reasoning must be true.
  The relevant entailment is in part secured by the verb chosen, and in part by what the verb is applied to.
  Here, `demonstrate' is a factive verb, if an agent demonstrates that \(\phi\), then it is true that \(\phi\).
  And, the existence of a chess strategy does not depend on the agent demonstrating that the relevant strategy exists.

  To take another example, you only have the ability to identify a typo on this page if there is a typo on this page.
  So, if I were to provide you with testimony that you have the ability to identify a typo on this page, you may begin searching for the typo, or you may note that there must be a typo in order for me to be in a position to provide you with testimony that you have the ability.
\end{note}

\begin{note}[Sketch of \AR{}]
  \begin{enumerate}[label=(\textsf{A}\arabic*), ref=(\textsf{A}\arabic*)]
  \item\label{AR:Sketch:1} I have the attribute of being able to \emph{V} that \(\phi\).
  \item\label{AR:Sketch:2} In order to have the attribute of being able to \emph{V} that \(\phi\), \(\phi\) must be the case independent of whether or not I witness the ability.
  \item\label{AR:Sketch:3} \(\phi\) is the case.
  \end{enumerate}

  To keep things simple, we will refer to the principle behind the pattern sketched as \AR{}.
  And agent may bundle~\ref{AR:Sketch:1} and~\ref{AR:Sketch:3} into a conditional, and avoid instantiating the reasoning pattern, but so long as the conditional is (implicitly) held on the basis of the intermediate premise~\ref{AR:Sketch:2}, we take use of such a conditional to be an instance of \AR{}.
\end{note}


\subsection{\WR{}}
\label{sec:wr-1}

\begin{note}[\WR{} def.]
  {
    \color{red}
    Include: observation that the entailment may come from some property of the agent.
    The point of \WR{} is that the agent claims support for details of the event.
  }

  We now turn to \WR{}.
  \begin{restatable}[\WR{}]{definition}{defWitnessing}\label{WR:def}
        An agent's reasoning with an instance of \aben{the} by claiming support for \(\phi\) from \emph{S} having ability to \emph{V} that \(\phi\) is an instance of \emph{\WR{}} when the agent holds that:
    \begin{enumerate}
    \item\label{WR:def:1} \emph{S} has the ability to \emph{V} that \(\phi\) \emph{if and only if} there is a potential event in which \emph{S} witnesses the act of \emph{V}ing that \(\phi\).
    \item\label{WR:def:2} Claim support for event or details of event.
    \item\label{WR:def:3} Details of the event in which \emph{S} witnesses the act of \emph{V}ing that \(\phi\), or part of the event, entails that \(\phi\) is the case.\nolinebreak
      \footnote{Again, intuitively, because there could not be a potential event in which \emph{S} witnesses the act of \emph{V}ing that \(\phi\) without \(\phi\) already being the case.
      The notion of entailment here does not require that \(\phi\) is true \emph{because} there is some potential event of the relevant kind.}
    \end{enumerate}
  \end{restatable}

  {
    \color{red}
    ~\textcite{Rebuschi:2011ub} talk about \emph{de objecto} attitudes.
    This might be helpful given that the events are potential.
  }

  {
    \color{red}
    Key idea with \WR{} is that the agent appeals to certain things which follow from the event being witnessed.
    Whereas, \AR{} appeals to certain things which must be the case in order for the event to be witnessed.
  }

  {
    \color{red}
    Difference between the existence of an event (~\ref{WR:def:1}) and details of the event (~\ref{WR:def:2}).
    To clarify.
    \(\exists e(V(e) \land \text{agent} = \emph{S} \dots)\).
    \(\phi\) follows.
    However, there are two ways to think about this.
    First, the existential, second the event.
    \emph{De dicto} and \emph{de re}.
    \WR{} is \emph{de re}.

    Consider existential of individuals.
  }

  {
    \color{green}
    Before going into the details, it'll be helpful to highlight the big idea, especially with respect to how things (will) work out with the `master property' from \AR{}.
  }

  \WR{} identifies instances of reasoning in which an agent applies \aben{the} by holding that \emph{S} having the ability to \emph{V} that \(\phi\) ensures there is a possible event in which \emph{S} \emph{V}s that \(\phi\).
  And, in turn, there is a possible event in which \emph{S} \emph{V}s that \(\phi\) entails that \(\phi\) is the case.
  In contrast to \AR{}, when an agent claims support as an instance of \WR{} an agent reasons about what must be the case in order for \emph{S} to witness some ability, rather than what must be the case in order for \emph{S} to have the property of possessing some ability.


  We use the term `potential' in place of `possible' when describing the relevant event to highlight that the existence of the event is tied to an ability attribution.
  One may hold that a possible event is any event which is not impossible, and hence it is possible for an arbitrary agent to prove Fermat's Last Theorem.
  Yet, it seems most agent's lack the ability to prove Fermat's Last Theorem, and so `potential' serves to restrict quantifier over events which an agent has the ability to witness --- however the details of that quantification are resolved.

  {
    \color{red}
    \WR{} is more complex than \AR{}.
    There is some action that \emph{S} may witness.
    And, understand what the result of that action is.
    So, we have something akin to a counterfactual.
    However, the counterfactual only relies on witnessing.
    Further, particular status of \(\phi\).
    Hence, as witnessing is the only issue, \(\phi\) is the case.

    Third, regardless.
    \(\phi\) holds regardless, but it does not follow from this that if the agent reasons via \WR{} then support claimed for \(\phi\) would be independent of ability information.
    The agent must recognise that \(\phi\) must be the case regardless, but this doesn't require that the agent has any way of reasoning to \(\phi\) other than by witnessing their ability.
    The point is clearer when considering witnessed instances of reasoning.
    \emph{X} testified that \emph{p}.
    Claim support for \emph{p}.
    \emph{p} is not the case because \emph{X} testified that \emph{p}, though my only path to claim support is by appeal to the testimony of \emph{X}.
  }
  To illustrate.

  \begin{illustration}
    I have the ability to calculate that \(243 \div 3 = 82\).
    Pen and paper to hand, etc.\
    Result of this will be a calculation that \(243 \div 3 = 82\).
    However, my calculation is irrelevant to whether it is the case that \(243 \div 3 = 82\).
    Hence, it follows that \(243 \div 3 = 82\).
  \end{illustration}

  \begin{illustration}
    Ability to discover that the ball is under the left cup.
    Raise the left cup, and identify the ball.
    Whether or not the ball is under the left cup is independent of this sequence of actions, and therefore it follows that the ball is under the left cup.
  \end{illustration}

  Compare to cases where only gets the counterfactual.
  I have the ability to make it so that the heating is turned out.
  Plausibly, the heating is not on, and depends on witnessing the action of `making it so'.
\end{note}

\begin{note}[`Available resources']
  Delicate.
  Focus is on the witnessing event.
  However, mere possibility isn't sufficient for \aben{the}.
  So, some restriction.
  That is, an account of what makes the witnessing event a \emph{potential} event rather than a \emph{possible} event.
  One way to express this idea is that included in appeal to potential witnessing event is that sufficient resources are available.
  Here, the idea is that nothing further is required for the event to take place.

  This redescription falls short of an analysis as we've shifted the work from `potential' to `available'.
  Still, room for an analogy.
  Consider running a 5K.
  Here, going to require a whole bunch of energy.
  The agent does not `have' the energy.
  However, resources to generate energy.
  Fat reserves, muscle density, and so on.
  In this sense, sufficient resources are available, but not something the agent has.

  \AR{}, whatever it is that generates the sufficient resources.
  \WR{}, the result of having generated the sufficient resources.

  So, the difference between \AR{} and \WR{} isn't necessarily with what the two interpretations reduce down to, but is rather a difference with respect to what the interpretations focus on.
  From \AR{}, the stuff that's true right now, the generator, does the work.
  From \WR{}, it's what will be generated.

  There's still an important difference, though.
  Our interest is in reasoning.
  We are interested in what the agent appeals to.

  Key difference.
  \AR{}, that there is stuff the agent has which will generate.
  \WR{}, that what is generated from the stuff the agent has will do the work.

  The impact of this distinction will be expanded up with respect to \gsi{}.
\end{note}


\subsection{Contrasting \AR{} and \WR{}}
\label{sec:contrasting-ar-wr}

\begin{note}[Difference]
  With \AR{} the important thing is some (possibly complex) property.
  With \WR{} the important thing is the witness.

  \[\text{Has}(S,\text{docs}) \land \text{Sufficient-to-show-guilt-of-defender}(\text{docs})\]

  Ability here is to ensure that there is some property of this kind.

  \[\exists e(\text{Calculating}(e) \land \text{agent}(e) = S \land \text{result}(e) = (243 \div 3 = 8))\]


  \AR{} focuses on whether something is true of the agent independent of what action they may perform.
  \(\phi\) follows from property.
  \WR{} focuses on an action the agent may perform.
  \(\phi\) follows from relation between \(\phi\) and possible witness of action.
\end{note}

\begin{note}
  Given \AR{}, conjecture that ability is not important for the entailment if complex.
  A useful shorthand, but in principle do not need to highlight the act.
  In contrast, the act is required for \WR{}.

  Still, agent is only given information about ability, so this will remain important for reasoning.
\end{note}

\begin{note}[Plausible equivalence]
  Generally speaking, switch between the two.
  Ball under cup.
  Hear that the birds are singing.
\end{note}

\begin{note}
  Primary purpose of the distinction is to ensure that things apply to any understanding of reasoning with \aben{the}.
  Further contrast after following distinction.
\end{note}

%%% Local Variables:
%%% mode: latex
%%% TeX-master: "master"
%%% End: