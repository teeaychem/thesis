\chapter{Claiming support, `use', and two conflicting ideas}
\label{cha:claiming-support-use}

\begin{note}
  Claiming support, seen in previous chapter.
  Basic requirement with respect to `defeaters'.
  Now turn to an intuitive sufficient idea, granting necessary.
  Then, state an idea of a necessary condition.
  And, conflicting sufficient condition.

  Arguing against the idea for a necessary is the rest of this paper.
\end{note}


\section{Claimed support and `use'}
\label{sec:claimed-support-use}

\begin{note}[Understanding of claiming support]
  Understanding of claiming support.

  Begin with a sufficient condition.
  In short, most instances of reasoning.
  Claiming support is common.

  Then, two types of defeaters.
  Mistaken and misled.
  Use to form a necessary condition.
  If claimed support, then agent deems that claimed support is not defeated.
  `Deem' is a placeholder.
  Strong or weak.
  Single constraint is that when claiming support, potential defeaters that aren't ruled out.\nolinebreak
  \footnote{
    At least two ways of viewing this.
    First, claiming support is restricted.
    Second, \emph{claiming} support only applies when there are potential defeaters, and some other relation to support when possible defeaters get ruled out.

    These are different, but I don't think the difference matter for resource bound agents of interest.
    Lack of resources is that always potential defeater, even if every possible defeater may be ruled out.
    }
  Finally, property, that claimed support does not depend on whether proposition the agent has claimed support for is true, or whether the claimed support \emph{does} support (if these are separated).
  Property will be important.
\end{note}

\begin{note}[Sufficient condition]
  We start with a sufficient condition for claiming support:
  \begin{restatable}[\USE{-} --- \USE{}]{idea}{ideaUSE}\label{prem:bP}\label{prop:USE}
    If an agent may claim support for premises and steps of reasoning, accesses those premises and traces claim to support through those steps of reasoning, then agent may claim support for conclusion on basis of the claimed support for the steps and premises of reasoning.
    (Given that the agent deems that the claims to support for premises and steps used are undefeated when drawing conclusion.)
  \end{restatable}

  \USE{} expresses an intuitive idea.

    In short, an agent may claim support when reasoning goes well.
  And, reasoning goes well when there are premises and steps of reasoning available to the agent, and the agent draws on these to claim support for the conclusion.
  The parenthetical remark is a simple safeguard, the agent does not lose a claim to support for the premises or steps in the process of reasoning.\nolinebreak
  \footnote{
    May think that this is the wrong safeguard.
    Consider the liar paradox:
    `This sentence is false.'
    \USE{} prevents agent from claiming support that the sentence is true or false.
    However, may think that agent is in a position to claim support that the sentence is both true and false.
    Indeed, standard reasoning associated with the liar suggests that the sentence is both true and false.
    Still, it's not obvious from demonstrating that the sentence is both true and false that one may claim support for the sentence being true and the sentence being false.
    That is, one may confine the paradox to the truth value of the sentence, rather than (associated) surplus of support.
  }

  The purpose of taking~\USE{} as basic is to fix a basic intuition regarding claiming support.

  Strictly speaking, we do not assume \USE{}.
  Main argument of the thesis is about what is sufficient for claiming support.
  {
    \color{red}
    \USE{} doesn't help me at any point.
    It almost helps, in the sense of that it provides a guarantee that agent would claim support by witnessing.
    However, it doesn't really help, as it's really not that clear.
    \USE{} talks about reasoning done, but you need additional assumptions that this remains true under a counterfactual.
  }
\end{note}

\begin{note}[Illustration of \USE{}]
  An illustration:

  \begin{illustration}\label{ill:rectangle:basic}
    Suppose an agent measures that the rectangle in front of them has the dimensions of \(19\text{cm}\) by \(7\text{cm}\).
    The agent understands how to calculate the area of a rectangle, and by performing some reasoning comes to hold that the area of the rectangle is \(133\text{cm}^{2}\).
    The support the agent has for holding that the area of the rectangle is \(133\text{cm}^{2}\) is obtained (at least in part) on the measurement of rectangle, understanding how to calculate the area of rectangle, and some grasp of mathematics.
  \end{illustration}

  Whether some (or all) of the required arithmetic is to be included as a premise or a step of reasoning may be set aside.
  Similarly, set aside whether further arguments, or whether some premises and steps are taken as basic.
  For example, perhaps some the agent requires some further claim to support for using the ruler to measure the rectangle such as comparison to a standard, or perhaps the agent's claim to support terminates by noting that their use of the ruler is a reliable process.
\end{note}

\begin{note}
  \color{red}
  Problem is, \USE{} does not tell us much about what claimed support is.
\end{note}

\section{Two (conflicting) ideas regarding claimed support}
\label{sec:inter-with-claim}

\begin{note}
  In this section we introduce two propositions which characterise what we are arguing against and what we are arguing for.
  \ESU{-} and \EAS{-}, respectively.
  Argue against \ESU{} by cases involving ability.
  Argue for \EAS{} which outlines the way in which ability conflicts with \ESU{}.

  Start with introduction of \ESU{}.
  And, motivate with reference to literature on the basing relation and rationality as responding to reasons.

  Move to \EAS{}, clarify relation to \ESU{} and contrast to related principle argued for by \citeauthor{Moretti:2019wx}.
\end{note}

\subsection{\ESU{} --- \ESU{-}}
\label{sec:esu}

\begin{note}[Recap of \USE{}]
  Brief recap of \USE{}.
  Introduced the idea, and then expanded on the details.
\end{note}

\begin{note}[Focus]
  We will argue against the converse of~\USE{}:

  \targetESU*

  \ESU{}, as the converse of~\USE{} focuses on reasoning.
  {
    \color{red}
    Assumption here is that there's no other way to claim support.
  }
  To clarify,~\gESU{} is a generalisation of~\ESU{}.


  \begin{restatable}[\gESU{}]{target}{targeGESU}\label{denied-claim}
    An agent may claim support for some proposition \(\phi\) by appealing to some materia\nolinebreak
    \footnote{Latin.
      Material, matter, basis, information, foundation, ground, etc.
    }
    \emph{M} only if the agent uses \emph{M} in the reasoning which culminates in claiming support for \(\phi\).
  \end{restatable}
  Our focus is with whether an agent is required to have \emph{used} something in order to appeal to that thing when claiming support.
  No fixed understanding of `use' is assumed in the statement of~\ESU{} and~\gESU{}, and we will offer some disambiguation below.
  First, a basic illustration.
\end{note}

\begin{note}
  Three brief notes on~\ESU{}:

      First, the `has' in~\ESU{} only requires `at some point in the past'.
      Hence,~\ESU{} does not require the agent to reason from premises to conclusion each time the agent claims support for the conclusion.
      For example, if an agent proved the Deduction Theorem for propositional logic last week, then the agent would not be in conflict with~\ESU{} if they claimed support for the Deduction Theorem on the basis of the premises and reasoning they performed in the past.

      Second, and following from the first,~\ESU{} will also hold for any stronger statement --- for example if `has' is read as `has just'.
      For example, requiring that the agent's memory of proving the Deduction Theorem allows the agent to claim support, rather than the premises and steps used in the past.
      The argument (stated below) denies that, given certain information, the agent needs witnesses any reasoning in order to claim support for the result of witnessing the reasoning.

      Third, as~\ESU{} is about when an agent may \emph{claim} support, it is compatible with~\ESU{} to hold that the agent \emph{has} support --- regardless of whether the agent has witnessing the reasoning.
\end{note}

\begin{note}[Simplest]
  \color{red}
  Difference between \(\phi\) therefore \(\psi\) and \(\phi\) and \(\phi \rightarrow \psi\) therefore \(\psi\).
  Possible for agent to reason from \(\phi\) to \(\psi\), so in principle possible for agent to claim support for \(\psi\) from \(\phi\).
  However, \ESU{} denies this if the agent doesn't do the reasoning.
  Instead, agent also needs \(\phi \rightarrow \psi\), and then they're fine.

  Key point of the suggested revision is that \ESU{} doesn't need to focus on claiming support for \(\phi\), specifically.
  Rather, it's just about establishing a relation of support between \(\phi\) and \(\psi\).

  Then, big idea is that ability is not understood as an instance of \(\phi \rightarrow \psi\), which it might otherwise seem to be.

  Indeed, viewed from the perspective of propositional logic, deduction theorem.
  If \(\vdash \phi \rightarrow \psi\) then \(\phi \vdash \psi\).
  \ESU{} denies that the same holds for claimed support.
  Seems quite sensible.
\end{note}

\begin{note}[Illustration]
  To illustrate~\ESU{}, consider the illustration provided for~\USE{}.

    If the agent did not measure the rectangle, understand how to calculate the area of a rectangle, or perform the required calculations, then the agent would not be in a position to claim support that area of the rectangle is \(133\text{cm}^{2}\).
  A lucky guess that the area of the rectangle is \(133\text{cm}^{2}\) would not allow the agent to claim support that the area of the rectangle is  \(133\text{cm}^{2}\) on the basis of the dimensions of the rectangle, the agent's understanding of how to calculate the area of a rectangle, and the relevant mathematics.
  And, it seems the agent is not in a position to base their lucky guess in such a way because the agent did not reason from the dimensions of the rectangle, the agent's understanding of how to calculate the area of a rectangle, and the relevant mathematics.\nolinebreak
  \footnote{
    Moving to another agent, observe doing the work, get report.
    Easy to resist, by adding in additional premise.
    Still, no presupposing that this needs to be done.
  }
  Similarly, if an agent learns that a rectangle with dimensions of \(19\text{cm}\) by \(7\text{cm}\) may be calculated to have an area of \(133\text{cm}^{2}\), then the agent may not claim support for the area of the rectangle on the basis of the calculation.
  If the agent has not performed the calculation, then the agent may not appeal to the use of the calculation when claiming support --- rather, the agent mentions that the calculation is true.\nolinebreak
  \footnote{
    Slight weakening of~\ESU{} may be made.
    So long as \emph{some} agent has performed the calculation.
    Argue against~\ESU{}, and the argument made will hold for this weakening.
  }
\end{note}

\subsubsection{Intuition}

\begin{note}[Intuition]
  \ESU{} and~\gESU{} seems quite plausible, at least to me.
  The proposition is a careful statement of an intuitive ideas:

  Whether or not an agent claims support is the result of the structure of the reasoning process, and if some premises or step is not used, then it is irrelevant to the structure of the process.
  Hence, the only premises and steps of interest when claiming support are those used in the reasoning process.

  Rests on the broader idea from~\gESU{}.
  Claiming support is the result of some agentive process, and the result of an agentive process is explained by the constituents of the process.\nolinebreak
  \footnote{
    Ah, the homonculus.

    Question about whether the agent is important.

    This gets difficult.

    Consider clocks.
    Clock does not keep track of time.
    Rather, mechanical system designed to change in constant with some passage of time. (Cf.\ \textcite{Smith:1988aa}.)

    Agent may be like this.
    Distinction is intentionality.
    When I go about keeping track of the time, I'm attempting (at least typically) to maintain reference to what the time is.
    Figure out a way to approximate a second, and that's what's happening.
    Approximation.
    If it is noted that I requarly sigh every minute, use this, but I wouldn't be keep tracking of time, though you may be using regularity to do so.
    So, in the former case, using understanding of time, while in the latter not doing so.
  }

  As~\gESU{} is restricted to an agent claiming support, things seem a little easier.
  Problems with interpretation, however.
  Transparency.
  Familiar, if debatable, illustration.
  Freud.
  (Here, adjourning the meeting by saying something mistaken.)
\end{note}

\begin{note}[Analogy]
  By analogy, whether or not my mug of (once cold) coffee overheats in the microwave is the result of some process involving electromagnetic radiation.
  My desire that the mug of coffee does not overheat is not used as part of the process of heating the coffee, and so is irrelevant to the structure of the process.

  My desire may explain why the mug of coffee is taking part in a certain process, and an unused premise or step may explain why an agent performed so reasoning.
  Still, a premise or step must be used as part of the process of reasoning to stand in explanation for the result of reasoning.

  Press the analogy further: Reasoning is a causal process.
  And, any property of reasoning reduces to cause and effect.
  If premises or steps are not used, then those premises or steps stands outside the relevant causal trace, and may not be appealed to when accounting for some structural property of the conclusion of the instance of reasoning (here, that the agent claims support for the conclusion).
\end{note}

\subsubsection{\ESU{} in the literature}

\begin{note}
  \color{red}
  Given proposed revision to \ESU{} this section should be expanded a little.
  For, most of the cases talk about claiming support for \(\phi\) directly, while \ESU{} is more general in that it talks about claiming support for any entailment between \(\phi\) and \(\psi\).
\end{note}

\begin{note}[Causal theories of basing]
  Indeed,~\ESU{} seems to be implied by various causal theories of basing.

  \citeauthor{Pollock:1999tm} introduce the basing relation with the following observation:
  \begin{quote}
    To be justified in believing something it is not sufficient merely to \emph{have} a good reason for believing it.
    One could have a good reason at one's disposal but never make the connection.
    \dots
    Surely, you are not justified in believing [something], despite the fact that you have impeccable reasons for it at your disposal.
    What is lacking is that you do not believe the conclusion on the basis of those reasons.\nolinebreak
    \mbox{}\hfill\mbox{(\cite[35]{Pollock:1999tm})}
  \end{quote}
  The observation falls short of being an account of the basing relation, but the intuition \citeauthor{Pollock:1999tm} appeal to is instructive.
  It seems that an agent must connect reasons and the content of a belief in order for the belief to be formed on the basis of those reasons, and hence be justified by those reasons.
  In turn, if a connection is made between reasons and the content of belief, then those reasons are used by the agent.

  For a concrete instance, consider \citeauthor{Moser:1989tv}'s account of the basing relation:
  \begin{quote}
    \emph{S}'s believing or assenting to \emph{P} is based on his justifying propositional reason \emph{Q} \(=_{\text{df}}\) \emph{S}'s believing or assenting to \emph{P} is causally sustained in a nondeviant manner by his believing or assenting to \emph{Q}, and by his associating \emph{P} and \emph{Q}.\nolinebreak
    \mbox{}\hfill\mbox{(\cite*[157]{Moser:1989tv})}
  \end{quote}

  Suppose we have a conclusion from some premises and steps of reasoning.
  If the agent has not witnessed the relevant reasoning, then it seems the conclusion is not causally sustained in a nondeviant manner by his believing or assenting to the premises of the reasoning, nor has the agent associated the conclusion with the premises by witnessing the relevant steps of reasoning.

  To illustrate, claim support that 173 is prime.
  It's possible that I did the prime factorisation, and possible that I took that representation to be part of the reason why I claim that 173 is prime.
  However, represented query of whether prime to wolfram alpha as justifying, and that's why I claimed support.
  So, definitely not from okay to appeal to the reasoning I have not witnessed.
  And, if infer that 173 is prime from claimed support that I have the ability to demonstrate that 173 is prime, the same issue.
  As I've not witnessed, then no role for \emph{Q}, whatever that turns out to be.

  This is a quick argument, and borders on conjecture, so let us examine the relevant association in greater detail.
  \citeauthor{Moser:1989tv} distinguishes between occurrent and non-occurrent satisfaction of association relations.

  We start with occurrent satisfaction of an association relation:
  \begin{quote}
    \emph{S} occurrently satisfies an association relation between \emph{E} and \emph{P} \(=_{\text{df}}\)
    \begin{enumerate*}[label=(\roman*), ref=(\roman*)]
    \item\label{moser:oar:i} S has a \emph{de re} awareness of \emph{E}'s supporting \emph{P}, and
    \item\label{moser:oar:ii} as a nondeviant result of this awareness, \emph{S} is in a dispositional state whereby if he were to focus his attention only on his evidence for \emph{P} (while all else remained the same), he would focus his attention on \emph{E}.\newline
    \mbox{}\hfill\mbox{(\Citeyear[141--142]{Moser:1989tv})}
    \end{enumerate*}
  \end{quote}

  \emph{de re} awareness is a non-propositional, direct awareness of \emph{E} supporting \emph{P}.

  \ESU{} follows from~\ref{moser:oar:i}.
  \emph{de re} awareness, but this doesn't rule out use.
  \ESU{} does not require that the agent engages in propositional reasoning.

  In cases where the agent has not witnessed reasoning, there is no \emph{de re} awareness.
  Without the reasoning taking place, the agent is not directly aware of what the reasoning consists of.

  Following, the definition of non-occurrent satisfaction of an association relations is derived from occurrent satisfaction of an association relations by allowing~\ref{moser:oar:i} to be satisfied at some point in the past while requiring that~\ref{moser:oar:ii} continues to be satisfied in the present.
  As noted, \ESU{} is compatible with the agent having witnessed the reasoning at some point in the past.
  Therefore, \ESU{} is entailed given both occurrent and non-occurrent satisfaction of association relations
\end{note}

{
  \color{red}
  This doesn't make sense\dots
  I think the idea I had was that the agent has to use the represented relation.
  Hm, so, the idea is that in the case of \(\phi \vdash \psi\), the agent hasn't represented how to get from \(\phi\) to \(\psi\), and therefore the agent isn't allowed to base \(C\) or \(R\) given \citeauthor{Neta:2019aa}'s account.
  I don't think this is sufficiently clear from what I have written.
  However, it does seem relevant.
  And, in also, basing doesn't necessarily need to be between beliefs.
  This could just be a relation of justification\dots though this isn't necessarily the case.
  So care is need.
  Still, with a little rewriting this looks useful.

  The tricky part is understanding what it is to represent R as justifying C.
  What I need is the idea expressed above, that the relevant representation is sufficiently detailed.
  I think this should be in \citeauthor{Neta:2019aa}.
  For, intuitively representing R as \emph{justifying} C is stronger than a representation with the content that R justifies C.
}

\begin{note}[Representationalism]
  \citeauthor{Neta:2019aa} generalises (purely) epistemic interest in basing relations to cover the explanatory relation between reasons and (rationally evaluable) states held, or actions performed, by an agent.

  On the way to a novel proposal, \citeauthor{Neta:2019aa} sketches a broad characterisation of representationalist theories of (generalised) basing:
  \begin{quote}
    \begin{enumerate}[label=(R\arabic*), ref=(R\arabic*)]
    \item\label{neta:RC:b} \emph{basing} C on R involves the agent's representing R as justifying C, and
    \item\label{neta:RC:jb} \emph{justifying basing} of C on R consists in the adroitness of this representation.\nolinebreak
          \mbox{}\hfill\mbox{(\Citeyear[192]{Neta:2019aa})}
    \end{enumerate}
  \end{quote}
  As \ESU{} does not distinguish between successful and unscuccesful instances of claiming support, our interest is with~\ref{neta:RC:b}.
  And, in contrast to \citeauthor{Moser:1989tv}, a representationalist theory may lack a causal component.
  Indeed, \citeauthor{Neta:2019aa} considers scenario in which an agent receives information from some source, forms a belief which is supported by the received information, and represents the received information as justifying the belief.
  The twist, however, is that the agent forming the belief was caused by some other source.
  For example, an agent may listen to a speech given by a talented orator and form a belief in response to the speech.
  The agent may represent the content of the speech as justifying the conclusion, while the cause of the belief being formed is the emotional impact with which the orator stated the conclusion.
  Following the representationalist characterisation, the agent would base the content of the belief on the content of the speech rather than the emotional impact with which the speech concluded.
  Indeed, the agent may do so even if they recognise that they were swayed by emotion.

  As before, consider a conclusion of some reasoning that the agent has not witnessed.
  If the agent has not witnessed the reasoning, then the agent has not represented some or all of the relevant premises and steps of reasoning.
  Therefore, it seems that it is not possible for the agent to represent the premises and steps of reasoning as justifying the relevant conclusion.
  In other words, a representationalist account requires (minimally) that an agent represents premises and steps of reasoning as justifying when claiming support for some conclusion of reasoning, and hence use of those premises and steps.

  {
    \color{red}
    What is going on here\dots
    The point is that if we follow \citeauthor{Neta:2019aa} then there needs to be a representation.
    In turn, the issue is that it's not clear that the agent needs to reason from \(\psi\) to \(\phi\) in order to obtain the relevant representation.
    So, it's not clear that \citeauthor{Neta:2019aa} actually is relevant.

    So, it's this previous paragraph that needs attention.
    No use, then no representation.
    This is the only point that really matters.
    So, I need to find something in \citeauthor{Neta:2019aa} that supports this, or somehow provide a much better argument.

    Then, in the following paragraph is redundant.
    The issue is with how the relevant representation is obtained.
    The part where I'm getting confused is that \citeauthor{Neta:2019aa} doesn't hold that the agent needs to do the reasoning each time the representation is used.
  }

  As an aside, it is not clear whether representing an entailment or inference is the same as reasoning with an entailment, and therefore it does not seem to follow from the representationalist characterisation that the agent must witness the relevant reasoning.
  However, the interpretation of `use' is intended to be sufficiently broad to cover such cases.\nolinebreak
  \footnote{
    Alternatively, a clause may be added to~\ESU{} which denies that the agent represents the relevant premises and steps of reasoning.
    The argument made against~\ESU{} is compatible with the use of representations, or mere representation even if unused --- though it is unclear to me what an unused but represented premise or step would matter when claiming support.
  }

  Further, \citeauthor{Neta:2019aa}'s discussion is instructive because the response \citeauthor{Neta:2019aa} offers to some problematic scenarios focus on \emph{how} a representation is used.
  One may hold that the agent in the example given did not base their belief in the conclusion on the content of the speech in view of the fact that the agent was swayed by emotion.
  If so, \citeauthor{Neta:2019aa} proposes the following revision:
  \begin{quote}
    \begin{enumerate}[label=(R\arabic*\('\)), ref=(R\arabic*\('\))]
    \item\label{neta:RC:jp} for an agent to C based on reason R involves not merely the agent's representing R as justifying C---it also involves \emph{this latter representation (or its content) being part of the reason why the agent C's}.\nolinebreak
      \mbox{}\hfill\mbox{(\Citeyear[197]{Neta:2019aa})}
    \end{enumerate}
  \end{quote}
  The added clause states that the relevant representation must explain why the agent formed a belief.
  Hence, given~\ref{neta:RC:jp} the agent would not be permitted to base their belief in the content of the speech given that they were swayed by emotion.
  Intuitively,~\ref{neta:RC:jp} expands on what it is for premises and steps of reasoning to be use when forming a belief.
  So, given that representation requires use, the expanded clause may be seen as focusing on \emph{how} the representation is used.
\end{note}

\begin{note}[Responding to reasons]
  As final motivation, consider the proposal at the core of \citeauthor{Lord:2018aa}'s (\Citeyear{Lord:2018aa}) thesis that being rational is to correctly respond to reasons.

  \begin{quote}
    \textbf{Correctly Responding:} What it is for A's \(\phi\)-ing to be ex post rational is for A to possess sufficient reason S to \(\phi\) and for A's \(\phi\)-ing to be a manifestation of knowledge about how to use S as sufficient reason to \(\phi\).\nolinebreak
    \mbox{}\hfill\mbox{(\Citeyear[143]{Lord:2018aa})}
  \end{quote}

  An agent's action is rational only if the action is a manifestation of some know-how.
  \citeauthor{Lord:2018aa} summaries:

  \begin{quote}
    \dots when one manifests one's know-how, dispositions that are directly sensitive to normative facts are manifesting. Thus, the competences involved in the relevant know-how make one directly sensitive to the normative facts\nolinebreak
    \mbox{}\hfill\mbox{(\Citeyear[16]{Lord:2018aa})}
  \end{quote}

  For our purposes, following example of manifesting know-how directly relates to reasoning:

  \begin{quote}
    The most salient disposition [when appealing to \emph{p} as a reason]\nolinebreak
    \footnote{Note, \citeauthor{Lord:2018aa} (explicitly) not talking about believing that \emph{p} is a reason, but argues that the cited disposition to present both when appealing to p as a reason and believing that \emph{p} is a reason.}
    is the disposition to (competently) use \emph{p} as a premise in reasoning.\nolinebreak
    \mbox{}\hfill\mbox{(\Citeyear[25]{Lord:2018aa})}
  \end{quote}

  Hence, suppose an agent appeals to a premise of reasoning in order to claim support for some conclusion.
  Then, if the agent does not use the premise of reasoning, it seems the agent does not manifest know-how, which is required for the appeal to meet \citeauthor{Lord:2018aa}'s account of rational action.

  Of course, that the noted disposition is the most salient does not rule out alternative, less noteworthy, dispositions.
  However, it is unclear to me how to \emph{manifest} know-how without use.
  Looking ahead, it does not seem to be the case that I manifest my ability to show that a certain rule of inference is sound when skipping over details in a completeness proof.
  However, I may manifest know-how regarding the (presumed) truth of the ability attribution.

  Likewise with my ability to establish a preference for tofu over any other kind of miso when ordering soup.
\end{note}

\begin{note}[Summarising illustrations]
  Three examples of claiming or establishing relations of support have been given.
  Each example suggests that if an agent does not use a premises or steps when claiming support, then an agent may not claim support by appeal to the unused premises or steps.

  Stepping back,~\ESU{} may be seen as a desiderata for any account of (successfully) claiming support.
  For:
  If an agent (successfully) claims support for some conclusion of reasoning, then the premises and steps used with respect to that claim of support is itself the result of some reasoning --- the reasoning that culminated with the claim to support itself used premises and steps of reasoning.
  So, given that the agent used certain premises and steps when claiming support for conclusion, some property of the premises and steps used, an adequate account of claiming support must explain how the premises and steps used permit the agent to claim support.\nolinebreak
  \footnote{
    Note, however, that this argument does not imply that support for the conclusion must be accounted for in terms of the premises and steps used by the agent to claim support.
    As we will note below, one may hold that an enthymematic argument permits an agent to claim support, while the relevant relation of support is secured by the corresponding non-enthymematic argument.
    Cf.\ \textcite{Moretti:2019wx} for suggestions along these lines.
  }
  In turn, if an agent appeals to premises and steps that they did not use, then those premises and steps must be redundant.

  Turning to ability.
  Suppose and agent appeals to
  \begin{enumerate*}
  \item their ability to demonstrate that \(\phi\) is the case, and
  \item that \(\phi\) must be the case in order for the agent to have the ability to demonstrate that \(\phi\)
  \end{enumerate*}
  in order to claim support for \(\phi\).
  Then, the premises and steps involved in a full account of reasoning from the two claims must be sufficient to claim support that \(\phi\) is the case.
  So, as the agent does not witness their ability to demonstrate that \(\phi\) in such reasoning, it must be the case that claimed support for (the property of) having the ability to demonstrate that \(\phi\) is sufficient for such reasoning.
\end{note}

\subsection{\EAS{} --- \EAS{-}}
\label{sec:eas}

{
  \color{red}
  Perhaps include a note about how the argument relates to \EAS{}.
  I don't provide a direct argument, but this is the best way I see of resolving the tension.
}

\begin{note}[Alternative]
  \ESU{} is a universal claim, and so applies to all instances in which an agent may claim support for conclusion on basis of support for premises and steps of reasoning --- an agent may only claim support if the agent reasoned from the premises via the steps to the conclusion.

  Our goal is to motivate the following exception to \gESU{}, and hence \ESU{}:

  \goalEAS*
\end{note}

\begin{note}[Intuition for \EAS{}]
  \EAS{} is a conditional.
  Antecedent is claimed support for ability.
  Consequent is that it may be permissible to violate \gESU{}.
\end{note}

\begin{note}
  Now, started with \USE{}, and then looked at \ESU{}, the converse.
  Both of these we have a particular instance of reasoning in mind.
  Now, \EAS{} may, intuitively, be understood to states that whatever that reasoning is, if an agent has claimed support that they're able to witness such reasoning, then the agent may claim support.

  However, things are a little more complex.
  \EAS{} is about the ability to claim support to reason to some conclusion.
  However, \EAS{} does not state that the agent may claim support for the conclusion on the basis of the premises that they would reason from were they to witness the ability.

  Issue here is that the substance of \EAS{} --- what the relevant materia amounts to --- depends on two things:
  \begin{itemize}
  \item How (appeal to) ability is understood, and
  \item The kind of reasoning involved in the appeal to ability.
  \end{itemize}

  We will outline the basics, then reformulate \EAS{} using one what in which (appeal to) ability is understood.

  Start, how ability is understood.
  Lead naturally to the kind of reasoning involved.

  The argument for \EAS{} will not depend on how ability is understood, but the kind of reasoning involved.
  Still, kind of reasoning involved when combined with how ability is understood.
\end{note}

\begin{note}
  Briefly stated,
  \AR{} understands ability in terms of some (complex) property.
  \WR{} understands ability in terms of possible witnessing events.

  For example, \AR{} may involve the property (attribution) of understanding geometry, perhaps broken down into the understanding or availability of various definitions, propositions, lemmas, theorems, and steps of reasoning.
  While, \WR{} would involve reasoning with particular definitions, propositions, lemmas, theorems, and steps of reasoning.

  So, agent appeals to property, or the reasoning itself.

  The purpose of this distinction is to ensure that our argument against \gESU{} does not rest on a particular way of understanding ability that may not extend to other ways of understanding ability.

  Conjecture that these are fundamentally connected.
  Witnessing event only if understanding.
  And, understanding only if possible to witness reasoning.

  Still, difference.
  Relevant properties are properties of the agent as they are.
  The witnessing event, by contrast, is a possible event.\nolinebreak
  \footnote{
    Property of there being a possible event involving the agent.
    In this case, still distinct from \WR{} as that the agent is part of possible event is still distinct from the reasoning that the agent would witness in the relevant event.
  }

  These are brief characterisations, but enough for now.
  Both~\AR{}~and~\WR{} will be considered at length in~\autoref{sec:ar-wr-1}.
  In addition to a more thorough treatment of the core ideas, \autoref{sec:ar-wr-1} includes additional examples, and an argument that~\AR{}~and~\WR{} are exhaustive --- any way of understanding ability will conform to either~\AR{}~and~\WR{}.
\end{note}

\begin{note}
  Now turn to the kind of reasoning involved.

  Motivated \AR{} in terms of understanding of premises and steps of reasoning, and \WR{} in terms of a possible event in which agent reasons with particular premises and steps.

  However, a further distinction in terms of what appeal to the relevant premises and steps or instance of reasoning amounts to.

  First, there is the \emph{existence} of premises and steps, or the \emph{possibility} of the witnessing event.
  Second, there is the premises and steps themselves, or the witnessing event.

  Difference from perspective of step of reasoning.
\end{note}


\begin{note}[Types of reasoning]
  Consider proofs.

  \(p \lor q\)
  \(\lnot q\)
  \(p\)

  Premises alone do not establish \(p\).
  Combined they do.

  Claim support individually, then \(p\).
  Alone, these don't require \(p\).
  More in \autoref{sec:ability-ads-adc}.
\end{note}

\begin{note}[\EASw{}]
    \begin{restatable}[\EASw{-} --- \EASw{}]{thought}{thoughtEASw}\label{thought:EASw}
    If an agent has claimed support that they have the ability to (adequately) reason to some conclusion, then it may be permissible for the agent to claim support for the conclusion by claiming support for the premises and steps of reasoning that the agent would use to witness their ability to reason to the conclusion.
  \end{restatable}

  Loosely restated,~\EASw{} holds that if an agent may claim support for having the ability to witness some reasoning, and is aware of the conclusion of that reasoning, then the way in which the agent claims support for the conclusion of that reasoning may mirror the way in which the agent would claim support for the conclusion by witnessing the reasoning (and hence using the relevant premises and steps).
\end{note}

\begin{note}[Just an idea]
  \emph{Idea} as this is preferred way of thinking about ability.
  However, argument will not depend on this way of thinking.
\end{note}

\begin{note}
  The (possible) event of the agent witnessing their ability to demonstrate \(\phi\) involves reasoning with various premises and steps which culminate in claiming support for \(\phi\).
  So, if~\EASw{} is true, then the agent may appeal to those premises and steps which are used in the (possible) witnessing event.

  One way to think about~\EASw{} (which we will explore in more details later) is in terms of propositional support.
  For, if an agent has the ability to demonstrate that \(\phi\) is the case, then the agent has propositional support for \(\phi\) as there is a way for the agent to demonstrate that \(\phi\) is the case.
  In addition, that the agent has the ability to demonstrate that \(\phi\) is the case ensure that the agent is in a position to make use of the available propositional support for \(\phi\).
  In turn,~\EAS{} may be interpreted to hold that so long as the agent has such information about their position to make use of the available propositional support for \(\phi\) then the agent does not need to reason with the relevant propositional support in order to claim support for \(\phi\) in virtue of the available propositional support for \(\phi\).
\end{note}

\begin{note}[Conditional]
  Here, note that it's a conditional, but also that it only states there are instances.
  It doesn't follow that ability will always allow the agent to claim support.

  The conditional is weak primarily because it is not at all clear that it holds in general.
  There are various cases in which it seems appeal to ability is blocked.

  Easiest cases involve claiming support in some public setting.
  Of course, success in a public setting is not necessarily required for private success.
  Same problem with testimony.
  I'm confident in a source and you're not.
  I fail to convince you, but I remain convinced myself.

  Still, seems as though similar considerations extend.
  For example, doing a PSET where I'm allowed to use theorems I've already proved.
  Have notes of what those theorems are.
  And, ability to prove them.
  Still, might refrains from using them until I've proven them once again.

  More could be said here, and it may be possible to argue for a stronger variant of \EAS{}.

  Even though it's weak, the condition is still interesting.
\end{note}


\begin{note}
  So, if~\EAS{} is true, then there are cases in which an agent is not required to reason from premises they may claim support for to some conclusion in order to obtain support for the conclusion on the basis the support the agent has for the premises.\nolinebreak
  \footnote{
    Stated~\EAS{} as an exception to~\ESU{}.
    And, we will argue that~\EAS{} is true.
    However, we will not argue that~\EAS{} \emph{is an exception} to~\ESU{}.
    To do so would require an argument that \ESU{} holds for other cases.
    Likewise, no argument that~\EAS{} is the only exception, as to do so would require argument that~\ESU{} holds for all other cases.
    Take~\ESU{} to be plausible, and suspect that there are few, if any, further exceptions, but~\EAS{} may stand independently on any further statements about claiming support.
  }
\end{note}

\begin{note}
  \color{red}
  I want to clarify \EAS{} a little.
  The use of `may' is problematic.
  It could be read as `it's always okay, but it's up to the agent'.
  Or, `it's possible, given appropriate context'.
  The latter is what I want, and is important for cases where doubts are plausibly raised about the ability.
\end{note}

\begin{note}[\EAS{} illustration]
  To illustrate \EAS{}

  \begin{illustration}\label{ill:rectangle:ability}
    Suppose you provide me with novel information that:
    \begin{enumerate}[label=\emph{A}\arabic*., ref=(\emph{A}\arabic*), series=EAS_counter]
    \item\label{EAS:ex:box:if} If I have ability to calculate the area of a box, then I have the ability to demonstrate that a rectangle with dimensions \(19\text{cm}\) by \(7\text{cm}\) has area \(133\text{cm}^{2}\).
    \end{enumerate}
    The information is `novel' because I have not been previously informed (in any way) about the area of a rectangle with dimensions \(19\text{cm}\) by \(7\text{cm}\).

    Still, (I claim support that):
    \begin{enumerate}[label=\emph{A}\arabic*., ref=(\emph{A}\arabic*), resume*=EAS_counter]
    \item\label{EAS:ex:box:gen} I have the ability to calculate the area of a rectangle.
    \end{enumerate}
    Therefore:
    \begin{enumerate}[label=\emph{A}\arabic*., ref=(\emph{A}\arabic*), resume*=EAS_counter]
    \item\label{EAS:ex:box:spec} I have the ability to demonstrate that a rectangle with dimensions \(19\text{cm}\) by \(7\text{cm}\) has area \(133\text{cm}^{2}\).
    \end{enumerate}
    From~\ref{EAS:ex:box:spec} it follows that:
    \begin{enumerate}[label=\emph{A}\arabic*., ref=(\emph{A}\arabic*), resume*=EAS_counter]
    \item\label{EAS:ex:box:fact} A rectangle with dimensions \(19\text{cm}\) by \(7\text{cm}\) has area \(133\text{cm}^{2}\).
    \end{enumerate}
  \end{illustration}
  \EAS{} holds that, when I claim support for~\ref{EAS:ex:box:fact} from~\ref{EAS:ex:box:spec}, I may appeal to dimensions and formula, though as I do not witness the ability, I do not use the premise and step.
  For, if~\ref{EAS:ex:box:spec} is the case then it is possible for me to witness reasoning in which I demonstrate that~\ref{EAS:ex:box:fact} is the case, and it is the premises and steps of reasoning used in such reasoning that establishes~\ref{EAS:ex:box:fact} is the case.
  I have not used those steps and premises, as I have not witnessed the relevant ability, but may I appeal to those steps and premises regardless --- or so we will argue.
\end{note}

\begin{note}[More detail]
  \color{details}
  I do not expect \EAS{} to be intuitive.
  Indeed, we are not interested in \EAS{} because it is a more-or-less intuitive principle which conflicts the intuitive \ESU{}.
  Rather, we are interested in \EAS{} primarily because \EAS{} is a consequence of tension arising from three things:

  \begin{enumerate}
  \item\label{incomp:tri:q:1} \ESU{}
  \item\label{incomp:tri:q:2} scenarios involving an agent reasoning with information about an their own ability,
  \item\label{incomp:tri:q:3} and a principle concerning when an agent is permitted to claim support
  \end{enumerate}

  To briefly expand on~\ref{incomp:tri:q:2} and~\ref{incomp:tri:q:3}:

  Information that one has some specific ability so long as one has some general ability --- such as the (specific) ability to show that \(25^{\circ}\text{C} = 77^{\circ}\text{F}\) given the (general) ability to convert between Celsius and Fahrenheit.
  And, an agent is never permitted to claim support for proposition having a certain value if the agent requires the proposition to have value \emph{in order to} claim support.
  (As an instance, an agent is not permitted to claim support for the truth of a proposition if the agent requires the proposition to be true \emph{in order to} claim support that the proposition is true.)\nolinebreak
  \footnote{
    The emphasis on `in order to' is important.
    The instance of the principle does not state that an agent is not permitted to claim support for the truth of a proposition if the agent requires the proposition to be true when claiming support that the proposition is true.
    I plausibly require that \(2 + 2 = 4\) when I claim support that \(2 + 2 = 4\), and this does not prevent me from claiming support by simple arithmetic.
    However, it would be impermissible (or so we will argue) to claim support that \(2 + 2 = 4\) by reasoning that the calculator is functional only if \(2 + 2 = 4\), and as the calculator states that \(2 + 2 = 4\) it is the case that \(2 + 2 = 4\).
  }
  The details matter, and we postpone detailing this argument to~\autoref{sec:broad-argum-overv}.

  In short, assuming the scenarios exist, there is tension between intuitive principles governing what an agent appeals to when reasoning and structural principles governing the relation between what the agent appeals to when reasoning.
\end{note}

\begin{note}
  For the moment we attempt to clarify \EAS{} to some degree.
  Three subsections follow:

  \begin{enumerate}
  \item We will outline alternative reasoning patterns from~\ref{EAS:ex:box:if} to~\ref{EAS:ex:box:fact}, clarify why we focus on a particular type of reasoning pattern, and examine some initial objections to~\EAS{} and canvas some responses.
  \item We will consider parallels between abilities and dispositions.
    The parallel will provide some additional intuition for why an agent may appeal to premises and steps that have no been used, and help further clarify our interest with ability.
  \item We will consider a related proposition argued for by \citeauthor{Moretti:2019wx} which holds that a belief need not be based (exclusively) on the premises and steps of reasoning used to arrive at the belief.
    The comparison will help highlight what is distinctive about~\EAS{} while at the same to introducing some ideas which suggest a way of understanding~\EAS{}.
  \end{enumerate}
\end{note}

\subsubsection{Against \EAS{}}

\begin{note}[Alternatives]
  The alternative reasoning pattern we will focus on in some detail holds that appealing to having the ability noted in \ref{EAS:ex:box:spec} is sufficient to claim support for \ref{EAS:ex:box:fact}.
  In line with \ESU{}, the agent would use the proposition that they have the relevant ability noted in~\ref{EAS:ex:box:spec} to claim support for~\ref{EAS:ex:box:fact}
  This reasoning pattern, along with the pattern suggested by \EAS{} will be considered in \autoref{sec:wr-ar} and we will argue that it conflicts with an intuitive principle regarding claiming support in \ref{sec:second-conditional}.

  Alternatively, on may argue that though the syntactic form of \ref{EAS:ex:box:if} is a conditional, it does not (necessarily) follow that the semantic content of~\ref{EAS:ex:box:if} is a (also) conditional.
  And that~\ref{EAS:ex:box:if} may (plausibly) be interpreted to explicitly state that~\ref{EAS:ex:box:spec} is an ability that an agent may have.
  For example:
  \begin{enumerate}[label=\emph{A}\arabic*., ref=(\emph{A}\arabic*), resume*=EAS_counter]
  \item\label{EAS:ex:box:if:R:state} The ability to demonstrate that a rectangle with dimensions \(19\text{cm}\) by \(7\text{cm}\) has area \(133\text{cm}^{2}\) is an ability an agent may have and it is an ability an agent has if they have ability to calculate the area of a rectangle.
  \end{enumerate}
  Hence, \ref{EAS:ex:box:if} is interpreted so that \ref{EAS:ex:box:spec} is accessible without endorsing the antecedent of \ref{EAS:ex:box:if}.
  \ref{EAS:ex:box:if:R:state} states that there is some ability that it is possible for an agent to have, and in addition provides sufficient conditions for having the relevant ability.
  The important part of \ref{EAS:ex:box:if:R:state} is the former conjunct:
  \begin{enumerate}[label=\emph{A}\arabic*., ref=(\emph{A}\arabic*), resume*=EAS_counter]
  \item\label{EAS:ex:box:spec:R:state} The ability to demonstrate that a rectangle with dimensions \(19\text{cm}\) by \(7\text{cm}\) has area \(133\text{cm}^{2}\) is an ability an agent may have.
  \end{enumerate}
  And, \ref{EAS:ex:box:fact} follows from \ref{EAS:ex:box:spec:R:state} by the observation that it is not possible to demonstrate that \(\phi\) if \(\phi\) is not the case --- there is no need for the agent to appeal to witnessing their ability.
  Therefore,~\ref{EAS:ex:box:spec:R:state} (and~\ref{EAS:ex:box:if:R:state}) implicitly includes information that~\ref{EAS:ex:box:fact} is the case --- an agent does not need to reason from~\ref{EAS:ex:box:spec} to~\ref{EAS:ex:box:fact}, because they have already been informed that~\ref{EAS:ex:box:fact} is the case.

  Note also that analogous reasoning applies if `I' is replaced with `an agent'.
  Likewise, if I know that you that you know that I have the ability to calculate the area of a rectangle.
  For, it then follows that you know that \ref{EAS:ex:box:spec} is the case, and therefore you know that \ref{EAS:ex:box:fact} is the case.
  Again there is no need for me to appeal to witnessing an ability.
\end{note}

\begin{note}[Box]
  The existence of alternative reasoning patterns is the issue at hand.
  For, so long as there are reasoning patterns \emph{R} which conform to \ESU{} it is open to the defender of \ESU{} to hold that if an agent is permitted to claim support, then the agent is required to reason via some member of \emph{R}.
  For, if there are reasoning patterns \emph{R} which conform to \ESU{} then there a no counterexamples to \ESU{} --- scenarios in which an agent claims support by appeal to premises or steps of reasoning that the agent has not used.

  Of course, an argument against a general principle such as \ESU{} is not required to be a counterexample.
  For example, it may be possible to argue that the reasoning patterns \ESU{} requires are sufficiently implausible.
  Hence, a restricted variant of \ESU{} compatible with \EAS{} would to be preferred.
  However, there are two issues with attempting such an argument.

  First, given the intuitive plausibility to \ESU{}, it seems unlikely that any violation of \ESU{} would be more plausible than an alternative reasoning pattern compatible with \ESU{}.
  Second, even if there are plausible reasoning pattern that are incompatible with \ESU{}, it is not clear that these should be incorporated in a theory of claiming support.
  For, meta-theoretical issues such as complexity or predictive power may still favour \ESU{}.
  Following \citeauthor{Box:1987vn}: `\dots all models are wrong; the practical question is how wrong do they have to be to not be useful.' (\Citeyear[74]{Box:1987vn})

  Indeed, the second point suggest a counterexample proper to show that \ESU{} is false is not necessarily an adequate argument against \ESU{} either.
  Observations in the spirit of \citeauthor{Box:1987vn} are trite, but also true.
  Even if \EAS{} is true and \ESU{} is false, would \EAS{} be useful?
\end{note}

\begin{note}[Responding to Box]
  With respect to idealised agents with unbounded resources, the answer appears to be no.
  For, with unbounded resources the agent the option of (attempting to) witnessing any ability (to reason) without cost.
  And, it seems that for such an agent witnessing a relevant ability would always be preferable to reasoning about an unwitnessed ability as the agent would minimally (subjectively) resolve any uncertainty about whether they have the ability.

  However, for limited agents, ability is abundant, while the resources required to witness abilities are scarce.
  That the exception to~\ESU{} is narrow does not entail that there are few occurrences of the exception.

  Information about ability may be abundant while the resources for witnessing abilities are either scarce or temporarily unavailable.
  So, for example, agent has the option of conserving or deferring use of resources.

  This observation suggests an initial line of response to an objection which focuses on whether \EAS{} would be useful.

  For, given that we are resource bound agents, it seems that possible instances of \EAS{} are widespread.
  From a functional perspective, reasoning with (the relevant instances of) ability just is reasoning about the result of expending available resources.
  Hence, if~\EAS{} is true, then the truth of \EAS{} would provide a novel perspective on resource bound agents.
  And, it is yet to be seen whether such a perspective is useful.

  In addition, there is a second indirect line of response.
  We observed above that \ESU{} seems prevalent in various theories which relate to reasoning, such as basing and responding to reasons.
  If \EAS{} is true, then there may be alternative conclusions to arguments that appeal to~\ESU{} as a premise.
  And, likewise, there may be interesting observations made in premises of arguments which establish \ESU{} as a foundation for further theorising.\nolinebreak
  \footnote{
    As an exception, even if~\EAS{}, conclusion of arguments which appeal to or assume \ESU{} may be restricted.
  }

  Taking stock:
  I doubt that \EAS{} is of interest if there are no reasoning patterns which require \EAS{} to be true.
  Still, if there are reasoning patterns which require \EAS{} to be true, then \EAS{} may be of interest.
  Further, I think there are good reasons to hold that there are reasoning patterns which require \EAS{} to be true.
  Hence, my goal is to motivate further research into whether \EAS{} is of interest.

  We now briefly turn to the type of scenarios which are a premise in our argument for the existence of such counterexamples.
\end{note}

\begin{note}[Types of scenario]
  The type of scenario we will focus on is designed to ensure that an agent is required to reason to (and from) information about a (specific) ability that they have.
  If the agent is required to reason \emph{to} (specific) ability information then rephrasing \ref{EAS:ex:box:if} as \ref{EAS:ex:box:if:R:state} will not be possible --- the agent will be required to reason from some premises by some steps to the (specific) ability information.
  And, as before, the type of scenario will preserve the requirement of the agent to reason \emph{from} the (specific) ability information in line with~\ref{EAS:ex:box:spec} and~\ref{EAS:ex:box:if:R:state}.
  Hence, by establishing such scenarios are possible we may restrict our attention to the steps from~\ref{EAS:ex:box:if} to~\ref{EAS:ex:box:spec} and from~\ref{EAS:ex:box:spec} to~\ref{EAS:ex:box:fact}.
\end{note}

\begin{note}
  To illustrate, let us add some context to the example scenario we've been focusing on.

  Suppose it is common knowledge between you and I that
  \begin{enumerate*}
  \item you have looked through my notes, and have applied my formula for calculating the area of a rectangle, and
  \item my notes are the only source of information you have regarding how to calculate the area of a rectangle.
  \end{enumerate*}
  We may now restate the semantic content of~\ref{EAS:ex:box:if} as follows:
  \begin{enumerate}[label=\emph{A}\arabic*., ref=(\emph{A}\arabic*), resume*=EAS_counter]
  \item\label{EAS:ex:box:inf:R} You have some general ability \(\gamma\), and a specific ability \(\varsigma\) (as an instance of that general ability).
    And, if \(\gamma\) is the ability to calculate the area of a rectangle, then \(\varsigma\) is the ability to demonstrate that a rectangle with dimensions \(19\text{cm}\) by \(7\text{cm}\) has area \(133\text{cm}^{2}\).
  \end{enumerate}
  The formula in my notes indicates that I have the ability to do something, and I have indicated what I think the appropriate characterisation of the ability is.
  Still, you are not in a position to offer information as to whether my characterisation of the ability is correct or not.\nolinebreak
  \footnote{
    Consider in reverse.
    One is often attributed abilities that I deny I have.
    For example, I do not have the ability to process information by means of mental images, but \citeauthor{Hume:2011aa} (arguably) holds that I do have such an ability.
    I lack the ability to reason in a particular way.
    (Not that \citeauthor{Hume:2011aa}'s arguments rest on visual as opposed to any other kind of imagination, but the point stands.)

    Similarly, you may claim that I have the ability to tell you whether or not \nagent{7} is coming to tea.
    However, and in contrast to your assumption, \nagent{7} has not replied to my invitation and so I lack a required premise in order to reason to a relevant conclusion.
  }
  The key feature of~\ref{EAS:ex:box:inf:R} it that I, the agent, am required to claim support that \(\gamma\) and \(\varsigma\) are the abilities of interest.
  The focus is not on whether or not an agent may perform some action.
  Rather, our interest is with what the action is.
  It is up to me, the agent, to claim support that:
  \begin{enumerate}[label=\emph{A}\arabic*., ref=(\emph{A}\arabic*), resume*=EAS_counter]
  \item\label{EAS:ex:box:gen:R} The general ability \(\gamma\) \emph{is} the ability to calculate the area of a rectangle.
  \end{enumerate}
  If so, I may then claim support that:
  \begin{enumerate}[label=\emph{A}\arabic*., ref=(\emph{A}\arabic*), resume*=EAS_counter]
  \item\label{EAS:ex:box:spec:R} \(\varsigma\) is the specific ability to demonstrate that a rectangle with dimensions \(19\text{cm}\) by \(7\text{cm}\) has area \(133\text{cm}^{2}\).
  \end{enumerate}
  Note, there is no route to \ref{EAS:ex:box:spec:R} other than by claiming support for~\ref{EAS:ex:box:gen:R} as I have no information about what the (specific) ability \(\varsigma\) is amounts to if \ref{EAS:ex:box:gen:R} is not the case.
  If \(\varsigma\) is some other ability, then~\ref{EAS:ex:box:fact} does not follow, witnessing the relevant ability would not demonstrate that a rectangle with
  dimensions \(19\text{cm}\) by \(7\text{cm}\) has area \(133\text{cm}^{2}\), and so a rectangle with dimensions \(19\text{cm}\) by \(7\text{cm}\) may (from my epistemic perspective) have some other area.
\end{note}

\begin{note}[Point]
  We will say more in \autoref{sec:cases-interest}.
  For the moment it is sufficient to observe that the agent is required to reason to and from specific ability.
  The scenario requires the agent to claim support by reasoning from~\ref{EAS:ex:box:if} to~\ref{EAS:ex:box:spec} and from~\ref{EAS:ex:box:spec}~to~\ref{EAS:ex:box:fact}.
  And, while the context added to force the reasoning pattern of interest is narrow, the principle behind the context is simple:
  The agent is required to claim support that they have the relevant general ability.
  Hence, any scenario which consists of \gsi{-} which requires the agent to claim support that they have the relevant general ability will require the same kind of reasoning pattern.

  Further, this arguably captures a general puzzle about ability.
  An agent is not required to have witnessed all instances of a general ability to claim support that they have the general ability.
  However, so long as the agent may claim support for having some general ability, then it follows that the agent will have the option of claiming support for each instance of the general ability.

  The primary issue, though, is whether there is an account of such reasoning that does not require \EAS{} to be true.
  We will shortly turn to this argument in \autoref{sec:broad-argum-overv}.
  Prior to doing so, we close this section with some further clarification on the motivation behind \EAS{}, what distinguishes \EAS{} from nearby principles, and a suggestion on how to conceptualise \EAS{}.
\end{note}

\subsubsection{Ability and dispositions}

\begin{note}[Parallel]
  To further clarify the motivation for \EAS{} we introduce a parallel between abilities and dispositions.
  The primary function of the parallel will be to highlight the importance of reasoning about an event.
  In the case of dispositions the event is the manifestation of the disposition, and in the case of ability the event is the agent witnessing the ability.

  The parallel is of interest because \EAS{} concerns the premises and steps of reasoning that the agent would use to witness the relevant ability.
  We will suggest that claiming support that some object has some disposition and that some agent has some ability may both be understood in terms of claiming support that the relevant event is a possible event.

  In turn, if reasoning \emph{to} a specific ability is understood in terms of claiming support that it is possible for the agent to witness the event, then reasoning \emph{from} a specific ability may be understood in terms of claiming support from what would happen in the possible event.
  \end{note}

\begin{note}[Parallel between dispositions and ability]
  Consider \citeauthor{Choi:2021wg}'s characterisation of the Simple Conditional Analysis of dispositions:
  \begin{quote}
    An object is disposed to \emph{M} when \emph{C} iff it would \emph{M} if it were the case that \emph{C}.\nolinebreak
    \mbox{}\hfill\mbox{(\Citeyear{Choi:2021wg})}
  \end{quote}
  For example, an object is disposed to dissolve when it is placed in water iff the object would dissolve if it were the case that it is placed in water.

  The Simple Conditional Analysis may be challenged, but for our purposes it is adequate.
  We are interested in the broad form of the truth condition, and various more refined analyses share the same broad form.
  Note, in particular, that it being the case that \emph{C} and \emph{M} happening describes an event.
  Given appropriate conditions; salt dissolves, glass breaks, and I mumble when I am tired.
  The key idea is that the property of being disposed to \emph{M} when \emph{C} is analysed in terms of the (possible) event of \emph{M} happening when \emph{C}.

  The parallel to ability is established by noting that ability may also be analysed in terms of a (possible) event, as we have seen.
  In particular, by incorporating volition in the analysans of the Simple Conditional Analysis.
  To illustrate, \citeauthor{Mandelkern:2017aa} trace the Conditional Analysis of ability  to \textcite{Hume:1748tp} and \textcite{Moore:1912te}, among others:
  \begin{quote}
    S can \(\phi\) iff S would \(\phi\) if S tried to \(\phi\)\nolinebreak
    \mbox{}\hfill\mbox{(\Citeyear[Cf.][308]{Mandelkern:2017aa})}
  \end{quote}
  Compare to the Simple Conditional Analysis of dispositions:
  The object is some agent \emph{S}, \emph{C} is `S tried to \(\phi\)' and \emph{M} is `S \(\phi\)s' --- it is volition alone which distinguishes the analyses.
  For example, I have the ability to demonstrate that a rectangle with dimensions \(19\text{cm}\) by \(7\text{cm}\) has area \(133\text{cm}^{2}\) only if I would demonstrate that a rectangle with dimensions \(19\text{cm}\) by \(7\text{cm}\) has area \(133\text{cm}^{2}\) if it were the case that I tried that a rectangle with dimensions \(19\text{cm}\) by \(7\text{cm}\) has area \(133\text{cm}^{2}\).
\end{note}

\begin{note}[Claiming support]
  Parallel analyses in hand, we now turn to claiming support.
  We start with dispositions.

  As with ability, there are various ways in which an agent may claim support that some object is disposed to \emph{M} when \emph{C}.
  For example, I may claim support that my shoes are disposed to squeak when wet because I have had sufficient occasion to observe the phenomenon.
  Likewise, I may claim support that any shoe of the same model is disposed to squeak when wet because I have traced the source of the squeak to a manufacturing choice.
  In short, support may be claimed by past event and shared properties.

  Still, take a novel act and a object pair.
  Personally, I have a empty fountain pen that I haven't placed in water.
  I claim that the fountain pen is disposed to float when placed in water.
  My reasoning is fairly simple.
  The fountain pen is quite light, especially so while empty of ink.
  And, the cap and loading mechanism seem to be quite well sealed, so the weight of the fountain pen will not increase by taking on water.
  So, given that the weight of the fountain pen will be unchanged, and given how light the pen is, it seems that the upward force exerted by the water against the fountain pen will be sufficient to keep the pen afloat.

  In short, I've noted a few properties of the pen, claimed support for a handful of others, and then considered what would happen.
  Our interest is with the last step.
  I appeal to, and use, the possible event.\nolinebreak
  \footnote{
    I may be wrong about the event, but that isn't at issue.
    It remains the case that I appeal to it.
  }
  The noted properties are relevant because they suggest that the event of floating would happen if it were the case that the fountain pen were placed in water.
\end{note}

\begin{note}
  The fountain pen is not the only object on my desk.
  Beside the fountain pen is a collection of instruments that I may use to investigate the fountain pen.
  And, stored in my mind is a basic understanding of fluid dynamics.

  If I were to measure the fountain pen, ensure that it is airtight, and appeal to some known facts, then an application of Archimedes' principle would allow me to demonstrate that the fountain pen is disposed to float when placed in water (of some specified density).
  Indeed, such a demonstration would be a straightforward refinement of the way in which I claimed support for the proposition that the pen is disposed to float when placed in water.

  Now, by similar reasoning I have claimed support for the proposition that I have the ability to demonstrate that the proposition that the pen is disposed to float when placed in water is true.
  Here, in addition to appealing to properties of the fountain pen, I also appealed to various mental properties.
  There is an important difference, however, regarding the relevant event.
  When reasoning about the disposition, the event is the fountain pen floating in water, but when reasoning about my ability to demonstrate the event is the demonstration --- a series of measurements and calculations.
\end{note}

\begin{note}[Diverge]
  Now to turn to \EAS{}.

  If I have the ability, then it follows that the fountain floats in water.
  As noted above, it is not possible for me to demonstrate something that is not the case.

  Claim support for the proposition that the fountain floats in water.

  Still, disposition, fountain pen is not floating in water.
  Likewise with respect to ability, I have not demonstrated that the fountain pen floats in water.
  I noted various things, but did not piece these together into a demonstration.

  Yet, in claiming support, there's the event of demonstrating.
  And, so I appeal to those premises and steps I would use in the event.
  This is \EAS{}.

  Appeal to what happens in the event.
  And, reasoning to claim possible event is viewed in terms of ensuring that the resources are available.
  I have not used the relevant premises and steps of reasoning, nor am I clear on the specific form they will take.
  Still, they are available.

  Final point of interest, then.
  In both cases, there's an appeal to an event.
  If \EAS{} holds with respect to ability, does something similar hold with respect to dispositions?

  First, important clarification.
  The reasoning outlined for disposition was claiming support for event.
  Here, no clear issue with \ESU{}.
  Similarly, no clear issue with \ESU{} with respect to claiming support for having an ability.
  Tension with \ESU{} arises when using ability as a premise in further reasoning.

  Second, key divergence.
  Conclusion obtained is something that is true independent of ability.
  Unclear to me whether similar reasoning with dispositions.
  For, ability is about an event involving the agent.

  In addition, there is no issue with supposing that the agent reasons with (and hence uses) to all the relevant features of the event.\nolinebreak
  \footnote{There may me details of reasoning that one is not easily able to express, but it doesn't follow that those details are not used.}
  Ability is in part interesting because it is clear that an agent does not witness the relevant event.
  This is not to say that a variant of \EAS{} does not hold with respect to dispositions.
  Rather, I am expressing
  \begin{enumerate*}
  \item hesitancy that there are comparable entailments, and
  \item concern that there is no clear argumentative path.
  \end{enumerate*}

  There is a related question about the ability of other agents.
  Here, \EAS{} does not entail.
  In turn, one may conjecture that reasoning from one's own ability is similar.
  I find this plausible.
  It is important to stress again that \EAS{} expresses a way in which an agent may claim support.
  Hence, \EAS{} is compatible with there being other ways in which an agent may claim support.
  It may be the case that the same holds with respect to other agents.\nolinebreak
  \footnote{
    For example, \citeauthor{Owens:2006tw} argues for a belief expression model of assertion in which the rationality of a belief formed by an agent on the basis of testimony depends whatever justification the speaker has for the relevant propositional content.
    \begin{quote}
      Trusting an expression of belief by accepting what a speaker says involves entering a state of mind which gets its rationality from the rationality of the belief expressed. This state's rationality depends on the speaker's justification for the belief he expresses, not on his justification for the action of expressing it. And to hear a speaker as making a sincere assertion, as expressing a belief, is \emph{ceteris paribus} to feel able to tap into \emph{that} justification (whether or not his assertion was directed at you) by accepting what he says.\nolinebreak
      \mbox{}\hfill\mbox{(\Citeyear[123]{Owens:2006tw})}
    \end{quote}
    \color{red} Some more
  }
  However, this is not an immediate consequence.
  \EAS{} permits exceptions to \ESU{}, but it does not require all instances of reasoning with ability is an exception to \ESU{}.
  And, our focus will be on cases in which an agent reasons about their own ability to reason.
  The weak quantifier `there are cases' is designed to leave such issues open.
\end{note}

\begin{note}[Concluding parallel]
  To summarise.
  \begin{itemize}
  \item Parallel between analysis of dispositions and abilities.
  \item Event in analysis of both.
  \item Reason about event.
  \item Motivation for \EAS{} by considering reason to and from event.
  \item This doesn't provide anything close to a clear theoretical account of the reasoning performed if \EAS{} is true, but it does hint at such at how developing such an account may be approached.
  \item Now turn to related conclusion.
  \item In turn, fill in some details on the account.
  \end{itemize}
\end{note}

\subsubsection{Enthymematic inferences}

\begin{note}[\citeauthor{Moretti:2019wx}]
  Above we considered how various account of the basing relation seem to imply \ESU{}.
  Roughly, because such accounts of the basing relation required a premise or step of reasoning to be used in order to be a candidate member of the base of some conclusion of reasoning --- motivated by either causal and representational considerations.
  In contrast, \citeauthor{Moretti:2019wx} argue for an account of the basing relation which does not entail \ESU{}.

  In our terminology, \citeauthor{Moretti:2019wx} argue that: A belief held by an agent may be \emph{based} on premises that the agent did not use when forming the belief.

  The following is a fragment of the general principle relating propositional justification to well-grounded belief (alternatively doxastistcally justified belief) containing the two clauses of interest:

  \begin{quote}
    IF

    \dots

    OR

    \begin{enumerate*}[label=(\arabic*.2\(^{\ast}\))]
    \item\label{LT:1.2} Q is propositionally justified for S in virtue of P1, P2, \(\dots\), Pn being justifiedly true from her perspective because S justifiedly believes P1, P2, \(\dots\), Pn, and in virtue of her being aware that Q is an inductive or deductive consequence of P1, P2, \(\dots\), Pn jointly, and
    \item\label{LT:2.2} S carries out a \emph{plain} inference from P1, P2, \(\dots\), Pn to Q.
    \end{enumerate*}

    OR

    \begin{enumerate*}[label=(\arabic*.3), ref=(\arabic*.3)]
    \item\label{LT:1.3} Q is propositionally justified for S in virtue of P1, P2, \(\dots\), Pn being justifiedly true from her perspective, though S doesn't believe at least some P1, P2, \(\dots\), Pn, and in virtue of S being aware that Q is an inductive or deductive consequence of P1, P2, \(\dots\), Pn jointly, and
    \item\label{LT:2.3} S carries out a (fully or partly) \emph{enthymematic inference} from P1, P2, \(\dots\), Pn to Q.
    \end{enumerate*}

    THEN
    \begin{enumerate}[label=(3)]
    \item S's belief that Q is well-grounded.\nolinebreak
      \mbox{}\hfill\mbox{(\Citeyear[87]{Moretti:2019wx})}
    \end{enumerate}
  \end{quote}

  The `plain' inference of~\ref{LT:1.2} and~\ref{LT:2.2} corresponds to cases in which an agent uses P1, P2, \(\dots\), Pn to reason to Q.
  By contrast, the `enthymematic' inference of~\ref{LT:1.3} and~\ref{LT:2.3} involves reasoning in which an agent does not use some or all of P1, P2, \(\dots\), Pn to reason to Q as the agent does not believe some of P1, P2, \(\dots\), Pn (though the agent has propositional support for each of P1, P2, \(\dots\), Pn).

  To illustrate the distinction between `plain' and `enthymematic' inferences (\Citeyear[Cf.][85]{Moretti:2019wx}) consider reasoning from the premise that \nagent{5} is shorter than \nagent{6} to the conclusion that someone is taller than \nagent{5}.
  An instance of plain (non-enthymematic) may take the intermediary step that \nagent{6} is taller than \nagent{5} before abstracting from \nagent{6}.
  In contrast, an instance of enthymematic reasoning consists of the (single) premise and conclusion noted without forming the belief that \nagent{6} is taller than \nagent{5}.\nolinebreak
  \footnote{Cf.\ (\Citeyear[87--89]{Moretti:2019wx}) for examples given by \citeauthor{Moretti:2019wx}.}

  The key idea is that if an agent reasons enthymematically, then the agent's belief may be based on those premises that the agent would use in the corresponding plain inference.
  (\Citeyear[Cf.][86--87]{Moretti:2019wx})
  Hence, we have a proposal on which an agent's belief may be supported by premises and steps of reasoning that an agent has not used.
  And, in addition, because S carries out a (fully or partly) enthymematic inference \ref{LT:2.3}, it seems S \emph{may} appeal to P1, P2, \(\dots\), Pn when reasoning to Q, in conflict with \ESU{}.

  Whether or not \citeauthor{Moretti:2019wx}'s account is correct is not of interest.
  Rather, \emph{grating} that \citeauthor{Moretti:2019wx}'s account is correct allows us to make two (related) observations.
  First, \citeauthor{Moretti:2019wx} account does not conflict with \ESU{} and so the account does not require \EAS{} to be true.
  And, second, how \citeauthor{Moretti:2019wx}'s account suggests a broader theoretical account of \EAS{}.
\end{note}

\begin{note}[First point]
  To establish the first point we require further details about how \citeauthor{Moretti:2019wx} define a (fully or partly) enthymematic inference.
  The following quote combines the relevant definitions:
  \begin{quote}
    \textbf{(}[\textbf{Partly}/\textbf{Fully}] \textbf{Enthymematic Inference)}

    S carries out a [\emph{partly}/\emph{fully}] \emph{enthymematic} inference from P1, P2, \(\dots\), Pn to Q if and only if
    \begin{enumerate}[label=(\alph*), ref=(\alph*)]
       \setcounter{enumi}{1}
    \item \emph{S doesn't actually believe} [\emph{at least some of the premises}/\emph{any of}] P1, P2, \(\dots\), Pn, though some constituents M1, M2, \(\dots\), Mm of S's perspective cause in S the \emph{disposition} to believe P1, P2, \(\dots\), Pn, and
    \item M1, M2, \(\dots\), Mm [together with the premises believed by S jointly/jointly] cause S's belief that Q through a process that is shaped by S's taking Q to be a consequence of P1, P2, \(\dots\), Pn at a personal level.\nolinebreak
      \mbox{}\hfill\mbox{(\Citeyear[85]{Moretti:2019wx})}
    \end{enumerate}
  \end{quote}

  In short, an enthymematic inference involves reasoning with premises M1, M2, \(\dots\), Mm which are related to the premises P1, P2, \(\dots\), Pn of some corresponding plain inference.
  In order to complete the definition, we require an account of what it is for S to take Q to be a consequence of P1, P2, \(\dots\), Pn at a personal level:

  \begin{quote}
    \textbf{(Personal Level\(^{\ast}\))}

    S's mental states M1, M2, \(\dots\), Mm and any premises believed by S, among P1, P2, \(\dots\), Pn, jointly cause S's belief that Q through a process shaped by S's taking Q to be a consequence of P1, P2, \(\dots\), Pn at a personal level if and only if M1, M2, \(\dots\), Mm and any premise believed by S, among P1, P2, \(\dots\), Pn, jointly cause S to believe Q and S would adduce the reasons that P1, P2, \(\dots\), Pn and that Q is a consequence of P1, P2, \(\dots\), Pn in response to a request to explain why she believes Q.\nolinebreak
    \mbox{}\hfill\mbox{(\Citeyear[85--86]{Moretti:2019wx})}
  \end{quote}

  So, loosely reconstructed an enthymematic inference involves constituents M1, M2, \(\dots\), Mm of S's perspective which ensure that S has the disposition to believe P1, P2, \(\dots\), Pn.
  And, the way in which M1, M2, \(\dots\), Mm lead to S forming the belief that Q allow S to explain that they believe Q on the basis of P1, P2, \(\dots\), Pn.
  In short, an enthymematic inference is an inference in which may be \emph{post hoc} expanded to some corresponding plain inference (in part) because performing the enthymematic inference requires the agent to be disposed to believe the required premises of the corresponding plain inference.
  And, as such the premises of the corresponding plain inference may be considered as (constitutive of) the basis of S's belief that Q.

  In contrast, \ESU{} concerns the way in which M1, M2, \(\dots\), Mm lead to S forming the belief that Q do not necessarily require the agent to appeal to P1, P2, \(\dots\), Pn.
  It is consistent with \citeauthor{Moretti:2019wx} account that the reasoning from M1, M2, \(\dots\), Mm to Q may only appeal to premises and steps of reasoning used.
  That Q may be based on P1, P2, \(\dots\), Pn is due to the requirement that S is disposed to believe P1, P2, \(\dots\), Pn and the possibility of S retroactively appealing to Q being a consequence of P1, P2, \(\dots\), Pn.
  Hence, the account does not conflict with \ESU{}, and in turn does not require \EAS{} to be true.

  The insight offered is that there does not necessarily need to be a structure preserving mapping between premises and steps providing propositional support for a belief and the premises and steps appealed to when forming the belief.
  However, this does not constrain what the agent appeals to when forming a belief.

  From a broader perspective, \citeauthor{Moretti:2019wx}'s proposal considers what an agent was able to do (i.e.\ reason by some plain inference) and holds that a basing relation follows but is silent of the way in which an agent claim support.
  In contrast, \EAS{} looks at what an agent is able to do, and holds that a way of claiming support follows, but is silent on issues concerning the basing relation.
\end{note}

\begin{note}[Second point]
  Still, this broader perspective together with the above discussion of dispositions suggests a way to understand \EAS{}.
  For, one may hold that if an agent has the ability to reason to some conclusion, then the agent is disposed to use relevant premises and steps of reasoning to reason to the conclusion.
  In parallel to \citeauthor{Moretti:2019wx}, then, one may hold that the agent has the ability to reason to some conclusion if (and only if) they are suitably related to some collection of relevant premises and steps of reasoning.
  In turn, the agent may appeal to those premises and steps of reasoning to claim support for the conclusion.
  Indeed, if we adopt a parallel understanding of the basing relation, then it follows (so long as the agent has the ability) that  the agent has sufficient propositional support for the conclusion, and may be well-grounded.
  The (possible) event of reasoning to the conclusion is important both for establishing that the agent has the ability and for determining which premises and steps of reasoning the agent appeals to, but the event is not important for determining that the relevant premises and steps of reasoning are available to the agent.

  This suggestion falls far short of a theory satisfying \EAS{}, I suspect \EAS{} may be motivated in part by distinguishing between what occurs in the event of reasoning, and sufficient resources required for such an event to occur.
  An event of reasoning will always make use of sufficient resources for the event to occur, but an agent may have sufficient resources for the event to occur even if the event does not occur.
  (Specific) abilities, then, fix an particular event and determine sufficient resources and the agent does not need to witness the event in order to appeal to those resources.

  Or perhaps not.
\end{note}

\begin{note}[Segue]
  Our goal is to establish that an adequate account of reasoning which extends to ability must satisfy \EAS{}.
  This goal does not require the above suggestion to be on the right track, nor does this goal require that there is a unique theory that satisfies \EAS{}.
  For now, we close the present section with a few remarks concerning ability and~\EAS{}.
\end{note}

\begin{note}[Actual support]
  As with~\ESU{}, \EAS{} does not entail that the agent \emph{has} support.
  Our focus is on reasoning, and as argued above, it seems the issue of whether an agent has support is distinct from whether an agent may claim support.
  Claiming support is the result of some reasoning, and whether or not an agent has support requires an evaluation of that reasoning.
  This means that, strictly speaking,~\EAS{} does not carry any implications regarding whether or not the agent has support by claiming support in line with~\EAS{}.
  It is possible that the agent would fail to establish support, or establishes a support relation other than between the conclusion and the premises and steps of reasoning appealed to.

  Still, it take it to be plausible that support traces a successful claim.
  From this perspective,~\EAS{} may seem a little more intuitive.
  Given an intuitive understanding of support, if an agent does have the ability to reason to some conclusion, then the conclusion stands in the relation of being support by certain premises and steps of reasoning, whether or not the agent witnesses their ability.
  In turn, if the agent may claim support for the having the relevant ability then the agent may claim support for the conclusion from the premises and steps that would be used to witness their ability.
  For, witnessing does not contribute to the relation of support between the conclusion and the relevant premises and steps --- witnessing would only clarify to the agent the specifics of the relation.

  Of course, the agent may be mistaken or misled about having ability.
  For example, the relevant premises and steps may fail to establish the conclusion, or the agent may not have sufficient resources to carry out reasoning from the premises and steps, etc.
  In turn, witnessing may be expected to highlight that the claimed support for having the ability is mistaken or misled.

  Two points:
  \begin{itemize}
  \item Such issues are not different to being mistaken or misled and using that one has the ability as a premise, so apply to any reasoning that makes use of ability without witnessing ability.
  \item Attempting to witness the ability might reveal that the agent is mistaken or misled about having the ability does not show that the agent may not claim support for having the ability.

    Reasoning typically involves premises and steps of reasoning that could be investigated further, but this does not prevent an agent from appealing to those steps and premises.
  For example, it is (almost) to check the definition of any word used against a dictionary, and doing so might reveal that I have been mistaken or mislead about the meaning that I will convey by using the word.
  I rarely do this, though.
  Most of the time it is sufficient to expect that I am not mistaken or have not been mislead about the meaning I would convey by using the word.
  \end{itemize}
\end{note}

\subsubsection{Application}
\label{sec:application}

\begin{note}[Desire]
  Finally, while the examples of reasoning given have concluded with the truth of some proposition --- that a rectangle has some specific area, or that a given fountain pen floats in water, etc.\ --- our interest with \EAS{} is broader.

  In many cases the assigned value truth, falsity, or something in between.
  However, claiming support, and in turn; \USE{}, \ESU{}, and \EAS{} are all neutral with respect to the value assigned to the proposition.
  Therefore, we may consider other values while investigating, and as an application of \ESU{} and \EAS{}.
  In particular, consider reasoning which concludes with the desirability of some proposition.\nolinebreak
  \footnote{
    \color{red}
    Mistaken or misled.
    Yes, I think this holds up.

    Strong view on which an agent may be mistaken about desires in the same way as an agent may be mistaken about evidence.
    View on which desires are independent of representation.
    Hence, misleading or mistaken support when an agent fails to represent desire.
  }

  To illustrate this point, consider temptation.\nolinebreak
  \footnote{
    \color{red}
    Whether or not this is `genuine' temptation isn't of \dots
  }
  Specifically we will consider a slight variation on \citeauthor{Bratman:1999ac}'s `two glasses of wine' (\Citeyear[38]{Bratman:1999ac}) case of temptation.\nolinebreak
  \footnote{
    \color{red}
    See also \textcite{Bratman:2007ab}
  }
\end{note}

\begin{note}[The Pianist]
  Consider a pianist who frequently performs at a club.
  Before each performance the pianist gets nervous and has the option of drinking a glass of wine.
  A glass of wine would also lead to a worse performance.
  However, the glass of wine would help with the pianist's nerves.
  Both are learnt with some experience.
  Hence, if the pianist reasons about what to do:
  \begin{itemize}
  \item When the pianist does not feel the nerves of an upcoming performance they reason to a preference abstaining from drinking a glass of wine.
  \item Yet, when nerves are felt the pianist reasons to a preference for drinking a glass of wine.
  \end{itemize}
  The pattern is stable, and has held over many performances.

  Still, while nerves sometimes get to the pianist, they abstain from drinking a glass of wine most of the time.

  That the our pianist abstains is not necessarily surprising --- it is not uncommon to resist temptation.
  Though it is puzzling.
  The pianist's reasoning is unwavering throughout the span of time in which the pianist has the option of drinking the wine; they reason to preference for drinking a glass of wine.
  So, if the pianist abstains, the pianist acts in opposition to their preference when given the option of drinking a glass of wine, and does so purposefully.
\end{note}

\begin{note}[Reasoning and desire]
  To clarify the puzzle, let us state a basic conjecture regarding preferences and acts.

  \begin{conjecture}\label{conj:resolve-issue-act}
    Any instance of purposeful rational action performed by an agent is the result of the agent resolving the issue of how to act.
    Where:
    \begin{enumerate}
    \item An act is an candidate resolution for how to act only if the agent has claimed support for preference for some proposition and has an expectation that act would bring about the proposition.
    \item An act is an admissible resolution only if there no other candidate resolutions for which the agent has a stronger (combined) preference with respect to the proposition(s) that the agent expects to be brought about by performing the act.
    \end{enumerate}
  \end{conjecture}

  \autoref{conj:resolve-issue-act} understands rational action as the result of an agent resolving the issue of how to act --- choosing which act from a collection of options to perform.
  By understanding rational action as the result of an agent resolving the issue of how to act we may break down the reasoning involved in purposeful rational action into two steps.

  First, what makes an act a candidate resolution, and second what makes an act an admissible resolution.

  An act is a possible resolution just in case the agent links the result of acting to some proposition the agent has a preference for.

  And, an  act is an admissible resolution just in case the agent has no stronger preference for some other proposition that the agent expect could be brought about by some other candidate action.
  (Or, more generally, when an agent is uncertain about which proposition may be brought about by some act, a combined preference regarding each potential proposition.)

  In short, \autoref{conj:resolve-issue-act} is more-or-less the core of an standard decision theoretic account of maximising expected utility without commitment to particular details.\nolinebreak
  \footnote{
    \color{red}
    Cf.\ \textcite{Steele:2020tr}.
    \citeauthor{Davidson:1963aa} `Primary reason' (\Citeyear{Davidson:1963aa})
  }
\end{note}

\begin{note}[Use of conjecture]
  \autoref{conj:resolve-issue-act} fixes an understanding of purposeful rational action, and in turn establishes two ways in which the pianist may resist drinking a glass of wine:
  \begin{enumerate}
  \item Drinking the glass of wine is not a candidate resolution.
    \autoref{conj:resolve-issue-act} states a necessary condition for a candidate resolution, but further conditions may rule out possible resolutions which satisfy the necessary condition stated.
  \item Drinking the glass of wine is not an admissible resolution.
    In particular, because the pianist has a claims support for a stronger preference toward the result of abstaining.
  \end{enumerate}

  We will provide a brief argument that \ESU{} requires the former to be the case and provide an example of how further conditions may rule out possible resolutions.
  In short, \ESU{} requires an agent to witness reasoning to a conclusion in order to claim support for such a conclusion, and as the pianist reasons to a preference for having drunk a glass of wine when performing, abstaining is not an admissible resolution.
  Then, we will turn to \EAS{}, and suggest that it allows the latter to be the case while granting that the only reasoning that the pianist witnesses establishes a preference for having drunk a glass of wine when performing.
  In short, so long as the pianist may claim support for the ability to reason to a stronger preference for the result of abstaining, then by \EAS{} the agent may claim support for a stronger preference for the result of abstaining.
\end{note}

\begin{note}
  Suppose \ESU{} is true.
  By \autoref{conj:resolve-issue-act} an agent resolves an issue of how to act by determining candidate resolutions.
  In turn, a candidate resolution results from the agent claiming support for a preference toward some proposition.
  And, \ESU{} requires that an agent must witness some instance of reasoning in order to claim support for the conclusion of the instance of reasoning.

  Turning to the pianist, we have assumed that before taking to the stage the pianist reasons to preference that favours drinking a glass of wine.
  So, given \autoref{conj:resolve-issue-act} abstaining is not an admissible resolution because the agent has a stronger preference from drinking a glass of wine before taking the stage.
  And, by \ESU{} it is not possible for the pianist to claim support for a preference that would lead to abstention because the pianist must witness the relevant reasoning in order to claim support.
  Therefore, the pianist must rule out drinking a glass of wine as a candidate resolution for how to act.
\end{note}

\begin{note}[Intention and \ESU{}]
  \color{red}

  \citeauthor{Bratman:2007ab} argues that such cases may be understood through an theory on which intentions constrain reasoning.
  If the pianist intends no to drink the wine, and this intention persists, then drinking the wine is no an available conclusion of reasoning.
  Key here is that intention does not interact with the pianist's preferences.
  It remains the case that the pianist would prefer.

  The role of intention in the role of \citeauthor{Bratman:2007ab} account is to constrain possible resolutions for how to act.
  An intention to not drink prevents drinking a glass of wine from being a candidate resolution for how to act (see in particular \textcite[\S3.3]{Bratman:1987aa}).

  However, because ruled out, the pianist does not have the option of acting on that preference.
  Rather, act in a way that is compatible with intention.
  For the pianist, we may assume abstention is the only act compatible with the intention.\nolinebreak
  \footnote{
    \color{red}
    Variation.
    Block contribution of nerves.
    So, intention to not allow nerves to contribute to reasoning.
    Compatible with drinking, but given the way the scenario has been constructed, will result in not drinking.
  }
  So, \citeauthor{Bratman:2007ab} is an example of how to resolve weakness of will given relation between reasoning and action expressed by \autoref{conj:resolve-issue-act} and \ESU{}.
\end{note}

\begin{note}[Broader]
  I think, in broad strokes, phenomena fit this kind of theory.
  Abstracting from the details of any particular theory, it seems plausible that candidate resolutions to the issue of how to act are subject to conditions that extend beyond whether the agent has claimed support for preference for some proposition and has an expectation that act would bring about the proposition --- \autoref{conj:resolve-issue-act} only stipulates that the given constraint is a necessary condition for candidate resolutions.

  However, not clear to me that all phenomena fit such a theory.
  The pianist's reasoning seems distorted.
  The nerves felt before taking to the stage plausibly interfere with the pianists reasoning about candidate resolutions to the issue of how to act, and so the pianist's reasoning plausibly does not resolve the issue of how to act in line with the premises and steps of reasoning that are available to the agent.
  So much, I suspect, is intuitive.
  However, any interpretation of the pianist compatible with \ESU{} is committed to the agent resolving the issue of how to act given unreliable reasoning.
  A \citeauthor{Bratman:1987aa}-like intention would rule out drinking a glass of wine as a resolution to the issue of how to act, but whatever reasoning the agent performs given the intention is still influenced by the nerves felt.

  In the following paragraphs we will suggest that the pianist may resist the conclusion of reasoning performed given nerves because it is distorted.
  We start with two additional conjectures.
\end{note}

\begin{note}[Preferences change with reasoning]
  \begin{conjecture}\label{conj:pref-vs-reasoning}
    Whether, or to what degree, an agent claims support for a preference toward a proposition may differ from whether, or to what degree, the agent would claim support for a preference toward the proposition given varying information.
  \end{conjecture}

  Loosely paraphrased,~\autoref{conj:pref-vs-reasoning} states that preferences an agent reasons to are subject to change given change in the information that the agent reasons from.
  Hence, it seems~\autoref{conj:pref-vs-reasoning} may be considered a truism rather than a conjecture.
  Indeed, varying information is a common component in the construction of cyclical preferences (cf.\ \cite{Sobel:1997wt},\cite{Schumm:1987wx},\cite{Davidson:1955wo}, etc.)

  To illustrate, we consider a case in which difference arises from information that does not contribute to the agent's preferential evaluation of the proposition.

  Suppose an agent has a established preference for meeting person who is Black Panther over meeting the person who is Storm by reasoning.
  However, the agent is not aware that Black Panther is T'Challa nor that Storm is Ororo Monroe.
  Indeed, the agent does not have any information about the referent of `T'Challa' or `Ororo Monroe' and so does establish a preference for meeting T'Challa or Ororo Monroe by reasoning (nor vice-versa).

  Still, it seems that if the agent were provided with the information that Black Panther is T'Challa and that Storm is Ororo Monroe then the agent would reason to a preference for meeting T'Challa or Ororo Monroe.
  And, such information would not contribute to the agent's preferential evaluation of the relevant propositions.
  The agent's preferences for the relevant propositions are determined by considerations that are independent of the terms used to refer to the relevant individuals --- e.g.\ the person who helped defeat Thanos and the person who helped defeat Magneto.

  With relation to \autoref{conj:resolve-issue-act}, whether or not an act is a possible resolution for issue of how to act may depend on what information agent reasons from.
  We now introduce a further conjecture:

  \begin{conjecture}\label{conj:more-info-is-good}
    Generally speaking: When resolving how to act, claimed support for some preference toward a proposition given more information is given greater weight than claimed support for preference toward the (same) proposition given less information.\nolinebreak
  \footnote{
    Note, `more' and `less' information are relative, and it may not be possible to compare distinct bodies of information.
    If so, \autoref{conj:more-info-is-good} does not state anything about the importance of either body of information.
    For example, an agent may have information about the subjective taste of a meal and about the nutritional value of the meal.
    Still, without a way to compare information about subjective taste to information nutritional value in an information there would be no sense in which the former could be considered to hold `more' information that the latter (or vice-versa).
    Though this is not to rule out such a comparison --- we do not place constraints on what comparisons an agent may make.
  }
\end{conjecture}

  \autoref{conj:more-info-is-good} speaks more generally than~\autoref{conj:pref-vs-reasoning} and as a result may be closer to or further from a truism depending on your point of view.
  Still, the core idea is simple:
  Claimed support for some preference toward a proposition given more information is worth more than claimed support for some preference toward a proposition given less information because more information typically increases the reduces the likelihood that the claimed support is either mistaken or misled.\nolinebreak
  \footnote{
    Cf.\ \cite{Good:1966wx} for a related idea --- though see also \cite{Bradley:2016wo}.
  }
  In other words, strength of preference in the sense of \autoref{conj:resolve-issue-act} is proportional to information used to establish preference given a fixed proposition.

  To illustrate, consider an agent resolving whether to banded or off-brand multi-vitamins.
  The agent's method of establishing a preference is to look on each container, and work through the lists of vitamins comparing whether a vitamin is included, and if so to what quantity, weighing some vitamins more heavily than others.
  As the agent works through the lists of vitamins the agent moves from less information to more.
  Still, after each vitamin on the list the agent marks a preference.
  For example, the branded multi-vitamins have 2,500 IU of vitamin A while the off-brand have 2,000 IU, so the agent's initial preference leads to purchasing the branded multi-vitamins.
  However, the branded multi-vitamins have 50mg of vitamin C while the off-brand have 60mg, and given information about vitamins A and C the agent's preference leads to purchasing the off-brand multi-vitamins.
  And so on until the agent has compared the contents of the branded and off-brand multi-vitamins.

  \autoref{conj:more-info-is-good} holds because it seems implausible that the agent could be understood as acting rationally by purchasing either multi-vitamin from a preference determined by a partial comparison between the multi-vitamins given that the agent has established a preference given a full comparison between the multi-vitamins.
\end{note}

\begin{note}[Back to the pianist]
  To summarise the two conjectures:
  \autoref{conj:pref-vs-reasoning} holds that an agent's preferences may vary with the information that the agent uses to claim support for those preference.
  And, \autoref{conj:more-info-is-good} holds, generally speaking, that claimed support for a preference from more information is given greater weight than claimed support for a preference from less information.

  Now, we return to the pianist.
  We make two observations.
  First, given \autoref{conj:pref-vs-reasoning} it may be possible to trace the change in the preference that the pianist reasons to (from abstention to a glass of wine, and vice-versa) to variation in the information that the pianist reasons with.
  In particular, it may be the case that the pianist's nerves prevent them from reasoning with information that they would otherwise reason with, hence the preference to abstain arrived at by reasoning before and after the span of time in which the pianist has the opportunity to drink a glass of wine is a preference arrived at given more information than the preference to drink a glass of wine.
  Second, given \autoref{conj:more-info-is-good} and the present interpretation, the agent may give greater weight to the preference to reasoning that results in a preference for abstention.

  In short, it may be true that pianist's nerves interfere with the reasoning they perform, and without such interference the agent would reason to abstaining from drinking a glass of wine before any given performance.\nolinebreak
  \footnote{
    \color{red}
    Plausible, though not immediate.
    To clear things up a little, consider a hangover.
    Reasoning sucks, I did not desire to miss lunch with a friend, I forgot, because of the hangover.
    However, I did desire to go to bed early, because of the hangover.
    Question is whether the nerves and the glass are like missing lunch or going to bed early.
    If the former, then reasoning is at issue.
    If the latter, then reasoning is at the issue.

    Conjecture, the former.
    No additional information from nerves.
    Still, when the nerves are present the pianist has a hard time reasoning with them.
  }

  The difficulty for the pianist is that such interference is always present;
  It is not straightforward for the pianist to give greater weight to abstaining, as the pianist does not reason to abstaining given their nerves.
  However, if the pianist may claim support for  ability to reason to abstaining, then by \EAS{} the agent may claim support for a preference for abstaining.

  Indeed, it seems the pianist may claim support for having the ability to establish a preference for abstaining when presented with the option of drinking a glass of wine.
  Two observations:
  \begin{itemize}
  \item First, the pianist has performed such reasoning many times, both before and after performances, and the result of such reasoning is stable: abstention.
  \item Second, given the assumptions made about the agent's nerves, the agent retains the relevant premises and steps of reasoning when presented with the choice to drink a glass of wine.
    The nerves felt when presented with the choice to drink a glass of wine only ensure that witnessing this ability is difficult to a degree sufficient for the agent to fail each attempt at witnessing the ability.
  \end{itemize}

  So, when presented with the choice to drink a glass of wine:
  If the  pianist may claim support for having the ability to establish a preference for abstaining, then by \EAS{} the pianist may claim support for a preference for abstaining.
  And, in turn, by \autoref{conj:more-info-is-good} the pianist may give greater weight to their preference for abstaining because the claim of support for a preference to abstain would be arrived at by taking into consideration additional information.

  Given this interpretation of the pianist, `giving into temptation' would be for the pianist to disregard the reasoning that the pianist is able to perform.
  Conversely, `resisting temptation' is for the pianist act in accordance with the preference that they would reason to given the information available to them.
  So, in contrast to accounts of temptation constrained by \ESU{} the pianist need not prevent themselves from reasoning to drinking a glass of wine.
  Rather, the pianist need only reflect on what they are able to reason to.
\end{note}

\begin{note}[Quick objection]
  There is a quick objection to consider before moving on:
  Does the agent have the ability to reason to stronger preference for the result of abstaining?
  For, it seems natural for the pianist to express that they do not have the ability to reason to a preference other than for having drunk a glass of wine when performing \emph{because} of how nerves interfere with their reasoning.

  Ability fluctuates.
  At present I claim support that I have the ability to prove that S4 is sound and complete with respect to transitive frames.
  Part of my claim is that I understand the details of Lindenbaum's Lemma.
  And, if I were to forget the details ofLindenbaum's Lemma then I would lack the ability to construct the relevant proof.
  So, I may lose the ability, but even if I do lose the ability due to forgetting the details of Lindenbaum's Lemma, I may regain the ability by revising the relevant details.

  However, there may be a difference between my loss of ability due to forgetting the details of Lindenbaum's Lemma and the pianist's nerves.
  Forgetting the details of Lindenbaum's Lemma ensures that I lack premises (or steps of reasoning) required to witness the proof.
  By contrast, it is not clear that the pianist's nerves entail that the pianist lack premises (or steps of reasoning) required to reason to stronger preference for the result of abstaining.

  Let us distinguish to ways in which an agent may be said to lack an ability to reason to some conclusion.
  First, the agent lacks sufficient resources; premises and steps of reasoning.
  Second, impediments to the agent using sufficient resources.

  From the first, I have the ability to enumerate all of the positive integers in decimal representation as I have sufficient resources to produce a decimal representation of the first positive integer and I have sufficient resources to produce a decimal representation of any successor integer.
  From the second, I am clearly bounded to enumerate only a finite collection of the positive integers given my mortality and so lack such an ability.

  The success of the quick objection relies on an ability to reason to some conclusion entailing that there are impediments to the agent using sufficient resources to reason to the conclusion.
  I suspect this entailment does not hold for the sense of ability at issue.
  Rather, I suggest that what matters is that the conclusion follows from the premises and steps of reasoning.

  Whether or not this a compelling suggestion is up to you.
  Whether or not \EAS{} holds does not depend on impediments to the agent using sufficient resources entail lack of ability.
  Though, interest in the details of ability and whether they relate to cases of temptation such as the pianist do depend on whether or not \EAS{} holds.
  Therefore, our primary focus will be on showing that \EAS{} holds.
\end{note}



%%% Local Variables:
%%% mode: latex
%%% TeX-master: "master"
%%% End: