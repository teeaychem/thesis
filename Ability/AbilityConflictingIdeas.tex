\chapter{Claiming support, `use', and two conflicting ideas}
\label{cha:claiming-support-use}

\begin{note}
  Claiming support, seen in previous chapter.
  Basic requirement with respect to `defeaters'.
  Now turn to an intuitive sufficient idea, granting necessary.
  Then, state an idea of a necessary condition.
  And, conflicting sufficient condition.

  Arguing against the idea for a necessary is the rest of this paper.
\end{note}


\section{Claimed support and `use'}
\label{sec:claimed-support-use}

\begin{note}[Sufficient condition]
  We start with a sufficient condition for claiming support:
  \begin{restatable}[\USE{0} --- \USE{}]{idea}{ideaUSE}\label{prem:bP}\label{prop:USE}
    If agent appeals to premises such that premises are part of reasoning, then the instance of reasoning may be an instance of claiming support.
  \end{restatable}

  \USE{} expresses an intuitive idea.

  In short, an agent may claim support when there are premises available to the agent, and the agent explicitly draws on those premises to claim support for the conclusion.

  Main argument of the thesis is about what is necessary for claiming support.

  We express \USE{} as an idea, rather than as assumption as we will not require \USE{}.
\end{note}

\begin{note}
  Two \illu{1}
\end{note}

\begin{note}[Illustration of \USE{}]
  \begin{illustration}
    \label{ill:rectangle:basic}
    \mbox{}
    \vspace{-\baselineskip}
    \begin{enumerate}
    \item The length of this rectangle measures \(19\text{cm}\) and the breadth of this rectangle measures \(7\text{cm}\).
    \item So, the length of this rectangle is \(19\text{cm}\) and the breadth of this rectangle is \(7\text{cm}\).
    \item
      It is possible that my measuring device in inaccurate, but I purchased it from a reliable hardware store.
    \item
      To calculate the area of a rectangle, one multiples the length of the rectangle by breadth.
    \item
      \label{ill:rectangle:basic:reasoning}
      \(19\text{cm}\) multiplied by \(7\text{cm}\) is \(133\text{cm}^{2}\).
    \item
      So, the area of the rectangle is \(133\text{cm}^{2}\).
    \end{enumerate}
    \vspace{-\baselineskip}
  \end{illustration}
\end{note}

\begin{note}
  \begin{illustration}[Waiting for a shop to open]
    \label{ill:waiting-for-shop}
    \mbox{}
    \vspace{-\baselineskip}
    \begin{enumerate}
    \item The sign in the door says that the shop will open at 10a.
    \item So, the shop will open at 10a.
    \item\label{ill:waiting-for-shop:reasoning} Of course, the sign may not be accurate, but it looks as though the store is well kept, so even if the sign is inaccurate, it is sufficient to indicate that the shop will open at 10a.
    \item\label{ill:waiting-for-shop:current-time} It is currently 9:45a.
    \item If 30 minutes pass, it will be 10:15a.
    \item 10:15a is after 10a.
    \item\label{ill:waiting-for-shop:return} So, if I return in 30m, then the shop will be open.
    \end{enumerate}
    \vspace{-\baselineskip}
  \end{illustration}
\end{note}

\begin{note}
  Both \illu{1} involves reasoning that intuitively amounts to claiming support.
  In particular, the third step of each \illu{0} involves reasoning about \requ{1} --- that the measure is accurate, and that the sign in the shop is truthful.
  Important from the perspective of \USE{}, however, is that the reasoning of both \illu{1} does not appeal to any premises other than those which form part of the instance of reasoning.

  In \autoref{ill:waiting-for-shop}, the agent's reasoning explicitly includes premises concerning the length and breadth of a rectangle and a statement of how to calculate the area of a rectangle given the rectangle's length and breadth.
  And, in \autoref{ill:waiting-for-shop} the agent reasons through premises concerning when the shop will open, what time it currently is, and what the result of some interval of time passing would be.

  Of course, one may hold that some of the steps in \autoref{ill:waiting-for-shop} and \autoref{ill:waiting-for-shop} are redundant.
  For example, it may be possible for an agent may move directly from \ref{ill:waiting-for-shop:current-time} to \ref{ill:waiting-for-shop:return}.

  Or, conversely, that steps in \autoref{ill:waiting-for-shop} and \autoref{ill:waiting-for-shop} are omitted.
  For example, it may be that \ref{ill:rectangle:basic:reasoning} should be expanded to state the operation of abstracting from \(\text{cm}\), multiplying \(19\) by \(7\), and applying \(\text{cm}^{2}\) to \(133\).

  However you may wish to revise these \illu{1}, the motivation behind \USE{} is hopefully clear:
  If appealing to a collection of premises is sufficient for claiming support for some conclusion, then so long as \ideaS{} and \ideaCS{} hold with respect to the instance of reasoning, the agent will succeed in claiming support.
\end{note}

\begin{note}
  In addition, though the \illu{1} of {\color{red} some section} fail to be instances of claiming support, the failure is not due to the agent appealing to premises that are not part of their reasoning --- rather, the failure is due to agent failing to reason about some \requ{}.
\end{note}

\begin{note}
  Further, though other \illu{1} that involve are not necessarily instances of claiming support, a variant of \USE{} which replaces `claiming support' and ideas \ideaS{} and \ideaCS{} with some other attitude and constraints.
\end{note}

\begin{note}[`Use']
  `Use'
\end{note}

\section{Two (conflicting) ideas as to claiming support}
\label{sec:inter-with-claim}

\begin{note}
  In this section we introduce two propositions which characterise what we are arguing against and what we are arguing for.
  \ESU{0} and \EAS{0}, respectively.
  Argue against \ESU{} by cases involving ability.
  Argue for \EAS{} which outlines the way in which ability conflicts with \ESU{}.

  Start with introduction of \ESU{}.
  And, motivate with reference to literature on the basing relation and rationality as responding to reasons.

  Move to \EAS{}, clarify relation to \ESU{} and contrast to related principle argued for by \citeauthor{Moretti:2019wx}.
\end{note}

\subsection{\ESU{0} --- \ESU{}}
\label{sec:esu}

\begin{note}[Recap of \USE{}]
  Brief recap of \USE{}.
  Introduced the idea, and then expanded on the details.
\end{note}

\begin{note}[Focus]
  We will argue against:

  \targetESU*

  In other words, only an instance of claiming support \emph{only} if agent has used the premises that they appeal to.
\end{note}

\begin{note}
  \ESU{} is equivalent to the converse of~\USE{}:\nolinebreak
  \footnote{
    Let:
    \(\phi\) {\color{red}???????}


    Note the logical form of \USE{} is \((\phi \text{ and } \psi) \rightarrow \xi\), the logical form of \ESU{} is \(\xi \rightarrow \psi\), and the logical form of \autoref{idea:ESUasUSE} is \(\xi \rightarrow (\phi \text{ and } \psi)\).
    So, \autoref{idea:ESUasUSE} is the converse of \USE{}.

    To establish equivalence between \ESU{} and \autoref{idea:ESUasUSE} we break this down into two directions:

    First, the implication from \autoref{idea:ESUasUSE} to \ESU{} is immediate.
    For, \(\phi\) implies \(\xi\) whenever \(\phi\) implies \(\psi \text{ and } \xi\).

    However, the implication from \ESU{} to \autoref{idea:ESUasUSE} is a little more complex.
    If \(\xi\) implies \(\phi\) and \(\xi\) implies \(\psi\), then \(\xi\) implies \(\phi \text{ and } \psi\).
    \ESU{} immediately ensures \(\xi\) implies \(\psi\).
    And, to observe that \(\xi\) implies \(\phi\) observe that an instance of reasoning is an instance of claiming support only if \ideaS{} and \ideaCS{} hold.
  }

  \begin{restatable}[]{idea}{ideaESUasUSE}
    \label{idea:ESUasUSE}
    An instance of reasoning is an instance of claiming support, \emph{only if}, \ideaS{} and \ideaCS{} hold and agent appeals to premises such that premises are part of reasoning.
  \end{restatable}

\end{note}

\begin{note}
  \USE{} includes assumption that \ideaS{} and \ideaCS{} hold with respect to the relevant instance of reasoning.
  However, as \ESU{} expresses a necessary, rather than sufficient, condition on claiming support, such assumptions are optional.
\end{note}

\begin{note}
  Focus is with whether an agent is required to have \emph{used} something in order to appeal to that thing when claiming support.
  No fixed understanding of `use' is assumed in the statement of~\ESU{}, and we will offer some disambiguation below.
\end{note}

\begin{note}
  Broad motivation for \ESU{} is somewhat complex.
  At issue is not instances of straightforwardly problematic reasoning.

  To illustrate, consider variant of illustration provided for~\USE{}.

  A lucky guess that the area of the rectangle is \(133\text{cm}^{2}\) would not allow the agent to claim support that the area of the rectangle is \(133\text{cm}^{2}\).
  However, the luck guess is part of the agent's reasoning.
  So, failure comes down to \ideaS{} or \ideaCS{}.
  Plausibly \ideaCS{}.
\end{note}

\begin{note}[Illustration]
  One way to generalise is to view claiming support as an instance of basing.
  So, doesn't get to base on the dimensions of the rectangle, the agent's understanding of how to calculate the area of a rectangle, and the relevant mathematics.

  And, it seems the agent is not in a position to base their lucky guess in such a way because the agent did not reason from the dimensions of the rectangle, the agent's understanding of how to calculate the area of a rectangle, and the relevant mathematics

  I think this is a plausible variant.
  Argument will also go against such an idea.

  However, stick with reasoning.

  Still, variant of this.

  Moving to another agent, observe doing the work, get report.
  Easy to resist, by adding in additional premise.
  Still, no presupposing that this needs to be done.

  That the other agent performed the calculation.
  However, what the other agent appealed to when reasoning, the agent does not get to appeal to \emph{that}.

  If the agent has not performed the calculation, then the agent may not appeal to the use of the calculation when claiming support --- rather, the agent mentions that the calculation is true.\nolinebreak
  \footnote{
    Slight weakening of~\ESU{} may be made.
    So long as \emph{some} agent has performed the calculation.
    Argue against~\ESU{}, and the argument made will hold for this weakening.
  }
\end{note}

\begin{note}
  To illustrate, claim support that 173 is prime.
  It's possible that I did the prime factorisation, and possible that I took that representation to be part of the reason why I claim that 173 is prime.
  However, represented query of whether prime to wolfram alpha as justifying, and that's why I claimed support.
  So, definitely not from okay to appeal to the reasoning I have not witnessed.
  And, if infer that 173 is prime from claimed support that I have the ability to demonstrate that 173 is prime, the same issue.
\end{note}

\begin{note}
  Three brief notes on~\ESU{}:
\end{note}

\begin{note}
  First, the `has' in~\ESU{} only requires `at some point in the past'.
  Hence,~\ESU{} does not require the agent to reason from premises to conclusion each time the agent claims support for the conclusion.
  For example, if an agent proved the Deduction Theorem for propositional logic last week, then the agent would not be in conflict with~\ESU{} if they claimed support for the Deduction Theorem on the basis of the premises and reasoning they performed in the past.
\end{note}

\begin{note}
  Second, and following from the first,~\ESU{} will also hold for any stronger statement --- for example if `has' is read as `has just'.
  For example, requiring that the agent's memory of proving the Deduction Theorem allows the agent to claim support, rather than the premises and steps used in the past.
  The argument (stated below) denies that, given certain information, the agent needs witnesses any reasoning in order to claim support for the result of witnessing the reasoning.
\end{note}

\begin{note}
  Third, as~\ESU{} is about when an agent may \emph{claim} support, it is compatible with~\ESU{} to hold that the agent \emph{has} support --- regardless of whether the agent has witnessing the reasoning.
\end{note}



\subsubsection{Intuition}

\begin{note}[Intuition]
  \ESU{} seems quite plausible, at least to me.
  The proposition is a careful statement of an intuitive ideas:

  Whether or not an agent claims support is the result of the structure of the reasoning process, and if some premises or step is not used, then it is irrelevant to the structure of the process.
  Hence, the only premises and steps of interest when claiming support are those used in the reasoning process.

  Claiming support is the result of some process, and the result of an process is explained by the constituents of the process.\nolinebreak
  \footnote{
    Ah, the homonculus.

    Question about whether the agent is important.

    This gets difficult.

    Consider clocks.
    Clock does not keep track of time.
    Rather, mechanical system designed to change in constant with some passage of time. (Cf.\ \textcite{Smith:1988aa}.)

    Agent may be like this.
    Distinction is intentionality.
    When I go about keeping track of the time, I'm attempting (at least typically) to maintain reference to what the time is.
    Figure out a way to approximate a second, and that's what's happening.
    Approximation.
    If it is noted that I requarly sigh every minute, use this, but I wouldn't be keep tracking of time, though you may be using regularity to do so.
    So, in the former case, using understanding of time, while in the latter not doing so.
  }

  As~\ESU{} is restricted to an agent claiming support, things seem a little easier.
  Problems with interpretation, however.
  Transparency.
  Familiar, if debatable, \illu{0}.
  Freud.
  (Here, adjourning the meeting by saying something mistaken.)
\end{note}

\begin{note}[Analogy]
  By analogy, whether or not my mug of (once cold) coffee overheats in the microwave is the result of some process involving electromagnetic radiation.
  My desire that the mug of coffee does not overheat is not used as part of the process of heating the coffee, and so is irrelevant to the structure of the process.

  My desire may explain why the mug of coffee is taking part in a certain process, and an unused premise or step may explain why an agent performed so reasoning.
  Still, a premise or step must be used as part of the process of reasoning to stand in explanation for the result of reasoning.

  Press the analogy further: Reasoning is a causal process.
  And, any property of reasoning reduces to cause and effect.
  If premises or steps are not used, then those premises or steps stands outside the relevant causal trace, and may not be appealed to when accounting for some structural property of the conclusion of the instance of reasoning (here, that the agent claims support for the conclusion).
\end{note}

\subsubsection{\ESU{} from the literature}

\begin{note}
  Our goal is to motivate \ESU{}.
  Strictly speaking, \ESU{} is about instances of reasoning which amount to claiming support.
  However, we will motivate \ESU{} by accounts of what reasoning involves.

  In other words, motivate \ESU{} by considerations about reasoning, rather than considerations about claiming support.

  Two broad ideas.
  \begin{enumerate}
  \item Causal accounts of reasoning.
  \item Something like \citeauthor{Boghossian:2014aa}'s Taking Condition.
  \end{enumerate}

  We being with causation in \autoref{sec:motivating-ESU:causation}.
  Here, argue for entailment.
  Specifically, causal accounts of reasoning imply \ESU{} with respect to what the accounts classify as reasoning.

  Strength of this motivation is entailment and prevalence of causal accounts.

  However, weakness is causation alone does not distinguish reasoning from other activities.
  We being \autoref{sec:motivating-ESU:beyond-causation} by motivating the previous point.
  And, then consider how various proposals for what is distinctive about reasoning motivate \ESU{}.
  The strength of such motivation is that it follows from considerations about what is distinctive about reasoning.
  The weakness is that we will not necessarily obtain anything as strong as entailment.
  Indeed, there may be some interest as to whether the rejection of \ESU{} is compatible with the accounts of reasoning considered.
\end{note}

\begin{note}[Qualification]
  However, an important qualification.

  We haven't said much about what reasoning is.
  And, without fixing an account of what reasoning is, there is no guarantee that any given account of reasoning will strictly entail \ESU{}.
  For, it may be that `reasoning' as mentioned in \ESU{} has broader scope than `reasoning' as mentioned in some theory of reasoning cited in motivation for \ESU{}, or conversely.
  In other words, only entailment if equivocation, and I do not wish to suggest that `reasoning' should be equivocated.

  Still, to the extent that \ESU{} and each account of reasoning concern some phenomena, and that each account of reasoning \emph{at least} concerns a phenomena closely related to the other accounts of reasoning, we suggest motivation for \ESU{} from closely related phenomena should motivate \ESU{}.\nolinebreak
  \footnote{
    One may think this is overly cautious.
    So, let me expand a little with reference to causal accounts of reasoning.

    The overall goal is to provide an argument against \ESU{}.
    If a causal account of reasoning implies \ESU{}, then the argument against \ESU{} will also be an argument against a causal account of reasoning.
    However, the argument relies on there being certain instances of reasoning that motivate rejecting \ESU{}.
    And, while one may reject that such purported instances of reasoning really are reasoning, one may also reject that such purported instances of reasoning are instances of reasoning to which one's favoured causal account applies.
    Hence, a rejection of \ESU{} may be compatible with a causal account of reasoning to the extent that \ESU{} fails for a broader sense of `reasoning'.

    Indeed, causal accounts of reasoning may distinguish an important part of a broader phenomena, and therefore it may be that rejecting \ESU{} (and hence causation) should merely amount to limiting the scope of such accounts.
  }
\end{note}

\paragraph{Causation}
\label{sec:motivating-ESU:causation}

\begin{note}
  We start with the easy case of causation.
  Motivation is simple.
  A causal account of reasoning implies \ESU{}.

  Causal accounts of reasoning are common.
  So, such causal accounts of reasoning imply \ESU{} with respect to what the accounts classify as reasoning.

  Note the somewhat cautious phrasing.
\end{note}

\begin{note}
  We being with~\cite{Armstrong:1968vh}.
  \begin{quote}
    We are not concerned here with logicians' questions about inference, but solely with the psychological process of inferring.
    The primary sense of the word is that in which it involves acquiring a belief on the basis of a belief already held.

    \mbox{}\hfill\(\vdots\)\hfill\mbox{}

    \dots to say that A infers \emph{p} from \emph{q} is simply to say that A's believing \emph{q} \emph{causes} him to acquire the belief \emph{p}.
    And the sense of `cause' employed here is the common or billiardball sense of `cause', whatever that sense is.\nolinebreak
    \mbox{}\hfill\mbox{(\citeyear[194]{Armstrong:1968vh})}
  \end{quote}
  \cite{Armstrong:1968vh} goes on to consider some issues, but these reflect the `simplicity' of saying that inference is a matter of causation between beliefs, rather than causation.
  `With these qualifications, it seems that our causal account of inferring can stand.'
  (\citeyear[197]{Armstrong:1968vh})

  Of course, \citeauthor{Armstrong:1968vh} talks about `inference' rather than `reasoning', and in doing so restricts the scope of the proposal from arbitrary ways in which a proposition may be evaluated.

  However, restricted implies \ESU{}.
  For, cause, then part.
\end{note}

\begin{note}
  The following two passages from \citeauthor{Wedgwood:2006ui} and \citeauthor{Broome:2013aa} (respectively) explicitly cast reasoning as a causal process:

  \begin{quote}
    Reasoning is a causal process, in which one mental event (say, one's accepting the conclusion of a certain argument) is caused by an antecedent mental event (say, one's considering the premises of the argument).\nolinebreak
    \mbox{}\hfill\mbox{(\cite[660]{Wedgwood:2006ui})}
  \end{quote}

  \begin{quote}
    So far as I can see, then, no further conditions need be added.
    I have arrived at necessary and sufficient conditions for a process to be active reasoning.
    Active reasoning is a particular sort of process by which conscious premise-attitudes cause you to acquire a conclusion-attitude.
    The process is that you operate on the contents of your premise-attitudes following a rule, to construct the conclusion, which is the content of a new attitude of yours that you acquire in the process.\nolinebreak
    \mbox{}\hfill\mbox{(\cite[234]{Broome:2013aa})}
  \end{quote}
  As we appealed to nothing other than causation when arguing that (a restricted form) \ESU{} follows from \citeauthor{Armstrong:1968vh} account of inference, the same considerations apply to both \citeauthor{Wedgwood:2006ui} and \citeauthor{Broome:2013aa}'s account of reasoning.
\end{note}

\begin{note}
  Finally, consider the following passage from \citeauthor{Boghossian:2014aa}:
  \begin{quote}
    \dots the property of a person’s thinking something \emph{for a reason} is not response-dependent.
    To say that R was S's reason for A'ing implies that S took R to support his A'ing at the time that he A'ed, and that his so taking it led to his A'ing.

    I don't see how S's being disposed to \emph{say} that R was his reason for A'ing could \emph{make it the case} that he took R to support his A'ing and that this taking had a certain causal impact.
    His saying it might be very good evidence that R was his reason.
    But saying that R was his reason can’t be \emph{constitutive} of R's being his reason.
    Causation is part of the idea of R's being his reason---and causation can't be a response-dependent property.\nolinebreak
    \mbox{}\hfill\mbox{(\citeyear[10--11]{Boghossian:2014aa})}
  \end{quote}
  \citeauthor{Boghossian:2014aa} is here arguing against dispositional accounts of the Taking Condition (which we will highlight shortly).
  However, in contrast to the previous citations \citeauthor{Boghossian:2014aa} appeals not the process of reasoning\nolinebreak
  \footnote{
    Or, strictly speaking, inferring in \citeauthor{Boghossian:2014aa}'s case.
  }
  but of what it is for some to think of something for a reason.
  Following \citeauthor{Boghossian:2014aa}, appealing to some proposition \(\phi\) as a reason for \emph{V}ing that \(\psi\) implies that there was some event in which the agent appeal to \(\phi\) in order to \emph{V} that \(\psi\).
  Granting that such event is an instance of reasoning, then \(\phi\) must have had some causal impact, and to have such causal impact it seems \(\phi\) must have been part of the instance of reasoning.
\end{note}

\begin{note}
  Causation is necessary.
  So, any instance of appealing is an instance of causation.
  Hence, premise is part of agent's reasoning.
  For, if not part, then no causation.
  No way for something that is not a part to cause.

  So, implication from causation to \ESU{}.

  At best, the appeal itself has causal role.

  Possible to read causation so that rejecting \ESU{} does not require rejecting causation.
  For, it may be case that agent appeals, but still has some causal role.
  Think of the testimony cases.
  Appeal to the content.
  Possible to state that there is some causal relation.
  However, distinct from proposals considered.
  And, have not found any proposal of this kind in the literature.

  Converse does not hold.
  Need not think that being part implies causal role.
  {\color{red} this is further grouping}.
\end{note}

\paragraph{Beyond causation}
\label{sec:motivating-ESU:beyond-causation}

\begin{note}
  Causation implies \ESU{}.
  However, question the role of causation.
\end{note}

\begin{note}
  \begin{quote}
    ``Plenty of blank leaves, I see!'' the Tortoise cheerily remarked.
    ``We shall need them \emph{all}!''
    (Achilles shuddered.)
    ``Now write as I dictate:---

    \begin{enumerate}[label=(\emph{\Alph*})]
    \item Things that arc equal to the same are equal to each other.
    \item The two sides of this Triangle are things that are equal to the same.
    \item If \emph{A} and \emph{B} are true, \emph{Z} must be true.
      \setcounter{enumi}{25}
    \item The two sides of this Triangle are equal to each other.''
    \end{enumerate}

    ``You should call it \emph{D}, not \emph{Z},'' said Achilles.
    ``It comes \emph{next} to the other three.
    If you accept \emph{A} and \emph{B} and \emph{C}, you \emph{must} accept Z.''

    ``And why \emph{must} I?''

    ``Because it follows \emph{logically} from them.
    If A and B and C are true, Z \emph{must} be true.
    You don't dispute \emph{that}, I imagine?''

    ``If \emph{A} and \emph{B} and \emph{C} are true, \emph{Z} \emph{must} be true,'' the Tortoise thoughtfully repeated.
    ``That's \emph{another} Hypothetical, isn't it?
    And, if I failed to see its truth, I might accept \emph{A} and \emph{B} and \emph{C}, and \emph{still} not accept \emph{Z}, mightn't I ?''

    \mbox{}\hfill\(\vdots\)\hfill\mbox{}

    ``Then Logic would take you by the throat, and force you to do it!''
    Achilles triumphantly replied. ``Logic would tell you 'You ca'n't help yourself.
    \dots''\nolinebreak
    \mbox{}\hfill\mbox{(\citeyear[279--280]{Carroll:1895uj})}
  \end{quote}
  Achilles seems to be arguing that accepting \emph{A} and \emph{B} and \emph{C} would be sufficient.
  Tortoise denying that \emph{A} and \emph{B} and \emph{C} are sufficient.
  In particular, possible to accept \emph{A} and \emph{B} and \emph{C} without accepting \emph{Z}.

  Achilles is not citing a causal relation between [\emph{A} and \emph{B} and \emph{C}] and \emph{Z} (or, rather, \emph{D}).
  However, Achilles does seem to be citing a causal relation between accepting [\emph{A} and \emph{B} and \emph{C}] and accepting \emph{Z}.

  The Tortoise would violate some causal law.
  However, despair sets in as Achilles observes that the Tortoise is violating the proposed causal law, and hence it is no law at all.

  The issue is that though causation implies \ESU{}, have not motivated \ESU{} from perspective of whatever it is that distinguishes reasoning from any other process.
\end{note}

\begin{note}
  Consider \citeauthor{Boghossian:2014aa}'s Taking Condition:
  \begin{quote}
    (Taking Condition): Inferring necessarily involves the thinker \emph{taking} his premises to support his conclusion and drawing his conclusion because of that fact.\nolinebreak
    \mbox{}\hfill\mbox{(\citeyear[5]{Boghossian:2014aa})}
  \end{quote}
  As with \citeauthor{Armstrong:1968vh}, inference --- and hence reasoning with beliefs --- rather than reasoning more broadly (\citeyear[cf][2]{Boghossian:2014aa}).
  

  \begin{quote}
    The intuition behind the Taking Condition is that no causal process counts as inference, unless it consists in an attempt to arrive at a belief by figuring out what, in some suitably broad sense, is supported by other things one believes.

    In the relevant sense, reasoning is something we \emph{do}, not just something that happens to us.
    And it is something \emph{we} do, not just something that is done by sub-personal bits of us.
    And it is something that we do with an \emph{aim}---that of figuring out what follows or is supported by other things one believes.
    It’s hard to see how to respect these features of reasoning without something like the Taking Condition.\nolinebreak
    \mbox{}\hfill\mbox{(\citeyear[5]{Boghossian:2014aa})}
  \end{quote}

  Though the Tortoise has accepted \emph{A} and \emph{B} and \emph{C}, the Tortoise has not \emph{taken} \emph{A} and \emph{B} and \emph{C} to support \emph{Z}.

  For \citeauthor{Boghossian:2014aa}, The Taking Condition expands on a causal process.
  There are various objections to the taking condition.\nolinebreak
  \footnote{
    See, for example,~\textcite{Hlobil:2014tq},~\textcite{Wright:2014tt}, and~\textcite{McHugh:2016vp}.

    \citeauthor{Hlobil:2014tq} argues against the Taking Condition as it distracts from what accounts of reasoning out to explain, rather than arguing against the Taking Condition directly.

    \citeauthor{Wright:2014tt} denies that reasoning must involve a state which connects premises to conclusions. (\citeyear[Cf.][33-34]{Wright:2014tt})

    And, \citeauthor{McHugh:2016vp} argue against the Taking Condition as the taking in question is that of \emph{support} between premises and conclusion.
  }

  Still, for our purposes the Taking Condition is simply a concise account of a necessary condition that distinguishes reasoning from other mental processes.

  And, while \citeauthor{Boghossian:2014aa} motivates the Taking Condition as a restriction on some causal process, the `because' present in the Taking Condition may be interpreted independently of causation.
\end{note}

\begin{note}
  Now, there is no immediate entailment from the Taking Condition (or something like it) to \ESU{}.
  For, without some account of what `taking' amounts to, we have merely fixed an adjective to capture some part of some phenomena.
  However, at issue is not whether the Taking Condition (or something like it) entails \ESU{}, but whether accepting the Taking Condition (or something like it) motivates \ESU{}.
  And, it seems plausible that the Taking Condition (or something like it) does.

  For, if



  However, important that the Taking Condition (or something like it) does not entail \ESU{}.
  For, plausible constraint on reasoning.
\end{note}

\begin{note}
  Two instances from the literature:
\end{note}

\begin{note}
  \citeauthor{Thomson:1965vv} suggests a doxastic variant of the Taking Condition, where are agent reasons from \(\phi\) to \(\psi\) just in case the agent believes that \(\phi\) is a reason for \(\psi\).
  \begin{quote}
    The claim which the 'formula' of p.\ 285\nolinebreak
    \footnote{
      The `formula' in question:
      \begin{quote}
    Now reasoning should surely involve drawing a conclusion from a set of premisses.
    But you can't be said to draw the conclusion that \emph{q} from \emph{p} if for all you know in knowing that \emph{p} it would at best be a matter of luck if \emph{q} as well.
    So to ``reason'' from \emph{p} by itself to \emph{q} isn't really to be reasoning; it's like saying one thing, and then taking a chance on it that something else is also true---like taking a leap in the dark, or more prosaically, like guessing.'
    (From here on I shall refer to this as the `\emph{formula}'.)\nolinebreak
    \mbox{}\hfill\mbox{(\citeyear[285]{Thomson:1965vv})}
  \end{quote}
}
    above was to support was this:
    suppose \emph{p} does not imply \emph{q}, and suppose a man says `\emph{p}, so \emph{q}';
    then he is not reasoning in saying this unless he believes that \emph{r}, where the conjunction of \emph{p} and \emph{r} implies \emph{q}, and \emph{r} is a suppressed premiss of his reasoning.\par
     But suppose such a man believes that \emph{p} is reason for \emph{q}; would this not be enough?
    `It would if ``\emph{p} is reason for \emph{q}'' were construed as a suppressed premiss of his argument'.
    Then let us so construe it.\newline
    \mbox{}\hfill\mbox{(\citeyear[294]{Thomson:1965vv})}
  \end{quote}
  Causation is absent from \citeauthor{Thomson:1965vv}.
  Does not imply that \citeauthor{Thomson:1965vv}'s proposal is independent of causation, but motivated does not appeal to causation.
\end{note}

\begin{note}
  And more recently \cite{Valaris:2014un} has argued for an explicitly non-causal doxastic account of the taking condition.

  \begin{quote}
    Explicitly reasoning from R to p just is, in part, believing that p follows from R.

    \mbox{}\hfill\(\vdots\)\hfill\mbox{}

    There is no suggestion that the belief that p follows from R plays a causal role in one’s reasoning from R to p, and so no room to wonder about how it could possibly play that role.\nolinebreak
    \mbox{}\hfill\mbox{(\citeyear[117--118]{Valaris:2014un})}
  \end{quote}
\end{note}

\begin{note}
  And \citeauthor{Wright:2014tt} offers a similar suggestion\dots

  For example, \citeauthor{Wright:2014tt}'s `Simple Proposal' may be substituted for the Taking Condition.
  \begin{quote}
    But consider instead the proposal, not that the status of the transition as inferential depends on the thinker’s judgments about his reasons, but that it depends on \emph{what his reasons are}.
    We want his acceptance of the premises to supply his \emph{actual} reasons for accepting the conclusion.

    \mbox{}\hfill\(\vdots\)\hfill\mbox{}

    Call this the Simple Proposal.
    It says that a thinker infers q from p\(_{1}\) \(\cdots\) p\(_{\text{n}}\) when he accepts each of p\(_{1}\) \(\cdots\) p\(_{\text{n}}\), moves to accept q, and does so for the reason that he accepts p\(_{1}\) \(\cdots\) p\(_{\text{n}}\).\newline
      \mbox{}\hfill\mbox{(\Citeyear[33]{Wright:2014tt})}
    \end{quote}

    However, \citeauthor{Wright:2014tt} denies that reasoning must involve a state which connects premises to conclusions and so however \citeauthor{Wright:2014tt}'s Simple Proposal is developed, it will not inolve a doxastic state:

    \begin{quote}
      What is needed, then, is an account of, or at least some insight into, what it is for certain intentional states of a thinker to be his actual reasons for his transition to another intentional state.

      [Which avoids] committing to the notion that doing something for certain reasons must involve a state that somehow registers those reasons as reasons for what one does.\nolinebreak
      \mbox{}\hfill\mbox{(\Citeyear[34]{Wright:2014tt})}
    \end{quote}
\end{note}

\begin{note}
  Question is whether this requires that premise is part of the agent's reasoning?
  Unlike causal, there is no simple argument.
  Details on what the representation amounts to.

  Still, if agent does not consider premise, then doubt that this is possible.
\end{note}

\paragraph{Correctly responding}

\begin{note}[Responding to reasons]
  As final motivation, consider the proposal at the core of \citeauthor{Lord:2018aa}'s (\Citeyear{Lord:2018aa}) thesis that being rational is to correctly respond to reasons.

  \begin{quote}
    \textbf{Correctly Responding:} What it is for A's \(\phi\)-ing to be ex post rational is for A to possess sufficient reason S to \(\phi\) and for A's \(\phi\)-ing to be a manifestation of knowledge about how to use S as sufficient reason to \(\phi\).\nolinebreak
    \mbox{}\hfill\mbox{(\Citeyear[143]{Lord:2018aa})}
  \end{quote}

  An agent's action is rational only if the action is a manifestation of some know-how.
  \citeauthor{Lord:2018aa} summaries:

  \begin{quote}
    \dots when one manifests one's know-how, dispositions that are directly sensitive to normative facts are manifesting. Thus, the competences involved in the relevant know-how make one directly sensitive to the normative facts\nolinebreak
    \mbox{}\hfill\mbox{(\Citeyear[16]{Lord:2018aa})}
  \end{quote}

  For our purposes, following example of manifesting know-how directly relates to reasoning:

  \begin{quote}
    The most salient disposition [when appealing to \emph{p} as a reason]\nolinebreak
    \footnote{Note, \citeauthor{Lord:2018aa} (explicitly) not talking about believing that \emph{p} is a reason, but argues that the cited disposition to present both when appealing to p as a reason and believing that \emph{p} is a reason.}
    is the disposition to (competently) use \emph{p} as a premise in reasoning.\nolinebreak
    \mbox{}\hfill\mbox{(\Citeyear[25]{Lord:2018aa})}
  \end{quote}

  Hence, suppose an agent appeals to a premise of reasoning in order to claim support for some conclusion.
  Then, if the agent does not use the premise of reasoning, it seems the agent does not manifest know-how, which is required for the appeal to meet \citeauthor{Lord:2018aa}'s account of rational action.

  Of course, that the noted disposition is the most salient does not rule out alternative, less noteworthy, dispositions.
  However, it is unclear to me how to \emph{manifest} know-how without use.
  Looking ahead, it does not seem to be the case that I manifest my ability to show that a certain rule of inference is sound when skipping over details in a completeness proof.
  However, I may manifest know-how regarding the (presumed) truth of the ability attribution.

  Likewise with my ability to establish a preference for tofu over any other kind of miso when ordering soup.
\end{note}

\begin{note}[Summarising illustrations]
  Stepping back,~\ESU{} may be seen as a desiderata for any account of claiming support.

  For:
  If an agent claims support for some conclusion of reasoning, then result of reasoning.
  So, premises.

  An adequate account of claiming support must explain how the premises used permit the agent to claim support.\nolinebreak
  \footnote{
    Note, however, that this argument does not imply that support for the conclusion must be accounted for in terms of the premises and steps used by the agent to claim support.
    As we will note below, one may hold that an enthymematic argument permits an agent to claim support, while the relevant relation of support is secured by the corresponding non-enthymematic argument.
    Cf.\ \textcite{Moretti:2019wx} for suggestions along these lines.
  }
  In turn, if an agent appeals to premises and steps that they did not use, then those premises and steps must be redundant.
\end{note}


\paragraph{Basing}

\begin{note}[Theories of basing]
  Connexion between \ESU{} and basing.
\end{note}

\begin{note}
  \citeauthor{Pollock:1999tm} introduce the basing relation with the following observation:
  \begin{quote}
    To be justified in believing something it is not sufficient merely to \emph{have} a good reason for believing it.
    One could have a good reason at one's disposal but never make the connection.
    \dots
    Surely, you are not justified in believing [something], despite the fact that you have impeccable reasons for it at your disposal.
    What is lacking is that you do not believe the conclusion on the basis of those reasons.\linebreak
    \mbox{}\hfill\mbox{(\Citeyear[35]{Pollock:1999tm})}
  \end{quote}
  The observation falls short of being an account of the basing relation, but the intuition \citeauthor{Pollock:1999tm} appeal to is instructive.
  It seems that an agent must connect reasons and the content of a belief in order for the belief to be formed on the basis of those reasons, and hence be justified by those reasons.
  In turn, if a connection is made between reasons and the content of belief, then those reasons are used by the agent.
\end{note}


\paragraph{Summary of motivation for \ESU{}}



\subsection{\EAS{0} --- \EAS{}}
\label{sec:eas}

\begin{note}
  Turning to ability.
  Suppose and agent appeals to
  \begin{enumerate*}
  \item their ability to demonstrate that \(\phi\) is the case, and
  \item that \(\phi\) must be the case in order for the agent to have the ability to demonstrate that \(\phi\)
  \end{enumerate*}
  in order to claim support for \(\phi\).
  Then, the premises and steps involved in a full account of reasoning from the two claims must be sufficient to claim support that \(\phi\) is the case.
  So, as the agent does not witness their ability to demonstrate that \(\phi\) in such reasoning, it must be the case that claimed support for (the property of) having the ability to demonstrate that \(\phi\) is sufficient for such reasoning.
\end{note}


{
  \color{red}
  Perhaps include a note about how the argument relates to \EAS{}.
  I don't provide a direct argument, but this is the best way I see of resolving the tension.
}

\begin{note}[Alternative]
  \ESU{} is a universal claim, and so applies to all instances in which an agent may claim support for conclusion on basis of support for premises and steps of reasoning --- an agent may only claim support if the agent reasoned from the premises via the steps to the conclusion.

  Our goal is to motivate the following exception to \ESU{}:

  \goalEAS*
\end{note}

\begin{note}[Intuition for \EAS{}]
  \EAS{} is a conditional.
  Antecedent is claimed support for ability.
  Consequent is that it may be permissible to violate \ESU{}.
\end{note}

\begin{note}
  Now, started with \USE{}, and then looked at \ESU{}, the converse.
  Both of these we have a particular instance of reasoning in mind.
  Now, \EAS{} may, intuitively, be understood to states that whatever that reasoning is, if an agent has claimed support that they're able to witness such reasoning, then the agent may claim support.

  However, things are a little more complex.
  \EAS{} is about the ability to claim support to reason to some conclusion.
  However, \EAS{} does not state that the agent may claim support for the conclusion on the basis of the premises that they would reason from were they to witness the ability.

  Issue here is that the substance of \EAS{} --- what the relevant materia amounts to --- depends on two things:
  \begin{itemize}
  \item How (appeal to) ability is understood, and
  \item The kind of reasoning involved in the appeal to ability.
  \end{itemize}

  We will outline the basics, then reformulate \EAS{} using one what in which (appeal to) ability is understood.

  Start, how ability is understood.
  Lead naturally to the kind of reasoning involved.

  The argument for \EAS{} will not depend on how ability is understood, but the kind of reasoning involved.
  Still, kind of reasoning involved when combined with how ability is understood.
\end{note}

\begin{note}
  Briefly stated,
  \AR{} understands ability in terms of some (complex) property.
  \WR{} understands ability in terms of possible witnessing events.

  For example, \AR{} may involve the property (attribution) of understanding geometry, perhaps broken down into the understanding or availability of various definitions, propositions, lemmas, theorems, and steps of reasoning.
  While, \WR{} would involve reasoning with particular definitions, propositions, lemmas, theorems, and steps of reasoning.

  So, agent appeals to property, or the reasoning itself.

  The purpose of this distinction is to ensure that our argument against \ESU{} does not rest on a particular way of understanding ability that may not extend to other ways of understanding ability.

  Conjecture that these are fundamentally connected.
  Witnessing event only if understanding.
  And, understanding only if possible to witness reasoning.

  Still, difference.
  Relevant properties are properties of the agent as they are.
  The witnessing event, by contrast, is a possible event.\nolinebreak
  \footnote{
    Property of there being a possible event involving the agent.
    In this case, still distinct from \WR{} as that the agent is part of possible event is still distinct from the reasoning that the agent would witness in the relevant event.
  }

  These are brief characterisations, but enough for now.
  Both~\AR{}~and~\WR{} will be considered at length in~\autoref{sec:ar-wr-1}.
  In addition to a more thorough treatment of the core ideas, \autoref{sec:ar-wr-1} includes additional examples, and an argument that~\AR{}~and~\WR{} are exhaustive --- any way of understanding ability will conform to either~\AR{}~and~\WR{}.
\end{note}

\begin{note}
  Now turn to the kind of reasoning involved.

  Motivated \AR{} in terms of understanding of premises and steps of reasoning, and \WR{} in terms of a possible event in which agent reasons with particular premises and steps.

  However, a further distinction in terms of what appeal to the relevant premises and steps or instance of reasoning amounts to.

  First, there is the \emph{existence} of premises and steps, or the \emph{possibility} of the witnessing event.
  Second, there is the premises and steps themselves, or the witnessing event.

  Difference from perspective of step of reasoning.
\end{note}


\begin{note}[Types of reasoning]
  Consider proofs.

  \(p \lor q\)
  \(\lnot q\)
  \(p\)

  Premises alone do not establish \(p\).
  Combined they do.

  Claim support individually, then \(p\).
  Alone, these don't require \(p\).
  More in \autoref{sec:ability-ads-adc}.
\end{note}

\begin{note}[\EASw{}]
    \begin{restatable}[\EASw{0} --- \EASw{}]{thought}{thoughtEASw}\label{thought:EASw}
    If an agent has claimed support that they have the ability to (adequately) reason to some conclusion, then it may be permissible for the agent to claim support for the conclusion by claiming support for the premises and steps of reasoning that the agent would use to witness their ability to reason to the conclusion.
  \end{restatable}

  Loosely restated,~\EASw{} holds that if an agent may claim support for having the ability to witness some reasoning, and is aware of the conclusion of that reasoning, then the way in which the agent claims support for the conclusion of that reasoning may mirror the way in which the agent would claim support for the conclusion by witnessing the reasoning (and hence using the relevant premises and steps).
\end{note}

\begin{note}[Just an idea]
  \emph{Idea} as this is preferred way of thinking about ability.
  However, argument will not depend on this way of thinking.
\end{note}

\begin{note}
  The (possible) event of the agent witnessing their ability to demonstrate \(\phi\) involves reasoning with various premises and steps which culminate in claiming support for \(\phi\).
  So, if~\EASw{} is true, then the agent may appeal to those premises and steps which are used in the (possible) witnessing event.

  One way to think about~\EASw{} (which we will explore in more details later) is in terms of propositional support.
  For, if an agent has the ability to demonstrate that \(\phi\) is the case, then the agent has propositional support for \(\phi\) as there is a way for the agent to demonstrate that \(\phi\) is the case.
  In addition, that the agent has the ability to demonstrate that \(\phi\) is the case ensure that the agent is in a position to make use of the available propositional support for \(\phi\).
  In turn,~\EAS{} may be interpreted to hold that so long as the agent has such information about their position to make use of the available propositional support for \(\phi\) then the agent does not need to reason with the relevant propositional support in order to claim support for \(\phi\) in virtue of the available propositional support for \(\phi\).
\end{note}

\begin{note}[Conditional]
  Here, note that it's a conditional, but also that it only states there are instances.
  It doesn't follow that ability will always allow the agent to claim support.

  The conditional is weak primarily because it is not at all clear that it holds in general.
  There are various cases in which it seems appeal to ability is blocked.

  Easiest cases involve claiming support in some public setting.
  Of course, success in a public setting is not necessarily required for private success.
  Same problem with testimony.
  I'm confident in a source and you're not.
  I fail to convince you, but I remain convinced myself.

  Still, seems as though similar considerations extend.
  For example, doing a PSET where I'm allowed to use theorems I've already proved.
  Have notes of what those theorems are.
  And, ability to prove them.
  Still, might refrains from using them until I've proven them once again.

  More could be said here, and it may be possible to argue for a stronger variant of \EAS{}.

  Even though it's weak, the condition is still interesting.
\end{note}


\begin{note}
  So, if~\EAS{} is true, then there are cases in which an agent is not required to reason from premises they may claim support for to some conclusion in order to obtain support for the conclusion on the basis the support the agent has for the premises.\nolinebreak
  \footnote{
    Stated~\EAS{} as an exception to~\ESU{}.
    And, we will argue that~\EAS{} is true.
    However, we will not argue that~\EAS{} \emph{is an exception} to~\ESU{}.
    To do so would require an argument that \ESU{} holds for other cases.
    Likewise, no argument that~\EAS{} is the only exception, as to do so would require argument that~\ESU{} holds for all other cases.
    Take~\ESU{} to be plausible, and suspect that there are few, if any, further exceptions, but~\EAS{} may stand independently on any further statements about claiming support.
  }
\end{note}

\begin{note}
  \color{red}
  I want to clarify \EAS{} a little.
  The use of `may' is problematic.
  It could be read as `it's always okay, but it's up to the agent'.
  Or, `it's possible, given appropriate context'.
  The latter is what I want, and is important for cases where doubts are plausibly raised about the ability.
\end{note}

\begin{note}[\EAS{} \illu{0}]
  To illustrate \EAS{}

  \begin{illustration}\label{ill:rectangle:ability}
    Suppose you provide me with novel information that:
    \begin{enumerate}[label=\emph{A}\arabic*., ref=(\emph{A}\arabic*), series=EAS_counter]
    \item\label{EAS:ex:box:if} If I have ability to calculate the area of a box, then I have the ability to demonstrate that a rectangle with dimensions \(19\text{cm}\) by \(7\text{cm}\) has area \(133\text{cm}^{2}\).
    \end{enumerate}
    The information is `novel' because I have not been previously informed (in any way) about the area of a rectangle with dimensions \(19\text{cm}\) by \(7\text{cm}\).

    Still, I am confident that:
    \begin{enumerate}[label=\emph{A}\arabic*., ref=(\emph{A}\arabic*), resume*=EAS_counter]
    \item\label{EAS:ex:box:gen} I have the ability to calculate the area of a rectangle.
    \end{enumerate}
    Therefore, from \ref{EAS:ex:box:if} as an instance of \ref{EAS:ex:box:gen}:
    \begin{enumerate}[label=\emph{A}\arabic*., ref=(\emph{A}\arabic*), resume*=EAS_counter]
    \item\label{EAS:ex:box:spec} I have the ability to demonstrate that a rectangle with dimensions \(19\text{cm}\) by \(7\text{cm}\) has area \(133\text{cm}^{2}\).
    \end{enumerate}
    From~\ref{EAS:ex:box:spec} it follows that:
    \begin{enumerate}[label=\emph{A}\arabic*., ref=(\emph{A}\arabic*), resume*=EAS_counter]
    \item\label{EAS:ex:box:fact} A rectangle with dimensions \(19\text{cm}\) by \(7\text{cm}\) has area \(133\text{cm}^{2}\).
    \end{enumerate}
  \end{illustration}

  \EAS{} holds that, when I claim support for~\ref{EAS:ex:box:fact}, I may appeal to dimensions and formula.

  For, if~\ref{EAS:ex:box:spec} is the case then it is possible for me to witness reasoning in which I demonstrate that~\ref{EAS:ex:box:fact} is the case, and it is the premises and steps of reasoning used in such reasoning that establishes~\ref{EAS:ex:box:fact} is the case.
  I have not used those steps and premises, as I have not witnessed the relevant ability, but may I appeal to those steps and premises regardless --- or so we will argue.

  There is an important subtlety here.
  {
    \color{red}
    Not appealing to either~\ref{EAS:ex:box:gen} or~\ref{EAS:ex:box:spec} to claim support for~\ref{EAS:ex:box:fact}.

    An alternative.
    Indirectly claim support for premises via claimed support for~\ref{EAS:ex:box:gen} and hence~\ref{EAS:ex:box:spec} via~\ref{EAS:ex:box:if}.
    However, argue that this would not amount to instance of claiming support.

    Indeed, argument also rules out~\ref{EAS:ex:box:spec} to~\ref{EAS:ex:box:fact}.
  }
\end{note}

\begin{note}[More detail]
  \color{details}
  I do not expect \EAS{} to be intuitive.
  Indeed, we are not interested in \EAS{} because it is a more-or-less intuitive principle which conflicts the intuitive \ESU{}.
  Rather, we are interested in \EAS{} primarily because \EAS{} is a consequence of tension arising from three things:

  \begin{enumerate}
  \item\label{incomp:tri:q:1} \ESU{}
  \item\label{incomp:tri:q:2} scenarios involving an agent reasoning with information about an their own ability,
  \item\label{incomp:tri:q:3} and a principle concerning when an agent is permitted to claim support
  \end{enumerate}

  To briefly expand on~\ref{incomp:tri:q:2} and~\ref{incomp:tri:q:3}:

  Information that one has some specific ability so long as one has some general ability --- such as the (specific) ability to show that \(25^{\circ}\text{C} = 77^{\circ}\text{F}\) given the (general) ability to convert between Celsius and Fahrenheit.
  And, an agent is never permitted to claim support for proposition having a certain value if the agent requires the proposition to have value \emph{in order to} claim support.
  (As an instance, an agent is not permitted to claim support for the truth of a proposition if the agent requires the proposition to be true \emph{in order to} claim support that the proposition is true.)\nolinebreak
  \footnote{
    The emphasis on `in order to' is important.
    The instance of the principle does not state that an agent is not permitted to claim support for the truth of a proposition if the agent requires the proposition to be true when claiming support that the proposition is true.
    I plausibly require that \(2 + 2 = 4\) when I claim support that \(2 + 2 = 4\), and this does not prevent me from claiming support by simple arithmetic.
    However, it would be impermissible (or so we will argue) to claim support that \(2 + 2 = 4\) by reasoning that the calculator is functional only if \(2 + 2 = 4\), and as the calculator states that \(2 + 2 = 4\) it is the case that \(2 + 2 = 4\).
  }
  The details matter, and we postpone detailing this argument to~\autoref{sec:broad-argum-overv}.

  In short, assuming the scenarios exist, there is tension between intuitive principles governing what an agent appeals to when reasoning and structural principles governing the relation between what the agent appeals to when reasoning.
\end{note}

\begin{note}
  For the moment we attempt to clarify \EAS{} to some degree.
  Three subsections follow:

  \begin{enumerate}
  \item We will outline alternative reasoning patterns from~\ref{EAS:ex:box:if} to~\ref{EAS:ex:box:fact}, clarify why we focus on a particular type of reasoning pattern, and examine some initial objections to~\EAS{} and canvas some responses.
  \item We will consider parallels between abilities and dispositions.
    The parallel will provide some additional intuition for why an agent may appeal to premises and steps that have no been used, and help further clarify our interest with ability.
  \item We will consider a related proposition argued for by \citeauthor{Moretti:2019wx} which holds that a belief need not be based (exclusively) on the premises and steps of reasoning used to arrive at the belief.
    The comparison will help highlight what is distinctive about~\EAS{} while at the same to introducing some ideas which suggest a way of understanding~\EAS{}.
  \end{enumerate}
\end{note}

\subsubsection{Against \EAS{}}

\begin{note}[Alternatives]
  The alternative reasoning pattern we will focus on in some detail holds that appealing to having the ability noted in \ref{EAS:ex:box:spec} is sufficient to claim support for \ref{EAS:ex:box:fact}.
  In line with \ESU{}, the agent would use the proposition that they have the relevant ability noted in~\ref{EAS:ex:box:spec} to claim support for~\ref{EAS:ex:box:fact}
  This reasoning pattern, along with the pattern suggested by \EAS{} will be considered in \autoref{sec:wr-ar} and we will argue that it conflicts with an intuitive principle regarding claiming support in \ref{sec:second-conditional}.

  Alternatively, on may argue that though the syntactic form of \ref{EAS:ex:box:if} is a conditional, it does not (necessarily) follow that the semantic content of~\ref{EAS:ex:box:if} is a (also) conditional.
  And that~\ref{EAS:ex:box:if} may (plausibly) be interpreted to explicitly state that~\ref{EAS:ex:box:spec} is an ability that an agent may have.
  For example:
  \begin{enumerate}[label=\emph{A}\arabic*., ref=(\emph{A}\arabic*), resume*=EAS_counter]
  \item\label{EAS:ex:box:if:R:state} The ability to demonstrate that a rectangle with dimensions \(19\text{cm}\) by \(7\text{cm}\) has area \(133\text{cm}^{2}\) is an ability an agent may have and it is an ability an agent has if they have ability to calculate the area of a rectangle.
  \end{enumerate}
  Hence, \ref{EAS:ex:box:if} is interpreted so that \ref{EAS:ex:box:spec} is accessible without endorsing the antecedent of \ref{EAS:ex:box:if}.
  \ref{EAS:ex:box:if:R:state} states that there is some ability that it is possible for an agent to have, and in addition provides sufficient conditions for having the relevant ability.
  The important part of \ref{EAS:ex:box:if:R:state} is the former conjunct:
  \begin{enumerate}[label=\emph{A}\arabic*., ref=(\emph{A}\arabic*), resume*=EAS_counter]
  \item\label{EAS:ex:box:spec:R:state} The ability to demonstrate that a rectangle with dimensions \(19\text{cm}\) by \(7\text{cm}\) has area \(133\text{cm}^{2}\) is an ability an agent may have.
  \end{enumerate}
  And, \ref{EAS:ex:box:fact} follows from \ref{EAS:ex:box:spec:R:state} by the observation that it is not possible to demonstrate that \(\phi\) if \(\phi\) is not the case --- there is no need for the agent to appeal to witnessing their ability.
  Therefore,~\ref{EAS:ex:box:spec:R:state} (and~\ref{EAS:ex:box:if:R:state}) implicitly includes information that~\ref{EAS:ex:box:fact} is the case --- an agent does not need to reason from~\ref{EAS:ex:box:spec} to~\ref{EAS:ex:box:fact}, because they have already been informed that~\ref{EAS:ex:box:fact} is the case.

  Note also that analogous reasoning applies if `I' is replaced with `an agent'.
  Likewise, if I know that you that you know that I have the ability to calculate the area of a rectangle.
  For, it then follows that you know that \ref{EAS:ex:box:spec} is the case, and therefore you know that \ref{EAS:ex:box:fact} is the case.
  Again there is no need for me to appeal to witnessing an ability.
\end{note}

\begin{note}[Box]
  The existence of alternative reasoning patterns is the issue at hand.
  For, so long as there are reasoning patterns \emph{R} which conform to \ESU{} it is open to the defender of \ESU{} to hold that if an agent is permitted to claim support, then the agent is required to reason via some member of \emph{R}.
  For, if there are reasoning patterns \emph{R} which conform to \ESU{} then there a no counterexamples to \ESU{} --- scenarios in which an agent claims support by appeal to premises or steps of reasoning that the agent has not used.

  Of course, an argument against a general principle such as \ESU{} is not required to be a counterexample.
  For example, it may be possible to argue that the reasoning patterns \ESU{} requires are sufficiently implausible.
  Hence, a restricted variant of \ESU{} compatible with \EAS{} would to be preferred.
  However, there are two issues with attempting such an argument.

  First, given the intuitive plausibility to \ESU{}, it seems unlikely that any violation of \ESU{} would be more plausible than an alternative reasoning pattern compatible with \ESU{}.
  Second, even if there are plausible reasoning pattern that are incompatible with \ESU{}, it is not clear that these should be incorporated in a theory of claiming support.
  For, meta-theoretical issues such as complexity or predictive power may still favour \ESU{}.
  Following \citeauthor{Box:1987vn}: `\dots all models are wrong; the practical question is how wrong do they have to be to not be useful.' (\Citeyear[74]{Box:1987vn})

  Indeed, the second point suggest a counterexample proper to show that \ESU{} is false is not necessarily an adequate argument against \ESU{} either.
  Observations in the spirit of \citeauthor{Box:1987vn} are trite, but also true.
  Even if \EAS{} is true and \ESU{} is false, would \EAS{} be useful?
\end{note}

\begin{note}[Responding to Box]
  With respect to idealised agents with unbounded resources, the answer appears to be no.
  For, with unbounded resources the agent the option of (attempting to) witnessing any ability (to reason) without cost.
  And, it seems that for such an agent witnessing a relevant ability would always be preferable to reasoning about an unwitnessed ability as the agent would minimally (subjectively) resolve any uncertainty about whether they have the ability.

  However, for limited agents, ability is abundant, while the resources required to witness abilities are scarce.
  That the exception to~\ESU{} is narrow does not entail that there are few occurrences of the exception.

  Information about ability may be abundant while the resources for witnessing abilities are either scarce or temporarily unavailable.
  So, for example, agent has the option of conserving or deferring use of resources.

  This observation suggests an initial line of response to an objection which focuses on whether \EAS{} would be useful.

  For, given that we are resource bound agents, it seems that possible instances of \EAS{} are widespread.
  From a functional perspective, reasoning with (the relevant instances of) ability just is reasoning about the result of expending available resources.
  Hence, if~\EAS{} is true, then the truth of \EAS{} would provide a novel perspective on resource bound agents.
  And, it is yet to be seen whether such a perspective is useful.

  In addition, there is a second indirect line of response.
  We observed above that \ESU{} seems prevalent in various theories which relate to reasoning, such as basing and responding to reasons.
  If \EAS{} is true, then there may be alternative conclusions to arguments that appeal to~\ESU{} as a premise.
  And, likewise, there may be interesting observations made in premises of arguments which establish \ESU{} as a foundation for further theorising.\nolinebreak
  \footnote{
    As an exception, even if~\EAS{}, conclusion of arguments which appeal to or assume \ESU{} may be restricted.
  }
\end{note}

\begin{note}[Taking stock]
  Taking stock:
  I doubt that \EAS{} is of interest if there are no reasoning patterns which require \EAS{} to be true.
  Still, if there are reasoning patterns which require \EAS{} to be true, then \EAS{} may be of interest.
  Further, I think there are good reasons to hold that there are reasoning patterns which require \EAS{} to be true.
  Hence, my goal is to motivate further research into whether \EAS{} is of interest.
\end{note}

\paragraph{Type of scenario}

\begin{note}
  We now briefly turn to the type of scenarios which are a premise in our argument for the existence of such counterexamples.
\end{note}

\begin{note}[Types of scenario]
  The type of scenario we will focus on is designed to ensure that an agent is required to reason to (and from) information about a (specific) ability that they have.
  If the agent is required to reason \emph{to} (specific) ability information then rephrasing \ref{EAS:ex:box:if} as \ref{EAS:ex:box:if:R:state} will not be possible --- the agent will be required to reason from some premises by some steps to the (specific) ability information.
  And, as before, the type of scenario will preserve the requirement of the agent to reason \emph{from} the (specific) ability information in line with~\ref{EAS:ex:box:spec} and~\ref{EAS:ex:box:if:R:state}.
  Hence, by establishing such scenarios are possible we may restrict our attention to the steps from~\ref{EAS:ex:box:if} to~\ref{EAS:ex:box:spec} and from~\ref{EAS:ex:box:spec} to~\ref{EAS:ex:box:fact}.
\end{note}

\begin{note}
  To illustrate, let us add some context to the example scenario we've been focusing on.

  Suppose it is common knowledge between you and I that
  \begin{enumerate*}
  \item you have looked through my notes, and have applied my formula for calculating the area of a rectangle, and
  \item my notes are the only source of information you have regarding how to calculate the area of a rectangle.
  \end{enumerate*}
  We may now restate the semantic content of~\ref{EAS:ex:box:if} as follows:
  \begin{enumerate}[label=\emph{A}\arabic*., ref=(\emph{A}\arabic*), resume*=EAS_counter]
  \item\label{EAS:ex:box:inf:R} You have some general ability \(\gamma\), and a specific ability \(\varsigma\) (as an instance of that general ability).
    And, if \(\gamma\) is the ability to calculate the area of a rectangle, then \(\varsigma\) is the ability to demonstrate that a rectangle with dimensions \(19\text{cm}\) by \(7\text{cm}\) has area \(133\text{cm}^{2}\).
  \end{enumerate}
  The formula in my notes indicates that I have the ability to do something, and I have indicated what I think the appropriate characterisation of the ability is.
  Still, you are not in a position to offer information as to whether my characterisation of the ability is correct or not.\nolinebreak
  \footnote{
    Consider in reverse.
    One is often attributed abilities that I deny I have.
    For example, I do not have the ability to process information by means of mental images, but \citeauthor{Hume:2011aa} (arguably) holds that I do have such an ability.
    I lack the ability to reason in a particular way.
    (Not that \citeauthor{Hume:2011aa}'s arguments rest on visual as opposed to any other kind of imagination, but the point stands.)

    Similarly, you may claim that I have the ability to tell you whether or not \nagent{7} is coming to tea.
    However, and in contrast to your assumption, \nagent{7} has not replied to my invitation and so I lack a required premise in order to reason to a relevant conclusion.
  }
  The key feature of~\ref{EAS:ex:box:inf:R} it that I, the agent, am required to claim support that \(\gamma\) and \(\varsigma\) are the abilities of interest.
  The focus is not on whether or not an agent may perform some action.
  Rather, our interest is with what the action is.
  It is up to me, the agent, to claim support that:
  \begin{enumerate}[label=\emph{A}\arabic*., ref=(\emph{A}\arabic*), resume*=EAS_counter]
  \item\label{EAS:ex:box:gen:R} The general ability \(\gamma\) \emph{is} the ability to calculate the area of a rectangle.
  \end{enumerate}
  If so, I may then claim support that:
  \begin{enumerate}[label=\emph{A}\arabic*., ref=(\emph{A}\arabic*), resume*=EAS_counter]
  \item\label{EAS:ex:box:spec:R} \(\varsigma\) is the specific ability to demonstrate that a rectangle with dimensions \(19\text{cm}\) by \(7\text{cm}\) has area \(133\text{cm}^{2}\).
  \end{enumerate}
  Note, there is no route to \ref{EAS:ex:box:spec:R} other than by claiming support for~\ref{EAS:ex:box:gen:R} as I have no information about what the (specific) ability \(\varsigma\) is amounts to if \ref{EAS:ex:box:gen:R} is not the case.
  If \(\varsigma\) is some other ability, then~\ref{EAS:ex:box:fact} does not follow, witnessing the relevant ability would not demonstrate that a rectangle with
  dimensions \(19\text{cm}\) by \(7\text{cm}\) has area \(133\text{cm}^{2}\), and so a rectangle with dimensions \(19\text{cm}\) by \(7\text{cm}\) may (from my epistemic perspective) have some other area.
\end{note}

\begin{note}[Point]
  We will say more in \autoref{sec:cases-interest}.
  For the moment it is sufficient to observe that the agent is required to reason to and from specific ability.
  The scenario requires the agent to claim support by reasoning from~\ref{EAS:ex:box:if} to~\ref{EAS:ex:box:spec} and from~\ref{EAS:ex:box:spec}~to~\ref{EAS:ex:box:fact}.
  And, while the context added to force the reasoning pattern of interest is narrow, the principle behind the context is simple:
  The agent is required to claim support that they have the relevant general ability.
  Hence, any scenario which consists of \gsi{0} which requires the agent to claim support that they have the relevant general ability will require the same kind of reasoning pattern.
\end{note}

\begin{note}[Puzzle about ability]
  Further, this arguably captures a general puzzle about ability.
  An agent is not required to have witnessed all instances of a general ability to claim support that they have the general ability.
  However, so long as the agent may claim support for having some general ability, then it follows that the agent will have the option of claiming support for each instance of the general ability.

  The primary issue, though, is whether there is an account of such reasoning that does not require \EAS{} to be true.
  We will shortly turn to this argument in \autoref{sec:broad-argum-overv}.
  Prior to doing so, we close this section with some further clarification on the motivation behind \EAS{}, what distinguishes \EAS{} from nearby principles, and a suggestion on how to conceptualise \EAS{}.
\end{note}

\subsubsection{Ability and dispositions}

\begin{note}[Parallel]
  To further clarify the motivation for \EAS{} we introduce a parallel between abilities and dispositions.
  The primary function of the parallel will be to highlight the importance of reasoning about an event.
  In the case of dispositions the event is the manifestation of the disposition, and in the case of ability the event is the agent witnessing the ability.

  The parallel is of interest because \EAS{} concerns the premises and steps of reasoning that the agent would use to witness the relevant ability.
  We will suggest that claiming support that some object has some disposition and that some agent has some ability may both be understood in terms of claiming support that the relevant event is a possible event.

  In turn, if reasoning \emph{to} a specific ability is understood in terms of claiming support that it is possible for the agent to witness the event, then reasoning \emph{from} a specific ability may be understood in terms of claiming support from what would happen in the possible event.
  \end{note}

\begin{note}[Parallel between dispositions and ability]
  Consider \citeauthor{Choi:2021wg}'s characterisation of the Simple Conditional Analysis of dispositions:
  \begin{quote}
    An object is disposed to \emph{M} when \emph{C} iff it would \emph{M} if it were the case that \emph{C}.\nolinebreak
    \mbox{}\hfill\mbox{(\Citeyear{Choi:2021wg})}
  \end{quote}
  For example, an object is disposed to dissolve when it is placed in water iff the object would dissolve if it were the case that it is placed in water.

  The Simple Conditional Analysis may be challenged, but for our purposes it is adequate.
  We are interested in the broad form of the truth condition, and various more refined analyses share the same broad form.
  Note, in particular, that it being the case that \emph{C} and \emph{M} happening describes an event.
  Given appropriate conditions; salt dissolves, glass breaks, and I mumble when I am tired.
  The key idea is that the property of being disposed to \emph{M} when \emph{C} is analysed in terms of the (possible) event of \emph{M} happening when \emph{C}.

  The parallel to ability is established by noting that ability may also be analysed in terms of a (possible) event, as we have seen.
  In particular, by incorporating volition in the analysans of the Simple Conditional Analysis.
  To illustrate, \citeauthor{Mandelkern:2017aa} trace the Conditional Analysis of ability  to \textcite{Hume:1748tp} and \textcite{Moore:1912te}, among others:
  \begin{quote}
    S can \(\phi\) iff S would \(\phi\) if S tried to \(\phi\)\nolinebreak
    \mbox{}\hfill\mbox{(\Citeyear[Cf.][308]{Mandelkern:2017aa})}
  \end{quote}
  Compare to the Simple Conditional Analysis of dispositions:
  The object is some agent \emph{S}, \emph{C} is `S tried to \(\phi\)' and \emph{M} is `S \(\phi\)s' --- it is volition alone which distinguishes the analyses.
  For example, I have the ability to demonstrate that a rectangle with dimensions \(19\text{cm}\) by \(7\text{cm}\) has area \(133\text{cm}^{2}\) only if I would demonstrate that a rectangle with dimensions \(19\text{cm}\) by \(7\text{cm}\) has area \(133\text{cm}^{2}\) if it were the case that I tried that a rectangle with dimensions \(19\text{cm}\) by \(7\text{cm}\) has area \(133\text{cm}^{2}\).
\end{note}

\begin{note}[Claiming support]
  Parallel analyses in hand, we now turn to claiming support.
  We start with dispositions.

  As with ability, there are various ways in which an agent may claim support that some object is disposed to \emph{M} when \emph{C}.
  For example, I may claim support that my shoes are disposed to squeak when wet because I have had sufficient occasion to observe the phenomenon.
  Likewise, I may claim support that any shoe of the same model is disposed to squeak when wet because I have traced the source of the squeak to a manufacturing choice.
  In short, support may be claimed by past event and shared properties.

  Still, take a novel act and a object pair.
  Personally, I have a empty fountain pen that I haven't placed in water.
  I claim that the fountain pen is disposed to float when placed in water.
  My reasoning is fairly simple.
  The fountain pen is quite light, especially so while empty of ink.
  And, the cap and loading mechanism seem to be quite well sealed, so the weight of the fountain pen will not increase by taking on water.
  So, given that the weight of the fountain pen will be unchanged, and given how light the pen is, it seems that the upward force exerted by the water against the fountain pen will be sufficient to keep the pen afloat.

  In short, I've noted a few properties of the pen, claimed support for a handful of others, and then considered what would happen.
  Our interest is with the last step.
  I appeal to, and use, the possible event.\nolinebreak
  \footnote{
    I may be wrong about the event, but that isn't at issue.
    It remains the case that I appeal to it.
  }
  The noted properties are relevant because they suggest that the event of floating would happen if it were the case that the fountain pen were placed in water.
\end{note}

\begin{note}
  The fountain pen is not the only object on my desk.
  Beside the fountain pen is a collection of instruments that I may use to investigate the fountain pen.
  And, stored in my mind is a basic understanding of fluid dynamics.

  If I were to measure the fountain pen, ensure that it is airtight, and appeal to some known facts, then an application of Archimedes' principle would allow me to demonstrate that the fountain pen is disposed to float when placed in water (of some specified density).
  Indeed, such a demonstration would be a straightforward refinement of the way in which I claimed support for the proposition that the pen is disposed to float when placed in water.

  Now, by similar reasoning I have claimed support for the proposition that I have the ability to demonstrate that the proposition that the pen is disposed to float when placed in water is true.
  Here, in addition to appealing to properties of the fountain pen, I also appealed to various mental properties.
  There is an important difference, however, regarding the relevant event.
  When reasoning about the disposition, the event is the fountain pen floating in water, but when reasoning about my ability to demonstrate the event is the demonstration --- a series of measurements and calculations.
\end{note}

\begin{note}[Diverge]
  Now to turn to \EAS{}.

  If I have the ability, then it follows that the fountain floats in water.
  As noted above, it is not possible for me to demonstrate something that is not the case.

  Claim support for the proposition that the fountain floats in water.

  Still, disposition, fountain pen is not floating in water.
  Likewise with respect to ability, I have not demonstrated that the fountain pen floats in water.
  I noted various things, but did not piece these together into a demonstration.

  Yet, in claiming support, there's the event of demonstrating.
  And, so I appeal to those premises and steps I would use in the event.
  This is \EAS{}.

  Appeal to what happens in the event.
  And, reasoning to claim possible event is viewed in terms of ensuring that the resources are available.
  I have not used the relevant premises and steps of reasoning, nor am I clear on the specific form they will take.
  Still, they are available.

  Final point of interest, then.
  In both cases, there's an appeal to an event.
  If \EAS{} holds with respect to ability, does something similar hold with respect to dispositions?

  First, important clarification.
  The reasoning outlined for disposition was claiming support for event.
  Here, no clear issue with \ESU{}.
  Similarly, no clear issue with \ESU{} with respect to claiming support for having an ability.
  Tension with \ESU{} arises when using ability as a premise in further reasoning.

  Second, key divergence.
  Conclusion obtained is something that is true independent of ability.
  Unclear to me whether similar reasoning with dispositions.
  For, ability is about an event involving the agent.

  In addition, there is no issue with supposing that the agent reasons with (and hence uses) to all the relevant features of the event.\nolinebreak
  \footnote{There may me details of reasoning that one is not easily able to express, but it doesn't follow that those details are not used.}
  Ability is in part interesting because it is clear that an agent does not witness the relevant event.
  This is not to say that a variant of \EAS{} does not hold with respect to dispositions.
  Rather, I am expressing
  \begin{enumerate*}
  \item hesitancy that there are comparable entailments, and
  \item concern that there is no clear argumentative path.
  \end{enumerate*}

  There is a related question about the ability of other agents.
  Here, \EAS{} does not entail.
  In turn, one may conjecture that reasoning from one's own ability is similar.
  I find this plausible.
  It is important to stress again that \EAS{} expresses a way in which an agent may claim support.
  Hence, \EAS{} is compatible with there being other ways in which an agent may claim support.
  It may be the case that the same holds with respect to other agents.\nolinebreak
  \footnote{
    For example, \citeauthor{Owens:2006tw} argues for a belief expression model of assertion in which the rationality of a belief formed by an agent on the basis of testimony depends whatever justification the speaker has for the relevant propositional content.
    \begin{quote}
      Trusting an expression of belief by accepting what a speaker says involves entering a state of mind which gets its rationality from the rationality of the belief expressed. This state's rationality depends on the speaker's justification for the belief he expresses, not on his justification for the action of expressing it. And to hear a speaker as making a sincere assertion, as expressing a belief, is \emph{ceteris paribus} to feel able to tap into \emph{that} justification (whether or not his assertion was directed at you) by accepting what he says.\nolinebreak
      \mbox{}\hfill\mbox{(\Citeyear[123]{Owens:2006tw})}
    \end{quote}
    \color{red} Some more
  }
  However, this is not an immediate consequence.
  \EAS{} permits exceptions to \ESU{}, but it does not require all instances of reasoning with ability is an exception to \ESU{}.
  And, our focus will be on cases in which an agent reasons about their own ability to reason.
  The weak quantifier `there are cases' is designed to leave such issues open.
\end{note}

\begin{note}[Concluding parallel]
  To summarise.
  \begin{itemize}
  \item Parallel between analysis of dispositions and abilities.
  \item Event in analysis of both.
  \item Reason about event.
  \item Motivation for \EAS{} by considering reason to and from event.
  \item This doesn't provide anything close to a clear theoretical account of the reasoning performed if \EAS{} is true, but it does hint at such at how developing such an account may be approached.
  \item Now turn to related conclusion.
  \item In turn, fill in some details on the account.
  \end{itemize}
\end{note}

\subsubsection{Enthymematic inferences}

\begin{note}[\citeauthor{Moretti:2019wx}]
  Above we considered how various account of the basing relation seem to imply \ESU{}.
  Roughly, because such accounts of the basing relation required a premise or step of reasoning to be used in order to be a candidate member of the base of some conclusion of reasoning --- motivated by either causal and representational considerations.
  In contrast, \citeauthor{Moretti:2019wx} argue for an account of the basing relation which does not entail \ESU{}.

  In our terminology, \citeauthor{Moretti:2019wx} argue that: A belief held by an agent may be \emph{based} on premises that the agent did not use when forming the belief.

  The following is a fragment of the general principle relating propositional justification to well-grounded belief (alternatively doxastistcally justified belief) containing the two clauses of interest:

  \begin{quote}
    IF

    \dots

    OR

    \begin{enumerate*}[label=(\arabic*.2\(^{\ast}\))]
    \item\label{LT:1.2} Q is propositionally justified for S in virtue of P1, P2, \(\dots\), Pn being justifiedly true from her perspective because S justifiedly believes P1, P2, \(\dots\), Pn, and in virtue of her being aware that Q is an inductive or deductive consequence of P1, P2, \(\dots\), Pn jointly, and
    \item\label{LT:2.2} S carries out a \emph{plain} inference from P1, P2, \(\dots\), Pn to Q.
    \end{enumerate*}

    OR

    \begin{enumerate*}[label=(\arabic*.3), ref=(\arabic*.3)]
    \item\label{LT:1.3} Q is propositionally justified for S in virtue of P1, P2, \(\dots\), Pn being justifiedly true from her perspective, though S doesn't believe at least some P1, P2, \(\dots\), Pn, and in virtue of S being aware that Q is an inductive or deductive consequence of P1, P2, \(\dots\), Pn jointly, and
    \item\label{LT:2.3} S carries out a (fully or partly) \emph{enthymematic inference} from P1, P2, \(\dots\), Pn to Q.
    \end{enumerate*}

    THEN
    \begin{enumerate}[label=(3)]
    \item S's belief that Q is well-grounded.\nolinebreak
      \mbox{}\hfill\mbox{(\Citeyear[87]{Moretti:2019wx})}
    \end{enumerate}
  \end{quote}

  The `plain' inference of~\ref{LT:1.2} and~\ref{LT:2.2} corresponds to cases in which an agent uses P1, P2, \(\dots\), Pn to reason to Q.
  By contrast, the `enthymematic' inference of~\ref{LT:1.3} and~\ref{LT:2.3} involves reasoning in which an agent does not use some or all of P1, P2, \(\dots\), Pn to reason to Q as the agent does not believe some of P1, P2, \(\dots\), Pn (though the agent has propositional support for each of P1, P2, \(\dots\), Pn).

  To illustrate the distinction between `plain' and `enthymematic' inferences (\Citeyear[Cf.][85]{Moretti:2019wx}) consider reasoning from the premise that \nagent{5} is shorter than \nagent{6} to the conclusion that someone is taller than \nagent{5}.
  An instance of plain (non-enthymematic) may take the intermediary step that \nagent{6} is taller than \nagent{5} before abstracting from \nagent{6}.
  In contrast, an instance of enthymematic reasoning consists of the (single) premise and conclusion noted without forming the belief that \nagent{6} is taller than \nagent{5}.\nolinebreak
  \footnote{Cf.\ (\Citeyear[87--89]{Moretti:2019wx}) for examples given by \citeauthor{Moretti:2019wx}.}

  The key idea is that if an agent reasons enthymematically, then the agent's belief may be based on those premises that the agent would use in the corresponding plain inference.
  (\Citeyear[Cf.][86--87]{Moretti:2019wx})
  Hence, we have a proposal on which an agent's belief may be supported by premises and steps of reasoning that an agent has not used.
  And, in addition, because S carries out a (fully or partly) enthymematic inference \ref{LT:2.3}, it seems S \emph{may} appeal to P1, P2, \(\dots\), Pn when reasoning to Q, in conflict with \ESU{}.

  Whether or not \citeauthor{Moretti:2019wx}'s account is correct is not of interest.
  Rather, \emph{grating} that \citeauthor{Moretti:2019wx}'s account is correct allows us to make two (related) observations.
  First, \citeauthor{Moretti:2019wx} account does not conflict with \ESU{} and so the account does not require \EAS{} to be true.
  And, second, how \citeauthor{Moretti:2019wx}'s account suggests a broader theoretical account of \EAS{}.
\end{note}

\begin{note}[First point]
  To establish the first point we require further details about how \citeauthor{Moretti:2019wx} define a (fully or partly) enthymematic inference.
  The following quote combines the relevant definitions:
  \begin{quote}
    \textbf{(}[\textbf{Partly}/\textbf{Fully}] \textbf{Enthymematic Inference)}

    S carries out a [\emph{partly}/\emph{fully}] \emph{enthymematic} inference from P1, P2, \(\dots\), Pn to Q if and only if
    \begin{enumerate}[label=(\alph*), ref=(\alph*)]
       \setcounter{enumi}{1}
    \item \emph{S doesn't actually believe} [\emph{at least some of the premises}/\emph{any of}] P1, P2, \(\dots\), Pn, though some constituents M1, M2, \(\dots\), Mm of S's perspective cause in S the \emph{disposition} to believe P1, P2, \(\dots\), Pn, and
    \item M1, M2, \(\dots\), Mm [together with the premises believed by S jointly/jointly] cause S's belief that Q through a process that is shaped by S's taking Q to be a consequence of P1, P2, \(\dots\), Pn at a personal level.\nolinebreak
      \mbox{}\hfill\mbox{(\Citeyear[85]{Moretti:2019wx})}
    \end{enumerate}
  \end{quote}

  In short, an enthymematic inference involves reasoning with premises M1, M2, \(\dots\), Mm which are related to the premises P1, P2, \(\dots\), Pn of some corresponding plain inference.
  In order to complete the definition, we require an account of what it is for S to take Q to be a consequence of P1, P2, \(\dots\), Pn at a personal level:

  \begin{quote}
    \textbf{(Personal Level\(^{\ast}\))}

    S's mental states M1, M2, \(\dots\), Mm and any premises believed by S, among P1, P2, \(\dots\), Pn, jointly cause S's belief that Q through a process shaped by S's taking Q to be a consequence of P1, P2, \(\dots\), Pn at a personal level if and only if M1, M2, \(\dots\), Mm and any premise believed by S, among P1, P2, \(\dots\), Pn, jointly cause S to believe Q and S would adduce the reasons that P1, P2, \(\dots\), Pn and that Q is a consequence of P1, P2, \(\dots\), Pn in response to a request to explain why she believes Q.\nolinebreak
    \mbox{}\hfill\mbox{(\Citeyear[85--86]{Moretti:2019wx})}
  \end{quote}

  So, loosely reconstructed an enthymematic inference involves constituents M1, M2, \(\dots\), Mm of S's perspective which ensure that S has the disposition to believe P1, P2, \(\dots\), Pn.
  And, the way in which M1, M2, \(\dots\), Mm lead to S forming the belief that Q allow S to explain that they believe Q on the basis of P1, P2, \(\dots\), Pn.
  In short, an enthymematic inference is an inference in which may be \emph{post hoc} expanded to some corresponding plain inference (in part) because performing the enthymematic inference requires the agent to be disposed to believe the required premises of the corresponding plain inference.
  And, as such the premises of the corresponding plain inference may be considered as (constitutive of) the basis of S's belief that Q.

  In contrast, \ESU{} concerns the way in which M1, M2, \(\dots\), Mm lead to S forming the belief that Q do not necessarily require the agent to appeal to P1, P2, \(\dots\), Pn.
  It is consistent with \citeauthor{Moretti:2019wx} account that the reasoning from M1, M2, \(\dots\), Mm to Q may only appeal to premises and steps of reasoning used.
  That Q may be based on P1, P2, \(\dots\), Pn is due to the requirement that S is disposed to believe P1, P2, \(\dots\), Pn and the possibility of S retroactively appealing to Q being a consequence of P1, P2, \(\dots\), Pn.
  Hence, the account does not conflict with \ESU{}, and in turn does not require \EAS{} to be true.

  The insight offered is that there does not necessarily need to be a structure preserving mapping between premises and steps providing propositional support for a belief and the premises and steps appealed to when forming the belief.
  However, this does not constrain what the agent appeals to when forming a belief.

  From a broader perspective, \citeauthor{Moretti:2019wx}'s proposal considers what an agent was able to do (i.e.\ reason by some plain inference) and holds that a basing relation follows but is silent of the way in which an agent claim support.
  In contrast, \EAS{} looks at what an agent is able to do, and holds that a way of claiming support follows, but is silent on issues concerning the basing relation.
\end{note}

\begin{note}[Second point]
  Still, this broader perspective together with the above discussion of dispositions suggests a way to understand \EAS{}.
  For, one may hold that if an agent has the ability to reason to some conclusion, then the agent is disposed to use relevant premises and steps of reasoning to reason to the conclusion.
  In parallel to \citeauthor{Moretti:2019wx}, then, one may hold that the agent has the ability to reason to some conclusion if (and only if) they are suitably related to some collection of relevant premises and steps of reasoning.
  In turn, the agent may appeal to those premises and steps of reasoning to claim support for the conclusion.
  Indeed, if we adopt a parallel understanding of the basing relation, then it follows (so long as the agent has the ability) that  the agent has sufficient propositional support for the conclusion, and may be well-grounded.
  The (possible) event of reasoning to the conclusion is important both for establishing that the agent has the ability and for determining which premises and steps of reasoning the agent appeals to, but the event is not important for determining that the relevant premises and steps of reasoning are available to the agent.

  This suggestion falls far short of a theory satisfying \EAS{}, I suspect \EAS{} may be motivated in part by distinguishing between what occurs in the event of reasoning, and sufficient resources required for such an event to occur.
  An event of reasoning will always make use of sufficient resources for the event to occur, but an agent may have sufficient resources for the event to occur even if the event does not occur.
  (Specific) abilities, then, fix an particular event and determine sufficient resources and the agent does not need to witness the event in order to appeal to those resources.

  Or perhaps not.
\end{note}

\subsection{Closing}
\label{sec:closing}

\begin{note}[Segue]
  Our goal is to establish that an adequate account of reasoning which extends to ability must satisfy \EAS{}.
  This goal does not require the above suggestion to be on the right track, nor does this goal require that there is a unique theory that satisfies \EAS{}.
  For now, we close the present section with a few remarks concerning ability and~\EAS{}.
\end{note}

\begin{note}[Actual support]
  As with~\ESU{}, \EAS{} does not entail that the agent \emph{has} support.
  Our focus is on reasoning, and as argued above, it seems the issue of whether an agent has support is distinct from whether an agent may claim support.
  Claiming support is the result of some reasoning, and whether or not an agent has support requires an evaluation of that reasoning.
  This means that, strictly speaking,~\EAS{} does not carry any implications regarding whether or not the agent has support by claiming support in line with~\EAS{}.
  It is possible that the agent would fail to establish support, or establishes a support relation other than between the conclusion and the premises and steps of reasoning appealed to.

  Still, it take it to be plausible that support traces a successful claim.
  From this perspective,~\EAS{} may seem a little more intuitive.
  Given an intuitive understanding of support, if an agent does have the ability to reason to some conclusion, then the conclusion stands in the relation of being support by certain premises and steps of reasoning, whether or not the agent witnesses their ability.
  In turn, if the agent may claim support for the having the relevant ability then the agent may claim support for the conclusion from the premises and steps that would be used to witness their ability.
  For, witnessing does not contribute to the relation of support between the conclusion and the relevant premises and steps --- witnessing would only clarify to the agent the specifics of the relation.

  Of course, the agent may be mistaken or misled about having ability.
  For example, the relevant premises and steps may fail to establish the conclusion, or the agent may not have sufficient resources to carry out reasoning from the premises and steps, etc.
  In turn, witnessing may be expected to highlight that the claimed support for having the ability is mistaken or misled.

  Two points:
  \begin{itemize}
  \item Such issues are not different to being mistaken or misled and using that one has the ability as a premise, so apply to any reasoning that makes use of ability without witnessing ability.
  \item Attempting to witness the ability might reveal that the agent is mistaken or misled about having the ability does not show that the agent may not claim support for having the ability.

    Reasoning typically involves premises and steps of reasoning that could be investigated further, but this does not prevent an agent from appealing to those steps and premises.
  For example, it is (almost) to check the definition of any word used against a dictionary, and doing so might reveal that I have been mistaken or mislead about the meaning that I will convey by using the word.
  I rarely do this, though.
  Most of the time it is sufficient to expect that I am not mistaken or have not been mislead about the meaning I would convey by using the word.
  \end{itemize}
\end{note}


%%% Local Variables:
%%% mode: latex
%%% TeX-master: "master"
%%% End: