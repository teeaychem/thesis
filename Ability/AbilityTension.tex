%%% Local Variables:
%%% TeX-master: "master"
%%% End:

\chapter{Tension}
\label{cha:tension}

\subsection{Overview of argument for tension}
\label{sec:overv-argum-tens}

\begin{note}[Conflict with the other two principles]
  The two different ways of understanding ability lead to conflict with the two other principles.
\end{note}

\begin{note}[Witnessing]
  The conflict between \ref{denied-claim} and witnessing is straightforward.

  For,~\ref{denied-claim} denies the possibility of witnessing in any case of ability.
  As the agent has not performed the reasoning, the agent is not in a position to appeal to the premises.
\end{note}

\begin{note}[Attribution]
  The conflict between~\nI{} and attribution requires the assumption from~\ref{prem:ab}.

  For, then the agent is appealing to the `leadingness' of their support for the general ability.
\end{note}

\begin{note}[Result from tension]
  So long as there's tension, this is a useful result, I think.
\end{note}

\begin{note}[Preferred resolution]
  Defend both~\nI{} and~\ref{prem:ab}.
  Therefore, not~\ref{denied-claim}.

  Given that both~\ref{denied-claim} and~\nI{} are quite intuitive, deny~\ref{prem:ab}.

  Sketched a straightforward argument from~\ref{denied-claim} to denying witnessing.
  Further, there are some subtleties with \nI{} and \ref{prem:ab} independent of \ref{denied-claim}.

  Picture on which there are exceptions to~\ref{denied-claim} so long as the agent is able to appeal to premises that they would trace support from in the relevant witnessing event of ability.

  Two lines of argument for denying~\ref{denied-claim}.
  \begin{enumerate}
  \item We have a motivated restriction on~\ref{prem:ability}.
  \item Witnessing provides an intuitive understanding of cases in which agent has the option of appealing to ability.
  \end{enumerate}
  These two lines of argument work together.
  The tension generates interest in witnessing that may be flatly rejected by prior endorsement of~\ref{denied-claim}.
  The intuitive understanding of scenarios involving ability suggests there's more to witnessing than resolving the tension in narrow cases.

  To help situate, begin by sketching restriction on~\ref{denied-claim}.
\end{note}

\subsection{A little more detail}
\label{sec:little-more-detail}

The argument is by counter-example.
Rejection of~\ref{denied-claim} is somewhat narrow.

\begin{enumerate}[]
\item\label{access} If premises imply conclusion, then if agent accesses support they have for premises and traces implication through reasoning, then agent obtains support for conclusion on basis of support for premises
\end{enumerate}

Basic.
Understanding of implication is put in the background.
May understand this as something quite trivial.

\begin{enumerate}[]
\item\label{access:exception} If agent is able to obtain support for conclusion on basis of support for premises by accesses support they have for premises and reasoning to conclusion, then if agent accesses support they have for premises and traces implication through reasoning, then agent obtains support for conclusion on basis of support for premises.
\end{enumerate}

If agent may do something to achieve a result, then they would achieve the result by doing the thing.
Embedded conditional captures a witnessing event of the ability.

Agent is required to witness the ability, if the agent is appealing to witnessing.

Compatible with the agent reasoning from their ability.

This is where the tension comes.
Scenarios in which it is permissible for agent to appeal to reasoning they are able to do in order to support a conclusion.

This narrows significantly.

Still, the type of counterexample is further constrained.
Here, the argument splits.

First, the counterexample with \nI{}.
Second, the upshots of denying~\ref{denied-claim}.

Counterexample requires alternative in some cases of ability, and given alternative is required, it expands to other cases of ability.

Remains a somewhat narrow exception to~\ref{denied-claim}.



\subsection{Denying ability}
\label{sec:denying-ability}

\begin{note}[Difficult corollary of argument]
Given that both~\ref{denied-claim} and~\nI{} are quite intuitive, deny~\ref{prem:ab}.
Strengthen the argument by observing a corollary.

\begin{enumerate}
\item\label{prem:ab:cor:a} The agent appeals to a potential witnessing event of an ability that the agent has in order to obtain support for some proposition without having appropriate support for the ability to generate the witnessing event.
\end{enumerate}

This is why~\ref{prem:ab} is phrased in terms of information, rather than support.

\nI{} does not prevent the agent from using the proposition.
\ref{prem:ab} only denies that the agent obtains support.

By appealing to witness, the agent isn't appealing to support.

So, in certain situations, propositions on the basis of such information may still be used.

Key idea here is that the agent doesn't have support either way.
And, it's the application of the general ability (which the agent does have support for) which does the work.

Corollary~\ref{prem:ab:cor:a} and an exception to~\ref{denied-claim} when dealing with ability is preferable to an exception to~\nI{}.
\end{note}

\begin{note}[Second aspect of corollary]
  Generalise~\ref{prem:ab:cor:a}:
  \begin{enumerate}
  \item\label{prem:ab:cor:b} Use a proposition \(\phi\) to obtain support for some other proposition \(\psi\) without support for \(\phi\).
  \end{enumerate}
  However, there are various instances where this seems fine.
  There's Wright, at least.
\end{note}

\begin{note}[Narrowing understanding of support]
A second option is to deny that the agent obtains support.
If so, no tension.
Quite plausible, at least on some ways of understanding support.

However, then left with a variant of support for which~\ref{denied-claim} does not hold.
The main focus on~\ref{denied-claim} is not support, but the accessibility requirement.
Our use of `support' is something of a generic placeholder.
\end{note}

\begin{note}[Too narrow]
  Hold that the exception made by~\ref{prem:ab} is too narrow to be of general interest.
  Supplemental argument that witnessing offers natural interpretation, even when there's the option of attribution.

  (And possibly that attribution seems to over-generate support.)
\end{note}

\begin{note}[More on too narrow]
  Idealised agents have no need for abilities.
However, for non-ideal agents, abilities seems useful.
Information about ability may be abundant while the resources for witnessing abilities are either scarce or temporarily unavailable.
So, agent is able to conserve or defer use of resources.
\end{note}


\subsection{Strength of \uRa{}}
\label{sec:appeal-main-premise}

Denying the \uRa{} is difficult.
The general principle provides a recipe for dealing with every other case.
However, given that the agent does some reasoning, seems there's always the option to identify the reasoning done with the reasons the agent appeals to.
So, given that there's always something for the \uRa{} to use, it seems that in the absence of any serious difficulty with the \uRa{}, alternatives lose out.

Further, simple failure of \uRa{} isn't particularly informative.
Seems good in many cases.
And, rules out many bad cases.
If the \uRa{} fails in general, then an account of why.

The strategy is split into two parts:
\begin{enumerate}
\item Identify a kind of case where the assumption is problematic, motivating an exception to the general principle.
\item Argue that if the exception is granted, then there are further cases of similar kind in which the alternative is compelling, even if the general principle holds.
\end{enumerate}

The strategy is straightforward.
Identifying a kind of counterexample means that one is not in a position to apply the general principle universally.
Given counterexample, then we have an alternative account for at least one kind of case.
Consider other cases in which the alternative may apply.
It is not the case that we don't have an argument for accessibility based on the application of a general principle.
Suggest that the alternative fares well in the absence of a general principle.

Then, so long as the structure in place for the alternative is fairly common, there is potential to re-evaluate arguments that involve the general principle.


\begin{note}[Denying \nI{} and \WR{}, a bullet to bite?]
  It looks as though I end up with:

  The agent appeals to a potential witnessing event of an ability that the agent has in order to obtain support for some proposition without having appropriate support for the ability to generate the witnessing event.

  Well, the ability is not the thing providing the support.
  So, lacking support for the ability is not the issue.

  The question is about why the agent is in a position to appeal to those reasons.

  Well, there's an additional constraint, possibly.
  Roughly, it is impermissible for an agent to hold a contrary attitude to a proposition that they're committed to.

  Well, then the agent does not have support for possessing the ability.

  This is somewhat disconcerting, but not particularly problematic.

  The key observation is that \nI{} means that the agent does not obtain support.
  This does not mean that the agent does not have the option of using the information.
\end{note}

\newpage



Problematic cases, are, roughly put, useful for understanding how things are.

Still, because particular case, then the rejection of~\ref{denied-claim} is limited.
I think the core of~\ref{denied-claim} often holds.
The concern is that~\ref{denied-claim} applies to all instances of reasoning.

Interest is in the role of~\ref{denied-claim} in arguments and understanding phenomena.
If there are exceptions to~\ref{denied-claim}, then if~\ref{denied-claim} is a premise, there may be variants on the relevant conclusion, and if~\ref{denied-claim} is a conclusion, then potential issues with premises, or links between them.




Well, this is the strong way of putting the argument.
The slightly weaker version is to hold that there is not a sense of support which holds for both reasoning and the use of abilities.

In this sense, the notion of support captured by~\ref{denied-claim} is not the only notion of support of interest.

Familiar with assuming and so on.
Key difference here is that the agent appeals to support.

Before continuing, narrow down.



\section{Ability}
\label{sec:ability}

First explain how ability relates, and how it suggests an alternative.

\subsection{Why ability is interesting}
\label{sec:why-abil-inter}

\begin{enumerate}
\item Reasoning is an action, a particular event.
\item Ability grants a potential witnessing event.
\end{enumerate}

Reasoning is some event, and ability allows us to attribute the agent the potential to witness the relevant event.

Certain kinds of ability.

The interest here is that we have a clear understanding of how reasoning works, related to the general premise.
With ability to reason, there are two things:
\begin{enumerate}
\item The reasoning which the agent may witness.
\item Having the ability to generate the witnessing event.
\end{enumerate}

By the general premise, it seems that if the agent were to witness the reasoning, then those things appealed to in reasoning would support the premise.
However, as the agent has not yet reasoned, having the ability should do the work.

So, this is the suggested alternative.
In some cases, witnessing.

If witnessing, then the conflict with the general principle is that as the agent has not done the reasoning, the agent doesn't have access to those reasons.
However, the reasons are there for the agent.

Two things about why ability is interesting.

Understand that if the agent has the ability, then they may witness reasoning.
However, ability secures only the potential.
Not in the sense that the there is a potential for a coin flip to land heads, but in that a coin with heads on both sides has the potential to land heads up when flipped --- the coin merely needs to be flipped.

Things follow from witnessing abilities.
Factive verbs work best here.
Key is that the relevant proposition is true whether or not the agent witnesses the ability.
Some propositions are only true after witnessing.

Simple example is \(\phi\), and `that I have \emph{V}'d that \(\psi\)'.
\(\phi\) must be true in order for the agent to \emph{V} that \(\phi\).
Still, `that I have \emph{V}'d that \(\psi\)' is only true when the agent has \emph{V}'d that \(\phi\).

Two different ways of understanding ability provide different perspectives.

If the agent reasons, then the agent obtains support for \(\phi\) which do not necessarily depend on a premise that the agent has the ability.
Easy to see with logic examples.
The agent witnesses the ability, but in so exercising does not need to appeal to the observation that they are witnessing.
Many cases where recognition of ability is only after witnessing it.

If the agent appeals to the attribute, then ability is a key premise.
The agent obtains \(\phi\) because if \(\phi\) were not the case, the agent wouldn't have the ability to \emph{V} that \(\phi\).

So, in principle there's two ways in which ability may be put to work.

One further complication.
Agent may not need to put ability to work.

Possible for agent to be provided with support for \(\phi\) and support for the ability to \emph{V} that \(\phi\).
For example, testimony, say.
Distinct from \(A(\phi)\) therefore \(\phi\) must be the case.

If so, ability is not of interest in obtaining a conclusion.

\begin{enumerate}
\item\label{cases-of-i-ex} There are cases in which \(A(\phi)\) is required for \(\phi\).
\end{enumerate}

\ref{cases-of-i-ex} holds that there are cases in which the agent has information that \(A(\phi)\), understands that \(A(\phi) \rightarrow \phi\), and has no other way of obtaining \(\phi\) other than by \(A(\phi)\).