\chapter{\fc{3} and ability}
\label{cha:sec:fcs-def:ability}

\begin{note}
  \fc{3} in terms of events in progress.
  Alternative suggestion is to say \fc{} just in case agent `can \(\alpha\)' or `has the ability to \(\alpha\)'.
  Preferable, ability.

  \begin{quote}
    \(\pv{\phi}{v}\) from \(\Phi\) is a \fc{} just in case agent has the ability to conclude \(\pv{\phi}{v}\) from \(\Phi\) (and has the ability to avoid concluding something incompatible).
  \end{quote}

  This chapter explores the idea, and highlights how \fc{1} are incompatible with contemporary accounts of ability.

  This chapter is optional.
  The main line of argument does not depend on any of the observations made in this chapter.
\end{note}


\begin{note}
  \begin{TOCEnum}
  \item
    \TOCLine{cha:sec:fcs-def:ability:abil-gener-spec}

    Distinguish general and specific abilities.
  \item
    \TOCLine{cha:sec:fcs-def:ability:past}

    Specific ability and the past.
  \item
    \TOCLine{cha:sec:fcs-def:ability:control-intuition}

    `Control'.
  \end{TOCEnum}
\end{note}

\section{Ability, general and specific}
\label{cha:sec:fcs-def:ability:abil-gener-spec}

\begin{note}
  Particular sense of ability.

  % {
  %   \color{blue}
  % In general, not a \fc{} that \(\mathbf{GL}\) is weakly complete with respect to the class of finite strict partial orders.
  % For, method relies on a syntactic proof \(\vdash_{\mathbf{GL}} \Box p \rightarrow \Box \Box p\).

  % In this respect, it seems I do not have the ability to prove \(\mathbf{GL}\) is weakly complete with respect to the class of finite strict partial orders.

  % However, if I have just (by some luck) completed or (by some studying) rehearsed a syntactic proof \(\vdash_{\mathbf{GL}} \Box p \rightarrow \Box \Box p\), then the relevant theorem is a \fc{}.
  % }

  In short, it may be true that \(\pvp{\psi}{v'}{\Psi}\) is a \fc{} for an agent while it is false that the agent has the ability to conclude \(\pv{\psi}{v'}\) from \(\Psi\).
\end{note}

\begin{note}
  Still, while there may not be an \emph{immediate} link, whether or not \(\pvp{\psi}{v'}{\Psi}\) is a \fc{} may still reduce to ability, when ability is appropriately understood.

  \phantlabel{ability-g-s-dist}%
  \nocite{Maier:2018uo}
  For, we may distinguish between `general', `categorical' or `global' abilities and `specific' or `local' abilities.

  Following \textcite[2]{Whittle:2010wr} the distinction is roughly as follows:%
  \footnote{
    Though, see~\textcite[esp.\ \S4]{Kittle:2015tb} and~\textcite[1--2]{Kikkert:2022wp} for additional discussion.%
  }
  \begin{itemize}[noitemsep]
  \item
    General (or global) abilities concern `what an agent is able to do in a large range of circumstances', while
  \item
    Specific (or local) ability concern `what the agent is able to do now, in some particular circumstances'.
  \end{itemize}

  General is just given in terms of specific.
  Not conversely, where specific is general and circumstances permit.%
  \footnote{
    For an example of this approach, see \citeauthor{Austin:1961vz}'s (\citeyear{Austin:1961vz}) `categorical' abilities and opportunities:

    \begin{quote}
      Consider the case where what we wish to assert is that somebody had the opportunity to do something but lacked the ability---`He could have smashed that lob, if he had been any good at the smash':
      here the \emph{if}-clause, which may of course be suppressed and understood, relates not to opportunity but to ability.
      [\dots]
      `He could have read \emph{Emma}, if he had had a copy', does seem to assert `categorically' that he had a certain ability, although he lacked the opportunity to exercise it.%
      \mbox{ }\hfill\mbox{(\citeyear[177]{Austin:1961vz})}
    \end{quote}
  }

  Example of what \textcite{Hackl:1998tt} terms `opportunity-can' (\citeyear[14]{Hackl:1998tt}):

  \begin{quote}
    \begin{enumerate}
    \item[(92)]
      \begin{enumerate}[label=\alph*., ref=(\alph*)]
      \item
        \label{Hackl:OC:a}
        A star gazer can see the solar eclipse of this year from the Cayman islands.\newline
        So if you were a star gazer and if you were on the Cayman islands at the right time you would see this year's solar eclipse.
      \item
        \label{Hackl:OC:b}
        John can see Mary from where he is standing.\newline
        So if you were standing in his place, you would see Mary.
      \end{enumerate}
    \end{enumerate}

    [\ref{Hackl:OC:b}] says that whoever is in this situation located at John's position and has normal eyesight and directs his/her gaze towards Mary will succeed in seeing Mary.%
    \mbox{ }\hfill\mbox{(\citeyear[39]{Hackl:1998tt})}
  \end{quote}
  \citeauthor{Hackl:1998tt}'s analysis straightforwardly extends to \ref{Hackl:OC:a}:
  A star gazer who is in the Cayman islands at the right time this year and looks for the solar eclipse will succeed in seeing the solar eclipse.

  So, a tentative proposal is to understand whether or not \(\pvp{\psi}{v'}{\Psi}\) is a \fc{} for an agent in terms of whether or not the agent has the \emph{specific} ability to conclude \(\pv{\psi}{v'}\) from \(\Psi\).

  Hence, we set aside \citeauthor{Austin:1961vz}'s `categorical' ability.
  Likewise we set aside `general' accounts of ability such as~\citeauthor{Carter:2021wd}'s~(\citeyear{Carter:2021wd}) `fallibilist',~\citeauthor{Kikkert:2022wp}'s~(\citeyear{Kikkert:2022wp}) `robust', and \citeauthor{Maier:2013vk}'s (\citeyear{Maier:2013vk}) `general' account, among others.

  Two issues.
  Specific ability and the past.
  `Control'

  Discussion focuses \textcite{Boylan:2020aa}.
  Clear that specific ability (\citeyear[23, fn.3]{Boylan:2020aa})
\end{note}

\section{(Specific) ability and the past}
\label{cha:sec:fcs-def:ability:past}

\begin{note}
  First is specific ability and what actually happens.
  Two entailments.
  First, \BoyPS{} following \textcite{Boylan:2020aa}.
  Second, \BoyPSC{} the converse of \BoyPS{}.

  Combined, had the ability to if and only if did.

  Embedded in the past.
  So, doesn't say too much.
  However, difficulty with this is \fc{} depends on something not happening.
\end{note}

\begin{note}
  The first entailment is termed `\BoyPS{}'.

  \begin{enumerate}[label=]
  \item
    \label{Boylan:Past-Success}
    \BoyPS{}: \(\text{Past}(S\text{ does }\phi) \Rightarrow \text{Past}(S\text{ is able to }\phi)\)%
    \mbox{ }\hfill\mbox{(\citeyear[\S1.1]{Boylan:2020aa})}
  \end{enumerate}

  \citeauthor{Boylan:2020aa} motivates \BoyPS{} in the following way:
  \begin{quote}
    \begin{quote}
      \textbf{Fluky Dartboard}.
      I am a terrible dartplayer.
      I struggle to even hit the board whenever I take a shot.
      However, I take my shot and I flukily hit the bullseye.
    \end{quote}

    Once I have taken the shot and hit the bullseye, I can compellingly argue:

    \begin{enumerate}
      \setcounter{enumi}{2}
    \item
      I hit the bullseye on that throw.\newline
      So, I was able to hit the bullseye on that throw.
    \end{enumerate}

    If you know that I have been successful, you must concede I was able to.%
    \mbox{ }\hfill\mbox{(\citeyear[2]{Boylan:2020aa})}
  \end{quote}

  Intuitions regarding \citeauthor{Boylan:2020aa}'s case may be unclear.
  However, recall we are interested in \emph{specific} ability.
  Therefore, the argument provided is consistent with \citeauthor{Boylan:2020aa} failing to have the \emph{general} ability to hit the bullseye.%
  \footnote{
    \textcite{Bhatt:2008aa} observes:
    \begin{quote}
      \begin{enumerate}[label=(\arabic*)]
        \setcounter{enumi}{314}
      \item
        (from~\cite{Thalberg:1969ta})
        \begin{enumerate}[label=\alph*., ref=(315\alph*)]
        \item
          \label{Bhatt:Thal-a}
          Yesterday, Brown hit three bulls-eyes in a row.
          Before he hit three bulls-eyes, he fired 600 rounds, without coming close to the bullseye; and his subsequent tries were equally wild.
        \item
          \label{Bhatt:Thal-b}
          Brown was able to hit three bulls-eyes in a row.
        \item
          \label{Bhatt:Thal-c}
          Brown had the ability to hit three bulls-eyes in a row.
        \end{enumerate}
      \end{enumerate}
      From~\ref{Bhatt:Thal-a}, we can conclude~\ref{Bhatt:Thal-b} but not~\ref{Bhatt:Thal-c}.
      Brown could have hit the target three times in a row by pure chance and he does not need to have had any ability for~\ref{Bhatt:Thal-b} to be true.%
      \mbox{ }\hfill\mbox{(\citeyear[167]{Bhatt:2008aa})}
    \end{quote}
    Distinction between `was able' and `had the ability'.
    \citeauthor{Boylan:2020aa} only `was able', and so agrees with \citeauthor{Bhatt:2008aa}.

    Still, the distinction between~\ref{Bhatt:Thal-b} but not~\ref{Bhatt:Thal-c} is due to the `specific'/`general' divide.
    And, indeed, \citeauthor{Bhatt:2008aa}'s proposal, \emph{to my understanding}, identifies `was able' with the specific reading of ability and `had the ability' with the general reading of ability.

    Indeed, \citeauthor{Boylan:2020aa} makes a similar observation with respect to \citeauthor{Maier:2018uo}'s (\citeyear{Maier:2018uo}) hybrid (modal/generic) account of ability. (\citeyear[23, fn.3]{Boylan:2020aa})
  }%
  \(^{,}\)%
  \footnote{
    Finding additional instances of \BoyPS{} has difficult.

    \textcite[1]{Boylan:2020aa} mentions \citeauthor{Austin:1961vz}'s  remark that `it follows merely from the premiss that he does it, that he has the ability to do it, according to ordinary English' (\citeyear[175]{Austin:1961vz}).
    However, reasoning patterns are often not made explicit.
    Most instances of `was able to' seem to correspond to the converse entailment, \BoyPSC{}, discussed below.

    Still, a clear instance comes from \textcite{Taylor:2011uh}:
    \begin{quote}
      Consider, then, the R-statement (S):

      \begin{quote}
        Stilpo walks through the Diomean Gate at t\textsubscript{2}
      \end{quote}

      and assume that statement, tenselessly expressed so as to avoid ambiguity in what follows, to be true.

      [\dots]
      if S is true, then it follows that Stilpo was able to be walking through the gate at t\textsubscript{2}, that being, in fact, precisely what he was doing.%
      \mbox{ }\hfill\mbox{(\cite[139--143]{Taylor:2011uh})}
    \end{quote}

    More generally, one may consider the following to be instance of \BoyPS{}, in which the cited proof explains(?) why the ability attribution is true:

    \begin{quote}
      Cantor was able to show (by a proof we will not reproduce here) that \([0, 1]\) is equivalent to the power set of the integers, and thus its cardinal number is \(2^{\aleph_{0}}\).\newline
      \mbox{ }\hfill\mbox{(\cite[65]{Partee:1990tu})}
    \end{quote}

    \begin{quote}
      Blok [\hyperlink{cite.Blok:1980th}{16}] was able to give a detailed analysis of frame incompleteness by drawing on algebraic methods.
      In particular, he did so by investigating splittings (a concept from lattice theory) of the lattice of normal modal logics [\dots]
      \mbox{ }\hfill\mbox{(\cite[74]{Blackburn:2007wa})}
    \end{quote}
  }
\end{note}


\begin{note}
  Second is the converse of \BoyPS{}
  \phantlabel{BoyPSC:Start}

  \begin{enumerate}[label=]
  \item
    \label{Boylan:Past-Success:C}
    \BoyPSC{}: \(\text{Past}(S\text{ is able to }\phi) \Rightarrow \text{Past}(S\text{ does }\phi)\)
  \end{enumerate}

  \citeauthor{Bhatt:2008aa}

  \begin{quote}
    The two readings associated with \emph{be able to} allow different interpretive possibilities for indefinite/bare plural subjects.

    \begin{enumerate}[label=(\arabic*), ref=(\arabic*)]
      \setcounter{enumi}{300}
    \item
      A fireman was/Firemen were able to eat five apples.
      \begin{enumerate}[label=\alph*., ref=(301\alph*)]
      \item
        \label{Bhatt:apples:ae}
        Yesterday at the apple eating contest, a fireman was/firemen were able to eat five apples.
        (Past episodic, actuality implication, existentially interpreted subject)
      \item
        In those days, a fireman were/firemen were able to eat five apples in an hour (Generic, no actuality implication, generically interpreted subject)%
        \mbox{}\hfill\mbox{(\citeyear[160]{Bhatt:2008aa})}
      \end{enumerate}
    \end{enumerate}
  \end{quote}

  \ref{Bhatt:apples:ae} `actuality implication'.%
  \footnote{
    Following \textcite{Alxatib:2019wf}:
    %
\begin{quote}
      Actuality Entailments (AEs) are inferences from premises that appear to be modal, like~\ref{Alxatib:a}, but their content is that the modality is effectuated in the evaluation world~---~\ref{Alxatib:b}.
      %
      \begin{enumerate}[label=(\arabic*)]
      \item
        \begin{enumerate}[label=\alph*., ref=(1\alph*)]
        \item
          \label{Alxatib:a}
          Pierre a dû \phantom{to.pfv} prendre le \phantom{e} train \newline
          Pierre had.to.\textsc{pfv} take \phantom{dre} the train\newline
          `Pierre had to take the train'
        \item
          \label{Alxatib:b}
          \emph{Inference}: Pierre took the train.%
          \mbox{}\hfill\mbox{(\citeyear[701]{Alxatib:2019wf})}
        \end{enumerate}
      \end{enumerate}
    \end{quote}
    %
    \citeauthor{Alxatib:2019wf} stresses the reading of `had' in (1a) is `unambiguously deontic' (\citeyear[703]{Alxatib:2019wf}).

    See \textcite{Asher:2012vr}, \textcite{Bhatt:2008aa}, \textcite{Hacquard:2006to,Hacquard:2009ta}, \textcite{Palmer:1977wb}, \textcite{Pinon:2003te}, and~\textcite{Werner:2011tp} for examples and additional discussion of actuality entailments.
  }
  Follows that a fireman/firemen ate five apples.%
  \footnote{
    In contrast, to \BoyPS{}, examples of \BoyPSC{} are plentiful.
    Two examples involving reasoning follow:

    \begin{quote}
      One can then, because of the special ``linear'' nature of the electrical process, calculate the distortion of a very complicated signal, such as Uncle Fred's voice, simply by treating it as a series of gradual ``turnings on'' and ``turnings off'' of the unit step response and adding up their combined causal influence.
      Using his operational calculus, Heaviside was able to calculate the unit step response in very quick order and then solve more complicated cases in the manner suggested.%
      \mbox{ }\hfill\mbox{(\cite[316]{Wilson:1988wx})}
    \end{quote}

     \begin{quote}
       One senses from a reading of Russell how he was able to overlook this point:
       the trouble was his failure to focus upon the distinction between ``propositional functions'' as attributes, or relations-in intension, and ``propositional functions'' as expressions [\dots]%
      \mbox{ }\hfill\mbox{(\cite[152]{Quine:1967tv})}
    \end{quote}
  }

  % Put these together, and specific ability, embedded under past tense reduces to what happened.

  % \begin{enumerate}[label=]
  % \item
  %   \label{Boylan:Past-Success:IFF}
  %   \BoyPSIFF{}: \(\text{Past}(S\text{ is able to }\phi) \Longleftrightarrow \text{Past}(S\text{ does }\phi)\)
  % \end{enumerate}

  The difficulty with respect to \fc{1} is that concluding \(\pv{\phi}{v}\) from \(\Phi\) doesn't entail that \(\pvp{\phi}{v}{\Phi}\) was a \fc{}.

  In short, \(\pvp{\phi}{v}{\Phi}\) is a \fc{} only if there is some action such that the agent is concluding.
  However, possible for an agent to conclude without the progressive being true.

  % For, given \BoyPS{} it is not possible to express Clause~\ref{def:fc:no-pe-bad} as:
  % \begin{enumerate}[label=\emph{n}., ref=(\emph{n})]
  % \item
  %   \label{Ability:past:narrow}
  %   The agent has the ability to \emph{not} [conclude something incompatible with concluding \(\pv{\phi}{v}\) from \(\Phi\)].
  % \end{enumerate}
  % As, so long as the agent concludes \(\pv{\phi}{v}\) from \(\Phi\), then the agent (plausibly) won't have concluded something incompatible, and hence the ability attribution will be true.

  % Hence, negation must scope over the ability attribution, e.g.:

  % \begin{enumerate}[label=\emph{w}., ref=(\emph{w})]
  % \item
  %   \label{Ability:past:wide}
  %   The agent does \emph{not} [have the ability to conclude something incompatible with concluding \(\pv{\phi}{v}\) from \(\Phi\)].
  % \end{enumerate}

  % Still, \ref{Ability:past:wide} will differ in truth value from \ref{Ability:past:narrow} only if the negated variant of \BoyPS{} does not hold:
  % \begin{enumerate}[label=]
  % \item
  %   \label{Boylan:Past-Success:CQC}
  %   \BoyPSCQC{}: \(\text{Past}(S\text{ does \emph{not} }\phi) \Rightarrow \text{Past}(S\text{ is \emph{not} able to }\phi)\)
  % \end{enumerate}
  % \BoyPSCQC{} may seems false on first glace.
  % However, the entailment may be motivated in parallel to \BoyPS{}.
  % For, suppose \citeauthor{Boylan:2020aa} takes the shot and does not hit the bullseye.
  % It seems we may then argue that:

  % \begin{enumerate}[label=\arabic*\('\).]
  %   \setcounter{enumi}{2}
  % \item
  %   You didn't hit the bullseye on that throw.\newline
  %   So, you were not able to to hit the bullseye on that throw.
  % \end{enumerate}
  % So, it is by no means clear that \BoyPSCQC{} fails for the relevant sense of ability.%
  % \footnote{
  %   Further, observe the same substitution applied to \BoyPSC{} intuitively holds.
  %   In particular, consider \citeauthor{Bhatt:2008aa}'s \ref{Bhatt:apples:ae} where the firemen \emph{weren't} able to eat five apples.
  % }
\end{note}


\begin{note}
  To summarise, immediate issue is with entailments.
  Seems these don't coincide with \fc{}.

  However, \BoyPS{}, \BoyPSC{} only concern past.

  Possible to give an indirect account of ability, tying specifically to present.

  Assuming unique sense of specific ability, explore.

  For, it is not immediate that \BoyPS{} and \BoyPSC{} to hold where `Future' or `Present' is substituted for `Past'.

  Additional motivation required.

  If do, then significant problem.
  If do not, though, remains a general question about relationship.

  And, if distinct senses of specific ability, issue is identifying relevant sense for reduction.
  In particular, though \BoyPS{} is difficult, \BoyPSC{} is robust, and raises issue.

  Following section will focus on \BoyPS{} and idea of control.
\end{note}

\section{(Specific) ability and control}
\label{cha:sec:fcs-def:ability:control-intuition}

\begin{note}[Segue]
  \autoref{cha:sec:fcs-def:ability:past} raised concerns about specific ability and what does (or does not) happen.

  In this section we present and focus on idea of `control' common to a various analyses of specific ability.
  And, we will argue that idea of control as captured by the act conditional analysis of ability is incompatible with an account of \(\pvp{\phi}{v}{\Phi}\) being a \fc{1} for an agent in terms of the agent have the ability to conclude \(\pv{\phi}{v}\) from \(\Phi\).%
  \footnote{
    Part of the interest of \textcite{Boylan:2020aa} is combining the validity of \BoyPS{} with the failure of `Present Success'.
    However, the combination isn't of interest to us.
    Though, ensures specific ability.
  }
\end{note}

\begin{note}[\AbControl{}]
  \phantlabel{ability:control}
  \textcite{Mandelkern:2017aa} express the idea of control as follows:%
  \footnote{
    \label{fn:control-accounts}
    A similar account of the control intuition is found in \textcite{Jaster:2020wv}:

  \begin{quote}
    [\dots] think of the ability to sing a song, to build a shag, to play tennis --- all have an action as their manifestation: the agent controls what is going on and she also controls whether to exercise the ability at all.%
    \mbox{ }\hfill\mbox{(\cite[34]{Jaster:2020wv})}
  \end{quote}

  And, \citeauthor{Boylan:2020aa}'s (\citeyear{Boylan:2020aa}) statement of the control intuition is limited to a specific example:
      \begin{quote}
        Imagine a great wave is rising and I have dashed into the sea with my surfboard.
        You know nothing about me: perhaps I am one of the world’s great surfers; perhaps I am a fool.
        [\dots]

    When said before the fact, the claim that I can surf that wave is strong it says that surfing that wave is within my control.
    This intuition, call it the \emph{control intuition} [\dots]%
    \mbox{ }\hfill\mbox{(\citeyear[1]{Boylan:2020aa})}
  \end{quote}

  Similar accounts may be also be found in~\textcite{Brown:1988tl},~\textcite{Kikkert:2022wp}, and~\textcite{Horty:1995wu}.
  }
  {
    \newbox\qqBoxA
    \newdimen\qqCornerHgt
    \setbox\qqBoxA=\hbox{$\ulcorner$}
    \global\qqCornerHgt=\ht\qqBoxA
    \newdimen\qqArgHgt
    \def\Quinequote #1{%
      \setbox\qqBoxA=\hbox{$#1$}%
      \qqArgHgt=\ht\qqBoxA%
      \ifnum     \qqArgHgt<\qqCornerHgt \qqArgHgt=0pt%
      \else \advance \qqArgHgt by -\qqCornerHgt%
      \fi \raise\qqArgHgt\hbox{$\ulcorner$} \box\qqBoxA %
      \raise\qqArgHgt\hbox{$\urcorner$}}

    \begin{quote}
      When someone says \(\Quinequote{\text{I [am able to] }\varphi}\), she is assuring her interlocutors that \(\sem[c]{\varphi}\) is within her control in a certain way.%
      \mbox{ }\hfill\mbox{(\citeyear[326]{Mandelkern:2017aa})}
    \end{quote}
  }
  \noindent%
  For ease of reference we will refer to the idea expressed via `\AbControl{}'.
\end{note}

\begin{note}[Control via \citeauthor{Schwarz:2020aa}]
  Similar to \citeauthor{Mandelkern:2017aa}, \textcite{Schwarz:2020aa} motivates \AbControl{} as follows:

  \begin{quote}
    Suppose Cyril does not know the first 10 digits of \(\pi\).
    Intuitively,~\ref{Schwarz:pi} is then false.

    \begin{enumerate}[label=(\arabic*), ref=(\arabic*)]
      \setcounter{enumi}{2}
    \item
      \label{Schwarz:pi}
      Cyril can recite the first 10 digits of \(\pi\).
    \end{enumerate}

    [\dots]
    when we say that someone can recite the first 10 digits of \(\pi\), we don't just mean that no relevant facts preclude them from uttering `three, one, four,' etc.
    Rather, the agent must have a certain kind of intentional control over performing the act under the description of `reciting digits of \(\pi\)'.\newline
    \mbox{ }\hfill\mbox{(\citeyear[2]{Schwarz:2020aa})}
  \end{quote}
\end{note}

\begin{note}[Control via \citeauthor{Boylan:2020aa}]
  Likewise, \citeauthor{Boylan:2020aa} (\citeyear{Boylan:2020aa}), inspired by~\textcite{Kenny:1976vh} motivates the idea of control with the following scenario:
  \begin{quote}
    \begin{quote}
      \textbf{Unreliable Dartboard}.
      I am a fairly bad dartplayer.
      I regularly hit the bottom half when I aim for the top; and vice versa.
      But I never miss the board entirely.
    \end{quote}

    I am about to take a shot.
    I am skilled enough to know I will hit the board; so I know the following:

    \begin{enumerate}[label=(\arabic*)]
      \setcounter{enumi}{6}
    \item
      I will hit the top half of the board on this throw or I will hit the bottom half of the board on this throw.
    \end{enumerate}

    But it does not seem that I should ascribe myself either of the following abilities here:

    \begin{enumerate}[label=(\arabic*), ref=(\arabic*), resume]
    \item
      I can hit the top on the throw.
    \item
      I can hit the bottom on this throw.
    \end{enumerate}

    Even the disjunction does not seem true:

    \begin{enumerate}[label=(\arabic*), ref=(\arabic*), resume]
    \item
      \label{Boylan:10}
      I can hit the top of the board on this throw or I can hit the bottom of the board on this throw.%
      \mbox{ }\hfill\mbox{(\citeyear[3]{Boylan:2020aa})}
    \end{enumerate}
  \end{quote}

  Intuitively, \citeauthor{Boylan:2020aa} lacks control over where the dart lands on the board, the exercises control over whether the dart lands on the dartboard.
  (\citeyear[\S2,19--20]{Boylan:2020aa})
\end{note}

\begin{note}[\BoyVS{}]
    As \citeauthor{Boylan:2020aa} observes,~\ref{Boylan:10} may be further expanded into a more complex disjunction of regions on the dartboard.
  (\citeyear[4]{Boylan:2020aa})
  For example, intuitively it is not the case that:
  \begin{enumerate}[label=(\arabic*'), resume]
    \setcounter{enumi}{10}
  \item
    I can hit outside the bullseye this throw or I can hit the upper-left-quadrant of the bullseye on this throw or I can hit the lower-right-quadrant of the bullseye on this throw or \dots
  \end{enumerate}

  Indeed, \AbControl{} leads to the \emph{invalidity} of \BoyVS{}:

  \begin{enumerate}[label=]
  \item
    \label{Boylan:Or-Success}
    \BoyVS{}: \(S\text{ will }\phi \lor S\text{ will }\psi \Rightarrow S\text{ is able to }\phi \lor S\text{ is able to }\psi\)%
    \mbox{ }\hfill\mbox{(\citeyear[\S1.2]{Boylan:2020aa})}
  \end{enumerate}
\end{note}

\begin{note}[Need to get precise]
  Now, \AbControl{} is an idea, but is under-specified by the motivation provided.
  \citeauthor{Mandelkern:2017aa} hedge with `in a certain way', \citeauthor{Schwarz:2020aa} hedges with `certain kind', and \citeauthor{Boylan:2020aa} does not provide an explicit statement of the idea (see Footnote~\ref{fn:control-accounts}).
  Hence, \(\pvp{\phi}{v}{\Phi}\) being a \fc{1} for an agent may be equivalent to the agent having the (controlled) ability to conclude \(\pv{\phi}{v}\) from \(\Phi\).
  Or, the proposed equivalence may fail.

  So, interest turns to details of the accounts of `is able to' advanced by \textcite{Mandelkern:2017aa} and \textcite{Boylan:2020aa} in order to obtain sufficient clarity on what \AbControl{} amounts to on their understanding.
\end{note}

\begin{note}[ACA]
  We present a generalised account of the `act conditional' analysis of ability, common to \textcite{Boylan:2020aa}, \textcite{Mandelkern:2017aa}, and \textcite{Schwarz:2020aa}.%
  \footnote{
    Though \citeauthor{Schwarz:2020aa} is non-committal with respect to a formal account of ability (\citeyear[cf.][13]{Schwarz:2020aa}), the spirit of \citeauthor{Schwarz:2020aa}'s analysis is sufficiently close to \citeauthor{Boylan:2020aa}'s for the issue to arise:
    `[A]n agent has the ability to \(\phi\) iff there are accessible worlds at which she \(\phi\)s simply by deciding to \(\phi\).' (\citeyear[19]{Schwarz:2020aa})
    Decision to action, but then the decision itself must sufficiently determine the action.
  }

  \[%
    \sem[c,w]{\text{S is able to }\varphi} = 1\text{ iff }\exists A \in \mathcal{A}_{S,c,w,t}\colon \forall v \in f_{c}(\text{S does }A,w),  \sem[c,v]{\varphi(S)} = 1%
  \]

  Where:
  %
  \begin{itemize}
  \item
    \(f_{c}\) is a selection function from proposition-world pairings to set of worlds.
  \item
    \(\mathcal{A}_{S,c,w}\) is the set of actions that are available to \(S\) in context \(c\) and world \(w\).
  \end{itemize}
  %
  So, \(S\text{ is able to }\varphi\) is true at some world \(w\) in context \(c\), just in case there is some action available to the agent, such that for every world in which it is true that \(S\text{ tries to}A\) determined by the selection function \(f_{c}\), it is the case that \(S \varphi\text{s}\).%
  \footnote{
    Strictly, both \citeauthor{Mandelkern:2017aa} and \citeauthor{Boylan:2020aa} omit universal quantification over worlds returned by the selection function.
    For \citeauthor{Mandelkern:2017aa}, as discussed below, the selection function returns a unique world, though, as discussed below, this assumption is problematic.
    For \citeauthor{Boylan:2020aa}, universal quantification is implicit by embedding `\(\varphi(S)\)' under a modal `\(\mathcal{W}\)' corresponding to `will'.
    % Issue is `if performs act, then \dots'
    % Restrictor semantics for conditional.
    % \(\sem[c,w,f]{\text{if }\phi,\psi} = 1 \text{ iff } \sem[c,w,f^{\sem[c,w,f]{\phi}}]{\psi} = 1\).
    % With universal, effectively inserting a modal.
    % Complexity of \citeauthor{Boylan:2020aa}'s account is getting the right modal.
    % Simplicity of \citeauthor{Mandelkern:2017aa}'s account is avoiding modal by assuming unique world.
  }

  Paraphrased, the act conditional analysis of ability holds:
  `\(S\) is able to \(\varphi\)' is true just in case there is some action \(A\) available to \(S\) such that if \(S\) tried to \(A\) then S would \(\varphi\).
\end{note}

\begin{note}[Selection functions]
  The primary difference between the analyses of \citeauthor{Mandelkern:2017aa} and \citeauthor{Boylan:2020aa} is the specification of \(f_{c}\), though in practice the difference seems minor:
  \begin{itemize}
  \item
    For \citeauthor{Mandelkern:2017aa},
    \(f_{c}\) is~\citeauthor{Stalnaker:1968vt}'s selection function.
    I.e.\ \(f_{c}(\psi, w) = \{v\}\) where \(v\) is the `closest' world to \(w\) where \(\psi\) is true.
    (\citeyear[Cf.][314]{Mandelkern:2017aa})

    However, the assumption of a unique `closest' world is clearly problematic given \BoyVS{}.
    For, an agent has the ability to throw a dart at the dartboard.
    Hence, in the closest possible world where the agent attempts to throw a dart, the agent succeeds.
    Further, the dart lands at some exact region of the dartboard.
    Hence, as there is only one closest world to consider, `\(S\) it able to throw a dart at the dartboard' is strengthened to `\(S\) it able to throw a dart at the \emph{exact region of the} dartboard'.%
    \footnote{
      In particular, \citeauthor{Mandelkern:2017aa} do not require that what the agent tries to do and what the agent does satisfy the same description (\citeyear[310,314]{Mandelkern:2017aa}).

      The same problem applies to the `orthodox approach' of~\textcite{Hilpinen:1969vw}, \textcite{Kratzer:1977aa,Kratzer:1981vn}~and~\textcite{Lewis:1976us}.
      See \textcite[\S1.3]{Boylan:2020aa} and \textcite[\S2]{Mandelkern:2017aa} for more on the orthodox approach.
    }

    Still, a \citeauthor{Lewis:1973th}ian approach where the selection function return a set of `closest' worlds resolves this issue.
    For, we may assume that the closest possible worlds determine some inexact region of the dartboard.
  \item
    For \citeauthor{Boylan:2020aa}, rather than selecting `close' worlds, \(f_{c}\) selects all worlds which are identical to \(w\) up until time \(t\) (in which \(S\) does \(A\)).%
    \footnote{
      Strictly, there is more.
      For, Non-classical Strong Kleene account of disjunction.
      (\citeyear[\S5]{Boylan:2020aa})
      Though, I really don't get it.
      Just think of \(\mathcal{W}\) as \(G\).
      `Indeterminate' is just \(F \phi \land F \lnot \phi\).
    }
  \end{itemize}
  %
  For present purposes, the key part of the act conditional analysis for capturing \AbControl{} is that \(S \varphi\)s \emph{follows from} \(S\text{ does }A\) in all worlds captured by the selection function.

  In this respect, that the agent \(\varphi\)s is a \emph{consequence} of performing \(A\).
  In particular, it is not possible for \(A\) to set `\(\varphi\)ing in motion'.
  For, we have seen with progressive and the imperfective paradox, \(\varphi\)ing does not entail an agent \(\varphi\)s.
\end{note}

\begin{note}[Availability]
  What counts as an available action is a more complex issue.
  Thankfully, the details of \citeauthor{Mandelkern:2017aa} and \citeauthor{Boylan:2020aa} may be avoided.%
  \footnote{
    Specifically, \citeauthor{Boylan:2020aa} says little on what makes it the case that an action is available to an agent:
    \begin{quote}
      I think of an agent's available actions as their options.
      And, for simplicity at least, we can typically think of options as a set of tryings.%
      \mbox{ }\hfill\mbox{(\citeyear[14]{Boylan:2020aa})}
    \end{quote}
    In contrast, \citeauthor{Mandelkern:2017aa} consider the issue in detail.
    In short:
    \begin{quote}
      [A]n action counts as practically available only if the agent knows that it is a way of bringing about the prejacent \emph{relative to a given description of her practical situation}.%
      \mbox{ }\hfill\mbox{(\citeyear[321]{Mandelkern:2017aa})}
    \end{quote}
    On my understanding, there is still some gap between knowing an action is a way of bringing something about and performing the action.

    Hence, the account allows for the possibility that an agent has the ability and fails.
    For, may fail to perform the relevant action.
      See \citeauthor{Maier:2013vk} the importance of allowing for failure.
  }

  For present purposes, a sufficient understanding of when an action is available to an agent by considering \citeauthor{Boylan:2020aa}'s scenario illustrating the invalidity of \BoyVS{}.
  For, if the agent throws a dart and it hits a certain region of the dartboard, then the agent performed the act of throwing the dart at that region.
  However, throwing the dart at that region could not have been an action available to the agent on pain of \BoyVS{} having a true premise and true conclusion.
  Hence, if an agent \emph{lacks} the ability to \(\varphi\), then it cannot be the case that there is an action \(A\) available in which the agent \(\varphi\)s by doing \(A\).
\end{note}

\begin{note}[Summary of \AbControl{}]
  To summarise, it seems that on an act conditional analysis of ability, \AbControl{} amounts to the availability of some action \(A\) such that the agent \(\varphi\)ing is a consequence of performing \(A\).
\end{note}

\begin{note}[Difficulty with \fc{1}]
  With understanding of \AbControl{} in hand, we now turn to \fc{1}.

  We make the simple observation that \(\pvp{\phi}{v}{\Phi}\) may be a \fc{} without the agent having appropriate control (in the sense of \AbControl{}) over concluding \(\pv{\phi}{v}\) from \(\Phi\).

  Specifically, cases where \(\pvp{\phi}{v}{\Phi}\) is a \fc{} but for any action \(A\), it is either the case that \(A\) is inconsequential or unavailable.

  The particular \fc{} is of little importance, so we take the abstract \(\pvp{\phi}{v}{\Phi}\)-pairing.
  The key to failure of \AbControl{} is the assumption that the agent does \emph{not} have the ability to avoid distraction.%
  \footnote{
    E.g.\ the agent may be interrupted at any time, become bored, or think of something else they would prefer to do.
  }
  In particular, consider relatively simple tasks such as long but simple calculations, simple sudoku puzzles, basic chess problems, or routine proofs.

  Now, suppose \(\pvp{\phi}{v}{\Phi}\) being a \fc{0} for an agent is equivalent to the agent having the ability to conclude \(\pv{\phi}{v}\) from \(\Phi\) (where \AbControl{} holds for the relevant sense of ability).

  Given \AbControl{} it must be the case that concluding \(\pv{\phi}{v}\) from \(\Phi\) is a result of performing some action \(A\).
  However, by assumption, the agent does not have the ability of avoid distraction, and so \(A\) is not an action available to the agent.
  For, if \(A\) were available to the agent, the agent would \emph{have} the ability to avoid distraction.

  Conversely, suppose any available action allows for the possibility of distraction.
  Then, it straightforwardly follows that concluding \(\pv{\phi}{v}\) from \(\Phi\) is \emph{not} a result of performing that action.
  For, if the agent gets distracted, then the do not conclude \(\pv{\phi}{v}\) from \(\Phi\).

  In short, \AbControl{} requires an available action such concluding \(\pv{\phi}{v}\) from \(\Phi\) is a result of performing some action.
  However, \(\pvp{\phi}{v}{\Phi}\) being a \fc{} tolerates the absence of any such action.

  We may express the difference in either of two ways.
  \begin{enumerate}[label=\arabic*.]
  \item
    In order for \(\pvp{\phi}{v}{\Phi}\) to be a \fc{} it need only be the case that there is some action which results in the agent concluding \(\pv{\phi}{v}\) from \(\Phi\) (and no action where the agent does not conclude anything incompatible), though this action does not need to be \emph{available} to the agent.
  \item
    In order for \(\pvp{\phi}{v}{\Phi}\) to be a \fc{} there must be some action available to the agent, but it need not be the case that concluding \(\pv{\phi}{v}\) from \(\Phi\) (and no action where the agent does not conclude anything incompatible), is a \emph{consequence} of performing the action.
  \end{enumerate}
  As indicated by interest in the progressive, I think the second expression is correct, but either is sufficient observe that \(\pvp{\phi}{v}{\Phi}\) being a \fc{} does not require \AbControl{} over concluding \(\pv{\phi}{v}\) from \(\Phi\).
\end{note}

\section{Summary}

\begin{note}
  Entertained reducing \fc{1} to ability.
  Specific, rather than general ability.
  However, questions about specific ability.
  Specifically, with what happens, and the past.
  And, \AbControl{}.
\end{note}

\begin{note}
  I have not argued that there is no sense of `ability' such that \(\pv{\phi}{v}\) being a \fc{0} from \(\Phi\) for an agent is equivalent to the agent having the ability to conclude \(\pv{\phi}{v}\) from \(\Phi\).
  However, identifying the (or a) sense of ability suitable for the equivalence is difficult.

  Hence, we pursue an account of \(\pv{\phi}{v}\) being a \fc{0} from \(\Phi\) in terms of events in progress.
\end{note}

\section[Independent difficulty]{Independent difficulty \hfill (Optional)}

\begin{note}
  This is the `plan' account of ability.
  It's kind of insane.
  Whether or not ability reduces to action such that choice and secure outcome.
\end{note}

\begin{note}
  This is kind of wild.
  For, actions are kind of huge.
  Similar to that paper with minimalism about intentions.
\end{note}

\begin{note}
  Our direct interest with account finishes with universal.
  However, clear additional problem.
  Co-operation.
\end{note}

\begin{note}
  Uh, think.
  Has the ability to X with my help.
  There's no action in advance.
  For, whatever is chosen, I intervene prior, changing the course.
  Well, the point is, I only help if the agent gives up on whatever they had been planning to do.

  This isn't odd, cooperative activity.
  So, actually, refine example a little.
  For, point is that there's the cooperation condition.
\end{note}




%%% Local Variables:
%%% mode: latex
%%% TeX-master: "master"
%%% TeX-engine: luatex
%%% End:
