\makeatletter
\renewcommand{\PackageInfo}[2]{}% Remove package information
\renewcommand{\@font@info}[1]{}% Remove font information
% \renewcommand{\@latex@info}[1]{}% Remove LaTeX information
\makeatother

\PassOptionsToPackage{unicode}{hyperref}

\documentclass[10pt]{article}
\usepackage[british]{babel}

\usepackage[margin=.75in]{geometry}

\usepackage{amsthm}         % (in part) For the defined environments
\usepackage{mathtools}      % Improves  on amsmaths/mtpro2
\usepackage{amssymb}

% % % My packages % % %
\usepackage{myNotation}
\usepackage{ThesisCustom}
\usepackage{CustomEnvs}
\usepackage{ThesisFig}
% % % % % % % % % % % %

% \usepackage{selnolig}% For suppressing certain typographic ligatures automatically
% % % % % % %
\usepackage{xfrac}
\usepackage{array}
\usepackage{arydshln}
\usepackage{multirow}
% https://tex.stackexchange.com/questions/12703/how-to-create-fixed-width-table-columns-with-text-raggedright-centered-raggedlef
\newcolumntype{L}[1]{>{\raggedright\let\newline\\\arraybackslash\hspace{0pt}}m{#1}}
\newcolumntype{C}[1]{>{\centering\let\newline\\\arraybackslash\hspace{0pt}}m{#1}}
\newcolumntype{R}[1]{>{\raggedleft\let\newline\\\arraybackslash\hspace{0pt}}m{#1}}

% \usepackage{silence}
% \WarningsOff[marginnote]% Suppress warnings related to package


% % % The bibliography % % %
\usepackage[%
backend=biber,
style=authoryear-comp,
bibstyle=authoryear,
citestyle=authoryear-comp,
uniquename=false,
backref=false,
hyperref=true,
url=false,
isbn=false,
doi=false,
useprefix=true,
maxbibnames=99,
]{biblatex}
\DeclareFieldFormat{postnote}{#1}
\DeclareFieldFormat{multipostnote}{#1}
% \setlength\bibitemsep{1.5\itemsep}
\newcommand{\noopsort}[1]{}
\addbibresource{Ability.bib}
\DefineBibliographyExtras{british}{\def\finalandcomma{\addcomma}} % Enable Oxford Comma
% % % % % % % % % % % % % % % 

\usepackage[inline]{enumitem}
\SetEnumitemValue{labelindent}{standard}{.25\parindent}
\setlist[itemize]{labelindent=standard} %, leftmargin=1.5em}
\setlist[enumerate]{labelindent=standard} %, leftmargin=1.5em}

\newlist{itenum}{enumerate}{1}
\setlist[itenum]{
  style=standard,
  font=\normalfont,
  labelwidth = \widthof{\emph{Then}:},
} 

\usepackage[export]{adjustbox}
\usepackage{subcaption}
\usepackage{float}

% % % % % % % % % % % % TIKZ
\usepackage{tikz}
\usetikzlibrary{bending,arrows,calc,arrows.meta,patterns,fadings}
\usetikzlibrary{trees}
\usetikzlibrary{backgrounds, positioning, fit, backgrounds}
\usetikzlibrary{math}
\usetikzlibrary{tikzmark}
% % % % % % % % % % % % 


\usepackage{graphicx} % for images (png/jpeg etc.)
\usepackage{caption} % for \caption* command

% % % % % % % % % % % % MY COMMANDS

\makeatletter
\renewcommand\paragraph{\@startsection{paragraph}{4}{\z@}%
  {-3.25ex\@plus -1ex \@minus -.2ex}%
  {1.5ex \@plus .2ex}%
  {\normalfont\normalsize\bfseries}}
\makeatother

\makeatletter
\renewcommand\subparagraph{\@startsection{subparagraph}{4}{\z@}%
  {-3.25ex\@plus -1ex \@minus -.2ex}%
  {1.5ex \@plus .2ex}%
  {\normalfont\normalsize\bfseries}}
\makeatother


% % % Fonts% %
\usepackage[no-math]{fontspec}
\defaultfontfeatures{Ligatures={NoDiscretionary, NoHistoric, NoRequired, NoContextual}}
% \protrudechars=2 %
\adjustspacing=2 %
\newfontfeature{Microtype}{protrusion=default;expansion=default;}
\setmainfont[Microtype]{Libertinus Serif}
\setsansfont[Microtype,Scale=MatchLowercase]{Libertinus Sans}
\setmonofont{Libertinus Mono}

\usepackage[math-style=ISO]{unicode-math}
\setmathfont{Libertinus Math}

\usepackage[
breaklinks,
bookmarks=false,
hidelinks,
linkcolor=highlight,
citecolor=highlight,
% colorlinks=true,
colorlinks=false,
]{hyperref}

\usepackage{csquotes}

\addto\extrasbritish{
  \def\chapterautorefname{Chapter}
  \def\sectionautorefname{Section}
  \def\subsectionautorefname{Section}
  \def\subsubsectionautorefname{Section}
  \def\paragraphautorefname{Section}
  \def\subparagraphautorefname{Section}
  \def\AlgoLineautorefname{Line}
  \def\pageautorefname{Page}
  \def\footnoteautorefname{Footnote}
  \def\scenarioCounterautorefname{Scenario}
  \def\observationCounterautorefname{Observation}
  \def\specificationCounterautorefname{Specification}
  \def\illustrationCounterautorefname{Illustration}
  \def\sketchCounterautorefname{Sketch}
  \def\linkCounterautorefname{Link}
  \def\constraintCounterautorefname{Constraint}
  \def\questionCounterautorefname{Question}
  \def\assumptionCounterautorefname{Assumption}
  \def\defiCounterautorefname{Definition}
  \def\propCounterautorefname{Proposition}
  \def\ideaCounterautorefname{Idea}
  \def\condCounterautorefname{Condition}
  \def\intuitionCounterautorefname{Intuition}
  \def\notationCounterautorefname{Notation}
  \def\applicationCounterautorefname{Application}
}

\title{
  Foregone-conclusions \quad / \quad Handout notes
}
% \author{Ben Sparkes}
\date{ }

\begin{document}

\maketitle





\section{\eiw{3} an agent concludes}
\label{sec:overview}

\subsection{Abstractions}

  % Think of this as a more flexible approach to propositional attitudes.

  \noindent
  Two relations:

  \begin{itemize}
  \item
    An agent concludes a \prop{0} has some \val{0} \underline{\emph{from}} some \pool{}.
  \item
    A \prop{0} having some \val{0} \underline{\emph{\fof{}}} some \pool{} (from the \agpe{}, relative to an event).
  \end{itemize}

  Note, a \fof{} relation may hold between arbitrary \prop{0}-\val{0}-\pool{0} pairings.
  % If you like, think of this as a relation of support.
  % However, there's nothing normative here.
  % This is what distinguishes \fofr{} from rationalisations.
  % For the moment, intuitive.
  % Shortly give a pair of sufficient conditions for a \fofr{}.
\end{note}

% \newpage

\section{Two questions and a constraint \hfill (\qWhy{}, \qHow{}, and \issueInclusion{})}
\label{sec:target}


\begin{note}
  Relationship between two questions about \eiw{3} an agent concludes.

  \qWhy{} \qHow{}, \issueInclusion{}.
\end{note}

\begin{note}
  \begin{question}{questionWhy}{\qWhy{}}
    Given \(e\) is an \eiw{0} \vAgent{} concludes \(\pv{\phi}{v}\) from \(\Phi\):
    \begin{itemize}
    \item
      Which \fofr{1} are included in (partial) explanations of \emph{why} \(e\) is an \eiw{0} \vAgent{} concludes \(\pv{\phi}{v}\) from \(\Phi\) (rather than an \eiw{0} some other thing happens)?
    \end{itemize}
    \vspace{-1\baselineskip}
  \end{question}

  Shuffling cards.
  If random shuffle, no answer to why.
  If magician performing sleight of hand, answer to why for chosen card.

  \qWhy{} is about \fingfr{}.
  Which \fingfr{} provide some insight.
\end{note}

\begin{note}
  \begin{question}{questionHow}{\qHow{}}
    Given \(e\) is an \eiw{0} \vAgent{} concludes \(\pv{\phi}{v}\) from \(\Phi\):
    \begin{itemize}
    \item
      Which past or present events (partially) explain \emph{how} \(e\) is an \eiw{0} \vAgent{} concludes \(\pv{\phi}{v}\) (rather than an \eiw{0} some other thing happens)?
    \end{itemize}
    \vspace{-1.5\baselineskip}
  \end{question}
\end{note}


\begin{note}
  \begin{constraint}{consInclusion}{\issueInclusion{}}
    \mbox{ }
    \vspace{-\baselineskip}
    \begin{itenum}
    \item[\emph{If}:]
      \fingfb{\(\pv{\psi}{v'}\)}{\(\Psi\)} answers \qWhy{}.
    \item[\emph{Then}:]
      An \eiw{0} the agent concludes \(\pv{\psi}{v'}\) from \(\Psi\) answers \qHow{}.
    \end{itenum}
    \vspace{-\baselineskip}
  \end{constraint}

  For some intuition, think of Davidsonian style rationalisations.
  Davidson's argument is that rationalisation are causal relations.
  So, constraint holds.
  Difficultly with approach via rationalisations is that rationalisations are a specific abstraction.
  I'm interested in something a little different.
\end{note}


\begin{note}
  This is where some important work is done.
  The thing is, this is set up in a careful way.
  \fofr{3} answer both \qWhy{} and \qHow{}.
  So, it's possible for a \fofr{} between a \prop{0}-\val{0} pair and \pool{} distinct from the \prop{0}-\val{0}-\pool{0} pair directly related to the conclusion to answer both \qWhy{} and \qHow{}.

  Key observation is that if the agent has not concluded \(\pv{\psi}{v'}\) from \(\Psi\) then \(\pv{\psi}{v'}\) \fof{} \(\Psi\) does not answer \qHow{}.
\end{note}


\begin{note}
  Might like to view \qWhy{} and \qHow{} as, respectively, asking about constitutive and causal explanations.
  Still, I see no need to relate these questions to broader types of explanation.
  If they do correspond, then great.
  But, I'm not interested in whether a connexion between these types of explanation holds in general, or in the specific case of an \eiw{} an agent concludes.

  As focus on getting answers to \qWhy{}, maybe you'll tell me more in the Q\&A.
\end{note}


\section{Approach}
\label{sec:approach}

\begin{note}
  I argue \issueInclusion{} fails to hold.
\end{note}


\begin{note}
  For \issueInclusion{} to fail, need a counterexample.
  Though this is more involved.

  \begin{enumerate}
  \item
    Sense of `why' and `how'.
  \item
    Answers to \qWhy{}.
  \item
    Way to identify \fingfr{}.
  \end{enumerate}
\end{note}

\begin{note}
  Not really about the result.
  Argument.
  Fairly technical.
  Framework so it's possible to create a direct argument.
  Only issue is whether the definitions, ideas, etc.\ work out.
  I.e.\ whether you agree.

  Stress this point.
  Well, what do I want to stress?
  `Makes sense'.
  So, think of epistemology.
  Various cases, intuition about whether knows.
  Intuition is taken as input for creating theory.
  This is the sort of Rawlsian reflective equilibrium.

  Important constraint.
  Should not be the case framework presupposes answer.
  See how these definitions come together.
  
\end{note}

\newpage

\section{Answers to \qWhy{}}
\label{sec:answers-qwhy}

\begin{note}
  Basic idea: Obtain answers to \qWhy{} by considering conclusions in progress.
\end{note}


\begin{note}
  Part of why this is interesting.
  Maybe.
  No subjunctives, with the exception of thinking about events in progress.
  If you can do events in progress without subjunctives, then, things go through.
  Even projections.
  These talk about possibility, and entail various subjunctives, but that's about it.
\end{note}


\subsection{Events, in progress}
\label{sec:events-progress}

\begin{note}
  Roughly, think of events in progress in terms of the (English) progressive aspect.
  %
  \begin{enumerate}
  \item
    The agent is making soup.\newline
    \mbox{ } \hfill \(\leadsto\) An \eiw{0} the agent makes soup is in progress.
  \item
    The agent is reading Henley's `Invictus'.\newline
    \mbox{ } \hfill \(\leadsto\) An \eiw{0} the agent reads Henley's `Invictus' is in progress.
  \item
    The agent is riding the slope.\newline
    \mbox{ } \hfill \(\leadsto\) An \eiw{0} the agent rides the slope is in progress.
  \end{enumerate}
  %
  This isn't quite right, as the (English) progressive has some quirks.
  Though, it's good enough.
\end{note}

\begin{note}
  Key feature of events in progress is captured by the term.
  It is the case that some other event is in progress.
\end{note}

\subsection{Events and descriptions}
\label{sec:events-descriptions}


\begin{note}
  Davidsonian.
  However, a little loose to help smooth some ideas.
  Basically, fix and event and definition of that event.

  This is particularly useful for talking about events in progress.
  If have an \eiw{} an agent concludes then there are some aspects of the event which were in progress, and other things that were not in progress.

  Consider \autoref{illu:gist:roots:F}.
  Agent's conclusion was in progress, but plausible order of steps 4 and 5 was not in progress.
  Same with Step 5 of \autoref{illu:gist:roots:QF}, and here unspecified --- add or subtract first.

  So distinct between these two things helps fix an event fairly easily, while narrowing down certain parts of the event.
\end{note}

\begin{note}
  \begin{itemize}
    \item
      \(\edn{}\) picks out a specific event (i.e.\ an event under some unique description).
    \item
      \(\ed{}\) captures (the specific event) \(\edn{}\) under the particular description \(\edo{}\).
    \end{itemize}
\end{note}


\subsection{\se{3}}

\begin{note}
  \begin{rdefinition}{def:se}{\se{3}}
    \vspace{-\baselineskip}
    \begin{itemize}
    \item
      \(\ed{\flat}\) is a \emph{\se{0}} of \(\ed{}\).
    \end{itemize}
    \emph{If and only if}:
    \begin{itemize}
    \item
      Clauses~\ref{assu:p:se:prog} and \ref{assu:p:se:hCon} hold:
      \begin{enumerate}[label=\Alph*., ref=\Alph*]
      \item
        \label{assu:p:se:prog}
        \(\ed{\flat}\) is such that \(\ed{}\) is in progress.
      \item
        \label{assu:p:se:hCon}
        \(\ed{}\) partly happens as a result of \(\ed{\flat}\).
      \end{enumerate}
    \end{itemize}
    \vspace{-\baselineskip}
  \end{rdefinition}

  \noindent%
  \(\ed{\flat}\) being a \emph{\se{0}} of \(\ed{}\) concerns specific events \(\edn{\flat}\) and \(\edn{}\).
  And, of interest is whether specific descriptions \(\edo{}\) and \(\edo{\flat}\) capture a particular connexion between \(\edn{\flat}\) and \(\edn{}\).

  Still, our interest is only with parts of \(\edn{\flat}\) and \(\edn{}\), respectively.
  Intuitively:
  %
  \begin{itemize}
  \item
    Clause~\ref{assu:p:se:prog} `looks forward':

    The description \(\edo{\flat}\) of \(\edn{\flat}\) captures that an event described by \(\edo{}\) is in progress.
  \item
    Clause~\ref{assu:p:se:hCon} `looks backward':

    \(\edn{}\) as described by \(\edo{}\) happened in part as a result of \(\edn{\flat}\) as described by \(\edo{\flat}\).
  \end{itemize}

  The role of Clause~\ref{assu:p:se:prog} is to ensure the event \(\edn{}\) is favoured over some other event.

  And, the role of Clause~\ref{assu:p:se:hCon} is to ensure \(\edn{}\) happens as a result of being favoured.
\end{note}

\subsection{\progEx{2}}
\label{sec:progex}

\begin{note}
  Answers to \qWhy{}.

  This is where things get a little involved.

  \progEx{}.

  Idea is, over any other event.
  Event over any other event.
  So, events in progress.
  Some examples.
  A little more careful.
  Event in progress, an event of interest happens as a result.
  Given this, event in progress is a partial answer to why the event happened, given the sense of why outlined.
\end{note}


\begin{note}
    \begin{rproposition}{prop:PEbasic}{\progExI{}}%
    Given \(\ed{}\) is an \eiw{0} \vAgent{} does \(a\), and the sense of `why' present in \qWhy{}:

    \begin{itenum}
    \item[\emph{If}:]
      \(\ed{\flat}\) is a \se{} of \(\ed{}\).
    \item[\emph{Then:}]
      \(\edo{\flat}\) being true of \(\edn{\flat}\) explains `why' \(\ed{}\) happened.
    \end{itenum}
    \vspace{-\baselineskip}
  \end{rproposition}

  \noindent%
  Given \(\ed{\flat}\) is a \se{} of \(\ed{}\), \(\edo{\flat}\) being true of \(\ed{}\) explains `why' \(\ed{}\) happened in a basic sense, as by Clause~\ref{assu:p:se:hCon} \(\ed{}\) partly happened as a result of \(\ed{\flat}\).
  And, Clause~\ref{assu:p:se:prog} allows us to expand this observation to see \(\ed{}\) partly happened as a result of something which favoured \(\ed{}\) happening over any (incompatible) event.
\end{note}

\paragraph{Example}

\begin{note}
  Step 2 of the scenarios.
\end{note}

\subsubsection{\progEx{2}, \fingfr{1}, and \fc{1}}
\label{sec:progex-fingfr1}

\begin{note}
  So, question is whether \eiw{} conclusion is in progress entails a \fingfr{}.
  I think this is the case.

  In short, often, to describe event in progress, generality.

  Entails various other conclusions are \fc{1}.
\end{note}

\begin{note}
  Everything is worked through in a fairly general way, and relevant ideas, definitions, and so on are applied to \scen{1} to demonstrate the way \issueInclusion{} fails.

  And, the thesis closes with the reader themselves creating a \scen{0} for which \issueInclusion{} fails.
\end{note}

\newpage

\section*{Presentation notes}
\label{sec:presentation-notes}

\paragraph*{Broad introduction}

\begin{note}
  \begin{itemize}
  \item
    Goal for this talk is to cover the foundations of the thesis.
    If things go well, at the end of the presentation you will have a fairly good idea of:
    \begin{enumerate}
    \item
      Broadly, what the thesis is about.
    \item
      What I argued for.
    \item
      The `core' of the argument.

      The thesis is very detail oriented.
      However, most of the thesis amounts to refining a particular idea to obtain a specific result.
      The way the idea is refined is beyond the scope of a presentation like this, but I think the idea itself almost fits.
    \end{enumerate}

    I don't really like slides, so I've prepared a handout.
  \item
    Slightly high level:
    \begin{itemize}
    \item 
      Given an event in which an agent concludes, two broad questions.
      `Why' and `how'.
      And, may think answers to why are constrained by answers to how.

      A fairly well-known instance of this is \citeauthor{Davidson:1963aa}'s account of rationalisations.
    \item
      Briefly look at an instance.
      Introduce some abstractions, state `why' and `how', and the constraint.
      I argue constraint fails to hold.
    \item
      Core part of the argument is a way to obtain answers to why.
    \end{itemize}
  \end{itemize}
\end{note}

\paragraph{\eiw{3} an agent concludes}

\begin{note}
  Two scenarios.

  Same conclusion, slightly different result.
\end{note}

\paragraph{Abstractions}

\begin{note}
  \begin{itemize}
  \item
    \prop{3}
  \item
    \val{3}
  \item
    \pool{3}
  \item
    Roughly, characterising inputs and outputs of reasoning.

    If you like, think of propositional attitudes.
    \prop{3} correspond to propositions and \val{1} correspond to attitudes.
  \end{itemize}

  With \prop{1}, \val{1}, and \pool{1} in hand, two relations of interest:

  \begin{itemize}
  \item
    From relation
  \item
    \fofr{}, from the \agpe{}.
  \end{itemize}

  Relative to an \eiw{} agent concludes.

  From, and \fof{}.
  Coincide.
  However, \fofr{} are more general.
  With respect to \scen{1}, two from relations, only hold relative to the \scen{0}.
  Still, the \fofr{} relation may hold in both \scen{1}.
  I.e.\ agent concluded by factorisation, but had the option to use the quadratic formula.
  For additional examples, consider practice problems.
  If you've been learning the basics of factorisation, then \fofr{} should (hopefully) hold for practice problems.
\end{note}

\begin{note}
  These are the abstractions.
  And, with these in hand, formulate two questions.
\end{note}

\paragraph{Two questions and a constraint \hfill (\qWhy{}, \qHow{}, and \issueInclusion{})}

\begin{note}
  Why, how, and constraint.
\end{note}

\begin{note}
  Notes on \qWhy{}:
  \begin{itemize}
  \item
    Included in (partial) explanations.

    Idea is, broad answers to why.
    These may be partial.
    And, whether this explanation `entails' a \fofr{0}.
  \end{itemize}
\end{note}


\begin{note}
  Phenomena.

  Event where an agent concludes.
  Two instances.

  As an aside, argument is independent of \prop{}, \val{}.
  Deal with \valI{True}, but other things are good.

  From relation and \fofr{}.
\end{note}

\end{document}

%%% Local Variables:
%%% mode: latex
%%% TeX-master: t
%%% TeX-engine: luatex
%%% TeX-master: "handoutNotes"
%%% End:

