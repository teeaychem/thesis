\chapter{Wrangling}
\label{cha:var:wrang}

\begin{note}
  \autoref{cha:introduction} introduced two questions, \qWhy{} and \qHow{}, and motivated a constraint between answers to \qWhy{} and \qHow{}.

  \autoref{cha:var:sec:vars} introduced variants of \qWhy{} and \qHow{}, and a variant constraint.

  \autoref{cha:clar:sec:literature}, in addition to intuition, constraint seems to often be a theoretical assumption.

  Purpose of variants is to motivate counterexamples to constraint.
  Specifically in terms of answers to \qWhyVnP{} which are not answers to \qHowV{}.
  In other words, \ros{} such that \ros{} explains, in part, why agent concludes but is such that the agent does not have a \wit{} for the \ros{}.

  In this section we outline in rough form how we will (attempt) to provide counterexamples.

  In short, need:
  An agent, event in which agent concludes \(\pv{\phi}{v}\) from \(\Phi\), and \ros{} between \(\pv{\psi}{v'}\) and \(\Psi\) such that:

  \begin{itemize}
  \item
    The agent does not have a \wit{} for the \ros{} between \(\pv{\psi}{v'}\) and \(\Psi\).
  \item
    The \ros{} between \(\pv{\psi}{v'}\) and \(\Psi\), in part, answers \qWhyVnP{}.
  \end{itemize}

  Our goal is motivate a general method for generating examples in which some \ros{} for which an agent does not have a \wit{} such that the \ros{} answers \qWhyVnP{}.
\end{note}

\subsection{\ros{3} without a \wit{}}

\begin{note}
  Immediate that an agent may not have a \wit{} for some \ros{}.

  Novel conclusions, as understood in this document, are common.
  Pair a proposition with some value.

  For example, enumerate all the tautologies of propositional logic.
  As \citeauthor{Harman:1973ww} notes, clutter, and there may be little point in deriving the tautologies.
  However, regardless of worth, it is not the case that have a \wit{} for most.

  Likewise, conclusions with respect to actions.
  For example, which particular style of coffee would like as the queue shortens and the time to place an order approaches.
\end{note}

\begin{note}
  More difficult, is \ros{}.

  As sketched in \autoref{cha:var:ros}, idea of a \fc{}.
\end{note}

\begin{note}
  \fc{} is such that action such that after performing action, event in which concludes is in progress.
  In other worlds, the agent would be concluding by performing the action.

  Delicacy.
  \fc{1} and \ros{} from the \agpe{}.

  Two issues:

  \begin{enumerate}
  \item
    If independent of \agpe{}, then it may be the case that \fc{} without any prior recognition from agent.
  \item
    If dependent on \agpe{}, then it may be the case that not a \fc{}.
  \end{enumerate}

  Two issues are important.
  Without \agpe{}, then unclear that get \ros{} of interest for answer to \qWhyVnP{}.
  Given \agpe{}, unclear that we get a \ros{}.
\end{note}

\begin{note}
  Second issue is particularly pressing.
  For, explanations.
  \emph{Factive}.

  Get \ros{} by \fc{}.
  However, may not be the case that \fc{}.
\end{note}

\begin{note}
  Our strategy is to avoid both problems by focusing on cases in which the agent \emph{knows} that \(\pv{\psi}{v'}\) from \(\Psi\) is a \fc{}.

  Hence, suitable link to the \agpe{}, as the agent is aware that \fc{}.
  And, avoid failure of \fc{} by factivity of knowledge.

  In this respect the understanding of \fc{1} in terms of action such that concluding is important.
  With the exception of more-or-less instantaneous actions, future may develop in surprising ways.

  For example, plausible that an agent knows when they strike the cue ball in a certain way, a particular red ball will land in a pocket.
  However, not plausible that the agent knows where the cue ball will come to rest after the red ball lands in the pocket.
  Hence, agent does not know their following more, and so on.

  In parallel, an agent may have no guarantee that they will not be interrupted, etc.
  Hence, in most cases it seems implausible that an agent knows they will concluded.
  Yet, to be concluding does not require completion.

  With respect to \fc{}, whether event in which the agent concludes would be in progress.

  As seen in \autoref{cha:var:ros:II:fcs} \dots

  \fc{1} obtain instances of a \ros{1} holding from an \agpe{} without the agent having a \wit{0} for the \ros{0}.
  Hence, candidate \ros{} that may be answers to \qWhyVnP{} such that the agent does not have a \wit{0} for the \ros{0}.

  However, in order for \ros{} to be an answer, in part, to \qWhyVnP{}, dependence.
\end{note}

\begin{note}
  This does not provide a complete solution to problem of factivity.
  For, what distinguishes one case from the other?

  However, this is nothing unique to cases under consideration, so long as relevant instances of \fc{} are plausibly knowledge.

  Though, this still differs from attitudes.

\end{note}

\subsection{Dependence}

\begin{note}
  Now, need it to be the case that if \ros{} failed to hold, then would not conclude.

  Suitable link between conclusion of \(\pv{\phi}{v}\) form \(\Phi\) and \ros{} between \(\pv{\psi}{v'}\) and \(\Psi\).

  \fc{3} serve an equally important role.

  For, if \fc{} then \ros{}.
  Conversely, if no \ros{} then not \fc{}.
  Hence, if fails to be \ros{}, then fails to be \fc{}.
  In turn, agent concludes only due to \fc{}, then suitable link.
\end{note}

\begin{note}
  Difficulty.
  How is it the case that the agent concludes only due to \fc{}?

  Method is to consider dependence of \qWhyVnP{}, from the \agpe{}:

  \begin{restatable}[\qWhyV{}]{question}{questionWhyV}
    \label{q:why:v}
    Given an agent \vAgent{}, proposition-value pair \(\pv{\phi}{v}\), \poP{} \(\Phi\), and event \(e\) in which \vAgent{} concludes \(\pv{\phi}{v}\) from \(\Phi\):

    \begin{quote}
      Which proposition-value-premises pairings \(\pvp{\psi}{v'}{\Psi}\) are such that, when \vAgent{} pairs \(\phi\) with \(v\):

      \begin{enumerate}[label=]
      \item
        \begin{enumerate}[label=\alph*., ref=(\alph*), series=qWhyVDef]
        \item
          A \ros{0} between \(\pv{\psi}{v'}\) and \(\Psi\) holds, from \agpe{\vAgent{}'}.
        \end{enumerate}
      \end{enumerate}

      And, from \agpe{\vAgent{}'}:

      \begin{enumerate}
      \item[\emph{If}:]
        \begin{enumerate}[label=\alph*., ref=(\alph*), resume*=qWhyVDef]
        \item
          The \ros{0} between \(\pv{\psi}{v'}\) and \(\Psi\) when pairing \(\phi\) with \(v\) failed to hold, from \agpe{\vAgent{}'}.%
          \footnote{
            It seems more natural to omit `from \agpe{\vAgent{}'}':
            \begin{itemize}
            \item
              The \ros{0} between \(\pv{\psi}{v'}\) and \(\Psi\) when pairing \(\phi\) with \(v\) failed to hold.
            \end{itemize}
            Indeed, evaluating conditional from \agpe{}.
            However, distinction between whether \ros{} and whether \ros{} from \agpe{}.
            Latter, and keeping the qualifier helps clarify.
          }
        \end{enumerate}
      \item[\emph{Then}:]
        \begin{enumerate}[label=\alph*., ref=(\alph*), resume*=qWhyVDef]
        \item
          \(e\) would not have been an event in which \vAgent{} concluded \(\pv{\phi}{v}\) from \(\Phi\).
        \end{enumerate}
      \end{enumerate}
    \end{quote}
    \vspace{-\baselineskip}
  \end{restatable}
\end{note}

\begin{note}
  The idea is:

  Counterexamples.

  One option is to specify (apparent) counterexamples so that the truth of the \emph{if-then} conditional follows.
  Then, at issue is whether analysis is correct.
  It may be that the truth of the conditional does not follow.

  Other option, specify (apparent) counterexamples so that the \emph{if-then} conditional holds from the \agpe{}.
  Then, at issue is whether the \agpe{} is correct.

  Trade a issue about analysis of counterexamples for a issue about what the counterexample achieves.

  Preference for the second option is ease of specifying examples.
  Build up an understanding of how and why such examples arise, and then try to figure out whether they result in anything substantial rather than attempting defend position that analysis of an example counters.

  In short:

  On first, whether analysis is correct.

  On second, whether agent is correct.

  Preference for whether agent is correct.
  Though, equivalent.
\end{note}

\begin{note}
  In order for \qWhyV{} to be of interest, must be cases where the \agpe{} is correct.

  \begin{proposition}
    \label{prop:why-n-p-link}
    Instances where \(\pvp{\psi}{v'}{\Psi}\) answers \qWhyVnP{} in virtue of answering \qWhyV{}
  \end{proposition}

  Find instances which witness the truth of \autoref{prop:why-n-p-link}.

  Stress, \autoref{prop:why-n-p-link} does not amount to an entailment.
  They may be cases where \agpe{} returns an answer to \qWhyV{} which is not an answer to \qWhyVnP{}.
\end{note}

\subsubsection{\fc{3}}
\label{sec:fc3}

\begin{note}
  \fc{3} will do the work:

  \begin{itemize}
  \item
    \ros{1} between \(\pv{\psi}{v'}\) and \(\Psi\) when pairing \(\phi\) with \(v\) failed to hold, from \agpe{\vAgent{}'}.
  \item
    \(\pv{\psi}{v'}\) failed to be a \fc{} from \(\Psi\).
  \item
    \(e\) would not have been an event in which \vAgent{} concluded \(\pv{\phi}{v}\) from \(\Phi\).
  \end{itemize}

  If \fc{} failed to hold, then no conclusion.
  Agent, implicitly, recognises link between \ros{1} and \fc{1}.
  However, given that \ros{1} are something of an abstraction, interest is really in whether \fc{}.

  So, dependence is captured from the \agpe{} and task is to construct \scen{1} such that if no \fc{0} then no conclusion.

  We will not sketch \scen{1} here.
\end{note}

\subsubsection{Concerns}
\label{sec:pitfalls}

\begin{note}
  Subdivide into two concerns.

  \begin{enumerate}
  \item
    Is \agpe{} informative?
  \item
    Granting true from \agpe{}, is informative, may it still be the case that \ros{} fails to answer \qWhyVnP{}.
  \end{enumerate}
\end{note}

\paragraph{Perspective alone}

\begin{note}
  It may be the case that that \emph{if-then} conditional holds from the \agpe{} but does not hold independently of the \agpe{}.
  Hence, the sense of dependence captured by \qWhyV{} is not equivalent with the intuitive sense of dependence captured by considering whether or not the \emph{if-then} conditional holds independently of the \agpe{}.

  The observation that the \emph{if-then} conditional may hold from the \agpe{} while failing to hold independently of the \agpe{} is clearest when considering conditionals more general.

  For example, suppose an agent has taken a gamble on a coin landing heads.
  The coin lands heads, and the agent receives a prize.
  From the \agpe{}, if the coin failed to lands heads, then the agent would not have received the prize.
  However, the agent was set to receive the prize for participating in the gamble, regardless of whether the coin landed heads.%
  \footnote{
    The present point is similar to issues raised by \citeauthor{Harman:1973ww} (\citeyear{Harman:1973ww}) regarding the proposed equivalence between reasons for which an agent believes something with reasons the agent would offer if asked to justify their belief.
  As \citeauthor{Harman:1973ww} notes, an agent may offer reasons because they think they will convince their audience, not because the agent is compelled by the reasons, etc.
  (\citeyear[Ch.2]{Harman:1973ww})

  To the extent that \citeauthor{Harman:1973ww}'s point is that what holds from an \agpe{} need not actually be the case, the point in the same.
  However, to the extent that \citeauthor{Harman:1973ww} relies on an under-specification of what holds from an \agpe{} --- i.e.\ the distinction between whether \(\phi\) has value \(v\) from the \agpe{} or whether the agent evaluates as true the proposition that their audience is responsive to \(\phi\) having value \(v\), the point is distinct.
  }

  So, switching back to \qWhyV{}, it may be the case that, though from the \agpe{} they would not have concluded \(\pv{\phi}{v}\) from \(\Phi\) if \support{} failed to hold between \(\pv{\psi}{v'}\) and \(\Psi\), the agent would have concluded \(\pv{\phi}{v}\) from \(\Phi\) regardless.
\end{note}

\begin{note}
  Consider \citeauthor{Davidson:1963aa}.
  Account of reason is in terms of attitudes.
  \emph{Not} what is the case from the \agpe{}.
  Attitudes, rather than the contents of attitudes.

  \begin{quote}
    Davidson[ asserts] a demand for a more ordinary form of explanation:
    an explanation which shows, not merely what, from another's point of view, \emph{could} count in favour of acting, but why that person did, in fact, act.%
    \mbox{}\hfill\mbox{(\cite[417]{Hieronymi:2011aa})}
  \end{quote}

  In general, difficult to make the switch.
  \textcite{Dancy:2000aa} argues for contents.
  However, problem of non-factivity.

  \citeauthor{Dancy:2000aa}'s position is difficult.
  On the one hand, reason from the \agpe{}.
  If this is the case, then compatible.
  In this respect, when shift to the \agpe{}, it is not the \agents{} belief, but what the agent believes.
  In short, reason from \agpe{} need not be the same as reason independent of \agpe{}.

  However, understanding from literature is that \citeauthor{Dancy:2000aa} holds that reason from \agpe{} \emph{is} reason.
  This, I find strange.
  And, this is not what we are interested in.
  Though, it is a delicate balance.
  Important thing to keep in mind is that we are two levels deep at this point.
  If dependence holds, along with proposition, then still in terms of \ros{} from \agpe{}.
  So, in dependence, then not pushing as far as \citeauthor{Dancy:2000aa}.
  No general entailment.
\end{note}

\begin{note}
  So, the only thing to do is ensure the \agpe{} is correct.

  However, motivation in similar style to \fc{1}.
  \fc{3} focus on whether agent would conclude.
  Pair is knowing that the agent would not conclude.
  So, in parallel fashion, knowledge.
\end{note}

\begin{note}
  So, avoid failure of factivity for a second time.
  However, problem of deviance.
  General disconnect.
  Agent is right, in general, but way in which they would not conclude is tangential.
  I have no solution to this.
  Grant that deviance does not occur.
\end{note}

\subsection{\qWhyVnP{} and \qWhyV{}}
\label{cha:var:expand:qWhy:variant}

\subsubsection{The \ros{} between \(\pv{\phi}{v}\) and \(\Phi\)}

\begin{note}
  Before turning to dependence, let us briefly observe that the \ros{} between \(\pv{\phi}{v}\) and \(\Phi\) should always, intuitively, be, in part, an answer to \qWhyV{}.

  For, when agent pairs \(\phi\) with \(v\), then by \supportI{}, it is the case that \ros{} between \(\pv{\phi}{v}\) and \(\Phi\) holds from the \agpe{}.
  Hence, \ref{q:why:v:a} must be true.

  Likewise, immediately the case that if \support{} failed to hold, then the agent would not have concluded \(\pv{\phi}{v}\) from \(\Phi\).
  For, by the same reasoning, when the agent pairs \(\phi\) with \(v\), then by \supportI{}, it is the case that \ros{} between \(\pv{\phi}{v}\) and \(\Phi\) holds from the \agpe{}.
  Therefore, if \support{} fails to hold between \(\pv{\phi}{v}\) and \(\Phi\), then \(e\) is not an event in which the agent concludes \(\pv{\phi}{v}\) from \(\Phi\).

  Now, the conditional when set aside from the \agpe{} may be true, it need not be the case that the conditional is true from the \agpe{}.
  However, we are working at some level of abstraction, and hence we assume the agent recognises the truth of the \emph{if-then} conditional.
  Indeed, we have merely expressed in artificial terms a truism:
  The agent would not have concluded \(\pv{\phi}{v}\) from \(\Phi\) if the agent had failed to conclude \(\pv{\phi}{v}\) from \(\Phi\).
\end{note}

\begin{note}
  With the core case of \support{} between \(\pv{\phi}{v}\) and \(\Phi\) being an answer to \qWhyV{} in hand, we turn to an in depth discussion of the kind of dependence captured by the \emph{if-then} conditional of \qWhyV{}.

  Our task is to balance a form of dependence that \emph{may} lead to counterexamples to \issueInclusion{} with an account of dependence that is compatible with the intuition that motivates \issueInclusion{}.
  In other words, the dependence should be such that answers to \qWhyV{} are constrained by the condition that the agent has a \wit{} for the relevant \ros{}.
\end{note}

\subsection{Summary}
\label{cha:var:expand:issue:summary}


%%% Local Variables:
%%% mode: latex
%%% TeX-master: "master"
%%% End:
