\chapter{\ideaCS{} and ability}
\label{sec:second-conditional}
\label{sec:LCS-applied}

\begin{note}
  So, two main tasks.
  \begin{enumerate}
  \item Why tension with \adA{}.
  \item Why tension does not arise for \adB{}.
  \end{enumerate}
\end{note}

\begin{note}[Finding tension, still]
  \color{red}
  We have outlined a type of scenario built primarily on an agent receiving information that the agent has some specific ability so long as the agent has some general ability.
  The agent has support for having the general ability, but there are two ways in which the agent's support for having the general ability may be used to establish support for {\color{red} the result of having the specific ability} --- \AR{} and \WR{}.

  The previous section argued that~\ESU{} constrains how an agent may use the received information.
  If an agent is required to traces support from premises to conclusion through reasoning, then an agent may not appeal to the support for the premises and steps of reasoning that the agent would use to witness the specific ability.

  The (initial) plausibility of~\ESU{}, then, suggests that the agent may only establish support for having the {\color{red} result of the specific ability} from the support they have for the general ability by \AR{}:
  The support the agent has for the general ability is support that it is true that the agent has the general ability.
  In turn, given the information received it is true that the agent has the specific ability, and it is only possible for the agent to have the specific ability if the result of witnessing the specific ability is true.

  The argument of this section is that the sketch of \AR{} given conflicts with a different, but equally plausible, premise.
  The premise concerns the way in which the agent obtains support for having the specific ability from the support for the general ability.
  We state conditional, the proceed to the premise.
  The initial statement of the premise is abstract and after providing a handful of clarifications we then link the premise to the type of scenario of interest.
\end{note}

\section{\LCS{} and ability}
\label{sec:ni-ability}

\begin{note}[Notes]
  The key here is \gsi{} and \adA{}.
  This is going to lead to a problem, given \LCS{}.

\end{note}

\begin{note}
  So, the general plan is to observe that if \ESU{} holds, then need to appeal to having the ability.
  And, this involves appeal to moving to having the ability from prior reasoning.
  Without the second step here, the argument breaks down.
  However, it seems unavoidable.
  There's no other reasoning possible for the agent.
  Ability seems required.
\end{note}

\subsection{\LCS{}}
\label{sec:ability-and-lcs}

\begin{note}
  \color{red}
  \begin{itemize}
  \item So, get \requ{}.
  \item This much is fine.
  \item Question is whether it is possible for the agent to do anything about the \requ{}.
  \item Arguably no.
  \item One big idea is that the agent has claimed support for general ability on some `core' such that this provides strong indication that general ability extends to all cases.
  \item This much is fine, then problem, however, is that we've still got specific information about what is outside of the core.
  \item So, the probability of any possible defeater is super low.
  \item And, this is enough to hold onto claimed support regardless.
  \item Because, I considered the possibility of those unknown defeaters, and still gathered enough to claim support regardless.
  \item So, this seems to allow claiming support for specific from general.
  \item But at the same time this seems bad.
  \item It has the same feel as the problems with \requ{1}.
  \item Because, all the stuff gathered was without recognition of this possibility.
  \end{itemize}

  \begin{itemize}
  \item I mean, problem is before, got the probability low as unrecognised.
  \item Question is whether this remains the case now recognised.
  \item Well, nothing really follows from probability being low.
  \item In a sense, this should already be the case.
  \item The issue isn't that these possible defeaters a \emph{likely}.
  \item The issue is that the agent should think that support holds regardless of whether they hold.
  \item ``Category mistake''
  \end{itemize}

  Okay, so this kind of works against low probability.
  Hence, argument here is that there's no way to get rid of this \requ{} if agent only relies on claimed support for general ability.
  Of course think it's unlikely, but the worry is not that the defeater is there, rather than it's not clear how the evidence goes against the defeater.

  The redux, then, is that this idea of a `core' doesn't really rely on the probability idea.
  But then this just goes against the initial assumption.
  Of course, this is kind of what \citeauthor{Pryor:2000tl} does.
  However, this seems to conflict with the idea of claimed support.
  If we've got some kind of dogmatist position, then it doesn't seem that the possibility of being \mom{} is such an issue.
  Indeed, the problem here is how to make something like this consistent with that assumption.
\end{note}

\subsection{\adA{}}
\label{sec:lcs-and-adA}

\begin{note}
  Observed \LCS{}.
  The important thing here is that \FCS{}.

  So, goal here is to argue that the additional constraints of \FCS{} also hold with respect to ability.
\end{note}

\subsubsection{\gsi{}, \adA{}, and \FCS{}}
\label{sec:ni-ability:adA}

\begin{note}
  The idea here is simple.
  \begin{itemize}
  \item The thing with \adA{} is that the agent appeals to ability `as a whole', so to speak.
  \item This gives us the relevant \(\phi\) instance for \nI{}.
  \item And, \aben{the} is such that the agent needs to appeal to ability, rather than mere claimed support.
  \item Then, the key focus is {\color{red} inclusion}.
  \end{itemize}
\end{note}

\begin{note}
  Now, the basic observation is that with \adA{} one moves from general to specific, and from ability to proposition.

  Here, only really interested in \aben{the}.
  However, as we've observed, goes from either general or specific.

  I mean, the basic observation is that the agent doesn't reason about general or specific ability.
  So, reasoning follows from it being the case that agent has attribute, or that there is a witnessing event.

  Ohhhh, the point is that the agent is relying on these conditionals.
  First, to move from general to specific.
  Second, to move from ability to proposition.

  With respect to these conditionals, it's \adA{}, so there's no way to move between these things without using the value of one thing to constrain the value of the other.

  So, instance of \adA{}, generally.
  And, because of the construction of the scenarios, the case of \adA{} we're interested involves appeal to the value of the proposition.
\end{note}

\subsubsection{\nI{} and \adB{} (excluding \ARB{})}
\label{sec:ni-ability:adB}

\begin{note}
  Key observation is that \adB{} doesn't go by value.

  However, there is a problem.

  For, it may seems as though the agent \emph{does} go by value because they require the premises, etc.

  This is clearest with the idea that:
  \begin{itemize}
  \item If \(\phi\) isn't the case, then some premise or step isn't part of ability.
  \end{itemize}
  Question about whether this gets a violation of \ideaCS{}.

  But, point is that agent at present is okay with claiming support that the reference resolves.

  So, this really isn't that problematic.

  Obviously it could break down.

  The point is that the agent at present outlines claim to support even if \mom{}.
\end{note}

\subsubsection{\requ{3}}
\label{sec:requ1}

\begin{note}
  Objection here is that we've identified failure due to \requ{1}, roughly.
  And, ability to claim support, and this is going to involve some \requ{1}.
  However, no use so no reasoning about.

  Yes, this is somewhat difficult.

  Unsatisfactory response would be to observe that \requ{} is only defined with respect to reasoning performed.
  Unsatisfactory because only necessary conditions, and plausible that there are additional necessary conditions.

  Possible line of response is appeal to ability.
  However, this will lead to an instance of \nI{} all over again.

  Instead, witnessing ability is itself sufficient for claiming support.
  And, as would amount to claiming support, this will involve reasoning about \requ{}.
  Important, \EAS{} does not necessarily hold for anything weaker.
  \nI{} has been stated for reasoning quite broadly, so problem if the agent appeals to claiming support, or any other reasoning with interdependence.
  However, witnessing is not appealing in this way.
\end{note}

\subsubsection{Appeal to premises and steps requires appeal to ability}

\begin{note}
  It's true that the combination implies the ability, and so the combination seems to lead to the same problem.
  We get \(\psi\) as an \requ{} of combining all of the premises and steps.

  However, what we're relying on is appeal to the individual components.
  The thing here is that it seems fine for the agent to witness.
  This doesn't block claiming support.

  Hence, if this is the case then it can't be that the problem is simply what follows from the combination.
  Rather, it must be something about not witnessing.
  However, this returns us to \ESU{}.
  This is the very intuition that we're arguing against.
  Hence, the question is whether this really is something that is the case.
\end{note}

%%% Local Variables:
%%% mode: latex
%%% TeX-master: "master"
%%% End: