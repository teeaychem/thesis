\chapter{Overview}
\label{cha:overview}

\section{Outline}
\label{sec:outline}

\begin{note}
  In this chapter we provide a high level overview of the main arguments made in this thesis.
  A significant part of the high level overview of arguments is an overview of the premises and assumptions that those arguments rest on.

  By given high level overview, clarify on how premises, assumptions, and conclusions relate.
  In the main body of the thesis, afford to elaborate.
  And, allow choice of where to seek elaboration.
\end{note}

\begin{note}
  Following introduction, interest is with ability.
  In particular, observation that \gsi{} information, and confidence in general ability seems to allow agent to claim support for result.

  Question on what basis the agent claims support.

  Slightly more general statement.
  \begin{quote}
    When and why an agent may claim support for the result of reasoning that the agent has not witnessed.
  \end{quote}
  Ability claims of interest provide some answer to `when'.
  For, ability provides information about what the result is, and also that the agent has the opportunity to perform the reasoning.\nolinebreak
  \footnote{No claim to necessity}

  Suggested in introduction, answer to `why' is more complex.
  % \begin{quote}
  %   Our interest is in when an agent may claim support for some conclusion of some instance of reasoning on the basis of the support the agent may claim for the premises of the instances of reasoning.
  % \end{quote}
\end{note}

\section{Main things}
\label{sec:main-things}

\begin{note}
  \begin{restatable}[\ESU{0} --- \ESU{}]{target}{targetESU}
    \label{denied-claim}
    An instance of reasoning is an instance of claiming support \emph{only if}, for any proposition-value pair \(\psi\) having value \(v\) appealed to, the agent has performed reasoning which indicates that \(\psi\) has value \(v\).
  \end{restatable}


  Some instance of reasoning from premises which indicates conclusion.
  Reasoning is compatible with the agent claiming support for the conclusion only if some agent has witnessed the reasoning.

  Necessary condition on claiming support.
  Key point is that, whether or not there are genuine cases of claiming support, \autoref{denied-claim} provides a clear condition for rejecting the possibility that some instance of reasoning is an instance of claiming support.
\end{note}

\begin{note}
  \begin{restatable}[\EAS{0} --- \EAS{}]{goal}{goalEAS}
    \label{prop:EAS}
    {
      \color{red}
      Ability, reasoning from premises to conclusion, compatible with claiming support.
    }

    If an agent is confident have the ability to reason to some conclusion, then it may be possible for the agent claim support for the conclusion by appealing to some premises that do not form part of the agent's reasoning.
  \end{restatable}
\end{note}

\begin{note}
  The distinction here is whether some instance of reasoning needs to be witnessed in order for the reasoning to be compatible with the agent claiming support.

  Ability is interesting here.
  Cases of interest, ability to reason.
  However, this alone does not imply that the reasoning is important for claiming support.
  Some property.
  The ability, or alternatively the potential for the agent to reason.
  \autoref{prop:EAS} is different.
  Not only the potential, but that reasoning.
\end{note}

\begin{note}
  Claiming support is in the background here.
  Purpose of claiming support is that it places a constraint on reasoning.

  In short\dots

  Whether this is satisfied in general, however, is not of interest.
  Instead, whether certain steps of reasoning violate.
  Hence, the possibility of claiming support.
  Step from ability to \(\psi\) to \(\psi\) need not violate.
\end{note}

\begin{note}
  Important that \autoref{denied-claim} is weak in two ways.
  First, \indicateN{0}.
  Does not need to be the case that agent recognises.

  Second, some past reasoning, by some agent.
  Plausible to place stronger requirements.
  Still, focus is on witnessing, rather than any qualified form of witnessing.
  Hence, stronger result granting these options.
\end{note}

\begin{note}
  \EAS{} is a qualified statement.
  It does not state that an agent having claimed support that they have the ability to claim support for some conclusion is \emph{always permissible} to claim support for the conclusion by appealing to some premises that do not form part of the agent's reasoning.
  Instead, it states that \emph{may be permissible} for the agent claim support in a certain way.

  Two brief clarifications:
\end{note}

\begin{note}
  First, we use the term `permissible' rather than `sufficient' as we are describing some volitional activity.
  We will not argue that an agent is required to perform some reasoning.
  If \EAS{} were rephrased to assume that the agent were already reasoning to the conclusion, then we could say that appealing to some premises that do not form part of the agent's reasoning would be sufficient.
  However, I find expressions of permissibility about actions cleaner than counterfactual expressions of what would be sufficient to complete some activity in progress.
\end{note}

\begin{note}
  Second, though `sometimes' is the more natural contrast to `always', `sometimes' implies that cases exist.
  We will argue for \EAS{} by considering a certain type of case that involves reasoning with ability.
  And, in certain token instances of those type of cases it will be permissible for the agent claim support in a certain way.
  However, the further step of arguing for instantiations of those token instances is of little interest.
  Claiming support identifies some general phenomenon which is not restricted to extant instantiations.
  And \EAS{} implies that a plausible constraint (\ESU{}) on that phenomenon is not the case.
  At issue is our understanding of the phenomenon, not whether some instance of the phenomenon has been realised.
  Even so, I doubt that it will be difficult to find instantiations of the token instances of the type of cases.
\end{note}

\paragraph{Matrix}

\begin{note}
  \begin{figure}[H]
    \centering
    \saMtxInterpreted{}
    \caption{Distinction matrix with \aben{the}}
    \label{fig:saMtxInterpreted:outline}
  \end{figure}
\end{note}

\begin{note}[Matrix, ruled out]
  \begin{figure}[H]
    \centering
    \saMtxRuledOut{}
    \caption{Distinction matrix}
    \label{fig:saMtxRuledOut:outline}
  \end{figure}
\end{note}

\begin{note}
  Recap.

  Claiming support.
  Constraint.

  Ability.
  In order to be compatible, satisfy constraint.
  Either of three options.
  Basic, ignore this.
  Property. Incompatible with constraint.
  Witness. Compatible.

  Here, display the matrix.
  I think this is the easiest way to visualise what is going on.
\end{note}

\paragraph{Outline}

\begin{note}
  How things are divided.
\end{note}

%%% Local Variables:
%%% mode: latex
%%% TeX-master: "master"
%%% End: