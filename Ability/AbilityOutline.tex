\chapter{Overview}
\label{cha:overview}

\section{Outline}
\label{sec:outline}

\begin{note}
  In this chapter we provide a high level overview of the main arguments made in this thesis.
  A significant part of the high level overview of arguments is an overview of the premises and assumptions that those arguments rest on.

  By given high level overview, clarify on how premises, assumptions, and conclusions relate.
  In the main body of the thesis, afford to elaborate.
  And, allow choice of where to seek elaboration.
\end{note}

\begin{itemize}
\item Start with claiming support, used throughout, so important.
\item Introduce and motivate plausible constraint on support, to be argued against/exception for.
\item Outline exception.
\item High level overview of argument for exception.
\item Major and minor.
\item Largely fairly high level sketch of major argument.
\item Type of ability information.
\item Understanding \emph{de re} ability reading.
\item \AR{} and \WR{}.
\item For the moment, brief, much more detail in relevant chapter.
\item Relation between \AR{} and \ESU{}.
\item Constraint on reading of ability.
\item Requires argument for \WR{}.
\item Introduce \nI{}.
\item Sketch argument for \nI{}.
\item Link \nI{} and \AR{}.
\item Completes overview of major argument.
\end{itemize}

\section{Ability and access to claimed support}
\label{sec:abil-access-supp}

\begin{note}
  Following introduction, interest is with ability.
  In particular, observation that \gsi{} information, and confidence in general ability seems to allow agent to claim support for result.

  Question on what basis the agent claims support.

  Slightly more general statement.
  \begin{quote}
    When and why an agent may claim support for the result of reasoning that the agent has not witnessed.
  \end{quote}
  Ability claims of interest provide some answer to `when'.
  For, ability provides information about what the result is, and also that the agent has the opportunity to perform the reasoning.\nolinebreak
  \footnote{No claim to necessity}

  Suggested in introduction, answer to `why' is more complex.
  % \begin{quote}
  %   Our interest is in when an agent may claim support for some conclusion of some instance of reasoning on the basis of the support the agent may claim for the premises of the instances of reasoning.
  % \end{quote}
\end{note}

\begin{note}[Introducing support]
  Initial clarification is with respect to claiming support.
  Emphasis on `\emph{claim}'.

  The thesis is not about when and why an agent \emph{has} support for the result of reasoning that the agent has not witnessed.

  There are three primary reasons why we focus on claimed support.

  First, neutral for main thread of argument on what support amounts to.
  Interest is with structure of claim, and background assumption that if success in claiming then structure of support follows structure of claim.

  Second, whether or not an agent has support often seems secondary.
  To illustrate:
  It may be that any claimed support for a proposition is support for that proposition, but perhaps not.
  Suppose `flan' is written on the side of a container.
  I may claim support that the container contains flan.
  And, it may be that the writing on the side of the container is support for the box containing flan.
  However, the straps ensuring the container remains closed is unfortunately placed, and if moved would reveal the side of the container reads `flannels'.
  The unfortunate placing of the straps does not seem to prevent \emph{claiming} support, but I'm not sure whether it is right to say that the writing on the side of the box (straps in place) \emph{does} provide support that the box containing flan.
  So, in what follows I will speak in terms of claiming support, and leave open whether what is claimed reflects on whether an agent has support.\nolinebreak
  \footnote{
    In particular, claiming allows focus on internal constrains, while remaining silent on whether having support is (in part) determined by external factors.
  }
  \(^{,}\)\nolinebreak
  \footnote{
    Distinction between propositional and doxastic support.
    Propositional, support agent has whether or not made a claim.
    Doxastic is successful claim and propositional support.
    So, both require that the agent has support.
    Claimed support is the agentive component of doxastic support.
    Not interested in whether the agent also has propositional support, though more or less assume.
  }
  \(^{,}\)\nolinebreak
  \footnote{
    {
      \color{red}
      English is somewhat difficult.
      It is somewhat unfortunate that `an agent has claimed support for \(\phi\)' may be read `there is support which the agent has claimed for \(\phi\)'.
      Still, this seems to follow more easily from `support claimed'.
      So, `claimed support' emphasises the claim, while `support claimed' emphasises support.
    }
  }

  Third, and following from the second, focusing on claimed support allows us to make no assumption about the relationship between claimed support and support.
  To elaborate, consider enthymematic inferences.
  One may hold that an agent may claim support for some conclusion via enthymematic inference, but hold that the support the agent has is explicated in terms of the (corresponding) complete inference.\nolinebreak
  \footnote{
    Cf.\ \textcite{Moretti:2019wx}.
  }
  Alternatively, one may hold that the enthymematic argument is an adequate support relation (at least with respect to context in which the inference was made).
  Hence, one may question whether the structure of claimed support follows the structure of support.
  Without any assumption concerning this relation, we will not be committed to any position on why an agent has support for an ability given a position on how an agent claims support for an ability.
  In turn, an important implication of this final point is that why or when an agent \emph{has} an ability, only why (and when) an agent claims support for and from ability.

  {
    \color{red}
    Reference \citeauthor{Moretti:2019wx} here in terms of support for the enthymematic stuff, then reference the conditions that \citeauthor{Moretti:2019wx} outline as an example \ESU{}, and finally in some later chapter as a comparison to \EAS{}.
    \citeauthor{Moretti:2019wx}

    Note, \citeauthor{Moretti:2019wx} is somewhat different, as it's not about appeal.
    So, agent may form well-grounded belief, but additional steps required to show that the agent may appeal to the expanded inference.
    Key, then, is the relationship between the basing relation and what the agent appeals to.
    So, it seems there's not really any discussion of how the agent reasons in these cases.
    Rather, it seems that \emph{how} the agent reasons is left out, and the relevant enthymematic inferences are in some sense automatic in terms of appealing to the way in which propositional support expands without requiring the agent to perform any inferences.
  }


  To illustrate, consider dispositions.
  Simple conditional analysis.
  Perhaps I claim support because I imagine the event, and in the event the glass breaks.
\end{note}

\subsection{Claiming support and reasoning}
\label{sec:claimed-support}

\begin{note}[Understanding of claiming support]
  Understanding of claiming support.

  Begin with a sufficient condition.
  In short, most instances of reasoning.
  Claiming support is common.

  Then, two types of defeaters.
  Mistaken and misled.
  Use to form a necessary condition.
  If claimed support, then agent deems that claimed support is not defeated.
  `Deem' is a placeholder.
  Strong or weak.
  Single constraint is that when claiming support, potential defeaters that aren't ruled out.\nolinebreak
  \footnote{
    At least two ways of viewing this.
    First, claiming support is restricted.
    Second, \emph{claiming} support only applies when there are potential defeaters, and some other relation to support when possible defeaters get ruled out.

    These are different, but I don't think the difference matter for resource bound agents of interest.
    Lack of resources is that always potential defeater, even if every possible defeater may be ruled out.
    }
  Finally, property, that claimed support does not depend on whether proposition the agent has claimed support for is true, or whether the claimed support \emph{does} support (if these are separated).
  Property will be important.
\end{note}

\begin{note}[Sufficient condition]
  We start with a sufficient condition for claiming support:
  \begin{proposition}[\USE{-} --- \USE{}]\label{prem:bP}\label{prop:USE}
    If an agent may claim support for premises and steps of reasoning, accesses those premises and traces claim to support through those steps of reasoning, then agent may claim support for conclusion on basis of the claimed support for the steps and premises of reasoning.
    (Given that the agent deems that the claims to support for premises and steps used are undefeated when drawing conclusion.)
  \end{proposition}

  The purpose of taking~\USE{} as basic is to fix a basic understanding of when an agent may claim support.
  In short, an agent may claim support when reasoning goes well.
  And, reasoning goes well when there are premises and steps of reasoning available to the agent, and the agent draws on these to claim support for the conclusion.
  The parenthetical remark is a simple safeguard, the agent does not lose a claim to support for the premises or steps in the process of reasoning.\nolinebreak
  \footnote{
    May think that this is the wrong safeguard.
    Consider the liar paradox:
    `This sentence is false.'
    \USE{} prevents agent from claiming support that the sentence is true or false.
    However, may think that agent is in a position to claim support that the sentence is both true and false.
    Indeed, standard reasoning associated with the liar suggests that the sentence is both true and false.
    Still, it's not obvious from demonstrating that the sentence is both true and false that one may claim support for the sentence being true and the sentence being false.
    That is, one may confine the paradox to the truth value of the sentence, rather than (associated) surplus of support.
  }
  We consider defeaters below.
  First, an illustration.

  \begin{illustration}\label{ill:rectangle:basic}
    Suppose an agent measures that the rectangle in front of them has the dimensions of \(19\text{cm}\) by \(7\text{cm}\).
    The agent understands how to calculate the area of a rectangle, and by performing some reasoning comes to hold that the area of the rectangle is \(133\text{cm}^{2}\).
    The support the agent has for holding that the area of the rectangle is \(133\text{cm}^{2}\) is obtained (at least in part) on the measurement of rectangle, understanding how to calculate the area of rectangle, and some grasp of mathematics.
  \end{illustration}

  Whether some (or all) of the required arithmetic is to be included as a premise or a step of reasoning may be set aside.
  Similarly, set aside whether further arguments, or whether some premises and steps are taken as basic.
  For example, perhaps some the agent requires some further claim to support for using the ruler to measure the rectangle such as comparison to a standard, or perhaps the agent's claim to support terminates by noting that their use of the ruler is a reliable process.
\end{note}

\begin{note}[Value proposition]
  Reasoning and claims to support focus.
  Briefly introduce a pair of propositions to clarify claim to support and reasoning.

  \begin{proposition}[Claimed support is for the value of a proposition]\label{prop:csifvoap}
    When an agent claims support for some proposition, the agent claims that the proposition has some value.
    Where:
      \begin{itemize}
      \item A proposition is some information. And,
      \item A value is an assessment of that information.
      \end{itemize}
      \vspace{-\topsep}\vspace{-\topsep}
  \end{proposition}
  \autoref{prop:csifvoap} fixes terminology.
  To illustrate, when stating the conclusion of the reasoning sketched above we used the proposition that \emph{the area of the rectangle is \(133\text{cm}^{2}\)}.
  The proposition refers to the state of affairs in which the area of the rectangle is \(133\text{cm}^{2}\), and speaking a little more precisely, the agent claimed that the proposition has the value `true' --- though it may be the value turns out to be `false'.
  Or, perhaps if the agent was a little unsure about the accuracy of the ruler, that the proposition has the value `likely', `probable', or some quantitative credence.
  And, some other instance of reasoning may have concluded that the proposition has the value `desirable' --- e.g.\ if the agent was searching for a rectangle of some approximate size.\nolinebreak
  \footnote{
    Nothing in particular hangs on the distinction between different values.
    If you prefer, you may expand the proposition (state of affairs) to include additional factors, and consider only the values `true' and `false'.
    For example, the proposition that \emph{I desire the bath to be warm} is false, as opposed to the proposition that \emph{the bath is warm} is valued undesirable by me.
  }

  Core idea is that claim of support is that things are a certain way.
  Proposition, what the thing is.
  Value, the way it is.
  In most cases the value will be clear (i.e. that the proposition is true, though sometimes that the proposition is desirable), and so we will talk of claiming support for the proposition.
  A handful of additional examples will be provided when illustrating the next proposition.
\end{note}

\begin{note}[Reasoning proposition]
  \begin{proposition}[Reasoning as claiming support]\label{prop:RisTV}
    (For our purposes) an instance of reasoning culminates with the agent claims support for some proposition \(\phi\) having some value \(v\).
    % Reasoning, tracing value through propositions/establishing that proposition has value.
  \end{proposition}

  \autoref{prop:RisTV} fixes an understanding of instances reasoning for present purposes.
  Instances of reasoning may culminate in other ways, and if so we are only interested in a specific type of instance.
  Still, this type of instance is quite general, and the following examples highlight:

  \begin{itemize}
  \item Testimony, so claim support that \emph{p} is true.
  \item Unreliable, so claim support that \emph{p} improbable.
  \item Dialect, so claim support that \emph{p} ought to be the case.
  \item Producer, so claim support that album is desirable.
  \end{itemize}
\end{note}

\begin{note}[Claiming support]
  Expand on this below.
  Briefly mention that this falls short of \emph{establishing} that \(\phi\) has value \(v\).
  \emph{Claimed} support means there's always some possible defeater.
\end{note}

\begin{note}[Understanding `having value \(v\)']
  In a deductive case, if the premises are true, then the conclusion is true.
  Means-end reasoning for desire.
  The value is important.
  If it is true that it past 6pm, then it is true the shop is closed.
  Provides value of shop being closed.

  However, if agent desires that it is past 6pm, then it doesn't follow that the agent desires that the shop is closed.
  Question an agent as to why they think their desires conform to truth --- is-ought problem.

  Means-end reasoning.
  It is true that there is cheese at the centre of the maze.
  And, it is desirable that I obtain the cheese at the centre of the maze.
  Further, it is true that I may only obtain the cheese at the centre of the maze by solving the maze.
  Therefore, it is desirable that I solve the maze.
\end{note}

\begin{note}[Moving to value]
  \color{red}
  Important to note is that way in which agent claims support is important.

  Preface paradox.
  Here, move to value, but these are isolated.
  Don't get to combine these together, no allowed to hold that these are simultaneously true.
  Or, simultaneously assign value to the individual propositions.

  So, move isn't free, so to speak.
\end{note}

\begin{note}
  \begin{proposition}[Defeaters for claimed support]\label{prop:defs-for-CS}
    There are at least two ways in which a claim to support may be defeated.
      \begin{itemize}
      \item Claimed support may (discovered to be) be \emph{misled} by indicating that a proposition has some value that it does not (in fact) have. And,
      \item Claimed support may (discovered to be) be \emph{mistaken} by relying on factors that do not indicate the value of the proposition.
      \end{itemize}
      \vspace{-\topsep}\vspace{-\topsep}
  \end{proposition}
  Misleading support may indicate value, and value may have value indicated by mistaken support.
  Both are defeaters in the sense that, were the agent to learn that the claim to support was misleading or mistaken, then the agent would not hold that the proposition has the value indicated by the (problematic) claim to support or the basis of that (problematic) claim to support.

  Common to distinguish between countervailing and undercutting defeaters.
  Countervailing, support for some other proposition.
  Undercutting, force of claimed support is mitigated.

  Both misled and mistaken are instances of being undercut.
  For, if mistaken or misled then the support is no good.

  Neither are clearly cases of countervailing.
  For, no relation to other support is required.
  However, presence of countervailing implies misled.
  Countervailing does not imply mistaken, though may in some cases demonstrate so.

  Below suggest that when claiming support the agent excepts that claimed support is not misleading or mistaken.
  Hence, given that countervailing implies misled, implicitly take claimed support for a proposition to indicate absence.
  (This is another purpose from separation from support.)

  ???
\end{note}

\begin{note}[M\&M Illustration]
  To illustrate:

  Suppose I glance at the clock on the wall.
  The clock reads 11:45a, so I claim support that it is 11:45a.
  However, it may be the case that the clock is incorrectly set, and the time is 11:15a, or 12:15p, etc.\
  By claiming support from the time expressed by the clock, I would have been \emph{misled} about what the time actually is.
  For, it is not true that the time is 11:45a.
  Though, in all other respects, there may be no fault with claiming that the time is as expressed by the clock and so the claim to support is not mistaken.

  By contrast, suppose I glace at the clock on the wall.
  The clock reads 11:45a, so I claim support that it is 11:45a.
  By claiming support from the time expressed by the clock, I would have been \emph{mistaken} about what the time actually is.
  For, the time expressed by a broken clock is not a good indicator of what the time actually is.
  Though, despite the clock being broken, it is 11:45a and so the claim to support is not misleading.

  And, claimed support for the time from a broken clock expressing the wrong time would be both misled \emph{and} mistaken.\nolinebreak
  \footnote{
    A second illustration:
    Consider a smoke detector, designed to sound an alarm if and only if sufficient levels of smoke are detected.
    Hence, if the alarm sounds, one may claim support there being smoke in the room where the alarm is installed.
    One may be misled; the alarm may have malfunctioned, so no fire.
    Or, one may be mistaken; the same type of alarm may be installed in a different room, wouldn't be a useful indicator.
  }
\end{note}

\begin{note}[Subjectively sound]
  Of course, clocks are typically glanced at, and a glance at a clock is often insufficient to determine whether the clock is incorrectly set or broken.
  Hence, the \emph{possibility} that a clock is incorrectly set or broken --- or more broadly the possibility that claimed support is misleading or mistaken --- does not prevent an agent from claiming support.
  So, ensuring that to-be-claimed support would be neither mistaken or misleading is not a necessary condition for claiming support.
  Rather, we endorse the following condition with respect to these types of defeaters:

  \begin{proposition}[Adequacy of claimed support]\label{prop:CSNMORM}
    If agent claims support for some proposition, then
    % from the perspective of the agent,
    the agent deems that the claimed support
    % adequate whether or not the claimed support may (in fact) be
    is not misleading nor mistaken --- the agent has some expectation that possible defeaters (of the two types of noted) do not obtain.\nolinebreak
    \footnote{
      Stronger than distinct claim that the agent does not deem that the claimed support is misleading or mistaken.
      Stronger requires, presence of some deeming.
      Latter does not.
    }
  \end{proposition}
  Our understanding of `expectation' contrasts to claiming support.
  If an agent claims support that some proposition \(\phi\) has value \(v\) then the agent appeals to information that indicates that \(\phi\) has value \(v\).
  By contrast, we assume that an agent to noting that they have no information that indicates a possible defeater obtains is sufficient to expect that the possible defeater does not obtain.
  For example, there appears to be a bowl of fruit in the centre of the table, and so I claim support by visual inspection that there is a bowl of fruit in the centre of the table.
  I may be misled, what appears to be fruit may be plastic replicas, but I may expect that this possible defeater does not obtain as I have no information which indicates that the items are plastic replicas.

  If some expectation that they do, then this seems enough to deny support.
  If no attention to defeaters, questionable whether any claimed support.

  If the claimed support is not misleading, then the proposition has the value the claimed support indicates the proposition has.
  And, if the claimed support is not mistaken, then the claimed support indicates the value.
\end{note}

\begin{note}[Agents are fallible]
  The missing piece of \autoref{prop:CSNMORM} is an account of what `deem' amounts to.
  The following proposition (\ref{prop:fallibility}) states an assumption, which allows a (general) functional characterisation of `deem'.

  \begin{proposition}[Agents are fallible]\label{prop:fallibility}
    When claiming support for a proposition and agent is never in a position to rule out the (epistemic) possibility that the claimed support is not misled or mistaken.
  \end{proposition}

  I take it to be intuitive that agents are fallible in many cases of claiming support.
  It is not too difficult to think of ways in which claimed support may be misleading or mistaken.
  As noted above, claiming support for what the time is from glancing at a clock seems sufficient, but clocks may be incorrectly set (misled) or broken (mistake).
  Similarly, a sample of \(1,000\) rolls may mislead me into thinking that a die is unbiased, or an overloaded operator may lead to a mistake in claiming support for the proposition that \(x = 4\) is an expression of equality rather than variable assignment.

  \autoref{prop:fallibility} states that an agent is never in a position to rule out the possibility that the claimed support is not misled or mistaken.
  This does not entail that there are defeaters, nor that there are any possible defeaters --- only that defeaters are an epistemic possibility.

  Still, in some cases, this may seem absurd.
  Suppose in front of me are two apples and two pears.
  So, four pieces of fruit.

  However, those appears and pears may not be real pieces of fruit, they may be replicas.
  So beings the process of attempting to quarantine fallibility from infallibility.

  There are two pairs of objects in front of me, and the objects appear to be fruit.
  So, there are four objects in front of me, which appear to be fruit.

  There are two pairs of objects hence there are four objects.

  But I may be hallucinating.

  There appear to be two pairs of objects in front of me, which appear to be fruit.
  So, there appear to be four objects in front of me, which appear to be fruit.

  Whenever there are two pairs of objects, there are four objects.

  Even so, I'm not in a position to rule out the possibility that arithmetic is inconsistent.

  Perhaps identity is a stronger candidate: Any object is identical with itself --- but I doubt one needs to claim support for reflexivity of equality.
  Similarly, it doesn't seem to be the case that I need to claim support for the proposition that my name is Humpty --- no matter what my birth certificate says, I get to decide what my name is.

  Rather than fix a specific account of the `possibility' modal used, here are a handful compatible interpretations:

  \begin{enumerate}[label=\Alph*., ref=(\Alph*)]
  \item\label{CS:I:Never} \emph{In principle} it is not possible for any agent to rule out the possibility that claimed support is not misleading or mistaken.
  \item\label{CS:I:Resources} It is not possible for a \emph{resource bound agent} to rule out the possibility that claimed support is not misleading or mistaken.
  \item\label{CS:I:Class} There is a restricted class of propositions for which and agent is required to claim support, and it is not possible for any agent to rule out the possibility that claimed support for a proposition belonging the class is not misleading or mistaken.
  \end{enumerate}

  To illustrate, consider the proposition that there is an external world:
  \ref{CS:I:Never} denies that there could be, e.g.\ proof of an external world.
  \ref{CS:I:Resources} denies that agents of interest could not demonstrate such a proof even if it were to exist.
  \ref{CS:I:Class} allows an agent may not be required to claim support for the existence of an external world.

  I expect the intended application of claimed support will be compatible with each interpretation, and specifically with respect to \ref{CS:I:Class}, that the propositions will belong to the highlighted class.\nolinebreak
  \footnote{
    I favour the combination of \ref{CS:I:Resources} and \ref{CS:I:Class}, and to leave open whether an idealised agent may rule out the possibility of being misled or mistaken with respect to some propositions when claiming support.
  }\(^{,}\)\nolinebreak
  \footnote{
    In particular, true of ability.
  }
\end{note}

\begin{note}
  Given some interpretation of the possibility modal, the functional role of `deem' is to provide resistance to possible defeaters that an agent is not in a position to rule out.
  How resistant the agent's claimed support needs to be is up to you.

  Provide a test below to help with intuition.
  Before doing so, final proposition, which follows as a corollary from previous.
\end{note}

\begin{note}[\eiS{}]
  Subjectively sound, claimed support indicates value.

  Final proposition.
  Definition.

  % \begin{definition}[Dependence and independence]
  %   An agent's claimed support for \(\phi\) \emph{depends} on some value of \(\psi\) just in case the agent would not claim support for \(\phi\) given any other value of \(\psi\).

  %   An agent's claimed support for \(\phi\) is \emph{independent} of \(\psi\) just in case the agent's claimed support does not depend on the value of \(\psi\).
  % \end{definition}

  Claiming support is independent of value.

  \begin{proposition}[\eiS{-} --- \eiS{}]\label{prop:supp:independence}
    Claimed support indicates the value of \(\phi\)
    % independently of the value of \(\phi\) --- or
    regardless of whether the claimed support is \mom{}.
\nolinebreak
    \footnote{
      Possibly goes against externalism, but I don't think this is right.
      External circumstances may impact the support the agent has.
      However, as these are external, it seems this condition plausibly holds for \emph{claiming} support.
      This is how you get puzzles for externalism.
      In both cases, it's fine for the agent to claim support, but the external circumstances impact whether the agent \emph{has} support.
      The internalist/externalist divide would seem to affect the conditions on claiming.

      Way to expand on this is reconstructing bootstrapping examples with and without \eiS{}.
      If the agent would only get basic support if reliable, then it's not clear that bootstrapping is a problem.
    }\(^{,}\)\nolinebreak
    \footnote{
      One way independence.
      Not clear that value is independent of support.
      So long as sufficiently strong support, not possible for proposition to have value other than claimed support.
    }
  \end{proposition}
  An immediate consequence of \eiS{} is that there is no requirement that claimed support for some proposition \(\phi\) is \nmom{}.

  \eiS{} follows from Propositions~\ref{prop:CSNMORM} and~\ref{prop:fallibility}.

  From Proposition~\ref{prop:CSNMORM}, deemed that claimed support is \nmom{}.
  From Proposition~\ref{prop:fallibility}, always possibility.

  Suppose depends on value of \(\phi\).
  Still, from Proposition~\ref{prop:fallibility}, (epistemic) possibility that value of \(\phi\) differs.
  If differs, then claim to support is misleading.
  By Proposition~\ref{prop:CSNMORM} agent deems not mistaken or misled, and so deems that possible defeaters do not obtain.
  However, requirement of \(\phi\) denies the relevant possibility.
  For, if different value of \(\phi\) then no claim to support.
  Therefore, agent does not deem the claimed support not misleading.


  For, possibility that \(\phi\) does not have value, or that claimed support does not indicate.
  And, given claim to support requires that the possibility does not obtain, agent has not deemed that the claimed support is not misled or mistaken --- rather the agent requires that the claimed support is not misled or mistaken.

  \eiS{} does not deny that things may need to be a certain way for an agent to claim, or to be in a position to, claim support.
  It may be the case that no agent would be in a position to claim support that the speed of light is constant if the speed of light were not constant, but in claiming support an agent must deem that possible defeaters do not obtain, e.g.\ that the laws of nature are constant, and that no mistakes have been made when observing relevant phenomena.

  The force of the corollary is that agent does not require \(\phi\) related things being a certain way in order to claim support for \(\phi\).

  See with failures of `even if\dots' test.
\end{note}

\begin{note}[Quick clarification on \eiS{}]
  A quick clarification may be in order.

  \eiS{} is only about value/support for\(\phi\).
  So, \eiS{} does not prevent agent from claiming support for \(\psi\) from value of \(\xi\) given claimed support that \(\xi\) has that value.
  For example, an agent may claim support that \emph{p} is true from claimed support that \emph{S} knows \emph{p}.
  And, the agent may do so because the proposition that \emph{S} knows \emph{p} is true only if \emph{p} is true.
  That is, so long as the agent does not require \emph{p} to be true in order to claim support for the proposition that \emph{S} knows \emph{p} is true.
  We will return to \eiS{}, expand on this quick clarification, and note related observations in Section~\ref{sec:second-conditional}.
\end{note}

\begin{note}[Adequate reasoning]
  Term this \emph{adequate} reasoning.
  May be good, may involve mistakes, may be bad.
  Kind of reasoning that we, the folk, do.
  Distinction for claiming support is that this is different from whether the agent has support, and we may set issues about whether the agent has support.

  Our interest is what is required for an agent to \emph{claim} support for (premises and) steps of reasoning, rather than what is required for an agent to \emph{have} support for (premises and) steps of reasoning.

  Use support as opposed to justification.
  Initial focus is on epistemic/doxastic attitudes.
  However, practical reasoning.
  For example, means-end.
  Support considered quite general to also include this.
\end{note}

\begin{note}
  To help fix intuition, I suggest a (hypothetical) test to clarify what is meant by `deem': The `even if\dots' test.
  So long as an agent may provide an adequate responses to the test, the agent will be in a position to claim support.
\end{note}

\begin{note}[The `Even if\dots' test]
  The `Even if\dots' test queries whether an agent's claimed support permits an agent to expect that some (epistemically) possible defeater fails to obtain `even if' it does obtain.

  For example, even if \(0.999\dots = 1\), there must be \emph{some} difference between \(0.999\dots\) and \(1\) --- no matter how small --- and some difference between to things is sufficient to establish that they are not equal.

  Implied in this response is something like the observation that \(0.9 = (1 - 0.1)\) and \(0.99 = (1 - 0.01)\), and so \(0.999\dots = (1 - 0.000\dots 1)\), hence \(1 = (0.999\dots + 0.000\dots 1)\), and because \(0.999\dots\) refers to some quantity, \(0.000\dots 1\) likewise refers to some quantity.
  It seems reasonable for an agent to expect that the Archimedean property does not hold for real numbers.

  The example given is an instance of the applied to the possibility that the agent's claimed support that \(0.999\dots \ne 1\) may be misleading, as the antecedent supposes that \(0.999\dots = 1\).

  Generalising, we have outlined two kinds of defeaters that would prevent an agent from claiming support.
  The two types of defeaters suggest two basic instances of the test:
  \begin{enumerate}
  \item[(ML)] Even if \(\phi\) does not have value my claimed support indicates, I deem it to be the case that\dots
  \item[(MT)] Even if I some part (or whole) of my claimed support for the value of \(\phi\) is mistaken, I deem it to be the case that\dots
  \end{enumerate}
  Below we provide three examples for each basic instance of the test, two (plausibly) successful responses and one (plausibly) unsuccessful response..\nolinebreak
  \footnote{
    You may think that some of the adequate responses I suggest are too weak, but for future purposes I require only that some positive answer many be given, and so you may strengthen the requirements on a positive answer as you see fit.
  }
\end{note}

\begin{note}[Even if: misled]

  We being with two plausibly satisfactory responses to being misled.

  \begin{enumerate}[label=(ML\arabic*), ref=(ML\arabic*), series=ML_counter]
  \item\label{ML:asleep} Even if that person is not sleeping, their eyes have been closed for a long time and their breathing is slow.
  \item\label{ML:lying} Even if you are telling the truth, the scientific consensus is against you.
  \end{enumerate}

  With \ref{ML:asleep} the agent has claimed support for the proposition that the person is sleeping.
  It's not too hard to give the impression of being asleep, so there is some possibility that the person is awake and support claimed is misleading.
  Still, even if the person is awake, the person is exhibiting sufficient signs of being asleep for the agent to expect that they are not misled.

  Turning to \ref{ML:lying}, it may be that the person is telling the truth and if the person is indeed telling the truth then any claimed support for a conflicting proposition must be mistaken.
  However, scientific consensus seems sufficient to claim support for the relevant conflicting proposition --- one expects that scientific consensus is not misleading given the rigours of the scientific process.
  Scientific consensus does not (at least typically) require that the person is not telling the truth (though will imply that to be the case).

  In contrast, consider an unsatisfactory response.

  \begin{enumerate}[label=(ML\arabic*), ref=(ML\arabic*), resume*=ML_counter]
  \item\label{ML:forgery} Even if this certificate is a forgery, it professes to be the real thing.
  \end{enumerate}
  If the certificate is a forgery, then the claimed support for the proposition that the certificate is not a forgery would be misleading.

  The response to the (epistemic) possibility that the certificate is a forgery is unsatisfactory because the agent depends on the certificate not being a forgery.
  Hence, that the certificate self-certifies it's authenticity is no response to the (possibility) that it is a forgery.
  Immediate conflict with \eiS{}, and traces back to \autoref{prop:CSNMORM} because it seems quite unreasonable to expect that the certificate is not a forgery based on it's self-certification.

  Of course, many certificates do self-certify (it would be excessive effort to identify a certificate and then be required to find information about what the certificate is for), and perhaps a simple observation that there are no signs of tampering may be a sufficient response to the `even if\dots' test.
\end{note}

\begin{note}[Even if: mistaken]
  Turning to the possibility of mistaken support, consider the following two instances of the `even if\dots' test.

  \begin{enumerate}[label=(MT\arabic*), ref=(MT\arabic*), series=MT_counter]
  \item\label{MT:fake-wound} Even if that is a fake wound, I have no way to tell and the actions of the (apparently) wounded would be a feat of acting.
  \item\label{MT:misquote} Even if the newspaper has quoted the wrong person, the paper has a strong record of accurate reporting.
  \end{enumerate}

  With respect to~\ref{MT:fake-wound}, it seems a mistake to treat a fake wound as indicate the presence of an actual wound, as a fake wound does not require a genuine would but likewise a fake wound may cover a genuine wound.
  The response to the `Even if\dots' test notes that the behaviour of the (apparently) wounded person is sufficiently consistent with their expectations of the behaviour of a person with the (apparent) wound, and would lead to be surprised if the person was not in fact wounded.

  Turning to~\ref{MT:misquote}, if the paper has quoted the wrong person then it would be a mistake to claim support that the person said whatever-it-is-they-said, though it may still be the case that the person did say whatever-it-is-they-said.
  Even so, the strong record of the paper seems sufficient for the agent to expect that the newspaper has not misattributed or imagined the quote on the relevant occasion.

  In contrast, consider an unsatisfactory response.

  \begin{enumerate}[label=(MT\arabic*), ref=(MT\arabic*), resume*=MT_counter]
  \item Even if this library does not using LCC indexing, the library does not have a copy of `Measurement Theory' because as search for `H61 .R593' returns no results.
  \end{enumerate}
  Holding that a library does not have a copy of a book because a search for the book under a particular indexing system would be a mistake.
  For, if the library does not use the particular indexing system then a search using that indexing system will always fail, regardless of whether or not the library has a copy of the book.

  In turn, a failed search for an LCC index in the library's database does not seems sufficient for an agent to claim that the library does not have a copy of the book unless the agent is in a position to claim support that the library uses LCC indexing.
  Following, it seems the failed response to the `Even if\dots' test may be supplemented by noting that the library is a research library, and therefore likely uses LCC indexing, etc.\
\end{note}

\begin{note}[Even if: more]
  Primary observation from these examples is that in positive cases provided responses indicate that some response to possibility of being misled or mistaken is available to the agent.
  In the failure cases, no response.

  % Cases of entailment, preface paradox.
  % Mistake somewhere.
  % Here, support is good for each of the claims made in the preface, but these do not combine to make a case that no mistake has been made across any of the claims.
  % May come down to familiar concerns, too significant possibility of being misled.
  % May also think that claimed support for each might require a mistake in one.
  % I.e. source for claim includes further claims which state that source for some other claim is mistaken.
  % Problem with using both sources, even if for distinct propositions.

  % Interesting problem later.
  % For now, simple example.
  An interesting case for misled is the preface paradox.
  Claimed support for everything in the preface, but also claimed support for mistake.
  Credence resolves tension, remains noteworthy that even if confident of some potential defeater, claimed support is sufficient to resist undermining claimed support.
\end{note}

\begin{note}[Bootstrapping]
  Question about how \eiS{} relates to bootstrapping.

  In particular, worry that endorsing \eiS{} rules out reliabilism, and so on.

  First, difference between claiming support and support proper.

  So, nothing in particular follows for theories concerned with support proper.

  Further, no clear issue, I think.
  Consider the gas gauge problem.

  Way to formulate this as an issue for \eiS{} is to hold that the agent doesn't get to reason about the gas gauge without taking it to be reliable.
  Indeed, this seems obvious.
  Problem is that it looks as though the reasoning only goes through if the agent is \nmom{} about the reliability of the gauge.

  So, problem.

  Though, let's be clear.
  The issue is not that the reasoning seems problematic.
  No, that's the problem for other theories.
  Rather, other theories don't see this reasoning as problematic.
  In turn, this leads to tension between \eiS{} (and our broad understanding of claiming support) and those theories.

  So, this leads to an interesting tension.
  Seems to need that either it's okay or that other theories are wrong.

  There's a third option, though.
  Argue that \eiS{} isn't enough to find a problem.
  If so, other theories may be right, but also don't need to endorse reasoning.

  Observation: It follows from \eiS{} that agent would be misled if not reliable, and would fail to gain support --- intuitively, anyway --- but that doesn't immediately rule out \emph{claiming} support.

  So, question is whether agent can move from gauge to gas even if \mom{}.
  In other words, recognise the problem, but still go for the reasoning.
  Well, that seems possible.

  Variant of basic knowledge.

  This is kind of okay to motivate with respect to certain domains.
  Esp.\ vision.

  So, to summarise.
  The issue here is getting the link.
  There's no requirement from \eiS{} alone that the link is going to be broken if the gauge isn't (actually) reliable.
\end{note}

\begin{note}[Closing support]
  To summarise, claim of support.
  Certain kind of independence.
  Only interested in support, and not how this relates to attitudes.
  Somewhat intuitive, but no claims that this is the only understanding of support.

  For the moment, this provides clarity for understanding of support.
  Below, use to argue for failure to claim support.
\end{note}

\begin{note}[Something to emphasise]
  \color{red}
  Something to emphasise here is that this means that there's a way for an agent to claim support without being certain that \(\phi\) is the case.
  I don't have any answers for what this is.
  However, I do take this to be highly intuitive.
\end{note}

\subsection{Claiming support, reasoning, and ability}
\label{sec:inter-with-claim}

\begin{note}
  In this section we introduce two propositions which characterise what we are arguing against and what we are arguing for.
  \ESU{-} and \EAS{-}, respectively.
  Argue against \ESU{} by cases involving ability.
  Argue for \EAS{} which outlines the way in which ability conflicts with \ESU{}.

  Start with introduction of \ESU{}.
  And, motivate with reference to literature on the basing relation and rationality as responding to reasons.

  Move to \EAS{}, clarify relation to \ESU{} and contrast to related principle argued for by \citeauthor{Moretti:2019wx}.
\end{note}

\subsubsection{\ESU{}}
\label{sec:esu}

\begin{note}[Recap of \USE{}]
  Brief recap of \USE{}.
  Introduced the idea, and then expanded on the details.
\end{note}

\begin{note}[Focus]
  We will argue against the converse of~\USE{}:

  \begin{proposition}[\ESU{-} --- \ESU{}]\label{denied-claim}
    An agent may claim support for some conclusion of reasoning by claiming that the conclusion of reasoning is supported by premises and steps of reasoning \emph{only if} the agent has witnessed the reasoning (e.g.\ traced the claimed support for those premises and steps used to claim support for the conclusion).\nolinebreak
      \footnote{
    Three brief notes on~\ESU{}:

    First, the `has' in~\ESU{} only requires `at some point in the past'.
    Hence,~\ESU{} does not require the agent to reason from premises to conclusion each time the agent claims support for the conclusion.
    For example, if an agent proved the Deduction Theorem for propositional logic last week, then the agent would not be in conflict with~\ESU{} if they claimed support for the Deduction Theorem on the basis of the premises and reasoning they performed in the past.

    Second, and following from the first,~\ESU{} will also hold for any stronger statement --- for example if `has' is read as `has just'.
    For example, requiring that the agent's memory of proving the Deduction Theorem allows the agent to claim support, rather than the premises and steps used in the past.
    The argument (stated below) denies that, given certain information, the agent needs witnesses any reasoning in order to claim support for the result of witnessing the reasoning.

    Third, as~\ESU{} is about when an agent may \emph{claim} support, it is compatible with~\ESU{} to hold that the agent \emph{has} support --- regardless of whether the agent has witnessing the reasoning.
  }
  \end{proposition}

  \ESU{}, as the converse of~\USE{} focuses on reasoning.
  {
    \color{red}
    Assumption here is that there's no other way to claim support.
  }
  To clarify,~\gESU{} is a generalisation of~\ESU{}.

  \begin{proposition}[\gESU{}]
    An agent may claim support for some proposition \(\phi\) by appealing to some materia\nolinebreak
    \footnote{Latin.
      Material, matter, basis, information, foundation, ground, etc.
    }
    \emph{M} only if the agent uses \emph{M} in the reasoning which culminates in claiming support for \(\phi\).
  \end{proposition}
  Our focus is with whether an agent is required to have \emph{used} something in order to appeal to that thing when claiming support.
  No fixed understanding of `use' is assumed in the statement of~\ESU{} and~\gESU{}, and we will offer some disambiguation below.
  First, a basic illustration.
\end{note}

\begin{note}[Simplest]
  \color{red}
  Difference between \(\phi\) therefore \(\psi\) and \(\phi\) and \(\phi \rightarrow \psi\) therefore \(\psi\).
  Possible for agent to reason from \(\phi\) to \(\psi\), so in principle possible for agent to claim support for \(\psi\) from \(\phi\).
  However, \ESU{} denies this if the agent doesn't do the reasoning.
  Instead, agent also needs \(\phi \rightarrow \psi\), and then they're fine.

  Key point of the suggested revision is that \ESU{} doesn't need to focus on claiming support for \(\phi\), specifically.
  Rather, it's just about establishing a relation of support between \(\phi\) and \(\psi\).

  Then, big idea is that ability is not understood as an instance of \(\phi \rightarrow \psi\), which it might otherwise seem to be.

  Indeed, viewed from the perspective of propositional logic, deduction theorem.
  If \(\vdash \phi \rightarrow \psi\) then \(\phi \vdash \psi\).
  \ESU{} denies that the same holds for claimed support.
  Seems quite sensible.
\end{note}

\begin{note}[Illustration]
  To illustrate~\ESU{}, consider the illustration provided for~\USE{}.

    If the agent did not measure the rectangle, understand how to calculate the area of a rectangle, or perform the required calculations, then the agent would not be in a position to claim support that area of the rectangle is \(133\text{cm}^{2}\).
  A lucky guess that the area of the rectangle is \(133\text{cm}^{2}\) would not allow the agent to claim support that the area of the rectangle is  \(133\text{cm}^{2}\) on the basis of the dimensions of the rectangle, the agent's understanding of how to calculate the area of a rectangle, and the relevant mathematics.
  And, it seems the agent is not in a position to base their lucky guess in such a way because the agent did not reason from the dimensions of the rectangle, the agent's understanding of how to calculate the area of a rectangle, and the relevant mathematics.\nolinebreak
  \footnote{
    Moving to another agent, observe doing the work, get report.
    Easy to resist, by adding in additional premise.
    Still, no presupposing that this needs to be done.
  }
  Similarly, if an agent learns that a rectangle with dimensions of \(19\text{cm}\) by \(7\text{cm}\) may be calculated to have an area of \(133\text{cm}^{2}\), then the agent may not claim support for the area of the rectangle on the basis of the calculation.
  If the agent has not performed the calculation, then the agent may not appeal to the use of the calculation when claiming support --- rather, the agent mentions that the calculation is true.\nolinebreak
  \footnote{
    Slight weakening of~\ESU{} may be made.
    So long as \emph{some} agent has performed the calculation.
    Argue against~\ESU{}, and the argument made will hold for this weakening.
  }
\end{note}

\paragraph{Intuition}

\begin{note}[Intuition]
  \ESU{} and~\gESU{} seems quite plausible, at least to me.
  The proposition is a careful statement of an intuitive ideas:

  Whether or not an agent claims support is the result of the structure of the reasoning process, and if some premises or step is not used, then it is irrelevant to the structure of the process.
  Hence, the only premises and steps of interest when claiming support are those used in the reasoning process.

  Rests on the broader idea from~\gESU{}.
  Claiming support is the result of some agentive process, and the result of an agentive process is explained by the constituents of the process.\nolinebreak
  \footnote{
    Ah, the homonculus.

    Question about whether the agent is important.

    This gets difficult.

    Consider clocks.
    Clock does not keep track of time.
    Rather, mechanical system designed to change in constant with some passage of time. (Cf.\ \textcite{Smith:1988aa}.)

    Agent may be like this.
    Distinction is intentionality.
    When I go about keeping track of the time, I'm attempting (at least typically) to maintain reference to what the time is.
    Figure out a way to approximate a second, and that's what's happening.
    Approximation.
    If it is noted that I requarly sigh every minute, use this, but I wouldn't be keep tracking of time, though you may be using regularity to do so.
    So, in the former case, using understanding of time, while in the latter not doing so.
  }

  As~\gESU{} is restricted to an agent claiming support, things seem a little easier.
  Problems with interpretation, however.
  Transparency.
  Familiar, if debatable, illustration.
  Freud.
  (Here, adjourning the meeting by saying something mistaken.)
\end{note}

\begin{note}[Analogy]
  By analogy, whether or not my mug of (once cold) coffee overheats in the microwave is the result of some process involving electromagnetic radiation.
  My desire that the mug of coffee does not overheat is not used as part of the process of heating the coffee, and so is irrelevant to the structure of the process.

  My desire may explain why the mug of coffee is taking part in a certain process, and an unused premise or step may explain why an agent performed so reasoning.
  Still, a premise or step must be used as part of the process of reasoning to stand in explanation for the result of reasoning.

  Press the analogy further: Reasoning is a causal process.
  And, any property of reasoning reduces to cause and effect.
  If premises or steps are not used, then those premises or steps stands outside the relevant causal trace, and may not be appealed to when accounting for some structural property of the conclusion of the instance of reasoning (here, that the agent claims support for the conclusion).
\end{note}

\paragraph{\ESU{} in the wild}

\begin{note}
  \color{red}
  Given proposed revision to \ESU{} this section should be expanded a little.
  For, most of the cases talk about claiming support for \(\phi\) directly, while \ESU{} is more general in that it talks about claiming support for any entailment between \(\phi\) and \(\psi\).
\end{note}

\begin{note}[Causal theories of basing]
  Indeed,~\ESU{} seems to be implied by various causal theories of basing.

  \citeauthor{Pollock:1999tm} introduce the basing relation with the following observation:
  \begin{quote}
    To be justified in believing something it is not sufficient merely to \emph{have} a good reason for believing it.
    One could have a good reason at one's disposal but never make the connection.
    \dots
    Surely, you are not justified in believing [something], despite the fact that you have impeccable reasons for it at your disposal.
    What is lacking is that you do not believe the conclusion on the basis of those reasons.\nolinebreak
    \mbox{}\hfill\mbox{(\cite[35]{Pollock:1999tm})}
  \end{quote}
  The observation falls short of being an account of the basing relation, but the intuition \citeauthor{Pollock:1999tm} appeal to is instructive.
  It seems that an agent must connect reasons and the content of a belief in order for the belief to be formed on the basis of those reasons, and hence be justified by those reasons.
  In turn, if a connection is made between reasons and the content of belief, then those reasons are used by the agent.

  For a concrete instance, consider \citeauthor{Moser:1989tv}'s account of the basing relation:
  \begin{quote}
    \emph{S}'s believing or assenting to \emph{P} is based on his justifying propositional reason \emph{Q} \(=_{\text{df}}\) \emph{S}'s believing or assenting to \emph{P} is causally sustained in a nondeviant manner by his believing or assenting to \emph{Q}, and by his associating \emph{P} and \emph{Q}.\nolinebreak
    \mbox{}\hfill\mbox{(\cite*[157]{Moser:1989tv})}
  \end{quote}

  Suppose we have a conclusion from some premises and steps of reasoning.
  If the agent has not witnessed the relevant reasoning, then it seems the conclusion is not causally sustained in a nondeviant manner by his believing or assenting to the premises of the reasoning, nor has the agent associated the conclusion with the premises by witnessing the relevant steps of reasoning.

  To illustrate, claim support that 173 is prime.
  It's possible that I did the prime factorisation, and possible that I took that representation to be part of the reason why I claim that 173 is prime.
  However, represented query of whether prime to wolfram alpha as justifying, and that's why I claimed support.
  So, definitely not from okay to appeal to the reasoning I have not witnessed.
  And, if infer that 173 is prime from claimed support that I have the ability to demonstrate that 173 is prime, the same issue.
  As I've not witnessed, then no role for \emph{Q}, whatever that turns out to be.

  This is a quick argument, and borders on conjecture, so let us examine the relevant association in greater detail.
  \citeauthor{Moser:1989tv} distinguishes between occurrent and non-occurrent satisfaction of association relations.

  We start with occurrent satisfaction of an association relation:
  \begin{quote}
    \emph{S} occurrently satisfies an association relation between \emph{E} and \emph{P} \(=_{\text{df}}\)
    \begin{enumerate*}[label=(\roman*), ref=(\roman*)]
    \item\label{moser:oar:i} S has a \emph{de re} awareness of \emph{E}'s supporting \emph{P}, and
    \item\label{moser:oar:ii} as a nondeviant result of this awareness, \emph{S} is in a dispositional state whereby if he were to focus his attention only on his evidence for \emph{P} (while all else remained the same), he would focus his attention on \emph{E}.\newline
    \mbox{}\hfill\mbox{(\Citeyear[141--142]{Moser:1989tv})}
    \end{enumerate*}
  \end{quote}

  \emph{de re} awareness is a non-propositional, direct awareness of \emph{E} supporting \emph{P}.

  \ESU{} follows from~\ref{moser:oar:i}.
  \emph{de re} awareness, but this doesn't rule out use.
  \ESU{} does not require that the agent engages in propositional reasoning.

  In cases where the agent has not witnessed reasoning, there is no \emph{de re} awareness.
  Without the reasoning taking place, the agent is not directly aware of what the reasoning consists of.

  Following, the definition of non-occurrent satisfaction of an association relations is derived from occurrent satisfaction of an association relations by allowing~\ref{moser:oar:i} to be satisfied at some point in the past while requiring that~\ref{moser:oar:ii} continues to be satisfied in the present.
  As noted, \ESU{} is compatible with the agent having witnessed the reasoning at some point in the past.
  Therefore, \ESU{} is entailed given both occurrent and non-occurrent satisfaction of association relations
\end{note}

{
  \color{red}
  This doesn't make sense\dots
  I think the idea I had was that the agent has to use the represented relation.
  Hm, so, the idea is that in the case of \(\phi \vdash \psi\), the agent hasn't represented how to get from \(\phi\) to \(\psi\), and therefore the agent isn't allowed to base \(C\) or \(R\) given \citeauthor{Neta:2019aa}'s account.
  I don't think this is sufficiently clear from what I have written.
  However, it does seem relevant.
  And, in also, basing doesn't necessarily need to be between beliefs.
  This could just be a relation of justification\dots though this isn't necessarily the case.
  So care is need.
  Still, with a little rewriting this looks useful.

  The tricky part is understanding what it is to represent R as justifying C.
  What I need is the idea expressed above, that the relevant representation is sufficiently detailed.
  I think this should be in \citeauthor{Neta:2019aa}.
  For, intuitively representing R as \emph{justifying} C is stronger than a representation with the content that R justifies C.
}

\begin{note}[Representationalism]
  \citeauthor{Neta:2019aa} generalises (purely) epistemic interest in basing relations to cover the explanatory relation between reasons and (rationally evaluable) states held, or actions performed, by an agent.

  On the way to a novel proposal, \citeauthor{Neta:2019aa} sketches a broad characterisation of representationalist theories of (generalised) basing:
  \begin{quote}
    \begin{enumerate}[label=(R\arabic*), ref=(R\arabic*)]
    \item\label{neta:RC:b} \emph{basing} C on R involves the agent's representing R as justifying C, and
    \item\label{neta:RC:jb} \emph{justifying basing} of C on R consists in the adroitness of this representation.\nolinebreak
          \mbox{}\hfill\mbox{(\Citeyear[192]{Neta:2019aa})}
    \end{enumerate}
  \end{quote}
  As \ESU{} does not distinguish between successful and unscuccesful instances of claiming support, our interest is with~\ref{neta:RC:b}.
  And, in contrast to \citeauthor{Moser:1989tv}, a representationalist theory may lack a causal component.
  Indeed, \citeauthor{Neta:2019aa} considers scenario in which an agent receives information from some source, forms a belief which is supported by the received information, and represents the received information as justifying the belief.
  The twist, however, is that the agent forming the belief was caused by some other source.
  For example, an agent may listen to a speech given by a talented orator and form a belief in response to the speech.
  The agent may represent the content of the speech as justifying the conclusion, while the cause of the belief being formed is the emotional impact with which the orator stated the conclusion.
  Following the representationalist characterisation, the agent would base the content of the belief on the content of the speech rather than the emotional impact with which the speech concluded.
  Indeed, the agent may do so even if they recognise that they were swayed by emotion.

  As before, consider a conclusion of some reasoning that the agent has not witnessed.
  If the agent has not witnessed the reasoning, then the agent has not represented some or all of the relevant premises and steps of reasoning.
  Therefore, it seems that it is not possible for the agent to represent the premises and steps of reasoning as justifying the relevant conclusion.
  In other words, a representationalist account requires (minimally) that an agent represents premises and steps of reasoning as justifying when claiming support for some conclusion of reasoning, and hence use of those premises and steps.

  {
    \color{red}
    What is going on here\dots
    The point is that if we follow \citeauthor{Neta:2019aa} then there needs to be a representation.
    In turn, the issue is that it's not clear that the agent needs to reason from \(\psi\) to \(\phi\) in order to obtain the relevant representation.
    So, it's not clear that \citeauthor{Neta:2019aa} actually is relevant.

    So, it's this previous paragraph that needs attention.
    No use, then no representation.
    This is the only point that really matters.
    So, I need to find something in \citeauthor{Neta:2019aa} that supports this, or somehow provide a much better argument.

    Then, in the following paragraph is redundant.
    The issue is with how the relevant representation is obtained.
    The part where I'm getting confused is that \citeauthor{Neta:2019aa} doesn't hold that the agent needs to do the reasoning each time the representation is used.
  }

  As an aside, it is not clear whether representing an entailment or inference is the same as reasoning with an entailment, and therefore it does not seem to follow from the representationalist characterisation that the agent must witness the relevant reasoning.
  However, the interpretation of `use' is intended to be sufficiently broad to cover such cases.\nolinebreak
  \footnote{
    Alternatively, a clause may be added to~\ESU{} which denies that the agent represents the relevant premises and steps of reasoning.
    The argument made against~\ESU{} is compatible with the use of representations, or mere representation even if unused --- though it is unclear to me what an unused but represented premise or step would matter when claiming support.
  }

  Further, \citeauthor{Neta:2019aa}'s discussion is instructive because the response \citeauthor{Neta:2019aa} offers to some problematic scenarios focus on \emph{how} a representation is used.
  One may hold that the agent in the example given did not base their belief in the conclusion on the content of the speech in view of the fact that the agent was swayed by emotion.
  If so, \citeauthor{Neta:2019aa} proposes the following revision:
  \begin{quote}
    \begin{enumerate}[label=(R\arabic*\('\)), ref=(R\arabic*\('\))]
    \item\label{neta:RC:jp} for an agent to C based on reason R involves not merely the agent's representing R as justifying C---it also involves \emph{this latter representation (or its content) being part of the reason why the agent C's}.\nolinebreak
      \mbox{}\hfill\mbox{(\Citeyear[197]{Neta:2019aa})}
    \end{enumerate}
  \end{quote}
  The added clause states that the relevant representation must explain why the agent formed a belief.
  Hence, given~\ref{neta:RC:jp} the agent would not be permitted to base their belief in the content of the speech given that they were swayed by emotion.
  Intuitively,~\ref{neta:RC:jp} expands on what it is for premises and steps of reasoning to be use when forming a belief.
  So, given that representation requires use, the expanded clause may be seen as focusing on \emph{how} the representation is used.
\end{note}

\begin{note}[Responding to reasons]
  As final motivation, consider the proposal at the core of \citeauthor{Lord:2018aa}'s (\Citeyear{Lord:2018aa}) thesis that being rational is to correctly respond to reasons.

  \begin{quote}
    \textbf{Correctly Responding:} What it is for A's \(\phi\)-ing to be ex post rational is for A to possess sufficient reason S to \(\phi\) and for A's \(\phi\)-ing to be a manifestation of knowledge about how to use S as sufficient reason to \(\phi\).\nolinebreak
    \mbox{}\hfill\mbox{(\Citeyear[143]{Lord:2018aa})}
  \end{quote}

  An agent's action is rational only if the action is a manifestation of some know-how.
  \citeauthor{Lord:2018aa} summaries:

  \begin{quote}
    \dots when one manifests one's know-how, dispositions that are directly sensitive to normative facts are manifesting. Thus, the competences involved in the relevant know-how make one directly sensitive to the normative facts\nolinebreak
    \mbox{}\hfill\mbox{(\Citeyear[16]{Lord:2018aa})}
  \end{quote}

  For our purposes, following example of manifesting know-how directly relates to reasoning:

  \begin{quote}
    The most salient disposition [when appealing to \emph{p} as a reason]\nolinebreak
    \footnote{Note, \citeauthor{Lord:2018aa} (explicitly) not talking about believing that \emph{p} is a reason, but argues that the cited disposition to present both when appealing to p as a reason and believing that \emph{p} is a reason.}
    is the disposition to (competently) use \emph{p} as a premise in reasoning.\nolinebreak
    \mbox{}\hfill\mbox{(\Citeyear[25]{Lord:2018aa})}
  \end{quote}

  Hence, suppose an agent appeals to a premise of reasoning in order to claim support for some conclusion.
  Then, if the agent does not use the premise of reasoning, it seems the agent does not manifest know-how, which is required for the appeal to meet \citeauthor{Lord:2018aa}'s account of rational action.

  Of course, that the noted disposition is the most salient does not rule out alternative, less noteworthy, dispositions.
  However, it is unclear to me how to \emph{manifest} know-how without use.
  Looking ahead, it does not seem to be the case that I manifest my ability to show that a certain rule of inference is sound when skipping over details in a completeness proof.
  However, I may manifest know-how regarding the (presumed) truth of the ability attribution.

  Likewise with my ability to establish a preference for tofu over any other kind of miso when ordering soup.
\end{note}

\begin{note}[Summarising illustrations]
  Three examples of claiming or establishing relations of support have been given.
  Each example suggests that if an agent does not use a premises or steps when claiming support, then an agent may not claim support by appeal to the unused premises or steps.

  Stepping back,~\ESU{} may be seen as a desiderata for any account of (successfully) claiming support.
  For:
  If an agent (successfully) claims support for some conclusion of reasoning, then the premises and steps used with respect to that claim of support is itself the result of some reasoning --- the reasoning that culminated with the claim to support itself used premises and steps of reasoning.
  So, given that the agent used certain premises and steps when claiming support for conclusion, some property of the premises and steps used, an adequate account of claiming support must explain how the premises and steps used permit the agent to claim support.\nolinebreak
  \footnote{
    Note, however, that this argument does not imply that support for the conclusion must be accounted for in terms of the premises and steps used by the agent to claim support.
    As we will note below, one may hold that an enthymematic argument permits an agent to claim support, while the relevant relation of support is secured by the corresponding non-enthymematic argument.
    Cf.\ \textcite{Moretti:2019wx} for suggestions along these lines.
  }
  In turn, if an agent appeals to premises and steps that they did not use, then those premises and steps must be redundant.

  Turning to ability.
  Suppose and agent appeals to
  \begin{enumerate*}
  \item their ability to demonstrate that \(\phi\) is the case, and
  \item that \(\phi\) must be the case in order for the agent to have the ability to demonstrate that \(\phi\)
  \end{enumerate*}
  in order to claim support for \(\phi\).
  Then, the premises and steps involved in a full account of reasoning from the two claims must be sufficient to claim support that \(\phi\) is the case.
  So, as the agent does not witness their ability to demonstrate that \(\phi\) in such reasoning, it must be the case that claimed support for (the property of) having the ability to demonstrate that \(\phi\) is sufficient for such reasoning.
\end{note}

\subsubsection{\EAS{}}
\label{sec:eas}

{
  \color{red}
  Perhaps include a note about how the argument relates to \EAS{}.
  I don't provide a direct argument, but this is the best way I see of resolving the tension.
}

\begin{note}[Alternative]
  \ESU{} is a universal claim, and so applies to all instances in which an agent may claim support for conclusion on basis of support for premises and steps of reasoning --- an agent may only claim support if the agent reasoned from the premises via the steps to the conclusion.

  Our goal is to motivate the following exception to \gESU{}, and hence \ESU{}:

  \begin{proposition}[\EAS{-} --- \EAS{}]\label{prop:EAS}
    If an agent has claimed support that they have the ability to (adequately) reason to some conclusion, then it may be permissible for the agent claim support for the conclusion by appealing to some materia \(M\) to claim support for \(\phi\) without using \(M\) in the reasoning which culminates in claiming support for \(\phi\).
  \end{proposition}
\end{note}

\begin{note}[Intuition for \EAS{}]
  \EAS{} is a conditional.
  Antecedent is claimed support for ability.
  Consequent is that it may be permissible to violate \gESU{}.
\end{note}

\begin{note}
  Now, started with \USE{}, and then looked at \ESU{}, the converse.
  Both of these we have a particular instance of reasoning in mind.
  Now, \EAS{} may, intuitively, be understood to states that whatever that reasoning is, if an agent has claimed support that they're able to witness such reasoning, then the agent may claim support.

  However, things are a little more complex.
  \EAS{} is about the ability to claim support to reason to some conclusion.
  However, \EAS{} does not state that the agent may claim support for the conclusion on the basis of the premises that they would reason from were they to witness the ability.

  Issue here is that the substance of \EAS{} --- what the relevant materia amounts to --- depends on two things:
  \begin{itemize}
  \item How (appeal to) ability is understood, and
  \item The kind of reasoning involved in the appeal to ability.
  \end{itemize}

  We will outline the basics, then reformulate \EAS{} using one what in which (appeal to) ability is understood.

  Start, how ability is understood.
  Lead naturally to the kind of reasoning involved.

  The argument for \EAS{} will not depend on how ability is understood, but the kind of reasoning involved.
  Still, kind of reasoning involved when combined with how ability is understood.
\end{note}

\begin{note}
  Briefly stated,
  \AR{} understands ability in terms of some (complex) property.
  \WR{} understands ability in terms of possible witnessing events.

  For example, \AR{} may involve the property (attribution) of understanding geometry, perhaps broken down into the understanding or availability of various definitions, propositions, lemmas, theorems, and steps of reasoning.
  While, \WR{} would involve reasoning with particular definitions, propositions, lemmas, theorems, and steps of reasoning.

  So, agent appeals to property, or the reasoning itself.

  The purpose of this distinction is to ensure that our argument against \gESU{} does not rest on a particular way of understanding ability that may not extend to other ways of understanding ability.

  Conjecture that these are fundamentally connected.
  Witnessing event only if understanding.
  And, understanding only if possible to witness reasoning.

  Still, difference.
  Relevant properties are properties of the agent as they are.
  The witnessing event, by contrast, is a possible event.\nolinebreak
  \footnote{
    Property of there being a possible event involving the agent.
    In this case, still distinct from \WR{} as that the agent is part of possible event is still distinct from the reasoning that the agent would witness in the relevant event.
  }

  These are brief characterisations, but enough for now.
  Both~\AR{}~and~\WR{} will be considered at length in~\autoref{sec:ar-wr-1}.
  In addition to a more thorough treatment of the core ideas, \autoref{sec:ar-wr-1} includes additional examples, and an argument that~\AR{}~and~\WR{} are exhaustive --- any way of understanding ability will conform to either~\AR{}~and~\WR{}.
\end{note}

\begin{note}
  Now turn to the kind of reasoning involved.

  Motivated \AR{} in terms of understanding of premises and steps of reasoning, and \WR{} in terms of a possible event in which agent reasons with particular premises and steps.

  However, a further distinction in terms of what appeal to the relevant premises and steps or instance of reasoning amounts to.

  First, there is the \emph{existence} of premises and steps, or the \emph{possibility} of the witnessing event.
  Second, there is the premises and steps themselves, or the witnessing event.

  Difference from perspective of step of reasoning.
  


  \footnote{
    Different from \dd{} \dr{} distinction.
    I believe that XXXXXX is a spy.

    Two ways of disambiguating.
    I believe that \emph{there is a} person called XXXXXX and that that person is a spy.
    I believe that \emph{the} person called XXXXX is a spy.

    Former, referring to whatever it is satisfying description, with belief that there is something which satisfies description.
    Latter, describing something what belief refers to.
  }
\end{note}


\begin{note}[Types of reasoning]
  In the case of \AR{}, it's those properties involved in having the ability.
  However, in case of \EAS{} agent hasn't reasoned from those.
  Instead, the agent claimed support for ability allows agent to appeal to these.

  The difficulty here is that this rests on an important, but delicate, distinction.
  Roughly, the difference between appealing to a property and appealing to the existence of a property.
  This will be detailed in~\autoref{sec:reas-dd-reas}

  Here, the former.

  Same applies to \WR{}.
  There's a difference between appealing to the existence of a witnessing event, and the reasoning that is part of the witnessing event.
\end{note}

\begin{note}[\EASw{}]
  \begin{proposition}[\EASw{-} --- \EASw{}]\label{prop:EASw}
    If an agent has claimed support that they have the ability to (adequately) reason to some conclusion, then it may be permissible for the agent to claim support for the conclusion by claiming support for the premises and steps of reasoning that the agent would use to witness their ability to reason to the conclusion.
  \end{proposition}

  Loosely restated,~\EASw{} holds that if an agent may claim support for having the ability to witness some reasoning, and is aware of the conclusion of that reasoning, then the way in which the agent claims support for the conclusion of that reasoning may mirror the way in which the agent would claim support for the conclusion by witnessing the reasoning (and hence using the relevant premises and steps).

  The (possible) event of the agent witnessing their ability to demonstrate \(\phi\) involves reasoning with various premises and steps which culminate in claiming support for \(\phi\).
  So, if~\EASw{} is true, then the agent may appeal to those premises and steps which are used in the (possible) witnessing event.

  One way to think about~\EASw{} (which we will explore in more details later) is in terms of propositional support.
  For, if an agent has the ability to demonstrate that \(\phi\) is the case, then the agent has propositional support for \(\phi\) as there is a way for the agent to demonstrate that \(\phi\) is the case.
  In addition, that the agent has the ability to demonstrate that \(\phi\) is the case ensure that the agent is in a position to make use of the available propositional support for \(\phi\).
  In turn,~\EAS{} may be interpreted to hold that so long as the agent has such information about their position to make use of the available propositional support for \(\phi\) then the agent does not need to reason with the relevant propositional support in order to claim support for \(\phi\) in virtue of the available propositional support for \(\phi\).
\end{note}

\begin{note}[Conditional]
  Here, note that it's a conditional, but also that it only states there are instances.
  It doesn't follow that ability will always allow the agent to claim support.

  The conditional is weak primarily because it is not at all clear that it holds in general.
  There are various cases in which it seems appeal to ability is blocked.

  Easiest cases involve claiming support in some public setting.
  Of course, success in a public setting is not necessarily required for private success.
  Same problem with testimony.
  I'm confident in a source and you're not.
  I fail to convince you, but I remain convinced myself.

  Still, seems as though similar considerations extend.
  For example, doing a PSET where I'm allowed to use theorems I've already proved.
  Have notes of what those theorems are.
  And, ability to prove them.
  Still, might refrains from using them until I've proven them once again.

  More could be said here, and it may be possible to argue for a stronger variant of \EAS{}.

  Even though it's weak, the condition is still interesting.
\end{note}


\begin{note}
  So, if~\EAS{} is true, then there are cases in which an agent is not required to reason from premises they may claim support for to some conclusion in order to obtain support for the conclusion on the basis the support the agent has for the premises.\nolinebreak
  \footnote{
    Stated~\EAS{} as an exception to~\ESU{}.
    And, we will argue that~\EAS{} is true.
    However, we will not argue that~\EAS{} \emph{is an exception} to~\ESU{}.
    To do so would require an argument that \ESU{} holds for other cases.
    Likewise, no argument that~\EAS{} is the only exception, as to do so would require argument that~\ESU{} holds for all other cases.
    Take~\ESU{} to be plausible, and suspect that there are few, if any, further exceptions, but~\EAS{} may stand independently on any further statements about claiming support.
  }
\end{note}

\begin{note}
  \color{red}
  I want to clarify \EAS{} a little.
  The use of `may' is problematic.
  It could be read as `it's always okay, but it's up to the agent'.
  Or, `it's possible, given appropriate context'.
  The latter is what I want, and is important for cases where doubts are plausibly raised about the ability.
\end{note}

\begin{note}[\EAS{} illustration]
  To illustrate \EAS{}

  \begin{illustration}\label{ill:rectangle:ability}
    Suppose you provide me with `novel' information that:
    \begin{enumerate}[label=\emph{A}\arabic*., ref=(\emph{A}\arabic*), series=EAS_counter]
    \item\label{EAS:ex:box:if} If I have ability to calculate the area of a box, then I have the ability to demonstrate that a rectangle with dimensions \(19\text{cm}\) by \(7\text{cm}\) has area \(133\text{cm}^{2}\).
    \end{enumerate}
    The information is `novel' because I have not been previously informed (in any way) about the area of a rectangle with dimensions \(19\text{cm}\) by \(7\text{cm}\).

    Still, (I claim support that):
    \begin{enumerate}[label=\emph{A}\arabic*., ref=(\emph{A}\arabic*), resume*=EAS_counter]
    \item\label{EAS:ex:box:gen} I have the ability to calculate the area of a rectangle.
    \end{enumerate}
    Therefore:
    \begin{enumerate}[label=\emph{A}\arabic*., ref=(\emph{A}\arabic*), resume*=EAS_counter]
    \item\label{EAS:ex:box:spec} I have the ability to demonstrate that a rectangle with dimensions \(19\text{cm}\) by \(7\text{cm}\) has area \(133\text{cm}^{2}\).
    \end{enumerate}
    From~\ref{EAS:ex:box:spec} it follows that:
    \begin{enumerate}[label=\emph{A}\arabic*., ref=(\emph{A}\arabic*), resume*=EAS_counter]
    \item\label{EAS:ex:box:fact} A rectangle with dimensions \(19\text{cm}\) by \(7\text{cm}\) has area \(133\text{cm}^{2}\).
    \end{enumerate}
  \end{illustration}
  \EAS{} holds that, when I claim support for~\ref{EAS:ex:box:fact} from~\ref{EAS:ex:box:spec}, I may appeal to dimensions and formula, though as I do not witness the ability, I do not use the premise and step.
  For, if~\ref{EAS:ex:box:spec} is the case then it is possible for me to witness reasoning in which I demonstrate that~\ref{EAS:ex:box:fact} is the case, and it is the premises and steps of reasoning used in such reasoning that establishes~\ref{EAS:ex:box:fact} is the case.
  I have not used those steps and premises, as I have not witnessed the relevant ability, but may I appeal to those steps and premises regardless --- or so we will argue.
\end{note}

\begin{note}[More detail]
  \color{details}
  I do not expect \EAS{} to be intuitive.
  Indeed, we are not interested in \EAS{} because it is a more-or-less intuitive principle which conflicts the intuitive \ESU{}.
  Rather, we are interested in \EAS{} primarily because \EAS{} is a consequence of tension arising from three things:

  \begin{enumerate}
  \item\label{incomp:tri:q:1} \ESU{}
  \item\label{incomp:tri:q:2} scenarios involving an agent reasoning with information about an their own ability,
  \item\label{incomp:tri:q:3} and a principle concerning when an agent is permitted to claim support
  \end{enumerate}

  To briefly expand on~\ref{incomp:tri:q:2} and~\ref{incomp:tri:q:3}:

  Information that one has some specific ability so long as one has some general ability --- such as the (specific) ability to show that \(25^{\circ}\text{C} = 77^{\circ}\text{F}\) given the (general) ability to convert between Celsius and Fahrenheit.
  And, an agent is never permitted to claim support for proposition having a certain value if the agent requires the proposition to have value \emph{in order to} claim support.
  (As an instance, an agent is not permitted to claim support for the truth of a proposition if the agent requires the proposition to be true \emph{in order to} claim support that the proposition is true.)\nolinebreak
  \footnote{
    The emphasis on `in order to' is important.
    The instance of the principle does not state that an agent is not permitted to claim support for the truth of a proposition if the agent requires the proposition to be true when claiming support that the proposition is true.
    I plausibly require that \(2 + 2 = 4\) when I claim support that \(2 + 2 = 4\), and this does not prevent me from claiming support by simple arithmetic.
    However, it would be impermissible (or so we will argue) to claim support that \(2 + 2 = 4\) by reasoning that the calculator is functional only if \(2 + 2 = 4\), and as the calculator states that \(2 + 2 = 4\) it is the case that \(2 + 2 = 4\).
  }
  The details matter, and we postpone detailing this argument to~\autoref{sec:broad-argum-overv}.

  In short, assuming the scenarios exist, there is tension between intuitive principles governing what an agent appeals to when reasoning and structural principles governing the relation between what the agent appeals to when reasoning.
\end{note}

\begin{note}
  For the moment we attempt to clarify \EAS{} to some degree.
  Three subsections follow:

  \begin{enumerate}
  \item We will outline alternative reasoning patterns from~\ref{EAS:ex:box:if} to~\ref{EAS:ex:box:fact}, clarify why we focus on a particular type of reasoning pattern, and examine some initial objections to~\EAS{} and canvas some responses.
  \item We will consider parallels between abilities and dispositions.
    The parallel will provide some additional intuition for why an agent may appeal to premises and steps that have no been used, and help further clarify our interest with ability.
  \item We will consider a related proposition argued for by \citeauthor{Moretti:2019wx} which holds that a belief need not be based (exclusively) on the premises and steps of reasoning used to arrive at the belief.
    The comparison will help highlight what is distinctive about~\EAS{} while at the same to introducing some ideas which suggest a way of understanding~\EAS{}.
  \end{enumerate}
\end{note}

\paragraph{Against \EAS{}}

\begin{note}[Alternatives]
  The alternative reasoning pattern we will focus on in some detail holds that appealing to having the ability noted in \ref{EAS:ex:box:spec} is sufficient to claim support for \ref{EAS:ex:box:fact}.
  In line with \ESU{}, the agent would use the proposition that they have the relevant ability noted in~\ref{EAS:ex:box:spec} to claim support for~\ref{EAS:ex:box:fact}
  This reasoning pattern, along with the pattern suggested by \EAS{} will be considered in \autoref{sec:wr-ar} and we will argue that it conflicts with an intuitive principle regarding claiming support in \ref{sec:second-conditional}.

  Alternatively, on may argue that though the syntactic form of \ref{EAS:ex:box:if} is a conditional, it does not (necessarily) follow that the semantic content of~\ref{EAS:ex:box:if} is a (also) conditional.
  And that~\ref{EAS:ex:box:if} may (plausibly) be interpreted to explicitly state that~\ref{EAS:ex:box:spec} is an ability that an agent may have.
  For example:
  \begin{enumerate}[label=\emph{A}\arabic*., ref=(\emph{A}\arabic*), resume*=EAS_counter]
  \item\label{EAS:ex:box:if:R:state} The ability to demonstrate that a rectangle with dimensions \(19\text{cm}\) by \(7\text{cm}\) has area \(133\text{cm}^{2}\) is an ability an agent may have and it is an ability an agent has if they have ability to calculate the area of a rectangle.
  \end{enumerate}
  Hence, \ref{EAS:ex:box:if} is interpreted so that \ref{EAS:ex:box:spec} is accessible without endorsing the antecedent of \ref{EAS:ex:box:if}.
  \ref{EAS:ex:box:if:R:state} states that there is some ability that it is possible for an agent to have, and in addition provides sufficient conditions for having the relevant ability.
  The important part of \ref{EAS:ex:box:if:R:state} is the former conjunct:
  \begin{enumerate}[label=\emph{A}\arabic*., ref=(\emph{A}\arabic*), resume*=EAS_counter]
  \item\label{EAS:ex:box:spec:R:state} The ability to demonstrate that a rectangle with dimensions \(19\text{cm}\) by \(7\text{cm}\) has area \(133\text{cm}^{2}\) is an ability an agent may have.
  \end{enumerate}
  And, \ref{EAS:ex:box:fact} follows from \ref{EAS:ex:box:spec:R:state} by the observation that it is not possible to demonstrate that \(\phi\) if \(\phi\) is not the case --- there is no need for the agent to appeal to witnessing their ability.
  Therefore,~\ref{EAS:ex:box:spec:R:state} (and~\ref{EAS:ex:box:if:R:state}) implicitly includes information that~\ref{EAS:ex:box:fact} is the case --- an agent does not need to reason from~\ref{EAS:ex:box:spec} to~\ref{EAS:ex:box:fact}, because they have already been informed that~\ref{EAS:ex:box:fact} is the case.

  Note also that analogous reasoning applies if `I' is replaced with `an agent'.
  Likewise, if I know that you that you know that I have the ability to calculate the area of a rectangle.
  For, it then follows that you know that \ref{EAS:ex:box:spec} is the case, and therefore you know that \ref{EAS:ex:box:fact} is the case.
  Again there is no need for me to appeal to witnessing an ability.
\end{note}

\begin{note}[Box]
  The existence of alternative reasoning patterns is the issue at hand.
  For, so long as there are reasoning patterns \emph{R} which conform to \ESU{} it is open to the defender of \ESU{} to hold that if an agent is permitted to claim support, then the agent is required to reason via some member of \emph{R}.
  For, if there are reasoning patterns \emph{R} which conform to \ESU{} then there a no counterexamples to \ESU{} --- scenarios in which an agent claims support by appeal to premises or steps of reasoning that the agent has not used.

  Of course, an argument against a general principle such as \ESU{} is not required to be a counterexample.
  For example, it may be possible to argue that the reasoning patterns \ESU{} requires are sufficiently implausible.
  Hence, a restricted variant of \ESU{} compatible with \EAS{} would to be preferred.
  However, there are two issues with attempting such an argument.

  First, given the intuitive plausibility to \ESU{}, it seems unlikely that any violation of \ESU{} would be more plausible than an alternative reasoning pattern compatible with \ESU{}.
  Second, even if there are plausible reasoning pattern that are incompatible with \ESU{}, it is not clear that these should be incorporated in a theory of claiming support.
  For, meta-theoretical issues such as complexity or predictive power may still favour \ESU{}.
  Following \citeauthor{Box:1987vn}: `\dots all models are wrong; the practical question is how wrong do they have to be to not be useful.' (\Citeyear[74]{Box:1987vn})

  Indeed, the second point suggest a counterexample proper to show that \ESU{} is false is not necessarily an adequate argument against \ESU{} either.
  Observations in the spirit of \citeauthor{Box:1987vn} are trite, but also true.
  Even if \EAS{} is true and \ESU{} is false, would \EAS{} be useful?
\end{note}

\begin{note}[Responding to Box]
  With respect to idealised agents with unbounded resources, the answer appears to be no.
  For, with unbounded resources the agent the option of (attempting to) witnessing any ability (to reason) without cost.
  And, it seems that for such an agent witnessing a relevant ability would always be preferable to reasoning about an unwitnessed ability as the agent would minimally (subjectively) resolve any uncertainty about whether they have the ability.

  However, for limited agents, ability is abundant, while the resources required to witness abilities are scarce.
  That the exception to~\ESU{} is narrow does not entail that there are few occurrences of the exception.

  Information about ability may be abundant while the resources for witnessing abilities are either scarce or temporarily unavailable.
  So, for example, agent has the option of conserving or deferring use of resources.

  This observation suggests an initial line of response to an objection which focuses on whether \EAS{} would be useful.

  For, given that we are resource bound agents, it seems that possible instances of \EAS{} are widespread.
  From a functional perspective, reasoning with (the relevant instances of) ability just is reasoning about the result of expending available resources.
  Hence, if~\EAS{} is true, then the truth of \EAS{} would provide a novel perspective on resource bound agents.
  And, it is yet to be seen whether such a perspective is useful.

  In addition, there is a second indirect line of response.
  We observed above that \ESU{} seems prevalent in various theories which relate to reasoning, such as basing and responding to reasons.
  If \EAS{} is true, then there may be alternative conclusions to arguments that appeal to~\ESU{} as a premise.
  And, likewise, there may be interesting observations made in premises of arguments which establish \ESU{} as a foundation for further theorising.\nolinebreak
  \footnote{
    As an exception, even if~\EAS{}, conclusion of arguments which appeal to or assume \ESU{} may be restricted.
  }

  Taking stock:
  I doubt that \EAS{} is of interest if there are no reasoning patterns which require \EAS{} to be true.
  Still, if there are reasoning patterns which require \EAS{} to be true, then \EAS{} may be of interest.
  Further, I think there are good reasons to hold that there are reasoning patterns which require \EAS{} to be true.
  Hence, my goal is to motivate further research into whether \EAS{} is of interest.

  We now briefly turn to the type of scenarios which are a premise in our argument for the existence of such counterexamples.
\end{note}

\begin{note}[Types of scenario]
  The type of scenario we will focus on is designed to ensure that an agent is required to reason to (and from) information about a (specific) ability that they have.
  If the agent is required to reason \emph{to} (specific) ability information then rephrasing \ref{EAS:ex:box:if} as \ref{EAS:ex:box:if:R:state} will not be possible --- the agent will be required to reason from some premises by some steps to the (specific) ability information.
  And, as before, the type of scenario will preserve the requirement of the agent to reason \emph{from} the (specific) ability information in line with~\ref{EAS:ex:box:spec} and~\ref{EAS:ex:box:if:R:state}.
  Hence, by establishing such scenarios are possible we may restrict our attention to the steps from~\ref{EAS:ex:box:if} to~\ref{EAS:ex:box:spec} and from~\ref{EAS:ex:box:spec} to~\ref{EAS:ex:box:fact}.
\end{note}

\begin{note}
  To illustrate, let us add some context to the example scenario we've been focusing on.

  Suppose it is common knowledge between you and I that
  \begin{enumerate*}
  \item you have looked through my notes, and have applied my formula for calculating the area of a rectangle, and
  \item my notes are the only source of information you have regarding how to calculate the area of a rectangle.
  \end{enumerate*}
  We may now restate the semantic content of~\ref{EAS:ex:box:if} as follows:
  \begin{enumerate}[label=\emph{A}\arabic*., ref=(\emph{A}\arabic*), resume*=EAS_counter]
  \item\label{EAS:ex:box:inf:R} You have some general ability \(\gamma\), and a specific ability \(\varsigma\) (as an instance of that general ability).
    And, if \(\gamma\) is the ability to calculate the area of a rectangle, then \(\varsigma\) is the ability to demonstrate that a rectangle with dimensions \(19\text{cm}\) by \(7\text{cm}\) has area \(133\text{cm}^{2}\).
  \end{enumerate}
  The formula in my notes indicates that I have the ability to do something, and I have indicated what I think the appropriate characterisation of the ability is.
  Still, you are not in a position to offer information as to whether my characterisation of the ability is correct or not.\nolinebreak
  \footnote{
    Consider in reverse.
    One is often attributed abilities that I deny I have.
    For example, I do not have the ability to process information by means of mental images, but \citeauthor{Hume:2011aa} (arguably) holds that I do have such an ability.
    I lack the ability to reason in a particular way.
    (Not that \citeauthor{Hume:2011aa}'s arguments rest on visual as opposed to any other kind of imagination, but the point stands.)

    Similarly, you may claim that I have the ability to tell you whether or not \nagent{7} is coming to tea.
    However, and in contrast to your assumption, \nagent{7} has not replied to my invitation and so I lack a required premise in order to reason to a relevant conclusion.
  }
  The key feature of~\ref{EAS:ex:box:inf:R} it that I, the agent, am required to claim support that \(\gamma\) and \(\varsigma\) are the abilities of interest.
  The focus is not on whether or not an agent may perform some action.
  Rather, our interest is with what the action is.
  It is up to me, the agent, to claim support that:
  \begin{enumerate}[label=\emph{A}\arabic*., ref=(\emph{A}\arabic*), resume*=EAS_counter]
  \item\label{EAS:ex:box:gen:R} The general ability \(\gamma\) \emph{is} the ability to calculate the area of a rectangle.
  \end{enumerate}
  If so, I may then claim support that:
  \begin{enumerate}[label=\emph{A}\arabic*., ref=(\emph{A}\arabic*), resume*=EAS_counter]
  \item\label{EAS:ex:box:spec:R} \(\varsigma\) is the specific ability to demonstrate that a rectangle with dimensions \(19\text{cm}\) by \(7\text{cm}\) has area \(133\text{cm}^{2}\).
  \end{enumerate}
  Note, there is no route to \ref{EAS:ex:box:spec:R} other than by claiming support for~\ref{EAS:ex:box:gen:R} as I have no information about what the (specific) ability \(\varsigma\) is amounts to if \ref{EAS:ex:box:gen:R} is not the case.
  If \(\varsigma\) is some other ability, then~\ref{EAS:ex:box:fact} does not follow, witnessing the relevant ability would not demonstrate that a rectangle with
  dimensions \(19\text{cm}\) by \(7\text{cm}\) has area \(133\text{cm}^{2}\), and so a rectangle with dimensions \(19\text{cm}\) by \(7\text{cm}\) may (from my epistemic perspective) have some other area.
\end{note}

\begin{note}[Point]
  We will say more in \autoref{sec:cases-interest}.
  For the moment it is sufficient to observe that the agent is required to reason to and from specific ability.
  The scenario requires the agent to claim support by reasoning from~\ref{EAS:ex:box:if} to~\ref{EAS:ex:box:spec} and from~\ref{EAS:ex:box:spec}~to~\ref{EAS:ex:box:fact}.
  And, while the context added to force the reasoning pattern of interest is narrow, the principle behind the context is simple:
  The agent is required to claim support that they have the relevant general ability.
  Hence, any scenario which consists of \gsi{-} which requires the agent to claim support that they have the relevant general ability will require the same kind of reasoning pattern.

  Further, this arguably captures a general puzzle about ability.
  An agent is not required to have witnessed all instances of a general ability to claim support that they have the general ability.
  However, so long as the agent may claim support for having some general ability, then it follows that the agent will have the option of claiming support for each instance of the general ability.

  The primary issue, though, is whether there is an account of such reasoning that does not require \EAS{} to be true.
  We will shortly turn to this argument in \autoref{sec:broad-argum-overv}.
  Prior to doing so, we close this section with some further clarification on the motivation behind \EAS{}, what distinguishes \EAS{} from nearby principles, and a suggestion on how to conceptualise \EAS{}.
\end{note}

\paragraph{Ability and dispositions}

\begin{note}[Parallel]
  To further clarify the motivation for \EAS{} we introduce a parallel between abilities and dispositions.
  The primary function of the parallel will be to highlight the importance of reasoning about an event.
  In the case of dispositions the event is the manifestation of the disposition, and in the case of ability the event is the agent witnessing the ability.

  The parallel is of interest because \EAS{} concerns the premises and steps of reasoning that the agent would use to witness the relevant ability.
  We will suggest that claiming support that some object has some disposition and that some agent has some ability may both be understood in terms of claiming support that the relevant event is a possible event.

  In turn, if reasoning \emph{to} a specific ability is understood in terms of claiming support that it is possible for the agent to witness the event, then reasoning \emph{from} a specific ability may be understood in terms of claiming support from what would happen in the possible event.
  \end{note}

\begin{note}[Parallel between dispositions and ability]
  Consider \citeauthor{Choi:2021wg}'s characterisation of the Simple Conditional Analysis of dispositions:
  \begin{quote}
    An object is disposed to \emph{M} when \emph{C} iff it would \emph{M} if it were the case that \emph{C}.\nolinebreak
    \mbox{}\hfill\mbox{(\Citeyear{Choi:2021wg})}
  \end{quote}
  For example, an object is disposed to dissolve when it is placed in water iff the object would dissolve if it were the case that it is placed in water.

  The Simple Conditional Analysis may be challenged, but for our purposes it is adequate.
  We are interested in the broad form of the truth condition, and various more refined analyses share the same broad form.
  Note, in particular, that it being the case that \emph{C} and \emph{M} happening describes an event.
  Given appropriate conditions; salt dissolves, glass breaks, and I mumble when I am tired.
  The key idea is that the property of being disposed to \emph{M} when \emph{C} is analysed in terms of the (possible) event of \emph{M} happening when \emph{C}.

  The parallel to ability is established by noting that ability may also be analysed in terms of a (possible) event, as we have seen.
  In particular, by incorporating volition in the analysans of the Simple Conditional Analysis.
  To illustrate, \citeauthor{Mandelkern:2017aa} trace the Conditional Analysis of ability  to \textcite{Hume:1748tp} and \textcite{Moore:1912te}, among others:
  \begin{quote}
    S can \(\phi\) iff S would \(\phi\) if S tried to \(\phi\)\nolinebreak
    \mbox{}\hfill\mbox{(\Citeyear[Cf.][308]{Mandelkern:2017aa})}
  \end{quote}
  Compare to the Simple Conditional Analysis of dispositions:
  The object is some agent \emph{S}, \emph{C} is `S tried to \(\phi\)' and \emph{M} is `S \(\phi\)s' --- it is volition alone which distinguishes the analyses.
  For example, I have the ability to demonstrate that a rectangle with dimensions \(19\text{cm}\) by \(7\text{cm}\) has area \(133\text{cm}^{2}\) only if I would demonstrate that a rectangle with dimensions \(19\text{cm}\) by \(7\text{cm}\) has area \(133\text{cm}^{2}\) if it were the case that I tried that a rectangle with dimensions \(19\text{cm}\) by \(7\text{cm}\) has area \(133\text{cm}^{2}\).
\end{note}

\begin{note}[Claiming support]
  Parallel analyses in hand, we now turn to claiming support.
  We start with dispositions.

  As with ability, there are various ways in which an agent may claim support that some object is disposed to \emph{M} when \emph{C}.
  For example, I may claim support that my shoes are disposed to squeak when wet because I have had sufficient occasion to observe the phenomenon.
  Likewise, I may claim support that any shoe of the same model is disposed to squeak when wet because I have traced the source of the squeak to a manufacturing choice.
  In short, support may be claimed by past event and shared properties.

  Still, take a novel act and a object pair.
  Personally, I have a empty fountain pen that I haven't placed in water.
  I claim that the fountain pen is disposed to float when placed in water.
  My reasoning is fairly simple.
  The fountain pen is quite light, especially so while empty of ink.
  And, the cap and loading mechanism seem to be quite well sealed, so the weight of the fountain pen will not increase by taking on water.
  So, given that the weight of the fountain pen will be unchanged, and given how light the pen is, it seems that the upward force exerted by the water against the fountain pen will be sufficient to keep the pen afloat.

  In short, I've noted a few properties of the pen, claimed support for a handful of others, and then considered what would happen.
  Our interest is with the last step.
  I appeal to, and use, the possible event.\nolinebreak
  \footnote{
    I may be wrong about the event, but that isn't at issue.
    It remains the case that I appeal to it.
  }
  The noted properties are relevant because they suggest that the event of floating would happen if it were the case that the fountain pen were placed in water.
\end{note}

\begin{note}
  The fountain pen is not the only object on my desk.
  Beside the fountain pen is a collection of instruments that I may use to investigate the fountain pen.
  And, stored in my mind is a basic understanding of fluid dynamics.

  If I were to measure the fountain pen, ensure that it is airtight, and appeal to some known facts, then an application of Archimedes' principle would allow me to demonstrate that the fountain pen is disposed to float when placed in water (of some specified density).
  Indeed, such a demonstration would be a straightforward refinement of the way in which I claimed support for the proposition that the pen is disposed to float when placed in water.

  Now, by similar reasoning I have claimed support for the proposition that I have the ability to demonstrate that the proposition that the pen is disposed to float when placed in water is true.
  Here, in addition to appealing to properties of the fountain pen, I also appealed to various mental properties.
  There is an important difference, however, regarding the relevant event.
  When reasoning about the disposition, the event is the fountain pen floating in water, but when reasoning about my ability to demonstrate the event is the demonstration --- a series of measurements and calculations.
\end{note}

\begin{note}[Diverge]
  Now to turn to \EAS{}.

  If I have the ability, then it follows that the fountain floats in water.
  As noted above, it is not possible for me to demonstrate something that is not the case.

  Claim support for the proposition that the fountain floats in water.

  Still, disposition, fountain pen is not floating in water.
  Likewise with respect to ability, I have not demonstrated that the fountain pen floats in water.
  I noted various things, but did not piece these together into a demonstration.

  Yet, in claiming support, there's the event of demonstrating.
  And, so I appeal to those premises and steps I would use in the event.
  This is \EAS{}.

  Appeal to what happens in the event.
  And, reasoning to claim possible event is viewed in terms of ensuring that the resources are available.
  I have not used the relevant premises and steps of reasoning, nor am I clear on the specific form they will take.
  Still, they are available.

  Final point of interest, then.
  In both cases, there's an appeal to an event.
  If \EAS{} holds with respect to ability, does something similar hold with respect to dispositions?

  First, important clarification.
  The reasoning outlined for disposition was claiming support for event.
  Here, no clear issue with \ESU{}.
  Similarly, no clear issue with \ESU{} with respect to claiming support for having an ability.
  Tension with \ESU{} arises when using ability as a premise in further reasoning.

  Second, key divergence.
  Conclusion obtained is something that is true independent of ability.
  Unclear to me whether similar reasoning with dispositions.
  For, ability is about an event involving the agent.

  In addition, there is no issue with supposing that the agent reasons with (and hence uses) to all the relevant features of the event.\nolinebreak
  \footnote{There may me details of reasoning that one is not easily able to express, but it doesn't follow that those details are not used.}
  Ability is in part interesting because it is clear that an agent does not witness the relevant event.
  This is not to say that a variant of \EAS{} does not hold with respect to dispositions.
  Rather, I am expressing
  \begin{enumerate*}
  \item hesitancy that there are comparable entailments, and
  \item concern that there is no clear argumentative path.
  \end{enumerate*}

  There is a related question about the ability of other agents.
  Here, \EAS{} does not entail.
  In turn, one may conjecture that reasoning from one's own ability is similar.
  I find this plausible.
  It is important to stress again that \EAS{} expresses a way in which an agent may claim support.
  Hence, \EAS{} is compatible with there being other ways in which an agent may claim support.
  It may be the case that the same holds with respect to other agents.\nolinebreak
  \footnote{
    For example, \citeauthor{Owens:2006tw} argues for a belief expression model of assertion in which the rationality of a belief formed by an agent on the basis of testimony depends whatever justification the speaker has for the relevant propositional content.
    \begin{quote}
      Trusting an expression of belief by accepting what a speaker says involves entering a state of mind which gets its rationality from the rationality of the belief expressed. This state's rationality depends on the speaker's justification for the belief he expresses, not on his justification for the action of expressing it. And to hear a speaker as making a sincere assertion, as expressing a belief, is \emph{ceteris paribus} to feel able to tap into \emph{that} justification (whether or not his assertion was directed at you) by accepting what he says.\nolinebreak
      \mbox{}\hfill\mbox{(\Citeyear[123]{Owens:2006tw})}
    \end{quote}
    \color{red} Some more
  }
  However, this is not an immediate consequence.
  \EAS{} permits exceptions to \ESU{}, but it does not require all instances of reasoning with ability is an exception to \ESU{}.
  And, our focus will be on cases in which an agent reasons about their own ability to reason.
  The weak quantifier `there are cases' is designed to leave such issues open.
\end{note}

\begin{note}[Concluding parallel]
  To summarise.
  \begin{itemize}
  \item Parallel between analysis of dispositions and abilities.
  \item Event in analysis of both.
  \item Reason about event.
  \item Motivation for \EAS{} by considering reason to and from event.
  \item This doesn't provide anything close to a clear theoretical account of the reasoning performed if \EAS{} is true, but it does hint at such at how developing such an account may be approached.
  \item Now turn to related conclusion.
  \item In turn, fill in some details on the account.
  \end{itemize}
\end{note}

\paragraph{Enthymematic inferences}

\begin{note}[\citeauthor{Moretti:2019wx}]
  Above we considered how various account of the basing relation seem to imply \ESU{}.
  Roughly, because such accounts of the basing relation required a premise or step of reasoning to be used in order to be a candidate member of the base of some conclusion of reasoning --- motivated by either causal and representational considerations.
  In contrast, \citeauthor{Moretti:2019wx} argue for an account of the basing relation which does not entail \ESU{}.

  In our terminology, \citeauthor{Moretti:2019wx} argue that: A belief held by an agent may be \emph{based} on premises that the agent did not use when forming the belief.

  The following is a fragment of the general principle relating propositional justification to well-grounded belief (alternatively doxastistcally justified belief) containing the two clauses of interest:

  \begin{quote}
    IF

    \dots

    OR

    \begin{enumerate*}[label=(\arabic*.2\(^{\ast}\))]
    \item\label{LT:1.2} Q is propositionally justified for S in virtue of P1, P2, \(\dots\), Pn being justifiedly true from her perspective because S justifiedly believes P1, P2, \(\dots\), Pn, and in virtue of her being aware that Q is an inductive or deductive consequence of P1, P2, \(\dots\), Pn jointly, and
    \item\label{LT:2.2} S carries out a \emph{plain} inference from P1, P2, \(\dots\), Pn to Q.
    \end{enumerate*}

    OR

    \begin{enumerate*}[label=(\arabic*.3), ref=(\arabic*.3)]
    \item\label{LT:1.3} Q is propositionally justified for S in virtue of P1, P2, \(\dots\), Pn being justifiedly true from her perspective, though S doesn't believe at least some P1, P2, \(\dots\), Pn, and in virtue of S being aware that Q is an inductive or deductive consequence of P1, P2, \(\dots\), Pn jointly, and
    \item\label{LT:2.3} S carries out a (fully or partly) \emph{enthymematic inference} from P1, P2, \(\dots\), Pn to Q.
    \end{enumerate*}

    THEN
    \begin{enumerate}[label=(3)]
    \item S's belief that Q is well-grounded.\nolinebreak
      \mbox{}\hfill\mbox{(\Citeyear[87]{Moretti:2019wx})}
    \end{enumerate}
  \end{quote}

  The `plain' inference of~\ref{LT:1.2} and~\ref{LT:2.2} corresponds to cases in which an agent uses P1, P2, \(\dots\), Pn to reason to Q.
  By contrast, the `enthymematic' inference of~\ref{LT:1.3} and~\ref{LT:2.3} involves reasoning in which an agent does not use some or all of P1, P2, \(\dots\), Pn to reason to Q as the agent does not believe some of P1, P2, \(\dots\), Pn (though the agent has propositional support for each of P1, P2, \(\dots\), Pn).

  To illustrate the distinction between `plain' and `enthymematic' inferences (\Citeyear[Cf.][85]{Moretti:2019wx}) consider reasoning from the premise that \nagent{5} is shorter than \nagent{6} to the conclusion that someone is taller than \nagent{5}.
  An instance of plain (non-enthymematic) may take the intermediary step that \nagent{6} is taller than \nagent{5} before abstracting from \nagent{6}.
  In contrast, an instance of enthymematic reasoning consists of the (single) premise and conclusion noted without forming the belief that \nagent{6} is taller than \nagent{5}.\nolinebreak
  \footnote{Cf.\ (\Citeyear[87--89]{Moretti:2019wx}) for examples given by \citeauthor{Moretti:2019wx}.}

  The key idea is that if an agent reasons enthymematically, then the agent's belief may be based on those premises that the agent would use in the corresponding plain inference.
  (\Citeyear[Cf.][86--87]{Moretti:2019wx})
  Hence, we have a proposal on which an agent's belief may be supported by premises and steps of reasoning that an agent has not used.
  And, in addition, because S carries out a (fully or partly) enthymematic inference \ref{LT:2.3}, it seems S \emph{may} appeal to P1, P2, \(\dots\), Pn when reasoning to Q, in conflict with \ESU{}.

  Whether or not \citeauthor{Moretti:2019wx}'s account is correct is not of interest.
  Rather, \emph{grating} that \citeauthor{Moretti:2019wx}'s account is correct allows us to make two (related) observations.
  First, \citeauthor{Moretti:2019wx} account does not conflict with \ESU{} and so the account does not require \EAS{} to be true.
  And, second, how \citeauthor{Moretti:2019wx}'s account suggests a broader theoretical account of \EAS{}.
\end{note}

\begin{note}[First point]
  To establish the first point we require further details about how \citeauthor{Moretti:2019wx} define a (fully or partly) enthymematic inference.
  The following quote combines the relevant definitions:
  \begin{quote}
    \textbf{(}[\textbf{Partly}/\textbf{Fully}] \textbf{Enthymematic Inference)}

    S carries out a [\emph{partly}/\emph{fully}] \emph{enthymematic} inference from P1, P2, \(\dots\), Pn to Q if and only if
    \begin{enumerate}[label=(\alph*), ref=(\alph*)]
       \setcounter{enumi}{1}
    \item \emph{S doesn't actually believe} [\emph{at least some of the premises}/\emph{any of}] P1, P2, \(\dots\), Pn, though some constituents M1, M2, \(\dots\), Mm of S's perspective cause in S the \emph{disposition} to believe P1, P2, \(\dots\), Pn, and
    \item M1, M2, \(\dots\), Mm [together with the premises believed by S jointly/jointly] cause S's belief that Q through a process that is shaped by S's taking Q to be a consequence of P1, P2, \(\dots\), Pn at a personal level.\nolinebreak
      \mbox{}\hfill\mbox{(\Citeyear[85]{Moretti:2019wx})}
    \end{enumerate}
  \end{quote}

  In short, an enthymematic inference involves reasoning with premises M1, M2, \(\dots\), Mm which are related to the premises P1, P2, \(\dots\), Pn of some corresponding plain inference.
  In order to complete the definition, we require an account of what it is for S to take Q to be a consequence of P1, P2, \(\dots\), Pn at a personal level:

  \begin{quote}
    \textbf{(Personal Level\(^{\ast}\))}

    S's mental states M1, M2, \(\dots\), Mm and any premises believed by S, among P1, P2, \(\dots\), Pn, jointly cause S's belief that Q through a process shaped by S's taking Q to be a consequence of P1, P2, \(\dots\), Pn at a personal level if and only if M1, M2, \(\dots\), Mm and any premise believed by S, among P1, P2, \(\dots\), Pn, jointly cause S to believe Q and S would adduce the reasons that P1, P2, \(\dots\), Pn and that Q is a consequence of P1, P2, \(\dots\), Pn in response to a request to explain why she believes Q.\nolinebreak
    \mbox{}\hfill\mbox{(\Citeyear[85--86]{Moretti:2019wx})}
  \end{quote}

  So, loosely reconstructed an enthymematic inference involves constituents M1, M2, \(\dots\), Mm of S's perspective which ensure that S has the disposition to believe P1, P2, \(\dots\), Pn.
  And, the way in which M1, M2, \(\dots\), Mm lead to S forming the belief that Q allow S to explain that they believe Q on the basis of P1, P2, \(\dots\), Pn.
  In short, an enthymematic inference is an inference in which may be \emph{post hoc} expanded to some corresponding plain inference (in part) because performing the enthymematic inference requires the agent to be disposed to believe the required premises of the corresponding plain inference.
  And, as such the premises of the corresponding plain inference may be considered as (constitutive of) the basis of S's belief that Q.

  In contrast, \ESU{} concerns the way in which M1, M2, \(\dots\), Mm lead to S forming the belief that Q do not necessarily require the agent to appeal to P1, P2, \(\dots\), Pn.
  It is consistent with \citeauthor{Moretti:2019wx} account that the reasoning from M1, M2, \(\dots\), Mm to Q may only appeal to premises and steps of reasoning used.
  That Q may be based on P1, P2, \(\dots\), Pn is due to the requirement that S is disposed to believe P1, P2, \(\dots\), Pn and the possibility of S retroactively appealing to Q being a consequence of P1, P2, \(\dots\), Pn.
  Hence, the account does not conflict with \ESU{}, and in turn does not require \EAS{} to be true.

  The insight offered is that there does not necessarily need to be a structure preserving mapping between premises and steps providing propositional support for a belief and the premises and steps appealed to when forming the belief.
  However, this does not constrain what the agent appeals to when forming a belief.

  From a broader perspective, \citeauthor{Moretti:2019wx}'s proposal considers what an agent was able to do (i.e.\ reason by some plain inference) and holds that a basing relation follows but is silent of the way in which an agent claim support.
  In contrast, \EAS{} looks at what an agent is able to do, and holds that a way of claiming support follows, but is silent on issues concerning the basing relation.
\end{note}

\begin{note}[Second point]
  Still, this broader perspective together with the above discussion of dispositions suggests a way to understand \EAS{}.
  For, one may hold that if an agent has the ability to reason to some conclusion, then the agent is disposed to use relevant premises and steps of reasoning to reason to the conclusion.
  In parallel to \citeauthor{Moretti:2019wx}, then, one may hold that the agent has the ability to reason to some conclusion if (and only if) they are suitably related to some collection of relevant premises and steps of reasoning.
  In turn, the agent may appeal to those premises and steps of reasoning to claim support for the conclusion.
  Indeed, if we adopt a parallel understanding of the basing relation, then it follows (so long as the agent has the ability) that  the agent has sufficient propositional support for the conclusion, and may be well-grounded.
  The (possible) event of reasoning to the conclusion is important both for establishing that the agent has the ability and for determining which premises and steps of reasoning the agent appeals to, but the event is not important for determining that the relevant premises and steps of reasoning are available to the agent.

  This suggestion falls far short of a theory satisfying \EAS{}, I suspect \EAS{} may be motivated in part by distinguishing between what occurs in the event of reasoning, and sufficient resources required for such an event to occur.
  An event of reasoning will always make use of sufficient resources for the event to occur, but an agent may have sufficient resources for the event to occur even if the event does not occur.
  (Specific) abilities, then, fix an particular event and determine sufficient resources and the agent does not need to witness the event in order to appeal to those resources.

  Or perhaps not.
\end{note}

\begin{note}[Segue]
  Our goal is to establish that an adequate account of reasoning which extends to ability must satisfy \EAS{}.
  This goal does not require the above suggestion to be on the right track, nor does this goal require that there is a unique theory that satisfies \EAS{}.
  For now, we close the present section with a few remarks concerning ability and~\EAS{}.
\end{note}

\begin{note}[Actual support]
  As with~\ESU{}, \EAS{} does not entail that the agent \emph{has} support.
  Our focus is on reasoning, and as argued above, it seems the issue of whether an agent has support is distinct from whether an agent may claim support.
  Claiming support is the result of some reasoning, and whether or not an agent has support requires an evaluation of that reasoning.
  This means that, strictly speaking,~\EAS{} does not carry any implications regarding whether or not the agent has support by claiming support in line with~\EAS{}.
  It is possible that the agent would fail to establish support, or establishes a support relation other than between the conclusion and the premises and steps of reasoning appealed to.

  Still, it take it to be plausible that support traces a successful claim.
  From this perspective,~\EAS{} may seem a little more intuitive.
  Given an intuitive understanding of support, if an agent does have the ability to reason to some conclusion, then the conclusion stands in the relation of being support by certain premises and steps of reasoning, whether or not the agent witnesses their ability.
  In turn, if the agent may claim support for the having the relevant ability then the agent may claim support for the conclusion from the premises and steps that would be used to witness their ability.
  For, witnessing does not contribute to the relation of support between the conclusion and the relevant premises and steps --- witnessing would only clarify to the agent the specifics of the relation.

  Of course, the agent may be mistaken or misled about having ability.
  For example, the relevant premises and steps may fail to establish the conclusion, or the agent may not have sufficient resources to carry out reasoning from the premises and steps, etc.
  In turn, witnessing may be expected to highlight that the claimed support for having the ability is mistaken or misled.

  Two points:
  \begin{itemize}
  \item Such issues are not different to being mistaken or misled and using that one has the ability as a premise, so apply to any reasoning that makes use of ability without witnessing ability.
  \item Attempting to witness the ability might reveal that the agent is mistaken or misled about having the ability does not show that the agent may not claim support for having the ability.

    Reasoning typically involves premises and steps of reasoning that could be investigated further, but this does not prevent an agent from appealing to those steps and premises.
  For example, it is (almost) to check the definition of any word used against a dictionary, and doing so might reveal that I have been mistaken or mislead about the meaning that I will convey by using the word.
  I rarely do this, though.
  Most of the time it is sufficient to expect that I am not mistaken or have not been mislead about the meaning I would convey by using the word.
  \end{itemize}
\end{note}

\begin{note}[Desire]
  Finally, while the examples of reasoning given have concluded with the truth of some proposition --- that a rectangle has some specific area, or that a given fountain pen floats in water, etc.\ --- our interest with \EAS{} is broader.
  Recall, in~\autoref{prop:RisTV} we stated that for the purposes of this paper we consider the conclusion of reasoning as assigning some value to some proposition.
  In many cases the assigned value truth, falsity, or something in between.
  However, claiming support, and in turn; \USE{}, \ESU{}, and \EAS{} are all neutral with respect to the value assigned to the proposition.
  Therefore, we may consider other values while investigating, and as an application of \ESU{} and \EAS{}.
  In particular, consider reasoning which concludes with the desirability of some proposition.\nolinebreak
  \footnote{
    \color{red}
    Mistaken or misled.
    Yes, I think this holds up.

    Strong view on which an agent may be mistaken about desires in the same way as an agent may be mistaken about evidence.
    View on which desires are independent of representation.
    Hence, misleading or mistaken support when an agent fails to represent desire.
  }

  To illustrate this point, consider temptation.\nolinebreak
  \footnote{
    \color{red}
    Whether or not this is `genuine' temptation isn't of \dots
  }
  Specifically we will consider a slight variation on \citeauthor{Bratman:1999ac}'s `two glasses of wine' (\Citeyear[38]{Bratman:1999ac}) case of temptation.\nolinebreak
  \footnote{
    \color{red}
    See also \textcite{Bratman:2007ab}
  }
\end{note}

\begin{note}[The Pianist]
  Consider a pianist who frequently performs at a club.
  Before each performance the pianist gets nervous and has the option of drinking a glass of wine.
  A glass of wine would also lead to a worse performance.
  However, the glass of wine would help with the pianist's nerves.
  Both are learnt with some experience.
  Hence, if the pianist reasons about what to do:
  \begin{itemize}
  \item When the pianist does not feel the nerves of an upcoming performance they reason to a preference abstaining from drinking a glass of wine.
  \item Yet, when nerves are felt the pianist reasons to a preference for drinking a glass of wine.
  \end{itemize}
  The pattern is stable, and has held over many performances.

  Still, while nerves sometimes get to the pianist, they abstain from drinking a glass of wine most of the time.

  That the our pianist abstains is not necessarily surprising --- it is not uncommon to resist temptation.
  Though it is puzzling.
  The pianist's reasoning is unwavering throughout the span of time in which the pianist has the option of drinking the wine; they reason to preference for drinking a glass of wine.
  So, if the pianist abstains, the pianist acts in opposition to their preference when given the option of drinking a glass of wine, and does so purposefully.
\end{note}

\begin{note}[Reasoning and desire]
  To clarify the puzzle, let us state a basic conjecture regarding preferences and acts.

  \begin{conjecture}\label{conj:resolve-issue-act}
    Any instance of purposeful rational action performed by an agent is the result of the agent resolving the issue of how to act.
    Where:
    \begin{enumerate}
    \item An act is an candidate resolution for how to act only if the agent has claimed support for preference for some proposition and has an expectation that act would bring about the proposition.
    \item An act is an admissible resolution only if there no other candidate resolutions for which the agent has a stronger (combined) preference with respect to the proposition(s) that the agent expects to be brought about by performing the act.
    \end{enumerate}
  \end{conjecture}

  \autoref{conj:resolve-issue-act} understands rational action as the result of an agent resolving the issue of how to act --- choosing which act from a collection of options to perform.
  By understanding rational action as the result of an agent resolving the issue of how to act we may break down the reasoning involved in purposeful rational action into two steps.

  First, what makes an act a candidate resolution, and second what makes an act an admissible resolution.

  An act is a possible resolution just in case the agent links the result of acting to some proposition the agent has a preference for.

  And, an  act is an admissible resolution just in case the agent has no stronger preference for some other proposition that the agent expect could be brought about by some other candidate action.
  (Or, more generally, when an agent is uncertain about which proposition may be brought about by some act, a combined preference regarding each potential proposition.)

  In short, \autoref{conj:resolve-issue-act} is more-or-less the core of an standard decision theoretic account of maximising expected utility without commitment to particular details.\nolinebreak
  \footnote{
    \color{red}
    Cf.\ \textcite{Steele:2020tr}.
    \citeauthor{Davidson:1963aa} `Primary reason' (\Citeyear{Davidson:1963aa})
  }
\end{note}

\begin{note}[Use of conjecture]
  \autoref{conj:resolve-issue-act} fixes an understanding of purposeful rational action, and in turn establishes two ways in which the pianist may resist drinking a glass of wine:
  \begin{enumerate}
  \item Drinking the glass of wine is not a candidate resolution.
    \autoref{conj:resolve-issue-act} states a necessary condition for a candidate resolution, but further conditions may rule out possible resolutions which satisfy the necessary condition stated.
  \item Drinking the glass of wine is not an admissible resolution.
    In particular, because the pianist has a claims support for a stronger preference toward the result of abstaining.
  \end{enumerate}

  We will provide a brief argument that \ESU{} requires the former to be the case and provide an example of how further conditions may rule out possible resolutions.
  In short, \ESU{} requires an agent to witness reasoning to a conclusion in order to claim support for such a conclusion, and as the pianist reasons to a preference for having drunk a glass of wine when performing, abstaining is not an admissible resolution.
  Then, we will turn to \EAS{}, and suggest that it allows the latter to be the case while granting that the only reasoning that the pianist witnesses establishes a preference for having drunk a glass of wine when performing.
  In short, so long as the pianist may claim support for the ability to reason to a stronger preference for the result of abstaining, then by \EAS{} the agent may claim support for a stronger preference for the result of abstaining.
\end{note}

\begin{note}
  Suppose \ESU{} is true.
  By \autoref{conj:resolve-issue-act} an agent resolves an issue of how to act by determining candidate resolutions.
  In turn, a candidate resolution results from the agent claiming support for a preference toward some proposition.
  And, \ESU{} requires that an agent must witness some instance of reasoning in order to claim support for the conclusion of the instance of reasoning.

  Turning to the pianist, we have assumed that before taking to the stage the pianist reasons to preference that favours drinking a glass of wine.
  So, given \autoref{conj:resolve-issue-act} abstaining is not an admissible resolution because the agent has a stronger preference from drinking a glass of wine before taking the stage.
  And, by \ESU{} it is not possible for the pianist to claim support for a preference that would lead to abstention because the pianist must witness the relevant reasoning in order to claim support.
  Therefore, the pianist must rule out drinking a glass of wine as a candidate resolution for how to act.
\end{note}

\begin{note}[Intention and \ESU{}]
  \color{red}

  \citeauthor{Bratman:2007ab} argues that such cases may be understood through an theory on which intentions constrain reasoning.
  If the pianist intends no to drink the wine, and this intention persists, then drinking the wine is no an available conclusion of reasoning.
  Key here is that intention does not interact with the pianist's preferences.
  It remains the case that the pianist would prefer.

  The role of intention in the role of \citeauthor{Bratman:2007ab} account is to constrain possible resolutions for how to act.
  An intention to not drink prevents drinking a glass of wine from being a candidate resolution for how to act (see in particular \textcite[\S3.3]{Bratman:1987aa}).

  However, because ruled out, the pianist does not have the option of acting on that preference.
  Rather, act in a way that is compatible with intention.
  For the pianist, we may assume abstention is the only act compatible with the intention.\nolinebreak
  \footnote{
    \color{red}
    Variation.
    Block contribution of nerves.
    So, intention to not allow nerves to contribute to reasoning.
    Compatible with drinking, but given the way the scenario has been constructed, will result in not drinking.
  }
  So, \citeauthor{Bratman:2007ab} is an example of how to resolve weakness of will given relation between reasoning and action expressed by \autoref{conj:resolve-issue-act} and \ESU{}.
\end{note}

\begin{note}[Broader]
  I think, in broad strokes, phenomena fit this kind of theory.
  Abstracting from the details of any particular theory, it seems plausible that candidate resolutions to the issue of how to act are subject to conditions that extend beyond whether the agent has claimed support for preference for some proposition and has an expectation that act would bring about the proposition --- \autoref{conj:resolve-issue-act} only stipulates that the given constraint is a necessary condition for candidate resolutions.

  However, not clear to me that all phenomena fit such a theory.
  The pianist's reasoning seems distorted.
  The nerves felt before taking to the stage plausibly interfere with the pianists reasoning about candidate resolutions to the issue of how to act, and so the pianist's reasoning plausibly does not resolve the issue of how to act in line with the premises and steps of reasoning that are available to the agent.
  So much, I suspect, is intuitive.
  However, any interpretation of the pianist compatible with \ESU{} is committed to the agent resolving the issue of how to act given unreliable reasoning.
  A \citeauthor{Bratman:1987aa}-like intention would rule out drinking a glass of wine as a resolution to the issue of how to act, but whatever reasoning the agent performs given the intention is still influenced by the nerves felt.

  In the following paragraphs we will suggest that the pianist may resist the conclusion of reasoning performed given nerves because it is distorted.
  We start with two additional conjectures.
\end{note}

\begin{note}[Preferences change with reasoning]
  \begin{conjecture}\label{conj:pref-vs-reasoning}
    Whether, or to what degree, an agent claims support for a preference toward a proposition may differ from whether, or to what degree, the agent would claim support for a preference toward the proposition given varying information.
  \end{conjecture}

  Loosely paraphrased,~\autoref{conj:pref-vs-reasoning} states that preferences an agent reasons to are subject to change given change in the information that the agent reasons from.
  Hence, it seems~\autoref{conj:pref-vs-reasoning} may be considered a truism rather than a conjecture.
  Indeed, varying information is a common component in the construction of cyclical preferences (cf.\ \cite{Sobel:1997wt},\cite{Schumm:1987wx},\cite{Davidson:1955wo}, etc.)

  To illustrate, we consider a case in which difference arises from information that does not contribute to the agent's preferential evaluation of the proposition.

  Suppose an agent has a established preference for meeting person who is Black Panther over meeting the person who is Storm by reasoning.
  However, the agent is not aware that Black Panther is T'Challa nor that Storm is Ororo Monroe.
  Indeed, the agent does not have any information about the referent of `T'Challa' or `Ororo Monroe' and so does establish a preference for meeting T'Challa or Ororo Monroe by reasoning (nor vice-versa).

  Still, it seems that if the agent were provided with the information that Black Panther is T'Challa and that Storm is Ororo Monroe then the agent would reason to a preference for meeting T'Challa or Ororo Monroe.
  And, such information would not contribute to the agent's preferential evaluation of the relevant propositions.
  The agent's preferences for the relevant propositions are determined by considerations that are independent of the terms used to refer to the relevant individuals --- e.g.\ the person who helped defeat Thanos and the person who helped defeat Magneto.

  With relation to \autoref{conj:resolve-issue-act}, whether or not an act is a possible resolution for issue of how to act may depend on what information agent reasons from.
  We now introduce a further conjecture:

  \begin{conjecture}\label{conj:more-info-is-good}
    Generally speaking: When resolving how to act, claimed support for some preference toward a proposition given more information is given greater weight than claimed support for preference toward the (same) proposition given less information.\nolinebreak
  \footnote{
    Note, `more' and `less' information are relative, and it may not be possible to compare distinct bodies of information.
    If so, \autoref{conj:more-info-is-good} does not state anything about the importance of either body of information.
    For example, an agent may have information about the subjective taste of a meal and about the nutritional value of the meal.
    Still, without a way to compare information about subjective taste to information nutritional value in an information there would be no sense in which the former could be considered to hold `more' information that the latter (or vice-versa).
    Though this is not to rule out such a comparison --- we do not place constraints on what comparisons an agent may make.
  }
\end{conjecture}

  \autoref{conj:more-info-is-good} speaks more generally than~\autoref{conj:pref-vs-reasoning} and as a result may be closer to or further from a truism depending on your point of view.
  Still, the core idea is simple:
  Claimed support for some preference toward a proposition given more information is worth more than claimed support for some preference toward a proposition given less information because more information typically increases the reduces the likelihood that the claimed support is either mistaken or misled.\nolinebreak
  \footnote{
    Cf.\ \cite{Good:1966wx} for a related idea --- though see also \cite{Bradley:2016wo}.
  }
  In other words, strength of preference in the sense of \autoref{conj:resolve-issue-act} is proportional to information used to establish preference given a fixed proposition.

  To illustrate, consider an agent resolving whether to banded or off-brand multi-vitamins.
  The agent's method of establishing a preference is to look on each container, and work through the lists of vitamins comparing whether a vitamin is included, and if so to what quantity, weighing some vitamins more heavily than others.
  As the agent works through the lists of vitamins the agent moves from less information to more.
  Still, after each vitamin on the list the agent marks a preference.
  For example, the branded multi-vitamins have 2,500 IU of vitamin A while the off-brand have 2,000 IU, so the agent's initial preference leads to purchasing the branded multi-vitamins.
  However, the branded multi-vitamins have 50mg of vitamin C while the off-brand have 60mg, and given information about vitamins A and C the agent's preference leads to purchasing the off-brand multi-vitamins.
  And so on until the agent has compared the contents of the branded and off-brand multi-vitamins.

  \autoref{conj:more-info-is-good} holds because it seems implausible that the agent could be understood as acting rationally by purchasing either multi-vitamin from a preference determined by a partial comparison between the multi-vitamins given that the agent has established a preference given a full comparison between the multi-vitamins.
\end{note}

\begin{note}[Back to the pianist]
  To summarise the two conjectures:
  \autoref{conj:pref-vs-reasoning} holds that an agent's preferences may vary with the information that the agent uses to claim support for those preference.
  And, \autoref{conj:more-info-is-good} holds, generally speaking, that claimed support for a preference from more information is given greater weight than claimed support for a preference from less information.

  Now, we return to the pianist.
  We make two observations.
  First, given \autoref{conj:pref-vs-reasoning} it may be possible to trace the change in the preference that the pianist reasons to (from abstention to a glass of wine, and vice-versa) to variation in the information that the pianist reasons with.
  In particular, it may be the case that the pianist's nerves prevent them from reasoning with information that they would otherwise reason with, hence the preference to abstain arrived at by reasoning before and after the span of time in which the pianist has the opportunity to drink a glass of wine is a preference arrived at given more information than the preference to drink a glass of wine.
  Second, given \autoref{conj:more-info-is-good} and the present interpretation, the agent may give greater weight to the preference to reasoning that results in a preference for abstention.

  In short, it may be true that pianist's nerves interfere with the reasoning they perform, and without such interference the agent would reason to abstaining from drinking a glass of wine before any given performance.\nolinebreak
  \footnote{
    \color{red}
    Plausible, though not immediate.
    To clear things up a little, consider a hangover.
    Reasoning sucks, I did not desire to miss lunch with a friend, I forgot, because of the hangover.
    However, I did desire to go to bed early, because of the hangover.
    Question is whether the nerves and the glass are like missing lunch or going to bed early.
    If the former, then reasoning is at issue.
    If the latter, then reasoning is at the issue.

    Conjecture, the former.
    No additional information from nerves.
    Still, when the nerves are present the pianist has a hard time reasoning with them.
  }

  The difficulty for the pianist is that such interference is always present;
  It is not straightforward for the pianist to give greater weight to abstaining, as the pianist does not reason to abstaining given their nerves.
  However, if the pianist may claim support for  ability to reason to abstaining, then by \EAS{} the agent may claim support for a preference for abstaining.

  Indeed, it seems the pianist may claim support for having the ability to establish a preference for abstaining when presented with the option of drinking a glass of wine.
  Two observations:
  \begin{itemize}
  \item First, the pianist has performed such reasoning many times, both before and after performances, and the result of such reasoning is stable: abstention.
  \item Second, given the assumptions made about the agent's nerves, the agent retains the relevant premises and steps of reasoning when presented with the choice to drink a glass of wine.
    The nerves felt when presented with the choice to drink a glass of wine only ensure that witnessing this ability is difficult to a degree sufficient for the agent to fail each attempt at witnessing the ability.
  \end{itemize}

  So, when presented with the choice to drink a glass of wine:
  If the  pianist may claim support for having the ability to establish a preference for abstaining, then by \EAS{} the pianist may claim support for a preference for abstaining.
  And, in turn, by \autoref{conj:more-info-is-good} the pianist may give greater weight to their preference for abstaining because the claim of support for a preference to abstain would be arrived at by taking into consideration additional information.

  Given this interpretation of the pianist, `giving into temptation' would be for the pianist to disregard the reasoning that the pianist is able to perform.
  Conversely, `resisting temptation' is for the pianist act in accordance with the preference that they would reason to given the information available to them.
  So, in contrast to accounts of temptation constrained by \ESU{} the pianist need not prevent themselves from reasoning to drinking a glass of wine.
  Rather, the pianist need only reflect on what they are able to reason to.
\end{note}

\begin{note}[Quick objection]
  There is a quick objection to consider before moving on:
  Does the agent have the ability to reason to stronger preference for the result of abstaining?
  For, it seems natural for the pianist to express that they do not have the ability to reason to a preference other than for having drunk a glass of wine when performing \emph{because} of how nerves interfere with their reasoning.

  Ability fluctuates.
  At present I claim support that I have the ability to prove that S4 is sound and complete with respect to transitive frames.
  Part of my claim is that I understand the details of Lindenbaum's Lemma.
  And, if I were to forget the details ofLindenbaum's Lemma then I would lack the ability to construct the relevant proof.
  So, I may lose the ability, but even if I do lose the ability due to forgetting the details of Lindenbaum's Lemma, I may regain the ability by revising the relevant details.

  However, there may be a difference between my loss of ability due to forgetting the details of Lindenbaum's Lemma and the pianist's nerves.
  Forgetting the details of Lindenbaum's Lemma ensures that I lack premises (or steps of reasoning) required to witness the proof.
  By contrast, it is not clear that the pianist's nerves entail that the pianist lack premises (or steps of reasoning) required to reason to stronger preference for the result of abstaining.

  Let us distinguish to ways in which an agent may be said to lack an ability to reason to some conclusion.
  First, the agent lacks sufficient resources; premises and steps of reasoning.
  Second, impediments to the agent using sufficient resources.

  From the first, I have the ability to enumerate all of the positive integers in decimal representation as I have sufficient resources to produce a decimal representation of the first positive integer and I have sufficient resources to produce a decimal representation of any successor integer.
  From the second, I am clearly bounded to enumerate only a finite collection of the positive integers given my mortality and so lack such an ability.

  The success of the quick objection relies on an ability to reason to some conclusion entailing that there are impediments to the agent using sufficient resources to reason to the conclusion.
  I suspect this entailment does not hold for the sense of ability at issue.
  Rather, I suggest that what matters is that the conclusion follows from the premises and steps of reasoning.

  Whether or not this a compelling suggestion is up to you.
  Whether or not \EAS{} holds does not depend on impediments to the agent using sufficient resources entail lack of ability.
  Though, interest in the details of ability and whether they relate to cases of temptation such as the pianist do depend on whether or not \EAS{} holds.
  Therefore, our primary focus will be on showing that \EAS{} holds.
\end{note}

\section{Structure of argument}
\label{sec:structure-argument}

\begin{note}[Structure of argument]
  Two lines of argument for endorsing~\EAS{}, and hence denying~\ESU{}.
  \begin{enumerate}[label=(L\arabic*), ref=(L\arabic*)]
  \item\label{arg:line:1} Motivate~\EAS{} as resolution to tension resulting from~\ESU{}.\newline
    Specifically:
    \begin{enumerate}[label=(L1\alph*)]
    \item\label{arg:line:1:a} Provide recipe for generating scenarios where~\ESU{} is in tension with particular scenarios involving information that an agent has the ability reason to some conclusion and a further claim regarding when it permissible for an agent to claim support for a proposition.
    \item\label{arg:line:1:b} Motivate~\EAS{} as a resolution to the tension.
    \end{enumerate}
  \item\label{arg:line:2} Argue that granting~\EAS{} as an exception to~\ESU{} allows for an intuitive understanding of cases in which agent has the option of appealing to ability, even if there are alternative ways of interpreting the scenario in line with~\ESU{}.
  \end{enumerate}
  These two lines of argument work together.
  The tension of~\ref{arg:line:1} generates interest in witnessing that may be flatly rejected by prior endorsement of~\ESU{}.
  The intuitive understanding of scenarios involving ability of~\ref{arg:line:2} suggests there's more to witnessing than resolving the tension in narrow cases.
\end{note}

\begin{note}[Details of \ref{arg:line:1}]
  The initial focus is on the first line of argument,~\ref{arg:line:1}.
  The tension developed in part~\ref{arg:line:1:a} is delicate, but hopefully informative.
  We will establish a number of corollaries regarding ability and the interaction between~\ESU{} and ability.
\end{note}

\subsection{Major argument}
\label{sec:major-argument}

\begin{itemize}
\item Type of case.
\item \gsi{-}.
  \begin{itemize}
  \item Two parts.
  \item Claiming support for specific ability
  \item Claiming support for result of specific ability.
  \end{itemize}
\item ability entailment.
\item Schematic interpretations of ability: \AR{} and \WR{}.
\item Exhaustive.
\item Relate to \ESU{} and \EAS{}.
\item Return to two parts of \gsi{}.
  \begin{itemize}
  \item \ESU{} requires that when agent reasons to specific ability, agent claims support for a property in line with \AR{}.
  \item However, \nI{} requires that when an agent reasons from specific ability, agent does not claim support from a property, in contrast to \AR{}.
  \end{itemize}
\item \ESU{} requires agent reasons with \AR{}.
\item Reasoning with \AR{} is incompatible with \nI{}.
\end{itemize}

So, the key thing is that we're looking at two different aspects of reasoning with ability.
Reasoning to and reasoning from.
And, it is not possible to give an account of \emph{both} reasoning to and from ability given \ESU{} and \nI{}.
So, the tension isn't, so to speak, `direct'.
Rather, tension arises due to some extended piece of reasoning.

So, one way to resolve tension is to deny extended reasoning.
First part argues that extended reasoning is plausible.


\begin{itemize}
\item Upshot:
  \begin{itemize}
  \item If scenarios, then either \AR{} or \WR{}.
  \item In turn, either not \nI{} or not \ESU{}.
  \item Alternatively, if scenarios then either \ESU{} or \nI{} (the scenarios give rise to this conflict).
  \item In turn, either \AR{} or \WR{}.
  \end{itemize}
\end{itemize}

We've already seen a decent amount of stuff regarding \ESU{} and \EAS{}.
These more or less correspond to \AR{} and \WR{}, kind of.

\subsection{Minor argument}
\label{sec:minor-argument}

???


% \begin{note}[Before turning to the argument\dots]
%   Before turning to the argument, we conclude this introduction with a handful of notes regarding~\ESU{} and~\EAS{}.
% \end{note}

% \begin{note}[Scope of \ESU{}]
%   \ESU{} does not say anything in particular about what the agent may claim support for, only what must be the case in order for an agent to appeal to support for some conclusion on the basis of support for premises.

%   Talking in terms of (support for) premises and conclusions restricts attention to reasoning.
%   There may be broader use of `premise' and `conclusion' where an agent is not required to reason from premise to conclusion in order for the premise to support the conclusion.
%   For example, if visual perception is immediate.
%   Perhaps it may be said that an agent's visual experience is a premise to the conclusion that a dog is sleeping.
%   Still, for present purposes, `conclusion' refers to the output of some process of reasoning performed by an agent which is either actual or potential, and `premises' to the input of that process.

%   Note, also, that in both cases the relation between premises and conclusion is important.
%   If agent does not reason, then neither~\USE{} nor~\ESU{} apply.
%   If there are multiple ways to obtain a conclusion, then~\ESU{} does not require the agent to reason from a particular set of premises.

%   Likewise,~\ESU{} does not require that an agent is required to obtain support for a proposition by valid and subjectively sound reasoning from some premises.

%   Rather,~\ESU{} requires that an agent reason from premises to conclusion in order to establishes support between premises and conclusion
%   By contrast,~\USE{} holds that reasoning is sufficient to establish such a relation.
% \end{note}

% \begin{note}[\ESU{} is intuitive]
%   \ESU{} is intuitive, and is quite common, though not without exceptions.
% (For example, there's views on testimony in which the testifier provides agent access to support the testifier has.
% One may understand this as conflicting with~\ESU{}, or that the fact that these are accessible is the relevant piece of support.)
% \end{note}

% \begin{note}[Alternative]
%   \EAS{} restricts~\ESU{}.
%   This is not to say the agent obtains support equivalent to that which would be obtained were the agent to do, or have done, the reasoning.
%   Nor, that the agent is aware of the relevant premises.

%   Intuitively, \EAS{} states that the agent may appeal to the reasoning they are able to perform in support for the conclusion of that reasoning, and as that reasoning moves from premises to conclusion, it is on the basis of the support for those premises that the agent would identify by reasoning that the agent obtains (some) support for the conclusion.

%   Hence, \EAS{} is in line with the spirit of~\USE{}.
%   For the exception to~\ESU{} is granted by the agent appealing to a witnessing event in which the antecedent (and consequent) of~\USE{} are satisfied.
% \end{note}

% \begin{note}[Ability ensures propositional?]
%   Plausible that if the agent has the ability, then the agent already has propositional support for the relevant proposition.
% \end{note}

\section{Major argument}
\label{sec:broad-argum-overv}

\begin{note}[Overview]
  Tension resulting from the unrestricted scope of~\ESU{}.
  We begin by introducing a particular type of scenario involving ability, and observe how~\ESU{} requires a unique interpretation of the scenario.
  We then introduce an additional principle regarding support, which conflicts with the interpretation of the type of scenario introduction required by~\ESU{}.
\end{note}

\begin{note}[Introducing key parts]
  Type of information and entailment.
  Two ways to understand entailment.
  Then, if information and entailment \dots
  Principles constrain understanding.
  \ESU{} and a second principle.
\end{note}

\subsection{Scenarios}
\label{sec:cases-interest}

Our goal is to argue for \EAS{} and against \ESU{}.
At the core of the argument is reasoning about ability.
Specifically, a certain type of scenario in which an agent reason to and from information that they have the ability to witness some specific act.
How the agent reasons with such (specific) ability information in the scenarios of interest will provide a type of counterexample to \ESU{} and in turn an argument for \EAS{}.

In this section we outline two key features of the scenarios we are interested in.
Subsection~\ref{sec:type-scenario} will introduce \gsi{-} to characterise how the agent reasons to the (specific) ability information.
Then, subsection~\ref{sec:ability-entailment} will introduce `\aben{the}' to characterise how the agent reason from the (specific) ability information.
Finally, subsection~\ref{sec:scenarios} will combine \gsi{-} and `\aben{the}' to provide an in-depth understanding of the type of scenarios we are interested in.

\subsubsection{\Gsi{}}
\label{sec:type-scenario}

\begin{note}[Tension, information]
  \begin{definition}[\gsi{}]
    \Gsi{-} is information that:\nolinebreak
    \footnote{
      Strictly speaking the formulation of \gsi{} as a conditional isn't important.
      What matters is that the agent is required to claim support for the general ability in order to claim support for the specific ability.
      For example, the conditional may be reformulated as:
      \begin{enumerate}[label=(\gsi{}\('\)), ref=(\gsi{}\('\))]
      \item Either \emph{S} does not have the general ability to \(\gamma\), or the agent has a specific ability to \(\varsigma\).
      \end{enumerate}
    }
    \begin{quote}
      If \emph{S} has a general ability to \(\gamma\), then \emph{S} has a specific ability to \(\varsigma\).
    \end{quote}
    Where \emph{S} is some agent, \(\gamma\) is some general ability, \(\varsigma\) is some specific ability, and it is either implicitly or explicitly stated that \(\varsigma\) is instance of \(\gamma\).
  \end{definition}

  The following pair of examples are instances of \gsi{}.
  \begin{enumerate}[label=(\gsi{}:\arabic*), ref=(\gsi{}:\arabic*)]
  \item\label{qe:cond} If you have the ability to reason with the rules of chess, then you have the ability to demonstrate that, given the arrangement of the board, there is a sequences of moves that will ensure a win for one of the players (as an instance of the general ability to reason with the rules of chess).
  \end{enumerate}

  \begin{enumerate}[label=(\gsi{}:\arabic*), ref=(\gsi{}:\arabic*), resume]
  \item\label{qe:cond:french} If you have the ability to read French, then you have the ability to read The Count of Monte Cristo without a translation (as an instance of the general ability to read French).
  \end{enumerate}
  In both examples an agent is informed that they have the ability to perform a specific act --- demonstrating a strategy or reading a book --- so long as they have some general ability --- an understanding of chess or French literacy --- because the witnessing the specific ability act would be an instance of witnessing the agent's general ability.

  \gsi{} does not directly provide the agent with the information that they have the specific ability.\nolinebreak
  \footnote{Nor (looking ahead to section~\ref{sec:ability-entailment}) does \gsi{} directly provide the agent with information that the result of witnessing the specific ability is when \aben{the} holds with respect to the specific ability.}
  The agent is not informed that they have the general ability and that therefore they have a specific ability.
  To illustrate, I am confident I have the ability to reason with the rules of chess, and so given \ref{qe:cond} I may be confident that I am able to demonstrate the existence of such a strategy.
  By contrast, I do not have the ability to read French, and so I do not have the ability to read The Count of Monte Cristo without a translation.

  Still, I may also be mistaken.
  It may be that I am overconfident, that I do not have the ability to reason with the rules of chess, and hence it may be the case that I do not have the ability to demonstrate the existence of the relevant chess strategy.
  Likewise, I may have the ability to read French, and may have the ability to read The Count of Monte Cristo without a translation.
  However unlikely this may be, I haven't tried to read French in quite some time.
\end{note}

\begin{note}[Not direct]
  \Gsi{} contrasts with what we term `\dsi{-}' --- information that the agent has some ability.
  \begin{definition}[\dsi{}]
    \Dsi{-} is information that:
    \begin{quote}
      \emph{S} has the ability to \(\varsigma\).
    \end{quote}
    Where \emph{S} is some agent and \(\varsigma\) is some specific ability.
  \end{definition}
  For example, the following is a `direct' reconstruction of~\ref{qe:cond}:

  \begin{enumerate}[label=(\dsi{}:\arabic*), ref=(\dsi{}:\arabic*), series=dsi_count]
  \item\label{qe:cons} You have the ability to demonstrate that there is a sequences of moves that will ensure a win for one of the players as an instance of your general ability to reason with the rules of chess.
  \end{enumerate}

  If~\ref{qe:cons} is true then the agent has the ability to demonstrate some strategy.
  And, in turn,~\ref{qe:cons} expands on why the agent has the relevant specific ability.
  By contrast,~\ref{qe:cond} may be true even if the agent does not have the ability to demonstrate some strategy.
  Hence, \dsi{} is not in general entailed by \gsi{}.\nolinebreak
  \footnote{
    However, if it is the case that an agent has the general ability mentioned in the antecedent of \gsi{}, then a corresponding instance of \dsi{} will be true.
    Note, this is ensured because the consequent of~\ref{qe:cond} ensures the relevant `instance of' relation obtains.
    % So, if I have the ability to reason with the rules of chess and~\ref{qe:cond} is true with respect to me, then \ref{qe:cons} will also be true with respect to me.
  }
\end{note}

\begin{note}[Important features of \gsi{}]
  \gsi{}, then, has two important features:
  \begin{enumerate}
  \item \gsi{} ensures that the agent is on the hook, so to speak, for claiming support they have the specific ability.
  \item If the agent may claim support for having the relevant general ability, then \gsi{} provides the agent with an account of why they may claim support for having some specific ability.
  \end{enumerate}
  Hence, \gsi{} ensure that an agent must themselves claim support that they have some specific ability while providing the agent with relevant information about why they may claim support for having the specific ability.
\end{note}

\begin{note}[Merging \gsi{} and \dsi{}]
  Finally, though we will focus on \gsi{}, there is a variant that merges \gsi{} and \dsi{} which could be substituted for \gsi{} in further discussion.
  This variant involves informing an agent that they have some general ability, and some specific ability as an instance of that general ability, but requires the agent to identify what the general ability is.

  Here is the variant applied to~\ref{qe:cond}.
  \begin{enumerate}[label=(\gsi{}\(^{'}\):\arabic*), ref=(\gsi{}\(^{'}\):\arabic*)]
  \item
    \begin{enumerate}
    \item You have some general ability \(\gamma\), and a specific ability \(\varsigma\) (as an instance of that general ability). And,
    \item If \(\gamma\) is the ability to reason with the rules of chess, then \(\varsigma\) is the ability to demonstrate that, given the arrangement of the board, there is a sequences of moves that will ensure a win for one of the players (as an instance of the general ability)
    \end{enumerate}
  \end{enumerate}
  The agent remains on the hook, so to speak, for claiming support that they have the relevant specific ability because it is up to the agent to identify the general ability \emph{as} the ability to reason with the rules of chess.
  And, likewise, if the agent may claim support for identifying the general ability in a particular way, then the variant allows the agent to claim support that they have a particular specific ability.

  We favour \gsi{} given it's comparative structural simplicity, but the variant highlights that that the agent claiming support for having some specific ability is not of interest.
  Rather, what is interest is that \gsi{} allows the agent to claim support for the particulars of some specific ability.

  In section~\ref{sec:ability-entailment} we will highlight why the particulars matter.
\end{note}

\subsubsection{An ability entailment}
\label{sec:ability-entailment}

\begin{note}[\aben{(The)}]
  The second component in scenarios of interest is the availability of an entailment from the specific ability.

  We term an instance of the entailment as an `\aben{}'.

  \begin{definition}[Ability entailment]\label{def:aben}
    \aben{The} is any entailment of the form:
    \begin{quote}
      \emph{S} has the (specific) ability to \emph{V} that \(\phi\) \emph{therefore} \(\phi\) is the case.
    \end{quote}
    Where \emph{S} is an agent, \emph{V} is some action, and \(\phi\) is some proposition.
  \end{definition}

  The rough intuition behind instances of \aben{the} is that \(\phi\) being the case does not depend on \emph{S} witnessing the (specific) ability to \emph{V} that \(\phi\).
  So, \aben{the} links ability and something that must be the case in order to have ability and the result of witnessing ability must be the case in order for the agent to have the ability

  For example, \aben{the} holds with respect to the (specific) ability to demonstrate the existence of a chess strategy from \ref{qe:cond} as whether or not a given chess strategy exists depends on the moves permitted by the rules of chess --- a strategy that has not been demonstrated is a strategy.
  Likewise, \emph{S} has the (specific) ability to discover that their keys are in their jacket pocket only if it is the case that their keys are in their jacket pocket --- whether or not \emph{S}'s keys are in their jacket pocket does not depend on \emph{S} discovering that to be the case.

  By contrast, `to read The Count of Monte Cristo without a translation' is an action and so \aben{the} does not apply to the specific ability of~\ref{qe:cond:french}.
  Even so, \aben{the} apply to nearby variants and not others.
  \emph{S} may have the specific ability to read that Dantès was a merchant sailor, and it follows that Dantès was a merchant sailor.
  In contrast, while \emph{S} may have the ability to believe that certain passages cannot be adequately translated, it does not follow that those passages cannot be adequately translated.
  Similarly, \emph{S} may have the ability to hope that they will employ the chess strategy discovered in a competitive game, but it does not follow that \emph{S} will employ the strategy.

  More broadly, \aben{the} holds with respect to factive verbs, such as `see', `know', `understand', and so on.
  Though, I doubt factive verbs are an adequate explanation for \aben{the}.
  Consider `read'.
  I have the ability to read that Elvis Presley was born in 1935, but I also have the ability to read that Elvis is working undercover for the DEA.
  What matters, then, is not the verb used, but how the agent would witness the relevant ability.
  I have the ability to read that Elvis was born in 1935 from a reliable source, and hence \aben{the} applies.
  The same is not true for my ability to read that Elvis is working for the DEA.

  Indeed, \aben{the} merely identifies an entailment.
  It does not provide an account of when or why such entailments hold.
  We identify entailments of this type because our interest is in how (in certain cases) agent's reason with instances of \aben{the}.
\end{note}

\subsubsection{Details of scenarios}
\label{sec:scenarios}

\begin{note}[Both things are important]
  The scenarios we are interested in combine \gsi{} with \aben{the}.

  The role of \gsi{} is to ensure that the agent is not provided with direct information about specific ability.
  And the role of \aben{the} is to highlight that the agent is in a position to claim support for some further proposition if they claim support for specific ability.
  Hence, scenarios combine claiming support \emph{for} specific ability and claiming support \emph{from} specific ability.

  To illustrate, consider the following pattern of reasoning:
  \begin{enumerate}[label=\arabic*., ref=(\arabic*)]
  \item\label{scen:rp:1} I have the general ability to reason with the rules of chess.
  \item\label{scen:rp:2} I received \gsi{} information that if they have the general ability to reason with the rules of chess then they have the ability to demonstrate the existence of some strategy.
  \item\label{scen:rp:3} So, from~\ref{scen:rp:1} and~\ref{scen:rp:2} it follows that I have the ability to demonstrate the existence of some strategy.
  \item\label{scen:rp:4} And, as \aben{the} hold with respect to~\ref{scen:rp:3}, the relevant strategy exists.
  \end{enumerate}
  I reason to (\ref{scen:rp:1} --- \ref{scen:rp:3}) and from (\ref{scen:rp:3} --- \ref{scen:rp:4}) a specific ability.
  The reasoning pattern seems sound.
  And, at no point do I need to witness their general ability to reason with the rules of chess, or the specific application of the general ability to demonstrate the existence of the strategy.
\end{note}

\begin{note}
  Both components are important.
  Focus on \gsi{} will restrict the interpretation of what the agent claims support for.
  And, in turn, what the agent has claimed support for will determine what the agent appeals to when appealing to \aben{the} entailment.\nolinebreak
  \footnote{
    I suspect it may be possible to focus only on \gsi{}.
    As we will see, this is where the key step of the argument takes place.
    However, this is not trivial.
    Would require more focus on how the agent gets to specific from general.
    By splitting in this way, we avoid details.
    Instead, focus on what it is that the agent gets, and then \aben{the} is forced to work with this.
  }
  \gsi{} and \aben{the} combine to provide a (partial) functional characterisation of reasoning with specific ability.
\end{note}

\begin{note}
  Note, however, that there is a distinction between how an agent reasons about ability, and what ability is.
  We are interested in how agent's reason about (specific) ability, and not what makes it true that an agent has a (specific) ability.
  Our focus will shortly turn to how to interpret (specific) ability when appealed to in the type of scenario described.
  We will outline two general schematic interpretations of ability, argue that these are exhaustive, and note how general constraints such as \ESU{} constrain which interpretation is available.
\end{note}

\begin{note}[Scenario proposition]
  For ease of reference, we wrap scenarios involving the limited information as a proposition.
  \begin{proposition}[\eA{-} --- \eA{}]\label{prop:SE}
    There are scenarios in which an agent \emph{S} receives \gsi{} information of the form:
    % \mbox{ }\vspace{5pt}
    \begin{center}
      If \emph{S} has a general ability to \(\gamma\), then \emph{S} has a specific ability to \emph{V} that \(\phi\).
    \end{center}
    % \mbox{ }\vspace{5pt}

    \noindent Such that \aben{the} applies to the specific ability to \emph{V} that \(\phi\).

    In turn:
    \begin{enumerate}
    \item \emph{S} may reason from claimed support that they have the general ability to \(\gamma\) in order to claim support for having the specific ability to \emph{V} that \(\phi\). And,
    \item \emph{S} may reason from their claimed support that they have the ability to \emph{V} that \(\phi\) to claim support that \(\phi\) is the case by appealing to \aben{the}.
    \end{enumerate}
    \vspace{-\topsep}\vspace{-\topsep}
  \end{proposition}
\end{note}

\begin{note}[Possible restrictions]
  First, \eA{} holds only that there are cases in which the agent may appeal to ability to obtain support.
  It is therefore consistent with~\eA{} that there are cases in which the details of the cases outlined are satisfied, but where kind of support is unsuitable for certain purposes.
  For example, some witness of ability may be demanded by a third-party.
  In this respect, the content of \eA{} is similar to an analogous claim with respect to memory.
  If an agent remembers proving that \(\phi\), then \(\phi\) is the case.
  Still, one may still request that an agent provides you with a proof of \(\phi\) in order to for you to be satisfied that \(\phi\) is the case --- many exams are like this.
  So, that an agent may not always and in any context claim support for \(\phi\) from claimed support for their ability to \emph{V} that \(\phi\) is not an objection to~\eA{}.
\end{note}

\begin{note}
  Second, \eA{} does not require that an agent reason in the way described given \gsi{} and availability of \aben{the}.

  For example, the following statement is an instance of \gsi{}:
  \begin{enumerate}
  \item Any person who has the (general) ability to reason with the rules of chess has the (specific) ability to identify Alekhine's Defense as a fine opening move.
  \end{enumerate}
  The universal quantifier implies that the statement is true with respect to me, among others.
  Still, I am confident that there is at least one other person who has the ability to reason with the rules of chess, and may therefore infer that Alekhine's Defense as a fine opening move without appealing to my own ability.
  Indeed, if I am inclined to doubt my own (general) ability in contrast to a Grandmaster, then I may be more confident that Alekhine's Defense as a fine opening move if I appeal to the existence of a Grandmaster.

  Again, it is consistent with \eA{} that an agent may reason in such a way.
  Still, in defence of \eA{} it is important to note that \gsi{} information may be limited to the agent in question.
  For example, I may have studied your notes on how to play chess and identified a strategy which follows from those notes.
  I have no doubt that you have the ability to identify the same strategy, so when I provide \gsi{} my emphasis is on whether you have the ability to reason with \emph{chess}, rather than some closely related game.

  There are many ways to build context so that an agents is required to reason with \gsi{} and \aben{the} if the agent is to reason with (specific) ability at all, but I doubt these are required.
  The reasoning described by \eA{} (and illustrated above) seems plain and permissible.
\end{note}

\begin{note}
  Finally, \gsi{} and \aben{the} are constraints which do not hold in all cases of reasoning with specific ability.

  For example, one may be told that a gift of a metal detector grants the ability to discover if there is buried treasure in the garden.
  The former does not entail that there is buried treasure in the garden, and testimony or the metal detector may be claimed as support for the ability.

  So, question about whether this really does anything for general understanding of ability.
  \gsi{} and \aben{the} combine to require a particular interpretation.
  However, interpretation with general applicability is not restricted to instances in which it is forced.
  The role of a counterexample is not (typically) to establish that every instance of a theory is mistaken, but to identify a gap.
  And, even if the original theory may be restricted to non-problematic cases, the alternative theory may compete with the original theory.
  So, given that the particular interpretation is required to hold given additional stipulations, interest is in whether it holds without additional stipulations.
\end{note}

\subsubsection{Reasoning in the scenarios}
\label{sec:reasoning-scenarios}

\begin{note}
  {
    \color{red}
    Two different ways of reasoning here.
    Either from specific to \(\phi\) or from general to \(\phi\).
  }
\end{note}

\begin{note}
  With general ability, this is a `derived ability entailment'.
\end{note}

\subsection{Two (schematic) interpretations of (specific) ability}
\label{sec:wr-ar}

\begin{note}
  In the previous section we introduced \gsi{-} and \aben{the}.
  In the present section we motivate two interpretations of (specific) ability in the context of reasoning to (specific) ability from \gsi{} and reasoning from (specific) ability with \aben{the}.

  The two interpretations are termed `\AR{}' and `\WR{}' in turn, and are schematic.
  Roughly:
  \AR{} holds that when appealing to (specific) ability an agent appeals to some property or attribute that they have.
  And, by contrast, \WR{} holds that when appealing to (specific) ability an agent appeals to the action that they would perform by witnessing the relevant ability.
  \AR{} and \WR{} are distinguished, then, by whether an agent reasons with a property (\AR{}) or an event (\WR{}).

  To illustrate by analogy, consider a mechanical clock.
  The clock has the property of displaying the correct time, by it is also involved in the event of changing it's configuration as time passes.
  The property that the clock is displaying the correct time is important for determining whether one is late for a meeting.
  By contrast, the event of changing it's configuration as time passes is important for determining when to remove a brewing teabag.
  A meeting starts at a certain point in time, while tea is brewed over a period of time.
  If the clock does not represent the correct time, then three minutes passing will not, in general, help determine whether one is late to the meeting.
  And, whether or not it is 3pm is not, in general, important with respect to whether or not the tea has finished brewing.
  The qualifier `in general' is important.
  Measuring the passage of is useful if I know the length of time before the meeting is due, and the correct time is useful if I know when I started brewing the tea.

  The distinction between \AR{} and \WR{} is similar.
  Both interpretations may be more or less useful in certain circumstances, and interchangeable in others.
  Still, the combination of \gsi{} and \aben{the} identify a pattern of reasoning in which we may elaborate how the relevant interpretation of (specific) ability is important, and in turn broader principles (\ESU{} and, to be introduced below, \nI{}) will constrain whether the interpretations are permissible.

  We begin by introducing \AR{} and \WR{} with respect to \aben{the} in section~\ref{sec:ar-wr-1}.
  Then, in subsection~\ref{sec:ar-wr-gsi} we will relate \AR{} and \WR{} to \gsi{}.
  Finally, in subsection~\ref{sec:ar-wr-are} we will argue that \AR{} and \WR{} are exhaustive, though not exclusive, interpretations of (specific) ability.
  The following two sections~\ref{sec:first-conditional} and~\ref{sec:second-conditional} will then demonstrate how \WR{} and \AR{} conflict with broader principles.
\end{note}

\begin{note}[\dd{} and \dr{}]
  \color{red}
  \AR{} and \WR{} make use of the \dd{} and \dr{} distinction.
  A note on how this distinction is applied.

  For, we've talked about reasoning in terms of assigning values to propositions.
  Familiar case.
  \(\exists xPx\) vs \(Pa\).
  Here, the distinction is with respect to reference.
  With \(Pa\) the reference is to \(a\).
  With \(\exists xPx\) there's no reference to any particular object.

  The way in which we use the distinction is different.
  With respect to entailment.

  \(\exists xPx\) entails something \(\phi\).
  Two ways of viewing this.
  First, there is some referent of \(x\).
  And, the referent does the work in establishing \(\phi\).
  In this sense, \dr{}.
  Alternative is that \(\exists xPx\) entails \(\phi\).
  In this sense, \dd{}.

  The contrast is \dd{} uses the value assigned to the proposition to establish \(\phi\).
  By contrast, \dr{} uses the referent of the proposition to establish \(\phi\).

  So, the distinction also applies to \(Pa\).
  \dd{}, the value assigned to \(Pa\).
  \dr{}, that \(P\) holds of \(a\).

  Key difference is with how entailment works.
  Entailment because of value assigned --- \dd{}
  Entailment because of what is the case given value assigned -- \dr{}.

  So, apply to reasoning.
  \(\phi \vdash \psi\).
  Claim support for \(\phi\).
  It is true that there are premises and steps of reasoning that ensure that \(\psi\) is true.
  This is \dd{}.
  Note, there's an implicit conditional here.
  By contrast, \(\phi\) and steps of reasoning entail \(\psi\).
  This is \dr{}.
  There's no implicit conditional.

  The most basic way to view this distinction is in terms of syntactic and semantic entailment.
  Syntactic is \dd{} and semantic is \dr{}.
  However, this is hard to apply.
  For, syntactic entailment is just a matter of syntax.
  By contrast, when an agent reasons \dd{} it's because the proposition has a certain value --- not that the relevant premises has a certain form.

  Indeed, standard modal semantics doesn't help.
  Here, function of the modal may be seen as shifting reference.

  Rather, distinction is captured by semantics of propositional versus first (or higher order) logic.

  I suspect this may be seen as related to the more familiar distinction.
  However, not sure how to fix the relation at this point in time.
  Intuitively, any reasoning follows from description, and breaks down with co-reference, etc.
  However, with respect to the standard modal semantics, things don't really work out this way.
\end{note}

\begin{note}[Table]
  \begin{figure}[H]
    \centering
    \begin{tblr}{abovesep=8pt, belowsep=8pt, width=0.95\textwidth, colspec={Q[c,m]|Q[c,m]|Q[1.8,c,m]|Q[1.8,c,m]}}
      \multicolumn{2}{c}{} & \nr{} & \ur{} \\
      \hline
      \multicolumn{2}{c}{\WR{}} & ? & ? \\
      \hline
      \multirow{2}{*}{\AR{}} & Basic & ? & ? \\
      \cline[dashed]{2-4}
      & Derived & ?  & ? \\
    \end{tblr}
    \caption{Distinction matrix for interpretations of \aben{the}. \\ Rows are interpretations of ability, columns are type of reasoning regarding ability.}
  \end{figure}
\end{note}


\subsection{\AR{} and \WR{}}
\label{sec:ar-wr-1}

\begin{note}[\WR{} and \AR{}]
  We term the two schematic interpretations of \aben{the} `\AR{}' and `\WR{}', respectively.
  Brief descriptions from detached perspective.
  Given that the interpretations are schematic, they fall short of a full account of how an agent claims support by \aben{an}.
  However, the arguments to follow are of interest in part because they concern any way in which the schematic interpretations are filled out.
\end{note}

{
  \color{red}
  I should emphasise that here we're interested in reasoning.

  Also, the distinction is important to ensure that the argument's don't depend on a specific reading of ability.
}

\subsubsection{\AR{}}
\label{sec:ar-1}

\begin{note}
  \begin{definition}[\AR{}]\label{AR:def}
    An agent's reasoning with an instance of \aben{the} by claiming support for \(\phi\) from \emph{S} having ability to \emph{V} that \(\phi\) is an instance of \emph{\AR{}} when the agent holds that:

    \emph{S} has the ability to \emph{V} that \(\phi\)
    \begin{enumerate*}[label=(\textsf{A}\arabic*), ref=(\textsf{A}\arabic*)]
    \item\label{A:s:1} is or reduces to some (possibly complex) property \emph{P} of \emph{S}, and
    \item\label{A:s:2} \emph{P}, or some part of \emph{P}, entails that \(\phi\) is the case.\nolinebreak
      \footnote{Intuitively, because the agent could not have \emph{P} without \(\phi\) already being the case.
      The notion of entailment here does not require that \(\phi\) is true \emph{because} of \emph{P}.}
    \end{enumerate*}
  \end{definition}

  {
    \color{red}
    \AR{} identifies instances of reasoning in which an agent applies \aben{the} by holding the ability to \emph{V} that \(\phi\) is a property of an agent.\nolinebreak
    \footnote{
      Note, this does not say anything about what the ability to \(\phi\) is.
      Rather, way in which the agent claims support.
    }
    Note, when appealing to \aben{the} an agent need not be aware of what the (potentially complex) property of \emph{S} is.
    Rather, claimed support that \emph{S} has the ability to \emph{V} that \(\phi\) allows the agent to claim support for the existence of some property of \emph{S} which in turn entails \(\phi\).
  }

  Now, generally speaking properties are things which may be predicated or attributed of other things.
  The coffee is hot, I am thirsty, my mouth is sensitive to heat, I am reckless, I am in pain, and so on\dots
  And, properties come cheap.
  For example, the participation of an agent in some event gives rise to a property that may be attributed to the agent.
  Specifically, the property of participating in the event.
  Moments ago I participated in the event of recklessly drinking hot coffee with a mouth that is sensitive to heat.
  Therefore, I have the property of participating in such an event.

  So,~\ref{A:s:1} is trivially true.
  When we speak of an agent having some ability we are predicating or attributing ability to an agent.
  However,~\ref{A:s:2} requires that the property entails that \(\phi\) is the case.
  And, it is not clear that an entailment which follows from an event is always reflected in the property of being a participant in the event.
  For example, it seems that I am in pain because I participated in the event of drinking hot coffee, \emph{not} because I have the property of having participated in the event of drinking hot coffee.
  By contrast, that I have the property of having participated in the event of drinking hot coffee entails that I have the property of having participated in the event of drinking something.

  % From~\ref{A:s:2} it must be the case that the relevant property entails \(\phi\).
  % And, from~\ref{A:s:3} the property must not analysed in terms of there being a potential event in which \emph{S} witnesses the act of \emph{V}ing that \(\phi\).
  % This is, from one perspective, an arbitrary restriction.
  % For example, if there is a potential event in which an agent witnesses the act of \emph{V}ing that \(\phi\), then the agent has the property of being a participant of that potential event.
  % From a different perspective,~\ref{A:s:3}

  Roughly, we may expect the property of interest is akin to having a heart, possessing ¥500, being of a certain age, and so on\dots

  {
    \color{red}
    Key idea with \AR{} is that the agent `directly' claims support for a property when using \aben{the}.
  }

  To illustrate \AR{} we focus on the idea of reducing the ability to \emph{V} that \(\phi\) to some (potentially complex) property of \emph{S}.
  Again, when appealing to \aben{the} an agent need not be aware of what the (potentially complex) property of \emph{S} is.
  Rather, these illustrations suggest that such properties exist.

  \begin{illustration}
    Consider the proposition that \emph{S} has the ability to hear that the birds are signing.
    Again, it seems \aben{the} holds, and one may infer that birds are singing.

    So, by \AR{} there is some (complex) property \emph{P} of \emph{S} such that \emph{P}, or some part of \emph{P}, entails the the birds are signing.

    Consider the complex property of a well-functioning auditory system and sufficient proximity to the birds singing.
    The property of having well-functioning auditory system ensures that \emph{S} has the ability to hear nearby noises.
    And, having well-functioning auditory system together sufficient proximity to the birds singing together ensure that \emph{S} has the ability to hear the nearby noise of the birds singing.

    \aben{the} follows from part of this complex property.
    If the agent has the property of being in sufficient proximity to the birds singing, then it follows that there are birds singing.
  \end{illustration}

  \begin{illustration}
    Consider the proposition that the prosecution has the ability to demonstrate that the defendant is guilty.
    Intuitively, \aben{the} holds, as it is not possible to demonstrate the guilt of an innocent person.\nolinebreak
    \footnote{
      It is a different matter to convince a jury of the guilt of an innocent person.
      And, \aben{the} does not seem to hold with respect to the ability to convince a jury that the defendant is guilty.
    }
    By \AR{} there is some (complex) property \emph{P} of the lawyer such that \emph{P}, or some part of \emph{P}, entails the guilt of the defendant.
    Say, the lawyer is in possession of evidence sufficient to establish guilt of the defendant.
    If so, it is a property of the lawyer that they are in possession of such evidence, and by assumption the evidence entails that the defendant is guilty.

    It seems possession of evidence alone may not be sufficient to establish that the lawyer has the ability to prove that the defendant is guilty.
    For, it is plausible that a lawyer may be in possession of evidence that they do not understand.
    However, as our interest is with \aben{the} it is sufficient to observe that the evidence alone entails the guilt of the defendant.
  \end{illustration}

  Again, these illustrations highlight ways in which \emph{S} having the ability to \emph{V} that \(\phi\) may be reduced to some (complex) property of \emph{S}.
  \AR{} does not hold that an agent identifies such a property when claimed support by an instance of \aben{the}.
  Rather, \AR{} holds that the agent reasons with ability as a property of the agent.
  Indeed, while these suggestions reduce ability to complex properties, \AR{} also admits of the possibility that the ability to \emph{V} that \(\phi\) is a basic property which does not admit of further analysis.
  If so, then it seems that \aben{the} must also be taken as basic.\nolinebreak
  \footnote{
    I lack any suggestion for how to understand \AR{} if the property is indeed basic, but there is no need to rule out this option ---  no part of the following arguments depend on whether or how these schemas may be substantiated.
  }
  So, to summarise.
  The distinguishing feature of \AR{} is that there are instances when an agent claims support for \(\phi\) from claimed support that \emph{S} has the ability to \emph{V} that \(\phi\) because the latter ensures that there is some property \emph{P} holds of \emph{S} and \emph{P} entails \(\phi\).
  If the agent has the ability to \emph{V} that \(\phi\), then there may also be some action, \emph{V}ing, that the agent may witness.
  However, as \AR{} appeals to some property, the witnessing event is irrelevant to the way in which the agent claims support for \(\phi\).
\end{note}

\begin{note}
  {
    \color{red}
    Some additional notes on \AR{} that haven't been merged with the above follow.
  }
\end{note}

\begin{note}[Compatibility]
  \AR{} suggests an alternative way to obtain support for the conclusion of reasoning the agent is able to do.
  Specifically, if order for the agent to \emph{have} the attribute of being able to reason to the conclusion, the conclusion of the reasoning must be true.
  The relevant entailment is in part secured by the verb chosen, and in part by what the verb is applied to.
  Here, `demonstrate' is a factive verb, if an agent demonstrates that \(\phi\), then it is true that \(\phi\).
  And, the existence of a chess strategy does not depend on the agent demonstrating that the relevant strategy exists.

  To take another example, you only have the ability to identify a typo on this page if there is a typo on this page.
  So, if I were to provide you with testimony that you have the ability to identify a typo on this page, you may begin searching for the typo, or you may note that there must be a typo in order for me to be in a position to provide you with testimony that you have the ability.
\end{note}

\begin{note}[Sketch of \AR{}]
  \begin{enumerate}[label=(\textsf{A}\arabic*), ref=(\textsf{A}\arabic*)]
  \item\label{AR:Sketch:1} I have the attribute of being able to \emph{V} that \(\phi\).
  \item\label{AR:Sketch:2} In order to have the attribute of being able to \emph{V} that \(\phi\), \(\phi\) must be the case independent of whether or not I witness the ability.
  \item\label{AR:Sketch:3} \(\phi\) is the case.
  \end{enumerate}

  To keep things simple, we will refer to the principle behind the pattern sketched as \AR{}.
  And agent may bundle~\ref{WR:Sketch:1} and~\ref{WR:Sketch:3} into a conditional, and avoid instantiating the reasoning pattern, but so long as the conditional is (implicitly) held on the basis of the intermediate premise~\ref{WR:Sketch:2}, we take use of such a conditional to be an instance of \AR{}.
\end{note}


\subsubsection{\WR{}}
\label{sec:wr-1}

\begin{note}[\WR{} def.]
  {
    \color{red}
    Include: observation that the entailment may come from some property of the agent.
    The point of \WR{} is that the agent claims support for details of the event.
  }

  We now turn to \WR{}.
  \begin{definition}[\WR{}]\label{WR:def}
        An agent's reasoning with an instance of \aben{the} by claiming support for \(\phi\) from \emph{S} having ability to \emph{V} that \(\phi\) is an instance of \emph{\WR{}} when the agent holds that:
    \begin{enumerate}
    \item\label{WR:def:1} \emph{S} has the ability to \emph{V} that \(\phi\) \emph{if and only if} there is a potential event in which \emph{S} witnesses the act of \emph{V}ing that \(\phi\).
    \item\label{WR:def:2} Claim support for event or details of event.
    \item\label{WR:def:3} Details of the event in which \emph{S} witnesses the act of \emph{V}ing that \(\phi\), or part of the event, entails that \(\phi\) is the case.\nolinebreak
      \footnote{Again, intuitively, because there could not be a potential event in which \emph{S} witnesses the act of \emph{V}ing that \(\phi\) without \(\phi\) already being the case.
      The notion of entailment here does not require that \(\phi\) is true \emph{because} there is some potential event of the relevant kind.}
    \end{enumerate}
  \end{definition}

  {
    \color{red}
    ~\textcite{Rebuschi:2011ub} talk about \emph{de objecto} attitudes.
    This might be helpful given that the events are potential.
  }

  {
    \color{red}
    Key idea with \WR{} is that the agent appeals to certain things which follow from the event being witnessed.
    Whereas, \AR{} appeals to certain things which must be the case in order for the event to be witnessed.
  }

  {
    \color{red}
    Difference between the existence of an event (~\ref{WR:def:1}) and details of the event (~\ref{WR:def:2}).
    To clarify.
    \(\exists e(V(e) \land \text{agent} = \emph{S} \dots)\).
    \(\phi\) follows.
    However, there are two ways to think about this.
    First, the existential, second the event.
    \emph{De dicto} and \emph{de re}.
    \WR{} is \emph{de re}.

    Consider existential of individuals.
  }

  {
    \color{green}
    Before going into the details, it'll be helpful to highlight the big idea, especially with respect to how things (will) work out with the `master property' from \AR{}.
  }

  \WR{} identifies instances of reasoning in which an agent applies \aben{the} by holding that \emph{S} having the ability to \emph{V} that \(\phi\) ensures there is a possible event in which \emph{S} \emph{V}s that \(\phi\).
  And, in turn, there is a possible event in which \emph{S} \emph{V}s that \(\phi\) entails that \(\phi\) is the case.
  In contrast to \AR{}, when an agent claims support as an instance of \WR{} an agent reasons about what must be the case in order for \emph{S} to witness some ability, rather than what must be the case in order for \emph{S} to have the property of possessing some ability.


  We use the term `potential' in place of `possible' when describing the relevant event to highlight that the existence of the event is tied to an ability attribution.
  One may hold that a possible event is any event which is not impossible, and hence it is possible for an arbitrary agent to prove Fermat's Last Theorem.
  Yet, it seems most agent's lack the ability to prove Fermat's Last Theorem, and so `potential' serves to restrict quantifier over events which an agent has the ability to witness --- however the details of that quantification are resolved.

  {
    \color{red}
    \WR{} is more complex than \AR{}.
    There is some action that \emph{S} may witness.
    And, understand what the result of that action is.
    So, we have something akin to a counterfactual.
    However, the counterfactual only relies on witnessing.
    Further, particular status of \(\phi\).
    Hence, as witnessing is the only issue, \(\phi\) is the case.

    Third, regardless.
    \(\phi\) holds regardless, but it does not follow from this that if the agent reasons via \WR{} then support claimed for \(\phi\) would be independent of ability information.
    The agent must recognise that \(\phi\) must be the case regardless, but this doesn't require that the agent has any way of reasoning to \(\phi\) other than by witnessing their ability.
    The point is clearer when considering witnessed instances of reasoning.
    \emph{X} testified that \emph{p}.
    Claim support for \emph{p}.
    \emph{p} is not the case because \emph{X} testified that \emph{p}, though my only path to claim support is by appeal to the testimony of \emph{X}.
  }
  To illustrate.

  \begin{illustration}
    I have the ability to calculate that \(243 \div 3 = 82\).
    Pen and paper to hand, etc.\
    Result of this will be a calculation that \(243 \div 3 = 82\).
    However, my calculation is irrelevant to whether it is the case that \(243 \div 3 = 82\).
    Hence, it follows that \(243 \div 3 = 82\).
  \end{illustration}

  \begin{illustration}
    Ability to discover that the ball is under the left cup.
    Raise the left cup, and identify the ball.
    Whether or not the ball is under the left cup is independent of this sequence of actions, and therefore it follows that the ball is under the left cup.
  \end{illustration}

  Compare to cases where only gets the counterfactual.
  I have the ability to make it so that the heating is turned out.
  Plausibly, the heating is not on, and depends on witnessing the action of `making it so'.
\end{note}

\begin{note}[`Available resources']
  Delicate.
  Focus is on the witnessing event.
  However, mere possibility isn't sufficient for \aben{the}.
  So, some restriction.
  That is, an account of what makes the witnessing event a \emph{potential} event rather than a \emph{possible} event.
  One way to express this idea is that included in appeal to potential witnessing event is that sufficient resources are available.
  Here, the idea is that nothing further is required for the event to take place.

  This redescription falls short of an analysis as we've shifted the work from `potential' to `available'.
  Still, room for an analogy.
  Consider running a 5K.
  Here, going to require a whole bunch of energy.
  The agent does not `have' the energy.
  However, resources to generate energy.
  Fat reserves, muscle density, and so on.
  In this sense, sufficient resources are available, but not something the agent has.

  \AR{}, whatever it is that generates the sufficient resources.
  \WR{}, the result of having generated the sufficient resources.

  So, the difference between \AR{} and \WR{} isn't necessarily with what the two interpretations reduce down to, but is rather a difference with respect to what the interpretations focus on.
  From \AR{}, the stuff that's true right now, the generator, does the work.
  From \WR{}, it's what will be generated.

  There's still an important difference, though.
  Our interest is in reasoning.
  We are interested in what the agent appeals to.

  Key difference.
  \AR{}, that there is stuff the agent has which will generate.
  \WR{}, that what is generated from the stuff the agent has will do the work.

  The impact of this distinction will be expanded up with respect to \gsi{}.
\end{note}


\subsubsection{Contrasting \AR{} and \WR{}}
\label{sec:contrasting-ar-wr}

\begin{note}[Difference]
  With \AR{} the important thing is some (possibly complex) property.
  With \WR{} the important thing is the witness.

  \[\text{Has}(S,\text{docs}) \land \text{Sufficient-to-show-guilt-of-defender}(\text{docs})\]

  Ability here is to ensure that there is some property of this kind.

  \[\exists e(\text{Calculating}(e) \land \text{agent}(e) = S \land \text{result}(e) = (243 \div 3 = 8))\]


  \AR{} focuses on whether something is true of the agent independent of what action they may perform.
  \(\phi\) follows from property.
  \WR{} focuses on an action the agent may perform.
  \(\phi\) follows from relation between \(\phi\) and possible witness of action.
\end{note}

\begin{note}
  Given \AR{}, conjecture that ability is not important for the entailment if complex.
  A useful shorthand, but in principle do not need to highlight the act.
  In contrast, the act is required for \WR{}.

  Still, agent is only given information about ability, so this will remain important for reasoning.
\end{note}

\begin{note}[Plausible equivalence]
  Generally speaking, switch between the two.
  Ball under cup.
  Hear that the birds are singing.
\end{note}

\begin{note}
  Primary purpose of the distinction is to ensure that things apply to any understanding of reasoning with \aben{the}.
  Further contrast after following distinction.
\end{note}

\subsection{Reasoning \nr{} and reasoning \ur{}}
\label{sec:reas-dd-reas}

\begin{note}
  We now turn to the distinction between reasoning \emph{using reference} and reasoning \emph{not using reference}.
  Or, from now on, reasoning \ur{} and reasoning \nr{}.

  The use of (probably mistranslated) Latin is in part due to some similarity between the distinction under discussion and variety of distinctions made by using the terms `\dd{}' and `\dr{}'.
  In larger part, though, Latin is used because I had a hard time finding a combination of English terms which are not suggestive of some distinction.
  For example, you may already be considering whether `\emph{using reference}' and `\emph{not using reference}' correspond to `semantic' reasoning by way of some interpretation function and `syntactic' reasoning by rule governed syntactic manipulation.
  As with the \dd{}/\dr{} distinction, we will contrast \ur{} and \nr{} with `semantic' and `syntactic' reasoning below.
  For the moment I hope the use of Latin will allow for some flexibility.
\end{note}

\begin{note}[Plan]
  The plan for this section is as follows:
  \begin{itemize}
  \item First, situate the distinction.
  \item Background for distinction.
  \item Initial illustrations.
  \item Definitions.
  \item Illustrations for key type of reasoning.
    \begin{itemize}
    \item With logic
    \item Difference from syntax and semantics
    \item Existentials.
    \end{itemize}
  \item Contrast with \dd{}/\dr{}.
  \item Apply to ability/distinction matrix.
  \end{itemize}
\end{note}

\paragraph*{Situating the distinction}

\begin{note}
  To situate the distinction between reasoning \ur{} and \nr{} we will first outline how it applies to ideas that have already been introduced, and then note the purpose distinction in the argument ahead.
\end{note}

\subparagraph*{With respect to ideas already introduced}

\begin{note}
  In~\autoref{sec:abil-access-supp} we restricted our use of `reasoning' to involve an agent claiming support for some proposition \(\phi\) having some value \(v\) (\autoref{prop:RisTV}).
  For example, an instance of reasoning may culminate with an agent claiming support that it is true that the front door was locked, or that it is desirable that they take a walk.

  The distinction between reasoning \ur{} and \nr{} is targeted at the way in which an agent claims support that some proposition \(\phi\) has some value \(v\).
\end{note}

\begin{note}
  Further, in~\autoref{sec:cases-interest} we introduced particular instances of an agent claiming support for some conclusion which involved two key steps.
  The reasoning, in outline:
  \begin{enumerate}[label=\arabic*., ref=(\arabic*)]
  \item\label{NUR:ro:i} I have some general ability \(\gamma\).
  \item\label{NUR:ro:ii} If I have general ability \(\gamma\) then I have some specific ability \(\varsigma\) to \emph{V} that \(\phi\).
  \item\label{NUR:ro:iii} I have the specific ability \(\varsigma\) to \emph{V} that \(\phi\). \hfill (From~\ref{NUR:ro:i} and~\ref{NUR:ro:ii})
  \item\label{NUR:ro:iv} It is only possible to \emph{V} that \(\phi\) if \(\phi\) is already the case.
  \item\label{NUR:ro:v} \(\phi\) is the case. \hfill (From~\ref{NUR:ro:iii} and~\ref{NUR:ro:iv})
  \end{enumerate}

  The two key steps are from~\ref{NUR:ro:i} and~\ref{NUR:ro:ii} to~\ref{NUR:ro:iii} and from~\ref{NUR:ro:iii} and~\ref{NUR:ro:iv} to~\ref{NUR:ro:v}.

  The first key step involves the conditional of~\ref{NUR:ro:ii}, termed `\gsi{-}', and clarified in~\autoref{sec:type-scenario}.

  The second key step involves the conditional of~\ref{NUR:ro:iv}, termed `\aben{an}', and clarified in~\autoref{sec:ability-entailment}.

  Both these steps involve conditionals, and hence using the conditional to claimed support for the consequent of the conditional given claimed support for the antecedent of the conditional.
  And, the distinction between reasoning \ur{} and \nr{} is of interest to use with respect to these two instances of claiming support in particular.

  Finally, in~\autoref{sec:wr-ar} we introduced two (schematic) interpretations of ability --- \AR{} and \WR{} --- and as both key steps appeal to ability, the distinction between reasoning \ur{} and \nr{} will further inform how these interpretations of ability function in an instance of reasoning.
\end{note}

\begin{note}
  To summarise, from what we have seen, then, the distinction between reasoning \ur{} and \nr{} is targeted at the way in which an agent claims support that some proposition \(\phi\) has some value \(v\).
  So, the distinction is often interest when applied to the instance of reasoning which is the focus of this paper, and, hence, different ways in which the instance of reasoning may be understood.
\end{note}

\subparagraph*{With respect to the argument ahead}

\begin{note}
  Now, with the application to previously discussed ideas in hand, the distinction has to key purposes looking ahead:

  In~\autoref{sec:first-conditional} the distinction will separate an interpretation of the instance of reasoning which is incompatible with \ESU{} (\ur{}) form an interpretation which is compatible (\nr{}).

  And, in~\autoref{sec:second-conditional} the distinction will separate an interpretation of the instance of reasoning which is incompatible with \nI{} (\nr{}) form an interpretation which is compatible (\ur{}).
\end{note}

\paragraph*{Basic distinction}

\begin{note}[Reasoning]
  As noted, reasoning involves claiming support that a proposition has some value.

  Some propositions and parts of proposition, when appealed to or used in reasoning, \emph{refer}.

  In the two quick examples given above, propositions are that the front door was locked and that the agent takes a walk.
  When part of some instance of reasoning, propositions are about what the status of some particular door was, and a type of action that some particular agent may take.
  In these examples, the agent performing the reasoning for which the propositions are a part is aware of what they refer to.
  `The front door' is the front door of the agent's house, and `they' is either the agent or some close acquaintance.

  This is not always the case.

\end{note}

\begin{note}[Quick clarification]
  The stronger claim that propositions, when part of reasoning, refer is not without issue.
  Let me illustrate with a quick argument.

  \begin{itemize}
  \item All propositions refer.
  \item A proposition is something that may be assigned value.
  \item One such value is truth.
  \item Some conditionals are true.
  \item Some conditionals refer.
  \end{itemize}

  Three intuitive statements about propositions which combined with strong assumption lead to a problem.

  Settling whether or not conditionals (for example) refer --- or what conditionals refer to if the do refer --- is of no real interest for present purposes.
  Argument should go through either way.
  Trouble is I would to be committed to something quite strong.

  Would be nice to narrow down the appropriate sense of proposition.
  However, task with little reward.

  Nothing depends on borderline cases.

  Conditionals in which antecedent and consequent refer are viewed in terms of information about reference.

  In sort, what we're interested in those propositions which do refer.

  I had to unlock the door, so the door was locked.
  I am feeling stressed, so it is desirable that I take a walk.
\end{note}

\paragraph*{Illustrations for intuition}

\begin{note}[Illustration]
  Before defining \ur{} and \nr{}, let us work up some intuition by re-examining an instance of claiming support.

  \begin{quote}
    From claimed support that a rectangle measures \(19\text{cm}\) by \(7\text{cm}\), an agent claims support that the area of the rectangle is \(133\text{cm}^{2}\).
  \end{quote}
  In \autoref{ill:rectangle:basic} measurement, understanding of how to calculate the area of a rectangle, and some grasp of mathematics.\nolinebreak
  \footnote{
    In \autoref{ill:rectangle:ability} applied to ability, but not interested in that here.
  }
  The following two illustrations detail two different ways in which the claim support.

  The purpose of the pairing is to help develop some intuition for the ways in which an agent may or may not appeal to or use the referent of a proposition when reasoning.
  Beyond this, the subject matter and steps of reasoning hold no (direct) interest.

  Both illustrations start with a the same premise --- the rectangle measures \(19\text{cm}\) by \(7\text{cm}\) --- and end with the same conclusion --- the area of the rectangle is \(133\text{cm}^{2}\).

  The key difference between the two illustrations is how the agent claims support for the conclusion given claimed support for the premises.

  In the first illustration the agent will refer to a particular rectangle throughout the intermediate reasoning.
  And, by contrast, in the second the agent will not refer to a particular rectangle throughout the intermediate reasoning.
\end{note}

\begin{note}[]
  \begin{illustration}\label{ill:rectangle:ur}
    \vspace{-\baselineskip}
    \begin{enumerate}[label=\(\protect\tBox\)\space\arabic*., ref=(\(\protect\tBox\)\space\arabic*), align=left, leftmargin=*]
    \item[\(\protect\tBox\)\space P.]\label{tB:measure} This rectangle measures \(19\text{cm}\) by \(7\text{cm}\).
    \item\label{tB:width} Width is \(19\text{cm}\), so divide into \(19\) columns containing some number of unit squares, where unit is a centimetre.
    \item\label{tB:height} Height is \(7\text{cm}\).
    \item\label{tB:counting} So, there are seven unit squares in each column.
    \item\label{tB:total} Therefore, total of \(133\) square centimetres.
    \item[\(\protect\tBox\)\space C.]\label{tB:conclusion} Hence, the area of this rectangle is \(133\text{cm}^{2}\).
    \end{enumerate}
    \vspace{-\baselineskip}
  \end{illustration}

  The reasoning of~\autoref{ill:rectangle:ur} somewhat stilted in style, but plausible.

  The agent understands how to calculate the area of a triangle and applies the calculation to the specific rectangle.
  Key is that at no point does the agent abstract from the particular rectangle to an arbitrary rectangle with the same dimensions --- steps~\ref{tB:width} to~\ref{tB:total} are about the specific rectangle.
  Of course, the same reasoning may be applied to any rectangle with the same dimensions, but steps~\ref{tB:measure} to \ref{tB:conclusion} refer to \emph{that} particular rectangle, and the agent claims support because of what they have established about that triangle.

  In this sense, the propositions concerning rectangles refer, and the agent appeals to or uses the referent of those propositions to claim support.

  The `\(\tBox\)' prefix for each step is designed to indicate that the agent is reasoning about a particular object throughout.

  \begin{illustration}\label{ill:rectangle:nr}
    \vspace{-\baselineskip}
    \begin{enumerate}[label=\(\protect\tBoxd\)\space\arabic*., ref=(\(\protect\tBoxd\)\space\arabic*), align=left, leftmargin=*]
    \item[\(\protect\tBoxd\)\space P.]\label{tBd:measure} This rectangle measures \(19\text{cm}\) by \(7\text{cm}\).
    \item\label{tBd:calculate} Calculate the area of any two-dimensional object, take length and width in a common unit and multiply together to get area in common unit squared.
    \item\label{tBd:abstract} So, if object with \(19\) and \(7\), then area is \(19 \times 7\) cm2.
    \item\label{tBd:instantiate} Put together.
    \item[\(\protect\tBoxd\)\space C.]\label{tBd:conclusion} Hence, the area of this rectangle is \(133\text{cm}^{2}\).
    \end{enumerate}
    \vspace{-\baselineskip}
  \end{illustration}

  As with~\autoref{ill:rectangle:ur}, the reasoning of~\autoref{ill:rectangle:nr} is also somewhat stilted in style, but plausible.

  The agent understands how to calculate the area of a rectangle and although they conclude by claiming support with respect the specific rectangle, the agent quickly abstracts to any rectangle with the same dimensions.

  Key is that the agent abstracts from the particular rectangle to an arbitrary rectangle with the same dimensions for the most part of their reasoning --- steps~\ref{tBd:calculate} to~\ref{tBd:abstract} are not about any specific rectangle.

  In this sense, the propositions may refer, as they apply to any rectangle with the respective dimensions, but the agent neither appeals to nor uses possible referents of those propositions to claim support at key steps of the reasoning.

  The `\(\tBoxd\)' prefix for each step is designed to indicate that the agent is only sometimes reasoning about a particular object.
\end{note}

\begin{note}[Propositions, recap]
  {\color{red} \autoref{prop:RisTV} has reasoning as establishing a value}
  Two ways of tracing value.
  With and without reasoning about what the constituents of the proposition refer to.\nolinebreak
  \footnote{
    Footnote on Russel, Frege, etc.
  }

  Here, truth conditions, but more general evaluation.
  I.e.\ truth isn't the only evaluation of interest.

  Here, truth conditions.
  However, value.
  So, truth conditions + value, reasoning about something.
  Distinction is with respect to how the agent goes about reasoning about the something.

  Have the idea of a proposition.
  Something which gets a value.
  Reason in terms of preservation of value.

  
\end{note}

\begin{note}[The distinction]
  {
    \color{red}
    This is leading to a distinct between `there is a' and `the'.
    Though, that's just a gloss.
  }
  Focus on deductive reasoning.
\end{note}

\paragraph*{Definitions}

\begin{note}
  Have basic distinction, and a pair of illustrations.

  Following, apply this distinction to additional illustrations for which reasoning involved which will resemble reasoning of interest with respect to  ability scenarios.

  First, though, summarise the ideas by stating definitions.
\end{note}

\begin{note}[A pair of definitions]
  \begin{definition}[\ur{}]
    \vspace{-\baselineskip}
    \begin{itemize}
    \item Let \(t\) be some thing, and
    \item Let \(S\) be some step of reasoning that involves appeal to claimed support for propositions \(\psi_{1},\dots,\psi_{n}\) to claim support for some proposition \(\phi\).
    \end{itemize}

    The step of reasoning \(S\) is \ur{} with respect to \(t\) if the agent \emph{appeals to} \(t\) when claiming support for \(\phi\) by \(S\).
  \end{definition}

  \begin{definition}[\nr{}]
    \vspace{-\baselineskip}
    \begin{itemize}
    \item Let \(t\) be some thing, and
    \item Let \(S\) be some step of reasoning that involves appeal to claimed support for propositions \(\psi_{1},\dots,\psi_{n}\) to claim support for some proposition \(\phi\).
    \end{itemize}

    The step of reasoning \(S\) is \nr{} with respect to \(t\) if the agent \emph{does not appeal to} \(t\) when claiming support for \(\phi\) by \(S\).
  \end{definition}
\end{note}

\begin{note}
  Respective definitions differ only with respect to whether appeals --- or does not appeal to --- the referent of the proposition, or part.

  So, initial discussion of definitions covers both.

  Start with conditions.

  \(t\) fix some thing.
  Any thing, really.
  Possible referent of a referential term.
  Reasoning is complex, plausible that any step may involve a complex of factors.
  Only interested in specific terms, so definition avoids difficulties with classifying steps of reasoning as a whole.

  Don't require that \(t\) occurs, to keep things simple.
  Implicit that \nr{} with respect to \(t\) if \(t\) does not occur.

  However, possible for \(t\) to be in either premise(s) or conclusion of step.

  \(S\)
  When claiming support for some proposition from some other proposition.
  So, definitions cover a single step of reasoning --- not interested in classifying anything broader.

  Possible to claim support for a proposition from something non-propositional.
  However, only interested in proposition-to-proposition case, so this restriction is fine.

  Existential.
  Key idea is that \ur{} applies whenever appeals to referent.
  The agent claim support in this way.

  Finally, broad point.
  Claiming support.
  Doesn't say that this is successful, that there is not some other way, in particular going by \nr{} or \ur{} instead.
\end{note}

\begin{note}[Simpliciter]
  Definitions are with respect to some thing.
  That's what we're interested in.
  Further, they don't characterise steps of reasoning in general.
  Hence, no immediate (at least) consequences for what is involved in a step of reasoning.
  For example, possible \ur{} with respect to some thing and \nr{} with respect to some other thing.
\end{note}

\begin{note}[The distinction and entailment]
  The purpose of these definitions is to capture when the agent claim support by appeal to some thing.

  Applied to rectangles.
  \autoref{ill:rectangle:ur} as the agent references the particular triangle throughout the reasoning.
  So, \ur{} holds for each step with respect to the rectangle.
  The way in which the agent claims support is such that the rectangle does work.


  \nr{} holds for steps~\ref{tBd:calculate}~to~\ref{tBd:instantiate} of~\autoref{ill:rectangle:nr}.


  \emph{Why} something follows from something else.

  With \ur{} get an argument where the agent ensure that the reference for the consequence works out.
  This is the one to start with.
  The argument is that things \emph{are} a certain way, so to speak.
  That is, reasoning works out because reference works out.


  \nr{} doesn't do this.
  No reference.
  So, transformation to the information that the agent already has.
\end{note}

\begin{note}[To keep in mind]
  The distinction between \ur{} and \nr{} is not about the presence (or absence) of referential terms.
  Nor that the agent may claim support by noting that a term refers.

  In rectangle, important that refers to any rectangle.

  Main interest with this distinction will be whether agent appeals to referent of a referential term, so to speak.
\end{note}

\paragraph*{Illustrations for interest}

\begin{note}
  We now turn to additional illustrations.

  Two goals.

  First, relate to familiar stuff.
  Difference from distinction from syntactic and semantic reasoning.

  Second, to consider reasoning with existentials.
  This, then expanded on when turn to application of the distinction.

  Start with simple case.
  Then, turn to existentials.

  Get
  More familiarity with distinction.
  Additional considerations.

  Focused on deductive when giving examples.
  However, same applies to other kinds of reasoning.
\end{note}

\begin{note}[Instance of reasoning]
  The scenarios of interest:
  \begin{quote}
    From claimed support that a dog is \RIPa{} and \RIPb{}, an agent claims support that the dog is \RIPb{}.
  \end{quote}

  In outline, the reasoning is straightforward:
  \begin{quote}
    \begin{itemize}
    \item[P.] \nagent{10} is \RIPa{} and \RIPb{} dog.
    \item[---.] So, \nagent{10} is \RIPa{} dog and a \RIPb{} dog.
    \item[C.] Hence, \nagent{10} is a \RIPb{} dog.
    \end{itemize}
  \end{quote}

  Though straightforward, the reasoning is not completely trivial.
  `\RIPa{} and \RIPb{}' is the combination of two adjectives --- `\RIPa{}' and `\RIPb{}', respectively.
  And, it is not always the case that the attribution of a combination of adjectives to an object allows the attribution of the separate adjective.

  For example, it does not follow from the picture being black and white that the picture is white.\nolinebreak
  \footnote{
    You may prefer `black-and-white'.
    If so, I suggest `\RIPa{}-and-\RIPb{}'.
  }
  Nor does it follow from the weather being cloudy and sunny that the weather is sunny.

  And, of course, this isn't unique to `and' and adjectives.
  It does not follow from the transit time being an hour and five minutes that the transit time is five minutes.
  Nor does it follow from book being written by A and B that the book was written by B.

  The point, though a minor one, is that `and' behaves in a variety of ways and so some reasoning is required to move from the premise to the conclusion.
  \nolinebreak
  \footnote{
    Compare to obedient and \RIPb{}.
    May think obedience restricts \RIPb{}.
    Obedient and \RIPb{}, but not \RIPb{}.
    Not to say that the meaning of `and' works the same in both constructions.
    However, enough to require that syntax should be respected.
  }
\end{note}

\begin{note}
  Outline of reasoning.

  Focus for the moment is on relation to syntactic and semantic.

  
\end{note}


\begin{note}[Example, \nr{}]
  Recall, \nr{}.

  Idea is to illustrate this kind of reasoning in terms of transformations.
  Following steps in the first-order language.

  \begin{illustration}\label{ill:dog:C:nr}
    \vspace{-\baselineskip}
    \begin{enumerate}[label=\(\protect\iDogd\)\space\arabic*., ref=(\(\protect\iDogd\)\space\arabic*), align=left, leftmargin=*]
      % \item[\(\protect\iDogd\)\space P.] The dog is \RIPa{} and \RIPb{}.
    \item\label{ill:iDogd:abd} \(\text{\RIPa{-} \& \RIPb{} dog}(w)\)
    \item\label{ill:iDogd:sep-gen} \(\forall x(\text{\RIPa{-} \& \RIPb{} dog}(x) \rightarrow (\text{\RIPa{-} dog}(x) \land \text{\RIPb{-} dog})(x))\)
    \item\label{ill:iDogd:sep-app} \(\text{\RIPa{-} \& \RIPb{} dog}(w) \rightarrow ((\text{\RIPa{-} dog}(x) \land \text{\RIPb{-} dog})(w))\)
    \item\label{ill:iDogd:sep-con} \(\text{\RIPa{-} dog}(w) \land \text{\RIPb{-} dog}(w)\)
    \item\label{ill:iDogd:done} \(\text{\RIPb{-} dog}(w)\)
      % \item[\(\protect\iDogd\)\space C.] Hence, the dog is \RIPb{}.
    \end{enumerate}
    \vspace{-\baselineskip}
  \end{illustration}
\end{note}

\begin{note}[Main point]
  Moving between the steps by rules applied that do not depend on interpreting predicates and constants.
\end{note}

\begin{note}[Background]
  Key point here is that we have some background.

  Understand the syntax, and understand the intended interpretation of the syntax.
  `\RIPa{-} and \RIPb{} is a predicate, and applied to some constant `\(w\)' (abbreviating `\nagent{10}').
  For the moment, all that matters is that these are predicates and constants.

  Form of the argument.
  Two assumptions.
  Three premises obtained by applying rules following main connective of previous premise.
  In order, universal quantifier, conditional, conjunction.
\end{note}

\begin{note}[Walk-through]
  Walk through these rules.
  Observation is that each instance conforms to reasoning \nr{}.

  First step, from \ref{ill:iDogd:sep-gen} to \ref{ill:iDogd:sep-app} instance of universal instantiation.
  Applied to \(w\) because \(w\) is a constant.
  In order to apply this rule, it doesn't matter what \(w\) refers to, nor what the predicates involved refer to.
  Universal instantiation allows any constant.

  From \ref{ill:iDogd:sep-app} to \ref{ill:iDogd:sep-con} instance of conditional elimination.

  And, from \ref{ill:iDogd:sep-con} to \ref{ill:iDogd:done} conjunction elimination.

  If I may, I encourage you to check that the argument is valid.
  The task is instructive because in doing so you too will abstract away from the intended interpretation of the terms.
\end{note}

\begin{note}[Two things]
  Two questions here.

  Question about how to obtain \ref{ill:iDogd:abd} from initial premise, and how to obtain conclusion from \ref{ill:iDogd:done}, but there doesn't seem to be mystery about what is going on.
  Same as with illustration above.
  Move from measurements regarding a particular rectangle to the manipulation of symbols which may be given an interpretation.
  Here, instead of mathematics, we have first order logic.

  \ref{ill:iDogd:sep-gen}.
  Granted two assumptions, valid.

  Key thing is choice of predicates.
  Designed to ensure that the argument is sound when applied given intended interpretation.

  It's also not clear that any dog which is \RIPa{} and \RIPb{} is so independently of it being a dog.
  A \RIPb{} dog need not be \RIPb{} in the same way a small elephant need not be small.

  These aren't distractions for the sake a pedantry.

  Reasoning proceeds by applying rules, regardless of reference.
  But, want soundness so that they may be applied without reference.
  \nr{} is about the reasoning that takes place, but it doesn't hold that reference is irrelevant to reasoning.


  Of course, it's fair to say that the equivalence only holds because of what `\RIPa{}' and `\RIPb{}' refer to.
  However, key is that once the rule has been obtained an agent may reason about the terms, rather than reasoning about \RIPa{}ness and \RIPb{}ness.


  Well, 
\end{note}

\begin{note}[Summarise]
  Premise and conclusion about a particular dog, \nagent{10}.
  However, intermediary steps are instances of reasoning \nr{} as the agent does not go by referent.
  All the agent is concerned with is the logical form, so to speak.

  However, stress that although the steps of reasoning are independent of reference, it may still matter that parts refer.
  \nagent{10} may not feature, but that propositions are about \nagent{10} when given intended interpretation may matter.
\end{note}

\begin{note}
  Contrast to reasoning \ur{}.

  Following illustration traces follows the same general pattern as~\autoref{ill:dog:C:nr}.
  Difference, roughly stated, is that the agent will reason about \nagent{10}, \RIPa{} and \RIPb{} dogs, and so on.
\end{note}

\begin{note}[Example, \ur{}]

  \begin{illustration}
    \vspace{-\baselineskip}
    \begin{enumerate}[label=\(\protect\iDog\)\space\arabic*., ref=\arabic*, align=left, leftmargin=*]
    \item\label{ill:iDog:mixed} \nagent{10} is a big and playful dog.
    \item\label{ill:iDog:sep-gen} The thing of being a \RIPa{} and \RIPb{} dog is just the combination of being a \RIPa{} dog and being a \RIPb{} dog.
    \item\label{ill:iDog:sep-app} \nagent{10} being a \RIPa{} and \RIPb{} dog is the case when \nagent{10} is both a \RIPa{} dog and a \RIPb{} dog.
    \item\label{ill:iDog:sep-con} \nagent{10} is a \RIPa{} dog and \nagent{10} a \RIPb{} dog.
    \item\label{ill:iDog:sep-res} \nagent{10} is a \RIPb{} dog.
    \end{enumerate}
    \vspace{-\baselineskip}
  \end{illustration}
\end{note}

\begin{note}
  Natural language to ease discussion.
  Could go with semantic interpretation.\nolinebreak
  \footnote{
    Follow convention and use \(\sem{--}\) to capture the reference of some term `\(\text{--}\)'.
    \begin{illustration}
      \vspace{-\baselineskip}
      \begin{enumerate}[label=\(\protect\iDog\)\space\arabic*., ref=\arabic*, align=left, leftmargin=*]
      \item \(\sem{w} \in \sem{\RIPa{-} \& \RIPb{} dog}\)
      \item \(\sem{\RIPa{-} and \RIPb{} dog} \subseteq (\sem{\RIPa{-} dog} \cap \sem{\RIPb{-} dog})\)
      \item \emph{If} \(\sem{w} \in \sem{\RIPa{-} \& \RIPb{} dog}\), \emph{then} \(\sem{w} \in \sem{\RIPa{-} dog}\) \emph{and} \(\sem{w} \in \sem{\RIPb{-} dog}\)
      \item \(\sem{w} \in \sem{\RIPa{-} dog}\) and \(\sem{w} \in \sem{\RIPb{-} dog}\)
      \item \(\sem{w} \in \sem{\RIPb{-} dog}\)
      \end{enumerate}
      \vspace{-\baselineskip}
    \end{illustration}
    Imports set theoretical background from first order logic.
    Constants are individuals, predicates are treated extensionally.
    Read simply, \(\sem{The dog} \in \sem{\RIPa{-} and \RIPb{}}\) may suggest that the agent is reasoning that the reference of the term `the dog' is a member of the reference of `\RIPa{} and \RIPb{}'.

    \emph{\nagent{10}} is such that they are a \emph{\RIPa{} and \RIPb{}} dog.
  }
\end{note}

\begin{note}
  \ref{ill:iDog:mixed} meet \nagent{10} for the first time.
  \nagent{10} is playing with a ball.
  Person who takes care of \nagent{10} remarks that they're a \RIPa{} and \RIPb{} dog.
  You wonder about adjectives, what's being communicated here.
  Think about these two adjectives, \ref{ill:iDog:sep-gen}.
  Of course, apply to \nagent{10}, \ref{ill:iDog:sep-app}.
  And, observed, so \ref{ill:iDog:sep-con}.
  Hence, \ref{ill:iDog:sep-res}.
\end{note}

\begin{note}
  \ur{} with respect to \nagent{} and properties, roughly.
  So, four things.
  \nagent{10}, \RIPa{} \& \RIPb{}, \RIPa{}, and \RIPb{}.

  Throughout, properties.

  Variations.
  Focus on the adjectives and how these relate to properties, or abstract to concepts.

  Key point is that \ur{}.
  Agent is reasoning about some things, and claiming support by appeal to those things.
\end{note}

\begin{note}
  As with first pair of illustrations, claiming support by reference to something.
  This is all the distinction amounts to.
  Broader, applied to different kinds of reasoning, and differences that follow from these two kinds of reasoning.
\end{note}

\begin{note}[Difference]
  First illustration, logical structure of propositions.
  Don't need to appeal to reference.
  Logical structure for convenience.
  Claim support by transformations.

  Second illustration, surface presentation remains similar, but properties.

  \mom{}.
  First, \nagent{10} and transformations.
  World appears this way, and reasoning is independent.

  Second, \nagent{10} and understanding of properties involved.
  No reason to think that properties have been thought about badly, and testimony or observation seems good enough.
  World appears this way.
\end{note}

\begin{note}
  Overall similarity.
  Designed so that the explanation of the steps applies to both illustrations.
  Steps are missing from outline of reasoning present.
  Nothing about presentation.
  Indeed, second as semantic counterpart to syntactic reasoning of first.

  Alternatively, formal and non-formal.

  Something about this is right, but it's not quite right.
  Hopefully overlap will lend some clarity.
\end{note}

\paragraph*{Two related ideas}

\begin{note}
  Two related distinctions.
  Overlap and differences.
  Not an exhaustive discussion, just enough to identify similarities and differences.

  Similarity, then difference.
\end{note}

\begin{note}
  Should already be cautious.
  \ur{} and \nr{} is about whether some thing is appealed to when taking a step.
  It not obvious that anything else follows about the step, which is what the distinctions apply to.

  It's not obvious, but not immediate either.
  Still, difficult to work with.
  So, let's simplify for the moment.

  \begin{definition}
    Let \(S\) be some step of reasoning that involves appeal to claimed support for propositions \(\psi_{1},\dots,\psi_{n}\) to claim support for some proposition \(\phi\).

    The step of reasoning \(S\) is:
    \begin{itemize}
    \item \ur{} \emph{simpliciter}, if there is something thing for which \(S\) is \ur{} with respect to.
    \item \nr{} \emph{simpliciter}, otherwise.
    \end{itemize}
    \vspace{-\baselineskip}
  \end{definition}
  Now, apply to steps of reasoning as a whole.
\end{note}

\begin{note}[Issues]
  \nr{} for both sides of consequence.
  So, \nr{} does not imply syntactic reasoning.
  Still, seems formal then \nr{}.

  However, \nr{} does not imply formal reasoning.
\end{note}

\subparagraph*{Syntactic and semantic perspectives on logical consequence}

\begin{note}
  Similarity: well, syntactic transformation and something involving reference.

  Quick point is that role of reference in semantic perspective of logical consequence is different from \ur{}.

  Given some background, this is somewhat obvious.
  It's still logical consequence that we're talking about.
\end{note}

\begin{note}
  Using first order logic.
  Further, appeal to logical consequence.

  Presentations given are valid.

  Suggestion that the distinction I've been relying on to illustrate \ur{} and \nr{} just is the distinction at issue.

  On the one hand, syntactic reasoning and on the other semantic.

  Understood distinction between syntactic and semantic accounts of logical consequence.

  This is not to say that logical consequence is at issue.
  Problem with logical consequence is that it's always independent of content.
  This is ~\cite{Etchemendy:1990wo,Etchemendy:2008wz}.

  Instead, difference is the type of reasoning.
  E.g. the basics of the representational or interpretational approaches.

  Thing here is that when applied to logical consequence these approaches aren't really \ur{}.
  It's not about what the terms refer to, but possible referents of the terms.

  This is really important.
  \nr{} doesn't say that reference isn't relevant.
  And, get to the end of some semantic reasoning and it's possible to construct a counterexample.
  Reference is important.
  But, there's no need to appeal to that counterexample.

  Still, because semantics is about reference, get instances of \ur{} which follow semantic consequence.
  Every semantic consequence will lead to a valid instance of \ur{}, because here we're just fixing on one particular interpretation.

  Converse does not seem to hold.
  Consider illustration again.
  Middle steps are there to ensure that no matter the reference.
  But, as it's \nagent{10} it seems these aren't required.
  It's a logical consequence, but it's not at all clear that these steps are required.


  This is why syntax versus semantics as applied to logical consequence is tricky.
  Logical consequence is it's own thing.
  Two different interpretations of this.
  However, same consequences.
  And, the difference between illustrations is in part a difference in consequence.

  Really, we're looking at two 

  The point, really, of \ur{} is that the agent is not claiming support for the conclusion by appeal to logical consequence.
\end{note}

\begin{note}[Quick syntax vs semantics]
  Quick distinction is between syntactic and semantic reasoning.\nolinebreak
  \footnote{
    Avoid using term `logical form' as this doesn't distinguish between the two different things.
  }
  Familiar distinction.
  Consequence of reference.

  Two issues.
  First, background system.
  Second, and more important, suggests absence of reference.
  However, quite possible that the agent requires that some term refers.

  Note, also distinction doesn't rely on views of reference.
  Possible to have a view where reference is by the agent, or a view where reference is by language.
\end{note}

\begin{note}[Avoid syntax semantics terminology]
  Don't use syntax/semantics terminology because the focus of the contrast is reference, rather than two `systems' that falls from the contrast.
  Indeed, syntax/semantics requires a background system.
  Won't stray from first order logic, or at least reasoning that may be captured by first order logic.
  So, free to use this terminology is you prefer.
  However, caution, as will be seen in the following illustration.
\end{note}

\subparagraph*{Formal and non-formal reasoning}

 \cite{Beall:2019ty}

\begin{note}
  Above, different kinds of consequence.
  Issue was logical consequence.

  So, perhaps formal and non-formal.\nolinebreak
  \footnote{
    Distinct from informal logic~\textcite{Groarke:2021tk}.
  }

  Instances of \nr{} considered arithmetic and first order logic.
  Formal systems.
  Appeal to something beyond form(al system).

  Intuitive distinction, but beyond this I'm not sure how exactly to characterise the distinction.

  Well, seems that this distinction fits well.
  If logical is just no particular reference, then we've got a nice distinction.
\end{note}

\begin{note}[`Formal' and \nr{}, `non-formal' and \ur{}]
  Basic idea.
\end{note}

\begin{note}[Two cases]
  \begin{itemize}
  \item `Formal' and \ur{}: nonstandard models. Reasoning about formal properties of natural numbers. Just because these don't uniquely identify doesn't prevent this.
    However, still non-formal in the sense that the agent goes beyond formalism.
    So, this does end up being non-formal in a sense.
  \item `Non-formal' and \nr{}: time
  \end{itemize}
\end{note}

\begin{note}[`Formal' and \ur{}: nonstandard models]
  Example.

  For an illustration being put to use in a different context, consider intended and non-standard models.
  Specifically, with respect to arithmetic.
  Intended model example.

  The Peano Axioms are sound with respect to the natural numbers, addition, and multiplication.
  However, the Peano Axioms are also sound with respect to various non-standard structures.
  For example, by considering an object \(\omega\) which is larger than any natural number.

  So, there's a barrier to claiming that reasoning \nr{} is also reasoning \ur{}.
  If reasoning about the natural numbers, then reasoning \ur{}, because reasoning \nr{} does not do enough to fix on the natural numbers.

  More generally, incompleteness.
  (Nonstandard constructions based on incompleteness too.)

  From soundness, things still hold up.
  However, clear that there's a distinction between whether reference is being used.

  This is the thing that's important.

  Still, some caution.
  The way we've drawn the distinction requires some subtlety.
  For, it is a case of \nr{} for the agent to reason that `one' refers to one.
  Hence, nonstandard models don't follow from \nr{} reasoning alone.
  Still, follows that claiming support would not distinguish.
  And, that's where the key issue is.
\end{note}

\begin{note}[`Non-formal' and \nr{}: time]
  Well, rectangle.

  Though, this is a little tricky.

  So, reading the time from a clock.
\end{note}



\paragraph*{Existential illustrations}

\begin{note}[Example with an existential]
  The initial illustration was simple.
  Our interest is with something more complex.
  Existentials.

  \begin{quote}
    \begin{itemize}
    \item[P.] Some dog is \RIPa{} and \RIPb{}.
    \item[---] So, some dog is \RIPa{} and some dog is \RIPb{}.
    \item[C.] Hence, some dog is \RIPb{}.
    \end{itemize}
  \end{quote}
\end{note}

\begin{note}[Exists \nr{}]
  \nr{} is kind of straightforward.
  Not going to think about what this \(x\) refers to, just manipulate like any other individual.

    \begin{illustration}\label{ill:dog:E:nr}
    \vspace{-\baselineskip}
    \begin{enumerate}[label=\(\protect\iEDogd\)\space\arabic*., ref=(\(\protect\iEDogd\)\space\arabic*), align=left, leftmargin=*]
      % \item[\(\protect\iDogd\)\space P.] The dog is \RIPa{} and \RIPb{}.
    \item\label{ill:iDogd:E:abd} \(\exists x \text{\RIPa{-} \& \RIPb{} dog}(x)\)
    \item\label{ill:iDogd:E:sep-gen} \(\forall x(\text{\RIPa{-} \& \RIPb{} dog}(x) \rightarrow (\text{\RIPa{-} dog}(x) \land \text{\RIPb{-} dog}(x)))\)
    \item\label{ill:iDogd:E:abd} \(\text{\RIPa{-} \& \RIPb{} dog}(i)\)
    \item\label{ill:iDogd:E:sep-app} \(\text{\RIPa{-} \& \RIPb{} dog}(i) \rightarrow (\text{\RIPa{-} dog}(i) \land \text{\RIPb{-} dog}(i))\)
    \item\label{ill:iDogd:E:sep-con} \(\text{\RIPa{-} dog}(i) \land \text{\RIPb{-} dog}(i)\)
    \item\label{ill:iDogd:E:inst} \(\text{\RIPb{-} dog}(i)\)
    \item\label{ill:iDogd:E:done} \(\exists x\text{\RIPb{-} dog}(x)\)
      % \item[\(\protect\iDogd\)\space C.] Hence, the dog is \RIPb{}.
    \end{enumerate}
    \vspace{-\baselineskip}
  \end{illustration}
\end{note}

\begin{note}
  \autoref{ill:dog:E:nr} is \ref{ill:dog:C:nr}.

  Steps \ref{ill:iDogd:E:abd} to \ref{ill:iDogd:E:inst} mirror the reasoning with universal quantifier, conditionals, and conjunctions as before.

  Difference is these are with respect to some fresh constant.

  Standard.
  And, \nr{} with respect to any possible referent of \(i\).
  For, as before, agent doesn't require an interpretation of the non-logical vocabulary in order to apply these rules.
\end{note}

\begin{note}[Exists \ur{}]
  \begin{illustration}\label{ill:dog:E:ur}
    \vspace{-\baselineskip}
    \begin{enumerate}[label=\(\protect\iEDog\)\space\arabic*., ref=(\(\protect\iEDog\)\space\arabic*), align=left, leftmargin=*]
    \item Some dog is \RIPa{} \& \RIPb{}.
    \item The thing of being a \RIPa{} and \RIPb{} dog is just the combination of being a \RIPa{} dog and being a \RIPb{} dog.
    \item \emph{That dog} is a \RIPa{} and \RIPb{} dog
    \item \emph{That dog} being a \RIPa{} and \RIPb{} dog is the case when \emph{that dog} is both a \RIPa{} dog and a \RIPb{} dog.
    \item \emph{That dog} is a \RIPa{} dog and \emph{that dog} a \RIPb{} dog.
    \item \emph{That dog} is a \RIPb{} dog.
    \item Some dog is \RIPb{}.
    \end{enumerate}
    \vspace{-\baselineskip}
  \end{illustration}
\end{note}

\begin{note}
  \ur{} is far more interesting.
  First order logic, update to point to something.
  From \ur{} the agent is reasoning about that thing.
  About it's \RIPa{}ness and \RIPb{}ness.
  The conclusion doesn't require further information about the referent.
  So, remains at some level of generality.

  This may seem odd.
  Agent doesn't information about how reference is resolved.
  However, no different from common cases.
  For example, `Plato', `Grice', etc.
  Existential secures a reference, and that's all that's required.
\end{note}

\begin{note}[\emph{That dog} might not exist]
  Possible that \emph{that dog} does not exist, so the agent is not appealing to some existing thing.
  Well, sure, but same problem with `Plato'.
  Existential is sufficient to claim support.

  In other respects, the same.

  Parallel to \nagent{10}.
  Appealed to \nagent{10}'s \RIPa{} \& \RIPb{}-ness to claim support for \nagent{10}'s \RIPb{}-ness.
  In the same way, claiming support for \emph{that dog}'s \nagent{10}'s \RIPb{}-ness by appeal to \emph{that dog}'s \RIPa{} \& \RIPb{}-ness.

  Appeal to \emph{that dog} and those properties have same role.
\end{note}

\paragraph*{Similarity to \dd{}/\dr{} distinction.}

\begin{note}[Similarity to \dd{}/\dr{} distinction.]
  SEP
  Semantically de re/de dicto:
  A sentence is semantically de re just in case it permits substitution of co-designating terms salva veritate.
  Otherwise, it is semantically de dicto.

  Simple distinction in terms of co-designating terms.
  Here, issue is about how some referent contributes to truth conditions of proposition.
  However, comes down to the same idea.
  Whether or not the agent puts reference to use.
  Still, there's an important difference.
  \dd{} and \dr{} seems to talk about the status of the agent's relation to things.
  In particular, about the relation between reference applied to distinct terms.
  So, in cases with existentials, the point is that the agent doesn't have any candidate for co-reference.

  It's important to keep these two things separate.
  The agent's reasoning determines \ur{} and \nr{}.
  In contrast, it is not in general possible for the agent to determine \dd{} and \dr{}.

  The \dd{}/\dr{} distinction is more about the quality of reference.
  So, agent fails to fix a transparent reference relation.
  Yet, this doesn't uncover the way in which the agent reasons.
  So,~\citeauthor{Fitch:1981vg}'s example.
  Quite possible that the agent is reasoning about Cicero, not whatever satisfies the term.
  However, \dd{} because it's not possible to substitute.

  Some clear examples.
  Difference is with how the reference may be resolved from the perspective of the agent.
  \dr{}, object.
  \dd{}, satisfying attributions.
  So, whether the referent contributes to truth conditions.
  This makes sense.
  In the case of \dr{}, yes, in the case of \dd{}, no.

  And, in turn, this is different from how the agent proceeds to reason from proposition.

  Suppose \dr{}.
  Possible \nr{}, as the agent may not appeal to reference.
  And, possible \ur{} as agent might reason about the individual.

  Suppose \dd{}.
  Possible \nr{}, same as before.
  Also possible \ur{}.
  Still \dd{} as the referent is not \emph{directly} contributing to the truth conditions.
  However, the agent is still reasoning about that thing, whatever it is.

  So, easiest when thinking about the earlier example.
  Reasoning about some number, not something which satisfies the Peano axioms.

  Now, \dd{} and \nr{} are a natural pairing.
  If no fix on reference, then don't reason with reference.
  The important thing, however, is the way in which the proposition is evaluated.
\end{note}

\begin{note}[Metaphysical distinction.]
  Same issue.
  Details about predication don't seem to be up to the agent.
\end{note}

\begin{note}[rigid and non-rigid designation]
  Also different from rigid and non-rigid designation.
  This only applies when \ur{}.
  It may seem that non-rigid and \nr{} pair up.
  But this isn't right.
  Quite possible to fix that the reference relation is rigid, but also to never use the reference relation.
  Go through any example where one can use some kind of logical reasoning.
  It's not the case that the result could vary simply because one reasoned by logical form, so to speak.
\end{note}

\begin{note}[\ESU{}]
  Important to note that there's no issue with \ESU{} yet.
  Both types of reasoning are quite compatible.
  So, this distinction is important, but nothing follows from this distinction alone.

  I mean, \nr{} is clearly fine with both.
  The only difficulty is with \ur{}.
  However, this is also fine, as \ESU{} is only about what the agent appeals to when reasoning, and it's a different claim to hold that reference either is or is not important.

  This is a distinction with a difference, but the \emph{significance} of this difference will come later.
\end{note}

\begin{note}[Examples with properties and events]
  E.g.\ Brutus hugging Caesar.

  This is also the thing about Davidsonian event semantics.
  Arguably, \emph{de constructione} with respect to the event.
  I mean, motivated by logic.
  Still, doesn't prevent \emph{de materia}.
\end{note}

\subsection{Filling in the matrix}
\label{sec:filling-matrix}

\begin{note}
  Handful of distinctions.

  Fill in the matrix.

  First, recall \gsi{}.
\end{note}

\subsubsection{Recap}
\label{sec:recap-reasoning}

\begin{note}
  Ability.
  Instance of reasoning with ability.
  Two distinctions which apply here.
  \AR{} and \WR{} for ability.
  \ur{} and \nr{} for how ability is used to claim support.
\end{note}

\begin{note}[Two ways to get to \(\phi\)]
  Two key steps.
  \begin{itemize}
  \item \gsi{}.
  \item \aben{the}.
  \end{itemize}

  First, general to specific, then specific to \(\phi\).

  Second, general to specific, then general to \(\phi\).
\end{note}

\begin{note}
  Second is something like evidence of evidence is evidence.

  Here, the important difference is that the agent only needs to appeal to general ability.
  And, they've claimed support for this.

  The point is that it's not clear the agent is required to do anything too much with the specific ability.
\end{note}

\begin{note}[Focus]
  Common here is \gsi{}.
  Needed in both cases.

  However, it's also the case that the `final' bit of reasoning is more or less \aben{the}.

  So, in a sense both ways of reasoning depend on these two things.

  The only difference is the particular for \aben{the} would take.

  There is a difference.
  For, general and specific are different.
  I'm not clear on whether this amounts to a significant difference.
  Still, even if it does, argument will proceed even if there is something significant to be made of this.
\end{note}

\begin{note}[Summarising]
  Above we introduced \gsi{}.
  Limited information of the form `If \emph{S} has a (general) ability to \(\gamma\), then \emph{S} has a (specific) ability to \emph{V} that \(\phi\) (as an instance of the general ability).'
  We then noted that certain instances of the (specific) ability to \emph{V} that \(\phi\) entail that \(\phi\) is the case.
  Two interpretations of \aben{the}, \AR{} and \WR{}.

  Our focus now turns back to \gsi{}.
  For those instances of \gsi{} when \aben{the} holds, the interpretations \AR{} and \WR{} detail what the agent obtains by reasoning from general to specific ability.
  In other words, \emph{what} the agent is claiming support for.

  As noted, using a conditional such as \gsi{} is not automatic.
  The informer has not provided the agent with any additional way to claim support that the agent has the general ability.
  Rather, outlined something that follows \emph{if} the agent has the general ability.

  So, it is up to the agent to resolve in either way.
  If the agent wants to use the information, then the agent needs to reason from general to specific.
  The issue is that without any additional reasoning, it seems there's no clear way to determine which way the agent should go.
  Here is where the distinction between \AR{} and \WR{} is important.
  Interpretation of specific ability informs how the agent move from general to specific.

  Following two propositions outline combination.
  {
    \color{red}
    The key thing here is about claiming that one has a specific ability.
  }
\end{note}

\subsubsection{\nr{}}
\label{sec:nr}

\begin{note}
  Straightforward.

  Two points.

  General to specific.
  Here, using claimed support for implication.

  \aben{the}.
  Here, using \aben{the}
\end{note}

\begin{note}[\RBV{}]
  Important thing to note here is the type of reasoning involved.

  In these cases, because the agent goes with implications, needs to go beyond claimed support.
\end{note}

\begin{note}
  Discuss this at length in~\ref{sec:second-conditional}.
\end{note}

\begin{note}
  For the moment, quick intuition.

  Go with knowledge.
  Here, using factivity.

  Point is, claimed support is consistent with not knowing.
  If don't know, then factivity isn't of any use.
  So, need to go beyond claimed support to apply factivity.
\end{note}

\subsubsection{\ur{}}
\label{sec:ur-1}

\begin{note}
  Start with a broader example.
  Programming.
\end{note}

\begin{note}
  Similar to verifying an algorithm may be implemented.
  Break down all of the steps in the algorithm, and then ensure that it is possible to express each of the steps in the programming language of choice.

  \begin{quote}
    \textsc{factorial}(\(n\)):\newline
    \textbf{if} \(n = 1\)\newline
    \mbox{}\indent \textbf{return} \(1\)\newline
    \textbf{else}\newline
    \mbox{}\indent \textbf{return} \(n \times\) \textsc{factorial}(\(n-1\))
  \end{quote}

  Fortran 77 does not support recursion, a function may not call an instance of itself.\nolinebreak
  \footnote{
    This is not to say that one may not compute factorials using Fortran 77.
    It's a Turing complete language.
    However, would require a different (non-recursive) algorithm.
  }
  By contrast, the recursive factorial algorithm may implemented in languages that support recursion, such as Lisp or Python.

  \ur{}, it's those features that are important.
  \AR{}, properties of the language, \WR{} adds in particular event.
\end{note}

\begin{note}
  Turning to reasoning, very similar idea.
  Features of programming languages are resources for doing something, in the same way that premises and steps of reasoning are resources for reaching some conclusion.
\end{note}

\begin{note}[Unified idea]
  Claim support for premises and steps of reasoning.

  Easiest with \WR{}.
  What's missing here is the use of the premises in reasoning.
  Hence, contrast to \ESU{} which we'll talk in some detail about below.
  

  Same applies to \AR{} by general property reduction.

  So, \AR{} and \WR{} allow the agent to do the same thing, but in slightly different ways.
\end{note}

\begin{note}[\gsi{}++]
  First, \gsi{} applied to \AR{}
  \begin{proposition}[\textsf{|gs-I\space·\space H|}]
    % In order for \emph{S} to have the (specific) ability to \emph{V} that \(\phi\) for which \aben{the} holds, claimed support for general and claimed support for \gsi{} are sufficient to claim support that \emph{S} has the property of being able to \emph{V} that \(\phi\).
    Suppose an agent has:
    \begin{enumerate}
    \item Claimed support for some general ability \(\gamma\).
    \item Claimed support that if they have the general ability \(\gamma\) then they have some specific ability to \emph{V} that \(\phi\) (for which \aben{the} holds).
    \end{enumerate}
    Then:
    \begin{enumerate}[resume]
    \item \emph{S} may claim support for having the specific ability \(\sigma\) by reasoning that they have the property of being able to \emph{V} that \(\phi\).
    \end{enumerate}
    \vspace{-\topsep}\vspace{-\topsep}
  \end{proposition}
  Second, \gsi{} applied to \WR{}
  \begin{proposition}[\textsf{|gs-I\space·\space W|}]\label{W:s}
    % In order for \emph{S} to have the (specific) ability to \emph{V} that \(\phi\) for which \aben{the} holds, claimed support for general and claimed support for \gsi{} are sufficient to claim support that there is a potential witnessing event in which \emph{S} \emph{V}s that \(\phi\).
        Suppose an agent has claimed support for some general ability \(\gamma\) and has claimed support that if they have the general ability \(\gamma\) then they have some specific ability to \emph{V} that \(\phi\) for which \aben{the} holds.
    Then, an agent may claim support for having the specific ability \(\sigma\) by reasoning that there is a potential witnessing event in which \emph{S} \emph{V}s that \(\phi\).
  \end{proposition}
\end{note}

\begin{note}[Alternatives]
  Appeal to premises and steps is not required by either \AR{} or \WR{}.
  However, most plausible account of what is going on.

  Explored some alternatives for \AR{}, but unclear what is of importance other than reasoning, and hence premises and steps.
  And, in this respect, basic \AR{} seems like a dead end.
  Premises and steps allow the agent to claim support in the same way as they would allow the agent to claim support when used in reasoning.
  It's not at all clear to me that basic \AR{} makes sense from this perspective.
\end{note}

\begin{note}[Limitation of intuition]
  Focused on idea that claiming support in same way as reasoning.

  This is not to imply equivalence of claimed support.

  Said too little about claimed support to make any strong remarks about equivalence.
  Still, intuitive that additional way of being \mom{}.
  For, haven't done the reasoning, so \mom{} about this.
  Not the case if the agent has done the reasoning.
\end{note}

\paragraph{Old notes}

\begin{note}[\gsi{}++ applied : \AR{}]
  \AR{} doesn't need to much expansion.
  Silent on what the property is.
  One way to view is that general ability reduces to sufficient collection of specific.
  \gsi{} conditional informs the agent that specific instance, so required for general ability.
  \gsi{} is novel, but support claimed is for quantifier over all core instances.

  Similar to a standard induction principle.

  With respect to chess, this is one such principle.
\end{note}

\begin{note}[\gsi{}++ applied : \WR{}]
  \WR{} is different.
  Witnessing event.
  So, \emph{V}ing that \(\phi\).
  Break down \emph{V}ing that \(\phi\) into a series of actions performed by the agent.
  General ability secures performing each of those actions.



  Turning to the chess example.
  Here, appeal to sufficient understanding of the rules of chess, and the combination of these.
  More broadly, premises and steps of reasoning.

  It's this kind of stuff that \WR{} uses.
\end{note}

\begin{note}[Impact of distinction]
  Return to the impact of the distinction.

  \AR{}, focus on generator.
  Hence, task is to establish that the agent has resources to generate.
  So, in a sense, with \gsi{} we go from existence of general to existence of specific generator.
  Note, this isn't to say that there's something like a general generator.
  It may be the case by general ability we have quantification over specific abilities.
  If so, then claim support that there's a particular specific generator.
  Nor that specific generators are unique.
  The available resources may overlap.
  Still, some thing that is true of the agent, and claimed support for general ability is sufficient to claim support that the `some thing' holds.

  \WR{}, focus on generated event.
  So, agent doesn't necessarily need to establish a generator, but rather ensure that event may be generated.
  Hence, \gsi{}, general ability allows the agent to generate witnessing event for specific ability.
  Not looking to claim support that `some thing' is true of the agent.
  Rather, claimed support for general ability, and appeal to the actions that this allows the agent to perform.
\end{note}

\begin{note}[Intuition for \AR{} and \WR{}]
  Both \AR{} and \WR{} are ways to understand \aben{the}, which is in turn about what is entailed by an agent having a (certain kind of) specific ability.

  \AR{} focuses on the idea that the agent may claim support from having the attribute (or the truth) of the specific ability.
  \AR{} requires support for attribute, which in turn suggests in a position to claim support for premises and steps.
  \AR{} doesn't require agent to claim support for premises and steps.

  \WR{} focuses on the idea that the agent may claim support from witnessing (or using) the specific ability.
  \WR{} requires support for premises and steps, which in turn suggests in a position to claim support for attribute.
  \WR{} doesn't require agent to claim support for attribute.
\end{note}

\begin{note}[Quite brief]
  Sketches of \AR{} and \WR{} are brief.
  Expand on these in the following sections (\ref{sec:first-conditional} and~\ref{sec:second-conditional}) to some extent, and chapter~\ref{cha:potent-infer-attr} will focus on a detailed account of both.
\end{note}

\begin{note}
  Role of ability to secure witnessing event.

  The distinction may be highlighted by a distinct set of implications\nolinebreak
  \footnote{
    Though not necessarily entailments.
  }
  \nagent{4} is dehydrated, so \nagent{4} is tired.
  \nagent{4} took a long walk in the sun, so \nagent{4} is tired.

  First, implication follows from some property.
  Second, implication follows from the result of some action.

  As with ability, both implications may be true.
  Still, difference in terms of whether one appeals to some property of \nagent{4}, or some action that \nagent{4} performed.
  As with ability, there is some ambiguity.
  There's the fact that \nagent{4} took a walk in the sun, and there's the action of \nagent{4} taking a walk in the sun.
\end{note}

\begin{note}[Why]
  So, \AR{}, some property.
  With \WR{}, it's the event that matters.
  In turn, moving from premises to some conclusion.
  Appeal to the event involves appeal to constituents of event.

  Return to \ESU{}.
  No inherent conflict with either \AR{} or \WR{}.
  Difference between property and witness.
  Requirement is that claimed support for premises is sufficient to claim support for conclusion.
  With \AR{}, claimed support for property --- need enough to be sure property is adequate.
  With \WR{}, claimed support for witness --- need enough to make sure that witness is adequate.

  \nagent{4} is thirsty, no implication.
  \nagent{4} walked, no implication.

  Does not matter that thirst is part of the relevant instance of being dehydrated.
  Nor that the witnessing event of walking was a long walk in the sun.
  Deny claimed support as did not reason from such premises.
\end{note}

\subsubsection{Summary of distinctions}
\label{sec:summary-distinctions}

\begin{note}[Table]
  \begin{figure}[H]
    \centering
    \begin{tblr}{abovesep=8pt, belowsep=8pt, width=0.95\textwidth, colspec={Q[c,m]|Q[c,m]|Q[1.8,c,m]|Q[1.8,c,m]}}
      \multicolumn{2}{c}{} & \nr{} & \ur{} \\
      \hline
      \multicolumn{2}{c}{\WR{}} & That there is an event in which \emph{S} \emph{V}s that \(\phi\) entails \(\phi\) & Details of an event in which \emph{S} \emph{V}s that \(\phi\) entail \(\phi\) \\
      \hline
      \multirow[c]{2}{*}{\AR{}} & Basic  & That \emph{S} has the ability (to \emph{V} that \(\phi\)) entails \(\phi\) & \emph{S}'s ability (to \emph{V} that \(\phi\)) entails \(\phi\)? \\
      \cline[dashed]{2-4}
      & Derived & That there is some property \emph{P} (from \emph{S} having the ability to \emph{V} that \(\phi\)) entails \(\phi\) & (Some) property \emph{P} (from \emph{S} having the ability to \emph{V} that \(\phi\)) entails \(\phi\) \\
    \end{tblr}
    \caption{Distinction matrix with \aben{the}}
  \end{figure}
\end{note}

\begin{note}
  Basic \AR{} with \ur{} has `?'.
  For, as noted above it's not clear what this amounts to.
\end{note}

\subsubsection{The distinctions are exhaustive}
\label{sec:ar-wr-are}

\begin{note}
  \color{red}
  This section is now far more straightforward.

  \nr{} and \ur{} is straightforward, this is a binary distinction.

  \AR{} and \WR{} is a little more complex.
  Ability always with respect to some action, that's a constraint on the type of ability of interest.
  So, static versus dynamic.


  And, Basic and derived are easy given \AR{}.
\end{note}

\begin{note}
  The distinction between \AR{} and \WR{} sets up two (schematic) ways in which agent an agent may claim support given an instance of \aben{the}.
  We now argue that these two (schematic) methods are exhaustive.
  {
    \color{red}
    Important to keep in mind is that our interest is with claiming support.
    And, in particular, what the agent claims support for given \AR{} and \WR{}.
    So, the claim that \AR{} and \WR{} are exhaustive is a claim about how an agent reasons, not what ability reduces to.
  }
\end{note}

\begin{note}[Exhaustive]
  \begin{proposition}\label{prop:WR-and-AR-exhaustive}\label{either-AR-or-WR}
    Any interpretations of an agent's (specific) ability to \emph{V} that \(\phi\) (for which \aben{the} holds) conforms to either:
    \begin{enumerate}
    \item \AR{}: It is a property of the agent that they are able to \emph{V} that \(\phi\).
    \item \WR{}: There is a potential witnessing event in which the agent \emph{V}s that \(\phi\).
    \end{enumerate}
    \vspace{-\topsep}\vspace{-\topsep}
  \end{proposition}
\end{note}

\begin{note}[Argument]
  \color{red}
  Start with the basics.
  Have an instance of \aben{the}.
  So, the agent claim support for \(\phi\) given claimed support for \emph{S} having the ability to \emph{V} that \(\phi\).
  So, need to argue that the agent:
  Either claims support for some property of \emph{S} (\AR{}).
  Or, claim support for \(\phi\) as the result of the event of \emph{S} \emph{V}ing that \(\phi\), with 
\end{note}

\begin{note}[Old arguments]
  {
    \color{red}
    To argue for this\dots
    I can't say that these two things are equivalent.
    I mean, these aren't equivalent.
    \AR{} gets entailment from some property, while \WR{} gets entailment from some event.
    And, no clear entailment from property to event, nor vice-versa.
  }

  Switching between ability and potential events.
  This is not important, two ways of describing the same thing.
  The ability to \emph{V} that \(\phi\) is equivalent to there being a potential event in which the agent \emph{V}s that \(\phi\).
  For, if there is no such potential event, then the agent does not have the ability to \emph{V} that \(\phi\).
  Conversely, if there is a potential event in which the agent \emph{V}s that \(\phi\), then the agent has the ability to \emph{V} that \(\phi\).
  Rather, rewriting allows us to focus on the event.

  So, we may take `There is a potential witnessing event in which the agent \emph{V}s that \(\phi\)' as canonical.
  These are simply truth conditions.
  No commitment to what there is.

  Issue is how these truth conditions function to yield the potentive witness.
  Either:
  State of affairs such that there is a potential witnessing event.
  Or, refers to the witnessing event (which is potential).

  Illustrate.
  \emph{X} ran a 5K.
  To me, interpret as a state of affairs.
  \emph{X} buttered a slice of toast.
  Interpret as reference to an event that happened.

  To argue for exhaustivity, note the presence of the modal `potential'.\nolinebreak
  \footnote{
    May repeat the same argument here with ability, the particular representation doesn't matter.
  }
  Hence, distinction between what is, and what could be.
  What is: that there is the potential for a witnessing event.
  What could be: the witnessing event.

  Alternatively, what is the case, and what is not (at least yet) the case.

  Note, this distinction is not about what makes the proposition true.
  Rather, it is in how the truth conditions are used.

  Illustrate.
  S thinks that X and Y are different.
  Well, modal, thinks.
  So, there is some thought that S has.
  Or, the content of the thought.
  S did not say hello to Y.
  Explain by the content, so shift with the modal.
  That there is a thought which belongs to S isn't important from the point of view of S.
  Alternatively, therefore S considers X and Y referring terms.
  Here, we have no interest in the content of the thought.
  Instead, interest is limited to its functional properties, and use this to ascribe other properties to S.

  Hence, modal, and that's it.

  Remaining issue is details of the schemas.
  These talk about more than mere reference.
  \AR{}, agent, and \WR{} the result of the witnessing event.
  In turn, these are harmless and the only plausible option.

  \AR{} is simple.
  State of affairs, but as the agent is involved, then it is natural to attribute to the agent.
  Implausible that it's some event.

  \WR{} focuses attention to culmination of event.
  However, need culmination.
  Quirk of English that may `use' relevant verbs in this way.
  Imperfective paradox.
  May consider this a state, but only in the sense that it is a state bought about by some event.
  Focus on event, but given culmination, consider this a state.
  Still, state of culminated event.
  Possible that this is simply a state in which the agent has some appropriate relation.
  Problem is that an ability is the ability to do some thing.
  If abstract away from the act, then it's not clear how to understand conditions as identifying ability.
\end{note}

\subsubsection{Where the tension arises}
\label{sec:where-tension-arises}

\begin{note}
  {
    \color{red}
    Looking ahead.
  }
  The goal here is to clarify that the tension arises from the ability entailment.
  The role of general to specific is to ensure that agent gets to fact from specific.
\end{note}


\subsection{\ESU{}, \gsi{}, and \aben{the}}
\label{sec:first-conditional}

\begin{note}[Summary]
  In this section we argue that \ESU{} constrains what an agent may claim support for when reasoning from general to specific ability.

  In short, \ESU{} rules out claiming support by \ur{}.
\end{note}

\begin{note}
  Two ways corresponding to two sides of (specific) ability
  First, with respect to \aben{the}: appealing to (specific) ability.
  Second, with respect to \gsi{}: establishing (specific) ability from (general) ability.
\end{note}

\begin{note}[Expand: \gsi{}]
  Start with \gsi{}.

  Agent is claiming support for specific ability.
  Hence, claiming support that there is a potential event in which they \emph{V} that \(\phi\).
  Expanding potential event, claiming support that sufficient resources are available.
  Note, the agent may not (merely) \emph{expect} that sufficient resources are available, as availability of resources is part of claim for potential event.
  Rather, the agent may expect that there are no defeaters to claim that resources are available.

  To illustrate.
  Suppose I claim support that I know the train will be late.
  It's not (merely) that I expect that the train will be late.
  In order to claim support, some considerations sufficient to establish that the train will be late, and that there are no defeaters for these considerations.
  Expect would be absence of materia that train is on time.
  But absence alone doesn't push either way.\nolinebreak
  \footnote{
    Absence may be materia, though.
    For example, at least five minutes before train will arrive there is a message broadcast at the station.
    We are at the station, and it is a three minutes before the train is scheduled to arrive.
  }

  Task is to account for why an agent may claim support for availability of sufficient resources.
  In rough outline, answer is simple.
  Claimed support for general ability.
  Specific ability to \emph{V} that \(\phi\) is an `instance' of the general ability.
  So, given context, general ability supplements sufficient additional premises and steps of reasoning.

  However, without witnessing specific ability, agent is not aware of which additional premises and steps of reasoning are used.
\end{note}

\begin{note}[Moving to incompatibility]
  Incompatibly with \ESU{} will be from common point of appeal to sufficient resources.
  To this we now turn.
\end{note}

\subsubsection{Constraints on reasoning with \gsi{} given \ESU{}}
\label{sec:incomp-wr-ura}

\begin{note}[Argument outline]
  \color{red}
  There are two issues.
  \begin{itemize}
  \item \ESU{} means that the agent may not `directly' establish the existence of a witnessing event.
    For, in order to do so, the agent would need to claim support that the conclusion follows from some collection of premises and steps of reasoning.
    However, as the agent does not witness, then this isn't possible given \ESU{}.
    \begin{itemize}
    \item The objection here is that the agent doesn't necessarily need to go directly.
      It's possible that the agent claim support for some property, hence gets specific ability, and then reasons that this means that there's a witnessing event.
      So, to the extent that \WR{} needs first the existence of a witnessing event, \ESU{} might be okay.
    \end{itemize}
  \item Second, the agent can't reason with the details of the witnessing event.
    This then blocks \WR{}, and does so conclusively.
    For, the agent is not permitted to appeal to a relation of support between premises and conclusion.
    The agent is only permitted to appeal to the existence of an event that would establish such a relation.
    So this is the main objection.
  \end{itemize}
\end{note}

Key proposition of this section.

\begin{note}[Proposition]
  \begin{proposition}[\ur{} is incompatible with~\ESU{}]\label{mcA:WR-and-denied-claim}
    \emph{If} \ESU{} is true \emph{then} no claiming support by \ur{} with respect to derived \AR{} or \WR{}.
  \end{proposition}
\end{note}

\begin{note}
  Argument is straightforward.
  \ur{} then claiming support by property of witnessing event.
  But, agent has not used those things in reasoning.
\end{note}


\begin{note}
  So, the key thing with this proposition is that in cases where an agent reasons with \gsi{-}, the agent claims support for a property.
  Hence, if an agent reasons from specific ability via \aben{the}, then must be an instance of \AR{}.

  Now, \autoref{mcA:WR-and-denied-claim} doesn't say that \ESU{} and \WR{} are incompatible in general.
  We'll see this in the argument for~\autoref{mcA:WR-and-denied-claim}.

  The argument that \WR{} is an incorrect interpretation of (specific) abilities of the form \emph{S} has the ability to \emph{V} that \(\phi\) (for which \aben{the} entailment holds) has two components.
  First, difficulty establishing by \gsi{}.
  Second, rules out \aben{the}.

  Difficulties with \gsi{}.
  Result of claiming support by \gsi{} is that agent claims support for specific.
  And, given \WR{}, this involves claiming support that some sufficient collection of premises and steps of reasoning are available to the agent.
  Given \ESU{}, the agent is required to use these steps and premises in order to appeal.
  Therefore, \ESU{} requires a partial witnessing event.
  Partial only, as the depending on how premises and steps are understood, certain premises or steps may be reused, and a single use may be sufficient for \ESU{}.

  The issue is strengthened when turning to \aben{the}.
  For, the conclusion is that \(\phi\) is the case.
  And, a partial witnessing event does not establish that \(\phi\) is the case.
\end{note}

\subsubsection{Summary}
\label{sec:uRa-and-wr-summary}

\begin{note}[Table]
  \begin{figure}[H]
    \centering
    \begin{tblr}{abovesep=8pt, belowsep=8pt, width=0.95\textwidth, colspec={Q[c,m]|Q[c,m]|Q[1.8,c,m]|Q[1.8,c,m]}}
      \multicolumn{2}{c}{} & \nr{} & \ur{} \\
      \hline
      \multicolumn{2}{c}{\WR{}} &  & Ruled out by \ESU{} \\
      \hline
      \multirow{2}{*}{\AR{}} & Basic  &  &  \\
      \cline[dashed]{2-4}
      & Derived &  & Ruled out by \ESU{} \\
    \end{tblr}
    \caption{Distinction matrix}
  \end{figure}
\end{note}

\begin{note}[Summarising]
  ???
\end{note}

\subsection{\nI{}, \gsi{}, and \aben{the}}
\label{sec:second-conditional}

\begin{note}[Redo of section]
  Seen \ESU{} and \ur{}.
  Now turn to \nr{}.
\end{note}

\begin{note}[Broad sketch of section]
  Introduce a general constraint on claiming support.
  The general constraint will relate to moving from general to specific ability information --- agent is not in a position to claim support for having specific ability from information and claimed support for general ability.
  However, initial statement and motivation apply to all instances of claiming support.
  After statement and motivation, show how the constraint relates to \nr{}.
  If so, agent lacks support for having specific ability, and does not have the option of claiming support for result of specific ability by \nr{}.
\end{note}

\subsubsection{Overview of \nI{}}
\label{sec:ni-1}

\begin{note}
  We turn to the general limitation on claiming support.
\end{note}

\paragraph{Statement of \nI{}}

\begin{note}[\nI{}]
  \begin{proposition}[\nI{-}  --- \nI{}]\label{prem:ni}
    Suppose:
    \begin{enumerate}[ref=\named{\nItext{}:\arabic*}, series=nI_counter]
    \item\label{nI:claimed-support} \(S\) has claimed support that some proposition \(\phi\) has value \(v\) from materia \(M\) and continues to claim support for \(\phi\) having value \(v\) from materia \(M\).\nolinebreak
      \footnote{
        The agent has at some time in the past (perhaps a moment ago) claimed support for \(\phi\) having \(v\) from materia \(M\), and at the present time the agent continues to hold that \(\phi\) has value \(v\) from \(M\).
        In some cases, an agent may revise materia.
        E.g.\ this may be the case when appealing to memory, a new source, etc.
        Note, however, that this does not require that the agent initially claimed support for \(\phi\) from materia \(M\) --- may be in instance of memory.
      }
    \item\label{nI:received-info} \(S\) has claimed support that if proposition \(\phi\) has value \(v\) then proposition \(\psi\) has value \(v'\).
    \end{enumerate}
    And, suppose:
    \begin{enumerate}[ref=\named{\nItext{}:\arabic*}, resume*=nI_counter]
    \item\label{nI:inclusion}
       \(S\) is confident that:
      \begin{enumerate}[label=\alph*., ref=\named{\nItext{}3:\alph*}]
      \item\label{nI:inclusion:position} Their claimed support for \(\phi\) having value \(v\) is \mom{} if \(S\) is not (given present context) in a position to claim support that \(\psi\) has value \(v'\) (without appealing to \(\phi\) having value \(v\)).\nolinebreak
        \footnote{
          Agent considers.
          This should be stressed.
          And, in most cases, it is mistaken that's at issue.
          For, failure of \(\psi\) condition doesn't raise any issue in particular for \(\phi\).
        }\(^{,}\)\nolinebreak
        \footnote{
          Also, it may be the case that the statement of \nI{} needs further constraints on what the relation is between the claimed support for \(\phi\) and claiming support for \(\psi\) (without appeal to the value of \(\phi\)).
          I don't see a clear problem at the moment.
          However, I do expand below.
        }
      \item\label{nI:inclusion:bound} If their claimed support for \(\phi\) having value \(v\) is \nmom{} then they would be \nmom{} when claiming support for \(\psi\) having value \(v'\) (without appeal to the value of \(\phi\)).
      \end{enumerate}
    \end{enumerate}
    Then:
    \begin{enumerate}[ref=\named{\nItext{}:\arabic*}, resume*=nI_counter]
    \item\label{nI:going-by-value} \(S\) may not claim support for \(\psi\) having value \(v'\) by appeal (only) to:
      \begin{enumerate}[label=\alph*., ref=\named{\nItext{}4:\alph*}]
      \item\label{nI:going-by-value:phi} \(\phi\) having value \(v\) from~\ref{nI:claimed-support}, and
      \item\label{nI:goingbyvalue:psi} the corresponding implication that \(\psi\) has value \(v'\) from~\ref{nI:received-info}.
      \end{enumerate}
    \end{enumerate}
    \vspace{-\topsep}\vspace{-\topsep}
  \end{proposition}
\end{note}

\begin{note}[What \nI{} amounts to]
  \nI{} is a restriction on the way in which an agent may claim support for some proposition when certain conditions obtain.
  The proposition consists of three `background' conditions --- clauses~\ref{nI:claimed-support} to~\ref{nI:inclusion} --- and a limitation given those background conditions --- clause~\ref{nI:going-by-value}.
\end{note}

\begin{note}[Plan]
  To start:
  \begin{enumerate}
  \item Core idea of \nI{}.
  \item Role in overall argument.
  \item Some intuition.
  \item Key points.
  \item Some connexions to the literature.
  \end{enumerate}
  Then, details.
\end{note}

\begin{note}[Quick intuition]
  The basic idea behind \nI{} is that if clauses~\ref{nI:claimed-support} and~\ref{nI:received-info} hold, then clause~\ref{nI:inclusion} captures a condition which \emph{undercuts} the agent claiming support as captured by~\ref{nI:going-by-value}.

  The term `undercuts' is used in relation to the established idea of an undercutting defeater.

  \begin{quote}
    The second kind of defeater attacks the connection between P and Q rather than attacking Q directly.
    \dots

    This second kind of defeater is, roughly speaking, a reason for thinking that, under these circumstances, knowing-that-P is not a good way to find out whether Q.
    \dots

    A type II defeater is any reason for believing that \({\sim}(P => Q)\) which is not also a reason for believing that \({\sim}Q\).\nolinebreak
    \mbox{}\hfill\mbox{(\cite[43]{Pollock:1974uk})}
  \end{quote}
  Where `\(=>\)' is the subjunctive conditional. (\Citeyear[42]{Pollock:1974uk})

  With \citeauthor{Pollock:1974uk}'s original formulation, the defeater attacks the link.\nolinebreak
  \footnote{
    See \textcite[196,fn.166]{Pollock:1999tm} for a brief note of the history of undercutting defeaters.
    \textcite{Pollock:1974uk} is more-or-less a direct expansion of discussion in~\textcite{Pollock:1970un}.
  }

  However, generalise.
  That there's a relation of claimed support, evidence, or what have you between \(P\) and \(Q\).

  Generalisation by \citeauthor{Bergmann:2005ws}\nolinebreak
  \footnote{
    I am also inclined to \citeauthor{Worsnip:2018aa}'s sketch of undercutting defeaters, which references~\citeauthor{Bergmann:2005ws}.
    \begin{quote}
      Undercutting defeaters, which are easiest to think of in the context of the attitude of belief, are supposed to be considerations that undermine the justification of a belief in a proposition p not necessarily by providing (sufficient) positive evidence to think that p is false, but rather merely by suggesting (perhaps misleadingly) that one’s reasons for believing p are no good, in a way that neutralizes or mitigates their justificatory or evidential force.\linebreak
      \mbox{}\hfill\mbox{(\Citeyear[29]{Worsnip:2018aa})}
    \end{quote}
  }
  \begin{quote}
    \emph{d} is an \emph{undercutting defeater} for \emph{b} iff \emph{d} is a defeater for \emph{b} which is (or is an epistemically appropriate basis for) the belief that one's actual ground or reason for \emph{b} is not indicative of \emph{b}'s truth.\nolinebreak
    \mbox{}\hfill\mbox{(\cite[424]{Bergmann:2005ws})}
  \end{quote}

  Similar generalisation, but similar to \citeauthor{Pollock:1974uk} the relation is useful.
  So, generalise the subjunctive conditional.

  If~\nIOTT{} hold, then the way of claiming support captured by~\ref{nI:going-by-value} is undercut.

  Specifically, undercutting arises from something close to circular reasoning.
  Where, in order for agent to appeal to \(\phi\) having value \(v\) the agent must assume that \(\psi\) has value \(v'\).
  Hence, can't go to \(\phi\) without \(\psi\), which means that can't use \(\phi\) to claim support for \(\psi\).

  The specifics will wait until we're walked through the clauses of \nI{} in some detail.
\end{note}

\begin{note}[\eiS{}]
  Idea of an undercutting defeater is quite general.
  Here, we'll develop a specific account.
  Main thing will be \eiS{}.
  \begin{quote}
    \begin{proposition}[\eiS{-} --- \eiS{}]
      \color{blue}
      Claimed support is taken to indicate the value of \(\phi\) regardless of whether the claimed support is \mom{}.
    \end{proposition}
  \end{quote}
  Undercut when \eiS{} fails.
  This doesn't say anything about the value of \(\phi\).
  Rather, if \eiS{} doesn't hold then the agent doesn't get to claim support.
\end{note}

\begin{note}
  Purpose of conditions is to get undercutting of this kind.
  Of course, work is done by \eiS{}.
  And, in some respects, it is easier to reason directly.

  However, for our purposes this won't do.
  Our goal is to apply \nI{} to ability.

  Two problems.
  First, whether one gets this kind of failure.
  \nI{} address this by narrowing down on a small set of conditions that we can check.
  Second, whether the kind of failure matters.
  \nI{} address this by motivating the failure in general.
\end{note}

\begin{note}[A few illustrations]
  Let us now turn to a few illustrations before discussing \nI{} in further detail.

  We'll begin with a somewhat detailed illustration.
  \nI{} identifies a particular way in which an agent may fail to claim support, and the primary goal of the initial demonstration is to highlight why the agent would fail to claim support.
  Hence, the illustration treads a fine line between highlighting a problematic method, but not necessarily a problematic result.
  This is by design.
  And, I will continue to stress that \nI{} concerns a way of claiming support for some proposition, rather than the possibility of claiming support for some proposition.

  Following two illustrations will be variations on the initial.

  Still, it may be helpful to observe how \nI{} relates to an intuitively problematic result.
  Therefore, we will provide an additional, simple, illustration of a failure to claim support.

  The final illustration in the trio will complement the initial par of illustrations by highlighting an instance where~\nI{} does not apply.

  \phantlabel{dogmatism-introduction}
  The reader may note structural similarities between these illustrations and \citeauthor{Kripke:2011wv}'s Dogmatism paradox.
  We will discuss the relation after the illustrations.
\end{note}

\begin{note}[Brief illustration of \nI{}]
  The first illustration considers theories and counterexamples.

  \begin{illustration}\label{ill:CE:main}
    Suppose a researcher have constructed a theory of some general phenomenon.

    The theory seems to capture the phenomenon, and the researcher has claimed (inductive) support that the theory is adequate by applying it to various instances of the general phenomenon.
    Even if the theory isn't adequate, the theory has been (seemingly) successful applied to sufficient specific instances of the phenomenon.
    Hence, even if \mom{}.

    However, as the phenomenon is a \emph{general} phenomenon it also makes various predictions about what must happen in all other instances to which the researcher has not (yet, at least) applied the theory to.

    Hence, there is a possible counterexample to the theory associated with each instance the researcher has not (yet) applied the theory to.
    If some particular instance does not conform to the theory, the theory is inadequate.
    Conversely, if the theory is adequate, every particular instance of the phenomenon conforms the theory.
    In other words, if the theory is adequate, then there are no counterexamples to the theory.

    Of course, it may be simple to revise the theory is a counterexample exists, and the fundamental ideas of the theory may remain sound (Cf.~\textcite{Bonevac:2011tz}).
    And, the theory may have sufficient resources to explain why any apparent counterexample is not a counterexample.
    Yet, it remains the case that the theory would need to be revised in light of a counterexample.

    Now, to summarise, the researcher may claim support for two propositions which mirror~\ref{nI:received-info} and~\ref{nI:claimed-support}:

    \begin{itemize}
    \item The theory is adequate, and
    \item If the theory is adequate, then there are no counterexamples.
    \end{itemize}

    At issue is whether the researcher may claim support that there are no counterexamples to the theory from the claimed support for the two propositions in the following way:

    \begin{itemize}
    \item I have claimed support that the theory is adequate.
    \item So, given the claimed support, theory is adequate.
    \item Therefore, as the theory is adequate, given the claimed support, it follows that there are no counterexamples.
    \item Hence, I claim support that there are no counterexamples to the theory.
    \end{itemize}

    Seems problematic.
    Claimed support that the theory is adequate is qualified by the possibility of counterexamples.
    {
      \color{red}
      Note, agent is, here, only claiming support that there are no counterexamples.
      And, claiming support may be \mom{}.
      So, it does not follow that the agent is ruling out the possibility of counterexamples to the theory.
      Plausible that the agent \emph{may} claim support.
      Problem is the way in which the agent goes about this.
    }

    {
      Even if not convinced about support, this way of claiming support seems problematic.
      Relying on theory being adequate.
      However, if this is the case, then no possible counterexamples.
      Issue is that such counterexamples are possible given the state of your claimed support.
      Hence, claiming support in this way seems to take for granted that there are no counterexamples.
    }

    Problem is that the reasoning only works if there are no counterexamples.
    If there are counterexamples, misled.
    Hence, problem to go from the theory is adequate.
    However, without this step, researcher doesn't get to no counterexamples.

    So, this is \eiS{}.
  \end{illustration}

  So, relation between theory and counterexample \emph{undercuts} using way of using theory to get no counterexample.

  Now, given that the researcher has claimed support that the theory is adequate, the researcher may \emph{expect} that there are no counterexamples to the theory.
  And, it doesn't follow that the researcher may not claim support.
  Specific way --- reasoning captured by~\ref{nI:going-by-value}
  Plausible that details of the theory provide some way of claiming support.

  Indeed, it seems the researcher is require to take the alternative path --- to show that the proposed counterexample is accounted for by the contents of the theory, regardless of whether the theory is true.

  Fault here is with respect to \eiS{}.
  {
    \color{red}
    Here, conditions of~\ref{nI:inclusion} are satisfied, but we did not explicitly appeal to them.
    Purpose of~\ref{nI:inclusion} is conditions sufficient for this kind of problem to arise.
    So, to do in argument for \nI{} is to develop is why~\ref{nI:inclusion} does something similar.
    Upshot is that \nI{} is general.

    In the third illustration, we'll see why the way of claiming support is okay in some cases.
  }
  Difficult part is to account for why~\ref{nI:inclusion} sets things up and ensures that things don't go too far.
\end{note}

\begin{note}[Idea main part of \nI{} works]
  As noted above, it is unclear whether or not there may be some way for the researcher to claim that there are no counterexamples to the theory.
  And, if there is some way for the researcher to claim that there are no counterexamples to the theory, one may be inclined to wonder whether there really is a problem with claiming support in the way outlined by~\ref{nI:going-by-value}.

  In other words, one may be wondering whether \eiS{} is a plausible constraint on claiming support.
  We gave a general argument for \eiS{} in~\autoref{sec:abil-access-supp}.
  However, it may help to see how the issue highlighted relates to an intuitively problematic instance of reasoning, regardless of how support is claimed.

  \begin{illustration}\label{ill:CE:colleague}
    Suppose a colleague has studied the researcher's theory, and they (the colleague) thinks they have found a counterexample.

    The colleague has informed the researcher that they think they have observed a counterexample.

    However, the colleague has not provided the researcher with any further details about the counterexample.

    Now, the conditional of interest may be made more precise:
    \begin{itemize}
    \item If the theory is adequate, then the colleague has failed to identify a counterexample to the theory.
    \end{itemize}

    Now, let's replicate the way of claiming support from before.

    \begin{itemize}
    \item I have claimed support that the theory is adequate.
    \item So, given the claimed support, theory is adequate.
    \item Therefore, as the theory is adequate, given the claimed support, it follows that the colleague has failed to identify a counterexample to the theory.
    \item Hence, I claim support that the colleague has failed to identify a counterexample to the theory.
    \end{itemize}

    I take this illustration to be intuitively problematic.
    In short, if claim support, then doesn't need to examine counterexample to claim support that it is not a counterexample.

    Possible response is that researcher does claim support, but information colleague impacts claimed support for theory.
    However, this is also puzzling.
    Researcher has no information.
    Hence, if retain confidence, then equally against counterexample.
    And, if does not retain confidence, then down the theory in a way that seems implausible.

    Seems, instead, that claimed support for theory persists, but that this doesn't extend to counterexample.\nolinebreak
    \footnote{
      Inclined to apply this to previous illustration.

      However, there's a difference between two illustrations.
      Here, someone (the colleague) has reason to think there is a counterexample, and this seems a sufficiently important difference to draw any quick conclusions.
      And, as we don't require a resolution to this issue, I won't explore further.
    }

    Perhaps more detail is needed.
    I have some doubts that claiming support is always bad.
    However, clearer that developed in a way such that problem remains.

    Now, seems that the researcher doesn't get to claim support because if counterexample, then theory is bad.
    Hence, requires that counterexample is not true in order to progress.
    But, then, doesn't make the move regardless of whether or not there is a counterexample.

    So, it seems \eiS{} does the work.
  \end{illustration}

  Undercuts using \(\phi\) for \(\psi\).
  Same problem, failure of \eiS{}.

  For, the agent has already assumed \(\psi\).

  Problem is that the agent doesn't get to claim support for \(\psi\) because fail the \eiS{} thing.
  If \(\psi\) isn't really the case, then reasoning collapses.
  Key thing about our understanding of claimed support is that it holds up even if the agent is \mom{} about the value of the proposition.

  {
    \color{red}
    Note:
    There's possible tension here.
    It seems that if the first illustration is okay, then this (second) illustration should also be okay.
    Maybe.
    But, this is too quick.
    Additional information here.

    Now, still some difficulty, as I think \EAS{} might apply to the first.
    So, shouldn't it apply to this?
    Well, no.
    For, \EAS{} only suggests possibility in some cases.
    Fine to think of this additional information as constraint on appeal via ability.
    For, if the colleague thinks they've found a counterexample, then this suggests a problem with the agent's ability.
  }
\end{note}

\begin{note}[Case to worry about]
  \begin{illustration}\label{ill:CE:testimony}
    Suppose the researcher has published a paper containing the details of the theory.

    Our attention now turns to a novice who has read far enough into the paper to understand, at least, the general phenomenon that the theory applies to and that the researcher has claimed inductive support for the theory.
    We'll also assume that the novice does not possess the expertise required to apply the theory.\nolinebreak
    \footnote{
      Though I don't think this assumption is important.
    }

    The novice is thinking about instances of the general phenomenon, and identifies one.

    The conditional of interest is:
    \begin{itemize}
    \item If theory is adequate, it accounts for this instance of the phenomenon.
    \end{itemize}

    Of course, the novice also recognises that the theory is inadequate if it  does not account for the particular instance of the phenomenon.
    Still, the novice claims support in the familiar manner.

    \begin{itemize}
    \item I have claimed support that the theory is adequate (this time by reading a published paper).
    \item So, given the claimed support, the theory is adequate.
    \item Therefore, as the theory is adequate, given the claimed support, it follows that the theory accounts for this instance of the phenomenon.
    \item Hence, I claim support that the theory accounts for this instance of the phenomenon.
    \end{itemize}
    In contrast to the previous illustrations, it seems the novice may claim support in such a way.
  \end{illustration}

  Here, even if theory is bad\dots the novice still satisfies \eiS{}.

  Problem is that \ref{nI:going-by-value} just outlines this reasoning.

  Of course, given role of \eiS{}, might think to reformulate.
  However, not a good condition.
  Want sufficient condition for failure of \eiS{}.
\end{note}

\begin{note}
  These three illustrations.
  First, kind of scenario that's the main interest.
  Where claiming support in a certain way seems problematic, even if it not clear that the agent may claim support in some other way.

  To stress the problem, considered a cleaner case, where it seems agent may not claim support, and argued that same problem is a plausible account of why.

  Third illustration, way of claiming support is okay.
  As all instances of \nI{}, and hence the previous two illustrations, focus on particular way of claiming support illustrated that it's okay.
\end{note}

\begin{note}[Intuition]
  In short, \nI{} captures a limitation: An agent is not in a position to claim support for some proposition \(\psi\) by information that \(\phi\) `entails'\nolinebreak
  \footnote{Strictly speaking, \ref{nI:received-info} is more general than entailment.}
  \(\psi\) if failure to establish support for \(\psi\) independently of the value of \(\phi\) would reveal problem with the support claim for \(\phi\).

  Hence, \nI{} focuses on when an agent may claim support for some proposition by noting that (from the agent's perspective) that the value of the proposition is determined by further propositions the agent has claimed support for.
  From \ref{nI:received-info}, \(\phi\) to \(\psi\)
  And from \ref{nI:claimed-support} \(\phi\)
  So, \(\psi\), but \ref{nI:going-by-value} denies this given \ref{nI:inclusion}.
  Some other way of claiming support for \(\psi\).
  However, not merely an alternative path, but an alternative path that must be possible given claimed support for \(\phi\).

  Issue is that given \ref{nI:claimed-support} and~\ref{nI:inclusion}, agent expect that they have the resources, and hence expects that \(\psi\) is the case.

  So, that \(\phi\) has value \(v\).
  In doing so, resources to claim support for \(\psi\) has value \(v'\).
  Hence, that \(\psi\) has value \(v'\).
  So, \(\psi\) having value \(v'\) is a requirement on claimed support for \(\phi\) being any good.
  However, no support claimed for \(\psi = v'\).
  Even so, appeal to \ref{nI:received-info} does not help, because wouldn't get to \(\phi\) without \(\psi\).
  That value of \(\phi\) constrains value of \(\psi\) is useful information, but it useless because if \(\psi\) isn't already so constrained, then no appeal to \(\phi\).

  Similar to other principles, failure because establishing something that needs to be the case in order to be in a position to establish.
\end{note}

\begin{note}[Task]
  Two important tasks with respect to arguing for \nI{}:

  \begin{itemize}
  \item \ref{nI:inclusion} and~\ref{nI:inclusion:position}, how these function.
  \item And, why \nI{} holds.
  \end{itemize}

  Then, additional details.
\end{note}

\begin{note}
  Before continuing I would like to stress two general points.
  Follow up on relation to \citeauthor{Kripke:2011wv}'s Dogmatism paradox.
  Link to similar principles in the literature.
\end{note}

\begin{note}
  Two quick points to stress.
\end{note}

\begin{note}[First point to stress]
  First, \nI{} outlines sufficient conditions for a limitation on claiming support to obtain.
  Hence, there may be other limitations on claiming support.
  For example, \ESU{} implies a limitation of claiming support --- an agent may not claim support by appeal to some premises or step of reasoning that they have not used.

  Similarly, there may be other limitations on claiming support with obtain.
\end{note}

\begin{note}[Second point to stress]
  Second~\ref{nI:going-by-value} is a limitation of a \emph{way} of claiming support.
  So, \nI{} does not imply that the agent is not in a position to claim support for \(\psi\), only that one way of claiming support is ruled out when the other clauses of \nI{} hold.
  Hence, \nI{} is compatible with the agent claiming support for \(\psi\) having value \(v'\) --- so long the agent doesn't follow the pattern captured by~\ref{nI:going-by-value}.

  Indeed, I take the primary upshot of \nI{} to be a demand for understanding alternative ways of claiming support when each clause of \nI{} holds and it seems that the agent may claim support for \(\psi\) having value \(v'\).
  And, after arguing for \nI{} our attention will turn to examining how an agent may claim support by \EAS{} when clauses~\ref{nI:claimed-support} obtain.

  For now, however, our focus is on arguing for \nI{}.

  It is perhaps also helpful to flag that the following discussion of \nI{} will relate previous details on claiming support (in particular section~\ref{sec:abil-access-supp}) but will proceed without consideration of ability.
  Our goal is to motivate \nI{} as a general limitation on claiming support, and in turn draw a consequence from such a limitation when applied to ability.
\end{note}

\begin{note}[Relation to circular reasoning and Dogmatism paradox]
  \phantref{dogmatism-introduction}{Above} we noted some structural similarities between the illustrations and \citeauthor{Kripke:2011wv}'s dogmatism paradox.
  Let us briefly follow up on this observation.
  The key distinction will be the role of knowledge in the dogmatism paradox
\end{note}

\begin{note}[The dogmatism paradox]
  To refresh, the dogmatism paradox (\cite[39,43--45]{Kripke:2011wv};\cite[148]{Harman:1973ww}) concerns the factivity of knowledge and intuitions concerning evidence.

  Roughly stated, the paradox pairs the following two propositions:
  \begin{enumerate}[label=D\arabic*., ref=(D\arabic*)]
  \item\label{dog:1} If an agent is aware that they know that \(\phi\), then the agent may disregard any evidence against \(\phi\).\nolinebreak
    \footnote{
      Neither \citeauthor{Kripke:2011wv} (nor \citeauthor{Harman:1973ww}) make explicit mention of the agent being aware that they know \(\phi\) when formulating the Dogmatism paradox.
      Still, the paradox is clearer with this stated, as it's require addition work to find issue with a permission (to disregard evidence) if an agent is not aware that they have such a permission.

      More generally, I agree with \citeauthor{Zhaoqing:2015vj}'s (\Citeyear{Zhaoqing:2015vj}) proposal to understand the paradox in terms of knowledge attribution rather than of knowledge proper.
    }
  \item\label{dog:2} Rational agents respect their evidence
    (\cite[Cf.][\S2]{Kelly:2016wk})
  \end{enumerate}
  Given~\ref{dog:2}, it seems no agent should not disregard any instance of evidence, even if the antecedent of~\ref{dog:1} is satisfied.
\end{note}

\begin{note}
  Relation.
  Broad perspective, agent has adopted attitude toward proposition, but doesn't get to extend that to a consequence.
  Above, claiming support.
  Here, knowledge.
\end{note}

\begin{note}
  For, if the agent is aware that they know that \(\phi\) then the agent knows that \(\phi\).
  And, as knowledge is factive it follows that if the agent knows that \(\phi\), then it must be the case that \(\phi\).
  In turn, if it is the case that \(\phi\) then any evidence against \(\phi\) is evidence for something that is not the case.
  Hence, the agent may disregard any evidence against \(\phi\).\nolinebreak
  \footnote{
    Hard to reconcile with understanding of support.
    For, have~\ref{prop:fallibility}.
    I don't think this matters.
    Have~\ref{prop:fallibility} to simplify things.
    No problem with dropping fallibility, it would simply take more work.
    Though, see previous footnote.
    I don't think it's plausible that an agent is aware that they know beyond the Cartesian basics.
  }

  Still, the Dogmatism paradox requires knowledge.
  For, an agent is only (apparently) in a position disregard any evidence against \(\phi\) because there knowledge that \(\phi\) guarantees that \(\phi\) is the case.

  And, overlooked is that it is possible that the agent is not in a position to hold that they have successfully claimed support.
  May expect, but that's something different.
  Dogmatism paradox requires the factivity of knowledge.

  Indeed, limitations of claiming support that are at issue.
  And, reasoning suggested above captures gist of why agent would fail to claim support.
  Escalating claimed support.
  However, details matter.
\end{note}

\begin{note}[Literature]
  \nI{} is similar to \citeauthor{Wright:2011wn}'s Transmission Failure template (\Citeyear{Wright:2003aa,Wright:2011wn}) and \citeauthor{Weisberg:2010to}'s No Feedback condition (\Citeyear{Weisberg:2010to}).
  {
    \color{red}
    We will highlight the contrast between \nI{} and similar principles in Chapter~\ref{cha:inertia} // below.
    }

  For now, it may be helpful to highlight that \nI{} does not deny that any agent may claim support for \(\psi\) having value \(v'\).
  Rather,~\ref{nI:going-by-value} (only) denies that the agent may claim support for \(\psi\) in a specific way.
\end{note}

\newpage

\subsubsection{Re-do of \nI{} argument}
\label{sec:re-do-ni}


\begin{note}[Structure]
  First, clarify the conditions.
  Second, argument for how these fit together.
\end{note}

\begin{note}[Key point of argument]
  The agent doesn't get to claim support for through RBV because any move to value requires to already be the case given the information the agent has.
\end{note}

\begin{note}[The tension with \eiS{}]
  The tension with \eiS{} is that \eiS{} requires that the agent may claim support even if it's not actually a genuine relation of support.
  In the sense that only get to \(\phi\) if the claimed support for \(\phi\) is not mistaken or misled.
\end{note}


\begin{note}[Weakest sufficient condition]
  Important that \nI{} is a sufficient condition because it could be strengthened.

  However, also important that it's so weak, because it's compatible with Moore, etc.\
\end{note}

\subsubsection{Details of --- and argument for --- \nI{}}
\label{sec:details-ni}

\begin{note}[Plan for the section]
  \color{red}
  \begin{itemize}
  \item Application to reasoning about ability
    \begin{itemize}
    \item Interesting, because there's going to be conflict.
      For, \nI{} forces agent to expand.
      However, this seems to conflict going by ability.
      Indeed, this is where the motivation for \EAS{}.

      {
        \color{green}
        Also interesting to note that there's some indication of the source of the tension.
        \ESU{} requires agent to witness, so to speak, while \nI{} places some constraints how an agent may witness.
        I.e.\ this is the sort of clash that one might expect to get.
        So, it's not too surprising that there would be these kinds of clashes.

        Actually, this is sort of interesting.
        \nI{} requires expansion in certain cases.
        And, the interesting thing about ability is that it seems this expansion is not required.
      }
    \end{itemize}
  \end{itemize}
\end{note}


\begin{note}[Overview]
  In this section we will walk through the four clauses of \nI{} in some detail.

  Our goal is to clarify each clause to the extent that the remaining task in our argument for \nI{} is to explain why~\ref{nI:going-by-value} follows from~\nIOTT{}, the initial three clauses of \nI{}.

  With this in mind, the examination of each clause is directed at clarifying what the overall role of the clause with respect to \nI{} as a whole.
  Still, as each clause may be satisfied independently of the others, when considering the what and when of each clause we will consider the clause separate from its role in \nI{}.

  The map for this section is as follows:

  Start with a combined treatment of~\ref{nI:claimed-support} and~\ref{nI:received-info} as these are relatively straightforward background conditions.
  Importance is to set up information that the agent is reasoning with.

  Interest is with~\ref{nI:inclusion} and~\ref{nI:going-by-value}.

  Take~\ref{nI:going-by-value} before~\ref{nI:inclusion}.
  Helpful to see what~\ref{nI:inclusion} is targeted at.
  Allow a clearer treatment of some of the issues that arise with respect to~\ref{nI:inclusion}

  \indent\indent Clauses~\ref{nI:inclusion} and~\ref{nI:going-by-value} will then be examined separately as they are, and their role is, more complex.

  \begin{center}
    General pattern for these subsections
    \begin{itemize}
    \item Start with the core idea.
    \item Role in argument.
    \item Clarify details.
    \item Consider objections.
    \item Broader picture.
    \end{itemize}
  \end{center}
\end{note}


\paragraph{\ref{nI:claimed-support} and \ref{nI:received-info}}

\begin{note}[\ref{nI:claimed-support} and \ref{nI:received-info}]
  \ref{nI:claimed-support} and \ref{nI:received-info} detail two pieces of information an agent has claimed support for.

  Restated:
  \begin{quote}
    \color{blue}
        \begin{enumerate}
    \item[\ref{nI:claimed-support}] \(S\) has claimed support that some proposition \(\phi\) has value \(v\) from materia \(M\) and continues to claim support for \(\phi\) having value \(v\) from materia \(M\).
    \item[\ref{nI:received-info}] \(S\) has claimed support that if proposition \(\phi\) has value \(v\) then proposition \(\psi\) has value \(v'\).
    \end{enumerate}
  \end{quote}

  \ref{nI:claimed-support} identifies claimed support that some proposition \(\phi\) has value \(v\).
  And, \ref{nI:received-info} identifies claimed support that if \(\phi\) has vale \(v\) then some other proposition \(\psi\) has value \(v'\).

  If the agent satisfies clauses \ref{nI:claimed-support} and \ref{nI:received-info}, then the agent often has the option of reasoning to \(\psi\) having value \(v'\) by combining these two pieces of information.
\end{note}

\begin{note}[Often, but not always]
  Often, but not always.
  The relevant instances of \(\psi\) and \(\psi\) combined with the way in which the agent has claimed support may lead to trouble.
  \nI{} captures one instance of trouble, but there are others.

  \begin{illustration}
    Suppose \nagent{8} has received information from a younger and an older sibling.
    The older sibling has told \nagent{8} that the dog has escaped.
    And, the younger sibling has told \nagent{8} that if even the dog has escaped, the younger sibling has no idea whether the dog has actually escaped or not.
  \end{illustration}

  Here, \(\phi\) is that the dog has escaped, \(\psi\) is that the younger sibling has no idea whether the dog has actually escaped or not, and both \(v\) and \(v'\) are `true'.

  It seems that \nagent{8} will have a hard time combing the support claimed for \(\phi\) and \(\psi\) to reason to \(\psi\) being true.
  For, \nagent{8} claims support for \(\phi\) from the younger siblings testimony, but if \(\psi\) is the case then the younger siblings testimony is unreliable (with respect to whether the dog has run away, at least).
  Hence, rather than concluding that \(\psi\) is the case, \nagent{8} is tasked with resolving the tension between the support claimed for \(\phi\) by way of the younger sibling, and the support claimed for if \(\phi\) then \(\psi\) from the older sibling.

  The illustration merely highlights that it's not always straightforward to piece together and antecedent and a conditional.
  This observation is hardly novel, and similar illustrations may be found in ~\citeauthor{Harman:1986ux}'s~\citetitle{Harman:1986ux}, among others.
  Still, \nI{} is a sufficient, rather than necessary, condition for an agent failing to claim support (and failing in a certain way) and so that there may be other ways in which instances of \(\phi\) and \(\psi\) lead to trouble is not of particular interest.

  When discussing \nI{} we will assume that there is no tension between the claimed support for \(\phi\) and the claimed support for if \(\phi\) then \(\psi\).
  Instead, the difficult for the agent will follow (primarily) from \ref{nI:inclusion} --- a certain kind of relation between claimed support for \(\phi\) and claiming support for \(\psi\).
\end{note}

\begin{note}
  \color{red}
  Maybe some examples of good instances…
\end{note}

\begin{note}
  {
    \color{red}
    Something about intuition for \(\psi\) as a possible defeater.
  }
  Now, the agent may observe that if both~\ref{nI:claimed-support} and~\ref{nI:received-info} hold, then if \(\psi\) does \emph{not} have value \(v'\) then the support claimed in either condition is either \mom{}.

  First, if it is the case that \(\psi\) has value \(v'\) when \(\phi\) has value \(v\) and \(\psi\) does not have value \(v'\), then \(\phi\) does not have value \(v\).
  Hence, the agent's claimed support for \(\phi\) having value \(v\) must be misled.

  Second, if the agent's claimed support \(\phi\) having value \(v\) is not misled then \(\phi\) has value \(v\), but then if \(\psi\) does not have value \(v'\) it is not the case that \(\psi\) has value \(v'\) when \(\phi\) has value \(v\), and hence the agent's claimed support for the relation is misled.

  However, this observation alone is not particularly interesting.
  Take any case in which an agent claims support for receiving testimony regarding some matter the agent has no information regarding.
  And, observe that while the agent is mistaken is claiming support for testimony if what the interlocutor has said is not true, this does not prevent the agent for claiming support for the matter by appeal to the (apparent) testimony of their interlocutor.

  For example, if a logic instructor (unintentionally) misstates a theorem, a student may still claim support for the truth of the theorem by appeal to the instruction they received --- and even if the student reflects that they would be mistaken if the theorem is misstated.

  As stated in \eiS{}, claimed support may be misled or mistaken.
  At most, the observation under discussion in general only requires the agent to expect that \(\psi\) has value \(v'\).

%   Stated in terms of value because this also holds for desires, and for probabilistic statements.
%   Desires, means end is easiest to demonstrate with.
%   Probability, think in terms of conditionalization, or in terms of entailment.
%   Truth of \(\phi\) then probability of \(\psi\) is \(40\%\).
%   If the probability of \(\phi\) is \(70\%\) then the probability of \(\psi\) is \(40\%\) (though the probability of \(\phi \land \psi\) may be \(28\%\)).
\end{note}

\paragraph{\ref{nI:going-by-value}}

\begin{note}[Background to Limitation]
  Core idea of~\ref{nI:going-by-value} is that given~\nIOTT{} as background conditions, agent is prevented from claiming support in a specific way.
  So, primary role of~\ref{nI:going-by-value} is to capture what the way of reasoning is.

  Argument for why~\ref{nI:going-by-value} holds given~\nIOTT{} will be given below.
  Here, further details on what this way of claiming support amounts to.

  \ref{nI:going-by-value} prohibits this kind of reasoning given previous conditions obtain.
  For the moment we focus on what is prohibited, and defer why it is prohibited to argument below.
\end{note}

\begin{note}[\ref{nI:going-by-value} Restated]
  \ref{nI:going-by-value} describes a particular way of claiming support (and denies that the agent may claim support in such a way if certain conditions are met).

  Restated:
  \color{blue}
  \begin{quote}
    \begin{enumerate}
    \item[\ref{nI:going-by-value}] \(S\) may not claim support for \(\psi\) having value \(v'\) by appeal (only) to:
      \begin{enumerate}[label=\alph*.]
      \item[\ref{nI:going-by-value:phi}] \(\phi\) having value \(v\) from~\ref{nI:claimed-support}, and
      \item[\ref{nI:goingbyvalue:psi}] the corresponding implication that \(\psi\) has value \(v'\) from~\ref{nI:received-info}.
      \end{enumerate}
    \end{enumerate}
  \end{quote}
\end{note}

\begin{note}[`Just' \emph{modus ponens}]
From a glace, the type of reasoning captured by~\ref{nI:going-by-value} seems straightforward.
  From~\ref{nI:claimed-support} and~\ref{nI:received-info} the agent has claimed support that \(\phi\) has value \(v\) and that if \(\phi\) has value \(v\) then \(\psi\) has value \(v'\).

  So,~\ref{nI:going-by-value} is \emph{just} an instance of \emph{modus ponens}.

  Well, that's a little too fast.

  We saw above that \emph{modus ponens} as putting together an antecedent and a conditional, and here we'll provide further details of what such reasoning involves.
  Perhaps the details are just the details of an instance of \emph{modus ponens}, but we're not too interesting in broad classifications.
  Rather, our goal is to provide sufficient background to see why \ref{nI:claimed-support},~\ref{nI:received-info}, and~\ref{nI:inclusion} prevent claiming support by whatever name this kind of reasoning may go under.

  Indeed, our interest is primarily with respect to the agent moving from their claimed support for \(\phi\) having value \(v\) to \(\phi\) having value \(v\).
  Hence, why we have termed this type of reasoning `\RBV{-}'.
\end{note}

\begin{note}[Breakdown]
  So, have:
  \begin{enumerate}
  \item Claimed support for \(\phi\) having value \(v\), and
  \item Claimed support that if \(\phi\) has value \(v\) then \(\psi\) has value \(v'\)
  \end{enumerate}
  And, in turn.
  \begin{enumerate}
  \item \(\phi\) has value \(v\), therefore
  \item \(\psi\) has value \(v'\).
  \end{enumerate}
  Two important steps.
  \begin{enumerate}
  \item Move from claimed support for \(\phi\) having value \(v\) to \(\phi\) having value \(v\).
  \item Move from \(\psi\) having value \(v'\) to claiming support that \(\psi\) has value \(v'\).
  \end{enumerate}
\end{note}

\begin{note}[The two key moves]
  Focus now on the two key moves.
\end{note}

\begin{note}[First move]
  Distinction between claiming support that \(\phi\) has value \(v\) and reasoning about \(\phi\) having value \(v\).

  Okay, the key thing here is that this move involves no possibility of misled.
  So, claimed support, fine with possible defeaters, expect that not misled.

  This is the key thing to keep in mind.
  Important point is that the agent requires a bunch of stuff to hold that they have not necessarily claimed support for.
\end{note}

\begin{note}[Familiar from literature]
  Zebra.
  Light bulb.

  Exhibit this basic problem.
  There's no problem with the agent claiming support, but if the agent then goes by value they get more.
  In these cases, knowledge does a lot of the work (and we'll see why different later).
  Strengthens the problem, as knowledge is taken to be factive.
  However, we don't necessarily have this issue with respect to \nI{}.
  Here, more general, same move works with respect belief, etc.
\end{note}

\begin{note}[Second move]
  Claiming support for \(\psi\) having value \(v'\).

  Issue is a continuation of the previous.
  By reasoning from \(\phi\) having value \(v\) the agent has gone beyond claimed support.
  And, given that the agent has got a whole bunch of stuff from \(\phi\) having value \(v\), the agent needs to resolve these issues as the agent isn't thinking that these are the case when they claim support for \(\psi\) having value \(v'\).

  Look, this works a lot of the time, so there's some way of understanding this.
  With respect to \nI{}, interest is in whether this trivialises the reasoning.
\end{note}

\begin{note}[Example]
  Simple way to trace this is by thinking about Tarskian treatment of quantifiers.\nolinebreak
  \footnote{
    Conditionals are similar, but distinct.
    Antecedent is true, and then discharge assumption.
    Here, typically no claimed support that the antecedent is true.
  }
  Explored above.
  Here, \ur{}.
  Update the assignment function to point to a specific individual.
  Use this to then reason as normal.
  However, when we're done, discharge this assumption.
\end{note}

\begin{note}
  Slightly different example, here, where things seem fine.
  \begin{illustration}
    Searching for YYY in the building.
    If I search, there are various defeaters I expect not to obtain.
    Ask secretary.
    If not in LLL, then not in building.
    Now I only need to check LLL.
    From this, \RBV{} and then claim support for potential defeaters of obscure places.
    Why?
    Because regardless of my reasoning, XXX not being in LLL entails that XXX in not in the building.
    Given this type of value information, establish an also-is relation of support.
    XXX not in LLL also-is support for XXX not in building.
    Without, don't get this.
  \end{illustration}
  Here, the searching the room is fine.
  However, now extend by info from uketuke.
  This then no defeaters from uketuke.
  I mean, the agent has only checked one room.

  Claim support, and here it seems that claimed support now rests on expectation that these defeaters don't hold.
\end{note}

\begin{note}[Other types of reasoning]
  \RBV{} isn't the only way to reason.
  Key is that first step, often don't need it to be the case that one requires that possible defeaters don't hold.

  Handful of examples
\end{note}

\begin{note}[Disjunction]
  Going from \(\phi\) to \(\phi \lor \psi\).
  Don't need to \RBV{}.
  Claimed support for \(\phi\) also is claimed support for \(\phi \lor \psi\).
\end{note}

\begin{note}
  Consider again the rectangle.
  The rectangle has certain dimensions, and calculations are only relevant given dimensions.
  So, the reasoning proceeds independently.
  There's nothing in the core part of the reasoning that requires the rectangle itself to have the noted dimensions.
  At no point do I need to appeal to it being true.
  Rather, I background that it is true, and derive additional results.
  {
    \color{red}
    I offer a different perspective on this.
  }

  Contrast, if the alarm is beeping then there is a fire.
  It matters whether or not the alarm really is beeping.

  Here, a useful illustration is something like reliable then reliable on this instance.
  In this type of reasoning, it seems there's something missing between \(\phi\) and \(\psi\).
  So, this wouldn't be a case of \RBV{}.
  That is, there's something additional going on with \(\phi\) = reliable in general to \(\psi\) = true on this occasion.
  Indeed, it seems that depending on how \(\psi\) is understood, then either no \ref{nI:received-info} and hence no \RBV{}, or no \ref{nI:inclusion}, because a single failure is not sufficient to raise a problem with \(\phi\).
\end{note}

\begin{note}
  Finally, a variant on the zebra and red wall cases.

  It doesn't look as the wall \dots
  And, it doesn't look as though it's a cleverly disguised mule.

  This is admittedly an odd proposition.
  So, it seems plausible to obtain this by reasoning.
  A cleverly disguised mule doesn't look like a cleverly disguised mule --- it's cleverly disguised.

  Almost trivial.

  Important here is that no need to \RBV{}.
  Indeed, \RBV{} is strange.
  Isn't at all obvious why not being a cleverly disguised mule would allow one to claim support that it doesn't look like a cleverly disguised mule.

  Of course, the issue is that it's hard to do anything interesting with these propositions.
  Doesn't seem that the claimed support even permits \RBV{}.
  However, further exploration of this topic would go beyond present interest, so we'll stop here.
\end{note}

\begin{note}[Summary]
  In summary, we've expanded on the type of reasoning that~\ref{nI:going-by-value} captures.
  We termed this type of reasoning `\RBV{-}' as the distinguishing feature is reasoning from a proposition having a certain value.
  Two things which follow from this:
  \begin{itemize}
  \item Requires expectations to hold.
  \item And, this requires some care when claiming support for \(\psi\) having value \(v'\).
  \end{itemize}
  So, the task of \ref{nI:inclusion} is to narrow down the conditions to a specific problem --- given~\ref{nI:claimed-support} and~\ref{nI:received-info}.

  Broadly speaking, {\color{red} undercutting defeater}.
  So, \RBV{} won't do for claiming support.
  After discussing~\ref{nI:inclusion} in some detail, we'll turn to the argument.
\end{note}

\paragraph{\ref{nI:inclusion}}

\begin{note}[Core idea]
  The role of~\ref{nI:inclusion} is to capture a relation between an agent's claimed support for \(\phi\) having value \(v\) and claiming support for \(\psi\) having value \(v'\) such that the agent considers their claimed support for \(\phi\) having value \(v\) to depend on the possibility of claiming support for \(\psi\) having value \(v'\).

  Loosely phrased, the agent thinks that the support they've claimed for \(\phi\) isn't really worth much if it not also possible for them to claim support for \(\psi\).

  Bluntly, \ref{nI:inclusion} is here to ensure that claiming support for \(\psi\) having value \(v'\) is a possible and `pressing' defeater for the claimed support that \(\phi\) has value \(v\).
  {
    \color{red} Leads to undercutting\dots
  }
  Note, in contrast to the observation made above with respect to~\ref{nI:claimed-support} and~\ref{nI:received-info} it is \emph{claiming support for \(\psi\) having value \(v'\)} that is a possible defeater for the claimed support that \(\phi\) has value \(v\), rather than \(\psi\) not having value \(v'\).

  In turn, the relation captured by~\ref{nI:inclusion} will lead to a kind of circularity if the were to reason as described in~\ref{nI:going-by-value}.
\end{note}

\begin{note}[Example]
  Idea is captured with the idea of counterexamples.
\end{note}

\begin{note}[\ref{nI:inclusion}]
  Restated:
  \begin{quote}
    \color{blue}
    \begin{enumerate}
    \item[\ref{nI:inclusion}]
      Suppose \(S\) is confident that:
      \begin{enumerate}
      \item[\ref{nI:inclusion:position}] Their claimed support for \(\phi\) having value \(v\) is \mom{} if they are not (given present context) in a position to claim support that \(\psi\) has value \(v'\) (without appealing to \(\phi\) having value \(v\)).
      \item[\ref{nI:inclusion:bound}] If their claimed support for \(\phi\) having value \(v\) is \nmom{} then they would be \nmom{} when claiming support for \(\psi\) for value \(v'\). (when not going by value.)
      \end{enumerate}
    \end{enumerate}
  \end{quote}
\end{note}

\begin{note}[Two clauses]
  The two sub-clauses of~\ref{nI:inclusion} are separated by~\ref{nI:inclusion:position} being a conditional with an antecedent and consequent that describe present circumstances while \ref{nI:inclusion:bound} is a conditional with an antecedent that describe present circumstances and a subjunctive consequent.
  In turn, each sub-clause is stated from the perspective of possible defeaters, but as a result both sub-clauses carry certain implications if the agent is confident that they have successful claimed support for \(\phi\) having value \(v\).

  First, as stated, \ref{nI:inclusion:position} ensures that the agent considers their claimed support \(\phi\) having value \(v\) is mistaken or misled if they are not in a position to claim support that \(\psi\) has value \(v'\).
  In turn, if the agent is confident that they have claimed support \(\phi\) having value \(v\) then the agent should be equally confident that they are in a position to claim support that \(\psi\) has value \(v'\)

  Second,~\ref{nI:inclusion:bound} ensures that the agent is confident that if they were to claim support for \(\psi\) having value \(v'\) then such claimed support for would be \mom{} if their claimed support for \(\phi\) having value \(v\) is \mom{}.\nolinebreak
  \footnote{
    Note, however, that the subjective element of this contraposed form is restricted to the antecedent of the conditional.
  }
  In turn, if the agent is confident that they have claimed support \(\phi\) having value \(v\) then the agent should be equally confident that claimed support for \(\psi\) having value \(v'\) would not be \mom{} if they were to do so.

  The two sub-clauses are closely related, but are distinct.
  It is possible for either to hold without the other.

  \begin{illustration}
    Suppose taught addition.
    Has been told that multiplication reduces to addition, but has not been informed of the details.
  \end{illustration}
  \ref{nI:inclusion:bound} holds but~\ref{nI:inclusion:position} does not.

  \begin{illustration}
    Suppose calculator from scratch.
    Good with arithmetic, but less good with programming.
  \end{illustration}
  \ref{nI:inclusion:position} holds, but~\ref{nI:inclusion:bound} does not.
  For, a whole bunch of additional stuff.
  However, revise scenario so that~\ref{nI:inclusion:bound} does hold.

  So, distinct.
  \ref{nI:inclusion:position} in a position and~\ref{nI:inclusion:bound}, binds.
\end{note}

{
  \color{red}
  Seen examples above.
}

\begin{note}
  `Confidence'

  Then, each sub-clause separately.

  Finally, link to literature.
\end{note}

\begin{note}[`Confidence']
  Our use of the term `confidence' does not require the agent to have claimed support for the conditional content of \ref{nI:inclusion}.
  Nor does out use of `confidence' imply that the claimed support for \(\phi\) \emph{is} mistaken or misled given the identified conditions.
  We are interested only in what makes sense from the agent's perspective.
  Nearby reformulations of \ref{nI:inclusion} may also be true, but confidence is sufficient to recognise a problem.
  To illustrate: If I am confident that the water is poisoned, then regardless of whether I claimed support for the water being poisoned, I will not drink it.
\end{note}

\subparagraph{\ref{nI:inclusion:position}}

\begin{note}[Points that will be covered]
  We will focus on two parts of \ref{nI:inclusion:position}.

  First, on what it is for an agent to be in a position to claim support for some proposition (given present context), as this is an unfortunate source of imprecision.

  Second, we'll clarify the restriction that the agent does not appeal to \(\phi\) having value \(v\) --- though this will be further explored when discussing~\ref{nI:inclusion:bound}.
\end{note}

\begin{note}
  First, an admission.
  Key role here is that get a problem with the claimed support for \(\phi\) having value \(v\).
  Possible that some other condition could do the work.
  However, this is sufficient for the purposes of this paper.
\end{note}

\begin{note}[`(given present context)']
  Imprecise.
  No clear account of what it is to `be in a position' nor what `present context' amounts to.

  Unfortunately, I have no simple characterisation for either.
  Best I have to offer is that the agent is not prevented from claiming support by lack of resources.
  However, it seems to me that isn't really substantially different.
  Resource is broad enough to mean whatever is required for the agent to claim support.
  And, bound to context.

  Instead, walk through considerations with respect to some scenarios.
  Consider issues.
  Argue that don't need to do better than imprecision.
\end{note}

\begin{note}
  Pair of easy cases.
  Clarify to some extent in a position.

  Then, some harder cases.
\end{note}

\begin{note}[Flavours]
  Consider variations on a case where information relating \(\phi\) and \(\psi\) comes from a third party.

  You'll enjoy this flavour of ice cream.

  Claimed support from testimony, roughly.

  Seems fine when discussing things on the train.

  More difficult when tasters are available.
  Here, \ref{nI:inclusion} holds.
\end{note}

\begin{note}
  Difference is that in the first, no way of checking.
  In the latter, there's a way of checking.

  Only issue here is `testimony'.
  With respect to testimony, the response is that in cases of testimony, one can be thought of as appealing to a general truthful property, rather than always truthful.
  So, it's not obvious that testimony is always going to give rise to an instance of \ref{nI:inclusion}.
\end{note}

\begin{note}
  Receiving a letter.
  Unmarked, okay, it's for me.
  Marked, check the address.
\end{note}

\begin{note}[More difficult]
  \dots
\end{note}

\begin{note}[Illustration, testimony]
  To illustrate, consider expert testimony to a layperson.
  Suppose you, the expert, have testified to me, the layperson, that there are exactly five intermediate logics that have the interpolation property.\nolinebreak
  \footnote{Cf.\ \textcite{Maksimova:1977un}}
  From this it follows that there is one intermediate logics that have the interpolation property.
  However, I am quite confident that I would not be in a position to claim support for the latter proposition without your testimony.
  Given that I do not have the expertise involved, any failure by me to claim support that there is a intermediate logic with the interpolation property is uninformative.
  Likewise, given that I am a layperson I'm not in a position to rule out that there aren't intermediate logics with the interpolation property, and therefore I may consider this a potential defeater to your testimony.\nolinebreak
  \footnote{
    Additional example: reports of internal states.

    Without \ref{nI:received-info}, further difference.
    I have a virus scanner.
    Run this on your pc.
    Also, a test pc.
    Test PC contains a know virus, so if the virus scanner is good, then it will identify infection.
    However, no relation between your PC and my test PC.
    All that would be established is that the scanner is not good for claiming support.
    }

  Still, given \ref{nI:claimed-support}, agent may expect \(\psi\) to have value \(v'\), and may claim support.
  And, may expect to have the resources to claim support for \(\psi\) without appealing to \(\phi\) having value \(v\).
  To illustrate, suppose you and I are both experts.
  You claim to have developed a sound and complete proof system for an logic and presented me with a paper containing the system and a proof.
  Given that I have the paper and the expertise, I am confident that I would be mistaken or misled by your testimony if I am not in a position to claim support that the system is sound and complete by working through the paper.\nolinebreak
  \footnote{
    Here, complexity of understanding of having resources shows.
    For, it may be that the reader learns something new, a lemma etc.\ which could be considered a new resource.
    Likewise, one may think that it's fine to continue to follow testimony given a problematic proof as one is confident that the prover has the resources to revise the proof.
    If so, not clear whether conditional holds, and will depend having resources.
    If proof synthesises resources, then may still hold.
    If proof introduces new information, then conditional does not hold.

    No clear answer for these cases.
    Intend to be compatible with your understanding of resources.
    Will only take a stance on this when applying.
  }
\end{note}

\begin{note}
  Even more difficult
\end{note}

\begin{note}
  Coworker.
  Rely on colleague, as the agent doesn't have access to the file.
  But, access is granted quickly after hearing from the colleague.
\end{note}

\begin{note}
  These cases are harder within the broader context of \nI{}.
  Deny \RBV{}.
  Issue is that in both cases, result seems excessive.

  Well, first thing to do is to check that the agent really is claiming support.
  Fine, it seems, for the agent to stop at claiming support for the other agent, and not going any further.
  See no reason to hold that the agent must claim support.

  Still, this isn't quite satisfactory.
  Doesn't seem that bad, and the above suggestion requires a careful understanding of when an agent is required to claim support.
  So, what if the agent does claim support?

  As noted above, views on testimony can sort this out.
  First, by going for `weak' testimony.
  Second, by breaking \ref{nI:inclusion:bound} is the testimony turns out to be a mistake.
  Look, it's not obvious why it would make sense for the agent to claim support, but the point is that \nI{} wouldn't hold.

  Alternatively, Simple restriction is for first time claiming of support.
  Difficulty is a variation of the expert case.
  It isn't obvious that one gets to claim support for the stuff learnt as a layperson when one develops expertise.
  For example, translation between languages.
  Claim support for simple translation.
  When fluent, seems claimed support is distinct, based on broader understanding of the language.
  Not relying on simplifications in learner's dictionary.

  However, other ways of claiming support may also work.
  Arguing for one such way.
  Besides this, \RBV{} is quite strong.
  And, who knows about other types of reasoning.
  In particular, ways in which reasoning \nr{} might work.
\end{note}

\begin{note}[Uninspiring]
  These responses aren't particularly inspiring.

  However, let's look at this from a different angle.
  What's going to follow from insensitivity to context?
  End up with claiming support that does not depend on whether or not the agent is in a position to deal with defeaters.

  Well, the first option is that these never matter.

  Some kind of built in support.
  This comes up with ~\citeauthor{Pryor:2012tq}'s dogmatism (\cite{Pryor:2000tl,Pryor:2012tq}) and various ideas about entitlement (Wright, Burge, etc.)
  For example, Pryor's dogmatism for perception, just having the experience is good enough.
  Question about these kinds of defeaters.
  Reads to me that these kinds of things mean that \ref{nI:inclusion} will never hold.

  Question is whether this extends to all cases, so that \nI{} is trivial, but before pressing this seems too strong.
  Problems in various cases.
  The red room, but in the corner is a switch, flipped to off, but says it's broken.

  Context makes a difference.
  So, to this extent, looks as though there's going to be difficult cases.
\end{note}

\begin{note}[Following doesn't depend on difficult cases]
  Of course, this isn't a general defence of the clauses.
  Rather, that such difficulties can't be avoided.
  Upshot here is that we aren't really interested in such difficult cases.
\end{note}

\begin{note}[`(without appealing to \(\phi\) having value \(v\))']
  The parenthetical clause `(without appealing to \(\phi\) having value \(v\))' ensures that \ref{nI:inclusion} may only be true when the agent is confident that they are in a position to claim support for \(\psi\) having value \(v'\) independent of the conditional content of \ref{nI:received-info}.

  In this respect, \ref{nI:inclusion} requires an independent check on the claimed support for \(\phi\) having value \(v\).\nolinebreak
  \footnote{
    Note also that without the parenthetical clause, \nI{} would deny the possibility of any instance of the reasoning described in \ref{nI:going-by-value}.
  }

  Indeed, not being in a position to claim support for \(\psi\) having value \(v'\) (without appealing to \(\phi\) having value \(v\)) as a potential defeater to claimed support for \(\phi\) having value \(v\) is distinct from the potential defeater of \(\psi\) not having value \(v'\).
  For, an agent may consider \(\psi\) not having value \(v'\) is a potential defeater given \ref{nI:received-info} while being confident that they could not be in a position to claim support for \(\psi\) having value \(v'\) without claimed support for \(\phi\) having value \(v\).

  \begin{itemize}
  \item Illustration
  \item No restriction on why conditional is true.
    \begin{itemize}
    \item Distinction between two ways in which it might be true.
    \end{itemize}
  \end{itemize}
\end{note}



\begin{note}[Inclusion and Association]
  \color{red}
  The illustrations provided offer some intuition, and it seems these will have to do.
  For example, one may consider `in a position' to mean that the agent does not require any novel resources to claim support.
  However, an agent may need to synthesise more fundamental concepts when following a proof, and it is unclear whether the synthesis is `novel information'.
  Similarly, it is difficult to say what the present context is when an agent may phone a friend as a source of testimony.
  In some cases, corroborating testimony may be sufficient to claim support (another plausible instance of `novel information'), while in other cases at issue may be the agent's own understanding (e.g.\ with respect to cases of Inclusion).
  In defence of this latent ambiguity, the specifics will not matter when arguing for the truth of \nI{}.
  And, I suspect the cases to which we apply \nI{} will be sufficiently clear cut.
\end{note}

\subparagraph{\ref{nI:inclusion:bound}}

\begin{note}[Inclusion and Association]
  For \ref{nI:inclusion:bound} we will consider two general ways in which \ref{nI:inclusion:bound} may be true, and provide examples for both.

  The task, then, is to account for why claimed support for some proposition being \nmom{} may imply that claiming support for some other proposition would be \nmom{}.

  We term the ways \incl{} and \asso{}, respectively.
\end{note}


\begin{note}[Inclusion]
  \incl{} is when the same (primary) resources used to claim support for some proposition may be (re)applied to establish a distinct proposition.

  To see why \incl{} leads to instances of~\ref{nI:inclusion:bound} suppose:
  The agent is confident that support claimed for the initial proposition is \nmom{}.
  And, the agent is confident that \incl{} holds with respect to the initial proposition and some other proposition.
  In turn, the agent will use the same resources to claim support for the distinct proposition.
  Therefore, if the agent is confident that the claimed support is \nmom{} for the former proposition then the claim supported must be \nmom{} for the latter proposition, else the agent should not be confident that their claimed support for the former proposition is \nmom{}.

  \begin{illustration}
    Consider claimed support that \(6^{2} \times 6^{3} = 6^{5}\).
    Support has been claimed by understanding basic properties of exponents.
    Hence, an agent may be confident that they are in a position to claim support that \(3^{15} \times 3^{12} = 3^{27}\).
  \end{illustration}
  Indeed, working through problem exercises in a textbook is way of ensuring that one has understood such principles.
  Not that textbooks typically ask for the working out.\nolinebreak
  \footnote{
    Though sometimes.
    For a highly specific example, consider constructing canonical models to prove completeness for various normal modal logics.

    The exercises in textbooks such as~\citetitle{Blackburn:2002aa} require the reader to consider a specific system, so there's no surprise that, e.g., \textbf{K1.1} is sound and complete with respect to the class of frames with a relation function that mirrors a partial function.
    Rather, the task of the exercise is to ensure the reader understands how to reason with canonical models, and if the reader has claimed support with respect to \textbf{K1.1} which is \nmom{} then they should be confident that their claimed support will be \nmom{} when they tackle \textbf{K4.3}.

    See \textcite[210]{Blackburn:2002aa} for the respective exercises.
  }

  {
    \color{red}
    Note here that this is the sort of thing that seems most likely in counterexample type cases.
  }
\end{note}

\begin{note}[\asso{}]
  \asso{} is when claiming support for some proposition ensures the agent is in a position to appeal to some distinct collection of resources for some other proposition.

  \begin{illustration}
    Example here is something like storing a guide in a document.
    Here, the agent has created the document, \(\phi\) is just that the document has all of the info.
    So, if document is good, then contents for \(\psi\).
    This is different, as \(\phi\) is a now a check on the stuff in the document working out.
  \end{illustration}

  Also, ice cream example from above.
\end{note}

\begin{note}[Ways in which \ref{nI:inclusion} may fail to hold]
  Finally, the illustrations given have focused on instances for which \ref{nI:inclusion} holds.
  It is important that there are instances where \ref{nI:inclusion} holds, but it is equally important not to suggest that these are in abundance.
\end{note}

\begin{note}[Failures for~\ref{nI:inclusion:position}]
  There are various ways in which \ref{nI:inclusion} may fail to hold.
  For example, if \(\phi\) is sufficiently general or probabilistic.
  If so, not having the resources to claim support \(\psi\) may not establish much.
  Highlight here is that the issue isn't with \ref{nI:received-info}, values are constrained.
  Rather, issue is with \ref{nI:inclusion}.
  Still, there is a nearby issue with \ref{nI:received-info} if the agent goes from likely to true.
  Here, \ref{nI:received-info} would not hold.
  No interest in this, I think this type of reasoning is fine, but \nI{} simply fails to apply.
\end{note}

\subparagraph{Combined}

\begin{note}[Failure for zebra case]
  Here, not obvious that holds for zebra case, as it's not clear there is an alternative.
\end{note}

\begin{note}[Literature]
  {
    \color{red}
    Place this here as it helps clarify why~\ref{nI:inclusion:position} and~~\ref{nI:inclusion:bound} seem to work well together.
  }
  The circularity here is similar to that proposed by~\cite{Sgaravatti:2013wu}

  \begin{quote}
    (JBA) An argument A is circular relative to an evidential state E iff in order for a subject S in E to have a justified belief in each one of A’s premisses, it is necessary that S has a justified belief in A’s conclusion\nolinebreak
    \mbox{}\hfill\mbox{(\Citeyear[759]{Sgaravatti:2013wu})}
  \end{quote}
  Relative to an agent, really.

  Talk about claiming support, rather than evidence.
  And, details of why it's necessary.
  Also, contrast to some of the details.

  Well, interesting is:
  \begin{quote}
    For my present purposes it will suffice to say that a good test of A’s being necessary for B (and thus of B’s being sufficient for A) is the satisfaction of two subjunctive conditionals. First, if A did not hold, B would not hold; secondly, if B were to hold, A would hold.\nolinebreak
    \mbox{}\hfill\mbox{(\Citeyear[761]{Sgaravatti:2013wu})}
  \end{quote}
  This is very similar to \ref{nI:inclusion}.
  A is \(\psi\) and B is \(\phi\).\nolinebreak
  \footnote{
    \ref{nI:inclusion} was developed independently, though this is probably no surprise given how clumsy~\ref{nI:inclusion} is.
  }

  Still, relative to a certain way of claiming support.
  It's not the case that this idea holds in general.
\end{note}

\begin{note}[Different example.]
  \color{red}
  Truth functional completeness.
  Here, don't go by value.
  Rather, go by interdefinability.
  Indeed, interdefinability plays an important role in getting this condition.
  And, don't even need to go by value.
\end{note}



\newpage

\subsubsection{Argument for \nI{}}
\label{sec:argument-ni-1}

\begin{note}
  \color{red}
  
\end{note}

\begin{note}[Outline]
  As we have worked through the clauses of \nI{} in detail, it is relatively straightforward to demonstrate why \ref{nI:going-by-value} holds given \ref{nI:claimed-support},~\ref{nI:received-info}, and~\ref{nI:inclusion} all hold.

  \ref{nI:claimed-support},~\ref{nI:received-info} are background.
  Focus is~\ref{nI:inclusion}.
  And,~\ref{nI:inclusion} requires that the agent holds that \(\psi\) has value \(v'\) in order to hold that \(\phi\) has value \(\phi\).
  Given this, appealing to the conditional from~\ref{nI:received-info} does nothing.
  Circular.
  Alternatively, a case of \emph{petitio principii} (or Begging the question under a certain disambiguation)\nolinebreak
  \footnote{
    Look, English has moved on.
    It's fine to use `begging the question' in the sense of demanding that a question be asked.
  }

  Objection here is that circularity requires some care.
  Circularity might be fine.

  Simple argument, repeating.
  Don't get conclusion without assumption.
  However, there's no problem if agent has claimed support for assumption.

  Generalising, there's no real issue with the agent showing that they already have support to claim for \(\psi\) having value \(v'\).
  So, task is to show that the agent really is required to \emph{assume} \(\psi\) has value \(v'\) (i.e.\ no claim of support) when they move to reasoning from \(\phi\) having value \(v\).
\end{note}



\begin{note}[Outline]
  \begin{itemize}
  \item Key part of the argument is \RBV{} and \eiS{}
  \end{itemize}

  \begin{enumerate}
  \item Claimed support for \(\phi\)
  \item Aware of relation to \(\psi\)
  \item Hence, by \eiS{}, expect that \(\psi\) has value.
  \item Now, this gets to some tension.
    Key step, however, is linking to being in a position.
  \item In a position.
    So, as seen above this establishes a relation between claimed support for \(\phi\) and claiming support for \(\psi\).
  \item Somehow this is stronger than merely being aware of relation\dots well, it establishes that failing to claim support for \(\psi\) would really be a problem.
  \item So, suppose agent goes by value.
  \item Only works if not mistaken or misled.
  \item Hence, only works if they are in a position to claim support for \(\psi\).
  \item So, to move to value the agent already holds that they're in a position to claim support for \(\psi\).
  \item So, agent is getting \(\psi\) from both \(\phi\) and being in a position.
  \item However, agent doesn't get to discharge being in a position.
  \end{enumerate}
\end{note}


\begin{note}[Revised]
  \begin{enumerate}
  \item Going by value.
  \item Inherit constraints on claimed support for \(\phi\).
  \item So, in position to claim support for \(\psi\).
  \item Nothing particularly problematic here, generally speaking --- claiming support is hard.
  \item However, claiming support for \(\psi\).
  \item So, we now have the assumption of being in position as background for claiming support for \(\psi\).
  \item This is where the issue is, though it's not super obvious.
    \begin{enumerate}
    \item Arguing that there's no way to discharge assumption seems fine.
    \item For, nothing available to the agent.
    \item Difficulty is in understanding why this prevents claiming support for \(\psi\).
    \item Simple idea is that being in a position to claim support also requires the agent to hold that \(\psi\) is the case.
      \begin{enumerate}
      \item If this holds, then the agent hasn't done anything of interest.
      \item For, need to assume what one is trying to claim support for.
      \end{enumerate}
    \end{enumerate}
  \item So, from~\ref{nI:inclusion} if the agent goes by value to \(\phi\) then also go by value to \(\psi\).
    For,~\ref{nI:inclusion:bound} requires that the agent holds that they're not mistaken or misled with respect to \(\psi\) if not mistaken or misled with respect to \(\phi\).
    Hence, if agent is going to value \(\phi\), agent must also be going to value of \(\psi\).
  \item Note,~\ref{nI:inclusion:bound} is the main thing here, but it's benign without~\ref{nI:inclusion:position} (there's a note on this above).
  \item So, we've got \(\psi\) already, before even appealing to the conditional\dots
  \end{enumerate}
  So, now, the remaining issue is why having \(\psi\) is bad.
  \begin{enumerate}
  \item Well, goal is to appeal to the conditional.
  \item However, the conditional doesn't establish anything more than the agent has assumed.
  \item And, no way for \(\psi\) not to be an assumption.
  \end{enumerate}
\end{note}

\begin{note}[\RBV{}]
  First thing to stress is that \RBV{} requires that defeaters don't hold.
  For, the agent has moved from claimed support to value.

  This is difficult, actually.
  For, this is an additional assumption, that will get discharged.

  Need it to be the case that \RBV{} requires assumption that claimed support is neither mistaken or misled.

  This seems fine.
  However, delicate.
  Difference between expecting this not to be the case, and holding that this is not the case.

  Idea, then, could be that with \RBV{} agent assumes expectations are the case.
  And, any reasoning that follows is going to inherit those same expectations.
\end{note}

\begin{note}
  \color{red}
  Important for later!

  I'm going to argue that ability avoids \nI{} in some cases.
  Key here is that it's with \gsi{}, which provides the agent with more information about how to claim support for \(\psi\).
  Hence, this doesn't offer a general resolution to the constraint placed by \nI{}.

  Further, it's going to avoid the problem because the agent doesn't go by value.
  Yet, this can't just be a technicality.
  There's got to be some intuition.
  Here, it's because the agent is reducing \(\phi\) and \(\psi\) to the base basis.
  The issue with \RBV{} is that it separates.
\end{note}

\begin{note}[Two distinct arguments]
  \begin{enumerate}
  \item Focuses on the idea that by moving to \(\phi\) the agent also gets \(\psi\), so appealing to \(\phi \rightarrow \psi\) does no work.
    For, holding that CS for \(\phi\) is good, but then in position for \(\psi\), but then \(\psi\) must also be good.
    Wait, this is also an undischarged assumption.
  \item `Undischargable' assumption.
  \end{enumerate}
\end{note}

\begin{note}[\nI{} argument, state]
  Start with three conditions describing state of agent.

  By~\ref{nI:claimed-support}, agent has claimed support for \(\phi\), though recognises the support may be mistaken or misled.
  It is possible that there's a different value of \(\phi\) (misled), or even if that value, the claimed support is not an indicator (mistaken).

  By~\ref{nI:received-info}, agent has information that value of \(\phi\) constrains value of \(\psi\).
  Use of `information' allows for arbitrarily strong support.
  May assume that this is something known, though do not require for failure.

  Finally, by~\ref{nI:inclusion}, the agent is aware that the claimed support for \(\phi\) includes support for \(\psi\).
  That is, agent may reapply premises and steps to claim support for \(\psi\).
  Hence, the agent does not need to go by value of \(\phi\) to get \(\psi\), as premises and steps used to claim support for \(\phi\) did not require value of \(\phi\) (by \eiS{}).
\end{note}


\begin{note}[\eiS{}]
  \eiS{} again.
  If claimed support, then independent of value.
\end{note}

\subsubsection{Argument for \nI{}}
\label{sec:argument-ni}



\begin{note}[\nI{} argument, \RBV{}]
  From the three conditions describing state, it is possible for agent to claim support for \(\psi\).
  In particular, by reapplying and establishing how claimed support for \(\phi\) includes support for \(\psi\).

  \ref{nI:going-by-value} describes a way of obtaining support that does not appeal to inclusion of support.
  Agent moves from claimed support for \(\phi\) to value of \(\phi\), and given information, this would constrain value for \(\psi\), hence leading to support for \(\psi\) if \RBV{} is permitted.

  With the background provided, from a certain perspective, agent would be bypassing reapplication.
  Possible to claim support for \(\psi\) regardless of value of \(\phi\), but with information, observe constraint on value of \(\phi\) and hence on \(\psi\).
  As agent doesn't need to do work to establish how value is constrained by reasoning by value, this is quite easy.

  Issue is that \incl{} blocks \RBV{}.
  
  However, because \incl{}, problem moving from support for \(\phi\) to value of \(\phi\).
  For, in moving support for \(\phi\) to value of \(\phi\), agent is implicitly requiring value of \(\psi\).
  For, given \incl{}, if \(\psi\) does not have value, then claimed support for \(\phi\) would be mistaken or misleading, and hence would block move to value.
  So, any further reasoning by value is required to already have \(\psi\), and this conflicts with \eiS{} when the agent moves to claiming support for \(\psi\) as the result of \RBV{}.

  From a broader perspective, the agent doesn't get to claim support for \(\psi\) through \RBV{} because any move to value requires \(\psi\) to already be the case given the information the agent has.
  Hence, failure of \eiS{} if agent were to claim support for \(\psi\).
  Because of \incl{}, \RBV{} does not preserve \eiS{}.

  Illustrate, possible that the agent is mistaken/misled, and claimed support for \(\phi\) does not include support for \(\psi\).
  However, if \RBV{}, then claim support for \(\psi\).
  Problematic, because whether inclusion is a test for whether the claimed support for \(\phi\) is any good, but bypassing test when reasoning by value.

  No claiming support by noting value consequence, if failure to show value independent consequence would lead to revision of support.
\end{note}

\begin{note}[Summary, and testimony]
  Final case to summarise:
  Knowledge via testimony.
  This condition doesn't necessarily apply, as agent may not be in position to claim support for what follows from knowledge claim.

  Two reasons for this.
  First, agent may not be in a position to check.
  E.g.\ missing premise, or layperson, e.g.\ missing steps of reasoning.

  Second, agent may not need to \RBV{-}.
  For, if you've testified, then it follows from your statement.
  I don't need to appeal to me having heard from you.
  Instead, given the additional information that I have, you've already made the claim.
  Even if \(\phi\) doesn't have value, this is still an okay reinterpretation of the testimony you have provided.
  Here, to get the intuition, it's really not clear that I need to endorse that I do have the option to check.
  {
    \color{red}
    This point only really makes sense after the argument has been given.
  }
\end{note}

\subsubsection{Illustrations of \nI{}}
\label{sec:illustrations-ni}

\begin{note}[Abstract, so examples]
  Turn to illustrations, and then to how \nI{} applies to \gsi{}.
\end{note}

\begin{note}
  Here, only interest is in support.
  Hence, recognised by the agent that they may be mistaken or misled.
  From this perspective, the issue is not ruling out potential defeaters.
  Similar to knowledge, etc.\ but no requirement that there are no defeaters.
\end{note}

\begin{note}
  May think that this restricts any application of \RBV{} to claimed support for \(\phi\) without value independent.
  This isn't quite right.
  \eiS{} keeps focus on \(\psi\).
  Only committed to \(\psi\) being a problem.
  Potential issue is no worse than any other instance of claim to support --- possibility of being mistaken or misled.
  If \(\psi\) ends up being used, then there's going to be a gap, where agent isn't in position to claim support by value, but unless eventual consequence is in turn used for \(\psi\), no clear problem --- at least not without stronger assumption.
\end{note}

\begin{note}
  \ESU{} is going to require the agent to reason from premises and steps `included' in claimed support for \(\phi\) in order to claim support for \(\psi\).
\end{note}


\begin{note}[Examples]
  Examples are somewhat difficult, due to complexities of state.
\end{note}

\begin{note}[Picture book]
  \begin{illustration}
    If not genuine, then missing serial number.
  \end{illustration}
  No need to reinspect, faults are support, so no serial number.
\end{note}

\begin{note}[Logic proof]
  \begin{illustration}
    If conjunction and negation are truth functionally complete, then disjunction and negation are truth functionally complete.

    And, claim of inclusion.
  \end{illustration}

  Here, in proving completeness, expressing other connectives.
  So, inclusion because agent will have shown how to switch between conjunction and disjunction.

  Well, claimed support for conjunction and negation.
  So, yes.
\end{note}

\begin{note}[Treasure]
  \begin{illustration}
    Claimed treasure only if learnt secret.
  \end{illustration}
  A little more interesting, as here, agent is going to have done something to learn secret when claiming support for treasure, but may not recognise that they've learnt the information.
  Of course, may be wrong treasure.
  Again, seems bad.
  But, if treasure then sell for \$X, seems fine.

  Useful, as earlier examples may seem to rely on easy checks, but putting pieces together to reveal secret may be quite difficult.
\end{note}

\begin{note}[Problematic]
  \begin{illustration}
    I walked \(15k\) yesterday, and it would not be true that I walked \(15k\) if I had only walked \(14k\).
  \end{illustration}
  In contrast to other cases, the conclusion is fine.
  Problem here is the reasoning.

  Going by value.
  No way to go to \(15k\) without having walked more than \(14k\).
  So, already require that not only \(14k\) is true when reasoning-by-value.
  Claim to support fails because I'm highlighting what's got to be true given support.

  However, clear to observe that claim to support includes not only \(15k\).
  Easy to reason that no matter whether I did walk \(15k\), that the support claimed for walking \(15k\) extends to cover not only \(14k\).

  Reasoning-by-value \emph{is} strange in this scenario.
  Do not need it to be the case that \(15k\) is true in order to deny only \(14k\).


  Useful to note, as there is still some reasoning with the value.
  However, the reasoning does not depend on specific value.
\end{note}

\begin{note}[Knaves]
  \begin{illustration}
    If X is speaking falsely, then Y is speaking truthfully.

    Knave says a bunch of things that you've got could support for being false, but could be true.
  \end{illustration}
  Variation on Knave problems.
  Again, there may be intuition that solving the problem is easily in reach, but I think this is a mistake.
  Knave problems are hard, and the difficulty doesn't seems to make a difference.
\end{note}

\newpage

\subsubsection{Contrast to other conditions}
\label{sec:contr-other-cond}

\begin{note}
  Two conditions.

  First, \citeauthor{Wright:2011wn} on warrant transmission.

  Second, \citeauthor{Weisberg:2010to} on bootstrapping.
\end{note}

\begin{note}
  Use to argue that \nI{} is unique.

  Also, observe some interesting things about \nI{}.

  \citeauthor{Wright:2011wn} by difference in extension.
  \citeauthor{Weisberg:2010to} by difference in intension.

  Respective approaches are motivated by ease of demonstrating the relevant difference in extension and intension.
  \citeauthor{Wright:2011wn}'s template(s) match scenarios fairly well, and so extension.
  \citeauthor{Weisberg:2010to}'s make certain things explicit with work for difference in intension.

  However, will suggest that observations made with respect to \citeauthor{Weisberg:2010to} also extend to \citeauthor{Wright:2011wn}.
\end{note}

\paragraph{Wright on warrant transmission (failure)}

\begin{note}[How transmission failure relates]
  Inclined to think these are really the same.

  Note, in particular, \citeauthor{Wright:2000tq} is interested in transmission of \emph{second-order} warrant.
  So, not about whether the agent has warrant, but whether the agent may \emph{claim} to have warrant.
  (\Citeyear[89]{Wright:2011wn})

  In parallel, \nI{} is about claiming support, and not about whether the agent has support.
\end{note}

\begin{note}
  Basic ideas go back (at least) to the Proper Execution Principle of~\textcite{Wright:1991vn}, and in particular the \widt{} of~\textcite{Wright:2000tq} and (\Citeyear{Wright:2003aa}).\nolinebreak
  \footnote{See also~\textcite{Wright:1986ug,Wright:2002uk} and \textcite{Wright:2004uo}.}

  The \widt{} is as follows:
  \phantlabel{widt}
  \begin{quote}
    A body of evidence, \emph{e}, is an information-dependent warrant for a particular proposition P if regarding \emph{e} as warranting P rationally requires certain kinds of collateral information, \emph{I}.
    Some examples of such \emph{e}, P and \emph{I} [\dots] have the feature that elements of the relevant \emph{I} are themselves entailed by P (together perhaps with other warranted premises).
    In that case, any warrant supplied by \emph{e} for P will not be transmissible to those elements of \emph{I}.\nolinebreak
    \mbox{}\hfill\mbox{(\Citeyear[143]{Wright:2000tq})}
  \end{quote}

  The ellipses skip a quick illustration given by~\citeauthor{Wright:2000tq} in favour of the following illustration.
  \begin{quote}
    \vspace{-\baselineskip}
    \begin{illustration}
      You go to the zoo, see several zebras in an enclosure, and opine that these animals are zebras.
      Well, you know what zebras look like, and these animals look just like that.
      Surely you are fully warranted in your belief.
      But if the animals are zebras, then it follows that they are not mules painstakingly and skilfully disguised as zebras.\linebreak
      \mbox{}\hfill\mbox{(\Citeyear[154]{Wright:2000tq})}
      \newline\mbox{ }
    \end{illustration}
  \end{quote}

  Here, the body of evidence, \emph{e}, is what you've seen, the proposition P is that those animals are zebras.
  At issue is whether the warrant for P transmits to the proposition that those animals are not mules <adjectives> disguised to look just like zebras.
  In other words, at issue is whether the proposition that those animals are not mules <adjectives> disguised to look just like zebras is collateral information required for what you've seen to warrant those animals being zebras.

  From a broader perspective, the relevant collateral information is, well, there need be no specific collateral information across all possible ways of filling out the remaining details of the scenario, so let's say the collateral information is that things are as they appear.
  If so, the noted proposition is certainly required.

  More specifically, you're at a zoo, so something looking like a zebra seems sufficient to claim warrant that the thing is a zebra, and hence not a disguised mule.
  As such, \citeauthor{Wright:2000tq} holds there is a problem because, generally speaking, \dots

  \begin{quote}
    \dots\space there are external preconditions for the effectiveness of your\linebreak method---casual observation---whose satisfaction you will very likely, without compromise of the warrant you acquire for those beliefs, have done nothing special to ensure.

    [\dots]

    Can the warrants you acquire licitly be transmitted to the claim that those preconditions \emph{are} met---or at least that they are not frustrated in those specific respects?
    It should seem obvious that they cannot.\linebreak
    \mbox{}\hfill\mbox{(\Citeyear[154]{Wright:2000tq})}
  \end{quote}
  We won't go further into why \citeauthor{Wright:2000tq} thinks the result should seem obvious.
  Rather, the above should give you an idea of the phenomenon \citeauthor{Wright:2000tq} is interested in.
  And, with this, we can begin a comparison with \nI{}.
\end{note}

\begin{note}
  In relation to \nI{}, similar idea of undercutting.\nolinebreak
  \footnote{
    Indeed, \citeauthor{Wright:1991vn} notes that Stephen Yablo suggested the kind of defeat in question might be called undercutting in reference to \citeauthor{Pollock:1987un} (\Citeyear[95,fn.9]{Wright:1991vn}).

    I should perhaps note here that I developed~\nI{} after struggling to apply the ideas of \citeauthor{Wright:2011wn} to the scenarios of interest involving ability.
    And, after developing an initial draft of~\nI{} I took to the literature to see if there were any developed ideas that are either equivalent or imply \nI{}.
    This, quite naturally, led me to \citeauthor{Pollock:1987un}'s distinction between overriding and undercutting defeaters, and some of the references above which use `undercutting' in a broader sense than \citeauthor{Pollock:1987un}'s original formulation.
    Hence, it seemed to me that framing \nI{} in terms of identifying something of an undercutting defeater might be a helpful guide.
    If I had found this footnote earlier, I may have had an easier time developing the initial draft of~\nI{}.
  }
  {
    \color{expand}
    Information-dependence blocks transmission of warrant, but does not suggest anything about the relevant elements of \emph{I}.
  }

  Further, parallels between the first part of this and \eiS{}.
  different, but seem to go for the same idea.
  \eiS{} motivated by fallibility, and the first part amounts to fallibility.

  \citeauthor{Wright:2000tq} views this as a requirement, I haven't made this move.
  Significant part of \citeauthor{Wright:2000tq}\nolinebreak
  \footnote{
    See Pryor.
    This is what dogmatism denies.
  }
  , but I don't think this is the thing to focus on.\nolinebreak
  \footnote{
    You might be inclined to think understanding of claimed support should be strengthened.
    I don't want to take a stance on this, and hence problematic for me to distinguish on this basis.
  }

  {% to delete?
    Still, looking at the \emph{form} of \nI{} and \citeauthor{Wright:2000tq}'s \widt{}, there is a difference.
  }

  Instead
  \nI{} is concerned with the way in which in agent uses claimed support for a pair of propositions to claim support (or warrant) for some other proposition.
  By contrast, the \widt{} is concerned with preconditions for claiming warrant (or support) for a proposition that `might' be used to claim support.

  How agent claims support for \(\psi\) given claimed support for \(\phi\) and an implication from \(\phi\) to \(\psi\).

  \(\phi\) and \(\phi \rightarrow \psi\).
  Whether evidence, or claimed support, for this pair requires some collateral information entailed by \(\psi\).

  So, \nI{} doesn't care too much about \(\phi\) and \(\phi \rightarrow \psi\) whereas the template does.

  This is the thing of interest.

  Note, not possible to apply the template in a different way.

  So, in terms of \eiS{}.
  If going by value, require \(\psi\) to be the case.

  For template, need some warranted proposition P.
  Can't be \(\psi\), as template needs warrant, which we're denying.
  So, need something between \(\phi\) and \(\phi \rightarrow \psi\) and \(\psi\).
  Seems there's no proposition here.

  At issue is whether this difference in form corresponds to a difference in substance.
\end{note}

\begin{note}[The revised template]
  The \widt{} is intuitive, but has some downsides.

  Foremost, we would like additional clarity with respect to collateral information.
  As things stand, it's a little vague as to what constitutes and external preconditions for the effectiveness of a method.
  Further exposition might resolve this problem, but \dots

  More significant the \widt{} has been surpassed.\nolinebreak
  \footnote{
    This is a somewhat subtle issue.

    The revised template is, strictly speaking, a revision of the disjunctive template.
    And, \citeauthor{Wright:2002uk} initially distinguished the two templates:

    The \widt{} was designed to identify failures of transmission following from accumulation of defeasible evidence.

    And, by contrast, the disjunctive template was designed to identify failures of transmission following from by some faculty, such as perception or memory.

    See, for example, \textcite{Wright:2002uk} in which both templates are discussed separately, and \textcite[91]{Wright:2011wn} where the difference in motivation is restated.

    Still, \citeauthor{Wright:2011wn} observes that both templates the `base' for failure of transmission is the same in both cases.
    And, in turn, that the initial formulation of the disjunctive template yields unintuitive results when applied to cases covered by the \widt{} is a significant problem.
    (\Citeyear[91]{Wright:2011wn})

    So, \wrt{} is, strictly speaking, not a revision of the \widt{}, but rather the disjunctive template.
    However, \wrt{} is also designed to apply to the cases covered by the \widt{}, given that \citeauthor{Wright:2011wn}
    holds that both the \widt{} and the disjunctive template capture the same core phenomenon.
    Therefore, we have omitted these turns from the body of the paper.
  }
  (\Citeyear[90]{Wright:2011wn})
  This doesn't prevent a comparison, as such, but it may lead one to consider the comparison disingenuous.
  I don't think this is the case, and I began with the information-dependence as it is the spirit, rather than the letter, of the template which is at issue.

  So, to contrast \citeauthor{Wright:2000tq}'s template and \nI{} in detail, let's switch to \phantlabel{wrt}\citeauthor{Wright:2011wn}'s revised template:

  \begin{quote}
    Where A entails B, a rational claim to warrant for A is not transmissible to B if there is some proposition C such that:
    \begin{enumerate}[label=\roman*., ref=(\roman*)]
    \item\label{WT:i} The process/state of accomplishing the relevant putative warrant for A is subjectively compatible with C's holding: things could be with one in all respects exactly as they subjectively are yet C be true
    \item\label{WT:ii} C is incompatible (not necessarily with A but) with some presupposition of the cognitive project of obtaining a warrant for A in the relevant fashion, and
    \item\label{WT:iii} Not-B entails C\nolinebreak
      \mbox{}\hfill\mbox{(\Citeyear[93]{Wright:2011wn})}
    \end{enumerate}
  \end{quote}
  Where
  \begin{quote}
    A presupposition of a cognitive project is any condition P such that to doubt P (in advance of executing the project) would rationally commit one to doubting the significance, or competence of the\linebreak project, irrespective of its outcome.\nolinebreak
    \mbox{}\hfill\mbox{(\Citeyear[92]{Wright:2011wn})}
  \end{quote}

  In relation to the \widt{} discussed above:
  \citeauthor{Wright:2011wn}'s account of presuppositions of cognitive projects clarifies what collateral information amounts to.
  Cases of transmission failure are going to arise when one attempts to claim warrant for a condition for which doubt toward would undercut any outcome of the project.\nolinebreak
  \footnote{
    What matters is whether the relevant cognitive project has a presupposition of this kind, not whether the agent has done `anything special to ensure such presuppositions are satisfied.
    }

  In turn,~\ref{WT:i} to~\ref{WT:iii} detail how warrant for the relevant proposition depends on such collateral information.
\end{note}

\begin{note}
  The core intuition remains the same:
  Failure of transmission from a fixed proposition to conditions that need to be met in order for the agent to claim warrant for the fixed proposition.
\end{note}

\begin{note}[Applied to a case]
  Let's apply \wrt{} to the illustration used above to check:

  The relevant instances of A and B are, respectively:
  \begin{itemize}
  \item[A.] Those animals are zebras
  \item[B.] Those animals are not mules disguised to look like zebras
  \end{itemize}
  And, we may take C to be not-A. (\Citeyear[90]{Wright:2011wn})

  Now,~\ref{WT:i} to~\ref{WT:iii} are satisfied:

  \begin{itemize}
  \item[{\hyperref[WT:i]{i:}}] The process of accomplishing putative warrant for A is that the animals appear to be zebras, and things could be exactly as they \emph{appear} to be and yet the animals are not zebras.
  \item[{\hyperref[WT:ii]{ii:}}] The animals not being zebras is incompatible with A, of course --- given that C is not-A.\nolinebreak
    \footnote{
      For more details: (\Citeyear[90--96]{Wright:2011wn})
    }
    And, more broadly, the animals not being zebras is incompatible with moving from appearance to fact.
  \item[{\hyperref[WT:iii]{iii:}}] If those animals \emph{are} mules disguised to look like zebras, then those animals are not zebras.
  \end{itemize}
  Hence, \wrt{} identifies a failure of transmission in much the same way as we saw above with respect to the \widt{}.
\end{note}

\begin{note}[Entailment]
  Now, turning to the comparison proper.

  First, we'll map A and B from \wrt{} to \(\phi\) and \(\psi\), respectively, from \nI{}.

  An immediate difference is that \wrt{} requires an entailment between the relevant A and B while \nI{} requires that the agent has claimed support that if \(\phi\) has value \(v\) then \(\psi\) has value \(v'\).
  To keep things simple, we'll assume that \ref{nI:received-info} is strengthened to claimed support for an entailment.
  \wrt{} doesn't require that the agent has warrant, or has claimed support, for the entailment between A and B, and hence will apply in the case that the agent has.
    So, \ref{nI:received-info} may be seen as an instance of \wrt{}'s initial condition with some superfluous detail.

  Likewise, \wrt{} is concerned with whether a claim to warrant for A is transmissible to B in general, and so not assume the agent has claimed warrant for A.
  So, as~\ref{nI:claimed-support} requires that the agent has claimed support for \(\phi\), we may consider~\ref{nI:claimed-support} as superfluous from the perspective of \wrt{}.
\end{note}

\begin{note}[Gist of why these are different]
  Now, with \ref{nI:received-info} and~\ref{nI:claimed-support} sidelined, we're left with~\ref{nI:inclusion} and~\ref{nI:going-by-value}.
  And, it seems focus should be on~\ref{nI:inclusion}.
  For,~\ref{nI:going-by-value} is (primarily) about the way in which the agent may go about claiming support for \(\psi\) and \nI{} only limits an agent claiming support in such a way if~\ref{nI:inclusion} holds.

  Hence, it seems to me the question is whether~\ref{WT:i} -- \ref{WT:iii} from \wrt{} and~\ref{nI:inclusion} do different things.
\end{note}

\begin{note}[Two questions]
  Let's break this down into two questions.

  \begin{enumerate}
  \item Whether an instance of~\ref{nI:inclusion} obtaining means that the agent makes a presupposition of the kind identified by \wrt{}.
  \item And, conversely, whether an instance~\ref{nI:inclusion} obtains if the agent has made a presupposition of the kind identifies by \wrt{}.
  \end{enumerate}
\end{note}

\begin{note}[Second question]
  The second question is straightforward to answer in the negative.

  Testimony, sight, whatever.
  Here, doesn't need to be any other way for the agent to claim support for the relevant proposition.
  {
    \color{red}
    To illustrate, zebra.
    Not in a position to claim support by some other way that moving from appearance is bad.
  }
\end{note}

\begin{note}[First question]
  The first question is more involved, and requires some care.

  Consider again the general presupposition that things are as they appear.
  Or, even more generally, that one is not in some sceptical scenario, such as a dream or a vat (cf.~\Citeyear{Wright:2002uk}, \Citeyear[97--98]{Wright:2011wn}).
  The difficult here is that it's easy to trivialise the question if the relevant presupposition is any presupposition.
  For, it seems any cognitive project will require some presupposition.

  Instead, the question is whether~\ref{nI:inclusion} obtaining means there is some \emph{related} presupposition.

  And, it seems this need not the case.

  For,~\ref{nI:inclusion} is, intuitively, about whether an agent is confident they have some way to claim support for \(\psi\), other than appealing to \(\phi\) and an implication from \(\phi\) to \(\psi\).
  But, it doesn't seem to follow that doubt about whether the agent has some other way of claiming support for \(\psi\) prior to claiming support for \(\phi\) and the implication from \(\phi\) to \(\psi\) would undercut claiming support for \(\phi\) or the implication from \(\phi\) to \(\psi\).

  I suspect this point is best argued for by illustrations, and we will consider a handful below.
  Still, it may be helpful to first outline the target of such illustrations in some detail.

  \ref{nI:inclusion} is about whether an agent is confident they have some other way to claim support for \(\psi\), but consists of two parts.
  \ref{nI:inclusion:position} requires that the agent is confident that the support claimed for \(\phi\) and would be \mom{} if the agent is not in a position to claim support for \(\psi\) some other way.
  And,~\ref{nI:inclusion:bound} requires that the agent is confident that the claimed support for \(\phi\) is a guarantee of sorts for claiming support for \(\psi\).

  Now, we're interested in deriving a related presupposition from \ref{nI:inclusion}.
  Still, the presupposition needs to be with respect to claimed support.
  So, as~\ref{nI:inclusion:bound} is a condition which concerns (as yet) unclaimed support, the related should follow from~\ref{nI:inclusion:position} --- in particular, from the possibility of the claimed support for \(\phi\) being \mom{}.

  However, claimed support for \(\phi\) being \mom{} reduces to either the claimed support indicating \(\phi\) has some value it does not have (misled) or the claimed support relies on factors that do not indicate the value of the \(\phi\) (mistaken) --- Cf.~\autoref{prop:defs-for-CS}.

  The point here is that claimed support being \mom{} is a relatively broad phenomenon.
  For example, an instance of inductive support may indicate the value of \(\phi\), and hence not be mistaken, but misled due to constrained sampling.
  Consider, by way of quick illustration, testing a random number generator by sampling its output.
  It may take a significant sample size to identify a bias, and hence bug in the source code.

  Yet, \citeauthor{Wright:2011wn} notion of a presupposition requires that doubt about the presupposition is such that doubt about the presupposition, in advance of following through on the project, would \emph{require} doubt about the significance or competence of the project --- regardless of its outcome.

  So, suppose an instance of~\ref{nI:inclusion} may obtain because an agent has claimed inductive support for \(\phi\), has claimed support that \(\phi\) entails \(\psi\), and has some independent check on whether \(\psi\) is the case.

  If such an instance of~\ref{nI:inclusion} obtaining means that the agent makes a presupposition of the kind identified by \wrt{}, then the relevant presupposition should concern the nature of the claimed inductive support.

  However, it seems fundamental to claimed inductive support that one may doubt the inductive support is not misled without a requirement that one doubts the significance, or competence, of claiming such inductive support.

  I may doubt that I have obtained a sufficiently large sample to conclude that there are no bugs in the source code of the random number generator without being required to doubt the significance, or competence, of the sample acquired.
\end{note}

\begin{note}
  To summarise:
  \ref{nI:inclusion} concerns (an agent's confidence in) a particular kind of relationship holding between claimed support for \(\phi\) and claiming support for \(\psi\) from the perspective of whether the respective instances of support are (or would be) \mom{}.
  This relationship may arise from claimed inductive support for \(\phi\).
  If so, a positive answer to the first question would require a corresponding presupposition with respect to the claimed indicative support for \(\phi\).
  Yet, such a presupposition seems incompatible with the nature of inductive support.
\end{note}

\begin{note}
  Stress, briefly, that this does not indicate anything problematic about \citeauthor{Wright:2011wn}'s template.
  I'm inclined to think the template is sound.
  The issue is whether (at least some) of the instances captured by \nI{} fall outside the scope of \citeauthor{Wright:2011wn}'s template.
  Given that both \nI{} and \citeauthor{Wright:2011wn}'s template are sufficient, there's no tension between the two if it is the case.
\end{note}

\begin{note}
  Let's now return to~\autoref{ill:CE:main} from the start of this section in which we examined a researcher may claim support that there are no counterexamples to a theory they have developed.

  Given that we have already seen how \nI{} applies to both illustrations, and outlined the theoretical difference between \nI{} and \wrt{}, we will focus only on why \wrt{} does not seem to apply to the illustration.
  In particular, why it seems there is no plausible candidate for the required `C proposition' of \wrt{}.
\end{note}

\begin{note}[\autoref{ill:CE:main}]
  \autoref{ill:CE:main} considered a researcher who has claimed inductive support for some theory.
  The instance of reasoning we took interest with was as follows:

  \begin{itemize}
  \item I have claimed support that the theory is adequate.
  \item So, given the claimed support, theory is adequate.
  \item Therefore, as the theory is adequate, given the claimed support, it follows that there are no counterexamples.
  \item Hence, I claim support that there are no counterexamples to the theory.
  \end{itemize}

  As we assumed an entailment from an adequate theory to an absence of counterexamples to the theory, we have the following two instances of A and B with respect to \wrt{}:

  \begin{itemize}
  \item[A.] The theory is adequate
  \item[B.] There are no counterexamples to the theory.\nolinebreak
    \footnote{
      With respect to~\autoref{ill:CE:colleague}, we would have:
      \begin{itemize}
      \item[B.] The colleague has failed to identify a counterexample to the theory.
      \end{itemize}
    }
  \end{itemize}

  If we are to identify failure of warrant transmission from A to B via \wrt{}, then there must be some proposition C such that, paraphrased:

  \begin{enumerate}[label=\roman*., ref=(\roman*)]
  \item\label{wrt:CE:maini} The process of claiming warrant for the theory being adequate is subjectively compatible with C holding.
  \item\label{wrt:CE:mainii} C is incompatible with either the adequacy, or some presupposition of the cognitive project of claiming warrant for the adequacy, of the theory.
  \item\label{wrt:CE:mainiii} The existence of a counterexample to the theory entails C.
  \end{enumerate}

  Well,~\ref{wrt:CE:maini} seems okay.
  Interested in claimed inductive warrant/support.
  And, claiming inductive support seems subjectively compatible with an entailment from some counterexample holding.
  It seems possible that things could be exactly as they subjectively are, yet the theory is inadequate because there is an unobserved instance of the phenomenon which constitutes a counterexample to the theory.

  So,~\ref{wrt:CE:mainii} and~\ref{wrt:CE:mainiii}.

  Working backwards.

  From~\ref{wrt:CE:mainiii}:
  C needs to be entailed by the existence of a counterexample.

  Paired with~\ref{wrt:CE:mainii}, the existence of a counterexample needs to entail something that is incompatible with either the adequacy, or some presupposition of the cognitive project of claiming warrant for the adequacy, of the theory.


  The problem:
  Claiming inductive warrant.
  Seems compatible with some counterexample holding.
  Applied to various instances of the phenomenon, and the theory holds up.
  Possible that it doesn't hold up under some instance of the phenomenon.

  So,
  Suppose the existence of a counterexample entails something that is incompatible with either the adequacy, or some presupposition of the cognitive project of claiming warrant for the adequacy, of the theory.
  Then, it seems the theory denies the possibility of such a counterexample.

  The difficulty is that we're talking generally about some theory for which the researcher has claimed inductive warrant for.
  I see no reason to think that any theory which fits this broad description will deny the possibility of certain counterexamples.

  There, may be that there are assumptions.
  Theories are built on other theories.
  However, interest is in a counterexample to the theory --- not a counterexample to theoretical foundations.

  Claiming warrant that there are no counterexamples in general seems to be the issue, rather than the specific kind a counterexample that would be required for \wrt{} to apply.

  Indeed, we can revise the relevant B instance given \citeauthor{Wright:2011wn}'s notion of a presupposition:
  \begin{itemize}
  \item[B\('\).] No counterexample consistent with the presuppositions.
  \end{itemize}

  Evaluation of the reasoning seems the same.
  Indeed, natural assumption that there are no such presuppositions, so the concerns raised in~\autoref{ill:CE:main} remain.
\end{note}

\begin{note}[Looking ahead]
  \color{later}
  Difference is one thing, but also difference with respect to cases of interest.
  So, looking ahead, ability.

  Simple variation on second example.
  Ability to demonstrate that instance of phenomenon is covered by theory/not a counterexample.

  Follows from understanding of the theory.
  Seems just as bad.
  And, \citeauthor{Wright:2011wn} doesn't apply to either.
  Just need a little more work.
  Claiming support for general ability, so we add claimed (inductive) support for theory together with understanding of theory.
  Follows to specific ability as instance of general.

  So, while not focusing on cases involving ability from perspective of motivating \nI{}, differences here still relevant.
\end{note}

\paragraph{Weisberg}

\begin{note}[Intro to \wnf{}]
  Case.
  Condition.
  Contrast.

  Applies to inductive reasoning.
  \nI{} isn't strictly concerned with inductive reasoning.
  However, application is focused on this, and we have appealed to inductive reasoning extensively when contrasting \nI{} to \citeauthor{Wright:2011wn}'s templates.
\end{note}

\begin{note}[Bootstrapping]
  To illustrate \wnf{}, let's consider a case of bootstrapping introduced by~\textcite{Vogel:2000tl}'s --- here following \citeauthor{Weisberg:2010to}'s presentation:
  \begin{quote}
    \begin{illustration}\label{ill:gas-gauge}
      \emph{The Gas Gauge}. The gas gauge in \nagent{9}'s car is reliable, though she has no evidence about its reliability.
      On one occasion the gauge reads F, leading her to believe that the tank is full, which it is.
      She notes that on this occasion the tank reads F and is full.
      She then repeats this procedure many times on other occasions, eventually coming to believe that the gauge reliably indicates when the tank is full.\nolinebreak
      \mbox{}\hfill\mbox{(\Citeyear[526--527]{Weisberg:2010to})}\linebreak
      \mbox{}
    \end{illustration}
  \end{quote}
  \citeauthor{Vogel:2000tl} argued that kind of reasoning present in~\autoref{ill:gas-gauge} is a problem for reliabilist theories of knowledge, and others have argued the problem may be extended further (see \textcite[\S1]{Weisberg:2010to} for more details).

  However, our interest in the reasoning present in~\autoref{ill:gas-gauge} and \wnf{} is merely that the reasoning is intuitively problematic, \wnf{} is an account of why, and \wnf{} may capture the same phenomenon as \nI{}.
\end{note}

\begin{note}[No feedback]
  \begin{quote}\phantlabel{wnf}
    \textbf{No Feedback} If
    \begin{enumerate*}[label=(\roman*)]
    \item\label{W:NF:i} \(L_{1}-L_{n}\) are inferred from \(P_{1}-P_{m}\), and
    \item\label{W:NF:ii} \(C\) is inferred from \(L_{1}-L_{n}\) (and possibly some of \(P_{1}-P_{m}\)) by an argument whose justificatory power depends on making \(C\) at least \(x\) probable,\nolinebreak
      \footnote{
        There may be some ambiguity here.
        As we will see when examining an illustration below, the arguments justificatory power should be read in terms of depending on \emph{having made} \(C\) at least x probable rather than \emph{establishing that} \(C\) at least \(x\) probable.
        By contrast, the following clause requires that \(P_{1}-P_{m}\) \emph{are making} \(C\) at least \(x\) probable without the help of \(L_{1}-L_{n}\).
      }
      and
    \item\label{W:NF:iii} \(P_{1}-P_{m}\) do not make \(C\) at least \(x\) probable without the help of \(L_{1}-L_{n}\), then the argument for \(C\) is defeated.\linebreak
      \mbox{}\hfill\mbox{(\Citeyear[533--534]{Weisberg:2010to})}
    \end{enumerate*}
  \end{quote}
  Where `\(P\)' stands for a premise(s), and `\(L\)' for a lemma(s). (Cf.~\Citeyear[533]{Weisberg:2010to})

  Again, we have a condition in which an argument would be undercut.
  \wnf{} suggests only that the argument for \(C\) would be defeated, but leaves open the status of \(C\).
\end{note}

\begin{note}[\wnf{} intuition]
  \citeauthor{Weisberg:2010to} motivates with the following intuition.
  \begin{quote}
    The idea is that the amplification of an already amplified signal distorts the original signal, resulting in feedback, and bootstrapping is just ``epistemic feedback''.
    Bootstrapping is an undesirable result of amplifying the output of ampliative inference without restriction.\linebreak
    \mbox{}\hfill\mbox{(\Citeyear[534]{Weisberg:2010to})}
  \end{quote}

  To summarise.
  {
    \color{red}
    The basic idea here is that going from \(P_{1}-P_{m}\) to \(L_{1}-L_{n}\) is seen as an inductive step.
    The relevant \(L_{1}-L_{n}\) go beyond \(P_{1}-P_{m}\).
    And, this means that \(L_{1}-L_{n}\) can't be combined with \(P_{1}-P_{m}\) to draw further inferences.
  }
  {
    \color{red}
    \phantlabel{wnf:expectation}
    Note, doesn't rule out \(L_{1}-L_{n}\).
    Here, similar to expecting that defeaters don't hold.
    Or, following \citeauthor{Weisberg:2010to}, drawing conclusions from evidence.
  }
\end{note}

\begin{note}
  After walking through how \wnf{} applies to~\autoref{ill:gas-gauge} we will motivate a connexion between \wnf{} and \nI{}, before arguing that the two are sufficiently distinct.
\end{note}

\begin{note}
  The overall conclusion \nagent{9} draws in~\autoref{ill:gas-gauge} is that the gauge reliably indicates when the gas tank is full.
  Still, this overall conclusion is drawn from repeated instances of reasoning on particular occasions that concludes that the gauge is reliable on that occasion.
  And, the fault identified by \wnf{} concerns the reasoning on particular occasions.
  Intuitively, if \nagent{9} fails to establish the reliability of the gauge on any particular occasion by the particular instances of reasoning, then the conclusions of those particular instances of reasoning are unavailable for \nagent{9} to draw the general conclusion.

  So, to begin let us summarise the pattern to which each particular instances of reasoning conforms:

  \begin{enumerate}
  \item\label{W:GG:i} The gauge is reliable. \hfill (Background assumption)\nolinebreak
    \footnote{
      Have as background assumption because in the original, \nagent{9} skips over this as an explicit step.
      However, following \citeauthor{Weisberg:2010to} possible to reformulate to some level of probability sufficient to go to 3, such that the overall result of argument is to raise probability. (\Citeyear[528]{Weisberg:2010to})
    }
  \item\label{W:GG:v} It is sufficiently likely that the gauge is functioning correctly on this occasion. \hfill \mbox{(From~\ref{W:GG:i}, `Amplification')}
  \item\label{W:GG:ii} The gauge reads full. \hfill (Observation)
  \item\label{W:GG:iii} So, the tank is full. \hfill (From~\ref{W:GG:v} \&~\ref{W:GG:ii})
  \item\label{W:GG:iv} Hence, the gauge is functioning correctly on this occasion. \hfill (From~\ref{W:GG:ii} \&~\ref{W:GG:iii})
  \end{enumerate}

  The `feedback' in this reasoning pattern involves establishing (an instance of) the reliability of the gauge from an assumption that the gauge is reliable.

  From the perspective of \wnf{} we have:
  \begin{itemize}
  \item[P:] The gauge reads full.
  \item[L:] The tank is full.
  \item[C:] The gauge is functioning correctly on this occasion.
  \end{itemize}

  And, each of the clauses of \wrt{} are satisfied, for:
  \begin{itemize}[labelwidth=\widthof{(iii)}]
  \item[{\hyperref[W:NF:i]{i:}}] That the tank is full is inferred from the gauge reading full (together with the background assumption applied to the particular occasion).
  \item[{\hyperref[W:NF:ii]{ii:}}] That the gauge is functioning correctly on this occasion is inferred from the tank being full (and the gauge readings full) by an argument whose justificatory power depends on it being probable the gauge functioning correctly on this occasion.
  \item[{\hyperref[W:NF:iii]{iii:}}] That the gauge reads full does not make it probable the gauge functioning correctly on this occasion without the help of it being the case that the tank is full.
  \end{itemize}

  In short, the reasoning from~\ref{W:GG:i} to~\ref{W:GG:iv} captures the (intuitive) idea that \nagent{9}'s reasoning is flawed because and agent doesn't get to use reasoning that proceeds from an assumption to infer that the assumption holds.

  {
    In terms of \citeauthor{Weisberg:2010to}'s presentation, the agent makes an ampliative inference from \(P_{1}-P_{m}\) to \(L_{1}-L_{n}\), requires certain things to be the case, and, results of amplification inference don't provide one with an argument for source of distortion.
    }

  \nagent{9} requires the gauge functioning correctly on this occasion to infer that the gas tank is full, but observing that the gauge functioning correctly follows given the assumption that the gauge functioning correctly doesn't make it any more likely that the gauge really is functioning correctly.
\end{note}

\begin{note}[Different from \citeauthor{Wright:2011wn}]
    {
    Here, very similar to \citeauthor{Wright:2011wn}'s \wrt{}.
    Difference is with respect to \ref{WT:iii}.
    Not-\(C\) does not necessarily entail something incompatible.

    For, \nagent{9} needs sufficiently reliable.
    And, it doesn't follow from the gauge is not functioning on this occasion that it is not sufficiently likely, nor that it is not possible to move from sufficiently likely to working.

    Still, \wnf{} does seem to fall within the general scope of \widt{}.
  }
\end{note}

\begin{note}[In relation to \nI{}]
  We turn now to the relationship between \citeauthor{Weisberg:2010to}'s \wnf{} and \nI{}.

  Recall, \eiS{}:
  Claimed support indicates the value of a proposition regardless of whether the claimed support is \mom{}.

  The argument for \nI{} rests on \eiS{}, and \citeauthor{Weisberg:2010to}'s \wnf{} may, likewise, be seen to rest on \eiS{}.

  For, it seems that any claimed support for the conclusion of an argument that satisfies the clauses of \wnf{} would violate \eiS{}.

  Consider \wnf{} once again.
  An argument for a relevant instance of \(C\) is defeated because \(C\) being probable to some degree is required in order to obtain additional lemmas used to construct an argument for \(C\).
  Recast, then, the agent may not construct an argument for \(C\) if the agent requires \(C\) to be probable to some degree.
  Or, equivalently, the agent may not construct an argument for \(C\) if it is a requirement for the success of the argument that the agent is not misled about degree to which \(C\) is probable.
  I.e.\ the argument would indicate the value, or probability, of \(C\) regardless of whether the claimed support is misled because the argument is only successful if \(C\) is probable to the relevant degree.

  So, is it the case that \citeauthor{Weisberg:2010to}'s \wnf{} and \nI{} are equivalent accounts of how \eiS{} constrains claiming support, if some (perhaps) superficial details about `probability' or `being in a position to claim support' are either removed or revised?
\end{note}

\begin{note}[Technicality]
  Well, an important difference is a difference between scope of application.
  \wnf{} only applies to inductive reasoning (Cf.~\Citeyear[533]{Weisberg:2010to}), while \nI{} makes no such restriction.

  Still, I don't think too much should hang on this difference.
  We have motivated \nI{} primarily with respect to inductive reasoning, and reasoning with \gsi{-} is also, plausibly, an instance of inductive reasoning.
  So, even if there is room for a technicality, it doesn't matter for the cases of interest.
\end{note}

\begin{note}[Key difference]
  Barring technicalities, still, a fundamental difference is present.

  As we have seen, \wnf{} holds when an agent makes an inductive inference from some premises to some lemmas which require the relevant conclusion of the argument to be probable to a certain degree.

  We \hyperref[wnf:expectation]{noted} that \wnf{} does not require that the agent has claimed support that the conclusion of the argument is probable to the required degree.
  However, clause~\ref{W:NF:ii} of \wnf{} explicitly states the argument of interest depends on relevant conclusion being probable to the required degree.
  So, the reasoning outlined by \wnf{} involves the agent being committed to the relevant conclusion being probable to the required degree as a part of the instance of reasoning to which \wnf{} applies.
  And, hence, tension with \eiS{}.

  \nI{} is fundamentally different.

  In the case of \nI{},~\ref{nI:inclusion} combines with~\ref{nI:going-by-value} to ensure that when an agent moves from reasoning from claimed support for \(\phi\) having value \(v\) to reasoning from \(\phi\) having value \(v\) the agent assumes that \(\psi\) has value \(v'\).
  In turn, this assumption leads to tension with \eiS{} as any reasoning that proceeds from \(\phi\) having value \(v\) depends on \(\psi\) having value \(v'\).
  Hence, even if the agent reasons from the implication that \(\psi\) has value \(v'\) when \(\phi\) has value \(v\), such reasoning is constrained to a context in which it must already be assumed that \(\psi\) has value \(v'\).
  And, so, it is not possible for the agent to claim support that \(\psi\) has value \(v'\) that indicates that \(\psi\) has value \(v'\) regardless of whether or not the claimed support is \mom{} because the reasoning from \(\phi\) having value \(v\) to \(\psi\) having value \(v'\) fails if \(\psi\) does not have value \(v'\).

  This is, admittedly, complex.
  To simplify, \wnf{} holds when an agent assumes the relevant conclusion (is probable to a certain degree) as part of the reasoning to the relevant conclusion.
  And, in contrast, \nI{} holds when an agent is required to assume that the relevant conclusion (has some value) in order to reason to the relevant conclusion.

  Simplified further, for~\wnf{} the issue is with respect to the reasoning that would be performed by the agent.
  And, by contrast, for~\nI{} the issue is with respect to what the agent must in order to perform the relevant reasoning.

  Admittedly this is a somewhat delicate.
  However, this distinction is important as it highlights how \nI{} is concerned with the specific details of the way in which the agent reasons, rather than any pattern of reasoning itself.

  As we saw when contrasting Illustrations~\ref{ill:CE:main},~\ref{ill:CE:colleague}, and~\ref{ill:CE:testimony}, the reasoning identified by~\ref{nI:going-by-value} allows an agent to claim support in certain cases (\ref{ill:CE:testimony}), but not in others (\ref{ill:CE:main} and~\ref{ill:CE:colleague}).

  So, there is a significant difference between the cores of \wnf{} and \nI{}.
  Issues with reasoning, and issues with assumptions prior to reasoning, respectively.
\end{note}

\begin{note}[Generalising to \widt{}]
  Noted above that \wnf{} seems to fall in the scope of \widt{}.
  If so, same kind of problem for \widt{}.

  However, relies on additional background about claiming support for presuppositions.
    And, while it is true that \citeauthor{Wright:2011wn} holds such views, it was instructive to observe that difference regardless.

    No such alternative with relation to \citeauthor{Weisberg:2010to}, as it's built into \wnf{}.
\end{note}

\begin{note}[Hm]
  \color{return}
  This is brief, but it's not clear more needs to be said.
  These are distinct phenomenon, even if they turn out to be extensionally equivalent.
  Conjecture that these wouldn't be, as similar issue with respect to difference between \citeauthor{Wright:2011wn}'s \wrt{} and \nI{} arising from \ref{nI:inclusion}.
  Working through illustrations is significant due to difference in formulation.
  There, difference in extension for difference in intension.
  Here, directly observed difference in intension.

  Broader consequences are of some interest (at least to me).
  However, aren't of interest to core line of argument.
  So, pursuing this will be left for some other time.
\end{note}

% \paragraph{Pryor}

% \begin{note}[Objection wrt.\ \citeauthor{Pryor:2000tl}]
%   I've probably already mentioned this somewhere else, but there's a way of reading \citeauthor{Pryor:2000tl} that conflicts.
%   For, Dogmatism could be read as allowing for \RBV{} in cases where it applies.

%   However, this requires some care.

%   Initial point is that it's not clear whether there's a dogmatist position with respect to claiming support.

%   Fruther, there are two possible ways to approach the issue.
%   First, as above.

%   Second, there's some operator which is dogmatic, and consequences stay within the scope of this operator.
%   So, for example with the zebra, the agent sees that the animal is a zebra, and hence sees that it's not a cleverly disguised mule.
%   On this reading, the agent does not \RBV{}.
% \end{note}

\newpage


\subsubsection{Generalising \nI{}}
\label{sec:generalising-ni}

\begin{note}
  Core question about whether there's a generalisation of \nI{}.

  In particular, one might think that there's a requirement for the agent to witness the relevant reasoning in certain cases.
  I mean, that's the core of \nI{}.
  In some cases, the agent doesn't have the option of skipping this by appealing to claimed support for something.

  However, the difficulty is in finding an expansion which doesn't also prevent the agent from claiming support when they do witness.
  In all cases, it's clear that one may get things wrong.

  The way that \ur{} avoids this is by avoiding strong claims to the specific ability.
  Indeed, principle is the same as witnessing.
  So, there's no plausible way to expand \nI{} to cover the proposals without also denying the relevant instance of witnessing.

  Rather, objections here comes from supporting \ESU{}.
  That this isn't a way to claim support.
\end{note}

\newpage

\subsubsection{\nI{} applied to \gsi{} and \nr{}}
\label{sec:ni-applies-ar}

\begin{note}[Applying to type of scenario]
  Our attention now turns to how \nI{} applies to the use of \aben{the} in scenarios of interest.

  The focus of our attention is whether an agent may claim support for having a specific ability given the claimed support for having a general ability, given \gsi{}.
\end{note}


\begin{note}
  \begin{itemize}
  \item First is to go through \nI{} with ability.
    The highlight here is that \RBV{} is problematic.
  \item Then, the question is whether \nr{} requires using ability is this way.
  \item Finally, why \nI{} doesn't apply to \ur{}.
  \end{itemize}

  So, the big issue here is the relationship between \RBV{} and \nI{}.

  The goal is to show specific case of \nI{} ends up as an instance of \RBV{}.

  
\end{note}

\begin{note}
  Basic idea is simple.
  \begin{enumerate}
  \item \aben{The}.
    Here, needs to be the case that the agent has the ability.
  \item \nr{}.
    So, form the existence.
  \item Nothing about existence along which allows the agent to get to proposition.
  \item Therefore, must use conditional.
  \end{enumerate}

  Fits the letter of \nI{} and also fits the motivation.

  Problem is being \mom{}.
  If \(\phi\) has different value, then don't have a way of getting to \(\phi\).
  For, relied on this key step of existence.
\end{note}


\begin{note}
  Now, the basic observation is that with \nr{} one moves from general to specific, and from ability to proposition.

  Here, only really interested in \aben{the}.
  However, as we've observed, goes from either general or specific.

  I mean, the basic observation is that the agent doesn't reason about general or specific ability.
  So, reasoning follows from it being the case that agent has attribute, or that there is a witnessing event.

  Ohhhh, the point is that the agent is relying on these conditionals.
  First, to move from general to specific.
  Second, to move from ability to proposition.

  With respect to these conditionals, it's \nr{}, so there's no way to move between these things without using the value of one thing to constrain the value of the other.

  So, \RBV{} is an instance of \nr{}, generally.
  And, because of the construction of the scenarios, the case of \nr{} we're interested in is an instance of \RBV{}.
\end{note}

\newpage

\begin{note}[Checking conditions]
  Conditions \ref{nI:claimed-support} and~\ref{nI:received-info} are provided by the scenario.
  Condition~\ref{nI:inclusion} is obtained by reflection on ability.
  So, then, condition~\ref{nI:going-by-value} rules out a way of claiming support for specific ability.
\end{note}

\begin{note}
  If argument is successful, then agent will not be in a position to claim support for specific ability.
  This is the antecedent of the relevant use of \aben{the}.
  Pair \nI{} with following supplement.

  \begin{proposition}[\nIm{}]
    An agent must have claimed support for the antecedent of an entailment in order to claim support for the consequent of the entailment via the entailment.\nolinebreak
    \footnote{To clarify, entailment is only about value.
      Think of conditional.

      So, does not follow that there being an entailment is a required part of agent's reasoning.
      \nIm{} is talking about when the agent appeals to an entailment, rather than any understanding of entailment beyond it being the case.
    }
  \end{proposition}
  \nIm{} seems indisputable,\nolinebreak
  \footnote{
    An agent may have some other way of claiming support for the consequent of the entailment.
    However, if the agent is not in a position to claim support for the antecedent, then the agent is not in a position to claim support because there is an entailment from the antecedent to the consequent.\nolinebreak

    For example, that the coin landed heads is entailed by Sam knowing that the coin landed heads.
    Here, entailment from \(K\phi\) to \(\phi\).

    Second, this light being on entails that the printer is out of paper.
    If agent appeals to entailment, again, need the light to be on.
    However, could look in the paper drawer, or modify the wiring so that an alarm sounds.

    However, Taylor is not in a position to claim support for the coin landed heads because Sam knows if Taylor has no idea whether Sam knows --- though Taylor may claim support by looking at the coin.
  }
  and so not in a position to claim support for result of witnessing ability via \AR{}.
\end{note}

\begin{note}
  Note on two ways of reasoning.
  Getting support for specific ability by \RBV{} and also getting result by \RBV{}.
  \nI{} only explicitly rules out first.
  However, with \nIm{}, second is implicitly ruled out, as if agent claims for value, then need to be able to claim support for premise.
  And, definition of \AR{} is such that going for premise is attribute.
  This is not necessarily required --- e.g.\ \WR{}.
\end{note}

\hozline{}

\begin{note}[Notes for \AR{}]
    {
    \color{green}
    Well, with the first, it's getting to \(\phi\).
    Doesn't seem the agent is in a position to use factive inference.
    Because, the agent is going from not possible to have ability and for \(\phi\) to be false.

    With the second, different.
    Because, the agent is going from application of ability providing support for \(\phi\).
  }

  Use of \AR{} gets quick argument for \RBV{}.
\end{note}

\hozline{}

\begin{note}[Application of \nI{} to \AR{}]
  \AR{}, working with attribute.
  So long as you have general ability, you have specific ability.
\end{note}


\begin{note}[Application of \nI{} to \AR{} argument]
  \gsi{} information.
  For \AR{}, agent is required to claim support that they have the specific ability.
  That is, claim support for consequent because the agent claims support for specific ability.
  This is distinguishing feature of \AR{} --- the agent is only appealing to having attribute.
  General characterisation of \AR{}, all ability from appeal to attribute.
  Contrast to \WR{}, where some use of ability.

  So, this means attribute for general and specific in cases of interest.
  For, \AR{} is attribute for all instances of ability.




    If agent appeals to general and information, then agent is appealing to having general attribute, and not only support for general attribute.
\end{note}

\hozline{}

\begin{note}[Distinguishing features]
  Reasoning has distinguishing features that pair well.
  \begin{itemize}
  \item Extends support.
  \end{itemize}
\end{note}


\begin{note}[Important points]
  Two important points:

  The role of~\nI{} is to highlight that the agent is not in a position to obtain support for (specific) ability in a certain way.
  That is,~\nI{} does not state that the agent may not obtain support for (specific) ability some other way.

  Second, so long as agent holds that they have general ability, then committed to truth.

  May be tempted to say that the agent is not committed, but this seems implausible.
  Cases of transmission failure, it seems agent does remain committed, at least.

  May take issue with information provided, especially if ideal.
  If informer has information, then they should say.
  In turn, not problem with~\nI{} as the agent would have support (via testimony) for specific ability.
  However, informer may only have the conditional.

  Ordinary agents.
  Maxims are broken.
  And, interest effects.
  Up to the agent.

  Seems puzzling, but not paradoxical.
\end{note}

\begin{note}[Why this is important]
  The key idea, and the foundation of the objection, is that the agent is going indirectly.
  The agent fails to show how the general ability extends to specific ability.
  For, the only things available to the agent is the constraint.

  This is useful independently.
  For, even if not convinced by~\nI{}, clear that given \gsi{} and~\ESU{}, the agent goes directly.
  And, something a little puzzling about this.
  Or, so I think.
\end{note}

\begin{note}[Dogmatism]
  Continuing relation to issues with knowledge.
  \autoref{prem:ni} is quite close to dogmatism paradox.
  If one knows that \(\phi\), then any evidence for \(\lnot \phi\) is misleading.

  Distinct again, however.
  For, don't have knowledge in the antecedent.
  Get the dogmatism paradox from the factivity of knowledge.
  No requirement that support for (general) ability is factive.

  Hence, role of the informer is important again, because agent is not in a position to come to the conditional by themselves prior to reasoning.
\end{note}

\hozline{}

\begin{note}[Finding tension, still]
  We have outlined a type of scenario built primarily on an agent receiving information that the agent has some specific ability so long as the agent has some general ability.
  The agent has support for having the general ability, but there are two ways in which the agent's support for having the general ability may be used to establish support for {\color{red} the result of having the specific ability} --- \AR{} and \WR{}.

  The previous section argued that~\ESU{} constrains how an agent may use the received information.
  If an agent is required to traces support from premises to conclusion through reasoning, then an agent may not appeal to the support for the premises and steps of reasoning that the agent would use to witness the specific ability.
  
  The (initial) plausibility of~\ESU{}, then, suggests that the agent may only establish support for having the {\color{red} result of the specific ability} from the support they have for the general ability by \AR{}:
  The support the agent has for the general ability is support that it is true that the agent has the general ability.
  In turn, given the information received it is true that the agent has the specific ability, and it is only possible for the agent to have the specific ability if the result of witnessing the specific ability is true.

  The argument of this section is that the sketch of \AR{} given conflicts with a different, but equally plausible, premise.
  The premise concerns the way in which the agent obtains support for having the specific ability from the support for the general ability.
  We state conditional, the proceed to the premise.
  The initial statement of the premise is abstract and after providing a handful of clarifications we then link the premise to the type of scenario of interest.
\end{note}

\subsubsection{\nI{} and \ur{}}
\label{sec:ur}

\begin{note}
  Here, argue that \ur{} is okay with \nI{}.
\end{note}

\begin{note}
  Key point is that agent claims support for property or event.
  The agent doesn't move to value.

  So, \gsi{} it's the parts of the general ability.
  And \aben{the} it's the premises and so on.

  Key point is that given background information, these allow the agent to claim support, even if it turns out the agent is \mom{}.
  Information is that that stuff is sufficient to claim support.

  Easiest with \WR{}.
  As, this is just the same as an instance of reasoning.
  The only difference is that the agent isn't clear on what's going on.

  Everything the agent has claimed support for allows them to make this move.
  Even if turns out things aren't right, and \mom{}, the agent seems to have enough, and by \ur{} they don't require that they aren't \mom{}.

  This, to my mind, is the key idea with ability.
  It informs the agent of something they have the ability to do.
  And, that thing functions in just the same way as it would if the agent were to do the thing.

  Claim support by appeal to that reasoning.
  Only going to be truly successful if I have the ability, for sure.
  However, claim support for ability even if \mom{}.
\end{note}

\begin{note}
  Key observation is that \ur{} doesn't go by value.

  However, there is a problem.

  For, it may seems as though the agent \emph{does} go by value because they require the premises, etc.

  This is clearest with the idea that:
  \begin{itemize}
  \item If \(\phi\) isn't the case, then some premise or step isn't part of ability.
  \end{itemize}
  Question about whether this gets a violation of \eiS{}.

  But, point is that agent at present is okay with claiming support that the reference resolves.

  So, this really isn't that problematic.

  Obviously it could break down.

  The point is that the agent at present outlines claim to support even if \mom{}.
\end{note}

\subsubsection{Incompatibility of \nI{}, \gsi{}, and \nr{}}
\label{sec:ni-summary}

\begin{note}[Table]
    \begin{figure}[h]
    \centering
    \begin{tblr}{abovesep=8pt, belowsep=8pt, width=0.95\textwidth, colspec={Q[c,m]|Q[c,m]|Q[1.8,c,m]|Q[1.8,c,m]}}
      \multicolumn{2}{c}{} & \nr{} & \ur{} \\
      \hline
      \multicolumn{2}{c}{\WR{}} & Ruled out by \nI{}  &  \\
      \hline
      \multirow{2}{*}{\AR{}} & Basic  & Ruled out by \nI{}  & ???  \\
      \cline[dashed]{2-4}
      & Derived & Ruled out by \nI{}  &  \\
    \end{tblr}
    \caption{Distinction matrix}
  \end{figure}
\end{note}

\subsubsection{\nI{} isn't that strong}
\label{sec:ni-isnt-that}

\begin{note}
  Look, \nI{} rules out a way of claiming support quite broadly.
  However, this is because we're focusing on \aben{the}.
  This shouldn't be taken to suggest that there's general tension between \nI{} and \nr{}.
\end{note}

\subsection{Establishing tension/summary}
\label{sec:establishing-tension}

\begin{note}[Results of the distinctions]
  Recap.

  \begin{itemize}
  \item Kind of scenario involving ability.
  \item Distinction between \AR{} and \WR{}.
  \item Distinction between \nr{} and \ur{}.
  \item Distinction matrix.
  \item Point of this was to provide an exhaustive account of the ways in which the agent may claim support in the scenarios of interest.
  \end{itemize}

  Then, moved to figuring out whether the respective combinations of the distinction matrix are permissible.
  \begin{itemize}
  \item \ESU{} ruled out \ur{}, with the exception of \AR{} basic.
  \item \nI{} ruled out \nr{} no matter way in which ability was though about.
  \end{itemize}

  So, if this is correct, we've got three options.
  \begin{itemize}
  \item \AR{} basic, with \ur{}.
  \item Reject \nI{}.
  \item Reject \ESU{}.
  \end{itemize}
\end{note}

\begin{note}[Matrix]
  \begin{figure}[H]
    \centering
    \begin{tblr}{abovesep=8pt, belowsep=8pt, width=0.95\textwidth, colspec={Q[c,m]|Q[c,m]|Q[1.8,c,m]|Q[1.8,c,m]}}
      \multicolumn{2}{c}{} & \nr{} & \ur{} \\
      \hline
      \multicolumn{2}{c}{\WR{}} & Ruled out by \nI{}  & Ruled out by \ESU{} \\
      \hline
      \multirow{2}{*}{\AR{}} & Basic  & Ruled out by \nI{}  & ???  \\
      \cline[dashed]{2-4}
      & Derived & Ruled out by \nI{}  & Ruled out by \ESU{} \\
    \end{tblr}
    \caption{Distinction matrix}
  \end{figure}
\end{note}

\subsubsection{\AR{} Basic with \ur{}}
\label{sec:ar-basic-with}

\begin{note}
  Main issue here is that it doesn't seem as though this is a basic step of reasoning.

  For, ability breaks down into various components.

  Response here is that there do seem to be basic steps of reasoning with are similar in this respect.
  For example, it seems as though many cases of moving from cause to effect will do this.

  I think going back to dispositions might help here.
  
\end{note}

\subsubsection{Reject reasoning}
\label{sec:reject-reasoning}

\begin{note}
  If not basic, then reject reasoning.
\end{note}

\begin{note}
  Problem here is that this seems too strong.
\end{note}

\begin{note}[Main problem]
  The main problem is that it seems fine for the agent to claim support for specific ability.
  And, that \aben{the} applies.

  This gives rise to some tension.

  Possible resolution here is that agent expects things are as they would be if agent witnessed, but does not get to claim support.
  Issues only follow from claiming support.

  However, then the agent doesn't get to do anything with the proposition.

  Flipside is that claiming support is minimal.
  Agent does this to use the proposition in further reasoning, and only constraint is \eiS{}.
\end{note}

\begin{note}
  Addition:
  There's some parallel here with reflection.
\end{note}

\begin{note}[Reflection]
\begin{quote}
    Reflection states that agents should treat their future selves as experts or, roughly, that an agent’s current credence in any proposition A should equal his or her expected future credence in A.\linebreak
    \mbox{}\hfill\mbox{(\Citeyear[59]{Briggs:2009up})}
  \end{quote}
\end{note}

\begin{note}[Difference to reflection]
  Key difference is that in these cases, there's no guarantee that the agent will go through with ability.
  So, it's not necessarily a future self of the agent.
  Though, that's only on a quick surface reading of reflection.

  This is somewhat delicate.
  For, reflection has some strong background assumptions.
  Problem with ability is that agent might witness ability.
  With reflection, we don't consider restrictions on the reasoning the agent would do.

  Now, weakening reflection is difficult.

  One the one hand, can consider all evidence that the agent would reason through.
  If so, then it looks as though ability is going to fall within the scope.
  Problem here, however, because the argument for reflection is in terms of coherence.
  And, it's not clear how to apply conditionalisation to boundedness.
  Dutch books are about coherent credence functions.

  So, it is not clear that there's a way to derive instances of \aben{the} from principles which motivate reflection.
\end{note}

\begin{note}
  Taking a step back, in these kinds of cases it's something like evidence of evidence.
  This is in \textcite[2]{Tal:2017uw}, linking to another paper.

  And, this is kind of similar to what's going on with ability.

  This only works easily from \AR{} perspective.
  Still\dots

  Things get complex here.
  For, if this is the case, then it's not clear that the agent needs to worry about \aben{the}.
  So, the issues arising from the matrix don't really apply.

  Point here is that the agent could go straight for general ability.

  Problem is that agent still needs to get specific ability.

  Same issues with \ESU{} and \nI{} apply here.
  Still get something appealed to but not used.
\end{note}

\begin{note}
  Had scenarios.
  Here, just added that there are close things in the literature which seem fine.
\end{note}


\subsubsection{Reject \nI{}}
\label{sec:reject-ni}

\begin{note}
  Follows from \eiS{}, mostly.

  And, \eiS{} seems quite plausible.
\end{note}

\begin{note}
  Inclined to discount similarities to \citeauthor{Wright:2011wn} and \citeauthor{Weisberg:2010to}.
  Enough of a different.
  And, these kind of things are very difficult.
\end{note}

\begin{note}
  Rather, motivation and illustration.
\end{note}

\begin{note}
  Response here to focus on the difficult cases.
  But, as noted, because \nI{} is only about a way of claiming support, there are various ways in which one may deal with those cases.
\end{note}

\subsubsection{Reject \ESU{}}
\label{sec:reject-esu}

\begin{note}
  The literature.

  And, intuitive appeal.

  Maybe that's enough.
  Still, I want to see argumentation.
\end{note}

\subsection{Outlook}
\label{sec:outlook}

\begin{note}
  Reject \AR{} basic and \ESU{}.

  This leaves us with two options for understanding \aben{the}.
\end{note}

\subsection{\AR{}}
\label{sec:ar-2}

\begin{note}
  I don't have too much to say here.

  Outlined the idea of \AR{}.

  Considered this in terms of evidence of evidence.
  Ability then doing the work of making that evidence evidence for the agent.

  Role for ability here is securing that there is evidence.
  For, without ability, agent doesn't get to establish these background conditions.
\end{note}

\subsection{\WR{}}
\label{sec:wr-2}

\begin{note}
  Favoured
\end{note}



\section{Minor argument}
\label{sec:posit-argumn-overv}

\subsection{Cases}
\label{sec:cases}

\begin{note}
  Main role of minor argument is cases.
\end{note}

\begin{note}[Beyond belief]
  Application in particular to desire.
\end{note}


%%% Local Variables:
%%% mode: latex
%%% TeX-master: "master"
%%% End:
