\chapter{Overview}
\label{cha:overview}

\section{The issue}
\label{sec:issue}

\begin{note}[Main issue, positive resolution, quick argument for negative]
  Our interest is with the following issue:

  \begin{quote}
    May an agent conclude \(\phi\) has value \(v\) from certain premises without witnessing reasoning that concludes \(\phi\) has value \(v\) from those premises?
  \end{quote}

  The goal is to motivate a positive resolution:

  \begin{quote}
    \emph{There are cases in which} agent may conclude \(\phi\) has value \(v\) from certain premises without witnessing reasoning that concludes \(\phi\) has value \(v\) from those premises.
  \end{quote}

  This positive resolution is, by my estimation, not straightforward.
  At least, the positive resolution is not straightforward without placing some constraints on \emph{which} premises an agent may appeal to when reasoning, and we will motivate the positive resolution without such constraints.

  For, without constraints on which premises an agent may appeal to when reasoning, one may argue as follows:
  \begin{enumerate}
  \item Any instance of reasoning is some process with start and end points and intermediary steps.
  \item If an agent has concluded that \(\phi\) has value \(v\) by some reasoning, then the reasoning has start points and intermediary steps.
  \item Hence, the agent has concluded \(\phi\) has value \(v\) by witnessing some reasoning from some start points via some intermediary steps.
  \item In other words, the agent has concluded \(\phi\) has value \(v\) by witnessing some reasoning from some premises.
  \end{enumerate}

  In short, so long as an agent has concluded that \(\phi\) has value \(v\), the agent has always witnessed reasoning from some premises.

  \begin{enumerate}[resume]
  \item So, either the start points are the premises of interest mentioned in the issue, or the agent has concluded \(\phi\) has value \(v\) from a distinct set of premises.
  \end{enumerate}

  In other words, either the agent has witnessed reasoning from the premises of interest, or the premises of interest (and any reasoning from them) are not required to conclude \(\phi\) has value \(v\).
\end{note}

\begin{note}[More on the quick argument]
  The quick argument does not directly lead to a negative resolution to the issue.
  Still, the quick argument does suggest that any appeal to premises \emph{without} witnessing reasoning from those premises is redundant.

  Now, perhaps redundancy isn't so bad.
  I only need a single key to ensure I have the option of unlocking a door, but a second key is useful if the first is lost.

  Still, I take it to be the case that redundancy provides leverage for a wide range of arguments motivating a negative resolution in the case of reasoning.

  For, if appeal to some premises is redundant, then any argument that requires witnessing need only observe that a counterargument must find some role for something which is not needed.

  Reasoning is an event, and distinct way of concluding \(\phi\) has value \(v\) may be useful, it is unclear why the distinct way of concluding \(\phi\) has value \(v\) is of use when concluding \(\phi\) has value \(v\) from present premises.
  To push the analogy, a second key may have various uses, but the second key is irrelevant in the event of unlocking the door with the first key.
  That the second key is would unlock the door if the first was lost has no role in the event of unlocking the door with the first key.

  From a different perspective, if appeal to certain premises without witnessing reasoning from those premises is redundant, then it seems any positive role given to appeal to those premises may be redistributed to the premises of the reasoning the agent did witness.

  More concretely, even if I were to show that there was some benefit for concluding \(\phi\) has value \(v\) via unwitnessed reasoning with respect to some particular account of reasoning, it seems at least plausible that the account of reasoning may be reformulated to derive the same benefit from the premises of the reasoning the agent witnessed.

  More generally, it may seem (and I suspect it does seem) intuitive that the issue should be resolved negatively.
  Reasoning just is obtaining a conclusion by witnessing reasoning from premises.
  And, if the quick argument succeeds, then there surely is some way to preserve the intuition.

  So, the task is to show that the quick argument fails.
  Concluding \(\phi\) has value \(v\) from certain premises without witnessing reasoning that concludes \(\phi\) has value \(v\) from those premises has some role.
  Or, rather, that the quick argument is not without cost.
  Perhaps the issue really should be resolved in the negative, but this will require giving up one of two equally (I think) intuitive ideas.

  The result will be motivation for a positive resolution to the issue.
  However, the motivation will be somewhat narrow.
  To escape the tension, the positive resolution need only hold for a restricted pattern of reasoning.
  Still, with the existential motivated, I hope future work may expand the positive resolution to other patterns of reasoning.
  And, while such expansions may still need to argue that concluding \(\phi\) has value \(v\) from unwitnessed reasoning is makes sense with respect to the specific topic at hand, future expansions will at least have the option of observing that the broad idea of concluding \(\phi\) has value \(v\) from unwitnessed reasoning may not be dismissed without cost.
\end{note}

\begin{note}
  The remainder of this chapter will introduce the core ideas of this document, and provide a general sketch of how the mentioned tension will be developed.
\end{note}

\begin{note}
  Start with claiming support.
  Issue was stated with respect to reasoning in general.
  However, tension is developed by focusing on when the conclusion of reasoning satisfies a certain property.
  In such cases, we say that an agent has not only concluded \(\phi\) has value \(v\), but has moreover claimed support for \(\phi\) having value \(v\).

  We will then recast the issue in slightly more detail, with respect to claiming support.

  Finally, we will introduce some observations regarding ability.
  Specifically, our interest will be with general abilities to reason which have specific instances.
  For example, the general ability to do (basic) arithmetic, and a specific instance of the general ability, to add \(18\) and \(41\).
\end{note}

\begin{note}
  While this chapter serves as an introduction to these core ideas, they will each be worked through in more detail in the corresponding parts of this document.

  Specifically, \autoref{part:claimingSupportI} will concern claiming \support{}.
  \autoref{part:reasoning} will concern the issue in greater detail.
  And, \autoref{part:ability} will detail ability.
\end{note}

\section{Outline}
\label{sec:outline}

\begin{note}
  In this chapter we provide a high level overview of the main arguments made in this document.

  A significant part of the high level overview of arguments is an overview of the premises and assumptions that those arguments rest on.

  By given high level overview, clarify on how premises, assumptions, and conclusions relate.
  In the main body, afford to elaborate.
  And, allow choice of where to seek elaboration.
\end{note}

\begin{note}
  General statement.
  



  Ability claims of interest provide some answer to `when'.
  For, ability provides information about what the result is, and also that the agent has the opportunity to perform the reasoning.\nolinebreak
  \footnote{No claim to necessity}

  Suggested in introduction, answer to `why' is more complex.
\end{note}

\section{Main things}
\label{sec:main-things}

\begin{note}
  \begin{restatable}[\ESU{0} --- \ESU{}]{target}{targetESU}
    \label{denied-claim}
    An agent concluding \(\phi\) has value \(v\) is an instance of claiming support \emph{only if}:

    For any pool of proposition-value pairs \(\langle \chi_{1},v_{1} \rangle,\dots,\langle \chi_{k},v_{k} \rangle\) appealed to as premises, the agent has witnessed reasoning which concludes that \(\phi\) has value \(v\) from \(\chi_{i}\) having values \(v_{i}\).\nolinebreak
    \footnote{More generally, the agent has witnessed reasoning whose conclusion \emph{indicates} \(\phi\) has value \(v\).}
  \end{restatable}


  Some instance of reasoning from premises which indicates conclusion.
  Reasoning is compatible with the agent claiming support for the conclusion only if some agent has witnessed the reasoning.

  Necessary condition on claiming support.
  Key point is that, whether or not there are genuine cases of claiming support, \autoref{denied-claim} provides a clear condition for rejecting the possibility that some instance of reasoning is an instance of claiming support.
\end{note}

\begin{note}
  \begin{restatable}[\EAS{0} --- \EAS{}]{goal}{goalEAS}
    \label{prop:EAS}
    There are instances of reasoning in which an agent concludes that \(\phi\) has value \(v\) by appeal to some pool of proposition-value pairs \(\langle \chi_{1},v_{1} \rangle,\dots,\langle \chi_{k},v_{k} \rangle\) as premises without witnessing reasoning from \(\langle \chi_{1},v_{1} \rangle,\dots,\langle \chi_{k},v_{k} \rangle\) to \(\langle \phi,v \rangle\).

    And, on some occasions, the reasoning is an instance of claiming support.
  \end{restatable}
\end{note}

\begin{note}
  The distinction here is whether some instance of reasoning needs to be witnessed in order for the reasoning to be compatible with the agent claiming support.

  \EAS{} contains two existentials.
  Some instances of reasoning, and some of these are instances of claiming support.
  Of course, interest is with the intersection.
  Instances of reasoning which are instances of claiming support.

  Key is ability.
  Specifically, ability to reason.
  This provides information to the agent.
  Given premises, then conclusion.
  Ability, in these cases, functions as an `interpolant'.

  Expand after outlining argument.
\end{note}

\begin{note}
  While \EAS{} is our `goal', the argument will be indirect.
  Motivate \EAS{} by observing that tension follows from combining \USE{} with two additional ideas.
\end{note}

\begin{note}
  The first is a constraint on particular instances of reasoning.

  `Claiming \support{}'.
\end{note}

\begin{note}
  The second is that it is possible to claim \support{} for having some general ability.
\end{note}

\begin{note}
  Tension, roughly.
\end{note}

\begin{note}
  Note, the tension is not about whether \(\phi\) has value \(v\).
  Instead, the tension is about whether the agent would have a certain property if they were to conclude that \(\phi\) has value \(v\).
  Property of having claimed \support{}.
  Expanded, property of holding that any independent check is satisfied.
  Any other reasoning about whether \(\phi\) has value \(v\) would conclude that \(\phi\) has value \(v\).
\end{note}

\begin{note}
  Returning to \EAS{}.
  Specific instances of the general ability.
  In this sense, the instances of \EAS{} we argue for are narrow.
  Need strong sign that the agent has the general ability.

  Further.
  It does not state that an agent having claimed support that they have the ability to reason to some conclusion is \emph{always permissible} to claim support for the conclusion by appealing to some premises that do not form part of the agent's reasoning.
  Instead, it states that \emph{may be permissible} for the agent claim support in a certain way.

  In various respects, these aren't particularly interesting cases.
  However, the goal is to argue that such cases exist.
  Whether these are constrained to the type of cases we consider for the argument is a further question.

  There may be more interesting cases, but given that \ESU{} is incompatible with all such cases, I see no compelling reason to explore such cases without \emph{first} motivating a rejection of \ESU{}.
\end{note}



\begin{note}
  Important that \ESU{} is fairly weak.

  Has witnessed reasoning.

  Indeed, \ESU{} may be weakened further, \emph{some} agent.\nolinebreak
  \footnote{
    Simple example, know that some logic is decidable.
    Take a relatively simple fragment.
    Then, either proof or a countermodel.
    Same idea holds for tension, but there's no guarantee that anyone has concluded\dots
  }
  However, to keep things straightforward, focus on a single agent.
  Indeed, the tension arises from focus on the agent's own reasoning.
\end{note}

\paragraph{Matrix}

\begin{note}
  \begin{figure}[H]
    \centering
    \saMtxInterpreted{}
    \caption{Distinction matrix with \aben{the}}
    \label{fig:saMtxInterpreted:outline}
  \end{figure}
\end{note}

\begin{note}[Matrix, ruled out]
  \begin{figure}[H]
    \centering
    \saMtxRuledOut{}
    \caption{Distinction matrix}
    \label{fig:saMtxRuledOut:outline}
  \end{figure}
\end{note}

\begin{note}
  Recap.

  Claiming support.
  Constraint.

  Ability.
  In order to be compatible, satisfy constraint.
  Either of three options.
  Basic, ignore this.
  Property. Incompatible with constraint.
  Witness. Compatible.

  Here, display the matrix.
  I think this is the easiest way to visualise what is going on.
\end{note}

\paragraph{Outline}

\begin{note}
  How things are divided.
\end{note}

%%% Local Variables:
%%% mode: latex
%%% TeX-master: "master"
%%% End: