\chapter{Overview}
\label{cha:overview}

\section{Outline}
\label{sec:outline}

\begin{note}
  In this chapter we provide a high level overview of the main arguments made in this thesis.
  A significant part of the high level overview of arguments is an overview of the premises and assumptions that those arguments rest on.

  By given high level overview, clarify on how premises, assumptions, and conclusions relate.
  In the main body of the thesis, afford to elaborate.
  And, allow choice of where to seek elaboration.
\end{note}

\begin{note}
  Following introduction, interest is with ability.
  In particular, observation that \gsi{} information, and confidence in general ability seems to allow agent to claim support for result.

  Question on what basis the agent claims support.

  Slightly more general statement.
  \begin{quote}
    When and why an agent may claim support for the result of reasoning that the agent has not witnessed.
  \end{quote}
  Ability claims of interest provide some answer to `when'.
  For, ability provides information about what the result is, and also that the agent has the opportunity to perform the reasoning.\nolinebreak
  \footnote{No claim to necessity}

  Suggested in introduction, answer to `why' is more complex.
  % \begin{quote}
  %   Our interest is in when an agent may claim support for some conclusion of some instance of reasoning on the basis of the support the agent may claim for the premises of the instances of reasoning.
  % \end{quote}
\end{note}

\section{Main things}
\label{sec:main-things}

\begin{note}
  \begin{restatable}[\ESU{-} --- \ESU{}]{target}{targetESU}\label{denied-claim}
    An agent may claim support for some conclusion of reasoning by claiming that the conclusion of reasoning is supported by premises and steps of reasoning \emph{only if} the agent has witnessed the reasoning (e.g.\ traced the claimed support for those premises and steps used to claim support for the conclusion).
  \end{restatable}
\end{note}

\begin{note}
  \begin{restatable}[\EAS{-} --- \EAS{}]{goal}{goalEAS}\label{prop:EAS}
    If an agent has claimed support that they have the ability to (adequately) reason to some conclusion, then it may be permissible for the agent claim support for the conclusion by appealing to some materia \(M\) to claim support for \(\phi\) without using \(M\) in the reasoning which culminates in claiming support for \(\phi\).
  \end{restatable}
\end{note}

\begin{note}
  Where claimed support is\dots
\end{note}

\begin{note}
  \autoref{sec:abil-access-supp} works through claiming support amounts to in some detail.

  Enough to provide context for these things.
  And, foundation for core part of the argument.

  Then, turn to target and goal in some detail.

  Then, moving to the argument proper.
  Ability.
  Tension.
  Includes consequence of claiming support.
\end{note}

%%% Local Variables:
%%% mode: latex
%%% TeX-master: "master"
%%% End: