%%% Local Variables:
%%% TeX-master: "master"
%%% End:

\chapter{Overview}
\label{cha:overview}

\section{Ability and access to support}
\label{sec:abil-access-supp}

\begin{note}[Introducing main topic]
  Our interest is in when it is permissible for an agent to obtain support for some conclusion of some instance of reasoning on the basis of the support the agent has for the premises of the instances of reasoning.

  We take the following claim as basic:
  \begin{enumerate}[label=\bP{}, ref=\bP{}]
  \item\label{prem:bP} If agent accesses support they have for premises and traces implication through valid and subjectively sound reasoning, then agent obtains support for conclusion on basis of support for premises.
  \end{enumerate}
  For example, suppose an agent has support that one of the three boxes in front of them is red.
  And, the agent has support that the box in front of them has the dimensions of 19 cm x 28 cm x 7cm.
  Some reasoning, the volume of the box is roughly 3,724 cm\(^{3}\).
  Whether some (or all) of the required arithmetic is to be included as a premise may be set aside.
  The support the agent has for holding that the volume of the box is roughly 3,724 cm\(^{3}\) is obtained (at least in part) on the support the agent has for the box having the dimensions noted.
\end{note}

\begin{note}[Support]
  Intuition is, roughly, that agent has support for a proposition if agent has the option of using the proposition in some further reasoning.

  This is not to say that a proposition without support has no use in reasoning.
\end{note}

\begin{note}[Valid and subjectively sound]
  Valid and subjectively sound.
  Valid, in the sense that the agent may obtain support for the conclusion.
  In a deductive case, if the premises are true, then the conclusion is true.
  Subjectively sound in the sense that for each premise or step of reasoning used, the agent has support for the premise or step and does not have superior support for a contrary premise or step.
\end{note}

\begin{note}[Focus]
Our interest is with the converse of~\ref{prem:bP}.

\begin{enumerate}[label=\mp{}, ref=\mp{}]
\item\label{denied-claim} An agent obtains support for conclusion on basis of support the agent has for premises only if the agent accesses support they have for premises and traces implication through valid and subjectively sound reasoning.
\end{enumerate}
If the agent did not measure the box, nor perform the arithmetic, the agent would not obtain support.
A luck guess that the box is roughly 3,724 cm\(^{3}\) would not allow the agent to hold that the volume of the box is roughly 3,724 cm\(^{3}\) on the basis of the dimensions of the box.
\end{note}

\begin{note}[Alternative]
  Our interest with \ref{denied-claim} is that it is a universal claim --- \ref{denied-claim} applies to all instances in which an agent obtains support for conclusion on basis of support for premises.

  Our goal is to motivate an exception to \ref{denied-claim}.

  \begin{enumerate}[label=\rC{}, ref=\rC{}]
  \item\label{rC} If an agent has information that they have the ability to (soundly) reason to some conclusion, then the agent may obtain support for the conclusion on basis of support for premises that the agent would access by witnessing their ability to reason to the conclusion.
  \end{enumerate}

  \ref{rC} carves an exception to~\ref{denied-claim}.
  For, if~\ref{rC} is true, then there are cases in which an agent is not required to reason from premises to some conclusion in order to obtain support for the conclusion on the basis the support the agent has for the premises.
\end{note}

\begin{note}[Structure of argument]
  Two lines of argument for endorsing~\ref{rC}, and hence denying~\ref{denied-claim}.
  \begin{enumerate}[label=(L\arabic*), ref=(L\arabic*)]
  \item\label{arg:line:1} Motivate~\ref{rC} as resolution to tension resulting from~\ref{denied-claim}.\newline
    Specifically:
    \begin{enumerate}[label=(L1\alph*)]
    \item\label{arg:line:1:a} Provide recipe for generating scenarios where~\ref{denied-claim} is in tension with particular scenarios involving information that an agent has the ability to perform some reasoning and a further claim regarding support.
    \item\label{arg:line:1:b} Motivate~\ref{rC} as a resolution to the tension.
    \end{enumerate}
  \item\label{arg:line:2} Argue that granting~\ref{rC} as an exception to~\ref{denied-claim} allows for an intuitive understanding of cases in which agent has the option of appealing to ability, even if there are alternative ways of interpreting the scenario in line with~\ref{denied-claim}.
  \end{enumerate}
  These two lines of argument work together.
  The tension of~\ref{arg:line:1} generates interest in witnessing that may be flatly rejected by prior endorsement of~\ref{denied-claim}.
  The intuitive understanding of scenarios involving ability of~\ref{arg:line:2} suggests there's more to witnessing than resolving the tension in narrow cases.
\end{note}

\begin{note}[Details of \ref{arg:line:1}]
  The initial focus is on the first line of argument,~\ref{arg:line:1}.
  The tension developed in part~\ref{arg:line:1:a} is delicate, but hopefully informative.
  We will establish a number of corollaries regarding ability and the interaction between~\ref{denied-claim} and ability.
\end{note}

\begin{note}[Before turning to the argument\dots]
  Before turning to the argument, we conclude this introduction with a handful of notes regarding~\ref{denied-claim} and~\ref{rC}.
\end{note}

\begin{note}[Interest in ability]
  Idealised agents have no need to appeal to ability.
  However, for limited agents, ability is abundant, while the resources required to witness abilities are scarce.
  That the exception to~\ref{denied-claim} is narrow does not entail that there are few occurrences of the exception.

  Information about ability may be abundant while the resources for witnessing abilities are either scarce or temporarily unavailable.
  So, agent is able to conserve or defer use of resources.

  Broadening scope.
  Arguments involving~\ref{denied-claim}.
  Distinction between ideal and non-ideal.
  Potential alternative conclusions to arguments that appeal to~\ref{denied-claim} as a premise.
  Revise premises for arguments in which~\ref{denied-claim} is a conclusion.
\end{note}

\begin{note}[Scope of \mp{}]
  \mp{} does not say anything in particular about what the agent has support for, only what must be the case in order for an agent to appeal to support for some conclusion on the basis of support for premises.

  Talking in terms of (support for) premises and conclusions restricts attention to reasoning.
  There may be broader use of `premise' and `conclusion' where an agent is not required to reason from premise to conclusion in order for the premise to support the conclusion.
  For example, if visual perception is immediate.
  Perhaps it may be said that an agent's visual experience is a premise to the conclusion that a dog is sleeping.
  Still, for present purposes, `conclusion' refers to the output of some process of reasoning performed by an agent which is either actual or potential, and `premises' to the input of that process.

  Note, also, that in both cases the relation between premises and conclusion is important.
  If agent does not reason, then neither~\ref{prem:bP} nor~\ref{denied-claim} apply.
  If there are multiple ways to obtain a conclusion, then~\ref{denied-claim} does not require the agent to reason from a particular set of premises.

  Likewise,~\ref{denied-claim} does not require that an agent is required to obtain support for a proposition by valid and subjectively sound reasoning from some premises.

  Rather,~\ref{denied-claim} requires that an agent reason from premises to conclusion in order to establishes support between premises and conclusion
  By contrast,~\ref{prem:bP} holds that reasoning is sufficient to establish such a relation.
\end{note}

\begin{note}[\mp{} is intuitive]
  \ref{denied-claim} is intuitive, and is quite common, though not without exceptions.
(For example, there's views on testimony in which the testifier provides agent access to support the testifier has.
One may understand this as conflicting with~\ref{denied-claim}, or that the fact that these are accessible is the relevant piece of support.)
\end{note}

\begin{note}[Alternative]
  \ref{rC} restricts~\ref{denied-claim}.
  This is not to say the agent obtains support equivalent to that which would be obtained were the agent to do, or have done, the reasoning.
  Nor, that the agent is aware of the relevant premises.

  Intuitively, \rC{} states that the agent may appeal to the reasoning they are able to perform in support for the conclusion of that reasoning, and as that reasoning moves from premises to conclusion, it is on the basis of the support for those premises that the agent would identify by reasoning that the agent obtains (some) support for the conclusion.

  Hence, \rC{} is in line with the spirit of~\ref{prem:bP}.
  For the exception to~\ref{denied-claim} is granted by the agent appealing to a witnessing event in which the antecedent (and consequent) of~\ref{prem:bP} are satisfied.
\end{note}

\begin{note}[Doxastic (as opposed to propositional) support]
  This is `doxastic' as opposed to `propositional'.
  Agent need not reason in order to have propositional support.
  For example, \(A < B\) and \(B < C\), then propositional support for \(A < C\) even if I don't bother to reason.
  No further assumptions about the relation between propositional and doxastic support.
\end{note}

\begin{note}[Ability ensures propositional?]
  Plausible that if the agent has the ability, then the agent already has propositional support for the relevant proposition.
\end{note}

\section{Broad argument overview}
\label{sec:broad-argum-overv}

\begin{note}[Overview]
  Tension resulting from the unrestricted scope of~\ref{denied-claim}.
  We begin by introducing a particular type of scenario involving ability, and observe how~\ref{denied-claim} requires a unique interpretation of the scenario.
  We then introduce an additional principle regarding support, which conflicts with the interpretation of the type of scenario introduction required by~\ref{denied-claim}.
\end{note}

\section{Type of scenario}
\label{sec:type-scenario}

\begin{note}[Tension, information]
  The tension arises when an agent is provided with a piece of limited information that:
  \begin{enumerate}[label=(I\arabic*), ref=(I\arabic*)]
  \item So long as the agent has a general ability, then the agent has a specific ability.
  \end{enumerate}
  The information is limited because it does not directly provide the agent with the information that the agent has the specific ability, nor that the result of witnessing the specific ability is the case.

  For example,
  \begin{enumerate}[label=(I\arabic*), ref=(I\arabic*), resume]
  \item\label{qe:cond} So long as you have the ability to reason with the rules of chess, you have the ability to demonstrating that there is a sequences of moves that will ensure a win for one of the players (as an instance of the general ability).
  \end{enumerate}
  The conditional structure distinguishes this information from:
  \begin{enumerate}[label=(I\arabic*), ref=(I\arabic*), resume]
  \item\label{qe:cons} You have the ability to demonstrate that there is a sequences of moves that will ensure a win for one of the players.
  \end{enumerate}
  For the agent is required to obtain~\ref{qe:cons} from~\ref{qe:cond}.
  In turn, the agent is not provided with information that:
  \begin{enumerate}[label=(I\arabic*), ref=(I\arabic*), resume]
  \item\label{qe:result} There is a sequences of moves that will ensure a win for one of the players.
  \end{enumerate}
  For, it need not be the case that~\ref{qe:result} is true if~\ref{qe:cond} is true by virtue of a false antecedent.
  Of course, the antecedent of~\ref{qe:cond} need not be false.
  Still, the limited information requires the agent to appeal to their general ability in order to obtain information about how the agent's general ability extends to a particular case.

  However, if the agent hold that they have the ability to demonstrate that there is a sequences of moves that will ensure a win for one of the players, then the agent may reason to~\ref{qe:result}.

  \begin{enumerate}
  \item[\textsf{A}]\label{A:s} On the one had, a strategy must exist in order for the agent to possess the ability to demonstrate that a strategy exists.
  \item[\textsf{W}]\label{W:s} On the other hand, if the agent has the ability to demonstrate that a strategy exists, then there it is possible for the agent to witness an event in which they demonstrate that a strategy exists.
  \end{enumerate}
  We term these \AR{} and \WR{}, respectively.

  We will return to this kind of limited information in greater detail below.
  For now, the basic idea is that the agent is on the hook, so to speak, for holding that they have the specific ability.
  Perhaps the informer does not want the agent to rely on the informer's information for the existence of the strategy.
  Or, perhaps the agent only wants to appeal to their own understanding of chess.

  We don't need a fleshed out scenario, but if it helps,~\ref{qe:cond} may be read as a slight challenge.
  The relevant interpretation of `if you're smart enough, you can solve this problem' seems clear.
  `If your ability to reason is of sufficient worth, then by extension of that ability, you have the ability to solve this problem.'
  Paraphrased, `if you're smart enough, you have the ability to solve this problem'.
  So challenged, and confident in one's smarts, one may expect to solve the problem.
  The slight difference with the limited information of interest is that the informer provides information about what the solution to the problem is if the agent is `smart enough'.
\end{note}

\begin{note}[Scenario premise]
  For ease of reference, we wrap scenarios involving the limited information as a premise.
  \begin{enumerate}[label=\eA{}, ref=\eA{}]
  \item\label{prem:ab} It is possible for an agent to use information that they have some specific ability so long as the agent has some general ability to obtain support for what follows from the specific ability.
    (Where the agent lacks doxastic support for what follows, and for \(A(\psi)\) without information).
  \end{enumerate}
\end{note}

\begin{note}[Possible restrictions]
  The important aspect of premise~\ref{prem:ab} is that there are cases in which the agent may appeal to ability to obtain support.
  This is quite weak.

  Understanding of support here is primarily for the agent.

  It allows that there may be cases in which the details of the cases outlined are satisfied, but where kind of support is unsuitable for certain purposes.

  In particular, some witness of ability may be demanded by a third-party.
  Perhaps due to lack of confidence in agent, or contextual features of the scenario.
  This is no different from memory.
  Memory of proving \(\phi\) provides support for \(\phi\).
  Still, one may still demand a demonstration of \(\phi\).
  Perhaps the third-party considers the agent's memory unreliable, or perhaps context has been set so that memory is insufficient to add a proposition to the common ground, etc.
\end{note}

\section{First conditional}
\label{sec:first-conditional}

\begin{note}[Attribute]
  There is not yet tension between~\ref{denied-claim} and~\ref{prem:ab}.
  For, we noted that the agent may appeal to their ability to reason in either of two ways:
  \begin{enumerate}[label=\(\cdot\)]
  \item \AR{}: \(\phi\) must be the case in order to have the attribute of being able to reason to \(\phi\).
  \item \WR{}: There is a (potential) witnessing event in which \(\phi\) is demonstrated, and therefore \(\phi\) is the case.
  \end{enumerate}
  WR{} is an instance of~\ref{rC}, as the agent obtains support for the conclusion of the reasoning is able to do on the basis of the reasoning that would be performed in a witnessing event.
  Hence, the supported obtained for the conclusion is obtained on the basis of the support the agent has for the premises that would be used.
  Again, this does not imply that the agent obtains support for the conclusion which is equivalent to the support the agent would obtain by witnessing their ability by performing the reasoning.

  However, \AR{} suggests an alternative way to obtain support for the conclusion of reasoning the agent is able to do.
  Specifically, if order for the agent to \emph{have} the attribute of being able to reason to the conclusion, the conclusion of the reasoning must be true.
  The relevant entailment is in part secured by the verb chosen, and in part by what the verb is applied to.
  Here, `demonstrate' is a factive verb, if an agent demonstrates that \(\phi\), then it is true that \(\phi\).
  And, the existence of a chess strategy does not depend on the agent demonstrating that the relevant strategy exists.

  To take another example, you only have the ability to identify a typo on this page if there is a typo on this page.
  So, if I were to provide you with testimony that you have the ability to identify a typo on this page, you may begin searching for the typo, or you may note that there must be a typo in order for me to be in a position to provide you with testimony that you have the ability.

  The reasoning is summarised with the following sketch.

  \begin{enumerate}[label=(\textsf{A}\arabic*), ref=(\textsf{A}\arabic*)]
  \item\label{WR:Sketch:1} I have the attribute of being able to \emph{V} that \(\phi\).
  \item\label{WR:Sketch:2} In order to have the attribute of being able to \emph{V} that \(\phi\), \(\phi\) must be the case independent of whether or not I witness the ability.
  \item\label{WR:Sketch:3} \(\phi\) is the case.
  \end{enumerate}

  To keep things simple, we will refer to the principle behind the pattern sketched as \AR{}.
  And agent may bundle~\ref{WR:Sketch:1} and~\ref{WR:Sketch:3} into a conditional, and avoid instantiating the reasoning pattern, but so long as the conditional is (implicitly) held on the basis of the intermediate premise~\ref{WR:Sketch:2}, we take use of such a conditional to be an instance of \AR{}.

  \AR{} is compatible with \ref{denied-claim}.
  For, the two premises~\ref{WR:Sketch:1} and~\ref{WR:Sketch:2} are accessible to the agent, and obtaining \ref{WR:Sketch:3} from~\ref{WR:Sketch:1} and~\ref{WR:Sketch:2} appears to be straightforwardly sound reasoning.
\end{note}

\begin{note}[Conditional \ref{P:ab-and-dc:W}]
  We wrap the above observations in a conditional.
  \begin{quote}
    \begin{enumerate}[label=(C\arabic*), ref=(C\arabic*)]
    \item\label{P:ab-and-dc:W} If
      \begin{enumerate}[label=(\alph*)]
      \item\label{P:ab-and-dc:W:ab} an agent may obtain support for the conclusion of reasoning they are able to do in cases described by~\ref{prem:ab}, and
      \item\label{P:ab-and-dc:W:uRa} \ref{denied-claim} is true,
      \end{enumerate}
      then
      \begin{enumerate}[label=(\alph*), resume]
      \item\label{P:ab-and-dc:W:AR} the support for \(\phi\) is obtained on the basis of the agent having the attribute of being able to demonstrate that \(\phi\).
      \end{enumerate}
    \end{enumerate}
  \end{quote}
  The reasoning described in the consequent of the conditional, \ref{P:ab-and-dc:W:AR}, is in line with \AR{} --- the support the agent obtains for the conclusion of the reasoning that they are able to do is obtained from the support the agent has for having the attribute of being able to reason to the conclusion.
\end{note}

\section{Second conditional}
\label{sec:second-conditional}

\begin{note}[Finding tension, still]
  We have outlined a particular type of case, an type of entailment, and two ways in which the entailment may be used.

  At present,~\ref{P:ab-and-dc:W} (merely) establishes that~\ref{denied-claim} constrains how an agent may use information about reasoning they are able to do.
  

  Still, given that~\ref{P:ab-and-dc:W} requires the agent to obtains support in a specific way (\AR{}), we turn to whether it is permissible for the agent to obtain support in the context of~\ref{prem:ab}.
\end{note}

\begin{note}[Inertia]
  Our final main premise is:\nolinebreak
  \footnote{
    A few caveats may be required here.
  }
  \begin{enumerate}[label=\nI{}, ref=\nI{}]
  \item\label{prem:ni} An agent does not have the option of obtaining support for some proposition \(\psi\) on the basis of information that the support the agent has for \(\phi\) is misleading or mistaken if \(\psi\) is not the case.
  \end{enumerate}

  The key idea is that in order for support to go through, the agent needs to be clear on how the support the agent has for \(\phi\) is also support for \(\psi\).

  {
    \color{red}
    A slightly better way of putting things for my purposes is that \(\phi\) does not inherit support from \(\psi\).
    So, \(\phi\) may be supported, but no support is added.
  }

  {
    \color{red}
    The agent may obtain support for \(\phi\) from some other premise.
  }

  For example, explanatory connexion.

  % A restricted instance of \ref{prem:ni} is entailed by cases of misleading evidence about evidence.
  % Here, the higher-order support is misleading else the lower order support is misleading.
  % The agent has a problem, for sure.
  % Still, \ref{prem:ni} does not entail (a restricted instance) of misleading higher order evidence.
  % For,~\ref{prem:ni} constraints only how an agent's support may be applied --- it is compatible with there being no higher order support.

  I consider \ref{prem:ni} to be intuitive.
  However, argument may be provided.
\end{note}

\begin{note}[Some motivation for \ref{prem:ni}, 1]
  Some motivation for~\ref{prem:ni} may be found by constructing instances of reasoning which violate~\ref{prem:ni}.
  A couple of cases may help.

  For example, suppose our agent is out shopping for a gift for Sam with a friend and has not yet considered the items before them.
  The friend remarks that `Someone would only buy \emph{that} (some particular item) as a gift for Sam if they didn't know Sam very well.'
  The agent has support that they do know Sam very well.
  However, it does not seem permissible for the agent to obtains support that they wouldn't buy that as a gift, one the basis of the support they have their familiarity with Sam and the statement made by their friend.
  For, it is possible that they agent would have settled on the particular item if they had given it some consideration --- the support the agent has for their familiarity with Sam \emph{may} be misleading.

  Or, suppose the agent has parked their car on the street outside of Sam's place and reasons that:
  If my car is stolen then my support that this is a safe neighbourhood would be misleading.
  Therefore, as my support that this is a safe neighbourhood is not misleading, my car has not been stolen.
  The agent's reasoning seems confused.
  Perhaps the agent has the option of extending the support the have that the neighbourhood is safe to obtain support that they car remains parked outside, but that support the agent has would be misleading if their car has been stolen does not seem appropriate.
\end{note}

\begin{note}[Some motivation for \ref{prem:ni}, 2]
  For further motivation, consider undercutting defeaters.

  We take the following sketch from \textcite{Worsnip:2018aa}:
  \begin{quote}
    Undercutting defeaters, which are easiest to think of in the context of the attitude of belief, are supposed to be considerations that undermine the justification of a belief in a proposition p not necessarily by providing (sufficient) positive evidence to think that p is false, but rather merely by suggesting (perhaps misleadingly) that one’s reasons for believing p are no good, in a way that neutralizes or mitigates their justificatory or evidential force.\nolinebreak
    \mbox{}\hfill\mbox{(\citeyear[29]{Worsnip:2018aa})}
  \end{quote}

\end{note}

\begin{note}[Inertia and attribution]
  \ref{prem:ni} raises a problem for applying \AR{} to the specific ability of scenarios described by~\ref{prem:ab}.

  For, the agent is required to obtain specific ability from general ability.
  \begin{enumerate}
  \item Agent has support for the general ability to reason with the rules of chess.
  \item However, the agent has not demonstrated the existence of the strategy, and so the agent relies on the information provided by the informer to hold that they have the specific ability to demonstrate the existence of the strategy.
  \item Still, the informer provided by the informer requires the agent to endorse having the general ability to reason with the rules of chess.
  \item In turn, that whatever support the agent has for having the general ability to reason with the rules of chess is not misleading.
  \item For, it may be the case that the agent does not have the ability to demonstrate the existence of the particular strategy.
  \item Therefore, the agent does not obtain support for the ability to demonstrate the existence of the strategy.
  \item Hence, it is not an option for the agent to obtain support for the existence of the strategy on the basis of support for the ability to demonstrate the strategy in line with \AR{}
  \end{enumerate}
\end{note}

\begin{note}[Established conditional 2]
  We wrap the above observations in a conditional.
  \begin{enumerate}[label=(C\arabic*), ref=(C\arabic*)]
    \setcounter{enumi}{1}
  \item\label{P:ab-and-dc:A} If
    \begin{enumerate}[label=(C2\alph*), ref=(C2\alph*)]
    \item an agent obtains support for some proposition \(\phi\) on the basis of the agent's ability to demonstrate that \(\phi\) is the case, and
    \item \ref{prem:ab} is true,
    \end{enumerate}
    then the support for \(\phi\) \emph{may not be} obtained (in line with \AR{}) on the basis of the agent having the attribute of being able to demonstrate that \(\phi\).
  \end{enumerate}
\end{note}

\begin{note}[Established conditional 2]
  Something about, if \ref{prem:ab} then agent does not obtain support for the attribute.
\end{note}

\section{Establishing tension}
\label{sec:establishing-tension}

\begin{note}[Summary]
  Given the two established conditionals~\ref{P:ab-and-dc:W} and~\ref{P:ab-and-dc:A}, the combination of the key premises of \ref{denied-claim},~\ref{prem:ab}, and~\ref{prem:ni} are in tension.

  For, combining~\ref{P:ab-and-dc:W} and~\ref{P:ab-and-dc:A} we have:
  \begin{enumerate}[label=(CC), ref=(CC)]
  \item If \ref{prem:ab} is the case an agent obtains support for some proposition \(\phi\) on the basis of the agent's ability to demonstrate that \(\phi\) is the case then:
    \begin{enumerate}[label=(C\arabic*\(\sim\)), ]
    \item If \ref{denied-claim} is true, then the support for \(\phi\) is obtained on the basis of the agent having the attribute of being able to demonstrate that \(\phi\) (in line with \AR{}).
    \item If \ref{prem:ni} is true, then the support for \(\phi\) \emph{may not be} obtained (in line with \AR{}) on the basis of the agent having the attribute of being able to demonstrate that \(\phi\).
    \end{enumerate}
  \end{enumerate}
  In short, if~\ref{prem:ab} is the case then~\ref{denied-claim} requires a certain interpretation of the scenarios identified by~\ref{prem:ab} and~\ref{prem:ni} denies that the interpretation is plausible.
\end{note}

\begin{note}[Tension, choices]
  The combination of~\ref{P:ab-and-dc:W} and~\ref{P:ab-and-dc:A} is complex.
  However, the basic structure is straightforward:
  \[\ref{prem:ab} \rightarrow ((\ref{denied-claim} \rightarrow \AR{}) \land (\ref{prem:ni} \rightarrow \lnot \AR{}))\]
  Rewriting:
  \[\ref{prem:ab} \rightarrow ((\lnot \AR{} \rightarrow \lnot \ref{denied-claim}) \land (\AR{} \rightarrow \lnot \ref{prem:ni}))\]
  Hence:
    \[\ref{prem:ab} \rightarrow ((\WR{} \rightarrow \lnot \ref{denied-claim}) \land (\AR{} \rightarrow \lnot \ref{prem:ni}))\]
  Simplifying:
  \[\ref{prem:ab} \rightarrow ((\WR{} \land \lnot \ref{denied-claim}) \lor (\AR{} \land \lnot \ref{prem:ni}))\]
  A reformulating as a distinction:
  \[\lnot \ref{prem:ab} \lor (\WR{} \land \lnot \ref{denied-claim}) \lor (\AR{} \land\lnot \ref{prem:ni})\]

  In short, we have the following resolutions.
  \begin{enumerate}
  \item\label{ten:res:nS} Agent may not obtain support for result of witnessing ability, or
  \item\label{ten:res:nD} Agent obtains support for result on the basis on premises that the agent would use when witnessing ability --- incompatible with general application of~\ref{denied-claim}
  \item\label{ten:res:nI} Agent obtains support for result from attribute of having the ability on the basis that the support they have for general ability would be misleading --- incompatible with general application of~\ref{prem:ni}
  \end{enumerate}
  \ref{ten:res:nS} is incompatible with~\ref{ten:res:nD} and~\ref{ten:res:nI}.
  However, \ref{ten:res:nI} and~\ref{ten:res:nD} are compatible, as both~\ref{denied-claim} and~\ref{prem:ni} may be restricted.
\end{note}

\begin{note}[Argument sketch recap]
  Let us recap the main points of the argument so far.
  \begin{enumerate}
  \item Assume possibility of cases in which agent is provided with information that they have some specific ability so long as the agent has a general ability, such that the agent has support for having the general ability, but has not established support for possessing the specific ability.
  \item In such cases, it seems it is possible for the agent to obtain support for what follows from the agent witnessing their specific ability.
  \item If so, the agent appeals to having the specific ability in order to obtain support for what follows from the agent witnessing their specific ability.
  \item Attribution, and witnessing.
  \item If witnessing, then conflict with the requirement that an agent must access support for the premises appealed to in support of a conclusion.
  \item If attribution, then conflict with the restriction that an agent may not obtain support for some proposition on the basis that support the agent has for some other proposition would be misleading otherwise.
  \end{enumerate}

  To follow:
  \begin{enumerate}
  \item Restricting~\ref{denied-claim} in favour of~\ref{rC} works well.
  \end{enumerate}
\end{note}

\begin{note}[Meek outlook]
  This is not a clean argument.
  Take~\ref{denied-claim} and~\ref{prem:ni} and hold the first.
  The agent may not obtain support.

  While there may be tension if the agent obtains support, this tension is never instantiated.

  I am sympathetic.

  Still, endorsing the restriction does not require the agent to obtain support in this case.
  Harbour some hope that that there is scope to restrict \ref{denied-claim}, and that the argument provided for resolving tension in favour of \rC{}, along with later arguments, may serve as a source for reflection.
\end{note}

\section{Following structure}
\label{sec:following-structure}

\begin{note}[Structure]
  The initial portion develops tension in detail.
\end{note}