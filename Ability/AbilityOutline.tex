%%% Local Variables:
%%% TeX-master: "master"
%%% End:

\chapter{Overview}
\label{cha:overview}

\section{Outline}
\label{sec:outline}

\begin{note}
  In this chapter we provide a high level overview of the main arguments made in this thesis.
  A significant part of the high level overview of arguments is an overview of the premises and assumptions that those arguments rest on.

  By given high level overview, clarify on how premises, assumptions, and conclusions relate.
  In the main body of the thesis, afford to elaborate.
  And, allow choice of where to seek elaboration.
\end{note}

\begin{itemize}
\item Start with claiming support, used throughout, so important.
\item Introduce and motivate plausible constraint on support, to be argued against/exception for.
\item Outline exception.
\item High level overview of argument for exception.
\item Negative and positive.
\item Largely fairly high level sketch of negative argument.
\item Type of ability information.
\item Understanding \emph{de re} ability reading.
\item \AR{} and \WR{}.
\item For the moment, brief, much more detail in relevant chapter.
\item Relation between \AR{} and \uRa{}.
\item Constraint on reading of ability.
\item Requires argument for \WR{}.
\item Introduce \nI{}.
\item Sketch argument for \nI{}.
\item Link \nI{} and \AR{}.
\item Completes overview of negative argument.
\end{itemize}

\section{Ability and access to claimed support}
\label{sec:abil-access-supp}

\begin{note}
  Following introduction, interest is with ability.
  In particular, observation that \gsi{} information, and confidence in general ability seems to allow agent to claim support for result.

  Question on what basis the agent claims support.

  Slightly more general statement.
  \begin{quote}
    When and why an agent may claim support for the result of reasoning that the agent has not witnessed.
  \end{quote}
  Ability claims of interest provide some answer to `when'.
  For, ability provides information about what the result is, and also that the agent has the opportunity to perform the reasoning.\nolinebreak
  \footnote{No claim to necessity}

  Suggested in introduction, answer to `why' is more complex.
  % \begin{quote}
  %   Our interest is in when an agent may claim support for some conclusion of some instance of reasoning on the basis of the support the agent may claim for the premises of the instances of reasoning.
  % \end{quote}
\end{note}

\begin{note}[Introducing support]
  Initial clarification is with respect to claiming support.
  Emphasis on `\emph{claim}'.

  The thesis is not about when and why an agent \emph{has} support for the result of reasoning that the agent has not witnessed.

  Two reasons for this.
  First, neutral for main thread of argument on what support amounts to.
  Interest is with structure of claim, and background assumption that if success in claiming then structure of support follows structure of claim.

  Second, whether or not an agent has support often seems secondary.
  To illustrate:
  It may be that any claimed support for a proposition is support for that proposition, but perhaps not.
  Suppose `flan' is written on the side of a container.
  I may claim support that the container contains flan.
  And, it may be that the writing on the side of the container is support for the box containing flan.
  However, the straps ensuring the container remains closed is unfortunately placed, and if moved would reveal the side of the container reads `flannels'.
  The unfortunate placing of the straps does not seem to prevent \emph{claiming} support, but I'm not sure whether it is right to say that the writing on the side of the box (straps in place) \emph{does} the box containing flan.
  So, in what follows I will speak in terms of claiming support, and leave open whether what is claimed reflects on whether an agent has support.\nolinebreak
  \footnote{
    In particular, claiming allows focus on internal constrains, while remaining silent on whether having support is (in part) determined by external factors.
  }
  \(^{,}\)\nolinebreak
  \footnote{
    Distinction between propositional and doxastic support.
    Propositional, support agent has whether or not made a claim.
    Doxastic is successful claim and propositional support.
    So, both require that the agent has support.
    Claimed support is the agentive component of doxastic support.
    Not interested in whether the agent also has propositional support, though more or less assume.
  }
  \(^{,}\)\nolinebreak
  \footnote{
    {
      \color{red}
      English is somewhat difficult.
      It is somewhat unfortunate that `an agent has claimed support for \(\phi\)' may be read `there is support which the agent has claimed for \(\phi\)'.
      Still, this seems to follow more easily from `support claimed'.
      So, `claimed support' emphasises the claim, while `support claimed' emphasises support.
    }
  }
\end{note}

\subsection{Background for claiming support and reasoning}
\label{sec:claimed-support}

\begin{note}[Understanding of claiming support]
  Understanding of claiming support.

  Begin with a sufficient condition.
  In short, most instances of reasoning.
  Claiming support is common.

  Then, two types of defeaters.
  Mistaken and misled.
  Use to form a necessary condition.
  If claimed support, then agent deems that claimed support is not defeated.
  `Deem' is a placeholder.
  Strong or weak.
  Single constraint is that when claiming support, potential defeaters that aren't ruled out.\nolinebreak
  \footnote{
    At least two ways of viewing this.
    First, claiming support is restricted.
    Second, \emph{claiming} support only applies when there are potential defeaters, and some other relation to support when possible defeaters get ruled out.

    These are different, but I don't think the difference matter for resource bound agents of interest.
    Lack of resources is that always potential defeater, even if every possible defeater may be ruled out.
    }
  Finally, property, that claimed support does not depend on whether proposition the agent has claimed support for is true, or whether the claimed support \emph{does} support (if these are separated).
  Property will be important.
\end{note}

\begin{note}[Sufficient condition]
  We start with a sufficient condition for claiming support:
  \begin{proposition}[\bP{-} --- \bP{}]\label{prem:bP}
    If an agent may claim support for premises and steps of reasoning, accesses those premises and traces claim to support through those steps of reasoning, then agent may claim support for conclusion on basis of the claimed support for the steps and premises of reasoning.
    (Given that the agent deems that the claims to support for premises and steps used are undefeated when drawing conclusion.)
  \end{proposition}

  The purpose of taking~\bP{} as basic is to fix a basic understanding of when an agent may claim support.
  In short, an agent may claim support when reasoning goes well.
  And, reasoning goes well when there are premises and steps of reasoning available to the agent, and the agent draws on these to claim support for the conclusion.
  The parenthetical remark is a simple safeguard, the agent does not lose a claim to support for the premises or steps in the process of reasoning.\nolinebreak
  \footnote{
    May think that this is the wrong safeguard.
    Consider the liar paradox:
    `This sentence is false.'
    \bP{} prevents agent from claiming support that the sentence is true or false.
    However, may think that agent is in a position to claim support that the sentence is both true and false.
    Indeed, standard reasoning associated with the liar suggests that the sentence is both true and false.
    Still, it's not obvious from demonstrating that the sentence is both true and false that one may claim support for the sentence being true and the sentence being false.
    That is, one may confine the paradox to the truth value of the sentence, rather than (associated) surplus of support.
  }
  We consider defeaters below.
  First, an illustration.


  Suppose an agent measures that the box in front of them has the dimensions of \(19\text{cm}\) by \(7\text{cm}\).
  The agent understands how to calculate the area of a box, and by performing some reasoning comes to hold that the area of the box is \(133\text{cm}^{2}\).
  The support the agent has for holding that the area of the box is \(133\text{cm}^{2}\) is obtained (at least in part) on the measurement of box, understanding how to calculate the area of box, and some grasp of arithmetic.

  Whether some (or all) of the required arithmetic is to be included as a premise or a step of reasoning may be set aside.
  Similarly, set aside whether further arguments, or whether some premises and steps are taken as basic.
  For example, perhaps some the agent requires some further claim to support for using the ruler to measure the box such as comparison to a standard, or perhaps the agent's claim to support terminates by noting that their use of the ruler is a reliable process.
\end{note}

\begin{note}[Value proposition]
  Reasoning and claims to support focus.
  Briefly introduce a pair of propositions to clarify claim to support and reasoning.

  \begin{proposition}[Claimed support is for the value of a proposition]\label{prop:csifvoap}
    When an agent claims support for some proposition, the agent claims that the proposition has some value.
    Where:
    \begin{itemize}
    \item A proposition is some state of affairs. And,
    \item A value is an assessment of a state of affairs.
    \end{itemize}
  \end{proposition}
  The role of \autoref{prop:csifvoap} is primarily to fix terminology.
  To illustrate, when stating the conclusion of the reasoning sketched above we used the proposition that \emph{the area of the box is \(133\text{cm}^{2}\)}.
  The proposition refers to the state of affairs in which the area of the box is \(133\text{cm}^{2}\), and speaking a little more precisely, the agent claimed that the proposition has the value `true' --- though it may be the value turns out to be `false'.
  Or, perhaps if the agent was a little unsure about the accuracy of the ruler, that the proposition has the value `likely', `probable', or some quantitative credence.
  And, some other instance of reasoning may have concluded that the proposition has the value `desirable' --- e.g.\ if the agent was searching for a box of some approximate size.\nolinebreak
  \footnote{
    Nothing in particular hangs on the distinction between different values.
    If you prefer, you may expand the proposition (state of affairs) to include additional factors, and consider only the values `true' and `false'.
    For example, the proposition that \emph{I desire the bath to be warm} is false, as opposed to the proposition that \emph{the bath is warm} is valued undesirable by me.
  }

  Core idea is that claim of support is that things are a certain way.
  Proposition, what the thing is.
  Value, the way it is.
  In most cases the value will be clear (i.e. that the proposition is true, though sometimes that the proposition is desirable), and so we will talk of claiming support for the proposition.
  A handful of additional examples will be provided when illustrating the next proposition.
\end{note}

\begin{note}[Reasoning proposition]
  \begin{proposition}[Reasoning as establishing value]\label{prop:RisTV}
    Reasoning, tracing value through propositions/establishing that proposition has value.
  \end{proposition}

  \begin{itemize}
  \item Testimony, so claim support that \emph{p} is true.
  \item Unreliable, so claim support that \emph{p} improbable.
  \item Religious text, so claim support that \emph{p} ought to be the case.
  \item Producer, so claim support that album is desirable.
  \end{itemize}

  In a deductive case, if the premises are true, then the conclusion is true.
  Means-end reasoning for desire.
  The value is important.
  If it is true that it past 6pm, then it is true the shop is closed.
  Provides value of shop being closed.

  However, if agent desires that it is past 6pm, then it doesn't follow that the agent desires that the shop is closed.
  Question an agent as to why they think their desires conform to truth --- is-ought problem.

  Means-end reasoning.
  It is true that there is cheese at the centre of the maze.
  And, it is desirable that I obtain the cheese at the centre of the maze.
  Further, it is true that I may only obtain the cheese at the centre of the maze by solving the maze.
  Therefore, it is desirable that I solve the maze.
\end{note}

\begin{note}
  There are at least two ways in which a claim to support may be defeated.
  \begin{itemize}
  \item Claimed support may (discovered to be) be \emph{misled} by suggesting that a proposition has some value that it does not (in fact) have. And,
  \item Claimed support may (discovered to be) be \emph{mistaken} by appealing to factors that do not indicate the value of the proposition.
  \end{itemize}
  Misleading support may indicate value, and value may have value indicated by mistaken support.
  Both are defeaters in the sense that, were the agent to learn that the claim to support was misleading or mistaken, then the agent would not hold that the proposition has the value indicated by the (problematic) claim to support or the basis of that (problematic) claim to support.

  ???
\end{note}

\begin{note}[M\&M Illustration]
  To illustrate:

  Suppose I glance at the clock on the wall.
  The clock reads 11:45a, so I claim support that it is 11:45a.
  However, it may be the case that the clock is incorrectly set, and the time is 11:15a, or 12:15p, etc.\
  By claiming support from the time expressed by the clock, I would have been \emph{misled} about what the time actually is.
  For, it is not true that the time is 11:45a.
  Though, in all other respects, there may be no fault with claiming that the time is as expressed by the clock and so the claim to support is not mistaken.

  By contrast, suppose I glace at the clock on the wall.
  The clock reads 11:45a, so I claim support that it is 11:45a.
  By claiming support from the time expressed by the clock, I would have been \emph{mistaken} about what the time actually is.
  For, the time expressed by a broken clock is not a good indicator of what the time actually is.
  Though, despite the clock being broken, it is 11:45a and so the claim to support is not misleading.

  And, claimed support for the time from a broken clock expressing the wrong time would be both misled \emph{and} mistaken.\nolinebreak
  \footnote{
    A second illustration:
    Consider a smoke detector, designed to sound an alarm if and only if sufficient levels of smoke are detected.
    Hence, if the alarm sounds, one may claim support there being smoke in the room where the alarm is installed.
    One may be misled; the alarm may have malfunctioned, so no fire.
    Or, one may be mistaken; the same type of alarm may be installed in a different room, wouldn't be a useful indicator.
  }
\end{note}

\begin{note}[Subjectively sound]
  Of course, clocks are typically glanced at, and a glance at a clock is often insufficient to determine whether the clock is incorrectly set or broken.
  Hence, the \emph{possibility} that a clock is incorrectly set or broken --- or more broadly the possibility that claimed support is misleading or mistaken --- does not prevent an agent from claiming support.
  So, ensuring that to-be-claimed support would be neither mistaken or misleading is not a necessary condition for claiming support.
  Rather, we endorse the following condition with respect to these types of defeaters:
  \begin{proposition}\label{prop:CSNMORM}
    \textbf{(Claimed support is deemed neither misleading nor mistaken)}

    If agent claims support for some proposition, then from the perspective of the agent, the agent deems that the claimed support is not misleading nor mistaken.\nolinebreak
    \footnote{
      Stronger than distinct claim that the agent does not deem that the claimed support is misleading or mistaken.
      Stronger requires, presence of some deeming.
      Later does not.
    }
  \end{proposition}
  Idea of \autoref{prop:CSNMORM} is that in claiming support and agent has some expectation that possible defeaters (of the two types of noted) do not obtain.

  If the claimed support is not misleading, then the proposition has the value the claimed support indicates the proposition has.
  And, if the claimed support is not mistaken, then the claimed support indicates the value.
\end{note}

\begin{note}[Fallibility]
  The missing piece of \autoref{prop:CSNMORM} is an account of what `deem' amounts to.
  Thankfully, a detailed account is not required.
  Instead, \autoref{prop:fallibility} states an assumption, which allows a functional characterisation of `deem'.

  \begin{proposition}[Fallibility]\label{prop:fallibility}
    Agents are fallible.

    Specially, when claiming support for a proposition and agent is never in a position to rule out the possibility that the claimed support is not misled or mistaken.
  \end{proposition}

  I take it to be intuitive that agents are fallible in many cases of claiming support.
  It is not too difficult to think of ways in which claimed support may be misleading or mistaken.
  As noted above, claiming support for what the time is from glancing at a clock seems sufficient, but clocks may be incorrectly set (misled) or broken (mistake).
  Similarly, a sample of \(1,000\) rolls may mislead me into thinking that a die is unbiased, or an overloaded operator may lead to a mistake in claiming support for the proposition that \(x = 4\) is an expression of equality rather than variable assignment.

  However, \autoref{prop:fallibility} states that an agent is never in a position to rule out the possibility that the claimed support is not misled or mistaken.
  In some cases, this may seem absurd.
  Suppose in front of me are two apples and two pears.
  So, four pieces of fruit.

  However, those appears and pears may not be real pieces of fruit, they may be replicas.
  So beings the process of attempting to quarantine fallibility from infallibility.

  There are two pairs of objects in front of me, and the objects appear to be fruit.
  So, there are four objects in front of me, which appear to be fruit.

  There are two pairs of objects hence there are four objects.

  But I may be hallucinating.

  There appear to be two pairs of objects in front of me, which appear to be fruit.
  So, there appear to be four objects in front of me, which appear to be fruit.

  Whenever there are two pairs of objects, there are four objects.

  


  Sceptical scenarios.

  Reduce down to addition.
  Learning/rule following.

  I may state the rule.
  However, no need to claim support.
  Fixing value, rather than establishing that the proposition has some value.

  Still, a few ways to understand `possibility' modal used.

  First, that \emph{in principle} it is not possible for any agent to rule out the possibility that claimed support is not misleading or mistaken.
  Second, only with respect to certain propositions.
  Third, that it is not possible for a \emph{resource bound agent} to rule out the possibility that claimed support is not misleading or mistaken.

  The first denies that there could be, e.g.\ proof of an external world, while the second denies that agents of interest could not demonstrate such a proof even if it were to exist.
  The second allows for some propositions and restricts.
  Third, no matter proposition.

  Hume inspired.
  Process, memory.

  In any case, possibility does not imply much.
  However, possibility is sometimes enough.
  If clocks had changed, then glace isn't sufficient.
  Need something a little more, or claim support for a weaker proposition.

  What follows is compatible with each understanding.
  I hold the third.

  So, following \autoref{prop:fallibility}, the functional role of `deem' is to provide resistance to possible defeaters.
  How resistant the agent's claimed support needs to be is up to you.
\end{note}

\begin{note}
  Still, to help fix intuitions, I suggest a (hypothetical) test --- the `even if\dots' test --- to clarify what I mean by `deem'.
  So long as an agent may provide an adequate responses to the test, the agent will be in a position to claim support.
\end{note}

\begin{note}
  You may think that some of the adequate responses I suggest are too weak, but for future purposes I require only that some positive answer many be given, and so you may strengthen the requirements on a positive answer as you see fit.
\end{note}

\begin{note}[The `Even if\dots' test]
  The `even if\dots' test is simple.
  We have outlined two issues that would prevent an agent from claiming support.
  If misleading or mistaken.
  The `even if\dots' test requires an agent to provide an adequate response to a concern that the claimed support is misleading or mistaken.

  \begin{enumerate}
  \item Even if my claimed support for \(\phi\) is misleading, \dots
  \item Even if my claimed support for \(\phi\) is mistaken, \dots
  \end{enumerate}

  Hypothetical.
  In part because the test requires a response, issues about articulation.
  Indeed, similar issues with respect to formulating an instance of the test.

  Handful of examples to clarify.
\end{note}

\begin{note}[Even if: misled]
  Sunday afternoon: Even if not sleeping, eyes have been closed for a long time and breathing is slow.

  Even if it is a forgery, it seems impossible to detect any thing that would suggest so.

  Even if you're telling the truth, the scientific consensus is against you.

  An interesting case for misled is the preface paradox.
  Claimed support for everything in the preface, but also claimed support for mistake.
  Credence resolves tension, remains noteworthy that even if confident of some potential defeater, claimed support is sufficient to resist undermining claimed support.

  Fault:
  Even if forgery, professes to be the real thing.
  Self-verification, so no response to potential defeater, as requires that the potential defeater does not hold.
  However, no signs of tampering, okay.
\end{note}

\begin{note}[Even if: mistaken]
  Idea here is that claimed support is robust against having made mistakes.
  So, it should be the case that I can take any instance of good support, and highlight some possible mistake, and argue against this.
  The difficulty, of course, is that may be good support because it avoids possibility of being mistaken, so not so clear.

  Even if fake wound, support for calling ambulance.
  In a sense, the person got lucky, fake would and real wound.

  Newspaper may have quoted the wrong person.
  So, it would be a mistake to take word of person to support proposition.
  Even if one is mistaken, newspaper is proofread and checked, etc.
  So, there is no problem with claiming support from person and quote in newspaper.

  Recognise that this would be a problem, but it doesn't block claiming support.

  Fault:
  Even if this library does not using LCC indexing, the library does not have a copy of `Measurement Theory' because as search for `H61 .R593' returns no results.

  Mistaken, because searching for an LCC index in the library's database would not indicate whether or not the library has a copy of the book if the library does not use LCC indexing.
  However, may be supplement by noting that the library is a research library, and therefore likely uses LCC indexing, etc.\
\end{note}

\begin{note}[Even if: more]
  Primary observation from these examples is that in positive cases provided responses indicate that some response to possibility of being misled or mistaken is available to the agent.
  In the failure cases, no response.

  % Cases of entailment, preface paradox.
  % Mistake somewhere.
  % Here, support is good for each of the claims made in the preface, but these do not combine to make a case that no mistake has been made across any of the claims.
  % May come down to familiar concerns, too significant possibility of being misled.
  % May also think that claimed support for each might require a mistake in one.
  % I.e. source for claim includes further claims which state that source for some other claim is mistaken.
  % Problem with using both sources, even if for distinct propositions.

  % Interesting problem later.
  % For now, simple example.
\end{note}

\begin{note}[???]
  \color{red}
  Possible defeaters, so no claim that the reasoning is sound.
  However, agent deems that no defeaters, so may term this `\emph{subjectively} sound' reasoning.

  At least two worries.
  First, given general use of the term `support', considerations may suggest iterated support.
  Second, worries about over-intellectualisation of claiming support.
\end{note}

\begin{note}[\eiS{}]
  Subjectively sound, claimed support indicates value.

  Final proposition.
  Claiming support is independent of value.

  {
    \color{red}
    Corollary to even if test?
    Because, if required value of \(\phi\), then wouldn't be robust with respect to test?
    And, holds regardless the `strength' of the test.
    Because, if dependence, then immediately collapses and fails.
  }

  \begin{proposition}[\eiS{-} --- \eiS{}]\label{prop:supp:independence}
    If agent claims support for some proposition, \(\phi\), then the claimed support is taken to indicate the value of \(\phi\) independently of the value of \(\phi\) or whether the claimed support is `genuine' support
    Equivalently, the claimed support is taken to indicate the value of \(\phi\) does not require that the claimed support is not misled or mistaken.\nolinebreak
    \footnote{
      Possibly goes against externalism, but I don't think this is right.
      External circumstances may impact the support the agent has.
      However, as these are external, it seems this condition plausibly holds for \emph{claiming} support.
      This is how you get puzzles for externalism.
      In both cases, it's fine for the agent to claim support, but the external circumstances impact whether the agent \emph{has} support.
      The internalist/externalist divide would seem to affect the conditions on claiming.

      Way to expand on this is reconstructing bootstrapping examples with and without \eiS{}.
      If the agent would only get basic support if reliable, then it's not clear that bootstrapping is a problem.
    }\(^{,}\)\nolinebreak
    \footnote{
      One way independence.
      Not clear that value is independent of support.
      So long as sufficiently strong support, not possible for proposition to have value other than claimed support.
    }
  \end{proposition}
  \eiS{} follows from Propositions~\ref{prop:CSNMORM} and~\ref{prop:fallibility}.

  From Proposition~\ref{prop:CSNMORM}, deemed that claimed support is neither mistaken nor misled.
  From Proposition~\ref{prop:fallibility}, always possibility.

  Suppose depends on value of \(\phi\).
  If so, then no claim to support if mistaken or misled.
  By Proposition~\ref{prop:fallibility}, possibility that value or basis for \(\phi\) doesn't hold up.
  So, Proposition~\ref{prop:CSNMORM} denies support.
  For, possibility that \(\phi\) does not have value, or that claimed support does not indicate.
  And, given claim to support requires that the possibility does not obtain, agent has not deemed that the claimed support is not misled or mistaken --- rather the agent requires that the claimed support is not misled or mistaken.

  \eiS{} does not deny that things may need to be a certain way for an agent to claim, or to be in a position to, claim support.
  It may be the case that no agent would be in a position to claim support that the speed of light is constant if the speed of light were not constant, but in claiming support an agent must deem that possible defeaters do not obtain, e.g.\ that the laws of nature are constant, and that no mistakes have been made when observing relevant phenomena.

  The force of the corollary is that agent does not require \(\phi\) related things being a certain way in order to claim support for \(\phi\).

  See with failures of `even if\dots' test.
\end{note}

\begin{note}[Quick clarification on \eiS{}]
  A quick clarification may be in order.

  \eiS{} is only about value/support for\(\phi\).
  So, \eiS{} does not prevent agent from claiming support for \(\psi\) from value of \(\xi\) given claimed support that \(\xi\) has that value.
  For example, an agent may claim support that \emph{p} is true from claimed support that \emph{S} knows \emph{p}.
  And, the agent may do so because the proposition that \emph{S} knows \emph{p} is true only if \emph{p} is true.
  That is, so long as the agent does not require \emph{p} to be true in order to claim support for the proposition that \emph{S} knows \emph{p} is true.
  We will return to \eiS{}, expand on this quick clarification, and note related observations in Section~\ref{sec:second-conditional}.
\end{note}


\begin{note}[Adequate reasoning]
  Term this \emph{adequate} reasoning.
  May be good, may involve mistakes, may be bad.
  Kind of reasoning that we, the folk, do.
  Distinction for claiming support is that this is different from whether the agent has support, and we may set issues about whether the agent has support.

  Our interest is what is required for an agent to \emph{claim} support for (premises and) steps of reasoning, rather than what is required for an agent to \emph{have} support for (premises and) steps of reasoning.

  Use support as opposed to justification.
  Initial focus is on epistemic/doxastic attitudes.
  However, practical reasoning.
  For example, means-end.
  Support considered quite general to also include this.
\end{note}

\begin{note}[Closing support]
  To summarise, claim of support.
  Certain kind of independence.
  Only interested in support, and not how this relates to attitudes.
  Somewhat intuitive, but no claims that this is the only understanding of support.

  For the moment, this provides clarity for understanding of support.
  Below, use to argue for failure to claim support.
\end{note}

\subsection{Interest with claiming support and reasoning}
\label{sec:inter-with-claim}

\begin{note}[Focus]
  Consider the converse of~\bP{}:

  \begin{proposition}[\uRa{-} --- \uRa{}]\label{denied-claim}
    An agent may claim support for conclusion on basis of support the agent has for premises only if \emph{some} agent has accessed support they have for premises and traces implication through (at least) adequate reasoning.
  \end{proposition}

  \uRa{} seems quite plausible.\nolinebreak
  \footnote{
    Three brief notes on~\uRa{}:

    First, the `has' in~\uRa{} only requires `at some point in the past'.
    Hence,~\uRa{} does not require the agent to reason from premises to conclusion each time the agent claims support for the conclusion.
    For example, if an agent proved the Deduction Theorem for propositional logic last week, then the agent would not be in conflict with~\uRa{} if they claimed support for the Deduction Theorem on the basis of the premises and reasoning they performed in the past.

    Second, and following from the first, the against~\uRa{} will also hold for any stronger statement --- for example if `has' is read as `has just'.
    For example, requiring that the agent's memory of proving the Deduction Theorem allows the agent to claim support, rather than the premises and steps used in the past.
    The argument (stated below) denies that, given certain information, the agent needs witnesses any reasoning in order to claim support for the result of witnessing the reasoning.

    Third, as~\uRa{} is about when an agent may \emph{claim} support, it is compatible with~\uRa{} to hold that the agent \emph{has} support --- regardless of whether the agent has witnessing the reasoning.
  }

  \uRa{} might not be obvious, for \emph{some} agent rather than \emph{the} agent.
  Stronger implies \uRa{}, so this isn't too much of a problem.
  Might want to go for the weaker if there's something social.
  Here, additional conditions, but our interest with \uRa{} is as a necessary condition.

  Start with stronger example.

  If the agent did not measure the box, understand how to calculate the area of a box, or perform the arithmetic, the agent would not be in a position to claim support that area of the box is \(133\text{cm}^{2}\).
  A lucky guess that the area of the box is \(133\text{cm}^{2}\) would not allow the agent to hold that the area of the box is  \(133\text{cm}^{2}\) on the basis of the dimensions of the box, the agent's understanding of how to calculate the area of a box, and arithmetic.
  And, it seems the agent is not in a position to base their lucky guess in such a way because the agent did not reason from the dimensions of the box, the agent's understanding of how to calculate the area of a box, and arithmetic.

  Moving to another agent, observe doing the work, get report.
  Easy to resist, by adding in additional premise.
  Still, no presupposing that this needs to be done.

  {
  Again, appealing to \eiS{}.
  Tracing support in this way identifies what there is `to be said' for \(\phi\)/conclusion.
  }

  Further,~\uRa{} not only seems quite plausible, but is either explicitly or implicitly appealed to when characterising the support an agent may claim, or reasons, an agent has for some (reasoned to) proposition.\nolinebreak
  {\color{red}
    Intuitive understand of doxastic support.
    Following, two prominent accounts of the basing relation.
    Other examples in rationality.
    For now, take intuitive plausibility.
    Chapter~\ref{cha:access} (will have) more detail.
  }
\end{note}

\begin{note}[Alternative]
  \uRa{} is that it is a universal claim, and so applies to all instances in which an agent may claim support for conclusion on basis of support for premises and steps of reasoning --- an agent may only claim support if the reasoned from the premises via the steps.

  Our goal is to motivate the following exception to \uRa{}:
  \begin{proposition}[\rC{-} --- \rC{}]\label{rC}
    If an agent has information that they have the ability to (adequately) reason to some conclusion, then the agent may claim support for the conclusion on basis of support for premises that the agent would access and steps used by witnessing their ability to reason to the conclusion.
  \end{proposition}

  If~\rC{} is true, then there are cases in which an agent is not required to reason from premises they may claim support for to some conclusion in order to obtain support for the conclusion on the basis the support the agent has for the premises.

  Stated~\rC{} as an exception to~\uRa{}, but we are not arguing that~\rC{} is an exception to~\uRa{}, which would require an argument that \uRa{} holds for other cases.
  Nor that \rC{} is the only exception, which would require a stronger argument that~\uRa{} holds for all other cases.
  Take~\uRa{} to be plausible, and suspect that there are few, if any, further exceptions, but~\rC{} may stand independently on any further statements about (claiming) support.

  As with~\uRa{}, \rC{} does not entail that the agent \emph{has} support.
  {
    \color{red}
    Some more notes on why this is important.
    Basically, this means that~\rC{} isn't too strong.
  }
\end{note}

\begin{note}[Normative/weakness]
  Consider~\rC{} somewhat weak as it's only about claiming support.
  Doesn't show that the agent has support.
  Agent may be mistaken, in the sense used above.
  So, in particular, normative conflict.
  Ought not use information about ability.
  I don't think this is correct, normative considerations distinguish good instances from bad instances.
  Not all cases are good, though I think the ones focused on are.
  In any case, normative is a separate argument, and need to get clear on what the phenomena is before going in for normative considerations.

  So, only dealing with `adequate' reasoning.
\end{note}

\begin{note}[Ability]
  The exception is motivated by the agent having information that they have the ability to (adequately) reason to the conclusion.

  Idealised agents have no need to appeal to ability.
  However, for limited agents, ability is abundant, while the resources required to witness abilities are scarce.
  That the exception to~\uRa{} is narrow does not entail that there are few occurrences of the exception.

  Information about ability may be abundant while the resources for witnessing abilities are either scarce or temporarily unavailable.
  So, for example, agent has the option of conserving or deferring use of resources.

  If~\rC{} then a novel perspective on limited agents.
  When and why limited agent may claim support for result of reasoning.
  Here, conflict with appeal to~\uRa{} it its unrestricted form.

  And, in between propositional and doxastic support (or more commonly justification).
  Propositional support doesn't allow agent to claim support.
  Doxastic support, agent does claim support.
  If ability allows agent to claim support, \dots

  Secondary motivation is with ability to reason itself.
  Even if arguments for~\rC{} fail to convince, not much has been said on ability to reason, and first steps are useful.
\end{note}

\begin{note}[Interest in ability]
  Broadening scope.
  Arguments involving~\uRa{}.
  Distinction between ideal and non-ideal.
  Potential alternative conclusions to arguments that appeal to~\uRa{} as a premise.
  Revise premises for arguments in which~\uRa{} is a conclusion.
\end{note}

\begin{note}[Examples]
  Need quick truth and desire examples.
  I don't think I need to make these examples compelling, as giving borderline examples may help motivate interest.

  \uRa{} is common in basing literature.
  Support is general, extends to desires.\nolinebreak
  \footnote{
    Strong view on which an agent may be mistaken about desires in the same way as an agent may be mistaken about evidence.
    View on which desires are independent of representation.
    Hence, misleading or mistaken support when an agent fails to represent desire.
  }
  Here, temptation.
  Rely on ability to demonstrate that abstaining from a \(\text{n}^{\text{th}}\) glass of wine, even though the reasoning the agent performs after the \(\text{n-1}^{\text{th}}\) glass supports drinking another glass.
\end{note}

\begin{note}[Small corollary]
  Small corollary is that if the agent has reasoned, then don't need to rely on memory as premise for claiming support.
  Rather, memory works to provide information.

  To extent this is plausible, this is the `backwards' looking part of~\rC{}.
  It's the `forwards' looking part that we will focus on.
\end{note}

\begin{note}[\emph{Exception}]
  As an exception, this doesn't mean that the (appeal to) support is the same as would be if the agent had done the reasoning.
  One way to frame is that doxastic or no claim to support.
  See~\uRa{} as what it takes to obtain doxastic support.
  Deny this.
  There is some other kind of relation to support.
\end{note}

\begin{note}[No causality]
  Important to note, as seems easy to confuse.
  Not claiming that the agent's attitude is causally related to the premises.
  Cause and `basing' may come apart.
  What matters is relation, causation ensures this relation.
  Denied is that based only if caused.
  Information about ability identified as cause, but argue that this is insufficient in certain cases.
\end{note}

\begin{note}[Motivating idea, value]
  Thinking about attitudes.
  Some kind of value on the attitude.
  Doxastic, truth, maybe.
  Why should it matter that the reasoning has been performed prior to forming the attitude?

  Well, providing support regardless of what the value turns out to be.

  Here, a little different.
  Cases in which, whatever it turns out to be, agent is permitted to form such an attitude, constrain with additional condition which seems harder to give up.
\end{note}

\begin{note}[Motivating idea, normative]
  Seems as though there's a plausible normative dimension, in which an agent may be criticised on what they are able to do.
  In particular, ability to reason further, but did not do so.
  Difficult to understand this as a requirement to reason, given constraints on resources.
  Doesn't seem to excuse, or so one may think.

  That is, this is something in between
  \begin{enumerate}
  \item The agent not having the resources.
  \item The agent having the resources, but being unaware that they may use them.
  \item The agent having the resources, and being aware that they may use them.
  \item The agent using the resources.
  \end{enumerate}
\end{note}

\newpage

\section{Structure of argument}
\label{sec:structure-argument}

\begin{note}[Structure of argument]
  Two lines of argument for endorsing~\rC{}, and hence denying~\uRa{}.
  \begin{enumerate}[label=(L\arabic*), ref=(L\arabic*)]
  \item\label{arg:line:1} Motivate~\rC{} as resolution to tension resulting from~\uRa{}.\newline
    Specifically:
    \begin{enumerate}[label=(L1\alph*)]
    \item\label{arg:line:1:a} Provide recipe for generating scenarios where~\uRa{} is in tension with particular scenarios involving information that an agent has the ability reason to some conclusion and a further claim regarding support.
    \item\label{arg:line:1:b} Motivate~\rC{} as a resolution to the tension.
    \end{enumerate}
  \item\label{arg:line:2} Argue that granting~\rC{} as an exception to~\uRa{} allows for an intuitive understanding of cases in which agent has the option of appealing to ability, even if there are alternative ways of interpreting the scenario in line with~\uRa{}.
  \end{enumerate}
  These two lines of argument work together.
  The tension of~\ref{arg:line:1} generates interest in witnessing that may be flatly rejected by prior endorsement of~\uRa{}.
  The intuitive understanding of scenarios involving ability of~\ref{arg:line:2} suggests there's more to witnessing than resolving the tension in narrow cases.
\end{note}

\begin{note}[Details of \ref{arg:line:1}]
  The initial focus is on the first line of argument,~\ref{arg:line:1}.
  The tension developed in part~\ref{arg:line:1:a} is delicate, but hopefully informative.
  We will establish a number of corollaries regarding ability and the interaction between~\uRa{} and ability.
\end{note}

\begin{note}[Before turning to the argument\dots]
  Before turning to the argument, we conclude this introduction with a handful of notes regarding~\uRa{} and~\rC{}.
\end{note}

\begin{note}[Scope of \uRa{}]
  \uRa{} does not say anything in particular about what the agent may claim support for, only what must be the case in order for an agent to appeal to support for some conclusion on the basis of support for premises.

  Talking in terms of (support for) premises and conclusions restricts attention to reasoning.
  There may be broader use of `premise' and `conclusion' where an agent is not required to reason from premise to conclusion in order for the premise to support the conclusion.
  For example, if visual perception is immediate.
  Perhaps it may be said that an agent's visual experience is a premise to the conclusion that a dog is sleeping.
  Still, for present purposes, `conclusion' refers to the output of some process of reasoning performed by an agent which is either actual or potential, and `premises' to the input of that process.

  Note, also, that in both cases the relation between premises and conclusion is important.
  If agent does not reason, then neither~\bP{} nor~\uRa{} apply.
  If there are multiple ways to obtain a conclusion, then~\uRa{} does not require the agent to reason from a particular set of premises.

  Likewise,~\uRa{} does not require that an agent is required to obtain support for a proposition by valid and subjectively sound reasoning from some premises.

  Rather,~\uRa{} requires that an agent reason from premises to conclusion in order to establishes support between premises and conclusion
  By contrast,~\bP{} holds that reasoning is sufficient to establish such a relation.
\end{note}

\begin{note}[\uRa{} is intuitive]
  \uRa{} is intuitive, and is quite common, though not without exceptions.
(For example, there's views on testimony in which the testifier provides agent access to support the testifier has.
One may understand this as conflicting with~\uRa{}, or that the fact that these are accessible is the relevant piece of support.)
\end{note}

\begin{note}[Alternative]
  \rC{} restricts~\uRa{}.
  This is not to say the agent obtains support equivalent to that which would be obtained were the agent to do, or have done, the reasoning.
  Nor, that the agent is aware of the relevant premises.

  Intuitively, \rC{} states that the agent may appeal to the reasoning they are able to perform in support for the conclusion of that reasoning, and as that reasoning moves from premises to conclusion, it is on the basis of the support for those premises that the agent would identify by reasoning that the agent obtains (some) support for the conclusion.

  Hence, \rC{} is in line with the spirit of~\bP{}.
  For the exception to~\uRa{} is granted by the agent appealing to a witnessing event in which the antecedent (and consequent) of~\bP{} are satisfied.
\end{note}

\begin{note}[Ability ensures propositional?]
  Plausible that if the agent has the ability, then the agent already has propositional support for the relevant proposition.
\end{note}

\section{Negative argument overview}
\label{sec:broad-argum-overv}

\begin{note}[Overview]
  Tension resulting from the unrestricted scope of~\uRa{}.
  We begin by introducing a particular type of scenario involving ability, and observe how~\uRa{} requires a unique interpretation of the scenario.
  We then introduce an additional principle regarding support, which conflicts with the interpretation of the type of scenario introduction required by~\uRa{}.
\end{note}

\begin{note}[Introducing key parts]
  Type of information and entailment.
  Two ways to understand entailment.
  Then, if information and entailment \dots
  Principles constrain understanding.
  \uRa{} and a second principle.
\end{note}

\subsection{Type of scenario}
\label{sec:type-scenario}

\begin{note}[Tension, information]
  The tension arises when an agent receives (limited) information that:
  \begin{enumerate}[label=(\GSI{}), ref=(\GSI{})]
  \item If the agent has a general ability to \(\gamma\), then the agent has a specific ability to \(\varsigma\) (as an instance of the general ability).
  \end{enumerate}
  The information is limited because it does not directly provide the agent with the information that the agent has the specific ability, nor that the result of witnessing the specific ability is the case.

  \GSI{} as \gsi{}.
  Information is that if the agent has a general ability to \(\gamma\), then as an instance of the general ability the agent has the specific ability to \(\varsigma\).

  For example,
  \begin{enumerate}[label=(\GSI{}\arabic*), ref=(\GSI{}\arabic*)]
  \item\label{qe:cond} If you have the ability to reason with the rules of chess, you have the ability to demonstrating that there is a sequences of moves that will ensure a win for one of the players (as an instance of the general ability).
  \end{enumerate}
  The conditional structure\nolinebreak
  \footnote{
    Strictly speaking the formulation as a conditional isn't important.
    What matters is that the agent is required to endorse general ability.
    \begin{enumerate}[label=(\GSI{}\('\)), ref=(\GSI{}\('\))]
    \item Either the agent does not have the general ability to \(\gamma\), or the agent has a specific ability to \(\varsigma\).
    \end{enumerate}
  }
  of the information distinguishes \GSI{} from (\dSI{}):
  \begin{enumerate}[label=(\dSI{}), ref=(\dSI{})]
  \item You have the ability to \(\varsigma\).
  \end{enumerate}
  With respect to the example:
  \begin{enumerate}[label=(\dSI{}\arabic*), ref=(\dSI{}\arabic*)]
  \item\label{qe:cons} You have the ability to demonstrate that there is a sequences of moves that will ensure a win for one of the players.
  \end{enumerate}
  Where `\dSI{}' stands for direct specific ability information as the agent directly receives information that they have a specific ability.

  By contrast, with \GSI{} the agent is required to obtain~\ref{qe:cons} from~\ref{qe:cond} by endorsing the antecedent --- that they have the (general) ability to reason with the rules of chess --- and so it may not be the case that the agent has the specific ability.

  So, from \GSI{}, the agent is not provided with information that:
  \begin{enumerate}[label=(I\arabic*), ref=(I\arabic*), resume]
  \item\label{qe:result} There is a sequences of moves that will ensure a win for one of the players.
  \end{enumerate}
  For, it need not be the case that~\ref{qe:result} is true if~\ref{qe:cond} is true by virtue of a false antecedent.
  While, \ref{qe:result} does follow from~\ref{qe:cons}.

  Of course, the antecedent of~\ref{qe:cond} need not be false.
  Rather, the observation that the antecedent of~\ref{qe:cond} highlights how \GSI{} requires the agent to appeal to their general ability in order to obtain information about how the agent's general ability extends to a particular case.

  However, if the agent holds that they have the ability to demonstrate that there is a sequences of moves that will ensure a win for one of the players, then the agent may reason to~\ref{qe:result}.
\end{note}

\subsection{(The) Ability entailment}
\label{sec:ability-entailment}

\begin{note}[\aben{}]
  Note here on \aben{}.
  Then  ways of understanding \aben{}.

  \begin{proposition}[Ability entailment]
    Entailment of the from `S has (specific) ability to \emph{V} that \(\phi\)' and so `\(\phi\) is the case'.
  \end{proposition}

  \aben{} links ability and result.

  Intuitive that \aben{} holds.

  Cases of interest, \aben{} applies to obtaining result.
  Hence, specific ability to demonstrate the existence of a strategy, so a strategy exists.

  Focus is on why it holds.
  Distinguish two ways in which ability is used in some instance of reasoning, and then use these to elaborate on \aben{}.

  Outline two interpretations, and argue that these are exhaustive.

  Argue that only one understanding is compatible with \GSI{} and \aben{}.
\end{note}

\subsection{\WR{} and \AR{}}
\label{sec:wr-ar}

\begin{note}[\WR{} and \AR{}]
  We term these \AR{} and \WR{}, respectively.
  Brief descriptions from detached perspective.

  \begin{proposition}[\AR{}]\label{A:s}
    Appal to attribute of ability for each use of ability.
  \end{proposition}

  \AR{} applies to use of ability in some reasoning.
  In all cases, appeal the attribute.

  Support for general, so support for specific, and hence support for conclusion.
  With this, apply to \aben{}.

  \begin{proposition}[\WR{}]
    Appeal to witnessing ability at some point in reasoning.
  \end{proposition}

  Key idea of witnessing is that there is \emph{some} use.
  And, existential to allow use of attributes, as with use of general.

  \AR{} is always in line with \uRa{}, the role of witnessing in \WR{} is in conflict with \uRa{}.
\end{note}

\begin{note}[\PA{} and \PW{}]
  \begin{proposition}[\PA{}]
    A strategy must exist in order for it to be the case that an agent to possess the ability to demonstrate that a strategy exists.

    Therefore, if an agent may claim support for possessing the ability to demonstrate that a strategy exists, then the agent may claim support for the existence of a strategy, and support transmits over entailment.

    So, the attribute is sufficient to grant the entailment.
  \end{proposition}

  Applying to scenarios of interest, \AR{}, attribute of specific ability, and in turn from attribute of general ability.
  Agent claims support for general, so allows agent to claim support for specific, and in turn claim support for result.

  \begin{proposition}[\PW{}]\label{W:s}
    Event of reasoning from premises to conclusion.
    Specific ability, requires being in a position to claim support for premises and steps.
    Hence, appeal to support for premises and steps.
    (Rather than attribute.)
  \end{proposition}

  There is a slight alternative.
  Here, the agent diverges and claims that they have the attribute.
  Then, in line with \PA{}.
  Remains a case of \WR{}, as witnessing from general to specific.
\end{note}

\begin{note}[Applied to scenarios]
  With scenarios, \gsi{}.
  \AR{} is attributions all the way through, finish with \PA{}.

  \WR{} is witnessing at some point.
  General and information, to instance of specific.
  Claim support for each premise or step used, given attribute, and so claim support for conclusion, finish with \PW{}.
\end{note}

\begin{note}[Terminology]
  Use \AR{} and \WR{}, as these imply \PA{} and \PW{}.
  \PA{} and \PW{} to further disambiguate if needed.
\end{note}

\begin{note}[Intuition for \AR{} and \WR{}]
  Both \AR{} and \WR{} are ways to understand \aben{}, which is in turn about what is entailed by an agent having a (certain kind of) specific ability.

  \AR{} focuses on the idea that the agent may claim support from having the attribute (or the truth) of the specific ability.
  \WR{} focuses on the idea that the agent may claim support from witnessing (or using) the specific ability.

  \AR{} requires support for attribute, which in turn suggests in a position to claim support for premises and steps.
  \WR{} requires support for premises and steps, which in turn suggests in a position to claim support for attribute.

  \AR{} doesn't require agent to claim support for premises and steps.
  \WR{} doesn't require agent to claim support for attribute.

  Intuitive example.
  Answer on a test.
  Student puts in the correct answer.
  What matters is that the answer is correct, attribute of the student.
  What matters is that the student has performed the reasoning correctly, the result of some process.

  It seems there is some difference to strength of claimed support.
  However, interest is not in reasoning that has been witnessed.
  Rather, reasoning that the agent has the ability to witness.\nolinebreak
  \footnote{
    \uRa{} provides an explanation here.
    For, if agent did not reason, then wasn't in a position to claim support.
    Hence, trace of the process suggests this condition has been met.

    Arguing against \uRa{}, however.
    This doesn't block the observation, as we're restricting \uRa{} rather than denying \uRa{}.

    Imagine student being provided with information about what to use.
    Student clarifies their understanding.
    To me, this does seem distinct form the two mentioned observations.
  }

  Shortly distinguish these with \uRa{} --- bad for \WR{}.
\end{note}

\begin{note}[Quite brief]
  Sketches of \AR{} and \WR{} are brief.
  Expand on these in the following sections (\ref{sec:first-conditional} and~\ref{sec:second-conditional}) to some extent, and chapter~\ref{cha:potent-infer-attr} will focus on a detailed account of both.
\end{note}

\begin{note}[No third option]
  \begin{proposition}\label{prop:WR-and-AR-exhaustive}
    \AR{} and \WR{} seem exhaustive.
  \end{proposition}
  No alternative source of information about conclusion.
  Hence, specific ability is required, and is likewise novel.
  So general ability is a required premise.

  Ability is describing some action that may be witnessed by the agent.
  Unclear what else there is than mentioning or witnessing action.

  \begin{proposition}\label{either-AR-or-WR}
    Either \WR{} or \AR{} for \aben{}.
  \end{proposition}
\end{note}

\begin{note}[\WR{} alternative]
  Key is support for premises and steps, so invoke general ability in understanding of \WR{}.

  \begin{enumerate}
  \item Agent has support for general ability, and specific ability is an instance of general ability.
  \item So, given the \GSI{}, there is a potential event in which agent does reasoning, and given general ability, the agent is in a position to claim support for each premise and step of reasoning.
  \end{enumerate}

  Possible to reformulate with anything that places agent in a position to claim support.
  Likewise, general will be used to develop tension for \AR{}, but may be possible to formulate without.

  Hence, given \label{either-AR-or-WR}, if problem with \AR{} then \WR{} and not \uRa{}.
\end{note}

\begin{note}[Basic idea]
  We will return to \GSI{} in greater detail below.
  For now, the basic idea is that the agent is on the hook, so to speak, for holding that they have the specific ability.

  Scenarios of this kind are likely uncommon.
  If assume some Gricean maxims, then seems to require that the informer does not have stronger information, or that stronger information is not relevant.

  Perhaps the informer does not want the agent to rely on the informer's information for the existence of the strategy.
  Or, perhaps the agent only wants to appeal to their own understanding of chess.

  Or may be read as a slight challenge.
  The relevant interpretation of `if you're smart enough, you can solve this problem' seems clear.
  `If your ability to reason is of sufficient worth, then by extension of that ability, you have the ability to solve this problem.'
  Paraphrased, `if you're smart enough, you have the ability to solve this problem'.
  So challenged, and confident in one's smarts, one may expect to solve the problem.
  The slight difference with the limited information of interest is that the informer provides information about what the solution to the problem is if the agent is `smart enough'.

  Scenarios require the agent to use this information.
  May be the case that this kind of information is used when alternatives are available.
  E.g.\ hold \(\phi\) not only because testimony, but because ability.

  Main point is moving from general to specific.
  However, focus point is use of specific.
  How this works when the agent appeals to ability.
  Role of \GSI{} is to capture idea that agent may have reason to appeal to specific ability.
  Plausible that if agent appeals to specific then something like general in the background.
  Even if somewhat rare, then, fairly clean set-up for appeal to specific ability.

  Note, in particular with \ref{qe:cons} the agent may hold that \(\phi\) without holding that they have the specific ability to demonstrate that \(\phi\).
  For example, consider instructor telling a student that they have the ability to show that ? is the solution to the problem.
  Student may reason that the instructor is not wrong about the solution, but is wrong about the student's ability.

  If student is given \GSI{} then it's not clear that the agent gets the solution.
  Plausible that variant information, though.
  ? is the answer, and so long as general, then specific.
  Though, as before, agent doesn't need to consider ability with such a variant, as route to ? which is independent of the agent's ability.
\end{note}

\begin{note}[Scenario proposition]
  For ease of reference, we wrap scenarios involving the limited information as a proposition.
  \begin{proposition}[\eA{-} --- \eA{}]\label{prem:ab}
    {
      {\color{red}
        Reform this with \aben{}.
      }
      Interest in cases is an instance of \aben{} with certain type of support for specific ability.
    }
    It is possible for an agent to use information that they have some specific ability so long as the agent has some general ability to claim support for what follows from the specific ability.
    (Where the agent lacks doxastic support for what follows, and for \(A(\varsigma)\) without information).
  \end{proposition}
\end{note}

\begin{note}[Possible restrictions]
  The important aspect of premise~\eA{} is that there are cases in which the agent may appeal to ability to obtain support.
  This is quite weak.

  Understanding of support here is primarily for the agent.

  It allows that there may be cases in which the details of the cases outlined are satisfied, but where kind of support is unsuitable for certain purposes.
\end{note}

\begin{note}[Normative, again]
  In particular, some witness of ability may be demanded by a third-party.
  Perhaps due to lack of confidence in agent, or contextual features of the scenario.
  This is no different from memory.
  Memory of proving \(\phi\) provides support for \(\phi\).
  Still, one may still demand a demonstration of \(\phi\).
  Perhaps the third-party considers the agent's memory unreliable, or perhaps context has been set so that memory is insufficient to add a proposition to the common ground, etc.
\end{note}

\subsection{\uRa{} and \WR{}}
\label{sec:first-conditional}

\begin{note}[Summary]
  In this section, argue that \uRa{} conflicts with \WR{} understanding of \aben{}.
  And, show that \AR{} is compatible with \uRa{}.
  So, as long as disjunction holds, \uRa{} requires \AR{}.

  That's the main takeaway.
  Secondary takeaway is that \gsi{} needs allow the agent to claim support for having attribute for specific ability from claimed support for general ability.
  This will be important in the following section.
\end{note}

\subsection{Incompatibility of \WR{} and \uRa{}}
\label{sec:incomp-wr-ura}

\begin{note}[Proposition]
   \begin{proposition}\label{mcA:WR-and-denied-claim}
    \WR{} is incompatible with~\uRa{}.
  \end{proposition}
    For,~\ref{P:ab-and-dc:W:ab} and~\ref{P:ab-and-dc:W:uRa}.
  Then, agent obtains support by~\ref{P:ab-and-dc:W:ab}.
  As \uRa{}, then from \ref{mcA:WR-and-denied-claim} not \WR{}.
  So, from~\ref{either-AR-or-WR}, must be \AR{}.
\end{note}

\begin{note}[To argument]
  {
    \color{red}
    We provided a brief argument for~\ref{either-AR-or-WR} in the previous section.
  }
  So, what follows is a brief argument for~\ref{mcA:WR-and-denied-claim}.
\end{note}

\begin{note}[Attribute]
  \WR{} is an instance of~\rC{}, as the agent obtains support for the conclusion of the reasoning is able to do on the basis of the reasoning that would be performed in a witnessing event.
  Hence, the supported obtained for the conclusion is obtained on the basis of the support the agent has for the premises that would be used.
  Again, this does not imply that the agent obtains support for the conclusion which is equivalent to the support the agent would obtain by witnessing their ability by performing the reasoning.
\end{note}


\begin{note}[Compatibility]
  However, \AR{} suggests an alternative way to obtain support for the conclusion of reasoning the agent is able to do.
  Specifically, if order for the agent to \emph{have} the attribute of being able to reason to the conclusion, the conclusion of the reasoning must be true.
  The relevant entailment is in part secured by the verb chosen, and in part by what the verb is applied to.
  Here, `demonstrate' is a factive verb, if an agent demonstrates that \(\phi\), then it is true that \(\phi\).
  And, the existence of a chess strategy does not depend on the agent demonstrating that the relevant strategy exists.

  To take another example, you only have the ability to identify a typo on this page if there is a typo on this page.
  So, if I were to provide you with testimony that you have the ability to identify a typo on this page, you may begin searching for the typo, or you may note that there must be a typo in order for me to be in a position to provide you with testimony that you have the ability.

  The reasoning is summarised with the following sketch.

  \begin{enumerate}[label=(\textsf{A}\arabic*), ref=(\textsf{A}\arabic*)]
  \item\label{WR:Sketch:1} I have the attribute of being able to \emph{V} that \(\phi\).
  \item\label{WR:Sketch:2} In order to have the attribute of being able to \emph{V} that \(\phi\), \(\phi\) must be the case independent of whether or not I witness the ability.
  \item\label{WR:Sketch:3} \(\phi\) is the case.
  \end{enumerate}

  To keep things simple, we will refer to the principle behind the pattern sketched as \AR{}.
  And agent may bundle~\ref{WR:Sketch:1} and~\ref{WR:Sketch:3} into a conditional, and avoid instantiating the reasoning pattern, but so long as the conditional is (implicitly) held on the basis of the intermediate premise~\ref{WR:Sketch:2}, we take use of such a conditional to be an instance of \AR{}.

  \AR{} is compatible with \uRa{}.
  For, the two premises~\ref{WR:Sketch:1} and~\ref{WR:Sketch:2} are accessible to the agent, and obtaining \ref{WR:Sketch:3} from~\ref{WR:Sketch:1} and~\ref{WR:Sketch:2} appears to be straightforwardly sound reasoning.
\end{note}

\subsubsection{Summary}
\label{sec:uRa-and-wr-summary}

\begin{note}[Conditional A]
  The first conditional we establish highlights how \uRa{} constrains how an agent may use \gsi{} in the type of scenarios described by \eA{}.

  \begin{proposition}[\mcA{}]
  \begin{enumerate}[label=(C\Alph*), ref=(C\Alph*)]
  \item\label{P:ab-and-dc:W} If
    \begin{enumerate}[label=(\roman*), ref=(CA.\roman*)]
    \item\label{P:ab-and-dc:W:ab} an agent may claim support for the conclusion of reasoning they are able to do in cases described by~\eA{}, and
    \item\label{P:ab-and-dc:W:uRa} \uRa{} is true,
    \end{enumerate}
    then
    \begin{enumerate}[label=(\roman*), ref=(CA.\roman*), resume]
    \item\label{P:ab-and-dc:W:AR} the claimed support for \(\phi\) is obtained on the basis of the agent having the attribute of being able to demonstrate that \(\phi\) (i.e.\ \AR{}).
    \end{enumerate}
  \end{enumerate}
\end{proposition}

  The reasoning described in the consequent of the conditional, \ref{P:ab-and-dc:W:AR}, is in line with \AR{} --- the support the agent obtains for the conclusion of the reasoning that they are able to do is obtained from the support the agent has for having the attribute of being able to reason to the conclusion.
\end{note}

\begin{note}[Summarising]
  ???
\end{note}

\begin{note}[Note on why \AR{} does not conflict with \uRa{}]
  \AR{} and \uRa{} are compatible.
  For, \AR{} is attribute.
  So, if instance of \AR{} agent is claiming support for attribute.
  This is accessed.
\end{note}

\subsection{\nI{} and \AR{}}
\label{sec:second-conditional}

\begin{note}[Redo of section]
  Seen \uRa{} and \WR{}.
  Turn to \AR{}.
  Also saw in last section certain kind of support required.

  Introduce a general constraint on claiming support.
  The general constraint will relate to moving from general to specific ability information --- agent is not in a position to claim support for having specific ability from information and claimed support for general ability.
  However, initial statement and motivation apply to all instances of claiming support.
  After statement and motivation, show how the constraint relates to \AR{}.
  If so, agent lacks support for having specific ability, and does not have the option of claiming support for result of specific ability by \AR{}.
\end{note}

\subsubsection{Statement of \nI{}}
\label{sec:ni-1}

\begin{note}[\nI{}]
  We turn to the general constraint on claiming support.
  \begin{proposition}[\nI{-}  --- \nI{}]\label{prem:ni}
    Suppose an agent:
    \begin{enumerate}
    \item\label{nI:claimed-support} Has claimed support for \(\phi\) (recognised that may be misleading given information that agent has).
    \item\label{nI:received-info} Received novel information that if \(\phi\) then \(\psi\) is (also) the case/\(\psi\) has some value so long as \(\phi\) has so value.
    \item\label{nI:inclusion} If support for \(\phi\) does not also include (independent of value) resources to claim support for \(\psi\), then claimed support for \(\phi\) is mistaken or misled.
    \end{enumerate}
    Then:
    \begin{enumerate}[resume]
    \item\label{nI:going-by-value} No claim to (first time) support for \(\psi\) by appeal to value of \(\phi\) from \ref{nI:claimed-support}, and relation between \(\phi\) and \(\psi\) from \ref{nI:received-info}.
    \end{enumerate}
  \end{proposition}
\end{note}

\begin{note}[Structure of \nI{} and plan]
  Structure of \nI{} is two conditions that allow an agent to claim support in a certain way (\ref{nI:claimed-support} and~\ref{nI:received-info}), detailed by~\ref{nI:going-by-value}, but blocked when a defeater condition holds (~\ref{nI:inclusion}).

  Conditions to be clarified.
  Inclusion, and reasoning by value.
  Roughly:
  \incl{} reapply (some of the) premises and steps of reasoning used in claimed support for \(\phi\) to claim support for \(\psi\).
  E.g.\ simple: \(p \rightarrow q\) and \(q \rightarrow r\), to claim support for \(p \rightarrow r\).
  Reapply, \(p \rightarrow (q \land r)\).

  \RBV{} is a type of reasoning where agent appeals to value of proposition claimed support for.
  E.g.\ \(K_{E}p\) to \(p\).

  As \nI{} depends on these, little intuition prior to clarifying.
  The gist, however, is that \nI{} captures intuitive constraint that agent is not in a position to claim support for some proposition \(\psi\) by information that \(\phi\) entails \(\psi\) if failure to establish support for \(\psi\) independently of the value of \(\phi\) would reveal problem with the support claim for \(\phi\).

  Possible defeater for support for \(\phi\).
  However, possible defeater isn't necessarily a defeater.
  Key part to explaining \nI{} is why reasoning by value is bad, given that the agent doesn't have a clear problem with the claimed support for \(\phi\).

  Clarifying \incl{} and \RBV{} will allow a clearer statement.
  \nI{} will then follow from \incl{} and \RBV{} together with basic constraint on support \eiS{}.

  Important clarification, \nI{} is sufficient condition.
  Further, \RBV{} is \emph{a way} of claiming support.
  So, \nI{} does not imply that the agent is not in a position to claim support for \(\psi\), only that one way of claiming support is ruled out given \ref{nI:claimed-support}--\ref{nI:inclusion}.
  Following, as \nI{} is about claiming support, this says nothing about whether the agent has support --- in particular, if claimed support for \(\phi\) is support, then by \incl{}, plausible that the agent has support for \(\psi\) even if not in a position to claim when \RBV{}.

  Finally, no appeal to what the values of \(\phi\) and \(\psi\) (actually) are.
  \nI{} is `internal' in this sense.

  Looking forward, argument will be that \gsi{} sets up \incl{} and \AR{} requires \RBV{}.
  Hence, \nI{} rules out agent claiming support for specific ability by \AR{}.
  However, primary motivation will be independent of ability.

  Begin with clarifying \incl{} and \RBV{}, then general account of \nI{}, followed by a handful of examples, and concluding with ability.

  More in Chapter~\ref{sec:inertia}.
  Including, related literature. (No Feedback and Wright.)
\end{note}

\subsubsection{Inclusion of support}
\label{sec:inclusion-support}

\begin{note}[Inclusion of support]
  \begin{proposition}[Inclusion of support --- \incl{}]
    Support for \(\phi\) includes/requires support for \(\psi\).
    If, possible to reapply premises that establish \(\phi\) to establish \(\psi\) (without appealing to value of \(\phi\)).
  \end{proposition}
  Basic idea of \incl{} is reapplication.
  Hence, \incl{}\dots

  Without going by \(\phi\).
  This is not an additional condition, but clarification.
  For, given \eiS{}, agent did not need value of \(\phi\) in order to claim support for \(\phi\), and as agent is reapplying claimed support for \(\phi\), such reapplication does not require value of \(\phi\) either.

    \begin{figure}[H]
    \begin{subfigure}{.45\textwidth}
      \centering
      \begin{tikzpicture}[->, >=stealth', node distance=0cm, every text node part/.style={align=center}, scale=0.75]
        \node [] (1) at (-1,4) {};
        \node [] (2) at (-1,-1) {};
        \node [] (3) at (4,-1) {};
        \node [] (4) at (4,4) {};

        \node [] (a) at (1.5,3) {\(P/S\)};
        \node [] (b) at (1.5,0) {\(S(\phi)\)};
        \node [] (c) at (3,0) {};
        \node [] (d) at (3,3) {};

        \draw [->,-{Square[open]}, xshift=4] (a) to  node[right] {} (b);
        \draw [rounded corners=5pt, opacity=.1] (-.75,3.75) -- (-.75,-.75) -- (3.75,-.75) -- (3.75,3.75) -- cycle;
      \end{tikzpicture}
      \caption{Support for \(S(\phi)\) from premises/steps}
    \end{subfigure}
    \hfill
    \begin{subfigure}{.45\textwidth}
      \centering
      \begin{tikzpicture}[->, >=stealth', node distance=0cm, every text node part/.style={align=center}, scale=0.75]
        \node [] (1) at (-1,4) {};
        \node [] (2) at (-1,-1) {};
        \node [] (3) at (4,-1) {};
        \node [] (4) at (4,4) {};

        \node [opacity=.33] (a) at (0,3) {\(P/S\)};
        \node [opacity=.33] (b) at (0,0) {\(S(\phi)\)};
        \node [] (x) at (1.5,3) {\(\supseteq\)};
        \node [] (c) at (3,0) {\(S(\psi)\)};
        \node [] (d) at (3,3) {\(P/S\)};

        \draw [->,-{Square[open]}, opacity=.33] (a) to  node[right] {} (b);
        \draw [->,-{Square[open]},] (d) to  node[right] {} (c);
        \draw [rounded corners=5pt, opacity=.1] (-.75,3.75) -- (-.75,-.75) -- (3.75,-.75) -- (3.75,3.75) -- cycle;
      \end{tikzpicture}
      \caption{Reapply premises/steps to claim support for \(\psi\)}
    \end{subfigure}
    \caption{\incl{} diagram}
  \end{figure}

  Examples:
  \begin{itemize}
  \item \(p \rightarrow r\), includes support for \(p \rightarrow (q \land r)\) with premises \(p \rightarrow q\) and \(q \rightarrow r\).
  \item More complex example of entailment.
  \item Combining support for novel entailments.
  \item If understand language, then parse: some sentence.
  \item Optimal route to some place, then stop at some intermediate point at some time.
  \item Calculation of area, then calculation of diagonal. (A little trickier, as diagonal is background, and it's only the reuse of measurements.)
  \end{itemize}
  In these cases, some consequence.
  However, it's not simply any consequence.
  Rather, there is some resource that the agent will have used which may be refined.

  Clear failures for arbitrary entailment.
  Things used for proving completeness do not necessarily ensure finite model property, though in certain cases they may.

  And, finally, cases of ability.
  Ability is quite natural here, as when looking at the agent doing something, so perform some action.

  \begin{quote}
    If you've done X, then in a position to do Y.
  \end{quote}

  Doesn't hold for many cases.
  Two examples:
  \begin{itemize}
  \item Knowledge, support that \(S\) knows \(p\) does not include support for \(p\).
  \item Going by a supplied conditional.
    E.g.\ If not in London then in Paris.
    Searched London, but this doesn't already establish that the person is in Paris, the person could be anywhere other than London.
  \end{itemize}
\end{note}

\begin{note}[\incl{} and \nI{}]
  So, with respect to \nI{}, \incl{} highlights tight relation between support for \(\phi\) and support for \(\psi\).

  Because focus is on inclusion, issue of \(\psi\) is something of an (indirect, partial) test on the claimed support for \(\phi\).\nolinebreak
  \footnote{
    Partial, because it doesn't say too much about mistaken or misleading support that allows the agent to claim support for too much.
    However, pair with `not claim support for \(\psi\)' type conditions.
  }
  This is why issue of support for \(\psi\) is significant.
\end{note}

\subsubsection{Reasoning-by-value}
\label{sec:reasoning-value}

\begin{note}[\RBV{}]
  Second point is reasoning-by-value.

  \begin{proposition}[Reasoning by value (\RBV{})]
    An agent \emph{reasons by value} if the agent moves from claimed support for propositions \(\phi_{i}\) to \(\phi_{i}\) having value \(v_{i}\), which constrains value of \(\psi\).
  \end{proposition}

  \RBV{} is common.
  Agent has claimed support for \(\phi\) having value \(v\), then reason about what follows from \(\phi\) having value \(v\).
  Distinguishing feature is that it is \(\phi\) having value \(v\) which constrains value of \(\phi\).
  Purpose of going by value is that claimed support for \(\phi\) may not be sufficient to provide constraint of value of \(\psi\) without value of \(\phi\).

  First, entailments sometimes require value.
  Example with knowledge.
  It is true that an agent knows that \(p\) only if \(p\) is true.
  So, agent knowing \(p\) constrains value of \(p\).

  Support for agent knowing \(p\) without independent support for \(p\).
  For example, novice and an expert.
  Novice in position to claim support for expertise of expert, but not in a position to reason to \(p\) independently of expert.
  Appeal to claim of support.
  Factivity.
  Factivity requires that the expert knows, not merely that the agent has claimed support that the expert knows.
  Claimed support won't do this alone, for claimed support doesn't get \(p\) without \(K_{E}p\), nor does it need to be the case that \(K_{E}p\) in order to claim support for \(p\).\nolinebreak
  \footnote{
    Decline to link support to attitudes, but for clearer intuition, consider belief.
    \(B(K_{E}p)\) gets to \(B(p)\), but simply believing \(K_{E}p\) isn't enough to get \(K_{E}p\) and hence \(p\).
    So, \(K_{E}p\) given belief, hence \(p\), resulting in \(B(p)\).
  }
  Probe and find issues.

  Second, following entailment, information provided to agents is often about values.
  Information provides to an officer worker by company secretary that if their manager is not in the London office today, they are in the Paris office.
  Secretary does not provide information other than constraints on location.
  Employee searches London office, doesn't find manager.
  So, claim support that manager is not in the London office, and with the secretary's information, claim support that the manager is in the Paris office.

  As secretary doesn't provide information other than constraints, need not in London office to go to Paris office.

  Similarly, condition~\ref{nI:received-info} is about value, though abstract stated.

  \begin{figure}[H]
    \begin{subfigure}{.45\textwidth}
      \centering
      \begin{tikzpicture}[->, >=stealth', node distance=0cm, every text node part/.style={align=center}, scale=0.75]
        \node [] (1) at (-1,4) {};
        \node [] (2) at (-1,-1) {};
        \node [] (3) at (4,-1) {};
        \node [] (4) at (4,4) {};

        \node [] (a) at (0,3) {\(S(\phi)\)};
        \node [] (b) at (0,0) {\(\phi\)};
        \node [] (c) at (3,0) {\(\psi\)};
        \node [] (d) at (3,3) {};

        \draw [->, xshift=4] (b) to  node[right] {} (c);
        \draw [rounded corners=5pt, opacity=.1] (-.75,3.75) -- (-.75,-.75) -- (3.75,-.75) -- (3.75,3.75) -- cycle;
      \end{tikzpicture}
      \caption{Support for \(S(\phi)\) and info}
    \end{subfigure}
    \hfill
    \begin{subfigure}{.45\textwidth}
      \centering
      \begin{tikzpicture}[->, >=stealth', node distance=0cm, every text node part/.style={align=center}, scale=0.75]
        \node [] (1) at (-1,4) {};
        \node [] (2) at (-1,-1) {};
        \node [] (3) at (4,-1) {};
        \node [] (4) at (4,4) {};

        \node [] (a) at (0,3) {\(S(\phi)\)};
        \node [] (b) at (0,0) {\(\phi\)};
        \node [] (c) at (3,0) {\(\psi\)};
        \node [] (d) at (3,3) {};

        \draw [->] (b) to  node[right] {} (c);
        \draw [->, -{Square[open]}] (a) to  node[right] {} (b);
        \draw [rounded corners=5pt, opacity=.1] (-.75,3.75) -- (-.75,-.75) -- (3.75,-.75) -- (3.75,3.75) -- cycle;
      \end{tikzpicture}
      \caption{Move to value of \(\phi\)}
    \end{subfigure}

    \begin{subfigure}{.45\textwidth}
      \centering
      \begin{tikzpicture}[>=stealth', node distance=0cm, every text node part/.style={align=center}, scale=0.75]
        \node [] (1) at (-1,4) {};
        \node [] (2) at (-1,-1) {};
        \node [] (3) at (4,-1) {};
        \node [] (4) at (4,4) {};

        \node [] (a) at (0,3) {\(S(\phi)\)};
        \node [opacity=0.33] (b) at (0,0) {\(\phi\)};
        \node [] (c) at (3,0) {\(\psi\)};
        \node [] (d) at (3,3) {};

        \draw [->, opacity=0.33] (b) to  node[right] {} (c);
        \draw [->,-{Square[open]}, opacity=0.33] (a) to  node[right] {} (b);
        \draw [->,-{Square[open]}] (a) to  node[right] {} (c);
        \draw [rounded corners=5pt, opacity=.1] (-.75,3.75) -- (-.75,-.75) -- (3.75,-.75) -- (3.75,3.75) -- cycle;
      \end{tikzpicture}
      \caption{Given support, value of \(\psi\) is determined}
    \end{subfigure}
    \hfill
    \begin{subfigure}{.45\textwidth}
      \centering
      \begin{tikzpicture}[>=stealth', node distance=0cm, every text node part/.style={align=center}, scale=0.75]
        \node [] (1) at (-1,4) {};
        \node [] (2) at (-1,-1) {};
        \node [] (3) at (4,-1) {};
        \node [] (4) at (4,4) {};

        \node [opacity=0.66] (a) at (0,3) {\(S(\phi)\)};
        \node [opacity=0.33] (b) at (0,0) {\(\phi\)};
        \node [opacity=0.66] (c) at (3,0) {\(\psi\)};
        \node [] (d) at (3,3) {\(S(\psi)\)};

        \draw [->, opacity=0.33] (b) to  node[right] {} (c);
        \draw [->,-{Square[open]}, opacity=0.33] (a) to  node[right] {} (b);

        \draw [rounded corners=10pt] (-.5,3.5) -- (-.5,-.5) -- (3.5,-.5) -- (3.5,.5) -- (.5,.5) -- (.5,3.5) -- cycle;
        \draw [->,-{Square[open]}] (.64,.64) to node[right] {} (d);

        \draw [rounded corners=5pt, opacity=.1] (-.75,3.75) -- (-.75,-.75) -- (3.75,-.75) -- (3.75,3.75) -- cycle;
      \end{tikzpicture}
      \caption{Support for \(\psi\) from constraint on \(\psi\) following from support for \(\phi\)}
    \end{subfigure}
    \caption{\RBV{} diagram}
  \end{figure}
\end{note}


\begin{note}[Non-\RBV{} reasoning]
  Not all reasoning is like this.

  Return to examples used to illustrate inclusion of support.
  \begin{itemize}
  \item Conjunction elimination.
    Here, working with the logical form, it doesn't matter what \(p\), \(q\), nor \(r\) are.
  \item Same for efficient route example.
    Agent doesn't need to appeal to having found the most efficient route to produce variation.
  \end{itemize}

  Some instance of reasoning may go either way.
  For example.
  Machine.
  Told that the machine is designed to perform some function, but not told what the function is.

  Question, \(f(X,Y)\)

  One option is to input into machine.
  So, relying on machine implementing function.
  Going by value.

  Second option, abstract function from machine by inspection.
  Then, calculate \(f(X,Y)\).
  Not going by value, doesn't matter what the machine produced.

  Risks for each.
  For the first, potential malfunction on specific input, but without malfunction then that's the result.
  For the second, no concern about potential malfunction, but possible that abstracted function is faulty.

  So, difference is not tied to proposition.

  And, complex chain of reasoning may include both.
  \(p \rightarrow (q \land K_{S}r)\) so \(p \rightarrow q \land r\).
\end{note}

\subsubsection{Argument for \nI{}}
\label{sec:argument-ni}

\begin{note}[Review of \nI{}]
  \incl{} and \RBV{}, now turn to why conflict with \incl{} and \RBV{}, as stated by \nI{}.

  \ref{nI:going-by-value}, going by value.
  So, going from support for \(\phi\) to \(\phi\), and then from \ref{nI:received-info} to \(\psi\) by value.

  However, denied claim of support for \(\psi\) by \incl{}.
  Turn to argument for this.
\end{note}

\begin{note}[\nI{} argument, state]
  Start with three conditions describing state of agent.

  By~\ref{nI:claimed-support}, agent has claimed support for \(\phi\), though recognises the support may be fallible.
  It is possible that there's a different value of \(\phi\) (misled), or even if that value, the claimed support is not an indicator (mistaken).

  And, by~\ref{nI:received-info}, agent has information that value of \(\phi\) constrains value of \(\psi\).
  Use of `information' allows for arbitrarily strong support.
  May assume that this is something known, though do not require for failure.

  Finally, by~\ref{nI:inclusion}, the agent is aware that the claimed support for \(\phi\) includes support for \(\psi\).
  That is, agent may reapply premises and steps to claim support for \(\psi\).
  Hence, the agent does not need to go by value of \(\phi\) to get \(\psi\), as premises and steps used to claim support for \(\phi\) did not require value of \(\phi\) (by \eiS{}).
\end{note}


\begin{note}[\nI{} argument, \RBV{}]
  From the three conditions describing state, it is possible for agent to claim support for \(\psi\).
  In particular, by reapplying and establishing how claimed support for \(\phi\) includes support for \(\psi\).

  \ref{nI:going-by-value} describes a way of obtaining support that does not appeal to inclusion of support.
  Agent moves from claimed support for \(\phi\) to value of \(\phi\), and given information, this would constrain value for \(\psi\), hence leading to support for \(\psi\) if \RBV{} is permitted.

  With the background provided, from a certain perspective, agent would be bypassing reapplication.
  Possible to claim support for \(\psi\) regardless of value of \(\phi\), but with information, observe constraint on value of \(\phi\) and hence on \(\psi\).
  As agent doesn't need to do work to establish how value is constrained by reasoning by value, this is quite easy.

  Issue is that \incl{} blocks \RBV{}.
  \eiS{} again.
  If claimed support, then independent of value.
  However, because \incl{}, problem moving from support for \(\phi\) to value of \(\phi\).
  For, in moving support for \(\phi\) to value of \(\phi\), agent is implicitly requiring value of \(\psi\).
  For, given \incl{}, if \(\psi\) does not have value, then claimed support for \(\phi\) would be mistaken or misleading, and hence would block move to value.
  So, any further reasoning by value is required to already have \(\psi\), and this conflicts with \eiS{} when the agent moves to claiming support for \(\psi\) as the result of \RBV{}.

  From a broader perspective, the agent doesn't get to claim support for \(\psi\) through \RBV{} because any move to value requires \(\psi\) to already be the case given the information the agent has.
  Hence, failure of \eiS{} if agent were to claim support for \(\psi\).
  Because of \incl{}, \RBV{} does not preserve \eiS{}.

  Illustrate, possible that the agent is mistaken/misled, and claimed support for \(\phi\) does not include support for \(\psi\).
  However, if \RBV{}, then claim support for \(\psi\).
  Problematic, because whether inclusion is a test for whether the claimed support for \(\phi\) is any good, but bypassing test when reasoning by value.

  No claiming support by noting value consequence, if failure to show value independent consequence would lead to revision of support.
\end{note}

\subsubsection{Illustrations of \nI{}}
\label{sec:illustrations-ni}

\begin{note}
  Nothing particularly problematic about this, as given \incl{}, agent applies whatever support they have for \(\phi\) to get \(\psi\).

  Further, no issue with using \(\phi\) for other things.
  Here, only interest is in support.
  Hence, recognised by the agent that they may be misled.
  From this perspective, the issue is not ruling out potential defeaters.
  Similar to knowledge, etc.\ but no requirement that there are no defeaters.
\end{note}

\begin{note}[Abstract, so examples]
  The above highlights the problem.
  However, abstract.
  Turn to illustrations, and then to how \nI{} applies to \gsi{}.
\end{note}


\begin{note}
  May think that this restricts any application of \RBV{} to claimed support for \(\phi\) without value independent.
  This isn't quite right.
  \eiS{} keeps focus on \(\psi\).
  Only committed to \(\psi\) being a problem.
  Potential issue is no worse than any other instance of claim to support --- possibility of being mistaken or misled.
  If \(\psi\) ends up being used, then there's going to be a gap, where agent isn't in position to claim support by value, but unless eventual consequence is in turn used for \(\psi\), no clear problem --- at least not without stronger assumption.
\end{note}

\begin{note}
  \uRa{} is going to require the agent to reason from premises and steps `included' in claimed support for \(\phi\) in order to claim support for \(\psi\).
\end{note}


\begin{note}[Examples]
  Examples are somewhat difficult, due to complexities of state.
\end{note}

\begin{note}[Picture book]
  \begin{scenario}
    Picture book.
    Finding a particular character.
    Go through, and claim support.
    Then receive information that if found character, then wearing a striped shirt.
    So, visual support, relation of inclusion.
    Intuitively problematic, because, if not wearing striped shirt, then claimed support is bad.
  \end{scenario}
  Observe, here, that other things by value are fine.
  If found, then actually one of a number of solutions are the author made a mistake.
  Inclusion fails here, because may have moved on after single solution.

  Another visual
  \begin{scenario}
    If not genuine, then missing serial number.
  \end{scenario}
  No need to reinspect, faults are support, so no serial number.
\end{note}

\begin{note}[Logic proof]
  \begin{scenario}
    If conjunction and negation are truth functionally complete, then disjunction and negation are truth functionally complete.

    And, claim of inclusion.
  \end{scenario}

  Here, in proving completeness, expressing other connectives.
  So, inclusion because agent will have shown how to switch between conjunction and disjunction.

  Well, claimed support for conjunction and negation.
  So, yes.
\end{note}

\begin{note}[Treasure]
  \begin{scenario}
    Claimed treasure only if learnt secret.
  \end{scenario}
  A little more interesting, as here, agent is going to have done something to learn secret when claiming support for treasure, but may not recognise that they've learnt the information.
  Of course, may be wrong treasure.
  Again, seems bad.
  But, if treasure then sell for \$X, seems fine.

  Useful, as earlier examples may seem to rely on easy checks, but putting pieces together to reveal secret may be quite difficult.
\end{note}

\begin{note}[Problematic]
  \begin{scenario}
    If walked \(15k\), then walked \(13k\).
  \end{scenario}
  Doesn't seem so much of a problem, but at the same time, it seems \RBV{} traces \incl{}.
  Well, no, this is ruled out by some earlier condition.
  For, plausible that it's not really possible to have claimed support for \(15\) without already claiming support for \(13\).
\end{note}

\begin{note}[Knaves]
  \begin{scenario}
    If X is speaking falsely, then Y is speaking truthfully.

    Knave says a bunch of things that you've got could support for being false, but could be true.
  \end{scenario}
  Variation on Knave problems.
  Again, there may be intuition that solving the problem is easily in reach, but I think this is a mistake.
  Knave problems are hard, and the difficulty doesn't seems to make a difference.
\end{note}

\newpage

\subsubsection{\nI{} applies to \AR{}}
\label{sec:ni-applies-ar}

\begin{note}[Applying to type of scenario]
  Our attention now turns to how \nI{} applies to the use of \aben{} in scenarios of interest.

  The focus of our attention is whether an agent may claim support for having a specific ability given the claimed support for having a general ability, given \gsi{}.
\end{note}

\begin{note}[Checking conditions]
  Conditions \ref{nI:claimed-support} and~\ref{nI:received-info} are provided by the scenario.
  Condition~\ref{nI:inclusion} is obtained by reflection on ability.
  So, then, condition~\ref{nI:going-by-value} rules out a way of claiming support for specific ability.

  To argue for is that \AR{} implies \RBV{}.
\end{note}

\begin{note}
  If argument is successful, then agent will not be in a position to claim support for specific ability.
  This is the antecedent of the relevant use of \aben{}.
  Pair \nI{} with following supplement.

  \begin{proposition}[\nIm{}]
    An agent must have claimed support for the antecedent of an entailment in order to claim support for the consequent of the entailment via the entailment.\nolinebreak
    \footnote{To clarify, entailment is only about value.
      Think of conditional.

      So, does not follow that there being an entailment is a required part of agent's reasoning.
      \nIm{} is talking about when the agent appeals to an entailment, rather than any understanding of entailment beyond it being the case.
    }
  \end{proposition}
  \nIm{} seems indisputable,\nolinebreak
  \footnote{
    An agent may have some other way of claiming support for the consequent of the entailment.
    However, if the agent is not in a position to claim support for the antecedent, then the agent is not in a position to claim support because there is an entailment from the antecedent to the consequent.\nolinebreak

    For example, that the coin landed heads is entailed by Sam knowing that the coin landed heads.
    Here, entailment from \(K\phi\) to \(\phi\).

    Second, this light being on entails that the printer is out of paper.
    If agent appeals to entailment, again, need the light to be on.
    However, could look in the paper drawer, or modify the wiring so that an alarm sounds.

    However, Taylor is not in a position to claim support for the coin landed heads because Sam knows if Taylor has no idea whether Sam knows --- though Taylor may claim support by looking at the coin.
  }
  and so not in a position to claim support for result of witnessing ability via \AR{}.
\end{note}

\begin{note}
  Note on two ways of reasoning.
  Getting support for specific ability by \RBV{} and also getting result by \RBV{}.
  \nI{} only explicitly rules out first.
  However, with \nIm{}, second is implicitly ruled out, as if agent claims for value, then need to be able to claim support for premise.
  And, definition of \AR{} is such that going for premise is attribute.
  This is not necessarily required --- e.g.\ \WR{}.
\end{note}

\newpage

\begin{note}[Notes for \AR{}]
    {
    \color{green}
    Well, with the first, it's getting to \(\phi\).
    Doesn't seem the agent is in a position to use factive inference.
    Because, the agent is going from not possible to have ability and for \(\phi\) to be false.

    With the second, different.
    Because, the agent is going from application of ability providing support for \(\phi\).
  }

  Use of \AR{} gets quick argument for \RBV{}.
\end{note}

\begin{note}[Application of \nI{} to \AR{}]
  \AR{}, working with attribute.
  So long as you have general ability, you have specific ability.
\end{note}

\newpage

\begin{note}[Application of \nI{} to \AR{} argument]
  \gsi{} information.
  For \AR{}, agent is required to claim support that they have the specific ability.
  That is, claim support for consequent because the agent claims support for specific ability.
  This is distinguishing feature of \AR{} --- the agent is only appealing to having attribute.
  General characterisation of \AR{}, all ability from appeal to attribute.
  Contrast to \WR{}, where some use of ability.

  So, this means attribute for general and specific in cases of interest.
  For, \AR{} is attribute for all instances of ability.




    If agent appeals to general and information, then agent is appealing to having general attribute, and not only support for general attribute.
\end{note}

\newpage

\begin{note}[Distinguishing features]
  Reasoning has distinguishing features that pair well.
  \begin{itemize}
  \item Extends support.
  \end{itemize}
\end{note}


\begin{note}[Important points]
  Two important points:

  The role of~\nI{} is to highlight that the agent is not in a position to obtain support for (specific) ability in a certain way.
  That is,~\nI{} does not state that the agent may not obtain support for (specific) ability some other way.

  Second, so long as agent holds that they have general ability, then committed to truth.

  May be tempted to say that the agent is not committed, but this seems implausible.
  Cases of transmission failure, it seems agent does remain committed, at least.

  May take issue with information provided, especially if ideal.
  If informer has information, then they should say.
  In turn, not problem with~\nI{} as the agent would have support (via testimony) for specific ability.
  However, informer may only have the conditional.

  Ordinary agents.
  Maxims are broken.
  And, interest effects.
  Up to the agent.

  Seems puzzling, but not paradoxical.
\end{note}

\begin{note}[Why this is important]
  The key idea, and the foundation of the objection, is that the agent is going indirectly.
  The agent fails to show how the general ability extends to specific ability.
  For, the only things available to the agent is the constraint.

  This is useful independently.
  For, even if not convinced by~\nI{}, clear that given \gsi{} and~\uRa{}, the agent goes directly.
  And, something a little puzzling about this.
  Or, so I think.
\end{note}

\begin{note}[Dogmatism]
  Continuing relation to issues with knowledge.
  \autoref{prem:ni} is quite close to dogmatism paradox.
  If one knows that \(\phi\), then any evidence for \(\lnot \phi\) is misleading.

  Distinct again, however.
  For, don't have knowledge in the antecedent.
  Get the dogmatism paradox from the factivity of knowledge.
  No requirement that support for (general) ability is factive.

  Hence, role of the informer is important again, because agent is not in a position to come to the conditional by themselves prior to reasoning.
\end{note}

\newpage

\begin{note}[Finding tension, still]
  We have outlined a type of scenario built primarily on an agent receiving information that the agent has some specific ability so long as the agent has some general ability.
  The agent has support for having the general ability, but there are two ways in which the agent's support for having the general ability may be used to establish support for {\color{red} the result of having the specific ability} --- \AR{} and \WR{}.

  The previous section argued that~\uRa{} constrains how an agent may use the received information.
  If an agent is required to traces support from premises to conclusion through reasoning, then an agent may not appeal to the support for the premises and steps of reasoning that the agent would use when witnessing the specific ability.
  {\color{red} This is summarised in~\ref{P:ab-and-dc:W}.}

  The (initial) plausibility of~\uRa{}, then, suggests that the agent may only establish support for having the {\color{red} result of the specific ability} from the support they have for the general ability by \AR{}:
  The support the agent has for the general ability is support that it is true that the agent has the general ability.
  In turn, given the information received it is true that the agent has the specific ability, and it is only possible for the agent to have the specific ability if the result of witnessing the specific ability is true.

  The argument of this section is that the sketch of \AR{} given conflicts with a different, but equally plausible, premise.
  The premise concerns the way in which the agent obtains support for having the specific ability from the support for the general ability.
  We state conditional, the proceed to the premise.
  The initial statement of the premise is abstract and after providing a handful of clarifications we then link the premise to the type of scenario of interest.
\end{note}

\subsubsection{Summary}
\label{sec:ni-summary}

\begin{note}[Conditional B]
  \begin{proposition}[\mcB{}]
    \begin{enumerate}[label=(C\Alph*), ref=(C\Alph*)]
      \setcounter{enumi}{1}
    \item\label{P:ab-and-dc:A} If
      \begin{enumerate}[label=(\roman*), ref=(CB.\roman*)]
      \item\label{P:ab-and-dc:A:ab} an agent obtains support for some proposition \(\phi\) on the basis of the agent's ability to demonstrate that \(\phi\) is the case, and
      \item\label{P:ab-and-dc:A:ni} \nI{} is true,
      \end{enumerate}
      then
      \begin{enumerate}[label=(\roman*), ref=(CB.\roman*), resume]
      \item\label{P:ab-and-dc:A:AR} the support for \(\phi\) \emph{is not} claimed on the basis of the agent having the attribute of being able to demonstrate that \(\phi\) (in line with \AR{}).
      \end{enumerate}
    \end{enumerate}
  \end{proposition}
\end{note}

\begin{note}[Relating to other conditional]
  Given \mcA{}, the immediate interest with \mcB{} is that~\ref{P:ab-and-dc:W:AR} and~\ref{P:ab-and-dc:W:AR} are incompatible.
  So, if both conditionals are true, then (at least) one of the antecedents from either conditional is false.

  \ref{P:ab-and-dc:A:ab} repeats~\ref{P:ab-and-dc:W:ab}, roughly \eA{}.\nolinebreak
  \footnote{
    As with \mcA{}, could rephrase without \ref{P:ab-and-dc:A:ab}.
    If \nI{} is true, then not possible for agent to claim support for \(\phi\) on the basis of the agent having the attribute of being able to demonstrate that \(\phi\).
  }
  % Still,~\ref{P:ab-and-dc:A:ni} differs from~\ref{P:ab-and-dc:W:uRa}.
  \mcA{} assumed \uRa{}, principle we are proposing an exception for.
  \mcB{} assumes a principle, \nI{} to be argued for.
  Hence, if both \mcA{} and \mcB{} are true, either \eA{}, \uRa{}, or \nI{} is false.

  Before turning to \nI{}, observe general picture.
  If \eA{} and \nI{}, then \uRa{} is false.
  Hence, argument against \uRa{}.
  Further, \ref{either-AR-or-WR} then \WR{}.
  Motivates \rC{}.
\end{note}



\begin{note}[Established conditional 2]
  Something about, if \eA{} then agent does not obtain support for the attribute.
\end{note}

\subsection{Establishing tension/summary}
\label{sec:establishing-tension}

\begin{note}[Summary]
  Given the two established conditionals~\ref{P:ab-and-dc:W} and~\ref{P:ab-and-dc:A}, the combination of the key premises of \uRa{},~\eA{}, and~\nI{} are in tension.

  For, combining~\ref{P:ab-and-dc:W} and~\ref{P:ab-and-dc:A} we have:
  \begin{enumerate}[label=(CC), ref=(CC)]
  \item If \eA{} is the case an agent obtains support for some proposition \(\phi\) on the basis of the agent's ability to demonstrate that \(\phi\) is the case then:
    \begin{enumerate}[label=(C\arabic*\(\sim\)), ]
    \item If \uRa{} is true, then the support for \(\phi\) is obtained on the basis of the agent having the attribute of being able to demonstrate that \(\phi\) (in line with \AR{}).
    \item If \nI{} is true, then the support for \(\phi\) \emph{may not be} obtained (in line with \AR{}) on the basis of the agent having the attribute of being able to demonstrate that \(\phi\).
    \end{enumerate}
  \end{enumerate}
  In short, if~\eA{} is the case then~\uRa{} requires a certain interpretation of the scenarios identified by~\eA{} and~\nI{} denies that the interpretation is plausible.
\end{note}


\begin{note}[Creating tension]
  Have:
  \begin{enumerate}
  \item \(\eA{}\)
  \item \(\uRa{} \rightarrow \lnot\WR{}\)
  \item \(\nI{} \rightarrow \lnot\AR{}\)
  \end{enumerate}
  Argued for these.

  Then,
  \begin{enumerate}
  \item \(\eA{} \rightarrow (\AR{} \lor \WR{})\)
  \end{enumerate}
  If agent claims support, either \AR{} or \WR{}.

  Then use Propositions.
  \begin{enumerate}
  \item \(\eA{} \rightarrow ((\AR{} \land \lnot\nI{}) \lor (\WR{} \land \lnot\uRa{}))\)
  \end{enumerate}
  Simplify by exclusive proposition again.
    \begin{enumerate}
  \item \(\eA{} \rightarrow ((\lnot\nI{}) \lor (\lnot\uRa{}))\)
  \end{enumerate}

  So,
  \begin{enumerate}
  \item \(\lnot\eA{} \lor \lnot\uRa{} \lor \lnot\nI{}\)
  \end{enumerate}
\end{note}

\begin{note}[Weak points]
  \eA{} and \(\eA{} \rightarrow (\AR{} \lor \WR{})\).

  Perhaps there is a third option.
  Though unclear what this would be, even in outline.

  Hence, that the agent has the option of claiming support.
  If so, though, it seems surprising that \aben{} is never used to claim support.
\end{note}

\begin{note}[Tension, choices]
  In short, we have the following resolutions.
  \begin{enumerate}
  \item\label{ten:res:nS} Agent may not obtain support for result of witnessing ability, or
  \item\label{ten:res:nD} Agent obtains support for result on the basis on premises that the agent would use when witnessing ability --- incompatible with general application of~\uRa{}
  \item\label{ten:res:nI} Agent obtains support for result from attribute of having the ability on the basis that the support they have for general ability would be misleading --- incompatible with general application of~\nI{}
  \end{enumerate}
  \ref{ten:res:nS} is incompatible with~\ref{ten:res:nD} and~\ref{ten:res:nI}.
  However, \ref{ten:res:nI} and~\ref{ten:res:nD} are compatible, as both~\uRa{} and~\nI{} may be restricted.
\end{note}

\begin{note}[Argument sketch recap]
  Let us recap the main points of the argument so far.
  \begin{enumerate}
  \item Assume possibility of cases in which agent is provided with information that they have some specific ability so long as the agent has a general ability, such that the agent has support for having the general ability, but has not established support for possessing the specific ability.
  \item In such cases, it seems it is possible for the agent to obtain support for what follows from the agent witnessing their specific ability.
  \item If so, the agent appeals to having the specific ability in order to obtain support for what follows from the agent witnessing their specific ability.
  \item Attribution, and witnessing.
  \item If witnessing, then conflict with the requirement that an agent must access support for the premises appealed to in support of a conclusion.
  \item If attribution, then conflict with the restriction that an agent may not obtain support for some proposition on the basis that support the agent has for some other proposition would be misleading otherwise.
  \end{enumerate}

  To follow:
  \begin{enumerate}
  \item Restricting~\uRa{} in favour of~\rC{} works well.
  \end{enumerate}
\end{note}

\begin{note}[Meek outlook]
  This is not a clean argument.
  Take~\uRa{} and~\nI{} and hold the first.
  The agent may not obtain support.

  While there may be tension if the agent obtains support, this tension is never instantiated.

  I am sympathetic.

  Still, endorsing the restriction does not require the agent to obtain support in this case.
  Harbour some hope that that there is scope to restrict \uRa{}, and that the argument provided for resolving tension in favour of \rC{}, along with later arguments, may serve as a source for reflection.
\end{note}

\begin{note}[\WR{} isn't required for interest]
  \mcB{} is quite interesting itself.
\end{note}

\section{Positive argument overview}
\label{sec:posit-argumn-overv}

\subsection{Cases}
\label{sec:cases}

\begin{note}
  Main role of positive argument is cases.
\end{note}

\begin{note}[Beyond belief]
  Application in particular to desire.
\end{note}