%%% Local Variables:
%%% TeX-master: "master"
%%% End:

\chapter{Overview}
\label{cha:overview}

{
  \begin{note}[To add]
    Seems there's some intuitive tension that may be found in the cases.
    For example, agent gets the information, and then finds some note that says such a strategy does not exist.
    If agent does not have support, and the note is somewhat plausible, then it seems things should be easily resolved.
    However, this doesn't seem quite right.
    So, it seems there is something to be said in terms of the agent having support.

    This sort of observation is \emph{quite} important for motivating things\dots
  \end{note}
}

\section{Ability and access to support}
\label{sec:abil-access-supp}

\begin{note}[Introducing main topic]
  Our interest is in when it is permissible for an agent to obtain support for some conclusion of some instance of reasoning on the basis of the support the agent has for the premises of the instances of reasoning.

  We take the following claim as basic:
  \begin{enumerate}[label=\bP{}, ref=\bP{}]
  \item\label{prem:bP} If agent accesses support they have for premises and traces implication through valid and subjectively sound reasoning, then agent obtains support for conclusion on basis of support for premises.
  \end{enumerate}
  For example, suppose an agent has support that one of the three boxes in front of them is red.
  And, the agent has support that the box in front of them has the dimensions of 19 cm x 28 cm x 7cm.
  Some reasoning, the volume of the box is roughly 3,724 cm\(^{3}\).
  Whether some (or all) of the required arithmetic is to be included as a premise may be set aside.
  The support the agent has for holding that the volume of the box is roughly 3,724 cm\(^{3}\) is obtained (at least in part) on the support the agent has for the box having the dimensions noted.
\end{note}

\begin{note}[Support]
  Intuition is, roughly, that agent has support for a proposition if agent has the option of using the proposition in some further reasoning.

  This is not to say that a proposition without support has no use in reasoning.

  Use support as opposed to justification.
  Initial focus is on epistemic/doxastic attitudes.
  However, practical reasoning.
  For example, means-end.
  Support considered quite general to also include this.
\end{note}

\begin{note}[Valid and subjectively sound]
  Valid and subjectively sound.
  Valid, in the sense that the agent may obtain support for the conclusion.
  In a deductive case, if the premises are true, then the conclusion is true.
  Subjectively sound in the sense that for each premise or step of reasoning used, the agent has support for the premise or step and does not have superior support for a contrary premise or step.
\end{note}

\begin{note}[Focus]
Our interest is with the converse of~\ref{prem:bP}.

\begin{enumerate}[label=\mp{}, ref=\mp{}]
\item\label{denied-claim} An agent obtains support for conclusion on basis of support the agent has for premises only if the agent accesses support they have for premises and traces implication through valid and subjectively sound reasoning.
\end{enumerate}
If the agent did not measure the box, nor perform the arithmetic, the agent would not obtain support.
A luck guess that the box is roughly 3,724 cm\(^{3}\) would not allow the agent to hold that the volume of the box is roughly 3,724 cm\(^{3}\) on the basis of the dimensions of the box.
\end{note}

\begin{note}[Alternative]
  Our interest with \ref{denied-claim} is that it is a universal claim --- \ref{denied-claim} applies to all instances in which an agent obtains support for conclusion on basis of support for premises.

  Our goal is to motivate an exception to \ref{denied-claim}.

  \begin{enumerate}[label=\rC{}, ref=\rC{}]
  \item\label{rC} If an agent has information that they have the ability to (soundly) reason to some conclusion, then the agent may obtain support for the conclusion on basis of support for premises that the agent would access by witnessing their ability to reason to the conclusion.
  \end{enumerate}

  \ref{rC} carves an exception to~\ref{denied-claim}.
  For, if~\ref{rC} is true, then there are cases in which an agent is not required to reason from premises to some conclusion in order to obtain support for the conclusion on the basis the support the agent has for the premises.
\end{note}

\begin{note}[Alternative is not \emph{too} strong]
  Quite note to observe that the `has' in~\ref{denied-claim} only requires `at some point in the past'.
  Consideration here is Goldman, for example.

  Arguments against~\ref{denied-claim} will also hold for anything stronger.
  In particular, where the `has' is somewhat immediate.

  Stronger may be more plausible to some, but exception is forward looking, and~\ref{denied-claim} is appropriate.
\end{note}

\begin{note}[Examples]
  Need quick truth and desire examples.
  I don't think I need to make these examples compelling, as giving borderline examples may help motivate interest.
\end{note}

\begin{note}[Motivating idea, designated value]
  Thinking about attitudes.
  Some kind of designated value on the attitude.
  Doxastic, truth, maybe.
  Why should it matter that the reasoning has been performed prior to forming the attitude?

  Arguing in this way requires some account.
  Perhaps, in hindsight, this would have been preferable.
  Here, a little different.
  Cases in which, whatever it turns out to be, agent is permitted to form such an attitude, constrain with additional condition which seems harder to give up.
\end{note}

\begin{note}[Motivating idea, normative]
  Seems as though there's a plausible normative dimension, in which an agent may be criticised on what they are able to do.
  In particular, ability to reason further, but did not do so.
  Difficult to understand this as a requirement to reason, given constraints on resources.
  Doesn't seem to excuse, or so one may think.

  That is, this is something in between
  \begin{enumerate}
  \item The agent not having the resources.
  \item The agent having the resources, but being unaware that they may use them.
  \item The agent having the resources, and being aware that they may use them.
  \item The agent using the resources.
  \end{enumerate}
\end{note}

\begin{note}[\emph{Exception}]
  As an exception, this doesn't mean that the (appeal to) support is the same as would be if the agent had done the reasoning.
  One way to frame is that doxastic or no claim to support.
  See~\ref{denied-claim} as what it takes to obtain doxastic support.
  Deny this.
  There is some other kind of relation to support.
\end{note}

{
  \begin{note}[No causality]
    Important to note, as seems easy to confuse.
    Not claiming that the agent's attitude is causally related.
    Cause and `basing' may come apart.
  \end{note}

  \begin{note}[Not clear infinitsm is an exception]
    Infinitism is an exception to~\ref{denied-claim}.
    Indeed, sketch follows some infinist ideas, to a degree.
    Inifinist may require that the agent has done enough reasoning.
    (E.g.\ \textcite[10]{Klein:2007ve})
  \end{note}
}

\begin{note}[Structure of argument]
  Two lines of argument for endorsing~\ref{rC}, and hence denying~\ref{denied-claim}.
  \begin{enumerate}[label=(L\arabic*), ref=(L\arabic*)]
  \item\label{arg:line:1} Motivate~\ref{rC} as resolution to tension resulting from~\ref{denied-claim}.\newline
    Specifically:
    \begin{enumerate}[label=(L1\alph*)]
    \item\label{arg:line:1:a} Provide recipe for generating scenarios where~\ref{denied-claim} is in tension with particular scenarios involving information that an agent has the ability to perform some reasoning and a further claim regarding support.
    \item\label{arg:line:1:b} Motivate~\ref{rC} as a resolution to the tension.
    \end{enumerate}
  \item\label{arg:line:2} Argue that granting~\ref{rC} as an exception to~\ref{denied-claim} allows for an intuitive understanding of cases in which agent has the option of appealing to ability, even if there are alternative ways of interpreting the scenario in line with~\ref{denied-claim}.
  \end{enumerate}
  These two lines of argument work together.
  The tension of~\ref{arg:line:1} generates interest in witnessing that may be flatly rejected by prior endorsement of~\ref{denied-claim}.
  The intuitive understanding of scenarios involving ability of~\ref{arg:line:2} suggests there's more to witnessing than resolving the tension in narrow cases.
\end{note}

\begin{note}[Details of \ref{arg:line:1}]
  The initial focus is on the first line of argument,~\ref{arg:line:1}.
  The tension developed in part~\ref{arg:line:1:a} is delicate, but hopefully informative.
  We will establish a number of corollaries regarding ability and the interaction between~\ref{denied-claim} and ability.
\end{note}

\begin{note}[Before turning to the argument\dots]
  Before turning to the argument, we conclude this introduction with a handful of notes regarding~\ref{denied-claim} and~\ref{rC}.
\end{note}

\begin{note}[Interest in ability]
  Idealised agents have no need to appeal to ability.
  However, for limited agents, ability is abundant, while the resources required to witness abilities are scarce.
  That the exception to~\ref{denied-claim} is narrow does not entail that there are few occurrences of the exception.

  Information about ability may be abundant while the resources for witnessing abilities are either scarce or temporarily unavailable.
  So, agent is able to conserve or defer use of resources.

  Broadening scope.
  Arguments involving~\ref{denied-claim}.
  Distinction between ideal and non-ideal.
  Potential alternative conclusions to arguments that appeal to~\ref{denied-claim} as a premise.
  Revise premises for arguments in which~\ref{denied-claim} is a conclusion.
\end{note}

\begin{note}[Scope of \mp{}]
  \mp{} does not say anything in particular about what the agent has support for, only what must be the case in order for an agent to appeal to support for some conclusion on the basis of support for premises.

  Talking in terms of (support for) premises and conclusions restricts attention to reasoning.
  There may be broader use of `premise' and `conclusion' where an agent is not required to reason from premise to conclusion in order for the premise to support the conclusion.
  For example, if visual perception is immediate.
  Perhaps it may be said that an agent's visual experience is a premise to the conclusion that a dog is sleeping.
  Still, for present purposes, `conclusion' refers to the output of some process of reasoning performed by an agent which is either actual or potential, and `premises' to the input of that process.

  Note, also, that in both cases the relation between premises and conclusion is important.
  If agent does not reason, then neither~\ref{prem:bP} nor~\ref{denied-claim} apply.
  If there are multiple ways to obtain a conclusion, then~\ref{denied-claim} does not require the agent to reason from a particular set of premises.

  Likewise,~\ref{denied-claim} does not require that an agent is required to obtain support for a proposition by valid and subjectively sound reasoning from some premises.

  Rather,~\ref{denied-claim} requires that an agent reason from premises to conclusion in order to establishes support between premises and conclusion
  By contrast,~\ref{prem:bP} holds that reasoning is sufficient to establish such a relation.
\end{note}

\begin{note}[\mp{} is intuitive]
  \ref{denied-claim} is intuitive, and is quite common, though not without exceptions.
(For example, there's views on testimony in which the testifier provides agent access to support the testifier has.
One may understand this as conflicting with~\ref{denied-claim}, or that the fact that these are accessible is the relevant piece of support.)
\end{note}

\begin{note}[Alternative]
  \ref{rC} restricts~\ref{denied-claim}.
  This is not to say the agent obtains support equivalent to that which would be obtained were the agent to do, or have done, the reasoning.
  Nor, that the agent is aware of the relevant premises.

  Intuitively, \rC{} states that the agent may appeal to the reasoning they are able to perform in support for the conclusion of that reasoning, and as that reasoning moves from premises to conclusion, it is on the basis of the support for those premises that the agent would identify by reasoning that the agent obtains (some) support for the conclusion.

  Hence, \rC{} is in line with the spirit of~\ref{prem:bP}.
  For the exception to~\ref{denied-claim} is granted by the agent appealing to a witnessing event in which the antecedent (and consequent) of~\ref{prem:bP} are satisfied.
\end{note}

\begin{note}[Doxastic (as opposed to propositional) support]
  This is `doxastic' as opposed to `propositional'.
  Agent need not reason in order to have propositional support.
  For example, \(A < B\) and \(B < C\), then propositional support for \(A < C\) even if I don't bother to reason.
  No further assumptions about the relation between propositional and doxastic support.
\end{note}

\begin{note}[Ability ensures propositional?]
  Plausible that if the agent has the ability, then the agent already has propositional support for the relevant proposition.
\end{note}

\section{Broad argument overview}
\label{sec:broad-argum-overv}

\begin{note}[Overview]
  Tension resulting from the unrestricted scope of~\ref{denied-claim}.
  We begin by introducing a particular type of scenario involving ability, and observe how~\ref{denied-claim} requires a unique interpretation of the scenario.
  We then introduce an additional principle regarding support, which conflicts with the interpretation of the type of scenario introduction required by~\ref{denied-claim}.
\end{note}

\section{Type of scenario}
\label{sec:type-scenario}

\begin{note}[Tension, information]
  The tension arises when an agent is provided with a piece of limited information that:
  \begin{enumerate}[label=(I\arabic*), ref=(I\arabic*)]
  \item So long as the agent has a general ability, then the agent has a specific ability.
  \end{enumerate}
  The information is limited because it does not directly provide the agent with the information that the agent has the specific ability, nor that the result of witnessing the specific ability is the case.

  For example,
  \begin{enumerate}[label=(I\arabic*), ref=(I\arabic*), resume]
  \item\label{qe:cond} So long as you have the ability to reason with the rules of chess, you have the ability to demonstrating that there is a sequences of moves that will ensure a win for one of the players (as an instance of the general ability).
  \end{enumerate}
  The conditional structure distinguishes this information from:
  \begin{enumerate}[label=(I\arabic*\('\), ref=(I\arabic*\('\)), resume]
    \setcounter{enumi}{1}
  \item\label{qe:cons} You have the ability to demonstrate that there is a sequences of moves that will ensure a win for one of the players.
  \end{enumerate}
  For the agent is required to obtain~\ref{qe:cons} from~\ref{qe:cond} by endorsing the antecedent --- that they have the (general) ability to reason with the rules of chess.
  In turn, the agent is not provided with information that:
  \begin{enumerate}[label=(I\arabic*), ref=(I\arabic*), resume]
  \item\label{qe:result} There is a sequences of moves that will ensure a win for one of the players.
  \end{enumerate}
  For, it need not be the case that~\ref{qe:result} is true if~\ref{qe:cond} is true by virtue of a false antecedent.
  Of course, the antecedent of~\ref{qe:cond} need not be false.
  Still, the limited information requires the agent to appeal to their general ability in order to obtain information about how the agent's general ability extends to a particular case.

  However, if the agent hold that they have the ability to demonstrate that there is a sequences of moves that will ensure a win for one of the players, then the agent may reason to~\ref{qe:result}.

  \begin{enumerate}
  \item[\textsf{A}]\label{A:s} On the one had, a strategy must exist in order for the agent to possess the ability to demonstrate that a strategy exists.
  \item[\textsf{W}]\label{W:s} On the other hand, if the agent has the ability to demonstrate that a strategy exists, then it is possible for the agent to witness an event in which they demonstrate that a strategy exists.
  \end{enumerate}
  We term these \AR{} and \WR{}, respectively.

  We will return to this kind of limited information in greater detail below.
  For now, the basic idea is that the agent is on the hook, so to speak, for holding that they have the specific ability.
  Perhaps the informer does not want the agent to rely on the informer's information for the existence of the strategy.
  Or, perhaps the agent only wants to appeal to their own understanding of chess.

  We don't need a fleshed out scenario, but if it helps,~\ref{qe:cond} may be read as a slight challenge.
  The relevant interpretation of `if you're smart enough, you can solve this problem' seems clear.
  `If your ability to reason is of sufficient worth, then by extension of that ability, you have the ability to solve this problem.'
  Paraphrased, `if you're smart enough, you have the ability to solve this problem'.
  So challenged, and confident in one's smarts, one may expect to solve the problem.
  The slight difference with the limited information of interest is that the informer provides information about what the solution to the problem is if the agent is `smart enough'.
\end{note}

\begin{note}[Scenario premise]
  For ease of reference, we wrap scenarios involving the limited information as a premise.
  \begin{enumerate}[label=\eA{}, ref=\eA{}]
  \item\label{prem:ab} It is possible for an agent to use information that they have some specific ability so long as the agent has some general ability to obtain support for what follows from the specific ability.
    (Where the agent lacks doxastic support for what follows, and for \(A(\psi)\) without information).
  \end{enumerate}
\end{note}

\begin{note}[Possible restrictions]
  The important aspect of premise~\ref{prem:ab} is that there are cases in which the agent may appeal to ability to obtain support.
  This is quite weak.

  Understanding of support here is primarily for the agent.

  It allows that there may be cases in which the details of the cases outlined are satisfied, but where kind of support is unsuitable for certain purposes.

  In particular, some witness of ability may be demanded by a third-party.
  Perhaps due to lack of confidence in agent, or contextual features of the scenario.
  This is no different from memory.
  Memory of proving \(\phi\) provides support for \(\phi\).
  Still, one may still demand a demonstration of \(\phi\).
  Perhaps the third-party considers the agent's memory unreliable, or perhaps context has been set so that memory is insufficient to add a proposition to the common ground, etc.
\end{note}

\section{First conditional}
\label{sec:first-conditional}

\begin{note}[Attribute]
  There is not yet tension between~\ref{denied-claim} and~\ref{prem:ab}.
  For, we noted that the agent may appeal to their ability to reason in either of two ways:
  \begin{enumerate}[label=\(\cdot\)]
  \item \AR{}: \(\phi\) must be the case in order to have the attribute of being able to reason to \(\phi\).
  \item \WR{}: There is a (potential) witnessing event in which \(\phi\) is demonstrated, and therefore \(\phi\) is the case.
  \end{enumerate}
  WR{} is an instance of~\ref{rC}, as the agent obtains support for the conclusion of the reasoning is able to do on the basis of the reasoning that would be performed in a witnessing event.
  Hence, the supported obtained for the conclusion is obtained on the basis of the support the agent has for the premises that would be used.
  Again, this does not imply that the agent obtains support for the conclusion which is equivalent to the support the agent would obtain by witnessing their ability by performing the reasoning.

  However, \AR{} suggests an alternative way to obtain support for the conclusion of reasoning the agent is able to do.
  Specifically, if order for the agent to \emph{have} the attribute of being able to reason to the conclusion, the conclusion of the reasoning must be true.
  The relevant entailment is in part secured by the verb chosen, and in part by what the verb is applied to.
  Here, `demonstrate' is a factive verb, if an agent demonstrates that \(\phi\), then it is true that \(\phi\).
  And, the existence of a chess strategy does not depend on the agent demonstrating that the relevant strategy exists.

  To take another example, you only have the ability to identify a typo on this page if there is a typo on this page.
  So, if I were to provide you with testimony that you have the ability to identify a typo on this page, you may begin searching for the typo, or you may note that there must be a typo in order for me to be in a position to provide you with testimony that you have the ability.

  The reasoning is summarised with the following sketch.

  \begin{enumerate}[label=(\textsf{A}\arabic*), ref=(\textsf{A}\arabic*)]
  \item\label{WR:Sketch:1} I have the attribute of being able to \emph{V} that \(\phi\).
  \item\label{WR:Sketch:2} In order to have the attribute of being able to \emph{V} that \(\phi\), \(\phi\) must be the case independent of whether or not I witness the ability.
  \item\label{WR:Sketch:3} \(\phi\) is the case.
  \end{enumerate}

  To keep things simple, we will refer to the principle behind the pattern sketched as \AR{}.
  And agent may bundle~\ref{WR:Sketch:1} and~\ref{WR:Sketch:3} into a conditional, and avoid instantiating the reasoning pattern, but so long as the conditional is (implicitly) held on the basis of the intermediate premise~\ref{WR:Sketch:2}, we take use of such a conditional to be an instance of \AR{}.

  \AR{} is compatible with \ref{denied-claim}.
  For, the two premises~\ref{WR:Sketch:1} and~\ref{WR:Sketch:2} are accessible to the agent, and obtaining \ref{WR:Sketch:3} from~\ref{WR:Sketch:1} and~\ref{WR:Sketch:2} appears to be straightforwardly sound reasoning.
\end{note}

\begin{note}[Conditional \ref{P:ab-and-dc:W}]
  We wrap the above observations in a conditional.
  \begin{quote}
    \begin{enumerate}[label=(C\arabic*), ref=(C\arabic*)]
    \item\label{P:ab-and-dc:W} If
      \begin{enumerate}[label=(\alph*)]
      \item\label{P:ab-and-dc:W:ab} an agent may obtain support for the conclusion of reasoning they are able to do in cases described by~\ref{prem:ab}, and
      \item\label{P:ab-and-dc:W:uRa} \ref{denied-claim} is true,
      \end{enumerate}
      then
      \begin{enumerate}[label=(\alph*), resume]
      \item\label{P:ab-and-dc:W:AR} the support for \(\phi\) is obtained on the basis of the agent having the attribute of being able to demonstrate that \(\phi\).
      \end{enumerate}
    \end{enumerate}
  \end{quote}
  The reasoning described in the consequent of the conditional, \ref{P:ab-and-dc:W:AR}, is in line with \AR{} --- the support the agent obtains for the conclusion of the reasoning that they are able to do is obtained from the support the agent has for having the attribute of being able to reason to the conclusion.
\end{note}

\section{Second conditional}
\label{sec:second-conditional}

\begin{note}[Finding tension, still]
  We have outlined a particular type of case, an type of entailment, and two ways in which the entailment may be used.

  At present,~\ref{P:ab-and-dc:W} (merely) establishes that~\ref{denied-claim} constrains how an agent may use information about reasoning they are able to do.
  

  Still, given that~\ref{P:ab-and-dc:W} requires the agent to obtains support in a specific way (\AR{}), we turn to whether it is permissible for the agent to obtain support in the context of~\ref{prem:ab}.
\end{note}

\begin{note}[Inertia]
  Our final main premise is:\nolinebreak
  \footnote{
    A few caveats may be required here.
  }
  

  A variant of~\ref{prem:ni} is\dots
  \begin{enumerate}[label=\nI{}, ref=\nI{}]
  \item\label{prem:ni} An agent does not have the option of obtaining support for some proposition \(\psi\) from information that the support the agent has for \(\phi\) is misleading or mistaken if \(\psi\) is not the case.
  \end{enumerate}


  From \nI{} we get:
  \begin{enumerate}[label=\nIp{}, ref=\nIp{}]
  \item\label{prem:ni:p} An agent does not have the option of obtaining support for some proposition \(\xi\) entailed by \(\psi\) from information that the support the agent has for \(\phi\) is misleading or mistaken if \(\psi\) is not the case.
  \end{enumerate}
  With the additional premise that an agent requires support for antecedent of entailment in order to get support for consequent.

  With respect to scenarios of interest.
  \(\phi = \) general ability.
  \(\psi = \) specific ability.
  \(\xi = \) strategy exists.

  So, application of \nI{} is that the agent does not have option of obtaining support for specific ability from information that the support they have for general ability is misleading or mistaken if the agent does not have the specific ability.

  \nIp{} is weaker than \nP{} and is what is of interest.
  So, possible to hold \nIp{} while denying \nI{}.
  However, unclear on why \nIp{} without stronger \nI{}.
\end{note}

\begin{note}[Quick motivation]
  Question for scenarios of interest is why this is relevant.

  Focus on \AR{} and finial step.
  \begin{itemize}
  \item Have specific ability only if strategy exists.
  \end{itemize}
  Here, agent is appealing to entailment.
  This, then, requires the agent to have support for having the specific ability.

  It is not (`merely') the case that the agent would do some reasoning that leads them to conclusion.
  Agent may do reasoning and have support for premise even if they are mistaken.

  Rather, because the agent is appealing to having ability, needs to be the case that it is true that they have the ability.

  Intuitive difference.
  In \AR{} case, the agent is relying on truth and entailment.
  If agent were to reason, the agent would be relying on support from premises and steps extending.

  Difference between appealing to having ability, and to instance of ability.

  Simple example.
  I am mistaken, but I have the ability to demonstrate that \(\psi\) --- so support for \(\psi\).
  I am mistaken, but this my demonstration of \(\psi\) --- so support for \(\psi\).

  First, mistaken about being able to demonstrate that \(\psi\), so blocks what follows from ability by entailment, though the agent still retains some support for having the ability.
  Second, have shown how support extends, therefore granting there is a mistake, still retain support for \(\phi\).

  Distinction between appealing to \(\phi\) in order to get \(\psi\) and extending the support one has for \(\phi\) to \(\psi\).

  In turn, loops back to understanding of \nI{}.
  For, \nI{} doesn't extend support.
\end{note}


\begin{note}[Motivation: examples]
  Handful of examples from literature on failure of transmission.

  Don't need to hold that these are cases of transmission failure.
  Agent may get support.
  However, agent does not get support by denying~\ref{prem:ni}.

  In part, the distinction above helps understand.
  The agent does not get that it's not a cleverly disguised mule from vision.
  Rather, it's the fact that the agent is seeing, and what follows from this.

  {
    \color{red}
    Here note important point that~\ref{prem:ni} does not prevent the agent from obtaining support.
  }

  However, our interest in~\ref{prem:ni}.
  Information.
  Hence, the agent may be in a position to claim support for \(\phi\), and the information need not be something that the agent is in a position to identify by themselves.

  ``If that's a zebra and that's a ?, then you're in a position to distinguish species X from species Y.''

  As above, may be that the agent obtains support, but not via denying~\ref{prem:ni}.
\end{note}

\begin{note}[Distinguishing features]
  Reasoning has distinguishing features that pair well.
  \begin{itemize}
  \item Extends support.
  \end{itemize}
\end{note}

\begin{note}[Positive idea of support?]
  The key idea is that in order for support to go through, the agent needs to be clear on how the support the agent has for \(\phi\) is also support for \(\psi\).

  {
    \color{green}
    Motivation here is that because (specific) ability would be application of general, this is a potential (and quite direct) counterexample --- assume that the agent receives testimony.
  }

  {
    \color{red}
    A slightly better way of putting things for my purposes is that \(\phi\) does not inherit support from \(\psi\).
    So, \(\phi\) may be supported, but no support is added.
  }

  {
    \color{red}
    The agent may obtain support for \(\phi\) from some other premise.
  }

  For example, explanatory connexion.

  % A restricted instance of \ref{prem:ni} is entailed by cases of misleading evidence about evidence.
  % Here, the higher-order support is misleading else the lower order support is misleading.
  % The agent has a problem, for sure.
  % Still, \ref{prem:ni} does not entail (a restricted instance) of misleading higher order evidence.
  % For,~\ref{prem:ni} constraints only how an agent's support may be applied --- it is compatible with there being no higher order support.

  I consider~\ref{prem:ni} to be intuitive.
  However, argument may be provided.
\end{note}

\begin{note}[~\ref{prem:ni} and \AR{}]
  Why, then, does \AR{} seem to require a violation of~\ref{prem:ni}?

  First, need the conditions to match up.
  This is somewhat simple.

  The agent would be applying general ability.
  Therefore, if the agent does not have the specific ability, then something has gone with the support the agent has for the general.
  This is explicitly stated.
  Important, because do not hold the stronger position that there's a broad general to specific entailment.
  Some things may follow from rules of chess, but go beyond agent's ability, but be within some other agent's ability.
  More than simply understanding the rules.

  \AR{} fits in.
  Novel information --- no other resources for conclusion, so the agent is required to use the general ability as a premise.
  Premise is that the agent has the attribute of being able to \dots.
  Conclusion is that the agent has attribute of being able to\dots.

  So, one attribute from another.
  `Extends' type of support relation.
  However, agent isn't appealing to how support extends.
  There's no additional reasoning, other than the conditional.

  Observe that this is not compatible with the agent holding that they may be mistaken.
  Support for general ability may be misleading or mistaken, but given support, then this extends to specific ability.
  ? This seems bad.

  Contrast to.
  Support for general ability may be misleading or mistaken, but given support and information, then reasoning I have support for extends to conclusion.
  Agent is not relying on support for general ability to provide support for specific ability.
  Instead, agent is appealing to support may appeal to as a use of the general ability.

  Compare with simple cases.
  


  Mistaken on the first may involve no support for \(\phi\).
  Mistaken on the second is mistake in use of support in demonstrating \(\phi\).
  {
    \color{green}
    Well, with the first, it's getting to \(\phi\).
    Doesn't seem the agent is in a position to use factive inference.
    Because, the agent is going from not possible to have ability and for \(\phi\) to be false.

    \textbf{
      This is the part I'm missing.
      It's the use of this particular inference that really characterises \AR{}.
      It's possible for appeal to attribute without this inference, but that doesn't matter to me.
    }

    With the second, different.
    Because, the agent is going from application of ability providing support for \(\phi\).
  }

  {
    \color{red}
    This is in contrast to \WR{}, where the agent may hold that they may be mistaken, but the reasoning extends.
    Hence, may be mistaken, but given the support I have, support for conclusion.
    Lack an account of how reasoning extends.
  }
\end{note}

\begin{note}[A little hyperbole]
  The use of \AR{} is a little hyperbolic.
  For, have not provided an argument that there's no way for reasoning via attributes to avoid~\ref{prem:ni}.
  It's the other way round.
  And, that it's not clear that there's no other interpretation.
\end{note}

\begin{note}[Indirect]
  The really important idea is that in the cases of interest, if the agent goes from general ability, then it's \emph{indirect}.
  The agent is not using their general ability.
  Rather, the agent is relying on having the general ability.

  Further, the consequent is a potential counterexample.
  This is what really distinguishes things, I think.

  The idea, then, is that if \(\psi\) is not the case, then the agent has an counterexample to \(\phi\).
  Hence, as the agent doesn't `do' anything to show that \(\psi\) is the case, they can't simply transmit support for \(\phi\) to \(\psi\).
\end{note}

\begin{note}[More than indirect]
  Example.
  If not in London, then in Paris.

  This is a `is-also' support.
  Conditional states that support for not in London is also support for Paris.
  Does the agent obtain support for Paris on the basis that the support for not London is misleading or mistaken if Paris is not the case?
  Well, would not show that the \emph{support} is mistaken or misleading.
  Because, the conditional establishes something that doesn't follow from the antecedent without information.

  By contrast, the general-to-specific is such that the support the agent has for general must also extend to specific.

  So, not London to Paris has this `is-also' understanding.
  If not in Paris, then it seems the agent is still going to have support for not in London.
  Puzzle here is contraposition.
  For, it follows that if not in Paris then in London.
  However, this is not so obvious.
  May give up conditional.

  This contraposition observation is a really clear difference.
  For, contraposition does hold in the general-to-specific case.
  If agent doesn't have specific, then doesn't have general.
\end{note}


\begin{note}[`includes']
  General to specific is an `includes' conditional.
  If support for general, then extends to specific.
  Key idea here is that general and specific reduce down to similar things.

  So, the additional component of interest is that the agent should be, in some sense, committed to have support for any specific instance of general ability.
  Something like \(\forall S(G \rightarrow S)\).
  However, there's something puzzling about this.
  Because, the agent has no idea what those applications are.

  \ref{prem:ni} suggests.
  For, agent would be relying on support for general not being misleading, and a specific may show that it is.


  
  So, it seems this doesn't provide the agent with support.
  So, something about support from some subset extending to other elements of set.

  Gives an intuitive example.
  Picking balls out of a bag.
  Well, it's the case that all the balls are red, otherwise I've picked out a non-representative sample.
  Quite possible that this is the case.
  Agent doesn't get support for next being red.
  Though, agent has support for a probability distribution.

  This really goes back to the idea that the agent has no idea whether the support they have will be shown to be mistaken.

  If so, the LP example isn't so clear cut.
  For, if the agent's got enough support that wouldn't be shown to be mistaken, the use of conditional goes through.
  Further support added, but what they have remains robust.
  Right, this seems correct.

  Same for balls in a bag.
  It's possible that the next ball revises the probability distribution, if the agent isn't aware of what's going on in the bag.
  No, this doesn't quite work.
  It only really works if the agent makes an assumption about the colours or something of the kind.
  See this would be a bad assumption to make, because the next ball would show the assumption to be mistaken.

  Hence, because specific would be an application of general, then failure of specific would show that there's something wrong with general.
\end{note}

\begin{note}[\WR{} and \AR{}]
  Role of~\ref{prem:ni} should be clearest with the difference between \AR{} and \WR{}.

  With \AR{}, the agent appeals to support they have for general ability.
  This support, we assume, is accessible to the agent.
  And, the difficulty is that the agent would find that some support is misleading or mistaken if they don't have the general ability.
  The issue with~\ref{prem:ni} is that the agent appeals to support for general ability, not the support for items of application of general ability.

  One way to look at this is in terms of whether the agent has the ability.
  For, with \AR{}, this seems to be how things go.
  Support for general ability, therefore one has the general ability, and so one has the specific ability, and hence one has support for specific ability.
  So, projection of belief.

  Another perspective is that the agent is not \emph{applying} general ability in order for specific ability.
  Which, is roughly what is meant by `indirectness'.
  There should be something here, as this really is the core difference between the two approaches.
  For, with \WR{}, then, the agent traces support through reasoning\dots
  Relying on the support that application provides for conclusion.
  Not, that there's something wrong if the agent doesn't get support for conclusion.
  And this ties in with the useful point.
  If lack, then the support is misleading or mistaken.
  However, from agent's perspective, they've extended support from premises to conclusion, and have not relied on indirect claim that this can be done.
  Of course, the agent doesn't have access, and hence conflict with~\ref{denied-claim}.

  With \WR{} the agent relies on information, needs to be true that they may witness.
  Agent does not require support for this.
  Rather, it is true, and then support builds throughout the process.


  With \WR{}, the agent applies support as they would in reasoning.
  So, the agent is not appealing to support for general ability, but support general ability provides for premises and steps of reasoning.
  The difficult, and conflict with~\ref{denied-claim} is that the agent does not have access to what they are claiming support for.
  However, support functions the same way as if the agent were to witness the reasoning.

  Return to the idea of projection of belief.
  With \WR{}, the agent does not project belief.
  For, agent `witnesses' \dots but this is difficult, because agent still relies on information for \(W\) attitude.
  


  A useful point is that the support being misleading or mistaken if the agent lacks the specific ability does not mean that the agent is obtaining support for specific ability on the basis that the support for the general ability is not misleading or mistaken.

  This useful point is clearest to see when considering an instance of reasoning.
  
\end{note}

\begin{note}[Overview]
  From explanation:
  \begin{itemize}
  \item Indirect/\AR{} is bad.
  \item Reasoning would be fine.
  \item \WR{} mirrors reasoning for the most part.
  \item Information isn't used to transfer support, but provides structure for \WR{}.
  \end{itemize}
  
\end{note}

\begin{note}[Important points]
  Two important points:

  The role of~\ref{prem:ni} is to highlight that the agent is not in a position to obtain support for (specific) ability in a certain way.
  That is,~\ref{prem:ni} does not state that the agent may not obtain support for (specific) ability some other way.

  Second, some long as agent holds that they have general ability, then committed to truth.
  The \gen{} may be taken as testimony.
  So, either one or the other.
  A kind of transmission failure.
  Distinct from case of knowledge, as no factivity.
  In part, role of informer.

  May be tempted to say that the agent is not committed, but this seems implausible.
  Cases of transmission failure, it seems agent does remain committed, at least.

  May take issue with information provided, especially if ideal.
  If informer has information, then they should say.
  In turn, not problem with~\ref{prem:ni} as the agent would have support (via testimony) for specific ability.
  However, informer may only have the conditional.

  Ordinary agents.
  Maxims are broken.
  And, interest effects.
  Up to the agent.

  Seems puzzling, but not paradoxical.
\end{note}

\begin{note}[Why this is important]
  The key idea, and the foundation of the objection, is that the agent is going indirectly.
  The agent is \emph{not} showing how the general ability extends to specific ability.
  For, the only things available to the agent is the constraint.

  This is useful independently.
  For, even if not convinced by~\ref{prem:ni}, clear that given \gsi{} and~\ref{denied-claim}, the agent goes directly.
  And, something a little puzzling about this.
  Or, so I think.
\end{note}

\begin{note}[Aside]
  I don't see this as too different from simply stating that general ability extends.
  That is, the sort of \gsi{} is what is actually communicated in most cases of this kind.
  However, nothing rests on this.
\end{note}

\begin{note}[Dogmatism]
  Continuing relation to issues with knowledge.
  \autoref{prem:ni} is quite close to dogmatism paradox.
  If one knows that \(\phi\), then any evidence for \(\lnot \phi\) is misleading.

  Distinct again, however.
  For, don't have knowledge in the antecedent.
  Get the dogmatism paradox from the factivity of knowledge.
  No requirement that support for (general) ability is factive.

  Hence, role of the informer is important again, because agent is not in a position to come to the conditional by themselves prior to reasoning.
\end{note}


\begin{note}[Requires some care]
  \ref{prem:ni} requires a little care.
  It is a constraint on whether the agent has the option of obtaining support.
  It does not constrain whether the agent is able to make use of the information in other ways.

  Useful example may be commitment.
  For example, committed to \(\psi\), but no support.

  In particular, support for \(\lnot\psi\) seems difficult.
  The agent doesn't have the option of obtaining support for \(\psi\).
  However, it seems that with the information there is some resistance to support for \(\lnot\psi\).

  This point will be important later.
\end{note}

\begin{note}[Inertia and attribution]
  \ref{prem:ni} raises a problem for applying \AR{} to the specific ability of scenarios described by~\ref{prem:ab}.

  For, the agent is required to obtain specific ability from general ability.
  \begin{enumerate}
  \item Agent has support for the general ability to reason with the rules of chess.
  \item However, the agent has not demonstrated the existence of the strategy, and so the agent relies on the information provided by the informer to hold that they have the specific ability to demonstrate the existence of the strategy.
  \item Still, the informer provided by the informer requires the agent to endorse having the general ability to reason with the rules of chess.
  \item In turn, that whatever support the agent has for having the general ability to reason with the rules of chess is not misleading.
  \item For, it may be the case that the agent does not have the ability to demonstrate the existence of the particular strategy.
  \item Therefore, the agent does not obtain support for the ability to demonstrate the existence of the strategy.
  \item Hence, it is not an option for the agent to obtain support for the existence of the strategy on the basis of support for the ability to demonstrate the strategy in line with \AR{}
  \end{enumerate}
\end{note}

\begin{note}[Established conditional 2]
  We wrap the above observations in a conditional.
  \begin{enumerate}[label=(C\arabic*), ref=(C\arabic*)]
    \setcounter{enumi}{1}
  \item\label{P:ab-and-dc:A} If
    \begin{enumerate}[label=(C2\alph*), ref=(C2\alph*)]
    \item an agent obtains support for some proposition \(\phi\) on the basis of the agent's ability to demonstrate that \(\phi\) is the case, and
    \item \ref{prem:ab} is true,
    \end{enumerate}
    then the support for \(\phi\) \emph{may not be} obtained (in line with \AR{}) on the basis of the agent having the attribute of being able to demonstrate that \(\phi\).
  \end{enumerate}
\end{note}

\begin{note}[Established conditional 2]
  Something about, if \ref{prem:ab} then agent does not obtain support for the attribute.
\end{note}

\section{Establishing tension}
\label{sec:establishing-tension}

\begin{note}[Summary]
  Given the two established conditionals~\ref{P:ab-and-dc:W} and~\ref{P:ab-and-dc:A}, the combination of the key premises of \ref{denied-claim},~\ref{prem:ab}, and~\ref{prem:ni} are in tension.

  For, combining~\ref{P:ab-and-dc:W} and~\ref{P:ab-and-dc:A} we have:
  \begin{enumerate}[label=(CC), ref=(CC)]
  \item If \ref{prem:ab} is the case an agent obtains support for some proposition \(\phi\) on the basis of the agent's ability to demonstrate that \(\phi\) is the case then:
    \begin{enumerate}[label=(C\arabic*\(\sim\)), ]
    \item If \ref{denied-claim} is true, then the support for \(\phi\) is obtained on the basis of the agent having the attribute of being able to demonstrate that \(\phi\) (in line with \AR{}).
    \item If \ref{prem:ni} is true, then the support for \(\phi\) \emph{may not be} obtained (in line with \AR{}) on the basis of the agent having the attribute of being able to demonstrate that \(\phi\).
    \end{enumerate}
  \end{enumerate}
  In short, if~\ref{prem:ab} is the case then~\ref{denied-claim} requires a certain interpretation of the scenarios identified by~\ref{prem:ab} and~\ref{prem:ni} denies that the interpretation is plausible.
\end{note}

\begin{note}[Tension, choices]
  The combination of~\ref{P:ab-and-dc:W} and~\ref{P:ab-and-dc:A} is complex.
  However, the basic structure is straightforward:
  \[\ref{prem:ab} \rightarrow ((\ref{denied-claim} \rightarrow \AR{}) \land (\ref{prem:ni} \rightarrow \lnot \AR{}))\]
  Rewriting:
  \[\ref{prem:ab} \rightarrow ((\lnot \AR{} \rightarrow \lnot \ref{denied-claim}) \land (\AR{} \rightarrow \lnot \ref{prem:ni}))\]
  Hence:
    \[\ref{prem:ab} \rightarrow ((\WR{} \rightarrow \lnot \ref{denied-claim}) \land (\AR{} \rightarrow \lnot \ref{prem:ni}))\]
  Simplifying:
  \[\ref{prem:ab} \rightarrow ((\WR{} \land \lnot \ref{denied-claim}) \lor (\AR{} \land \lnot \ref{prem:ni}))\]
  A reformulating as a distinction:
  \[\lnot \ref{prem:ab} \lor (\WR{} \land \lnot \ref{denied-claim}) \lor (\AR{} \land\lnot \ref{prem:ni})\]

  In short, we have the following resolutions.
  \begin{enumerate}
  \item\label{ten:res:nS} Agent may not obtain support for result of witnessing ability, or
  \item\label{ten:res:nD} Agent obtains support for result on the basis on premises that the agent would use when witnessing ability --- incompatible with general application of~\ref{denied-claim}
  \item\label{ten:res:nI} Agent obtains support for result from attribute of having the ability on the basis that the support they have for general ability would be misleading --- incompatible with general application of~\ref{prem:ni}
  \end{enumerate}
  \ref{ten:res:nS} is incompatible with~\ref{ten:res:nD} and~\ref{ten:res:nI}.
  However, \ref{ten:res:nI} and~\ref{ten:res:nD} are compatible, as both~\ref{denied-claim} and~\ref{prem:ni} may be restricted.
\end{note}

\begin{note}[Argument sketch recap]
  Let us recap the main points of the argument so far.
  \begin{enumerate}
  \item Assume possibility of cases in which agent is provided with information that they have some specific ability so long as the agent has a general ability, such that the agent has support for having the general ability, but has not established support for possessing the specific ability.
  \item In such cases, it seems it is possible for the agent to obtain support for what follows from the agent witnessing their specific ability.
  \item If so, the agent appeals to having the specific ability in order to obtain support for what follows from the agent witnessing their specific ability.
  \item Attribution, and witnessing.
  \item If witnessing, then conflict with the requirement that an agent must access support for the premises appealed to in support of a conclusion.
  \item If attribution, then conflict with the restriction that an agent may not obtain support for some proposition on the basis that support the agent has for some other proposition would be misleading otherwise.
  \end{enumerate}

  To follow:
  \begin{enumerate}
  \item Restricting~\ref{denied-claim} in favour of~\ref{rC} works well.
  \end{enumerate}
\end{note}

\begin{note}[Meek outlook]
  This is not a clean argument.
  Take~\ref{denied-claim} and~\ref{prem:ni} and hold the first.
  The agent may not obtain support.

  While there may be tension if the agent obtains support, this tension is never instantiated.

  I am sympathetic.

  Still, endorsing the restriction does not require the agent to obtain support in this case.
  Harbour some hope that that there is scope to restrict \ref{denied-claim}, and that the argument provided for resolving tension in favour of \rC{}, along with later arguments, may serve as a source for reflection.
\end{note}

\section{Following structure}
\label{sec:following-structure}

\begin{note}[Structure]
  The initial portion develops tension in detail.
\end{note}