\chapter{Overview}
\label{cha:overview}

\section{Outline}
\label{sec:outline}

\begin{note}
  In this chapter we provide a high level overview of the main arguments made in this thesis.
  A significant part of the high level overview of arguments is an overview of the premises and assumptions that those arguments rest on.

  By given high level overview, clarify on how premises, assumptions, and conclusions relate.
  In the main body of the thesis, afford to elaborate.
  And, allow choice of where to seek elaboration.
\end{note}

\begin{itemize}
\item Start with claiming support, used throughout, so important.
\item Introduce and motivate plausible constraint on support, to be argued against/exception for.
\item Outline exception.
\item High level overview of argument for exception.
\item Major and minor.
\item Largely fairly high level sketch of major argument.
\item Type of ability information.
\item Understanding \emph{de re} ability reading.
\item \AR{} and \WR{}.
\item For the moment, brief, much more detail in relevant chapter.
\item Relation between \AR{} and \ESU{}.
\item Constraint on reading of ability.
\item Requires argument for \WR{}.
\item Introduce \nI{}.
\item Sketch argument for \nI{}.
\item Link \nI{} and \AR{}.
\item Completes overview of major argument.
\end{itemize}

\section{Ability and access to claimed support}
\label{sec:abil-access-supp}

\begin{note}
  Following introduction, interest is with ability.
  In particular, observation that \gsi{} information, and confidence in general ability seems to allow agent to claim support for result.

  Question on what basis the agent claims support.

  Slightly more general statement.
  \begin{quote}
    When and why an agent may claim support for the result of reasoning that the agent has not witnessed.
  \end{quote}
  Ability claims of interest provide some answer to `when'.
  For, ability provides information about what the result is, and also that the agent has the opportunity to perform the reasoning.\nolinebreak
  \footnote{No claim to necessity}

  Suggested in introduction, answer to `why' is more complex.
  % \begin{quote}
  %   Our interest is in when an agent may claim support for some conclusion of some instance of reasoning on the basis of the support the agent may claim for the premises of the instances of reasoning.
  % \end{quote}
\end{note}

\begin{note}[Introducing support]
  Initial clarification is with respect to claiming support.
  Emphasis on `\emph{claim}'.

  The thesis is not about when and why an agent \emph{has} support for the result of reasoning that the agent has not witnessed.

  There are three primary reasons why we focus on claimed support.

  First, neutral for main thread of argument on what support amounts to.
  Interest is with structure of claim, and background assumption that if success in claiming then structure of support follows structure of claim.

  Second, whether or not an agent has support often seems secondary.
  To illustrate:
  It may be that any claimed support for a proposition is support for that proposition, but perhaps not.
  Suppose `flan' is written on the side of a container.
  I may claim support that the container contains flan.
  And, it may be that the writing on the side of the container is support for the box containing flan.
  However, the straps ensuring the container remains closed is unfortunately placed, and if moved would reveal the side of the container reads `flannels'.
  The unfortunate placing of the straps does not seem to prevent \emph{claiming} support, but I'm not sure whether it is right to say that the writing on the side of the box (straps in place) \emph{does} provide support that the box containing flan.
  So, in what follows I will speak in terms of claiming support, and leave open whether what is claimed reflects on whether an agent has support.\nolinebreak
  \footnote{
    In particular, claiming allows focus on internal constrains, while remaining silent on whether having support is (in part) determined by external factors.
  }
  \(^{,}\)\nolinebreak
  \footnote{
    Distinction between propositional and doxastic support.
    Propositional, support agent has whether or not made a claim.
    Doxastic is successful claim and propositional support.
    So, both require that the agent has support.
    Claimed support is the agentive component of doxastic support.
    Not interested in whether the agent also has propositional support, though more or less assume.
  }
  \(^{,}\)\nolinebreak
  \footnote{
    {
      \color{red}
      English is somewhat difficult.
      It is somewhat unfortunate that `an agent has claimed support for \(\phi\)' may be read `there is support which the agent has claimed for \(\phi\)'.
      Still, this seems to follow more easily from `support claimed'.
      So, `claimed support' emphasises the claim, while `support claimed' emphasises support.
    }
  }

  Third, and following from the second, focusing on claimed support allows us to make no assumption about the relationship between claimed support and support.
  To elaborate, consider enthymematic inferences.
  One may hold that an agent may claim support for some conclusion via enthymematic inference, but hold that the support the agent has is explicated in terms of the (corresponding) complete inference.\nolinebreak
  \footnote{
    Cf.\ \textcite{Moretti:2019wx}.
  }
  Alternatively, one may hold that the enthymematic argument is an adequate support relation (at least with respect to context in which the inference was made).
  Hence, one may question whether the structure of claimed support follows the structure of support.
  Without any assumption concerning this relation, we will not be committed to any position on why an agent has support for an ability given a position on how an agent claims support for an ability.
  In turn, an important implication of this final point is that why or when an agent \emph{has} an ability, only why (and when) an agent claims support for and from ability.

  {
    \color{red}
    Reference \citeauthor{Moretti:2019wx} here in terms of support for the enthymematic stuff, then reference the conditions that \citeauthor{Moretti:2019wx} outline as an example \ESU{}, and finally in some later chapter as a comparison to \EAS{}.
    \citeauthor{Moretti:2019wx}

    Note, \citeauthor{Moretti:2019wx} is somewhat different, as it's not about appeal.
    So, agent may form well-grounded belief, but additional steps required to show that the agent may appeal to the expanded inference.
    Key, then, is the relationship between the basing relation and what the agent appeals to.
    So, it seems there's not really any discussion of how the agent reasons in these cases.
    Rather, it seems that \emph{how} the agent reasons is left out, and the relevant enthymematic inferences are in some sense automatic in terms of appealing to the way in which propositional support expands without requiring the agent to perform any inferences.
  }


  To illustrate, consider dispositions.
  Simple conditional analysis.
  Perhaps I claim support because I imagine the event, and in the event the glass breaks.
\end{note}

\subsection{Claiming support and reasoning}
\label{sec:claimed-support}

\begin{note}[Understanding of claiming support]
  Understanding of claiming support.

  Begin with a sufficient condition.
  In short, most instances of reasoning.
  Claiming support is common.

  Then, two types of defeaters.
  Mistaken and misled.
  Use to form a necessary condition.
  If claimed support, then agent deems that claimed support is not defeated.
  `Deem' is a placeholder.
  Strong or weak.
  Single constraint is that when claiming support, potential defeaters that aren't ruled out.\nolinebreak
  \footnote{
    At least two ways of viewing this.
    First, claiming support is restricted.
    Second, \emph{claiming} support only applies when there are potential defeaters, and some other relation to support when possible defeaters get ruled out.

    These are different, but I don't think the difference matter for resource bound agents of interest.
    Lack of resources is that always potential defeater, even if every possible defeater may be ruled out.
    }
  Finally, property, that claimed support does not depend on whether proposition the agent has claimed support for is true, or whether the claimed support \emph{does} support (if these are separated).
  Property will be important.
\end{note}

\begin{note}[Sufficient condition]
  We start with a sufficient condition for claiming support:
  \begin{proposition}[\USE{-} --- \USE{}]\label{prem:bP}\label{prop:USE}
    If an agent may claim support for premises and steps of reasoning, accesses those premises and traces claim to support through those steps of reasoning, then agent may claim support for conclusion on basis of the claimed support for the steps and premises of reasoning.
    (Given that the agent deems that the claims to support for premises and steps used are undefeated when drawing conclusion.)
  \end{proposition}

  The purpose of taking~\USE{} as basic is to fix a basic understanding of when an agent may claim support.
  In short, an agent may claim support when reasoning goes well.
  And, reasoning goes well when there are premises and steps of reasoning available to the agent, and the agent draws on these to claim support for the conclusion.
  The parenthetical remark is a simple safeguard, the agent does not lose a claim to support for the premises or steps in the process of reasoning.\nolinebreak
  \footnote{
    May think that this is the wrong safeguard.
    Consider the liar paradox:
    `This sentence is false.'
    \USE{} prevents agent from claiming support that the sentence is true or false.
    However, may think that agent is in a position to claim support that the sentence is both true and false.
    Indeed, standard reasoning associated with the liar suggests that the sentence is both true and false.
    Still, it's not obvious from demonstrating that the sentence is both true and false that one may claim support for the sentence being true and the sentence being false.
    That is, one may confine the paradox to the truth value of the sentence, rather than (associated) surplus of support.
  }
  We consider defeaters below.
  First, an illustration.


  Suppose an agent measures that the box in front of them has the dimensions of \(19\text{cm}\) by \(7\text{cm}\).
  The agent understands how to calculate the area of a box, and by performing some reasoning comes to hold that the area of the box is \(133\text{cm}^{2}\).
  The support the agent has for holding that the area of the box is \(133\text{cm}^{2}\) is obtained (at least in part) on the measurement of box, understanding how to calculate the area of box, and some grasp of arithmetic.

  Whether some (or all) of the required arithmetic is to be included as a premise or a step of reasoning may be set aside.
  Similarly, set aside whether further arguments, or whether some premises and steps are taken as basic.
  For example, perhaps some the agent requires some further claim to support for using the ruler to measure the box such as comparison to a standard, or perhaps the agent's claim to support terminates by noting that their use of the ruler is a reliable process.
\end{note}

\begin{note}[Value proposition]
  Reasoning and claims to support focus.
  Briefly introduce a pair of propositions to clarify claim to support and reasoning.

  \begin{proposition}[Claimed support is for the value of a proposition]\label{prop:csifvoap}
    When an agent claims support for some proposition, the agent claims that the proposition has some value.
    Where:
      \begin{itemize}
      \item A proposition is some state of affairs. And,
      \item A value is an assessment of a state of affairs.
      \end{itemize}
      \vspace{-\topsep}\vspace{-\topsep}
  \end{proposition}
  \autoref{prop:csifvoap} fixes terminology.
  To illustrate, when stating the conclusion of the reasoning sketched above we used the proposition that \emph{the area of the box is \(133\text{cm}^{2}\)}.
  The proposition refers to the state of affairs in which the area of the box is \(133\text{cm}^{2}\), and speaking a little more precisely, the agent claimed that the proposition has the value `true' --- though it may be the value turns out to be `false'.
  Or, perhaps if the agent was a little unsure about the accuracy of the ruler, that the proposition has the value `likely', `probable', or some quantitative credence.
  And, some other instance of reasoning may have concluded that the proposition has the value `desirable' --- e.g.\ if the agent was searching for a box of some approximate size.\nolinebreak
  \footnote{
    Nothing in particular hangs on the distinction between different values.
    If you prefer, you may expand the proposition (state of affairs) to include additional factors, and consider only the values `true' and `false'.
    For example, the proposition that \emph{I desire the bath to be warm} is false, as opposed to the proposition that \emph{the bath is warm} is valued undesirable by me.
  }

  Core idea is that claim of support is that things are a certain way.
  Proposition, what the thing is.
  Value, the way it is.
  In most cases the value will be clear (i.e. that the proposition is true, though sometimes that the proposition is desirable), and so we will talk of claiming support for the proposition.
  A handful of additional examples will be provided when illustrating the next proposition.
\end{note}

\begin{note}[Reasoning proposition]
  \begin{proposition}[Reasoning as establishing value]\label{prop:RisTV}
    Reasoning, tracing value through propositions/establishing that proposition has value.
  \end{proposition}

  \begin{itemize}
  \item Testimony, so claim support that \emph{p} is true.
  \item Unreliable, so claim support that \emph{p} improbable.
  \item Religious text, so claim support that \emph{p} ought to be the case.
  \item Producer, so claim support that album is desirable.
  \end{itemize}

  In a deductive case, if the premises are true, then the conclusion is true.
  Means-end reasoning for desire.
  The value is important.
  If it is true that it past 6pm, then it is true the shop is closed.
  Provides value of shop being closed.

  However, if agent desires that it is past 6pm, then it doesn't follow that the agent desires that the shop is closed.
  Question an agent as to why they think their desires conform to truth --- is-ought problem.

  Means-end reasoning.
  It is true that there is cheese at the centre of the maze.
  And, it is desirable that I obtain the cheese at the centre of the maze.
  Further, it is true that I may only obtain the cheese at the centre of the maze by solving the maze.
  Therefore, it is desirable that I solve the maze.
\end{note}

\begin{note}[Moving to value]
  \color{red}
  Important to note is that way in which agent claims support is important.

  Preface paradox.
  Here, move to value, but these are isolated.
  Don't get to combine these together, no allowed to hold that these are simultaneously true.
  Or, simultaneously assign value to the individual propositions.

  So, move isn't free, so to speak.
\end{note}

\begin{note}
  \begin{proposition}[Defeaters for claimed support]
    There are at least two ways in which a claim to support may be defeated.
      \begin{itemize}
      \item Claimed support may (discovered to be) be \emph{misled} by suggesting that a proposition has some value that it does not (in fact) have. And,
      \item Claimed support may (discovered to be) be \emph{mistaken} by appealing to factors that do not indicate the value of the proposition.
      \end{itemize}
      \vspace{-\topsep}\vspace{-\topsep}
  \end{proposition}
  Misleading support may indicate value, and value may have value indicated by mistaken support.
  Both are defeaters in the sense that, were the agent to learn that the claim to support was misleading or mistaken, then the agent would not hold that the proposition has the value indicated by the (problematic) claim to support or the basis of that (problematic) claim to support.

  Common to distinguish between countervailing and undercutting defeaters.
  Countervailing, support for some other proposition.
  Undercutting, force of claimed support is mitigated.

  Both misled and mistaken are instances of being undercut.
  For, if mistaken or misled then the support is no good.

  Neither are clearly cases of countervailing.
  For, no relation to other support is required.
  However, presence of countervailing implies misled.
  Countervailing does not imply mistaken, though may in some cases demonstrate so.

  Below suggest that when claiming support the agent excepts that claimed support is not misleading or mistaken.
  Hence, given that countervailing implies misled, implicitly take claimed support for a proposition to indicate absence.
  (This is another purpose from separation from support.)

  ???
\end{note}

\begin{note}[M\&M Illustration]
  To illustrate:

  Suppose I glance at the clock on the wall.
  The clock reads 11:45a, so I claim support that it is 11:45a.
  However, it may be the case that the clock is incorrectly set, and the time is 11:15a, or 12:15p, etc.\
  By claiming support from the time expressed by the clock, I would have been \emph{misled} about what the time actually is.
  For, it is not true that the time is 11:45a.
  Though, in all other respects, there may be no fault with claiming that the time is as expressed by the clock and so the claim to support is not mistaken.

  By contrast, suppose I glace at the clock on the wall.
  The clock reads 11:45a, so I claim support that it is 11:45a.
  By claiming support from the time expressed by the clock, I would have been \emph{mistaken} about what the time actually is.
  For, the time expressed by a broken clock is not a good indicator of what the time actually is.
  Though, despite the clock being broken, it is 11:45a and so the claim to support is not misleading.

  And, claimed support for the time from a broken clock expressing the wrong time would be both misled \emph{and} mistaken.\nolinebreak
  \footnote{
    A second illustration:
    Consider a smoke detector, designed to sound an alarm if and only if sufficient levels of smoke are detected.
    Hence, if the alarm sounds, one may claim support there being smoke in the room where the alarm is installed.
    One may be misled; the alarm may have malfunctioned, so no fire.
    Or, one may be mistaken; the same type of alarm may be installed in a different room, wouldn't be a useful indicator.
  }
\end{note}

\begin{note}[Subjectively sound]
  Of course, clocks are typically glanced at, and a glance at a clock is often insufficient to determine whether the clock is incorrectly set or broken.
  Hence, the \emph{possibility} that a clock is incorrectly set or broken --- or more broadly the possibility that claimed support is misleading or mistaken --- does not prevent an agent from claiming support.
  So, ensuring that to-be-claimed support would be neither mistaken or misleading is not a necessary condition for claiming support.
  Rather, we endorse the following condition with respect to these types of defeaters:

  \begin{proposition}[Adequacy of claimed support]\label{prop:CSNMORM}
    If agent claims support for some proposition, then
    % from the perspective of the agent,
    the agent deems that the claimed support
    % adequate whether or not the claimed support may (in fact) be
    is not misleading nor mistaken --- the agent has some expectation that possible defeaters (of the two types of noted) do not obtain.\nolinebreak
    \footnote{
      Stronger than distinct claim that the agent does not deem that the claimed support is misleading or mistaken.
      Stronger requires, presence of some deeming.
      Later does not.
    }
  \end{proposition}
  Our understanding of `expectation' contrasts to claiming support.
  If an agent claims support that some proposition \(\phi\) has value \(v\) then the agent appeals to information that indicates that \(\phi\) has value \(v\).
  By contrast, we assume that an agent to noting that they have no information that indicates a possible defeater obtains is sufficient to expect that the possible defeater does not obtain.
  For example, there appears to be a bowl of fruit in the centre of the table, and so I claim support by visual inspection that there is a bowl of fruit in the centre of the table.
  I may be misled, what appears to be fruit may be plastic replicas, but I may expect that this possible defeater does not obtain as I have no information which indicates that the items are plastic replicas.

  If some expectation that they do, then this seems enough to deny support.
  If no attention to defeaters, questionable whether any claimed support.

  If the claimed support is not misleading, then the proposition has the value the claimed support indicates the proposition has.
  And, if the claimed support is not mistaken, then the claimed support indicates the value.
\end{note}

\begin{note}[Agents are fallible]
  The missing piece of \autoref{prop:CSNMORM} is an account of what `deem' amounts to.
  The following proposition (\ref{prop:fallibility}) states an assumption, which allows a (general) functional characterisation of `deem'.

  \begin{proposition}[Agents are fallible]\label{prop:fallibility}
    When claiming support for a proposition and agent is never in a position to rule out the (epistemic) possibility that the claimed support is not misled or mistaken.
  \end{proposition}

  I take it to be intuitive that agents are fallible in many cases of claiming support.
  It is not too difficult to think of ways in which claimed support may be misleading or mistaken.
  As noted above, claiming support for what the time is from glancing at a clock seems sufficient, but clocks may be incorrectly set (misled) or broken (mistake).
  Similarly, a sample of \(1,000\) rolls may mislead me into thinking that a die is unbiased, or an overloaded operator may lead to a mistake in claiming support for the proposition that \(x = 4\) is an expression of equality rather than variable assignment.

  \autoref{prop:fallibility} states that an agent is never in a position to rule out the possibility that the claimed support is not misled or mistaken.
  This does not entail that there are defeaters, nor that there are any possible defeaters --- only that defeaters are an epistemic possibility.

  Still, in some cases, this may seem absurd.
  Suppose in front of me are two apples and two pears.
  So, four pieces of fruit.

  However, those appears and pears may not be real pieces of fruit, they may be replicas.
  So beings the process of attempting to quarantine fallibility from infallibility.

  There are two pairs of objects in front of me, and the objects appear to be fruit.
  So, there are four objects in front of me, which appear to be fruit.

  There are two pairs of objects hence there are four objects.

  But I may be hallucinating.

  There appear to be two pairs of objects in front of me, which appear to be fruit.
  So, there appear to be four objects in front of me, which appear to be fruit.

  Whenever there are two pairs of objects, there are four objects.

  Even so, I'm not in a position to rule out the possibility that arithmetic is inconsistent.

  Perhaps identity is a stronger candidate: Any object is identical with itself --- but I doubt one needs to claim support for reflexivity of equality.
  Similarly, it doesn't seem to be the case that I need to claim support for the proposition that my name is ミスタ --- no matter what my birth certificate says, I get to decide what my name is.

  Rather than fix a specific account of the `possibility' modal used, here are a handful compatible interpretations:

  \begin{enumerate}[label=\Alph*., ref=(\Alph*)]
  \item\label{CS:I:Never} \emph{In principle} it is not possible for any agent to rule out the possibility that claimed support is not misleading or mistaken.
  \item\label{CS:I:Resources} It is not possible for a \emph{resource bound agent} to rule out the possibility that claimed support is not misleading or mistaken.
  \item\label{CS:I:Class} There is a restricted class of propositions for which and agent is required to claim support, and it is not possible for any agent to rule out the possibility that claimed support for a proposition belonging the class is not misleading or mistaken.
  \end{enumerate}

  To illustrate, consider the proposition that there is an external world:
  \ref{CS:I:Never} denies that there could be, e.g.\ proof of an external world.
  \ref{CS:I:Resources} denies that agents of interest could not demonstrate such a proof even if it were to exist.
  \ref{CS:I:Class} allows an agent may not be required to claim support for the existence of an external world.

  I expect the intended application of claimed support will be compatible with each interpretation, and specifically with respect to \ref{CS:I:Class}, that the propositions will belong to the highlighted class.\nolinebreak
  \footnote{
    I favour the combination of \ref{CS:I:Resources} and \ref{CS:I:Class}, and to leave open whether an idealised agent may rule out the possibility of being misled or mistaken with respect to some propositions when claiming support.
  }\(^{,}\)\nolinebreak
  \footnote{
    In particular, true of ability.
  }
\end{note}

\begin{note}
  Given some interpretation of the possibility modal, the functional role of `deem' is to provide resistance to possible defeaters that an agent is not in a position to rule out.
  How resistant the agent's claimed support needs to be is up to you.

  Provide a test below to help with intuition.
  Before doing so, final proposition, which follows as a corollary from previous.
\end{note}

\begin{note}[\eiS{}]
  Subjectively sound, claimed support indicates value.

  Final proposition.
  Definition.

  \begin{definition}[Dependence and independence]
    An agent's claimed support for \(\phi\) \emph{depends} on some value of \(\psi\) just in case the agent would not claim support for \(\phi\) given any other value of \(\psi\).

    An agent's claimed support for \(\phi\) is \emph{independent} of \(\psi\) just in case the agent's claimed support does not depend on the value of \(\psi\).
  \end{definition}

  Claiming support is independent of value.

  \begin{proposition}[\eiS{-} --- \eiS{}]\label{prop:supp:independence}
    If agent claims support for some proposition, \(\phi\), then the claimed support is taken to indicate the value of \(\phi\) independently of the value of \(\phi\) or whether the claimed support is `genuine' support.
    Equivalently, the claimed support is taken to indicate the value of \(\phi\) does not require that the claimed support is not misled or mistaken.\nolinebreak
    \footnote{
      Possibly goes against externalism, but I don't think this is right.
      External circumstances may impact the support the agent has.
      However, as these are external, it seems this condition plausibly holds for \emph{claiming} support.
      This is how you get puzzles for externalism.
      In both cases, it's fine for the agent to claim support, but the external circumstances impact whether the agent \emph{has} support.
      The internalist/externalist divide would seem to affect the conditions on claiming.

      Way to expand on this is reconstructing bootstrapping examples with and without \eiS{}.
      If the agent would only get basic support if reliable, then it's not clear that bootstrapping is a problem.
    }\(^{,}\)\nolinebreak
    \footnote{
      One way independence.
      Not clear that value is independent of support.
      So long as sufficiently strong support, not possible for proposition to have value other than claimed support.
    }
  \end{proposition}
  \eiS{} follows from Propositions~\ref{prop:CSNMORM} and~\ref{prop:fallibility}.

  From Proposition~\ref{prop:CSNMORM}, deemed that claimed support is neither mistaken nor misled.
  From Proposition~\ref{prop:fallibility}, always possibility.

  Suppose depends on value of \(\phi\).
  Still, from Proposition~\ref{prop:fallibility}, (epistemic) possibility that value of \(\phi\) differs.
  If differs, then claim to support is misleading.
  By Proposition~\ref{prop:CSNMORM} agent deems not mistaken or misled, and so deems that possible defeaters do not obtain.
  However, requirement of \(\phi\) denies the relevant possibility.
  For, if different value of \(\phi\) then no claim to support.
  Therefore, agent does not deem the claimed support not misleading.


  For, possibility that \(\phi\) does not have value, or that claimed support does not indicate.
  And, given claim to support requires that the possibility does not obtain, agent has not deemed that the claimed support is not misled or mistaken --- rather the agent requires that the claimed support is not misled or mistaken.

  \eiS{} does not deny that things may need to be a certain way for an agent to claim, or to be in a position to, claim support.
  It may be the case that no agent would be in a position to claim support that the speed of light is constant if the speed of light were not constant, but in claiming support an agent must deem that possible defeaters do not obtain, e.g.\ that the laws of nature are constant, and that no mistakes have been made when observing relevant phenomena.

  The force of the corollary is that agent does not require \(\phi\) related things being a certain way in order to claim support for \(\phi\).

  See with failures of `even if\dots' test.
\end{note}

\begin{note}[Quick clarification on \eiS{}]
  A quick clarification may be in order.

  \eiS{} is only about value/support for\(\phi\).
  So, \eiS{} does not prevent agent from claiming support for \(\psi\) from value of \(\xi\) given claimed support that \(\xi\) has that value.
  For example, an agent may claim support that \emph{p} is true from claimed support that \emph{S} knows \emph{p}.
  And, the agent may do so because the proposition that \emph{S} knows \emph{p} is true only if \emph{p} is true.
  That is, so long as the agent does not require \emph{p} to be true in order to claim support for the proposition that \emph{S} knows \emph{p} is true.
  We will return to \eiS{}, expand on this quick clarification, and note related observations in Section~\ref{sec:second-conditional}.
\end{note}


\begin{note}[Adequate reasoning]
  Term this \emph{adequate} reasoning.
  May be good, may involve mistakes, may be bad.
  Kind of reasoning that we, the folk, do.
  Distinction for claiming support is that this is different from whether the agent has support, and we may set issues about whether the agent has support.

  Our interest is what is required for an agent to \emph{claim} support for (premises and) steps of reasoning, rather than what is required for an agent to \emph{have} support for (premises and) steps of reasoning.

  Use support as opposed to justification.
  Initial focus is on epistemic/doxastic attitudes.
  However, practical reasoning.
  For example, means-end.
  Support considered quite general to also include this.
\end{note}




\begin{note}
  To help fix intuition, I suggest a (hypothetical) test to clarify what is meant by `deem': The `even if\dots' test.
  So long as an agent may provide an adequate responses to the test, the agent will be in a position to claim support.
\end{note}

\begin{note}[The `Even if\dots' test]
  The `Even if\dots' test queries whether an agent's claimed support permits an agent to expect that some (epistemically) possible defeater fails to obtain `even if' it does obtain.

  For example, even if \(0.999\dots = 1\), there must be \emph{some} difference between \(0.999\dots\) and \(1\) --- no matter how small --- and some difference between to things is sufficient to establish that they are not equal.

  Implied in this response is something like the observation that \(0.9 = (1 - 0.1)\) and \(0.99 = (1 - 0.01)\), and so \(0.999\dots = (1 - 0.000\dots 1)\), hence \(1 = (0.999\dots + 0.000\dots 1)\), and because \(0.999\dots\) refers to some quantity, \(0.000\dots 1\) likewise refers to some quantity.
  It seems reasonable for an agent to expect that the Archimedean property does not hold for real numbers.

  The example given is an instance of the applied to the possibility that the agent's claimed support that \(0.999\dots \ne 1\) may be misleading, as the antecedent supposes that \(0.999\dots = 1\).

  Generalising, we have outlined two kinds of defeaters that would prevent an agent from claiming support.
  The two types of defeaters suggest two basic instances of the test:
  \begin{enumerate}
  \item[(ML)] Even if \(\phi\) does not have value my claimed support indicates, I deem it to be the case that\dots
  \item[(MT)] Even if I some part (or whole) of my claimed support for the value of \(\phi\) is mistaken, I deem it to be the case that\dots
  \end{enumerate}
  Below we provide three examples for each basic instance of the test, two (plausibly) successful responses and one (plausibly) unsuccessful response..\nolinebreak
  \footnote{
    You may think that some of the adequate responses I suggest are too weak, but for future purposes I require only that some positive answer many be given, and so you may strengthen the requirements on a positive answer as you see fit.
  }
\end{note}

\begin{note}[Even if: misled]

  We being with two plausibly satisfactory responses to being misled.

  \begin{enumerate}[label=(ML\arabic*), ref=(ML\arabic*), series=ML_counter]
  \item\label{ML:asleep} Even if that person is not sleeping, their eyes have been closed for a long time and their breathing is slow.
  \item\label{ML:lying} Even if you are telling the truth, the scientific consensus is against you.
  \end{enumerate}

  With \ref{ML:asleep} the agent has claimed support for the proposition that the person is sleeping.
  It's not too hard to give the impression of being asleep, so there is some possibility that the person is awake and support claimed is misleading.
  Still, even if the person is awake, the person is exhibiting sufficient signs of being asleep for the agent to expect that they are not misled.

  Turning to \ref{ML:lying}, it may be that the person is telling the truth and if the person is indeed telling the truth then any claimed support for a conflicting proposition must be mistaken.
  However, scientific consensus seems sufficient to claim support for the relevant conflicting proposition --- one expects that scientific consensus is not misleading given the rigours of the scientific process.
  Scientific consensus does not (at least typically) require that the person is not telling the truth (though will imply that to be the case).

  In contrast, consider an unsatisfactory response.

  \begin{enumerate}[label=(ML\arabic*), ref=(ML\arabic*), resume*=ML_counter]
  \item\label{ML:forgery} Even if this certificate is a forgery, it professes to be the real thing.
  \end{enumerate}
  If the certificate is a forgery, then the claimed support for the proposition that the certificate is not a forgery would be misleading.

  The response to the (epistemic) possibility that the certificate is a forgery is unsatisfactory because the agent depends on the certificate not being a forgery.
  Hence, that the certificate self-certifies it's authenticity is no response to the (possibility) that it is a forgery.
  Immediate conflict with \eiS{}, and traces back to \autoref{prop:CSNMORM} because it seems quite unreasonable to expect that the certificate is not a forgery based on it's self-certification.

  Of course, many certificates do self-certify (it would be excessive effort to identify a certificate and then be required to find information about what the certificate is for), and perhaps a simple observation that there are no signs of tampering may be a sufficient response to the `even if\dots' test.
\end{note}

\begin{note}[Even if: mistaken]
  Turning to the possibility of mistaken support, consider the following two instances of the `even if\dots' test.

  \begin{enumerate}[label=(MT\arabic*), ref=(MT\arabic*), series=MT_counter]
  \item\label{MT:fake-wound} Even if that is a fake wound, I have no way to tell and the actions of the (apparently) wounded would be a feat of acting.
  \item\label{MT:misquote} Even if the newspaper has quoted the wrong person, the paper has a strong record of accurate reporting.
  \end{enumerate}

  With respect to~\ref{MT:fake-wound}, it seems a mistake to treat a fake wound as indicate the presence of an actual wound, as a fake wound does not require a genuine would but likewise a fake wound may cover a genuine wound.
  The response to the `Even if\dots' test notes that the behaviour of the (apparently) wounded person is sufficiently consistent with their expectations of the behaviour of a person with the (apparent) wound, and would lead to be surprised if the person was not in fact wounded.

  Turning to~\ref{MT:misquote}, if the paper has quoted the wrong person then it would be a mistake to claim support that the person said whatever-it-is-they-said, though it may still be the case that the person did say whatever-it-is-they-said.
  Even so, the strong record of the paper seems sufficient for the agent to expect that the newspaper has not misattributed or imagined the quote on the relevant occasion.

  In contrast, consider an unsatisfactory response.

  \begin{enumerate}[label=(MT\arabic*), ref=(MT\arabic*), resume*=MT_counter]
  \item Even if this library does not using LCC indexing, the library does not have a copy of `Measurement Theory' because as search for `H61 .R593' returns no results.
  \end{enumerate}
  Holding that a library does not have a copy of a book because a search for the book under a particular indexing system would be a mistake.
  For, if the library does not use the particular indexing system then a search using that indexing system will always fail, regardless of whether or not the library has a copy of the book.

  In turn, a failed search for an LCC index in the library's database does not seems sufficient for an agent to claim that the library does not have a copy of the book unless the agent is in a position to claim support that the library uses LCC indexing.
  Following, it seems the failed response to the `Even if\dots' test may be supplemented by noting that the library is a research library, and therefore likely uses LCC indexing, etc.\
\end{note}

\begin{note}[Even if: more]
  Primary observation from these examples is that in positive cases provided responses indicate that some response to possibility of being misled or mistaken is available to the agent.
  In the failure cases, no response.

  % Cases of entailment, preface paradox.
  % Mistake somewhere.
  % Here, support is good for each of the claims made in the preface, but these do not combine to make a case that no mistake has been made across any of the claims.
  % May come down to familiar concerns, too significant possibility of being misled.
  % May also think that claimed support for each might require a mistake in one.
  % I.e. source for claim includes further claims which state that source for some other claim is mistaken.
  % Problem with using both sources, even if for distinct propositions.

  % Interesting problem later.
  % For now, simple example.
  An interesting case for misled is the preface paradox.
  Claimed support for everything in the preface, but also claimed support for mistake.
  Credence resolves tension, remains noteworthy that even if confident of some potential defeater, claimed support is sufficient to resist undermining claimed support.

\end{note}

\begin{note}[???]
  \color{red}
  Possible defeaters, so no claim that the reasoning is sound.
  However, agent deems that no defeaters, so may term this `\emph{subjectively} sound' reasoning.

  At least two worries.
  First, given general use of the term `support', considerations may suggest iterated support.
  Second, worries about over-intellectualisation of claiming support.
\end{note}


\begin{note}[Closing support]
  To summarise, claim of support.
  Certain kind of independence.
  Only interested in support, and not how this relates to attitudes.
  Somewhat intuitive, but no claims that this is the only understanding of support.

  For the moment, this provides clarity for understanding of support.
  Below, use to argue for failure to claim support.
\end{note}

\subsection{Claiming support, reasoning, and ability}
\label{sec:inter-with-claim}

\begin{note}
  In this section we introduce two propositions which characterise what we are arguing against and what we are arguing for.
  \ESU{-} and \EAS{-}, respectively.
  Argue against \ESU{} by cases involving ability.
  Argue for \EAS{} which outlines the way in which ability conflicts with \ESU{}.

  Start with introduction of \ESU{}.
  And, motivate with reference to literature on the basing relation and rationality as responding to reasons.

  Move to \EAS{}, clarify relation to \ESU{} and contrast to related principle argued for by \citeauthor{Moretti:2019wx}.
\end{note}

\subsubsection{\ESU{}}
\label{sec:esu}

\begin{note}[Focus]
  We will argue against the converse of~\USE{}:

  \begin{proposition}[\ESU{-} --- \ESU{}]\label{denied-claim}
    An agent may claim support for some conclusion of reasoning by claiming that the conclusion of reasoning is supported by premises and steps of reasoning \emph{only if} the agent has witnessed the reasoning (e.g.\ traced the claimed support for those premises and steps used to claim support for the conclusion).\nolinebreak
      \footnote{
    Three brief notes on~\ESU{}:

    First, the `has' in~\ESU{} only requires `at some point in the past'.
    Hence,~\ESU{} does not require the agent to reason from premises to conclusion each time the agent claims support for the conclusion.
    For example, if an agent proved the Deduction Theorem for propositional logic last week, then the agent would not be in conflict with~\ESU{} if they claimed support for the Deduction Theorem on the basis of the premises and reasoning they performed in the past.

    Second, and following from the first,~\ESU{} will also hold for any stronger statement --- for example if `has' is read as `has just'.
    For example, requiring that the agent's memory of proving the Deduction Theorem allows the agent to claim support, rather than the premises and steps used in the past.
    The argument (stated below) denies that, given certain information, the agent needs witnesses any reasoning in order to claim support for the result of witnessing the reasoning.

    Third, as~\ESU{} is about when an agent may \emph{claim} support, it is compatible with~\ESU{} to hold that the agent \emph{has} support --- regardless of whether the agent has witnessing the reasoning.
  }
  \end{proposition}

  \ESU{}, as the converse of~\USE{} focuses on reasoning, and our focus will be with~\ESU{} because our interest is with reasoning.
  However, the key claims of~\ESU{} are independent of reasoning.
  To clarify,~\gESU{} is a generalisation of~\ESU{}.

  \begin{proposition}[\gESU{}]
    An agent may claim support for some proposition \(\phi\) by appealing to some materia\nolinebreak
    \footnote{Latin.
      Material, matter, basis, information, foundation, ground, etc.
    }
    \emph{M} only if the agent uses \emph{M} when claiming support for \(\phi\).
  \end{proposition}
  Our focus is with whether an agent is required to have \emph{used} something in order to appeal to that thing when claiming support.
  No fixed understanding of `use' is assumed in the statement of~\ESU{} and~\gESU{}, and we will offer some disambiguation below.
  First, a basic illustration.
\end{note}

\begin{note}[Simplest]
  \color{red}
  Difference between \(\phi\) therefore \(\psi\) and \(\phi\) and \(\phi \rightarrow \psi\) therefore \(\psi\).
  Possible for agent to reason from \(\phi\) to \(\psi\), so in principle possible for agent to claim support for \(\psi\) from \(\phi\).
  However, \ESU{} denies this if the agent doesn't do the reasoning.
  Instead, agent also needs \(\phi \rightarrow \psi\), and then they're fine.

  Key point of the suggested revision is that \ESU{} doesn't need to focus on claiming support for \(\phi\), specifically.
  Rather, it's just about establishing a relation of support between \(\phi\) and \(\psi\).

  Then, big idea is that ability is not understood as an instance of \(\phi \rightarrow \psi\), which it might otherwise seem to be.

  Indeed, viewed from the perspective of propositional logic, deduction theorem.
  If \(\vdash \phi \rightarrow \psi\) then \(\phi \vdash \psi\).
  \ESU{} denies that the same holds for claimed support.
  Seems quite sensible.
\end{note}

\begin{note}[Illustration]
  To illustrate~\ESU{}, consider the illustration provided for~\USE{}.

    If the agent did not measure the box, understand how to calculate the area of a box, or perform the arithmetic, the agent would not be in a position to claim support that area of the box is \(133\text{cm}^{2}\).
  A lucky guess that the area of the box is \(133\text{cm}^{2}\) would not allow the agent to hold that the area of the box is  \(133\text{cm}^{2}\) on the basis of the dimensions of the box, the agent's understanding of how to calculate the area of a box, and arithmetic.
  And, it seems the agent is not in a position to base their lucky guess in such a way because the agent did not reason from the dimensions of the box, the agent's understanding of how to calculate the area of a box, and arithmetic.\nolinebreak
  \footnote{
    Moving to another agent, observe doing the work, get report.
    Easy to resist, by adding in additional premise.
    Still, no presupposing that this needs to be done.
  }
  Similarly, if an agent learns that a box with dimensions of \(19\text{cm}\) by \(7\text{cm}\) may be calculated to have an area of \(133\text{cm}^{2}\), then the agent may not claim support for the area of the box on the basis of the calculation.
  If the agent has not performed the calculation, then the agent may not appeal to the use of the calculation when claiming support --- rather, the agent mentions that the calculation is true.\nolinebreak
  \footnote{
    Slight weakening of~\ESU{} may be made.
    So long as \emph{some} agent has performed the calculation.
    Argue against~\ESU{}, and the argument made will hold for this weakening.
  }
\end{note}

\paragraph{Intuition}

\begin{note}[Intuition]
  \ESU{} and~\gESU{} seems quite plausible, at least to me.
  The proposition is a careful statement of an intuitive ideas:

  Whether or not an agent claims support is the result of the structure of the reasoning process, and if some premises or step is not used, then it is irrelevant to the structure of the process.
  Hence, the only premises and steps of interest when claiming support are those used in the reasoning process.

  Rests on the broader idea from~\gESU{}.
  Claiming support is the result of some agentive process, and the result of an agentive process is explained by the constituents of the process.\nolinebreak
  \footnote{
    Ah, the homonculus.

    Question about whether the agent is important.

    This gets difficult.

    Consider clocks.
    Clock does not keep track of time.
    Rather, mechanical system designed to change in constant with some passage of time. (Cf.\ \textcite{Smith:1988aa}.)

    Agent may be like this.
    Distinction is intentionality.
    When I go about keeping track of the time, I'm attempting (at least typically) to maintain reference to what the time is.
    Figure out a way to approximate a second, and that's what's happening.
    Approximation.
    If it is noted that I requarly sigh every minute, use this, but I wouldn't be keep tracking of time, though you may be using regularity to do so.
    So, in the former case, using understanding of time, while in the latter not doing so.
  }

  As~\gESU{} is restricted to an agent claiming support, things seem a little easier.
  Problems with interpretation, however.
  Transparency.
  Familiar, if debatable, illustration.
  Freud.
  (Here, adjourning the meeting by saying something mistaken.)
\end{note}

\begin{note}[Analogy]
  By analogy, whether or not my mug of (once cold) coffee overheats in the microwave is the result of some process involving electromagnetic radiation.
  My desire that the mug of coffee does not overheat is not used as part of the process of heating the coffee, and so is irrelevant to the structure of the process.

  My desire may explain why the mug of coffee is taking part in a certain process, and an unused premise or step may explain why an agent performed so reasoning.
  Still, a premise or step must be used as part of the process of reasoning to stand in explanation for the result of reasoning.

  Press the analogy further: Reasoning is a causal process.
  And, any property of reasoning reduces to cause and effect.
  If premises or steps are not used, then those premises or steps stands outside the relevant causal trace, and may not be appealed to when accounting for some structural property of the conclusion of the instance of reasoning (here, that the agent claims support for the conclusion).
\end{note}

\paragraph{\ESU{} in the wild}

\begin{note}
  \color{red}
  Given proposed revision to \ESU{} this section should be expanded a little.
  For, most of the cases talk about claiming support for \(\phi\) directly, while \ESU{} is more general in that it talks about claiming support for any entailment between \(\phi\) and \(\psi\).
\end{note}

\begin{note}[Causal theories of basing]
  Indeed,~\ESU{} seems to be implied by various causal theories of basing.

  \citeauthor{Pollock:1999tm} introduce the basing relation with the following observation:
  \begin{quote}
    To be justified in believing something it is not sufficient merely to \emph{have} a good reason for believing it.
    One could have a good reason at one's disposal but never make the connection.
    \dots
    Surely, you are not justified in believing [something], despite the fact that you have impeccable reasons for it at your disposal.
    What is lacking is that you do not believe the conclusion on the basis of those reasons.\nolinebreak
    \mbox{}\hfill\mbox{(\cite[35]{Pollock:1999tm})}
  \end{quote}
  The observation falls short of being an account of the basing relation, but the intuition \citeauthor{Pollock:1999tm} appeal to is instructive.
  It seems that an agent must connect reasons and the content of a belief in order for the belief to be formed on the basis of those reasons, and hence be justified by those reasons.
  In turn, if a connection is made between reasons and the content of belief, then those reasons are used by the agent.

  For a concrete instance, consider \citeauthor{Moser:1989tv}'s account of the basing relation:
  \begin{quote}
    \emph{S}'s believing or assenting to \emph{P} is based on his justifying propositional reason \emph{Q} \(=_{\text{df}}\) \emph{S}'s believing or assenting to \emph{P} is causally sustained in a nondeviant manner by his believing or assenting to \emph{Q}, and by his associating \emph{P} and \emph{Q}.\nolinebreak
    \mbox{}\hfill\mbox{(\cite*[157]{Moser:1989tv})}
  \end{quote}

  Suppose we have a conclusion from some premises and steps of reasoning.
  If the agent has not witnessed the relevant reasoning, then it seems the conclusion is not causally sustained in a nondeviant manner by his believing or assenting to the premises of the reasoning, nor has the agent associated the conclusion with the premises by witnessing the relevant steps of reasoning.

  To illustrate, claim support that 173 is prime.
  It's possible that I did the prime factorisation, and possible that I took that representation to be part of the reason why I claim that 173 is prime.
  However, represented query of whether prime to wolfram alpha as justifying, and that's why I claimed support.
  So, definitely not from okay to appeal to the reasoning I have not witnessed.
  And, if infer that 173 is prime from claimed support that I have the ability to demonstrate that 173 is prime, the same issue.
  As I've not witnessed, then no role for \emph{Q}, whatever that turns out to be.

  This is a quick argument, and borders on conjecture, so let us examine the relevant association in greater detail.
  \citeauthor{Moser:1989tv} distinguishes between occurrent and non-occurrent satisfaction of association relations.

  We start with occurrent satisfaction of an association relation:
  \begin{quote}
    \emph{S} occurrently satisfies an association relation between \emph{E} and \emph{P} \(=_{\text{df}}\)
    \begin{enumerate*}[label=(\roman*), ref=(\roman*)]
    \item\label{moser:oar:i} S has a \emph{de re} awareness of \emph{E}'s supporting \emph{P}, and
    \item\label{moser:oar:ii} as a nondeviant result of this awareness, \emph{S} is in a dispositional state whereby if he were to focus his attention only on his evidence for \emph{P} (while all else remained the same), he would focus his attention on \emph{E}.\newline
    \mbox{}\hfill\mbox{(\citeyear[141--142]{Moser:1989tv})}
    \end{enumerate*}
  \end{quote}

  \emph{de re} awareness is a non-propositional, direct awareness of \emph{E} supporting \emph{P}.

  \ESU{} follows from~\ref{moser:oar:i}.
  \emph{de re} awareness, but this doesn't rule out use.
  \ESU{} does not require that the agent engages in propositional reasoning.

  In cases where the agent has not witnessed reasoning, there is no \emph{de re} awareness.
  Without the reasoning taking place, the agent is not directly aware of what the reasoning consists of.

  Following, the definition of non-occurrent satisfaction of an association relations is derived from occurrent satisfaction of an association relations by allowing~\ref{moser:oar:i} to be satisfied at some point in the past while requiring that~\ref{moser:oar:ii} continues to be satisfied in the present.
  As noted, \ESU{} is compatible with the agent having witnessed the reasoning at some point in the past.
  Therefore, \ESU{} is entailed given both occurrent and non-occurrent satisfaction of association relations
\end{note}

{
  \color{red}
  This doesn't make sense\dots
  I think the idea I had was that the agent has to use the represented relation.
  Hm, so, the idea is that in the case of \(\phi \vdash \psi\), the agent hasn't represented how to get from \(\phi\) to \(\psi\), and therefore the agent isn't allowed to base \(C\) or \(R\) given \citeauthor{Neta:2019aa}'s account.
  I don't think this is sufficiently clear from what I have written.
  However, it does seem relevant.
  And, in also, basing doesn't necessarily need to be between beliefs.
  This could just be a relation of justification\dots though this isn't necessarily the case.
  So care is need.
  Still, with a little rewriting this looks useful.

  The tricky part is understanding what it is to represent R as justifying C.
  What I need is the idea expressed above, that the relevant representation is sufficiently detailed.
  I think this should be in \citeauthor{Neta:2019aa}.
  For, intuitively representing R as \emph{justifying} C is stronger than a representation with the content that R justifies C.
}

\begin{note}[Representationalism]
  \citeauthor{Neta:2019aa} generalises (purely) epistemic interest in basing relations to cover the explanatory relation between reasons and (rationally evaluable) states held, or actions performed, by an agent.

  On the way to a novel proposal, \citeauthor{Neta:2019aa} sketches a broad characterisation of representationalist theories of (generalised) basing:
  \begin{quote}
    \begin{enumerate}[label=(R\arabic*), ref=(R\arabic*)]
    \item\label{neta:RC:b} \emph{basing} C on R involves the agent's representing R as justifying C, and
    \item\label{neta:RC:jb} \emph{justifying basing} of C on R consists in the adroitness of this representation.\nolinebreak
          \mbox{}\hfill\mbox{(\citeyear[192]{Neta:2019aa})}
    \end{enumerate}
  \end{quote}
  As \ESU{} does not distinguish between successful and unscuccesful instances of claiming support, our interest is with~\ref{neta:RC:b}.
  And, in contrast to \citeauthor{Moser:1989tv}, a representationalist theory may lack a causal component.
  Indeed, \citeauthor{Neta:2019aa} considers scenario in which an agent receives information from some source, forms a belief which is supported by the received information, and represents the received information as justifying the belief.
  The twist, however, is that the agent forming the belief was caused by some other source.
  For example, an agent may listen to a speech given by a talented orator and form a belief in response to the speech.
  The agent may represent the content of the speech as justifying the conclusion, while the cause of the belief being formed is the emotional impact with which the orator stated the conclusion.
  Following the representationalist characterisation, the agent would base the content of the belief on the content of the speech rather than the emotional impact with which the speech concluded.
  Indeed, the agent may do so even if they recognise that they were swayed by emotion.

  As before, consider a conclusion of some reasoning that the agent has not witnessed.
  If the agent has not witnessed the reasoning, then the agent has not represented some or all of the relevant premises and steps of reasoning.
  Therefore, it seems that it is not possible for the agent to represent the premises and steps of reasoning as justifying the relevant conclusion.
  In other words, a representationalist account requires (minimally) that an agent represents premises and steps of reasoning as justifying when claiming support for some conclusion of reasoning, and hence use of those premises and steps.

  {
    \color{red}
    What is going on here\dots
    The point is that if we follow \citeauthor{Neta:2019aa} then there needs to be a representation.
    In turn, the issue is that it's not clear that the agent needs to reason from \(\psi\) to \(\phi\) in order to obtain the relevant representation.
    So, it's not clear that \citeauthor{Neta:2019aa} actually is relevant.

    So, it's this previous paragraph that needs attention.
    No use, then no representation.
    This is the only point that really matters.
    So, I need to find something in \citeauthor{Neta:2019aa} that supports this, or somehow provide a much better argument.

    Then, in the following paragraph is redundant.
    The issue is with how the relevant representation is obtained.
    The part where I'm getting confused is that \citeauthor{Neta:2019aa} doesn't hold that the agent needs to do the reasoning each time the representation is used.
  }

  As an aside, it is not clear whether representing an entailment or inference is the same as reasoning with an entailment, and therefore it does not seem to follow from the representationalist characterisation that the agent must witness the relevant reasoning.
  However, the interpretation of `use' is intended to be sufficiently broad to cover such cases.\nolinebreak
  \footnote{
    Alternatively, a clause may be added to~\ESU{} which denies that the agent represents the relevant premises and steps of reasoning.
    The argument made against~\ESU{} is compatible with the use of representations, or mere representation even if unused --- though it is unclear to me what an unused but represented premise or step would matter when claiming support.
  }

  Further, \citeauthor{Neta:2019aa}'s discussion is instructive because the response \citeauthor{Neta:2019aa} offers to some problematic scenarios focus on \emph{how} a representation is used.
  One may hold that the agent in the example given did not base their belief in the conclusion on the content of the speech in view of the fact that the agent was swayed by emotion.
  If so, \citeauthor{Neta:2019aa} proposes the following revision:
  \begin{quote}
    \begin{enumerate}[label=(R\arabic*\('\)), ref=(R\arabic*\('\))]
    \item\label{neta:RC:jp} for an agent to C based on reason R involves not merely the agent's representing R as justifying C---it also involves \emph{this latter representation (or its content) being part of the reason why the agent C's}.\nolinebreak
      \mbox{}\hfill\mbox{(\citeyear[197]{Neta:2019aa})}
    \end{enumerate}
  \end{quote}
  The added clause states that the relevant representation must explain why the agent formed a belief.
  Hence, given~\ref{neta:RC:jp} the agent would not be permitted to base their belief in the content of the speech given that they were swayed by emotion.
  Intuitively,~\ref{neta:RC:jp} expands on what it is for premises and steps of reasoning to be use when forming a belief.
  So, given that representation requires use, the expanded clause may be seen as focusing on \emph{how} the representation is used.
\end{note}

\begin{note}[Responding to reasons]
  As final motivation, consider the proposal at the core of \citeauthor{Lord:2018aa}'s (\citeyear{Lord:2018aa}) thesis that being rational is to correctly respond to reasons.

  \begin{quote}
    \textbf{Correctly Responding:} What it is for A's \(\phi\)-ing to be ex post rational is for A to possess sufficient reason S to \(\phi\) and for A's \(\phi\)-ing to be a manifestation of knowledge about how to use S as sufficient reason to \(\phi\).\nolinebreak
    \mbox{}\hfill\mbox{(\citeyear[143]{Lord:2018aa})}
  \end{quote}

  An agent's action is rational only if the action is a manifestation of some know-how.
  \citeauthor{Lord:2018aa} summaries:

  \begin{quote}
    \dots when one manifests one's know-how, dispositions that are directly sensitive to normative facts are manifesting. Thus, the competences involved in the relevant know-how make one directly sensitive to the normative facts\nolinebreak
    \mbox{}\hfill\mbox{(\citeyear[16]{Lord:2018aa})}
  \end{quote}

  For our purposes, following example of manifesting know-how directly relates to reasoning:

  \begin{quote}
    The most salient disposition [when appealing to \emph{p} as a reason]\nolinebreak
    \footnote{Note, \citeauthor{Lord:2018aa} (explicitly) not talking about believing that \emph{p} is a reason, but argues that the cited disposition to present both when appealing to p as a reason and believing that \emph{p} is a reason.}
    is the disposition to (competently) use \emph{p} as a premise in reasoning.\nolinebreak
    \mbox{}\hfill\mbox{(\citeyear[25]{Lord:2018aa})}
  \end{quote}

  Hence, suppose an agent appeals to a premise of reasoning in order to claim support for some conclusion.
  Then, if the agent does not use the premise of reasoning, it seems the agent does not manifest know-how, which is required for the appeal to meet \citeauthor{Lord:2018aa}'s account of rational action.

  Of course, that the noted disposition is the most salient does not rule out alternative, less noteworthy, dispositions.
  However, it is unclear to me how to \emph{manifest} know-how without use.
  Looking ahead, it does not seem to be the case that I manifest my ability to show that a certain rule of inference is sound when skipping over details in a completeness proof.
  However, I may manifest know-how regarding the (presumed) truth of the ability attribution.

  Likewise with my ability to establish a preference for tofu over any other kind of miso when ordering soup.
\end{note}

\begin{note}[Summarising illustrations]
  Three examples of claiming or establishing relations of support have been given.
  Each example suggests that if an agent does not use a premises or steps when claiming support, then an agent may not claim support by appeal to the unused premises or steps.

  Stepping back,~\ESU{} may be seen as a desiderata for any account of (successfully) claiming support.
  For:
  If an agent (successfully) claims support for some conclusion of reasoning, then the premises and steps used with respect to that claim of support is itself the result of some reasoning --- the reasoning that culminated with the claim to support itself used premises and steps of reasoning.
  So, given that the agent used certain premises and steps when claiming support for conclusion, some property of the premises and steps used, an adequate account of claiming support must explain how the premises and steps used permit the agent to claim support.\nolinebreak
  \footnote{
    Note, however, that this argument does not imply that support for the conclusion must be accounted for in terms of the premises and steps used by the agent to claim support.
    As we will note below, one may hold that an enthymematic argument permits an agent to claim support, while the relevant relation of support is secured by the corresponding non-enthymematic argument.
    Cf.\ \textcite{Moretti:2019wx} for suggestions along these lines.
  }
  In turn, if an agent appeals to premises and steps that they did not use, then those premises and steps must be redundant.

  Turning to ability.
  Suppose and agent appeals to
  \begin{enumerate*}
  \item their ability to demonstrate that \(\phi\) is the case, and
  \item that \(\phi\) must be the case in order for the agent to have the ability to demonstrate that \(\phi\)
  \end{enumerate*}
  in order to claim support for \(\phi\).
  Then, the premises and steps involved in a full account of reasoning from the two claims must be sufficient to claim support that \(\phi\) is the case.
  So, as the agent does not witness their ability to demonstrate that \(\phi\) in such reasoning, it must be the case that claimed support for (the property of) having the ability to demonstrate that \(\phi\) is sufficient for such reasoning.
\end{note}

\subsubsection{\EAS{}}
\label{sec:eas}

{
  \color{red}
  Perhaps include a note about how the argument relates to \EAS{}.
  I don't provide a direct argument, but this is the best way I see of resolving the tension.
}

\begin{note}[Alternative]
  \ESU{} is a universal claim, and so applies to all instances in which an agent may claim support for conclusion on basis of support for premises and steps of reasoning --- an agent may only claim support if the agent reasoned from the premises via the steps to the conclusion.

  Our goal is to motivate the following exception to \ESU{}:
  \begin{proposition}[\EAS{-} --- \EAS{}]\label{prop:EAS}
    If an agent has claimed support that they have the ability to (adequately) reason to some conclusion, then the agent may claim support for the conclusion by claiming support for the premises and steps of reasoning that the agent would use to witness their ability to reason to the conclusion.
  \end{proposition}

  Loosely restated,~\EAS{} holds that if an agent may claim support for having the ability to witness some reasoning, and is aware of the conclusion of that reasoning, then the way in which the agent claims support for the conclusion of that reasoning may mirror the way in which the agent would claim support for the conclusion by witnessing the reasoning (and hence using the relevant premises and steps).

  The (possible) event of the agent witnessing their ability to demonstrate \(\phi\) involves reasoning with various premises and steps which culminate in claiming support for \(\phi\).
  So, if~\EAS{} is true, then the agent may appeal to those premises and steps which are used in the (possible) witnessing event.

  One way to think about~\EAS{} (which we will explore in more details later) is in terms of propositional support.
  For, if an agent has the ability to demonstrate that \(\phi\) is the case, then the agent has propositional support for \(\phi\) as there is a way for the agent to demonstrate that \(\phi\) is the case.
  In addition, that the agent has the ability to demonstrate that \(\phi\) is the case ensure that the agent is in a position to make use of the available propositional support for \(\phi\).
  In turn,~\EAS{} may be interpreted to hold that so long as the agent has such information about their position to make use of the available propositional support for \(\phi\) then the agent does not need to reason with the relevant propositional support in order to claim support for \(\phi\) in virtue of the available propositional support for \(\phi\).

  So, if~\EAS{} is true, then there are cases in which an agent is not required to reason from premises they may claim support for to some conclusion in order to obtain support for the conclusion on the basis the support the agent has for the premises.\nolinebreak
  \footnote{
    Stated~\EAS{} as an exception to~\ESU{}.
    And, we will argue that~\EAS{} is true.
    However, we will not argue that~\EAS{} \emph{is an exception} to~\ESU{}.
    To do so would require an argument that \ESU{} holds for other cases.
    Likewise, no argument that~\EAS{} is the only exception, as to do so would require argument that~\ESU{} holds for all other cases.
    Take~\ESU{} to be plausible, and suspect that there are few, if any, further exceptions, but~\EAS{} may stand independently on any further statements about claiming support.
  }
\end{note}

\begin{note}[\EAS{} illustration]
  To illustrate, suppose you provide me with `novel' information that:
  \begin{enumerate}[label=\emph{A}\arabic*., ref=(\emph{A}\arabic*), series=EAS_counter]
  \item\label{EAS:ex:box:if} If I have ability to calculate the area of a box, then I have the ability to demonstrate that a box with dimensions \(19\text{cm}\) by \(7\text{cm}\) has area \(133\text{cm}^{2}\).
  \end{enumerate}
  The information is `novel' because I have not been previously informed (in any way) about the area of a box with dimensions \(19\text{cm}\) by \(7\text{cm}\).

  Still, (I claim support that):
  \begin{enumerate}[label=\emph{A}\arabic*., ref=(\emph{A}\arabic*), resume*=EAS_counter]
  \item\label{EAS:ex:box:gen} I have the ability to calculate the area of a box
  \end{enumerate}
  Therefore:
  \begin{enumerate}[label=\emph{A}\arabic*., ref=(\emph{A}\arabic*), resume*=EAS_counter]
  \item\label{EAS:ex:box:spec} I have the ability to demonstrate that a box with dimensions \(19\text{cm}\) by \(7\text{cm}\) has area \(133\text{cm}^{2}\).
  \end{enumerate}
  From~\ref{EAS:ex:box:spec} it follows that:
  \begin{enumerate}[label=\emph{A}\arabic*., ref=(\emph{A}\arabic*), resume*=EAS_counter]
  \item\label{EAS:ex:box:fact} A box with dimensions \(19\text{cm}\) by \(7\text{cm}\) has area \(133\text{cm}^{2}\).
  \end{enumerate}
  \EAS{} holds that, when I claim support for~\ref{EAS:ex:box:fact} from~\ref{EAS:ex:box:spec}, I may appeal to dimensions and formula, though as I do not witness the ability, I do not use the premise and step.
  For, if~\ref{EAS:ex:box:spec} is the case then it is possible for me to witness reasoning in which I demonstrate that~\ref{EAS:ex:box:fact} is the case, and it is the premises and steps of reasoning used in such reasoning that establishes~\ref{EAS:ex:box:fact} is the case.
  I have not used those steps and premises, as I have not witnessed the relevant ability, but may I appeal to those steps and premises regardless --- or so we will argue.
\end{note}

\begin{note}[More detail]
  I do not expect \EAS{} to be intuitive.
  Indeed, we are not interested in \EAS{} because it is a more-or-less intuitive principle which conflicts the intuitive \ESU{}.
  Rather, we are interested in \EAS{} primarily because \EAS{} is a consequence of tension arising from three things:
  \begin{enumerate*}
  \item\label{incomp:tri:q:1} \ESU{}
  \item\label{incomp:tri:q:2} scenarios involving an agent reasoning with information about an their own ability,
  \item\label{incomp:tri:q:3} and a principle concerning when an agent is permitted to claim support
  \end{enumerate*}
  To briefly expand on~\ref{incomp:tri:q:2} and~\ref{incomp:tri:q:3}:
  Information that one has some specific ability so long as one has some general ability --- such as the (specific) ability to show that \(25^{\circ}\text{C} = 77^{\circ}\text{F}\) given the (general) ability to convert between Celsius and Fahrenheit.
  And, agent is never permitted to claim support for proposition having a certain value if the agent requires the proposition to have value \emph{in order to} claim support.
  (As an instance, an agent is not permitted to claim support for the truth of a proposition if the agent requires the proposition to be true \emph{in order to} claim support that the proposition is true.)\nolinebreak
  \footnote{
    The emphasis on `in order to' is important.
    The instance of the principle does not state that an agent is not permitted to claim support for the truth of a proposition if the agent requires the proposition to be true when claiming support that the proposition is true.
    I plausibly require that \(2 + 2 = 4\) when I claim support that \(2 + 2 = 4\), and this does not prevent me from claiming support by simple arithmetic.
    However, it would be impermissible (or so we will argue) to claim support that \(2 + 2 = 4\) by reasoning that the calculator is functional only if \(2 + 2 = 4\), and as the calculator states that \(2 + 2 = 4\) it is the case that \(2 + 2 = 4\).
  }
  The details matter, and we postpone detailing this argument to~\autoref{sec:broad-argum-overv}.

  In short, assuming the scenarios exist, there is tension between intuitive principles governing what an agent appeals to when reasoning and structural principles governing the relation between what the agent appeals to when reasoning.
\end{note}

\begin{note}
  For the moment we attempt to clarify \EAS{} to some degree.
  Three subsections follow:

  \begin{enumerate}
  \item We will outline alternative reasoning patterns from~\ref{EAS:ex:box:if} to~\ref{EAS:ex:box:fact}, clarify why we focus on a particular type of reasoning pattern, and examine some initial objections to~\EAS{} and canvas some responses.
  \item We will consider parallels between abilities and dispositions.
    The parallel will provide some additional intuition for why an agent may appeal to premises and steps that have no been used, and help further clarify our interest with ability.
  \item We will consider a related proposition argued for by \citeauthor{Moretti:2019wx} which holds that a belief need not be based (exclusively) on the premises and steps of reasoning used to arrive at the belief.
    The comparison will help highlight what is distinctive about~\EAS{} while at the same to introducing some ideas which suggest a way of understanding~\EAS{}.
  \end{enumerate}
\end{note}

\paragraph{Against \EAS{}}

\begin{note}[Alternatives]
  The alternative reasoning pattern we will focus on in some detail holds that appealing to having the ability noted in \ref{EAS:ex:box:spec} is sufficient to claim support for \ref{EAS:ex:box:fact}.
  In line with \ESU{}, the agent would use the proposition that they have the relevant ability noted in~\ref{EAS:ex:box:spec} to claim support for~\ref{EAS:ex:box:fact}
  This reasoning pattern, along with the pattern suggested by \EAS{} will be considered in \autoref{sec:wr-ar} and we will argue that it conflicts with an intuitive principle regarding claiming support in \ref{sec:second-conditional}.

  Alternatively, on may argue that though the syntactic form of \ref{EAS:ex:box:if} is a conditional, it does not (necessarily) follow that the semantic content of~\ref{EAS:ex:box:if} is a (also) conditional.
  And that~\ref{EAS:ex:box:if} may (plausibly) be interpreted to explicitly state that~\ref{EAS:ex:box:spec} is an ability that an agent may have.
  For example:
  \begin{enumerate}[label=\emph{A}\arabic*., ref=(\emph{A}\arabic*), resume*=EAS_counter]
  \item\label{EAS:ex:box:if:R:state} The ability to demonstrate that a box with dimensions \(19\text{cm}\) by \(7\text{cm}\) has area \(133\text{cm}^{2}\) is an ability an agent may have and it is an ability an agent has if they have ability to calculate the area of a box.
  \end{enumerate}
  Hence, \ref{EAS:ex:box:if} is interpreted so that \ref{EAS:ex:box:spec} is accessible without endorsing the antecedent of \ref{EAS:ex:box:if}.
  \ref{EAS:ex:box:if:R:state} states that there is some ability that it is possible for an agent to have, and in addition provides sufficient conditions for having the relevant ability.
  The important part of \ref{EAS:ex:box:if:R:state} is the former conjunct:
  \begin{enumerate}[label=\emph{A}\arabic*., ref=(\emph{A}\arabic*), resume*=EAS_counter]
  \item\label{EAS:ex:box:spec:R:state} The ability to demonstrate that a box with dimensions \(19\text{cm}\) by \(7\text{cm}\) has area \(133\text{cm}^{2}\) is an ability an agent may have.
  \end{enumerate}
  And, \ref{EAS:ex:box:fact} follows from \ref{EAS:ex:box:spec:R:state} by the observation that it is not possible to demonstrate that \(\phi\) if \(\phi\) is not the case --- there is no need for the agent to appeal to witnessing their ability.
  Therefore,~\ref{EAS:ex:box:spec:R:state} (and~\ref{EAS:ex:box:if:R:state}) implicitly includes information that~\ref{EAS:ex:box:fact} is the case --- an agent does not need to reason from~\ref{EAS:ex:box:spec} to~\ref{EAS:ex:box:fact}, because they have already been informed that~\ref{EAS:ex:box:fact} is the case.

  Note also that analogous reasoning applies if `I' is replaced with `an agent'.
  Likewise, if I know that you that you know that I have the ability to calculate the area of a box.
  For, it then follows that you know that \ref{EAS:ex:box:spec} is the case, and therefore you know that \ref{EAS:ex:box:fact} is the case.
  Again there is no need for me to appeal to witnessing an ability.
\end{note}

\begin{note}[Box]
  The existence of alternative reasoning patterns is the issue at hand.
  For, so long as there are reasoning patterns \emph{R} which conform to \ESU{} it is open to the defender of \ESU{} to hold that if an agent is permitted to claim support, then the agent is required to reason via some member of \emph{R}.
  For, if there are reasoning patterns \emph{R} which conform to \ESU{} then there a no counterexamples to \ESU{} --- scenarios in which an agent claims support by appeal to premises or steps of reasoning that the agent has not used.

  Of course, an argument against a general principle such as \ESU{} is not required to be a counterexample.
  For example, it may be possible to argue that the reasoning patterns \ESU{} requires are sufficiently implausible.
  Hence, a restricted variant of \ESU{} compatible with \EAS{} would to be preferred.
  However, there are two issues with attempting such an argument.

  First, given the intuitive plausibility to \ESU{}, it seems unlikely that any violation of \ESU{} would be more plausible than an alternative reasoning pattern compatible with \ESU{}.
  Second, even if there are plausible reasoning pattern that are incompatible with \ESU{}, it is not clear that these should be incorporated in a theory of claiming support.
  For, meta-theoretical issues such as complexity or predictive power may still favour \ESU{}.
  Following \citeauthor{Box:1987vn}: `\dots all models are wrong; the practical question is how wrong do they have to be to not be useful.' (\citeyear[74]{Box:1987vn})

  Indeed, the second point suggest a counterexample proper to show that \ESU{} is false is not necessarily an adequate argument against \ESU{} either.
  Observations in the spirit of \citeauthor{Box:1987vn} are trite, but also true.
  Even if \EAS{} is true and \ESU{} is false, would \EAS{} be useful?
\end{note}

\begin{note}[Responding to Box]
  With respect to idealised agents with unbounded resources, the answer appears to be no.
  For, with unbounded resources the agent the option of (attempting to) witnessing any ability (to reason) without cost.
  And, it seems that for such an agent witnessing a relevant ability would always be preferable to reasoning about an unwitnessed ability as the agent would minimally (subjectively) resolve any uncertainty about whether they have the ability.

  However, for limited agents, ability is abundant, while the resources required to witness abilities are scarce.
  That the exception to~\ESU{} is narrow does not entail that there are few occurrences of the exception.

  Information about ability may be abundant while the resources for witnessing abilities are either scarce or temporarily unavailable.
  So, for example, agent has the option of conserving or deferring use of resources.

  This observation suggests an initial line of response to an objection which focuses on whether \EAS{} would be useful.

  For, given that we are resource bound agents, it seems that possible instances of \EAS{} are widespread.
  From a functional perspective, reasoning with (the relevant instances of) ability just is reasoning about the result of expending available resources.
  Hence, if~\EAS{} is true, then the truth of \EAS{} would provide a novel perspective on resource bound agents.
  And, it is yet to be seen whether such a perspective is useful.

  In addition, there is a second indirect line of response.
  We observed above that \ESU{} seems prevalent in various theories which relate to reasoning, such as basing and responding to reasons.
  If \EAS{} is true, then there may be alternative conclusions to arguments that appeal to~\ESU{} as a premise.
  And, likewise, there may be interesting observations made in premises of arguments which establish \ESU{} as a foundation for further theorising.\nolinebreak
  \footnote{
    As an exception, even if~\EAS{}, conclusion of arguments which appeal to or assume \ESU{} may be restricted.
  }

  Taking stock:
  I doubt that \EAS{} is of interest if there are no reasoning patterns which require \EAS{} to be true.
  Still, if there are reasoning patterns which require \EAS{} to be true, then \EAS{} may be of interest.
  Further, I think there are good reasons to hold that there are reasoning patterns which require \EAS{} to be true.
  Hence, my goal is to motivate further research into whether \EAS{} is of interest.

  We now briefly turn to the type of scenarios which are a premise in our argument for the existence of such counterexamples.
\end{note}

\begin{note}[Types of scenario]
  The type of scenario we will focus on is designed to ensure that an agent is required to reason to (and from) information about a (specific) ability that they have.
  If the agent is required to reason \emph{to} (specific) ability information then rephrasing \ref{EAS:ex:box:if} as \ref{EAS:ex:box:if:R:state} will not be possible --- the agent will be required to reason from some premises by some steps to the (specific) ability information.
  And, as before, the type of scenario will preserve the requirement of the agent to reason \emph{from} the (specific) ability information in line with~\ref{EAS:ex:box:spec} and~\ref{EAS:ex:box:if:R:state}.
  Hence, by establishing such scenarios are possible we may restrict our attention to the steps from~\ref{EAS:ex:box:if} to~\ref{EAS:ex:box:spec} and from~\ref{EAS:ex:box:spec} to~\ref{EAS:ex:box:fact}.
\end{note}

\begin{note}
  To illustrate, let us add some context to the example scenario we've been focusing on.

  Suppose it is common knowledge between you and I that
  \begin{enumerate*}
  \item you have looked through my notes, and have applied my formula for calculating the area of a box, and
  \item my notes are the only source of information you have regarding how to calculate the area of a box.
  \end{enumerate*}
  We may now restate the semantic content of~\ref{EAS:ex:box:if} as follows:
  \begin{enumerate}[label=\emph{A}\arabic*., ref=(\emph{A}\arabic*), resume*=EAS_counter]
  \item\label{EAS:ex:box:inf:R} You have some general ability \(\gamma\), and a specific ability \(\varsigma\) (as an instance of that general ability).
    And, if \(\gamma\) is the ability to calculate the area of a box, then \(\varsigma\) is the ability to demonstrate that a box with dimensions \(19\text{cm}\) by \(7\text{cm}\) has area \(133\text{cm}^{2}\).
  \end{enumerate}
  The formula in my notes indicates that I have the ability to do something, and I have indicated what I think the appropriate characterisation of the ability is.
  Still, you are not in a position to offer information as to whether my characterisation of the ability is correct or not.\nolinebreak
  \footnote{
    Consider in reverse.
    One is often attributed abilities that I deny I have.
    For example, I do not have the ability to process information by means of mental images, but \citeauthor{Hume:2011aa} (arguably) holds that I do have such an ability.
    I lack the ability to reason in a particular way.
    (Not that \citeauthor{Hume:2011aa}'s arguments rest on visual as opposed to any other kind of imagination, but the point stands.)

    Similarly, you may claim that I have the ability to tell you whether or not \nagent{7} is coming to tea.
    However, and in contrast to your assumption, \nagent{7} has not replied to my invitation and so I lack a required premise in order to reason to a relevant conclusion.
  }
  The key feature of~\ref{EAS:ex:box:inf:R} it that I, the agent, am required to claim support that \(\gamma\) and \(\varsigma\) are the abilities of interest.
  The focus is not on whether or not an agent may perform some action.
  Rather, our interest is with what the action is.
  It is up to me, the agent, to claim support that:
  \begin{enumerate}[label=\emph{A}\arabic*., ref=(\emph{A}\arabic*), resume*=EAS_counter]
  \item\label{EAS:ex:box:gen:R} The general ability \(\gamma\) \emph{is} the ability to calculate the area of a box.
  \end{enumerate}
  If so, I may then claim support that:
  \begin{enumerate}[label=\emph{A}\arabic*., ref=(\emph{A}\arabic*), resume*=EAS_counter]
  \item\label{EAS:ex:box:spec:R} \(\varsigma\) is the specific ability to demonstrate that a box with dimensions \(19\text{cm}\) by \(7\text{cm}\) has area \(133\text{cm}^{2}\).
  \end{enumerate}
  Note, there is no route to \ref{EAS:ex:box:spec:R} other than by claiming support for~\ref{EAS:ex:box:gen:R} as I have no information about what the (specific) ability \(\varsigma\) is amounts to if \ref{EAS:ex:box:gen:R} is not the case.
  If \(\varsigma\) is some other ability, then~\ref{EAS:ex:box:fact} does not follow, witnessing the relevant ability would not demonstrate that a box with
  dimensions \(19\text{cm}\) by \(7\text{cm}\) has area \(133\text{cm}^{2}\), and so a box with dimensions \(19\text{cm}\) by \(7\text{cm}\) may (from my epistemic perspective) have some other area.
\end{note}

\begin{note}[Point]
  We will say more in \autoref{sec:cases-interest}.
  For the moment it is sufficient to observe that the agent is required to reason to and from specific ability.
  The scenario requires the agent to claim support by reasoning from~\ref{EAS:ex:box:if} to~\ref{EAS:ex:box:spec} and from~\ref{EAS:ex:box:spec}~to~\ref{EAS:ex:box:fact}.
  And, while the context added to force the reasoning pattern of interest is narrow, the principle behind the context is simple:
  The agent is required to claim support that they have the relevant general ability.
  Hence, any scenario which consists of \gsi{-} which requires the agent to claim support that they have the relevant general ability will require the same kind of reasoning pattern.

  Further, this arguably captures a general puzzle about ability.
  An agent is not required to have witnessed all instances of a general ability to claim support that they have the general ability.
  However, so long as the agent may claim support for having some general ability, then it follows that the agent will have the option of claiming support for each instance of the general ability.

  The primary issue, though, is whether there is an account of such reasoning that does not require \EAS{} to be true.
  We will shortly turn to this argument in \autoref{sec:broad-argum-overv}.
  Prior to doing so, we close this section with some further clarification on the motivation behind \EAS{}, what distinguishes \EAS{} from nearby principles, and a suggestion on how to conceptualise \EAS{}.
\end{note}

\paragraph{Ability and dispositions}

\begin{note}[Parallel]
  To further clarify the motivation for \EAS{} we introduce a parallel between abilities and dispositions.
  The primary function of the parallel will be to highlight the importance of reasoning about an event.
  In the case of dispositions the event is the manifestation of the disposition, and in the case of ability the event is the agent witnessing the ability.

  The parallel is of interest because \EAS{} concerns the premises and steps of reasoning that the agent would use to witness the relevant ability.
  We will suggest that claiming support that some object has some disposition and that some agent has some ability may both be understood in terms of claiming support that the relevant event is a possible event.

  In turn, if reasoning \emph{to} a specific ability is understood in terms of claiming support that it is possible for the agent to witness the event, then reasoning \emph{from} a specific ability may be understood in terms of claiming support from what would happen in the possible event.
  \end{note}

\begin{note}[Parallel between dispositions and ability]
  Consider \cite{Choi:2021wg}'s characterisation of the Simple Conditional Analysis of dispositions:
  \begin{quote}
    An object is disposed to \emph{M} when \emph{C} iff it would \emph{M} if it were the case that \emph{C}.\nolinebreak
    \mbox{}\hfill\mbox{(\citeyear{Choi:2021wg})}
  \end{quote}
  For example, an object is disposed to dissolve when it is placed in water iff the object would dissolve if it were the case that it is placed in water.

  The Simple Conditional Analysis may be challenged, but for our purposes it is adequate.
  We are interested in the broad form of the truth condition, and various more refined analyses share the same broad form.
  Note, in particular, that it being the case that \emph{C} and \emph{M} happening describes an event.
  Given appropriate conditions; salt dissolves, glass breaks, and I mumble when I am tired.
  The key idea is that the property of being disposed to \emph{M} when \emph{C} is analysed in terms of the (possible) event of \emph{M} happening when \emph{C}.

  The parallel to ability is established by noting that ability may also be analysed in terms of a (possible) event, as we have seen.
  In particular, by incorporating volition in the analysans of the Simple Conditional Analysis.
  To illustrate, \citeauthor{Mandelkern:2017aa} trace the Conditional Analysis of ability  to \textcite{Hume:1748tp} and \textcite{Moore:1912te}, among others:
  \begin{quote}
    S can \(\phi\) iff S would \(\phi\) if S tried to \(\phi\)\nolinebreak
    \mbox{}\hfill\mbox{(\citeyear[Cf.][308]{Mandelkern:2017aa})}
  \end{quote}
  Compare to the Simple Conditional Analysis of dispositions:
  The object is some agent \emph{S}, \emph{C} is `S tried to \(\phi\)' and \emph{M} is `S \(\phi\)s' --- it is volition alone which distinguishes the analyses.
  For example, I have the ability to demonstrate that a box with dimensions \(19\text{cm}\) by \(7\text{cm}\) has area \(133\text{cm}^{2}\) only if I would demonstrate that a box with dimensions \(19\text{cm}\) by \(7\text{cm}\) has area \(133\text{cm}^{2}\) if it were the case that I tried that a box with dimensions \(19\text{cm}\) by \(7\text{cm}\) has area \(133\text{cm}^{2}\).
\end{note}

\begin{note}[Claiming support]
  Parallel analyses in hand, we now turn to claiming support.
  We start with dispositions.

  As with ability, there are various ways in which an agent may claim support that some object is disposed to \emph{M} when \emph{C}.
  For example, I may claim support that my shoes are disposed to squeak when wet because I have had sufficient occasion to observe the phenomenon.
  Likewise, I may claim support that any shoe of the same model is disposed to squeak when wet because I have traced the source of the squeak to a manufacturing choice.
  In short, support may be claimed by past event and shared properties.

  Still, take a novel act and a object pair.
  Personally, I have a empty fountain pen that I haven't placed in water.
  I claim that the fountain pen is disposed to float when placed in water.
  My reasoning is fairly simple.
  The fountain pen is quite light, especially so while empty of ink.
  And, the cap and loading mechanism seem to be quite well sealed, so the weight of the fountain pen will not increase by taking on water.
  So, given that the weight of the fountain pen will be unchanged, and given how light the pen is, it seems that the upward force exerted by the water against the fountain pen will be sufficient to keep the pen afloat.

  In short, I've noted a few properties of the pen, claimed support for a handful of others, and then considered what would happen.
  Our interest is with the last step.
  I appeal to, and use, the possible event.\nolinebreak
  \footnote{
    I may be wrong about the event, but that isn't at issue.
    It remains the case that I appeal to it.
  }
  The noted properties are relevant because they suggest that the event of floating would happen if it were the case that the fountain pen were placed in water.
\end{note}

\begin{note}
  The fountain pen is not the only object on my desk.
  Beside the fountain pen is a collection of instruments that I may use to investigate the fountain pen.
  And, stored in my mind is a basic understanding of fluid dynamics.

  If I were to measure the fountain pen, ensure that it is airtight, and appeal to some known facts, then an application of Archimedes' principle would allow me to demonstrate that the fountain pen is disposed to float when placed in water (of some specified density).
  Indeed, such a demonstration would be a straightforward refinement of the way in which I claimed support for the proposition that the pen is disposed to float when placed in water.

  Now, by similar reasoning I have claimed support for the proposition that I have the ability to demonstrate that the proposition that the pen is disposed to float when placed in water is true.
  Here, in addition to appealing to properties of the fountain pen, I also appealed to various mental properties.
  There is an important difference, however, regarding the relevant event.
  When reasoning about the disposition, the event is the fountain pen floating in water, but when reasoning about my ability to demonstrate the event is the demonstration --- a series of measurements and calculations.
\end{note}

\begin{note}[Diverge]
  Now to turn to \EAS{}.

  If I have the ability, then it follows that the fountain floats in water.
  As noted above, it is not possible for me to demonstrate something that is not the case.

  Claim support for the proposition that the fountain floats in water.

  Still, disposition, fountain pen is not floating in water.
  Likewise with respect to ability, I have not demonstrated that the fountain pen floats in water.
  I noted various things, but did not piece these together into a demonstration.

  Yet, in claiming support, there's the event of demonstrating.
  And, so I appeal to those premises and steps I would use in the event.
  This is \EAS{}.

  Appeal to what happens in the event.
  And, reasoning to claim possible event is viewed in terms of ensuring that the resources are available.
  I have not used the relevant premises and steps of reasoning, nor am I clear on the specific form they will take.
  Still, they are available.

  Final point of interest, then.
  In both cases, there's an appeal to an event.
  If \EAS{} holds with respect to ability, does something similar hold with respect to dispositions?

  First, important clarification.
  The reasoning outlined for disposition was claiming support for event.
  Here, no clear issue with \ESU{}.
  Similarly, no clear issue with \ESU{} with respect to claiming support for having an ability.
  Tension with \ESU{} arises when using ability as a premise in further reasoning.

  Second, key divergence.
  Conclusion obtained is something that is true independent of ability.
  Unclear to me whether similar reasoning with dispositions.
  For, ability is about an event involving the agent.

  In addition, there is no issue with supposing that the agent reasons with (and hence uses) to all the relevant features of the event.\nolinebreak
  \footnote{There may me details of reasoning that one is not easily able to express, but it doesn't follow that those details are not used.}
  Ability is in part interesting because it is clear that an agent does not witness the relevant event.
  This is not to say that a variant of \EAS{} does not hold with respect to dispositions.
  Rather, I am expressing
  \begin{enumerate*}
  \item hesitancy that there are comparable entailments, and
  \item concern that there is no clear argumentative path.
  \end{enumerate*}

  There is a related question about the ability of other agents.
  Here, \EAS{} does not entail.
  In turn, one may conjecture that reasoning from one's own ability is similar.
  I find this plausible.
  It is important to stress again that \EAS{} expresses a way in which an agent may claim support.
  Hence, \EAS{} is compatible with there being other ways in which an agent may claim support.
  It may be the case that the same holds with respect to other agents.\nolinebreak
  \footnote{
    For example, \citeauthor{Owens:2006tw} argues for a belief expression model of assertion in which the rationality of a belief formed by an agent on the basis of testimony depends whatever justification the speaker has for the relevant propositional content.
    \begin{quote}
      Trusting an expression of belief by accepting what a speaker says involves entering a state of mind which gets its rationality from the rationality of the belief expressed. This state's rationality depends on the speaker's justification for the belief he expresses, not on his justification for the action of expressing it. And to hear a speaker as making a sincere assertion, as expressing a belief, is \emph{ceteris paribus} to feel able to tap into \emph{that} justification (whether or not his assertion was directed at you) by accepting what he says.\nolinebreak
      \mbox{}\hfill\mbox{(\citeyear[123]{Owens:2006tw})}
    \end{quote}
    \color{red} Some more
  }
  However, this is not an immediate consequence.
  \EAS{} permits exceptions to \ESU{}, but it does not require all instances of reasoning with ability is an exception to \ESU{}.
  And, our focus will be on cases in which an agent reasons about their own ability to reason.
  The weak quantifier `there are cases' is designed to leave such issues open.
\end{note}

\begin{note}[Concluding parallel]
  To summarise.
  \begin{itemize}
  \item Parallel between analysis of dispositions and abilities.
  \item Event in analysis of both.
  \item Reason about event.
  \item Motivation for \EAS{} by considering reason to and from event.
  \item This doesn't provide anything close to a clear theoretical account of the reasoning performed if \EAS{} is true, but it does hint at such at how developing such an account may be approached.
  \item Now turn to related conclusion.
  \item In turn, fill in some details on the account.
  \end{itemize}
\end{note}

\paragraph{Enthymematic inferences}

\begin{note}[\citeauthor{Moretti:2019wx}]
  Above we considered how various account of the basing relation seem to imply \ESU{}.
  Roughly, because such accounts of the basing relation required a premise or step of reasoning to be used in order to be a candidate member of the base of some conclusion of reasoning --- motivated by either causal and representational considerations.
  In contrast, \citeauthor{Moretti:2019wx} argue for an account of the basing relation which does not entail \ESU{}.

  In our terminology, \citeauthor{Moretti:2019wx} argue that: A belief held by an agent may be \emph{based} on premises that the agent did not use when forming the belief.

  The following is a fragment of the general principle relating propositional justification to well-grounded belief (alternatively doxastistcally justified belief) containing the two clauses of interest:

  \begin{quote}
    IF

    \dots

    OR

    \begin{enumerate*}[label=(\arabic*.2\(^{\ast}\))]
    \item\label{LT:1.2} Q is propositionally justified for S in virtue of P1, P2, \(\dots\), Pn being justifiedly true from her perspective because S justifiedly believes P1, P2, \(\dots\), Pn, and in virtue of her being aware that Q is an inductive or deductive consequence of P1, P2, \(\dots\), Pn jointly, and
    \item\label{LT:2.2} S carries out a \emph{plain} inference from P1, P2, \(\dots\), Pn to Q.
    \end{enumerate*}

    OR

    \begin{enumerate*}[label=(\arabic*.3), ref=(\arabic*.3)]
    \item\label{LT:1.3} Q is propositionally justified for S in virtue of P1, P2, \(\dots\), Pn being justifiedly true from her perspective, though S doesn't believe at least some P1, P2, \(\dots\), Pn, and in virtue of S being aware that Q is an inductive or deductive consequence of P1, P2, \(\dots\), Pn jointly, and
    \item\label{LT:2.3} S carries out a (fully or partly) \emph{enthymematic inference} from P1, P2, \(\dots\), Pn to Q.
    \end{enumerate*}

    THEN
    \begin{enumerate}[label=(3)]
    \item S's belief that Q is well-grounded.\nolinebreak
      \mbox{}\hfill\mbox{(\citeyear[87]{Moretti:2019wx})}
    \end{enumerate}
  \end{quote}

  The `plain' inference of~\ref{LT:1.2} and~\ref{LT:2.2} corresponds to cases in which an agent uses P1, P2, \(\dots\), Pn to reason to Q.
  By contrast, the `enthymematic' inference of~\ref{LT:1.3} and~\ref{LT:2.3} involves reasoning in which an agent does not use some or all of P1, P2, \(\dots\), Pn to reason to Q as the agent does not believe some of P1, P2, \(\dots\), Pn (though the agent has propositional support for each of P1, P2, \(\dots\), Pn).

  To illustrate the distinction between `plain' and `enthymematic' inferences (\citeyear[Cf.][85]{Moretti:2019wx}) consider reasoning from the premise that \nagent{5} is shorter than \nagent{6} to the conclusion that someone is taller than \nagent{5}.
  An instance of plain (non-enthymematic) may take the intermediary step that \nagent{6} is taller than \nagent{5} before abstracting from \nagent{6}.
  In contrast, an instance of enthymematic reasoning consists of the (single) premise and conclusion noted without forming the belief that \nagent{6} is taller than \nagent{5}.\nolinebreak
  \footnote{Cf.\ (\citeyear[87--89]{Moretti:2019wx}) for examples given by \citeauthor{Moretti:2019wx}.}

  The key idea is that if an agent reasons enthymematically, then the agent's belief may be based on those premises that the agent would use in the corresponding plain inference.
  (\citeyear[Cf.][86--87]{Moretti:2019wx})
  Hence, we have a proposal on which an agent's belief may be supported by premises and steps of reasoning that an agent has not used.
  And, in addition, because S carries out a (fully or partly) enthymematic inference \ref{LT:2.3}, it seems S \emph{may} appeal to P1, P2, \(\dots\), Pn when reasoning to Q, in conflict with \ESU{}.

  Whether or not \citeauthor{Moretti:2019wx}'s account is correct is not of interest.
  Rather, \emph{grating} that \citeauthor{Moretti:2019wx}'s account is correct allows us to make two (related) observations.
  First, \citeauthor{Moretti:2019wx} account does not conflict with \ESU{} and so the account does not require \EAS{} to be true.
  And, second, how \citeauthor{Moretti:2019wx}'s account suggests a broader theoretical account of \EAS{}.
\end{note}

\begin{note}[First point]
  To establish the first point we require further details about how \citeauthor{Moretti:2019wx} define a (fully or partly) enthymematic inference.
  The following quote combines the relevant definitions:
  \begin{quote}
    \textbf{(}[\textbf{Partly}/\textbf{Fully}] \textbf{Enthymematic Inference)}

    S carries out a [\emph{partly}/\emph{fully}] \emph{enthymematic} inference from P1, P2, \(\dots\), Pn to Q if and only if
    \begin{enumerate}[label=(\alph*), ref=(\alph*)]
       \setcounter{enumi}{1}
    \item \emph{S doesn't actually believe} [\emph{at least some of the premises}/\emph{any of}] P1, P2, \(\dots\), Pn, though some constituents M1, M2, \(\dots\), Mm of S's perspective cause in S the \emph{disposition} to believe P1, P2, \(\dots\), Pn, and
    \item M1, M2, \(\dots\), Mm [together with the premises believed by S jointly/jointly] cause S's belief that Q through a process that is shaped by S's taking Q to be a consequence of P1, P2, \(\dots\), Pn at a personal level.\nolinebreak
      \mbox{}\hfill\mbox{(\citeyear[85]{Moretti:2019wx})}
    \end{enumerate}
  \end{quote}

  In short, an enthymematic inference involves reasoning with premises M1, M2, \(\dots\), Mm which are related to the premises P1, P2, \(\dots\), Pn of some corresponding plain inference.
  In order to complete the definition, we require an account of what it is for S to take Q to be a consequence of P1, P2, \(\dots\), Pn at a personal level:

  \begin{quote}
    \textbf{(Personal Level\(^{\ast}\))}

    S's mental states M1, M2, \(\dots\), Mm and any premises believed by S, among P1, P2, \(\dots\), Pn, jointly cause S's belief that Q through a process shaped by S's taking Q to be a consequence of P1, P2, \(\dots\), Pn at a personal level if and only if M1, M2, \(\dots\), Mm and any premise believed by S, among P1, P2, \(\dots\), Pn, jointly cause S to believe Q and S would adduce the reasons that P1, P2, \(\dots\), Pn and that Q is a consequence of P1, P2, \(\dots\), Pn in response to a request to explain why she believes Q.\nolinebreak
    \mbox{}\hfill\mbox{(\citeyear[85--86]{Moretti:2019wx})}
  \end{quote}

  So, loosely reconstructed an enthymematic inference involves constituents M1, M2, \(\dots\), Mm of S's perspective which ensure that S has the disposition to believe P1, P2, \(\dots\), Pn.
  And, the way in which M1, M2, \(\dots\), Mm lead to S forming the belief that Q allow S to explain that they believe Q on the basis of P1, P2, \(\dots\), Pn.
  In short, an enthymematic inference is an inference in which may be \emph{post hoc} expanded to some corresponding plain inference (in part) because performing the enthymematic inference requires the agent to be disposed to believe the required premises of the corresponding plain inference.
  And, as such the premises of the corresponding plain inference may be considered as (constitutive of) the basis of S's belief that Q.

  In contrast, \ESU{} concerns the way in which M1, M2, \(\dots\), Mm lead to S forming the belief that Q do not necessarily require the agent to appeal to P1, P2, \(\dots\), Pn.
  It is consistent with \citeauthor{Moretti:2019wx} account that the reasoning from M1, M2, \(\dots\), Mm to Q may only appeal to premises and steps of reasoning used.
  That Q may be based on P1, P2, \(\dots\), Pn is due to the requirement that S is disposed to believe P1, P2, \(\dots\), Pn and the possibility of S retroactively appealing to Q being a consequence of P1, P2, \(\dots\), Pn.
  Hence, the account does not conflict with \ESU{}, and in turn does not require \EAS{} to be true.

  The insight offered is that there does not necessarily need to be a structure preserving mapping between premises and steps providing propositional support for a belief and the premises and steps appealed to when forming the belief.
  However, this does not constrain what the agent appeals to when forming a belief.

  From a broader perspective, \citeauthor{Moretti:2019wx}'s proposal considers what an agent was able to do (i.e.\ reason by some plain inference) and holds that a basing relation follows but is silent of the way in which an agent claim support.
  In contrast, \EAS{} looks at what an agent is able to do, and holds that a way of claiming support follows, but is silent on issues concerning the basing relation.
\end{note}

\begin{note}[Second point]
  Still, this broader perspective together with the above discussion of dispositions suggests a way to understand \EAS{}.
  For, one may hold that if an agent has the ability to reason to some conclusion, then the agent is disposed to use relevant premises and steps of reasoning to reason to the conclusion.
  In parallel to \citeauthor{Moretti:2019wx}, then, one may hold that the agent has the ability to reason to some conclusion if (and only if) they are suitably related to some collection of relevant premises and steps of reasoning.
  In turn, the agent may appeal to those premises and steps of reasoning to claim support for the conclusion.
  Indeed, if we adopt a parallel understanding of the basing relation, then it follows (so long as the agent has the ability) that  the agent has sufficient propositional support for the conclusion, and may be well-grounded.
  The (possible) event of reasoning to the conclusion is important both for establishing that the agent has the ability and for determining which premises and steps of reasoning the agent appeals to, but the event is not important for determining that the relevant premises and steps of reasoning are available to the agent.

  This suggestion falls far short of a theory satisfying \EAS{}, I suspect \EAS{} may be motivated in part by distinguishing between what occurs in the event of reasoning, and sufficient resources required for such an event to occur.
  An event of reasoning will always make use of sufficient resources for the event to occur, but an agent may have sufficient resources for the event to occur even if the event does not occur.
  (Specific) abilities, then, fix an particular event and determine sufficient resources and the agent does not need to witness the event in order to appeal to those resources.

  Or perhaps not.
\end{note}

\begin{note}[Segue]
  Our goal is to establish that an adequate account of reasoning which extends to ability must satisfy \EAS{}.
  This goal does not require the above suggestion to be on the right track, nor does this goal require that there is a unique theory that satisfies \EAS{}.
  For now, we close the present section with a few remarks concerning ability and~\EAS{}.
\end{note}

\begin{note}[Actual support]
  As with~\ESU{}, \EAS{} does not entail that the agent \emph{has} support.
  Our focus is on reasoning, and as argued above, it seems the issue of whether an agent has support is distinct from whether an agent may claim support.
  Claiming support is the result of some reasoning, and whether or not an agent has support requires an evaluation of that reasoning.
  This means that, strictly speaking,~\EAS{} does not carry any implications regarding whether or not the agent has support by claiming support in line with~\EAS{}.
  It is possible that the agent would fail to establish support, or establishes a support relation other than between the conclusion and the premises and steps of reasoning appealed to.

  Still, it take it to be plausible that support traces a successful claim.
  From this perspective,~\EAS{} may seem a little more intuitive.
  Given an intuitive understanding of support, if an agent does have the ability to reason to some conclusion, then the conclusion stands in the relation of being support by certain premises and steps of reasoning, whether or not the agent witnesses their ability.
  In turn, if the agent may claim support for the having the relevant ability then the agent may claim support for the conclusion from the premises and steps that would be used to witness their ability.
  For, witnessing does not contribute to the relation of support between the conclusion and the relevant premises and steps --- witnessing would only clarify to the agent the specifics of the relation.

  Of course, the agent may be mistaken or misled about having ability.
  For example, the relevant premises and steps may fail to establish the conclusion, or the agent may not have sufficient resources to carry out reasoning from the premises and steps, etc.
  In turn, witnessing may be expected to highlight that the claimed support for having the ability is mistaken or misled.

  Two points:
  \begin{itemize}
  \item Such issues are not different to being mistaken or misled and using that one has the ability as a premise, so apply to any reasoning that makes use of ability without witnessing ability.
  \item Attempting to witness the ability might reveal that the agent is mistaken or misled about having the ability does not show that the agent may not claim support for having the ability.

    Reasoning typically involves premises and steps of reasoning that could be investigated further, but this does not prevent an agent from appealing to those steps and premises.
  For example, it is (almost) to check the definition of any word used against a dictionary, and doing so might reveal that I have been mistaken or mislead about the meaning that I will convey by using the word.
  I rarely do this, though.
  Most of the time it is sufficient to expect that I am not mistaken or have not been mislead about the meaning I would convey by using the word.
  \end{itemize}
\end{note}

\begin{note}[Desire]
  Finally, while the examples of reasoning given have concluded with the truth of some proposition --- that a box has some specific area, or that a given fountain pen floats in water, etc.\ --- our interest with \EAS{} is broader.
  Recall, in~\autoref{prop:RisTV} we stated that for the purposes of this paper we consider the conclusion of reasoning as assigning some value to some proposition.
  In many cases the assigned value truth, falsity, or something in between.
  However, claiming support, and in turn; \USE{}, \ESU{}, and \EAS{} are all neutral with respect to the value assigned to the proposition.
  Therefore, we may consider other values while investigating, and as an application of \ESU{} and \EAS{}.
  In particular, consider reasoning which concludes with the desirability of some proposition.\nolinebreak
  \footnote{
    \color{red}
    Mistaken or misled.
    Yes, I think this holds up.

    Strong view on which an agent may be mistaken about desires in the same way as an agent may be mistaken about evidence.
    View on which desires are independent of representation.
    Hence, misleading or mistaken support when an agent fails to represent desire.
  }

  To illustrate this point, consider temptation.\nolinebreak
  \footnote{
    \color{red}
    Whether or not this is `genuine' temptation isn't of \dots
  }
  Specifically we will consider a slight variation on \citeauthor{Bratman:1999ac}'s `two glasses of wine' (\citeyear[38]{Bratman:1999ac}) case of temptation.\nolinebreak
  \footnote{
    \color{red}
    See also \textcite{Bratman:2007ab}
  }
\end{note}

\begin{note}[The Pianist]
  Consider a pianist who frequently performs at a club.
  Before each performance the pianist gets nervous and has the option of drinking a glass of wine.
  A glass of wine would also lead to a worse performance.
  However, the glass of wine would help with the pianist's nerves.
  Both are learnt with some experience.
  Hence, if the pianist reasons about what to do:
  \begin{itemize}
  \item When the pianist does not feel the nerves of an upcoming performance they reason to a preference abstaining from drinking a glass of wine.
  \item Yet, when nerves are felt the pianist reasons to a preference for drinking a glass of wine.
  \end{itemize}
  The pattern is stable, and has held over many performances.

  Still, while nerves sometimes get to the pianist, they abstain from drinking a glass of wine most of the time.

  That the our pianist abstains is not necessarily surprising --- it is not uncommon to resist temptation.
  Though it is puzzling.
  The pianist's reasoning is unwavering throughout the span of time in which the pianist has the option of drinking the wine; they reason to preference for drinking a glass of wine.
  So, if the pianist abstains, the pianist acts in opposition to their preference when given the option of drinking a glass of wine, and does so purposefully.
\end{note}

\begin{note}[Reasoning and desire]
  To clarify the puzzle, let us state a basic conjecture regarding preferences and acts.

  \begin{conjecture}\label{conj:resolve-issue-act}
    Any instance of purposeful rational action performed by an agent is the result of the agent resolving the issue of how to act.
    Where:
    \begin{enumerate}
    \item An act is an candidate resolution for how to act only if the agent has claimed support for preference for some proposition and has an expectation that act would bring about the proposition.
    \item An act is an admissible resolution only if there no other candidate resolutions for which the agent has a stronger (combined) preference with respect to the proposition(s) that the agent expects to be brought about by performing the act.
    \end{enumerate}
  \end{conjecture}

  \autoref{conj:resolve-issue-act} understands rational action as the result of an agent resolving the issue of how to act --- choosing which act from a collection of options to perform.
  By understanding rational action as the result of an agent resolving the issue of how to act we may break down the reasoning involved in purposeful rational action into two steps.

  First, what makes an act a candidate resolution, and second what makes an act an admissible resolution.

  An act is a possible resolution just in case the agent links the result of acting to some proposition the agent has a preference for.

  And, an  act is an admissible resolution just in case the agent has no stronger preference for some other proposition that the agent expect could be brought about by some other candidate action.
  (Or, more generally, when an agent is uncertain about which proposition may be brought about by some act, a combined preference regarding each potential proposition.)

  In short, \autoref{conj:resolve-issue-act} is more-or-less the core of an standard decision theoretic account of maximising expected utility without commitment to particular details.\nolinebreak
  \footnote{
    \color{red}
    Cf.\ \textcite{Steele:2020tr}.
    \citeauthor{Davidson:1963aa} `Primary reason' (\citeyear{Davidson:1963aa})
  }
\end{note}

\begin{note}[Use of conjecture]
  \autoref{conj:resolve-issue-act} fixes an understanding of purposeful rational action, and in turn establishes two ways in which the pianist may resist drinking a glass of wine:
  \begin{enumerate}
  \item Drinking the glass of wine is not a candidate resolution.
    \autoref{conj:resolve-issue-act} states a necessary condition for a candidate resolution, but further conditions may rule out possible resolutions which satisfy the necessary condition stated.
  \item Drinking the glass of wine is not an admissible resolution.
    In particular, because the pianist has a claims support for a stronger preference toward the result of abstaining.
  \end{enumerate}

  We will provide a brief argument that \ESU{} requires the former to be the case and provide an example of how further conditions may rule out possible resolutions.
  In short, \ESU{} requires an agent to witness reasoning to a conclusion in order to claim support for such a conclusion, and as the pianist reasons to a preference for having drunk a glass of wine when performing, abstaining is not an admissible resolution.
  Then, we will turn to \EAS{}, and suggest that it allows the latter to be the case while granting that the only reasoning that the pianist witnesses establishes a preference for having drunk a glass of wine when performing.
  In short, so long as the pianist may claim support for the ability to reason to a stronger preference for the result of abstaining, then by \EAS{} the agent may claim support for a stronger preference for the result of abstaining.
\end{note}

\begin{note}
  Suppose \ESU{} is true.
  By \autoref{conj:resolve-issue-act} an agent resolves an issue of how to act by determining candidate resolutions.
  In turn, a candidate resolution results from the agent claiming support for a preference toward some proposition.
  And, \ESU{} requires that an agent must witness some instance of reasoning in order to claim support for the conclusion of the instance of reasoning.

  Turning to the pianist, we have assumed that before taking to the stage the pianist reasons to preference that favours drinking a glass of wine.
  So, given \autoref{conj:resolve-issue-act} abstaining is not an admissible resolution because the agent has a stronger preference from drinking a glass of wine before taking the stage.
  And, by \ESU{} it is not possible for the pianist to claim support for a preference that would lead to abstention because the pianist must witness the relevant reasoning in order to claim support.
  Therefore, the pianist must rule out drinking a glass of wine as a candidate resolution for how to act.
\end{note}

\begin{note}[Intention and \ESU{}]
  \color{red}

  \citeauthor{Bratman:2007ab} argues that such cases may be understood through an theory on which intentions constrain reasoning.
  If the pianist intends no to drink the wine, and this intention persists, then drinking the wine is no an available conclusion of reasoning.
  Key here is that intention does not interact with the pianist's preferences.
  It remains the case that the pianist would prefer.

  The role of intention in the role of \citeauthor{Bratman:2007ab} account is to constrain possible resolutions for how to act.
  An intention to not drink prevents drinking a glass of wine from being a candidate resolution for how to act (see in particular \textcite[\S3.3]{Bratman:1987aa}).

  However, because ruled out, the pianist does not have the option of acting on that preference.
  Rather, act in a way that is compatible with intention.
  For the pianist, we may assume abstention is the only act compatible with the intention.\nolinebreak
  \footnote{
    \color{red}
    Variation.
    Block contribution of nerves.
    So, intention to not allow nerves to contribute to reasoning.
    Compatible with drinking, but given the way the scenario has been constructed, will result in not drinking.
  }
  So, \citeauthor{Bratman:2007ab} is an example of how to resolve weakness of will given relation between reasoning and action expressed by \autoref{conj:resolve-issue-act} and \ESU{}.
\end{note}

\begin{note}[Broader]
  I think, in broad strokes, phenomena fit this kind of theory.
  Abstracting from the details of any particular theory, it seems plausible that candidate resolutions to the issue of how to act are subject to conditions that extend beyond whether the agent has claimed support for preference for some proposition and has an expectation that act would bring about the proposition --- \autoref{conj:resolve-issue-act} only stipulates that the given constraint is a necessary condition for candidate resolutions.

  However, not clear to me that all phenomena fit such a theory.
  The pianist's reasoning seems distorted.
  The nerves felt before taking to the stage plausibly interfere with the pianists reasoning about candidate resolutions to the issue of how to act, and so the pianist's reasoning plausibly does not resolve the issue of how to act in line with the premises and steps of reasoning that are available to the agent.
  So much, I suspect, is intuitive.
  However, any interpretation of the pianist compatible with \ESU{} is committed to the agent resolving the issue of how to act given unreliable reasoning.
  A \citeauthor{Bratman:1987aa}-like intention would rule out drinking a glass of wine as a resolution to the issue of how to act, but whatever reasoning the agent performs given the intention is still influenced by the nerves felt.

  In the following paragraphs we will suggest that the pianist may resist the conclusion of reasoning performed given nerves because it is distorted.
  We start with two additional conjectures.
\end{note}

\begin{note}[Preferences change with reasoning]
  \begin{conjecture}\label{conj:pref-vs-reasoning}
    Whether, or to what degree, an agent claims support for a preference toward a proposition may differ from whether, or to what degree, the agent would claim support for a preference toward the proposition given varying information.
  \end{conjecture}

  Loosely paraphrased,~\autoref{conj:pref-vs-reasoning} states that preferences an agent reasons to are subject to change given change in the information that the agent reasons from.
  Hence, it seems~\autoref{conj:pref-vs-reasoning} may be considered a truism rather than a conjecture.
  Indeed, varying information is a common component in the construction of cyclical preferences (cf.\ \cite{Sobel:1997wt},\cite{Schumm:1987wx},\cite{Davidson:1955wo}, etc.)

  To illustrate, we consider a case in which difference arises from information that does not contribute to the agent's preferential evaluation of the proposition.

  Suppose an agent has a established preference for meeting person who is Black Panther over meeting the person who is Storm by reasoning.
  However, the agent is not aware that Black Panther is T'Challa nor that Storm is Ororo Monroe.
  Indeed, the agent does not have any information about the referent of `T'Challa' or `Ororo Monroe' and so does establish a preference for meeting T'Challa or Ororo Monroe by reasoning (nor vice-versa).

  Still, it seems that if the agent were provided with the information that Black Panther is T'Challa and that Storm is Ororo Monroe then the agent would reason to a preference for meeting T'Challa or Ororo Monroe.
  And, such information would not contribute to the agent's preferential evaluation of the relevant propositions.
  The agent's preferences for the relevant propositions are determined by considerations that are independent of the terms used to refer to the relevant individuals --- e.g.\ the person who helped defeat Thanos and the person who helped defeat Magneto.

  With relation to \autoref{conj:resolve-issue-act}, whether or not an act is a possible resolution for issue of how to act may depend on what information agent reasons from.
  We now introduce a further conjecture:

  \begin{conjecture}\label{conj:more-info-is-good}
    Generally speaking: When resolving how to act, claimed support for some preference toward a proposition given more information is given greater weight than claimed support for preference toward the (same) proposition given less information.\nolinebreak
  \footnote{
    Note, `more' and `less' information are relative, and it may not be possible to compare distinct bodies of information.
    If so, \autoref{conj:more-info-is-good} does not state anything about the importance of either body of information.
    For example, an agent may have information about the subjective taste of a meal and about the nutritional value of the meal.
    Still, without a way to compare information about subjective taste to information nutritional value in an information there would be no sense in which the former could be considered to hold `more' information that the latter (or vice-versa).
    Though this is not to rule out such a comparison --- we do not place constraints on what comparisons an agent may make.
  }
\end{conjecture}

  \autoref{conj:more-info-is-good} speaks more generally than~\autoref{conj:pref-vs-reasoning} and as a result may be closer to or further from a truism depending on your point of view.
  Still, the core idea is simple:
  Claimed support for some preference toward a proposition given more information is worth more than claimed support for some preference toward a proposition given less information because more information typically increases the reduces the likelihood that the claimed support is either mistaken or misled.\nolinebreak
  \footnote{
    Cf.\ \cite{Good:1966wx} for a related idea --- though see also \cite{Bradley:2016wo}.
  }
  In other words, strength of preference in the sense of \autoref{conj:resolve-issue-act} is proportional to information used to establish preference given a fixed proposition.

  To illustrate, consider an agent resolving whether to banded or off-brand multi-vitamins.
  The agent's method of establishing a preference is to look on each container, and work through the lists of vitamins comparing whether a vitamin is included, and if so to what quantity, weighing some vitamins more heavily than others.
  As the agent works through the lists of vitamins the agent moves from less information to more.
  Still, after each vitamin on the list the agent marks a preference.
  For example, the branded multi-vitamins have 2,500 IU of vitamin A while the off-brand have 2,000 IU, so the agent's initial preference leads to purchasing the branded multi-vitamins.
  However, the branded multi-vitamins have 50mg of vitamin C while the off-brand have 60mg, and given information about vitamins A and C the agent's preference leads to purchasing the off-brand multi-vitamins.
  And so on until the agent has compared the contents of the branded and off-brand multi-vitamins.

  \autoref{conj:more-info-is-good} holds because it seems implausible that the agent could be understood as acting rationally by purchasing either multi-vitamin from a preference determined by a partial comparison between the multi-vitamins given that the agent has established a preference given a full comparison between the multi-vitamins.
\end{note}

\begin{note}[Back to the pianist]
  To summarise the two conjectures:
  \autoref{conj:pref-vs-reasoning} holds that an agent's preferences may vary with the information that the agent uses to claim support for those preference.
  And, \autoref{conj:more-info-is-good} holds, generally speaking, that claimed support for a preference from more information is given greater weight than claimed support for a preference from less information.

  Now, we return to the pianist.
  We make two observations.
  First, given \autoref{conj:pref-vs-reasoning} it may be possible to trace the change in the preference that the pianist reasons to (from abstention to a glass of wine, and vice-versa) to variation in the information that the pianist reasons with.
  In particular, it may be the case that the pianist's nerves prevent them from reasoning with information that they would otherwise reason with, hence the preference to abstain arrived at by reasoning before and after the span of time in which the pianist has the opportunity to drink a glass of wine is a preference arrived at given more information than the preference to drink a glass of wine.
  Second, given \autoref{conj:more-info-is-good} and the present interpretation, the agent may give greater weight to the preference to reasoning that results in a preference for abstention.

  In short, it may be true that pianist's nerves interfere with the reasoning they perform, and without such interference the agent would reason to abstaining from drinking a glass of wine before any given performance.\nolinebreak
  \footnote{
    \color{red}
    Plausible, though not immediate.
    To clear things up a little, consider a hangover.
    Reasoning sucks, I did not desire to miss lunch with a friend, I forgot, because of the hangover.
    However, I did desire to go to bed early, because of the hangover.
    Question is whether the nerves and the glass are like missing lunch or going to bed early.
    If the former, then reasoning is at issue.
    If the latter, then reasoning is at the issue.

    Conjecture, the former.
    No additional information from nerves.
    Still, when the nerves are present the pianist has a hard time reasoning with them.
  }

  The difficulty for the pianist is that such interference is always present;
  It is not straightforward for the pianist to give greater weight to abstaining, as the pianist does not reason to abstaining given their nerves.
  However, if the pianist may claim support for  ability to reason to abstaining, then by \EAS{} the agent may claim support for a preference for abstaining.

  Indeed, it seems the pianist may claim support for having the ability to establish a preference for abstaining when presented with the option of drinking a glass of wine.
  Two observations:
  \begin{itemize}
  \item First, the pianist has performed such reasoning many times, both before and after performances, and the result of such reasoning is stable: abstention.
  \item Second, given the assumptions made about the agent's nerves, the agent retains the relevant premises and steps of reasoning when presented with the choice to drink a glass of wine.
    The nerves felt when presented with the choice to drink a glass of wine only ensure that witnessing this ability is difficult to a degree sufficient for the agent to fail each attempt at witnessing the ability.
  \end{itemize}

  So, when presented with the choice to drink a glass of wine:
  If the  pianist may claim support for having the ability to establish a preference for abstaining, then by \EAS{} the pianist may claim support for a preference for abstaining.
  And, in turn, by \autoref{conj:more-info-is-good} the pianist may give greater weight to their preference for abstaining because the claim of support for a preference to abstain would be arrived at by taking into consideration additional information.

  Given this interpretation of the pianist, `giving into temptation' would be for the pianist to disregard the reasoning that the pianist is able to perform.
  Conversely, `resisting temptation' is for the pianist act in accordance with the preference that they would reason to given the information available to them.
  So, in contrast to accounts of temptation constrained by \ESU{} the pianist need not prevent themselves from reasoning to drinking a glass of wine.
  Rather, the pianist need only reflect on what they are able to reason to.
\end{note}

\begin{note}[Quick objection]
  There is a quick objection to consider before moving on:
  Does the agent have the ability to reason to stronger preference for the result of abstaining?
  For, it seems natural for the pianist to express that they do not have the ability to reason to a preference other than for having drunk a glass of wine when performing \emph{because} of how nerves interfere with their reasoning.

  Ability fluctuates.
  At present I claim support that I have the ability to prove that S4 is sound and complete with respect to transitive frames.
  Part of my claim is that I understand the details of Lindenbaum's Lemma.
  And, if I were to forget the details ofLindenbaum's Lemma then I would lack the ability to construct the relevant proof.
  So, I may lose the ability, but even if I do lose the ability due to forgetting the details of Lindenbaum's Lemma, I may regain the ability by revising the relevant details.

  However, there may be a difference between my loss of ability due to forgetting the details of Lindenbaum's Lemma and the pianist's nerves.
  Forgetting the details of Lindenbaum's Lemma ensures that I lack premises (or steps of reasoning) required to witness the proof.
  By contrast, it is not clear that the pianist's nerves entail that the pianist lack premises (or steps of reasoning) required to reason to stronger preference for the result of abstaining.

  Let us distinguish to ways in which an agent may be said to lack an ability to reason to some conclusion.
  First, the agent lacks sufficient resources; premises and steps of reasoning.
  Second, impediments to the agent using sufficient resources.

  From the first, I have the ability to enumerate all of the positive integers in decimal representation as I have sufficient resources to produce a decimal representation of the first positive integer and I have sufficient resources to produce a decimal representation of any successor integer.
  From the second, I am clearly bounded to enumerate only a finite collection of the positive integers given my mortality and so lack such an ability.

  The success of the quick objection relies on an ability to reason to some conclusion entailing that there are impediments to the agent using sufficient resources to reason to the conclusion.
  I suspect this entailment does not hold for the sense of ability at issue.
  Rather, I suggest that what matters is that the conclusion follows from the premises and steps of reasoning.

  Whether or not this a compelling suggestion is up to you.
  Whether or not \EAS{} holds does not depend on impediments to the agent using sufficient resources entail lack of ability.
  Though, interest in the details of ability and whether they relate to cases of temptation such as the pianist do depend on whether or not \EAS{} holds.
  Therefore, our primary focus will be on showing that \EAS{} holds.
\end{note}

\section{Structure of argument}
\label{sec:structure-argument}

\begin{note}[Structure of argument]
  Two lines of argument for endorsing~\EAS{}, and hence denying~\ESU{}.
  \begin{enumerate}[label=(L\arabic*), ref=(L\arabic*)]
  \item\label{arg:line:1} Motivate~\EAS{} as resolution to tension resulting from~\ESU{}.\newline
    Specifically:
    \begin{enumerate}[label=(L1\alph*)]
    \item\label{arg:line:1:a} Provide recipe for generating scenarios where~\ESU{} is in tension with particular scenarios involving information that an agent has the ability reason to some conclusion and a further claim regarding when it permissible for an agent to claim support for a proposition.
    \item\label{arg:line:1:b} Motivate~\EAS{} as a resolution to the tension.
    \end{enumerate}
  \item\label{arg:line:2} Argue that granting~\EAS{} as an exception to~\ESU{} allows for an intuitive understanding of cases in which agent has the option of appealing to ability, even if there are alternative ways of interpreting the scenario in line with~\ESU{}.
  \end{enumerate}
  These two lines of argument work together.
  The tension of~\ref{arg:line:1} generates interest in witnessing that may be flatly rejected by prior endorsement of~\ESU{}.
  The intuitive understanding of scenarios involving ability of~\ref{arg:line:2} suggests there's more to witnessing than resolving the tension in narrow cases.
\end{note}

\begin{note}[Details of \ref{arg:line:1}]
  The initial focus is on the first line of argument,~\ref{arg:line:1}.
  The tension developed in part~\ref{arg:line:1:a} is delicate, but hopefully informative.
  We will establish a number of corollaries regarding ability and the interaction between~\ESU{} and ability.
\end{note}

\subsection{Major argument}
\label{sec:major-argument}

\begin{itemize}
\item Type of case.
\item \gsi{-}.
  \begin{itemize}
  \item Two parts.
  \item Claiming support for specific ability
  \item Claiming support for result of specific ability.
  \end{itemize}
\item ability entailment.
\item Schematic interpretations of ability: \AR{} and \WR{}.
\item Exhaustive.
\item Relate to \ESU{} and \EAS{}.
\item Return to two parts of \gsi{}.
  \begin{itemize}
  \item \ESU{} requires that when agent reasons to specific ability, agent claims support for a property in line with \AR{}.
  \item However, \nI{} requires that when an agent reasons from specific ability, agent does not claim support from a property, in contrast to \AR{}.
  \end{itemize}
\item \ESU{} requires agent reasons with \AR{}.
\item Reasoning with \AR{} is incompatible with \nI{}.
\end{itemize}

So, the key thing is that we're looking at two different aspects of reasoning with ability.
Reasoning to and reasoning from.
And, it is not possible to give an account of \emph{both} reasoning to and from ability given \ESU{} and \nI{}.
So, the tension isn't, so to speak, `direct'.
Rather, tension arises due to some extended piece of reasoning.

So, one way to resolve tension is to deny extended reasoning.
First part argues that extended reasoning is plausible.


\begin{itemize}
\item Upshot:
  \begin{itemize}
  \item If scenarios, then either \AR{} or \WR{}.
  \item In turn, either not \nI{} or not \ESU{}.
  \item Alternatively, if scenarios then either \ESU{} or \nI{} (the scenarios give rise to this conflict).
  \item In turn, either \AR{} or \WR{}.
  \end{itemize}
\end{itemize}

We've already seen a decent amount of stuff regarding \ESU{} and \EAS{}.
These more or less correspond to \AR{} and \WR{}, kind of.

\subsection{Minor argument}
\label{sec:minor-argument}

???


% \begin{note}[Before turning to the argument\dots]
%   Before turning to the argument, we conclude this introduction with a handful of notes regarding~\ESU{} and~\EAS{}.
% \end{note}

% \begin{note}[Scope of \ESU{}]
%   \ESU{} does not say anything in particular about what the agent may claim support for, only what must be the case in order for an agent to appeal to support for some conclusion on the basis of support for premises.

%   Talking in terms of (support for) premises and conclusions restricts attention to reasoning.
%   There may be broader use of `premise' and `conclusion' where an agent is not required to reason from premise to conclusion in order for the premise to support the conclusion.
%   For example, if visual perception is immediate.
%   Perhaps it may be said that an agent's visual experience is a premise to the conclusion that a dog is sleeping.
%   Still, for present purposes, `conclusion' refers to the output of some process of reasoning performed by an agent which is either actual or potential, and `premises' to the input of that process.

%   Note, also, that in both cases the relation between premises and conclusion is important.
%   If agent does not reason, then neither~\USE{} nor~\ESU{} apply.
%   If there are multiple ways to obtain a conclusion, then~\ESU{} does not require the agent to reason from a particular set of premises.

%   Likewise,~\ESU{} does not require that an agent is required to obtain support for a proposition by valid and subjectively sound reasoning from some premises.

%   Rather,~\ESU{} requires that an agent reason from premises to conclusion in order to establishes support between premises and conclusion
%   By contrast,~\USE{} holds that reasoning is sufficient to establish such a relation.
% \end{note}

% \begin{note}[\ESU{} is intuitive]
%   \ESU{} is intuitive, and is quite common, though not without exceptions.
% (For example, there's views on testimony in which the testifier provides agent access to support the testifier has.
% One may understand this as conflicting with~\ESU{}, or that the fact that these are accessible is the relevant piece of support.)
% \end{note}

% \begin{note}[Alternative]
%   \EAS{} restricts~\ESU{}.
%   This is not to say the agent obtains support equivalent to that which would be obtained were the agent to do, or have done, the reasoning.
%   Nor, that the agent is aware of the relevant premises.

%   Intuitively, \EAS{} states that the agent may appeal to the reasoning they are able to perform in support for the conclusion of that reasoning, and as that reasoning moves from premises to conclusion, it is on the basis of the support for those premises that the agent would identify by reasoning that the agent obtains (some) support for the conclusion.

%   Hence, \EAS{} is in line with the spirit of~\USE{}.
%   For the exception to~\ESU{} is granted by the agent appealing to a witnessing event in which the antecedent (and consequent) of~\USE{} are satisfied.
% \end{note}

% \begin{note}[Ability ensures propositional?]
%   Plausible that if the agent has the ability, then the agent already has propositional support for the relevant proposition.
% \end{note}

\section{Major argument}
\label{sec:broad-argum-overv}

\begin{note}[Overview]
  Tension resulting from the unrestricted scope of~\ESU{}.
  We begin by introducing a particular type of scenario involving ability, and observe how~\ESU{} requires a unique interpretation of the scenario.
  We then introduce an additional principle regarding support, which conflicts with the interpretation of the type of scenario introduction required by~\ESU{}.
\end{note}

\begin{note}[Introducing key parts]
  Type of information and entailment.
  Two ways to understand entailment.
  Then, if information and entailment \dots
  Principles constrain understanding.
  \ESU{} and a second principle.
\end{note}

\subsection{Scenarios}
\label{sec:cases-interest}

Our goal is to argue for \EAS{} and against \ESU{}.
At the core of the argument is reasoning about ability.
Specifically, a certain type of scenario in which an agent reason to and from information that they have the ability to witness some specific act.
How the agent reasons with such (specific) ability information in the scenarios of interest will provide a type of counterexample to \ESU{} and in turn an argument for \EAS{}.

In this section we outline two key features of the scenarios we are interested in.
Subsection~\ref{sec:type-scenario} will introduce \gsi{-} to characterise how the agent reasons to the (specific) ability information.
Then, subsection~\ref{sec:ability-entailment} will introduce `the \aben{}' to characterise how the agent reason from the (specific) ability information.
Finally, subsection~\ref{sec:scenarios} will combine \gsi{-} and `the \aben{}' to provide an in-depth understanding of the type of scenarios we are interested in.

\subsubsection{\Gsi{-}}
\label{sec:type-scenario}

\begin{note}[Tension, information]
  \begin{definition}[\gsi{}]
    \Gsi{-} is information with the semantic content that:\nolinebreak
    \footnote{
      The formulation of \gsi{} as a conditional isn't important.
      What matters is that the agent is required to claim support for the general ability in order to claim support for the specific ability.
      For example, the conditional may be reformulated as:
      \begin{enumerate}[label=(\gsi{}\('\)), ref=(\gsi{}\('\))]
      \item Either \emph{S} does not have the general ability to \(\gamma\), or the agent has a specific ability to \(\varsigma\).
      \end{enumerate}
    }
    \begin{quote}
      If \emph{S} has a general ability to \(\gamma\), then \emph{S} has a specific ability to \(\varsigma\).
    \end{quote}
    Where \emph{S} is some agent, \(\gamma\) is some general ability, \(\varsigma\) is some specific ability, and it is either implicitly or explicitly stated that \(\varsigma\) is instance of \(\gamma\).
  \end{definition}

  The following pair of examples are instances of \gsi{}.
  \begin{enumerate}[label=(\gsi{}:\arabic*), ref=(\gsi{}:\arabic*)]
  \item\label{qe:cond} If you have the ability to reason with the rules of chess, then you have the ability to demonstrate that, given the arrangement of the board, there is a sequences of moves that will ensure a win for one of the players (as an instance of the general ability).
  \end{enumerate}

  \begin{enumerate}[label=(\gsi{}:\arabic*), ref=(\gsi{}:\arabic*), resume]
  \item\label{qe:cond:french} If you have the ability to read French, then you have the ability to read The Count of Monte Cristo without a translation (as an instance of the general ability).
  \end{enumerate}
  In both examples an agent is informed that they have the ability to perform a specific act --- demonstrating a strategy or reading a book --- so long as they have some general ability --- an understanding of chess or French literacy --- (in part) because the witnessing the specific ability act would be an instance of witnessing the agent's general ability.

  \gsi{} is limited because it does not directly provide the agent with the information that the agent has the specific ability, nor that the result of witnessing the specific ability is the case.
  The agent is not informed that they have the general ability and that therefore they have a specific ability.
  To illustrate, I am confident I have the ability to reason with the rules of chess, and so given \ref{qe:cond} I may be confident that I am able to demonstrate the existence of such a strategy.
  By contrast, I do not have the ability to read French, and so I do not have the ability to read The Count of Monte Cristo without a translation.

  Still, I may also be mistaken.
  It may be that I am overconfident, that I do not have the ability to reason with the rules of chess, and hence it may be the case that I do not have the ability to demonstrate the existence of the relevant chess strategy.
  Likewise, I may have the ability to read French, and may have the ability to read The Count of Monte Cristo without a translation.
  However unlikely this may be, I haven't tried to read French in quite some time.
\end{note}

\begin{note}[Not direct]
  \Gsi{} contrasts with what we term `\dsi{-}' --- information which states that the agent has a specific ability.
  \begin{definition}[\dsi{}]
    \Dsi{-} is information with the semantic content that:
    \begin{quote}
      \emph{S} has the ability to \(\varsigma\).
    \end{quote}
    Where \emph{S} is some agent and \(\varsigma\) is some specific ability.
  \end{definition}
  For example, the following is a direct variant of~\ref{qe:cond}:

  \begin{enumerate}[label=(\dsi{}\arabic*), ref=(\dsi{}\arabic*), series=dsi_count]
  \item\label{qe:cons} You have the ability to demonstrate that there is a sequences of moves that will ensure a win for one of the players as an instance of your general ability to reason with the rules of chess.
  \end{enumerate}

  Note that~\ref{qe:cons} states that the agent has the ability to demonstrate some strategy, and expands on why the agent has the specific ability.
  Hence, \dsi{} is not in general entailed by \gsi{}.
  However, if it is the case that an agent has the general ability mentioned in the antecedent of \gsi{}, then a corresponding instance of \dsi{} will be true.
  If I have the ability to reason with the rules of chess and \ref{qe:cond} is true with respect to me, then \ref{qe:cons} will also be true with respect to me.
\end{note}

\begin{note}
  \gsi{}, then, has two important features:
  \begin{enumerate}
  \item \gsi{} ensures that the agent is on the hook, so to speak, for holding that they have the specific ability.
  \item If the agent does have the relevant general ability, then \gsi{} provides the agent with an account of why the have some specific ability.
  \end{enumerate}
  Hence, \gsi{} ensure that an agent must themselves claim support that they have some specific ability while providing the agent with relevant information about why they may claim support for having the specific ability.
\end{note}

\begin{note}
  \color{red}
  \begin{itemize}
  \item You have some general ability \(\gamma\), and a specific ability \(\varsigma\) (as an instance of that general ability).
    And, if \(\gamma\) is the ability to calculate the area of a box, then \(\varsigma\) is the ability to demonstrate that a box with dimensions \(19\text{cm}\) by \(7\text{cm}\) has area \(133\text{cm}^{2}\).
  \end{itemize}
\end{note}

\subsubsection{An ability entailment}
\label{sec:ability-entailment}

\begin{note}[\aben{}]
  The second component in scenarios of interest is the availability of an entailment from the specific ability.
  We term the entailment `the \aben{}'.

  \begin{definition}[Ability entailment]
    The \aben{} is any entailment of the form:
    \begin{quote}
      \emph{S} has the (specific) ability to \emph{V} that \(\phi\) \emph{therefore} \(\phi\) is the case.
    \end{quote}
    Where \emph{S} is an agent, \emph{V} is some action, and \(\phi\) is some proposition.
  \end{definition}

  The rough intuition behind instances of the \aben{} is that \(\phi\) being the case does not depend on \emph{S} witnessing the (specific) ability to \emph{V} that \(\phi\).
  So, the \aben{} links ability and something that must be the case in order to have ability and the result of witnessing ability must be the case in order for the agent to have the ability

  For example, the \aben{} holds with respect to the (specific) ability to demonstrate the existence of a chess strategy from \ref{qe:cond} as whether or not a given chess strategy exists depends on the moves permitted by the rules of chess --- a strategy that has not been demonstrated is a strategy.
  Likewise, \emph{S} has the (specific) ability to discover that their keys are in their jacket pocket only if it is the case that their keys are in their jacket pocket --- whether or not \emph{S}'s keys are in their jacket pocket does not depend on \emph{S} discovering that to be the case.

  By contrast, `to read The Count of Monte Cristo without a translation' is an action and so the \aben{} does not apply to the specific ability of~\ref{qe:cond:french}.
  Even so, the \aben{} apply to nearby variants and not others.
  \emph{S} may have the specific ability to read that Dantès was a merchant sailor, and it follows that Dantès was a merchant sailor.
  In contrast, while \emph{S} may have the ability to believe that certain passages cannot be adequately translated, it does not follow that those passages cannot be adequately translated.
  Similarly, \emph{S} may have the ability to hope that they will employ the chess strategy discovered in a competitive game, but it does not follow that \emph{S} will employ the strategy.

  More broadly, the \aben{} holds with respect to factive verbs, such as `see', `know', `understand', and so on.
  Though, I doubt factive verbs are an adequate explanation for the \aben{}.
  Consider `read'.
  I have the ability to read that Elvis Presley was born in 1935, but I also have the ability to read that Elvis is working undercover for the DEA.
  What matters, then, is not the verb used, but how the agent would witness the relevant ability.
  I have the ability to read that Elvis was born in 1935 from a reliable source, and hence the \aben{} applies.
  The same is not true for my ability to read that Elvis is working for the DEA.

  Indeed, the \aben{} merely identifies an entailment.
  It does not provide an account of when or why such entailments hold.
  We identify entailments of this type because our interest is in how (in certain cases) agent's reason with instances of the \aben{}.
\end{note}

\subsubsection{Details of scenarios}
\label{sec:scenarios}

\begin{note}[Both things are important]
  The scenarios we are interested in combine \gsi{} with the \aben{}.
  The role of \gsi{} is to ensure that the agent is not provided with direct information about specific ability.
  And the role of the \aben{} is to highlight that the agent is in a position to claim support for some further proposition if they claim support for specific ability.
  Hence, scenarios combine claiming support \emph{for} specific ability and claiming support \emph{from} specific ability.

  To illustrate, consider the following pattern of reasoning:
  \begin{enumerate}[label=\arabic*., ref=(\arabic*)]
  \item\label{scen:rp:1} I have the general ability to reason with the rules of chess.
  \item\label{scen:rp:2} I received \gsi{} information that if they have the general ability to reason with the rules of chess then they have the ability to demonstrate the existence of some strategy.
  \item\label{scen:rp:3} So, from~\ref{scen:rp:1} and~\ref{scen:rp:2} it follows that I have the ability to demonstrate the existence of some strategy.
  \item\label{scen:rp:4} And, as the \aben{} hold with respect to~\ref{scen:rp:3}, the relevant strategy exists.
  \end{enumerate}
  I reason to (\ref{scen:rp:1} --- \ref{scen:rp:3}) and from (\ref{scen:rp:3} --- \ref{scen:rp:4}) a specific ability.
  The reasoning pattern seems sound.
  And, at no point do I need to witness their general ability to reason with the rules of chess, or the specific application of the general ability to demonstrate the existence of the strategy.
\end{note}

\begin{note}
  Both components are important.
  Focus on \gsi{} will restrict the interpretation of what the agent claims support for.
  And, in turn, what the agent has claimed support for will determine what the agent appeals to when appealing to the \aben{} entailment.\nolinebreak
  \footnote{
    I suspect it may be possible to focus only on \gsi{}.
    As we will see, this is where the key step of the argument takes place.
    However, this is not trivial.
    Would require more focus on how the agent gets to specific from general.
    By splitting in this way, we avoid details.
    Instead, focus on what it is that the agent gets, and then the \aben{} is forced to work with this.
  }
  \gsi{} and the \aben{} combine to provide a (partial) functional characterisation of reasoning with specific ability.
\end{note}

\begin{note}
  Note, however, that there is a distinction between how an agent reasons about ability, and what ability is.
  We are interested in how agent's reason about (specific) ability, and not what makes it true that an agent has a (specific) ability.
  Our focus will shortly turn to how to interpret (specific) ability when appealed to in the type of scenario described.
  We will outline two general schematic interpretations of ability, argue that these are exhaustive, and note how general constraints such as \ESU{} constrain which interpretation is available.
\end{note}

\begin{note}[Scenario proposition]
  For ease of reference, we wrap scenarios involving the limited information as a proposition.
  \begin{proposition}[\eA{-} --- \eA{}]\label{prem:ab}\label{prop:SE}
    There are scenarios in which an agent \emph{S} receives \gsi{} information of the form:
    % \mbox{ }\vspace{5pt}
    \begin{center}
      If \emph{S} has a general ability to \(\gamma\), then \emph{S} has a specific ability to \emph{V} that \(\phi\).
    \end{center}
    % \mbox{ }\vspace{5pt}

    \noindent Such that the \aben{} applies to the specific ability to \emph{V} that \(\phi\).

    In turn:
    \begin{enumerate}
    \item \emph{S} may reason from claimed support that they have the general ability to \(\gamma\) in order to claim support for having the specific ability to \emph{V} that \(\phi\). And,
    \item \emph{S} may reason from their claimed support that they have the ability to \emph{V} that \(\phi\) to claim support that \(\phi\) is the case (in part) by appealing to the \aben{}.
    \end{enumerate}
    \vspace{-\topsep}\vspace{-\topsep}
  \end{proposition}
\end{note}

\begin{note}[Possible restrictions]
  First, \eA{} holds only that there are cases in which the agent may appeal to ability to obtain support.
  It is therefore consistent with~\eA{} that there are cases in which the details of the cases outlined are satisfied, but where kind of support is unsuitable for certain purposes.
  For example, some witness of ability may be demanded by a third-party.
  In this respect, the content of \eA{} is similar to an analogous claim with respect to memory.
  If an agent remembers proving that \(\phi\), then \(\phi\) is the case.
  Still, one may still request that an agent provides you with a proof of \(\phi\) in order to for you to be satisfied that \(\phi\) is the case --- many exams are like this.
  So, that an agent may not always and in any context claim support for \(\phi\) from claimed support for their ability to \emph{V} that \(\phi\) is not an objection to~\eA{}.
\end{note}

\begin{note}
  Second, \eA{} does not require that an agent reason in the way described given \gsi{} and availability of the \aben{}.

  For example, the following statement is an instance of \gsi{}:
  \begin{enumerate}
  \item Any person who has the (general) ability to reason with the rules of chess has the (specific) ability to identify Alekhine's Defense as a fine opening move.
  \end{enumerate}
  The universal quantifier implies that the statement is true with respect to me, among others.
  Still, I am confident that there is at least one other person who has the ability to reason with the rules of chess, and may therefore infer that Alekhine's Defense as a fine opening move without appealing to my own ability.
  Indeed, if I am inclined to doubt my own (general) ability in contrast to a Grandmaster, then I may be more confident that Alekhine's Defense as a fine opening move if I appeal to the existence of a Grandmaster.

  Again, it is consistent with \eA{} that an agent may reason in such a way.
  Still, in defence of \eA{} it is important to note that \gsi{} information may be limited to the agent in question.
  For example, I may have studied your notes on how to play chess and identified a strategy which follows from those notes.
  I have no doubt that you have the ability to identify the same strategy, so when I provide \gsi{} my emphasis is on whether you have the ability to reason with \emph{chess}, rather than some closely related game.

  There are many ways to build context so that an agents is required to reason with \gsi{} and the \aben{} if the agent is to reason with (specific) ability at all, but I doubt these are required.
  The reasoning described by \eA{} (and illustrated above) seems plain and permissible.
\end{note}

\begin{note}
  Finally, \gsi{} and the \aben{} are constraints which do not hold in all cases of reasoning with specific ability.

  For example, one may be told that a gift of a metal detector grants the ability to discover if there is buried treasure in the garden.
  The former does not entail that there is buried treasure in the garden, and testimony or the metal detector may be claimed as support for the ability.

  So, question about whether this really does anything for general understanding of ability.
  \gsi{} and the \aben{} combine to require a particular interpretation.
  However, interpretation with general applicability is not restricted to instances in which it is forced.
  The role of a counterexample is not (typically) to establish that every instance of a theory is mistaken, but to identify a gap.
  And, even if the original theory may be restricted to non-problematic cases, the alternative theory may compete with the original theory.
  So, given that the particular interpretation is required to hold given additional stipulations, interest is in whether it holds without additional stipulations.
\end{note}

\subsection{Two (schematic) interpretations of (specific) ability}
\label{sec:wr-ar}

\begin{note}
  In the previous section we introduced \gsi{-} and the \aben{}.
  In the present section we motivate two interpretations of (specific) ability in the context of reasoning to (specific) ability from \gsi{} and reasoning from (specific) ability with the \aben{}.

  The two interpretations are termed `\AR{}' and `\WR{}' in turn, and are schematic.
  Roughly:
  \AR{} holds that when appealing to (specific) ability an agent appeals to some property or attribute that they have.
  And, by contrast, \WR{} holds that when appealing to (specific) ability an agent appeals to the action that they would perform by witnessing the relevant ability.
  \AR{} and \WR{} are distinguished, then, by whether an agent reasons with a property (\AR{}) or an event (\WR{}).

  To illustrate by analogy, consider a mechanical clock.
  The clock has the property of displaying the correct time, by it is also involved in the event of changing it's configuration as time passes.
  The property that the clock is displaying the correct time is important for determining whether one is late for a meeting.
  By contrast, the event of changing it's configuration as time passes is important for determining when to remove a brewing teabag.
  A meeting starts at a certain point in time, while tea is brewed over a period of time.
  If the clock does not represent the correct time, then three minutes passing will not, in general, help determine whether one is late to the meeting.
  And, whether or not it is 3pm is not, in general, important with respect to whether or not the tea has finished brewing.
  The qualifier `in general' is important.
  Measuring the passage of is useful if I know the length of time before the meeting is due, and the correct time is useful if I know when I started brewing the tea.

  The distinction between \AR{} and \WR{} is similar.
  Both interpretations may be more or less useful in certain circumstances, and interchangeable in others.
  Still, the combination of \gsi{} and the \aben{} identify a pattern of reasoning in which we may elaborate how the relevant interpretation of (specific) ability is important, and in turn broader principles (\ESU{} and, to be introduced below, \nI{}) will constrain whether the interpretations are permissible.

  We begin by introducing \AR{} and \WR{} with respect to the \aben{} in section~\ref{sec:ar-wr-1}.
  Then, in subsection~\ref{sec:ar-wr-gsi} we will relate \AR{} and \WR{} to \gsi{}.
  Finally, in subsection~\ref{sec:ar-wr-are} we will argue that \AR{} and \WR{} are exhaustive, though not exclusive, interpretations of (specific) ability.
  The following two sections~\ref{sec:first-conditional} and~\ref{sec:second-conditional} will then demonstrate how \WR{} and \AR{} conflict with broader principles.
\end{note}

\begin{note}[\dd{} and \dr{}]
  \color{red}
  \AR{} and \WR{} make use of the \dd{} and \dr{} distinction.
  A note on how this distinction is applied.

  For, we've talked about reasoning in terms of assigning values to propositions.
  Familiar case.
  \(\exists xPx\) vs \(Pa\).
  Here, the distinction is with respect to reference.
  With \(Pa\) the reference is to \(a\).
  With \(\exists xPx\) there's no reference to any particular object.

  The way in which we use the distinction is different.
  With respect to entailment.

  \(\exists xPx\) entails something \(\phi\).
  Two ways of viewing this.
  First, there is some referent of \(x\).
  And, the referent does the work in establishing \(\phi\).
  In this sense, \dr{}.
  Alternative is that \(\exists xPx\) entails \(\phi\).
  In this sense, \dd{}.

  The contrast is \dd{} uses the value assigned to the proposition to establish \(\phi\).
  By contrast, \dr{} uses the referent of the proposition to establish \(\phi\).

  So, the distinction also applies to \(Pa\).
  \dd{}, the value assigned to \(Pa\).
  \dr{}, that \(P\) holds of \(a\).

  Key difference is with how entailment works.
  Entailment because of value assigned --- \dd{}
  Entailment because of what is the case given value assigned -- \dr{}.

  So, apply to reasoning.
  \(\phi \vdash \psi\).
  Claim support for \(\phi\).
  It is true that there are premises and steps of reasoning that ensure that \(\psi\) is true.
  This is \dd{}.
  Note, there's an implicit conditional here.
  By contrast, \(\phi\) and steps of reasoning entail \(\psi\).
  This is \dr{}.
  There's no implicit conditional.

  The most basic way to view this distinction is in terms of syntactic and semantic entailment.
  Syntactic is \dd{} and semantic is \dr{}.
  However, this is hard to apply.
  For, syntactic entailment is just a matter of syntax.
  By contrast, when an agent reasons \dd{} it's because the proposition has a certain value --- not that the relevant premises has a certain form.

  Indeed, standard modal semantics doesn't help.
  Here, function of the modal may be seen as shifting reference.

  Rather, distinction is captured by semantics of propositional versus first (or higher order) logic.

  I suspect this may be seen as related to the more familiar distinction.
  However, not sure how to fix the relation at this point in time.
  Intuitively, any reasoning follows from description, and breaks down with co-reference, etc.
  However, with respect to the standard modal semantics, things don't really work out this way.
\end{note}

\begin{note}[Table]
  \begin{figure}[H]
    \centering
    \begin{tblr}{abovesep=8pt, belowsep=8pt, width=0.95\textwidth, colspec={Q[c,m]|Q[c,m]|Q[1.8,c,m]|Q[1.8,c,m]}}
      \multicolumn{2}{c}{} & \dd{} & \dr{} \\
      \hline
      \multicolumn{2}{c}{\WR{}} & ? & ? \\
      \hline
      \multirow{2}{*}{\AR{}} & Basic & ? & ? \\
      \cline[dashed]{2-4}
      & Derived & ?  & ? \\
    \end{tblr}
    \caption{Distinction matrix. \\ Rows are interpretations of ability, columns are type of reasoning regarding ability.}
  \end{figure}
\end{note}

\subsection{Reasoning \dd{} and reasoning \dr{}}
\label{sec:reas-dd-reas}

\emph{de constructione} \emph{de materia}
Roughly, of the construction and of the source.

So, reasoning that applies to a proposition as a whole.
Reasoning that applies to constituents of the proposition.

\begin{note}[Structure]
  The goal here is to build an understanding of the different types of reasoning.
  \begin{itemize}
  \item \dd{}/\dr{} distinction.
  \item Key is that this focuses on an object.
\end{note}

Have the idea of a proposition.
Something which gets a value.
Reason in terms of preservation of value.
Propositional logic as an example.
I don't need to consider the details of the propositions.
\(\exists xPx \land \exists xQx \vdash \exists x Px\)

However, not all reasoning is like this.
Consider first order logic.
Here, reason about constituents of a proposition.
So, for example, applying some quantifier rule.
\(\exists x(Px \land Qx) \vdash \exists x Px\)

This is quite broad.
Key is that this narrows down to parts.
So, in the last example, \(x\) is \emph{de materia} while predicates are \emph{de constructione}.
It doesn't matter what's predicate.
Though, I suspect this is an uncommon way of reasoning.
For most there's no need to generalise.

So, of interest is the reverse.
\emph{de materia} for predicate, \emph{de constructione} for objects.
Don't reason about any particular object.

I don't know the specifics.
However, parallel.
Idea is that there's no need to reason about an object.
That there is an object is sufficient.\nolinebreak
\footnote{
  Syntax and semantics in a way.
  However, tricky.
  Still, interesting to observe problems with identifying intended models.
}

{
  \color{red}
  This is leading to a distinct between `there is a' and `the'.
  Though, that's just a gloss.
}

So, the key is whether or not the agent appeals to reference.


This doesn't quite match the established \emph{de dicto} and \emph{de re} distinction.
Difference is that dd/dr talks about the type of reference.
While, the distinction being made talks about whether or not agent uses reference.

Though there are parallels.
\emph{de re} then object, then, typically thing of \emph{de constructione} in order for the substitution to go through.
However, \emph{de materia} from \emph{de re}.
There's no requirement that one needs to reason about the thing in order to arrive.
Agent could appeal to these terms being co-referential, without going through the reference.

\emph{de dicto} then no reference, so \emph{de materia}.
However, \emph{de constructione} from \emph{de dicto}.
As above, remove the quantifier and reason about whatever object satisfies requirements.





So, we have the idea of a term.
Terms refer.
Some object or property.
Don't require reference in order to reason.
So, \dd{} if agent doesn't go for interpretation, and \dr{} if agent does go for interpretation.

So, agent's reasoning about propositions.
So, something maps to propositions.
Question is the mapping.

First, break down propositions.
Suppose proposition \dots
With respect to this proposition \dots Book, table, myself, etc.
Distinction between these different things.
Okay, so the way I think about each of these influences proposition.
Reference.
Here, compositionality.
Agent's relation to a situation is determined by relation to constituents of situation.

Concerning constituent.
Concerning any constituent compatible with the rest of the situation.

Break down with example.
Key is that predicates are part of the situation, and these constraint the second.

So, this is how I'm drawing the distinction.


Relation to other conceptions.
Sensitivity to how agent is related to things.
Context of propositional attitude reports.
\dd{} and \dr{}.

SEP
Semantically de re/de dicto:
A sentence is semantically de re just in case it permits substitution of co-designating terms salva veritate.
Otherwise, it is semantically de dicto.

Difference, sentence rather than object.
Reformulate.
So, not restricted to sentences.
Slava veritate, so just truth, but reformulate.

Whether replace terms.
So, somewhat.

If \dr{} then direct.
Rest of the situation doesn't matter.
If \dd{} then likewise, different ways of constructing the situation will lead to changes.

However, if direct then might not be \dr{}.
For, 


If \dr{} then works out, because the term isn't important.
If \dd{} then likewise, if different terms influence differently.

However, going the other way.
If \dd{} then less clear.
Not possible to redescribe the situation, 





Distinction of important here is how reference functions.
\dr{} and \dd{} to some object.
\dr{} specific object.
\dd{} object satisfying constraints placed on reference.




Propositions are ways in which the world could be.
However, propositions can't be identified with a particular the way the world could be.
First, not specific enough.
Some proposition doesn't say anything about other parts of the world.
So, have some mapping between propositions and ways the world could be.




So, familiar example.
Tallest spy.
\dd{}, don't go for individual, \dr{} go for individual.
Different entailments from belief.
Examples from the literature.

The thing to keep in mind is that reference is/may remain important, but in case of \dd{} the agent is reasoning about constraints on resolution, rather \dots


\subsection{\AR{} and \WR{}}
\label{sec:ar-wr-1}

\begin{note}[\WR{} and \AR{}]
  We term the two schematic interpretations of the \aben{} `\AR{}' and `\WR{}', respectively.
  Brief descriptions from detached perspective.
  Given that the interpretations are schematic, they fall short of a full account of how an agent claims support by an \aben{}.
  However, the arguments to follow are of interest in part because they concern any way in which the schematic interpretations are filled out.
\end{note}

{
  \color{red}
  I should emphasise that here we're interested in reasoning.
}

\begin{note}
  \begin{definition}[\AR{}]\label{A:s}
    An agent's reasoning with an \aben{} by claiming support for \(\phi\) from \emph{S} having ability to \emph{V} that \(\phi\) is an instance of \emph{\AR{}} when the agent holds that:
    \begin{enumerate}[label=\textsf{A}\arabic*., ref=(\textsf{A}\arabic*)]
    \item\label{A:s:1} \emph{S} has the ability to \emph{V} that \(\phi\) is or reduces to some (potentially complex) property \emph{P} of \emph{S}, and
    \item\label{A:s:2} \emph{P}, or some part of \emph{P}, entails that \(\phi\) is the case.\nolinebreak
      \footnote{Intuitively, because the agent could not have \emph{P} without \(\phi\) already being the case.
      The notion of entailment here does not require that \(\phi\) is true \emph{because} of \emph{P}.}
    \end{enumerate}
  \end{definition}

  {
    \color{red}
    \AR{} identifies instances of reasoning in which an agent applies the \aben{} by holding the ability to \emph{V} that \(\phi\) is a property of an agent.\nolinebreak
    \footnote{
      Note, this does not say anything about what the ability to \(\phi\) is.
      Rather, way in which the agent claims support.
    }
    Note, when appealing to the \aben{} an agent need not be aware of what the (potentially complex) property of \emph{S} is.
    Rather, claimed support that \emph{S} has the ability to \emph{V} that \(\phi\) allows the agent to claim support for the existence of some property of \emph{S} which in turn entails \(\phi\).
  }

  Now, generally speaking properties are things which may be predicated or attributed of other things.
  The coffee is hot, I am thirsty, my mouth is sensitive to heat, I am reckless, I am in pain, and so on\dots
  And, properties come cheap.
  For example, the participation of an agent in some event gives rise to a property that may be attributed to the agent.
  Specifically, the property of participating in the event.
  Moments ago I participated in the event of recklessly drinking hot coffee with a mouth that is sensitive to heat.
  Therefore, I have the property of participating in such an event.

  So,~\ref{A:s:1} is trivially true.
  When we speak of an agent having some ability we are predicating or attributing ability to an agent.
  However,~\ref{A:s:2} requires that the property entails that \(\phi\) is the case.
  And, it is not clear that an entailment which follows from an event is always reflected in the property of being a participant in the event.
  For example, it seems that I am in pain because I participated in the event of drinking hot coffee, \emph{not} because I have the property of having participated in the event of drinking hot coffee.
  By contrast, that I have the property of having participated in the event of drinking hot coffee entails that I have the property of having participated in the event of drinking something.

  % From~\ref{A:s:2} it must be the case that the relevant property entails \(\phi\).
  % And, from~\ref{A:s:3} the property must not analysed in terms of there being a potential event in which \emph{S} witnesses the act of \emph{V}ing that \(\phi\).
  % This is, from one perspective, an arbitrary restriction.
  % For example, if there is a potential event in which an agent witnesses the act of \emph{V}ing that \(\phi\), then the agent has the property of being a participant of that potential event.
  % From a different perspective,~\ref{A:s:3}

  Roughly, we may expect the property of interest is akin to having a heart, possessing ¥500, being of a certain age, and so on\dots

  {
    \color{red}
    Key idea with \AR{} is that the agent `directly' claims support for a property when using the \aben{}.
  }

  To illustrate \AR{} we focus on the idea of reducing the ability to \emph{V} that \(\phi\) to some (potentially complex) property of \emph{S}.
  Again, when appealing to the \aben{} an agent need not be aware of what the (potentially complex) property of \emph{S} is.
  Rather, these illustrations suggest that such properties exist.

  \begin{illustration}
    Consider the proposition that \emph{S} has the ability to hear that the birds are signing.
    Again, it seems the \aben{} holds, and one may infer that birds are singing.

    So, by \AR{} there is some (complex) property \emph{P} of \emph{S} such that \emph{P}, or some part of \emph{P}, entails the the birds are signing.

    Consider the complex property of a well-functioning auditory system and sufficient proximity to the birds singing.
    The property of having well-functioning auditory system ensures that \emph{S} has the ability to hear nearby noises.
    And, having well-functioning auditory system together sufficient proximity to the birds singing together ensure that \emph{S} has the ability to hear the nearby noise of the birds singing.

    The \aben{} follows from part of this complex property.
    If the agent has the property of being in sufficient proximity to the birds singing, then it follows that there are birds singing.
  \end{illustration}

  \begin{illustration}
    Consider the proposition that the prosecution has the ability to demonstrate that the defendant is guilty.
    Intuitively, the \aben{} holds, as it is not possible to demonstrate the guilt of an innocent person.\nolinebreak
    \footnote{
      It is a different matter to convince a jury of the guilt of an innocent person.
      And, the \aben{} does not seem to hold with respect to the ability to convince a jury that the defendant is guilty.
    }
    By \AR{} there is some (complex) property \emph{P} of the lawyer such that \emph{P}, or some part of \emph{P}, entails the guilt of the defendant.
    Say, the lawyer is in possession of evidence sufficient to establish guilt of the defendant.
    If so, it is a property of the lawyer that they are in possession of such evidence, and by assumption the evidence entails that the defendant is guilty.

    It seems possession of evidence alone may not be sufficient to establish that the lawyer has the ability to prove that the defendant is guilty.
    For, it is plausible that a lawyer may be in possession of evidence that they do not understand.
    However, as our interest is with the \aben{} it is sufficient to observe that the evidence alone entails the guilt of the defendant.
  \end{illustration}

  Again, these illustrations highlight ways in which \emph{S} having the ability to \emph{V} that \(\phi\) may be reduced to some (complex) property of \emph{S}.
  \AR{} does not hold that an agent identifies such a property when claimed support by an \aben{}.
  Rather, \AR{} holds that the agent reasons with ability as a property of the agent.
  Indeed, while these suggestions reduce ability to complex properties, \AR{} also admits of the possibility that the ability to \emph{V} that \(\phi\) is a basic property which does not admit of further analysis.
  If so, then it seems that the \aben{} must also be taken as basic.\nolinebreak
  \footnote{
    I lack any suggestion for how to understand \AR{} if the property is indeed basic, but there is no need to rule out this option ---  no part of the following arguments depend on whether or how these schemas may be substantiated.
  }
  So, to summarise.
  The distinguishing feature of \AR{} is that there are instances when an agent claims support for \(\phi\) from claimed support that \emph{S} has the ability to \emph{V} that \(\phi\) because the latter ensures that there is some property \emph{P} holds of \emph{S} and \emph{P} entails \(\phi\).
  If the agent has the ability to \emph{V} that \(\phi\), then there may also be some action, \emph{V}ing, that the agent may witness.
  However, as \AR{} appeals to some property, the witnessing event is irrelevant to the way in which the agent claims support for \(\phi\).
\end{note}

\begin{note}[\WR{} def.]
  {
    \color{red}
    Include: observation that the entailment may come from some property of the agent.
    The point of \WR{} is that the agent claims support for details of the event.
  }

  We now turn to \WR{}.
  \begin{definition}[\WR{}]
        An agent's reasoning with an \aben{} by claiming support for \(\phi\) from \emph{S} having ability to \emph{V} that \(\phi\) is an instance of \emph{\WR{}} when the agent holds that:
    \begin{enumerate}
    \item\label{WR:def:1} \emph{S} has the ability to \emph{V} that \(\phi\) \emph{if and only if} there is a potential event in which \emph{S} witnesses the act of \emph{V}ing that \(\phi\).
    \item\label{WR:def:2} Claim support for event or details of event.
    \item\label{WR:def:3} Details of the event in which \emph{S} witnesses the act of \emph{V}ing that \(\phi\), or part of the event, entails that \(\phi\) is the case.\nolinebreak
      \footnote{Again, intuitively, because there could not be a potential event in which \emph{S} witnesses the act of \emph{V}ing that \(\phi\) without \(\phi\) already being the case.
      The notion of entailment here does not require that \(\phi\) is true \emph{because} there is some potential event of the relevant kind.}
    \end{enumerate}
  \end{definition}

  {
    \color{red}
    ~\textcite{Rebuschi:2011ub} talk about \emph{de objecto} attitudes.
    This might be helpful given that the events are potential.
  }

  {
    \color{red}
    Key idea with \WR{} is that the agent appeals to certain things which follow from the event being witnessed.
    Whereas, \AR{} appeals to certain things which must be the case in order for the event to be witnessed.
  }

  {
    \color{red}
    Difference between the existence of an event (~\ref{WR:def:1}) and details of the event (~\ref{WR:def:2}).
    To clarify.
    \(\exists e(V(e) \land \text{agent} = \emph{S} \dots)\).
    \(\phi\) follows.
    However, there are two ways to think about this.
    First, the existential, second the event.
    \emph{De dicto} and \emph{de re}.
    \WR{} is \emph{de re}.

    Consider existential of individuals.
  }

  {
    \color{green}
    Before going into the details, it'll be helpful to highlight the big idea, especially with respect to how things (will) work out with the `master property' from \AR{}.
  }

  \WR{} identifies instances of reasoning in which an agent applies the \aben{} by holding that \emph{S} having the ability to \emph{V} that \(\phi\) ensures there is a possible event in which \emph{S} \emph{V}s that \(\phi\).
  And, in turn, there is a possible event in which \emph{S} \emph{V}s that \(\phi\) entails that \(\phi\) is the case.
  In contrast to \AR{}, when an agent claims support as an instance of \WR{} an agent reasons about what must be the case in order for \emph{S} to witness some ability, rather than what must be the case in order for \emph{S} to have the property of possessing some ability.


  We use the term `potential' in place of `possible' when describing the relevant event to highlight that the existence of the event is tied to an ability attribution.
  One may hold that a possible event is any event which is not impossible, and hence it is possible for an arbitrary agent to prove Fermat's Last Theorem.
  Yet, it seems most agent's lack the ability to prove Fermat's Last Theorem, and so `potential' serves to restrict quantifier over events which an agent has the ability to witness --- however the details of that quantification are resolved.

  {
    \color{red}
    \WR{} is more complex than \AR{}.
    There is some action that \emph{S} may witness.
    And, understand what the result of that action is.
    So, we have something akin to a counterfactual.
    However, the counterfactual only relies on witnessing.
    Further, particular status of \(\phi\).
    Hence, as witnessing is the only issue, \(\phi\) is the case.

    Third, regardless.
    \(\phi\) holds regardless, but it does not follow from this that if the agent reasons via \WR{} then support claimed for \(\phi\) would be independent of ability information.
    The agent must recognise that \(\phi\) must be the case regardless, but this doesn't require that the agent has any way of reasoning to \(\phi\) other than by witnessing their ability.
    The point is clearer when considering witnessed instances of reasoning.
    \emph{X} testified that \emph{p}.
    Claim support for \emph{p}.
    \emph{p} is not the case because \emph{X} testified that \emph{p}, though my only path to claim support is by appeal to the testimony of \emph{X}.
  }
  To illustrate.

  I have the ability to calculate that \(243 \div 3 = 82\).
  Pen and paper to hand, etc.\
  Result of this will be a calculation that \(243 \div 3 = 82\).
  However, my calculation is irrelevant to whether it is the case that \(243 \div 3 = 82\).
  Hence, it follows that \(243 \div 3 = 82\).

  Ability to discover that the ball is under the left cup.
  Raise the left cup, and identify the ball.
  Whether or not the ball is under the left cup is independent of this sequence of actions, and therefore it follows that the ball is under the left cup.

  Compare to cases where only gets the counterfactual.
  I have the ability to make it so that the heating is turned out.
  Plausibly, the heating is not on, and depends on witnessing the action of `making it so'.
\end{note}

\begin{note}[Difference]
  \AR{} focuses on whether something is true of the agent independent of what action they may perform.
  \(\phi\) follows from property.
  \WR{} focuses on an action the agent may perform.
  \(\phi\) follows from relation between \(\phi\) and possible witness of action.

  Given \AR{}, conjecture that ability is not important.
  A useful shorthand, but in principle do not need to highlight the act.
  In contrast, the act is required for \WR{}.

  With \AR{} the important thing is the property.
  With \WR{} the important thing is the witness.

  \[\text{Has}(S,\text{docs}) \land \text{Sufficient-to-show-guilt-of-defender}(\text{docs})\]

  Ability here is to ensure that there is some property of this kind.

  \[\exists e(\text{Calculating}(e) \land \text{agent}(e) = S \land \text{result}(e) = (243 \div 3 = 8))\]

  Role of ability to secure witnessing event.

  The distinction may be highlighted by a distinct set of implications\nolinebreak
  \footnote{
    Though not necessarily entailments.
  }
  \nagent{4} is dehydrated, so \nagent{4} is tired.
  \nagent{4} took a long walk in the sun, so \nagent{4} is tired.

  First, implication follows from some property.
  Second, implication follows from the result of some action.
  {
    \color{red}
    The really difficult thing to clear up here is that the entailment is not from the \emph{existence} of an event, but rather from \emph{the} event.
    Consider \(\exists Fx \vdash \exists Gx\).
    Truth conditions.
    Need to find some individual.
    The entailment doesn't hold because of the quantifiers.
    Rather, it holds because of how we interpret the quantifiers.
    It goes down to individuals.

    In a sense, this is the core thing I've puzzled about in this thesis.
    The right way to reason about events, and while there are different ways of doing so, these do lead to different results.
  }
  As with ability, both implications may be true.
  Still, difference in terms of whether one appeals to some property of \nagent{4}, or some action that \nagent{4} performed.
  As with ability, there is some ambiguity.
  There's the fact that \nagent{4} took a walk in the sun, and there's the action of \nagent{4} taking a walk in the sun.
\end{note}

\begin{note}[More on \WR{}]
  \color{red}
  Using the event semantics representation.
  Easy to represent possible event, here existential quantification is replaced.
  In order to represent potential, things are a little more tricky.
  One option is a different quantifier.
  Another option is to restate components.
  Here, reasoning.
  For the verb, we have premises and steps.
  So, place these outside the scope of the quantifier.
\end{note}

\begin{note}[Why]
  So, \AR{}, the truth of some property.
  With \WR{}, it's the event that matters.
  In turn, moving from premises to some conclusion.
  Appeal to the event involves appeal to constituents of event.

  Return to \ESU{}.
  No inherent conflict with either \AR{} or \WR{}.
  Difference between property and witness.
  Requirement is that claimed support for premises is sufficient to claim support for conclusion.
  With \AR{}, claimed support for property --- need enough to be sure property is adequate.
  With \WR{}, claimed support for witness --- need enough to make sure that witness is adequate.

  \nagent{4} is thirsty, no implication.
  \nagent{4} walked, no implication.

  Does not matter that thirst is part of the relevant instance of being dehydrated.
  Nor that the witnessing event of walking was a long walk in the sun.
  Deny claimed support as did not reason from such premises.
\end{note}

\begin{note}[`Available resources']
  Delicate.
  Focus is on the witnessing event.
  However, mere possibility isn't sufficient for the \aben{}.
  So, some restriction.
  That is, an account of what makes the witnessing event a \emph{potential} event rather than a \emph{possible} event.
  One way to express this idea is that included in appeal to potential witnessing event is that sufficient resources are available.
  Here, the idea is that nothing further is required for the event to take place.

  This redescription falls short of an analysis as we've shifted the work from `potential' to `available'.
  Still, room for an analogy.
  Consider running a 5K.
  Here, going to require a whole bunch of energy.
  The agent does not `have' the energy.
  However, resources to generate energy.
  Fat reserves, muscle density, and so on.
  In this sense, sufficient resources are available, but not something the agent has.

  \AR{}, whatever it is that generates the sufficient resources.
  \WR{}, the result of having generated the sufficient resources.

  So, the difference between \AR{} and \WR{} isn't necessarily with what the two interpretations reduce down to, but is rather a difference with respect to what the interpretations focus on.
  From \AR{}, the stuff that's true right now, the generator, does the work.
  From \WR{}, it's what will be generated.

  There's still an important difference, though.
  Our interest is in reasoning.
  We are interested in what the agent appeals to.

  Key difference.
  \AR{}, that there is stuff the agent has which will generate.
  \WR{}, that what is generated from the stuff the agent has will do the work.

  The impact of this distinction will be expanded up with respect to \gsi{}.
\end{note}

\subsubsection{Summary of distinctions}
\label{sec:summary-distinctions}

\begin{note}[Table]
  \begin{figure}[H]
    \centering
    \begin{tblr}{abovesep=8pt, belowsep=8pt, width=0.95\textwidth, colspec={Q[c,m]|Q[c,m]|Q[1.8,c,m]|Q[1.8,c,m]}}
      \multicolumn{2}{c}{} & \dd{} & \dr{} \\
      \hline
      \multicolumn{2}{c}{\WR{}} & That there is an event in which \emph{S} \emph{V}s that \(\phi\) entails \(\phi\) & Details of an event in which \emph{S} \emph{V}s that \(\phi\) entail \(\phi\) \\
      \hline
      \multirow[c]{2}{*}{\AR{}} & Basic  & That \emph{S} has the ability to \emph{V} that \(\phi\) entails \(\phi\) & \emph{S}'s ability to \emph{V} that \(\phi\) entails \(\phi\) \\
      \cline[dashed]{2-4}
      & Derived & That there is some property \emph{P} (from \emph{S} having the ability to \emph{V} that \(\phi\)) entails \(\phi\) & (Some) property \emph{P} (from \emph{S} having the ability to \emph{V} that \(\phi\)) entails \(\phi\) \\
    \end{tblr}
    \caption{Distinction matrix}
  \end{figure}
\end{note}

\subsubsection{The distinctions are exhaustive}
\label{sec:ar-wr-are}

\begin{note}
  \color{red}
  This section is now far more straightforward.
  \dd{} and \dr{} should be easy.
  \AR{} and \WR{} should be easy.
  And, Basic and derived are easy given \AR{}.
\end{note}

\begin{note}
  The distinction between \AR{} and \WR{} sets up two (schematic) ways in which agent an agent may claim support given an instance of the \aben{}.
  We now argue that these two (schematic) methods are exhaustive.
  {
    \color{red}
    Important to keep in mind is that our interest is with claiming support.
    And, in particular, what the agent claims support for given \AR{} and \WR{}.
    So, the claim that \AR{} and \WR{} are exhaustive is a claim about how an agent reasons, not what ability reduces to.
  }
\end{note}

\begin{note}[Exhaustive]
  \begin{proposition}\label{prop:WR-and-AR-exhaustive}\label{either-AR-or-WR}
    Any interpretations of an agent's (specific) ability to \emph{V} that \(\phi\) (for which the \aben{} holds) conforms to either:
    \begin{enumerate}
    \item \AR{}: It is a property of the agent that they are able to \emph{V} that \(\phi\).
    \item \WR{}: There is a potential witnessing event in which the agent \emph{V}s that \(\phi\).
    \end{enumerate}
    \vspace{-\topsep}\vspace{-\topsep}
  \end{proposition}
\end{note}

\begin{note}[Argument]
  \color{red}
  Start with the basics.
  Have an instance of the \aben{}.
  So, the agent claim support for \(\phi\) given claimed support for \emph{S} having the ability to \emph{V} that \(\phi\).
  So, need to argue that the agent:
  Either claims support for some property of \emph{S} (\AR{}).
  Or, claim support for \(\phi\) as the result of the event of \emph{S} \emph{V}ing that \(\phi\), with 
\end{note}

\begin{note}[Old arguments]
  {
    \color{red}
    To argue for this\dots
    I can't say that these two things are equivalent.
    I mean, these aren't equivalent.
    \AR{} gets entailment from some property, while \WR{} gets entailment from some event.
    And, no clear entailment from property to event, nor vice-versa.
  }

  Switching between ability and potential events.
  This is not important, two ways of describing the same thing.
  The ability to \emph{V} that \(\phi\) is equivalent to there being a potential event in which the agent \emph{V}s that \(\phi\).
  For, if there is no such potential event, then the agent does not have the ability to \emph{V} that \(\phi\).
  Conversely, if there is a potential event in which the agent \emph{V}s that \(\phi\), then the agent has the ability to \emph{V} that \(\phi\).
  Rather, rewriting allows us to focus on the event.

  So, we may take `There is a potential witnessing event in which the agent \emph{V}s that \(\phi\)' as canonical.
  These are simply truth conditions.
  No commitment to what there is.

  Issue is how these truth conditions function to yield the potentive witness.
  Either:
  State of affairs such that there is a potential witnessing event.
  Or, refers to the witnessing event (which is potential).

  Illustrate.
  \emph{X} ran a 5K.
  To me, interpret as a state of affairs.
  \emph{X} buttered a slice of toast.
  Interpret as reference to an event that happened.

  To argue for exhaustivity, note the presence of the modal `potential'.\nolinebreak
  \footnote{
    May repeat the same argument here with ability, the particular representation doesn't matter.
  }
  Hence, distinction between what is, and what could be.
  What is: that there is the potential for a witnessing event.
  What could be: the witnessing event.

  Alternatively, what is the case, and what is not (at least yet) the case.

  Note, this distinction is not about what makes the proposition true.
  Rather, it is in how the truth conditions are used.

  Illustrate.
  S thinks that X and Y are different.
  Well, modal, thinks.
  So, there is some thought that S has.
  Or, the content of the thought.
  S did not say hello to Y.
  Explain by the content, so shift with the modal.
  That there is a thought which belongs to S isn't important from the point of view of S.
  Alternatively, therefore S considers X and Y referring terms.
  Here, we have no interest in the content of the thought.
  Instead, interest is limited to its functional properties, and use this to ascribe other properties to S.

  Hence, modal, and that's it.

  Remaining issue is details of the schemas.
  These talk about more than mere reference.
  \AR{}, agent, and \WR{} the result of the witnessing event.
  In turn, these are harmless and the only plausible option.

  \AR{} is simple.
  State of affairs, but as the agent is involved, then it is natural to attribute to the agent.
  Implausible that it's some event.

  \WR{} focuses attention to culmination of event.
  However, need culmination.
  Quirk of English that may `use' relevant verbs in this way.
  Imperfective paradox.
  May consider this a state, but only in the sense that it is a state bought about by some event.
  Focus on event, but given culmination, consider this a state.
  Still, state of culminated event.
  Possible that this is simply a state in which the agent has some appropriate relation.
  Problem is that an ability is the ability to do some thing.
  If abstract away from the act, then it's not clear how to understand conditions as identifying ability.
\end{note}

\subsubsection{Reflexive, reasoning}
\label{sec:reflexive}

{
  \color{red}
  Idea here is to narrow down \AR{} and \WR{} to reasoning.
  Then, I get an understanding of \WR{} in the agent claims support.
}

{
  \color{red}
  Okay, \ESU{} holds that claimed support for conclusion from steps and premises only if reasoned from steps and premises.
  Now, \WR{} holds that witnessing event.
  So, the agent needs to claim support that conclusion does follow from certain steps and premises.
  Therefore, \ESU{} and \WR{} are incompatible, because \ESU{} means that the agent is not in a position to claim support that the conclusion follows from premises and steps.
  So, there's an event in which \emph{S} reasons from premises to conclusion.
  So, conclusion is supported by premises and steps.


  Look, the idea here is that \ESU{} is going to rule out the agent claiming support for the conclusion.
  And, this happens because the agent appeals to a witnessing event.
  The agent only gets to claim \(\phi\) holds because there's reasoning which results in a \emph{V}ing of \(\phi\).
  So, the idea is that \ESU{} rules out making any claims about this kind of witnessing event.

  That is, so long as the agent isn't informed.
  So, it might be better to bundle \gsi{} and reasoning together.
}


\subsubsection{\AR{}, \WR{}, and \gsi{}}
\label{sec:ar-wr-gsi}

\begin{note}[Summarising]
  Above we introduced \gsi{}.
  Limited information of the form `If \emph{S} has a (general) ability to \(\gamma\), then \emph{S} has a (specific) ability to \emph{V} that \(\phi\) (as an instance of the general ability).'
  We then noted that certain instances of the (specific) ability to \emph{V} that \(\phi\) entail that \(\phi\) is the case.
  Two interpretations of the \aben{}, \AR{} and \WR{}.

  Our focus now turns back to \gsi{}.
  For those instances of \gsi{} when \aben{} holds, the interpretations \AR{} and \WR{} detail what the agent obtains by reasoning from general to specific ability.
  In other words, \emph{what} the agent is claiming support for.

  As noted, using a conditional such as \gsi{} is not automatic.
  The informer has not provided the agent with any additional way to claim support that the agent has the general ability.
  Rather, outlined something that follows \emph{if} the agent has the general ability.

  So, it is up to the agent to resolve in either way.
  If the agent wants to use the information, then the agent needs to reason from general to specific.
  The issue is that without any additional reasoning, it seems there's no clear way to determine which way the agent should go.
  Here is where the distinction between \AR{} and \WR{} is important.
  Interpretation of specific ability informs how the agent move from general to specific.

  Following two propositions outline combination.
  {
    \color{red}
    The key thing here is about claiming that one has a specific ability.
  }
\end{note}

\begin{note}[\gsi{}++]
  First, \gsi{} applied to \AR{}
  \begin{proposition}[\textsf{|gs-I\space·\space H|}]
    % In order for \emph{S} to have the (specific) ability to \emph{V} that \(\phi\) for which the \aben{} holds, claimed support for general and claimed support for \gsi{} are sufficient to claim support that \emph{S} has the property of being able to \emph{V} that \(\phi\).
    Suppose an agent has:
    \begin{enumerate}
    \item Claimed support for some general ability \(\gamma\).
    \item Claimed support that if they have the general ability \(\gamma\) then they have some specific ability to \emph{V} that \(\phi\) (for which the \aben{} holds).
    \end{enumerate}
    Then:
    \begin{enumerate}[resume]
    \item \emph{S} may claim support for having the specific ability \(\sigma\) by reasoning that they have the property of being able to \emph{V} that \(\phi\).
    \end{enumerate}
    \vspace{-\topsep}\vspace{-\topsep}
  \end{proposition}
  Second, \gsi{} applied to \WR{}
  \begin{proposition}[\textsf{|gs-I\space·\space W|}]\label{W:s}
    % In order for \emph{S} to have the (specific) ability to \emph{V} that \(\phi\) for which the \aben{} holds, claimed support for general and claimed support for \gsi{} are sufficient to claim support that there is a potential witnessing event in which \emph{S} \emph{V}s that \(\phi\).
        Suppose an agent has claimed support for some general ability \(\gamma\) and has claimed support that if they have the general ability \(\gamma\) then they have some specific ability to \emph{V} that \(\phi\) for which the \aben{} holds.
    Then, an agent may claim support for having the specific ability \(\sigma\) by reasoning that there is a potential witnessing event in which \emph{S} \emph{V}s that \(\phi\).
  \end{proposition}
\end{note}

\begin{note}[\gsi{}++ applied : \AR{}]
  \AR{} doesn't need to much expansion.
  Silent on what the property is.
  One way to view is that general ability reduces to sufficient collection of specific.
  \gsi{} conditional informs the agent that specific instance, so required for general ability.
  \gsi{} is novel, but support claimed is for quantifier over all core instances.

  Similar to a standard induction principle.

  With respect to chess, this is one such principle.
\end{note}

\begin{note}[\gsi{}++ applied : \WR{}]
  \WR{} is different.
  Witnessing event.
  So, \emph{V}ing that \(\phi\).
  Break down \emph{V}ing that \(\phi\) into a series of actions performed by the agent.
  General ability secures performing each of those actions.

  Similar to verifying an algorithm may be implemented.
  Break down all of the steps in the algorithm, and then ensure that it is possible to express each of the steps in the programming language of choice.

  \begin{quote}
    \textsc{factorial}(\(n\)):\newline
    \textbf{if} \(n = 1\)\newline
    \mbox{}\indent \textbf{return} \(1\)\newline
    \textbf{else}\newline
    \mbox{}\indent \textbf{return} \(n \times\) \textsc{factorial}(\(n-1\))
  \end{quote}

  Fortran 77 does not support recursion, a function may not call an instance of itself.\nolinebreak
  \footnote{
    This is not to say that one may not compute factorials using Fortran 77.
    It's a Turing complete language.
    However, would require a different (non-recursive) algorithm.
  }
  By contrast, the recursive factorial algorithm may implemented in languages that support recursion, such as Lisp or Python.

  Turning to the chess example.
  Here, appeal to sufficient understanding of the rules of chess, and the combination of these.
  More broadly, premises and steps of reasoning.

  It's this kind of stuff that \WR{} uses.
\end{note}

\begin{note}[Impact of distinction]
  Return to the impact of the distinction.

  \AR{}, focus on generator.
  Hence, task is to establish that the agent has resources to generate.
  So, in a sense, with \gsi{} we go from existence of general to existence of specific generator.
  Note, this isn't to say that there's something like a general generator.
  It may be the case by general ability we have quantification over specific abilities.
  If so, then claim support that there's a particular specific generator.
  Nor that specific generators are unique.
  The available resources may overlap.
  Still, some thing that is true of the agent, and claimed support for general ability is sufficient to claim support that the `some thing' holds.

  \WR{}, focus on generated event.
  So, agent doesn't necessarily need to establish a generator, but rather ensure that event may be generated.
  Hence, \gsi{}, general ability allows the agent to generate witnessing event for specific ability.
  Not looking to claim support that `some thing' is true of the agent.
  Rather, claimed support for general ability, and appeal to the actions that this allows the agent to perform.
\end{note}


\begin{note}[Use of terminology]
  \AR{} and \WR{} are favoured.
  For, \gsi{} and \aben{} are fixed for the remainder of the major argument.
\end{note}

\begin{note}[Intuition for \AR{} and \WR{}]
  Both \AR{} and \WR{} are ways to understand \aben{}, which is in turn about what is entailed by an agent having a (certain kind of) specific ability.

  \AR{} focuses on the idea that the agent may claim support from having the attribute (or the truth) of the specific ability.
  \AR{} requires support for attribute, which in turn suggests in a position to claim support for premises and steps.
  \AR{} doesn't require agent to claim support for premises and steps.

  \WR{} focuses on the idea that the agent may claim support from witnessing (or using) the specific ability.
  \WR{} requires support for premises and steps, which in turn suggests in a position to claim support for attribute.
  \WR{} doesn't require agent to claim support for attribute.
\end{note}

\begin{note}[Quite brief]
  Sketches of \AR{} and \WR{} are brief.
  Expand on these in the following sections (\ref{sec:first-conditional} and~\ref{sec:second-conditional}) to some extent, and chapter~\ref{cha:potent-infer-attr} will focus on a detailed account of both.
\end{note}


\subsection{\ESU{}, \gsi{}, and the \aben{}}
\label{sec:first-conditional}

\begin{note}[Summary]
  In this section we argue that \ESU{} constrains what an agent may claim support for when reasoning from general to specific ability.
  In short, the agent may only claim support for some property.
  In turn, if claimed support for specific ability, then any application of the \aben{} will be an instance of \AR{}.
  For, the agent has not claimed support for a witnessing event.



  In this section, argue that \ESU{} requires \AR{} because \ESU{} conflicts with \WR{}.

  Two ways corresponding to two sides of (specific) ability
  First, with respect to the \aben{}: appealing to (specific) ability.
  Second, with respect to \gsi{}: establishing (specific) ability from (general) ability.

  With \WR{}, it is potential witnessing event has central role.
  Issue is what this involves.
  Potential, so the only thing that's missing is to act.

  
  
  

  Now, issue is what is used when the agent appeals to potential witnessing event.
  That required resources are available.
  Interest here is that bare (specific) ability statement does not include specification of sufficient resources.
  Hence, conflict with \ESU{}.
  Appeal to sufficient resources, but agent has not used the resources.

  Here, something needs to be stressed.
  Sufficient resources is not a general property.
  We're interested in a witnessing event.
  Therefore, those resources that would be used to witness.
  If broad appeal to resources, then a case of \AR{}.

  To illustrate, back to factorial.
  How did I calculate the \(5! = 120\)?
  I used my ability to calculate factorials.
  Or, I recursively reasoning in line with algorithm.
  \(5 \times (4 \times (3 \times (2 \times 1)))\).
  and then\(2 \times 1 = 2\), \(3 \times 2 = 6\), \(4 \times 6 = 24\), and \(5 \times 24 = 120\).

  Key thing is the things that would be used in the event.
  Focus on the potential event, and so resources used in the event.

  Common issue: Potential event, so the agent doesn't identify or use resources.
  \gsi{} establishes potential witnessing event, ensure those resources are there.
  The \aben{} appeals to the potential witnessing event and so appeals to those resources.
\end{note}

\begin{note}[Expand: \gsi{}]
  Start with \gsi{}.

  Agent is claiming support for specific ability.
  Hence, claiming support that there is a potential event in which they \emph{V} that \(\phi\).
  Expanding potential event, claiming support that sufficient resources are available.
  Note, the agent may not (merely) \emph{expect} that sufficient resources are available, as availability of resources is part of claim for potential event.
  Rather, the agent may expect that there are no defeaters to claim that resources are available.

  To illustrate.
  Suppose I claim support that I know the train will be late.
  It's not (merely) that I expect that the train will be late.
  In order to claim support, some considerations sufficient to establish that the train will be late, and that there are no defeaters for these considerations.
  Expect would be absence of materia that train is on time.
  But absence alone doesn't push either way.\nolinebreak
  \footnote{
    Absence may be materia, though.
    For example, at least five minutes before train will arrive there is a message broadcast at the station.
    We are at the station, and it is a three minutes before the train is scheduled to arrive.
  }

  Task is to account for why an agent may claim support for availability of sufficient resources.
  In rough outline, answer is simple.
  Claimed support for general ability.
  Specific ability to \emph{V} that \(\phi\) is an `instance' of the general ability.
  So, given context, general ability supplements sufficient additional premises and steps of reasoning.

  However, without witnessing specific ability, agent is not aware of which additional premises and steps of reasoning are used.
\end{note}

\begin{note}[Expand: \aben{}]
  Conversely, with the \aben{} the agent claims support for \(\phi\).
  Similarly, claimed support, and the \aben{} takes (specific) ability to \emph{V} that \(\phi\) as main premise.
  Hence, the agent appeals to (and uses) (specific) ability to \emph{V} that \(\phi\) in order to claim support that \(\phi\).

  Here, agent uses appeals to sufficient resources.
  Key part of \WR{} interpretation is that the agent may witness \emph{V}ing that \(\phi\).
  Ability is understood in terms of potential witness.
  Therefore, appeal is made to sufficient resources.
  However, as the agent does not \emph{V} that \(\phi\), the agent is not aware of what those resources are.

  Stress again, at issue is not the \emph{de dicto} proposition that there are such sufficient resources.
  Rather, it's the \emph{de re} proposition that `those' resources are sufficient.
  The difficulty is that without witnessing, the agent has not determined what the referent of `those' is.

  Personally, I find the above the clearest statement of the issue with \WR{}.
  Focus on a witnessing event.
  So, refer to something.
  However, prior to witnessing, no referent is not fixed.
\end{note}

\begin{note}[Moving to incompatibility]
  Incompatibly with \ESU{} will be from common point of appeal to sufficient resources.
  To this we now turn.
\end{note}

\subsubsection{Constraints on reasoning with \gsi{} given \ESU{}}
\label{sec:incomp-wr-ura}

\begin{note}[Argument outline]
  \color{red}
  There are two issues.
  \begin{itemize}
  \item \ESU{} means that the agent may not `directly' establish the existence of a witnessing event.
    For, in order to do so, the agent would need to claim support that the conclusion follows from some collection of premises and steps of reasoning.
    However, as the agent does not witness, then this isn't possible given \ESU{}.
    \begin{itemize}
    \item The objection here is that the agent doesn't necessarily need to go directly.
      It's possible that the agent claim support for some property, hence gets specific ability, and then reasons that this means that there's a witnessing event.
      So, to the extent that \WR{} needs first the existence of a witnessing event, \ESU{} might be okay.
    \end{itemize}
  \item Second, the agent can't reason with the details of the witnessing event.
    This then blocks \WR{}, and does so conclusively.
    For, the agent is not permitted to appeal to a relation of support between premises and conclusion.
    The agent is only permitted to appeal to the existence of an event that would establish such a relation.
    So this is the main objection.
  \end{itemize}
\end{note}

Key proposition of this section.

\begin{note}[Proposition]
  \begin{proposition}[\WR{} is incompatible with~\ESU{}]\label{mcA:WR-and-denied-claim}
    \emph{If} \ESU{} is true \emph{then} \WR{} is an incorrect interpretation of (specific) abilities of the form \emph{S} has the ability to \emph{V} that \(\phi\) (for which the \aben{} entailment holds).
    {
      \color{red}
      If \ESU{} holds then when claiming support for specific ability by \gsi{} the agent claims support for some property.
    }
  \end{proposition}

  So, the key thing with this proposition is that in cases where an agent reasons with \gsi{-}, the agent claims support for a property.
  Hence, if an agent reasons from specific ability via the \aben{}, then must be an instance of \AR{}.

  Now, \autoref{mcA:WR-and-denied-claim} doesn't say that \ESU{} and \WR{} are incompatible in general.
  We'll see this in the argument for~\autoref{mcA:WR-and-denied-claim}.
  {
    \color{red}
    Simple argument is that \ESU{} would require the agent to witness the relevant ability.
    Objection to the simple argument is that premises and steps may be re-used, and therefore the agent may only need to produce a partial witness.
    However, then there's potential defeaters that the premises and steps used for the partial witness are not sufficient for a full witness.
    Is there a response to this?
    So, issue is whether partial witness would be enough to claim support.
    Well, the goal here is to argue that this would require the agent to have information that if these premises and steps hold up, then \(\phi\) must be the case.
  }


  {
    \color{green}
    Initial idea is that potential witnessing event is sufficient.
    The agent does not need to use any of the premises or steps.
    I think this is plausible.
    However, this is flawed as an argument.
    \WR{} only concerns the interpretation of specific ability.
    No restrictions have been placed on \gsi{} move or the \aben{} move.
    So, while it may be plausible that witnessing event is sufficient, this is not something which follows from the principles at issue.

    Different argument.
    Appeal to premises and steps.
    So, needs to use these.
    Requires the agent to witness the ability.

    Note, it seems a presupposition of this argument that premises and steps are fine grained.
    Each premise and step is considered distinct in order to achieve this.

    Not obvious.
    Propositional logic as an example.

    So: Objection, that it is not clear that the agent must fully witness.
    Therefore, partial or fragmented witness may be possible.

    However, logic motivation is less clear on inspection.
    When applying a rule it's always with reference to previous premises.
    Point here is that this is a type-token distinction.
    Conjunction introduction is a general sort of thing --- a type --- and there are instances --- tokens.
    The type conjunction introduction does not apply to any pair of propositions in particular.
    A token instance of conjunction introduction holds with respect to particular propositions.
    Contrast, the word type `Smith' does not refer to an individual, but a token may.
    So, in order to avoid witnessing, the agent would need to appeal to types, rather than tokens, but it is tokens that are important.
    It is rare for types to be used in reasoning.
    However, tokens are common.

    Two consequences.
    For \gsi{}, the agent needs to witness specific in order to claim support that they have specific.
    For the \aben{}, the agent needs to witness specific ability in order to claim support for conclusion.

    Both problem stem from common core of \WR{}: Witnessing events.

    In short. there's not plausible sense of `use' which doesn't require the agent to witness their ability with respect to \WR{}.
  }

  The argument that \WR{} is an incorrect interpretation of (specific) abilities of the form \emph{S} has the ability to \emph{V} that \(\phi\) (for which the \aben{} entailment holds) has two components.
  First, difficulty establishing by \gsi{}.
  Second, rules out the \aben{}.

  Difficulties with \gsi{}.
  Result of claiming support by \gsi{} is that agent claims support for specific.
  And, given \WR{}, this involves claiming support that some sufficient collection of premises and steps of reasoning are available to the agent.
  Given \ESU{}, the agent is required to use these steps and premises in order to appeal.
  Therefore, \ESU{} requires a partial witnessing event.
  Partial only, as the depending on how premises and steps are understood, certain premises or steps may be reused, and a single use may be sufficient for \ESU{}.

  The issue is strengthened when turning to the \aben{}.
  For, the conclusion is that \(\phi\) is the case.
  And, a partial witnessing event does not establish that \(\phi\) is the case.


  For,~\ref{P:ab-and-dc:W:ab} and~\ref{P:ab-and-dc:W:uRa}.
  Then, agent obtains support by~\ref{P:ab-and-dc:W:ab}.
  As \ESU{}, then from \ref{mcA:WR-and-denied-claim} not \WR{}.
  So, from~\ref{either-AR-or-WR}, must be \AR{}.
\end{note}

\begin{note}[To argument]
  {
    \color{red}
    We provided a brief argument for~\ref{either-AR-or-WR} in the previous section.
  }
  So, what follows is a brief argument for~\ref{mcA:WR-and-denied-claim}.
\end{note}

\begin{note}[Attribute]
  \WR{} is an instance of~\EAS{}, as the agent obtains support for the conclusion of the reasoning is able to do on the basis of the reasoning that would be performed in a witnessing event.
  Hence, the supported obtained for the conclusion is obtained on the basis of the support the agent has for the premises that would be used.
  Again, this does not imply that the agent obtains support for the conclusion which is equivalent to the support the agent would obtain by witnessing their ability by performing the reasoning.
\end{note}

\begin{note}[Compatibility]
  However, \AR{} suggests an alternative way to obtain support for the conclusion of reasoning the agent is able to do.
  Specifically, if order for the agent to \emph{have} the attribute of being able to reason to the conclusion, the conclusion of the reasoning must be true.
  The relevant entailment is in part secured by the verb chosen, and in part by what the verb is applied to.
  Here, `demonstrate' is a factive verb, if an agent demonstrates that \(\phi\), then it is true that \(\phi\).
  And, the existence of a chess strategy does not depend on the agent demonstrating that the relevant strategy exists.

  To take another example, you only have the ability to identify a typo on this page if there is a typo on this page.
  So, if I were to provide you with testimony that you have the ability to identify a typo on this page, you may begin searching for the typo, or you may note that there must be a typo in order for me to be in a position to provide you with testimony that you have the ability.

  The reasoning is summarised with the following sketch.

  \begin{enumerate}[label=(\textsf{A}\arabic*), ref=(\textsf{A}\arabic*)]
  \item\label{WR:Sketch:1} I have the attribute of being able to \emph{V} that \(\phi\).
  \item\label{WR:Sketch:2} In order to have the attribute of being able to \emph{V} that \(\phi\), \(\phi\) must be the case independent of whether or not I witness the ability.
  \item\label{WR:Sketch:3} \(\phi\) is the case.
  \end{enumerate}

  To keep things simple, we will refer to the principle behind the pattern sketched as \AR{}.
  And agent may bundle~\ref{WR:Sketch:1} and~\ref{WR:Sketch:3} into a conditional, and avoid instantiating the reasoning pattern, but so long as the conditional is (implicitly) held on the basis of the intermediate premise~\ref{WR:Sketch:2}, we take use of such a conditional to be an instance of \AR{}.

  \AR{} is compatible with \ESU{}.
  For, the two premises~\ref{WR:Sketch:1} and~\ref{WR:Sketch:2} are accessible to the agent, and obtaining \ref{WR:Sketch:3} from~\ref{WR:Sketch:1} and~\ref{WR:Sketch:2} appears to be straightforwardly sound reasoning.
\end{note}

\subsubsection{Summary}
\label{sec:uRa-and-wr-summary}

\begin{note}[Table]
  \begin{figure}[H]
    \centering
    \begin{tblr}{abovesep=8pt, belowsep=8pt, width=0.95\textwidth, colspec={Q[c,m]|Q[c,m]|Q[1.8,c,m]|Q[1.8,c,m]}}
      \multicolumn{2}{c}{} & \dd{} & \dr{} \\
      \hline
      \multicolumn{2}{c}{\WR{}} &  & Ruled out by \ESU{} \\
      \hline
      \multirow{2}{*}{\AR{}} & Basic  &  &  \\
      \cline[dashed]{2-4}
      & Derived &  & Ruled out by \ESU{} \\
    \end{tblr}
    \caption{Distinction matrix}
  \end{figure}
\end{note}

\begin{note}[Conditional A]
  The first conditional we establish highlights how \ESU{} constrains how an agent may use \gsi{} in the type of scenarios described by \eA{}.

  \begin{proposition}[\mcA{}]
  \begin{enumerate}[label=(C\Alph*), ref=(C\Alph*), series=CC_counter]
  \item\label{P:ab-and-dc:W} If
    \begin{enumerate}[label=(\roman*), ref=(CA.\roman*), series=CCA_counter]
    \item\label{P:ab-and-dc:W:ab} an agent may claim support for the conclusion of reasoning they are able to do in cases described by~\eA{}, and
    \item\label{P:ab-and-dc:W:uRa} \ESU{} is true,
    \end{enumerate}
    then
    \begin{enumerate}[label=(\roman*), ref=(CA.\roman*), resume*=CCA_counter]
    \item\label{P:ab-and-dc:W:AR} support for \(\phi\) is claimed from the agent having the attribute of being able to demonstrate that \(\phi\) (i.e.\ \AR{}).
    \end{enumerate}
  \end{enumerate}
  \vspace{-\topsep}\vspace{-\topsep}
\end{proposition}

  The reasoning described in the consequent of the conditional, \ref{P:ab-and-dc:W:AR}, is in line with \AR{} --- the support the agent obtains for the conclusion of the reasoning that they are able to do is obtained from the support the agent has for having the attribute of being able to reason to the conclusion.
\end{note}

\begin{note}[Summarising]
  ???
\end{note}

\begin{note}[Note on why \AR{} does not conflict with \ESU{}]
  \AR{} and \ESU{} are compatible.
  For, \AR{} is attribute.
  So, if instance of \AR{} agent is claiming support for attribute.
  This is accessed.
\end{note}

\subsection{\nI{}, \gsi{}, and \WR{}}
\label{sec:second-conditional}

\begin{note}[Redo of section]
  Seen \ESU{} and \WR{}.
  Turn to \AR{}.
  Also saw in last section certain kind of support required.

  Introduce a general constraint on claiming support.
  The general constraint will relate to moving from general to specific ability information --- agent is not in a position to claim support for having specific ability from information and claimed support for general ability.
  However, initial statement and motivation apply to all instances of claiming support.
  After statement and motivation, show how the constraint relates to \AR{}.
  If so, agent lacks support for having specific ability, and does not have the option of claiming support for result of specific ability by \AR{}.
\end{note}

\subsubsection{Statement of \nI{}}
\label{sec:ni-1}


\begin{note}[\nI{}]
  We turn to the general constraint on claiming support.
  \begin{proposition}[\nI{-}  --- \nI{}]\label{prem:ni}
    For any agent \(S\).
    Suppose:
    \begin{enumerate}[ref=(\textsf{NI}:\arabic*), series=nI_counter]
    \item\label{nI:claimed-support} \(S\) has claimed support that some proposition \(\phi\) has value \(v\) from materia \(M\) and continues to claim support for \(\phi\) having value \(v\) from materia \(M\).\nolinebreak
      \footnote{
        The agent has at some time in the past (perhaps a moment ago) claimed support for \(\phi\) having \(v\) from materia \(M\), and at the present time the agent continues to hold that \(\phi\) has value \(v\) from \(M\).
        In some cases, an agent may revise materia.
        E.g.\ this may be the case when appealing to memory, a new source, etc.
        Note, however, that this does not require that the agent initially claimed support for \(\phi\) from materia \(M\) --- may be in instance of memory.
      }
    \item\label{nI:received-info} \(S\) has claimed support that if proposition \(\phi\) has value \(v\) then proposition \(\psi\) has value \(v'\).
    \end{enumerate}
    And, suppose:
    \begin{enumerate}[ref=(\textsf{NI}:\arabic*), resume*=nI_counter]
    \item\label{nI:inclusion} \(S\) is confident that the claimed support for \(\phi\) having value \(v\) is mistaken or misled if \(S\) is not (given present context) in a position to claim support that \(\psi\) has value \(v'\) (without appealing to \(\phi\) having value \(v\)).\nolinebreak
      \footnote{
        Agent considers.
        This should be stressed.
        And, in most cases, it is mistaken that's at issue.
        For, failure of \(\psi\) condition doesn't raise any issue in particular for \(\phi\).
      }
    \end{enumerate}
    Then:
    \begin{enumerate}[ref=(\textsf{NI}:\arabic*), resume*=nI_counter]
    \item\label{nI:going-by-value} \(S\) is not permitted to claim support for \(\psi\) having value \(v'\) by appeal to:
      \begin{enumerate*}[label=(\roman*)]
      \item \(\phi\) having value \(v\) from~\ref{nI:claimed-support}, and
      \item the corresponding implication that \(\psi\) has value \(v'\) from~\ref{nI:received-info}.
      \end{enumerate*}
    \end{enumerate}
    \vspace{-\topsep}\vspace{-\topsep}
  \end{proposition}
\end{note}


\begin{note}
  \nI{} highlights a restriction on the way in which an agent may claim support for some proposition when certain conditions obtain.

  In this respect, \nI{} is similar to \citeauthor{Wright:2011wn}'s Transmission Failure template (\citeyear{Wright:2003aa,Wright:2011wn}) and \citeauthor{Weisberg:2010to}'s No Feedback condition (\citeyear{Weisberg:2010to}).
  We will highlight the contrast between \nI{} and similar principles in Chapter~\ref{cha:inertia}.
  For now, it may be helpful to highlight that \nI{} does not deny that any agent may claim support for \(\psi\) having value \(v'\).
  Rather,~\ref{nI:going-by-value} (only) denies that the agent may claim support for \(\psi\) in a specific way.

  In short, \nI{} captures an intuitive constraint that agent is not in a position to claim support for some proposition \(\psi\) by information that \(\phi\) `entails'\nolinebreak
  \footnote{Strictly speaking, \ref{nI:received-info} is more general than entailment.}
  \(\psi\) if failure to establish support for \(\psi\) independently of the value of \(\phi\) would reveal problem with the support claim for \(\phi\).

  Hence, \nI{} focuses on when an agent may claim support for some proposition by noting that (from the agent's perspective) that the value of the proposition is determined by further propositions the agent has claimed support for.
  From \ref{nI:received-info}, \(\phi\) to \(\psi\)
  And from \ref{nI:claimed-support} \(\phi\)
  So, \(\psi\), but \ref{nI:going-by-value} denies this given \ref{nI:inclusion}.
  Some other way of claiming support for \(\psi\).
  However, not merely an alternative path, but an alternative path that must be possible given claimed support for \(\phi\).

  Issue is that given \ref{nI:claimed-support} and~\ref{nI:inclusion}, agent expect that they have the resources, and hence expects that \(\psi\) is the case.

  So, that \(\phi\) has value \(v\).
  In doing so, resources to claim support for \(\psi\) has value \(v'\).
  Hence, that \(\psi\) has value \(v'\).
  So, \(\psi\) having value \(v'\) is a requirement on claimed support for \(\phi\) being any good.
  However, no support claimed for \(\psi = v'\).
  Even so, appeal to \ref{nI:received-info} does not help, because wouldn't get to \(\phi\) without \(\psi\).
  That value of \(\phi\) constrains value of \(\psi\) is useful information, but it useless because if \(\psi\) isn't already so constrained, then no appeal to \(\phi\).

  Similar to other principles, failure because establishing something that needs to be the case in order to be in a position to establish.

  To briefly illustrate:

  Some theory, proposed counterexample, some thing that the theory should deal with.
  So, if theory is true, explanation for why the proposed counterexample is not a genuine counterexample.
  Good support for theory.
  Still, problem to argue that the theory is true.
  This only works if the counterexample doesn't hold up.
  Need not a counterexample to get theory.

  I take this to be intuitively problematic.
  Indeed, it seems the researcher is require to take the alternative path --- to show that the proposed counterexample is accounted for by the contents of the theory, regardless of whether the theory is true.
  However, it is a little to quick.
  The brief illustration appeals to the general broad idea.
  Our interest with \nI{} is the details, the specific type of cases identified, and why.
  So, provide a more thorough example, then work through the conditions in some detail with further illustrations.
  With detailed understanding in hand, provide argument.
\end{note}

\subsubsection{Illustration of \nI{}}
\label{sec:illustration-ni}

\begin{note}[Illustration]
  To illustrate, let \(\phi\) be the proposition that a certain individual in a given picture is Wally.\nolinebreak
  \footnote{
    Or Waldo.
    Background on `Where's Wally?'.
  }
  I have searched the picture and identified an individual with (what I consider) a sufficient collection of identifying properties say: glasses, brown hair, blue trousers, and a red and white striped jumper).
  Hence, \ref{nI:claimed-support} is true, I have claimed support that the (relevant) individual is Wally.

  Let \(\psi\) be some other identifying trait, say that the individual is wearing \emph{round} glasses.
  Hence, \ref{nI:claimed-support} is true as I may claim support (via your testimony) that if a certain individual in a given illustration of a crowd of people is Wally then the individual is wearing round glasses.
  Note, it does not follow from information that Wally has an additional identifying property that the collection of properties I identified is insufficient.
  So I need not consider that my claimed support is mistaken (by being insufficient), though I may recognise that there is a potential defeater.

  The third condition~\ref{nI:inclusion} queries whether claimed support for finding Wally has a certain property.
  Would the claimed support for having found Wally be mistaken or misled if I am not in a position to claim support that the identified individual is not wearing round glasses?
  Clearly, I am mistaken or misled if the identified individual is not wearing round glasses, as Wally is always wearing a pair of round glasses.
  However, at issue is whether I am confident that I am mistaken or misled if I am not in a position to claim support that the individual is wearing round glasses (without appealing to the individual being Wally).
  Well, the book is nearby and I have the time to check.
  So, I \emph{expect} that am in a position to claim support that the individual is wearing glasses.
  Indeed, if I am not in such a position it must be because the individual is not wearing round glasses.
  And, if the individual is not wearing round glasses, then the individual is not Wally.
  If so, the support claimed that the individual is Wally would be both misleading and mistaken.
  Misleading as the individual is not Wally and mistaken as the traits identified were insufficient to identify the individual as Wally.\nolinebreak
  \footnote{
    \ref{nI:inclusion} restricts application of \nI{} to cases in which the agent `should' be in position to claim support for some other proposition.
    Hence: \emph{given} that you have the opportunity to reinspect the individual, there is a problem claiming support that the individual is Wally if you are not in position to claim support that the individual is not wearing round glasses.

    If I do not have the opportunity, then there may be no reason to consider the claimed support for the individual being Wally may be problematic.

    Variations:
    So, if the picture is destroyed, there's no interest.
    However, if faded, then okay.
    If rely on testimony, then still a problem.
    But, if sold the book and testimony, then fine.
  }

  Finally, the fourth condition~\ref{nI:going-by-value} denies that I may claim support that the individual is wearing glasses in a certain way.
  Specifically, I may not claim support that the individual is wearing rounded glasses from:
  \begin{enumerate*}
  \item it is true that the individual is Wally, and
  \item it is true that if the individual is Wally then the individual is wearing glasses, and the entailment that,
  \item it is true that the individual is wearing glasses.
  \end{enumerate*}
  We term this type of reasoning `\RBV{-}', and will expand below.
  For now, important observation is that in order to `reason by value' an agent infers from claimed support for a proposition having some value that the corresponding proposition has the value, and then reasons from the proposition having the value.

  So, I order to reason with value, use claimed support.
  Hence, I expect that I am in a position to claim support that the individual is wearing glasses (without appeal to the individual being Wally).

  Thing is, move to value, and because connected to support, then I need to hold it to be true.
  Path requires wearing glasses.
  So, then, need this as a background condition to move to value.
  Given that I am required to (implicitly) appeal to the individual wearing glasses in order to hold that the individual is Wally, then constraint on value does not allow me to establish wearing glasses solely from claimed support for being Wally.

  To ease intuition, consider again the `even if\dots' test with respect to proposed reasoning by value.
  Suppose the individual is not wearing glasses.
  Then, not in a position.
  Hence, misleading and mistaken.
  So, don't get Wally.
  Hence, RBV fails.

  Contrast.
  Suppose I've checked the picture and the individual is wearing glasses.
  Some other instance of \(\psi\).
  I am not confident that I am in a position to claim support for the proposition that this is the first time that Wally was drawn with before all the other characters in the picture.
  As before, potential defeater.
  Difference is that I do not require that this is the case to claim support that I have found Wally.
\end{note}

\begin{note}[\emph{Summary}]
  Detailed illustration.

  Potential defeater does not establish that claimed support is problematic, and hold that individual identified is Wally.
  However, I may not use to claim support that potential defeater does not obtain given that I (implicitly) require that potential defeater does not hold to use proposition claimed support for.

  Focus was on \ref{nI:inclusion} and~\ref{nI:going-by-value}, and in turn focus on this pair when providing details.
\end{note}

\subsubsection{Details of --- and argument for --- \nI{}}
\label{sec:details-ni}

\begin{note}
  \ref{nI:claimed-support} identifies claimed support that some proposition \(\phi\) has value \(v\).
  And \ref{nI:received-info} identifies claimed support that if \(\phi\) has vale \(v\) then some other proposition \(\psi\) has value \(v'\).

  It follows from \ref{nI:received-info} that \(\psi\) not having value \(v'\) is a potential defeater for claimed support for \(\phi\) having value \(v\).
  So, from this, the agent excepts \(\psi\) have value \(v'\).
  However, this is distinct from claiming support for \(\psi\) having value \(v'\).
  Again, we do not hold that an agent is required to claim support that potential defeaters do not obtain.
  At issue is whether the agent may use \ref{nI:claimed-support} and~\ref{nI:received-info} to claim support for \(\psi\) having value \(v'\).
  In this respect, \ref{nI:received-info} may be subsumed under \ref{nI:going-by-value}, but the independent condition allows a statement of the relevant instances of \(\phi\) and \(\psi\) (and values \(v\) and \(v'\)).

%   Stated in terms of value because this also holds for desires, and for probabilistic statements.
%   Desires, means end is easiest to demonstrate with.
%   Probability, think in terms of conditionalization, or in terms of entailment.
%   Truth of \(\phi\) then probability of \(\psi\) is \(40\%\).
%   If the probability of \(\phi\) is \(70\%\) then the probability of \(\psi\) is \(40\%\) (though the probability of \(\phi \land \psi\) may be \(28\%\)).
\end{note}

\begin{note}
  \ref{nI:inclusion} holds if the agent is confident that the claimed support for \(\phi\) having value \(v\) is mistaken or misled if the agent is not in a position to claim support for \(\psi\) having value \(v'\) (without appealing to \(\phi\) having value \(v\)), relative to the context the agent is in.

  Our use of the term `confidence' does not require the agent to have claimed support for the conditional content of \ref{nI:inclusion}.
  Nor does out use of `confidence' imply that the claimed support for \(\phi\) \emph{is} mistaken or misled given the identified conditions.
  We are interested only in what makes sense from the agent's perspective.
  Nearby reformulations of \ref{nI:inclusion} may also be true, but confidence is sufficient to recognise a problem.
  To illustrate: If I am confident that the water is poisoned, then regardless of whether I claimed support for the water being poisoned, I will not drink it.

  The parenthetical clause `(without appealing to \(\phi\) having value \(v\))' ensures that \ref{nI:inclusion} may only be true when the agent is confident that they are in a position to claim support for \(\psi\) having value \(v'\) independent of the conditional content of \ref{nI:received-info}.
  In this respect, \ref{nI:inclusion} requires an independent check on the claimed support for \(\phi\) having value \(v\).\nolinebreak
  \footnote{
    Note also that without the parenthetical clause, \nI{} would deny the possibility of any instance of the reasoning described in \ref{nI:going-by-value}.
  }

  Indeed, not being in a position to claim support for \(\psi\) having value \(v'\) (without appealing to \(\phi\) having value \(v\)) as a potential defeater to claimed support for \(\phi\) having value \(v\) is distinct from the potential defeater of \(\psi\) not having value \(v'\).
  For, an agent may consider \(\psi\) not having value \(v'\) is a potential defeater given \ref{nI:received-info} while being confident that they could not be in a position to claim support for \(\psi\) having value \(v'\) without claimed support for \(\phi\) having value \(v\).

  To illustrate, consider expert testimony to a layperson.
  Suppose you, the expert, have testified to me, the layperson, that there are exactly five intermediate logics that have the interpolation property.\nolinebreak
  \footnote{\cite{Maksimova:1977un}}
  From this it follows that there is one intermediate logics that have the interpolation property.
  However, I am quite confident that I would not be in a position to claim support for the latter proposition without your testimony.
  Given that I do not have the expertise involved, any failure by me to claim support that there is a intermediate logic with the interpolation property is uninformative.
  Likewise, given that I am a layperson I'm not in a position to rule out that there aren't intermediate logics with the interpolation property, and therefore I may consider this a potential defeater to your testimony.\nolinebreak
  \footnote{
    Additional example: reports of internal states.

    Without \ref{nI:received-info}, further difference.
    I have a virus scanner.
    Run this on your pc.
    Also, a test pc.
    Test PC contains a know virus, so if the virus scanner is good, then it will identify infection.
    However, no relation between your PC and my test PC.
    All that would be established is that the scanner is not good for claiming support.
    }

  Still, given \ref{nI:claimed-support}, agent may expect \(\psi\) to have value \(v'\), and may claim support.
  And, may expect to have the resources to claim support for \(\psi\) without appealing to \(\phi\) having value \(v\).
  To illustrate, suppose you and I are both experts.
  You claim to have developed a sound and complete proof system for an logic and presented me with a paper containing the system and a proof.
  Given that I have the paper and the expertise, I am confident that I would be mistaken or misled by your testimony if I am not in a position to claim support that the system is sound and complete by working through the paper.\nolinebreak
  \footnote{
    Here, complexity of understanding of having resources shows.
    For, it may be that the reader learns something new, a lemma etc.\ which could be considered a new resource.
    Likewise, one may think that it's fine to continue to follow testimony given a problematic proof as one is confident that the prover has the resources to revise the proof.
    If so, not clear whether conditional holds, and will depend having resources.
    If proof synthesises resources, then may still hold.
    If proof introduces new information, then conditional does not hold.

    No clear answer for these cases.
    Intend to be compatible with your understanding of resources.
    Will only take a stance on this when applying.
  }

  As examples demonstrate, difficulty is establishing this particular type of potential defeater given that context is required to determine whether agent is in a position.

  \ref{nI:inclusion} does not place any constraints on why the conditional is true.
  To help clarify, consider two broad ways in which conditional may be true.
  Inclusion, and Association.

  Inclusion is when a collection of resources may be (re)applied to establish a distinct proposition.
  If the support claimed for the initial proposition is neither misled or mistaken, then the agent will be in a position to appeal to the same resource used to claim support to establish the distinct proposition.
  To illustrate, consider claimed support that \(6^{2} \times 6^{3} = 6^{5}\).
  Support has been claimed by understanding basic properties of exponents.
  Hence, an agent may be confident that they are in a position to claim support that \(3^{15} \times 3^{12} = 3^{27}\).
  Indeed, working through problem exercises in a textbook is way of ensuring that one has understood such principles.

  Association is when claiming support for some proposition ensures the agent is in a position to appeal to some distinct collection of resources for some other proposition.
  To illustrate, consider claimed support that the WiFi is malfunctioning after troubleshooting one's own PC.
  Then, in a position to claim support that the public PC does not connect to the WiFi.
  Checking the public PC is a simple task, but distinct from troubleshooting one's own PC.\nolinebreak
  \footnote{
    Or, conflicting proofs and another branch of mathematics.
  }

  Still, neither Inclusion nor Association touch on how to determine when an agent is (given present context) in a position to claim support for some proposition.
  There are two issues.
  What the relevant context is, and what being in a position means.
  Unfortunately, I have no simple characterisation for either.
  The illustrations provided offer some intuition, and it seems these will have to do.
  For example, one may consider `in a position' to mean that the agent does not require any novel resources to claim support.
  However, an agent may need to synthesise more fundamental concepts when following a proof, and it is unclear whether the synthesis is `novel information'.
  Similarly, it is difficult to say what the present context is when an agent may phone a friend as a source of testimony.
  In some cases, corroborating testimony may be sufficient to claim support (another plausible instance of `novel information'), while in other cases at issue may be the agent's own understanding (e.g.\ with respect to cases of Inclusion).
  In defence of this latent ambiguity, the specifics will not matter when arguing for the truth of \nI{}.
  And, I suspect the cases to which we apply \nI{} will be sufficiently clear cut.

  Finally, the illustrations given have focused on true instances of \ref{nI:inclusion}.
  It is important that there are true instances of \ref{nI:inclusion}, but it is equally important not to suggest that these are in abundance.
  There are various ways in which \ref{nI:inclusion} may fail to be true.
  For example, if \(\phi\) is sufficiently general or probabilistic.
  If so, not having the resources to claim support \(\psi\) may not establish much.
  Highlight here is that the issue isn't with \ref{nI:received-info}, values are constrained.
  Rather, issue is with \ref{nI:inclusion}.
  Still, there is a nearby issue with \ref{nI:received-info} if the agent goes from likely to true.
  Here, \ref{nI:received-info} would not hold.
  No interest in this, I think this type of reasoning is fine, but \nI{} simply fails to apply.
\end{note}

\begin{note}[Background to Restriction]
  These are the background conditions.
  Final is the restriction.
\end{note}

\begin{note}[~\ref{nI:going-by-value}]
  \ref{nI:going-by-value} describes a particular way of claiming support.
  Termed this `\RBV{-}' above.
  For, reason about what follows from value.
  Same reasoning for why \(\psi\) is a potential defeater to \(\phi\).

  \ref{nI:going-by-value} prohibits this kind of reasoning given previous conditions obtain.
  For the moment we focus on what is prohibited, and defer why it is prohibited to argument below.

  Here, role of successful instance of \ref{nI:going-by-value} helps clarify.
  The function is to obtain something stronger than the expectation that potential defeater does not hold.
  If \RBV{} works out, then the agent does not merely expect, but provides materia to establish the potential defeater does not obtain.

  Conditional of \ref{nI:received-info} is important here, because the value, regardless of the state of the claimed support for the value, requires that potential defeater does not obtain.

  Illustration.
  Searching for YYY in the building.
  If I search, there are various defeaters I expect not to obtain.
  Ask secretary.
  If not in LLL, then not in building.
  Now I only need to check LLL.
  From this, \RBV{} and then claim support for potential defeaters of obscure places.
  Why?
  Because regardless of my reasoning, XXX not being in LLL entails that XXX in not in the building.
  Given this type of value information, establish an also-is relation of support.
  XXX not in LLL also-is support for XXX not in building.
  Without, don't get this.

  Following illustration:
  Distinctive is that \(\mathcal{V}(\phi) = v\).

  Highlight the distinction by considering alternative way of claiming support.
  Consider again the box.
  The box has certain dimensions, and calculations are only relevant given dimensions.
  So, the reasoning proceeds independently.
  There's nothing in the core part of the reasoning that requires the box itself to have the noted dimensions.
  At no point do I need to appeal to it being true.
  Rather, I background that it is true, and derive additional results.

  Contrast, if the alarm is beeping then there is a fire.
  It matters whether or not the alarm really is beeping.

  Here, a useful illustration is something like reliable then reliable on this instance.
  In this type of reasoning, it seems there's something missing between \(\phi\) and \(\psi\).
  So, this wouldn't be a case of \RBV{}.
  That is, there's something additional going on with \(\phi\) = reliable in general to \(\psi\) = true on this occasion.
  Indeed, it seems that depending on how \(\psi\) is understood, then either no \ref{nI:received-info} and hence no \RBV{}, or no \ref{nI:inclusion}, because a single failure is not sufficient to raise a problem with \(\phi\).
\end{note}

\newpage

\begin{note}[Structure of \nI{} and plan]


  Possible defeater for support for \(\phi\).
  However, possible defeater isn't necessarily a defeater.
  Key part to explaining \nI{} is why reasoning by value is bad, given that the agent doesn't have a clear problem with the claimed support for \(\phi\).

  Clarifying \incl{} and \RBV{} will allow a clearer statement.
  \nI{} will then follow from \incl{} and \RBV{} together with basic constraint on support \eiS{}.

  Important clarification, \nI{} is sufficient condition.
  Further, \RBV{} is \emph{a way} of claiming support.
  So, \nI{} does not imply that the agent is not in a position to claim support for \(\psi\), only that one way of claiming support is ruled out given \ref{nI:claimed-support}--\ref{nI:inclusion}.
  Following, as \nI{} is about claiming support, this says nothing about whether the agent has support --- in particular, if claimed support for \(\phi\) is support, then by \incl{}, plausible that the agent has support for \(\psi\) even if not in a position to claim when \RBV{}.

  Finally, no appeal to what the values of \(\phi\) and \(\psi\) (actually) are.
  \nI{} is `internal' in this sense.

  Looking forward, argument will be that \gsi{} sets up \incl{} and \AR{} requires \RBV{}.
  Hence, \nI{} rules out agent claiming support for specific ability by \AR{}.
  However, primary motivation will be independent of ability.

  Begin with clarifying \incl{} and \RBV{}, then general account of \nI{}, followed by a handful of examples, and concluding with ability.

  More in Chapter~\ref{sec:inertia}.
  Including, related literature. (No Feedback and Wright.)
\end{note}



\subsubsection{Argument for \nI{}}
\label{sec:argument-ni}

\begin{note}[Review of \nI{}]
  \incl{} and \RBV{}, now turn to why conflict with \incl{} and \RBV{}, as stated by \nI{}.

  \ref{nI:going-by-value}, going by value.
  So, going from support for \(\phi\) to \(\phi\), and then from \ref{nI:received-info} to \(\psi\) by value.

  However, denied claim of support for \(\psi\) by \incl{}.
  Turn to argument for this.
\end{note}

\begin{note}[\nI{} argument, state]
  Start with three conditions describing state of agent.

  By~\ref{nI:claimed-support}, agent has claimed support for \(\phi\), though recognises the support may be fallible.
  It is possible that there's a different value of \(\phi\) (misled), or even if that value, the claimed support is not an indicator (mistaken).

  And, by~\ref{nI:received-info}, agent has information that value of \(\phi\) constrains value of \(\psi\).
  Use of `information' allows for arbitrarily strong support.
  May assume that this is something known, though do not require for failure.

  Finally, by~\ref{nI:inclusion}, the agent is aware that the claimed support for \(\phi\) includes support for \(\psi\).
  That is, agent may reapply premises and steps to claim support for \(\psi\).
  Hence, the agent does not need to go by value of \(\phi\) to get \(\psi\), as premises and steps used to claim support for \(\phi\) did not require value of \(\phi\) (by \eiS{}).
\end{note}


\begin{note}[\nI{} argument, \RBV{}]
  From the three conditions describing state, it is possible for agent to claim support for \(\psi\).
  In particular, by reapplying and establishing how claimed support for \(\phi\) includes support for \(\psi\).

  \ref{nI:going-by-value} describes a way of obtaining support that does not appeal to inclusion of support.
  Agent moves from claimed support for \(\phi\) to value of \(\phi\), and given information, this would constrain value for \(\psi\), hence leading to support for \(\psi\) if \RBV{} is permitted.

  With the background provided, from a certain perspective, agent would be bypassing reapplication.
  Possible to claim support for \(\psi\) regardless of value of \(\phi\), but with information, observe constraint on value of \(\phi\) and hence on \(\psi\).
  As agent doesn't need to do work to establish how value is constrained by reasoning by value, this is quite easy.

  Issue is that \incl{} blocks \RBV{}.
  \eiS{} again.
  If claimed support, then independent of value.
  However, because \incl{}, problem moving from support for \(\phi\) to value of \(\phi\).
  For, in moving support for \(\phi\) to value of \(\phi\), agent is implicitly requiring value of \(\psi\).
  For, given \incl{}, if \(\psi\) does not have value, then claimed support for \(\phi\) would be mistaken or misleading, and hence would block move to value.
  So, any further reasoning by value is required to already have \(\psi\), and this conflicts with \eiS{} when the agent moves to claiming support for \(\psi\) as the result of \RBV{}.

  From a broader perspective, the agent doesn't get to claim support for \(\psi\) through \RBV{} because any move to value requires \(\psi\) to already be the case given the information the agent has.
  Hence, failure of \eiS{} if agent were to claim support for \(\psi\).
  Because of \incl{}, \RBV{} does not preserve \eiS{}.

  Illustrate, possible that the agent is mistaken/misled, and claimed support for \(\phi\) does not include support for \(\psi\).
  However, if \RBV{}, then claim support for \(\psi\).
  Problematic, because whether inclusion is a test for whether the claimed support for \(\phi\) is any good, but bypassing test when reasoning by value.

  No claiming support by noting value consequence, if failure to show value independent consequence would lead to revision of support.
\end{note}

\begin{note}[Summary, and testimony]
  Final case to summarise:
  Knowledge via testimony.
  This condition doesn't necessarily apply, as agent may not be in position to claim support for what follows from knowledge claim.

  Two reasons for this.
  First, agent may not be in a position to check.
  E.g.\ missing premise, or layperson, e.g.\ missing steps of reasoning.

  Second, agent may not need to \RBV{-}.
  For, if you've testified, then it follows from your statement.
  I don't need to appeal to me having heard from you.
  Instead, given the additional information that I have, you've already made the claim.
  Even if \(\phi\) doesn't have value, this is still an okay reinterpretation of the testimony you have provided.
  Here, to get the intuition, it's really not clear that I need to endorse that I do have the option to check.
  {
    \color{red}
    This point only really makes sense after the argument has been given.
  }
\end{note}

\subsubsection{Illustrations of \nI{}}
\label{sec:illustrations-ni}

\begin{note}
  Nothing particularly problematic about this, as given \incl{}, agent applies whatever support they have for \(\phi\) to get \(\psi\).

  Further, no issue with using \(\phi\) for other things.
  Here, only interest is in support.
  Hence, recognised by the agent that they may be misled.
  From this perspective, the issue is not ruling out potential defeaters.
  Similar to knowledge, etc.\ but no requirement that there are no defeaters.
\end{note}

\begin{note}[Abstract, so examples]
  The above highlights the problem.
  However, abstract.
  Turn to illustrations, and then to how \nI{} applies to \gsi{}.
\end{note}


\begin{note}
  May think that this restricts any application of \RBV{} to claimed support for \(\phi\) without value independent.
  This isn't quite right.
  \eiS{} keeps focus on \(\psi\).
  Only committed to \(\psi\) being a problem.
  Potential issue is no worse than any other instance of claim to support --- possibility of being mistaken or misled.
  If \(\psi\) ends up being used, then there's going to be a gap, where agent isn't in position to claim support by value, but unless eventual consequence is in turn used for \(\psi\), no clear problem --- at least not without stronger assumption.
\end{note}

\begin{note}
  \ESU{} is going to require the agent to reason from premises and steps `included' in claimed support for \(\phi\) in order to claim support for \(\psi\).
\end{note}


\begin{note}[Examples]
  Examples are somewhat difficult, due to complexities of state.
\end{note}

\begin{note}[Picture book]
  \begin{scenario}
    If not genuine, then missing serial number.
  \end{scenario}
  No need to reinspect, faults are support, so no serial number.
\end{note}

\begin{note}[Logic proof]
  \begin{scenario}
    If conjunction and negation are truth functionally complete, then disjunction and negation are truth functionally complete.

    And, claim of inclusion.
  \end{scenario}

  Here, in proving completeness, expressing other connectives.
  So, inclusion because agent will have shown how to switch between conjunction and disjunction.

  Well, claimed support for conjunction and negation.
  So, yes.
\end{note}

\begin{note}[Treasure]
  \begin{scenario}
    Claimed treasure only if learnt secret.
  \end{scenario}
  A little more interesting, as here, agent is going to have done something to learn secret when claiming support for treasure, but may not recognise that they've learnt the information.
  Of course, may be wrong treasure.
  Again, seems bad.
  But, if treasure then sell for \$X, seems fine.

  Useful, as earlier examples may seem to rely on easy checks, but putting pieces together to reveal secret may be quite difficult.
\end{note}

\begin{note}[Problematic]
  \begin{scenario}
    I walked \(15k\) yesterday, and it would not be true that I walked \(15k\) if I had only walked \(14k\).
  \end{scenario}
  In contrast to other cases, the conclusion is fine.
  Problem here is the reasoning.

  Going by value.
  No way to go to \(15k\) without having walked more than \(14k\).
  So, already require that not only \(14k\) is true when reasoning-by-value.
  Claim to support fails because I'm highlighting what's got to be true given support.

  However, clear to observe that claim to support includes not only \(15k\).
  Easy to reason that no matter whether I did walk \(15k\), that the support claimed for walking \(15k\) extends to cover not only \(14k\).

  Reasoning-by-value \emph{is} strange in this scenario.
  Do not need it to be the case that \(15k\) is true in order to deny only \(14k\).


  Useful to note, as there is still some reasoning with the value.
  However, the reasoning does not depend on specific value.
\end{note}

\begin{note}[Knaves]
  \begin{scenario}
    If X is speaking falsely, then Y is speaking truthfully.

    Knave says a bunch of things that you've got could support for being false, but could be true.
  \end{scenario}
  Variation on Knave problems.
  Again, there may be intuition that solving the problem is easily in reach, but I think this is a mistake.
  Knave problems are hard, and the difficulty doesn't seems to make a difference.
\end{note}

\newpage

\subsubsection{\nI{} applied to \gsi{} and \AR{}}
\label{sec:ni-applies-ar}

\begin{note}[Applying to type of scenario]
  Our attention now turns to how \nI{} applies to the use of \aben{} in scenarios of interest.

  The focus of our attention is whether an agent may claim support for having a specific ability given the claimed support for having a general ability, given \gsi{}.
\end{note}

\begin{note}[Checking conditions]
  Conditions \ref{nI:claimed-support} and~\ref{nI:received-info} are provided by the scenario.
  Condition~\ref{nI:inclusion} is obtained by reflection on ability.
  So, then, condition~\ref{nI:going-by-value} rules out a way of claiming support for specific ability.

  To argue for is that \AR{} implies \RBV{}.
\end{note}

\begin{note}
  If argument is successful, then agent will not be in a position to claim support for specific ability.
  This is the antecedent of the relevant use of \aben{}.
  Pair \nI{} with following supplement.

  \begin{proposition}[\nIm{}]
    An agent must have claimed support for the antecedent of an entailment in order to claim support for the consequent of the entailment via the entailment.\nolinebreak
    \footnote{To clarify, entailment is only about value.
      Think of conditional.

      So, does not follow that there being an entailment is a required part of agent's reasoning.
      \nIm{} is talking about when the agent appeals to an entailment, rather than any understanding of entailment beyond it being the case.
    }
  \end{proposition}
  \nIm{} seems indisputable,\nolinebreak
  \footnote{
    An agent may have some other way of claiming support for the consequent of the entailment.
    However, if the agent is not in a position to claim support for the antecedent, then the agent is not in a position to claim support because there is an entailment from the antecedent to the consequent.\nolinebreak

    For example, that the coin landed heads is entailed by Sam knowing that the coin landed heads.
    Here, entailment from \(K\phi\) to \(\phi\).

    Second, this light being on entails that the printer is out of paper.
    If agent appeals to entailment, again, need the light to be on.
    However, could look in the paper drawer, or modify the wiring so that an alarm sounds.

    However, Taylor is not in a position to claim support for the coin landed heads because Sam knows if Taylor has no idea whether Sam knows --- though Taylor may claim support by looking at the coin.
  }
  and so not in a position to claim support for result of witnessing ability via \AR{}.
\end{note}

\begin{note}
  Note on two ways of reasoning.
  Getting support for specific ability by \RBV{} and also getting result by \RBV{}.
  \nI{} only explicitly rules out first.
  However, with \nIm{}, second is implicitly ruled out, as if agent claims for value, then need to be able to claim support for premise.
  And, definition of \AR{} is such that going for premise is attribute.
  This is not necessarily required --- e.g.\ \WR{}.
\end{note}

\hozline{}

\begin{note}[Notes for \AR{}]
    {
    \color{green}
    Well, with the first, it's getting to \(\phi\).
    Doesn't seem the agent is in a position to use factive inference.
    Because, the agent is going from not possible to have ability and for \(\phi\) to be false.

    With the second, different.
    Because, the agent is going from application of ability providing support for \(\phi\).
  }

  Use of \AR{} gets quick argument for \RBV{}.
\end{note}

\hozline{}

\begin{note}[Application of \nI{} to \AR{}]
  \AR{}, working with attribute.
  So long as you have general ability, you have specific ability.
\end{note}


\begin{note}[Application of \nI{} to \AR{} argument]
  \gsi{} information.
  For \AR{}, agent is required to claim support that they have the specific ability.
  That is, claim support for consequent because the agent claims support for specific ability.
  This is distinguishing feature of \AR{} --- the agent is only appealing to having attribute.
  General characterisation of \AR{}, all ability from appeal to attribute.
  Contrast to \WR{}, where some use of ability.

  So, this means attribute for general and specific in cases of interest.
  For, \AR{} is attribute for all instances of ability.




    If agent appeals to general and information, then agent is appealing to having general attribute, and not only support for general attribute.
\end{note}

\hozline{}

\begin{note}[Distinguishing features]
  Reasoning has distinguishing features that pair well.
  \begin{itemize}
  \item Extends support.
  \end{itemize}
\end{note}


\begin{note}[Important points]
  Two important points:

  The role of~\nI{} is to highlight that the agent is not in a position to obtain support for (specific) ability in a certain way.
  That is,~\nI{} does not state that the agent may not obtain support for (specific) ability some other way.

  Second, so long as agent holds that they have general ability, then committed to truth.

  May be tempted to say that the agent is not committed, but this seems implausible.
  Cases of transmission failure, it seems agent does remain committed, at least.

  May take issue with information provided, especially if ideal.
  If informer has information, then they should say.
  In turn, not problem with~\nI{} as the agent would have support (via testimony) for specific ability.
  However, informer may only have the conditional.

  Ordinary agents.
  Maxims are broken.
  And, interest effects.
  Up to the agent.

  Seems puzzling, but not paradoxical.
\end{note}

\begin{note}[Why this is important]
  The key idea, and the foundation of the objection, is that the agent is going indirectly.
  The agent fails to show how the general ability extends to specific ability.
  For, the only things available to the agent is the constraint.

  This is useful independently.
  For, even if not convinced by~\nI{}, clear that given \gsi{} and~\ESU{}, the agent goes directly.
  And, something a little puzzling about this.
  Or, so I think.
\end{note}

\begin{note}[Dogmatism]
  Continuing relation to issues with knowledge.
  \autoref{prem:ni} is quite close to dogmatism paradox.
  If one knows that \(\phi\), then any evidence for \(\lnot \phi\) is misleading.

  Distinct again, however.
  For, don't have knowledge in the antecedent.
  Get the dogmatism paradox from the factivity of knowledge.
  No requirement that support for (general) ability is factive.

  Hence, role of the informer is important again, because agent is not in a position to come to the conditional by themselves prior to reasoning.
\end{note}

\hozline{}

\begin{note}[Finding tension, still]
  We have outlined a type of scenario built primarily on an agent receiving information that the agent has some specific ability so long as the agent has some general ability.
  The agent has support for having the general ability, but there are two ways in which the agent's support for having the general ability may be used to establish support for {\color{red} the result of having the specific ability} --- \AR{} and \WR{}.

  The previous section argued that~\ESU{} constrains how an agent may use the received information.
  If an agent is required to traces support from premises to conclusion through reasoning, then an agent may not appeal to the support for the premises and steps of reasoning that the agent would use to witness the specific ability.
  {\color{red} This is summarised in~\ref{P:ab-and-dc:W}.}

  The (initial) plausibility of~\ESU{}, then, suggests that the agent may only establish support for having the {\color{red} result of the specific ability} from the support they have for the general ability by \AR{}:
  The support the agent has for the general ability is support that it is true that the agent has the general ability.
  In turn, given the information received it is true that the agent has the specific ability, and it is only possible for the agent to have the specific ability if the result of witnessing the specific ability is true.

  The argument of this section is that the sketch of \AR{} given conflicts with a different, but equally plausible, premise.
  The premise concerns the way in which the agent obtains support for having the specific ability from the support for the general ability.
  We state conditional, the proceed to the premise.
  The initial statement of the premise is abstract and after providing a handful of clarifications we then link the premise to the type of scenario of interest.
\end{note}

\subsubsection{Incompatibility of \nI{}, \gsi{}, and \AR{}}
\label{sec:ni-summary}

\begin{note}[Table]
    \begin{figure}[h]
    \centering
    \begin{tblr}{abovesep=8pt, belowsep=8pt, width=0.95\textwidth, colspec={Q[c,m]|Q[c,m]|Q[1.8,c,m]|Q[1.8,c,m]}}
      \multicolumn{2}{c}{} & \dd{} & \dr{} \\
      \hline
      \multicolumn{2}{c}{\WR{}} & Ruled out by \nI{}  &  \\
      \hline
      \multirow{2}{*}{\AR{}} & Basic  & Ruled out by \nI{}  & Ruled out by \nI{}  \\
      \cline[dashed]{2-4}
      & Derived & Ruled out by \nI{}  &  \\
    \end{tblr}
    \caption{Distinction matrix}
  \end{figure}
\end{note}

\begin{note}[Conditional B]
  \begin{proposition}[\mcB{}]
    \begin{enumerate}[label=(C\Alph*), ref=(C\Alph*), resume*=CC_counter]
    \item\label{P:ab-and-dc:A} If
      \begin{enumerate}[label=(\roman*), ref=(CB.\roman*), series=CCB_counter]
      \item\label{P:ab-and-dc:A:ab} an agent obtains support for some proposition \(\phi\) on the basis of the agent's ability to demonstrate that \(\phi\) is the case, and
      \item\label{P:ab-and-dc:A:ni} \nI{} is true,
      \end{enumerate}
      then
      \begin{enumerate}[label=(\roman*), ref=(CB.\roman*), resume*=CCB_counter]
      \item\label{P:ab-and-dc:A:AR} the support for \(\phi\) \emph{is not} claimed on the basis of the agent having the attribute of being able to demonstrate that \(\phi\) (in line with \AR{}).
      \end{enumerate}
    \end{enumerate}
    \vspace{-\topsep}
  \end{proposition}
\end{note}

\begin{note}[Relating to other conditional]
  Given \mcA{}, the immediate interest with \mcB{} is that~\ref{P:ab-and-dc:W:AR} and~\ref{P:ab-and-dc:W:AR} are incompatible.
  So, if both conditionals are true, then (at least) one of the antecedents from either conditional is false.

  \ref{P:ab-and-dc:A:ab} repeats~\ref{P:ab-and-dc:W:ab}, roughly \eA{}.\nolinebreak
  \footnote{
    As with \mcA{}, could rephrase without \ref{P:ab-and-dc:A:ab}.
    If \nI{} is true, then not possible for agent to claim support for \(\phi\) on the basis of the agent having the attribute of being able to demonstrate that \(\phi\).
  }
  % Still,~\ref{P:ab-and-dc:A:ni} differs from~\ref{P:ab-and-dc:W:uRa}.
  \mcA{} assumed \ESU{}, principle we are proposing an exception for.
  \mcB{} assumes a principle, \nI{} to be argued for.
  Hence, if both \mcA{} and \mcB{} are true, either \eA{}, \ESU{}, or \nI{} is false.

  Before turning to \nI{}, observe general picture.
  If \eA{} and \nI{}, then \ESU{} is false.
  Hence, argument against \ESU{}.
  Further, \ref{either-AR-or-WR} then \WR{}.
  Motivates \EAS{}.
\end{note}



\begin{note}[Established conditional 2]
  Something about, if \eA{} then agent does not obtain support for the attribute.
\end{note}

\subsection{Establishing tension/summary}
\label{sec:establishing-tension}

\begin{note}[Table]
  \begin{figure}[H]
    \centering
    \begin{tblr}{abovesep=8pt, belowsep=8pt, width=0.95\textwidth, colspec={Q[c,m]|Q[c,m]|Q[1.8,c,m]|Q[1.8,c,m]}}
      \multicolumn{2}{c}{} & \dd{} & \dr{} \\
      \hline
      \multicolumn{2}{c}{\WR{}} & Ruled out by \nI{}  & Ruled out by \ESU{} \\
      \hline
      \multirow{2}{*}{\AR{}} & Basic  & Ruled out by \nI{}  & Ruled out by \nI{}  \\
      \cline[dashed]{2-4}
      & Derived & Ruled out by \nI{}  & Ruled out by \ESU{} \\
    \end{tblr}
    \caption{Distinction matrix}
  \end{figure}
\end{note}

\begin{note}[Summary]
  Given the two established conditionals~\ref{P:ab-and-dc:W} and~\ref{P:ab-and-dc:A}, the combination of the key premises of \ESU{},~\eA{}, and~\nI{} are in tension.

  For, combining~\ref{P:ab-and-dc:W} and~\ref{P:ab-and-dc:A} we have:
  \begin{enumerate}[label=(CC), ref=(CC)]
  \item If \eA{} is the case an agent obtains support for some proposition \(\phi\) on the basis of the agent's ability to demonstrate that \(\phi\) is the case then:
    \begin{enumerate}[label=(C\arabic*\(\sim\)), ]
    \item If \ESU{} is true, then the support for \(\phi\) is obtained on the basis of the agent having the attribute of being able to demonstrate that \(\phi\) (in line with \AR{}).
    \item If \nI{} is true, then the support for \(\phi\) \emph{may not be} obtained (in line with \AR{}) on the basis of the agent having the attribute of being able to demonstrate that \(\phi\).
    \end{enumerate}
  \end{enumerate}
  In short, if~\eA{} is the case then~\ESU{} requires a certain interpretation of the scenarios identified by~\eA{} and~\nI{} denies that the interpretation is plausible.
\end{note}


\begin{note}[Creating tension]
  Have:
  \begin{enumerate}
  \item \(\eA{}\)
  \item \(\ESU{} \rightarrow \lnot\WR{}\)
  \item \(\nI{} \rightarrow \lnot\AR{}\)
  \end{enumerate}
  Argued for these.

  Then,
  \begin{enumerate}
  \item \(\eA{} \rightarrow (\AR{} \lor \WR{})\)
  \end{enumerate}
  If agent claims support, either \AR{} or \WR{}.

  Then use Propositions.
  \begin{enumerate}
  \item \(\eA{} \rightarrow ((\AR{} \land \lnot\nI{}) \lor (\WR{} \land \lnot\ESU{}))\)
  \end{enumerate}
  Simplify by exclusive proposition again.
    \begin{enumerate}
  \item \(\eA{} \rightarrow ((\lnot\nI{}) \lor (\lnot\ESU{}))\)
  \end{enumerate}

  So,
  \begin{enumerate}
  \item \(\lnot\eA{} \lor \lnot\ESU{} \lor \lnot\nI{}\)
  \end{enumerate}
\end{note}

\begin{note}[Weak points]
  \eA{} and \(\eA{} \rightarrow (\AR{} \lor \WR{})\).

  Perhaps there is a third option.
  Though unclear what this would be, even in outline.

  Hence, that the agent has the option of claiming support.
  If so, though, it seems surprising that \aben{} is never used to claim support.
\end{note}

\begin{note}[Tension, choices]
  In short, we have the following resolutions.
  \begin{enumerate}
  \item\label{ten:res:nS} Agent may not obtain support for result of witnessing ability, or
  \item\label{ten:res:nD} Agent obtains support for result on the basis on premises that the agent would use to witness ability --- incompatible with general application of~\ESU{}
  \item\label{ten:res:nI} Agent obtains support for result from attribute of having the ability on the basis that the support they have for general ability would be misleading --- incompatible with general application of~\nI{}
  \end{enumerate}
  \ref{ten:res:nS} is incompatible with~\ref{ten:res:nD} and~\ref{ten:res:nI}.
  However, \ref{ten:res:nI} and~\ref{ten:res:nD} are compatible, as both~\ESU{} and~\nI{} may be restricted.
\end{note}

\begin{note}[Argument sketch recap]
  Let us recap the main points of the argument so far.
  \begin{enumerate}
  \item Assume possibility of cases in which agent is provided with information that they have some specific ability so long as the agent has a general ability, such that the agent has support for having the general ability, but has not established support for possessing the specific ability.
  \item In such cases, it seems it is possible for the agent to obtain support for what follows from the agent witnessing their specific ability.
  \item If so, the agent appeals to having the specific ability in order to obtain support for what follows from the agent witnessing their specific ability.
  \item Attribution, and witnessing.
  \item If witnessing, then conflict with the requirement that an agent must access support for the premises appealed to in support of a conclusion.
  \item If attribution, then conflict with the restriction that an agent may not obtain support for some proposition on the basis that support the agent has for some other proposition would be misleading otherwise.
  \end{enumerate}

  To follow:
  \begin{enumerate}
  \item Restricting~\ESU{} in favour of~\EAS{} works well.
  \end{enumerate}
\end{note}

\begin{note}[Meek outlook]
  This is not a clean argument.
  Take~\ESU{} and~\nI{} and hold the first.
  The agent may not obtain support.

  While there may be tension if the agent obtains support, this tension is never instantiated.

  I am sympathetic.

  Still, endorsing the restriction does not require the agent to obtain support in this case.
  Harbour some hope that that there is scope to restrict \ESU{}, and that the argument provided for resolving tension in favour of \EAS{}, along with later arguments, may serve as a source for reflection.
\end{note}

\begin{note}[\WR{} isn't required for interest]
  \mcB{} is quite interesting itself.
\end{note}

\section{Minor argument}
\label{sec:posit-argumn-overv}

\subsection{Cases}
\label{sec:cases}

\begin{note}
  Main role of minor argument is cases.
\end{note}

\begin{note}[Beyond belief]
  Application in particular to desire.
\end{note}


%%% Local Variables:
%%% mode: latex
%%% TeX-master: "master"
%%% End:
