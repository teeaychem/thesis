\chapter{Overview}
\label{cha:overview}

\section{Outline}
\label{sec:outline}

\begin{note}
  In this chapter we provide a high level overview of the main arguments made in this thesis.
  A significant part of the high level overview of arguments is an overview of the premises and assumptions that those arguments rest on.

  By given high level overview, clarify on how premises, assumptions, and conclusions relate.
  In the main body of the thesis, afford to elaborate.
  And, allow choice of where to seek elaboration.
\end{note}

\begin{note}
  Following introduction, interest is with ability.
  In particular, observation that \gsi{} information, and confidence in general ability seems to allow agent to claim support for result.

  Question on what basis the agent claims support.

  Slightly more general statement.
  \begin{quote}
    When and why an agent may claim support for the result of reasoning that the agent has not witnessed.
  \end{quote}
  Ability claims of interest provide some answer to `when'.
  For, ability provides information about what the result is, and also that the agent has the opportunity to perform the reasoning.\nolinebreak
  \footnote{No claim to necessity}

  Suggested in introduction, answer to `why' is more complex.
  % \begin{quote}
  %   Our interest is in when an agent may claim support for some conclusion of some instance of reasoning on the basis of the support the agent may claim for the premises of the instances of reasoning.
  % \end{quote}
\end{note}

\section{Main things}
\label{sec:main-things}

\begin{note}
  \begin{restatable}[\ESU{-} --- \ESU{}]{target}{targetESU}\label{denied-claim}
    An agent may claim support for some conclusion of reasoning by claiming that the conclusion of reasoning is supported by premises and steps of reasoning \emph{only if} the agent has witnessed the reasoning (e.g.\ traced the claimed support for those premises and steps used to claim support for the conclusion).
  \end{restatable}
\end{note}

\begin{note}
  \begin{restatable}[\EAS{-} --- \EAS{}]{goal}{goalEAS}\label{prop:EAS}
    If an agent has claimed support that they have the ability to (adequately) reason to some conclusion, then it may be permissible for the agent claim support for the conclusion by appealing to some materia \(M\) to claim support for \(\phi\) without using \(M\) in the reasoning which culminates in claiming support for \(\phi\).
  \end{restatable}
\end{note}

\begin{note}
  Where claimed support is\dots
\end{note}

\begin{note}
  \autoref{sec:abil-access-supp} works through claiming support amounts to in some detail.

  Enough to provide context for these things.
  And, foundation for core part of the argument.

  Then, turn to target and goal in some detail.

  Then, moving to the argument proper.
  Ability.
  Tension.
  Includes consequence of claiming support.
\end{note}

\section{Claiming support (and claimed support)}
\label{sec:abil-access-supp}

\paragraph*{Overview}

\begin{note}
  The present {\color{red} section} is about claiming support.
  Or, more precisely, claiming support and claimed support --- we are interested in both an activity and the result of that activity.

  The purpose of the following discussion of claiming support with respect to the {\color{red} overall project} is simple:

  Reasoning with ability, obtaining some conclusion.
  Incompatible with intuitive understanding of part of such reasoning and, in particular, an common idea appealed to in various arguments.

  In order to make such an argument we require some background constraints on such reasoning.
  So, this {\color{red} section} works though those background constraints.

  This then provides sufficiently precise context for what we arguing against, and what we are arguing for.

  And, as part of providing sufficiently precise context will include a rough template of the considerations which motivate argument.

  In other words, context, and also preview.
\end{note}

\begin{note}
  Two things to keep in mind.

  \begin{enumerate}
  \item These constraints are partial.
  \item Consequence of constraints that matters.
  \end{enumerate}

  Enough to identify certain ways in which an agent may fail to perform the reasoning of interest.

  Benefit of not building in `too much'.

  However, also possible to endorse additional constraints.
  In particular, rule out alternative.
  If so, also a problem.
  We will not give much attention to this.
  Argument rests on plausibility when applied to particular case.
  If agree with constraints and agree with particular case, then this is an indirect argument against stronger constraints.


  Consequence of constraints that matters.
  Preview of considerations that matter, but additional step.

  However, postpone final constraint until argument proper.
  Primarily because for the moment, enough to interpret what we are argument against and for --- consequence isn't required.
  Secondary, this constraint where it matters in the argument.
  Third, surrounding motivation and context within literature.

  Welcome to turn to \autoref{sec:second-conditional} following this section.
\end{note}

\paragraph*{Plan}

\begin{note}
  \begin{enumerate}
  \item `Claimed support' is a term, but with certain pragmatic implications.
  \item Two ideas.
  \item Assumptions from these ideas.
  \item Working through assumptions with respect to particular consequence.
  \item Additional comments.
  \end{enumerate}
\end{note}

\begin{note}
  Roughly speaking, the two ideas:
  \begin{enumerate}
  \item Possible that there are defeaters for claimed support.
  \item Claimed support involves reasoning that defeaters do not obtain.
  \end{enumerate}
\end{note}

\subsection{`Claiming'}
\label{sec:claiming}

\begin{note}[Introducing support]
  Initial clarification is with respect to claiming support.
  Emphasis on `\emph{claim}'.

  The thesis is not about when and why an agent has \emph{support}.

  There are three primary reasons why we focus on claimed support.

  First, neutral for main thread of argument on what support amounts to.
  Interest is with structure of claim, and background assumption that if success in claiming then structure of support follows structure of claim.

  Second, whether or not an agent has support often seems secondary.
  It may be that any claimed support for a proposition is support for that proposition, but perhaps not.
  \begin{illustration}[A box of flan(nels)]
    \label{illu:flan-nels}
    Suppose `flan' is written on the side of a container.
    I may claim support that the container contains flan.
    And, it may be that the writing on the side of the container is support for the box containing flan.
    However, the straps ensuring the container remains closed is unfortunately placed, and if moved would reveal the side of the container reads `flannels'.
  \end{illustration}

  The unfortunate placing of the straps does not seem to prevent \emph{claiming} support, but I'm not sure whether it is right to say that the writing on the side of the box (straps in place) \emph{does} provide support that the box containing flan.
  So, speaking in terms of claiming support leaves open whether what is claimed reflects on whether an agent has support.\nolinebreak
  \footnote{
    In particular, claiming allows focus on internal constrains, while remaining silent on whether having support is (in part) determined by external factors.
  }
  \(^{,}\)\nolinebreak
  \footnote{
    Distinction between propositional and doxastic support.
    Propositional, support agent has whether or not made a claim.
    Doxastic is successful claim and propositional support.
    So, both require that the agent has support.
    Claimed support is the agentive component of doxastic support.
    Not interested in whether the agent also has propositional support, though more or less assume.
  }
  \(^{,}\)\nolinebreak
  \footnote{
    {
      \color{red}
      English is somewhat difficult.
      It is somewhat unfortunate that `an agent has claimed support for \(\phi\)' may be read `there is support which the agent has claimed for \(\phi\)'.
      Still, this seems to follow more easily from `support claimed'.
      So, `claimed support' emphasises the claim, while `support claimed' emphasises support.
    }
  }

  Third, and following from the second, focusing on claimed support allows us to make no assumption about the relationship between claimed support and support.
  To elaborate, consider enthymematic inferences.
  One may hold that an agent may claim support for some conclusion via enthymematic inference, but hold that the support the agent has is understood from the perspective of the (corresponding) complete inference.\nolinebreak
  \footnote{
    Cf.\ \textcite{Moretti:2019wx}.
  }
  Alternatively, one may hold that the enthymematic argument is an adequate support relation (at least with respect to context in which the inference was made).

  Hence, one may question whether the structure of claimed support follows the structure of support.

  An important consequence of this final point is that we will only be interested in why (and when) an agent claims support for and from ability rather than why (or when) an agent \emph{has} an ability.
\end{note}

\subsection{Proposition and evaluation}

\begin{note}[Value proposition]
  Reasoning and claims to support focus.
  Briefly introduce a pair of propositions to clarify claim to support and reasoning.

  \begin{restatable}[Claimed support is for the value of a proposition]{assumption}{assuCSVP}\label{assu:CSVP}
    When an agent claims support for some proposition, the agent claims that the proposition has some value.
    Where:
    \begin{itemize}
    \item A proposition is some information. (Mode of presentation) And,
    \item A value is an assessment of that information.
    \end{itemize}
    \vspace{-\baselineskip}
  \end{restatable}
  \autoref{assu:CSVP} fixes terminology.
  To illustrate, when stating the conclusion of the reasoning sketched above we used the proposition that \emph{the area of the rectangle is \(133\text{cm}^{2}\)}.
  The proposition refers to the state of affairs in which the area of the rectangle is \(133\text{cm}^{2}\), and speaking a little more precisely, the agent claimed that the proposition has the value `true' --- though it may be the value turns out to be `false'.
  Or, perhaps if the agent was a little unsure about the accuracy of the ruler, that the proposition has the value `likely', `probable', or some quantitative credence.
  And, some other instance of reasoning may have concluded that the proposition has the value `desirable' --- e.g.\ if the agent was searching for a rectangle of some approximate size.\nolinebreak
  \footnote{
    Nothing in particular hangs on the distinction between different values.
    If you prefer, you may expand the proposition (state of affairs) to include additional factors, and consider only the values `true' and `false'.
    For example, the proposition that \emph{I desire the bath to be warm} is false, as opposed to the proposition that \emph{the bath is warm} is valued undesirable by me.
  }

  Core idea is that claim of support is that things are a certain way.
  Proposition, what the thing is.
  Value, the way it is.
\end{note}

\begin{note}
    \begin{itemize}
  \item \(p\) has the value `true'. \hfill (\emph{p} is true.)
  \item \(p\) has the value `ought to be'. \hfill (\emph{p} ought to be the case.)
  \item \(p\) has the value `desirable'. \hfill (\emph{p} is desirable.)
  \item \(p\) has the value `improbable'. \hfill (\emph{p} is improbable.)\nolinebreak
    \footnote{
      Probability is somewhat interesting.
      The value of the probability of \(p\) is below some threshold.
      E.g.\ the value of the probability of \(p\) is \(0.1\).
      So, on a surface reading the thing that is in a certain way is the probability of \(p\) rather than \(p\).
    }
  \end{itemize}
\end{note}

\begin{note}
In most cases the value will be clear (i.e. that the proposition is true, though sometimes that the proposition is desirable), and so we will talk of claiming support for the proposition.
  A handful of additional examples will be provided when illustrating the next proposition.
\end{note}

\begin{note}
  Nothing hangs on distinction between values.
  Reduce everything to truth and falsity.
  However, we do not assume this, and if you do not think this is the case either, then I would like to not suggest that the assumptions and arguments to follow concern only those propositions which may be evaluated as true or false.
\end{note}

\subsection{Culmination of reasoning}

\begin{note}[Reasoning proposition]
  \begin{restatable}[Claiming support is the result of reasoning]{assumption}{assuCSRR}\label{assu:CS-culmination-of-R}
    Claimed support some proposition \(\phi\) having value \(v\) is the culmination of some instance of reasoning.
  \end{restatable}
\end{note}

\begin{note}
  Key thing here is that result of premises and steps of reasoning.

  The role of~\autoref{assu:CS-culmination-of-R} is primarily to ensure that claiming support always guarantee the existence of premises and steps.
  With the exception of some broad constraints to be outlined in further assumptions below, we (will) have little to say about the specifics of what the reasoning consists of.
\end{note}

\begin{note}[Quick examples]
  \begin{itemize}
  \item \(S\) testified that \(p\), so \(p\) is true.
  \item \(p\) would satisfy every member of the group, so \(p\) ought to be the case.
  \item The song is produced by \(S\), so it is desirable that I listen to it.
  \item The device reads \(p\) and is reliable, so \emph{not}-\(p\) is improbable.
  \end{itemize}
\end{note}

\begin{note}
  Instances of reasoning may culminate in other ways, so we are only interested in a specific type of reasoning.
\end{note}

\begin{note}[Claiming support]
  Expand on this below.
  Briefly mention that this falls short of \emph{establishing} that \(\phi\) has value \(v\).
  \emph{Claimed} support means there's always some possible defeater.
\end{note}

\begin{note}[Understanding `having value \(v\)']
  In a deductive case, if the premises are true, then the conclusion is true.
  Means-end reasoning for desire.
  The value is important.
  If it is true that it past 6pm, then it is true the shop is closed.
  Provides value of shop being closed.

  However, if agent desires that it is past 6pm, then it doesn't follow that the agent desires that the shop is closed.
  Question an agent as to why they think their desires conform to truth --- is-ought problem.

  Means-end reasoning.
  It is true that there is cheese at the centre of the maze.
  And, it is desirable that I obtain the cheese at the centre of the maze.
  Further, it is true that I may only obtain the cheese at the centre of the maze by solving the maze.
  Therefore, it is desirable that I solve the maze.
\end{note}

\subsection{Summary}
\label{sec:summary}

\begin{note}
  Assumptions \ref{assu:CSVP} and \ref{assu:CS-culmination-of-R} are designed to be general and straightforward.
\end{note}

\begin{note}
  However, combined they contain a subtlety that will become important.
  The subtlety is this:
  From \autoref{assu:CSVP} claimed support is for the value of a proposition, and from \autoref{assu:CS-culmination-of-R}, claimed support is the result of some reasoning.
  In turn, if an agent has information that some proposition \(\phi\) having value \(v\) requires that some other proposition \(\psi\) has value \(v'\), then it is a consequence of the agent's claimed support for \(\phi\) having value \(v\) that \(\psi\) has value \(v'\).
  Yet, as claimed support is the result of some reasoning, it does not immediately follow that the agent has claimed support for \(\psi\) having value \(v'\).
  For, it need not be the case that the agent performs any reasoning about \(\psi\) having value \(v'\).
\end{note}

\begin{note}
  Further, not only is no claimed support, but it is not clear that reasoning for \(\phi\) having value \(v\) applies to \(\psi\) having value \(v'\).
  Claimed support is not simply a propositional attitude, it is tied to reasoning, and it is not immediate that reasoning about a proposition having some value extends to any (held) consequence of that proposition.

  This point will be implicitly expanded on through the following section, and explicitly noted in \autoref{sec:no-closure}.
\end{note}

\subsection{Limitations on claiming support}
\label{sec:claimed-support-1}

\subsubsection{Presence}
\label{sec:presence}

\begin{note}
  \color{red}
  Start with a definition of the terms.
\end{note}

\begin{note}
  \begin{restatable}[\mistaken{-} and \misled{}]{definition}{defMoM}\label{def:MoM}
    Some instance of reasoning that culminates with \(\phi\) having value \(v\).
    \begin{itemize}
      \item The reasoning is \emph{\mistaken{}} by involving appeal to some proposition \(\psi\) having value \(v'\) which does not have value \(v'\).
    \item The reasoning is \emph{\misled{}} if \(\phi\) does not have value \(v\).
    \end{itemize}
    \vspace{-\baselineskip}
  \end{restatable}
\end{note}

\begin{note}
  \begin{restatable}[\twodefeaters{-} --- \twodefeaters{}]{idea}{ideaDefeatersForCS}\label{idea:defs-for-CS}
    \mistaken{-} and \misled{} are possible defeaters for claimed support in the sense that:

    Were the agent to consider the claimed support was \mom{}, then the agent would retract claimed support is problematic.\nolinebreak
    \footnote{
      Common to distinguish between countervailing and undercutting defeaters.
      Countervailing, support for some other proposition.
      Undercutting, force of claimed support is mitigated.

      Both misled and mistaken are instances of being undercut.
      For, if mistaken or misled then the support is no good.

      Neither are clearly cases of countervailing.
      For, no relation to other support is required.
      However, presence of countervailing implies misled.
      Countervailing does not imply mistaken, though may in some cases demonstrate so.
    }
    And, always some way.
  \end{restatable}
\end{note}

\begin{note}
    Intuition here is that of \mistaken{} is instance of undercutting and \misled{} is a consequence of rebutting, mainly.

  This is mismatched.
  Rebutting implies \misled{}.
  Yet, may be undercutting which does not reduce to \mistaken{}.

  Purpose is that both are straightforward problems relate to basic assumption about claiming support.
  Relation to broader characterisation is present, but have nothing to say about these kinds of defeaters beyond relation to specific instances.
\end{note}

\phantlabel{first-mention-undercutting-defeater} % first mention of undercutting defeaters
\begin{note}[Quick intuition]
  The basic idea behind \autoref{idea:defs-for-CS} is that of \emph{undercuts} undercutting defeaters.

  \begin{quote}
    The second kind of defeater attacks the connection between P and Q rather than attacking Q directly.
    \dots

    This second kind of defeater is, roughly speaking, a reason for thinking that, under these circumstances, knowing-that-P is not a good way to find out whether Q.
    \dots

    A type II defeater is any reason for believing that \({\sim}(P => Q)\) which is not also a reason for believing that \({\sim}Q\).\nolinebreak
    \mbox{}\hfill\mbox{(\cite[43]{Pollock:1974uk})}
  \end{quote}
  Where `\(=>\)' is the subjunctive conditional. (\Citeyear[42]{Pollock:1974uk})

  With \citeauthor{Pollock:1974uk}'s original formulation, the defeater attacks the link.\nolinebreak
  \footnote{
    See \textcite[196,fn.166]{Pollock:1999tm} for a brief note of the history of undercutting defeaters.
    \textcite{Pollock:1974uk} is more-or-less a direct expansion of discussion in~\textcite{Pollock:1970un}.
  }

  However, generalise.
  That there's a relation of claimed support, evidence, or what have you between \(P\) and \(Q\).

  Generalisation by \citeauthor{Bergmann:2005ws}\nolinebreak
  \footnote{
    I am also inclined to \citeauthor{Worsnip:2018aa}'s sketch of undercutting defeaters, which references~\citeauthor{Bergmann:2005ws}.
    \begin{quote}
      Undercutting defeaters, which are easiest to think of in the context of the attitude of belief, are supposed to be considerations that undermine the justification of a belief in a proposition p not necessarily by providing (sufficient) positive evidence to think that p is false, but rather merely by suggesting (perhaps misleadingly) that one’s reasons for believing p are no good, in a way that neutralizes or mitigates their justificatory or evidential force.\linebreak
      \mbox{}\hfill\mbox{(\Citeyear[29]{Worsnip:2018aa})}
    \end{quote}
  }
  \begin{quote}
    \emph{d} is an \emph{undercutting defeater} for \emph{b} iff \emph{d} is a defeater for \emph{b} which is (or is an epistemically appropriate basis for) the belief that one's actual ground or reason for \emph{b} is not indicative of \emph{b}'s truth.\linebreak
    \mbox{}\hfill\mbox{(\cite[424]{Bergmann:2005ws})}
  \end{quote}

  Similar generalisation, but similar to \citeauthor{Pollock:1974uk} the relation is useful.
  So, generalise the subjunctive conditional.

  {\color{green}
  If~\nIBackground{} hold, then the way of claiming support captured by~\ref{nI:going-by-value} is undercut.

    Specifically, undercutting arises from \ref{nI:inclusion} establishing a form of interdependence between claimed support for \(\phi\) and claiming support for \(\phi\).

  Such that in order for agent to appeal to \(\phi\) having value \(v\) the agent must assume that \(\psi\) has value \(v'\).
  Hence, can't go to \(\phi\) without \(\psi\), which means that can't use \(\phi\) to claim support for \(\psi\).
  }
\end{note}

\begin{note}
  \autoref{def:MoM} is `only' a definition.
  Still, suggestive terminology.
  Hence, \twodefeaters{} makes explicit the idea behind the suggestive terminology.

  However, definition and \emph{idea} rather than assumption
  We do not assume this idea.
  In part, interested in the possibility of such `defeaters' and we have no interest in situations where an agent discovers that their claimed support is \mom{}.
  In larger part, unclear what the idea amounts to in practice.
  So, instead, provide {\color{red} assumptions} below that clarifies how the possibility of being \mom{} matters.

  Before these {\color{red} assumptions}, {\color{red} some more basic assumptions} regarding how being \mom{} relates to claimed support.

  Still, \twodefeaters{} is sufficiently intuitive that we will refer to \mom{} as `(two) types of defeaters'.
\end{note}

\begin{note}[Agents are fallible]
  The following assumption states how these two types of defeaters provide a limitation on claiming support.

  \begin{restatable}[\nfcs{--} -- \nfcs{}]{assumption}{assuNFCS}\label{assu:supp:nfactive}
    When claiming support for a proposition and agent is not in a position to rule out the (epistemic) possibility that the claimed support is \nmom{}.
  \end{restatable}

  In other words, claimed support is not `factive': Claimed support for \(\phi\) having value \(v\) does not entail \(\phi\) has value \(v\).
\end{note}

\begin{note}
  I take it to be intuitive that agents are fallible in many cases of claiming support.
  It is not too difficult to think of ways in which claimed support may be misleading or mistaken.
  As noted above, claiming support for what the time is from glancing at a clock seems sufficient, but clocks may be incorrectly set (misled) or broken (mistake).
  Similarly, a sample of \(1,000\) rolls may mislead me into thinking that a die is unbiased, or an overloaded operator may lead to a mistake in claiming support for the proposition that \(x = 4\) is an expression of equality rather than variable assignment.
\end{note}

\begin{note}
  \begin{itemize}
  \item Illustrations
  \item What `possibility' amounts to
  \item Relation to `genuine' defeaters
  \item Scope of possibility.
  \end{itemize}
\end{note}

\begin{note}[M\&M Illustration]
  To illustrate:

  Suppose I glance at the clock on the wall.
  The clock reads 11:45a, so I claim support that it is 11:45a.
  However, it may be the case that the clock is incorrectly set, and the time is 11:15a, or 12:15p, etc.\
  By claiming support from the time expressed by the clock, I would have been \emph{misled} about what the time actually is.
  For, it is not true that the time is 11:45a.
  Though, in all other respects, there may be no fault with claiming that the time is as expressed by the clock and so the claim to support is not mistaken.

  By contrast, suppose I glace at the clock on the wall.
  The clock reads 11:45a, so I claim support that it is 11:45a.
  By claiming support from the time expressed by the clock, I would have been \emph{mistaken} about what the time actually is.
  For, claimed support by appeal to a functioning clock.
  Though, despite the clock being broken, it is 11:45a and so the claim to support is not misleading.

  Combining, claimed support for the time from a broken clock expressing the wrong time would be both \misled{} \emph{and} \mistaken{}.\nolinebreak
  \footnote{
    A second illustration:
    Consider a smoke detector, designed to sound an alarm if and only if sufficient levels of smoke are detected.
    Hence, if the alarm sounds, one may claim support there being smoke in the room where the alarm is installed.
    One may be misled; the alarm may have malfunctioned, so no fire.
    Or, one may be mistaken; the same type of alarm may be installed in a different room, wouldn't be a useful indicator.
  }

  {
    \color{red}
    Of course, clocks are typically glanced at, and a glance at a clock is often insufficient to determine whether the clock is incorrectly set or broken.
    Hence, the \emph{possibility} that a clock is incorrectly set or broken --- or more broadly the possibility that claimed support is misleading or mistaken --- does not prevent an agent from claiming support.
    So, ensuring that to-be-claimed support would be \mom{} is not a necessary condition for claiming support.
  }
\end{note}


\begin{note}[`Possibility' modal]
  Rather than fix a specific account of the `possibility' modal used, here are a handful compatible interpretations:

  \begin{enumerate}[label=\Alph*., ref=(\Alph*)]
  \item\label{CS:I:Never} \emph{In principle} it is not possible for any agent to rule out the possibility that claimed support is not misleading or mistaken.
  \item\label{CS:I:Resources} It is not possible for a \emph{resource bound agent} to rule out the possibility that claimed support is not misleading or mistaken.
  \item\label{CS:I:Class} There is a restricted class of propositions for which and agent is required to claim support, and it is not possible for any agent to rule out the possibility that claimed support for a proposition belonging the class is \nmom{}.
  \end{enumerate}

  To illustrate, consider the proposition that there is an external world:
  \ref{CS:I:Never} denies that there could be, e.g.\ proof of an external world.
  \ref{CS:I:Resources} denies that agents of interest could not demonstrate such a proof even if it were to exist.
  \ref{CS:I:Class} allows an agent may not be required to claim support for the existence of an external world.

  I expect the intended application of claimed support will be compatible with each interpretation, and specifically with respect to \ref{CS:I:Class}, that the propositions will belong to the highlighted class.\nolinebreak
  \footnote{
    I favour the combination of \ref{CS:I:Resources} and \ref{CS:I:Class}, and to leave open whether an idealised agent may rule out the possibility of being misled or mistaken with respect to some propositions when claiming support.
  }\(^{,}\)\nolinebreak
  \footnote{
    In particular, true of ability.
  }
\end{note}

\begin{note}
  \nfcs{} states that an agent is never in a position to rule out the possibility that the claimed support is not misled or mistaken.
  This does not entail that there are defeaters, nor that there are any possible defeaters --- only that defeaters are an epistemic possibility.

  Still, in some cases, this may seem absurd.
  Suppose in front of me are two apples and two pears.
  So, four pieces of fruit.

  There are two responses here.

  First, not an instance of claiming support.

  Second, outline various way in which the claimed support may be \mom{}.
  Those apples and pears may not be real pieces of fruit, they may be replicas.\nolinebreak
  \footnote{
    So beings the process of attempting to quarantine fallibility from infallibility.

    There are two pairs of objects in front of me, and the objects appear to be fruit.
    So, there are four objects in front of me, which appear to be fruit.

    There are two pairs of objects hence there are four objects.

    But I may be hallucinating.

    There appear to be two pairs of objects in front of me, which appear to be fruit.
    So, there appear to be four objects in front of me, which appear to be fruit.

    Whenever there are two pairs of objects, there are four objects.

    Perhaps it is impossible to be \mom{}, but then stronger than claiming support.

    Likewise:

    Any object is identical with itself --- but I doubt one needs to claim support for reflexivity of equality.

    Similarly, it doesn't seem to be the case that I need to claim support for the proposition that my name is Humpty --- no matter what my birth certificate says, I get to decide what my name is.
  }
\end{note}

\begin{note}[Summarising \nfcs{}]
  Limitation on strength of claimed support.

  Plausibly holds with respect to various other attitudes.
  Belief because can't rule out.

  Issue here is that compatible with claimed support being a very weak attitude.
  For example, agent is allowed to assume some arbitrary proposition \(\psi\) has value \(v'\) when claiming support for \(\phi\) having value \(v\).
  Possibility of being \mistaken{} with respect to \(\psi\) has value \(v'\), and hence possibility of being \misled{} with respect to \(\phi\) having value \(v\).

  Still, interest is in something stronger than this.

  The (epistemic) possibility of being \mom{} is as a worry.
  Goal, roughly, is to establish good ground for appealing to \(\phi\) having value \(v\) in further reasoning.
  So, \nfcs{} limits the strength of claimed support in reasoning, but also need additional assumptions to ensure sufficiently strong.
\end{note}

\begin{note}[\eiS{}]
  Definition of \mom{}, and from \nfcs{}, (epistemic) possibility that claimed support is \mom{}.
  To ensure sufficiently strong, idea is that agent never requires that claimed support is not \mom{}.

  So, because epistemic possibility, should work out however things turn out.

  % \begin{definition}[Dependence and independence]
  %   An agent's claimed support for \(\phi\) \emph{depends} on some value of \(\psi\) just in case the agent would not claim support for \(\phi\) given any other value of \(\psi\).

  %   An agent's claimed support for \(\phi\) is \emph{independent} of \(\psi\) just in case the agent's claimed support does not depend on the value of \(\psi\).
  % \end{definition}

  \begin{restatable}[\eiS{-}]{idea}{ideaEIS}\label{idea:eiS}
    Claimed support for \(\phi\) having value \(v\) indicates that \(\phi\) has value \(v\) regardless (from the perspective of the agent) of whether the claimed support for \(\phi\) having value \(v\) is \mom{}.

    Entertain possibility that \(\phi\) does not have value \(v\) while maintaining claimed support.
  \end{restatable}
  {
    \color{red}
    Old footnotes\nolinebreak
    \footnote{
      Possibly goes against externalism, but I don't think this is right.
      External circumstances may impact the support the agent has.
      However, as these are external, it seems this condition plausibly holds for \emph{claiming} support.
      This is how you get puzzles for externalism.
      In both cases, it's fine for the agent to claim support, but the external circumstances impact whether the agent \emph{has} support.
      The internalist/externalist divide would seem to affect the conditions on claiming.

      Way to expand on this is reconstructing bootstrapping examples with and without \eiS{}.
      If the agent would only get basic support if reliable, then it's not clear that bootstrapping is a problem.
    }
  }
\end{note}

\begin{note}
  \autoref{idea:defs-for-CS} places limitation on claimed support.

  \autoref{idea:eiS} goes in the opposite direction.
  Claimed support to be a relatively strong attitude.
\end{note}

\begin{note}
  Internalist flavour, specifically `mentalism' in the sense of \citeauthor{Feldman:2001uy} where:
  \begin{quote}
    \dots a person's beliefs are justified only by things that are \dots internal to the person's mental life.\nolinebreak
    \mbox{}\hfill\mbox{(\citeyear[233]{Feldman:2001uy})}\nolinebreak
      \footnote{
        See also~\textcite[\SS4,9]{Pappas:2017vi}.
      }
    \end{quote}
    Result of reasoning, and indication is part of that reasoning.
    Reasoning is internal to an individual's mental life, and hence internalist in this sense.

    Indication of \autoref{idea:eiS} need not amount to justification --- it may, but this will not be important.
    This is also not to require sufficient.
    These two ideas may be developed in various ways.
    Not only indicates from perspective of the agent, but is neither \mom{} and ways in which \mom{} are counterfactually distant.
    However, ideas for the moment.
    Assumptions are what will matter.
\end{note}

\subsubsection{\requ{3} and Independence}
\label{sec:two-assumpt-regard}
\label{sec:claim-supp-requ}

\begin{note}
  Two key ideas: \ideaCSA{} and \ideaCSB{}.
  These provide some characterisation of claimed support.

  We revised \autoref{idea:defs-for-CS} to an explicit assumption: \autoref{assu:supp:nfactive}.
  From \autoref{assu:supp:nfactive} claimed support is such that possibility of being \mom{}.
  This limits the strength.

  In turn, \autoref{idea:eiS} pushes for claimed support to be sufficiently strong that it compensates for such possibilities.

  For this paper, significant interest with \autoref{idea:eiS}.
  Different ways of reasoning, and how these deal with the concern about possibility of being \mom{}.

  In particular, how defeaters work.

  Introduce an assumption in relation to \autoref{idea:eiS} that mirrors \autoref{assu:supp:nfactive} from \autoref{idea:defs-for-CS}.
\end{note}

\begin{note}
  However, \autoref{idea:eiS} are delicate.
  Following \autoref{idea:defs-for-CS} defeaters, but the idea leans on the agent `discovering' (in some non-factive sense) that the claimed support is \mom{}.
  So, would defeat, but this does not necessarily indicate a problem with claimed support.

  Now, claimed support is necessarily weak enough to leave open the possibility of being \mom{}.
  However, discovering after claiming support does not have any immediate consequences for how to deal with the possibilities when claiming support.

  Indeed, it may not simply be the case that entertaining the possibility that one \emph{is} \mom{} is not enough to highlight a problem with claimed support.
  For, claiming support is compatible with such possibility, in general (though perhaps not for certain instances).

  To illustrate:

  \begin{illustration}[Textbook answers]
    \label{illu:textbook-answers}
    Answer in textbook.
    Reasoning is only good if I get to that answer.
    So, if not answer, then reasoning is bad.
  \end{illustration}

  Recognise the possibility, but still succeed in claiming support.
  Indeed, possibility that the author was \mom{} in claimed support for exercise.
  Even so, if the author claimed support then there's no clear problem.

  Of course, this is not to say that there may be other constraints.
  If answer is not the case, then it may be that the agent's information state falls short of some ideal (i.e.\ only truths).
  Still, no problem from the perspective of claiming support alone.

  Might need additional information to recognise the problem.

  Again, indicating doesn't reduce to considerations against ways in which the agent may be \mom{}.

  And, while it may be the case that `discovering' being \mom{} amounts to a defeater for claimed support, there is no immediate link between this and what happens when claiming support.
\end{note}

\begin{note}
  Still, there are cases where the possibility of being \mom{} does present a problem.

  \begin{illustration}
    Suppose some conditional that admits of:
    \(P \rightarrow (Q \rightarrow R)\)

    To

    \((P \land Q) \rightarrow R\)

    Then, appeal to \(P \rightarrow (Q \rightarrow R)\).
    Question for whether the instance also holds with respect to \((P \land Q) \rightarrow R\).
  \end{illustration}

  Intuitively, possibility of being \mistaken{}.
  And, some consideration regarding whether \(P \rightarrow (Q \rightarrow R)\) holds for instances.
\end{note}

\begin{note}
  These illustrations are fairly distinct.
  Hence, various different dimensions to account for intuition diverging between the two illustrations.

  In particular, \mistaken{} versus \misled{}.
  This is difficult, though, as it may be the case that relevant thing applies to both.

  Being \mom{} is a way in which claimed support may be defeated, and these place indirect limits on reasoning itself.
  So, distinguish between the result of reasoning and reasoning (i.e.\ between claimed support and claiming support).
  Connect the possibility of being \mom{} to some part of reasoning.
  One such limitation.
\end{note}

\begin{note}
  First, the idea of a \requ{} of claimed support.

  \begin{restatable}[\requ{3} of claimed support]{definition}{defRequisite}\label{def:requisite}
    \(\psi\) has value \(v'\) is \requ{} of claimed support that \(\phi\) has value \(v\) if, from agent's perspective:
    \begin{itemize}
    \item \(\psi\) not having value \(v'\) is possible.
    \item If \(\psi\) were not to have value \(v'\) then some step of reasoning would involve being \mom{} --- no making that step of reasoning.
    \item That step has consequence of \(\psi\) having value \(v'\) which persists through to conclusion of reasoning.
    \end{itemize}
  \end{restatable}
  Intuitively, non-temporary consequence of some step of reasoning in isolation.

  So, problem is clear.
  If \requ{} fails, then reasoning fails as a whole.
\end{note}

\begin{note}
  \begin{restatable}[\eiS{-} --- \eiS{}]{assumption}{assuEIS}\label{assu:supp:independence}
    Claimed support for \(\phi\) having value \(v\) involves:
    \begin{itemize}
    \item Reasoning about \requ{1}.
    \end{itemize}
    \vspace{-\baselineskip}
  \end{restatable}

  From~\autoref{assu:CS-culmination-of-R}, claimed support is the result of reasoning.
  \autoref{assu:supp:independence} provides addition constraints on what such reasoning involves.
  This is quite weak and rather uninformative.
  Still, primary purpose of \eiS{} is that we may expand on what the relevant reasoning involves.

  {\color{blue}
    In other words, claimed support covers all ways in which something follows from claiming support and is such that reasoning would not work without it.
    Not possible to claim support against a restricted selection of possibilities.
  }

  {\color{red}
    With \misled{}, claiming support does not come for free.
    With \mistaken{}, do not get to take on board premises and steps for free.
    (So to speak.)
  }

  More details on \autoref{assu:supp:independence}, after talking through \autoref{def:requisite}.
  However, important to note that \autoref{assu:supp:independence} does not give an account of what the reasoning is.

  Plausible there are constraints on the reasoning.
  \emph{However}, for us, it is either the lack of reasoning, or that it is not possible to do the required reasoning --- though not necessarily not possible in general (\nI{} for details).
\end{note}

\begin{note}[`Unrecognised' \requ{1}]
  May add:
  \begin{itemize}
  \item Reasoning about `unrecognised' \requ{1}.
  \end{itemize}
  Difficulty is what this amounts to.

  An alternative way to strengthen is to have some requirement concerning a search for \requ{1}.
\end{note}

\begin{note}
  Well, this definition and assumption may seem arbitrary.
  That much I concede.
  The key point is that proposition having value is an unavoidable consequence of some step of reasoning.

  With respect to \autoref{illu:textbook-answers}, that the reasoning is good is not a consequence of a step of reasoning.
  Of course, the conclusion of reasoning will be a \requ{}.
  In this case, \autoref{assu:supp:independence} is trivial.

  However, we are not searching for a complete account of possible defeaters of interest.
  Rather, a clear account of something sufficiently troublesome.

  However complex things get when additional aspects of reasoning are taken into account, what this definition captures seems clear enough, even if part of that clarity is obtained by arbitrary simplification.
\end{note}

\begin{note}
  Regular conditional and subjunctive.
\end{note}

\begin{note}
  The subjunctive here does, generally speaking, highlight a possible defeater for reasoning.

  Subjunctive because what follows from entertaining possibility.
  This then allows `discovering'.

  Of course, given that subjunctive, questions about how to evaluate.
  However, not counterfactual.
  So, given that possibility, do not need to consider cases where entertaining possibility requires some revision to the reasoning that the agent has performed.

  The final clause seems to explicit a plausible implicit consequence of the first clause.

  For \(\lnot p \leadsto \lnot q\), \(\lnot p \rightarrow \lnot q\), so \(q \rightarrow p\).

  Plausible, but this is not quite right.
  Consider reasoning by cases.
  If not that case, then a problem, but does not have the consequence that the case holds.
  Hence, persists.\nolinebreak
  \footnote{
    This is not to deny a slightly more complex clause.
    Rather, easier this way.
  }
\end{note}

\begin{note}
  Really important here is that being a \requ{} does not imply that the reasoning involves direct appeal.

  {
    \color{red}
    Indeed, this distinguishes from \citeauthor{Sgaravatti:2013wu}, which requires belief in each of premises.
  }
\end{note}

\begin{note}
  Not the case that any thing that is appealed to is a \requ{}.
  For, may add in certain additional things.
  In other words, careful to make sure that the subjunctive part is there.
\end{note}

\begin{note}[\autoref{assu:supp:independence} and reasoning]
  However, important point is that claimed support is tied to instance of reasoning.

  Given \nfcs{} it need not be the case that an agent completely rules out possibility of defeater obtaining --- i.e.\ of being \mom{}.
  Appeal to claim support in reasoning that defeaters do not obtain.
  Hence, reasoning does not require ruling out defeaters.
  So, \autoref{assu:supp:independence} does not require claimed support against defeaters.

  Still worry that reasoning is too weak.
  Additional requirements are compatible with \autoref{assu:supp:independence}, such as investigating unrecognised defeaters.
  However, do not \emph{assume} this.

  Again, primary purpose of \eiS{} is that we may expand on what the relevant reasoning involves.

  No clear assumptions regarding this in any detail.
  Part of the interest is in what is required of such reasoning.

  Appeal in argument is with respect to absence of reasoning regarding defeater.
  In turn, argument for what the reasoning involved amounts to.
  And, argument against a (plausible) necessary condition.
\end{note}


\begin{note}
  \begin{restatable}[]{proposition}{propReasondefRequisite}\label{prop:reason-requ}
    {
      \color{red} This should be an argument for why a \requ{} amounts to something that needs consideration.
    }
    If \requ{} then possible defeater.
  \end{restatable}

  So, suppose some \(\psi\) is \requ{} of some reasoning.
  Then, not only would reasoning would be \mom{} if \(\psi\) is not the case.
  But, in addition, some structural component would be bad.
  That is, connect the possibility of being \mom{} to some part of reasoning.
\end{note}

\begin{note}
  \eiS{} does not deny that things may need to be a certain way for an agent to claim, or to be in a position to, claim support.
  It may be the case that no agent would be in a position to claim support that the speed of light is constant if the speed of light were not constant.
  Still, in claiming support an agent must expect that possible defeaters do not obtain, e.g.\ that the laws of nature are constant, and that no mistakes have been made when observing relevant phenomena.
\end{note}

{
  \color{blue}
  What I really need for \nI{} is that there's no way to go without expectation.
  Right, okay, and the point is not that really that there's nothing else for the agent to do.
  There might be, but the point of the ability case is that it doesn't seem there needs to be.
  \emph{In addition}, it also seem implausible that there could be, but the argument doesn't depend on ruling out this possibility.
}

\begin{note}
  Following is an immediate consequence of \autoref{assu:supp:independence}:

  \begin{restatable}[Reason about recognised \requ{1}]{proposition}{propRecogniseDefeaters}\label{prop:CS-only-if-reason-recognised-defeaters}
    If recognised \requ{} at time of reasoning and does not reason, not an instance of claiming support.
  \end{restatable}
\end{note}

  \begin{note}
  \begin{itemize}
  \item If result of reasoning to \(\phi\) having value \(v\) is such that agent considers that reasoning fails if \(\phi\) does not have value \(v\), then reasoning is not an instance of claiming support.
  \item Not possible that instance of reasoning to \(\phi\) having value \(v\) is claimed support only if \(\phi\) has value \(v\).
  \item Claimed support for \(\phi\) having value \(v\) never requires that \(\phi\) has value \(v\).
  \end{itemize}
\end{note}

\begin{note}
  \color{red}
    \begin{itemize}
    \item Always possibility of \mom{} from \nfcs{}.
    \item This means that the agent has no guarantee that \(\phi\) has value \(v\) --- or better put the agent considers it to be an (epistemic) possibility that their claimed support is \mom{}.
    \item However, if the agent requires that \(\phi\) is the case, then there is no possibility of the claimed support being \emph{mistaken}.
    \item Well, no reasoning against being mistaken with respect to claimed support for this expectation.
  \end{itemize}

  Again, it does not seem impossible for an agent to adopt an attitude that recognises the possibility but assumes regardless.
\end{note}

\subsection{No closure}
\label{sec:no-closure}

\begin{note}
  \begin{restatable}[No `closure' of reasoning]{assumption}{assuRClosure}\label{assu:R-closure}
    Reasoning about some proposition \(\phi\) having value \(v\) does not necessarily apply to any held consequence of \(\phi\) having value \(v\).
  \end{restatable}
  Held, to avoid requirement of consequence.

  Ambiguous.
  Past reasoning present consequence.
  Present reasoning, present consequence.

  Only the former.
\end{note}

\begin{note}
    `A proposition is some information.'
\end{note}

\begin{note}
  Applied to claiming support.
  It is not necessarily the case that claimed support for \(\phi\) having value \(v\) is claimed support for \(\psi\) having value \(v'\).
\end{note}

\begin{note}
  However, \autoref{assu:R-closure} is more general.

  Reasoning that proceeds from entertaining the possibility that \(\phi\) has value \(v\) does not necessarily apply to \(\psi\) having value \(v'\).
\end{note}

\begin{note}
  Examples:
  \begin{itemize}
  \item Possibility that the trains are on strike.
  \item No Indication of strike, so do not consider live possibility.
  \item Read newspaper.
  \item Newspaper reported strike.
  \item Consequence of possibility is that the newspaper misreported.
  \item Reasoning does not extend to newspaper.
  \end{itemize}

  \begin{itemize}
  \item Out of milk.
  \item Then come to hold that there is milk in the fridge.
  \item Hallucinating.
  \item Does not extend.
  \end{itemize}

  \begin{itemize}
  \item Turing machine reduction.
  \item If possible then also possible.
  \item So, give up.
  \end{itemize}
\end{note}

\begin{note}
  Important consequence, block:

  S did not reason about possibility that Q is false.
  If Q is false, then P must also be false.
  Hence, P may be false.
  S did not reason about the possibility that P is false.
\end{note}

\subsubsection{Expectation}
\label{sec:claim-supp-expect}
\label{sec:claim-supp-nai}

\begin{note}
  In the previous section introduced the notion of a \requ{} and used this notion to refine \ideaCSB{} into an assumption.
  Assumption is relatively weak.
  Some reasoning about recognised \requ{1}.
  However, no commitments to what such reasoning must amount to.

  In the present section we introduce a second definition and proposition.
\end{note}

\begin{note}
  Our primary interest is with the introduction of \requ{1}.
  For, this is a case where the agent will not necessarily have performed any reasoning about the \requ{}.

  Introduce definition, first instance of failure to claim support.

  Build on below after examples.
  In particular, some corollaries.
\end{note}

\begin{note}[Expectation, assumption]
  Start with the definition:

  \begin{restatable}[Expectations with respect to claimed support]{definition}{defExpectation}\label{def:expectation}
    If claimed support for \(\phi\) having value \(v\) and then come to consider \(\psi\) has value \(v'\) as \requ{} of claimed support that \(\phi\) has value \(v\), then \(\psi\) has value \(v'\) is an \emph{expectation} of the claimed support for \(\phi\) having value \(v\).
  \end{restatable}

  So, an expectation is a specific instance of a \requ{}.
  In claiming support for \(\phi\) having value \(v\), the reasoning about \(\psi\) having value \(v'\) indirectly as an unrecognised \requ{}.
  However, after claiming support.

  Claiming support is an instance of reasoning.
  From \eiS{}, enough to go even if \mom{}.
  At time of claiming support, agent may not recognise certain possibilities.
  This doesn't prevent the agent from claiming support.

  Claimed support, so reasoning.
  Part of that reasoning, unrecognised defeaters do not obtain.
  Once the agent recognises, this is now expected.

  Expectation that \(\phi\) relative to claimed support for \(\psi\) is, roughly, that \(\phi\) is a recognised possible part of previously unrecognised\dots
\end{note}

\begin{note}[What expectation amounts to]
  \(\phi\) has value \(v\).
  Expects \(\psi\) has value \(v'\) relative to further information about \(\phi\) and \(\psi\).

  Extension of \autoref{assu:supp:independence} from \requ{1} to \expec{1}.
\end{note}

\begin{note}
  \begin{restatable}[Reasoning about expectations]{assumption}{assuCSNoExp}\label{assu:independence-expec}
    Reasoning.
    Appeal to claimed support for \(\psi\) having value \(v'\) such that \(\xi\) having value \(v''\) is an \expec{} of that claimed support for \(\psi\) having value \(v'\).
    Then:

    Claimed support for \(\phi\) having value \(v\) involves:
    \begin{itemize}
    \item Reasoning about \(\xi\) having value \(v''\) as \expec{} to \(\psi\) having value \(v'\).
    \end{itemize}
  \end{restatable}
  So, reclaiming support for \(\psi\) having value \(v'\).
  Hence, reasoning about \(\xi\) having value \(v''\) without appealing to claimed support for \(\psi\) having value \(v'\).
\end{note}

\begin{note}
  Intuition here is that \expec{}, no direct consideration.
  So, reasoning to \(\phi\) having value \(v\), sure.
  However, reasoning does not address the expectation.
\end{note}

\begin{note}
  For example, there appears to be a bowl of fruit in the centre of the table, and so I claim support by visual inspection that there is a bowl of fruit in the centre of the table.

  Well, things are as they appear.

  Oh, plastic fruit in particular.
\end{note}

\begin{note}
  \begin{restatable}[Claiming support Nai]{proposition}{propCSNai}\label{prop:CS-nai}
    Reasoning that \(\phi\) has value \(v\) is not an instance of claiming support for \(\phi\) having value \(v\) if \expec{} with no reasoning about it as \expec{}.
  \end{restatable}

  For, expectation leads to a (recognised) \requ{} for current instance of claiming support.
  As, if \expec{} does not hold, then problem with appeal to that claimed support.
  Of course, the limited upshot given assumptions made is that the agent needs to do some reasoning.

  {
    \color{red}
    This is not a consequence of assumption \autoref{assu:supp:independence} alone.
    For, \autoref{assu:supp:independence} only talks about reasoning.
    And, in this case there is some reasoning.
  }
\end{note}

\begin{note}
  Expectation alone doesn't seem so bad.
  Haven't shown that \(\phi\) is really important to the reasoning.
  However, add in requisite and things are bad.
  Requisite alone also isn't enough, as may be able to deal with this.
  Possible defeaters.
  So, it's that \(\phi\) is involved and the agent hasn't dealt with the possible defeater that leads to \(\phi\) being involved.

  \begin{proof}
    This assumption expands on \eiS{}.
    Basically, if require \(\phi\) has value \(v\) then no account of why \(\phi\) given possibility that \(\phi\) does not have value \(v\).

    Point is that claiming support, so result of compatible with possibility that \(\phi\) is not the case.
    However, also such that still indicates \(\phi\).
    But, no account if this possibility obtains.
  \end{proof}

  This is the strongest assumption.
  Intuitively, independent grounds for dismissing defeater.

  Quite delicate.
  Good reasoning against unrecognised defeaters.
  However, this does not extend if one (or more) of those defeaters is recognised.

  The point is not that the defeater is unlikely.
  Rather, claimed support is such that it involves considerations against possible defeaters, and those considerations are such that they deal with the possible defeater.
\end{note}

\begin{note}
  Note, however:

  \begin{itemize}
  \item Haven't placed significant constraints on reasoning involved.
  \item That it is quite possible that the same reasoning for the unrecognised reapplies to the now recognised defeater.
  \end{itemize}
\end{note}

\begin{note}
  Consequence of this is that added reasoning about possible defeater is a `new' instance of claiming support.
\end{note}

\begin{note}
  Basic issue is that when claimed support for \(\phi\), did not consider \(\psi\).
  \autoref{prop:CS-nai} does not state that the agent is required to give up claimed support.
  Nor appealing to the claimed support to claim support for other propositions.
  Nor in other forms of reasoning.
  Denied is that result of reasoning is claiming support.
  In particular, because claimed support for \(\phi\) does not address possibility.
  So, not clear that indicates value.

  And,~\autoref{prop:CS-nai} adds in that reasoning against when unknown is not enough.
\end{note}

\subsubsection{Persistence}

\begin{note}
  Finally, two optional assumptions.
\end{note}

\begin{note}
  One assumption about reasoning.
  \begin{restatable}[May appeal to previously claimed support]{assumption}{assuCSbyPCS}\label{assu:appeal-to-previous-CS}
    May appeal to previously claimed support --- i.e.\ do not need to \emph{reclaim} support.
  \end{restatable}

  Note, this does not guarantee that appeal will be successful.
  Basically, excluding \expec{}, sufficient to reason that claimed support.
\end{note}

\begin{note}
  \begin{restatable}[Claiming support persists]{assumption}{assuCSPersists}\label{assu:CS-persists}
    {
      \color{red}
      I don't need this.
      It makes more sense to require reasoning about all \expec{} and then allow that something weaker is not an instance of claiming support.

      Alternative is to further restrict assumption.
      Nothing really hangs on this, so could do.
      Still, various way in which other assumptions could be weakened that would still allow for things to go through.
      Point is, this seems to conflict with motivating ideas.
    }
    Claimed support may persist given introduction of \expec{}.
  \end{restatable}

  This is a optional assumption.
  Alternative is that retract claimed support.
  Everything that follows is compatible with retraction.
  However, weaker.
  Hence, go with this assumption.
\end{note}

\begin{note}
  \autoref{assu:CS-persists} is important.
  Note the `may'.
  Motivation is that we get a \requ{}.
  However, there's no suggestion that the \requ{} is not the case.

  Claiming support is the result of an instance of reasoning.
  And, the instance was fine.
  So, would have been a \requ{} (and will be a \requ{} if reclaim).
  However, was not at the time.
  And, only a possibility.

  So, keep result, but relative to expectation.
  Now, you don't need to think that this continues to be claimed support.

  Pairs with \autoref{assu:appeal-to-previous-CS}.
  Intuitively, same expectations are inherited.
  However, the details do not matter to us.
  Again, what follows holds granting this assumption, rather than anything stronger.
\end{note}

\begin{note}
  Still, it may be helpful to state a general idea that will follow from argument.

  \begin{restatable}[]{thought}{thoughtDismissingDefeaters}\label{thought:dismissing-defeaters}
    If in position to deal with defeater, then need not rule out possibility to greatest extent possible.

    Reasoning against unconsidered defeaters, well, recognised defeaters were sufficient to consider, something like this.
  \end{restatable}

  Expand on this idea below with responses to the `even if' test.
  Emphasise is that the argument does not build in strong assumptions about what reasoning involves.
\end{note}

\subsubsection{Illustrations}

\begin{note}
  Collection of assumptions, expanding on two ideas, and propositions building on these.
  Work through how these assumptions combine in some cases.
  Also drawing out intuition.

  To start, familiar case, distinguish from circularity and related idea.
  Then, looking at \autoref{prop:CS-nai}.
  Pair of simple illustrations.
  Then, final illustration to highlight intuition and draw out some tension.
  After these, corollary of \autoref{prop:CS-nai} which summarises, some discussion, and an idea.
\end{note}

\paragraph{Illustrations with respect to ideas}

\begin{note}[Testimony 1]
  \begin{illustration}[Testimony 1]\label{illu:CS:test:basic}
    \mbox{}
    \begin{enumerate}[label=\arabic*., ref=(\arabic*)]
    \item\label{ex:eiS:tt:test} \nagent{11} testified that they are trustworthy when speaking on matters regarding their personal character.
    \item Any agent and proposition, agent testified that proposition is the case only if proposition is the case.
    \item \nagent{11} testified that p is the case only if p is the case.
    \item\label{ex:eiS:tt:ok} \nagent{11} is trustworthy when speaking on matters regarding their personal character.
    \end{enumerate}
  \end{illustration}

  I take the reasoning of~\autoref{illu:CS:test:basic} to be intuitively problematic.

  I am also somewhat confident that you will have seen some variant of the reasoning in relation to circularity before.

  The goal for the moment is to explain why the two ideas expressed in relation to claiming support (\nfcs{} and \eiS{}) highlight a way in which the reasoning is problematic.
  And, to distinguish the way in which the ideas highlight a problem from a pair of nearby considerations.
\end{note}

\begin{note}
  Observe, however, that the intuitive problem is not that the agent has any reason(ing) to think that \nagent{11} is \emph{not} trustworthy when speaking on matters regarding their personal character.

  Rather, the intuitive problem is that the agent does not have any reason(ing) to think that \nagent{11} \emph{is} trustworthy when speaking on matters regarding their personal character.

  In particular, that that \nagent{11} is not trustworthy when speaking on matters regarding their personal character is simply a possibility.
  It may be the case that \nagent{11} is trustworthy.\nolinebreak
  \footnote{
    \color{red}
    It's not like this suggests that they are not trustworthy.
    Asking for directions.
    These are fine, but addition is not.
  }
\end{note}

\begin{note}
  Following, let us consider why the reasoning of~\autoref{illu:CS:test:basic} is intuitively problematic from the perspective of claiming support.\nolinebreak
  \footnote{
    Of course, the reasoning of \autoref{illu:CS:test:basic} seems problematic without constraints on the purpose.
    However, our interest is with claiming support.
  }

  \ideaCSA{} and \ideaCSB{}.

  The role of \ideaCSA{} is simple: It is possible for the agent's reasoning to be \mom{}.
  For example.

  Possibility of \mistaken{}.
  Well, misheard, speaking sarcastically.
  Possibility of \misled{}.
  Possible that S is not trustworthy.

  It is the possibility of being \misled{} that is of interest.

  Here, \ideaCSB{}.
  In order for the reasoning to be an instance of claiming support, the reasoning should indicate that \nagent{11} is not trustworthy on matters regarding their personal character regardless of whether the reasoning that \nagent{11} is trustworthy on matters regarding their personal character is \mom{}.

  In particular, it should be possible for the agent to entertain the possibility that \nagent{11} is not trustworthy on matters regarding their personal character while maintaining that their reasoning indicates that \nagent{11} is trustworthy on matters regarding their personal character.
\end{note}

\begin{note}[The problem]
  The problem, then, is that it does not seem possible for the agent to entertain the possibility that \nagent{11} is not trustworthy when speaking on matters regarding their personal character while maintaining that their reasoning indicates that \nagent{11} is trustworthy when speaking on matters regarding their personal character.

  For, it may be the case that \nagent{11} is not trustworthy when speaking on matters regarding their personal character.
  And, \nagent{11}'s testimony involves speaking on a matter regarding their personal character.
  Hence, by entertaining the the possibility that \nagent{11} is not trustworthy on matters regarding their personal character, the agent is required to entertain the possibility that \nagent{11}'s statement did not amount to an instance of testimony.
  And, if \nagent{11}'s statement did not amount to an instance of testimony then it would seem the agent lacks a line of reasoning that indicates that \nagent{11} is trustworthy when speaking on matters regarding their personal character.
\end{note}

\begin{note}
  The preceding is only an expression of an intuition.
  \ideaCSA{} and \ideaCSB{} are ideas, and the arguments that follow will rest on the assumptions and propositions drawn from these ideas, rather than the ideas themselves.
  Still, to the extent motivation for those assumptions and propositions rest on these ideas, \autoref{illu:CS:test:basic} highlights the constraints the ideas place on claiming support.

  \begin{itemize}
  \item given \ideaCSA{}, the agent is required to consider the possibility that conclusion of their reasoning is not the case, and
  \item given \ideaCSB{}, the agent is required to hold that their reasoning indicates that the conclusion of their reasoning is not the case regardless of whether the possibility that conclusion of their reasoning is not the case obtains.
  \end{itemize}
  The reasoning of \autoref{illu:CS:test:basic} fails to be an instance of claiming support because the possibility that \nagent{11} is not trustworthy when speaking on matters regarding their personal character is sufficient to undercut the premise that \nagent{11} testified.
\end{note}

\begin{note}
  Now, granting that the above identifies a problem, an immediate question is:
  Is the problem an instance of (vicious) circularity?\nolinebreak
  \footnote{
    In advance of following discussion, variation on~\textcite{Sorensen:1991wh}.

    Claim support for the following proposition from the following sentence:

    \begin{quote}
      Some sentences are typed on a computer.
    \end{quote}

    In line with suggestions of \citeauthor{Sorensen:1991wh}, seems fine.
    No difficulty with ideas as compatible with something else happening.
    But, background that this is so incredibly unlikely.
  }

  Circularity is certainly in the ballpark, but I do not think there is a straightforward reduction.
\end{note}

\begin{note}
  First, the problem was motivated by view the agent's reasoning as an instance of claiming support and applying the two basic ideas of claiming support (\ideaCSA{} and \ideaCSB{}).
  And, in general, it seems that circularity extends beyond the scope of claiming support.

  For example, consider knowledge:
  The reasoning of \autoref{illu:CS:test:basic} seems problematic when view from the perspective of establishing knowledge.
  Yet, from such a viewpoint it seems \ideaCSA{} would not apply --- if the agent had come to know that \nagent{11} is trustworthy when speaking on matters regarding their personal character then it would not be possible (at least relative to the conclusion of their reasoning) that \nagent{11} is not trustworthy.
  Hence, neither \ideaCSA{} nor \ideaCSB{} would apply.

  To be clear, our interest is not that the claiming support explains why the reasoning is problematic.
  Rather, the point is that the sketch given of why the reasoning is problematic when viewed as an instance of claiming support is distinct from circularity, as it seems circularity would extend to cases for which \ideaCSA{} nor \ideaCSB{} would not apply.
\end{note}

\begin{note}
  Second, the term `circularity' suggests that the reasoner has taken the conclusion of the reasoning for granted.

  For example, consider what \citeauthor{Sgaravatti:2013wu} terms the `Justification Account' of circularity.\nolinebreak
  \footnote{
    As \citeauthor{Sgaravatti:2013wu} notes, the Justification Account of circularity is a rewriting of the third type of `epistemic dependence' considered by \citeauthor{Pryor:2004ws}~(\citeyear[359]{Pryor:2004ws}).
    Neither \citeauthor{Pryor:2004ws} nor \citeauthor{Sgaravatti:2013wu} endorse the Justification Account, but I take the spirit of the account to sufficient for interest.
    Still, the considerations which follow also apply to distinguish the {\color{red} problem identified} from \citeauthor{Sgaravatti:2013wu}'s favoured account (\Citeyear[\S3]{Sgaravatti:2013wu}) and the fifth type of `epistemic dependence' considered by \citeauthor{Pryor:2004ws}~(\citeyear[359]{Pryor:2004ws}).
  }

  \begin{quote}
    \begin{enumerate}[label=(JA), ref=(JA)]
    \item\label{sg:JA} An argument is circular if and only if for you to have justification to believe the premisses, it is necessary that you have justification to believe the conclusion.\nolinebreak
      \mbox{}\hfill\mbox{(\Citeyear[754]{Sgaravatti:2013wu})}
    \end{enumerate}
  \end{quote}
  Where `justification to believe' is to be read as in terms of having formed the belief in an epistemically appropriate way as opposed to (merely) possessing sufficient resources to form formed the belief in an epistemically appropriate way.\nolinebreak
  \footnote{
    Or, however you prefer to characterise \citeauthor{Firth:1978vi}'s (\Citeyear{Firth:1978vi}) distinction between doxastic and propositional justification (or warrant).
    See also \citeauthor{Silva:2020aa} (\Citeyear{Silva:2020aa}) --- esp.\ fn.\ 1.
  }
  (\citeauthor[Cf.][754--755]{Sgaravatti:2013wu})

  Observe, \ref{sg:JA} applies to \autoref{illu:CS:test:basic} only if the agent requires a justified belief that \nagent{11} is trustworthy prior to the conclusion of the agent's reasoning.

  Such may be the case, and plausibly is, but to the extent that the instance of claiming support is an instance of forming a justified belief, the problem highlighted by appeal to \ideaCSA{} and \ideaCSB{} relied only on entertaining the possibility that \nagent{11} is not trustworthy.
  Still, that it was necessary for the agent to have claimed support for \nagent{11} being trustworthy in order to claim support for the premises does not follow without additional argument --- no matter how plausible this may be.\nolinebreak
  \footnote{
    Though I have doubts about whether this really is the case.
  }

  % {
  % \color{red}
  % Necessary, as could argue that this is an implicit assumption.
  % Hence, .
  % Given plausibility, possible.

  % Yet, equally, that these two things don't go together regardless of perspective on premise.

  % And, in this sense order of explanation would be reversed.
  % Still a question of why required.
  % }

  %   Admittedly this is a somewhat delicate point.
  %   One may argue that \ref{sg:JA} (or some variant) explains why it is (intuitively) not possible for the agent to entertain possibility that \nagent{11} is not trustworthy given their reasoning.

  %   Indeed, even if it is the case that the agent is required to have claimed support (or a justified belief) for \nagent{11} being trustworthy to claim support for the premises, the problem identified via \ideaCSB{} would remain distinct and would seem at best to motivate such an additional restriction.
  Leaves open the possibility that problem highlighted by \ideaCSA{} and \ideaCSB{} does not reduce.
  Hence, seems that distinguish intuitions from \ideaCSA{} and \ideaCSB{} from intuitions about why or how the agent introduced \nagent{11} testifying as a premise.
\end{note}

\begin{note}
  The basic intuition, really, is that from \ideaCSA{} there is the possibility of being \mom{}.
  In particular, \mistaken{} about testimony.
  So, various ways in which this premise may fail.
  The conclusion not holding is one such way.
  And, if this is the case then whatever considerations the agent has for testimony, those considerations do not extend.
\end{note}

\begin{note}
  Still, rather than toy in the abstract, let's investigate further by granting the agent with a way of claiming support for the initial premise of the reasoning.
  The problem identified by \ideaCSA{} and \ideaCSB{} will remain, but to press the problem of circularity will require stronger assumptions.
\end{note}

\begin{note}[Testimony 2]
  \begin{illustration}[Testimony 2]\label{illu:CS:test:with-CS-for-premise}
    \mbox{}
    \begin{enumerate}[label=\arabic*., ref=(\arabic*)]
    \item I was assured that \nagent{11} will be honest with me throughout the meeting.
    \item\label{ex:eiS:tt:test} \nagent{11} testified that they are trustworthy on matters regarding their personal character.
    \item Any agent and proposition, agent testified that proposition is the case only if proposition is the case.
    \item \nagent{11} testified that p is the case only if p is the case.
    \item\label{ex:eiS:tt:ok} \nagent{11} is trustworthy on matters regarding their personal character.
    \end{enumerate}
  \end{illustration}

  \autoref{illu:CS:test:with-CS-for-premise} seems intuitively problematic to a similar degree as \ref{illu:CS:test:basic}.

  The prior assurance does not help, at least given context that the friend was not aware of what \nagent{11} would say.

  However, the prior assurance does provide a clear account of how the agent introduced \nagent{11} testifying as a premise.
\end{note}

\begin{note}
  Problem here remains.
  Now, two instances of claiming support.

  Observe, vouched for particular instance, but not general.
  Assurance went for statements in general, but now have information about particular statement.
  Seems okay in various cases.

  \begin{itemize}
  \item The venue will be crowded.
  \item A squirrel took the birdseed.
  \item \TeX is Turing-complete
  \end{itemize}
  (Observe, in these instances neither a \requ{} nor an \expec{}.)
  Assurance is sufficient.
  In general, so long as appeal to instance of testimony while granting that content of that instance of testimony (and hence the testimony itself) may fail, then there is no tension with the assumptions made regarding claimed support.

  Yet, the reasoning seems to remain problematic with respect to self-attribution of trustworthiness.
\end{note}

\begin{note}
  The issue for \ref{sg:JA} is that it does not seem necessary for the agent to have justification to believe that \nagent{11} is trustworthy on matters regarding their personal character in order to have justification to believe that \nagent{11} would be honest throughout the meeting.

  Instead, it seems that there is a limitation on the scope of the assurance that is sensitive to the content of what \nagent{11} said (or would say).
  In turn, the agent's reasoning is problematic because the reasoning exceeds the relevant limitation.
  And, while it may be the case that would need justification for conclusion to have unlimited justification for the premises, the limitation alone is sufficient identify the problem.

  This is preferred understanding of the problem raised by \ideaCSA{} and \ideaCSB{}.
\end{note}

\begin{note}
  However, granting that the problem arises from some limitation, it is important to keep in mind that the limitation arises from entertaining some possibility as opposed to assuming that the possibility obtains.

  An instance of a limitation arising from assuming that the possibility obtains is the fourth type of dependence between premise and conclusion considered by \citeauthor{Pryor:2004ws}.

  \begin{quote}
    [Type 4] dependence between premise and conclusion is that the conclusion be such that evidence \emph{against it} would (to at least some degree) undermine the kind of justification you purport to have for the premises.\nolinebreak
    \mbox{}\hfill\mbox{(\citeyear[359]{Pryor:2004ws})}
  \end{quote}

  Again, plausible.

  Issue:
  \begin{enumerate}
  \item Evidence undermines the kind of justification the agent purports to have for the premises.
  \end{enumerate}

  Seems to hold for \autoref{illu:CS:test:basic}.
  Would undermine appeal to \nagent{11}'s testimony.
  However, fails for \autoref{illu:CS:test:with-CS-for-premise}.
  Would not undermine the companion's assurance, even if viewed as an instance of testimony.

  Run this through the other instances suggested.

  And, as \citeauthor{Pryor:2004ws} notes, \emph{kind} is important.
  However, it seems kind is not the only problem.
\end{note}

\begin{note}
  Also compatible with \citeauthor{Pryor:2004ws}'s objection to type 4 being sufficient to identify problematic reasoning.
  Details are in the following footnote.\footnote{
  Compatible with \citeauthor{Pryor:2004ws}'s objection to type 4 dependence.

  % \begin{illustration}
    % \mbox{}
    % \vspace{-\baselineskip}
    \begin{quote}
      Suppose you're watching a cat stalk a mouse. Your visual experiences justify you in believing:

      \begin{enumerate}[label=(\arabic*), ref=(\arabic*)]
        \setcounter{enumi}{10}
      \item\label{illu:Pryor:cat:1} The cat sees the mouse.
      \end{enumerate}

      You reason:

      \begin{enumerate}[label=(\arabic*), ref=(\arabic*), resume]
      \item\label{illu:Pryor:cat:2} If the cat sees the mouse, then there are some cases of seeing.
      \item\label{illu:Pryor:cat:3} So there are some cases of seeing.\nolinebreak
        \mbox{}\hfill\mbox{(\citeyear[361]{Pryor:2004ws})}
      \end{enumerate}
    \end{quote}
  % \end{illustration}

  Setting aside whether this is fine.

  Following \citeauthor{Pryor:2004ws}:

  Bad, given proposal, as if no cases of seeing, then the cat is not seeing. (\citeyear[361]{Pryor:2004ws})

  \citeauthor{Pryor:2004ws}'s position is as follows:

  \begin{quote}
    I don't think you need antecedent justification to believe \ref{illu:Pryor:cat:3}, before your experiences can give you justification to believe \ref{illu:Pryor:cat:1}.
    I also think it's plausible that your perceptual justification to believe \ref{illu:Pryor:cat:1} contributes to the credibility of \ref{illu:Pryor:cat:3}.\nolinebreak
    \mbox{}\hfill\mbox{(\citeyear[361]{Pryor:2004ws})}
  \end{quote}

  This is compatible with \ideaCSA{} and \ideaCSB{}, and the assumptions and premises which have followed.
  You need not agree with \citeauthor{Pryor:2004ws}

  No clear trouble.
  The possibility alone isn't going to do enough.
  Still seems possible for 1 to hold up.
  Would count against, but not clear that entertaining possibility raises issue for claimed support for premise.
  Indeed, also compatible with the cat not seeing, but if this is the case then it's not clear why no cases of seeing is any more important than the singular case.
  }
\end{note}

\begin{note}
  Of course, still some question about why entertaining the possibility is sufficient.
  And, why there does not seem to be considerations for certain consequences of claimed support.

  Nothing insightful beyond broad account.

  Given this, the purpose of the assumptions made is to narrow down a clearer problem.
  No reasoning, and as an instance of this when there is the impossibility of such reasoning.

  Examples of the former.
\end{note}

\paragraph{Illustrations with respect to assumptions}

\begin{note}
  Previous section.
  \ideaCSA{} and \ideaCSB{}.
  Motivation for assumptions, and effort to clarify in contrast to other intuitions.
  Fell short of an account of the intuition.

  \autoref{assu:supp:nfactive}, \autoref{assu:supp:independence}.
  And, from these \autoref{prop:CS-nai}.

  Key here is lack of reasoning.
  This allows for the possibility that some variant instance of reasoning would succeed in claiming support.
  Indeed, we will suggest.
\end{note}

\begin{note}
  Two short, then one in some detail.
\end{note}

\subparagraph{Spot the difference}

\begin{note}[Spot the difference]
  \begin{illustration}[Spot the difference]\label{illu:CS:spot-the-diff}
    The agent has been working through a spot-the-difference to pass some time.

    Though the time is not completely passed, the agent examined the two images with what seems sufficient care to claim support that they have found all the differences.
    However, the agent did not keep track of the number of differences.

    The agent announces `I have found all the differences' and their companion responds `All fifteen?'.

    \begin{enumerate}[label=\arabic*., ref=(I\ref{illu:CS:spot-the-diff}.\arabic*)]
      \setcounter{enumi}{-1}
    \item\label{illu:CS:spot-the-diff:info} If I have found all the differences, I have found fifteen differences.
    \end{enumerate}

    The agent then reasons as follows:

    \begin{enumerate}[label=\arabic*., ref=(I\ref{illu:CS:spot-the-diff}.\arabic*), resume]
    \item Exhaustive search.
    \item\label{illu:CS:spot-the-diff:all} I have found for all the differences.
    % \item\label{illu:CS:spot-the-diff:info} My companion has testified that there are fifteen differences.
    % \item\label{illu:CS:spot-the-diff:cond} If I have found all the differences, I have found fifteen differences.
    \item\label{illu:CS:spot-the-diff:fif} So, I have found fifteen differences. \hfill (From \ref{illu:CS:spot-the-diff:info} and \ref{illu:CS:spot-the-diff:all})
    \end{enumerate}
  \end{illustration}

  Before going further, structure of this.

  Claimed support for having found all the differences.
  The reasoning is mentioned but not stated in the illustration.
  Rather, present is distinct instance of reasoning after being provided with information.
  If not 15, then problem.
  However, the present reasoning does not consider possibility.
  Instead, consequence of previously claimed support.
\end{note}



\begin{note}
    Providing additional information about what the agent has claimed support for.
  Recall, \autoref{assu:CSVP}, information rather than states of affairs.
  \nolinebreak
  \footnote{
    Still slight issue.
    Offering a redescription.
    You met Clark Kent, so you met Superman.
    In this case, rather than claiming support for meeting Superman, provided information is seen as an equivalent formulation.
    It is possible to read \autoref{illu:CS:spot-the-diff} in this way, and this might be the most natural interpretation.
    However, it is not the interpretation under which see the problem.
    Rather, problem is where the conditional is explicit.
    Unlike Superman case, proper conditional.
  }
\end{note}

\begin{note}
  Information leads to \requ{}.

  And, present reasoning does not include reasoning about \requ{}.



  Simple for \requ{}.
  \expec{} as agent did not have information about how many differences, and was not keeping track.
\end{note}

\begin{note}
  Problem:
  Possibility of not fifteen.
  No reasoning about this expectation.
  Problem is that the agent has not considered whether the novel information places a limitation on their appeal to having found all the differences.
\end{note}

\begin{note}
  Argued above against circularity.
  Here, additional consideration.

  If the agent were to have had the information first time, then plausibly an instance of circularity.
  And, may think that this is also circularity as must also all must amount to fifteen.
\end{note}

\begin{note}
  This doesn't rule out some additional reasoning.
  \begin{enumerate}
  \item Exhausted search.
  \end{enumerate}
  Difference here is that the agent is not only appealing to having found.
  In addition, considerations that lead them to claiming that support.

  Whether you think this is enough is up to you.
  On the one hand, claimed support for no more differences.\nolinebreak
  \footnote{
    Indeed, reasoning framed with all as I think it is much less clear here.
  }
  So, what matters is not that found all but rather that exhaustivity of search.

  On the other hand, the agent did not keep track of the number of differences.
  So, may hold that they should go back and count.\nolinebreak
  \footnote{
    Looking ahead, \nI{}.
    Difficulty here is that don't need to go to \(\phi\).
    Indeed, note somewhere that \nI{} really only clearly takes hold when need some sort of factivity in play.
    We'll return to this.
  }
\end{note}

\begin{note}
  So, doesn't seem like circularity should apply here.
  For, if it does, then consider the variant as no improvement.
\end{note}

\subparagraph{Where's Wally}

\begin{note}
  \autoref{illu:CS:spot-the-diff} had something that could be obtained from the reasoning if re-examined.
  Just need to add a counter.

  Now, something that follows if the reasoning was \nmom{}.
\end{note}

\begin{note}
  \begin{illustration}[Where's Wally]
    \label{illu:CS:wheres-wally}
    Searching for Wally.
    On front of book is an image of wally in contrast to a number of other characters.
    Takes not of a number of features.
    Glasses, hat, striped jumper.
    In isolation, necessary but insufficient.
    Combined, sufficient.

    Search through the image.
    I've found Wally.
    Did you spot the cane first?

    The question carries implicit information:
    \begin{enumerate}[label=\arabic*., ref=(I\ref{illu:CS:wheres-wally}.\arabic*)]
      \setcounter{enumi}{-1}
    \item\label{illu:CS:wheres-wally:info} The individual identified is Wally only if the individual is holding a cane.
    \end{enumerate}

    The agent then reasons as follows:

    \begin{enumerate}[label=\arabic*., ref=(I\ref{illu:CS:wheres-wally}.\arabic*), resume]
    \item Collection of features sufficient.
    \item\label{illu:CS:wheres-wally:ante} Had features.
    % \item The individual identified is Wally only if the individual is holding a cane.
    \item The individual was holding a cane. \hfill (From \ref{illu:CS:wheres-wally:info} and \ref{illu:CS:wheres-wally:ante})
    \end{enumerate}
  \end{illustration}

  As with \autoref{illu:CS:spot-the-diff}, the reasoning of \autoref{illu:CS:wheres-wally} is such that the individual was holding the cane is an expectation of the claimed support that the individual was Wally, and no reasoning about it.
\end{note}

\begin{note}
  Problem.
  \requ{} and \expec{}.
  It's not clear need to give up claimed support, but does not extend to having a cane.

  In contrast to \ref{illu:CS:spot-the-diff}, does not seem there is anything weaker to fall back on.
  However, no strong claim here.
  Rely on stronger principles about claiming support.
\end{note}

\subparagraph{A trip to the zoo}

\begin{note}
  Illustrations \ref{illu:CS:spot-the-diff} and \ref{illu:CS:wheres-wally} looked at \expec{1} and reasoning.
  Final illustration is no different, structurally similar to \autoref{illu:CS:wheres-wally}.
  However, surrounding discussion to clarify details.

  After discussion, a corollary and a conjecture.
\end{note}

\begin{note}
  Zebra.

  Dretske is about knowledge.
  Problem for knowledge as factive.

  Still, don't need factive move.
  Possible not zebra, but vision is sufficient to expect that such a possibility does not obtain.

  Key here is that claiming support is never going to be strong enough to establish knowledge, at least to the extent that knowledge is factive.
\end{note}

\begin{note}
  \begin{illustration}[A trip to the zoo]
    \label{illu:CS:dretske-zebra}
    \mbox{}
    \vspace{-\baselineskip}
  \begin{quote}
    You take your son to the zoo, see several zebras, and, when questioned by your son, tell him they are zebras.
    Do you know they are zebras?
    Well, most of us would have little hesitation in saying that we did know this.
    We know what zebras look like, and, besides, this is the city zoo and the animals are in a pen clearly marked ``Zebras.''
    Yet, something's being a zebra implies that it is not a mule and, in particular, not a mule cleverly disguised by the zoo authorities to look like a zebra.
    Do you know that these animals are not mules cleverly disguised by the zoo authorities to look like zebras?

    \mbox{ }\hfill \(\vdots\) \hfill\mbox{ }

    Did you examine the animals closely enough to detect such a fraud?\linebreak
    \mbox{}\hfill\mbox{(\citeyear[1015--1016]{Dretske:1970to})}
  \end{quote}
  \vspace{-\baselineskip}
  \end{illustration}
\end{note}

\begin{note}
  \citeauthor{Dretske:1970to}'s presentation focuses on knowledge, so let us briefly form a parallel with respect to claiming support:

  \begin{illustration}
    \label{illu:dretske-zebra-var}
    \mbox{}
    `What if those animals are mules cleverly disguised by the zoo authorities to look like zebras?'
    \begin{enumerate}[label=\arabic*., ref=(I\ref{illu:CS:wheres-wally}.\arabic*)]
      \setcounter{enumi}{-1}
    \item If zebra, then not cleverly disguised mule.
    \end{enumerate}
    Reasons as follows:
    \begin{enumerate}[label=\arabic*., ref=(I\ref{illu:CS:wheres-wally}.\arabic*)]
    \item That animal appears to be a zebra.
    \item That animal is a zebra.
    \item That animal is not a cleverly disguised mule.
    \end{enumerate}
  \end{illustration}

  As with Illustrations~\ref{illu:CS:spot-the-diff} and~\ref{illu:CS:wheres-wally}, no reasoning about the possibility that the animal is a cleverly disguised mule.
  However, given conditional, \expec{} of claimed support for zebra.

  Introduction of additional consequence, so same structure as~\autoref{illu:CS:wheres-wally}.
  And, plausible that the agent may reason about the possibility that the animal is a cleverly disguised mule in a way sufficient to claim support.\nolinebreak
  \footnote{
    \citeauthor{Dretske:1970to} has a number of suggestions.
    \begin{quote}
    You have some general uniformities on which you rely, regularities to which you give expression by such remarks as, ``That isn't very likely'' or ``Why should the zoo authorities do that?''
    Granted, the hypothesis (if we may call it that) is not very plausible, given what we know about people and zoos.
    But the question here is not whether this alternative is plausible, not whether it is more or less plausible than that there are real zebras in the pen, but whether you know that this alternative hypothesis is false.\nolinebreak
    \mbox{}\hfill\mbox{(\citeyear[1016]{Dretske:1970to})}
  \end{quote}
  }
\end{note}

\begin{note}
  if think that doesn't know, then not too much of an issue.
  However, does know?
  Indeed, \citeauthor{Dretske:1970to}.

  No clear tension.
  Knowledge and claiming support, different.

  However, problem identified.
\end{note}

\begin{note}
  \begin{enumerate}[label=K\Alph*., ref=K\Alph*]
  \item\label{Dretske:No-C:cond:no-k-then-ep} If an agent does not know that \(\phi\) is true, then \(\phi\) being false is an epistemic possibility for that agent.
  \item\label{Dretske:No-C:cond:ep-then-no-k} If \(\phi\) being false is an epistemic possibility for some agent, then that agent does not know that \(\phi\) is true.
  \end{enumerate}
  When taken together, (\ref{Dretske:No-C:cond:no-k-then-ep}) and (\ref{Dretske:No-C:cond:ep-then-no-k}) state that: An agent knows that \(\phi\) is true if and only if \(\phi\) being false is not an epistemic possibility for the agent.
  Still, our interest will primarily be with (\ref{Dretske:No-C:cond:ep-then-no-k}).\nolinebreak
  \footnote{
    \label{fn:factivity-two-readings}
    Contrast to `factivity':
    \begin{itemize}
    \item If \(S\) knows that \(\phi\), then \(\phi\).
    \end{itemize}
    This may be read in at least two different ways.
    \begin{itemize}
    \item First, relation between epistemic state of the agent and state of the world:\newline
      \mbox{} \qquad If \(S\) knows that \(\phi\), then the state of the world is such that \(\phi\) is the case.
    \item Second, how things appear from epistemic state:\newline
      \mbox{} \qquad If \(S\) knows that \(\phi\) then every way the state of the world may be for \(S\) includes \(\phi\).
    \end{itemize}

    The two reading are independent of one another.

    For example, suppose you walked to the shop but the only epistemic possibility entertained by your friend is that you drove to the shop.
    Here, it is not possible for your friend to know that you drove to the shop on the first reading of factivity, but the second reading is not ruled out.

    Conversely, suppose it is the case that you walked to the shop but your friend considers it epistemically possibly that you drove.
    Here, knowing on the second reading of factivity is ruled out, but the first reading is not ruled out.

    Of course, you may endorse both readings of factivity.
    Our focus is on the `weaker' reading as we have made no connexion between claiming support and the state of the word.
    (Perhaps it is of some interest to note that \citeauthor{Dretske:1970to} explicitly denies the second reading, but not the first.)
  }\(^{,}\)\nolinebreak
  \footnote{
    The dogmatism paradox (\cite[39,43--45]{Kripke:2011wv};\cite[148]{Harman:1973ww}) seems to concern the second reading of factivity from~\autoref{fn:factivity-two-readings}, and intuitions concerning evidence.

  Roughly stated, the paradox pairs the following two propositions:
  \begin{enumerate}[label=D\arabic*., ref=(D\arabic*)]
  \item\label{dog:1} If an agent is aware that they know that \(\phi\), then the agent may disregard any evidence against \(\phi\).
  \item\label{dog:2} Rational agents respect their evidence
    (\cite[Cf.][\S2]{Kelly:2016wk})
  \end{enumerate}
  Given~\ref{dog:2}, it seems no agent should not disregard any instance of evidence, even if the antecedent of~\ref{dog:1} is satisfied.

  And, it seems \ref{dog:1} is motivated by factivity.
  For, if the agent is aware that they know that \(\phi\) then the agent knows that \(\phi\).
  And, as knowledge is factive it follows (by second reading) that \(\phi\) is the case.
  In turn, if it is the case that \(\phi\) then any evidence against \(\phi\) is evidence for something that is not the case.
  Hence, the agent may disregard any evidence against \(\phi\).


  Indeed, the second reading of factivity seems required.
  For, it seems an agent is only (apparently) in a position disregard any evidence against \(\phi\) because there knowledge that \(\phi\) guarantees that \(\phi\) is the case.
  If \emph{not}-\(\phi\) is (merely) an epistemic impossibility, and it is not clear why evidence may require an agent to revise what they consider possible.

  Note:
  Neither \citeauthor{Kripke:2011wv} (nor \citeauthor{Harman:1973ww}) make explicit mention of the agent being aware that they know \(\phi\) when formulating the Dogmatism paradox.
      Still, the paradox is clearer with this stated, as it's require addition work to find issue with a permission (to disregard evidence) if an agent is not aware that they have such a permission.

      More generally, I agree with \citeauthor{Zhaoqing:2015vj}'s (\Citeyear{Zhaoqing:2015vj}) proposal to understand the paradox in terms of knowledge attribution rather than of knowledge proper.
  }
  \citeauthor{Dretske:1970to} observes that endorsing (\ref{Dretske:No-C:cond:no-k-then-ep}) and (\ref{Dretske:No-C:cond:ep-then-no-k}) leads to closure.
  The following is a reconstruction.\nolinebreak
  \footnote{
    Specifically, the following passage:
    \begin{quote}
      A slightly more elaborate form of the same argument goes like this:
      If \(S\) does not know whether or not \(Q\) is true, then for all he knows it might be false.
      If \(Q\) is false, however, then \(P\) must also be false.
      Hence, for all \(S\) knows, \(P\) may be false.
      Therefore, \(S\) does not know that \(P\) is true.\nolinebreak
      \mbox{}\hfill\mbox{(\citeyear[1011]{Dretske:1970to})}
    \end{quote}
    Note: (\ref{Dretske:No-C:cond:no-k-then-ep}) is a reformulation of the first conditional of the passage, while a formulation (\ref{Dretske:No-C:cond:ep-then-no-k}) seems required to move from `\(P\) may be false' to `\(S\) does not know that \(P\) is true'.
  }
\end{note}

\begin{note}[Closure argument]
  Let \(S\) be some agent and suppose:
  \begin{enumerate}[label=\arabic*., ref=\arabic*]
  \item\label{Dretske:No-C:k-entail} \(S\) knows that \(\phi\) entails \(\psi\).
  \item\label{Dretske:No-C:dunno-psi} \(S\) does not know that \(\psi\) is true.
  \end{enumerate}
  Consider the following argument:
  \begin{enumerate}[label=\arabic*., ref=\arabic*,resume]
  \item\label{Dretske:No-C:ep-not-psi} \(\psi\) being false is an epistemic possibility for \(S\).%
    \hfill (\ref{Dretske:No-C:cond:no-k-then-ep} \& \ref{Dretske:No-C:dunno-psi})
  \item\label{Dretske:No-C:no-ep-no-entail} \(\phi\) not entailing \(\psi\) is not an epistemic possibility for \(S\)%
    \hfill (\ref{Dretske:No-C:cond:ep-then-no-k} \& \ref{Dretske:No-C:k-entail})
  \item\label{Dretske:No-C:ep-not-psi-and-phi}  \(\phi\) being true while \(\psi\) is false is not an epistemic possibility for \(S\).%
    \hfill (\ref{Dretske:No-C:no-ep-no-entail})
  \item\label{Dretske:No-C:ep-not-phi} \(\phi\) may be false.%
    \hfill (\ref{Dretske:No-C:ep-not-psi} \& \ref{Dretske:No-C:ep-not-psi-and-phi})
  \item\label{Dretske:No-C:not-k-phi} \(S\) does not know that \(\phi\) is true.%
    \hfill (\ref{Dretske:No-C:cond:ep-then-no-k} \& \ref{Dretske:No-C:ep-not-phi})
  \end{enumerate}

  Hence, we have shown that, given (\ref{Dretske:No-C:cond:no-k-then-ep}) and (\ref{Dretske:No-C:cond:ep-then-no-k}), (\ref{Dretske:No-C:k-entail}) and (\ref{Dretske:No-C:dunno-psi}) imply (\ref{Dretske:No-C:not-k-phi}).

  That is to say, we have shown:
  \begin{enumerate}[label=K\Alph*., ref=(K\Alph*)]
    \setcounter{enumi}{2}
  \item\label{K:closure:from-arg} If \(S\) knows that \(\phi\) entails \(\psi\) and \(S\) does not know that \(\psi\) is true, then \(S\) does not know that \(\phi\) is true.%
    \mbox{} \hfill \((K_{S}(\phi \rightarrow \psi) \land \lnot K_{S}\psi) \rightarrow \lnot K_{S}\phi\)
  \end{enumerate}
  And rewriting:\nolinebreak
  \footnote{
    \((\phi \land \lnot\psi) \rightarrow \lnot\xi\) iff \(\phi \rightarrow (\lnot\psi \rightarrow \lnot\xi)\) iff \(\phi \rightarrow (\xi \rightarrow \psi)\).
  }
  \begin{enumerate}[label=K\Alph*\('\)., ref=(K\Alph*\('\))]
    \setcounter{enumi}{2}
  \item\label{K:closure:standard} If \(S\) knows that \(\phi\) entails \(\psi\), then if \(S\) knows that \(\phi\) is true then \(S\) knows that \(\psi\) is true.%
    \mbox{} \hfill \(K_{S}(\phi \rightarrow \psi) \rightarrow (K_{S}\phi \rightarrow K_{S}\psi)\)
  \end{enumerate}
\end{note}

\begin{note}
  Return to \autoref{illu:CS:dretske-zebra}.

  Let us assume you know that:
  \begin{itemize}
  \item If the animals are zebras then the animals are not cleverly disguised mules. And,
  \item The animals are zebras.
  \end{itemize}

  If \ref{K:closure:standard} holds, then you also know that the animals are not cleverly disguised mules.
  However, following \citeauthor{Dretske:1970to}'s intuition, you do not know that the animals are cleverly disguised mules.

  Hence, to accommodate \citeauthor{Dretske:1970to}'s intuition, either (\ref{Dretske:No-C:cond:no-k-then-ep}) or (\ref{Dretske:No-C:cond:ep-then-no-k}) must be rejected.
  \citeauthor{Dretske:1970to} rejects (\ref{Dretske:No-C:cond:ep-then-no-k}).\nolinebreak
  \footnote{
    See below.
  }
\end{note}

\begin{note}
  Observe, the result of substituting `claimed support' from `knowledge' in (\ref{Dretske:No-C:cond:ep-then-no-k}) conflicts with \autoref{assu:supp:nfactive}.
  For, \autoref{assu:supp:nfactive} requires that claiming support for \(\phi\) is compatible with the (epistemic) possibility that the claimed support is \nmom{}.
  And, the claimed support is \misled{} just in case \(\phi\) is not the case.
  Hence~\autoref{assu:supp:nfactive} requires that claimed support for \(\phi\) is compatible with the (epistemic) possibility that \(\phi\) is not the case.

  In the above respect,~\autoref{assu:supp:nfactive} parallels \citeauthor{Dretske:1970to}'s rejection of (\ref{Dretske:No-C:cond:ep-then-no-k}).
  However, \citeauthor{Dretske:1970to}'s rejection of (\ref{Dretske:No-C:cond:ep-then-no-k}) is motivated by a rejection of~\ref{K:closure:standard}.
\end{note}

\begin{note}[Link]
  We now return to claiming support.

  The refinement of~\autoref{idea:eiS} through Assumptions \ref{assu:supp:independence} and \ref{assu:independence-expec} has lead to something close to~\ref{K:closure:standard}.

  Both build on \autoref{def:requisite}.
  A \requ{}.
  This is a complex parallel to knowing that \(\phi\) entails \(\psi\).
  However, somewhat general, and applies to \(K\) also.
  {\color{red} (Strictly speaking, because of contraposition, make this more explicit below)}

  \autoref{assu:supp:independence} then requires reasoning.

  Turning to \ref{assu:independence-expec}, extend \requ{} to \expec{} through \autoref{def:expectation}.
  In turn, require reasoning when appealing to claimed support.

  In both cases, requiring something with respect to \(\psi\) having value \(v'\) given some state of the agent an principle which relates \(\psi\) having value \(v'\) to that state.

  For sure, appealing, or having claimed support is distinct, so the closure is not that with respect to an \emph{an} epistemic operator such as knowledge.
  However, the tension here is in terms of viewing the rejection of \ref{K:closure:standard} as the endorsement of a `locality constraint'.
  {\color{red} What this means.}
  And, seem to violate such a constraint.
\end{note}

\begin{note}[`Locality constraint']
  \begin{quote}
    To know that \(x\) is \(A\) is to know that \(x\) is \(A\) within a framework of relevant alternatives, \(B\), \(C\), and \(D\).
    This set of contrasts, together with the fact that \(x\) is \(A\), serve to define what it is that is known when one knows that \(x\) is \(A\).
    One cannot change this set of contrasts without changing what a person is said to know when he is said to know that \(x\) is \(A\).\nolinebreak
    \mbox{}\hfill\mbox{(\citeyear[1022]{Dretske:1970to})}
  \end{quote}

  Where:
  \begin{quote}
    A relevant alternative is an alternative that might have been realized in the existing circumstances if the actual state of affairs had not materialized.\nolinebreak
    \footnote{
      \citeauthor{Dretske:1970to} adds:
  \begin{quote}
    \dots alternatives that \emph{might} have been realized in the existing circumstances if the actual state of affairs had not materialized.
    \dots are not relevant alternatives.\nolinebreak
    \mbox{}\hfill\mbox{(\citeyear[fn.6][1021]{Dretske:1970to})}
  \end{quote}
    }
    \nolinebreak
    \mbox{}\hfill\mbox{(\citeyear[1021]{Dretske:1970to})}
  \end{quote}
  Different from a \requ{}.
\end{note}

\begin{note}
  Point is not direct clash.\nolinebreak
  \footnote{Two problems.

    First, \citeauthor{Dretske:1970to} seems to go with no recognition at time, compatible with claiming support at time.
    So, would need instance of appealing to knowledge for some other purpose.
    Restating is not sufficient, assumptions made are compatible with persistence (as noted).

    Second, only get no need to rule out from \citeauthor{Dretske:1970to}, which does not require no reasoning.
  }
  At issue, rather, it the degree to which possibility is compatible with the absence of reasoning.
  That is, if reject closure, is this due to reasoning or due to requirements placed on such reasoning?\nolinebreak
  \footnote{
    (Problem here, but also extends to \nI{}.)
  }

  There is some subtlety, however.
  Kind of closure seems to break for many attitudes.
  This is not at issue, distinguishing feature of claiming support is closure, and the constraints this places on an agent.
  However, reject for stronger attitudes, then why for weaker?
\end{note}

\begin{note}
  Reasoning, but haven't placed constraints.
  So, this allows for something that may be quite weak.
  % I am here only restating what has gone before, but the added context may help.
  No reasoning, then leaves open possibility that does lead to a problem with the reasoning performed.

  Consider again relevant alternatives from internalist perspective.
  It seems, agent determines whether relevant or not will require reasoning.

  Consider:
  \begin{quote}
    `What if those reports about the zoo authorities cleverly disguising animals to look like other animals?
    If there are, could those animals be cleverly disguised mules?'
  \end{quote}
  Perhaps not a relevant alternative if the sense of `might' is non-epistemic, but if epistemic, then seems a problem.
\end{note}

\begin{note}
  To summarise.

  \citeauthor{Dretske:1970to}'s case.
  Rejection of closure.
  Question whether assumptions are okay.
  Argued that these are given understanding of claimed support, and that they plausibly extend to knowledge (and other attitudes intuitively stronger than that of having claimed support).
  For, rejection of closure plausibly amounts to rejection on strength of reasoning, rather than requirement to reason.

  Focus on this point for two reasons:
  First, distinguishing feature of claiming support and following will be about claiming support.
  Second, appeal to similar limitation later.
\end{note}

\newpage



\paragraph{A corollary and a conjecture}

\begin{note}
  Talked about `closing principle'.
  Make this precise.
\end{note}

\begin{note}[Generalising point for K example]
  Generalise.
  Any time the agent appeals to a consequence.
  And, this is straightforward, because consequence is only going to hold with the antecedent.

  \begin{corollary}\label{corr:eiS:C:contraposition}
    Suppose an agent has prior claimed support for:
    \begin{enumerate}[label=\arabic*., ref=(\arabic*)]
    \item\label{corr:cond:p} \(\phi\) having value \(v'\). And,
    \item\label{corr:cond:pq} If \(\phi\) has value \(v\) then \(\psi\) has value \(v'\).
  \end{enumerate}
  Such that the agent claimed support for \ref{corr:cond:p} prior to \ref{corr:cond:pq} and as such the reasoning involved in claiming support for \ref{corr:cond:p} did not entertain the possibility of \(\psi\) not having value \(v'\).
  And, further suppose the agent holds that:
  \begin{enumerate}[label=\arabic*., ref=(\arabic*), resume]
    \item The claimed support for \ref{corr:cond:pq} (also) implies that if \(\psi\) does not have value \(v'\) then \(\phi\) does not have value \(v\).
  \end{enumerate}
  Then, if the agent engages in reasoning such that:
    \begin{enumerate}[label=\arabic*., ref=(\arabic*), resume]
    \item The agent does not appeal to anything other premises other than~\ref{corr:cond:p} and ~\ref{corr:cond:pq} from their prior claimed support, some form of conditional detachment applied to ~\ref{corr:cond:p} and~\ref{corr:cond:pq} to conclude that \(\psi\) has value \(v'\).
    \end{enumerate}
    Such that remains possible that \(\psi\) does not have value \(v'\).
    The reasoning is not an instance of claiming support.
  \end{corollary}

  \autoref{corr:eiS:C:contraposition} is a summary of the common problem between illustrations \ref{illu:CS:spot-the-diff},~\ref{illu:CS:wheres-wally}, and~\ref{illu:dretske-zebra-var}.
  And, as such captures sufficient conditions for \autoref{prop:CS-nai}.
\end{note}

\begin{note}
  Observe, from \autoref{prop:CS-nai} we have that reasoning that \(\psi\) has value \(v'\) is not an instance of claiming support {\color{red} if \expec{} with no reasoning about it as \expec{}}.

  By assumption, no reasoning about \(\psi\) not having value \(v'\) when claiming support for \(\phi\) having value \(v\).

  Therefore, to establish \autoref{corr:eiS:C:contraposition} we need only show that \(\psi\) having value \(v'\) is a \expec{} of the claimed support for \(\phi\) having value \(v\).

  And, the agent has come to consider \(\psi\) having value \(v'\) as a \requ{} of that claimed support as:
  \begin{itemize}
  \item Possible that \(\psi\) does not have value \(v'\), by assumption.
  \item If \(\psi\) does not have value \(v'\) then \(\phi\) does not have value \(v\), and hence the claimed support for \(\phi\) having value \(v\) would be \misled{}.
  \item And, as \(\phi\) having value \(v\) implies \(\psi\) has value \(v'\) it must be the case that \(\psi\) having value \(v'\) persists through to the conclusion of reasoning.
  \end{itemize}
\end{note}

\begin{note}
  Again, intuition is that claimed support for \(\phi\), but novel information about a possible defeater.
  So, it's no good to appeal to prior claimed support --- without additional reasoning about that claimed support --- as the claimed support did not take into account the possible defeater.
\end{note}

\begin{note}
  In general, expect this not to be much of a concern.

  Three considerations.
\end{note}

\begin{note}[Not claiming support]
  First, the obvious, claiming support and may not be the case that agent is claiming support in relevant reasoning.
\end{note}


\begin{note}[Deal with \requ{}]
  Second, in many cases claimed support for \(\phi\) is not going to hold up regardless of whether \(\psi\), so plausible reasoning about \(\psi\) as a \requ{}.

  For example:

  Alarm ringing, so fire.
  If no fire, not alarm ringing.
  However, even if consider the possibility that there isn't really a fire, clear that the alarm is ringing.
\end{note}

\begin{note}
  Strengthening, this will also not apply if the agent has claimed support that \(\phi \rightarrow \psi\) and \(\lnot \phi \rightarrow \psi\) and appeals.
  For, no \requ{} in this case.
\end{note}

\begin{note}[Requires contraposition]
  Finally, contraposition.
  It is not clear that contraposition always holds.

  And, without contraposition, retain claimed support for \(\phi\), and even if \mom{} \(\phi\) provides enough to think that \(\psi\) is the case.
  In particular, a variation of~\ref{corr:eiS:C:contraposition} does not \emph{necessarily} hold with respect to conditional probability.

  \begin{idea}\label{conj:eiS:C:probability}
    Assuming that sufficiently high probability is sufficient for claiming support.

    It is not necessarily the case that an agent may not claim support for \(\psi\) having value \(v'\) by appeal to:
    \begin{enumerate}[label=\arabic*., ref=(\arabic*)]
    \item\label{corr:prob:p} Claimed support for \(\phi\) having value \(v'\) such that no consideration of \(\psi\).
    \end{enumerate}

    Some time after \ref{corr:prob:p}:

    \begin{enumerate}[label=\arabic*., ref=(\arabic*), resume]
  \item\label{corr:prob:pq} Claimed support for \(\psi\) having value \(v'\) when \(\phi\) has value \(v\).
    \end{enumerate}

    If:
    \begin{enumerate}[label=\arabic*., ref=(\arabic*), resume]
    \item The agent does not appeal to any other premises other than claimed support for~\ref{corr:prob:p},~\ref{corr:prob:pq} and some principle regarding conditional probability.
    \end{enumerate}
    \vspace{-\baselineskip}
  \end{idea}

  Follows from the simple observation that conditional probability does not require that \(\phi\) is false when \(\psi\) is false.

  For example, set some threshold \(t\) and consider a probability distribution such that:
  \(P(\phi) > t\), \(P(\psi \mid \phi) > t\).
  It is consistent with such a distribution that \(P(\lnot\phi \mid \lnot\psi) = 0\).\nolinebreak
  \footnote{
    For example, if \(t = .0\), then let \(P(\phi) = .9\), \(P(\psi) = .92\), \(P(\phi \land \psi) = .81\), \(P(\lnot\phi \land \lnot\psi) = 0\).
  }
  And, therefore, from an agent's perspective it need not be the case that \(\phi\) would be false if \(\psi\) were false.
  In other words, it may be that entertaining possibility that \(\psi\) is false is just entertaining a restricted instance of \(\phi\) being true.
  Hence, \(\psi\) being true is not a \requ{} of \(\phi\) being true.

  Still,  \(\psi\) being true may be an `\emph{unrecognised} \requ{}' of \(\phi\) being true.
  However, we have made no assumptions regarding such `unrecognised \requ{1}'.

  {
    \color{red}
    Some care here, though.
    As the issue with possible defeaters is distinct from the probability of such defeaters.
    Indeed, given possibility, probability is typically low.
    Yet, this does not say anything about whether claimed support would hold up if the defeater turned out to be the case.

    The point of this {\color{red} idea} is to highlight an instance of a conditional that does not contrapose, and so does not lead to a \requ{}.
  }
\end{note}

\begin{note}
  These kinds of issues are fundamental to the argument.
  Issue is with a way of claiming support, rather than the possibility of claiming support.
  And, the issue is a result of the two assumptions made regarding claiming support.

  If something stronger or weaker, then the issue does not (necessarily) arise.
  Take it for granted that testimony.
  Knowledge excludes (epistemic) possibilities associated with defeaters.

  So, even if agent fails to claim support, may have done something interesting!

  However, focus is on claiming support.
  Possible defeaters, and some defence against those defeaters obtaining.

  Really important thing from these illustrations is that no particular assumptions about reasoning.
  Illustrations all relied on absence of reasoning.
  So, although somewhat strong, still plausibly weak.
  Mentioned at many times things that seem sufficient.
  However, did not rely on these.

  So, claiming support is this odd thing.
  We will argue for additional proposition.
  \nI{}.
  Again, this will not rely on assumptions about reasoning.
\end{note}

\subsection{Summary of assumptions}

\begin{note}
  Claiming support.

  Two key ideas:

  \begin{enumerate}
  \item From \nfcs{}, always possible.
  \item From \eiS{}, doesn't depend.
  \end{enumerate}
  Second in part motivated by the first.

  Intuitively, stronger than belief, but weaker than knowledge.

  {
    \color{red}
    Claiming support is strong in the sense that it requires the agent's reasoning to hold up even if things are not how the agent thinks.
    However, strength is mitigated as bar for expectation may be set low.
  }

  Consequence of the first is way of establishing that reasoning does not result in claiming support.
  These assumptions are important.
  Argument relies on these assumptions.

  May be that it is possible to revise, but seem to capture something sufficiently interesting.
\end{note}

\begin{note}
  Well, it's a little more complex.
  What the argument really depends on is \ESU{} and \nI{}.
  Specifically \nI{}.

  It's not clear to me that these assumptions are strictly speaking required for \nI{}.
  And, some motivation of \nI{} independently of these.
  Still, this is as far as I got with \nI{}.

  Just so happens that these two assumptions also offer a nice `functional' characterisation.
  Therefore, appeal to these in order to set the stage.
\end{note}

\hozline{}

\subsubsection{Test}

\begin{note}
  To help fix intuition, I suggest a test to clarify what is meant by `expect': The `even if\dots' test.
  So long as an agent may provide an adequate responses to the test, the agent will be in a position to claim support.
\end{note}

\begin{note}[The `Even if\dots' test]
  The `Even if\dots' test queries whether an agent's claimed support permits an agent to expect that some (epistemically) possible defeater fails to obtain `even if' it does obtain.

  For example, even if \(0.999\dots = 1\), there must be \emph{some} difference between \(0.999\dots\) and \(1\) --- no matter how small --- and some difference between to things is sufficient to establish that they are not equal.

  Implied in this response is something like the observation that \(0.9 = (1 - 0.1)\) and \(0.99 = (1 - 0.01)\), and so \(0.999\dots = (1 - 0.000\dots 1)\), hence \(1 = (0.999\dots + 0.000\dots 1)\), and because \(0.999\dots\) refers to some quantity, \(0.000\dots 1\) likewise refers to some quantity.
  It seems reasonable for an agent to expect that the Archimedean property does not hold for real numbers.

  The example given is an instance of the applied to the possibility that the agent's claimed support that \(0.999\dots \ne 1\) may be misleading, as the antecedent supposes that \(0.999\dots = 1\).

  Generalising, we have outlined two kinds of defeaters that would prevent an agent from claiming support.
  The two types of defeaters suggest two basic instances of the test:
  \begin{enumerate}
  \item[(ML)] Even if \(\phi\) does not have value my claimed support indicates, I consider it to be the case that\dots
  \item[(MT)] Even if I some part (or whole) of my claimed support for the value of \(\phi\) is mistaken, I consider it to be the case that\dots
  \end{enumerate}
  Below we provide three examples for each basic instance of the test, two (plausibly) successful responses and one (plausibly) unsuccessful response..\nolinebreak
  \footnote{
    You may think that some of the adequate responses I suggest are too weak, but for future purposes I require only that some positive answer many be given, and so you may strengthen the requirements on a positive answer as you see fit.
  }
\end{note}

\begin{note}[Even if: misled]
  We being with two plausibly satisfactory responses to being misled.

  \begin{enumerate}[label=(ML\arabic*), ref=(ML\arabic*), series=ML_counter]
  \item\label{ML:asleep} Even if that person is not sleeping, their eyes have been closed for a long time and their breathing is slow.
  \item\label{ML:lying} Even if you are telling the truth, the scientific consensus is against you.
  \end{enumerate}

  With \ref{ML:asleep} the agent has claimed support for the proposition that the person is sleeping.
  It's not too hard to give the impression of being asleep, so there is some possibility that the person is awake and support claimed is misleading.
  Still, even if the person is awake, the person is exhibiting sufficient signs of being asleep for the agent to expect that they are not misled.

  Turning to \ref{ML:lying}, it may be that the person is telling the truth and if the person is indeed telling the truth then any claimed support for a conflicting proposition must be mistaken.
  However, scientific consensus seems sufficient to claim support for the relevant conflicting proposition --- one expects that scientific consensus is not misleading given the rigours of the scientific process.
  Scientific consensus does not (at least typically) require that the person is not telling the truth (though will imply that to be the case).

  In contrast, consider an unsatisfactory response.

  \begin{enumerate}[label=(ML\arabic*), ref=(ML\arabic*), resume*=ML_counter]
  \item\label{ML:forgery} Even if plagiarised, it's not worth thinking about. (I.e.\ don't worry about claiming support).
  \end{enumerate}
  If the certificate is a forgery, then the claimed support for the proposition that the certificate is not a forgery would be misleading.

  Seems the reasoning here is too little.
  Hence, that the certificate self-certifies it's authenticity is no response to the (possibility) that it is a forgery.
  Immediate conflict with \eiS{} because it seems quite unreasonable to expect that the certificate is not a forgery based on it's self-certification.

  Of course, many certificates do self-certify (it would be excessive effort to identify a certificate and then be required to find information about what the certificate is for), and perhaps a simple observation that there are no signs of tampering may be a sufficient response to the `even if\dots' test.
\end{note}

\begin{note}[Even if: mistaken]
  Turning to the possibility of mistaken support, consider the following two instances of the `even if\dots' test.

  \begin{enumerate}[label=(MT\arabic*), ref=(MT\arabic*), series=MT_counter]
  \item\label{MT:fake-wound} Even if that is a fake wound, I have no way to tell and the actions of the (apparently) wounded would be a feat of acting.
  \item\label{MT:misquote} Even if the newspaper has quoted the wrong person, the paper has a strong record of accurate reporting.
  \end{enumerate}

  With respect to~\ref{MT:fake-wound}, it seems a mistake to treat a fake wound as indicate the presence of an actual wound, as a fake wound does not require a genuine would but likewise a fake wound may cover a genuine wound.
  The response to the `Even if\dots' test notes that the behaviour of the (apparently) wounded person is sufficiently consistent with their expectations of the behaviour of a person with the (apparent) wound, and would lead to be surprised if the person was not in fact wounded.

  Turning to~\ref{MT:misquote}, if the paper has quoted the wrong person then it would be a mistake to claim support that the person said whatever-it-is-they-said, though it may still be the case that the person did say whatever-it-is-they-said.
  Even so, the strong record of the paper seems sufficient for the agent to expect that the newspaper has not misattributed or imagined the quote on the relevant occasion.

  In contrast, consider an unsatisfactory response.

  \begin{enumerate}[label=(MT\arabic*), ref=(MT\arabic*), resume*=MT_counter]
  \item Even if this library does not using LCC indexing, the library does not have a copy of `Measurement Theory' because as search for `H61 .R593' returns no results.
  \end{enumerate}
  Holding that a library does not have a copy of a book because a search for the book under a particular indexing system would be a mistake.
  For, if the library does not use the particular indexing system then a search using that indexing system will always fail, regardless of whether or not the library has a copy of the book.

  In turn, a failed search for an LCC index in the library's database does not seems sufficient for an agent to claim that the library does not have a copy of the book unless the agent is in a position to claim support that the library uses LCC indexing.
  Following, it seems the failed response to the `Even if\dots' test may be supplemented by noting that the library is a research library, and therefore likely uses LCC indexing, etc.\
\end{note}


\subsection{Closing focus on claimed support}

\begin{note}[Closing support]
  To summarise, claim of support.
  Certain kind of independence.
  Only interested in support, and not how this relates to attitudes.
  Somewhat intuitive, but no claims that this is the only understanding of support.

  For the moment, this provides clarity for understanding of support.
  Below, use to argue for failure to claim support.
\end{note}

\begin{note}[Something to emphasise]
  \color{red}
  Something to emphasise here is that this means that there's a way for an agent to claim support without being certain that \(\phi\) is the case.
  I don't have any answers for what this is.
  However, I do take this to be highly intuitive.
\end{note}


\begin{note}[Adequate]
  Kind of reasoning that we, the folk, do.
  Distinction for claiming support is that this is different from whether the agent has support, and we may set issues about whether the agent has support.

  Our interest is what is required for an agent to \emph{claim} support for (premises and) steps of reasoning, rather than what is required for an agent to \emph{have} support for (premises and) steps of reasoning.

  Use support as opposed to justification.
  Initial focus is on epistemic/doxastic attitudes.
  However, practical reasoning.
  For example, means-end.
  Support considered quite general to also include this.
\end{note}

\begin{note}
  Highlight again \phantref{dogmatism-wrt-nI}{below}.
\end{note}

\section{Claimed support and `use'}
\label{sec:claimed-support-use}

\begin{note}[Understanding of claiming support]
  Understanding of claiming support.

  Begin with a sufficient condition.
  In short, most instances of reasoning.
  Claiming support is common.

  Then, two types of defeaters.
  Mistaken and misled.
  Use to form a necessary condition.
  If claimed support, then agent deems that claimed support is not defeated.
  `Deem' is a placeholder.
  Strong or weak.
  Single constraint is that when claiming support, potential defeaters that aren't ruled out.\nolinebreak
  \footnote{
    At least two ways of viewing this.
    First, claiming support is restricted.
    Second, \emph{claiming} support only applies when there are potential defeaters, and some other relation to support when possible defeaters get ruled out.

    These are different, but I don't think the difference matter for resource bound agents of interest.
    Lack of resources is that always potential defeater, even if every possible defeater may be ruled out.
    }
  Finally, property, that claimed support does not depend on whether proposition the agent has claimed support for is true, or whether the claimed support \emph{does} support (if these are separated).
  Property will be important.
\end{note}

\begin{note}[Sufficient condition]
  We start with a sufficient condition for claiming support:
  \begin{restatable}[\USE{-} --- \USE{}]{idea}{ideaUSE}\label{prem:bP}\label{prop:USE}
    If an agent may claim support for premises and steps of reasoning, accesses those premises and traces claim to support through those steps of reasoning, then agent may claim support for conclusion on basis of the claimed support for the steps and premises of reasoning.
    (Given that the agent deems that the claims to support for premises and steps used are undefeated when drawing conclusion.)
  \end{restatable}

  \USE{} expresses an intuitive idea.

    In short, an agent may claim support when reasoning goes well.
  And, reasoning goes well when there are premises and steps of reasoning available to the agent, and the agent draws on these to claim support for the conclusion.
  The parenthetical remark is a simple safeguard, the agent does not lose a claim to support for the premises or steps in the process of reasoning.\nolinebreak
  \footnote{
    May think that this is the wrong safeguard.
    Consider the liar paradox:
    `This sentence is false.'
    \USE{} prevents agent from claiming support that the sentence is true or false.
    However, may think that agent is in a position to claim support that the sentence is both true and false.
    Indeed, standard reasoning associated with the liar suggests that the sentence is both true and false.
    Still, it's not obvious from demonstrating that the sentence is both true and false that one may claim support for the sentence being true and the sentence being false.
    That is, one may confine the paradox to the truth value of the sentence, rather than (associated) surplus of support.
  }

  The purpose of taking~\USE{} as basic is to fix a basic intuition regarding claiming support.

  Strictly speaking, we do not assume \USE{}.
  Main argument of the thesis is about what is sufficient for claiming support.
  {
    \color{red}
    \USE{} doesn't help me at any point.
    It almost helps, in the sense of that it provides a guarantee that agent would claim support by witnessing.
    However, it doesn't really help, as it's really not that clear.
    \USE{} talks about reasoning done, but you need additional assumptions that this remains true under a counterfactual.
  }
\end{note}

\begin{note}[Illustration of \USE{}]
  An illustration:

  \begin{illustration}\label{ill:rectangle:basic}
    Suppose an agent measures that the rectangle in front of them has the dimensions of \(19\text{cm}\) by \(7\text{cm}\).
    The agent understands how to calculate the area of a rectangle, and by performing some reasoning comes to hold that the area of the rectangle is \(133\text{cm}^{2}\).
    The support the agent has for holding that the area of the rectangle is \(133\text{cm}^{2}\) is obtained (at least in part) on the measurement of rectangle, understanding how to calculate the area of rectangle, and some grasp of mathematics.
  \end{illustration}

  Whether some (or all) of the required arithmetic is to be included as a premise or a step of reasoning may be set aside.
  Similarly, set aside whether further arguments, or whether some premises and steps are taken as basic.
  For example, perhaps some the agent requires some further claim to support for using the ruler to measure the rectangle such as comparison to a standard, or perhaps the agent's claim to support terminates by noting that their use of the ruler is a reliable process.
\end{note}

\begin{note}
  \color{red}
  Problem is, \USE{} does not tell us much about what claimed support is.
\end{note}

\section{Two (conflicting) ideas regarding claimed support}
\label{sec:inter-with-claim}

\begin{note}
  In this section we introduce two propositions which characterise what we are arguing against and what we are arguing for.
  \ESU{-} and \EAS{-}, respectively.
  Argue against \ESU{} by cases involving ability.
  Argue for \EAS{} which outlines the way in which ability conflicts with \ESU{}.

  Start with introduction of \ESU{}.
  And, motivate with reference to literature on the basing relation and rationality as responding to reasons.

  Move to \EAS{}, clarify relation to \ESU{} and contrast to related principle argued for by \citeauthor{Moretti:2019wx}.
\end{note}

\subsection{\ESU{} --- \ESU{-}}
\label{sec:esu}

\begin{note}[Recap of \USE{}]
  Brief recap of \USE{}.
  Introduced the idea, and then expanded on the details.
\end{note}

\begin{note}[Focus]
  We will argue against the converse of~\USE{}:

  \targetESU*

  \ESU{}, as the converse of~\USE{} focuses on reasoning.
  {
    \color{red}
    Assumption here is that there's no other way to claim support.
  }
  To clarify,~\gESU{} is a generalisation of~\ESU{}.


  \begin{restatable}[\gESU{}]{target}{targeGESU}\label{denied-claim}
    An agent may claim support for some proposition \(\phi\) by appealing to some materia\nolinebreak
    \footnote{Latin.
      Material, matter, basis, information, foundation, ground, etc.
    }
    \emph{M} only if the agent uses \emph{M} in the reasoning which culminates in claiming support for \(\phi\).
  \end{restatable}
  Our focus is with whether an agent is required to have \emph{used} something in order to appeal to that thing when claiming support.
  No fixed understanding of `use' is assumed in the statement of~\ESU{} and~\gESU{}, and we will offer some disambiguation below.
  First, a basic illustration.
\end{note}

\begin{note}
  Three brief notes on~\ESU{}:

      First, the `has' in~\ESU{} only requires `at some point in the past'.
      Hence,~\ESU{} does not require the agent to reason from premises to conclusion each time the agent claims support for the conclusion.
      For example, if an agent proved the Deduction Theorem for propositional logic last week, then the agent would not be in conflict with~\ESU{} if they claimed support for the Deduction Theorem on the basis of the premises and reasoning they performed in the past.

      Second, and following from the first,~\ESU{} will also hold for any stronger statement --- for example if `has' is read as `has just'.
      For example, requiring that the agent's memory of proving the Deduction Theorem allows the agent to claim support, rather than the premises and steps used in the past.
      The argument (stated below) denies that, given certain information, the agent needs witnesses any reasoning in order to claim support for the result of witnessing the reasoning.

      Third, as~\ESU{} is about when an agent may \emph{claim} support, it is compatible with~\ESU{} to hold that the agent \emph{has} support --- regardless of whether the agent has witnessing the reasoning.
\end{note}

\begin{note}[Simplest]
  \color{red}
  Difference between \(\phi\) therefore \(\psi\) and \(\phi\) and \(\phi \rightarrow \psi\) therefore \(\psi\).
  Possible for agent to reason from \(\phi\) to \(\psi\), so in principle possible for agent to claim support for \(\psi\) from \(\phi\).
  However, \ESU{} denies this if the agent doesn't do the reasoning.
  Instead, agent also needs \(\phi \rightarrow \psi\), and then they're fine.

  Key point of the suggested revision is that \ESU{} doesn't need to focus on claiming support for \(\phi\), specifically.
  Rather, it's just about establishing a relation of support between \(\phi\) and \(\psi\).

  Then, big idea is that ability is not understood as an instance of \(\phi \rightarrow \psi\), which it might otherwise seem to be.

  Indeed, viewed from the perspective of propositional logic, deduction theorem.
  If \(\vdash \phi \rightarrow \psi\) then \(\phi \vdash \psi\).
  \ESU{} denies that the same holds for claimed support.
  Seems quite sensible.
\end{note}

\begin{note}[Illustration]
  To illustrate~\ESU{}, consider the illustration provided for~\USE{}.

    If the agent did not measure the rectangle, understand how to calculate the area of a rectangle, or perform the required calculations, then the agent would not be in a position to claim support that area of the rectangle is \(133\text{cm}^{2}\).
  A lucky guess that the area of the rectangle is \(133\text{cm}^{2}\) would not allow the agent to claim support that the area of the rectangle is  \(133\text{cm}^{2}\) on the basis of the dimensions of the rectangle, the agent's understanding of how to calculate the area of a rectangle, and the relevant mathematics.
  And, it seems the agent is not in a position to base their lucky guess in such a way because the agent did not reason from the dimensions of the rectangle, the agent's understanding of how to calculate the area of a rectangle, and the relevant mathematics.\nolinebreak
  \footnote{
    Moving to another agent, observe doing the work, get report.
    Easy to resist, by adding in additional premise.
    Still, no presupposing that this needs to be done.
  }
  Similarly, if an agent learns that a rectangle with dimensions of \(19\text{cm}\) by \(7\text{cm}\) may be calculated to have an area of \(133\text{cm}^{2}\), then the agent may not claim support for the area of the rectangle on the basis of the calculation.
  If the agent has not performed the calculation, then the agent may not appeal to the use of the calculation when claiming support --- rather, the agent mentions that the calculation is true.\nolinebreak
  \footnote{
    Slight weakening of~\ESU{} may be made.
    So long as \emph{some} agent has performed the calculation.
    Argue against~\ESU{}, and the argument made will hold for this weakening.
  }
\end{note}

\subsubsection{Intuition}

\begin{note}[Intuition]
  \ESU{} and~\gESU{} seems quite plausible, at least to me.
  The proposition is a careful statement of an intuitive ideas:

  Whether or not an agent claims support is the result of the structure of the reasoning process, and if some premises or step is not used, then it is irrelevant to the structure of the process.
  Hence, the only premises and steps of interest when claiming support are those used in the reasoning process.

  Rests on the broader idea from~\gESU{}.
  Claiming support is the result of some agentive process, and the result of an agentive process is explained by the constituents of the process.\nolinebreak
  \footnote{
    Ah, the homonculus.

    Question about whether the agent is important.

    This gets difficult.

    Consider clocks.
    Clock does not keep track of time.
    Rather, mechanical system designed to change in constant with some passage of time. (Cf.\ \textcite{Smith:1988aa}.)

    Agent may be like this.
    Distinction is intentionality.
    When I go about keeping track of the time, I'm attempting (at least typically) to maintain reference to what the time is.
    Figure out a way to approximate a second, and that's what's happening.
    Approximation.
    If it is noted that I requarly sigh every minute, use this, but I wouldn't be keep tracking of time, though you may be using regularity to do so.
    So, in the former case, using understanding of time, while in the latter not doing so.
  }

  As~\gESU{} is restricted to an agent claiming support, things seem a little easier.
  Problems with interpretation, however.
  Transparency.
  Familiar, if debatable, illustration.
  Freud.
  (Here, adjourning the meeting by saying something mistaken.)
\end{note}

\begin{note}[Analogy]
  By analogy, whether or not my mug of (once cold) coffee overheats in the microwave is the result of some process involving electromagnetic radiation.
  My desire that the mug of coffee does not overheat is not used as part of the process of heating the coffee, and so is irrelevant to the structure of the process.

  My desire may explain why the mug of coffee is taking part in a certain process, and an unused premise or step may explain why an agent performed so reasoning.
  Still, a premise or step must be used as part of the process of reasoning to stand in explanation for the result of reasoning.

  Press the analogy further: Reasoning is a causal process.
  And, any property of reasoning reduces to cause and effect.
  If premises or steps are not used, then those premises or steps stands outside the relevant causal trace, and may not be appealed to when accounting for some structural property of the conclusion of the instance of reasoning (here, that the agent claims support for the conclusion).
\end{note}

\subsubsection{\ESU{} in the literature}

\begin{note}
  \color{red}
  Given proposed revision to \ESU{} this section should be expanded a little.
  For, most of the cases talk about claiming support for \(\phi\) directly, while \ESU{} is more general in that it talks about claiming support for any entailment between \(\phi\) and \(\psi\).
\end{note}

\begin{note}[Causal theories of basing]
  Indeed,~\ESU{} seems to be implied by various causal theories of basing.

  \citeauthor{Pollock:1999tm} introduce the basing relation with the following observation:
  \begin{quote}
    To be justified in believing something it is not sufficient merely to \emph{have} a good reason for believing it.
    One could have a good reason at one's disposal but never make the connection.
    \dots
    Surely, you are not justified in believing [something], despite the fact that you have impeccable reasons for it at your disposal.
    What is lacking is that you do not believe the conclusion on the basis of those reasons.\nolinebreak
    \mbox{}\hfill\mbox{(\cite[35]{Pollock:1999tm})}
  \end{quote}
  The observation falls short of being an account of the basing relation, but the intuition \citeauthor{Pollock:1999tm} appeal to is instructive.
  It seems that an agent must connect reasons and the content of a belief in order for the belief to be formed on the basis of those reasons, and hence be justified by those reasons.
  In turn, if a connection is made between reasons and the content of belief, then those reasons are used by the agent.

  For a concrete instance, consider \citeauthor{Moser:1989tv}'s account of the basing relation:
  \begin{quote}
    \emph{S}'s believing or assenting to \emph{P} is based on his justifying propositional reason \emph{Q} \(=_{\text{df}}\) \emph{S}'s believing or assenting to \emph{P} is causally sustained in a nondeviant manner by his believing or assenting to \emph{Q}, and by his associating \emph{P} and \emph{Q}.\nolinebreak
    \mbox{}\hfill\mbox{(\cite*[157]{Moser:1989tv})}
  \end{quote}

  Suppose we have a conclusion from some premises and steps of reasoning.
  If the agent has not witnessed the relevant reasoning, then it seems the conclusion is not causally sustained in a nondeviant manner by his believing or assenting to the premises of the reasoning, nor has the agent associated the conclusion with the premises by witnessing the relevant steps of reasoning.

  To illustrate, claim support that 173 is prime.
  It's possible that I did the prime factorisation, and possible that I took that representation to be part of the reason why I claim that 173 is prime.
  However, represented query of whether prime to wolfram alpha as justifying, and that's why I claimed support.
  So, definitely not from okay to appeal to the reasoning I have not witnessed.
  And, if infer that 173 is prime from claimed support that I have the ability to demonstrate that 173 is prime, the same issue.
  As I've not witnessed, then no role for \emph{Q}, whatever that turns out to be.

  This is a quick argument, and borders on conjecture, so let us examine the relevant association in greater detail.
  \citeauthor{Moser:1989tv} distinguishes between occurrent and non-occurrent satisfaction of association relations.

  We start with occurrent satisfaction of an association relation:
  \begin{quote}
    \emph{S} occurrently satisfies an association relation between \emph{E} and \emph{P} \(=_{\text{df}}\)
    \begin{enumerate*}[label=(\roman*), ref=(\roman*)]
    \item\label{moser:oar:i} S has a \emph{de re} awareness of \emph{E}'s supporting \emph{P}, and
    \item\label{moser:oar:ii} as a nondeviant result of this awareness, \emph{S} is in a dispositional state whereby if he were to focus his attention only on his evidence for \emph{P} (while all else remained the same), he would focus his attention on \emph{E}.\newline
    \mbox{}\hfill\mbox{(\Citeyear[141--142]{Moser:1989tv})}
    \end{enumerate*}
  \end{quote}

  \emph{de re} awareness is a non-propositional, direct awareness of \emph{E} supporting \emph{P}.

  \ESU{} follows from~\ref{moser:oar:i}.
  \emph{de re} awareness, but this doesn't rule out use.
  \ESU{} does not require that the agent engages in propositional reasoning.

  In cases where the agent has not witnessed reasoning, there is no \emph{de re} awareness.
  Without the reasoning taking place, the agent is not directly aware of what the reasoning consists of.

  Following, the definition of non-occurrent satisfaction of an association relations is derived from occurrent satisfaction of an association relations by allowing~\ref{moser:oar:i} to be satisfied at some point in the past while requiring that~\ref{moser:oar:ii} continues to be satisfied in the present.
  As noted, \ESU{} is compatible with the agent having witnessed the reasoning at some point in the past.
  Therefore, \ESU{} is entailed given both occurrent and non-occurrent satisfaction of association relations
\end{note}

{
  \color{red}
  This doesn't make sense\dots
  I think the idea I had was that the agent has to use the represented relation.
  Hm, so, the idea is that in the case of \(\phi \vdash \psi\), the agent hasn't represented how to get from \(\phi\) to \(\psi\), and therefore the agent isn't allowed to base \(C\) or \(R\) given \citeauthor{Neta:2019aa}'s account.
  I don't think this is sufficiently clear from what I have written.
  However, it does seem relevant.
  And, in also, basing doesn't necessarily need to be between beliefs.
  This could just be a relation of justification\dots though this isn't necessarily the case.
  So care is need.
  Still, with a little rewriting this looks useful.

  The tricky part is understanding what it is to represent R as justifying C.
  What I need is the idea expressed above, that the relevant representation is sufficiently detailed.
  I think this should be in \citeauthor{Neta:2019aa}.
  For, intuitively representing R as \emph{justifying} C is stronger than a representation with the content that R justifies C.
}

\begin{note}[Representationalism]
  \citeauthor{Neta:2019aa} generalises (purely) epistemic interest in basing relations to cover the explanatory relation between reasons and (rationally evaluable) states held, or actions performed, by an agent.

  On the way to a novel proposal, \citeauthor{Neta:2019aa} sketches a broad characterisation of representationalist theories of (generalised) basing:
  \begin{quote}
    \begin{enumerate}[label=(R\arabic*), ref=(R\arabic*)]
    \item\label{neta:RC:b} \emph{basing} C on R involves the agent's representing R as justifying C, and
    \item\label{neta:RC:jb} \emph{justifying basing} of C on R consists in the adroitness of this representation.\nolinebreak
          \mbox{}\hfill\mbox{(\Citeyear[192]{Neta:2019aa})}
    \end{enumerate}
  \end{quote}
  As \ESU{} does not distinguish between successful and unscuccesful instances of claiming support, our interest is with~\ref{neta:RC:b}.
  And, in contrast to \citeauthor{Moser:1989tv}, a representationalist theory may lack a causal component.
  Indeed, \citeauthor{Neta:2019aa} considers scenario in which an agent receives information from some source, forms a belief which is supported by the received information, and represents the received information as justifying the belief.
  The twist, however, is that the agent forming the belief was caused by some other source.
  For example, an agent may listen to a speech given by a talented orator and form a belief in response to the speech.
  The agent may represent the content of the speech as justifying the conclusion, while the cause of the belief being formed is the emotional impact with which the orator stated the conclusion.
  Following the representationalist characterisation, the agent would base the content of the belief on the content of the speech rather than the emotional impact with which the speech concluded.
  Indeed, the agent may do so even if they recognise that they were swayed by emotion.

  As before, consider a conclusion of some reasoning that the agent has not witnessed.
  If the agent has not witnessed the reasoning, then the agent has not represented some or all of the relevant premises and steps of reasoning.
  Therefore, it seems that it is not possible for the agent to represent the premises and steps of reasoning as justifying the relevant conclusion.
  In other words, a representationalist account requires (minimally) that an agent represents premises and steps of reasoning as justifying when claiming support for some conclusion of reasoning, and hence use of those premises and steps.

  {
    \color{red}
    What is going on here\dots
    The point is that if we follow \citeauthor{Neta:2019aa} then there needs to be a representation.
    In turn, the issue is that it's not clear that the agent needs to reason from \(\psi\) to \(\phi\) in order to obtain the relevant representation.
    So, it's not clear that \citeauthor{Neta:2019aa} actually is relevant.

    So, it's this previous paragraph that needs attention.
    No use, then no representation.
    This is the only point that really matters.
    So, I need to find something in \citeauthor{Neta:2019aa} that supports this, or somehow provide a much better argument.

    Then, in the following paragraph is redundant.
    The issue is with how the relevant representation is obtained.
    The part where I'm getting confused is that \citeauthor{Neta:2019aa} doesn't hold that the agent needs to do the reasoning each time the representation is used.
  }

  As an aside, it is not clear whether representing an entailment or inference is the same as reasoning with an entailment, and therefore it does not seem to follow from the representationalist characterisation that the agent must witness the relevant reasoning.
  However, the interpretation of `use' is intended to be sufficiently broad to cover such cases.\nolinebreak
  \footnote{
    Alternatively, a clause may be added to~\ESU{} which denies that the agent represents the relevant premises and steps of reasoning.
    The argument made against~\ESU{} is compatible with the use of representations, or mere representation even if unused --- though it is unclear to me what an unused but represented premise or step would matter when claiming support.
  }

  Further, \citeauthor{Neta:2019aa}'s discussion is instructive because the response \citeauthor{Neta:2019aa} offers to some problematic scenarios focus on \emph{how} a representation is used.
  One may hold that the agent in the example given did not base their belief in the conclusion on the content of the speech in view of the fact that the agent was swayed by emotion.
  If so, \citeauthor{Neta:2019aa} proposes the following revision:
  \begin{quote}
    \begin{enumerate}[label=(R\arabic*\('\)), ref=(R\arabic*\('\))]
    \item\label{neta:RC:jp} for an agent to C based on reason R involves not merely the agent's representing R as justifying C---it also involves \emph{this latter representation (or its content) being part of the reason why the agent C's}.\nolinebreak
      \mbox{}\hfill\mbox{(\Citeyear[197]{Neta:2019aa})}
    \end{enumerate}
  \end{quote}
  The added clause states that the relevant representation must explain why the agent formed a belief.
  Hence, given~\ref{neta:RC:jp} the agent would not be permitted to base their belief in the content of the speech given that they were swayed by emotion.
  Intuitively,~\ref{neta:RC:jp} expands on what it is for premises and steps of reasoning to be use when forming a belief.
  So, given that representation requires use, the expanded clause may be seen as focusing on \emph{how} the representation is used.
\end{note}

\begin{note}[Responding to reasons]
  As final motivation, consider the proposal at the core of \citeauthor{Lord:2018aa}'s (\Citeyear{Lord:2018aa}) thesis that being rational is to correctly respond to reasons.

  \begin{quote}
    \textbf{Correctly Responding:} What it is for A's \(\phi\)-ing to be ex post rational is for A to possess sufficient reason S to \(\phi\) and for A's \(\phi\)-ing to be a manifestation of knowledge about how to use S as sufficient reason to \(\phi\).\nolinebreak
    \mbox{}\hfill\mbox{(\Citeyear[143]{Lord:2018aa})}
  \end{quote}

  An agent's action is rational only if the action is a manifestation of some know-how.
  \citeauthor{Lord:2018aa} summaries:

  \begin{quote}
    \dots when one manifests one's know-how, dispositions that are directly sensitive to normative facts are manifesting. Thus, the competences involved in the relevant know-how make one directly sensitive to the normative facts\nolinebreak
    \mbox{}\hfill\mbox{(\Citeyear[16]{Lord:2018aa})}
  \end{quote}

  For our purposes, following example of manifesting know-how directly relates to reasoning:

  \begin{quote}
    The most salient disposition [when appealing to \emph{p} as a reason]\nolinebreak
    \footnote{Note, \citeauthor{Lord:2018aa} (explicitly) not talking about believing that \emph{p} is a reason, but argues that the cited disposition to present both when appealing to p as a reason and believing that \emph{p} is a reason.}
    is the disposition to (competently) use \emph{p} as a premise in reasoning.\nolinebreak
    \mbox{}\hfill\mbox{(\Citeyear[25]{Lord:2018aa})}
  \end{quote}

  Hence, suppose an agent appeals to a premise of reasoning in order to claim support for some conclusion.
  Then, if the agent does not use the premise of reasoning, it seems the agent does not manifest know-how, which is required for the appeal to meet \citeauthor{Lord:2018aa}'s account of rational action.

  Of course, that the noted disposition is the most salient does not rule out alternative, less noteworthy, dispositions.
  However, it is unclear to me how to \emph{manifest} know-how without use.
  Looking ahead, it does not seem to be the case that I manifest my ability to show that a certain rule of inference is sound when skipping over details in a completeness proof.
  However, I may manifest know-how regarding the (presumed) truth of the ability attribution.

  Likewise with my ability to establish a preference for tofu over any other kind of miso when ordering soup.
\end{note}

\begin{note}[Summarising illustrations]
  Three examples of claiming or establishing relations of support have been given.
  Each example suggests that if an agent does not use a premises or steps when claiming support, then an agent may not claim support by appeal to the unused premises or steps.

  Stepping back,~\ESU{} may be seen as a desiderata for any account of (successfully) claiming support.
  For:
  If an agent (successfully) claims support for some conclusion of reasoning, then the premises and steps used with respect to that claim of support is itself the result of some reasoning --- the reasoning that culminated with the claim to support itself used premises and steps of reasoning.
  So, given that the agent used certain premises and steps when claiming support for conclusion, some property of the premises and steps used, an adequate account of claiming support must explain how the premises and steps used permit the agent to claim support.\nolinebreak
  \footnote{
    Note, however, that this argument does not imply that support for the conclusion must be accounted for in terms of the premises and steps used by the agent to claim support.
    As we will note below, one may hold that an enthymematic argument permits an agent to claim support, while the relevant relation of support is secured by the corresponding non-enthymematic argument.
    Cf.\ \textcite{Moretti:2019wx} for suggestions along these lines.
  }
  In turn, if an agent appeals to premises and steps that they did not use, then those premises and steps must be redundant.

  Turning to ability.
  Suppose and agent appeals to
  \begin{enumerate*}
  \item their ability to demonstrate that \(\phi\) is the case, and
  \item that \(\phi\) must be the case in order for the agent to have the ability to demonstrate that \(\phi\)
  \end{enumerate*}
  in order to claim support for \(\phi\).
  Then, the premises and steps involved in a full account of reasoning from the two claims must be sufficient to claim support that \(\phi\) is the case.
  So, as the agent does not witness their ability to demonstrate that \(\phi\) in such reasoning, it must be the case that claimed support for (the property of) having the ability to demonstrate that \(\phi\) is sufficient for such reasoning.
\end{note}

\subsection{\EAS{} --- \EAS{-}}
\label{sec:eas}

{
  \color{red}
  Perhaps include a note about how the argument relates to \EAS{}.
  I don't provide a direct argument, but this is the best way I see of resolving the tension.
}

\begin{note}[Alternative]
  \ESU{} is a universal claim, and so applies to all instances in which an agent may claim support for conclusion on basis of support for premises and steps of reasoning --- an agent may only claim support if the agent reasoned from the premises via the steps to the conclusion.

  Our goal is to motivate the following exception to \gESU{}, and hence \ESU{}:

  \goalEAS*
\end{note}

\begin{note}[Intuition for \EAS{}]
  \EAS{} is a conditional.
  Antecedent is claimed support for ability.
  Consequent is that it may be permissible to violate \gESU{}.
\end{note}

\begin{note}
  Now, started with \USE{}, and then looked at \ESU{}, the converse.
  Both of these we have a particular instance of reasoning in mind.
  Now, \EAS{} may, intuitively, be understood to states that whatever that reasoning is, if an agent has claimed support that they're able to witness such reasoning, then the agent may claim support.

  However, things are a little more complex.
  \EAS{} is about the ability to claim support to reason to some conclusion.
  However, \EAS{} does not state that the agent may claim support for the conclusion on the basis of the premises that they would reason from were they to witness the ability.

  Issue here is that the substance of \EAS{} --- what the relevant materia amounts to --- depends on two things:
  \begin{itemize}
  \item How (appeal to) ability is understood, and
  \item The kind of reasoning involved in the appeal to ability.
  \end{itemize}

  We will outline the basics, then reformulate \EAS{} using one what in which (appeal to) ability is understood.

  Start, how ability is understood.
  Lead naturally to the kind of reasoning involved.

  The argument for \EAS{} will not depend on how ability is understood, but the kind of reasoning involved.
  Still, kind of reasoning involved when combined with how ability is understood.
\end{note}

\begin{note}
  Briefly stated,
  \AR{} understands ability in terms of some (complex) property.
  \WR{} understands ability in terms of possible witnessing events.

  For example, \AR{} may involve the property (attribution) of understanding geometry, perhaps broken down into the understanding or availability of various definitions, propositions, lemmas, theorems, and steps of reasoning.
  While, \WR{} would involve reasoning with particular definitions, propositions, lemmas, theorems, and steps of reasoning.

  So, agent appeals to property, or the reasoning itself.

  The purpose of this distinction is to ensure that our argument against \gESU{} does not rest on a particular way of understanding ability that may not extend to other ways of understanding ability.

  Conjecture that these are fundamentally connected.
  Witnessing event only if understanding.
  And, understanding only if possible to witness reasoning.

  Still, difference.
  Relevant properties are properties of the agent as they are.
  The witnessing event, by contrast, is a possible event.\nolinebreak
  \footnote{
    Property of there being a possible event involving the agent.
    In this case, still distinct from \WR{} as that the agent is part of possible event is still distinct from the reasoning that the agent would witness in the relevant event.
  }

  These are brief characterisations, but enough for now.
  Both~\AR{}~and~\WR{} will be considered at length in~\autoref{sec:ar-wr-1}.
  In addition to a more thorough treatment of the core ideas, \autoref{sec:ar-wr-1} includes additional examples, and an argument that~\AR{}~and~\WR{} are exhaustive --- any way of understanding ability will conform to either~\AR{}~and~\WR{}.
\end{note}

\begin{note}
  Now turn to the kind of reasoning involved.

  Motivated \AR{} in terms of understanding of premises and steps of reasoning, and \WR{} in terms of a possible event in which agent reasons with particular premises and steps.

  However, a further distinction in terms of what appeal to the relevant premises and steps or instance of reasoning amounts to.

  First, there is the \emph{existence} of premises and steps, or the \emph{possibility} of the witnessing event.
  Second, there is the premises and steps themselves, or the witnessing event.

  Difference from perspective of step of reasoning.
\end{note}


\begin{note}[Types of reasoning]
  Consider proofs.

  \(p \lor q\)
  \(\lnot q\)
  \(p\)

  Premises alone do not establish \(p\).
  Combined they do.

  Claim support individually, then \(p\).
  Alone, these don't require \(p\).
  More in \autoref{sec:ability-ads-adc}.
\end{note}

\begin{note}[\EASw{}]
    \begin{restatable}[\EASw{-} --- \EASw{}]{thought}{thoughtEASw}\label{thought:EASw}
    If an agent has claimed support that they have the ability to (adequately) reason to some conclusion, then it may be permissible for the agent to claim support for the conclusion by claiming support for the premises and steps of reasoning that the agent would use to witness their ability to reason to the conclusion.
  \end{restatable}

  Loosely restated,~\EASw{} holds that if an agent may claim support for having the ability to witness some reasoning, and is aware of the conclusion of that reasoning, then the way in which the agent claims support for the conclusion of that reasoning may mirror the way in which the agent would claim support for the conclusion by witnessing the reasoning (and hence using the relevant premises and steps).
\end{note}

\begin{note}[Just an idea]
  \emph{Idea} as this is preferred way of thinking about ability.
  However, argument will not depend on this way of thinking.
\end{note}

\begin{note}
  The (possible) event of the agent witnessing their ability to demonstrate \(\phi\) involves reasoning with various premises and steps which culminate in claiming support for \(\phi\).
  So, if~\EASw{} is true, then the agent may appeal to those premises and steps which are used in the (possible) witnessing event.

  One way to think about~\EASw{} (which we will explore in more details later) is in terms of propositional support.
  For, if an agent has the ability to demonstrate that \(\phi\) is the case, then the agent has propositional support for \(\phi\) as there is a way for the agent to demonstrate that \(\phi\) is the case.
  In addition, that the agent has the ability to demonstrate that \(\phi\) is the case ensure that the agent is in a position to make use of the available propositional support for \(\phi\).
  In turn,~\EAS{} may be interpreted to hold that so long as the agent has such information about their position to make use of the available propositional support for \(\phi\) then the agent does not need to reason with the relevant propositional support in order to claim support for \(\phi\) in virtue of the available propositional support for \(\phi\).
\end{note}

\begin{note}[Conditional]
  Here, note that it's a conditional, but also that it only states there are instances.
  It doesn't follow that ability will always allow the agent to claim support.

  The conditional is weak primarily because it is not at all clear that it holds in general.
  There are various cases in which it seems appeal to ability is blocked.

  Easiest cases involve claiming support in some public setting.
  Of course, success in a public setting is not necessarily required for private success.
  Same problem with testimony.
  I'm confident in a source and you're not.
  I fail to convince you, but I remain convinced myself.

  Still, seems as though similar considerations extend.
  For example, doing a PSET where I'm allowed to use theorems I've already proved.
  Have notes of what those theorems are.
  And, ability to prove them.
  Still, might refrains from using them until I've proven them once again.

  More could be said here, and it may be possible to argue for a stronger variant of \EAS{}.

  Even though it's weak, the condition is still interesting.
\end{note}


\begin{note}
  So, if~\EAS{} is true, then there are cases in which an agent is not required to reason from premises they may claim support for to some conclusion in order to obtain support for the conclusion on the basis the support the agent has for the premises.\nolinebreak
  \footnote{
    Stated~\EAS{} as an exception to~\ESU{}.
    And, we will argue that~\EAS{} is true.
    However, we will not argue that~\EAS{} \emph{is an exception} to~\ESU{}.
    To do so would require an argument that \ESU{} holds for other cases.
    Likewise, no argument that~\EAS{} is the only exception, as to do so would require argument that~\ESU{} holds for all other cases.
    Take~\ESU{} to be plausible, and suspect that there are few, if any, further exceptions, but~\EAS{} may stand independently on any further statements about claiming support.
  }
\end{note}

\begin{note}
  \color{red}
  I want to clarify \EAS{} a little.
  The use of `may' is problematic.
  It could be read as `it's always okay, but it's up to the agent'.
  Or, `it's possible, given appropriate context'.
  The latter is what I want, and is important for cases where doubts are plausibly raised about the ability.
\end{note}

\begin{note}[\EAS{} illustration]
  To illustrate \EAS{}

  \begin{illustration}\label{ill:rectangle:ability}
    Suppose you provide me with novel information that:
    \begin{enumerate}[label=\emph{A}\arabic*., ref=(\emph{A}\arabic*), series=EAS_counter]
    \item\label{EAS:ex:box:if} If I have ability to calculate the area of a box, then I have the ability to demonstrate that a rectangle with dimensions \(19\text{cm}\) by \(7\text{cm}\) has area \(133\text{cm}^{2}\).
    \end{enumerate}
    The information is `novel' because I have not been previously informed (in any way) about the area of a rectangle with dimensions \(19\text{cm}\) by \(7\text{cm}\).

    Still, (I claim support that):
    \begin{enumerate}[label=\emph{A}\arabic*., ref=(\emph{A}\arabic*), resume*=EAS_counter]
    \item\label{EAS:ex:box:gen} I have the ability to calculate the area of a rectangle.
    \end{enumerate}
    Therefore:
    \begin{enumerate}[label=\emph{A}\arabic*., ref=(\emph{A}\arabic*), resume*=EAS_counter]
    \item\label{EAS:ex:box:spec} I have the ability to demonstrate that a rectangle with dimensions \(19\text{cm}\) by \(7\text{cm}\) has area \(133\text{cm}^{2}\).
    \end{enumerate}
    From~\ref{EAS:ex:box:spec} it follows that:
    \begin{enumerate}[label=\emph{A}\arabic*., ref=(\emph{A}\arabic*), resume*=EAS_counter]
    \item\label{EAS:ex:box:fact} A rectangle with dimensions \(19\text{cm}\) by \(7\text{cm}\) has area \(133\text{cm}^{2}\).
    \end{enumerate}
  \end{illustration}
  \EAS{} holds that, when I claim support for~\ref{EAS:ex:box:fact} from~\ref{EAS:ex:box:spec}, I may appeal to dimensions and formula, though as I do not witness the ability, I do not use the premise and step.
  For, if~\ref{EAS:ex:box:spec} is the case then it is possible for me to witness reasoning in which I demonstrate that~\ref{EAS:ex:box:fact} is the case, and it is the premises and steps of reasoning used in such reasoning that establishes~\ref{EAS:ex:box:fact} is the case.
  I have not used those steps and premises, as I have not witnessed the relevant ability, but may I appeal to those steps and premises regardless --- or so we will argue.
\end{note}

\begin{note}[More detail]
  \color{details}
  I do not expect \EAS{} to be intuitive.
  Indeed, we are not interested in \EAS{} because it is a more-or-less intuitive principle which conflicts the intuitive \ESU{}.
  Rather, we are interested in \EAS{} primarily because \EAS{} is a consequence of tension arising from three things:

  \begin{enumerate}
  \item\label{incomp:tri:q:1} \ESU{}
  \item\label{incomp:tri:q:2} scenarios involving an agent reasoning with information about an their own ability,
  \item\label{incomp:tri:q:3} and a principle concerning when an agent is permitted to claim support
  \end{enumerate}

  To briefly expand on~\ref{incomp:tri:q:2} and~\ref{incomp:tri:q:3}:

  Information that one has some specific ability so long as one has some general ability --- such as the (specific) ability to show that \(25^{\circ}\text{C} = 77^{\circ}\text{F}\) given the (general) ability to convert between Celsius and Fahrenheit.
  And, an agent is never permitted to claim support for proposition having a certain value if the agent requires the proposition to have value \emph{in order to} claim support.
  (As an instance, an agent is not permitted to claim support for the truth of a proposition if the agent requires the proposition to be true \emph{in order to} claim support that the proposition is true.)\nolinebreak
  \footnote{
    The emphasis on `in order to' is important.
    The instance of the principle does not state that an agent is not permitted to claim support for the truth of a proposition if the agent requires the proposition to be true when claiming support that the proposition is true.
    I plausibly require that \(2 + 2 = 4\) when I claim support that \(2 + 2 = 4\), and this does not prevent me from claiming support by simple arithmetic.
    However, it would be impermissible (or so we will argue) to claim support that \(2 + 2 = 4\) by reasoning that the calculator is functional only if \(2 + 2 = 4\), and as the calculator states that \(2 + 2 = 4\) it is the case that \(2 + 2 = 4\).
  }
  The details matter, and we postpone detailing this argument to~\autoref{sec:broad-argum-overv}.

  In short, assuming the scenarios exist, there is tension between intuitive principles governing what an agent appeals to when reasoning and structural principles governing the relation between what the agent appeals to when reasoning.
\end{note}

\begin{note}
  For the moment we attempt to clarify \EAS{} to some degree.
  Three subsections follow:

  \begin{enumerate}
  \item We will outline alternative reasoning patterns from~\ref{EAS:ex:box:if} to~\ref{EAS:ex:box:fact}, clarify why we focus on a particular type of reasoning pattern, and examine some initial objections to~\EAS{} and canvas some responses.
  \item We will consider parallels between abilities and dispositions.
    The parallel will provide some additional intuition for why an agent may appeal to premises and steps that have no been used, and help further clarify our interest with ability.
  \item We will consider a related proposition argued for by \citeauthor{Moretti:2019wx} which holds that a belief need not be based (exclusively) on the premises and steps of reasoning used to arrive at the belief.
    The comparison will help highlight what is distinctive about~\EAS{} while at the same to introducing some ideas which suggest a way of understanding~\EAS{}.
  \end{enumerate}
\end{note}

\subsubsection{Against \EAS{}}

\begin{note}[Alternatives]
  The alternative reasoning pattern we will focus on in some detail holds that appealing to having the ability noted in \ref{EAS:ex:box:spec} is sufficient to claim support for \ref{EAS:ex:box:fact}.
  In line with \ESU{}, the agent would use the proposition that they have the relevant ability noted in~\ref{EAS:ex:box:spec} to claim support for~\ref{EAS:ex:box:fact}
  This reasoning pattern, along with the pattern suggested by \EAS{} will be considered in \autoref{sec:wr-ar} and we will argue that it conflicts with an intuitive principle regarding claiming support in \ref{sec:second-conditional}.

  Alternatively, on may argue that though the syntactic form of \ref{EAS:ex:box:if} is a conditional, it does not (necessarily) follow that the semantic content of~\ref{EAS:ex:box:if} is a (also) conditional.
  And that~\ref{EAS:ex:box:if} may (plausibly) be interpreted to explicitly state that~\ref{EAS:ex:box:spec} is an ability that an agent may have.
  For example:
  \begin{enumerate}[label=\emph{A}\arabic*., ref=(\emph{A}\arabic*), resume*=EAS_counter]
  \item\label{EAS:ex:box:if:R:state} The ability to demonstrate that a rectangle with dimensions \(19\text{cm}\) by \(7\text{cm}\) has area \(133\text{cm}^{2}\) is an ability an agent may have and it is an ability an agent has if they have ability to calculate the area of a rectangle.
  \end{enumerate}
  Hence, \ref{EAS:ex:box:if} is interpreted so that \ref{EAS:ex:box:spec} is accessible without endorsing the antecedent of \ref{EAS:ex:box:if}.
  \ref{EAS:ex:box:if:R:state} states that there is some ability that it is possible for an agent to have, and in addition provides sufficient conditions for having the relevant ability.
  The important part of \ref{EAS:ex:box:if:R:state} is the former conjunct:
  \begin{enumerate}[label=\emph{A}\arabic*., ref=(\emph{A}\arabic*), resume*=EAS_counter]
  \item\label{EAS:ex:box:spec:R:state} The ability to demonstrate that a rectangle with dimensions \(19\text{cm}\) by \(7\text{cm}\) has area \(133\text{cm}^{2}\) is an ability an agent may have.
  \end{enumerate}
  And, \ref{EAS:ex:box:fact} follows from \ref{EAS:ex:box:spec:R:state} by the observation that it is not possible to demonstrate that \(\phi\) if \(\phi\) is not the case --- there is no need for the agent to appeal to witnessing their ability.
  Therefore,~\ref{EAS:ex:box:spec:R:state} (and~\ref{EAS:ex:box:if:R:state}) implicitly includes information that~\ref{EAS:ex:box:fact} is the case --- an agent does not need to reason from~\ref{EAS:ex:box:spec} to~\ref{EAS:ex:box:fact}, because they have already been informed that~\ref{EAS:ex:box:fact} is the case.

  Note also that analogous reasoning applies if `I' is replaced with `an agent'.
  Likewise, if I know that you that you know that I have the ability to calculate the area of a rectangle.
  For, it then follows that you know that \ref{EAS:ex:box:spec} is the case, and therefore you know that \ref{EAS:ex:box:fact} is the case.
  Again there is no need for me to appeal to witnessing an ability.
\end{note}

\begin{note}[Box]
  The existence of alternative reasoning patterns is the issue at hand.
  For, so long as there are reasoning patterns \emph{R} which conform to \ESU{} it is open to the defender of \ESU{} to hold that if an agent is permitted to claim support, then the agent is required to reason via some member of \emph{R}.
  For, if there are reasoning patterns \emph{R} which conform to \ESU{} then there a no counterexamples to \ESU{} --- scenarios in which an agent claims support by appeal to premises or steps of reasoning that the agent has not used.

  Of course, an argument against a general principle such as \ESU{} is not required to be a counterexample.
  For example, it may be possible to argue that the reasoning patterns \ESU{} requires are sufficiently implausible.
  Hence, a restricted variant of \ESU{} compatible with \EAS{} would to be preferred.
  However, there are two issues with attempting such an argument.

  First, given the intuitive plausibility to \ESU{}, it seems unlikely that any violation of \ESU{} would be more plausible than an alternative reasoning pattern compatible with \ESU{}.
  Second, even if there are plausible reasoning pattern that are incompatible with \ESU{}, it is not clear that these should be incorporated in a theory of claiming support.
  For, meta-theoretical issues such as complexity or predictive power may still favour \ESU{}.
  Following \citeauthor{Box:1987vn}: `\dots all models are wrong; the practical question is how wrong do they have to be to not be useful.' (\Citeyear[74]{Box:1987vn})

  Indeed, the second point suggest a counterexample proper to show that \ESU{} is false is not necessarily an adequate argument against \ESU{} either.
  Observations in the spirit of \citeauthor{Box:1987vn} are trite, but also true.
  Even if \EAS{} is true and \ESU{} is false, would \EAS{} be useful?
\end{note}

\begin{note}[Responding to Box]
  With respect to idealised agents with unbounded resources, the answer appears to be no.
  For, with unbounded resources the agent the option of (attempting to) witnessing any ability (to reason) without cost.
  And, it seems that for such an agent witnessing a relevant ability would always be preferable to reasoning about an unwitnessed ability as the agent would minimally (subjectively) resolve any uncertainty about whether they have the ability.

  However, for limited agents, ability is abundant, while the resources required to witness abilities are scarce.
  That the exception to~\ESU{} is narrow does not entail that there are few occurrences of the exception.

  Information about ability may be abundant while the resources for witnessing abilities are either scarce or temporarily unavailable.
  So, for example, agent has the option of conserving or deferring use of resources.

  This observation suggests an initial line of response to an objection which focuses on whether \EAS{} would be useful.

  For, given that we are resource bound agents, it seems that possible instances of \EAS{} are widespread.
  From a functional perspective, reasoning with (the relevant instances of) ability just is reasoning about the result of expending available resources.
  Hence, if~\EAS{} is true, then the truth of \EAS{} would provide a novel perspective on resource bound agents.
  And, it is yet to be seen whether such a perspective is useful.

  In addition, there is a second indirect line of response.
  We observed above that \ESU{} seems prevalent in various theories which relate to reasoning, such as basing and responding to reasons.
  If \EAS{} is true, then there may be alternative conclusions to arguments that appeal to~\ESU{} as a premise.
  And, likewise, there may be interesting observations made in premises of arguments which establish \ESU{} as a foundation for further theorising.\nolinebreak
  \footnote{
    As an exception, even if~\EAS{}, conclusion of arguments which appeal to or assume \ESU{} may be restricted.
  }

  Taking stock:
  I doubt that \EAS{} is of interest if there are no reasoning patterns which require \EAS{} to be true.
  Still, if there are reasoning patterns which require \EAS{} to be true, then \EAS{} may be of interest.
  Further, I think there are good reasons to hold that there are reasoning patterns which require \EAS{} to be true.
  Hence, my goal is to motivate further research into whether \EAS{} is of interest.

  We now briefly turn to the type of scenarios which are a premise in our argument for the existence of such counterexamples.
\end{note}

\begin{note}[Types of scenario]
  The type of scenario we will focus on is designed to ensure that an agent is required to reason to (and from) information about a (specific) ability that they have.
  If the agent is required to reason \emph{to} (specific) ability information then rephrasing \ref{EAS:ex:box:if} as \ref{EAS:ex:box:if:R:state} will not be possible --- the agent will be required to reason from some premises by some steps to the (specific) ability information.
  And, as before, the type of scenario will preserve the requirement of the agent to reason \emph{from} the (specific) ability information in line with~\ref{EAS:ex:box:spec} and~\ref{EAS:ex:box:if:R:state}.
  Hence, by establishing such scenarios are possible we may restrict our attention to the steps from~\ref{EAS:ex:box:if} to~\ref{EAS:ex:box:spec} and from~\ref{EAS:ex:box:spec} to~\ref{EAS:ex:box:fact}.
\end{note}

\begin{note}
  To illustrate, let us add some context to the example scenario we've been focusing on.

  Suppose it is common knowledge between you and I that
  \begin{enumerate*}
  \item you have looked through my notes, and have applied my formula for calculating the area of a rectangle, and
  \item my notes are the only source of information you have regarding how to calculate the area of a rectangle.
  \end{enumerate*}
  We may now restate the semantic content of~\ref{EAS:ex:box:if} as follows:
  \begin{enumerate}[label=\emph{A}\arabic*., ref=(\emph{A}\arabic*), resume*=EAS_counter]
  \item\label{EAS:ex:box:inf:R} You have some general ability \(\gamma\), and a specific ability \(\varsigma\) (as an instance of that general ability).
    And, if \(\gamma\) is the ability to calculate the area of a rectangle, then \(\varsigma\) is the ability to demonstrate that a rectangle with dimensions \(19\text{cm}\) by \(7\text{cm}\) has area \(133\text{cm}^{2}\).
  \end{enumerate}
  The formula in my notes indicates that I have the ability to do something, and I have indicated what I think the appropriate characterisation of the ability is.
  Still, you are not in a position to offer information as to whether my characterisation of the ability is correct or not.\nolinebreak
  \footnote{
    Consider in reverse.
    One is often attributed abilities that I deny I have.
    For example, I do not have the ability to process information by means of mental images, but \citeauthor{Hume:2011aa} (arguably) holds that I do have such an ability.
    I lack the ability to reason in a particular way.
    (Not that \citeauthor{Hume:2011aa}'s arguments rest on visual as opposed to any other kind of imagination, but the point stands.)

    Similarly, you may claim that I have the ability to tell you whether or not \nagent{7} is coming to tea.
    However, and in contrast to your assumption, \nagent{7} has not replied to my invitation and so I lack a required premise in order to reason to a relevant conclusion.
  }
  The key feature of~\ref{EAS:ex:box:inf:R} it that I, the agent, am required to claim support that \(\gamma\) and \(\varsigma\) are the abilities of interest.
  The focus is not on whether or not an agent may perform some action.
  Rather, our interest is with what the action is.
  It is up to me, the agent, to claim support that:
  \begin{enumerate}[label=\emph{A}\arabic*., ref=(\emph{A}\arabic*), resume*=EAS_counter]
  \item\label{EAS:ex:box:gen:R} The general ability \(\gamma\) \emph{is} the ability to calculate the area of a rectangle.
  \end{enumerate}
  If so, I may then claim support that:
  \begin{enumerate}[label=\emph{A}\arabic*., ref=(\emph{A}\arabic*), resume*=EAS_counter]
  \item\label{EAS:ex:box:spec:R} \(\varsigma\) is the specific ability to demonstrate that a rectangle with dimensions \(19\text{cm}\) by \(7\text{cm}\) has area \(133\text{cm}^{2}\).
  \end{enumerate}
  Note, there is no route to \ref{EAS:ex:box:spec:R} other than by claiming support for~\ref{EAS:ex:box:gen:R} as I have no information about what the (specific) ability \(\varsigma\) is amounts to if \ref{EAS:ex:box:gen:R} is not the case.
  If \(\varsigma\) is some other ability, then~\ref{EAS:ex:box:fact} does not follow, witnessing the relevant ability would not demonstrate that a rectangle with
  dimensions \(19\text{cm}\) by \(7\text{cm}\) has area \(133\text{cm}^{2}\), and so a rectangle with dimensions \(19\text{cm}\) by \(7\text{cm}\) may (from my epistemic perspective) have some other area.
\end{note}

\begin{note}[Point]
  We will say more in \autoref{sec:cases-interest}.
  For the moment it is sufficient to observe that the agent is required to reason to and from specific ability.
  The scenario requires the agent to claim support by reasoning from~\ref{EAS:ex:box:if} to~\ref{EAS:ex:box:spec} and from~\ref{EAS:ex:box:spec}~to~\ref{EAS:ex:box:fact}.
  And, while the context added to force the reasoning pattern of interest is narrow, the principle behind the context is simple:
  The agent is required to claim support that they have the relevant general ability.
  Hence, any scenario which consists of \gsi{-} which requires the agent to claim support that they have the relevant general ability will require the same kind of reasoning pattern.

  Further, this arguably captures a general puzzle about ability.
  An agent is not required to have witnessed all instances of a general ability to claim support that they have the general ability.
  However, so long as the agent may claim support for having some general ability, then it follows that the agent will have the option of claiming support for each instance of the general ability.

  The primary issue, though, is whether there is an account of such reasoning that does not require \EAS{} to be true.
  We will shortly turn to this argument in \autoref{sec:broad-argum-overv}.
  Prior to doing so, we close this section with some further clarification on the motivation behind \EAS{}, what distinguishes \EAS{} from nearby principles, and a suggestion on how to conceptualise \EAS{}.
\end{note}

\subsubsection{Ability and dispositions}

\begin{note}[Parallel]
  To further clarify the motivation for \EAS{} we introduce a parallel between abilities and dispositions.
  The primary function of the parallel will be to highlight the importance of reasoning about an event.
  In the case of dispositions the event is the manifestation of the disposition, and in the case of ability the event is the agent witnessing the ability.

  The parallel is of interest because \EAS{} concerns the premises and steps of reasoning that the agent would use to witness the relevant ability.
  We will suggest that claiming support that some object has some disposition and that some agent has some ability may both be understood in terms of claiming support that the relevant event is a possible event.

  In turn, if reasoning \emph{to} a specific ability is understood in terms of claiming support that it is possible for the agent to witness the event, then reasoning \emph{from} a specific ability may be understood in terms of claiming support from what would happen in the possible event.
  \end{note}

\begin{note}[Parallel between dispositions and ability]
  Consider \citeauthor{Choi:2021wg}'s characterisation of the Simple Conditional Analysis of dispositions:
  \begin{quote}
    An object is disposed to \emph{M} when \emph{C} iff it would \emph{M} if it were the case that \emph{C}.\nolinebreak
    \mbox{}\hfill\mbox{(\Citeyear{Choi:2021wg})}
  \end{quote}
  For example, an object is disposed to dissolve when it is placed in water iff the object would dissolve if it were the case that it is placed in water.

  The Simple Conditional Analysis may be challenged, but for our purposes it is adequate.
  We are interested in the broad form of the truth condition, and various more refined analyses share the same broad form.
  Note, in particular, that it being the case that \emph{C} and \emph{M} happening describes an event.
  Given appropriate conditions; salt dissolves, glass breaks, and I mumble when I am tired.
  The key idea is that the property of being disposed to \emph{M} when \emph{C} is analysed in terms of the (possible) event of \emph{M} happening when \emph{C}.

  The parallel to ability is established by noting that ability may also be analysed in terms of a (possible) event, as we have seen.
  In particular, by incorporating volition in the analysans of the Simple Conditional Analysis.
  To illustrate, \citeauthor{Mandelkern:2017aa} trace the Conditional Analysis of ability  to \textcite{Hume:1748tp} and \textcite{Moore:1912te}, among others:
  \begin{quote}
    S can \(\phi\) iff S would \(\phi\) if S tried to \(\phi\)\nolinebreak
    \mbox{}\hfill\mbox{(\Citeyear[Cf.][308]{Mandelkern:2017aa})}
  \end{quote}
  Compare to the Simple Conditional Analysis of dispositions:
  The object is some agent \emph{S}, \emph{C} is `S tried to \(\phi\)' and \emph{M} is `S \(\phi\)s' --- it is volition alone which distinguishes the analyses.
  For example, I have the ability to demonstrate that a rectangle with dimensions \(19\text{cm}\) by \(7\text{cm}\) has area \(133\text{cm}^{2}\) only if I would demonstrate that a rectangle with dimensions \(19\text{cm}\) by \(7\text{cm}\) has area \(133\text{cm}^{2}\) if it were the case that I tried that a rectangle with dimensions \(19\text{cm}\) by \(7\text{cm}\) has area \(133\text{cm}^{2}\).
\end{note}

\begin{note}[Claiming support]
  Parallel analyses in hand, we now turn to claiming support.
  We start with dispositions.

  As with ability, there are various ways in which an agent may claim support that some object is disposed to \emph{M} when \emph{C}.
  For example, I may claim support that my shoes are disposed to squeak when wet because I have had sufficient occasion to observe the phenomenon.
  Likewise, I may claim support that any shoe of the same model is disposed to squeak when wet because I have traced the source of the squeak to a manufacturing choice.
  In short, support may be claimed by past event and shared properties.

  Still, take a novel act and a object pair.
  Personally, I have a empty fountain pen that I haven't placed in water.
  I claim that the fountain pen is disposed to float when placed in water.
  My reasoning is fairly simple.
  The fountain pen is quite light, especially so while empty of ink.
  And, the cap and loading mechanism seem to be quite well sealed, so the weight of the fountain pen will not increase by taking on water.
  So, given that the weight of the fountain pen will be unchanged, and given how light the pen is, it seems that the upward force exerted by the water against the fountain pen will be sufficient to keep the pen afloat.

  In short, I've noted a few properties of the pen, claimed support for a handful of others, and then considered what would happen.
  Our interest is with the last step.
  I appeal to, and use, the possible event.\nolinebreak
  \footnote{
    I may be wrong about the event, but that isn't at issue.
    It remains the case that I appeal to it.
  }
  The noted properties are relevant because they suggest that the event of floating would happen if it were the case that the fountain pen were placed in water.
\end{note}

\begin{note}
  The fountain pen is not the only object on my desk.
  Beside the fountain pen is a collection of instruments that I may use to investigate the fountain pen.
  And, stored in my mind is a basic understanding of fluid dynamics.

  If I were to measure the fountain pen, ensure that it is airtight, and appeal to some known facts, then an application of Archimedes' principle would allow me to demonstrate that the fountain pen is disposed to float when placed in water (of some specified density).
  Indeed, such a demonstration would be a straightforward refinement of the way in which I claimed support for the proposition that the pen is disposed to float when placed in water.

  Now, by similar reasoning I have claimed support for the proposition that I have the ability to demonstrate that the proposition that the pen is disposed to float when placed in water is true.
  Here, in addition to appealing to properties of the fountain pen, I also appealed to various mental properties.
  There is an important difference, however, regarding the relevant event.
  When reasoning about the disposition, the event is the fountain pen floating in water, but when reasoning about my ability to demonstrate the event is the demonstration --- a series of measurements and calculations.
\end{note}

\begin{note}[Diverge]
  Now to turn to \EAS{}.

  If I have the ability, then it follows that the fountain floats in water.
  As noted above, it is not possible for me to demonstrate something that is not the case.

  Claim support for the proposition that the fountain floats in water.

  Still, disposition, fountain pen is not floating in water.
  Likewise with respect to ability, I have not demonstrated that the fountain pen floats in water.
  I noted various things, but did not piece these together into a demonstration.

  Yet, in claiming support, there's the event of demonstrating.
  And, so I appeal to those premises and steps I would use in the event.
  This is \EAS{}.

  Appeal to what happens in the event.
  And, reasoning to claim possible event is viewed in terms of ensuring that the resources are available.
  I have not used the relevant premises and steps of reasoning, nor am I clear on the specific form they will take.
  Still, they are available.

  Final point of interest, then.
  In both cases, there's an appeal to an event.
  If \EAS{} holds with respect to ability, does something similar hold with respect to dispositions?

  First, important clarification.
  The reasoning outlined for disposition was claiming support for event.
  Here, no clear issue with \ESU{}.
  Similarly, no clear issue with \ESU{} with respect to claiming support for having an ability.
  Tension with \ESU{} arises when using ability as a premise in further reasoning.

  Second, key divergence.
  Conclusion obtained is something that is true independent of ability.
  Unclear to me whether similar reasoning with dispositions.
  For, ability is about an event involving the agent.

  In addition, there is no issue with supposing that the agent reasons with (and hence uses) to all the relevant features of the event.\nolinebreak
  \footnote{There may me details of reasoning that one is not easily able to express, but it doesn't follow that those details are not used.}
  Ability is in part interesting because it is clear that an agent does not witness the relevant event.
  This is not to say that a variant of \EAS{} does not hold with respect to dispositions.
  Rather, I am expressing
  \begin{enumerate*}
  \item hesitancy that there are comparable entailments, and
  \item concern that there is no clear argumentative path.
  \end{enumerate*}

  There is a related question about the ability of other agents.
  Here, \EAS{} does not entail.
  In turn, one may conjecture that reasoning from one's own ability is similar.
  I find this plausible.
  It is important to stress again that \EAS{} expresses a way in which an agent may claim support.
  Hence, \EAS{} is compatible with there being other ways in which an agent may claim support.
  It may be the case that the same holds with respect to other agents.\nolinebreak
  \footnote{
    For example, \citeauthor{Owens:2006tw} argues for a belief expression model of assertion in which the rationality of a belief formed by an agent on the basis of testimony depends whatever justification the speaker has for the relevant propositional content.
    \begin{quote}
      Trusting an expression of belief by accepting what a speaker says involves entering a state of mind which gets its rationality from the rationality of the belief expressed. This state's rationality depends on the speaker's justification for the belief he expresses, not on his justification for the action of expressing it. And to hear a speaker as making a sincere assertion, as expressing a belief, is \emph{ceteris paribus} to feel able to tap into \emph{that} justification (whether or not his assertion was directed at you) by accepting what he says.\nolinebreak
      \mbox{}\hfill\mbox{(\Citeyear[123]{Owens:2006tw})}
    \end{quote}
    \color{red} Some more
  }
  However, this is not an immediate consequence.
  \EAS{} permits exceptions to \ESU{}, but it does not require all instances of reasoning with ability is an exception to \ESU{}.
  And, our focus will be on cases in which an agent reasons about their own ability to reason.
  The weak quantifier `there are cases' is designed to leave such issues open.
\end{note}

\begin{note}[Concluding parallel]
  To summarise.
  \begin{itemize}
  \item Parallel between analysis of dispositions and abilities.
  \item Event in analysis of both.
  \item Reason about event.
  \item Motivation for \EAS{} by considering reason to and from event.
  \item This doesn't provide anything close to a clear theoretical account of the reasoning performed if \EAS{} is true, but it does hint at such at how developing such an account may be approached.
  \item Now turn to related conclusion.
  \item In turn, fill in some details on the account.
  \end{itemize}
\end{note}

\subsubsection{Enthymematic inferences}

\begin{note}[\citeauthor{Moretti:2019wx}]
  Above we considered how various account of the basing relation seem to imply \ESU{}.
  Roughly, because such accounts of the basing relation required a premise or step of reasoning to be used in order to be a candidate member of the base of some conclusion of reasoning --- motivated by either causal and representational considerations.
  In contrast, \citeauthor{Moretti:2019wx} argue for an account of the basing relation which does not entail \ESU{}.

  In our terminology, \citeauthor{Moretti:2019wx} argue that: A belief held by an agent may be \emph{based} on premises that the agent did not use when forming the belief.

  The following is a fragment of the general principle relating propositional justification to well-grounded belief (alternatively doxastistcally justified belief) containing the two clauses of interest:

  \begin{quote}
    IF

    \dots

    OR

    \begin{enumerate*}[label=(\arabic*.2\(^{\ast}\))]
    \item\label{LT:1.2} Q is propositionally justified for S in virtue of P1, P2, \(\dots\), Pn being justifiedly true from her perspective because S justifiedly believes P1, P2, \(\dots\), Pn, and in virtue of her being aware that Q is an inductive or deductive consequence of P1, P2, \(\dots\), Pn jointly, and
    \item\label{LT:2.2} S carries out a \emph{plain} inference from P1, P2, \(\dots\), Pn to Q.
    \end{enumerate*}

    OR

    \begin{enumerate*}[label=(\arabic*.3), ref=(\arabic*.3)]
    \item\label{LT:1.3} Q is propositionally justified for S in virtue of P1, P2, \(\dots\), Pn being justifiedly true from her perspective, though S doesn't believe at least some P1, P2, \(\dots\), Pn, and in virtue of S being aware that Q is an inductive or deductive consequence of P1, P2, \(\dots\), Pn jointly, and
    \item\label{LT:2.3} S carries out a (fully or partly) \emph{enthymematic inference} from P1, P2, \(\dots\), Pn to Q.
    \end{enumerate*}

    THEN
    \begin{enumerate}[label=(3)]
    \item S's belief that Q is well-grounded.\nolinebreak
      \mbox{}\hfill\mbox{(\Citeyear[87]{Moretti:2019wx})}
    \end{enumerate}
  \end{quote}

  The `plain' inference of~\ref{LT:1.2} and~\ref{LT:2.2} corresponds to cases in which an agent uses P1, P2, \(\dots\), Pn to reason to Q.
  By contrast, the `enthymematic' inference of~\ref{LT:1.3} and~\ref{LT:2.3} involves reasoning in which an agent does not use some or all of P1, P2, \(\dots\), Pn to reason to Q as the agent does not believe some of P1, P2, \(\dots\), Pn (though the agent has propositional support for each of P1, P2, \(\dots\), Pn).

  To illustrate the distinction between `plain' and `enthymematic' inferences (\Citeyear[Cf.][85]{Moretti:2019wx}) consider reasoning from the premise that \nagent{5} is shorter than \nagent{6} to the conclusion that someone is taller than \nagent{5}.
  An instance of plain (non-enthymematic) may take the intermediary step that \nagent{6} is taller than \nagent{5} before abstracting from \nagent{6}.
  In contrast, an instance of enthymematic reasoning consists of the (single) premise and conclusion noted without forming the belief that \nagent{6} is taller than \nagent{5}.\nolinebreak
  \footnote{Cf.\ (\Citeyear[87--89]{Moretti:2019wx}) for examples given by \citeauthor{Moretti:2019wx}.}

  The key idea is that if an agent reasons enthymematically, then the agent's belief may be based on those premises that the agent would use in the corresponding plain inference.
  (\Citeyear[Cf.][86--87]{Moretti:2019wx})
  Hence, we have a proposal on which an agent's belief may be supported by premises and steps of reasoning that an agent has not used.
  And, in addition, because S carries out a (fully or partly) enthymematic inference \ref{LT:2.3}, it seems S \emph{may} appeal to P1, P2, \(\dots\), Pn when reasoning to Q, in conflict with \ESU{}.

  Whether or not \citeauthor{Moretti:2019wx}'s account is correct is not of interest.
  Rather, \emph{grating} that \citeauthor{Moretti:2019wx}'s account is correct allows us to make two (related) observations.
  First, \citeauthor{Moretti:2019wx} account does not conflict with \ESU{} and so the account does not require \EAS{} to be true.
  And, second, how \citeauthor{Moretti:2019wx}'s account suggests a broader theoretical account of \EAS{}.
\end{note}

\begin{note}[First point]
  To establish the first point we require further details about how \citeauthor{Moretti:2019wx} define a (fully or partly) enthymematic inference.
  The following quote combines the relevant definitions:
  \begin{quote}
    \textbf{(}[\textbf{Partly}/\textbf{Fully}] \textbf{Enthymematic Inference)}

    S carries out a [\emph{partly}/\emph{fully}] \emph{enthymematic} inference from P1, P2, \(\dots\), Pn to Q if and only if
    \begin{enumerate}[label=(\alph*), ref=(\alph*)]
       \setcounter{enumi}{1}
    \item \emph{S doesn't actually believe} [\emph{at least some of the premises}/\emph{any of}] P1, P2, \(\dots\), Pn, though some constituents M1, M2, \(\dots\), Mm of S's perspective cause in S the \emph{disposition} to believe P1, P2, \(\dots\), Pn, and
    \item M1, M2, \(\dots\), Mm [together with the premises believed by S jointly/jointly] cause S's belief that Q through a process that is shaped by S's taking Q to be a consequence of P1, P2, \(\dots\), Pn at a personal level.\nolinebreak
      \mbox{}\hfill\mbox{(\Citeyear[85]{Moretti:2019wx})}
    \end{enumerate}
  \end{quote}

  In short, an enthymematic inference involves reasoning with premises M1, M2, \(\dots\), Mm which are related to the premises P1, P2, \(\dots\), Pn of some corresponding plain inference.
  In order to complete the definition, we require an account of what it is for S to take Q to be a consequence of P1, P2, \(\dots\), Pn at a personal level:

  \begin{quote}
    \textbf{(Personal Level\(^{\ast}\))}

    S's mental states M1, M2, \(\dots\), Mm and any premises believed by S, among P1, P2, \(\dots\), Pn, jointly cause S's belief that Q through a process shaped by S's taking Q to be a consequence of P1, P2, \(\dots\), Pn at a personal level if and only if M1, M2, \(\dots\), Mm and any premise believed by S, among P1, P2, \(\dots\), Pn, jointly cause S to believe Q and S would adduce the reasons that P1, P2, \(\dots\), Pn and that Q is a consequence of P1, P2, \(\dots\), Pn in response to a request to explain why she believes Q.\nolinebreak
    \mbox{}\hfill\mbox{(\Citeyear[85--86]{Moretti:2019wx})}
  \end{quote}

  So, loosely reconstructed an enthymematic inference involves constituents M1, M2, \(\dots\), Mm of S's perspective which ensure that S has the disposition to believe P1, P2, \(\dots\), Pn.
  And, the way in which M1, M2, \(\dots\), Mm lead to S forming the belief that Q allow S to explain that they believe Q on the basis of P1, P2, \(\dots\), Pn.
  In short, an enthymematic inference is an inference in which may be \emph{post hoc} expanded to some corresponding plain inference (in part) because performing the enthymematic inference requires the agent to be disposed to believe the required premises of the corresponding plain inference.
  And, as such the premises of the corresponding plain inference may be considered as (constitutive of) the basis of S's belief that Q.

  In contrast, \ESU{} concerns the way in which M1, M2, \(\dots\), Mm lead to S forming the belief that Q do not necessarily require the agent to appeal to P1, P2, \(\dots\), Pn.
  It is consistent with \citeauthor{Moretti:2019wx} account that the reasoning from M1, M2, \(\dots\), Mm to Q may only appeal to premises and steps of reasoning used.
  That Q may be based on P1, P2, \(\dots\), Pn is due to the requirement that S is disposed to believe P1, P2, \(\dots\), Pn and the possibility of S retroactively appealing to Q being a consequence of P1, P2, \(\dots\), Pn.
  Hence, the account does not conflict with \ESU{}, and in turn does not require \EAS{} to be true.

  The insight offered is that there does not necessarily need to be a structure preserving mapping between premises and steps providing propositional support for a belief and the premises and steps appealed to when forming the belief.
  However, this does not constrain what the agent appeals to when forming a belief.

  From a broader perspective, \citeauthor{Moretti:2019wx}'s proposal considers what an agent was able to do (i.e.\ reason by some plain inference) and holds that a basing relation follows but is silent of the way in which an agent claim support.
  In contrast, \EAS{} looks at what an agent is able to do, and holds that a way of claiming support follows, but is silent on issues concerning the basing relation.
\end{note}

\begin{note}[Second point]
  Still, this broader perspective together with the above discussion of dispositions suggests a way to understand \EAS{}.
  For, one may hold that if an agent has the ability to reason to some conclusion, then the agent is disposed to use relevant premises and steps of reasoning to reason to the conclusion.
  In parallel to \citeauthor{Moretti:2019wx}, then, one may hold that the agent has the ability to reason to some conclusion if (and only if) they are suitably related to some collection of relevant premises and steps of reasoning.
  In turn, the agent may appeal to those premises and steps of reasoning to claim support for the conclusion.
  Indeed, if we adopt a parallel understanding of the basing relation, then it follows (so long as the agent has the ability) that  the agent has sufficient propositional support for the conclusion, and may be well-grounded.
  The (possible) event of reasoning to the conclusion is important both for establishing that the agent has the ability and for determining which premises and steps of reasoning the agent appeals to, but the event is not important for determining that the relevant premises and steps of reasoning are available to the agent.

  This suggestion falls far short of a theory satisfying \EAS{}, I suspect \EAS{} may be motivated in part by distinguishing between what occurs in the event of reasoning, and sufficient resources required for such an event to occur.
  An event of reasoning will always make use of sufficient resources for the event to occur, but an agent may have sufficient resources for the event to occur even if the event does not occur.
  (Specific) abilities, then, fix an particular event and determine sufficient resources and the agent does not need to witness the event in order to appeal to those resources.

  Or perhaps not.
\end{note}

\begin{note}[Segue]
  Our goal is to establish that an adequate account of reasoning which extends to ability must satisfy \EAS{}.
  This goal does not require the above suggestion to be on the right track, nor does this goal require that there is a unique theory that satisfies \EAS{}.
  For now, we close the present section with a few remarks concerning ability and~\EAS{}.
\end{note}

\begin{note}[Actual support]
  As with~\ESU{}, \EAS{} does not entail that the agent \emph{has} support.
  Our focus is on reasoning, and as argued above, it seems the issue of whether an agent has support is distinct from whether an agent may claim support.
  Claiming support is the result of some reasoning, and whether or not an agent has support requires an evaluation of that reasoning.
  This means that, strictly speaking,~\EAS{} does not carry any implications regarding whether or not the agent has support by claiming support in line with~\EAS{}.
  It is possible that the agent would fail to establish support, or establishes a support relation other than between the conclusion and the premises and steps of reasoning appealed to.

  Still, it take it to be plausible that support traces a successful claim.
  From this perspective,~\EAS{} may seem a little more intuitive.
  Given an intuitive understanding of support, if an agent does have the ability to reason to some conclusion, then the conclusion stands in the relation of being support by certain premises and steps of reasoning, whether or not the agent witnesses their ability.
  In turn, if the agent may claim support for the having the relevant ability then the agent may claim support for the conclusion from the premises and steps that would be used to witness their ability.
  For, witnessing does not contribute to the relation of support between the conclusion and the relevant premises and steps --- witnessing would only clarify to the agent the specifics of the relation.

  Of course, the agent may be mistaken or misled about having ability.
  For example, the relevant premises and steps may fail to establish the conclusion, or the agent may not have sufficient resources to carry out reasoning from the premises and steps, etc.
  In turn, witnessing may be expected to highlight that the claimed support for having the ability is mistaken or misled.

  Two points:
  \begin{itemize}
  \item Such issues are not different to being mistaken or misled and using that one has the ability as a premise, so apply to any reasoning that makes use of ability without witnessing ability.
  \item Attempting to witness the ability might reveal that the agent is mistaken or misled about having the ability does not show that the agent may not claim support for having the ability.

    Reasoning typically involves premises and steps of reasoning that could be investigated further, but this does not prevent an agent from appealing to those steps and premises.
  For example, it is (almost) to check the definition of any word used against a dictionary, and doing so might reveal that I have been mistaken or mislead about the meaning that I will convey by using the word.
  I rarely do this, though.
  Most of the time it is sufficient to expect that I am not mistaken or have not been mislead about the meaning I would convey by using the word.
  \end{itemize}
\end{note}

\subsubsection{Application}
\label{sec:application}

\begin{note}[Desire]
  Finally, while the examples of reasoning given have concluded with the truth of some proposition --- that a rectangle has some specific area, or that a given fountain pen floats in water, etc.\ --- our interest with \EAS{} is broader.

  In many cases the assigned value truth, falsity, or something in between.
  However, claiming support, and in turn; \USE{}, \ESU{}, and \EAS{} are all neutral with respect to the value assigned to the proposition.
  Therefore, we may consider other values while investigating, and as an application of \ESU{} and \EAS{}.
  In particular, consider reasoning which concludes with the desirability of some proposition.\nolinebreak
  \footnote{
    \color{red}
    Mistaken or misled.
    Yes, I think this holds up.

    Strong view on which an agent may be mistaken about desires in the same way as an agent may be mistaken about evidence.
    View on which desires are independent of representation.
    Hence, misleading or mistaken support when an agent fails to represent desire.
  }

  To illustrate this point, consider temptation.\nolinebreak
  \footnote{
    \color{red}
    Whether or not this is `genuine' temptation isn't of \dots
  }
  Specifically we will consider a slight variation on \citeauthor{Bratman:1999ac}'s `two glasses of wine' (\Citeyear[38]{Bratman:1999ac}) case of temptation.\nolinebreak
  \footnote{
    \color{red}
    See also \textcite{Bratman:2007ab}
  }
\end{note}

\begin{note}[The Pianist]
  Consider a pianist who frequently performs at a club.
  Before each performance the pianist gets nervous and has the option of drinking a glass of wine.
  A glass of wine would also lead to a worse performance.
  However, the glass of wine would help with the pianist's nerves.
  Both are learnt with some experience.
  Hence, if the pianist reasons about what to do:
  \begin{itemize}
  \item When the pianist does not feel the nerves of an upcoming performance they reason to a preference abstaining from drinking a glass of wine.
  \item Yet, when nerves are felt the pianist reasons to a preference for drinking a glass of wine.
  \end{itemize}
  The pattern is stable, and has held over many performances.

  Still, while nerves sometimes get to the pianist, they abstain from drinking a glass of wine most of the time.

  That the our pianist abstains is not necessarily surprising --- it is not uncommon to resist temptation.
  Though it is puzzling.
  The pianist's reasoning is unwavering throughout the span of time in which the pianist has the option of drinking the wine; they reason to preference for drinking a glass of wine.
  So, if the pianist abstains, the pianist acts in opposition to their preference when given the option of drinking a glass of wine, and does so purposefully.
\end{note}

\begin{note}[Reasoning and desire]
  To clarify the puzzle, let us state a basic conjecture regarding preferences and acts.

  \begin{conjecture}\label{conj:resolve-issue-act}
    Any instance of purposeful rational action performed by an agent is the result of the agent resolving the issue of how to act.
    Where:
    \begin{enumerate}
    \item An act is an candidate resolution for how to act only if the agent has claimed support for preference for some proposition and has an expectation that act would bring about the proposition.
    \item An act is an admissible resolution only if there no other candidate resolutions for which the agent has a stronger (combined) preference with respect to the proposition(s) that the agent expects to be brought about by performing the act.
    \end{enumerate}
  \end{conjecture}

  \autoref{conj:resolve-issue-act} understands rational action as the result of an agent resolving the issue of how to act --- choosing which act from a collection of options to perform.
  By understanding rational action as the result of an agent resolving the issue of how to act we may break down the reasoning involved in purposeful rational action into two steps.

  First, what makes an act a candidate resolution, and second what makes an act an admissible resolution.

  An act is a possible resolution just in case the agent links the result of acting to some proposition the agent has a preference for.

  And, an  act is an admissible resolution just in case the agent has no stronger preference for some other proposition that the agent expect could be brought about by some other candidate action.
  (Or, more generally, when an agent is uncertain about which proposition may be brought about by some act, a combined preference regarding each potential proposition.)

  In short, \autoref{conj:resolve-issue-act} is more-or-less the core of an standard decision theoretic account of maximising expected utility without commitment to particular details.\nolinebreak
  \footnote{
    \color{red}
    Cf.\ \textcite{Steele:2020tr}.
    \citeauthor{Davidson:1963aa} `Primary reason' (\Citeyear{Davidson:1963aa})
  }
\end{note}

\begin{note}[Use of conjecture]
  \autoref{conj:resolve-issue-act} fixes an understanding of purposeful rational action, and in turn establishes two ways in which the pianist may resist drinking a glass of wine:
  \begin{enumerate}
  \item Drinking the glass of wine is not a candidate resolution.
    \autoref{conj:resolve-issue-act} states a necessary condition for a candidate resolution, but further conditions may rule out possible resolutions which satisfy the necessary condition stated.
  \item Drinking the glass of wine is not an admissible resolution.
    In particular, because the pianist has a claims support for a stronger preference toward the result of abstaining.
  \end{enumerate}

  We will provide a brief argument that \ESU{} requires the former to be the case and provide an example of how further conditions may rule out possible resolutions.
  In short, \ESU{} requires an agent to witness reasoning to a conclusion in order to claim support for such a conclusion, and as the pianist reasons to a preference for having drunk a glass of wine when performing, abstaining is not an admissible resolution.
  Then, we will turn to \EAS{}, and suggest that it allows the latter to be the case while granting that the only reasoning that the pianist witnesses establishes a preference for having drunk a glass of wine when performing.
  In short, so long as the pianist may claim support for the ability to reason to a stronger preference for the result of abstaining, then by \EAS{} the agent may claim support for a stronger preference for the result of abstaining.
\end{note}

\begin{note}
  Suppose \ESU{} is true.
  By \autoref{conj:resolve-issue-act} an agent resolves an issue of how to act by determining candidate resolutions.
  In turn, a candidate resolution results from the agent claiming support for a preference toward some proposition.
  And, \ESU{} requires that an agent must witness some instance of reasoning in order to claim support for the conclusion of the instance of reasoning.

  Turning to the pianist, we have assumed that before taking to the stage the pianist reasons to preference that favours drinking a glass of wine.
  So, given \autoref{conj:resolve-issue-act} abstaining is not an admissible resolution because the agent has a stronger preference from drinking a glass of wine before taking the stage.
  And, by \ESU{} it is not possible for the pianist to claim support for a preference that would lead to abstention because the pianist must witness the relevant reasoning in order to claim support.
  Therefore, the pianist must rule out drinking a glass of wine as a candidate resolution for how to act.
\end{note}

\begin{note}[Intention and \ESU{}]
  \color{red}

  \citeauthor{Bratman:2007ab} argues that such cases may be understood through an theory on which intentions constrain reasoning.
  If the pianist intends no to drink the wine, and this intention persists, then drinking the wine is no an available conclusion of reasoning.
  Key here is that intention does not interact with the pianist's preferences.
  It remains the case that the pianist would prefer.

  The role of intention in the role of \citeauthor{Bratman:2007ab} account is to constrain possible resolutions for how to act.
  An intention to not drink prevents drinking a glass of wine from being a candidate resolution for how to act (see in particular \textcite[\S3.3]{Bratman:1987aa}).

  However, because ruled out, the pianist does not have the option of acting on that preference.
  Rather, act in a way that is compatible with intention.
  For the pianist, we may assume abstention is the only act compatible with the intention.\nolinebreak
  \footnote{
    \color{red}
    Variation.
    Block contribution of nerves.
    So, intention to not allow nerves to contribute to reasoning.
    Compatible with drinking, but given the way the scenario has been constructed, will result in not drinking.
  }
  So, \citeauthor{Bratman:2007ab} is an example of how to resolve weakness of will given relation between reasoning and action expressed by \autoref{conj:resolve-issue-act} and \ESU{}.
\end{note}

\begin{note}[Broader]
  I think, in broad strokes, phenomena fit this kind of theory.
  Abstracting from the details of any particular theory, it seems plausible that candidate resolutions to the issue of how to act are subject to conditions that extend beyond whether the agent has claimed support for preference for some proposition and has an expectation that act would bring about the proposition --- \autoref{conj:resolve-issue-act} only stipulates that the given constraint is a necessary condition for candidate resolutions.

  However, not clear to me that all phenomena fit such a theory.
  The pianist's reasoning seems distorted.
  The nerves felt before taking to the stage plausibly interfere with the pianists reasoning about candidate resolutions to the issue of how to act, and so the pianist's reasoning plausibly does not resolve the issue of how to act in line with the premises and steps of reasoning that are available to the agent.
  So much, I suspect, is intuitive.
  However, any interpretation of the pianist compatible with \ESU{} is committed to the agent resolving the issue of how to act given unreliable reasoning.
  A \citeauthor{Bratman:1987aa}-like intention would rule out drinking a glass of wine as a resolution to the issue of how to act, but whatever reasoning the agent performs given the intention is still influenced by the nerves felt.

  In the following paragraphs we will suggest that the pianist may resist the conclusion of reasoning performed given nerves because it is distorted.
  We start with two additional conjectures.
\end{note}

\begin{note}[Preferences change with reasoning]
  \begin{conjecture}\label{conj:pref-vs-reasoning}
    Whether, or to what degree, an agent claims support for a preference toward a proposition may differ from whether, or to what degree, the agent would claim support for a preference toward the proposition given varying information.
  \end{conjecture}

  Loosely paraphrased,~\autoref{conj:pref-vs-reasoning} states that preferences an agent reasons to are subject to change given change in the information that the agent reasons from.
  Hence, it seems~\autoref{conj:pref-vs-reasoning} may be considered a truism rather than a conjecture.
  Indeed, varying information is a common component in the construction of cyclical preferences (cf.\ \cite{Sobel:1997wt},\cite{Schumm:1987wx},\cite{Davidson:1955wo}, etc.)

  To illustrate, we consider a case in which difference arises from information that does not contribute to the agent's preferential evaluation of the proposition.

  Suppose an agent has a established preference for meeting person who is Black Panther over meeting the person who is Storm by reasoning.
  However, the agent is not aware that Black Panther is T'Challa nor that Storm is Ororo Monroe.
  Indeed, the agent does not have any information about the referent of `T'Challa' or `Ororo Monroe' and so does establish a preference for meeting T'Challa or Ororo Monroe by reasoning (nor vice-versa).

  Still, it seems that if the agent were provided with the information that Black Panther is T'Challa and that Storm is Ororo Monroe then the agent would reason to a preference for meeting T'Challa or Ororo Monroe.
  And, such information would not contribute to the agent's preferential evaluation of the relevant propositions.
  The agent's preferences for the relevant propositions are determined by considerations that are independent of the terms used to refer to the relevant individuals --- e.g.\ the person who helped defeat Thanos and the person who helped defeat Magneto.

  With relation to \autoref{conj:resolve-issue-act}, whether or not an act is a possible resolution for issue of how to act may depend on what information agent reasons from.
  We now introduce a further conjecture:

  \begin{conjecture}\label{conj:more-info-is-good}
    Generally speaking: When resolving how to act, claimed support for some preference toward a proposition given more information is given greater weight than claimed support for preference toward the (same) proposition given less information.\nolinebreak
  \footnote{
    Note, `more' and `less' information are relative, and it may not be possible to compare distinct bodies of information.
    If so, \autoref{conj:more-info-is-good} does not state anything about the importance of either body of information.
    For example, an agent may have information about the subjective taste of a meal and about the nutritional value of the meal.
    Still, without a way to compare information about subjective taste to information nutritional value in an information there would be no sense in which the former could be considered to hold `more' information that the latter (or vice-versa).
    Though this is not to rule out such a comparison --- we do not place constraints on what comparisons an agent may make.
  }
\end{conjecture}

  \autoref{conj:more-info-is-good} speaks more generally than~\autoref{conj:pref-vs-reasoning} and as a result may be closer to or further from a truism depending on your point of view.
  Still, the core idea is simple:
  Claimed support for some preference toward a proposition given more information is worth more than claimed support for some preference toward a proposition given less information because more information typically increases the reduces the likelihood that the claimed support is either mistaken or misled.\nolinebreak
  \footnote{
    Cf.\ \cite{Good:1966wx} for a related idea --- though see also \cite{Bradley:2016wo}.
  }
  In other words, strength of preference in the sense of \autoref{conj:resolve-issue-act} is proportional to information used to establish preference given a fixed proposition.

  To illustrate, consider an agent resolving whether to banded or off-brand multi-vitamins.
  The agent's method of establishing a preference is to look on each container, and work through the lists of vitamins comparing whether a vitamin is included, and if so to what quantity, weighing some vitamins more heavily than others.
  As the agent works through the lists of vitamins the agent moves from less information to more.
  Still, after each vitamin on the list the agent marks a preference.
  For example, the branded multi-vitamins have 2,500 IU of vitamin A while the off-brand have 2,000 IU, so the agent's initial preference leads to purchasing the branded multi-vitamins.
  However, the branded multi-vitamins have 50mg of vitamin C while the off-brand have 60mg, and given information about vitamins A and C the agent's preference leads to purchasing the off-brand multi-vitamins.
  And so on until the agent has compared the contents of the branded and off-brand multi-vitamins.

  \autoref{conj:more-info-is-good} holds because it seems implausible that the agent could be understood as acting rationally by purchasing either multi-vitamin from a preference determined by a partial comparison between the multi-vitamins given that the agent has established a preference given a full comparison between the multi-vitamins.
\end{note}

\begin{note}[Back to the pianist]
  To summarise the two conjectures:
  \autoref{conj:pref-vs-reasoning} holds that an agent's preferences may vary with the information that the agent uses to claim support for those preference.
  And, \autoref{conj:more-info-is-good} holds, generally speaking, that claimed support for a preference from more information is given greater weight than claimed support for a preference from less information.

  Now, we return to the pianist.
  We make two observations.
  First, given \autoref{conj:pref-vs-reasoning} it may be possible to trace the change in the preference that the pianist reasons to (from abstention to a glass of wine, and vice-versa) to variation in the information that the pianist reasons with.
  In particular, it may be the case that the pianist's nerves prevent them from reasoning with information that they would otherwise reason with, hence the preference to abstain arrived at by reasoning before and after the span of time in which the pianist has the opportunity to drink a glass of wine is a preference arrived at given more information than the preference to drink a glass of wine.
  Second, given \autoref{conj:more-info-is-good} and the present interpretation, the agent may give greater weight to the preference to reasoning that results in a preference for abstention.

  In short, it may be true that pianist's nerves interfere with the reasoning they perform, and without such interference the agent would reason to abstaining from drinking a glass of wine before any given performance.\nolinebreak
  \footnote{
    \color{red}
    Plausible, though not immediate.
    To clear things up a little, consider a hangover.
    Reasoning sucks, I did not desire to miss lunch with a friend, I forgot, because of the hangover.
    However, I did desire to go to bed early, because of the hangover.
    Question is whether the nerves and the glass are like missing lunch or going to bed early.
    If the former, then reasoning is at issue.
    If the latter, then reasoning is at the issue.

    Conjecture, the former.
    No additional information from nerves.
    Still, when the nerves are present the pianist has a hard time reasoning with them.
  }

  The difficulty for the pianist is that such interference is always present;
  It is not straightforward for the pianist to give greater weight to abstaining, as the pianist does not reason to abstaining given their nerves.
  However, if the pianist may claim support for  ability to reason to abstaining, then by \EAS{} the agent may claim support for a preference for abstaining.

  Indeed, it seems the pianist may claim support for having the ability to establish a preference for abstaining when presented with the option of drinking a glass of wine.
  Two observations:
  \begin{itemize}
  \item First, the pianist has performed such reasoning many times, both before and after performances, and the result of such reasoning is stable: abstention.
  \item Second, given the assumptions made about the agent's nerves, the agent retains the relevant premises and steps of reasoning when presented with the choice to drink a glass of wine.
    The nerves felt when presented with the choice to drink a glass of wine only ensure that witnessing this ability is difficult to a degree sufficient for the agent to fail each attempt at witnessing the ability.
  \end{itemize}

  So, when presented with the choice to drink a glass of wine:
  If the  pianist may claim support for having the ability to establish a preference for abstaining, then by \EAS{} the pianist may claim support for a preference for abstaining.
  And, in turn, by \autoref{conj:more-info-is-good} the pianist may give greater weight to their preference for abstaining because the claim of support for a preference to abstain would be arrived at by taking into consideration additional information.

  Given this interpretation of the pianist, `giving into temptation' would be for the pianist to disregard the reasoning that the pianist is able to perform.
  Conversely, `resisting temptation' is for the pianist act in accordance with the preference that they would reason to given the information available to them.
  So, in contrast to accounts of temptation constrained by \ESU{} the pianist need not prevent themselves from reasoning to drinking a glass of wine.
  Rather, the pianist need only reflect on what they are able to reason to.
\end{note}

\begin{note}[Quick objection]
  There is a quick objection to consider before moving on:
  Does the agent have the ability to reason to stronger preference for the result of abstaining?
  For, it seems natural for the pianist to express that they do not have the ability to reason to a preference other than for having drunk a glass of wine when performing \emph{because} of how nerves interfere with their reasoning.

  Ability fluctuates.
  At present I claim support that I have the ability to prove that S4 is sound and complete with respect to transitive frames.
  Part of my claim is that I understand the details of Lindenbaum's Lemma.
  And, if I were to forget the details ofLindenbaum's Lemma then I would lack the ability to construct the relevant proof.
  So, I may lose the ability, but even if I do lose the ability due to forgetting the details of Lindenbaum's Lemma, I may regain the ability by revising the relevant details.

  However, there may be a difference between my loss of ability due to forgetting the details of Lindenbaum's Lemma and the pianist's nerves.
  Forgetting the details of Lindenbaum's Lemma ensures that I lack premises (or steps of reasoning) required to witness the proof.
  By contrast, it is not clear that the pianist's nerves entail that the pianist lack premises (or steps of reasoning) required to reason to stronger preference for the result of abstaining.

  Let us distinguish to ways in which an agent may be said to lack an ability to reason to some conclusion.
  First, the agent lacks sufficient resources; premises and steps of reasoning.
  Second, impediments to the agent using sufficient resources.

  From the first, I have the ability to enumerate all of the positive integers in decimal representation as I have sufficient resources to produce a decimal representation of the first positive integer and I have sufficient resources to produce a decimal representation of any successor integer.
  From the second, I am clearly bounded to enumerate only a finite collection of the positive integers given my mortality and so lack such an ability.

  The success of the quick objection relies on an ability to reason to some conclusion entailing that there are impediments to the agent using sufficient resources to reason to the conclusion.
  I suspect this entailment does not hold for the sense of ability at issue.
  Rather, I suggest that what matters is that the conclusion follows from the premises and steps of reasoning.

  Whether or not this a compelling suggestion is up to you.
  Whether or not \EAS{} holds does not depend on impediments to the agent using sufficient resources entail lack of ability.
  Though, interest in the details of ability and whether they relate to cases of temptation such as the pianist do depend on whether or not \EAS{} holds.
  Therefore, our primary focus will be on showing that \EAS{} holds.
\end{note}

\section{Structure of argument}
\label{sec:structure-argument}

\begin{note}[Structure of argument]
  Two lines of argument for endorsing~\EAS{}, and hence denying~\ESU{}.
  \begin{enumerate}[label=(L\arabic*), ref=(L\arabic*)]
  \item\label{arg:line:1} Motivate~\EAS{} as resolution to tension resulting from~\ESU{}.\newline
    Specifically:
    \begin{enumerate}[label=(L1\alph*)]
    \item\label{arg:line:1:a} Provide recipe for generating scenarios where~\ESU{} is in tension with particular scenarios involving information that an agent has the ability reason to some conclusion and a further claim regarding when it permissible for an agent to claim support for a proposition.
    \item\label{arg:line:1:b} Motivate~\EAS{} as a resolution to the tension.
    \end{enumerate}
  \item\label{arg:line:2} Argue that granting~\EAS{} as an exception to~\ESU{} allows for an intuitive understanding of cases in which agent has the option of appealing to ability, even if there are alternative ways of interpreting the scenario in line with~\ESU{}.
  \end{enumerate}
  These two lines of argument work together.
  The tension of~\ref{arg:line:1} generates interest in witnessing that may be flatly rejected by prior endorsement of~\ESU{}.
  The intuitive understanding of scenarios involving ability of~\ref{arg:line:2} suggests there's more to witnessing than resolving the tension in narrow cases.
\end{note}

\begin{note}[Details of \ref{arg:line:1}]
  The initial focus is on the first line of argument,~\ref{arg:line:1}.
  The tension developed in part~\ref{arg:line:1:a} is delicate, but hopefully informative.
\end{note}

\section{Ability}
\label{sec:major-argument}
\label{sec:broad-argum-overv}
\label{sec:all-about-ability}

\begin{itemize}
\item Type of case.
\item \gsi{-}.
  \begin{itemize}
  \item Two parts.
  \item Claiming support for specific ability
  \item Claiming support for result of specific ability.
  \end{itemize}
\item ability entailment.
\item Schematic interpretations of ability: \AR{} and \WR{}.
\item Exhaustive.
\item Relate to \ESU{} and \EAS{}.
\item Return to two parts of \gsi{}.
  \begin{itemize}
  \item \ESU{} requires that when agent reasons to specific ability, agent claims support for a property in line with \AR{}.
  \item However, \nI{} requires that when an agent reasons from specific ability, agent does not claim support from a property, in contrast to \AR{}.
  \end{itemize}
\item \ESU{} requires agent reasons with \AR{}.
\item Reasoning with \AR{} is incompatible with \nI{}.
\end{itemize}

So, the key thing is that we're looking at two different aspects of reasoning with ability.
Reasoning to and reasoning from.
And, it is not possible to give an account of \emph{both} reasoning to and from ability given \ESU{} and \nI{}.
So, the tension isn't, so to speak, `direct'.
Rather, tension arises due to some extended piece of reasoning.

So, one way to resolve tension is to deny extended reasoning.
First part argues that extended reasoning is plausible.


\begin{itemize}
\item Upshot:
  \begin{itemize}
  \item If scenarios, then either \AR{} or \WR{}.
  \item In turn, either not \nI{} or not \ESU{}.
  \item Alternatively, if scenarios then either \ESU{} or \nI{} (the scenarios give rise to this conflict).
  \item In turn, either \AR{} or \WR{}.
  \end{itemize}
\end{itemize}

We've already seen a decent amount of stuff regarding \ESU{} and \EAS{}.
These more or less correspond to \AR{} and \WR{}, kind of.

% \begin{note}[Before turning to the argument\dots]
%   Before turning to the argument, we conclude this introduction with a handful of notes regarding~\ESU{} and~\EAS{}.
% \end{note}

% \begin{note}[Scope of \ESU{}]
%   \ESU{} does not say anything in particular about what the agent may claim support for, only what must be the case in order for an agent to appeal to support for some conclusion on the basis of support for premises.

%   Talking in terms of (support for) premises and conclusions restricts attention to reasoning.
%   There may be broader use of `premise' and `conclusion' where an agent is not required to reason from premise to conclusion in order for the premise to support the conclusion.
%   For example, if visual perception is immediate.
%   Perhaps it may be said that an agent's visual experience is a premise to the conclusion that a dog is sleeping.
%   Still, for present purposes, `conclusion' refers to the output of some process of reasoning performed by an agent which is either actual or potential, and `premises' to the input of that process.

%   Note, also, that in both cases the relation between premises and conclusion is important.
%   If agent does not reason, then neither~\USE{} nor~\ESU{} apply.
%   If there are multiple ways to obtain a conclusion, then~\ESU{} does not require the agent to reason from a particular set of premises.

%   Likewise,~\ESU{} does not require that an agent is required to obtain support for a proposition by valid and subjectively sound reasoning from some premises.

%   Rather,~\ESU{} requires that an agent reason from premises to conclusion in order to establishes support between premises and conclusion
%   By contrast,~\USE{} holds that reasoning is sufficient to establish such a relation.
% \end{note}

% \begin{note}[\ESU{} is intuitive]
%   \ESU{} is intuitive, and is quite common, though not without exceptions.
% (For example, there's views on testimony in which the testifier provides agent access to support the testifier has.
% One may understand this as conflicting with~\ESU{}, or that the fact that these are accessible is the relevant piece of support.)
% \end{note}

% \begin{note}[Alternative]
%   \EAS{} restricts~\ESU{}.
%   This is not to say the agent obtains support equivalent to that which would be obtained were the agent to do, or have done, the reasoning.
%   Nor, that the agent is aware of the relevant premises.

%   Intuitively, \EAS{} states that the agent may appeal to the reasoning they are able to perform in support for the conclusion of that reasoning, and as that reasoning moves from premises to conclusion, it is on the basis of the support for those premises that the agent would identify by reasoning that the agent obtains (some) support for the conclusion.

%   Hence, \EAS{} is in line with the spirit of~\USE{}.
%   For the exception to~\ESU{} is granted by the agent appealing to a witnessing event in which the antecedent (and consequent) of~\USE{} are satisfied.
% \end{note}

% \begin{note}[Ability ensures propositional?]
%   Plausible that if the agent has the ability, then the agent already has propositional support for the relevant proposition.
% \end{note}

\subsection{Scenarios}
\label{sec:cases-interest}

Our goal is to argue for \EAS{} and against \ESU{}.
At the core of the argument is reasoning about ability.
Specifically, a certain type of scenario in which an agent reason to and from information that they have the ability to witness some specific act.
How the agent reasons with such (specific) ability information in the scenarios of interest will provide a type of counterexample to \ESU{} and in turn an argument for \EAS{}.

In this section we outline two key features of the scenarios we are interested in.
Subsection~\ref{sec:type-scenario} will introduce \gsi{-} to characterise how the agent reasons to the (specific) ability information.
Then, subsection~\ref{sec:ability-entailment} will introduce `\aben{the}' to characterise how the agent reason from the (specific) ability information.
Finally, subsection~\ref{sec:scenarios} will combine \gsi{-} and `\aben{the}' to provide an in-depth understanding of the type of scenarios we are interested in.

\subsubsection{\Gsi{}}
\label{sec:type-scenario}

\begin{note}[Tension, information]
    \begin{restatable}[\gsi{}]{definition}{defGSI}\label{def:gsi}
    \Gsi{-} is information that:\nolinebreak
    \footnote{
      Strictly speaking the formulation of \gsi{} as a conditional isn't important.
      What matters is that the agent is required to claim support for the general ability in order to claim support for the specific ability.
      For example, the conditional may be reformulated as:
      \begin{enumerate}[label=(\gsi{}\('\)), ref=(\gsi{}\('\))]
      \item Either \emph{S} does not have the general ability to \(\gamma\), or the agent has a specific ability to \(\varsigma\).
      \end{enumerate}
    }
    \begin{quote}
      If \emph{S} has a general ability to \(\gamma\), then \emph{S} has a specific ability to \(\varsigma\).
    \end{quote}
    Where \emph{S} is some agent, \(\gamma\) is some general ability, \(\varsigma\) is some specific ability, and it is either implicitly or explicitly stated that \(\varsigma\) is instance of \(\gamma\).
  \end{restatable}
  
  The following pair of examples are instances of \gsi{}.
  \begin{enumerate}[label=(\gsi{}:\arabic*), ref=(\gsi{}:\arabic*)]
  \item\label{qe:cond} If you have the ability to reason with the rules of chess, then you have the ability to demonstrate that, given the arrangement of the board, there is a sequences of moves that will ensure a win for one of the players (as an instance of the general ability to reason with the rules of chess).
  \end{enumerate}

  \begin{enumerate}[label=(\gsi{}:\arabic*), ref=(\gsi{}:\arabic*), resume]
  \item\label{qe:cond:french} If you have the ability to read French, then you have the ability to read The Count of Monte Cristo without a translation (as an instance of the general ability to read French).
  \end{enumerate}
  In both examples an agent is informed that they have the ability to perform a specific act --- demonstrating a strategy or reading a book --- so long as they have some general ability --- an understanding of chess or French literacy --- because the witnessing the specific ability act would be an instance of witnessing the agent's general ability.

  \gsi{} does not directly provide the agent with the information that they have the specific ability.\nolinebreak
  \footnote{Nor (looking ahead to section~\ref{sec:ability-entailment}) does \gsi{} directly provide the agent with information that the result of witnessing the specific ability is when \aben{the} holds with respect to the specific ability.}
  The agent is not informed that they have the general ability and that therefore they have a specific ability.
  To illustrate, I am confident I have the ability to reason with the rules of chess, and so given \ref{qe:cond} I may be confident that I am able to demonstrate the existence of such a strategy.
  By contrast, I do not have the ability to read French, and so I do not have the ability to read The Count of Monte Cristo without a translation.

  Still, I may also be mistaken.
  It may be that I am overconfident, that I do not have the ability to reason with the rules of chess, and hence it may be the case that I do not have the ability to demonstrate the existence of the relevant chess strategy.
  Likewise, I may have the ability to read French, and may have the ability to read The Count of Monte Cristo without a translation.
  However unlikely this may be, I haven't tried to read French in quite some time.
\end{note}

\begin{note}[Not direct]
  \Gsi{} contrasts with what we term `\dsi{-}' --- information that the agent has some ability.
    \begin{restatable}[\dsi{}]{definition}{defDSI}\label{def:dsi}
    \Dsi{-} is information that:
    \begin{quote}
      \emph{S} has the ability to \(\varsigma\).
    \end{quote}
    Where \emph{S} is some agent and \(\varsigma\) is some specific ability.
  \end{restatable}
  For example, the following is a `direct' recreation of~\ref{qe:cond}:

  \begin{enumerate}[label=(\dsi{}:\arabic*), ref=(\dsi{}:\arabic*), series=dsi_count]
  \item\label{qe:cons} You have the ability to demonstrate that there is a sequences of moves that will ensure a win for one of the players as an instance of your general ability to reason with the rules of chess.
  \end{enumerate}

  If~\ref{qe:cons} is true then the agent has the ability to demonstrate some strategy.
  And, in turn,~\ref{qe:cons} expands on why the agent has the relevant specific ability.
  By contrast,~\ref{qe:cond} may be true even if the agent does not have the ability to demonstrate some strategy.
  Hence, \dsi{} is not in general entailed by \gsi{}.\nolinebreak
  \footnote{
    However, if it is the case that an agent has the general ability mentioned in the antecedent of \gsi{}, then a corresponding instance of \dsi{} will be true.
    Note, this is ensured because the consequent of~\ref{qe:cond} ensures the relevant `instance of' relation obtains.
    % So, if I have the ability to reason with the rules of chess and~\ref{qe:cond} is true with respect to me, then \ref{qe:cons} will also be true with respect to me.
  }
\end{note}

\begin{note}[Important features of \gsi{}]
  \gsi{}, then, has two important features:
  \begin{enumerate}
  \item \gsi{} ensures that the agent is on the hook, so to speak, for claiming support they have the specific ability.
  \item If the agent may claim support for having the relevant general ability, then \gsi{} provides the agent with an account of why they may claim support for having some specific ability.
  \end{enumerate}
  Hence, \gsi{} ensure that an agent must themselves claim support that they have some specific ability while providing the agent with relevant information about why they may claim support for having the specific ability.
\end{note}

\begin{note}[Merging \gsi{} and \dsi{}]
  Finally, though we will focus on \gsi{}, there is a variant that merges \gsi{} and \dsi{} which could be substituted for \gsi{} in further discussion.
  This variant involves informing an agent that they have some general ability, and some specific ability as an instance of that general ability, but requires the agent to identify what the general ability is.

  Here is the variant applied to~\ref{qe:cond}.
  \begin{enumerate}[label=(\gsi{}\(^{'}\):\arabic*), ref=(\gsi{}\(^{'}\):\arabic*)]
  \item
    \begin{enumerate}
    \item You have some general ability \(\gamma\), and a specific ability \(\varsigma\) (as an instance of that general ability). And,
    \item If \(\gamma\) is the ability to reason with the rules of chess, then \(\varsigma\) is the ability to demonstrate that, given the arrangement of the board, there is a sequences of moves that will ensure a win for one of the players (as an instance of the general ability)
    \end{enumerate}
  \end{enumerate}
  The agent remains on the hook, so to speak, for claiming support that they have the relevant specific ability because it is up to the agent to identify the general ability \emph{as} the ability to reason with the rules of chess.
  And, likewise, if the agent may claim support for identifying the general ability in a particular way, then the variant allows the agent to claim support that they have a particular specific ability.

  We favour \gsi{} given it's comparative structural simplicity, but the variant highlights that that the agent claiming support for having some specific ability is not of interest.
  Rather, what is interest is that \gsi{} allows the agent to claim support for the particulars of some specific ability.

  In section~\ref{sec:ability-entailment} we will highlight why the particulars matter.
\end{note}

\subsubsection{An ability entailment}
\label{sec:ability-entailment}

\begin{note}[\aben{(The)}]
  The second component in scenarios of interest is the availability of an entailment from the specific ability.

  We term an instance of the entailment as an `\aben{}'.

  \begin{restatable}[Ability entailment]{definition}{defAE}\label{def:aben}
    \aben{The} is any entailment of the form:
    \begin{quote}
      \emph{S} has the (specific) ability to \emph{V} that \(\phi\) \emph{therefore} \(\phi\) is the case.
    \end{quote}
    Where \emph{S} is an agent, \emph{V} is some action, and \(\phi\) is some proposition.
  \end{restatable}

  The rough intuition behind instances of \aben{the} is that \(\phi\) being the case does not depend on \emph{S} witnessing the (specific) ability to \emph{V} that \(\phi\).
  So, \aben{the} links ability and something that must be the case in order to have ability and the result of witnessing ability must be the case in order for the agent to have the ability

  For example, \aben{the} holds with respect to the (specific) ability to demonstrate the existence of a chess strategy from \ref{qe:cond} as whether or not a given chess strategy exists depends on the moves permitted by the rules of chess --- a strategy that has not been demonstrated is a strategy.
  Likewise, \emph{S} has the (specific) ability to discover that their keys are in their jacket pocket only if it is the case that their keys are in their jacket pocket --- whether or not \emph{S}'s keys are in their jacket pocket does not depend on \emph{S} discovering that to be the case.

  By contrast, `to read The Count of Monte Cristo without a translation' is an action and so \aben{the} does not apply to the specific ability of~\ref{qe:cond:french}.
  Even so, \aben{the} apply to nearby variants and not others.
  \emph{S} may have the specific ability to read that Dantès was a merchant sailor, and it follows that Dantès was a merchant sailor.
  In contrast, while \emph{S} may have the ability to believe that certain passages cannot be adequately translated, it does not follow that those passages cannot be adequately translated.
  Similarly, \emph{S} may have the ability to hope that they will employ the chess strategy discovered in a competitive game, but it does not follow that \emph{S} will employ the strategy.

  More broadly, \aben{the} holds with respect to factive verbs, such as `see', `know', `understand', and so on.
  Though, I doubt factive verbs are an adequate explanation for \aben{the}.
  Consider `read'.
  I have the ability to read that Elvis Presley was born in 1935, but I also have the ability to read that Elvis is working undercover for the DEA.
  What matters, then, is not the verb used, but how the agent would witness the relevant ability.
  I have the ability to read that Elvis was born in 1935 from a reliable source, and hence \aben{the} applies.
  The same is not true for my ability to read that Elvis is working for the DEA.

  Indeed, \aben{the} merely identifies an entailment.
  It does not provide an account of when or why such entailments hold.
  We identify entailments of this type because our interest is in how (in certain cases) agent's reason with instances of \aben{the}.
\end{note}

\subsubsection{Details of scenarios}
\label{sec:scenarios}

\begin{note}[Both things are important]
  The scenarios we are interested in combine \gsi{} with \aben{the}.

  The role of \gsi{} is to ensure that the agent is not provided with direct information about specific ability.
  And the role of \aben{the} is to highlight that the agent is in a position to claim support for some further proposition if they claim support for specific ability.
  Hence, scenarios combine claiming support \emph{for} specific ability and claiming support \emph{from} specific ability.

  To illustrate, consider the following pattern of reasoning:
  \begin{enumerate}[label=\arabic*., ref=(\arabic*)]
  \item\label{scen:rp:1} I have the general ability to reason with the rules of chess.
  \item\label{scen:rp:2} I received \gsi{} information that if they have the general ability to reason with the rules of chess then they have the ability to demonstrate the existence of some strategy.
  \item\label{scen:rp:3} So, from~\ref{scen:rp:1} and~\ref{scen:rp:2} it follows that I have the ability to demonstrate the existence of some strategy.
  \item\label{scen:rp:4} And, as \aben{the} hold with respect to~\ref{scen:rp:3}, the relevant strategy exists.
  \end{enumerate}
  I reason to (\ref{scen:rp:1} --- \ref{scen:rp:3}) and from (\ref{scen:rp:3} --- \ref{scen:rp:4}) a specific ability.
  The reasoning pattern seems sound.
  And, at no point do I need to witness their general ability to reason with the rules of chess, or the specific application of the general ability to demonstrate the existence of the strategy.
\end{note}

\begin{note}
  Both components are important.
  Focus on \gsi{} will restrict the interpretation of what the agent claims support for.
  And, in turn, what the agent has claimed support for will determine what the agent appeals to when appealing to \aben{the} entailment.\nolinebreak
  \footnote{
    I suspect it may be possible to focus only on \gsi{}.
    As we will see, this is where the key step of the argument takes place.
    However, this is not trivial.
    Would require more focus on how the agent gets to specific from general.
    By splitting in this way, we avoid details.
    Instead, focus on what it is that the agent gets, and then \aben{the} is forced to work with this.
  }
  \gsi{} and \aben{the} combine to provide a (partial) functional characterisation of reasoning with specific ability.
\end{note}

\begin{note}
  Note, however, that there is a distinction between how an agent reasons about ability, and what ability is.
  We are interested in how agent's reason about (specific) ability, and not what makes it true that an agent has a (specific) ability.
  Our focus will shortly turn to how to interpret (specific) ability when appealed to in the type of scenario described.
  We will outline two general schematic interpretations of ability, argue that these are exhaustive, and note how general constraints such as \ESU{} constrain which interpretation is available.
\end{note}

\begin{note}[Scenario proposition]
  For ease of reference, we wrap scenarios involving the limited information as a proposition.
    \begin{restatable}[\eA{-} --- \eA{}]{proposition}{propScenariosExist}\label{prop:SE}
    There are scenarios in which an agent \emph{S} receives \gsi{} information of the form:
    % \mbox{ }\vspace{5pt}
    \begin{center}
      If \emph{S} has a general ability to \(\gamma\), then \emph{S} has a specific ability to \emph{V} that \(\phi\).
    \end{center}
    % \mbox{ }\vspace{5pt}

    \noindent Such that \aben{the} applies to the specific ability to \emph{V} that \(\phi\).

    In turn:
    \begin{enumerate}
    \item \emph{S} may reason from claimed support that they have the general ability to \(\gamma\) in order to claim support for having the specific ability to \emph{V} that \(\phi\). And,
    \item \emph{S} may reason from their claimed support that they have the ability to \emph{V} that \(\phi\) to claim support that \(\phi\) is the case by appealing to \aben{the}.
    \end{enumerate}
    \vspace{-\baselineskip}
  \end{restatable}
\end{note}

\begin{note}[Possible restrictions]
  First, \eA{} holds only that there are cases in which the agent may appeal to ability to obtain support.
  It is therefore consistent with~\eA{} that there are cases in which the details of the cases outlined are satisfied, but where kind of support is unsuitable for certain purposes.
  For example, some witness of ability may be demanded by a third-party.
  In this respect, the content of \eA{} is similar to an analogous claim with respect to memory.
  If an agent remembers proving that \(\phi\), then \(\phi\) is the case.
  Still, one may still request that an agent provides you with a proof of \(\phi\) in order to for you to be satisfied that \(\phi\) is the case --- many exams are like this.
  So, that an agent may not always and in any context claim support for \(\phi\) from claimed support for their ability to \emph{V} that \(\phi\) is not an objection to~\eA{}.
\end{note}

\begin{note}
  Second, \eA{} does not require that an agent reason in the way described given \gsi{} and availability of \aben{the}.

  For example, the following statement is an instance of \gsi{}:
  \begin{enumerate}
  \item Any person who has the (general) ability to reason with the rules of chess has the (specific) ability to identify Alekhine's Defense as a fine opening move.
  \end{enumerate}
  The universal quantifier implies that the statement is true with respect to me, among others.
  Still, I am confident that there is at least one other person who has the ability to reason with the rules of chess, and may therefore infer that Alekhine's Defense as a fine opening move without appealing to my own ability.
  Indeed, if I am inclined to doubt my own (general) ability in contrast to a Grandmaster, then I may be more confident that Alekhine's Defense as a fine opening move if I appeal to the existence of a Grandmaster.

  Again, it is consistent with \eA{} that an agent may reason in such a way.
  Still, in defence of \eA{} it is important to note that \gsi{} information may be limited to the agent in question.
  For example, I may have studied your notes on how to play chess and identified a strategy which follows from those notes.
  I have no doubt that you have the ability to identify the same strategy, so when I provide \gsi{} my emphasis is on whether you have the ability to reason with \emph{chess}, rather than some closely related game.

  There are many ways to build context so that an agents is required to reason with \gsi{} and \aben{the} if the agent is to reason with (specific) ability at all, but I doubt these are required.
  The reasoning described by \eA{} (and illustrated above) seems plain and permissible.
\end{note}

\begin{note}
  Finally, \gsi{} and \aben{the} are constraints which do not hold in all cases of reasoning with specific ability.

  For example, one may be told that a gift of a metal detector grants the ability to discover if there is buried treasure in the garden.
  The former does not entail that there is buried treasure in the garden, and testimony or the metal detector may be claimed as support for the ability.

  So, question about whether this really does anything for general understanding of ability.
  \gsi{} and \aben{the} combine to require a particular interpretation.
  However, interpretation with general applicability is not restricted to instances in which it is forced.
  The role of a counterexample is not (typically) to establish that every instance of a theory is mistaken, but to identify a gap.
  And, even if the original theory may be restricted to non-problematic cases, the alternative theory may compete with the original theory.
  So, given that the particular interpretation is required to hold given additional stipulations, interest is in whether it holds without additional stipulations.
\end{note}

\subsubsection{Reasoning in the scenarios}
\label{sec:reasoning-scenarios}

\begin{note}
  {
    \color{red}
    Two different ways of reasoning here.
    Either from specific to \(\phi\) or from general to \(\phi\).
  }
\end{note}

\begin{note}
  With general ability, this is a `derived ability entailment'.
\end{note}

\subsection{Two (schematic) interpretations of (specific) ability}
\label{sec:wr-ar}

\begin{note}
  In the previous section we introduced \gsi{-} and \aben{the}.
  In the present section we motivate two interpretations of (specific) ability in the context of reasoning to (specific) ability from \gsi{} and reasoning from (specific) ability with \aben{the}.

  The two interpretations are termed `\AR{}' and `\WR{}' in turn, and are schematic.
  Roughly:
  \AR{} holds that when appealing to (specific) ability an agent appeals to some property or attribute that they have.
  And, by contrast, \WR{} holds that when appealing to (specific) ability an agent appeals to the action that they would perform by witnessing the relevant ability.
  \AR{} and \WR{} are distinguished, then, by whether an agent reasons with a property (\AR{}) or an event (\WR{}).

  To illustrate by analogy, consider a mechanical clock.
  The clock has the property of displaying the correct time, by it is also involved in the event of changing it's configuration as time passes.
  The property that the clock is displaying the correct time is important for determining whether one is late for a meeting.
  By contrast, the event of changing it's configuration as time passes is important for determining when to remove a brewing teabag.
  A meeting starts at a certain point in time, while tea is brewed over a period of time.
  If the clock does not represent the correct time, then three minutes passing will not, in general, help determine whether one is late to the meeting.
  And, whether or not it is 3pm is not, in general, important with respect to whether or not the tea has finished brewing.
  The qualifier `in general' is important.
  Measuring the passage of is useful if I know the length of time before the meeting is due, and the correct time is useful if I know when I started brewing the tea.

  The distinction between \AR{} and \WR{} is similar.
  Both interpretations may be more or less useful in certain circumstances, and interchangeable in others.
  Still, the combination of \gsi{} and \aben{the} identify a pattern of reasoning in which we may elaborate how the relevant interpretation of (specific) ability is important, and in turn broader principles (\ESU{} and, to be introduced below, \nI{}) will constrain whether the interpretations are permissible.

  {
    \color{red}
    Outline of subsections.
  }
\end{note}

\begin{note}[Table]
  \begin{figure}[H]
    \centering
    \begin{tblr}{abovesep=8pt, belowsep=8pt, width=0.95\textwidth, colspec={Q[c,m]|Q[c,m]|Q[1.8,c,m]|Q[1.8,c,m]}}
      \multicolumn{2}{c}{} & \adA{} & \adB{} \\
      \hline
      \multicolumn{2}{c}{\WR{}} & ? & ? \\
      \hline
      \multirow{2}{*}{\AR{}} & Basic & ? & ? \\
      \cline[dashed]{2-4}
      & Derived & ?  & ? \\
    \end{tblr}
    \caption{Distinction matrix for interpretations of \aben{the}. \\ Rows are interpretations of ability, columns are type of reasoning regarding ability.}
  \end{figure}
\end{note}


\subsection{\AR{} and \WR{}}
\label{sec:ar-wr-1}

\begin{note}[\WR{} and \AR{}]
  We term the two schematic interpretations of \aben{the} `\AR{}' and `\WR{}', respectively.
  Brief descriptions from detached perspective.
  Given that the interpretations are schematic, they fall short of a full account of how an agent claims support by \aben{an}.
  However, the arguments to follow are of interest in part because they concern any way in which the schematic interpretations are filled out.
\end{note}

{
  \color{red}
  I should emphasise that here we're interested in reasoning.

  Also, the distinction is important to ensure that the argument's don't depend on a specific reading of ability.
}

\subsubsection{\AR{}}
\label{sec:ar-1}

\begin{note}
  \begin{restatable}[\AR{}]{definition}{defAttribution}\label{AR:def}
    An agent's reasoning with an instance of \aben{the} by claiming support for \(\phi\) from \emph{S} having ability to \emph{V} that \(\phi\) is an instance of \emph{\AR{}} when the agent holds that:

    \emph{S} has the ability to \emph{V} that \(\phi\)
    \begin{enumerate*}[label=(\textsf{A}\arabic*), ref=(\textsf{A}\arabic*)]
    \item\label{A:s:1} is or reduces to some (possibly complex) property \emph{P} of \emph{S}, and
    \item\label{A:s:2} \emph{P}, or some part of \emph{P}, entails that \(\phi\) is the case.\nolinebreak
      \footnote{Intuitively, because the agent could not have \emph{P} without \(\phi\) already being the case.
      The notion of entailment here does not require that \(\phi\) is true \emph{because} of \emph{P}.}
    \end{enumerate*}
  \end{restatable}
  
  {
    \color{red}
    \AR{} identifies instances of reasoning in which an agent applies \aben{the} by holding the ability to \emph{V} that \(\phi\) is a property of an agent.\nolinebreak
    \footnote{
      Note, this does not say anything about what the ability to \(\phi\) is.
      Rather, way in which the agent claims support.
    }
    Note, when appealing to \aben{the} an agent need not be aware of what the (potentially complex) property of \emph{S} is.
    Rather, claimed support that \emph{S} has the ability to \emph{V} that \(\phi\) allows the agent to claim support for the existence of some property of \emph{S} which in turn entails \(\phi\).
  }

  Now, generally speaking properties are things which may be predicated or attributed of other things.
  The coffee is hot, I am thirsty, my mouth is sensitive to heat, I am reckless, I am in pain, and so on\dots
  And, properties come cheap.
  For example, the participation of an agent in some event gives rise to a property that may be attributed to the agent.
  Specifically, the property of participating in the event.
  Moments ago I participated in the event of recklessly drinking hot coffee with a mouth that is sensitive to heat.
  Therefore, I have the property of participating in such an event.

  So,~\ref{A:s:1} is trivially true.
  When we speak of an agent having some ability we are predicating or attributing ability to an agent.
  However,~\ref{A:s:2} requires that the property entails that \(\phi\) is the case.
  And, it is not clear that an entailment which follows from an event is always reflected in the property of being a participant in the event.
  For example, it seems that I am in pain because I participated in the event of drinking hot coffee, \emph{not} because I have the property of having participated in the event of drinking hot coffee.
  By contrast, that I have the property of having participated in the event of drinking hot coffee entails that I have the property of having participated in the event of drinking something.

  % From~\ref{A:s:2} it must be the case that the relevant property entails \(\phi\).
  % And, from~\ref{A:s:3} the property must not analysed in terms of there being a potential event in which \emph{S} witnesses the act of \emph{V}ing that \(\phi\).
  % This is, from one perspective, an arbitrary restriction.
  % For example, if there is a potential event in which an agent witnesses the act of \emph{V}ing that \(\phi\), then the agent has the property of being a participant of that potential event.
  % From a different perspective,~\ref{A:s:3}

  Roughly, we may expect the property of interest is akin to having a heart, possessing ¥500, being of a certain age, and so on\dots

  {
    \color{red}
    Key idea with \AR{} is that the agent `directly' claims support for a property when using \aben{the}.
  }

  To illustrate \AR{} we focus on the idea of reducing the ability to \emph{V} that \(\phi\) to some (potentially complex) property of \emph{S}.
  Again, when appealing to \aben{the} an agent need not be aware of what the (potentially complex) property of \emph{S} is.
  Rather, these illustrations suggest that such properties exist.

  \begin{illustration}
    Consider the proposition that \emph{S} has the ability to hear that the birds are signing.
    Again, it seems \aben{the} holds, and one may infer that birds are singing.

    So, by \AR{} there is some (complex) property \emph{P} of \emph{S} such that \emph{P}, or some part of \emph{P}, entails the the birds are signing.

    Consider the complex property of a well-functioning auditory system and sufficient proximity to the birds singing.
    The property of having well-functioning auditory system ensures that \emph{S} has the ability to hear nearby noises.
    And, having well-functioning auditory system together sufficient proximity to the birds singing together ensure that \emph{S} has the ability to hear the nearby noise of the birds singing.

    \aben{the} follows from part of this complex property.
    If the agent has the property of being in sufficient proximity to the birds singing, then it follows that there are birds singing.
  \end{illustration}

  \begin{illustration}
    Consider the proposition that the prosecution has the ability to demonstrate that the defendant is guilty.
    Intuitively, \aben{the} holds, as it is not possible to demonstrate the guilt of an innocent person.\nolinebreak
    \footnote{
      It is a different matter to convince a jury of the guilt of an innocent person.
      And, \aben{the} does not seem to hold with respect to the ability to convince a jury that the defendant is guilty.
    }
    By \AR{} there is some (complex) property \emph{P} of the lawyer such that \emph{P}, or some part of \emph{P}, entails the guilt of the defendant.
    Say, the lawyer is in possession of evidence sufficient to establish guilt of the defendant.
    If so, it is a property of the lawyer that they are in possession of such evidence, and by assumption the evidence entails that the defendant is guilty.

    It seems possession of evidence alone may not be sufficient to establish that the lawyer has the ability to prove that the defendant is guilty.
    For, it is plausible that a lawyer may be in possession of evidence that they do not understand.
    However, as our interest is with \aben{the} it is sufficient to observe that the evidence alone entails the guilt of the defendant.
  \end{illustration}

  Again, these illustrations highlight ways in which \emph{S} having the ability to \emph{V} that \(\phi\) may be reduced to some (complex) property of \emph{S}.
  \AR{} does not hold that an agent identifies such a property when claimed support by an instance of \aben{the}.
  Rather, \AR{} holds that the agent reasons with ability as a property of the agent.
  Indeed, while these suggestions reduce ability to complex properties, \AR{} also admits of the possibility that the ability to \emph{V} that \(\phi\) is a basic property which does not admit of further analysis.
  If so, then it seems that \aben{the} must also be taken as basic.\nolinebreak
  \footnote{
    I lack any suggestion for how to understand \AR{} if the property is indeed basic, but there is no need to rule out this option ---  no part of the following arguments depend on whether or how these schemas may be substantiated.
  }
  So, to summarise.
  The distinguishing feature of \AR{} is that there are instances when an agent claims support for \(\phi\) from claimed support that \emph{S} has the ability to \emph{V} that \(\phi\) because the latter ensures that there is some property \emph{P} holds of \emph{S} and \emph{P} entails \(\phi\).
  If the agent has the ability to \emph{V} that \(\phi\), then there may also be some action, \emph{V}ing, that the agent may witness.
  However, as \AR{} appeals to some property, the witnessing event is irrelevant to the way in which the agent claims support for \(\phi\).
\end{note}

\begin{note}
  {
    \color{red}
    Some additional notes on \AR{} that haven't been merged with the above follow.
  }
\end{note}

\begin{note}[Compatibility]
  \AR{} suggests an alternative way to obtain support for the conclusion of reasoning the agent is able to do.
  Specifically, if order for the agent to \emph{have} the attribute of being able to reason to the conclusion, the conclusion of the reasoning must be true.
  The relevant entailment is in part secured by the verb chosen, and in part by what the verb is applied to.
  Here, `demonstrate' is a factive verb, if an agent demonstrates that \(\phi\), then it is true that \(\phi\).
  And, the existence of a chess strategy does not depend on the agent demonstrating that the relevant strategy exists.

  To take another example, you only have the ability to identify a typo on this page if there is a typo on this page.
  So, if I were to provide you with testimony that you have the ability to identify a typo on this page, you may begin searching for the typo, or you may note that there must be a typo in order for me to be in a position to provide you with testimony that you have the ability.
\end{note}

\begin{note}[Sketch of \AR{}]
  \begin{enumerate}[label=(\textsf{A}\arabic*), ref=(\textsf{A}\arabic*)]
  \item\label{AR:Sketch:1} I have the attribute of being able to \emph{V} that \(\phi\).
  \item\label{AR:Sketch:2} In order to have the attribute of being able to \emph{V} that \(\phi\), \(\phi\) must be the case independent of whether or not I witness the ability.
  \item\label{AR:Sketch:3} \(\phi\) is the case.
  \end{enumerate}

  To keep things simple, we will refer to the principle behind the pattern sketched as \AR{}.
  And agent may bundle~\ref{AR:Sketch:1} and~\ref{AR:Sketch:3} into a conditional, and avoid instantiating the reasoning pattern, but so long as the conditional is (implicitly) held on the basis of the intermediate premise~\ref{AR:Sketch:2}, we take use of such a conditional to be an instance of \AR{}.
\end{note}


\subsubsection{\WR{}}
\label{sec:wr-1}

\begin{note}[\WR{} def.]
  {
    \color{red}
    Include: observation that the entailment may come from some property of the agent.
    The point of \WR{} is that the agent claims support for details of the event.
  }

  We now turn to \WR{}.
  \begin{restatable}[\WR{}]{definition}{defWitnessing}\label{WR:def}
        An agent's reasoning with an instance of \aben{the} by claiming support for \(\phi\) from \emph{S} having ability to \emph{V} that \(\phi\) is an instance of \emph{\WR{}} when the agent holds that:
    \begin{enumerate}
    \item\label{WR:def:1} \emph{S} has the ability to \emph{V} that \(\phi\) \emph{if and only if} there is a potential event in which \emph{S} witnesses the act of \emph{V}ing that \(\phi\).
    \item\label{WR:def:2} Claim support for event or details of event.
    \item\label{WR:def:3} Details of the event in which \emph{S} witnesses the act of \emph{V}ing that \(\phi\), or part of the event, entails that \(\phi\) is the case.\nolinebreak
      \footnote{Again, intuitively, because there could not be a potential event in which \emph{S} witnesses the act of \emph{V}ing that \(\phi\) without \(\phi\) already being the case.
      The notion of entailment here does not require that \(\phi\) is true \emph{because} there is some potential event of the relevant kind.}
    \end{enumerate}
  \end{restatable}

  {
    \color{red}
    ~\textcite{Rebuschi:2011ub} talk about \emph{de objecto} attitudes.
    This might be helpful given that the events are potential.
  }

  {
    \color{red}
    Key idea with \WR{} is that the agent appeals to certain things which follow from the event being witnessed.
    Whereas, \AR{} appeals to certain things which must be the case in order for the event to be witnessed.
  }

  {
    \color{red}
    Difference between the existence of an event (~\ref{WR:def:1}) and details of the event (~\ref{WR:def:2}).
    To clarify.
    \(\exists e(V(e) \land \text{agent} = \emph{S} \dots)\).
    \(\phi\) follows.
    However, there are two ways to think about this.
    First, the existential, second the event.
    \emph{De dicto} and \emph{de re}.
    \WR{} is \emph{de re}.

    Consider existential of individuals.
  }

  {
    \color{green}
    Before going into the details, it'll be helpful to highlight the big idea, especially with respect to how things (will) work out with the `master property' from \AR{}.
  }

  \WR{} identifies instances of reasoning in which an agent applies \aben{the} by holding that \emph{S} having the ability to \emph{V} that \(\phi\) ensures there is a possible event in which \emph{S} \emph{V}s that \(\phi\).
  And, in turn, there is a possible event in which \emph{S} \emph{V}s that \(\phi\) entails that \(\phi\) is the case.
  In contrast to \AR{}, when an agent claims support as an instance of \WR{} an agent reasons about what must be the case in order for \emph{S} to witness some ability, rather than what must be the case in order for \emph{S} to have the property of possessing some ability.


  We use the term `potential' in place of `possible' when describing the relevant event to highlight that the existence of the event is tied to an ability attribution.
  One may hold that a possible event is any event which is not impossible, and hence it is possible for an arbitrary agent to prove Fermat's Last Theorem.
  Yet, it seems most agent's lack the ability to prove Fermat's Last Theorem, and so `potential' serves to restrict quantifier over events which an agent has the ability to witness --- however the details of that quantification are resolved.

  {
    \color{red}
    \WR{} is more complex than \AR{}.
    There is some action that \emph{S} may witness.
    And, understand what the result of that action is.
    So, we have something akin to a counterfactual.
    However, the counterfactual only relies on witnessing.
    Further, particular status of \(\phi\).
    Hence, as witnessing is the only issue, \(\phi\) is the case.

    Third, regardless.
    \(\phi\) holds regardless, but it does not follow from this that if the agent reasons via \WR{} then support claimed for \(\phi\) would be independent of ability information.
    The agent must recognise that \(\phi\) must be the case regardless, but this doesn't require that the agent has any way of reasoning to \(\phi\) other than by witnessing their ability.
    The point is clearer when considering witnessed instances of reasoning.
    \emph{X} testified that \emph{p}.
    Claim support for \emph{p}.
    \emph{p} is not the case because \emph{X} testified that \emph{p}, though my only path to claim support is by appeal to the testimony of \emph{X}.
  }
  To illustrate.

  \begin{illustration}
    I have the ability to calculate that \(243 \div 3 = 82\).
    Pen and paper to hand, etc.\
    Result of this will be a calculation that \(243 \div 3 = 82\).
    However, my calculation is irrelevant to whether it is the case that \(243 \div 3 = 82\).
    Hence, it follows that \(243 \div 3 = 82\).
  \end{illustration}

  \begin{illustration}
    Ability to discover that the ball is under the left cup.
    Raise the left cup, and identify the ball.
    Whether or not the ball is under the left cup is independent of this sequence of actions, and therefore it follows that the ball is under the left cup.
  \end{illustration}

  Compare to cases where only gets the counterfactual.
  I have the ability to make it so that the heating is turned out.
  Plausibly, the heating is not on, and depends on witnessing the action of `making it so'.
\end{note}

\begin{note}[`Available resources']
  Delicate.
  Focus is on the witnessing event.
  However, mere possibility isn't sufficient for \aben{the}.
  So, some restriction.
  That is, an account of what makes the witnessing event a \emph{potential} event rather than a \emph{possible} event.
  One way to express this idea is that included in appeal to potential witnessing event is that sufficient resources are available.
  Here, the idea is that nothing further is required for the event to take place.

  This redescription falls short of an analysis as we've shifted the work from `potential' to `available'.
  Still, room for an analogy.
  Consider running a 5K.
  Here, going to require a whole bunch of energy.
  The agent does not `have' the energy.
  However, resources to generate energy.
  Fat reserves, muscle density, and so on.
  In this sense, sufficient resources are available, but not something the agent has.

  \AR{}, whatever it is that generates the sufficient resources.
  \WR{}, the result of having generated the sufficient resources.

  So, the difference between \AR{} and \WR{} isn't necessarily with what the two interpretations reduce down to, but is rather a difference with respect to what the interpretations focus on.
  From \AR{}, the stuff that's true right now, the generator, does the work.
  From \WR{}, it's what will be generated.

  There's still an important difference, though.
  Our interest is in reasoning.
  We are interested in what the agent appeals to.

  Key difference.
  \AR{}, that there is stuff the agent has which will generate.
  \WR{}, that what is generated from the stuff the agent has will do the work.

  The impact of this distinction will be expanded up with respect to \gsi{}.
\end{note}


\subsubsection{Contrasting \AR{} and \WR{}}
\label{sec:contrasting-ar-wr}

\begin{note}[Difference]
  With \AR{} the important thing is some (possibly complex) property.
  With \WR{} the important thing is the witness.

  \[\text{Has}(S,\text{docs}) \land \text{Sufficient-to-show-guilt-of-defender}(\text{docs})\]

  Ability here is to ensure that there is some property of this kind.

  \[\exists e(\text{Calculating}(e) \land \text{agent}(e) = S \land \text{result}(e) = (243 \div 3 = 8))\]


  \AR{} focuses on whether something is true of the agent independent of what action they may perform.
  \(\phi\) follows from property.
  \WR{} focuses on an action the agent may perform.
  \(\phi\) follows from relation between \(\phi\) and possible witness of action.
\end{note}

\begin{note}
  Given \AR{}, conjecture that ability is not important for the entailment if complex.
  A useful shorthand, but in principle do not need to highlight the act.
  In contrast, the act is required for \WR{}.

  Still, agent is only given information about ability, so this will remain important for reasoning.
\end{note}

\begin{note}[Plausible equivalence]
  Generally speaking, switch between the two.
  Ball under cup.
  Hear that the birds are singing.
\end{note}

\begin{note}
  Primary purpose of the distinction is to ensure that things apply to any understanding of reasoning with \aben{the}.
  Further contrast after following distinction.
\end{note}

\subsection{\adA{}, \adB{}, and Ability}
\label{sec:ability-ads-adc}

\begin{note}[Recall\dots]
  Let us briefly summarise our progress so far.

  In~\autoref{sec:cases-interest} we introduced particular instances of an agent claiming support for some conclusion which involved two key steps.
  The reasoning, in outline:
  \begin{enumerate}[label=\arabic*., ref=(\arabic*)]
  \item\label{NUR:ro:i} I have some general ability \(\gamma\).
  \item\label{NUR:ro:ii} If I have general ability \(\gamma\) then I have some specific ability \(\varsigma\) to \emph{V} that \(\phi\).
  \item\label{NUR:ro:iii} I have the specific ability \(\varsigma\) to \emph{V} that \(\phi\). \hfill (From~\ref{NUR:ro:i} and~\ref{NUR:ro:ii})
  \item\label{NUR:ro:iv} It is only possible to \emph{V} that \(\phi\) if \(\phi\) is already the case.
  \item\label{NUR:ro:v} \(\phi\) is the case. \hfill (From~\ref{NUR:ro:iii} and~\ref{NUR:ro:iv})
  \end{enumerate}

  The reasoning involves claiming support by two important steps.

  \begin{itemize}
  \item The first step, from~\ref{NUR:ro:i} and~\ref{NUR:ro:ii} to~\ref{NUR:ro:iii}, involves the conditional of~\ref{NUR:ro:ii}, termed `\gsi{-}', clarified in~\autoref{sec:type-scenario}. And,
  \item The second step, from~\ref{NUR:ro:iii} and~\ref{NUR:ro:iv} to~\ref{NUR:ro:v}, is an instance of `\aben{an}', clarified in~\autoref{sec:ability-entailment}.
  \end{itemize}

  Both steps involve reasoning, and in particular claiming support, by appeal to ability.
  The first step, claiming support from a specific ability from a general ability.
  The second step, claiming support for some proposition from specific ability.

  Issue is how the agent claims support.
  In turn, how an agent reasons with ability.
  The sketch captures key premises and steps, but does not provide an interpretation of those steps.

  In~\autoref{sec:wr-ar} we introduced two (schematic) interpretations of specific ability --- \AR{} and \WR{}.
  A few purposes for these (schematic) interpretations.
  First, some insight into how an agent may claim support.
  \AR{} some property, \WR{} some event.
  No stance on these.
  Distinction on one hand allows us to state in greater detail, and on the other hand ensures that the arguments to follow do not presuppose a particular (schematic) interpretation of ability.

  Now, two steps under either \AR{} or \WR{} may look straightforward.
  Both involve conditionals, so matter of something like \emph{modus ponens}.
  In this respect, distinction between \AR{} and \WR{} matters only for finer details of how the agent claims support, rather than the reasoning involved.
  `Something like \emph{modus ponens}' sufficiently similar so that it's the conditional that's doing the work.
  In the sense that the conditional form does most of the work.
  \AR{} and \WR{} why it's claiming support as opposed to some other kind, say subjunctive or suppositional, reasoning.

  However, this is not immediate.
  The sketch does not provide an interpretation of those steps.
  The purpose of this section is to outline an alternative way of claiming support that may be applied to the sketch.
  Key part in argument against \ESU{} (and for \EAS{}).

  Start with illustration.
  Then, definition.
  Applied to further illustrations.
  Applied to \AR{} and \WR{}.


  Distinction between reasoning \adA{} and \adB{} is of interest to use with respect to these two instances of claiming support in particular.

  So, pair \AR{} and \WR{} with \adA{} and \adB{} and we have various ways of understanding how agent claim support.

  Keep in mind that here we're elaborating on what this sketch amounts to.
\end{note}

\paragraph{Initial illustrations}

\begin{note}[What we're going to look at]

  {\color{red} footnote}\nolinebreak
  \footnote{
    Here, as with other examples, focus on existential, as this is relevant.
    However, question about semantic counterpart.
    Every model, or there does not exist a model.
    In contrast to existential, pointing to some specific thing won't do.
    Still, may extend to composite properties of any model.
    Either \(\phi\) or \(\lnot\phi\).
    So, \(\psi\), or trivial.
  }

  \begin{enumerate}[label=\named{\(\exists\mathord{\vdash}{,}\top\)}, ref=\named{\(\exists\mathord{\vdash}{,}\top\)}]
  \item\label{ill:Eproof:def} A syntactic proof of a formula (using a sound first-order system) is sufficient to establish the formula is a (syntactic) theorem of first-order logic.
  \end{enumerate}
\end{note}

\begin{note}[Memory]
  \begin{illustration}\label{ill:ad:proof:mem}
    \begin{enumerate}
    \item\label{ill:Eproof:mem} I remember having created a syntactic proof of \formula{\forall x Px \rightarrow \lnot \exists x \lnot P x} (using a sound first-order system).
    \item\label{ill:Eproof:exP} So, there exists a syntactic proof of \formula{\forall x Px \rightarrow \lnot \exists x \lnot P x} (using a sound first-order system)
    \item\label{ill:Eproof:thm} Hence, by \ref{ill:Eproof:def}, \formula{\forall x Px \rightarrow \lnot \exists x \lnot P x} is a theorem of first-order logic.
    \end{enumerate}
  \end{illustration}

  \autoref{ill:ad:proof:mem} seems a straightforward case of claiming support.\nolinebreak
    \footnote{
      Looking ahead, one may concerned that \nI{} rules out the agent claiming support in the way outlined by \ref{ill:ad:proof:mem}.

      For, \ref{nI:claimed-support} is be satisfied by taking \(\phi\) to be the proof, and \(\psi\) to be theorem-hood.
      For sure, requires collapsing steps~\ref{ill:ad:proof:eve} and~\ref{ill:ad:proof:eve:app} into a complex conditional so \nI{} does not apply to the specific reasoning.
      Still, at issue is whether the spirit of \nI{} applies, rather than the letter --- reasoning with the derived conditional doesn't seem to change much.
      And, given the derived conditional, the agent's reasoning would fit the pattern described by \ref{nI:going-by-value}.

      So, at issue is whether \ref{nI:inclusion:position} and~\ref{nI:inclusion:bound} hold with respect to the reasoning.
      In short, is the agent confident their claimed support for their memory of the proof to be \mom{} if they are not in position to claim support for theorem-hood without appealing to their memory?

      This may be the case.
      One may only trust their memory of proving if they consider themselves to be in position to (re)create the proof, as failure would suggest that failed to create an adequate proof in the past --- e.g.\ the agent may have created a proof as a student and is now (re)considering whether the formula is a theorem as an expert in the field.

      Of course, this may also not be the case, but the worry is that the reasoning seems fine regardless of how additional details are added, so long as the additional details do not conflict with the claimed support.

      So, our attention turns to~\ref{nI:inclusion:bound}.
      In short, is the agent is confident that they would be \nmom{} when claiming support for theorem-hood if their memory is \nmom{}?

      Here, any worry eases.
      Memory of creating the proof seems quite independent of whether the agent would be successful if they were to attempt to (re)create the proof.
      For, the memory is (merely) of the existence of the proof, rather than the details of the syntactic system, and so on.

      In particular, it seems clear that the agent would not need to assume that the formula is a theorem prior to attempting to recreate the proof.
    }

  First, the agent has claimed support for \ref{ill:Eproof:def} by their familiarity with systems of first order logic.

  Second, the agent remembers having created a syntactic proof of the relevant formula.
  And, it seems sufficient, generally speaking, to claim support for some proposition by appealing to memory, hence the agent claims support that there was some event which culminated in a syntactic proof of the formula.\nolinebreak
  \footnote{
    It may be more natural to say `I remember creating\dots' or `I remember proving\dots'.
    The particular phrasing is chosen to remove any ambiguity about whether the agent \emph{finished} the activity creating or proving.
  }
  Of course, the agent may have misremembered, but there seems no issue with the agent expecting that appeal to their memory is \nmom{}.

  Following, this allows the agent to claim support that a syntactic proof of the formula exists.
  As before, the agent may have been \mom{} about whether what they created really was a syntactic proof of the formula
  And, as before it seems the agent may expect that they were not \mom{}.

  Hence, finally, the agent claims support that the formula is a (syntactic) theorem of first-order logic.

  To concisely summarise, we may say that the agent claimed support for the formula being a (syntactic) theorem of first-order logic \emph{because} of their understanding of syntactic theorem-hood and their memory of proving the formula.

  For sure,~\autoref{ill:ad:proof:mem} is designed to be as straightforward as possible.
  Of interest is not whether the agent claims support, but how the role the agent gives to their memory in claiming support.

  The agent appeals to their memory to establish that there exists a syntactic proof of the formula, and then combines the existence of a syntactic proof with~\ref{ill:Eproof:def} to claim support that the formula is a theorem.
  Hence, the agent's memory is directly involved in their claimed support for the formula being a theorem.
\end{note}


\begin{note}
  \begin{illustration}\label{ill:ad:proof:eve}
    \begin{enumerate}
    \item I remember having created a syntactic proof of \formula{\forall x Px \rightarrow \lnot \exists x \lnot P x} (using a sound first-order system).
    \item\label{ill:ad:proof:eve:app} In creating the syntactic proof I appealed to various aspects of some sound first-order system.
    \item\label{ill:ad:proof:eve:pos} As I created a proof, those various aspects of the sound first-order system are sufficient to ensure there exists a proof.
    \item Hence, by \ref{ill:Eproof:def}, \formula{\forall x Px \rightarrow \lnot \exists x \lnot P x} is a theorem of first-order logic.
    \end{enumerate}
  \end{illustration}

  As with~\autoref{ill:ad:proof:mem}, the agent's memory has a role in~\autoref{ill:ad:proof:eve}, but the role is quite different.
  Above, the agent claimed support for the formula being a theorem primarily \emph{because} they remembered creating a proof.
  By contrast, here the agent claims support for the formula being a theorem primarily because of the properties of some sound first-order system.

  Step~\ref{ill:ad:proof:eve:app} appeals to various aspects of some sound first-order system and, in turn, step~\ref{ill:ad:proof:eve:app} observes that those aspects are sufficient to ensure a proof exists.
  The agent claims support for the existence of a proof by appeal to the various aspects of some first-order system they appealed to when constructing the proof, rather than their memory of constructing the proof.
\end{note}

\begin{note}
  To help clarify, let's fix a particular syntactic proof using the Fitch-style proof system of~\textcite[557--560]{Barwise:1999tu}:

  \begin{figure}[H]
    \centering
    \begin{quote}
      \fitchprf{}{
        \subproof{\pline[1.]{\forall x P x}}{
          \subproof{\pline[2.]{\exists x \lnot Px}}{
            \boxedsubproof[3.]{a}{\lnot Pa}{
              \pline[4.]{Pa}[\lalle{1}] \\
              \pline[5.]{\bot}[\lfalsei{3}{4}]
            }
            \pline[6.]{\bot}[\lexie{2}{3--5}]
          }
          \pline[7.]{\lnot \exists x \lnot Px}[\lnoti{2--6}]
        }
        \pline[8.]{\forall x Px \rightarrow \lnot \exists x \lnot Px}[\lifi{1--7}]
      }
    \end{quote}
    \caption{A syntactic proof}\label{fig:syntx-prf}
  \end{figure}

  The proof consists of single instances of five introduction or elimination rules.
  Each rule is part of the Fitch-style proof system, and the specific application of the rules constitute the proof.
\end{note}


\begin{note}[Before\dots]
  Before returning to~\autoref{ill:ad:proof:eve}, let us observe that with the proof in hand one may claim support that a proof of the formula exists via the contents of~\autoref{fig:syntx-prf}.

  Broadly stated:

  \begin{enumerate}
  \item The proof is constructed from a sound first-order proof system.
  \item And, the particular application of some rules of the system to formulae is such that the proof begins with no assumptions and the last line of the proof is not part of any assumption made during the course of the proof.
  \end{enumerate}
\end{note}

\begin{note}
  Note, appeal to creation of the proof involves appeal to various aspects of the Fitch-style proof system.

  The object itself is mute to whether or not it is a proof.

  For example, adding `\formula{Ba}' as an assumption would void the proof, but you would need to observe that the appeal to existential elimination on line 6 requires that `\formula{a}' does not appear in the proof prior to its introduction on line 3 in order to claim support that the proof is void.

  Indeed, the proof consists of eight steps, each step is permitted by the first-order system, the proof begins with no assumptions, the last line of the proof is not part of any assumption made during the course of the proof and the proof, and so on.

  Sparing the details, claimed support that~\autoref{fig:syntx-prf} is a syntactic proof of \formula{\forall x Px \rightarrow \lnot \exists x \lnot P x} from the creation of~\autoref{fig:syntx-prf} is a matter of claiming support for each step of the creation.

  Indeed, to spare the details in general, let us instead talk of some collection of propositions and steps of reasoning.
  Claiming support that a proof exists from the some creation in the way under discussion is an instance of reasoning from details of the creation to the conclusion that a proof exists.
  Hence, as an instance of reasoning involves certain premises and steps of reasoning.
  And, whatever these turn out to be, the proceed from the creation of the proof rather than from some other source such as memory, testimony, and so on.
\end{note}

\begin{note}
  In other words, one may claim support that a proof of \formula{\forall x Px \rightarrow \lnot \exists x \lnot P x} exists (primarily) \emph{because} of their reasoning from some collection of premises and steps of reasoning concerning the creation to the existence of a proof of \formula{\forall x Px \rightarrow \lnot \exists x \lnot P x}.
\end{note}

\begin{note}[Return to \ref{ill:ad:proof:eve}]
  Now let us return to the reasoning of~\autoref{ill:ad:proof:eve}, and in particular steps~\ref{ill:ad:proof:eve:app} and~\ref{ill:ad:proof:eve:pos}:
  \begin{quote}
    \begin{enumerate}
      \setcounter{enumi}{2}
    \item In creating the syntactic proof I appealed to various aspects of some sound first-order system.
    \item As I created a proof, those various aspects of the sound first-order system are sufficient to ensure there exists a proof.
    \end{enumerate}
  \end{quote}
  Given that the agent remembers having created a syntactic proof, the `various aspects of some sound first-order system' of step~\ref{ill:ad:proof:eve} may be taken as those aspects of the first-order system that were appealed to in the premises and steps of reasoning when the agent created the proof.
  And step \ref{ill:ad:proof:eve}, in turn, appeals to how those various aspects of some sound first-order system were sufficient for the agent to claim support that a proof exists by the reasoning that occurred.

  In short, the agent remembers creating a syntactic proof and claiming support that a proof exists from the creation.
  The instance of claiming support involved reasoning from premises via steps to the relevant conclusion.
  Hence, it is possible to claim support for the conclusion by those premises and steps of reasoning.
  So, in~\ref{ill:ad:proof:eve} the agent observes that those premises and steps of reasoning are sufficient to claim support by way of their memory, and in turn appeals to those premises and steps of reasoning to claim support for the relevant conclusion.
\end{note}

\begin{note}
  {
    \color{red}
    Propositional support.
    (If I talk about this, it should be after the definitions.)
  }
\end{note}

\begin{note}
  Generalising, the way in which the agent claims support in~\autoref{ill:ad:proof:eve} is of interest because the agent appeals to premises and steps of reasoning that are not `part' of their present reasoning.
  The role of memory in the illustration is (merely) a way for the agent to recognise that there are such premises and steps of reasoning.
  And, in the definitions that follow, we will abstract from any particular way in which the recognises that relevant premises and steps of reasoning are available.

  Still, even though memory is contingent, we may briefly observe that the way in which the agent claim support in~\autoref{ill:ad:proof:eve} is compatible with \ESU{}.
  For, \ESU{} requires that an agent may claim support for some conclusion from premises and steps of reasoning only if the agent has witnessed reasoning to the conclusion from those premises via those steps of reasoning.
  So, if the initial instance of claiming support conformed to \ESU{} then the agent will have witnessed reasoning from those steps and premises to the conclusion --- the instance of claiming support in~\autoref{ill:ad:proof:eve} does not involve such witnessing, but the agent's memory would be about how the relevant premises and steps were used to claim support.

  Of course, the way in which the agent claim support in~\autoref{ill:ad:proof:eve} is incompatible with a strengthened variant of \ESU{} which requires the agent to use any premises and steps they appeal to in the \emph{present} instance of reasoning, but the point for the moment is that the way in which the agent claims support in~\autoref{ill:ad:proof:eve} does not already require what we are arguing against: \ESU{}.
\end{note}

\paragraph{Definitions}

\begin{note}
  With a somewhat detailed pair of contrasting illustrations in hand, we now turn to fixing a pair of definitions which capture the general way in which the agent claims support in the respective illustrations.

  The two ways will be termed `\adA{}' and `\adB{}', respectively.
\end{note}

% \begin{note}
%   Appealing to a conditional, and appealing to something `more fundamental' that underwrites the conditional.

%   This is okay, to some extent.
%   However, unsatisfactory.
%   First, conditional, so particular type of reasoning, but what really matters is that something is obtained from something else.
%   Second, no idea what `more fundamental' amounts to.
%   What is clear is that it's a different way, and that's what we capture.
% \end{note}

\begin{note}
  \begin{restatable}[\adA{}]{definition}{defADA}\label{AR:adA}\label{def:adA}
    Fix an agent and suppose:
    \begin{enumerate}[label=\textsf{S:\arabic*}., ref=(\textsf{S}:\arabic*), series=adA_counter]
    % \item\label{def:adA:p-to-p} Agent has claimed support that \(\psi\) has value \(v'\) when \(\phi\) has value \(v\).
    \item\label{def:adA:phi} The agent has claimed support for \(\phi\) having value \(v\).
    \end{enumerate}
    Then, agent claims support for \(\psi\) having value \(v'\) by `\adA{}' from \(\phi\) having value \(v\) when:
    \begin{enumerate}[label=\textsf{S:\arabic*}., ref=(\textsf{S}:\arabic*), resume*=adA_counter]
    \item\label{def:adA:psi} The agent claims support for \(\psi\) having value \(v'\) (in part) by via their claimed support that for \(\phi\) having value \(v\).
    \end{enumerate}
    \vspace{-\baselineskip}
  \end{restatable}
\end{note}

\begin{note}
  \adA{} does not outline a specific way of reasoning.
  Rather, captures the role of claimed support for \(\phi\) having value \(v\) in some instance of reasoning when the agent claims support for \(\psi\) having value \(v'\).

  Intuitive idea is that claimed support for \(\phi\) having value \(v\)~\ref{def:adA:phi} provides agent with resource to claim support for \(\psi\) having value \(v'\) to~\ref{def:adA:psi}.
\end{note}

\begin{note}
  Applied to the two sketches seen, claim support by existence of proof, or by specific ability to \emph{V} that \(\phi\).
  Key thing is that claimed support for existence of proof or the specific ability to \emph{V} that \(\phi\) rather than something else.

  \phantlabel{abstract-adA}
  Indeed, noting and abstracting from the role of conditionals in these two illustrations, basic (abstract) instance of \adA{}:

  {
    \small
    \begin{enumerate}[label=\arabic*., ref=\arabic*]
    \item\label{def:adA:ex:C:Cp} I have claimed support that \(\phi\) has value \(v\).
    \item\label{def:adA:ex:C:p} So, if my claimed support is \nmom{}, \(\phi\) has value \(v\). \hfill(From~\ref{def:adA:ex:C:Cp})
    \item\label{def:adA:ex:C:Cps} Likewise, I have claimed support that \(\psi\) has value \(v'\) when \(\phi\) has value \(v\).
    \item\label{def:adA:ex:C:ps} So, if my claimed support is \nmom{}, \(\psi\) has value \(v'\) when \(\phi\) has value \(v\). \hfill(From~\ref{def:adA:ex:C:Cps})
    \item\label{def:adA:ex:C:T} If \(\psi\) has value \(v'\) when \(\phi\) has value \(v\) and \(\phi\) has value \(v\), then it must be the case that \(\psi\) has value \(v'\). \hfill (From understanding of `if\dots then\dots')
    \item\label{def:adA:ex:C:s} Hence, if my claimed support is \nmom{}, \(\psi\) has value \(v'\).\newline
      \mbox{}\hfill (From \ref{def:adA:ex:C:p},~\ref{def:adA:ex:C:ps}~and~\ref{def:adA:ex:C:T})
    \item Therefore, I claim support that \(\psi\) has value \(v'\) as I expect the claimed support of (\ref{def:adA:ex:C:Cp}) and (\ref{def:adA:ex:C:Cps}) is, respectively, \nmom{}. \hfill (From \ref{def:adA:ex:C:Cp} -- \ref{def:adA:ex:C:s})
    \end{enumerate}
  }
  The reasoning is a verbose because claimed support is not necessarily factive
  \nolinebreak
  \footnote{
    It may be that an agent has claimed support for \(\phi\) having value \(v\) while \(\phi\) has value \(v'\).
  }
  yet the agent has claimed support about \(\phi\) having value \(v\) and how that relates to \(\psi\) having value \(v'\), rather than how claimed support for \(\phi\) having \(v\) relates to claimed support for \(\psi\) having value \(v'\),
  (Consider parallel reasoning with knowledge, rather than (mere) claimed support.\nolinebreak
  \footnote{The parallel reasoning in full:
    \begin{enumerate}[label=\arabic*., ref=\arabic*]
    \item\label{def:adA:ex:K:Kp} I know that \(\phi\) has value \(v\).
    \item\label{def:adA:ex:K:p} So, \(\phi\) has value \(v\). \hfill (From~\ref{def:adA:ex:K:Kp})
    \item\label{def:adA:ex:K:Kps} I know that \(\psi\) has value \(v'\) when \(\phi\) has value \(v\).
    \item\label{def:adA:ex:K:ps} So, \(\psi\) has value \(v'\) when \(\phi\) has value \(v\). \hfill(From~\ref{def:adA:ex:K:Kps})
    \item\label{def:adA:ex:K:T} If \(\psi\) has value \(v'\) when \(\phi\) has value \(v\) and \(\phi\) has value \(v\), then it must be the case that \(\psi\) has value \(v'\). \hfill (From understanding of `if\dots then\dots')
    \item\label{def:adA:ex:K:s} Hence, \(\psi\) has value \(v'\). \hfill (From \ref{def:adA:ex:C:p},~\ref{def:adA:ex:C:ps}~and~\ref{def:adA:ex:C:T})
    \item So, I know that \(\psi\) has value \(v'\) as \(\psi\) having value \(v'\) follows from~(\ref{def:adA:ex:K:Kp}) and~(\ref{def:adA:ex:K:Kps}).
      \mbox{}\hfill (From \ref{def:adA:ex:K:Kp} -- \ref{def:adA:ex:K:s})
    \end{enumerate}
  }%
  )
  Still, the reasoning is a clear instance of claiming support for \(\psi\) having value \(v'\) by \adA{} from \(\phi\) having value \(v\) as the agent claims support for \(\psi\) having value \(v'\) by appealing to their claimed support for \(\phi\) having value \(v\) to satisfy the antecedent of a conditional.

  Still, \adA{} need not involve a conditional.
  Consider, for example, claiming support that they have claimed support for a contradiction from claimed support that \(\phi\) and not-\(\phi\) are both true.
  It seems plausible that so claiming need only involve a reflection on what \(\phi\) and not-\(\phi\) amounts to.
\end{note}

\begin{note}

    \begin{restatable}[\adB{}]{definition}{defADB}\label{AR:adB}\label{def:adB}
    Fix an agent and suppose:
    \begin{enumerate}[label=\textsf{I:\arabic*}., ref=(\textsf{I}:\arabic*), series=adB_counter]
    % \item Agent has claimed support that \(\psi\) has value \(v'\) when \(\phi\) has value \(v\).\nolinebreak
      % \footnote{
      %   It doesn't matter whether claiming support for \(\phi\) is sufficient to claim support for \(\psi\).
      % }
    \item\label{def:adB:poss} Claimed support for \(\phi\) ensures there is some (distinct) collection of propositions premises \(\rho_{1},\dots,\rho_{k}\) with respective values and steps \(\delta_{1},\dots,\delta_{m}\), such that it is possible to claim support for \(\psi\) having value \(v'\) by appeal to \(\rho_{1},\dots,\rho_{k}\), with respective values, and \(\delta_{1},\dots,\delta_{m}\).
    \end{enumerate}
    Then, agent claims support for \(\psi\) having value \(v'\) by `\adB{}'\nolinebreak
    \footnote{
      Motivated in part by similarity to Craig's interpolation theorem.
      Find something `in-between' that does the work.
      However, that's the extent of the relation.
    }
    from \(\phi\) having value \(v\) when:
    \begin{enumerate}[label=\textsf{I}:\arabic*., ref=(\textsf{I}:\arabic*), resume*=adB_counter]
    \item\label{def:adB:inter} Claim support for \(\psi\) having value \(v'\) by appeal to \(\rho_{1},\dots,\rho_{k}\) with respective values and the possibility of claiming support for \(\psi\) having value \(v'\) by appeal to \(\rho_{1},\dots,\rho_{k}\) with respective values and steps \(\delta_{1},\dots,\delta_{m}\).
    \end{enumerate}
    \vspace{-\baselineskip}
  \end{restatable}
\end{note}

\begin{note}
  With \adA{} \(\phi\) having value \(v\).
  \adB{} does not involve the agent claiming support for \(\phi\) having value \(v'\) by \(\phi\) having value \(v\).
  Instead, some (distinct) collection of premises \(\rho_{1},\dots,\rho_{k}\) with respective values and steps \(\delta_{1},\dots,\delta_{m}\).

  Key difference is that agent isn't directly claiming support for \(\psi\) from \(\phi\).
  From \(\rho_{1},\dots,\rho_{k}\) to \(\psi\)
  Though, this is not to say that \(\phi\) is irrelevant.
  For the definition to be satisfied, \(\phi\) needs only be involved to the extent that it provides the link.
  In turn, the definition does not stated how the agent claims support for \(\rho_{1},\dots,\rho_{k}\) having respective values.

  In other words, \adA{} specifies a `because' while \adB{} does not.

  Two reasons.
  First, it doesn't matter.
  Some other way.
  Second, there are at least two possibilities:

  \begin{enumerate}
  \item Agent to claims support for \(\rho_{1},\dots,\rho_{n}\) by \(\phi\) (though this isn't the one I'm interested in).
  \item Agent has already claiming support for \(\rho_{1},\dots,\rho_{n}\) (this is the one I'm interested in).
  \end{enumerate}

  With respect to `because'
  \begin{itemize}
  \item Because \(\phi\) ensures \(\rho_{1},\dots,\rho_{n}\) and \(\psi\) from \(\rho_{1},\dots,\rho_{n}\).
  \item Because \(\rho_{1},\dots,\rho_{n}\) and \(\phi\) ensures possibility for \(\psi\) from \(\rho_{1},\dots,\rho_{n}\).
  \end{itemize}
\end{note}


\begin{note}
  Only seen \adB{} with respect to proof illustration.
  Remembered proving \(\phi\), that secures the possibility, but claiming support from the details of the proof itself.

  Intuitively, applies to ability in the same way.
  Premises and steps of reasoning work in the same way as components of a proof.
  However, we will delay details until we've seen a few more illustrations.
\end{note}

\begin{note}
  Broad distinction, agent may claim support by appeal to some thing, but it is also possible to break that thing down in to parts or elements such that the agent may claim support by appeal to those parts or elements (and how they compose).

  `Break down' is metaphorical.

  In some cases, the thing itself, in other cases, more basic stuff that must be the case in order for the thing to be the case.

  Break down does the work.
  Agent will typically recognise.
  Break down is not required.

  In this sense, break down is more fundamental.

  `Because\dots'

  Unifying feature is that \adA{} allows claim support for \adB{}, so not clear that need to go via \adB{}.
  Indeed, unclear, given \ESU{}, that may claim support by \adB{}.
  We will only push this question with respect to ability, though.
\end{note}

\begin{note}[\ESU{}]
  We noted above that the reasoning of~\ref{ill:ad:proof:eve} was compatible with \ESU{}.
  The reasoning of~\ref{ill:ad:proof:eve} is an instance of \adB{}.
  Hence, there are instances of \adB{} which are compatible with \ESU{}.

  Argue that there are instances which are not compatible.
\end{note}

\paragraph{Additional illustrations}

\begin{note}

  \begin{illustration}
    \begin{itemize}
    \item If bag are overweight then they can't be taken on the flight.
    \item Machine reads\dots
    \item Bag can't be taken on the flight.
    \end{itemize}
  \end{illustration}
  Contents of the bag are overweight.

  Combined weight of the items versus the combination of the individual weights.

  Compare, filling the bag and weighing it, versus summing the weight of the items as you fill the bag.

  Now, seems possible to fill the bag and weight it, then appeal to the sum of the items.

  So, this is a little more subtle.
  The bag has been weighed, and the distinction is between the weight of the contents of the bag, and the combined weight of the items that make up the contents of the bag.

  This is particularly interesting.
  Because, it seems clear that something is strange if someone talks about the weight of the contents of the bag without recognising that this is a function of the combined weight of all the individual elements of the bag.
  However, no idea what the contents of the bag are.

  So, claiming support from what is has been observed, the combined weight, rather than what must be the case in order to have made the observation.
\end{note}

\begin{note}[Fire alarm]
  \begin{illustration}
    \begin{itemize}
    \item Fire alarm is ringing.
    \item Fire in the building.
    \item Should leave by the nearest exit.
    \end{itemize}
  \end{illustration}
  So, claiming for getting out of the building.
  Fire alarm.
  Or, fire, fire alarm has picked this up.

  So, difference between that there is a fire in the building, and \emph{the} fire in the building.

  The point here is that, okay, you need to go from alarm to fire, that's all fine, but fire itself is sufficient to claim support.
\end{note}

\begin{note}

  \begin{illustration}\label{ill:ad:factorial}
    \begin{itemize}
    \item It is possible to write recursive functions in C.
    \item It is possible to write a recursive implementation of the factorial function in C.
    \end{itemize}
  \end{illustration}
  With proofs, abstract objects.

  Consider programming.

  Recursive implementation of factorial in C (chosen to make the implication clear).

  So, \adA{} is just the fact, so to speak.
  But, \adB{} points to the key step of calling function.
  Of course, this is just recursion, but appeal here is to the concept, so to speak, rather than the truth of the statement.

  Don't need to understand details.
  Go by form, so to speak.

  Claiming support by logical relation, rather than the states of affairs that ensure those logical relations hold up.

  Or, the definition is such that\dots
\end{note}

\begin{note}[Existentials]
  In a sense, the point here is that \adA{} cases of interest mean that there's something more.
  This is viewed in terms of some complex of more basic things existing.
  And, \adB{} follows the reference.
\end{note}

\subsubsection{\adA{}}
\label{sec:ads}

\subsubsection{\adB{}}
\label{sec:adc}

\begin{note}
  Core idea is that claim support by what follows from ability.

  Helpful to highlight parallel distinction with respect to other instances of claiming support.
\end{note}

\begin{note}
  Similar to verifying an algorithm may be implemented.
  Break down all of the steps in the algorithm, and then ensure that it is possible to express each of the steps in the programming language of choice.

  \begin{quote}
    \textsc{factorial}(\(n\)):\newline
    \textbf{if} \(n = 1\)\newline
    \mbox{}\indent \textbf{return} \(1\)\newline
    \textbf{else}\newline
    \mbox{}\indent \textbf{return} \(n \times\) \textsc{factorial}(\(n-1\))
  \end{quote}

  Fortran 77 does not support recursion, a function may not call an instance of itself.\nolinebreak
  \footnote{
    This is not to say that one may not compute factorials using Fortran 77.
    It's a Turing complete language.
    However, would require a different (non-recursive) algorithm.
  }
  By contrast, the recursive factorial algorithm may implemented in languages that support recursion, such as Lisp or Python.

  \adA{} and \adB{}, recursion, or function calling an instance of itself.

  \AR{}, properties of the language, \WR{} adds in particular event.
\end{note}

\begin{note}
  Turning to reasoning, very similar idea.
  Features of programming languages are resources for doing something, in the same way that premises and steps of reasoning are resources for reaching some conclusion.
\end{note}

\subsection{Recap}
\label{sec:recap-reasoning}

\begin{note}
  Ability.
  Instance of reasoning with ability.
  Two distinctions which apply here.
  \AR{} and \WR{} for ability.
  \adB{} and \adA{} for how ability is used to claim support.
\end{note}

\begin{note}[Two ways to get to \(\phi\)]
  Two key steps.
  \begin{itemize}
  \item \gsi{}.
  \item \aben{the}.
  \end{itemize}

  First, general to specific, then specific to \(\phi\).

  Second, general to specific, then general to \(\phi\).
\end{note}

\begin{note}
  Second is something like evidence of evidence is evidence.

  Here, the important difference is that the agent only needs to appeal to general ability.
  And, they've claimed support for this.

  The point is that it's not clear the agent is required to do anything too much with the specific ability.
\end{note}

\begin{note}[Focus]
  Common here is \gsi{}.
  Needed in both cases.

  However, it's also the case that the `final' bit of reasoning is more or less \aben{the}.

  So, in a sense both ways of reasoning depend on these two things.

  The only difference is the particular for \aben{the} would take.

  There is a difference.
  For, general and specific are different.
  I'm not clear on whether this amounts to a significant difference.
  Still, even if it does, argument will proceed even if there is something significant to be made of this.
\end{note}

\begin{note}[Summarising]
  Above we introduced \gsi{}.
  Limited information of the form `If \emph{S} has a (general) ability to \(\gamma\), then \emph{S} has a (specific) ability to \emph{V} that \(\phi\) (as an instance of the general ability).'
  We then noted that certain instances of the (specific) ability to \emph{V} that \(\phi\) entail that \(\phi\) is the case.
  Two interpretations of \aben{the}, \AR{} and \WR{}.

  Our focus now turns back to \gsi{}.
  For those instances of \gsi{} when \aben{the} holds, the interpretations \AR{} and \WR{} detail what the agent obtains by reasoning from general to specific ability.
  In other words, \emph{what} the agent is claiming support for.

  As noted, using a conditional such as \gsi{} is not automatic.
  The informer has not provided the agent with any additional way to claim support that the agent has the general ability.
  Rather, outlined something that follows \emph{if} the agent has the general ability.

  So, it is up to the agent to resolve in either way.
  If the agent wants to use the information, then the agent needs to reason from general to specific.
  The issue is that without any additional reasoning, it seems there's no clear way to determine which way the agent should go.
  Here is where the distinction between \AR{} and \WR{} is important.
  Interpretation of specific ability informs how the agent move from general to specific.

  Following two propositions outline combination.
  {
    \color{red}
    The key thing here is about claiming that one has a specific ability.
  }
\end{note}

\begin{note}[Unified idea]
  Claim support for premises and steps of reasoning.

  Easiest with \WR{}.
  What's missing here is the use of the premises in reasoning.
  Hence, contrast to \ESU{} which we'll talk in some detail about below.

  Same applies to \AR{} by general property reduction.

  So, \AR{} and \WR{} allow the agent to do the same thing, but in slightly different ways.
\end{note}

\begin{note}[\gsi{}++]
  First, \gsi{} applied to \AR{}
  \begin{restatable}[\textsf{|gs-I\space·\space H|}]{idea}{ideaCSbyAR}\label{idea:CS-by-AR}
    % In order for \emph{S} to have the (specific) ability to \emph{V} that \(\phi\) for which \aben{the} holds, claimed support for general and claimed support for \gsi{} are sufficient to claim support that \emph{S} has the property of being able to \emph{V} that \(\phi\).
    Suppose an agent has:
    \begin{enumerate}
    \item Claimed support for some general ability \(\gamma\).
    \item Claimed support that if they have the general ability \(\gamma\) then they have some specific ability to \emph{V} that \(\phi\) (for which \aben{the} holds).
    \end{enumerate}
    Then:
    \begin{enumerate}[resume]
    \item \emph{S} may claim support for having the specific ability \(\sigma\) by reasoning that they have the property of being able to \emph{V} that \(\phi\).
    \end{enumerate}
    \vspace{-\baselineskip}
  \end{restatable}

  Second, \gsi{} applied to \WR{}

  \begin{restatable}[\textsf{|gs-I\space·\space W|}]{idea}{ideaCSbyWR}\label{idea:CS-by-WR}\label{W:s}
    % In order for \emph{S} to have the (specific) ability to \emph{V} that \(\phi\) for which \aben{the} holds, claimed support for general and claimed support for \gsi{} are sufficient to claim support that there is a potential witnessing event in which \emph{S} \emph{V}s that \(\phi\).
    Suppose an agent has claimed support for some general ability \(\gamma\) and has claimed support that if they have the general ability \(\gamma\) then they have some specific ability to \emph{V} that \(\phi\) for which \aben{the} holds.
    Then, an agent may claim support for having the specific ability \(\sigma\) by reasoning that there is a potential witnessing event in which \emph{S} \emph{V}s that \(\phi\).
  \end{restatable}
\end{note}

\begin{note}[Alternatives]
  Appeal to premises and steps is not required by either \AR{} or \WR{}.
  However, most plausible account of what is going on.

  Explored some alternatives for \AR{}, but unclear what is of importance other than reasoning, and hence premises and steps.
  And, in this respect, basic \AR{} seems like a dead end.
  Premises and steps allow the agent to claim support in the same way as they would allow the agent to claim support when used in reasoning.
  It's not at all clear to me that basic \AR{} makes sense from this perspective.
\end{note}

\begin{note}[Limitation of intuition]
  Focused on idea that claiming support in same way as reasoning.

  This is not to imply equivalence of claimed support.

  Said too little about claimed support to make any strong remarks about equivalence.
  Still, intuitive that additional way of being \mom{}.
  For, haven't done the reasoning, so \mom{} about this.
  Not the case if the agent has done the reasoning.
\end{note}

\paragraph{Old notes}

\begin{note}[\gsi{}++ applied : \AR{}]
  \AR{} doesn't need to much expansion.
  Silent on what the property is.
  One way to view is that general ability reduces to sufficient collection of specific.
  \gsi{} conditional informs the agent that specific instance, so required for general ability.
  \gsi{} is novel, but support claimed is for quantifier over all core instances.

  Similar to a standard induction principle.

  With respect to chess, this is one such principle.
\end{note}

\begin{note}[\gsi{}++ applied : \WR{}]
  \WR{} is different.
  Witnessing event.
  So, \emph{V}ing that \(\phi\).
  Break down \emph{V}ing that \(\phi\) into a series of actions performed by the agent.
  General ability secures performing each of those actions.

  Turning to the chess example.
  Here, appeal to sufficient understanding of the rules of chess, and the combination of these.
  More broadly, premises and steps of reasoning.

  It's this kind of stuff that \WR{} uses.
\end{note}

\begin{note}[Impact of distinction]
  Return to the impact of the distinction.

  \AR{}, focus on generator.
  Hence, task is to establish that the agent has resources to generate.
  So, in a sense, with \gsi{} we go from existence of general to existence of specific generator.
  Note, this isn't to say that there's something like a general generator.
  It may be the case by general ability we have quantification over specific abilities.
  If so, then claim support that there's a particular specific generator.
  Nor that specific generators are unique.
  The available resources may overlap.
  Still, some thing that is true of the agent, and claimed support for general ability is sufficient to claim support that the `some thing' holds.

  \WR{}, focus on generated event.
  So, agent doesn't necessarily need to establish a generator, but rather ensure that event may be generated.
  Hence, \gsi{}, general ability allows the agent to generate witnessing event for specific ability.
  Not looking to claim support that `some thing' is true of the agent.
  Rather, claimed support for general ability, and appeal to the actions that this allows the agent to perform.
\end{note}

\begin{note}[Intuition for \AR{} and \WR{}]
  Both \AR{} and \WR{} are ways to understand \aben{the}, which is in turn about what is entailed by an agent having a (certain kind of) specific ability.

  \AR{} focuses on the idea that the agent may claim support from having the attribute (or the truth) of the specific ability.
  \AR{} requires support for attribute, which in turn suggests in a position to claim support for premises and steps.
  \AR{} doesn't require agent to claim support for premises and steps.

  \WR{} focuses on the idea that the agent may claim support from witnessing (or using) the specific ability.
  \WR{} requires support for premises and steps, which in turn suggests in a position to claim support for attribute.
  \WR{} doesn't require agent to claim support for attribute.
\end{note}

\begin{note}[Quite brief]
  Sketches of \AR{} and \WR{} are brief.
  Expand on these in the following sections (\ref{sec:first-conditional} and~\ref{sec:second-conditional}) to some extent, and chapter~\ref{cha:potent-infer-attr} will focus on a detailed account of both.
\end{note}

\begin{note}
  Role of ability to secure witnessing event.

  The distinction may be highlighted by a distinct set of implications\nolinebreak
  \footnote{
    Though not necessarily entailments.
  }
  \nagent{4} is dehydrated, so \nagent{4} is tired.
  \nagent{4} took a long walk in the sun, so \nagent{4} is tired.

  First, implication follows from some property.
  Second, implication follows from the result of some action.

  As with ability, both implications may be true.
  Still, difference in terms of whether one appeals to some property of \nagent{4}, or some action that \nagent{4} performed.
  As with ability, there is some ambiguity.
  There's the fact that \nagent{4} took a walk in the sun, and there's the action of \nagent{4} taking a walk in the sun.
\end{note}

\begin{note}[Why]
  So, \AR{}, some property.
  With \WR{}, it's the event that matters.
  In turn, moving from premises to some conclusion.
  Appeal to the event involves appeal to constituents of event.

  Return to \ESU{}.
  No inherent conflict with either \AR{} or \WR{}.
  Difference between property and witness.
  Requirement is that claimed support for premises is sufficient to claim support for conclusion.
  With \AR{}, claimed support for property --- need enough to be sure property is adequate.
  With \WR{}, claimed support for witness --- need enough to make sure that witness is adequate.

  \nagent{4} is thirsty, no implication.
  \nagent{4} walked, no implication.

  Does not matter that thirst is part of the relevant instance of being dehydrated.
  Nor that the witnessing event of walking was a long walk in the sun.
  Deny claimed support as did not reason from such premises.
\end{note}

\subsubsection{Summary of distinctions}
\label{sec:summary-distinctions}

\begin{note}
  \begin{figure}[H]
    \centering
    \begin{tblr}{abovesep=8pt, belowsep=8pt, width=0.95\textwidth, colspec={Q[c,m]|Q[c,m]|Q[1.8,c,m]|Q[1.8,c,m]}}
      \multicolumn{2}{c}{} & \adA{} & \adB{} \\
      \hline
      \multicolumn{2}{c}{\WR{}} & That there is an event in which \emph{S} \emph{V}s that \(\phi\) entails \(\phi\) & Parts of an event in which \emph{S} \emph{V}s that \(\phi\) entail \(\phi\) \\
      \hline
      \multirow[c]{2}{*}{\AR{}} & Basic  & That \emph{S} has the ability (to \emph{V} that \(\phi\)) entails \(\phi\) & --- \\
      \cline[dashed]{2-4}
      & Derived & That there is some property \emph{P} (from \emph{S} having the ability to \emph{V} that \(\phi\)) entails \(\phi\) & Parts of some property \emph{P} (from \emph{S} having the ability to \emph{V} that \(\phi\)) entails \(\phi\) \\
    \end{tblr}
    \caption{Distinction matrix with \aben{the}}
  \end{figure}
\end{note}

\begin{note}
  Basic \AR{} with \adB{} has `?'.
  For, as noted above it's not clear what this amounts to.
\end{note}

\subsubsection{The distinctions are (sufficiently) exhaustive}
\label{sec:ar-wr-are}

\begin{note}
  Two pairs.
  \AR{} and \WR{}, \adA{} and \adB{}.

  Goal is to argue that:
  \begin{itemize}
  \item \AR{} and \WR{} are exhaustive.
  \item \adA{} and \adB{}, sufficient, as relative to \(\phi\).
  \end{itemize}
\end{note}

\paragraph{\AR{} and \WR{}}

\begin{note}
  Key idea is that \AR{} and \WR{} are different perspectives on the same thing.

  Switching between ability and potential events.
  This is not important, two ways of describing the same thing.

  The ability to \emph{V} that \(\phi\) is equivalent to there being a potential event in which the agent \emph{V}s that \(\phi\).
  For, if there is no such potential event, then the agent does not have the ability to \emph{V} that \(\phi\).
  Conversely, if there is a potential event in which the agent \emph{V}s that \(\phi\), then the agent has the ability to \emph{V} that \(\phi\).
\end{note}

\begin{note}[Exhaustive]
    \begin{restatable}[]{proposition}{propAbilityExuastive}\label{prop:WR-and-AR-exhaustive}\label{either-AR-or-WR}
    Any interpretations of an agent's (specific) ability to \emph{V} that \(\phi\) (for which \aben{the} holds) conforms to either:
    \begin{enumerate}
    \item \AR{}: It is a property of the agent that they are able to \emph{V} that \(\phi\).
    \item \WR{}: There is a potential witnessing event in which the agent \emph{V}s that \(\phi\).
    \end{enumerate}
    \vspace{-\baselineskip}
  \end{restatable}
\end{note}

\begin{note}
  The distinction between \AR{} and \WR{} sets up two (schematic) ways in which agent an agent may claim support given an instance of \aben{the}.
  We now argue that these two (schematic) methods are exhaustive.
  {
    \color{red}
    Important to keep in mind is that our interest is with claiming support.
    And, in particular, what the agent claims support for given \AR{} and \WR{}.
    So, the claim that \AR{} and \WR{} are exhaustive is a claim about how an agent reasons, not what ability reduces to.
  }
\end{note}

\begin{note}
  \color{red}
  This section is now far more straightforward.
  \AR{} and \WR{} is a little more complex.
  Ability always with respect to some action, that's a constraint on the type of ability of interest.
  So, static versus dynamic.


  And, Basic and derived are easy given \AR{}.
\end{note}

\begin{note}[Argument]
  \color{red}
  Start with the basics.
  Have an instance of \aben{the}.
  So, the agent claim support for \(\phi\) given claimed support for \emph{S} having the ability to \emph{V} that \(\phi\).
  So, need to argue that the agent:
  Either claims support for some property of \emph{S} (\AR{}).
  Or, claim support for \(\phi\) as the result of the event of \emph{S} \emph{V}ing that \(\phi\), with 
\end{note}

\begin{note}[Old arguments]
  Remaining issue is details of the schemas.
  These talk about more than mere reference.
  \AR{}, agent, and \WR{} the result of the witnessing event.
  In turn, these are harmless and the only plausible option.

  \AR{} is simple.
  State of affairs, but as the agent is involved, then it is natural to attribute to the agent.
  Implausible that it's some event.

  \WR{} focuses attention to culmination of event.
  However, need culmination.
  Quirk of English that may `use' relevant verbs in this way.
  Imperfective paradox.
  May consider this a state, but only in the sense that it is a state bought about by some event.
  Focus on event, but given culmination, consider this a state.
  Still, state of culminated event.
  Possible that this is simply a state in which the agent has some appropriate relation.
  Problem is that an ability is the ability to do some thing.
  If abstract away from the act, then it's not clear how to understand conditions as identifying ability.
\end{note}

\paragraph{\adA{} and \adB{}}

\begin{note}[Style of argument]
  Well, with respect to claiming support for \(\psi\) such that \(\phi\) is involved.

  Either \adA{} or \adB{}.
  \adA{} seems sufficiently clear, so:
  Transform this to: If not \adA{} then \adB{}.
  Equivalent.
\end{note}

\begin{note}[Idea]
  So, \(\phi\) is involved, but isn't \adA{}.
  Hence, agent claims support by something else.
  If \(\phi\) isn't related to that stuff in any way, then completely redundant.
  Note, from agent's perspective, rather than possibility of revising without.
  So, seems it can only be about how those other things relate to the conclusion.

  Okay, so idea is that if no \adA{} then \(\phi\) isn't part of claiming support.
  If other stuff without \(\phi\) then redundant.
  So, if \(\phi\) is involved, about how the other stuff relates.
\end{note}

\subsubsection{Where the tension arises}
\label{sec:where-tension-arises}

\begin{note}
  {
    \color{red}
    Looking ahead.
  }
  The goal here is to clarify that the tension arises from the ability entailment.
  The role of general to specific is to ensure that agent gets to fact from specific.
\end{note}


\section{\ESU{}, \gsi{}, and \aben{the}}
\label{sec:first-conditional}

\begin{note}[Summary]
  In this section we argue that \ESU{} constrains what an agent may claim support for when reasoning from general to specific ability.

  In short, \ESU{} rules out claiming support by \adB{}.
\end{note}

\begin{note}
  Two ways corresponding to two sides of (specific) ability
  First, with respect to \aben{the}: appealing to (specific) ability.
  Second, with respect to \gsi{}: establishing (specific) ability from (general) ability.
\end{note}

\begin{note}[Expand: \gsi{}]
  Start with \gsi{}.

  Agent is claiming support for specific ability.
  Hence, claiming support that there is a potential event in which they \emph{V} that \(\phi\).
  Expanding potential event, claiming support that sufficient resources are available.
  Note, the agent may not (merely) \emph{expect} that sufficient resources are available, as availability of resources is part of claim for potential event.
  Rather, the agent may expect that there are no defeaters to claim that resources are available.

  To illustrate.
  Suppose I claim support that I know the train will be late.
  It's not (merely) that I expect that the train will be late.
  In order to claim support, some considerations sufficient to establish that the train will be late, and that there are no defeaters for these considerations.
  Expect would be absence of materia that train is on time.
  But absence alone doesn't push either way.\nolinebreak
  \footnote{
    Absence may be materia, though.
    For example, at least five minutes before train will arrive there is a message broadcast at the station.
    We are at the station, and it is a three minutes before the train is scheduled to arrive.
  }

  Task is to account for why an agent may claim support for availability of sufficient resources.
  In rough outline, answer is simple.
  Claimed support for general ability.
  Specific ability to \emph{V} that \(\phi\) is an `instance' of the general ability.
  So, given context, general ability supplements sufficient additional premises and steps of reasoning.

  However, without witnessing specific ability, agent is not aware of which additional premises and steps of reasoning are used.
\end{note}

\begin{note}[Moving to incompatibility]
  Incompatibly with \ESU{} will be from common point of appeal to sufficient resources.
  To this we now turn.
\end{note}

\subsubsection{Constraints on reasoning with \gsi{} given \ESU{}}
\label{sec:incomp-wr-ura}

\begin{note}[Argument outline]
  \color{red}
  There are two issues.
  \begin{itemize}
  \item \ESU{} means that the agent may not `directly' establish the existence of a witnessing event.
    For, in order to do so, the agent would need to claim support that the conclusion follows from some collection of premises and steps of reasoning.
    However, as the agent does not witness, then this isn't possible given \ESU{}.
    \begin{itemize}
    \item The objection here is that the agent doesn't necessarily need to go directly.
      It's possible that the agent claim support for some property, hence gets specific ability, and then reasons that this means that there's a witnessing event.
      So, to the extent that \WR{} needs first the existence of a witnessing event, \ESU{} might be okay.
    \end{itemize}
  \item Second, the agent can't reason with the details of the witnessing event.
    This then blocks \WR{}, and does so conclusively.
    For, the agent is not permitted to appeal to a relation of support between premises and conclusion.
    The agent is only permitted to appeal to the existence of an event that would establish such a relation.
    So this is the main objection.
  \end{itemize}
\end{note}

Key proposition of this section.

\begin{note}[Proposition]
    \begin{restatable}[\ESU{} and \adB{}]{proposition}{propNoESUandADB}\label{mcA:WR-and-denied-claim}
    \emph{If} \ESU{} is true \emph{then} no claiming support by \adB{} with respect to derived \AR{} or \WR{}.
  \end{restatable}
\end{note}

\begin{note}
  Argument is straightforward.
  \adB{} then claiming support by property of witnessing event.
  But, agent has not used those things in reasoning.
\end{note}

\begin{note}
  So, the key thing with this proposition is that in cases where an agent reasons with \gsi{-}, the agent claims support for a property.
  Hence, if an agent reasons from specific ability via \aben{the}, then must be an instance of \AR{}.

  Now, \autoref{mcA:WR-and-denied-claim} doesn't say that \ESU{} and \WR{} are incompatible in general.
  We'll see this in the argument for~\autoref{mcA:WR-and-denied-claim}.

  The argument that \WR{} is an incorrect interpretation of (specific) abilities of the form \emph{S} has the ability to \emph{V} that \(\phi\) (for which \aben{the} entailment holds) has two components.
  First, difficulty establishing by \gsi{}.
  Second, rules out \aben{the}.

  Difficulties with \gsi{}.
  Result of claiming support by \gsi{} is that agent claims support for specific.
  And, given \WR{}, this involves claiming support that some sufficient collection of premises and steps of reasoning are available to the agent.
  Given \ESU{}, the agent is required to use these steps and premises in order to appeal.
  Therefore, \ESU{} requires a partial witnessing event.
  Partial only, as the depending on how premises and steps are understood, certain premises or steps may be reused, and a single use may be sufficient for \ESU{}.

  The issue is strengthened when turning to \aben{the}.
  For, the conclusion is that \(\phi\) is the case.
  And, a partial witnessing event does not establish that \(\phi\) is the case.
\end{note}

\subsubsection{Summary}
\label{sec:uRa-and-wr-summary}

\begin{note}[Table]
  \begin{figure}[H]
    \centering
    \begin{tblr}{abovesep=8pt, belowsep=8pt, width=0.95\textwidth, colspec={Q[c,m]|Q[c,m]|Q[1.8,c,m]|Q[1.8,c,m]}}
      \multicolumn{2}{c}{} & \adA{} & \adB{} \\
      \hline
      \multicolumn{2}{c}{\WR{}} &  & Ruled out by \ESU{} \\
      \hline
      \multirow{2}{*}{\AR{}} & Basic  &  & --- \\
      \cline[dashed]{2-4}
      & Derived &  & Ruled out by \ESU{} \\
    \end{tblr}
    \caption{Distinction matrix}
  \end{figure}
\end{note}

\begin{note}[Summarising]
  ???
\end{note}

\section{\nI{}, \gsi{}, and \aben{the}}
\label{sec:second-conditional}

\begin{note}[Redo of section]
  Seen \ESU{} and \adB{}.
  Now turn to \adA{}.
\end{note}

\begin{note}[Broad sketch of section]
  Introduce a general constraint on claiming support.
  The general constraint will relate to moving from general to specific ability information --- agent is not in a position to claim support for having specific ability from information and claimed support for general ability.
  However, initial statement and motivation apply to all instances of claiming support.
  After statement and motivation, show how the constraint relates to \adA{}.
  If so, agent lacks support for having specific ability, and does not have the option of claiming support for result of specific ability by \adA{}.
\end{note}

\begin{note}
  \color{red}
  \large

  The point of \nI{} isn't to find something that gets me something about ability.
  Given the assumptions made, it's fairly easy to figure out the problem.
  Rather, the point on \nI{} is to generalise the problem.
  The ideas we're appealing to aren't relying on anything in particular about ability.
\end{note}

\subsubsection{Overview of \nI{}}
\label{sec:ni-1}

\begin{note}
  We turn to the general limitation on claiming support.
\end{note}

\paragraph{Statement of \nI{}}

\begin{note}[\nI{}]

  \begin{restatable}[\nI{-}  --- \nI{}]{proposition}{propNI}\label{prem:ni}
    Let \(\phi\), \(\psi\) be propositions and \(v\), \(v'\) values:
    \begin{enumerate}[ref=\named{\nIacro{}:\arabic*}, series=nI_counter]
    \item\label{nI:claimed-support}
      Suppose it is the case that:
      \begin{enumerate}[label=\alph*., ref=\named{\nIacro{}1:\alph*}]
      \item \nIClauseClaimedSupport{}\nolinebreak
        \footnote{
          The agent has at some time in the past (perhaps a moment ago) claimed support for \(\phi\) having \(v\), and at the present time the agent continues to hold that \(\phi\) has value \(v\).
          In some cases, an agent may revise basis for claimed support.
          E.g.\ this may be the case when appealing to memory, a new source, etc.
        }
        % \item\label{nI:received-info} \nIClauseReceivedInfo{}
      \item\label{nI:psi-is-new} {\color{red} \(S\) did not have information about \(\psi\) when claiming support for \(\phi\).}
      \end{enumerate}
    \end{enumerate}
    And, suppose:
    \begin{enumerate}[ref=\named{\nIacro{}:\arabic*}, resume*=nI_counter]
    \item\label{nI:inclusion} \nIClauseInclusion{}
      \begin{enumerate}[label=\alph*., ref=\named{\nIacro{}2:\alph*}]
      \item\label{nI:inclusion:position} \nIClauseInclusionPosition{}\nolinebreak
        \footnote{
          Agent considers.
          This should be stressed.
          And, in most cases, it is mistaken that's at issue.
          For, failure of \(\psi\) condition doesn't raise any issue in particular for \(\phi\).
        }\(^{,}\)\nolinebreak
        \footnote{
          Also, it may be the case that the statement of \nI{} needs further constraints on what the relation is between the claimed support for \(\phi\) and claiming support for \(\psi\) (without appeal to the value of \(\phi\)).
          I don't see a clear problem at the moment.
          However, I do expand below.
        }
      \item\label{nI:inclusion:bound} \nIClauseInclusioBound{}
      \end{enumerate}
    \end{enumerate}
    For any reasoning such that:
    \begin{enumerate}[ref=\named{\nIacro{}:\arabic*}, resume*=nI_counter]
    \item\label{nI:going-by-value} \nIClauseValue{}
      \begin{enumerate}[label=\alph*., ref=\named{\nIacro{}3:\alph*}]
      \item\label{nI:going-by-value:phi} \nIClauseValuePhi{}
      \item\label{nI:goingbyvalue:psi} \nIClauseValuePsi{}
      \end{enumerate}
    \end{enumerate}
    Result of reasoning is not an instance of \(S\) claimed support that \(\psi\) has value \(v'\).
    \vspace{-\baselineskip}
  \end{restatable}
\end{note}

\begin{note}[What \nI{} amounts to]
  \nI{} is a restriction on the way in which an agent may claim support for some proposition when certain conditions obtain.
  The proposition consists of a collection of `background' conditions --- clauses~\ref{nI:claimed-support} to~\ref{nI:inclusion} --- and a limitation given those background conditions --- clause~\ref{nI:going-by-value}.

  \ref{nI:claimed-support} and~\ref{nI:inclusion} combine to ensure that if \(\phi\) having value \(v\), then claimed support for \(\psi\) having value \(v'\) would violate \eiS{}.
  {
    \color{red} In particular, \ref{nI:psi-is-new} and~\ref{nI:inclusion} combine to ensure that \(\psi\) would be an expectation of the claimed support for \(\phi\).

    Argument for~\nI{} is similar to argument from~\ref{sec:claim-supp-expect}.
  }
\end{note}

\begin{note}[Expanding on~\ref{nI:inclusion}]
  Indeed,~\ref{nI:inclusion} is point of interest.

  Main thing will be expanding on:

  \begin{quote}
    \vspace{-\baselineskip}
    \ideaEIS*
  \end{quote}

  In particular:

  \begin{quote}
    \vspace{-\baselineskip}
    \assuEIS*
  \end{quote}

  Interdependence between claimed support for \(\phi\) having value \(v\) and claiming support for \(\psi\) having value \(v'\) given \ref{nI:inclusion}.
  If \(\phi\) has value \(v\), then claimed support for \(\phi\) is \nmom{}.
  Hence, claimed support for \(\psi\) having value \(v'\) would be \nmom{}.
  So, given background, \(\phi\) having value \(v\) requires \(\psi\) having value \(v'\).

  Problem, as claimed support for \(\psi\) having value \(v'\) --- via claimed support for \(\phi\) having value \(v\) --- `requires' \(\psi\) having value \(v\).

  Though, leaves open claiming support that does not going via the claimed support for \(\phi\) having value \(v\).

  Two possibilities:
  \begin{itemize}
  \item Different approach to claiming support.
  \item Reasoning that removes expectation.
  \end{itemize}

  So, \nI{} is not particularly strong.
  However, not too weak either.
  \autoref{prop:CS-nai} with respect to second point.
\end{note}

\begin{note}[Intuitive idea]
  To bring out the intuitive idea:

  If agent considers possibility that they're \mom{}, then tension.
  For, \(\psi\), but `requiring' something agent is \mom{} about.
  Problem because if agent is \mom{}, then claimed support involves appeal to something which is no the case.
  Claimed support would be \mom{}.

  Hence, trouble with \autoref{assu:supp:independence}.

  This is fast.
  When turn to argument proper we will move more slowly.
  Further, offer variation on the argument which make use of additional stuff of \ref{nI:going-by-value}.\nolinebreak
  \footnote{
    Preview:
    Worry may be that \(\psi\) having value \(v'\) is not sufficiently involved in the reasoning.
    If from claimed support for \(\phi\), then get that this is \mom{}, and hence claiming support involves appeal to \mom{} premise, which is again sufficient for claimed support to be mistaken.
  }
\end{note}

\begin{note}[Plan]
  To start:
  \begin{enumerate}
  \item Core idea of \nI{}.
  \item Role in overall argument.
  \item Some intuition.
  \item Key points.
  \item Some connexions to the literature.
  \end{enumerate}
  Then, details.
\end{note}

\begin{note}[Kind of defeater]
  The basic idea behind \nI{} is that given clause~\ref{nI:claimed-support}, clause~\ref{nI:inclusion} captures a condition which \emph{undercuts} the agent claiming support as captured by~\ref{nI:going-by-value}.

  Mentioned undercutting {\color{red} \hyperref[first-mention-undercutting-defeater]{above}}

  If~\nIBackground{} hold, then the way of claiming support captured by~\ref{nI:going-by-value} is undercut.

  {
    \color{red}
    Specifically, undercutting arises from \ref{nI:inclusion} establishing a form of interdependence between claimed support for \(\phi\) and claiming support for \(\phi\).
  }
  Such that in order for agent to appeal to \(\phi\) having value \(v\) the agent must assume that \(\psi\) has value \(v'\).
  Hence, can't go to \(\phi\) without \(\psi\), which means that can't use \(\phi\) to claim support for \(\psi\).
\end{note}

\begin{note}[\eiS{}]
  Idea of an undercutting defeater is quite general.
  Here, we'll develop a specific account.

  Undercut because failure of \eiS{}.
  This doesn't say anything about whether \(\psi\) has value \(v'\).
  Rather, if \eiS{} doesn't hold then the agent doesn't get to claim support {\color{red} due to lack of reasoning regarding recognised possible defeater}.
\end{note}

\begin{note}[Point of \nI{} is to motivate `independent' of ability]
  Purpose of conditions is to get undercutting of this kind.
  Of course, work is done by \eiS{}.
  And, in some respects, it is easier to reason directly.

  However, for our purposes this won't do.
  Our goal is to apply \nI{} to ability.

  Two problems.
  First, whether one gets this kind of failure.
  \nI{} address this by narrowing down on a small set of conditions that we can check.
  Second, whether the kind of failure matters.
  \nI{} address this by motivating the failure in general.
\end{note}

\paragraph{Illustrations for intuition}

\begin{note}[A few illustrations]
  Let us now turn to a few illustrations before discussing \nI{} in further detail.

  We'll begin with a somewhat detailed illustration.
  \nI{} identifies a particular way in which an agent may fail to claim support, and the primary goal of the initial demonstration is to highlight why the agent would fail to claim support.
  Hence, the illustration treads a fine line between highlighting a problematic method, but not necessarily a problematic result.
  This is by design.
  And, I will continue to stress that \nI{} concerns a way of claiming support for some proposition, rather than the possibility of claiming support for some proposition.

  Following two illustrations will be variations on the initial.

  Still, it may be helpful to observe how \nI{} relates to an intuitively problematic result.
  Therefore, we will provide an additional, simple, illustration of a failure to claim support.

  The final illustration in the trio will complement the initial par of illustrations by highlighting an instance where~\nI{} does not apply.

  \phantlabel{dogmatism-wrt-nI}
  The reader may note structural similarities between these illustrations and \citeauthor{Kripke:2011wv}'s Dogmatism paradox.
  We will discuss the relation after the illustrations.
\end{note}

\begin{note}[Brief illustration of \nI{}]
  The first illustration considers theories and counterexamples.

  \begin{illustration}\label{ill:CE:main}
    Suppose a researcher have constructed a theory of some general phenomenon.

    The theory seems to capture the phenomenon, and the researcher has claimed (inductive) support that the theory is adequate by applying it to various instances of the general phenomenon.
    Even if the theory isn't adequate, the theory has been (seemingly) successful applied to sufficient specific instances of the phenomenon.
    Hence, even if \mom{}.

    However, as the phenomenon is a \emph{general} phenomenon it also makes various predictions about what must happen in all other instances to which the researcher has not (yet, at least) applied the theory to.

    Hence, there is a possible counterexample to the theory associated with each instance the researcher has not (yet) applied the theory to.
    If some particular instance does not conform to the theory, the theory is inadequate.
    Conversely, if the theory is adequate, every particular instance of the phenomenon conforms the theory.
    In other words, if the theory is adequate, then there are no counterexamples to the theory.

    Of course, it may be simple to revise the theory is a counterexample exists, and the fundamental ideas of the theory may remain sound (Cf.~\textcite{Bonevac:2011tz}).
    And, the theory may have sufficient resources to explain why any apparent counterexample is not a counterexample.
    Yet, it remains the case that the theory would need to be revised in light of a counterexample.

    Now, to summarise, the researcher may claim support for two propositions allow the agent to claim support that there are no counterexamples.

    \begin{itemize}
    \item The theory is adequate, and
    \item If the theory is adequate, then there are no counterexamples.
    \end{itemize}

    At issue is whether the researcher may claim support that there are no counterexamples to the theory from the claimed support for the two propositions in the following way:

    \begin{itemize}
    \item I have claimed support that the theory is adequate.
    \item So, given the claimed support, theory is adequate.
    \item Therefore, as the theory is adequate, given the claimed support, it follows that there are no counterexamples.
    \item Hence, I claim support that there are no counterexamples to the theory.
    \end{itemize}

    Seems problematic.
    Claimed support that the theory is adequate is qualified by the possibility of counterexamples.
    {
      \color{red}
      Note, agent is, here, only claiming support that there are no counterexamples.
      And, claiming support may be \mom{}.
      So, it does not follow that the agent is ruling out the possibility of counterexamples to the theory.
      Plausible that the agent \emph{may} claim support.
      Problem is the way in which the agent goes about this.
    }

    {
      Even if not convinced about support, this way of claiming support seems problematic.
      Relying on theory being adequate.
      However, if this is the case, then no possible counterexamples.
      Issue is that such counterexamples are possible given the state of your claimed support.
      Hence, claiming support in this way seems to take for granted that there are no counterexamples.
    }

    Problem is that the reasoning only works if there are no counterexamples.
    If there are counterexamples, misled.
    Hence, problem to go from the theory is adequate.
    However, without this step, researcher doesn't get to no counterexamples.

    So, this is \eiS{}.
  \end{illustration}

  So, relation between theory and counterexample \emph{undercuts} using way of using theory to get no counterexample.

  Now, given that the researcher has claimed support that the theory is adequate, the researcher may \emph{expect} that there are no counterexamples to the theory.
  And, it doesn't follow that the researcher may not claim support.
  Specific way --- reasoning captured by~\ref{nI:going-by-value}
  Plausible that details of the theory provide some way of claiming support.

  Indeed, it seems the researcher is require to take the alternative path --- to show that the proposed counterexample is accounted for by the contents of the theory, regardless of whether the theory is true.

  Fault here is with respect to \eiS{}.
  {
    \color{red}
    Here, conditions of~\ref{nI:inclusion} are satisfied, but we did not explicitly appeal to them.
    Purpose of~\ref{nI:inclusion} is conditions sufficient for this kind of problem to arise.
    So, to do in argument for \nI{} is to develop is why~\ref{nI:inclusion} does something similar.
    Upshot is that \nI{} is general.

    In the third illustration, we'll see why the way of claiming support is okay in some cases.
  }
  Difficult part is to account for why~\ref{nI:inclusion} sets things up and ensures that things don't go too far.
\end{note}

\begin{note}[Idea main part of \nI{} works]
  As noted above, it is unclear whether or not there may be some way for the researcher to claim that there are no counterexamples to the theory.
  And, if there is some way for the researcher to claim that there are no counterexamples to the theory, one may be inclined to wonder whether there really is a problem with claiming support in the way outlined by~\ref{nI:going-by-value}.

  In other words, one may be wondering whether \eiS{} is a plausible constraint on claiming support.
  We gave a general argument for \eiS{} in~\autoref{sec:abil-access-supp}.
  However, it may help to see how the issue highlighted relates to an intuitively problematic instance of reasoning, regardless of how support is claimed.

  \begin{illustration}\label{ill:CE:colleague}
    Suppose a colleague has studied the researcher's theory, and they (the colleague) thinks they have found a counterexample.

    The colleague has informed the researcher that they think they have observed a counterexample.

    However, the colleague has not provided the researcher with any further details about the counterexample.

    Now, the conditional of interest may be made more precise:
    \begin{itemize}
    \item If the theory is adequate, then the colleague has failed to identify a counterexample to the theory.
    \end{itemize}

    Now, let's replicate the way of claiming support from before.

    \begin{itemize}
    \item I have claimed support that the theory is adequate.
    \item So, given the claimed support, theory is adequate.
    \item Therefore, as the theory is adequate, given the claimed support, it follows that the colleague has failed to identify a counterexample to the theory.
    \item Hence, I claim support that the colleague has failed to identify a counterexample to the theory.
    \end{itemize}
  \end{illustration}

  I take this illustration to be intuitively problematic.
  In short, if claim support, then doesn't need to examine counterexample to claim support that it is not a counterexample.

  Possible response is that researcher does claim support, but information colleague impacts claimed support for theory.
  However, this is also puzzling.
  Researcher has no information.
  Hence, if retain confidence, then equally against counterexample.
  And, if does not retain confidence, then down the theory in a way that seems implausible.

  Seems, instead, that claimed support for theory persists, but that this doesn't extend to counterexample.\nolinebreak
  \footnote{
    Inclined to apply this to previous illustration.

    However, there's a difference between two illustrations.
    Here, someone (the colleague) has reason to think there is a counterexample, and this seems a sufficiently important difference to draw any quick conclusions.
    And, as we don't require a resolution to this issue, I won't explore further.
  }

  Perhaps more detail is needed.
  I have some doubts that claiming support is always bad.
  However, clearer that developed in a way such that problem remains.

  Now, seems that the researcher doesn't get to claim support because if counterexample, then theory is bad.
  Hence, requires that counterexample is not true in order to progress.
  But, then, doesn't make the move regardless of whether or not there is a counterexample.

  So, it seems \eiS{} does the work.
\end{note}


\begin{note}
  Undercuts using \(\phi\) for \(\psi\).
  Same problem, failure of \eiS{}.

  For, the agent has already assumed \(\psi\).

  Problem is that the agent doesn't get to claim support for \(\psi\) because fail the \eiS{} thing.
  If \(\psi\) isn't really the case, then reasoning collapses.
  Key thing about our understanding of claimed support is that it holds up even if the agent is \mom{} about the value of the proposition.

  {
    \color{red}
    Note:
    There's possible tension here.
    It seems that if the first illustration is okay, then this (second) illustration should also be okay.
    Maybe.
    But, this is too quick.
    Additional information here.

    Now, still some difficulty, as I think \EAS{} might apply to the first.
    So, shouldn't it apply to this?
    Well, no.
    For, \EAS{} only suggests possibility in some cases.
    Fine to think of this additional information as constraint on appeal via ability.
    For, if the colleague thinks they've found a counterexample, then this suggests a problem with the agent's ability.
  }
\end{note}

\begin{note}[Variation where \nI{} does not apply]
  \begin{illustration}\label{ill:CE:testimony}
    Suppose the researcher has published a paper containing the details of the theory.

    Our attention now turns to a novice who has read far enough into the paper to understand, at least, the general phenomenon that the theory applies to and that the researcher has claimed inductive support for the theory.
    We'll also assume that the novice does not possess the expertise required to apply the theory.\nolinebreak
    \footnote{
      Though I don't think this assumption is important.
    }

    The novice is thinking about instances of the general phenomenon, and identifies one.

    The conditional of interest is:
    \begin{itemize}
    \item If theory is adequate, it accounts for this instance of the phenomenon.
    \end{itemize}

    Of course, the novice also recognises that the theory is inadequate if it  does not account for the particular instance of the phenomenon.
    Still, the novice claims support in the familiar manner.

    \begin{itemize}
    \item I have claimed support that the theory is adequate (this time by reading a published paper).
    \item So, given the claimed support, the theory is adequate.
    \item Therefore, as the theory is adequate, given the claimed support, it follows that the theory accounts for this instance of the phenomenon.
    \item Hence, I claim support that the theory accounts for this instance of the phenomenon.
    \end{itemize}
  \end{illustration}

  In contrast to the previous illustrations, it seems the novice may claim support in such a way.

  Possibility of being either \mom{} remains.
  Still, not in position to reason through theory and phenomena.
  Hence, claiming support from something like status of peer review --- or testimony.
  And, not accounting would not show peer review is bad.
\end{note}

\begin{note}
  These three illustrations.
  First, kind of scenario that's the main interest.
  Where claiming support in a certain way seems problematic, even if it not clear that the agent may claim support in some other way.

  To stress the problem, considered a cleaner case, where it seems agent may not claim support, and argued that same problem is a plausible account of why.

  Third illustration, way of claiming support is okay.
  As all instances of \nI{}, and hence the previous two illustrations, focus on particular way of claiming support illustrated that it's okay.
\end{note}

\begin{note}[Intuition]
  In short, \nI{} captures a limitation: An agent is not in a position to claim support for some proposition \(\psi\) when circumstances are such that the claimed support requires (from agent's point of view) that the agent is already in position to claim support for \(\psi\).

  No claiming support for\(\psi\) if failure to establish support for \(\psi\) independently of the value of \(\phi\) would reveal problem with the support claim for \(\phi\).

  Hence, \nI{} focuses on when an agent may claim support for some proposition by noting that (from the agent's perspective) that the value of the proposition is determined by further propositions the agent has claimed support for.

  Some other way of claiming support for \(\psi\).
  However, not merely an alternative path, but an alternative path that must be possible given claimed support for \(\phi\).

  Issue is that given \ref{nI:claimed-support} and~\ref{nI:inclusion}, agent expect that they have the resources, and hence expects that \(\psi\) is the case.

  So, that \(\phi\) has value \(v\).
  In doing so, resources to claim support for \(\psi\) has value \(v'\).
  Hence, that \(\psi\) has value \(v'\).
  So, \(\psi\) having value \(v'\) is a requirement on claimed support for \(\phi\) being any good.
  However, no support claimed for \(\psi\) having value \(v'\).

  In cases of reasoning with a conditional, such as the illustrations given, that value of \(\phi\) constrains value of \(\psi\) is in general helpful information, but in these specific cases it does not help the agent claim support for \(\psi\) having value \(v'\) because if \(\psi\) isn't already so constrained, then no appeal to \(\phi\) having value \(v\).

  Similar to other principles, failure because establishing something that needs to be the case in order to be in a position to establish.
\end{note}

\paragraph{Moving on\dots}

\begin{note}[Task]
  Two important tasks with respect to arguing for \nI{}:

  \begin{itemize}
  \item \ref{nI:inclusion} and~\ref{nI:inclusion:position}, how these function.
  \item And, why \nI{} holds.
  \end{itemize}

  Then, additional details.
\end{note}

\begin{note}
  Before continuing I would like to stress two general points.
  Follow up on relation to \citeauthor{Kripke:2011wv}'s Dogmatism paradox.
  Link to similar principles in the literature.
\end{note}

\begin{note}
  Two quick points to stress.
\end{note}

\begin{note}[First point to stress]
  First, \nI{} outlines sufficient conditions for a limitation on claiming support to obtain.
  Hence, there may be other limitations on claiming support.
  For example, \ESU{} implies a limitation of claiming support --- an agent may not claim support by appeal to some premises or step of reasoning that they have not used.

  Similarly, there may be other limitations on claiming support with obtain.
\end{note}

\begin{note}[Second point to stress]
  Second~\ref{nI:going-by-value} is a limitation of a \emph{way} of claiming support.
  So, \nI{} does not imply that the agent is not in a position to claim support for \(\psi\), only that one way of claiming support is ruled out when the other clauses of \nI{} hold.
  Hence, \nI{} is compatible with the agent claiming support for \(\psi\) having value \(v'\) --- so long the agent doesn't follow the pattern captured by~\ref{nI:going-by-value}.

  Indeed, I take the primary upshot of \nI{} to be a demand for understanding alternative ways of claiming support when each clause of \nI{} holds and it seems that the agent may claim support for \(\psi\) having value \(v'\).
  And, after arguing for \nI{} our attention will turn to examining how an agent may claim support by \EAS{} when clauses~\ref{nI:claimed-support} obtain.

  For now, however, our focus is on arguing for \nI{}.

  It is perhaps also helpful to flag that the following discussion of \nI{} will relate previous details on claiming support (in particular section~\ref{sec:abil-access-supp}) but will proceed without consideration of ability.
  Our goal is to motivate \nI{} as a general limitation on claiming support, and in turn draw a consequence from such a limitation when applied to ability.
\end{note}

\begin{note}[Literature]
  We will highlight the contrast between \nI{} and similar principles in~\autoref{sec:contr-other-cond}, below.
  In particular, \citeauthor{Wright:2011wn}'s work on transmission failure (\Citeyear{Wright:2003aa,Wright:2011wn}) and \citeauthor{Weisberg:2010to}'s No Feedback condition (\Citeyear{Weisberg:2010to}).

  For now, it may be helpful to highlight that \nI{} does not deny that any agent may claim support for \(\psi\) having value \(v'\).
  Rather,~\ref{nI:going-by-value} (only) denies that the agent may claim support for \(\psi\) in a specific way.
\end{note}

\newpage

\subsubsection{Details of --- and argument for --- \nI{}}
\label{sec:details-ni}
\label{sec:re-do-ni}

\paragraph*{Argument outline}

\begin{note}
  Our task is now to demonstrate why~\ref{nI:going-by-value} is the case when \nIBackground{} obtain.

  We begin with a brief statement of how the argument is structured, together with the basic idea behind the argument.
  Then, we turn to the argument proper.
  Finally, we finish with a handful of observations.

  In the following section (\ref{sec:illustrations-ni}), with the argument complete, we will consider a number of additional illustrations of \nI{}.
  Then, in~\ref{sec:contr-other-cond} we will contrast~\nI{} to a pair of similar conditions from the literature.
  And, at last, we will apply~\nI{} to ability in~\autoref{sec:ni-ability}.
  In particular, why~\nI{} applies to ability paired with \adA{} (\S\ref{sec:ni-ability:adA}) but not when ability is paired with \adB{} (\S\ref{sec:ni-ability:adB}).
\end{note}

\begin{note}
  Key proposition for \nI{} is:

  \begin{quote}
    \vspace{-\baselineskip}
    \propCSNai*
  \end{quote}

  For, our goal is to show that move to \(\phi\) brings \(\psi\) with.
  Hence, getting to \(\psi\), even if \(\psi\) is not explicit, means that we get a failure of claiming support.

  To get here, relationship between claimed support for \(\phi\) and claimed support for \(\psi\).

  \begin{enumerate}
  \item Claimed support for \(\psi\) having value \(v'\) is an expectation of the claimed support for \(\phi\) having value \(v\).
  \item Going to \(\phi\) having value \(v\) includes \(\psi\) having value \(v'\) as a result of previous.
  \end{enumerate}
  Hence, \ref{prop:CS-nai} applies.
  More broadly, \eiS{}.
  Claiming support for \(\psi\) so recognise \(\psi\) might not be the case.
  However, claimed support is such that it requires \(\psi\) is the case.
  Hence, no account of \(\psi\) given this possibility.
\end{note}

\begin{note}
  In broad outline.

  \begin{itemize}
  \item Claimed support for \(\phi\).
  \item No reasoning about \(\psi\).
    \begin{itemize}
    \item No claimed support for \(\psi\).
    \item No account of why \(\phi\) regardless of \(\psi\). (I.e.\ weaker than claimed support, if think this is possible).
    \end{itemize}
  \item From inclusion, this means that claiming support for \(\psi\) is an expectation of claimed support for \(\phi\).
  \item Consequence of expectation is that \(\psi\) is a requisite of the move to \(\phi\).
    In sense that:
    \begin{itemize}
    \item Move to \(\phi\) is not compatible with possibility that \(\psi\) is not the case.
    \item Also, as result of this, move to \(\phi\) requires \(\psi\) to be the case.
    \end{itemize}
  \item So, reasoning to \(\psi\) is not compatible with the possibility of \(\psi\) not holding.
  \item And, as way of claiming support needs this move, failure.
  \end{itemize}

  This is somewhat complex.
  So, to summarise.
  The key problem is that reasoning to \(\psi\) is not compatible with the possibility that \(\psi\) does not have value \(v'\).
  And, as a result of this the reasoning is not an instance of claiming support.

  Of course, still an instance of reasoning, and result may be of interest.
  However, no independence.

  {
    \color{red}
    Compare to conditional with contraposition.
  }
\end{note}

\begin{note}
  To do is expand on this quick sketch.

  At issue is not only how these clauses link together, but what the clauses amount to.
  Both these things are important.
  Still, to simplify matters split in to two sections.

  In the first section, walk through the clauses --- in particular \ref{nI:inclusion}.
  In the second, link the clauses together.

  Read these in either order.
  Split is in part so that the first section may expand on some aspects of the clauses which are important for understanding the scope of \nI{}, but are not required to obtain the key consequence.
\end{note}



\begin{note}
  Following outline given.
  Assume clauses hold.

  Main thing is `requisite'.

  So,
  \begin{enumerate}
  \item Preliminaries
  \item Requisite
  \end{enumerate}
  Then, tension.
\end{note}

\subsubsection{Two things}
\label{sec:nI:arg:clauses}

\begin{note}
  Two subsections.

  First,~\ref{nI:inclusion}.
  Second,~\ref{nI:going-by-value}.

  Add further information regarding these.
  Not required for argument.
\end{note}

\paragraph{~\ref{nI:inclusion}}

\begin{note}[\ref{nI:inclusion}]
  Restated:
  \begin{quote}
    \begin{enumerate}
    \item[\ref{nI:inclusion}]
      \nIClauseInclusion{}
      \begin{enumerate}
      \item[\ref{nI:inclusion:position}] \nIClauseInclusionPosition{}
      \item[\ref{nI:inclusion:bound}] \nIClauseInclusioBound{}
      \end{enumerate}
    \end{enumerate}
  \end{quote}
\end{note}

\begin{note}
  \begin{itemize}
  \item \ref{nI:inclusion:position}.
    \begin{itemize}
    \item \(\psi\) having value \(v'\) is the case, granting claimed support \(\phi\) having value \(v\) is the case goes through.
    \item Not the result of appealing to \(\phi\) having value \(v\).
      \begin{itemize}
      \item So, agent considers it possible to claim support for \(\psi\) having value \(v'\) without \(\phi\).
        Hence, way of removing expectation of \(\psi\) from claimed support for \(\phi\).
      \end{itemize}
    \end{itemize}
  \end{itemize}
\end{note}

\begin{note}
  {
    \color{red}
    The point of these parenthesis is that the relevant way of claiming support isn't the one that's going to give rise to the issue.
    Otherwise, it's somewhat trivial.
    Still, I feel there is a much better way of expressing the parenthetical parts of these clauses.
    I mean, the point would be that any instance is always going to be self defeating.
    But, the problem isn't \(\phi\) having value \(v\), rather it's there being some other instance that `forces' \(\psi\).
  }
  {
    \color{blue}
    Well, the interdependence here is really about the way of claiming support for \(\phi\) requiring that agent also claims support for \(\psi\).
    In this sense, \(\psi\) is a presupposition for the method of claim support.

    If agent requires implication, then this should break down, because it won't be the case that the agent needs \(\psi\) to hold up in order for the way in which the agent claims support for \(\phi\) to be okay.

    In this sense, the difficulty is whether these two conditions really capture this idea.
    It's the way of claiming support that's related, rather than \(\phi\) and \(\psi\).

    Could I make this explicit?
    Position to claim for \(\psi\) in same way as \(\phi\).
    And, that's successful only if claiming for \(\psi\) is also good.
  }
\end{note}

\begin{note}[Two clauses]
  The two sub-clauses of~\ref{nI:inclusion} are separated by~\ref{nI:inclusion:position} being a conditional with an antecedent and consequent that describe present circumstances while \ref{nI:inclusion:bound} is a conditional with an antecedent that describe present circumstances and a subjunctive consequent.

  {
    \color{red}
    The key this here is that if claimed support for \(\phi\) is not misled, then \(\psi\) has value \(v'\) --- that's the importance of being in position.
  }

  In turn, each sub-clause is stated from the perspective of possible defeaters, but as a result both sub-clauses carry certain implications if the agent is confident that they have successful claimed support for \(\phi\) having value \(v\).

  First, as stated, \ref{nI:inclusion:position} ensures that the agent considers their claimed support \(\phi\) having value \(v\) is mistaken or misled if they are not in a position to claim support that \(\psi\) has value \(v'\).
  In turn, if the agent is confident that they have claimed support \(\phi\) having value \(v\) then the agent should be equally confident that they are in a position to claim support that \(\psi\) has value \(v'\)

  Second,~\ref{nI:inclusion:bound} ensures that the agent is confident that if they were to claim support for \(\psi\) having value \(v'\) then such claimed support for would be \mom{} if their claimed support for \(\phi\) having value \(v\) is \mom{}.\nolinebreak
  \footnote{
    Note, however, that the subjective element of this contraposed form is restricted to the antecedent of the conditional.
  }
  In turn, if the agent is confident that they have claimed support \(\phi\) having value \(v\) then the agent should be equally confident that claimed support for \(\psi\) having value \(v'\) would not be \mom{} if they were to do so.

  The two sub-clauses are closely related, but are distinct.
  It is possible for either to hold without the other.

  \begin{illustration}
    Suppose taught addition.
    Has been told that multiplication reduces to addition, but has not been informed of the details.
  \end{illustration}
  \ref{nI:inclusion:bound} holds but~\ref{nI:inclusion:position} does not.

  \begin{illustration}
    Suppose calculator from scratch.
    Good with arithmetic, but less good with programming.
  \end{illustration}
  \ref{nI:inclusion:position} holds, but~\ref{nI:inclusion:bound} does not.
  For, a whole bunch of additional stuff.
  However, revise scenario so that~\ref{nI:inclusion:bound} does hold.

  So, distinct.
  \ref{nI:inclusion:position} in a position and~\ref{nI:inclusion:bound}, binds.
\end{note}

{
  \color{red}
  Seen examples above.
}

\begin{note}
  `Confidence'

  Then, each sub-clause separately.

  Finally, link to literature.
\end{note}

\begin{note}[`Confidence']
  Our use of the term `confidence' does not require the agent to have claimed support for the conditional content of \ref{nI:inclusion}.
  Nor does out use of `confidence' imply that the claimed support for \(\phi\) \emph{is} mistaken or misled given the identified conditions.
  We are interested only in what makes sense from the agent's perspective.
  Nearby reformulations of \ref{nI:inclusion} may also be true, but confidence is sufficient to recognise a problem.
  To illustrate: If I am confident that the water is poisoned, then regardless of whether I claimed support for the water being poisoned, I will not drink it.
\end{note}

\subparagraph{\ref{nI:inclusion:position}}

\begin{note}[Points that will be covered]
  We will focus on two parts of \ref{nI:inclusion:position}.

  First, on what it is for an agent to be in a position to claim support for some proposition (given present context), as this is an unfortunate source of imprecision.

  Second, we'll clarify the restriction that the agent does not appeal to \(\phi\) having value \(v\) --- though this will be further explored when discussing~\ref{nI:inclusion:bound}.
\end{note}

\begin{note}
  First, an admission.
  Key role here is that get a problem with the claimed support for \(\phi\) having value \(v\).
  Possible that some other condition could do the work.
  However, this is sufficient for the purposes of this paper.
\end{note}

\begin{note}[`(given present context)']
  Imprecise.
  No clear account of what it is to `be in a position' nor what `present context' amounts to.

  Unfortunately, I have no simple characterisation for either.
  Best I have to offer is that the agent is not prevented from claiming support by lack of resources.
  However, it seems to me that isn't really substantially different.
  Resource is broad enough to mean whatever is required for the agent to claim support.
  And, bound to context.

  Instead, walk through considerations with respect to some scenarios.
  Consider issues.
  Argue that don't need to do better than imprecision.
\end{note}

\begin{note}
  Pair of easy cases.
  Clarify to some extent in a position.

  Then, some harder cases.
\end{note}

\begin{note}[Flavours]
  Consider variations on a case where information relating \(\phi\) and \(\psi\) comes from a third party.

  You'll enjoy this flavour of ice cream.

  Claimed support from testimony, roughly.

  Seems fine when discussing things on the train.

  More difficult when tasters are available.
  Here, \ref{nI:inclusion} holds.
\end{note}

\begin{note}
  Difference is that in the first, no way of checking.
  In the latter, there's a way of checking.

  Only issue here is `testimony'.
  With respect to testimony, the response is that in cases of testimony, one can be thought of as appealing to a general truthful property, rather than always truthful.
  So, it's not obvious that testimony is always going to give rise to an instance of \ref{nI:inclusion}.
\end{note}

\begin{note}
  Receiving a letter.
  Unmarked, okay, it's for me.
  Marked, check the address.
\end{note}

\begin{note}[More difficult]
  \dots
\end{note}

\begin{note}[Illustration, testimony]
  To illustrate, consider expert testimony to a layperson.
  Suppose you, the expert, have testified to me, the layperson, that there are exactly five intermediate logics that have the interpolation property.\nolinebreak
  \footnote{Cf.\ \textcite{Maksimova:1977un}}
  From this it follows that there is an intermediate logics that has the interpolation property.
  However, I am quite confident that I would not be in a position to claim support for the latter proposition without your testimony.
  Given that I do not have the expertise involved, any failure by me to claim support that there is a intermediate logic with the interpolation property is uninformative.
  Likewise, given that I am a layperson I'm not in a position to rule out that there aren't intermediate logics with the interpolation property, and therefore I may consider this a potential defeater to your testimony.\nolinebreak
  \footnote{
    Additional example: reports of internal states.

    I have a virus scanner.
    Run this on your pc.
    Also, a test pc.
    Test PC contains a know virus, so if the virus scanner is good, then it will identify infection.
    However, no relation between your PC and my test PC.
    All that would be established is that the scanner is not good for claiming support.
    }

  Still, given \ref{nI:claimed-support}, agent may expect \(\psi\) to have value \(v'\), and may claim support.
  And, may expect to have the resources to claim support for \(\psi\) without appealing to \(\phi\) having value \(v\).
  To illustrate, suppose you and I are both experts.
  You claim to have developed a sound and complete proof system for an logic and presented me with a paper containing the system and a proof.
  Given that I have the paper and the expertise, I am confident that I would be mistaken or misled by your testimony if I am not in a position to claim support that the system is sound and complete by working through the paper.\nolinebreak
  \footnote{
    Here, complexity of understanding of having resources shows.
    For, it may be that the reader learns something new, a lemma etc.\ which could be considered a new resource.
    Likewise, one may think that it's fine to continue to follow testimony given a problematic proof as one is confident that the prover has the resources to revise the proof.
    If so, not clear whether conditional holds, and will depend having resources.
    If proof synthesises resources, then may still hold.
    If proof introduces new information, then conditional does not hold.

    No clear answer for these cases.
    Intend to be compatible with your understanding of resources.
    Will only take a stance on this when applying.
  }
\end{note}

\begin{note}
  Even more difficult
\end{note}

\begin{note}
  Coworker.
  Rely on colleague, as the agent doesn't have access to the file.
  But, access is granted quickly after hearing from the colleague.
\end{note}

\begin{note}
  These cases are harder within the broader context of \nI{}.
  Deny \RBV{}.
  Issue is that in both cases, result seems excessive.

  Well, first thing to do is to check that the agent really is claiming support.
  Fine, it seems, for the agent to stop at claiming support for the other agent, and not going any further.
  See no reason to hold that the agent must claim support.

  Still, this isn't quite satisfactory.
  Doesn't seem that bad, and the above suggestion requires a careful understanding of when an agent is required to claim support.
  So, what if the agent does claim support?

  As noted above, views on testimony can sort this out.
  First, by going for `weak' testimony.
  Second, by breaking \ref{nI:inclusion:bound} is the testimony turns out to be a mistake.
  Look, it's not obvious why it would make sense for the agent to claim support, but the point is that \nI{} wouldn't hold.

  Alternatively, Simple restriction is for first time claiming of support.
  Difficulty is a variation of the expert case.
  It isn't obvious that one gets to claim support for the stuff learnt as a layperson when one develops expertise.
  For example, translation between languages.
  Claim support for simple translation.
  When fluent, seems claimed support is distinct, based on broader understanding of the language.
  Not relying on simplifications in learner's dictionary.

  However, other ways of claiming support may also work.
  Arguing for one such way.
  Besides this, \RBV{} is quite strong.
  And, who knows about other types of reasoning.
  In particular, ways in which reasoning \adA{} might work.
\end{note}

\begin{note}[Uninspiring]
  These responses aren't particularly inspiring.

  However, let's look at this from a different angle.
  What's going to follow from insensitivity to context?
  End up with claiming support that does not depend on whether or not the agent is in a position to deal with defeaters.

  Well, the first option is that these never matter.

  Some kind of built in support.
  This comes up with ~\citeauthor{Pryor:2012tq}'s dogmatism (\cite{Pryor:2000tl,Pryor:2012tq}) and various ideas about entitlement (Wright, Burge, etc.)
  For example, Pryor's dogmatism for perception, just having the experience is good enough.
  Question about these kinds of defeaters.
  Reads to me that these kinds of things mean that \ref{nI:inclusion} will never hold.

  Question is whether this extends to all cases, so that \nI{} is trivial, but before pressing this seems too strong.
  Problems in various cases.
  The red room, but in the corner is a switch, flipped to off, but says it's broken.

  Context makes a difference.
  So, to this extent, looks as though there's going to be difficult cases.
\end{note}

\begin{note}[Following doesn't depend on difficult cases]
  Of course, this isn't a general defence of the clauses.
  Rather, that such difficulties can't be avoided.
  Upshot here is that we aren't really interested in such difficult cases.
\end{note}

\begin{note}[`(without appealing to \(\phi\) having value \(v\))']
  The parenthetical clause `(without appealing to \(\phi\) having value \(v\))' ensures that \ref{nI:inclusion} may only be true when the agent is confident that they are in a position to claim support for \(\psi\) having value \(v'\) independent of whether \(\phi\) has value \(v\).

  In this respect, \ref{nI:inclusion} requires an independent check on the claimed support for \(\phi\) having value \(v\).\nolinebreak
  \footnote{
    Note also that without the parenthetical clause, \nI{} would deny the possibility of any instance of the reasoning described in \ref{nI:going-by-value}.
  }

  Indeed, not being in a position to claim support for \(\psi\) having value \(v'\) (without appealing to \(\phi\) having value \(v\)) as a potential defeater to claimed support for \(\phi\) having value \(v\) is distinct from the potential defeater of \(\psi\) not having value \(v'\).
  For, an agent may consider \(\psi\) not having value \(v'\) is a potential defeater given \(\phi \rightarrow \psi\) while being confident that they could not be in a position to claim support for \(\psi\) having value \(v'\) without claimed support for \(\phi\) having value \(v\).

  \begin{itemize}
  \item Illustration
  \item No restriction on why conditional is true.
    \begin{itemize}
    \item Distinction between two ways in which it might be true.
    \end{itemize}
  \end{itemize}
\end{note}

\begin{note}[Inclusion and Association]
  \color{red}
  The illustrations provided offer some intuition, and it seems these will have to do.
  For example, one may consider `in a position' to mean that the agent does not require any novel resources to claim support.
  However, an agent may need to synthesise more fundamental concepts when following a proof, and it is unclear whether the synthesis is `novel information'.
  Similarly, it is difficult to say what the present context is when an agent may phone a friend as a source of testimony.
  In some cases, corroborating testimony may be sufficient to claim support (another plausible instance of `novel information'), while in other cases at issue may be the agent's own understanding (e.g.\ with respect to cases of Inclusion).
  In defence of this latent ambiguity, the specifics will not matter when arguing for the truth of \nI{}.
  And, I suspect the cases to which we apply \nI{} will be sufficiently clear cut.
\end{note}

\subparagraph{\ref{nI:inclusion:bound}}

\begin{note}[Inclusion and Association]
  For \ref{nI:inclusion:bound} we will consider two general ways in which \ref{nI:inclusion:bound} may be true, and provide examples for both.

  The task, then, is to account for why claimed support for some proposition being \nmom{} may imply that claiming support for some other proposition would be \nmom{}.

  We term the ways \incl{} and \asso{}, respectively.
\end{note}


\begin{note}[Inclusion]
  \incl{} is when the same (primary) resources used to claim support for some proposition may be (re)applied to establish a distinct proposition.

  To see why \incl{} leads to instances of~\ref{nI:inclusion:bound} suppose:
  The agent is confident that support claimed for the initial proposition is \nmom{}.
  And, the agent is confident that \incl{} holds with respect to the initial proposition and some other proposition.
  In turn, the agent will use the same resources to claim support for the distinct proposition.
  Therefore, if the agent is confident that the claimed support is \nmom{} for the former proposition then the claim supported must be \nmom{} for the latter proposition, else the agent should not be confident that their claimed support for the former proposition is \nmom{}.

  \begin{illustration}
    Consider claimed support that \(6^{2} \times 6^{3} = 6^{5}\).
    Support has been claimed by understanding basic properties of exponents.
    Hence, an agent may be confident that they are in a position to claim support that \(3^{15} \times 3^{12} = 3^{27}\).
  \end{illustration}
  Indeed, working through problem exercises in a textbook is way of ensuring that one has understood such principles.
  Not that textbooks typically ask for the working out.\nolinebreak
  \footnote{
    Though sometimes.
    For a highly specific example, consider constructing canonical models to prove completeness for various normal modal logics.

    The exercises in textbooks such as~\citetitle{Blackburn:2002aa} require the reader to consider a specific system, so there's no surprise that, e.g., \textbf{K1.1} is sound and complete with respect to the class of frames with a relation function that mirrors a partial function.
    Rather, the task of the exercise is to ensure the reader understands how to reason with canonical models, and if the reader has claimed support with respect to \textbf{K1.1} which is \nmom{} then they should be confident that their claimed support will be \nmom{} when they tackle \textbf{K4.3}.

    See \textcite[210]{Blackburn:2002aa} for the respective exercises.
  }

  {
    \color{red}
    Note here that this is the sort of thing that seems most likely in counterexample type cases.
  }
\end{note}

\begin{note}[\asso{}]
  \asso{} is when claiming support for some proposition ensures the agent is in a position to appeal to some distinct collection of resources for some other proposition.

  \begin{illustration}
    Example here is something like storing a guide in a document.
    Here, the agent has created the document, \(\phi\) is just that the document has all of the info.
    So, if document is good, then contents for \(\psi\).
    This is different, as \(\phi\) is a now a check on the stuff in the document working out.
  \end{illustration}

  Also, ice cream example from above.
\end{note}

\begin{note}[Ways in which \ref{nI:inclusion} may fail to hold]
  Finally, the illustrations given have focused on instances for which \ref{nI:inclusion} holds.
  It is important that there are instances where \ref{nI:inclusion} holds, but it is equally important not to suggest that these are in abundance.
\end{note}

\begin{note}[Failures for~\ref{nI:inclusion:position}]
  There are various ways in which \ref{nI:inclusion} may fail to hold.
  For example, if \(\phi\) is sufficiently general or probabilistic.
  If so, not having the resources to claim support \(\psi\) may not establish much.
\end{note}

\subparagraph{\ref{nI:inclusion:position} and~\ref{nI:inclusion:bound} combined}

\begin{note}[Failure for zebra case]
  Here, not obvious that holds for zebra case, as it's not clear there is an alternative.
\end{note}

\begin{note}[Literature]
  {
    \color{red}
    Place this here as it helps clarify why~\ref{nI:inclusion:position} and~\ref{nI:inclusion:bound} seem to work well together.
  }
  The circularity here is similar to that proposed by~\cite{Sgaravatti:2013wu}

  \begin{quote}
    \begin{itemize}
    \item[(JBA)] An argument A is circular relative to an evidential state E iff in order for a subject S in E to have a justified belief in each one of A’s premisses, it is necessary that S has a justified belief in A’s conclusion\nolinebreak \mbox{}\hfill\mbox{(\Citeyear[759]{Sgaravatti:2013wu})}
    \end{itemize}
  \end{quote}
  Relative to an agent, really.

  Talk about claiming support, rather than evidence.
  And, details of why it's necessary.
  Also, contrast to some of the details.

  Well, interesting is:
  \begin{quote}
    For my present purposes it will suffice to say that a good test of A’s being necessary for B (and thus of B’s being sufficient for A) is the satisfaction of two subjunctive conditionals. First, if A did not hold, B would not hold; secondly, if B were to hold, A would hold.\nolinebreak
    \mbox{}\hfill\mbox{(\Citeyear[761]{Sgaravatti:2013wu})}
  \end{quote}
  This is very similar to \ref{nI:inclusion}.
  A is \(\psi\) and B is \(\phi\).\nolinebreak
  \footnote{
    \ref{nI:inclusion} was developed independently, though this is probably no surprise given how clumsy~\ref{nI:inclusion} is.
  }
  However, need \(A \leadsto B \vdash A \rightarrow B\).

  Still, relative to a certain way of claiming support.
  It's not the case that this idea holds in general.

  The difference is that we're not dealing with circularity nor justified beliefs.
  The issue here is not that the agent needs a justified belief that \(\psi\) is the case.
  Rather, it's the claimed support is such that expectation is a problem.
  Intuitively, the task of the agent is to avoid expectation.
  But doing so does not amount to forming a justified belief.
  Instead, it not mattering whether \(\psi\) has value \(v'\).

  Applied to illustrations.
  Don't need justified belief against each counterexample for theory.

  Still, similar with respect to strength of connexion.
\end{note}

\begin{note}[Different example.]
  \color{red}
  Recall \autoref{illu:CS:tfc} --- truth functional completeness.
  Instead, don't go by value.
  Rather, go by interdefinability.
  Indeed, interdefinability plays an important role in getting this condition.
\end{note}

\paragraph{Value}

\begin{note}
  `\RBV{-}'

  More or less factive reasoning.
\end{note}

\subsubsection{Linking}
\label{sec:nI:arg:linking}

\paragraph{Laying out the clauses}

\subparagraph{\ref{nI:claimed-support}}

\begin{note}
\ref{nI:claimed-support} and \ref{nI:psi-is-new} are simple background conditions.

First, \ref{nI:claimed-support} ensures the agent has claimed support for \(\phi\) having value \(v\).
Hence, agent may appeal to \(\phi\) having value \(v\) by the claimed support for \(\phi\) having value \(v\) in further reasoning, and in particular reasoning which culminates in claiming support for some other proposition.

Second, \ref{nI:psi-is-new} ensures that the agent has not considered whether \(\psi\) has value \(v'\).

As a result, it is possible that \(\psi\) not having value \(v'\) is an unrecognised defeater, and hence it is possible that \(\phi\) having value \(v'\) will come to be an expectation of the agent's claimed support for \(\phi\) having value \(v\).

Note, though, that as the agent has not considered \(\psi\), these observations about \(\psi\) were not made when claimed support for \(\phi\) having value \(v\).

Little else of interest follows from \ref{nI:claimed-support} and \ref{nI:psi-is-new} alone.
For example, the agent may never come to consider \(\psi\).
Or, \(\psi\) may be such that it wouldn't show \mom{}.

I hadn't considered that that author has a habit of misquoting articles, but I am quoting directly from the original article\dots

Rather, possibility for \(\psi\) having value \(v'\) to become an expectation of the claimed support for \(\phi\) having value \(v\).
\end{note}

\paragraph{\ref{nI:inclusion}}

\begin{note}
  Important here is that of possible defeater.
  Have that agent didn't consider \(\psi\), but this doesn't establish that possible defeater.
\end{note}

\begin{note}[Core idea]
  The role of~\ref{nI:inclusion} is to capture a relation between an agent's claimed support for \(\phi\) having value \(v\) and claiming support for \(\psi\) having value \(v'\) such that the agent considers their claimed support for \(\phi\) having value \(v\) to depend on the possibility of claiming support for \(\psi\) having value \(v'\).

  Loosely phrased, the agent thinks that the support they've claimed for \(\phi\) isn't really worth much if it not also possible for them to claim support for \(\psi\).
\end{note}



\subparagraph{~\ref{nI:going-by-value}}

\begin{note}
  \ref{nI:going-by-value} serves two purposes.

  First, claiming support for \(\psi\) having value \(v'\).
  Second, claimed support for \(\phi\) having value \(v\) as part of such reasoning, from claimed support.
  And, that this is something that it is not possible for the agent to avoid.
  {
    \color{red} Because here it's not the case that \(\phi\) is an non-important part.
  }

  Some interest.
  Saw about with conditionals and contraposition.
  However, also saw with conditionals without contraposition.

  So, it is possible that the details of the reasoning introduce \(\psi\) having value \(v\) as an expectation.
  Still, this is not necessarily the case.

  Hence, that agent is claiming support and claimed support for \(\phi\) having value \(v\) is involved so not provide anything immediate.

  Hence, the task of~\ref{nI:inclusion} is to point to conflict with assumptions \emph{given} the details of this reasoning.
\end{note}

\paragraph{The argument}

\begin{note}
  \color{red}
  Maybe restate sketch.
\end{note}

\begin{note}
  Before starting the argument, recall:
  \begin{quote}
    \vspace{-\baselineskip}
    \propRecogniseDefeaters*
  \end{quote}
  \nI{} is a variation of \autoref{prop:CS-only-if-reason-recognised-defeaters}.

  Three things:
  \begin{itemize}
  \item Less general, requires various clauses.
  \item Stronger, as does not require assumption that agent does not reason.
  \item Does not strictly follow from \requ{}.
  \end{itemize}

  However, same motivating intuition.
  With \autoref{prop:CS-only-if-reason-recognised-defeaters}, no reasoning about \requ{} at time of reasoning, then no reasoning regarding possible defeater.
  Hence, conflict with \autoref{idea:eiS}.

  With \nI{}.
  Claiming support for \(\psi\) having value \(v'\) is \requ{} of claimed support for \(\phi\) having value \(v\).
  And, as a consequence of claimed support for \(\psi\) having value \(v'\) being a \requ{} it is not possible for the agent to entertain the possibility that \(\psi\) does not have value \(v'\) while appealing to \(\phi\) having value \(v\).
  Yet, assumption that only get to \(\psi\) having value \(v'\) from \(\phi\) having value \(v\).
  Therefore, conflict with \autoref{idea:eiS} --- Agent needs to have indication of why even if \mom{}.
  The problem is that not possible to conclude that \(\psi\) has value \(v'\) from a line of reasoning that entertains the possibility that \(\psi\) does not have value \(v'\).
  \requ{} does not work, as this is expectation.
\end{note}

\begin{note}
  First, \(\psi\) having value \(v'\) is \requ{} of moving from claimed support for \(\phi\) having value \(v\) to \(\phi\) having value \(v\).

  Second, consequence of claimed support for \(\psi\) having value \(v'\) being a \requ{} it is not possible for the agent to entertain the possibility that \(\psi\) does not have value \(v'\) while appealing to \(\phi\) having value \(v\).

  Third, why this consequence prevents claiming support.
\end{note}

\paragraph{First, \requ{}}

\begin{note}[Key proposition]
  First, showing that claimed support for \(\psi\) having value \(v'\) is a \requ{} of claimed support for \(\phi\) having value \(v\).

  \begin{note}
    Definition of interest.
    \begin{quote}
      \vspace{-\baselineskip}
      \defRequisite*
    \end{quote}
  \end{note}
\end{note}

\begin{note}
  So, relevant instance of~\autoref{assu:supp:independence}:
  \begin{enumerate}
  \item (\(\text{CS}\phi \vdash \phi\)) \(\rightarrow \psi\)
  \item \(\lnot\psi \leadsto\) \(\lnot\)(\(\text{CS}\phi \vdash \phi\))
  \end{enumerate}

  Point here is that we do get \(\psi\), and it is also the case that \(\psi\) is not a `mere contingency'.
\end{note}

\begin{note}
  Two immediate consequences from~\ref{nI:claimed-support}.
  \begin{enumerate}
  \item \(S\) is (given present context) in position to claim support that \(\psi\) has value \(v'\) (without appeal to \(\phi\) having value \(v\))
  \item \(S\) would be \nmom{} were they to claim support.
  \end{enumerate}
\end{note}

\paragraph{Second, move to \(\phi\), need \(\psi\)}

\begin{note}
  Basic idea:
  \(\psi\) having value \(v'\) is inherited from claimed support for \(\psi\) having value \(v'\) as a \requ{} of claimed support for \(\phi\) having value \(v\) \emph{when} claimed support for \(\phi\) having value \(v\) is appealed to in order to reason from \(\phi\) having value \(v\).

  In this sense, \(\psi\) having value \(v'\) is an `indirect' \requ{} given background.
  For, \(\phi\) having value \(v\) gets \(\psi\) having value \(v'\).
  And, failure of reasoning if \(\psi\) does not have value \(v'\).
\end{note}

\begin{note}
  Suppose agent has moved to \(\phi\) having value \(v\) from claimed support for \(\phi\) having value \(v\).

  What we get here is that claimed support for \(\phi\) is not \misled{}.
  Also, would not be \mistaken{}.
  For, claimed support is doing the work for \(\phi\) having value \(v\).


  Hence, if \(S\) would not be \mistaken{}, then it must (also) be the case that \(\phi\).
  Note here that we only rely on not being \misled{}.

  State this simply:
  Given that \(S\) has taken claimed support to be enough to get to \(\phi\), then because of relation to claiming support for \(\psi\), also \(\psi\).

  What is important to keep in mind is that it is the move to \(\phi\) that requires perspective on claimed support for \(\phi\), in turn perspective on claimed support for \(\psi\), and hence same move.

  So, moving to \(\phi\) is sufficient to fix value for \(\psi\).
\end{note}

\begin{note}
  Observation:

  \(\psi\) is not a `mere contingency'.

  This move is such that it requires ruling out possibility of \(\psi\) not having value \(v'\), roughly.
\end{note}


\paragraph{Entertaining \(\lnot\psi\)}

\begin{note}
  Not possible to hold that \(\phi\) has value \(v\).
  For, seen above.
\end{note}

\begin{note}
  True, claimed support for \(\phi\) having value \(v\).
  However, this doesn't also work for \(\psi\) having value \(v'\) as this claimed support for \(\psi\) having value \(v'\) is \expec{}.

  Right, so if giving up \(\phi\) having value \(v\), then not appealing to claimed support for \(\phi\) having value \(v\).
  Yet, if this is the case then no move to claimed support for \(\psi\) having value \(v'\).

  So, not only is claimed support for \(\psi\) having value \(v'\) a \requ{} and an \expec{}, it is also the case that there's no way to appeal to this to secure that \(\psi\) has value \(v'\) regardless of how things actually are.
\end{note}


\paragraph{Summary}

\begin{note}[To summarise]
  Look, the problem is that it is impossible for the agent to hold that \(\phi\) has value \(v\) without also holding that \(\psi\) has value \(v'\).

  \emph{Therefore}, when the agent entertains the possibility that \(\psi\) does not have value \(v'\) (as they must do in order to claim support), this includes \(\phi\) not having value \(v\).

  Yet, by assumption, the only way for the agent to conclude some line of reasoning with \(\psi\) having value \(v'\) is by appeal to \(\phi\) having value \(v\).

  Therefore, when entertaining this possibility, the agent \emph{must} fail to reason to \(\psi\) having value \(v'\).
  (Which is not to say that the agent ends up reasoning to some contrary evaluation --- only that \(\psi\) having value \(v'\) is not a possible conclusion.)

  The reason that we're interested in the relationship between the ways of claiming support is because it leads to this consequence.
  It is true that in the way of reasoning outlined claiming support for \(\psi\) having value \(v'\) ends up being both an expectation and a \requ{}.
  However, this is not the main focus.
  Rather, it is what follows from this.
  I.e.\ that \(\psi\) having value \(v'\) is a \requ{}.
\end{note}

\newpage





\paragraph*{A handful of observations}

\begin{note}
  Argument is complete.
  A handful of observations.
\end{note}

\begin{note}
  Perspective on claimed support is key.
  \nfcs{} and \eiS{}.

  May challenge these.
  However, \nI{} is going to apply when both these things hold.
\end{note}

\begin{note}[Unused component of argument]
  There is an additional component to the clauses of \nI{} that may be added to strengthen the argument.
  Not only is \(\psi\) not having value \(v'\) a possible defeater, but given \ref{nI:inclusion} it is a possible defeater that the agent is confident that they can claim support about.

  ``One perspective on \ref{nI:inclusion} is that it ensures that claiming support for \(\psi\) having value \(v'\) is a possible and `pressing' defeater for the claimed support that \(\phi\) has value \(v\).''

  {
    \color{red}
     In turn, links to \autoref{assu:supp:independence} to motivate some account of reasoning against recognised defeater.

  I think that there is a successful argument that follows this pattern.
  However, this is not the argument we present.
  }

  Without saying more about reasoning regarding recognised possible defeaters, it is hard to say how important this is.
  Add in assumption that if possible to claim support then no dismissing by `weaker' reasoning.
  Not necessary, as clause of \nI{} is that \(\psi\) only from \(\phi\).
  Instead, suggestions along these lines would suggest that \nI{} stronger constraint would allow final clause to be weaker.
  Upshot for application of \nI{} is that need to identify this link between \(\phi\) and \(\psi\).
\end{note}


\begin{note}
  It is not a consequence of this argument that in order to claim support for \(\psi\) having value \(v'\) from \(\phi\) having value \(v\) from claimed support that \(\phi\) has value \(v\) that the agent must have an account of why \(\psi\) has value \(v'\) granting that the claimed support for \(\phi\) having value \(v\) is \mom{}.
  It may be the case that the claimed support for \(v\) requires that the claimed support for \(\phi\) is \nmom{}.
  The problem is solely that given the clauses, premises of claimed support for \(\psi\) having value \(v'\) `require' that \(\psi\) has value \(v'\).
  It may be case that \(\psi\) only if CS\(\phi\) is \nmom{} (as relying on \(\phi\)).
  However, CS\(\phi\) holds up against being \mom{}, and so even if requiring \(\phi\) from CS, still have a plausible account of \(\psi\).
  So, CS\(\psi\) needs something stronger than CS\(\phi\) grants, but the agent is still fine to hold on to what comes from CS\(\phi\).
  Though, on a related point, it does seem the considerations presented here do rule out the agent from strengthening claimed support.
  (I.e.\ given CS\(\psi\) I now have an even stronger case for \(\phi\).)
\end{note}

\begin{note}
  Important limitation is that going by \(\phi\).
  This means that \nI{} does not rule out not going by \(\phi\).
  Still, expectation which is a worry.
  And, in any case, if there is a possibility, we don't need to worry about it as we're interested in applying \nI{} to cases where \(\phi\) is made use of.
\end{note}

\begin{note}[Generalising \nI{}]
  Core question about whether there's a generalisation of \nI{}.

  In particular, one might think that there's a requirement for the agent to witness the relevant reasoning in certain cases.
  I mean, that's the core of \nI{}.
  In some cases, the agent doesn't have the option of skipping this by appealing to claimed support for something.

  However, the difficulty is in finding an expansion which doesn't also prevent the agent from claiming support when they do witness.
  In all cases, it's clear that one may get things wrong.

  The way that \adB{} avoids this is by avoiding strong claims to the specific ability.
  Indeed, principle is the same as witnessing.
  So, there's no plausible way to expand \nI{} to cover the proposals without also denying the relevant instance of witnessing.

  Rather, objections here comes from supporting \ESU{}.
  That this isn't a way to claim support.
\end{note}

\begin{note}[Summary, and testimony]
  Final case to summarise:
  Knowledge via testimony.
  This condition doesn't necessarily apply, as agent may not be in position to claim support for what follows from knowledge claim.

  Two reasons for this.
  First, agent may not be in a position to check.
  E.g.\ missing premise, or layperson, e.g.\ missing steps of reasoning.

  Second, agent may not need to \RBV{-}.
  For, if you've testified, then it follows from your statement.
  I don't need to appeal to me having heard from you.
  Instead, given the additional information that I have, you've already made the claim.
  Even if \(\phi\) doesn't have value, this is still an okay reinterpretation of the testimony you have provided.
  Here, to get the intuition, it's really not clear that I need to endorse that I do have the option to check.
  {
    \color{red}
    This point only really makes sense after the argument has been given.
  }
\end{note}

\subsubsection{Illustrations of \nI{}}
\label{sec:illustrations-ni}

\begin{note}[Abstract, so examples]
  Turn to illustrations, and then to how \nI{} applies to \gsi{}.
\end{note}

\begin{note}
  Here, only interest is in support.
  Hence, recognised by the agent that they may be mistaken or misled.
  From this perspective, the issue is not ruling out potential defeaters.
  Similar to knowledge, etc.\ but no requirement that there are no defeaters.
\end{note}

\begin{note}
  May think that this restricts any application of \RBV{} to claimed support for \(\phi\) without value independent.
  This isn't quite right.
  \eiS{} keeps focus on \(\psi\).
  Only committed to \(\psi\) being a problem.
  Potential issue is no worse than any other instance of claim to support --- possibility of being mistaken or misled.
  If \(\psi\) ends up being used, then there's going to be a gap, where agent isn't in position to claim support by value, but unless eventual consequence is in turn used for \(\psi\), no clear problem --- at least not without stronger assumption.
\end{note}

\begin{note}
  \ESU{} is going to require the agent to reason from premises and steps `included' in claimed support for \(\phi\) in order to claim support for \(\psi\).
\end{note}

\begin{note}[Examples]
  Examples are somewhat difficult, due to complexities of state.
\end{note}

\begin{note}
  First, seeing exactly why the theory examples fail.
\end{note}

\begin{note}[Serial number]
  \begin{illustration}
    \label{illu:number-check}
    Genuine, only if serial number \dots (think credit cards).
  \end{illustration}
  No need to reinspect.

  Key idea here is that really relying on the product being genuine.
  Did not check for number when claiming support.
  So, without move to genuine, this breaks down.
\end{note}


\begin{note}[Logician]
  Here, novice logician, so limited to claiming support for sure.
  In principle, proof is stronger.
  However, possibility of \mistaken{}, and as a result \misled{}.
  \begin{illustration}\label{illu:CS:tfc}
    Novice logician.
    \begin{enumerate}
    \item Claimed support that \(\{\land,\lnot\}\) are truth functionally complete.
    \item If \(\{\land,\lnot\}\) are truth functionally complete then \(\{\lor,\lnot\}\) are truth functionally complete.
    \item So, \(\{\land,\lnot\}\) are truth functionally complete.
    \item Hence, \(\{\lor,\lnot\}\) are truth functionally complete.
    \end{enumerate}
  \end{illustration}

  First, reasoning for \(\{\land,\lnot\}\) did not depend on \(\{\lor,\lnot\}\).
  But this is quite complex.
  Not explicit assumption, but perhaps implicit.
  Still, going back through reasoning, it seems this is fair.

  Second, interdefinability.
  Hence, good account of why deals with possibility, as in general what holds for \(\{\land,\lnot\}\) will hold form \(\{\lor,\lnot\}\).

  Indeed, this suggests an alternative way of getting to the conclusion.
\end{note}

\begin{note}[Programming]
  \begin{illustration}
    \label{illu:programming}
    Writing a program to automate some reasoning/processing of data.
  \end{illustration}
  Various test cases.
  In these, possible to do the reasoning oneself.
  Therefore, no appeal to program for these simple cases, at least.
  This is quite similar to the logic illustration in this sense.

  However, interest here as interdependence breaks down in interesting ways.
  For, may break down due to resource constraints.
  E.g.\ available time or complexity of inputs.

  And, after enough time with the program, failure to obtain the same result is not clearly going to indicate a problem with the program.
  Rather, one's reasoning.
  Though, in turn, this may be reversed after enough checking of the reasoning.
\end{note}

\paragraph{Variations on earlier examples}

\begin{note}
  Seen \nI{}, and in how builds on ideas which motivate \autoref{prop:CS-nai}.

  To round of the illustrations, consider variations on the illustrations of \ref{sec:abil-access-supp} which related to \autoref{prop:CS-nai}.
\end{note}

\begin{note}[Spot the difference]
  Back to \autoref{illu:CS:spot-the-diff}

  Spot the difference, think all.
  Okay, so found seven.
  Well \dots

  This turns out to be a very natural extension.
  And, strengthens the initial by avoiding the need to go to finding all of the differences.
  The reasoning alone doesn't do enough.
\end{note}

\begin{note}[Wally]
  Recall \autoref{illu:CS:wheres-wally}.

  Here, seems to apply.
  For, need it to be the case that Wally.
  However, somewhat less interesting.
  For, at the moment of completing.
  \nI{} will not find fault if, for example, in variation where book was returned to the library.
  Else, turns to be a variation on \autoref{illu:number-check}
\end{note}

\begin{note}[A trip to the zoo]
  Here, there seems no plausible variation.
  For, the interdependency fails given that there's no way to tell if it's a cleverly disguised mule by sight.

  Even in case where appeal to zebra, it need not be the case that there is interdependence.
  And, even if position to claim support by some other method (e.g.\ talking to a zoo keeper), it does not seem that sight does anything to ensure this.
\end{note}

\paragraph{Illustrations where \nI{} does not apply}

\begin{note}[Treasure --- failure of interdependence]
  \begin{illustration}
    Claimed treasure only if learnt secret.
  \end{illustration}
  A little more interesting, as here, agent is going to have done something to learn secret when claiming support for treasure, but may not recognise that they've learnt the information.

  Of course, may be wrong treasure.

  However, there's too little information here to establish interdependence.
  That's the key point.

  Useful, as earlier examples may seem to rely on easy checks, but putting pieces together to reveal secret may be quite difficult.
\end{note}

\begin{note}
  The novice instance of the theory.

  And, testimony in general.
  Problem is that interdependence breaks down in these cases.
\end{note}


\newpage

\subsubsection{Other notes}
\label{sec:other-notes}

% \begin{note}[Other ways to fail, intuitively]
%   \nI{} captures an instance of trouble, but is only a sufficient condition --- there are others.

%   \begin{illustration}
%     Suppose \nagent{8} has received information from a younger and an older sibling.
%     The older sibling has told \nagent{8} that the dog has escaped.
%     And, the younger sibling has told \nagent{8} that if even the dog has escaped, the younger sibling has no idea whether the dog has actually escaped or not.
%   \end{illustration}

%   Here, \(\phi\) is that the dog has escaped, \(\psi\) is that the younger sibling has no idea whether the dog has actually escaped or not, and both \(v\) and \(v'\) are `true'.

%   It seems that \nagent{8} will have a hard time combing the support claimed for \(\phi\) and \(\psi\) to reason to \(\psi\) being true.

%   For, \nagent{8} claims support for \(\phi\) from the younger siblings testimony, but if \(\psi\) is the case then the younger siblings testimony is unreliable (with respect to whether the dog has run away, at least).
%   Hence, rather than concluding that \(\psi\) is the case, \nagent{8} is tasked with resolving the tension between the support claimed for \(\phi\) by way of the younger sibling, and the support claimed for if \(\phi\) then \(\psi\) from the older sibling.

%   The illustration merely highlights that it's not always straightforward to piece together and antecedent and a conditional.
%   This observation is hardly novel, and similar illustrations may be found in ~\citeauthor{Harman:1986ux}'s~\citetitle{Harman:1986ux}, among others.

%   Still, \nI{} is a sufficient, rather than necessary, condition for an agent failing to claim support (and failing in a certain way) and so that there may be other ways in which instances of \(\phi\) and \(\psi\) lead to trouble is not of particular interest.

%   When discussing \nI{} we will assume that there is no tension between the claimed support for \(\phi\) and the claimed support for if \(\phi\) then \(\psi\).
%   Instead, the difficult for the agent will follow (primarily) from \ref{nI:inclusion} --- a certain kind of relation between claimed support for \(\phi\) and claiming support for \(\psi\).
% \end{note}

\begin{note}[Implication wouldn't necessarily raise the problem]
  {
    \color{red}
    Point here is that couldn't \ref{nI:inclusion} be simplified to a conditional?
    Well, it could, as we've seen.
    The point is that \nI{} is something of a generalisation of this.
  }
  Now, the agent may observe that if both~\ref{nI:claimed-support} and~\(\phi \rightarrow \psi\) hold, then if \(\psi\) does \emph{not} have value \(v'\) then the support claimed in either condition is either \mom{}.

  First, if it is the case that \(\psi\) has value \(v'\) when \(\phi\) has value \(v\) and \(\psi\) does not have value \(v'\), then \(\phi\) does not have value \(v\).
  Hence, the agent's claimed support for \(\phi\) having value \(v\) must be misled.

  Second, if the agent's claimed support \(\phi\) having value \(v\) is not misled then \(\phi\) has value \(v\), but then if \(\psi\) does not have value \(v'\) it is not the case that \(\psi\) has value \(v'\) when \(\phi\) has value \(v\), and hence the agent's claimed support for the relation is misled.

  However, this observation alone is not particularly interesting.
  Take any case in which an agent claims support for receiving testimony regarding some matter the agent has no information regarding.
  And, observe that while the agent is mistaken is claiming support for testimony if what the interlocutor has said is not true, this does not prevent the agent for claiming support for the matter by appeal to the (apparent) testimony of their interlocutor.

  For example, if a logic instructor (unintentionally) misstates a theorem, a student may still claim support for the truth of the theorem by appeal to the instruction they received --- and even if the student reflects that they would be mistaken if the theorem is misstated.

  As stated in \eiS{}, claimed support may be misled or mistaken.
  At most, the observation under discussion in general only requires the agent to expect that \(\psi\) has value \(v'\).

%   Stated in terms of value because this also holds for desires, and for probabilistic statements.
%   Desires, means end is easiest to demonstrate with.
%   Probability, think in terms of conditionalization, or in terms of entailment.
%   Truth of \(\phi\) then probability of \(\psi\) is \(40\%\).
%   If the probability of \(\phi\) is \(70\%\) then the probability of \(\psi\) is \(40\%\) (though the probability of \(\phi \land \psi\) may be \(28\%\)).
\end{note}


\subsubsection{Contrast to other conditions}
\label{sec:contr-other-cond}

\begin{note}
  Two conditions.

  First, \citeauthor{Wright:2011wn} on warrant transmission.

  Second, \citeauthor{Weisberg:2010to} on bootstrapping.
\end{note}

\begin{note}
  Use to argue that \nI{} is unique.

  Also, observe some interesting things about \nI{}.

  \citeauthor{Wright:2011wn} by difference in extension.
  \citeauthor{Weisberg:2010to} by difference in intension.

  Respective approaches are motivated by ease of demonstrating the relevant difference in extension and intension.
  \citeauthor{Wright:2011wn}'s template(s) match scenarios fairly well, and so extension.
  \citeauthor{Weisberg:2010to}'s make certain things explicit with work for difference in intension.

  However, will suggest that observations made with respect to \citeauthor{Weisberg:2010to} also extend to \citeauthor{Wright:2011wn}.
\end{note}

\paragraph{Wright on warrant transmission (failure)}

\begin{note}[How transmission failure relates]
  Inclined to think these are really the same.

  Note, in particular, \citeauthor{Wright:2000tq} is interested in transmission of \emph{second-order} warrant.
  So, not about whether the agent has warrant, but whether the agent may \emph{claim} to have warrant.
  (\Citeyear[89]{Wright:2011wn})

  In parallel, \nI{} is about claiming support, and not about whether the agent has support.
\end{note}

\begin{note}
  Basic ideas go back (at least) to the Proper Execution Principle of~\textcite{Wright:1991vn}, and in particular the \widt{} of~\textcite{Wright:2000tq} and (\Citeyear{Wright:2003aa}).\nolinebreak
  \footnote{See also~\textcite{Wright:1986ug,Wright:2002uk} and \textcite{Wright:2004uo}.}

  The \widt{} is as follows:
  \phantlabel{widt}
  \begin{quote}
    A body of evidence, \emph{e}, is an information-dependent warrant for a particular proposition P if regarding \emph{e} as warranting P rationally requires certain kinds of collateral information, \emph{I}.
    Some examples of such \emph{e}, P and \emph{I} [\dots] have the feature that elements of the relevant \emph{I} are themselves entailed by P (together perhaps with other warranted premises).
    In that case, any warrant supplied by \emph{e} for P will not be transmissible to those elements of \emph{I}.\nolinebreak
    \mbox{}\hfill\mbox{(\Citeyear[143]{Wright:2000tq})}
  \end{quote}

  The ellipses skip a quick illustration given by~\citeauthor{Wright:2000tq} in favour of the following illustration.
  \begin{quote}
    \vspace{-\baselineskip}
    \begin{illustration}
      You go to the zoo, see several zebras in an enclosure, and opine that these animals are zebras.
      Well, you know what zebras look like, and these animals look just like that.
      Surely you are fully warranted in your belief.
      But if the animals are zebras, then it follows that they are not mules painstakingly and skilfully disguised as zebras.\linebreak
      \mbox{}\hfill\mbox{(\Citeyear[154]{Wright:2000tq})}
      \newline\mbox{ }
    \end{illustration}
  \end{quote}

  Here, the body of evidence, \emph{e}, is what you've seen, the proposition P is that those animals are zebras.
  At issue is whether the warrant for P transmits to the proposition that those animals are not mules <adjectives> disguised to look just like zebras.
  In other words, at issue is whether the proposition that those animals are not mules <adjectives> disguised to look just like zebras is collateral information required for what you've seen to warrant those animals being zebras.

  From a broader perspective, the relevant collateral information is, well, there need be no specific collateral information across all possible ways of filling out the remaining details of the scenario, so let's say the collateral information is that things are as they appear.
  If so, the noted proposition is certainly required.

  More specifically, you're at a zoo, so something looking like a zebra seems sufficient to claim warrant that the thing is a zebra, and hence not a disguised mule.
  As such, \citeauthor{Wright:2000tq} holds there is a problem because, generally speaking, \dots

  \begin{quote}
    \dots\space there are external preconditions for the effectiveness of your\linebreak method---casual observation---whose satisfaction you will very likely, without compromise of the warrant you acquire for those beliefs, have done nothing special to ensure.

    [\dots]

    Can the warrants you acquire licitly be transmitted to the claim that those preconditions \emph{are} met---or at least that they are not frustrated in those specific respects?
    It should seem obvious that they cannot.\linebreak
    \mbox{}\hfill\mbox{(\Citeyear[154]{Wright:2000tq})}
  \end{quote}
  We won't go further into why \citeauthor{Wright:2000tq} thinks the result should seem obvious.
  Rather, the above should give you an idea of the phenomenon \citeauthor{Wright:2000tq} is interested in.
  And, with this, we can begin a comparison with \nI{}.
\end{note}

\begin{note}
  In relation to \nI{}, similar idea of undercutting.\nolinebreak
  \footnote{
    Indeed, \citeauthor{Wright:1991vn} notes that Stephen Yablo suggested the kind of defeat in question might be called undercutting in reference to \citeauthor{Pollock:1987un} (\Citeyear[95,fn.9]{Wright:1991vn}).

    I should perhaps note here that I developed~\nI{} after struggling to apply the ideas of \citeauthor{Wright:2011wn} to the scenarios of interest involving ability.
    And, after developing an initial draft of~\nI{} I took to the literature to see if there were any developed ideas that are either equivalent or imply \nI{}.
    This, quite naturally, led me to \citeauthor{Pollock:1987un}'s distinction between overriding and undercutting defeaters, and some of the references above which use `undercutting' in a broader sense than \citeauthor{Pollock:1987un}'s original formulation.
    Hence, it seemed to me that framing \nI{} in terms of identifying something of an undercutting defeater might be a helpful guide.
    If I had found this footnote earlier, I may have had an easier time developing the initial draft of~\nI{}.
  }
  {
    \color{expand}
    Information-dependence blocks transmission of warrant, but does not suggest anything about the relevant elements of \emph{I}.
  }

  Further, parallels between the first part of this and \eiS{}.
  different, but seem to go for the same idea.
  \eiS{} motivated by fallibility, and the first part amounts to fallibility.

  \citeauthor{Wright:2000tq} views this as a requirement, I haven't made this move.
  Significant part of \citeauthor{Wright:2000tq}\nolinebreak
  \footnote{
    See Pryor.
    This is what dogmatism denies.
  }
  , but I don't think this is the thing to focus on.\nolinebreak
  \footnote{
    You might be inclined to think understanding of claimed support should be strengthened.
    I don't want to take a stance on this, and hence problematic for me to distinguish on this basis.
  }

  {% to delete?
    Still, looking at the \emph{form} of \nI{} and \citeauthor{Wright:2000tq}'s \widt{}, there is a difference.
  }

  Instead
  \nI{} is concerned with the way in which in agent uses claimed support for a pair of propositions to claim support (or warrant) for some other proposition.
  By contrast, the \widt{} is concerned with preconditions for claiming warrant (or support) for a proposition that `might' be used to claim support.

  How agent claims support for \(\psi\) given claimed support for \(\phi\) and an implication from \(\phi\) to \(\psi\).

  \(\phi\) and \(\phi \rightarrow \psi\).
  Whether evidence, or claimed support, for this pair requires some collateral information entailed by \(\psi\).

  So, \nI{} doesn't care too much about \(\phi\) and \(\phi \rightarrow \psi\) whereas the template does.

  This is the thing of interest.

  Note, not possible to apply the template in a different way.

  So, in terms of \eiS{}.
  If going by value, require \(\psi\) to be the case.

  For template, need some warranted proposition P.
  Can't be \(\psi\), as template needs warrant, which we're denying.
  So, need something between \(\phi\) and \(\phi \rightarrow \psi\) and \(\psi\).
  Seems there's no proposition here.

  At issue is whether this difference in form corresponds to a difference in substance.
\end{note}

\begin{note}[The revised template]
  The \widt{} is intuitive, but has some downsides.

  Foremost, we would like additional clarity with respect to collateral information.
  As things stand, it's a little vague as to what constitutes and external preconditions for the effectiveness of a method.
  Further exposition might resolve this problem, but \dots

  More significant the \widt{} has been surpassed.\nolinebreak
  \footnote{
    This is a somewhat subtle issue.

    The revised template is, strictly speaking, a revision of the disjunctive template.
    And, \citeauthor{Wright:2002uk} initially distinguished the two templates:

    The \widt{} was designed to identify failures of transmission following from accumulation of defeasible evidence.

    And, by contrast, the disjunctive template was designed to identify failures of transmission following from by some faculty, such as perception or memory.

    See, for example, \textcite{Wright:2002uk} in which both templates are discussed separately, and \textcite[91]{Wright:2011wn} where the difference in motivation is restated.

    Still, \citeauthor{Wright:2011wn} observes that both templates the `base' for failure of transmission is the same in both cases.
    And, in turn, that the initial formulation of the disjunctive template yields unintuitive results when applied to cases covered by the \widt{} is a significant problem.
    (\Citeyear[91]{Wright:2011wn})

    So, \wrt{} is, strictly speaking, not a revision of the \widt{}, but rather the disjunctive template.
    However, \wrt{} is also designed to apply to the cases covered by the \widt{}, given that \citeauthor{Wright:2011wn}
    holds that both the \widt{} and the disjunctive template capture the same core phenomenon.
    Therefore, we have omitted these turns from the body of the paper.
  }
  (\Citeyear[90]{Wright:2011wn})
  This doesn't prevent a comparison, as such, but it may lead one to consider the comparison disingenuous.
  I don't think this is the case, and I began with the information-dependence as it is the spirit, rather than the letter, of the template which is at issue.

  So, to contrast \citeauthor{Wright:2000tq}'s template and \nI{} in detail, let's switch to \phantlabel{wrt}\citeauthor{Wright:2011wn}'s revised template:

  \begin{quote}
    Where A entails B, a rational claim to warrant for A is not transmissible to B if there is some proposition C such that:
    \begin{enumerate}[label=\roman*., ref=(\roman*)]
    \item\label{WT:i} The process/state of accomplishing the relevant putative warrant for A is subjectively compatible with C's holding: things could be with one in all respects exactly as they subjectively are yet C be true
    \item\label{WT:ii} C is incompatible (not necessarily with A but) with some presupposition of the cognitive project of obtaining a warrant for A in the relevant fashion, and
    \item\label{WT:iii} Not-B entails C\nolinebreak
      \mbox{}\hfill\mbox{(\Citeyear[93]{Wright:2011wn})}
    \end{enumerate}
  \end{quote}
  Where
  \begin{quote}
    A presupposition of a cognitive project is any condition P such that to doubt P (in advance of executing the project) would rationally commit one to doubting the significance, or competence of the\linebreak project, irrespective of its outcome.\nolinebreak
    \mbox{}\hfill\mbox{(\Citeyear[92]{Wright:2011wn})}
  \end{quote}

  In relation to the \widt{} discussed above:
  \citeauthor{Wright:2011wn}'s account of presuppositions of cognitive projects clarifies what collateral information amounts to.
  Cases of transmission failure are going to arise when one attempts to claim warrant for a condition for which doubt toward would undercut any outcome of the project.\nolinebreak
  \footnote{
    What matters is whether the relevant cognitive project has a presupposition of this kind, not whether the agent has done `anything special to ensure such presuppositions are satisfied.
    }

  In turn,~\ref{WT:i} to~\ref{WT:iii} detail how warrant for the relevant proposition depends on such collateral information.
\end{note}

\begin{note}
  The core intuition remains the same:
  Failure of transmission from a fixed proposition to conditions that need to be met in order for the agent to claim warrant for the fixed proposition.
\end{note}

\begin{note}[Applied to a case]
  Let's apply \wrt{} to the illustration used above to check:

  The relevant instances of A and B are, respectively:
  \begin{itemize}
  \item[A.] Those animals are zebras
  \item[B.] Those animals are not mules disguised to look like zebras
  \end{itemize}
  And, we may take C to be not-A. (\Citeyear[90]{Wright:2011wn})

  Now,~\ref{WT:i} to~\ref{WT:iii} are satisfied:

  \begin{itemize}
  \item[{\hyperref[WT:i]{i:}}] The process of accomplishing putative warrant for A is that the animals appear to be zebras, and things could be exactly as they \emph{appear} to be and yet the animals are not zebras.
  \item[{\hyperref[WT:ii]{ii:}}] The animals not being zebras is incompatible with A, of course --- given that C is not-A.\nolinebreak
    \footnote{
      For more details: (\Citeyear[90--96]{Wright:2011wn})
    }
    And, more broadly, the animals not being zebras is incompatible with moving from appearance to fact.
  \item[{\hyperref[WT:iii]{iii:}}] If those animals \emph{are} mules disguised to look like zebras, then those animals are not zebras.
  \end{itemize}
  Hence, \wrt{} identifies a failure of transmission in much the same way as we saw above with respect to the \widt{}.
\end{note}

\begin{note}[Entailment]
  Now, turning to the comparison proper.

  First, we'll map A and B from \wrt{} to \(\phi\) and \(\psi\), respectively, from \nI{}.

  An immediate difference is that \wrt{} requires an entailment between the relevant A and B while \nI{} does not include such a requirement.
  % requires that the agent has claimed support that if \(\phi\) has value \(v\) then \(\psi\) has value \(v'\).
  Still, to keep things simple, we'll assume that the agent has claimed support for an entailment from \(\phi\) having value \(v\) to \(\psi\) having value \(v'\).
  \wrt{} doesn't require that the agent has warrant, or has claimed support, for the entailment between A and B, and hence will apply in the case that the agent has.
    % So, \ref{nI:received-info} may be seen as an instance of \wrt{}'s initial condition with some superfluous detail.

  Likewise, \wrt{} is concerned with whether a claim to warrant for A is transmissible to B in general, and so not assume the agent has claimed warrant for A.
  So, as~\ref{nI:claimed-support} requires that the agent has claimed support for \(\phi\), we may consider~\ref{nI:claimed-support} as superfluous from the perspective of \wrt{}.
\end{note}

\begin{note}[Gist of why these are different]
  We're interested with~\ref{nI:inclusion} and~\ref{nI:going-by-value}.
  And, it seems focus should be on~\ref{nI:inclusion}.
  For,~\ref{nI:going-by-value} is (primarily) about the way in which the agent may go about claiming support for \(\psi\) and \nI{} only limits an agent claiming support in such a way if~\ref{nI:inclusion} holds.

  Hence, it seems to me the question is whether~\ref{WT:i} -- \ref{WT:iii} from \wrt{} and~\ref{nI:inclusion} do different things.
\end{note}

\begin{note}[Two questions]
  Let's break this down into two questions.

  \begin{enumerate}
  \item Whether an instance of~\ref{nI:inclusion} obtaining means that the agent makes a presupposition of the kind identified by \wrt{}.
  \item And, conversely, whether an instance~\ref{nI:inclusion} obtains if the agent has made a presupposition of the kind identifies by \wrt{}.
  \end{enumerate}
\end{note}

\begin{note}[Second question]
  The second question is straightforward to answer in the negative.

  Testimony, sight, whatever.
  Here, doesn't need to be any other way for the agent to claim support for the relevant proposition.
  {
    \color{red}
    To illustrate, zebra.
    Not in a position to claim support by some other way that moving from appearance is bad.
  }
\end{note}

\begin{note}[First question]
  The first question is more involved, and requires some care.

  Consider again the general presupposition that things are as they appear.
  Or, even more generally, that one is not in some sceptical scenario, such as a dream or a vat (cf.~\Citeyear{Wright:2002uk}, \Citeyear[97--98]{Wright:2011wn}).
  The difficult here is that it's easy to trivialise the question if the relevant presupposition is any presupposition.
  For, it seems any cognitive project will require some presupposition.

  Instead, the question is whether~\ref{nI:inclusion} obtaining means there is some \emph{related} presupposition.

  And, it seems this need not the case.

  For,~\ref{nI:inclusion} is, intuitively, about whether an agent is confident they have some way to claim support for \(\psi\), other than appealing to \(\phi\) and an implication from \(\phi\) to \(\psi\).
  But, it doesn't seem to follow that doubt about whether the agent has some other way of claiming support for \(\psi\) prior to claiming support for \(\phi\) and the implication from \(\phi\) to \(\psi\) would undercut claiming support for \(\phi\) or the implication from \(\phi\) to \(\psi\).

  I suspect this point is best argued for by illustrations, and we will consider a handful below.
  Still, it may be helpful to first outline the target of such illustrations in some detail.

  \ref{nI:inclusion} is about whether an agent is confident they have some other way to claim support for \(\psi\), but consists of two parts.
  \ref{nI:inclusion:position} requires that the agent is confident that the support claimed for \(\phi\) and would be \mom{} if the agent is not in a position to claim support for \(\psi\) some other way.
  And,~\ref{nI:inclusion:bound} requires that the agent is confident that the claimed support for \(\phi\) is a guarantee of sorts for claiming support for \(\psi\).

  Now, we're interested in deriving a related presupposition from \ref{nI:inclusion}.
  Still, the presupposition needs to be with respect to claimed support.
  So, as~\ref{nI:inclusion:bound} is a condition which concerns (as yet) unclaimed support, the related should follow from~\ref{nI:inclusion:position} --- in particular, from the possibility of the claimed support for \(\phi\) being \mom{}.

  However, claimed support for \(\phi\) being \mom{} reduces to either the claimed support indicating \(\phi\) has some value it does not have (misled) or the claimed support relies on factors that do not indicate the value of the \(\phi\) (mistaken).

  The point here is that claimed support being \mom{} is a relatively broad phenomenon.
  For example, an instance of inductive support may indicate the value of \(\phi\), and hence not be mistaken, but misled due to constrained sampling.
  Consider, by way of quick illustration, testing a random number generator by sampling its output.
  It may take a significant sample size to identify a bias, and hence bug in the source code.

  Yet, \citeauthor{Wright:2011wn}'s notion of a presupposition requires that doubt about the presupposition is such that doubt about the presupposition, \emph{in advance of following through on the project}, would \emph{require} doubt about the significance or competence of the project --- regardless of its outcome.

  So, suppose an instance of~\ref{nI:inclusion} may obtain because an agent has claimed inductive support for \(\phi\), has claimed support that \(\phi\) entails \(\psi\), and has some independent check on whether \(\psi\) is the case.

  If such an instance of~\ref{nI:inclusion} obtaining means that the agent makes a presupposition of the kind identified by \wrt{}, then the relevant presupposition should concern the nature of the claimed inductive support.

  However, it seems fundamental to claimed inductive support that one may doubt the inductive support is not misled without a requirement that one doubts the significance, or competence, of claiming such inductive support.

  I may doubt that I have obtained a sufficiently large sample to conclude that there are no bugs in the source code of the random number generator without being required to doubt the significance, or competence, of the sample acquired.
\end{note}

\begin{note}
  \color{red}
  The big difference is doubt versus an expectation.
  Expectation is weaker, so applies more generally.
  However, as expectation is weaker it might also be easier to deal with.
  (As kind of seen in the zebra case.)
\end{note}

\begin{note}
  To summarise:
  \ref{nI:inclusion} concerns (an agent's confidence in) a particular kind of relationship holding between claimed support for \(\phi\) and claiming support for \(\psi\) from the perspective of whether the respective instances of support are (or would be) \mom{}.
  This relationship may arise from claimed inductive support for \(\phi\).
  If so, a positive answer to the first question would require a corresponding presupposition with respect to the claimed indicative support for \(\phi\).
  Yet, such a presupposition seems incompatible with the nature of inductive support.
\end{note}

\begin{note}
  Stress, briefly, that this does not indicate anything problematic about \citeauthor{Wright:2011wn}'s template.
  I'm inclined to think the template is sound.
  The issue is whether (at least some) of the instances captured by \nI{} fall outside the scope of \citeauthor{Wright:2011wn}'s template.
  Given that both \nI{} and \citeauthor{Wright:2011wn}'s template are sufficient, there's no tension between the two if it is the case.
\end{note}

\begin{note}
  Let's now return to~\autoref{ill:CE:main} from the start of this section in which we examined a researcher may claim support that there are no counterexamples to a theory they have developed.

  Given that we have already seen how \nI{} applies to both illustrations, and outlined the theoretical difference between \nI{} and \wrt{}, we will focus only on why \wrt{} does not seem to apply to the illustration.
  In particular, why it seems there is no plausible candidate for the required `C proposition' of \wrt{}.
\end{note}

\begin{note}[\autoref{ill:CE:main}]
  \autoref{ill:CE:main} considered a researcher who has claimed inductive support for some theory.
  The instance of reasoning we took interest with was as follows:

  \begin{itemize}
  \item I have claimed support that the theory is adequate.
  \item So, given the claimed support, theory is adequate.
  \item Therefore, as the theory is adequate, given the claimed support, it follows that there are no counterexamples.
  \item Hence, I claim support that there are no counterexamples to the theory.
  \end{itemize}

  As we assumed an entailment from an adequate theory to an absence of counterexamples to the theory, we have the following two instances of A and B with respect to \wrt{}:

  \begin{enumerate}[label=\Alph*., ref=(\Alph*)]
  \item\label{wrt:difference:theory:A} The theory is adequate
  \item\label{wrt:difference:theory:B} There are no counterexamples to the theory.\nolinebreak
    \footnote{
      With respect to~\autoref{ill:CE:colleague}, we would have:
      \begin{enumerate}[label=\Alph*., ref=(\Alph*)]
      \item The colleague has failed to identify a counterexample to the theory.
      \end{enumerate}
    }
  \end{enumerate}

  If we are to identify failure of warrant transmission from~\ref{wrt:difference:theory:A} to~\ref{wrt:difference:theory:B} via \wrt{}, then there must be some proposition C such that (paraphrased):

  \begin{enumerate}[label=\roman*., ref=(\roman*)]
  \item\label{wrt:CE:maini} The process of claiming warrant for the theory being adequate is subjectively compatible with C holding.
  \item\label{wrt:CE:mainii} C is incompatible with either the adequacy, or some presupposition of the cognitive project of claiming warrant for the adequacy, of the theory.
  \item\label{wrt:CE:mainiii} The existence of a counterexample to the theory entails C.
  \end{enumerate}

  Well,~\ref{wrt:CE:maini} seems okay.
  Interested in claimed inductive warrant/support.
  And, claiming inductive support seems subjectively compatible with an entailment from some counterexample holding.
  It seems possible that things could be exactly as they subjectively are, yet the theory is inadequate because there is an unobserved instance of the phenomenon which constitutes a counterexample to the theory.

  So,~\ref{wrt:CE:mainii} and~\ref{wrt:CE:mainiii}.

  Working backwards.

  From~\ref{wrt:CE:mainiii}:
  C needs to be entailed by the existence of a counterexample.

  Paired with~\ref{wrt:CE:mainii}, the existence of a counterexample needs to entail something that is incompatible with either the adequacy, or some presupposition of the cognitive project of claiming warrant for the adequacy, of the theory.

  The problem:
  Claiming inductive warrant.
  Seems compatible with some counterexample holding.
  Applied to various instances of the phenomenon, and the theory holds up.
  Possible that it doesn't hold up under some instance of the phenomenon.

  So,
  Suppose the existence of a counterexample entails something that is incompatible with either the adequacy, or some presupposition of the cognitive project of claiming warrant for the adequacy, of the theory.
  Then, it seems the theory denies the possibility of such a counterexample.

  The difficulty is that we're talking generally about some theory for which the researcher has claimed inductive warrant for.
  I see no reason to think that any theory which fits this broad description will deny the possibility of certain counterexamples.

  There, may be that there are assumptions.
  Theories are built on other theories.
  However, interest is in a counterexample to the theory --- not a counterexample to theoretical foundations.

  Claiming warrant that there are no counterexamples in general seems to be the issue, rather than the specific kind a counterexample that would be required for \wrt{} to apply.

  Indeed, we can revise the relevant B instance given \citeauthor{Wright:2011wn}'s notion of a presupposition:
  \begin{enumerate}[label=\Alph*\('\)., ref=(\Alph*\('\))]
    \setcounter{enumi}{1}
  \item No counterexample consistent with the presuppositions.
  \end{enumerate}

  Evaluation of the reasoning seems the same.
  Indeed, natural assumption that there are no such presuppositions, so the concerns raised in~\autoref{ill:CE:main} remain.
\end{note}

\begin{note}[Looking ahead]
  \color{later}
  Difference is one thing, but also difference with respect to cases of interest.
  So, looking ahead, ability.

  Simple variation on second example.
  Ability to demonstrate that instance of phenomenon is covered by theory/not a counterexample.

  Follows from understanding of the theory.
  Seems just as bad.
  And, \citeauthor{Wright:2011wn} doesn't apply to either.
  Just need a little more work.
  Claiming support for general ability, so we add claimed (inductive) support for theory together with understanding of theory.
  Follows to specific ability as instance of general.

  So, while not focusing on cases involving ability from perspective of motivating \nI{}, differences here still relevant.
\end{note}

\paragraph{Weisberg}

\begin{note}
  \color{red}
  Difference to \wnf{} is that there's no clear account of why \(\psi\) is needed, the problem, instead, is that it is not possible for the agent to get rid of \(\psi\).
\end{note}

\begin{note}[Intro to \wnf{}]
  Case.
  Condition.
  Contrast.

  Applies to inductive reasoning.
  \nI{} isn't strictly concerned with inductive reasoning.
  However, application is focused on this, and we have appealed to inductive reasoning extensively when contrasting \nI{} to \citeauthor{Wright:2011wn}'s templates.
\end{note}

\begin{note}[Bootstrapping]
  To illustrate \wnf{}, let's consider a case of bootstrapping introduced by~\textcite{Vogel:2000tl}'s --- here following \citeauthor{Weisberg:2010to}'s presentation:
  \begin{quote}
    \begin{illustration}\label{ill:gas-gauge}
      \emph{The Gas Gauge}. The gas gauge in \nagent{9}'s car is reliable, though she has no evidence about its reliability.
      On one occasion the gauge reads F, leading her to believe that the tank is full, which it is.
      She notes that on this occasion the tank reads F and is full.
      She then repeats this procedure many times on other occasions, eventually coming to believe that the gauge reliably indicates when the tank is full.\nolinebreak
      \mbox{}\hfill\mbox{(\Citeyear[526--527]{Weisberg:2010to})}\linebreak
      \mbox{}
    \end{illustration}
  \end{quote}
  \citeauthor{Vogel:2000tl} argued that kind of reasoning present in~\autoref{ill:gas-gauge} is a problem for reliabilist theories of knowledge, and others have argued the problem may be extended further (see \textcite[\S1]{Weisberg:2010to} for more details).

  However, our interest in the reasoning present in~\autoref{ill:gas-gauge} and \wnf{} is merely that the reasoning is intuitively problematic, \wnf{} is an account of why, and \wnf{} may capture the same phenomenon as \nI{}.
\end{note}

\begin{note}[No feedback]
  \begin{quote}\phantlabel{wnf}
    \textbf{No Feedback} If
    \begin{enumerate*}[label=(\roman*)]
    \item\label{W:NF:i} \(L_{1}-L_{n}\) are inferred from \(P_{1}-P_{m}\), and
    \item\label{W:NF:ii} \(C\) is inferred from \(L_{1}-L_{n}\) (and possibly some of \(P_{1}-P_{m}\)) by an argument whose justificatory power depends on making \(C\) at least \(x\) probable,\nolinebreak
      \footnote{
        There may be some ambiguity here.
        As we will see when examining an illustration below, the arguments justificatory power should be read in terms of depending on \emph{having made} \(C\) at least x probable rather than \emph{establishing that} \(C\) at least \(x\) probable.
        (It is in this sense that \(C\) is being `amplified'.)
        By contrast, the following clause requires that \(P_{1}-P_{m}\) \emph{are making} \(C\) at least \(x\) probable without the help of \(L_{1}-L_{n}\).
      }
      and
    \item\label{W:NF:iii} \(P_{1}-P_{m}\) do not make \(C\) at least \(x\) probable without the help of \(L_{1}-L_{n}\), then the argument for \(C\) is defeated.\linebreak
      \mbox{}\hfill\mbox{(\Citeyear[533--534]{Weisberg:2010to})}
    \end{enumerate*}
  \end{quote}
  Where `\(P\)' stands for a premise(s), and `\(L\)' for a lemma(s). (Cf.~\Citeyear[533]{Weisberg:2010to})

  Again, we have a condition in which an argument would be undercut.
  \wnf{} suggests only that the argument for \(C\) would be defeated, but leaves open the status of \(C\).
\end{note}

\begin{note}[\wnf{} intuition]
  \citeauthor{Weisberg:2010to} motivates with the following intuition.
  \begin{quote}
    The idea is that the amplification of an already amplified signal distorts the original signal, resulting in feedback, and bootstrapping is just ``epistemic feedback''.
    Bootstrapping is an undesirable result of amplifying the output of ampliative inference without restriction.\linebreak
    \mbox{}\hfill\mbox{(\Citeyear[534]{Weisberg:2010to})}
  \end{quote}

  To summarise.
  {
    \color{red}
    So, the point is that there's some amplification applied to \(C\) in order to get \(L\) from \(P\).
    And, this in turn blocks an argument to \(C\).
    For, already included amplification to \(C\).
    
  }
  {
    \color{red}
    \phantlabel{wnf:expectation}
    Note, doesn't rule out \(L_{1}-L_{n}\).
    Here, similar to expecting that defeaters don't hold.
    Or, following \citeauthor{Weisberg:2010to}, drawing conclusions from evidence.
  }
\end{note}

\begin{note}
  After walking through how \wnf{} applies to~\autoref{ill:gas-gauge} we will motivate a connexion between \wnf{} and \nI{}, before arguing that the two are sufficiently distinct.
\end{note}

\begin{note}
  The overall conclusion \nagent{9} draws in~\autoref{ill:gas-gauge} is that the gauge reliably indicates when the gas tank is full.
  Still, this overall conclusion is drawn from repeated instances of reasoning on particular occasions that concludes that the gauge is reliable on that occasion.
  And, the fault identified by \wnf{} concerns the reasoning on particular occasions.
  Intuitively, if \nagent{9} fails to establish the reliability of the gauge on any particular occasion by the particular instances of reasoning, then the conclusions of those particular instances of reasoning are unavailable for \nagent{9} to draw the general conclusion.

  So, to begin let us summarise the pattern to which each particular instances of reasoning conforms:

  \begin{enumerate}
  \item\label{W:GG:i} The gauge is reliable. \hfill (Background assumption)\nolinebreak
    \footnote{
      Have as background assumption because in the original, \nagent{9} skips over this as an explicit step.
      However, following \citeauthor{Weisberg:2010to} possible to reformulate to some level of probability sufficient to go to 3, such that the overall result of argument is to raise probability. (\Citeyear[528]{Weisberg:2010to})
    }
  \item\label{W:GG:v} It is sufficiently likely that the gauge is functioning correctly on this occasion. \hfill \mbox{(From~\ref{W:GG:i}, `Amplification')}
  \item\label{W:GG:ii} The gauge reads full. \hfill (Observation)
  \item\label{W:GG:iii} So, the tank is full. \hfill (From~\ref{W:GG:v} \&~\ref{W:GG:ii})
  \item\label{W:GG:iv} Hence, the gauge is functioning correctly on this occasion. \hfill (From~\ref{W:GG:ii} \&~\ref{W:GG:iii})
  \end{enumerate}

  The `feedback' in this reasoning pattern involves establishing (an instance of) the reliability of the gauge from an assumption that the gauge is reliable.

  From the perspective of \wnf{} we have:
  \begin{itemize}
  \item[P:] The gauge reads full.
  \item[L:] The tank is full.
  \item[C:] The gauge is functioning correctly on this occasion.
  \end{itemize}

  And, each of the clauses of \wrt{} are satisfied, for:
  \begin{itemize}[labelwidth=\widthof{(iii)}]
  \item[{\hyperref[W:NF:i]{i:}}] That the tank is full is inferred from the gauge reading full (together with the background assumption applied to the particular occasion).
  \item[{\hyperref[W:NF:ii]{ii:}}] That the gauge is functioning correctly on this occasion is inferred from the tank being full (and the gauge readings full) by an argument whose justificatory power depends on it being probable the gauge functioning correctly on this occasion.
  \item[{\hyperref[W:NF:iii]{iii:}}] That the gauge reads full does not make it probable the gauge functioning correctly on this occasion without the help of it being the case that the tank is full.
  \end{itemize}

  In short, the reasoning from~\ref{W:GG:i} to~\ref{W:GG:iv} captures the (intuitive) idea that \nagent{9}'s reasoning is flawed because and agent doesn't get to use reasoning that proceeds from an assumption to infer that the assumption holds.

  {
    In terms of \citeauthor{Weisberg:2010to}'s presentation, the agent makes an ampliative inference from \(P_{1}-P_{m}\) to \(L_{1}-L_{n}\), requires certain things to be the case, and, results of amplification inference don't provide one with an argument for source of distortion.
    }

  \nagent{9} requires the gauge functioning correctly on this occasion to infer that the gas tank is full, but observing that the gauge functioning correctly follows given the assumption that the gauge functioning correctly doesn't make it any more likely that the gauge really is functioning correctly.
\end{note}

\begin{note}[Different from \citeauthor{Wright:2011wn}]
    {
    Here, very similar to \citeauthor{Wright:2011wn}'s \wrt{}.
    Difference is with respect to \ref{WT:iii}.
    Not-\(C\) does not necessarily entail something incompatible.

    For, \nagent{9} needs sufficiently reliable.
    And, it doesn't follow from the gauge is not functioning on this occasion that it is not sufficiently likely, nor that it is not possible to move from sufficiently likely to working.

    Still, \wnf{} does seem to fall within the general scope of \widt{}.
  }
\end{note}

\begin{note}[In relation to \nI{}]
  We turn now to the relationship between \citeauthor{Weisberg:2010to}'s \wnf{} and \nI{}.

  Recall, \eiS{}:
  Claimed support indicates the value of a proposition regardless of whether the claimed support is \mom{}.

  The argument for \nI{} rests on \eiS{}, and \citeauthor{Weisberg:2010to}'s \wnf{} may, likewise, be seen to rest on \eiS{}.

  For, it seems that any claimed support for the conclusion of an argument that satisfies the clauses of \wnf{} would violate \eiS{}.

  Consider \wnf{} once again.
  An argument for a relevant instance of \(C\) is defeated because \(C\) being probable to some degree is required in order to obtain additional lemmas used to construct an argument for \(C\).
  Recast, then, the agent may not construct an argument for \(C\) if the agent requires \(C\) to be probable to some degree.
  Or, equivalently, the agent may not construct an argument for \(C\) if it is a requirement for the success of the argument that the agent is not misled about degree to which \(C\) is probable.
  I.e.\ the argument would indicate the value, or probability, of \(C\) regardless of whether the claimed support is misled because the argument is only successful if \(C\) is probable to the relevant degree.

  So, is it the case that \citeauthor{Weisberg:2010to}'s \wnf{} and \nI{} are equivalent accounts of how \eiS{} constrains claiming support, if some (perhaps) superficial details about `probability' or `being in a position to claim support' are either removed or revised?
\end{note}

\begin{note}[Technicality]
  There is an initial difference with respect to scope of application.
  \wnf{} only applies to inductive reasoning (Cf.~\Citeyear[533]{Weisberg:2010to}), while \nI{} makes no such restriction.

  Still, I don't think too much should hang on this difference.
  We have motivated \nI{} primarily with respect to inductive reasoning, and reasoning with \gsi{-} is also, plausibly, an instance of inductive reasoning.
  So, even if there is room for a technicality, it doesn't matter for the cases of interest.
\end{note}

\begin{note}
  \color{red}
  Intuition is that in case of \nI{} the agent needs to make \(\psi\) probable to some degree.
  For, intuitively, needs to be that \(\lnot\psi\) is not an actual defeater.
  I.e.\ the probability of \(\lnot\psi\) needs to be low, and so the probability of \(\psi\) needs to be high.
  If this is right, then it looks as though \nI{} collapses into an instance of \wnf{}.
\end{note}

\begin{note}
  \color{red}
  The problem here is that in order to get \nI{} from \wnf{} we need to assume that any argument relies on making possible defeaters sufficiently improbable and hence that the negation of the defeater is sufficiently probable.

  Ugh.
  This is more complex.
  First, it's not clear that \wnf{} applies to the kind of cases of interest.
  For, need three parts.

  So, would need to be the case that general ability does not make \(\psi\) sufficiently probable.

  Well, it's hard to evaluate this.
  Given the information, it seems it does.
  But, suppose it does not.
  Then, would need it to be the case that general to specific requires \(\psi\).
  This goes back to assumption before.

  However, this then seems problematic.
  For, then run through the novice.
  It seems that here the novice would require that the phenomenon is not a counterexample in order to apply.
  And that seems really odd.

  I mean, there is an issue with the novice as the phenomena does come up as an expectation, my solution is to allow the agent to consider this, as it's testimony, or something, but then the parallel for \wnf{} would be to grant that theory alone is sufficient.
  However, if this is the case, then it seems general alone is going to be sufficient.

  And, \nI{} is also going to apply even when \(P\) alone makes \(C\) sufficiently probable.
  Because, it's all about whether something has been considered, not what difference it makes.
  I mean, that's kind of tough, because without considering \(C\) it's hard to know what the probability distribution is.

  \nI{} is more about the impact of unconsidered information.
\end{note}

\begin{note}[Key difference]
  Barring technicalities, still, a fundamental difference is present.

  As we have seen, \wnf{} holds when an agent makes an inductive inference from some premises to some lemmas which require the relevant conclusion of the argument to be probable to a certain degree.

  We \hyperref[wnf:expectation]{noted} that \wnf{} does not require that the agent has claimed support that the conclusion of the argument is probable to the required degree.
  However, clause~\ref{W:NF:ii} of \wnf{} explicitly states the argument of interest depends on relevant conclusion being probable to the required degree.
  So, the reasoning outlined by \wnf{} involves the agent being committed to the relevant conclusion being probable to the required degree as a part of the instance of reasoning to which \wnf{} applies.
  And, hence, tension with \eiS{}.

  \nI{} is fundamentally different.

  In the case of \nI{},~\ref{nI:inclusion} combines with~\ref{nI:going-by-value} to ensure that when an agent moves from reasoning from claimed support for \(\phi\) having value \(v\) to reasoning from \(\phi\) having value \(v\) the agent assumes that \(\psi\) has value \(v'\).
  In turn, this assumption leads to tension with \eiS{} as any reasoning that proceeds from \(\phi\) having value \(v\) depends on \(\psi\) having value \(v'\).
  Hence, even if the agent reasons from the implication that \(\psi\) has value \(v'\) when \(\phi\) has value \(v\), such reasoning is constrained to a context in which it must already be assumed that \(\psi\) has value \(v'\).
  And, so, it is not possible for the agent to claim support that \(\psi\) has value \(v'\) that indicates that \(\psi\) has value \(v'\) regardless of whether or not the claimed support is \mom{} because the reasoning from \(\phi\) having value \(v\) to \(\psi\) having value \(v'\) fails if \(\psi\) does not have value \(v'\).

  This is, admittedly, complex.
  To simplify, \wnf{} holds when an agent assumes the relevant conclusion (is probable to a certain degree) as part of the reasoning to the relevant conclusion.
  And, in contrast, \nI{} holds when an agent is required to assume that the relevant conclusion (has some value) in order to reason to the relevant conclusion.

  Simplified further, for~\wnf{} the issue is with respect to the reasoning that would be performed by the agent.
  And, by contrast, for~\nI{} the issue is with respect to what the agent must hold in order to perform the relevant reasoning.

  Admittedly this is a somewhat delicate.
  However, this distinction is important as it highlights how \nI{} is concerned with the specific details of the way in which the agent reasons, rather than any pattern of reasoning itself.

  As we saw when contrasting Illustrations~\ref{ill:CE:main},~\ref{ill:CE:colleague}, and~\ref{ill:CE:testimony}, the reasoning identified by~\ref{nI:going-by-value} allows an agent to claim support in certain cases (\ref{ill:CE:testimony}), but not in others (\ref{ill:CE:main} and~\ref{ill:CE:colleague}).

  So, there is a significant difference between the cores of \wnf{} and \nI{}.
  Issues with reasoning, and issues with assumptions prior to reasoning, respectively.
\end{note}

\begin{note}[Generalising to \widt{}]
  Noted above that \wnf{} seems to fall in the scope of \widt{}.
  If so, same kind of problem for \widt{}.

  However, relies on additional background about claiming support for presuppositions.
    And, while it is true that \citeauthor{Wright:2011wn} holds such views, it was instructive to observe that difference regardless.

    No such alternative with relation to \citeauthor{Weisberg:2010to}, as it's built into \wnf{}.
\end{note}

\begin{note}[Hm]
  \color{return}
  This is brief, but it's not clear more needs to be said.
  These are distinct phenomenon, even if they turn out to be extensionally equivalent.
  Conjecture that these wouldn't be, as similar issue with respect to difference between \citeauthor{Wright:2011wn}'s \wrt{} and \nI{} arising from \ref{nI:inclusion}.
  Working through illustrations is significant due to difference in formulation.
  There, difference in extension for difference in intension.
  Here, directly observed difference in intension.

  Broader consequences are of some interest (at least to me).
  However, aren't of interest to core line of argument.
  So, pursuing this will be left for some other time.
\end{note}

% \paragraph{Pryor}

% \begin{note}[Objection wrt.\ \citeauthor{Pryor:2000tl}]
%   I've probably already mentioned this somewhere else, but there's a way of reading \citeauthor{Pryor:2000tl} that conflicts.
%   For, Dogmatism could be read as allowing for \RBV{} in cases where it applies.

%   However, this requires some care.

%   Initial point is that it's not clear whether there's a dogmatist position with respect to claiming support.

%   Fruther, there are two possible ways to approach the issue.
%   First, as above.

%   Second, there's some operator which is dogmatic, and consequences stay within the scope of this operator.
%   So, for example with the zebra, the agent sees that the animal is a zebra, and hence sees that it's not a cleverly disguised mule.
%   On this reading, the agent does not \RBV{}.
% \end{note}

\newpage

\subsubsection{\nI{} and ability}
\label{sec:ni-ability}

\begin{note}
  \color{red}
  What I end up using here is:

  \begin{quote}
    \vspace{-\baselineskip}
    \propCSNai*
  \end{quote}
  As this is what ends up blocking any way of removing expectation of \(\psi\) from claimed support for general ability.

  At least when ability is view `as a whole'.

  So, need section on why considerations as unrecognised defeater are distinct from recognised.
\end{note}

\paragraph{\nI{} applied to \gsi{} and \adA{}}
\label{sec:ni-ability:adA}

\begin{note}
  The idea here is simple.
  \begin{itemize}
  \item Checking that the conditions for \nI{} hold.
  \item The thing with \adA{} is that the agent appeals to ability `as a whole', so to speak.
  \item This gives us the relevant \(\phi\) instance for \nI{}.
  \item And, \aben{the} is such that the agent needs to appeal to ability, rather than mere claimed support.
  \item Then, the key focus is \ref{nI:inclusion}.
  \end{itemize}

  While we focus on \aben{the}, note that it seems the problem extends to going from general to specific.

  And, while argued for \nI{} independently, also motivate why the application to ability `makes sense'.
\end{note}

\begin{note}[Applying to type of scenario]
  Our attention now turns to how \nI{} applies to the use of \aben{the} in scenarios of interest.

  The focus of our attention is whether an agent may claim support for having a specific ability given the claimed support for having a general ability, given \gsi{}.
\end{note}

\begin{note}
  Now, the basic observation is that with \adA{} one moves from general to specific, and from ability to proposition.

  Here, only really interested in \aben{the}.
  However, as we've observed, goes from either general or specific.

  I mean, the basic observation is that the agent doesn't reason about general or specific ability.
  So, reasoning follows from it being the case that agent has attribute, or that there is a witnessing event.

  Ohhhh, the point is that the agent is relying on these conditionals.
  First, to move from general to specific.
  Second, to move from ability to proposition.

  With respect to these conditionals, it's \adA{}, so there's no way to move between these things without using the value of one thing to constrain the value of the other.

  So, instance of \adA{}, generally.
  And, because of the construction of the scenarios, the case of \adA{} we're interested involves appeal to the value of the proposition.
\end{note}

\newpage

\begin{note}[Checking conditions]
  Conditions \ref{nI:claimed-support} is provided by the scenario.
  And, scenario also provide information about how the agent claims support for \(\psi\) having value \(v'\) when \(\phi\) has value \(v\).
  Key is that \aben{the} requires that the agent has the specific ability, not (merely) that the agent has claimed support that they have the specific ability.
  Condition~\ref{nI:inclusion} is obtained by reflection on ability.
  So, then, condition~\ref{nI:going-by-value} rules out a way of claiming support for specific ability.
\end{note}

\begin{note}
  If argument is successful, then agent will not be in a position to claim support for specific ability.
  This is the antecedent of the relevant use of \aben{the}.
  Pair \nI{} with following supplement.

  \begin{restatable}[]{assumption}{assuDetachToClaim}\label{assu:detach-to-claim}
    An agent must have claimed support for the antecedent of an entailment in order to claim support for the consequent of the entailment via the entailment.\nolinebreak
    \footnote{To clarify, entailment is only about value.
      Think of conditional.

      So, does not follow that there being an entailment is a required part of agent's reasoning.
      \nIm{} is talking about when the agent appeals to an entailment, rather than any understanding of entailment beyond it being the case.
    }
  \end{restatable}
  \nIm{} seems indisputable,\nolinebreak
  \footnote{
    An agent may have some other way of claiming support for the consequent of the entailment.
    However, if the agent is not in a position to claim support for the antecedent, then the agent is not in a position to claim support because there is an entailment from the antecedent to the consequent.\nolinebreak

    For example, that the coin landed heads is entailed by Sam knowing that the coin landed heads.
    Here, entailment from \(K\phi\) to \(\phi\).

    Second, this light being on entails that the printer is out of paper.
    If agent appeals to entailment, again, need the light to be on.
    However, could look in the paper drawer, or modify the wiring so that an alarm sounds.

    However, Taylor is not in a position to claim support for the coin landed heads because Sam knows if Taylor has no idea whether Sam knows --- though Taylor may claim support by looking at the coin.
  }
  and so not in a position to claim support for result of witnessing ability via \AR{}.
\end{note}

\hozline{}

\begin{note}[Important points]
  Two important points:

  The role of~\nI{} is to highlight that the agent is not in a position to obtain support for (specific) ability in a certain way.
  That is,~\nI{} does not state that the agent may not obtain support for (specific) ability some other way.

  Second, so long as agent holds that they have general ability, then committed to truth.

  May take issue with information provided, especially if ideal.
  If informer has information, then they should say.
  In turn, not problem with~\nI{} as the agent would have support (via testimony) for specific ability.
  However, informer may only have the conditional.
\end{note}

\hozline{}

\begin{note}[Finding tension, still]
  We have outlined a type of scenario built primarily on an agent receiving information that the agent has some specific ability so long as the agent has some general ability.
  The agent has support for having the general ability, but there are two ways in which the agent's support for having the general ability may be used to establish support for {\color{red} the result of having the specific ability} --- \AR{} and \WR{}.

  The previous section argued that~\ESU{} constrains how an agent may use the received information.
  If an agent is required to traces support from premises to conclusion through reasoning, then an agent may not appeal to the support for the premises and steps of reasoning that the agent would use to witness the specific ability.

  The (initial) plausibility of~\ESU{}, then, suggests that the agent may only establish support for having the {\color{red} result of the specific ability} from the support they have for the general ability by \AR{}:
  The support the agent has for the general ability is support that it is true that the agent has the general ability.
  In turn, given the information received it is true that the agent has the specific ability, and it is only possible for the agent to have the specific ability if the result of witnessing the specific ability is true.

  The argument of this section is that the sketch of \AR{} given conflicts with a different, but equally plausible, premise.
  The premise concerns the way in which the agent obtains support for having the specific ability from the support for the general ability.
  We state conditional, the proceed to the premise.
  The initial statement of the premise is abstract and after providing a handful of clarifications we then link the premise to the type of scenario of interest.
\end{note}

\begin{note}
  \large
  \begin{itemize}
  \item So, get expectation.
  \item This much is fine.
  \item Question is whether it is possible for the agent to do anything about the expectation.
  \item Arguably no.
  \item One big idea is that the agent has claimed support for general ability on some `core' such that this provides strong indication that general ability extends to all cases.
  \item This much is fine, then problem, however, is that we've still got specific information about what is outside of the core.
  \item So, the probability of any possible defeater is super low.
  \item And, this is enough to hold onto claimed support regardless.
  \item Because, I considered the possibility of those unknown defeaters, and still gathered enough to claim support regardless.
  \item So, this seems to allow claiming support for specific from general.
  \item But at the same time this seems bad.
  \item It has the same feel as the problems with expectations.
  \item Because, all the stuff gathered was without recognition of this possibility.
  \end{itemize}

  \begin{itemize}  \item I mean, problem is before, got the probability low as unrecognised.
  \item Question is whether this remains the case now recognised.
  \item Well, nothing really follows from probability being low.
  \item In a sense, this should already be the case.
  \item The issue isn't that these possible defeaters a \emph{likely}.
  \item The issue is that the agent should think that support holds regardless of whether they hold.
  \item ``Category mistake''
  \end{itemize}

  Okay, so this kind of works against low probability.
  Hence, argument here is that there's no way to get rid of this expectation if agent only relies on claimed support for general ability.
  Of course think it's unlikely, but the worry is not that the defeater is there, rather than it's not clear how the evidence goes against the defeater.

  The redux, then, is that this idea of a `core' doesn't really rely on the probability idea.
  But then this just goes against the initial assumption.
  Of course, this is kind of what \citeauthor{Pryor:2000tl} does.
  However, this seems to conflict with the idea of claimed support.
  If we've got some kind of dogmatist position, then it doesn't seem that the possibility of being \mom{} is such an issue.
  Indeed, the problem here is how to make something like this consistent with that assumption.
\end{note}

\paragraph{\nI{} and \adB{} (excluding basic \AR{})}
\label{sec:ni-ability:adB}

\begin{note}
  \large
  \begin{itemize}
  \item Have \gsi{} information.
  \item This means that we get a sort of conditional
    \begin{itemize}
    \item so long as premises and steps are available, then witnessing event.
    \end{itemize}
  \item Now, the task is to claim support for premises and steps.
  \item Key idea here is that agent does not appeal to general ability.
  \item Instead, agent is appealing to those premises and steps in the same way they would do when witnessing \emph{and this doesn't require appeal to general ability}.
    \begin{itemize}
    \item it's not the case that I go `I can do arithmetic, so \(2 + 2 = 4\)'
    \item Rather, it's understanding \(2,+,=,4\), etc.
    \end{itemize}
  \item So, ability as a whole carries the expectation, but appeal to distinct components does not.
  \item This is really important to stress.
  \item The witnessing conditional (so to speak) comes from the information, not from the general ability.
    And therefore don't need to appeal to the general ability to get the witnessing event --- only issue is whether it can be `made actual'.
  \item I mean, this is what is kind of puzzling about \EAS{}.
  \item I haven't \emph{used} any of this stuff, but claiming support by it anyway.
  \end{itemize}

  \begin{itemize}
  \item Objection here is that there's still a question about missing steps.
  \item Well, information is that all this stuff is sufficient.
  \item The only issue is whether the thing would really amount to a proof.
  \end{itemize}

  \begin{note}
    Interesting is that the above gets to specific ability.
    It's then not clear that the agent is required to get \(\psi\) from specific by the witnessing kind of thing, but this seems natural.
  \end{note}
\end{note}

{
  \color{red}
  \begin{itemize}
  \item Key lesson learnt from \nI{} and \adA{} is that if the agent goes to having the ability, then they bring \(\phi\) with them (due to interdependence of claiming support).
    So, \adB{} needs to avoid going to ability.
  \item Key with \adB{} is that it breaks up this interdependence.
  \item Instead of using the ability as a whole, everything gets broken up into premises and steps.
  \item This is motivated by general understanding of reasoning.
  \item For, general point of reasoning is breaking things down so that the conclusion doesn't follow from any particular step or premise, but rather the combination.
  \end{itemize}

  So, the real thing I need for \adA{} is that ability gets treated as a whole.
  And, this then extends to basic \AR{}.

  Do I still need the \adB{} vs.\ \adA{} distinction?
  Probably, as it helps with motivating the key ideas.
  I mean, yes as there isn't a good distinction between \AR{} and \WR{} alone.
}

\begin{note}[Setting expectations]
  So far we have seen \ESU{} requires \adA{}, and that \nI{} rules out \adA{}.
  The final thing to check, then, is whether \adB{} is compatible with \nI{}.

  Some care to be taken here.
  \adB{} only holds that the agent claims support by appeal to ability --- \AR{} and \WR{}.
  So, it is not obvious that \adB{} alone provides us with enough information to provide a complete defence that the agent does claim support.
  Rather, our goal is to provide an `in principle' defence that \adB{} need not conflict with \nI{}.

  The upshot of this is an avenue for further research.
  If \nI{} and scenarios, then either \adB{} or basic \AR{}.
  We will say more on this in section~\ref{sec:establishing-tension}.

  For now, I hope to have appropriately set expectations with respect to following argument.
\end{note}

\begin{note}[Main idea]
  \nI{} is about interdependence of claiming support between \(\phi\) and \(\psi\) undermining a way of claiming support for \(\psi\).

  So, the task is to show that this interdependence need not hold when reasoning \adB{}.

  This may not be immediately obvious.
  Saw \nI{} applied to \adA{}.
  Going \adB{} doesn't necessarily make a difference.
  {
    \color{red}
    Problem is, comes from appealing to ability.
  }

  However, what \adB{} \dots
\end{note}

\begin{note}[Working through the details]
  \ref{nI:claimed-support} \dots



  \ref{nI:inclusion} are satisfied in the scenarios of interest.
  So, question is \ref{nI:going-by-value}.

  Letter, but more importantly the spirit.

  Spirit goes back to \eiS{}.
  Saw about that \eiS{} was central to argument for \nI{}.
  In other words, the question is whether the agent claims support which holds up `even if' \(\phi\) turns out not to be the case.

  In other words, \adB{} does not lead to similar conflict with \eiS{}.

  Hence, the `in principle' defence that \adB{} need not conflict with \nI{} rests on showing that \ref{nI:going-by-value} is not necessarily satisfied when steps of reasoning are \adB{} with respect to ability.

  Focus step is ability to \(\phi\).

  So, agent claims from specific ability.
  Idea has been noted.
  Clearest with respect to \WR{}.
  Appeals to event, in particular the premises and steps of reasoning.
  In turn, reduces to observation that claiming support in this way does not require \RBV{}.
\end{note}

\begin{note}
  So, broad idea is claiming support from same premises and steps they would if they were to witness their ability.

  \adB{}, claim support by appeal to thing, and \WR{} provides sufficient detail about what that thing is.

  Question is whether \(\phi\) needs to be the case.

  Recall, \(\phi\) because moved to it being true that agent has the ability.

  This move doesn't happen with \adB{}.


  Key point is that agent claims support for property or event.
  The agent doesn't move to value.

  So, \gsi{} it's the parts of the general ability.
  And \aben{the} it's the premises and so on.

  Key point is that given background information, these allow the agent to claim support, even if it turns out the agent is \mom{}.
  Information is that that stuff is sufficient to claim support.

  Easiest with \WR{}.
  As, this is just the same as an instance of reasoning.
  The only difference is that the agent isn't clear on what's going on.

  Everything the agent has claimed support for allows them to make this move.
  Even if turns out things aren't right, and \mom{}, the agent seems to have enough, and by \adB{} they don't require that they aren't \mom{}.

  This, to my mind, is the key idea with ability.
  It informs the agent of something they have the ability to do.
  And, that thing functions in just the same way as it would if the agent were to do the thing.

  Claim support by appeal to that reasoning.
  Only going to be truly successful if I have the ability, for sure.
  However, claim support for ability even if \mom{}.
\end{note}

\begin{note}
  Key observation is that \adB{} doesn't go by value.

  However, there is a problem.

  For, it may seems as though the agent \emph{does} go by value because they require the premises, etc.

  This is clearest with the idea that:
  \begin{itemize}
  \item If \(\phi\) isn't the case, then some premise or step isn't part of ability.
  \end{itemize}
  Question about whether this gets a violation of \eiS{}.

  But, point is that agent at present is okay with claiming support that the reference resolves.

  So, this really isn't that problematic.

  Obviously it could break down.

  The point is that the agent at present outlines claim to support even if \mom{}.
\end{note}

\paragraph{Objections}

\begin{note}
  The main objection here is something along the lines of a stronger requirement on claiming support.
  Agent gets to keep claimed support for general or specific ability.
  Possible defeater isn't enough, though information introduces an expectation.
  Well, this should prevent appealing even to the premises and steps of reasoning.
  Avoided this because these do not require expectation of \(\psi\).
\end{note}

\subparagraph{Deny claimed support for ability}

\begin{note}
  So, if witnessing, then premises and steps are good enough to go for the conclusion.

  Main problem is that possibility that the relevant witnessing event is not possible.
  However, claimed support that it is.

  Still, suggestion that given possible defeater, the agent doesn't even get to appeal to claimed support for general/specific.

  Then, wouldn't get the details of the witnessing event.

  So, this is much stronger than \autoref{assu:CS-persists}.
  Indeed, something like this would prevent event \ESU{}.
  Find this sufficiently implausible.
\end{note}

\subparagraph{Appeal to premises and steps requires appeal to ability}

\begin{note}
  It's true that the combination implies the ability, and so the combination seems to lead to the same problem.
  We get \(\psi\) as an expectation of combining all of the premises and steps.

  However, what we're relying on is appeal to the individual components.
  The thing here is that it seems fine for the agent to witness.
  This doesn't block claiming support.

  Hence, if this is the case then it can't be that the problem is simply what follows from the combination.
  Rather, it must be something about not witnessing.
  However, this returns us to \ESU{}.
  This is the very intuition that we're arguing against.
  Hence, the question is whether this really is something that is the case.
\end{note}

\subsubsection{Incompatibility of \nI{}, \gsi{}, and \adA{}}
\label{sec:ni-summary}

\begin{note}[Table]
    \begin{figure}[h]
    \centering
    \begin{tblr}{abovesep=8pt, belowsep=8pt, width=0.95\textwidth, colspec={Q[c,m]|Q[c,m]|Q[1.8,c,m]|Q[1.8,c,m]}}
      \multicolumn{2}{c}{} & \adA{} & \adB{} \\
      \hline
      \multicolumn{2}{c}{\WR{}} & Ruled out by \nI{}  &  \\
      \hline
      \multirow{2}{*}{\AR{}} & Basic  & Ruled out by \nI{}  & ---  \\
      \cline[dashed]{2-4}
      & Derived & Ruled out by \nI{}  &  \\
    \end{tblr}
    \caption{Distinction matrix}
  \end{figure}
\end{note}

\subsubsection{\nI{} isn't that strong}
\label{sec:ni-isnt-that}

\begin{note}
  Look, \nI{} rules out a way of claiming support quite broadly.
  However, this is because we're focusing on \aben{the}.
  This shouldn't be taken to suggest that there's general tension between \nI{} and \adA{}.
\end{note}

\section{Establishing tension/summary}
\label{sec:establishing-tension}

\begin{note}[Results of the distinctions]
  Recap.

  \begin{itemize}
  \item Kind of scenario involving ability.
  \item Distinction between \AR{} and \WR{}.
  \item Distinction between \adA{} and \adB{}.
  \item Distinction matrix.
  \item Point of this was to provide an exhaustive account of the ways in which the agent may claim support in the scenarios of interest.
  \end{itemize}

  Then, moved to figuring out whether the respective combinations of the distinction matrix are permissible.
  \begin{itemize}
  \item \ESU{} ruled out \adB{}, with the exception of \AR{} basic.
  \item \nI{} ruled out \adA{} no matter way in which ability was though about.
  \end{itemize}

  So, if this is correct, we've got three options.
  \begin{itemize}
  \item \AR{} basic, with \adB{}.
  \item Reject \nI{}.
  \item Reject \ESU{}.
  \end{itemize}
\end{note}

\begin{note}[Matrix]
  \begin{figure}[H]
    \centering
    \begin{tblr}{abovesep=8pt, belowsep=8pt, width=0.95\textwidth, colspec={Q[c,m]|Q[c,m]|Q[1.8,c,m]|Q[1.8,c,m]}}
      \multicolumn{2}{c}{} & \adA{} & \adB{} \\
      \hline
      \multicolumn{2}{c}{\WR{}} & Ruled out by \nI{}  & Ruled out by \ESU{} \\
      \hline
      \multirow{2}{*}{\AR{}} & Basic  & Ruled out by \nI{}  & ---  \\
      \cline[dashed]{2-4}
      & Derived & Ruled out by \nI{}  & Ruled out by \ESU{} \\
    \end{tblr}
    \caption{Distinction matrix}
  \end{figure}
\end{note}

\subsubsection{\AR{} Basic with \adB{}}
\label{sec:ar-basic-with}

\begin{note}
  Main issue here is that it doesn't seem as though this is a basic step of reasoning.

  For, ability breaks down into various components.

  Response here is that there do seem to be basic steps of reasoning with are similar in this respect.
  For example, it seems as though many cases of moving from cause to effect will do this.

  I think going back to dispositions might help here.
  
\end{note}

\subsubsection{Reject reasoning}
\label{sec:reject-reasoning}

\begin{note}
  If not basic, then reject reasoning.
\end{note}

\begin{note}
  Problem here is that this seems too strong.
\end{note}

\begin{note}[Main problem]
  The main problem is that it seems fine for the agent to claim support for specific ability.
  And, that \aben{the} applies.

  This gives rise to some tension.

  Possible resolution here is that agent expects things are as they would be if agent witnessed, but does not get to claim support.
  Issues only follow from claiming support.

  However, then the agent doesn't get to do anything with the proposition.

  Flipside is that claiming support is minimal.
  Agent does this to use the proposition in further reasoning, and only constraint is \eiS{}.
\end{note}

\begin{note}
  Addition:
  There's some parallel here with reflection.
\end{note}

\begin{note}[Reflection]
\begin{quote}
    Reflection states that agents should treat their future selves as experts or, roughly, that an agent’s current credence in any proposition A should equal his or her expected future credence in A.\linebreak
    \mbox{}\hfill\mbox{(\Citeyear[59]{Briggs:2009up})}
  \end{quote}
\end{note}

\begin{note}[Difference to reflection]
  Key difference is that in these cases, there's no guarantee that the agent will go through with ability.
  So, it's not necessarily a future self of the agent.
  Though, that's only on a quick surface reading of reflection.

  This is somewhat delicate.
  For, reflection has some strong background assumptions.
  Problem with ability is that agent might witness ability.
  With reflection, we don't consider restrictions on the reasoning the agent would do.

  Now, weakening reflection is difficult.

  One the one hand, can consider all evidence that the agent would reason through.
  If so, then it looks as though ability is going to fall within the scope.
  Problem here, however, because the argument for reflection is in terms of coherence.
  And, it's not clear how to apply conditionalisation to boundedness.
  Dutch books are about coherent credence functions.

  So, it is not clear that there's a way to derive instances of \aben{the} from principles which motivate reflection.
\end{note}

\begin{note}
  Taking a step back, in these kinds of cases it's something like evidence of evidence.
  This is in \textcite[2]{Tal:2017uw}, linking to another paper.

  And, this is kind of similar to what's going on with ability.

  This only works easily from \AR{} perspective.
  Still\dots

  Things get complex here.
  For, if this is the case, then it's not clear that the agent needs to worry about \aben{the}.
  So, the issues arising from the matrix don't really apply.

  Point here is that the agent could go straight for general ability.

  Problem is that agent still needs to get specific ability.

  Same issues with \ESU{} and \nI{} apply here.
  Still get something appealed to but not used.
\end{note}

\begin{note}
  Had scenarios.
  Here, just added that there are close things in the literature which seem fine.
\end{note}


\subsubsection{Reject \nI{}}
\label{sec:reject-ni}

\begin{note}
  Follows from \eiS{}, mostly.

  And, \eiS{} seems quite plausible.
\end{note}

\begin{note}
  Inclined to discount similarities to \citeauthor{Wright:2011wn} and \citeauthor{Weisberg:2010to}.
  Enough of a different.
  And, these kind of things are very difficult.
\end{note}

\begin{note}
  Rather, motivation and illustration.
\end{note}

\begin{note}
  Response here to focus on the difficult cases.
  But, as noted, because \nI{} is only about a way of claiming support, there are various ways in which one may deal with those cases.
\end{note}

\subsubsection{Reject \ESU{}}
\label{sec:reject-esu}

\begin{note}
  The literature.

  And, intuitive appeal.

  Maybe that's enough.
  Still, I want to see argumentation.
\end{note}

\subsection{Outlook}
\label{sec:outlook}

\begin{note}
  Reject \AR{} basic and \ESU{}.

  This leaves us with two options for understanding \aben{the}.
\end{note}

\subsection{\AR{}}
\label{sec:ar-2}

\begin{note}
  I don't have too much to say here.

  Outlined the idea of \AR{}.

  Considered this in terms of evidence of evidence.
  Ability then doing the work of making that evidence evidence for the agent.

  Role for ability here is securing that there is evidence.
  For, without ability, agent doesn't get to establish these background conditions.
\end{note}

\subsection{\WR{}}
\label{sec:wr-2}

\begin{note}
  Favoured
\end{note}


%%% Local Variables:
%%% mode: latex
%%% TeX-master: "master"
%%% End: