\chapter{\tC{2}}
\label{cha:typical}
\nocite{Wilson:1994aa}
\nocite{Goodman:1983aa}

\begin{note}
  This chapter introduces and develops the idea of an agent \emph{\typeAdj{}} concluding some \prop{0}-\val{0} pair from some \pool{}.
  Intuitively, an agent is \typeAdj{} concluding just in case there is some generality to the agent's reasoning which concluding.
  For example, the agent is reasoning by modus ponens, arithmetic, or the categorical imperative, and so on.

  The role of an agent \tCV{} in the overall argument is motivation.
  Counterexamples to \issueConstraint{} without \tCV{}.
  However, depend on way in which \scen{0} is understood.
  If agent is \tCV{}, then leverage.

  \autoref{cha:requs} introduces the key idea of a \requ{}, and instances of \requ{1} are motivated by the issue of whether or not an agent is \tCV{}.
\end{note}

\begin{note}
  The chapter has two main sections:
  \begin{TOCEnum}
  \item
    \TOCLine{cha:typical:int}

    General idea, \illu{1}
  \item
    \TOCLine{cha:typical:tCDef}

    Link to \fc{1}.
    The link to \fc{1} is important for motivating instances of \requ{} in \autoref{cha:requs}.
  \end{TOCEnum}
\end{note}



\section{\tC{2}}
\label{cha:typical:int}

\begin{note}
  Our interest is characterising an event in which an agent is concluding.
  By~\autoref{assu:ConRea} (\autopageref{assu:ConRea}), whenever an concludes, and agent reasons.
  Hence, whenever an agent is concluding, and agent is reasoning.
  And, some instances of reasoning are, intuitively, of a type:
  There are sufficiently similar characteristics between two or more events in which an agent reasons for the agent to be reasoning by `type of reasoning \(T\)', where `type of reasoning \(T\)' may be `modus ponens', `means-end reasoning',%
  % \footnote{
  %   To illustrate, consider the following passage:
  %   \begin{quote}
  %     \indent ``I'm giving this to Eeyore,'' he explained, ``as a present.
  %     What are you going to give?''

  %     ``Couldn't I give it too?'' said Piglet.
  %     ``From both of us?''

  %     ``No,'' said Pooh.
  %     ``That would not be a good plan.''

  %     ``All right, then, I'll give him a balloon.
  %     I've got one left from my party.
  %     I'll go and get it now, shall I?''

  %     ``That, Piglet, is a very good idea.
  %     It is just what Eeyore wants to cheer him up.
  %     Nobody can be uncheered with a balloon.''%
  %     \mbox{ }\hfill\mbox{(\cite[78--79]{Milne:2009aa})}\newline
  %     \mbox{ }
  %   \end{quote}

  %   Two instances of means-end reasoning by Piglet.
  %   Common end of cheering up Eeyore.
  %   First, jointly giving a gift with Pooh.
  %   Second, giving a balloon as a present.
  % }
  `arithmetic', `consequentialism', and so on.

%   For example, \assumptionName{3}~\ref{assu:concluding:pools},~\ref{assu:ConRea},~\ref{assu:PP}~and \definitionName{3}~\ref{def:witnessing},~\ref{def:fc},~and~\ref{def:NScon} all involve a material conditional.%
%   \footnote{
%     Pages: \pageref{assu:concluding:pools},~\pageref{assu:ConRea},~\pageref{assu:PP} and~\pageref{def:witnessing},~\pageref{def:fc},~and~\pageref{def:NScon}, respectively.
%   }%
%   \(^{,}\)%
%   \footnote{
%     The indicative conditional is more complex.
%     See, for example,~(\cite{McGee:1985tz}),~(\cite{Kolodny:2010aa}), and the discussion of selection tasks starting on \autopageref{par:selection-tasks}, below.
%   }

%   Two key pieces of reasoning:

%   \begin{itemize}
%   \item
%     \emph{From} \pv{\propI{If }\phi\propI{ then }\psi}{\valI{True}} \emph{and} \pv{\propI{\phi}}{\valI{True}}, \emph{get} \pv{\propI{\psi}}{\valI{True}}.
%   \item
%     \emph{From} \pv{\propI{\phi}}{\valI{True}} \emph{and} \pv{\propI{\psi}}{\valI{False}}, \emph{get} \pv{\propI{If }\phi\propI{ then }\psi}{\valI{False}}
%   \end{itemize}

%   First is useful for applying.
%   Second is useful for rejecting.

%   For example:
%   First when considering arguments for propositions.
%   Second when determining whether there is a problem with one or more.

%   Definitions etc. modus ponens with assumption this pattern is recognised.
%   Additional work to fill in the content, but nothing about what connects the content.
\end{note}

\begin{note}
  I expect this observation is intuitive, and specific examples will follow.
\end{note}

\begin{note}
  Our interest is stating a necessary condition for whether or not an event in which an agent is concluding is an event in which an agent's reasoning is of some type.

  Provide idea.
  \illu{3} of idea.
  Expand, dispositions.
\end{note}


\subsection{Idea}
\label{sec:idea}

\begin{note}
  Abstractly, consider `\vAgent{} is \emph{\tCV{}} \(\pv{\phi}{v}\) from \(\Phi\) by type of reasoning \(T\)' as a predicate of an event.
  Necessary condition for predicate.

  \begin{idea}[\tCN{2}]
    \label{idea:tC}
    \cenLine{
      \begin{itemize*}[noitemsep, label=\(\circ\)]
      \item
        Agent: \vAgent{}
      \item
        \prop{2}: \(\phi\)
      \item
        \val{2}: \(v\)
      \item
        \pool{2}: \(\Phi\)
      \item
        Event: \(e\)
      \item
        \mbox{ }
      \end{itemize*}
    }

    \begin{itemize}
    \item
      \(e\) is an event in which \vAgent{} is \emph{\tCV{}} \(\pv{\phi}{v}\) from \(\Phi\).\newline
      \hfill(By some type of reasoning \(T\).)
    \end{itemize}

    \emph{Only if}:

    \begin{itemize}
    \item
      For some collections; \({\cal E}\) of events, \({\cal X}\) of \prop{0}-\val{0}-\pool{0}~pairings:
      \begin{itemize}
      \item
        For every event \(e'\), there is some \prop{0}-\val{} pair \(\pv{\psi}{v'}\) and \pool{0} \(\Psi\) in \(\mathcal{X}\) such that:
        \begin{itenum}
        \item[\emph{If}:]
          \(e'\) is in the collection of events \({\cal E}\).
        \item[\emph{Then}:]
          \(e'\) is an event in which \vAgent{} is concluding \(\pv{\psi}{v'}\) from \(\Psi\).
        \end{itenum}
      \end{itemize}
    \end{itemize}
    \vspace{-\baselineskip}
  \end{idea}

  \noindent%
  Intuitively, \autoref{idea:tC} expresses the idea that the predicate `\vAgent{} is \tCV{} \(\pv{\phi}{v}\) from \(\Phi\) by type of reasoning \(T\)' applies to an event only if some `law' holds.%
  \footnote{
    Note, this is distinct from the position that concluding/reasoning is rule governed.
    \cite{Boghossian:2008vf,Boghossian:2012vb}, \cite{Broome:2002aa}.

    When reasoning, following a rule.
    We are talking about \emph{type} of concluding, rather than concluding.
    But, the object is not following the predicate.
  }
  Where, `law' is understood in the colloquial sense of a universally quantified material conditional.%
  \footnote{
    For example, consider:

    \citeauthor{Helmholtz:1977aa}'s characterisation of laws of nature:%
    \begin{quote}
      \nocite{Wilson:2006aa}
      Every law of nature asserts that upon preconditions alike in a certain respect, there always follow consequences that are alike in a certain other respect.%
      \mbox{ }\hfill\mbox{(\citeyear[122]{Helmholtz:1977aa})}
    \end{quote}
    The law of large numbers:
    \begin{quote}
      Things of every kind of nature are subject to a universal law which one may well call \emph{the Law of Large Numbers}.
      It consists in that if one observes large numbers of events of the same nature depending on causes which are constant and causes which vary irregularly, \dots, one finds that the proportions of occurrence are almost constant \dots\newline
      \mbox{ }\hfill\mbox{(\citeauthor{Seneta:2013aa}'s (\citeyear[9--10]{Seneta:2013aa}) translation of (\cite[7]{Poisson:1837aa}))}
    \end{quote}
    The law of truly large numbers:
    \begin{quote}
      [W]hen enormous numbers of events and people and their interactions cumulate over time, almost any outrageous event is bound to occur.%
      \mbox{ }\hfill\mbox{(\cite[853]{Diaconis:1989aa})}
    \end{quote}
    \citeauthor{Hempel:1965aa}'s Deductive-Nomological account of scientific explanation, \citeauthor{Boole:1854aa}'s laws of thought, etc.
  }

  \autoref{idea:tC} does not specify the content of the relevant law.
  However, intuitively, the law is to capture other cases where the agent is expected to be concluding by the same type of reasoning, and requires the agent \emph{is} concluding by the same type of reasoning.
\end{note}

\begin{note}
  Before turning to \illu{1}, note:
  \autoref{idea:tC} is not an analysis.
    Rather, \autoref{idea:tC} connects two truth functional statements by a conditional.
\end{note}



\subsection{\illu{3}}
\label{sec:illu3-1}

\begin{note}
  A few \illu{1}.
\end{note}


\paragraph*{Selection tasks}
\nocite{Wason:1968aa}
\nocite{Wason:1971aa}
\label{par:selection-tasks}

\begin{note}
  \citeauthor{Wason:1966aa} details their initial section task as follows:

  \begin{quote}
    The subjects (students) were presented with an array of cards and told that every card had a letter on one side and a number on the other side, and that either would be face upwards.
    They were then instructed to decide which cards they would need to turn over in order to determine whether the experimenter was lying in uttering the following statement:
    \emph{if a card has a vowel on one side then it has an even number on the other side}.%
    \mbox{ }\hfill\mbox{(\citeyear[145--146]{Wason:1966aa})}
  \end{quote}

  An example task is given in \autoref{fig:sectionTask}.

  \begin{figure}[H]
    \centering
    \begin{tikzpicture}[
      cardnode/.style={
        rectangle,
        minimum width=10mm,
        minimum height=14mm,
        align=center,
        rounded corners,
        font = {\Large\sffamily},
        very thick,
      },
      node distance=5mm,
      ]

      \node[cardnode, draw] (1) {2};
      \node[cardnode, draw, right = of 1] (2) {N};
      \node[cardnode, draw, right = of 2] (3) {E};
      \node[cardnode, draw, right = of 3] (4) {1};
    \end{tikzpicture}
    \caption{A selection task}
    \label{fig:sectionTask}
  \end{figure}

  \citeauthor{Wason:1966aa} observes the results are consistent with the following hypothesis:%
  \footnote{
    \citeauthor{Wason:1966aa} does not provide a detailed summary of the results.
    However, \citeauthor{Johnson-Laird:1969aa} detail results of \emph{twenty four} University College London students!
    Specifically, 19 of the 24 responded as excepted given \citeauthor{Wason:1966aa}'s hypothesis.
    (\citeyear[369--370]{Johnson-Laird:1969aa}).
  }
  \begin{quote}
    Subjects assume implicitly that a conditional statement has, not two truth values, but three: true, false and `irrelevant'.
    Vowels with even numbers verify, vowels with odd numbers falsify and consonants with any number are irrelevant.%
    \mbox{ }\hfill\mbox{(\citeyear[146]{Wason:1966aa})}
  \end{quote}
\end{note}

\begin{note}
  With respect to each subject, four distinct instances of \autoref{idea:tC}, corresponding to the four possible features of a card.
  We specify these with a variable `\(C\)' to represent the relevant card:

  \begin{center}
    \begin{tabular}{R{.45\textwidth} L{.45\textwidth}}
      \prop{2}-\val{0} pair & \pool{2} \\
      \hline
      \pv{C\propI{ needs to be turned over}}{\valI{True}} & \pv{C\propI{ has a vowel}}{\valI{True}} \\
      \pv{C\propI{ needs to be turned over}}{\valI{True}} & \pv{C\propI{ has an odd number}}{\valI{True}} \\
      \pv{C\propI{ needs to be turned over}}{\valI{False}} & \pv{C\propI{ has consonant}}{\valI{True}} \\
      \pv{C\propI{ needs to be turned over}}{\valI{False}} & \pv{C\propI{ has an even number}}{\valI{True}} \\
    \end{tabular}
  \end{center}

  \noindent%
  Short argument.
  Begin with two premises.

  \begin{enumerate}[label=\Alph*., ref=(\Alph*), noitemsep]
  \item
    \label{WasArg:results}
    Fail to express conclusions.
  \item
    \label{WasArg:IdeaI}
    \autoref{idea:tC} as described.
  \end{enumerate}

  Link these together:

  \begin{enumerate}[label=\arabic*., ref=(\arabic*), noitemsep]
  \item
    \label{WasArg:nC}
    Subjects did not conclude.

    From \ref{WasArg:results}.
  \item
    \label{WasArg:nCing}
    Subjects were not concluding.

    From \ref{WasArg:nC}, nothing to prevent the agent's from concluding.
  \item
    \label{WasArg:Done}
    The agent's were not \tCV{} by truth functional reasoning.

    From \ref{WasArg:nCing} and \ref{WasArg:IdeaI}
  \end{enumerate}

  Strength of the result depends on which instances of reasoning.

  {
    \color{red}

    Even if there are cases of truth-functional reasoning, something separates this from the reasoning found in selection tasks --- does not extend to all circumstances.
  }
\end{note}

\begin{note}
  Various other arguments may be seen to parallel the broad argument from \ref{WasArg:results} and \ref{WasArg:IdeaI} to \ref{WasArg:Done}.
  For example, consider the \citeauthor{Allais:1979aa} paradox (\cite{Allais:1979aa}),
  the Ellsberg paradox (\cite{Ellsberg:1961aa}), \citeauthor{Makinson:1965aa}'s Paradox of the Preface (\citeyear{Makinson:1965aa}), \citeauthor{Kyburg:1997aa}'s Lottery Paradox (\citeyear{Kyburg:1997aa}), \citeauthor{Quinn:1990aa}'s  puzzle of the self-torturer (\citeyear{Quinn:1990aa}), \citeauthor{Bratman:1981aa}'s arguments against the desire-belief model of practical reasoning (\citeyear{Bratman:1981aa,Bratman:1987aa}), and so on.%
  \footnote{
    Consider also \citeauthor{Harman:1984aa}'s (\citeyear{Harman:1984aa,Harman:1986ux}) arguments against a strong connexion between logical principles and principles of belief revision.

    \begin{quote}
      Logical principles are not directly rules of belief revision.
      [\dots]
      Logical principles hold universally, without exception, whereas the corresponding principles of belief revision would be at best prima facie principles, which do not always hold.%
      \mbox{ }\hfill\mbox{(\citeyear[107--108]{Harman:1984aa})}
    \end{quote}
  }
  Common to each mentioned is the idea that an agent failing to conclude something shows that other instances of the agent's reasoning does not have some general characteristic.
\end{note}


\paragraph*{Rules}

\begin{note}
  Selection tasks are events which happen, and \citeauthor{Wason:1966aa}'s hypothesis is that people \emph{in general} do not do not reason about conditionals using only \valI{True} and \valI{False}.
  However, the events the law of \autoref{idea:tC} quantifies over need not happen, and the event at issue may be a single event.
\end{note}

\begin{note}
  \begin{scenario}[Addition]%
    \label{illu:quus}%
    An agent is given pairs of numbers \(x\) and \(y\) and asked to respond with \(x + y\).
    The table below represents the event as the agent responds to the pairs.

    \medskip
    \hspace{2.8em}%
    \(
      \begin{array}{ccccccc}
      x & 3 & 54 & 21 & 3 & 17 & 0 \\
      y & 7 & 32 & 64 & 2 & 25 & 6 \\
      \hline
      \text{Response} & 10 & 86 & 85 & 5 & 42 & 6 \\
    \end{array}
    \)
    \medskip

    \noindent%
    The agent is distracted.
    However, if the agent had not been distracted, they would have continued as follows:

    \medskip
    \hfill%
    \(
    \begin{array}{ccccc}
      \cdots & 8 & 68 & 21 & 58 \\
      \cdots & 92 & 57 & 23 & 92 \\
      \hline
      \cdots & 100 & 5 & 44 & 5 \\
    \end{array}
    \)%
    \hspace{2.8em}%
    \mbox{ }%
    \newline%
  \end{scenario}

  \noindent%
  It seems the agent was not reasoning by addition.%
  \footnote{
    Rather, it seems the agent was reasoning by quaddition.
    Consider \citeauthor{Kripke:1982aa}'s (\citeyear{Kripke:1982aa}) def.\ of `quss':
    \begin{align*}
      x \text{ quss y} &= x \text{ plus } y, \text{ if } x,y < 57 \\
                       &= 5 \phantom{pl if x,,,} \text{ otherwise }
    \end{align*}
    \vspace{-\baselineskip}
  }

  For, consider the counterfactual event.
  The agent concluded \pv{\propI{x + y is 5}}{\valI{True}} from some \pool{} containing \pv{\propI{x is 68}}{\valI{True}} and \pv{\propI{y is 57}}{\valI{True}}.
  Hence, it seems the agent was not concluding \pv{\propI{x + y is 125}}{\valI{True}}.

  Further, though the counterfactual event suggests the agent was not reasoning by addition, it is less clear that the agent never reasons by addition.
  The agent's interpretation of `\(+\)' as something other than `plus' may be no different from interpreting `\(A \land B\)' as `\(A\) and \(B\)' rather than `the meet of sets \(A\) and \(B\)'.
\end{note}


\paragraph*{Powers}

\begin{note}
  Selection tasks and addition.
  Final \illu{0}, not all failures show an agent reasoning fails to be of some type.
\end{note}

\begin{note}
  \begin{scenario}[Powerful multiplication]
    \label{illu:tR:powers}
    A student has been studying algebra and has just been introduced to the rule of multiplication for powers (\(a^{n} \cdot a^{m} = a^{n + m}\)).

    At hand are a handful of exercises (from \cite[32]{Gelfand:1993aa}):

    \begin{quote}
      \begin{enumerate}[label=(\alph*), ref=(\alph*)]
      \item
        \label{mfp:a}
        You know that \(2^{1001} \cdot 2^{n} = 2^{2000}\).
        What is \(n\)?
      \item
        \label{mfp:b}
        You know that \(2^{1001} \cdot 2^{n} = \sfrac{1}{4}\).
        What is \(n\)?
      \item
        \label{mfp:c}
        Which is bigger: \(10^{-3}\) or \(2^{-10}\)?
      \item
        \label{mfp:d}
        You know that \(\sfrac{2^{1000}}{2^{n}} = 2^{501}\).
        What is \(n\)?
      \item
        \label{mfp:e}
        You know that \(\sfrac{2^{1000}}{2^{n}} = \sfrac{1}{16}\).
        What is \(n\)?
      \item
        \label{mfp:f}
        You know that \(4^{100} = 2^{n}\).
        What is \(n\)?
      \item
        \label{mfp:g}
        You know that \(2^{100} \cdot 3^{100} = a^{100}\).
        What is \(a\)?
      \item
        \label{mfp:h}
        You know that \((2^{10})^{15} = 2^{n}\).
        What is \(n\)?
      \end{enumerate}
    \end{quote}

    The student starts work on exercise~\ref{mfp:f}, and is concluding \(\pv{n\propI{ is }200}{\valI{True}}\).
  \end{scenario}

  Is it the case that the student is reasoning with an understanding of multiplication for powers?

  Consider the event in which the agent starts work on~\ref{mfp:g}.
  Does it need to be the case that the agent is concluding \pv{\propI{a is 6}}{\valI{True}} from \pv{\propI{\(2^{100} \cdot 3^{100} = a^{100}\)}}{\valI{True}} in order for the agent to be reasoning by the rule of multiplication for powers when working on~\ref{mfp:f}?

  I think arguments may be made either way.
  On the one hand, the solution to \ref{mfp:f} may be obtained by straightforward pattern matching, while \ref{mfp:g} requires some insight.
  Hence, reasoning to the solution to \ref{mfp:g} may be required for the agent to be reasoning `with' the rule, rather than `by' the rule.
  On the other hand, the rule of multiplication for powers immediately follows from a basic understanding of exponentiation, and hence reasoning `with' the rule is simply reasoning by exponentiation --- there is no need for the agent to follow a rule.
\end{note}

\begin{note}
  \begin{observation}
    \label{obs:typeLim}
    \cenLine{
      \begin{itemize*}[noitemsep, label=\(\circ\)]
      \item
        Agent: \vAgent{}
      \item
        \prop{2}: \(\phi\)
      \item
        \val{2}: \(v\)
      \item
        \pool{2}: \(\Phi\)
      \item
        Event: \(e\)
      \item
        \mbox{ }
      \end{itemize*}
    }%

    \begin{itemize}
    \item
      It is possible that:
      \begin{itemize}
      \item
        \(e\) is an event in which \vAgent{} is \tCV{} \(\pv{\phi}{v}\) from \(\Phi\).
      \item
        For some \(\pv{\psi}{v'}\), \(\Psi\) in \(\mathcal{X}\):
        \begin{itemize}
        \item
          There is no event \(e'\) in which \vAgent{} is concluding \(\pv{\psi}{v'}\) from \(\Psi\).
        \end{itemize}
      \end{itemize}
    \end{itemize}

    \vspace{-\baselineskip}
  \end{observation}

  \begin{motivation}{obs:typeLim}
    Consider \autoref{illu:tR:powers}.
    Exercises~\ref{mfp:b},~\ref{mfp:d}, and~\ref{mfp:e} all involve fractions.
    However, agent is shaky on fractions.
    Rule of multiplication is good, but not with fractions.%
  \footnote{
    Consider also \citeauthor{Chomsky:2015aa}'s distinction between competence and performance.

    \begin{quote}
      Arithmetical competence yields the correct number z for every pair~(x,~y) under addition or multiplication.
      But only a small finite subpart of arithmetical competence can be exhibited without external aids (by calculating in one's head).
      Obviously, that fact does not imply that arithmetical competence is correspondingly limited.%
      \mbox{ }\hfill\mbox{(\citeyear[xii]{Chomsky:2015aa})}
    \end{quote}

    Though, \citeauthor{Chomsky:2015aa} also motivates the distinction by errors (\citeyear[2]{Chomsky:2015aa}).

    For us, errors and mistakes are dealt with via the progressive and the relevant \torN{}.
    An agent may make errors or mistakes, while concluding --- so long as the agent is concluding.
    However, an agent must be concluding.
    Hence, if an agent is permitted to fail to be concluding some \(\pv{\psi}{v'}\) from \(\Psi\) while \tCV{} \(\pv{\phi}{v}\) from \(\Phi\), then \(\pvp{\psi}{v'}{\Psi}\) is not a \tI{} of the relevant \torNa{}.
    (See the discussion of \autoref{illu:tR:powers} on \autopageref{illu:tR:powers}.)
}
  \end{motivation}

\end{note}


\subsection[Dispositions]{Dispositions \hfill (Optional)}
\label{sec:dispositions}

\begin{note}
  In this optional section we clarify \autoref{idea:tC} by applying the same to dispositions.
\end{note}

\begin{note}
  Consider the `\dBCA{0}' of dispositions:%

  \begin{sketch}[\dBCA{2} --- \dBCAa{0}]
    \label{sketch:dBCA}
    \vspace{-\baselineskip}
    \begin{itemize}
    \item
      Object \(o\) has disposition \(d\)
    \end{itemize}
    \emph{If and only if}:
    \begin{itemize}
    \item
      There are descriptors \(C\)(ondition) and \(M\)(anifestation) such that:
      \begin{itemize}
      \item
        \emph{If} it were the case that \(C\), \emph{then} \(o\) would \(M\).
      \end{itemize}
    \end{itemize}
    \vspace{-\baselineskip}
  \end{sketch}

  \noindent%
  The \dBCA{0} is common.
  For example:

  \begin{quote}
    To say that an object \(a\) is (water-) \emph{soluble} at time \(t\) is to say that if \(a\) were in water at \(t\), \(a\) would dissolve at \(t\).%
    \mbox{ }\hfill\mbox{(\cite[203]{Quine:2013aa})}
  \end{quote}

  \begin{quote}
    Dispositional words like `know', `believe', `aspire', `clever' and `humorous' are determinable dispositional words.
    They signify abilities, tendencies or pronenesses to do, not things of one unique kind, but things of lots of different kinds.%
    \mbox{ }\hfill\mbox{(\cite[118]{Ryle:1949aa})}
  \end{quote}

  \begin{quote}
    [A] statement like

    \(w\) is inflammable

    amounts to [\dots] some such fainthearted counterfactual as

    If all conditions had been propitious and \(w\) had been heated enough, it would have burned.%
    \mbox{ }\hfill\mbox{(\cite[39]{Goodman:1983aa})}
  \end{quote}
\end{note}

\begin{note}
  \begin{proposition}[Basic proposition]%
    \label{obs:disp:basic}%
    The \dBCA{0} entails:

    \begin{itenum}
    \item[\emph{If}:]
      Object \(o\) has disposition \(d\)
    \item[\emph{Then}:]
      There are descriptors \(C'\) and \(M'\) such that:
      \begin{itemize}
      \item
        For every \scen{0}:
        \emph{If} \(C'\) is the case, \emph{then} \(o\) manifests \(M'\).
      \end{itemize}
    \end{itenum}
    \vspace{-\baselineskip}
  \end{proposition}

  \begin{argument}{obs:disp:basic}
    Suppose the \dBCA{0} holds and condition some object \(o\) with disposition \(d\).
    Given the \dBCAa{0}, there are descriptors \(C\) and \(M\) such that:
    \emph{If} it were the case that \(C\), \emph{then} \(o\) would \(M\).

    Consider the \scen{1} where \(C\) are the case, and what \(M\) captures.
    Now, re-express \(C\) as \(C'\) and \(M\) as \(M'\) such that \(C'\) and \(M'\) do not depend on context (e.g.\ world) of evaluation.%
    \footnote{
      We understand the \dBCAa{} as specifying \(C\) and \(M\) relative to a context, and hence \(C'\) and \(M'\) are given with respect to some context.
      However, it does not follow that \(C\) and \(M\) need to take the relevant context as an argument.
      (\autoref{idea:tC} is also understood this way.)
    }%
    \(^{,}\)%
    \footnote{
      Though, in general, \(C'\) need not capture all conditions.
      For, \autoref{obs:disp:basic} is a \emph{only if} statement.
    }
  \end{argument}
  The parallel between the consequent of \autoref{obs:disp:basic} and \autoref{idea:tC} is clear by inspection.%
  \footnote{
    However, \autoref{obs:disp:basic} does not (immediately, at least) suggest \tCV{} may be understood as a disposition.
    For, \autoref{obs:disp:basic} is an \emph{only if} statement.
    At least, no more than a law suggests a calculator is disposed to display \(345\) after \(23 \cdot 15\) and a button marked `\(=\)' is pressed.
  }
\end{note}

\begin{note}
  \begin{observation}
    \label{obs:disp:partial}
    Given \autoref{obs:disp:basic}, a partial grasp on conditions under which disposition manifests (i.e.\ \(C'\) and \(M'\)) is sufficient to establish an object does not have disposition \(d\).
  \end{observation}

  \begin{motivation}{obs:disp:partial}
    Sufficient for `law' on which disposition depends is false.
    As law is universally quantified conditional, one needs only find a \scen{0} where \(C'\) obtains and \(o\) does not manifest \(M'\).
  \end{motivation}

  For example, drop, doesn't break.
  Well, it's not fragile.%
  \footnote{
    Likewise, the existence of some law indicates whether some property is a dispositional property.
    E.g.\ are there laws for `likey' and `hatey'?
  }
\end{note}

\begin{note}
  With the parallel between the \dBCAa{} and \autoref{idea:tC} in hand, briefly observe a number of issues for the \emph{simple} conditional analysis of dispositions do not apply to the \dBCAa{}.
  And, hence, similar issues do not extend to \autoref{idea:tC}.

  The \dSCA{} is as follows:%
  \footnote{
    Compare to, e.g. \citeauthor{Lewis:1997wg}'s account:
    \textquote{%
      Something \(x\) is disposed at time \(t\) to give response \(r\) to stimulus \(s\) iff, if \(x\) were to undergo stimulus \(s\) at time \(t\), \(x\) would give response \(r\).%
    }
    (\citeyear[143]{Lewis:1997wg})
  }

  \begin{sketch}[The \dSCA{} --- \dSCAa{}, cf.\ \cite[\S1.2]{Choi:2021wg}]%
    \label{sketch:dSCA}
    \vspace{-\baselineskip}
    \begin{itemize}
    \item
      An object \(o\) disposed to \(M\) when \(C\)
    \end{itemize}
    \emph{If and only if}:
    \begin{itemize}
    \item
      \(o\) would \(M\) if it were the case that \(C\).
    \end{itemize}
    \vspace{-\baselineskip}
  \end{sketch}

  Observe, \(C\) and \(M\) are used to characterise the disposition, and therefore the choice of \(C\) and \(M\) in the counterfactual is fixed.
  By contrast, the \dBCAa{} analysed a disposition \(d\) which did not contain \(C\) and \(M\), and hence allowed free choice of \(C\) and \(M\) in the corresponding counterfactual.

  Further, observe well-know counterexamples to the \dSCAa{} require the choice of \(C\) and \(M\) being fixed.

  For example, consider `masks'.%
  \footnote{
    Parallel observations apply to (reverse) finks.
    (\cite{Martin:1994aa})
  }
  Following~\citeauthor{Clarke:2010aa} a mask is \textquote{something that prevents a disposition from manifesting despite the occurrence of that disposition's characteristic stimulus} (\citeyear[153]{Clarke:2010aa}).%
  \footnote{
    \cite{Johnston:1992aa} provides counterexamples to a specific instance of the \dBCAa{}.
  }

  For example, consider the following conditional with respect to `edible':
  If (were) ingested then (would be) digested.

  A sweet is digested when ingested, but a sweet wrapped in plastic is not.
  In this respect, being wrapped in plastic is a mask of the proposed analysis.

  Masks are a problem for the \dSCAa{}.
  For, the \dSCAa{} concerns the disposition to be digested when ingested, but there is little preventing someone from ingesting a sweet wrapped in plastic.
  Still, masks are not a problem for the \dBCAa{}.
  For, the \dBCAa{} concerns `edible', and the conditional may be rejected as an adequate analysis as it fails to specify the appropriate conditions \(C'\) under which \(M'\) manifests.

  For sure, a masks may be counterexamples to \emph{instances} of the \dBCAa{}, but in contrast to the \dSCAa{}, masks are not counterexamples to the (basic conditional) \emph{analysis}.%
  \footnote{
    As \citeauthor{Bonevac:2011tz} stress:
    \begin{quote}
      Counterexamples must be deployed as counterexamples to specific proposals.
      The example of a glass packed in styrofoam can perhaps show that fragile cannot be analysed as would break if struck, but it shows nothing about a proposed analysis of fragile as would break if struck when unwrapped, and certainly shows nothing about any proposed analysis of a different dispositional term, such as irascible.%
      \mbox{ }\hfill\mbox{(\citeyear[1144]{Bonevac:2011tz})}
    \end{quote}

    \nocite{Manley:2007aa}
    In response~\citeauthor{Manley:2011aa} (\citeyear{Manley:2011aa}) seem to argue that counterexamples concern whether it is possible for the \dBCAa{0} to function as an analysis of disposition \emph{ascriptions} --- not whether the \dBCAa{0} is true (\citeyear[cf.][\S1.3]{Manley:2011aa}).
  }%
  \(^{,}\)%
  \footnote{
    Examples of \dBCA{1} are those cited by \citeauthor{Choi:2021wg} as endorsements of \dSCA{1}.
    This is clearly not the case.

    The same is true of \citeauthor{Manley:2008aa} (\citeyear[60]{Manley:2008aa}).
    Though, \citeauthor{Manley:2008aa}'s discussion is almost identical to that of \citeauthor{Fara:2006aa}~(\citeyear[\S2.1]{Fara:2006aa})\dots

    Anyway, \citeauthor{Choi:2021wg} distinguish the \dSCAa{} from `Entailment':
    \begin{quote}
      \emph{F} is a disposition iff there are an associated stimulus condition and manifestation such that, necessarily, \emph{x} has \emph{F} only if \emph{x} would produce the manifestation if it were in the stimulus condition.%
      \mbox{ }\hfill\mbox{(\citeyear[\S2.1]{Choi:2021wg})}
    \end{quote}
    And, `Entailment' is equivalent to the \dBCAa{0}.
    However, \citeauthor{Choi:2021wg} add:
    \begin{quote}
      If disposition ascriptions do not entail corresponding counterfactual conditionals, then Entailment is hopeless.
      Note that the apparent counterexamples to [the \dSCA{}] may seem to show just that.
      But let's leave this claim aside for the sake of argument.
    \end{quote}
    There is nothing to set aside here.
  }%
  \(^{,}\)%
  \footnote{
    This observation and predates masks.
    Consider, e.g., the following passage from \citeauthor{Goodman:1983aa}:

    \begin{quote}
      [W]e can define ``flexible'' if we find an auxiliary manifest predicate that is suitably related to ``flexes'' through `causal' principles or laws.
      The problem of dispositions is to define the nature of the connection involved here:
      the problem of characterizing a relation such that if the initial manifest predicate ``Q'' stands in this relation to another manifest predicate or conjunction of manifest predicates ``A'', then ``A'' may be equated with the dispositional counterpart---``Q-able'' or ``Q\textsc{d}''---of the predicate ``Q''.\nolinebreak
      \mbox{ }\hfill\mbox{(\citeyear[45]{Goodman:1983aa} --- first published in 1955)}
    \end{quote}
  }
\end{note}

\section{\tC{2} and \fc{1}}
\label{cha:typical:tCDef}

\begin{note}
  The purpose of this section is to link \tCV{} to \fc{1}.

  Broadly, the link is somewhat straightforward:
  By \autoref{idea:tC}, \tCV{} only if some law.
  In certain cases, some events will be \pevent{1}.
  As law requires agent concludes, \fc{}.

  To be precise, observe the relevant link is a consequence of three definitions.
  The initial three sections cover the definitions.
  Fourth section links definitions to \tCV{}.
  Section links \tCV{} to \fc{1} via the definitions.
\end{note}


\subsection{\torN{3}}
\label{cha:typical:tCDef:ToRdef}

\begin{note}
  \torN{3} are defined in terms of \prop{1}, \val{1}, and \pool{1}:

  \begin{definition}[\torN{2}]
    \label{def:tor}
    \mbox{ }
    \vspace{-\baselineskip}
    \begin{itemize}
    \item
      \(T\) is a \torN{}.
    \end{itemize}

    \emph{If and only if}:

    \begin{itemize}
    \item
      \(T\) is a collection of \prop{0}-\val{0}-\pool{0} pairings.
    \end{itemize}
    \vspace{-\baselineskip}
  \end{definition}

  \noindent%
  Intuitively, a \torN{} (as defined) as the \emph{extension} of a \torN{}.
  The motivation for \autoref{def:tor} has two parts:

  First, by \autoref{assu:concluding:pvp} (\autopageref{assu:concluding:pvp}) and agent conclusion is a \prop{0}-\val{0} pairing, and by \autoref{assu:concluding:pools} (\autopageref{assu:concluding:pools}) an agent always concludes from some \pool{}.
  Hence, whenever an agent is concluding there is always some relevant \prop{0}-\val{0}-\pool{0} pairing.

  Second, by \autoref{assu:ConRea} (\autopageref{assu:ConRea}), if an agent concludes, the agent reasons to the \prop{0}-\val{0} pair from the \pool{0}.
  However, we place no additional constraints on reasoning.
  Hence, the framework with which we work does not allow finer grain.
\end{note}

\begin{note}
  To \illu{0}, consider the type `multiplication by powers' with respect to \autoref{illu:tR:powers}:
  \begin{center}
    \begin{tabular}{R{.45\textwidth} L{.45\textwidth}}
      \multicolumn{2}{c}{\prop{2}-\val{}-\pool{} pairings in type `multiplication by powers'} \\
      \hline\hline
      Proposition-value pair & \pool{2} \\
      \hline
      \pv{\propI{n is 999}}{\valI{True}} & \pv{\propI{\(2^{1001} \cdot 2^n = 2^{2000}\)}}{\valI{True}}, \dots \\
      \pv{\propI{n is -1003}}{\valI{True}} & \pv{\propI{\(2^{1001} \cdot 2^{n} = \sfrac{1}{4}\)}}{\valI{True}}, \dots \\
      \pv{\propI{\(10^{-3}\) is bigger than \(2^{-10}\)}}{\valI{True}} & \dots \\
      \pv{\propI{n is 500}}{\valI{True}} & \pv{\propI{\(\sfrac{2^{1000}}{2^{n}} = 2^{501}\)}}{\valI{True}}, \dots \\
    \end{tabular}
  \end{center}

  \begin{center}
    \begin{tabular}{R{.45\textwidth} L{.45\textwidth}}
      \multicolumn{2}{c}{\prop{2}-\val{}-\pool{} pairings not in type `multiplication by powers'} \\
      \hline\hline
      Proposition-value pair & \pool{2} \\
      \hline
      \hline
      \pv{\propI{n is 996}}{\valI{False}} & \pv{\propI{\(\sfrac{2^{1000}}{2^{n}} = \sfrac{1}{16}\)}}{\valI{True}}, \dots \\
      \pv{\propI{n is 2000}}{\valI{True}} & \pv{\propI{\(4^{100} = 2^{n}\)}}{\valI{True}}, \dots \\
      \pv{\propI{a is 5}}{\valI{True}} & \pv{\propI{\(2^{100} \cdot 3^{100} = a^{100}\)}}{\valI{True}}, \dots \\
      \pv{\propI{n is \(150\)}}{\valI{Want}} & \pv{\propI{\((2^{10})^{15} = 2^{n}\)}}{\valI{True}}, \dots \\
    \end{tabular}
  \end{center}

  The collection may be understood representative of the broad type `multiplication by powers'.
  In this respect, many other \prop{0}-\val{0}-\pool{0} pairs, such as \pv{\propI{n is 10}}{\valI{True}} paired with \pv{\propI{\(2^{10} \cdot 2^n = 2^{10}\)}}{\valI{True}}, \dots, and \pv{\propI{n is -7}}{\valI{True}} paired with \pv{\propI{\(2^{10} \cdot 2^n = 2^{3}\)}}{\valI{True}}, \dots.

  Still, the collection may also be understood \tI{1} of the narrow type `multiplication by powers with respect to problems \ref{mfp:a}--\ref{mfp:h}' --- here additional pairings which cover problems \ref{mfp:e} to \ref{mfp:h} may be included.
  Or, understood as \tI{1} of the very narrow type `multiplication by powers with respect to problems \ref{mfp:a}--\ref{mfp:d}' --- here the collection seems complete.

  \torN{3} are, for our interests, collections of \prop{0}-\val{0}-\pool{0} pairings.
  There need be no `natural' description of the type.
\end{note}

\subsection{\ptC{2}}
\label{sec:ptr0}

\begin{note}
  With a \torN{} in hand, we define what it is for an agent to be `\ptypeAdv{0}' concluding:

  \begin{definition}[\ptC{2}]
    \label{def:ptC}
    \cenLine{
      \begin{itemize*}[noitemsep, label=\(\circ\)]
      \item
        Agent: \vAgent{}
      \item
        \prop{3}: \(\phi\), \(\psi\)
      \item
        \val{3}: \(v\), \(v'\)
      \item
        \pool{3}: \(\Phi\), \(\Psi\)
      \item
        \mbox{ }
      \end{itemize*}
    }

    \noindent%
    \cenLine{
      \begin{itemize*}[noitemsep, label=\(\circ\)]
      \item
        Event: \(e\)
      \item
        Type of reasoning: \(T\)
      \item
        \mbox{ }
      \end{itemize*}
    }

    \begin{itemize}
    \item
      \(e\) is an event in which \vAgent{} is \emph{\ptCV{0}} \(\pv{\phi}{v}\) from \(\Phi\) by type \(T\).
    \end{itemize}

    \emph{If and only if}:

    \begin{itemize}
    \item
      For every \tI{} \(\pvp{\psi}{v'}{\Psi}\) of \(T\).
      \begin{itemize}
      \item
        There is some action \(a\) available to \vAgent{} such that:
        \begin{itemize}
        \item
          \vAgent{} is concluding \(\pv{\psi}{v'}\) from \(\Psi\) when \vAgent{} does \(a\).
        \end{itemize}
      \end{itemize}
    \end{itemize}
    \vspace{-\baselineskip}
  \end{definition}

  \noindent%
  An agent \ptCV{0} \(\pv{\phi}{v}\) from \(\Phi\) by type \(T\) is a strong condition.
  In short, it must be the case that for every \tI{} of the type, there is a \pevent{} in which the agent is concluding the \tI{}.
\end{note}

\begin{note}
  To illustrate, consider the agent of \autoref{illu:tR:powers} when concluding \pv{\propI{a is 6}}{\valI{True}} from \pv{\propI{\(2^{100} \cdot 3^{100} = a^{100}\)}}{\valI{True}}.

  With respect to the type `multiplication by powers', it is plausible the agent is not \ptCV{}.
  For, it is plausible there is no \pevent{} in which the agent is concluding, say, \pv{\propI{n is 5}}{\valI{True}} from \pv{\propI{\(3^{10} \cdot 3^{n} = 14,348,907\)}}{\valI{True}}.%
  \footnote{
    Even if the agent infers there is some natural number \(m\) such that \(3^{m} = 14,348,907\), the agent will plausible choose to do something else other than calculate the number.
  }

  However, with respect to the narrow type `multiplication by powers with respect to problems \ref{mfp:a}--\ref{mfp:h}' it seems plausible the agent is \ptCV{}.
  For, if the agent were to begin working on a different problem, the agent would be concluding the appropriate answer.
\end{note}

\begin{note}
  So, an agent \ptCV{0} is a strong condition.
  Still, two important features:

  First, there is a close connexion between an agent \ptCV{} \(\pv{\phi}{v}\) from \(\Phi\) and \(\pv{\phi}{v}\) from \(\Phi\) being a \fc{}, as both ideas build on \pevent{1} in which an agent is concluding.

  Second, there is a straightforward relation between an agent \tCV{} and the agent \ptCV{0}:

  \begin{proposition}[\typeAdv{2} \ptypeAdj{}]%
    \label{prop:tCV-ptCV}%
    \cenLine{
      \begin{itemize*}[noitemsep, label=\(\circ\)]
      \item
        Agent: \vAgent{}
      \item
        \prop{2}: \(\phi\)
      \item
        \val{2}: \(v\)
      \item
        \pool{2}: \(\Phi\)
      \item
        \mbox{ }
      \end{itemize*}
    }

    \noindent%
    \cenLine{
      \begin{itemize*}[noitemsep, label=\(\circ\)]
      \item
        Event: \(e\)
      \item
        Type of reasoning: \(T\)
      \item
        \mbox{ }
      \end{itemize*}
    }

    \begin{itenum}
    \item[\emph{If}:]
      \(e\) is an event in which \vAgent{} is \tCV{} \(\pv{\phi}{v}\) from \(\Phi\) by \torNa{} \(T\).
    \item[\emph{Then}:]
      There exists some \torN{} \(T'\) such that:
      \begin{itemize}
      \item
        \(e\) is an event in which \vAgent{} is \ptCV{} \(\pv{\phi}{v}\) from \(\Phi\) by \torNa{} \(T'\).
      \end{itemize}
    \end{itenum}
    \vspace{-\baselineskip}
  \end{proposition}

  \begin{argument}{prop:tCV-ptCV}
    Suppose  \(e\) is an event in which \vAgent{} is \tCV{} \(\pv{\phi}{v}\) from \(\Phi\) by \torN{} \(T\).
    Then, by \autoref{idea:tC} there is some collections \({\cal E}\) of events and \({\cal X}\) of \prop{0}-\val{0}-\pool{0}~pairings such that:

    \begin{itemize}[noitemsep]
    \item
      For every event \(e'\), there is some \prop{0}-\val{} pair \(\pv{\psi}{v'}\) and \pool{0} \(\Psi\) in \(\mathcal{X}\) such that:
      \begin{itenum}[noitemsep]
      \item[\emph{If}:]
        \(e'\) is in the collection of events \({\cal E}\).
      \item[\emph{Then}:]
        \(e'\) is an event in which \vAgent{} is concluding \(\pv{\psi}{v'}\) from \(\Psi\).
      \end{itenum}
    \end{itemize}

    \noindent%
    Now, consider the collection of events \(\mathcal{E}'\) such that:
    \begin{itemize}[noitemsep]
    \item
      For some \(\pvp{\psi}{v'}{\Psi}\) in \(\mathcal{X}\):
      \begin{itemize}[noitemsep]
      \item
        \(e'\) is an event in which \vAgent{} is concluding \(\pv{\psi}{v'}\) from \(\Psi\) after doing some action \(a\) available to \vAgent{}.
      \end{itemize}
    \end{itemize}

    \noindent%
    Finally, consider the type \(T'\) given by pairings \(\pvp{\psi}{v'}{\Psi}\) such that:
    \begin{itemize}[noitemsep]
    \item
      For some event \(e''\) in \(\mathcal{E}\):
      \begin{itemize}[noitemsep]
      \item
        \(e''\) is an event in which \vAgent{} is concluding \(\pv{\psi}{v'}\) from \(\Psi\) after doing some action \(a'\) available to \vAgent{}.
      \end{itemize}
    \end{itemize}

    \noindent%
    It is immediate by construction that:
    \begin{itemize}[noitemsep]
    \item
      For every \tI{} \(\pvp{\psi}{v'}{\Psi}\) of \(T'\).
      \begin{itemize}[noitemsep]
      \item
        There is some action \(a\) available to \vAgent{} such that:
        \begin{itemize}
        \item
          \vAgent{} is concluding \(\pv{\psi}{v'}\) from \(\Psi\) when \vAgent{} does \(a\).
        \end{itemize}
      \end{itemize}
    \end{itemize}
    \vspace{-1.5\baselineskip}
  \end{argument}
\end{note}

\begin{note}
  Observe, \rotoc{} only states what must be the case if the agent is \tCV{}.
  \autoref{def:rotoc} does not provide sufficient resources to infer an agent is \tCV{} by some \torN{} when an agent is \ptCV{} by some type.%
  \footnote{
    Indeed, not even when the relevant types are identical.
  }
\end{note}


\subsection{\rotoc{2}}
\label{sec:rotoc}

\begin{note}
  The final definition :
\end{note}

\begin{note}
  \begin{definition}[A \rotoc{}]
    \label{def:rotoc}
    \cenLine{
      \begin{itemize*}[noitemsep, label=\(\circ\)]
      \item
        Agent: \vAgent{}
      \item
        Proposition: \(\phi\)
      \item
        Value: \(v\)
      \item
        \pool{2}: \(\Phi\)
      \item
        \mbox{ }
      \end{itemize*}
    }\newline
    \cenLine{
      \begin{itemize*}[noitemsep, label=\(\circ\)]
      \item
        Event: \(e\)
      \item
        Type of reasoning: \(T\)
      \item
        \mbox{ }
      \end{itemize*}
    }

    \begin{itemize}
    \item
      \(T'\) is a \emph{\tRep{}} of \vAgent{} \tCV{} \(\pv{\phi}{v}\) from \(\Phi\) by type \(T\) in \(e\).
    \end{itemize}

    \emph{If and only if:}

    \begin{itemize}
    \item
      \begin{itenum}
      \item[\emph{If}:]
        \(e\) is an event in which \vAgent{} is \tCV{0} \(\pv{\phi}{v}\) from \(\Phi\)~by~type~\(T\).
      \item[\emph{Then}:]
        \(e\) is an event in which \vAgent{} is \ptCV{0} \(\pv{\phi}{v}\) from \(\Phi\)~by~type~\(T'\).
      \end{itenum}
    \end{itemize}
    \vspace{-.5\baselineskip}
  \end{definition}

  \noindent%
  From \autoref{prop:tCV-ptCV}, a \tRep{} always exists.
  The role of \autoref{def:rotoc} is to easily talk about any if-then connexion between an agent \tCV{} \(\pv{\phi}{v}\) from \(\Phi\) by type \(T\) and \ptCV{} \(\pv{\phi}{v}\) from \(\Phi\) by type \(T'\).
\end{note}

\begin{note}
  A \rotoc{} links an agent \ptCV{} to an agent \tCV{}.

  With \ptCV{}, observed parallels to \autoref{idea:tC}.
  In particular, events such that agent is concluding.
  Intuitively, \rotoc{} captures this.

  The function of \rotoc{} is to capture these events.
\end{note}

\begin{note}
  Simple proposition:

  \begin{proposition}
    \label{prop:tRepptCVtCV}
    \cenLine{
      \begin{itemize*}[noitemsep, label=\(\circ\)]
      \item
        Agent: \vAgent{}
      \item
        \prop{3}: \(\phi\), \(\psi\)
      \item
        \val{3}: \(v\), \(v'\)
      \item
        \pool{3}: \(\Phi\), \(\Psi\)
      \item
        \mbox{ }
      \end{itemize*}
    }

    \noindent%
    \cenLine{
      \begin{itemize*}[noitemsep, label=\(\circ\)]
      \item
        Event: \(e\)
      \item
        Types of reasoning: \(T\), \(T'\)
      \item
        \mbox{ }
      \end{itemize*}
    }
    \begin{itenum}
    \item[\emph{If}:]
      \(T'\) is a \tRep{} of \vAgent{} \tCV{} \(\pv{\phi}{v}\) from \(\Phi\) by type \(T\) in \(e\).
    \item[\emph{And}:]
      \(e\) is an event in which \vAgent{} is \tCV{0} \(\pv{\phi}{v}\) from \(\Phi\) by~type~\(T\).
    \item[\emph{Then}:]
      \(e\) is an event in which \vAgent{} is \ptCV{0} \(\pv{\phi}{v}\) from \(\Phi\) by~type~\(T'\).
    \end{itenum}
    \vspace{-.5\baselineskip}
  \end{proposition}

  \noindent%
  Note, if both antecedents of \autoref{prop:tRepptCVtCV} hold, then this entails the relevant \tRep{} is non-empty.
\end{note}

\begin{note}
  In the case of selection tasks, close to the idea of a \tRep{}.
  However, slightly different.
\end{note}

\begin{note}
  \begin{illustration}[ジョジョリオン]%
    \nocite{huangmufeiluyan:2011aa}%
    Book.
    List of chapters.
    Is the agent reading the chapter titles?

    \begin{center}
      \bgroup
      \def\arraystretch{1.125}
      \begin{tabular}{R{.45\textwidth} L{.45\textwidth}}
        \multicolumn{2}{c}{Translations and chapter titles in `reading'} \\
        \hline\hline
        Translation & Title \\
        \hline
        Soft and wet & ソフト&ウェット \\
        \hdashline
        Safety above everything else & \multirow{3}*{無事が何より} \\
        Safety first & \\
        Gotta be safe & \\
        \hdashline
        Every day is summer vacation & 毎日が夏休み \\
        \hdashline
        A hair clip from ??? period & 清の時代の髪留め \\
      \end{tabular}
      \egroup
    \end{center}

    First and third examples are straightforward.
    Difference is katakana and common kanji with a little grammar.
    Second, overly literal and slightly different translations due to lack of information about context.
    Fourth, do not need complete translation.

    \noindent%
    However, agent may fail to translate certain chapter titles such as:

    \begin{center}
      \bgroup
      \def\arraystretch{1.125}
      \begin{tabular}{R{.45\textwidth} L{.45\textwidth}}
        \multicolumn{2}{c}{Translations and chapter titles not in `reading'} \\
        \hline\hline
        Translation & Title \\
        \hline
        Software and wet & ソフト&ウェット \\
        \hdashline
        The Qing Dynasty Hair Clip & 清の時代の髪留め \\
        \hdashline
        ??? & 母と子
      \end{tabular}
      \egroup
    \end{center}

    First, translation ruled out by context, though possible.
    Second and third, require knowledge that may be expected for fluency, but not reading.
    Second, did not require complete translation.
    Third, failure to translate \textquote{Mother and child}.
  \end{illustration}
\end{note}

% \begin{note}
%   \begin{observation}[Trivial \tRep{1}]%
%     \label{obs:tR:trivialRep}%
%     It may be the case that the only \rotoc{} is type which contains \(\pv{\phi}{v}\), etc.
%   \end{observation}

%   \begin{motivation}{obs:tR:trivialRep}
%     Exam.
%     Consider any other question, then immediately return to present question.%
%     \footnote{
%       To go further, plausibly get to counterfactuals.

%       For example, rather than action to the agent, go via prompting.
%       I give question, you get answer.
%     }
%     Here, \tRep{} is just the question.
%   \end{motivation}
% \end{note}


\subsection{\tCV{3} linked to \fc{1}}
\label{cha:typical:tC-fc}

\begin{note}
  We now link an agent \tCV{} \(\pv{\phi}{v}\) from \(\Phi\) to \fc{1}.

  \begin{proposition}[\tCV{2} and \fc{1}]
    \label{prop:tC-and-fc}
    \cenLine{
      \begin{itemize*}[noitemsep, label=\(\circ\)]
      \item
        Agent: \vAgent{}
      \item
        \prop{3}: \(\phi\), \(\psi\)
      \item
        \val{3}: \(v\), \(v'\)
      \item
        \pool{3}: \(\Phi\), \(\Psi\)
      \item
        \mbox{ }
      \end{itemize*}
    }

    \noindent%
    \cenLine{
      \begin{itemize*}[noitemsep, label=\(\circ\)]
      \item
        Event: \(e\)
      \item
        Type of reasoning: \(T\)
      \item
        \mbox{ }
      \end{itemize*}
    }

    \begin{itenum}
    \item[\emph{If}:]
      Clauses~\ref{prop:tC-and-fc:A:tC},~\ref{prop:tC-and-fc:A:rep},~\ref{prop:tC-and-fc:A:tI},~and~\ref{prop:tC-and-fc:A:novel} jointly hold:
      \begin{enumerate}[label=\arabic*., ref=(\arabic*)]
      \item
        \label{prop:tC-and-fc:A:tC}
        \(e\) is an event in which \vAgent{} is \tCV{0} \(\pv{\phi}{v}\) from \(\Phi\) by~type~\(T\).
      \item
        \label{prop:tC-and-fc:A:rep}
        \(T'\) is a \tRep{} of \vAgent{} \tCV{} \(\pv{\phi}{v}\) from \(\Phi\) by type \(T\) in \(e\).
      \item
        \label{prop:tC-and-fc:A:tI}
        \(\pvp{\psi}{v'}{\Psi}\) is a \tI{} of \(T'\)
      \item
        \label{prop:tC-and-fc:A:novel}
        The following conditional is true:
        \begin{itemize}
        \item[\emph{If}:]
          There is some available action \(a\) such that \vAgent{} is concluding \(\pv{\psi}{v'}\) from \(\Psi\), when \vAgent{} does \(a\).
        \item[\emph{Then}:]
          There is some available action \(a'\) such that \vAgent{} is concluding \(\pv{\psi}{v'}\) from \(\Psi\) without use of any novel information obtained by doing \(a'\), when \vAgent{} does \(a'\).
        \end{itemize}
      \end{enumerate}
    \item[\emph{Then}:]
      \(\pv{\psi}{v'}\) is a \fc{} from \(\Psi\).
    \end{itenum}
    \vspace{-\baselineskip}
  \end{proposition}

  \noindent%
  Clause~\ref{prop:tC-and-fc:A:tC}, \tCV{}.

  Clause~\ref{prop:tC-and-fc:A:rep} \ptCV{0}.
  So, must be the case that action, agent is concluding.

  Clause~\ref{prop:tC-and-fc:A:tI}, \prop{0}-\val{0}-\pool{0} pair.

  Important thing in Clause \autoref{prop:tC-and-fc:A:novel}.
  This says that if some action, then without novel information.
  This is what is required for \fc{}.

  \autoref{prop:tC-and-fc} follows by applying the relevant definitions:

  \begin{argument}{prop:tC-and-fc}%
    Assume Clauses \ref{prop:tC-and-fc:A:tC}, \ref{prop:tC-and-fc:A:rep}, \ref{prop:tC-and-fc:A:tI}, and~\ref{prop:tC-and-fc:A:novel} hold.

    \noindent%
    By Clause~\ref{prop:tC-and-fc:A:rep} and \autoref{def:rotoc}:

    \begin{itenum}[noitemsep]
    \item[\emph{If}:]
      \(e\) is an event in which \vAgent{} is \tCV{0} \(\pv{\phi}{v}\) from \(\Phi\)~by~type~\(T\).
    \item[\emph{Then}:]
      \(e\) is an event in which \vAgent{} is \ptCV{0} \(\pv{\phi}{v}\) from \(\Phi\)~by~type~\(T'\).
    \end{itenum}

    \noindent%
    And, by Clause~\ref{prop:tC-and-fc:A:tC} the antecedent of this conditional is true.
    So:

    \begin{itemize}[noitemsep]
    \item
      \(e\) is an event in which \vAgent{} is \ptCV{0} \(\pv{\phi}{v}\) from \(\Phi\)~by~type~\(T'\).
    \end{itemize}

    \noindent%
    Hence, by \autoref{def:ptC}:

    \begin{itemize}[noitemsep]
    \item
      For every \tI{} \(\pvp{\chi}{v''}{X}\) of \(T'\).
      \begin{itemize}[noitemsep]
      \item
        There is some action \(a\) available to \vAgent{} such that:
        \begin{itemize}[noitemsep]
        \item
          \vAgent{} is concluding \(\pv{\psi}{v'}\) from \(\Psi\) when \vAgent{} does \(a\).
        \end{itemize}
      \end{itemize}
    \end{itemize}

    \noindent%
    By Clause~\ref{prop:tC-and-fc:A:tI}, \(\pvp{\psi}{v'}{\Psi}\) is a \tI{} of \(T'\).
    Hence:

    \begin{itemize}[noitemsep]
    \item
      There is some action \(a\) available to \vAgent{} such that:
      \begin{itemize}[noitemsep]
      \item
        \vAgent{} is concluding \(\pv{\psi}{v'}\) from \(\Psi\) when \vAgent{} does \(a\).
      \end{itemize}
    \end{itemize}

    \noindent%
    Now, we have some action \(a\) such that:%

    \begin{itemize}[noitemsep]
    \item
      \(a\) is available to \vAgent{}.
    \item
      \vAgent{} is concluding \(\pv{\psi}{v'}\) from \(\Psi\), when \vAgent{} does \(a\).
    \end{itemize}

    \noindent%
    And, by Clause~\ref{prop:tC-and-fc:A:novel}:

    \begin{itemize}[noitemsep]
    \item
      \vAgent{} is concluding \(\pv{\psi}{v'}\) from \(\Psi\), when \vAgent{} does \(a\) without use of any novel information obtained by doing \(a\).
    \end{itemize}

    \noindent%
    So, by \autoref{def:fc} (\autopageref{def:fc}), \(\pv{\psi}{v'}\) is a \fc{0} from \(\Psi\), for \vAgent{}.
  \end{argument}
\end{note}



\section*{Summary}

\begin{note}
  \tCN{2}.

  Extension account of \torN{}.
  Due to abstracting over theories.

  Then, necessary condition on \tC{}.
\end{note}

\begin{note}
  Only motivated \tC{} by intuition.
  Have not argued that this intuition is correct.
  \rotoc{2}.
  Key piece, and intuitive.
\end{note}


% \section[\citeauthor{Carroll:1895uj}]{\citeauthor{Carroll:1895uj}\hfill(Optional)}

% \nocite{Black:1951aa}

% \begin{note}
%   The point here is that with Carroll, generality that goes beyond any single instance.
%   Must apply to all instances, to be valid.
%   But, cannot hope to cover all instances in a single move.
% \end{note}

% \begin{note}
%   A difficulty found on a reading of \citeauthor{Carroll:1895uj}'s \citetitle{Carroll:1895uj}.
% \end{note}

% \begin{note}
%   \begin{quote}
%     ``Plenty of blank leaves, I see!'' the Tortoise cheerily remarked.
%     ``We shall need them \emph{all}!''
%     (Achilles shuddered.)
%     ``Now write as I dictate:---

%     \begin{enumerate}[label=(\emph{\Alph*}), ref=\emph{\Alph*}]
%     \item
%       \label{AatT:a}
%       Things that are equal to the same are equal to each other.
%     \item
%       \label{AatT:b}
%       The two sides of this Triangle are things that are equal to the same.
%     \item
%       \label{AatT:c}
%       If~\ref{AatT:a} and~\ref{AatT:b} are true,~\ref{AatT:z} must be true.
%       \setcounter{enumi}{25}
%     \item
%       \label{AatT:z}
%       The two sides of this Triangle are equal to each other.''
%     \end{enumerate}

%     ``You should call it~\ref{AatT:d}, not~\ref{AatT:z},'' said Achilles.
%     ``It comes \emph{next} to the other three.
%     If you accept~\ref{AatT:a} and~\ref{AatT:b} and~\ref{AatT:c}, you \emph{must} accept~\ref{AatT:z}.''

%     ``And why \emph{must} I?''

%     ``Because it follows \emph{logically} from them.
%     If~\ref{AatT:a} and~\ref{AatT:b} and~\ref{AatT:c} are true,~\ref{AatT:z} \emph{must} be true.
%     You don't dispute \emph{that}, I imagine?''

%     ``If~\ref{AatT:a} and~\ref{AatT:b} and~\ref{AatT:c} are true,~\ref{AatT:z} \emph{must} be true,'' the Tortoise thoughtfully repeated.
%     ``That's \emph{another} Hypothetical, isn't it?
%     And, if I failed to see its truth, I might accept~\ref{AatT:a} and~\ref{AatT:b} and~\ref{AatT:c}, and \emph{still} not accept~\ref{AatT:z}, mightn't I ?''

%     \mbox{}\hfill\(\vdots\)\hfill\mbox{}

%     ``Then Logic would take you by the throat, and force you to do it!''
%     Achilles triumphantly replied.
%     ``Logic would tell you 'You ca'n't help yourself.''%
%     \mbox{ }\hfill\mbox{(\citeyear[279--280]{Carroll:1895uj})}
%   \end{quote}

%   The Tortoise has written down three premises,~\ref{AatT:a},~\ref{AatT:b}, and~\ref{AatT:c}.
%   Achilles holds that~\ref{AatT:z} follows from~\ref{AatT:a},~\ref{AatT:b}, and~\ref{AatT:c}.
%   The Tortoise observes they have the possibility of refraining to accept~\ref{AatT:z} follows from~\ref{AatT:a},~\ref{AatT:b}, and~\ref{AatT:c}.
%   And (initially), the Tortoise does not accept~\ref{AatT:z} follows from~\ref{AatT:a},~\ref{AatT:b}, and~\ref{AatT:c}.
%   Achilles requests the Tortoise accepts that~\ref{AatT:z} follows from~\ref{AatT:a},~\ref{AatT:b}, and~\ref{AatT:c}, and the Tortoise complies.
%   Specifically, the Tortoise grants:

%   \begin{quote}
%     \begin{enumerate}[label=(\emph{\Alph*}), ref=\emph{\Alph*}]
%       \setcounter{enumi}{3}
%     \item
%       \label{AatT:d}
%       If~\ref{AatT:a} and~\ref{AatT:b} and~\ref{AatT:c} are true,~\ref{AatT:z} must be true.%
%       \mbox{ }\hfill\mbox{(\citeyear[279]{Carroll:1895uj})}
%     \end{enumerate}
%   \end{quote}

%   But, does not accept~\ref{AatT:z} follows from~\ref{AatT:a},~\ref{AatT:b},~\ref{AatT:c}, and~\ref{AatT:d}.
% \end{note}

% \begin{note}
%   Modus ponens.

%   \begin{quote}
%     From \(\phi\) and \emph{if} \(\phi\) then \(\psi\), infer \(\psi\).
%   \end{quote}

%   Modus ponens is general.
%   For \emph{any} \(\phi\), \(\psi\).

%   Now, there is a difference between \emph{modus ponens} and conditional.

%   However, take any instance.
%   Then, if \(P\), \(P \rightarrow Q\), \(Q\) must be true.
%   But, then this means that the conditional is true.

%   Consequence of the deduction theorem.

%   Likewise, deduction theorem goes the other way.

%   However, going from \(P\), \(P \rightarrow Q\) to \(Q\) need not be an instance of \emph{modus ponens}.
% \end{note}

% \begin{note}
%   Well, this is a headache.
%   \citeauthor{Carroll:1895uj} is talking about a specific A, B, and Z.
%   There is no clear generality.
% \end{note}

% \begin{note}
%   So, consider at issue is modus ponens.
%   For any specific instance accept, there is a further instance.
%   For, \(A, (A \rightarrow B) \vDash B\).
%   Then, \(\vDash (A \land (A \rightarrow B) \rightarrow B)\).
%   However, now, \(A \land (A \rightarrow B), (A \land (A \rightarrow B) \rightarrow B) \vDash B\).
%   And, so on.

%   The general pattern, get conditional, but then this gives a new instance of modus ponens, which must be true in order for modus ponens to be valid rule of inference.

%   \citeauthor{Carroll:1895uj}, by contrast, starts with \(A \vDash B\).
%   This is different.
%   However, rather than focusing on a single rule of inference, the puzzle turns on what validity amounts to.

%   Validity is a general thing, with specific instances.
%   However, grant any particular instance of validity without employing validity in general.
% \end{note}

% \begin{note}
%   \begin{quote}
%     My paradox \dots turns on the fact that, in a Hypothetical, the \emph{truth} of the Protasis, the \emph{truth} of the Apodosis, and the \emph{validity of the sequence}, are 3 distinct Propositions.

%     \mbox{}\hfill\(\vdots\)\hfill\mbox{}

%     Suppose I say ``I grant~\ref{AatT:a} and~\ref{AatT:b} and~\ref{AatT:c}, but I do \emph{not} grant that I am thereby \emph{obliged} to grant~\ref{AatT:z}.''
%     Surely, my granting~\ref{AatT:z} must \emph{wait} until I have been made to see the validity of this sequence: i.e.\ in order to grant~\ref{AatT:z}, I must grant~\ref{AatT:a},~\ref{AatT:b},~\ref{AatT:c}, and~\ref{AatT:d}! And so on.%
%     \mbox{ }\hfill\mbox{(\citeyear[472]{Carroll:1977wl})}
%   \end{quote}

%   My interpretation of the point \citeauthor{Carroll:1895uj} makes in this passage is that the truth of A B and the truth of C is distinct from the validity of A B C.
%   Granting is substantial, not merely moving.
%   But, in order to grant, this means granting all other cases.

%   So, the paradox is that, on the one hand, don't need validity for any specific true things.
%   But, on the other hand, only of interest if via validity.

%   The Tortoise is slowly working through each instance, but this has no hope of getting the Tortoise to general validity.
%   So, how does the Tortoise ever make it there?
% \end{note}

% \begin{note}
%   This point differs from received interpretation.

%   \citeauthor{Wieland:2013vf} (\citeyear{Wieland:2013vf}) characterises the general understanding of \textcite{Carroll:1895uj} in terms of two lessons:
%   \begin{quote}
%     [T]he negative lesson is that if you add ever more premises to an argument \dots, then you will never demonstrate that its conclusion follows logically.\newline
%     \mbox{ }\hfill\mbox{(\citeyear[984]{Wieland:2013vf})}
%   \end{quote}

%   Parallel, static answers, still option for concluding otherwise.

%   \begin{quote}
%     [T]he positive lesson is that rules of inference, rather than premises of the form `if premises such and such are true, then the conclusion is true', will do the job.\newline
%     \mbox{ }\hfill\mbox{(\citeyear[984]{Wieland:2013vf})}
%   \end{quote}

%   \begin{quote}
%     [\citeauthor{Carroll:1895uj}] simply lacked any distinct conception of a deduction as opposed to the assertion (``granting'' of) a hypothetical proposition.
%     \dots
%     Any attempt by Carroll to tackle the question of inference was bound to begin in confusion and end in constipation-all those premises piling up, but no motion.
%   \end{quote}
% \end{note}

% \paragraph{The Dichotomy}

% \begin{note}
%   Achilles and the Tortoise, Zeno's argument.

%   Surely, right?

%   Two ways to understand.
%   Does the Tortoise move at all, or does the Tortoise arrive at the end?
%   I mean, as formulated by Zeno, it's about catching up, no matter how much one moves.

%   It is different from Zeno's Dichotomy paradox.


%   If so, then we should expect the Tortoise to be making some movement.
%   Adding rules of inference is of no help, because the problem is not movement, it's about how to move so much in a single step.
% \end{note}

% \begin{note}
%   \color{red}
%   Something about logic forcing.
%   The Tortoise hasn't arrived.

%   Nothing hangs on validity.
%   Same issue with testimony.
%   `A'.
%   Why?
%   Testified A, so A.
%   Okay, but another instance of testimony.
%   Testified(Testified A, so A), so Testified A, so A.
% \end{note}

% \begin{note}
%   \begin{quote}
%     But if we who wish to represent his belief in Q as based on P are to write in our notebook everything his having that belief on that basis consists in then when we have written only P and Q we will not have written enough.
%     Someone can believe P and believe Q and still not believe Q on the basis of P whatever the relations between the propositions P and Q happen to be.
%     He might believe Q for some reason completely unconnected with P, or perhaps for no reason at all (if that is possible).%
%     \mbox{ }\hfill\mbox{(\citeyear[185]{Stroud:1979aa})}
%   \end{quote}
%   However, the moral drawn is narrow
%   \begin{quote}
%     The moral is that for every proposition or set of propositions the belief or acceptance of which is involved in someone's believing one proposition on the basis of another there must be something else, not simply a further proposition accepted, that is responsible for the one belief's being based on the other.%
%     \mbox{ }\hfill\mbox{(\citeyear[187]{Stroud:1979aa})}
%   \end{quote}

%   Even if we grant each individual is \ros{}, rather than an instance of the material conditional, \emph{logic} hasn't done anything.
% \end{note}

% \paragraph{General and specific: Contrast}

% \begin{note}
%   Use \citeauthor{Carroll:1895uj} to illustrate this point.

%   However, given the worry, various other things may be understood this way.

%   Hume, constant conjunction.
%   Part of the problem is identifying cause.
%   We get the famous line about observing.
%   However, Hume goes on.
%   It's not only no observation, but no generality.

%   Right, so more narrow than Hume.
%   Because, with Hume, at issue is whether we have grounds for this general thing.
%   With Carroll, it's whether we even really get to the general thing.
% \end{note}

% \begin{note}
%   \begin{quote}
%     Let me ask this: what has the expression of a rule—say a sign-post—got to do with my actions?
%     What sort of connexion is there here?%
%     ---%
%     Well, perhaps this one:
%     I have been trained to react to this sign in a particular way, and now I do so react to it.

%     But that is only to give a causal connexion; to tell how it has come about that we now go by the sign-post; not what this going-by-the sign really consists in.
%     On the contrary; I have further indicated that a person goes by a sign-post only in so far as there exists a regular use of sign-posts, a custom.%
%     \mbox{ }\hfill\mbox{(\citeyear[\S198]{Wittgenstein:1958aa})}
%   \end{quote}

%   Regular use of sign-posts, custom.

%   Ugh, this is ambiguous.
% \end{note}


% %
%   \(^{,}\)
%   \footnote{
%     \citeauthor{Hlobil:2014tq}'s ``Inferential Moorean Phenomenon'':
%   \begin{quote}
%     \begin{enumerate}
%     \item[(IMP)]
%       It is either impossible or seriously irrational to infer \emph{P} from \emph{Q} and to judge, at the same time, that the inference from \emph{Q} to \emph{P} is not a good inference.
%     \end{enumerate}
%     \dots
%     By the ``goodness'' of an inference I mean the feature that makes the relevant inference permissible. Thus, if the inference under consideration is an inductive inference, the relevant kind of goodness is not deductive validity.%
%     \mbox{ }\hfill\mbox{(\citeyear[\S1]{Hlobil:2014tq})}
%   \end{quote}
%   Though, this really isn't more basic given the interest in \tR{}.
%   For, the puzzle is what it is to `infer'.

%   Rationality isn't part of the picture.
%   And, this is a significant drawback of \citeauthor{Hlobil:2014tq}'s approach.
% }


% \subsection{Types and explanation}
% \label{cha:typical:sec:tor:bkgd}

% \begin{note}
%   There is a related, stronger claim, that generality derives from rule following.

%   For this, \citeauthor{Boghossian:2008vf}:

%   \begin{quote}
%     [O]ur internalization of general epistemic rules---like Modus Ponens and Induction---explain and rationalize why we form the beliefs that we form.
%     And that seems intuitively correct.

%     As in the case of our linguistic and conceptual abilities, our ability to form rational beliefs is \emph{productive}: on the basis of finite learning, we are able to form rational beliefs under a potential infinity of novel circumstances.
%     The only plausible explanation for this is that we have, somehow, internalized a rule that tells us, in some general way, what it would be rational to believe under varying epistemic circumstances.%
%     \mbox{ }\hfill\mbox{(\citeyear[483]{Boghossian:2008vf})}
%   \end{quote}

%   Strictly, \citeauthor{Boghossian:2008vf}, rules \textquote{represent our conception of how it would be most rational for a thinker to form beliefs under different epistemic circumstances} (\citeyear[473]{Boghossian:2008vf}).

%   The difference in approach is clearest with \citeauthor{Boghossian:2008vf}'s account of modus ponens:%
%   \footnote{
%     \citeauthor{Boghossian:2008vf} notes the rule is distinct from modus pones as found in textbooks.
%     Remarks: \textquote{It is actually quite mysterious what the logic textbook rule is supposed to be} (\citeyear[472,fn.2]{Boghossian:2008vf})
%     I don't think there is any mystery about the rule in most logic textbooks.
%     Instead, the mystery is the way in which logic relates to reasoning.
%     (Cf.~\cite{Harman:1986ux,MacFarlane:2004aa,Steinberger:2022aa}, etc.)
%     % Issue for the presentation.
%     % Literature is full of issues.
%     % The most well known, Gricean pragmatics.
%     % Though, also McGee, McFarlane, sweet conditonals, the miners paradox, etc.
%   }

%   \begin{quote}
%     (Modus Ponens):
%     If you are rationally permitted to believe both that \emph{p} and that `If \emph{p}, then \emph{q}', then, you are prima facie rationally permitted to believe that \emph{q}.%
%     \mbox{ }\hfill\mbox{(\citeyear[472]{Boghossian:2008vf})}
%   \end{quote}

%   Here, we have permissions.
%   What the agent is allowed to do.
%   However, this is distinct from what the agent does.
% \end{note}

% \begin{note}
%   \tR{} is distinct.
%   Whether came to \emph{q} from \emph{p} , if \emph{p} then \emph{q}.

%   Rationality is not part of our understanding.
%   Rather, generality.%
%   \footnote{
%     Observe, ~\cite{Kolodny:2005aa} is of no interest here.
%     Why be rational is distinct from whether there is some generality.
%   }
% \end{note}

% \begin{note}
%   Likewise, means-end reasoning is distinct from \citeauthor{Broome:2013aa}'s

%   \begin{quote}
%     \emph{End to Means Transmission}.
%     ((\emph{S} requires of \emph{N} that \emph{p}) \& necessarily \newline (\emph{p} \(\supset\) \emph{q}) \& \emph{q} is a means to \emph{p}) \(\supset\) (\emph{S} requires of \emph{N} that \emph{q}).%
%     \mbox{ }\hfill\mbox{(\citeyear[126]{Broome:2013aa})}
%   \end{quote}

%   \emph{S} is some source, such as morality.
%   \emph{N} is a person. (\citeyear[117]{Broome:2013aa})

%   Instead, the significantly weaker idea that the agent has reasoned from some end to a means to that end.
% \end{note}

% \begin{note}
%   On my understanding, this is, in part, the role of \citeauthor{Boghossian:2014aa}'s Taking Condition.

%   Way in which \dots

%   Indeed, \citeauthor{Boghossian:2014aa} highlights how condition allow to draw distinction between deductive and inductive.
%   With taking, get generality.

%   Indeed, \textcite{Boghossian:2014aa} is structured so that Taking is a generalisation of rule.
% \end{note}

% \begin{note}
%   However, \tor{} does not need to amount to a rule.
%   Rather, \tR{} only requires the rough phenomenon that \citeauthor{Boghossian:2008vf} argues rule following is the only plausible explanation of.%
%   \footnote{
%     Our interest with \tor{1} is independent of the worries about rule following raised by~\textcite{Kripke:1982aa}, to the extent that the worries raised by~\citeauthor{Kripke:1982aa} concern \emph{which} rule an agent is following, rather than \emph{whether} the agent is following a rule.
%     At interest is not whether the \tor{} corresponds to plus or quus, but whether the agent's reasoning is of some type.
%   }
% \end{note}

% \begin{note}
%   Same for modus ponens.

%   \citeauthor{Davies:2004aa} discussing~\textcite{Wright:2004aa} with respect to~\citeauthor{Moore:1959aa}'s proof of an external world (\citeyear{Moore:1959aa}):

%   \begin{quote}
%     Moore's argument can be set out as follows:
%     \begin{quote}
%       \begin{enumerate}[label=MOORE (\Roman*), ref=MOORE (\Roman*)]
%       \item
%         \label{MoorePoEW:1}
%         I am having an experience as of one hand [here] and another [here].
%       \item
%         \label{MoorePoEW:2}
%         I have hands.

%         If I have hands then an external world exists.
%       \end{enumerate}

%       Therefore:

%       \begin{enumerate}[label=MOORE (\Roman*), ref=MOORE (\Roman*), resume]
%       \item
%         \label{MoorePoEW:3}
%         An external world exists.
%       \end{enumerate}
%     \end{quote}

%     [\dots] the key question at this point in Wright's account is whether the support for~\ref{MoorePoEW:2} is transmitted to~\ref{MoorePoEW:3} across the modus ponens inference in which the conditional premise is supported by an elementary piece of philosophical theorising.\newline
%     \mbox{ }\hfill\mbox{(\citeyear[215]{Davies:2004aa})}
%   \end{quote}
% \end{note}

% \paragraph*{Preferences}

% \begin{note}
%   Type of reasoning isn't basic EU, etc.
% \end{note}

% \begin{note}
%   The situations, as presented by \citeauthor{Allais:1979aa}:%
%   \footnote{
%     The units in \citeauthor{Allais:1979aa}'s situations are French francs at 1952 prices (\citeauthor[138, fn.94]{Allais:1979aa}).\newline
%     100 French francs in 1952 is worth about the same as 50 US Dollars in 2022.
%   }
%   \begin{quote}
%     \begin{enumerate}[label=(\arabic*), ref=(\arabic*)]
%     \item
%       \emph{Do you prefer Situation A to Situation B}?

%       Situation A:
%         \begin{enumerate}[label=--]
%         \item
%           \emph{certainty} of receiving 100 million
%         \end{enumerate}
%         Situation B:
%         \begin{enumerate}[label=--]
%         \item
%           \emph{a} 10\% \emph{chance} of winning 500 million,
%         \item
%           \emph{an} 89\% \emph{chance} of winning 100 million,
%         \item
%           \emph{a} 1\% \emph{chance} of winning nothing.
%         \end{enumerate}
%       \item
%       \emph{Do you prefer Situation C to Situation D}?

%       Situation C:
%         \begin{enumerate}[label=--]
%         \item
%           \emph{a} 11\% \emph{chance} of winning 100 million,
%         \item
%           \emph{an} 89\% \emph{chance} of winning nothing.
%         \end{enumerate}
%         Situation D:
%         \begin{enumerate}[label=--]
%         \item
%           \emph{a} 10\% \emph{chance} of winning 500 million,
%         \item
%           \emph{a} 90\% \emph{chance} of winning nothing.%
%           \mbox{ }\hfill\mbox{(\citeyear[89]{Allais:1979aa})}
%         \end{enumerate}
%     \end{enumerate}
%   \end{quote}

%   \begin{quote}
%     The preference \(\text{A} > \text{B}\) should entail \(\text{C} > \text{D}\).%
%     \footnote{
%       First, write out expected utility for each situation.

%       \smallskip
%       \mbox{ }\hfill%
%       \EU{Sit.\ A} = \(\Util{100\text{m}}\)%
%       \hfill%
%       \EU{Sit.\ B} = \(.10 \cdot \Util{500\text{m}} + .89 \cdot \Util{100\text{m}} + .1 \cdot \Util{0\text{m}}\)%
%       \hfill\mbox{ }

%       \mbox{ }\hfill%
%       \EU{Sit.\ C} = \(.11 \cdot \Util{100\text{m}} + .89 \cdot \Util{0\text{m}}\)%
%       \hfill%
%       \EU{Sit.\ D} = \(.10 \cdot \Util{500\text{m}} + .90 \cdot \Util{0\text{m}}\)%
%       \hfill\mbox{ }
%       \smallskip

%       Suppose Situation A is preferred to Situation B.
%       Hence, \(\EU{Sit.\ A} > \EU{Sit.\ B}\).
%       Expanding, we obtain:
%       %
%       \[
%         \Util{100\text{m}} > .10 \cdot \Util{500\text{m}} + .89 \cdot \Util{100\text{m}} + .1 \cdot \Util{0\text{m}}
%       \]

%       Now, consider subtracting \(.89 \cdot \Util{100\text{m}}\) from each side of the inequality and adding \(.89 \cdot \Util{0\text{m}}\):
%       %
%       \begin{align*}
%         \Util{100\text{m}} - .89 \cdot \Util{100\text{m}} &> .10 \cdot \Util{500\text{m}} + .01 \cdot \Util{0\text{m}} \\
%          .11 \cdot \Util{100\text{m}} + .89 \cdot \Util{0\text{m}} &> .10 \cdot \Util{500\text{m}} + .90 \cdot \Util{0\text{m}}
%       \end{align*}

%       The left and right side of the inequality are \EU{Sit.\ C} and \EU{Sit.\ D}, respectively.
%       Therefore, it should be the case that Situation C is preferred to Situation D.
%     }
%   \end{quote}
%   However, this is not the case.
%   \citeauthor{Allais:1979aa} highlights pattern to the contrary:
%   \textquote{\emph{[T]he pattern for most highly prudent persons [\dots] who are considered generally as rational, is the pairing \(\text{A} > \text{B}\) and \(\text{C} < \text{D}\).}}
%   (\citeyear[89]{Allais:1979aa})

%   So, we have a something lawlike.%

%   Two things here.
%   \begin{itemize}
%   \item
%     Entailment between preferences.
%   \item
%     Circumstance in which conflicting preferences.
%   \end{itemize}

%   Both work, but the latter is of interest.%
%   \footnote{
%     \color{red}
%     The former \dots
%   }

%   For, if axioms, then preferences.
% \end{note}


\begin{note}
  Or, whether properly based.%
  \footnote{
    \citeauthor{Schaffer:2010vq}'s (\citeyear{Schaffer:2010vq}) Debasing demon.

    The debasing demon \textquote{throws her victims into the belief state on an improper basis, while leaving them with the impression as if they had proceeded properly}. (\citeyear[231]{Schaffer:2010vq})

    (However, see \textcite{Bondy:2018tk} for ways in which the \citeauthor{Schaffer:2010vq}'s demon fails.)
  }
\end{note}

% \begin{note}
%   To \illu{1} with the basic case of modus ponens:

%   \begin{center}
%     \begin{tabular}{R{.45\textwidth} L{.45\textwidth}}
%       \multicolumn{2}{c}{\prop{2}-\val{}-\pool{} pairings in type `by modus ponens'} \\
%       \hline\hline
%       Proposition-value pair & \pool{2} \\
%       \hline
%       \pv{Q}{\valI{True}} & \pv{P}{\valI{True}}, \pv{\propI{If }P\propI{ then }Q}{\valI{True}}, \dots \\
%     \end{tabular}
%   \end{center}

%   \noindent%
%   Where, \(P\), \(Q\) are arbitrary \prop{1}.
%   Likewise, in the following \(R\) is an arbitrary \prop{0}.

%   \begin{center}
%     \begin{tabular}{R{.45\textwidth} L{.45\textwidth}}
%       \multicolumn{2}{c}{\prop{2}-\val{}-\pool{} pairings not in type `by modus ponens'} \\
%       \hline\hline
%       Proposition-value pair & \pool{2} \\
%       \hline
%       \pv{R}{\valI{True}} & \pv{P}{\valI{True}}, \pv{\propI{If }P\propI{ or }Q\propI{ then }R}{\valI{True}}, \dots \\
%       \pv{Q}{\valI{False}} & \pv{P}{\valI{True}}, \pv{\propI{If }P\propI{ then }Q}{\valI{True}}, \dots \\
%     \end{tabular}
%   \end{center}
% \end{note}


%%% Local Variables:
%%% mode: latex
%%% TeX-master: "master"
%%% TeX-engine: luatex
%%% End:
