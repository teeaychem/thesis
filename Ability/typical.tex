\chapter{\tC{2}}
\label{cha:typical}

\nocite{Wilson:1994aa}
\nocite{Goodman:1983aa}

\begin{note}
  This chapter introduces and develops the idea of an agent concluding some \prop{0}-\val{0} pair from some \pool{} by some \emph{type} of reasoning --- colloquially, we say the agent is agent \emph{\typeAdj{}} concluding the \prop{0}-\val{0} pair from some \pool{}.

  Intuitively, an agent is \typeAdj{} concluding just in case there is some generality to the agent's reasoning which concluding.
  For example, the agent is reasoning by modus ponens, arithmetic, or the categorical imperative, etc.
\end{note}


\begin{note}
  The role of an agent \tCV{} in the overall argument to provide sufficient conditions for a \requ{}.
  The first section of this chapter introduces the idea of an agent \tCV{}, and the second section of this chapter provides a handful of definitions in order to link an agent \tCN{} to \fc{1}.
  In turn, \autoref{sec:typicalRequs} builds on the second section to provide sufficient conditions for a \requ{}.
\end{note}


\begin{note}
  At the close of the chapter all the idea relating to counterexamples to \issueConstraint{} will have been stated.
\end{note}


\section{\tC{2}}
\label{cha:typical:int}

\begin{note}
  Our interest is characterising an event in which an agent is concluding.
  By~\autoref{assu:ConRea} (\autopageref{assu:ConRea}), whenever an concludes, and agent reasons.
  Hence, whenever an agent is concluding, and agent is reasoning.
  And, some instances of reasoning are, intuitively, of a type:
  There are sufficiently similar characteristics between two or more events in which an agent reasons for the agent to be reasoning by `type of reasoning \(T\)', where `type of reasoning \(T\)' may be `modus ponens', `means-end reasoning', `arithmetic', `consequentialism', and so on.
\end{note}


\begin{note}
  I expect this observation is intuitive, and specific examples will follow.
\end{note}


\begin{note}
  Our interest is stating a necessary condition for whether or not an event in which an agent is concluding is an event in which an agent's reasoning is of some type.

  Provide idea.
  \illu{3} of idea.
\end{note}


\subsection{\torN{3}}
\label{cha:typical:tCDef:ToRdef}

\begin{note}
  \torN{3} are defined in terms of \prop{1}, \val{1}, and \pool{1}:

  \begin{definition}[A \torN{0}]
    \label{def:tor}
    \mbox{ }
    \vspace{-\baselineskip}
    \begin{itemize}
    \item
      \(T\) is a \torN{}.
    \end{itemize}

    \emph{If and only if}:

    \begin{itemize}
    \item
      \(T\) is a collection of \prop{0}-\val{0}-\pool{0} pairings.
    \end{itemize}
    \vspace{-\baselineskip}
  \end{definition}

  \noindent%
  Intuitively, a \torN{} (as defined) as the \emph{extension} of a \torN{}.
  The motivation for \autoref{def:tor} has two parts:

  First, by \autoref{assu:concluding:pvp} (\autopageref{assu:concluding:pvp}) and agent conclusion is a \prop{0}-\val{0} pairing, and by \autoref{assu:concluding:pools} (\autopageref{assu:concluding:pools}) an agent always concludes from some \pool{}.
  Hence, whenever an agent is concluding there is always some relevant \prop{0}-\val{0}-\pool{0} pairing.

  Second, by \autoref{assu:ConRea} (\autopageref{assu:ConRea}), if an agent concludes, the agent reasons to the \prop{0}-\val{0} pair from the \pool{0}.
  However, we place no additional constraints on reasoning.
  Hence, the framework with which we work does not allow finer grain.
\end{note}

\begin{note}
  For example, \autoref{illu:gist:roots:F}.

  The following \prop{2}-\val{}-\pool{} pairings are plausibly in the type `factorisation':
  \begin{center}
    \begin{tabular}{R{.45\textwidth} L{.45\textwidth}}
      \pv{\rootsCon{}}{\valI{True}} & \pv{\propM{2x^{2} - x - 1 = 0}}{\valI{True}}, \dots \\
      \pv{\propM{x \in \{-1,1\}}}{\valI{True}} & \pv{\propM{x^{2} - 1 = 0}}{\valI{True}}, \dots \\
      \pv{\propM{x \in \{4,2\}}}{\valI{True}} & \pv{\propM{x^{2} - 4x + 8 = 0}}{\valI{True}}, \dots \\
      \pv{\propM{x \in \{70,-2\}}}{\valI{True}} & \pv{\propM{x^{2} - 68x - 140 = 0}}{\valI{True}}, \dots \\
    \end{tabular}
  \end{center}
  %
  The following \prop{2}-\val{}-\pool{} pairings are plausibly \emph{not} in the type `factorisation':

  \begin{center}
    \begin{tabular}{R{.45\textwidth} L{.45\textwidth}}
      \pv{\rootsConBad{}}{\valI{True}} & \pv{\propM{2x^{2} - x - 1 = 0}}{\valI{True}}, \dots \\
      \pv{\propI{n is 2000}}{\valI{True}} & \pv{\propM{4^{100} = 2^{n}}}{\valI{True}}, \dots \\
      \pv{\propM{x \in \{70,-2\}}}{\valI{Want}} & \pv{\propM{x^{2} - 68x - 140 = 0}}{\valI{True}}, \dots \\
    \end{tabular}
  \end{center}

  The first \prop{0}-\val{0}-\pool{0} pairing is not the case, the second pairing is unrelated to factorisation, and the third pairing concerns a desire, rather than what is.
\end{note}


\begin{note}
  \torN{3} are, for our interests, collections of \prop{0}-\val{0}-\pool{0} pairings.
  There need be no `natural' description of the type.
\end{note}



\subsection{Definition}
\label{sec:idea}

\begin{note}
  Abstractly, consider `\vAgent{} is \emph{\tCV{}} \(\pv{\phi}{v}\) from \(\Phi\) by type of reasoning \(T\)' as a predicate of an event.
  Necessary condition for predicate.

  \begin{definition}[\tCN{2}]%
    \label{idea:tC}%
    \vspace{-\baselineskip}
    \begin{itemize}
    \item
      \(\ed{}\) is an event in which \vAgent{} is \emph{\tCV{}} \(\pv{\phi}{v}\) from \(\Phi\).\newline
      \hfill(By some type of reasoning \(T\).)
    \end{itemize}

    \emph{Only if}:

    \begin{itemize}
    \item
      For some collections; \(\mathcal{E}\) of possible events, \(\mathcal{X}\) of \prop{0}-\val{0}-\pool{0}~pairings:
      \begin{itemize}
      \item
        For every event \(\ed{\prime}\), there is some \prop{0}-\val{} pair \(\pv{\psi}{v'}\) and \pool{0} \(\Psi\) in \(\mathcal{X}\) such that:
        \begin{itenum}
        \item[\emph{If}:]
          \(\ed{\prime}\) is in the collection of events \(\mathcal{E}\).
        \item[\emph{Then}:]
          \(\ed{\prime}\) is an event in which \vAgent{} is concluding \(\pv{\psi}{v'}\) from \(\Psi\).
        \end{itenum}
      \end{itemize}
    \end{itemize}
    \vspace{-\baselineskip}
  \end{definition}

  \noindent%
  Intuitively, \autoref{idea:tC} expresses the idea that the predicate `\vAgent{} is \tCV{} \(\pv{\phi}{v}\) from \(\Phi\) by type of reasoning \(T\)' applies to an event only if some `law' holds.%
  \footnote{
    Note, this is distinct from the position that concluding/reasoning is rule governed.
    \cite{Boghossian:2008vf,Boghossian:2012vb}, \cite{Broome:2002aa}.

    When reasoning, following a rule.
    We are talking about \emph{type} of concluding, rather than concluding.
    But, the object is not following the predicate.
  }
  Where, `law' is understood in the colloquial sense of a universally quantified material conditional.%
  \footnote{
    For example, consider:

    \citeauthor{Helmholtz:1977aa}'s characterisation of laws of nature:%
    \begin{quote}
      \nocite{Wilson:2006aa}
      Every law of nature asserts that upon preconditions alike in a certain respect, there always follow consequences that are alike in a certain other respect.%
      \mbox{ }\hfill\mbox{(\citeyear[122]{Helmholtz:1977aa})}
    \end{quote}

    The law of large numbers:
    \begin{quote}
      Things of every kind of nature are subject to a universal law which one may well call \emph{the Law of Large Numbers}.
      It consists in that if one observes large numbers of events of the same nature depending on causes which are constant and causes which vary irregularly, \dots, one finds that the proportions of occurrence are almost constant \dots\newline
      \mbox{ }\hfill\mbox{(\citeauthor{Seneta:2013aa}'s (\citeyear[9--10]{Seneta:2013aa}) translation of (\cite[7]{Poisson:1837aa}))}
    \end{quote}

    The law of truly large numbers:
    \begin{quote}
      [W]hen enormous numbers of events and people and their interactions cumulate over time, almost any outrageous event is bound to occur.%
      \mbox{ }\hfill\mbox{(\cite[853]{Diaconis:1989aa})}
    \end{quote}
    \citeauthor{Hempel:1965aa}'s Deductive-Nomological account of scientific explanation, \citeauthor{Boole:1854aa}'s laws of thought, etc.
  }

  \autoref{idea:tC} does not specify the content of the relevant law.
  However, intuitively, the law is to capture other cases where the agent is expected to be concluding by the same type of reasoning, and requires the agent \emph{is} concluding by the same type of reasoning.
\end{note}


\begin{note}
  From a very broad perspective, it may help to think of an agent by some \torN{} in connexion to a \pool{}.

  For example, with respect to \autoref{illu:gist:roots:F} we understood the agent to conclude \pv{\propM{\rootsCon{}}}{\valI{True}} from some \pool{} which captures the \agents{} understanding of factorisation.
  This suggests the agent concluded \pv{\propM{\rootsCon{}}}{\valI{True}} \emph{by} factorising.
  For, if the agent did not conclude \pv{\propM{\rootsCon{}}}{\valI{True}} by factorising then various \prop{0}-\val{0} pairs of the relevant \pool{} would be irrelevant.
  That is to say, if the agent concluded \pv{\propM{\rootsCon{}}}{\valI{True}} on a whim, there would be no relevance of the \agents{} understanding of factorisation to the conclusion.

  Still, more may be said about what makes it the case that the agent concluded \pv{\propM{\rootsCon{}}}{\valI{True}} from some \pool{} which captures the \agents{} understanding of factorisation.
  In particular, some thing may be said about other conclusions the agent may make given their understanding of factorisation.
  For example, if the agent were to try and \emph{fail} to correctly solve various other basic factorisation problems, then it seems the agent did not conclude \pv{\propM{\rootsCon{}}}{\valI{True}} \emph{by} factorising.%
  \footnote{
    In other words, the agent may make other conclusions from \pool{1} which likewise capture the \agents{} understanding of factorisation.
  }

  This idea express here is an instance of a more general idea which amounts to something like modal robustness.
  Sensitivity (\cite{Nozick:1981aa}) and safety (\cite{Sosa:1999aa}) conditions on knowledge are an instance of this kind of idea, as is the general intuition that a skilfully made paper aeroplane flies in a variety of conditions, etc.

  \autoref{idea:tC} does not, strictly, require a fixed \pool{}, though the \illu{1} which follow build on the intuitive connexion between a \torN{} an a \pool{} which captures an \agents{} understanding of some thing.
\end{note}


\subsection{\illu{3}}
\label{sec:illu3-1}

\begin{note}
  We give two \illu{1} of \autoref{idea:tC}.

  The first \illu{0} considers selection tasks, and more broadly links \autoref{idea:tC} to negative results concerning reasoning found in the literature.

  The second \illu{0} considers multiplication by powers, and highlights our interest with \autoref{idea:tC}.
\end{note}



\paragraph*{Selection tasks}
\nocite{Wason:1968aa}
\nocite{Wason:1971aa}
\label{par:selection-tasks}


\begin{note}
  \citeauthor{Wason:1966aa} details their initial section task as follows:

  \begin{quote}
    The subjects (students) were presented with an array of cards and told that every card had a letter on one side and a number on the other side, and that either would be face upwards.
    They were then instructed to decide which cards they would need to turn over in order to determine whether the experimenter was lying in uttering the following statement:
    \emph{if a card has a vowel on one side then it has an even number on the other side}.%
    \mbox{ }\hfill\mbox{(\citeyear[145--146]{Wason:1966aa})}
  \end{quote}

  An example task is given in \autoref{fig:sectionTask}.

  \begin{figure}[H]
    \centering
    \begin{tikzpicture}[
      cardnode/.style={
        rectangle,
        minimum width=10mm,
        minimum height=14mm,
        align=center,
        rounded corners,
        font = {\Large\sffamily},
        very thick,
      },
      node distance=5mm,
      ]

      \node[cardnode, draw] (1) {2};
      \node[cardnode, draw, right = of 1] (2) {N};
      \node[cardnode, draw, right = of 2] (3) {E};
      \node[cardnode, draw, right = of 3] (4) {1};
    \end{tikzpicture}
    \caption{A selection task}
    \label{fig:sectionTask}
  \end{figure}

  \citeauthor{Wason:1966aa} observes the results are consistent with the following hypothesis:%
  \footnote{
    \citeauthor{Wason:1966aa} does not provide a detailed summary of the results.
    However, \citeauthor{Johnson-Laird:1969aa} detail results of \emph{twenty four} University College London students!
    Specifically, 19 of the 24 responded as excepted given \citeauthor{Wason:1966aa}'s hypothesis.
    (\citeyear[369--370]{Johnson-Laird:1969aa}).
  }
  \begin{quote}
    Subjects assume implicitly that a conditional statement has, not two truth values, but three: true, false and `irrelevant'.
    Vowels with even numbers verify, vowels with odd numbers falsify and consonants with any number are irrelevant.%
    \mbox{ }\hfill\mbox{(\citeyear[146]{Wason:1966aa})}
  \end{quote}
\end{note}

\begin{note}
  Selection tasks may be understood in connexion with \autoref{idea:tC}.
  In particular:
  %
  \begin{itemize}
  \item
    The relevant \tocN{} \(T\) is `reasoning via the material conditional'.%
    \footnote{
      \citeauthor{Wason:1966aa}'s hypothesis concerns a conditional statement having two truth values, but does not specify what these truth values are.
      Still, it is clear by inspection that these are the truth values of the material conditional.
    }
  \item
    The collection of possible events \(\mathcal{E}\) of interest contains selection tasks, plausibly taken under close to ideal circumstances.
    And,
  \item
    The collection of \prop{0}-\val{0}-\pool{0} pairings \(\mathcal{X}\) contains \prop{0}-\val{0}-\pool{0} pairings which capture appropriate \evalN{1} of what do.
    We specify these with a variable `\(C\)' to represent the relevant card:
    %
    \begin{center}
      \begin{tabular}{R{.40\textwidth} L{.40\textwidth}}
        \prop{2}-\val{0} pair & \prop{2}-\val{0} pairs in \pool{0} \\
        \hline
        \pv{C\propI{ needs to be turned over}}{\valI{True}} & \pv{C\propI{ has a vowel}}{\valI{True}} \\
        \pv{C\propI{ needs to be turned over}}{\valI{True}} & \pv{C\propI{ has an odd number}}{\valI{True}} \\
        \pv{C\propI{ needs to be turned over}}{\valI{False}} & \pv{C\propI{ has consonant}}{\valI{True}} \\
        \pv{C\propI{ needs to be turned over}}{\valI{False}} & \pv{C\propI{ has an even number}}{\valI{True}} \\
      \end{tabular}
    \end{center}
  \end{itemize}
  %
  As a consequence, \(\ed{}\) is an event when an agent is \tC{} some \prop{0}-\val{0} pair from some \pool{} by reasoning via the material conditional only if the agent would succeed at a selection task.

  \citeauthor{Wason:1966aa} observes most subjects do not succeed at selection tasks.
  Hence, \citeauthor{Wason:1966aa} suggests that subjects do not --- in general --- reason via the material conditional.
\end{note}


\begin{note}
  There is a important subtlety to this construction.
  Subjects clearly do reason in line with the material conditional.
  For example, various definitions and ideas (including \autoref{idea:tC}) in this document are given with an implicit expectation that any instance \emph{if} \dots \emph{then} \dots construction is understood in line with the material conditional.

  Hence, our reconstruction of \citeauthor{Wason:1966aa} does not state that subjects do not reason with the material conditional.
  Instead, the relevant \torN{} is intended to capture the idea that \emph{in general} subjects reason with conditionals via the material conditional.
  Rejecting that subjects `in general' reason with conditionals via the material conditional is compatible with subjects \emph{sometimes} reasoning with the material conditional.

  Specifying the relevant \torN{} is difficult.
  We understand there to be a difference between `reasoning via the material conditional' and `reasoning via the material conditional when implicitly or explicitly called on by the situation'.
  And, likewise, \citeauthor{Wason:1966aa}'s hypothesis of three truth values does not single out any particular \torN{}, as \citeauthor{Wason:1966aa} does not give a complete account of the truth values.

  Still, various other observations may be seen to parallel to our treatment of selection tasks via \autoref{idea:tC}.
  In particular, consider \citeauthor{Harman:1984aa}'s (\citeyear{Harman:1984aa,Harman:1986ux}) arguments against a strong connexion between logical principles and principles of belief revision.
  \citeauthor{Harman:1984aa} summarises:
  %
  \begin{quote}
    Logical principles are not directly rules of belief revision.
    [\dots]
    Logical principles hold universally, without exception, whereas the corresponding principles of belief revision would be at best prima facie principles, which do not always hold.%
    \mbox{ }\hfill\mbox{(\citeyear[107--108]{Harman:1984aa})}
  \end{quote}
    %
  Likewise, consider the \citeauthor{Allais:1979aa} paradox (\cite{Allais:1979aa}),
  the Ellsberg paradox (\cite{Ellsberg:1961aa}), \citeauthor{Makinson:1965aa}'s Paradox of the Preface (\citeyear{Makinson:1965aa}), \citeauthor{Kyburg:1997aa}'s Lottery Paradox (\citeyear{Kyburg:1997aa}), \citeauthor{Quinn:1990aa}'s  puzzle of the self-torturer (\citeyear{Quinn:1990aa}), \citeauthor{Bratman:1981aa}'s arguments against the desire-belief model of practical reasoning (\citeyear{Bratman:1981aa,Bratman:1987aa}), and so on.

  Common to each observation mentioned is the idea that an agent failing to conclude something shows that other instances of the agent's reasoning does not have some general characteristic.
\end{note}


% \paragraph*{Rules}

% \begin{note}
%   Selection tasks are events which happen, and \citeauthor{Wason:1966aa}'s hypothesis is that people \emph{in general} do not do not reason about conditionals using only \valI{True} and \valI{False}.
%   However, the events the law of \autoref{idea:tC} quantifies over need not happen, and the event at issue may be a single event.
% \end{note}

% \begin{note}
%   \begin{scenario}[Addition]%
%     \label{illu:quus}%
%     An agent is given pairs of numbers \(x\) and \(y\) and asked to respond with \(x + y\).
%     The table below represents the event as the agent responds to the pairs.

%     \medskip
%     \hspace{2.8em}%
%     \(
%       \begin{array}{ccccccc}
%       x & 3 & 54 & 21 & 3 & 17 & 0 \\
%       y & 7 & 32 & 64 & 2 & 25 & 6 \\
%       \hline
%       \text{Response} & 10 & 86 & 85 & 5 & 42 & 6 \\
%     \end{array}
%     \)
%     \medskip

%     \noindent%
%     The agent is distracted.
%     However, if the agent had not been distracted, they would have continued as follows:

%     \medskip
%     \hfill%
%     \(
%     \begin{array}{ccccc}
%       \cdots & 8 & 68 & 21 & 58 \\
%       \cdots & 92 & 57 & 23 & 92 \\
%       \hline
%       \cdots & 100 & 5 & 44 & 5 \\
%     \end{array}
%     \)%
%     \hspace{2.8em}%
%     \mbox{ }%
%     \newline%
%   \end{scenario}

%   \noindent%
%   It seems the agent was not reasoning by addition.%
%   \footnote{
%     Rather, it seems the agent was reasoning by quaddition, following \citeauthor{Kripke:1982aa}'s (\citeyear{Kripke:1982aa}) definition of `quss':
%     \begin{align*}
%       x \text{ quss y} &= x \text{ plus } y, \text{ if } x,y < 57 \\
%                        &= 5 \phantom{pl if x,,,} \text{ otherwise }
%     \end{align*}
%     \vspace{-\baselineskip}
%   }

%   For, consider the counterfactual event.
%   The agent concluded \pv{\propI{x + y is 5}}{\valI{True}} from some \pool{} containing \pv{\propI{x is 68}}{\valI{True}} and \pv{\propI{y is 57}}{\valI{True}}.
%   Hence, it seems the agent was not concluding \pv{\propI{x + y is 125}}{\valI{True}}.

%   Further, though the counterfactual event suggests the agent was not reasoning by addition, it is less clear that the agent never reasons by addition.
%   The agent's interpretation of `\(+\)' as something other than `plus' may be no different from interpreting `\(A \land B\)' as `\(A\) and \(B\)' rather than `the meet of sets \(A\) and \(B\)'.
% \end{note}


\paragraph*{Powers}

\begin{note}
  Selection tasks are events which happen and highlight an \agents{} reasoning is not of some particular type.
  However, our interest with \tCN{} primarily concerns events where the agent \emph{is} \tCV{} by some type, and \autoref{idea:tC} suggests the agent would make some conclusion if some possible event were to happen.
\end{note}

\begin{note}
  \begin{scenario}[Powerful multiplication]%
    \label{illu:tR:powers}%
    A student has been studying algebra and has just been introduced to the rule of multiplication for powers (\(a^{n} \cdot a^{m} = a^{n + m}\)).

    At hand are a handful of exercises (from \cite[32]{Gelfand:1993aa}):
    %
    \begin{quote}
      \begin{enumerate}[label=(\alph*), ref=(\alph*)]
      \item
        \label{mfp:a}
        You know that \(2^{1001} \cdot 2^{n} = 2^{2000}\).
        What is \(n\)?
      \item
        \label{mfp:b}
        You know that \(2^{1001} \cdot 2^{n} = \sfrac{1}{4}\).
        What is \(n\)?
      \item
        \label{mfp:c}
        Which is bigger: \(10^{-3}\) or \(2^{-10}\)?
      \item
        \label{mfp:d}
        You know that \(\sfrac{2^{1000}}{2^{n}} = 2^{501}\).
        What is \(n\)?
      \item
        \label{mfp:e}
        You know that \(\sfrac{2^{1000}}{2^{n}} = \sfrac{1}{16}\).
        What is \(n\)?
      \item
        \label{mfp:f}
        You know that \(4^{100} = 2^{n}\).
        What is \(n\)?
      \item
        \label{mfp:g}
        You know that \(2^{100} \cdot 3^{100} = a^{100}\).
        What is \(a\)?
      \item
        \label{mfp:h}
        You know that \((2^{10})^{15} = 2^{n}\).
        What is \(n\)?
      \end{enumerate}
    \end{quote}
    %
    The student starts work on exercise~\ref{mfp:f}, and is concluding \(\pv{n\propI{ is }200}{\valI{True}}\).
  \end{scenario}

  \noindent%
  Intuitively, the student is reasoning with an understanding of multiplication for powers.

  Further, it seems the agent is concluding \pv{\propI{n is 200}}{\valI{True}} by an understanding of multiplication for powers \emph{only if} the agent would be concluding \pv{\propI{a is 6}}{\valI{True}} from \pv{\propM{2^{100} \cdot 3^{100} = a^{100}}}{\valI{True}} were the agent to abandon \ref{mfp:g} and work on~\ref{mfp:f}.

  Indeed, it is plausible exercises \ref{mfp:a} through to \ref{mfp:h} were chosen by \citeauthor{Gelfand:1993aa} to test basic competence with multiplication for powers.
  Hence, if the agent would fail to be concluding \pv{\propI{a is 6}}{\valI{True}} from \pv{\propM{2^{100} \cdot 3^{100} = a^{100}}}{\valI{True}} when working on~\ref{mfp:f}, then it seems the agent does not have an adequate understanding of multiplication for powers to be concluding \emph{by} an understanding of multiplication for powers.%
  \footnote{
    This may be resisted.
    For example, exercises~\ref{mfp:b},~\ref{mfp:d}, and~\ref{mfp:e} all involve fractions.
    And, the agent may be shaky on fractions, but good with multiplication by powers.
    Hence, the relevant collection \prop{0}-\val{0}-\pool{0} pairings \(\mathcal{X}\) may only include \prop{0}-\val{0}-\pool{0} pairings which correspond to exercises \ref{mfp:a},  \ref{mfp:c},  \ref{mfp:f},  \ref{mfp:g}, and \ref{mfp:h}.

    For broader motivation for such restrictions, consider \citeauthor{Chomsky:2015aa}'s distinction between competence and performance.
    %
    \begin{quote}
      Arithmetical competence yields the correct number z for every pair~(x,~y) under addition or multiplication.
      But only a small finite subpart of arithmetical competence can be exhibited without external aids (by calculating in one's head).
      Obviously, that fact does not imply that arithmetical competence is correspondingly limited.%
      \mbox{ }\hfill\mbox{(\citeyear[xii]{Chomsky:2015aa})}
    \end{quote}
  }
\end{note}



\section{\tC{2} and \fc{1}}
\label{cha:typical:tCDef}


\begin{note}
  \autoref{sec:idea} introduced the idea of an agent \tCV{} (\autoref{idea:tC}).
  The present section links an agent \tCV{} to the idea of \fc{1}.
\end{note}

\begin{note}
  \begin{proposition}[\typeAdj{2} \fc{1}]%
    \label{prop:tCV-fc}%
    \vspace{-\baselineskip}
    \begin{itenum}
    \item[\emph{If}:]
      Conditions \ref{prop:tCV-fc:tC} and \ref{prop:tCV-fc:e} hold:
      \begin{enumerate}[label=\arabic*., ref=\arabic*]
      \item
        \label{prop:tCV-fc:tC}
        \(\ed{}\) is an event in which \vAgent{} is \tCV{} \(\pv{\phi}{v}\) from \(\Phi\) by \torNa{} \(T\).
      \item
        \label{prop:tCV-fc:e}
        There is some partition of \(\ed{}\) into sub-events \(\edn{1}, \dots, \edn{k}\) such that for some event \(\edn{i}\) in the partition:
        \begin{itemize}
        \item
          There is some possible event \(\edn{\sharp}\) in \(\mathcal{E}\) and \prop{0}-\val{0}-\pool{0} pairing \(\pvp{\psi}{v'}{\Psi}\) in \(\mathcal{X}\) such that:
          \begin{enumerate}[label=\alph*., ref=\theenumi\alph*]
          \item
            \label{prop:tCV-fc:e:act:i}
            \(\edn{\sharp}\) is the result of an action \(a\) done by \vAgent{} in \(\ed{i}\)
          \item
            \label{prop:tCV-fc:e:act:ii}
            \vAgent{} \evals{} \(\psi'\) as having value \(v''\) prior to doing \(a\), for each \prop{0}-\val{0} pair \(\pv{\psi'}{v''}\) in \(\Psi\).
          \end{enumerate}
        \end{itemize}
      \end{enumerate}
    \item[\emph{Then}:]
      There is some \prop{0}-\val{0}-\pool{0} pairing \(\pvp{\psi}{v'}{\Psi}\) such that:
      \begin{itemize}
      \item
        \(\pv{\psi}{v'}\) is a \fc{0} from \(\Psi\) for \vAgent{} throughout \(\ed{i}\).
      \end{itemize}
    \end{itenum}
    \vspace{-\baselineskip}
  \end{proposition}

  \noindent%
  \autoref{prop:tCV-fc} considers a \scen{0} in which an agent is \tCV{} and the collection of possible events and \prop{0}-\val{0}-\pool{0} pair satisfies conditions.

  First condition.
  \tCV{} a collection of possible events.
  In some cases, possible events correspond to result of the agent doing some action.

  As agent is \tCV{}, then for each event, \prop{0}-\val{0}-\pool{0} pairing such that the agent is concluding.
  Second condition, agent does not require novel information after action is done to conclude.

  For example, consider \autoref{illu:tR:powers}.
  Agent is concluding \pv{\propI{n is 200}}{\valI{True}} by an understanding of multiplication for powers \emph{only if} the agent would be concluding \pv{\propI{a is 6}}{\valI{True}} from \pv{\propM{2^{100} \cdot 3^{100} = a^{100}}}{\valI{True}} were the agent to abandon \ref{mfp:g} and work on~\ref{mfp:f}.
  So, we have some collection of events, result of action.
  And, second condition is satisfied.
\end{note}

\begin{note}
  The full argument for \autoref{prop:tCV-fc}, though merely amounts to generalising the above observation by citing the relevant definitions.

  \begin{argument}{prop:tCV-fc}
    Suppose conditions \ref{prop:tCV-fc:tC} and \ref{prop:tCV-fc:e} hold.

    By Condition~\ref{prop:tCV-fc:tC}, \(\ed{}\) is an event in which \vAgent{} is \tCV{} \(\pv{\phi}{v}\) from \(\Phi\) by \torNa{} \(T\).
    So, by \autoref{idea:tC} there is some collections \(\mathcal{E}\) of events and \(\mathcal{X}\) of \prop{0}-\val{0}-\pool{0}~pairings such that:
    %
    \begin{itemize}[noitemsep]
    \item
      For every event \(\ed{\prime}\), there is some \prop{0}-\val{} pair \(\pv{\psi}{v'}\) and \pool{0} \(\Psi\) in \(\mathcal{X}\) such that:
      \begin{itenum}[noitemsep]
      \item[\emph{If}:]
        \(\ed{\prime}\) is in the collection of events \(\mathcal{E}\).
      \item[\emph{Then}:]
        \(\ed{\prime}\) is an event in which \vAgent{} is concluding \(\pv{\psi}{v'}\) from \(\Psi\).
      \end{itenum}
    \end{itemize}
    %
    Further, Condition~\ref{prop:tCV-fc:e} concerns a sub-event \(\edn{i}\) of some partition of \(\ed{}\) into sub-events \(\edn{1}, \dots, \edn{k}\).
    In particular, there is some possible event \(\edn{\sharp}\) in \(\mathcal{E}\) and \prop{0}-\val{0}-\pool{0} pairing \(\pvp{\psi}{v'}{\Psi}\) in \(\mathcal{X}\) such that:
    \begin{enumerate}[label=\alph*., ref=\theenumi\alph*]
    \item
      \(\edn{\sharp}\) is the result of an action \(a\) done by the agent in \(\ed{}\).
    \item
      The agent \evals{} \(\psi'\) as having value \(v''\) prior to doing \(a\), for each \prop{0}-\val{0} pair \(\pv{\psi'}{v''}\) in \(\Psi\).
    \end{enumerate}
    By Condition \ref{prop:tCV-fc:e:act:i} and \autoref{idea:tC}, it follows that the agent is concluding \(\pv{\psi}{v'}\) from \(\Psi\) in \(\edn{\sharp}\).

    Now, \(\edn{i}\) is trivially a partition of \(\edn{i}\) into sub-events.
    And, as just observed there is some description \(d_{i}\) such that \(\ed{i}\) is an event in which the agent may do some action \(a_{i}\) where:
    %
    \begin{enumerate}[label=\Alph*., ref=\Alph*, series=tCVFCArg]
    \item
      The event \(\ed{i+}\) in which the agent does \(a_{i}\) is an event in which the agent is concluding \(\pv{\psi}{v'}\) from \(\Psi\).
    \end{enumerate}
    %
    Further, by Condition~\ref{prop:tCV-fc:e:act:ii}:
    %
    \begin{enumerate}[label=\Alph*., ref=\Alph*, resume*=tCVFCArg]
    \item
      The agent \evals{} \(\psi'\) as having value \(v''\) prior to doing \(a_{i}\), for each \prop{0}-\val{0} pair \(\pv{\psi'}{v''}\) in \(\Psi\).
    \end{enumerate}
    %
    Hence, the definition of \(\pv{\phi}{v}\) being a \fc{0} from \(\Psi\) throughout \(\ed{i}\) is satisfied (cf.~\autoref{def:fc}, \autopageref{def:fc}).
  \end{argument}
\end{note}


\section{An \illu{0}}


\begin{note}
  To close this chapter we offer a final \illu{0} of \tCV{}, though this time with a focus on \fc{1} given \autoref{prop:tCV-fc}.
\end{note}


\begin{note}
  \phantlabel{squish-elimination-proof}

  \begin{scenario}[Squish elimination]%
    \label{scen:squish}%
    Some time ago the agent showed \sqE{} is sound.

    It is now late morning on a sunny day.
    The agent ate a good breakfast, and drank some nice coffee and does the following syntactic proof:
    %
    \begin{center}
      \begin{fitch}
        \phantlabel{illu:sP:1}\fa (P \rightarrow Q) \rightarrow P \\
        \phantlabel{illu:sP:2}\fj R \\
        \phantlabel{illu:sP:3}\fa P & \sqE{}:\hyperref[illu:sP:1]{1} \\
        \phantlabel{illu:sP:4}\fa P \land R & \(\land\)\textbf{Intro:} \hyperref[illu:sP:2]{2},\hyperref[illu:sP:3]{3}
      \end{fitch}
    \end{center}
    %
    The agent concludes \((P \rightarrow Q) \rightarrow P, R \vdash P \land R\).
  \end{scenario}

  \noindent%
  Intuitively, agent \tCV{} \((P \rightarrow Q) \rightarrow P, R \vdash P \land R\) from some \pool{} which captures the \agents{} understanding of the relevant Fitch-style proof system.
  And, as the agent is \tCV{}, some \torNa{} captures the \agents{} understanding of the relevant Fitch-style proof system.
\end{note}


\begin{note}
  Still, in \autoref{scen:squish} the agent uses the non-standard \sqE{} rule.
  Specifically:

  \begin{definition}[\sqE{}]%
    \label{def:sque}%
    \sqE{} is the following rule:
    \begin{center}
      \begin{fitch}
        \ftag{\text{\scriptsize \emph{i}}}{\fa (\phi \rightarrow \psi) \rightarrow \phi} \\
        \ftag{\text{\scriptsize }}{\fa \vdots } \\
        \ftag{\text{\scriptsize \emph{j}}}{\fa \phi } & \sqE{}:\emph{i} \\
      \end{fitch}
    \end{center}
  \end{definition}

  \noindent%
  By stipulation, the agent has proved \sqE{} is sound.
  And, \sqE{} is indeed sound.%
  \footnote{
    \label{prop:sqE-sound}
    Rather than prove \sqE{} is sound (which would require a detailed statement of the proof system in question), we show that the key corresponding semantic entailment holds:

    Let \(v\) be an arbitrary (truth-functional) valuation, and assume \(v((\phi \rightarrow \psi) \rightarrow \phi) = \valI{True}\).
    Further, assume for contradiction \(v(\phi) = \valI{False}\).

    As \(v(\phi) = \valI{False}\), it immediately follows that \(v(\phi \rightarrow \psi) = \valI{True}\).
    Therefore, by the first assumption, it must be the case that \(v(\phi) = \valI{True}\).
    This contradictions the second assumption.
  }

  Still, \((P \rightarrow Q) \rightarrow P, R \vdash P \land R\) may be logic soup.
  The agent has proved \sqE{} is sound.
  Though, no guarantee that the \agents{} recollection of \sqE{} is tied to their understanding of the relevant Fitch-style proof system.

  In parallel to failing to identify correct cards, or failing solve exercise, something wrong.
\end{note}


\begin{note}
  Further, the proof of the soundness of \sqE{} is fairly straightforward.
  Hence, if the agent is \tCV{}, some action such that concluding \sqE{} is sound.%
  \footnote{
    Two options.
    Either directly show \sqE{} follows from the \agents{} understanding of the proof system via constructing a meta-proof of \sqE{} or indirectly show \sqE{} follows by providing a semantic argument and combine with completeness result.
    If neither direct or indirect, then the \agents{} use of \sqE{} does not follow from the \agents{} understanding of relevant Fitch-style proof system.
  }
  From \autoref{prop:tCV-fc}, \fc{}.%
  \footnote{
    Further, \sqaE{}, \sqoE{}, and \sqeE{} are all unsound.
    Respectively, the following rules:

    \mbox{ }\hfill
    \begin{minipage}[c]{0.25\textwidth}\vspace{0pt}
      \begin{fitch}
        \ftag{\text{\scriptsize \emph{i}}}{\fa \phi \rightarrow (\psi \rightarrow \phi)} \\
        \ftag{\text{\scriptsize }}{\fa \vdots } \\
        \ftag{\text{\scriptsize \emph{j}}}{\fa \phi } \\
      \end{fitch}
    \end{minipage}
    \hfill
    \begin{minipage}[c]{0.25\textwidth}\vspace{0pt}
      \begin{fitch}
        \ftag{\text{\scriptsize \emph{i}}}{\fa \psi \rightarrow (\phi \rightarrow \psi)} \\
        \ftag{\text{\scriptsize }}{\fa \vdots } \\
        \ftag{\text{\scriptsize \emph{j}}}{\fa \phi } \\
      \end{fitch}
    \end{minipage}
    \hfill
    \begin{minipage}[c]{0.25\textwidth}\vspace{0pt}
    \begin{fitch}
        \ftag{\text{\scriptsize \emph{i}}}{\fa (\psi \rightarrow \phi) \rightarrow \psi} \\
        \ftag{\text{\scriptsize }}{\fa \vdots } \\
        \ftag{\text{\scriptsize \emph{j}}}{\fa \phi } \\
      \end{fitch}
    \end{minipage}
    \hfill\mbox{ }
}
\end{note}


\begin{note}
  Does not require that the agent (re)proves \sqE{} is sound.
  At issue is that the agent has the option.
  Likewise, this does not state this is enough to state the agent is \tCV{}.
  \autoref{idea:tC} is a partial definition.
\end{note}

\section*{Summary}

\begin{note}
  \tCN{2}.

  Extension account of \torN{}.
  Due to abstracting over theories.

  Then, necessary condition on \tC{}.
\end{note}

\begin{note}
  Only motivated \tC{} by intuition.
  Have not argued that this intuition is correct.
  Key piece, and intuitive.
\end{note}




% \section[\citeauthor{Carroll:1895uj}]{\citeauthor{Carroll:1895uj}\hfill(Optional)}

% \nocite{Black:1951aa}

% \begin{note}
%   The point here is that with Carroll, generality that goes beyond any single instance.
%   Must apply to all instances, to be valid.
%   But, cannot hope to cover all instances in a single move.
% \end{note}

% \begin{note}
%   A difficulty found on a reading of \citeauthor{Carroll:1895uj}'s \citetitle{Carroll:1895uj}.
% \end{note}

% \begin{note}
%   \begin{quote}
%     ``Plenty of blank leaves, I see!'' the Tortoise cheerily remarked.
%     ``We shall need them \emph{all}!''
%     (Achilles shuddered.)
%     ``Now write as I dictate:---

%     \begin{enumerate}[label=(\emph{\Alph*}), ref=\emph{\Alph*}]
%     \item
%       \label{AatT:a}
%       Things that are equal to the same are equal to each other.
%     \item
%       \label{AatT:b}
%       The two sides of this Triangle are things that are equal to the same.
%     \item
%       \label{AatT:c}
%       If~\ref{AatT:a} and~\ref{AatT:b} are true,~\ref{AatT:z} must be true.
%       \setcounter{enumi}{25}
%     \item
%       \label{AatT:z}
%       The two sides of this Triangle are equal to each other.''
%     \end{enumerate}

%     ``You should call it~\ref{AatT:d}, not~\ref{AatT:z},'' said Achilles.
%     ``It comes \emph{next} to the other three.
%     If you accept~\ref{AatT:a} and~\ref{AatT:b} and~\ref{AatT:c}, you \emph{must} accept~\ref{AatT:z}.''

%     ``And why \emph{must} I?''

%     ``Because it follows \emph{logically} from them.
%     If~\ref{AatT:a} and~\ref{AatT:b} and~\ref{AatT:c} are true,~\ref{AatT:z} \emph{must} be true.
%     You don't dispute \emph{that}, I imagine?''

%     ``If~\ref{AatT:a} and~\ref{AatT:b} and~\ref{AatT:c} are true,~\ref{AatT:z} \emph{must} be true,'' the Tortoise thoughtfully repeated.
%     ``That's \emph{another} Hypothetical, isn't it?
%     And, if I failed to see its truth, I might accept~\ref{AatT:a} and~\ref{AatT:b} and~\ref{AatT:c}, and \emph{still} not accept~\ref{AatT:z}, mightn't I ?''

%     \mbox{}\hfill\(\vdots\)\hfill\mbox{}

%     ``Then Logic would take you by the throat, and force you to do it!''
%     Achilles triumphantly replied.
%     ``Logic would tell you 'You ca'n't help yourself.''%
%     \mbox{ }\hfill\mbox{(\citeyear[279--280]{Carroll:1895uj})}
%   \end{quote}

%   The Tortoise has written down three premises,~\ref{AatT:a},~\ref{AatT:b}, and~\ref{AatT:c}.
%   Achilles holds that~\ref{AatT:z} follows from~\ref{AatT:a},~\ref{AatT:b}, and~\ref{AatT:c}.
%   The Tortoise observes they have the possibility of refraining to accept~\ref{AatT:z} follows from~\ref{AatT:a},~\ref{AatT:b}, and~\ref{AatT:c}.
%   And (initially), the Tortoise does not accept~\ref{AatT:z} follows from~\ref{AatT:a},~\ref{AatT:b}, and~\ref{AatT:c}.
%   Achilles requests the Tortoise accepts that~\ref{AatT:z} follows from~\ref{AatT:a},~\ref{AatT:b}, and~\ref{AatT:c}, and the Tortoise complies.
%   Specifically, the Tortoise grants:

%   \begin{quote}
%     \begin{enumerate}[label=(\emph{\Alph*}), ref=\emph{\Alph*}]
%       \setcounter{enumi}{3}
%     \item
%       \label{AatT:d}
%       If~\ref{AatT:a} and~\ref{AatT:b} and~\ref{AatT:c} are true,~\ref{AatT:z} must be true.%
%       \mbox{ }\hfill\mbox{(\citeyear[279]{Carroll:1895uj})}
%     \end{enumerate}
%   \end{quote}

%   But, does not accept~\ref{AatT:z} follows from~\ref{AatT:a},~\ref{AatT:b},~\ref{AatT:c}, and~\ref{AatT:d}.
% \end{note}

% \begin{note}
%   Modus ponens.

%   \begin{quote}
%     From \(\phi\) and \emph{if} \(\phi\) then \(\psi\), infer \(\psi\).
%   \end{quote}

%   Modus ponens is general.
%   For \emph{any} \(\phi\), \(\psi\).

%   Now, there is a difference between \emph{modus ponens} and conditional.

%   However, take any instance.
%   Then, if \(P\), \(P \rightarrow Q\), \(Q\) must be true.
%   But, then this means that the conditional is true.

%   Consequence of the deduction theorem.

%   Likewise, deduction theorem goes the other way.

%   However, going from \(P\), \(P \rightarrow Q\) to \(Q\) need not be an instance of \emph{modus ponens}.
% \end{note}

% \begin{note}
%   Well, this is a headache.
%   \citeauthor{Carroll:1895uj} is talking about a specific A, B, and Z.
%   There is no clear generality.
% \end{note}

% \begin{note}
%   So, consider at issue is modus ponens.
%   For any specific instance accept, there is a further instance.
%   For, \(A, (A \rightarrow B) \vDash B\).
%   Then, \(\vDash (A \land (A \rightarrow B) \rightarrow B)\).
%   However, now, \(A \land (A \rightarrow B), (A \land (A \rightarrow B) \rightarrow B) \vDash B\).
%   And, so on.

%   The general pattern, get conditional, but then this gives a new instance of modus ponens, which must be true in order for modus ponens to be valid rule of inference.

%   \citeauthor{Carroll:1895uj}, by contrast, starts with \(A \vDash B\).
%   This is different.
%   However, rather than focusing on a single rule of inference, the puzzle turns on what validity amounts to.

%   Validity is a general thing, with specific instances.
%   However, grant any particular instance of validity without employing validity in general.
% \end{note}

% \begin{note}
%   \begin{quote}
%     My paradox \dots turns on the fact that, in a Hypothetical, the \emph{truth} of the Protasis, the \emph{truth} of the Apodosis, and the \emph{validity of the sequence}, are 3 distinct Propositions.

%     \mbox{}\hfill\(\vdots\)\hfill\mbox{}

%     Suppose I say ``I grant~\ref{AatT:a} and~\ref{AatT:b} and~\ref{AatT:c}, but I do \emph{not} grant that I am thereby \emph{obliged} to grant~\ref{AatT:z}.''
%     Surely, my granting~\ref{AatT:z} must \emph{wait} until I have been made to see the validity of this sequence: i.e.\ in order to grant~\ref{AatT:z}, I must grant~\ref{AatT:a},~\ref{AatT:b},~\ref{AatT:c}, and~\ref{AatT:d}! And so on.%
%     \mbox{ }\hfill\mbox{(\citeyear[472]{Carroll:1977wl})}
%   \end{quote}

%   My interpretation of the point \citeauthor{Carroll:1895uj} makes in this passage is that the truth of A B and the truth of C is distinct from the validity of A B C.
%   Granting is substantial, not merely moving.
%   But, in order to grant, this means granting all other cases.

%   So, the paradox is that, on the one hand, don't need validity for any specific true things.
%   But, on the other hand, only of interest if via validity.

%   The Tortoise is slowly working through each instance, but this has no hope of getting the Tortoise to general validity.
%   So, how does the Tortoise ever make it there?
% \end{note}

% \begin{note}
%   This point differs from received interpretation.

%   \citeauthor{Wieland:2013vf} (\citeyear{Wieland:2013vf}) characterises the general understanding of \textcite{Carroll:1895uj} in terms of two lessons:
%   \begin{quote}
%     [T]he negative lesson is that if you add ever more premises to an argument \dots, then you will never demonstrate that its conclusion follows logically.\newline
%     \mbox{ }\hfill\mbox{(\citeyear[984]{Wieland:2013vf})}
%   \end{quote}

%   Parallel, static answers, still option for concluding otherwise.

%   \begin{quote}
%     [T]he positive lesson is that rules of inference, rather than premises of the form `if premises such and such are true, then the conclusion is true', will do the job.\newline
%     \mbox{ }\hfill\mbox{(\citeyear[984]{Wieland:2013vf})}
%   \end{quote}

%   \begin{quote}
%     [\citeauthor{Carroll:1895uj}] simply lacked any distinct conception of a deduction as opposed to the assertion (``granting'' of) a hypothetical proposition.
%     \dots
%     Any attempt by Carroll to tackle the question of inference was bound to begin in confusion and end in constipation-all those premises piling up, but no motion.
%   \end{quote}
% \end{note}

% \paragraph{The Dichotomy}

% \begin{note}
%   Achilles and the Tortoise, Zeno's argument.

%   Surely, right?

%   Two ways to understand.
%   Does the Tortoise move at all, or does the Tortoise arrive at the end?
%   I mean, as formulated by Zeno, it's about catching up, no matter how much one moves.

%   It is different from Zeno's Dichotomy paradox.


%   If so, then we should expect the Tortoise to be making some movement.
%   Adding rules of inference is of no help, because the problem is not movement, it's about how to move so much in a single step.
% \end{note}

% \begin{note}
%   \color{red}
%   Something about logic forcing.
%   The Tortoise hasn't arrived.

%   Nothing hangs on validity.
%   Same issue with testimony.
%   `A'.
%   Why?
%   Testified A, so A.
%   Okay, but another instance of testimony.
%   Testified(Testified A, so A), so Testified A, so A.
% \end{note}

% \begin{note}
%   \begin{quote}
%     But if we who wish to represent his belief in Q as based on P are to write in our notebook everything his having that belief on that basis consists in then when we have written only P and Q we will not have written enough.
%     Someone can believe P and believe Q and still not believe Q on the basis of P whatever the relations between the propositions P and Q happen to be.
%     He might believe Q for some reason completely unconnected with P, or perhaps for no reason at all (if that is possible).%
%     \mbox{ }\hfill\mbox{(\citeyear[185]{Stroud:1979aa})}
%   \end{quote}
%   However, the moral drawn is narrow
%   \begin{quote}
%     The moral is that for every proposition or set of propositions the belief or acceptance of which is involved in someone's believing one proposition on the basis of another there must be something else, not simply a further proposition accepted, that is responsible for the one belief's being based on the other.%
%     \mbox{ }\hfill\mbox{(\citeyear[187]{Stroud:1979aa})}
%   \end{quote}

%   Even if we grant each individual is \ros{}, rather than an instance of the material conditional, \emph{logic} hasn't done anything.
% \end{note}

% \paragraph{General and specific: Contrast}

% \begin{note}
%   Use \citeauthor{Carroll:1895uj} to illustrate this point.

%   However, given the worry, various other things may be understood this way.

%   Hume, constant conjunction.
%   Part of the problem is identifying cause.
%   We get the famous line about observing.
%   However, Hume goes on.
%   It's not only no observation, but no generality.

%   Right, so more narrow than Hume.
%   Because, with Hume, at issue is whether we have grounds for this general thing.
%   With Carroll, it's whether we even really get to the general thing.
% \end{note}

% \begin{note}
%   \begin{quote}
%     Let me ask this: what has the expression of a rule—say a sign-post—got to do with my actions?
%     What sort of connexion is there here?%
%     ---%
%     Well, perhaps this one:
%     I have been trained to react to this sign in a particular way, and now I do so react to it.

%     But that is only to give a causal connexion; to tell how it has come about that we now go by the sign-post; not what this going-by-the sign really consists in.
%     On the contrary; I have further indicated that a person goes by a sign-post only in so far as there exists a regular use of sign-posts, a custom.%
%     \mbox{ }\hfill\mbox{(\citeyear[\S198]{Wittgenstein:1958aa})}
%   \end{quote}

%   Regular use of sign-posts, custom.

%   Ugh, this is ambiguous.
% \end{note}


% %
%   \(^{,}\)
%   \footnote{
%     \citeauthor{Hlobil:2014tq}'s ``Inferential Moorean Phenomenon'':
%   \begin{quote}
%     \begin{enumerate}
%     \item[(IMP)]
%       It is either impossible or seriously irrational to infer \emph{P} from \emph{Q} and to judge, at the same time, that the inference from \emph{Q} to \emph{P} is not a good inference.
%     \end{enumerate}
%     \dots
%     By the ``goodness'' of an inference I mean the feature that makes the relevant inference permissible. Thus, if the inference under consideration is an inductive inference, the relevant kind of goodness is not deductive validity.%
%     \mbox{ }\hfill\mbox{(\citeyear[\S1]{Hlobil:2014tq})}
%   \end{quote}
%   Though, this really isn't more basic given the interest in \tR{}.
%   For, the puzzle is what it is to `infer'.

%   Rationality isn't part of the picture.
%   And, this is a significant drawback of \citeauthor{Hlobil:2014tq}'s approach.
% }


% \subsection{Types and explanation}
% \label{cha:typical:sec:tor:bkgd}

% \begin{note}
%   There is a related, stronger claim, that generality derives from rule following.

%   For this, \citeauthor{Boghossian:2008vf}:

%   \begin{quote}
%     [O]ur internalization of general epistemic rules---like Modus Ponens and Induction---explain and rationalize why we form the beliefs that we form.
%     And that seems intuitively correct.

%     As in the case of our linguistic and conceptual abilities, our ability to form rational beliefs is \emph{productive}: on the basis of finite learning, we are able to form rational beliefs under a potential infinity of novel circumstances.
%     The only plausible explanation for this is that we have, somehow, internalized a rule that tells us, in some general way, what it would be rational to believe under varying epistemic circumstances.%
%     \mbox{ }\hfill\mbox{(\citeyear[483]{Boghossian:2008vf})}
%   \end{quote}

%   Strictly, \citeauthor{Boghossian:2008vf}, rules \textquote{represent our conception of how it would be most rational for a thinker to form beliefs under different epistemic circumstances} (\citeyear[473]{Boghossian:2008vf}).

%   The difference in approach is clearest with \citeauthor{Boghossian:2008vf}'s account of modus ponens:%
%   \footnote{
%     \citeauthor{Boghossian:2008vf} notes the rule is distinct from modus pones as found in textbooks.
%     Remarks: \textquote{It is actually quite mysterious what the logic textbook rule is supposed to be} (\citeyear[472,fn.2]{Boghossian:2008vf})
%     I don't think there is any mystery about the rule in most logic textbooks.
%     Instead, the mystery is the way in which logic relates to reasoning.
%     (Cf.~\cite{Harman:1986ux,MacFarlane:2004aa,Steinberger:2022aa}, etc.)
%     % Issue for the presentation.
%     % Literature is full of issues.
%     % The most well known, Gricean pragmatics.
%     % Though, also McGee, McFarlane, sweet conditonals, the miners paradox, etc.
%   }

%   \begin{quote}
%     (Modus Ponens):
%     If you are rationally permitted to believe both that \emph{p} and that `If \emph{p}, then \emph{q}', then, you are prima facie rationally permitted to believe that \emph{q}.%
%     \mbox{ }\hfill\mbox{(\citeyear[472]{Boghossian:2008vf})}
%   \end{quote}

%   Here, we have permissions.
%   What the agent is allowed to do.
%   However, this is distinct from what the agent does.
% \end{note}

% \begin{note}
%   \tR{} is distinct.
%   Whether came to \emph{q} from \emph{p} , if \emph{p} then \emph{q}.

%   Rationality is not part of our understanding.
%   Rather, generality.%
%   \footnote{
%     Observe, ~\cite{Kolodny:2005aa} is of no interest here.
%     Why be rational is distinct from whether there is some generality.
%   }
% \end{note}

% \begin{note}
%   Likewise, means-end reasoning is distinct from \citeauthor{Broome:2013aa}'s

%   \begin{quote}
%     \emph{End to Means Transmission}.
%     ((\emph{S} requires of \emph{N} that \emph{p}) \& necessarily \newline (\emph{p} \(\supset\) \emph{q}) \& \emph{q} is a means to \emph{p}) \(\supset\) (\emph{S} requires of \emph{N} that \emph{q}).%
%     \mbox{ }\hfill\mbox{(\citeyear[126]{Broome:2013aa})}
%   \end{quote}

%   \emph{S} is some source, such as morality.
%   \emph{N} is a person. (\citeyear[117]{Broome:2013aa})

%   Instead, the significantly weaker idea that the agent has reasoned from some end to a means to that end.
% \end{note}

% \begin{note}
%   On my understanding, this is, in part, the role of \citeauthor{Boghossian:2014aa}'s Taking Condition.

%   Way in which \dots

%   Indeed, \citeauthor{Boghossian:2014aa} highlights how condition allow to draw distinction between deductive and inductive.
%   With taking, get generality.

%   Indeed, \textcite{Boghossian:2014aa} is structured so that Taking is a generalisation of rule.
% \end{note}

% \begin{note}
%   However, \tor{} does not need to amount to a rule.
%   Rather, \tR{} only requires the rough phenomenon that \citeauthor{Boghossian:2008vf} argues rule following is the only plausible explanation of.%
%   \footnote{
%     Our interest with \tor{1} is independent of the worries about rule following raised by~\textcite{Kripke:1982aa}, to the extent that the worries raised by~\citeauthor{Kripke:1982aa} concern \emph{which} rule an agent is following, rather than \emph{whether} the agent is following a rule.
%     At interest is not whether the \tor{} corresponds to plus or quus, but whether the agent's reasoning is of some type.
%   }
% \end{note}

% \begin{note}
%   Same for modus ponens.

%   \citeauthor{Davies:2004aa} discussing~\textcite{Wright:2004aa} with respect to~\citeauthor{Moore:1959aa}'s proof of an external world (\citeyear{Moore:1959aa}):

%   \begin{quote}
%     Moore's argument can be set out as follows:
%     \begin{quote}
%       \begin{enumerate}[label=MOORE (\Roman*), ref=MOORE (\Roman*)]
%       \item
%         \label{MoorePoEW:1}
%         I am having an experience as of one hand [here] and another [here].
%       \item
%         \label{MoorePoEW:2}
%         I have hands.

%         If I have hands then an external world exists.
%       \end{enumerate}

%       Therefore:

%       \begin{enumerate}[label=MOORE (\Roman*), ref=MOORE (\Roman*), resume]
%       \item
%         \label{MoorePoEW:3}
%         An external world exists.
%       \end{enumerate}
%     \end{quote}

%     [\dots] the key question at this point in Wright's account is whether the support for~\ref{MoorePoEW:2} is transmitted to~\ref{MoorePoEW:3} across the modus ponens inference in which the conditional premise is supported by an elementary piece of philosophical theorising.\newline
%     \mbox{ }\hfill\mbox{(\citeyear[215]{Davies:2004aa})}
%   \end{quote}
% \end{note}



\begin{note}
  Or, whether properly based.%
  \footnote{
    \citeauthor{Schaffer:2010vq}'s (\citeyear{Schaffer:2010vq}) Debasing demon.

    The debasing demon \textquote{throws her victims into the belief state on an improper basis, while leaving them with the impression as if they had proceeded properly}. (\citeyear[231]{Schaffer:2010vq})

    (However, see \textcite{Bondy:2018tk} for ways in which the \citeauthor{Schaffer:2010vq}'s demon fails.)
  }
\end{note}


\begin{note}
  In turn, \autoref{sec:typicalRequs} links the sufficient conditions to the idea of a \requ{} introduced in \autoref{cha:requs}.
  As a result, our argument for the failure of \issueConstraint{} will primarily turn on whether there are \scen{1} in which an agent is \tCV{} (as understood by \autoref{idea:tC}).
\end{note}



%%% Local Variables:
%%% mode: latex
%%% TeX-master: "master"
%%% TeX-engine: luatex
%%% End:
