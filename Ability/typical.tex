\chapter{\tC{2}}
\label{cha:typical}
\nocite{Wilson:1994aa}

\begin{note}
  This chapter, \tC{}.

  Function in relation to counterexamples to \issueConstraint{} is conclusions.
  Relevant conclusion is the result of \tC{} --- or, \tR{}.
  That is, `part' of the relevant phenomena.

  Two goals.

  Motivate \tC{}.
  Motivate a necessary condition on \tC{}.
\end{note}

\begin{note}
  \begin{TOCEnum}
  \item
    \TOCLine{cha:typical:sec:g-s}

    General idea.
  \item
    \TOCLine{cha:typical:laws}

    Key idea.
  \item
    \TOCLine{cha:typical:sec:tR}

    Refinement of key idea.
  \item
    \TOCLine{cha:typical:sec:tC}

    Variation of key idea with respect to conclusions.
    Here, link \tR{} to \fc{1}.
  \end{TOCEnum}
\end{note}

\section{A type of reasoning}
\label{cha:typical:sec:g-s}

\begin{note}
  We begin with three \scen{0}.
  Each \scen{0} intuitively involves some type of reasoning.
  The \scen{0} may not be the clearest instance of such reasoning.
  The point, though, is to observe the way in which a type of reasoning may be identified between \scen{1} without those \scen{1} being designed to instantiate the type.
\end{note}

\begin{note}
  \begin{scenario}[Eeyore Has a Birthday]
    \indent ``I'm giving this to Eeyore,'' he explained, ``as a present.
    What are you going to give?''

    ``Couldn't I give it too?'' said Piglet.
    ``From both of us?''

    ``No,'' said Pooh.
    ``That would not be a good plan.''

    ``All right, then, I'll give him a balloon.
    I've got one left from my party.
    I'll go and get it now, shall I?''

    ``That, Piglet, is a very good idea.
    It is just what Eeyore wants to cheer him up.
    Nobody can be uncheered with a balloon.''%
    \mbox{ }\hfill\mbox{(\cite[78--79]{Milne:2009aa})}\newline
    \mbox{ }
  \end{scenario}

  \begin{scenario}[The Cicada and the Fox]
    A cicada was singing on top of a tall tree.
    The fox wanted to eat the cicada, so she came up with a trick.
    She stood in front of the tree and marvelled at the cicada's beautiful song.
    The fox then asked the cicada to come down and show himself, since the fox wanted to see how such a tiny creature could be endowed with such a sonorous voice.
    But the cicada saw through the fox's trick.
    He tore a leaf from the tree and let it fall to the ground.
    Thinking it was the cicada, the fox pounced and the cicada then said,
    `Hey, you must be crazy to think I would come down from here! I've been on my guard against foxes ever since I saw the wings of a cicada in the spoor of a fox.'%
    \mbox{ }\hfill\mbox{(\cite[136]{Aesop:2002aa})}\newline
    \mbox{ }
  \end{scenario}

  \begin{scenario}[Humpty Dumpty]
    `I'm afraid I ca'n't quite remember it,'
    Alice said, very politely.

    \noindent%
    `In that case we start afresh,'
    said Humpty Dumpty,
    `and it's my turn to choose a subject---'
    (`He talks about it just as if it was a game!' thought Alice.)
    `So here's a question for you. How old did you say you were?'

    \noindent%
    Alice made a short calculation, and said
    `Seven years and six months.'

    \noindent%
    `Wrong!'
    Humpty Dumpty exclaimed triumphantly.
    `You never said a word like it!'

    \noindent%
    `I thought you meant ``How old \emph{are} you?'''
    Alice explained.

    \noindent%
    `If I'd meant that, I'd have said it,'
    said Humpty Dumpty.

    \noindent%
    Alice didn't want to begin another argument, so she said nothing.\newline
    \mbox{ }\hfill\mbox{(\cite[188]{Carroll:2009aa})}\newline
    \mbox{ }
  \end{scenario}
\end{note}

\begin{note}
  Intuitively, the type is \emph{means-end} reasoning.%
  \footnote{
    Not designed to use `type' in the sense of type-token distinction.
    Common characteristic.
    Though, type distinct from instances.
  }
  Consider \citeauthor{Pollock:2002aa}'s characterisation:

  \begin{quote}
    Means-end reasoning is concerned with finding the means for achieving goals.
    [\dots]
    To achieve a goal, we consider an action that would achieve it under some specified circumstances and then try to find a way of putting ourselves in those circumstances in order to achieve the goal by doing the action.%
    \mbox{ }\hfill\mbox{(\citeyear[60]{Pollock:2002aa})}
  \end{quote}

  \noindent%
  Taking each \scen{0} in turn:
  \begin{itemize}
  \item
    Piglet reasons giving a balloon to Eeyore as a present is a means to the end of cheering up Eeyore.
  \item
    The fox reasons flattering the cicada is a means to eating the cicada.
  \item
    Alice reasons responding with `seven years and six months' is a means to answering Humpty Dumpty's question `how old did you say you were?'.
    And, Alice also reasons saying nothing is a means to avoid beginning another argument.
  \end{itemize}

  \noindent%
  In these respects, each \scen{0} involves means-end reasoning.
\end{note}

\begin{note}
  Further, the \scen{1} intuitively involve reasoning \emph{about} means-end~reasoning:

  \begin{itemize}
  \item
    Pooh reasons the conclusion of Piglet's means-end reasoning is very good.
  \item
    The cicada reasons the flattery of the fox is the result of the fox reasoning about means to the end of eating the cicada.
  \item
    And, Alice reasoned about what Humpty Dumpty's question `how old did you say you were?' was a means to.
    Though, Alice concluded the question was a means to learning Alice's age, while Humpty Dumpty's end was Alice repeating their previous answer.
  \end{itemize}

  \noindent%
  So, means-end reasoning is not the only type of reasoning in the \scen{1}.
  And, given some patience, further types of reasoning may be identified.
  For example, each \scen{0} involves two agent talking to one another, and so each \scen{0} (plausibly) involves reasoning by an agent about the ideas expressions convey, and so on.%
  \footnote{
    Room for skepticism.

  For example, arbitrary.
  Nothing that prevents the possibility.
  However, plausible.
  Perhaps these \scen{1} just happened.

  Or, a deviant causal chains. (\cite[cf.][79]{Davidson:1973vd})

  However, interest is in type, not defending a particular instance.
  So long as it is fine that are instances and type, then good.
  }
\end{note}

\begin{note}
  Our interest is in giving a necessary condition for an instance of reasoning to be an \emph{instance of a type} of reasoning.
\end{note}

\section{Types and laws}
\label{cha:typical:laws}

\begin{note}
  Our interest is in giving a necessary condition for an instance of reasoning to be an \emph{instance of a type} of reasoning.

  I think \citeauthor{Pollock:2002aa}'s characterisation is a good account of means-end reasoning.
  And, examples of reasoning about means-end reasoning are suggestive.
  However, something which tells us reasoning is of type, where type may be arbitrary.
\end{note}

\begin{note}
  Idea pursue, \tRN{} is `lawful':%
  \footnote{
    \autoref{idea:tR-law} parallels \qWhyV{}.
    In particular, \citeauthor{Hempel:1965aa}'s Deductive-Nomological account of scientific explanation (\autopageref{mention:Hempel:1})
  }
  {
    \color{red}
    Intuition.
  }

  To do this, abstract characterisation of a \tor{}.
  Then, idea which captures `lawfulness'.

  \autoref{cha:typical:sec:tR} narrows the idea further with definition.
\end{note}

\subsection{\tor{3}}
\label{cha:typical:sec:tR:ToRdef}

\begin{note}
  \tor{3} are defined in terms of propositions, values, and \pool{1}:

  \begin{definition}[\tor{2}]
    \label{def:tor}
    \mbox{ }
    \vspace{-\baselineskip}
    \begin{itemize}
    \item
      \(T\) is a \emph{\tor{0}}.
    \end{itemize}

    \emph{If and only if}:

    \begin{itemize}
    \item
      \(T\) is a collection of proposition-value-\pool{} pairings.
    \end{itemize}
    \vspace{-\baselineskip}
  \end{definition}

  Intuitively, think of a \tor{} (as defined) as the \emph{extension} of a \tor{}.
  No assumptions about reasoning.
  However, conclusions.
  Proposition-value pairs and \pool{1}.
  Interest is with conclusions, so narrow.%
  \footnote{
    Not events, as it may be the case that agent's reasoning is of type, even though there is no event in which the agent reasons.
    Reasoning, progressive.
    Need not be the case that agent reasons.

    For example, language comprehension.
    Novel sentences, even those which are never produced.
  }

  This is difficult.
  However, interest is not with any particular type.
  Only whether reasoning is of some type.

  As will be seen, types are flexible.
  Type of reasoning in general, type of reasoning for an agent, type of reasoning for an agent given particular circumstances, and so on.

  Further, necessary condition for \tR{}.
  This extensional approach is sufficient.
\end{note}

\begin{note}
  To \illu{1} with the basic case of modus ponens:

  \begin{center}
    \begin{tabular}{R{.45\textwidth} L{.45\textwidth}}
      \multicolumn{2}{c}{Proposition-value-\pool{} pairings in type `modus ponens'} \\
      \hline\hline
      Proposition-value pair & \pool{2} \\
      \hline
      \pv{Q}{\val{True}} & \pv{P}{\val{True}}, \pv{\prop{If }P\prop{ then }Q}{\val{True}}, \dots \\
    \end{tabular}
  \end{center}

  Where, \(P\), \(Q\).
  Likewise, in the following \(R\) are descriptions.

  \begin{center}
    \begin{tabular}{R{.45\textwidth} L{.45\textwidth}}
      \multicolumn{2}{c}{Proposition-value-\pool{} pairings not in type `modus ponens'} \\
      \hline\hline
      Proposition-value pair & \pool{2} \\
      \hline
      \pv{R}{\val{True}} & \pv{P}{\val{True}}, \pv{\prop{If }P\prop{ or }Q\prop{ then }R}{\val{True}}, \dots \\
      \pv{Q}{\val{False}} & \pv{P}{\val{True}}, \pv{\prop{If }P\prop{ then }Q}{\val{True}}, \dots \\
    \end{tabular}
  \end{center}

  Difficulty.
  As extension, no way to distinguish between two \tor{1} with the same extension.

  For example, modus ponens and hypothetical syllogism.

  However, our goal is a necessary condition.
  In line with idea, failure.
  So, if some circumstance, then neither modus ponens nor hypothetical syllogism.%
  \footnote{
    May distinguish between `quss' and `plus'. (\cite{Kripke:1982aa}.)
    However, to the extent this is justification, no interest.

    I think type is primarily descriptive.
    However, necessary.
    Therefore, no principled objection to individuating types by whether the agent is justified.
  }
\end{note}

\begin{note}
  For example, means-end reasoning.
  Some goal, reasons to an action.
  \pool{3} contains proposition-value pair which captures the goal, and proposition-value pair captures action.

  \begin{center}
    \begin{tabular}{R{.45\textwidth} L{.45\textwidth}}
      \multicolumn{2}{c}{Proposition-value-\pool{} pairings in type `means-end reasoning'} \\
      \hline\hline
      Proposition-value pair & \pool{2} \\
      \hline
      \pv{\prop{Give Eeyore a balloon}}{\val{Do}} & \pv{\prop{Cheer up Eeyore}}{\val{Want}}, \dots \\
      \pv{\prop{Flatter cicada}}{\val{Do}} & \pv{\prop{Eat cicada}}{\val{Want}}, \dots \\
      \pv{\prop{Respond `7 years and 6 months'}}{\val{Do}} & \pv{\prop{Answer Dumpty's question}}{\val{Want}}, \dots \\
    \end{tabular}
  \end{center}

  \begin{center}
    \begin{tabular}{R{.45\textwidth} L{.45\textwidth}}
      \multicolumn{2}{c}{Proposition-value-\pool{} pairings not in type `means-end reasoning'} \\
      \hline\hline
      Proposition-value pair & \pool{2} \\
      \hline
      \pv{Q}{\val{True}} & \pv{P}{\val{True}}, \pv{\prop{If }P\prop{ then }Q}{\val{True}}, \dots \\
    \end{tabular}
  \end{center}

  Fail, as these are not actions and goals.
  May be cases of actions and goals that are not means-end, but this is difficult.

  Though, type of reasoning \emph{for an agent}.

  Give joint present.
  Yes piglet, no Pooh.
\end{note}


\begin{note}
  With the definition of a \tor{} in hand, turn to a sketch of \tR{}.
  Following the sketch and \illu{0}, highlight the problem for the sketch.
  Then, work toward definition.
\end{note}

\subsection{Laws}
\label{sec:law}

\begin{note}
  \begin{idea}[\tRN{2} is lawful]
    \label{idea:tR-law}
    \cenLine{
      \begin{itemize*}[noitemsep, label=\(\circ\)]
      \item
        Agent: \vAgent{}
      \item
        Propositions: \(\phi\), \(\psi\)
      \item
        Values: \(v\), \(v'\)
      \item
        \pool{3}: \(\Phi\), \(\Psi\)
      \item
        Event: \(e\)
      \item
        \mbox{ }
      \end{itemize*}
    }

    \begin{itemize}
    \item[\emph{If}:]
      \(e\) is such that \vAgent{} is \emph{\tR{}} to \(\pv{\phi}{v}\) from \(\Phi\).
    \item[\emph{Then}:]
      For every event \(e'\):
      \begin{itemize}
      \item[\emph{If}:]
        \(e'\) is in a collection of events \(E\):
      \item[\emph{Then}:]
        For some \tI{} \(\pvp{\psi}{v'}{\Psi}\) of \(T\):
        \begin{itemize}
        \item
          \(e'\) is an event in which \vAgent{} is reasoning to \(\pv{\psi}{v'}\) from \(\Psi\).
        \end{itemize}
      \end{itemize}
    \end{itemize}
    \vspace{-\baselineskip}
  \end{idea}

  So, this does not specify what the law is.
  Only that the agent is reasoning.

  So, particular instance, just in case, for every event, instance of that type.

  Modals.

  Two.

  \begin{itemize}
  \item
    Allow failure to be reasoning.
  \item
    Allow failure of reasoning to be of type.
  \end{itemize}
\end{note}

\begin{note}
  Dispositions.

  Simple conditional analysis.

  \begin{quote}
    An object is disposed to \emph{M} when \emph{C} iff it would \emph{M} if it were the case that \emph{C}.\newline
    \mbox{ }\hfill\mbox{(\citeauthor[\S1.2]{Choi:2021wg})}
  \end{quote}

  Phrased in terms of a counterfactual.
  However:
  \begin{itemize}
  \item
    For any case:
    \begin{itemize}
    \item[\emph{If}:]
      Conditions \emph{C}\('\) obtain.
    \item[\emph{Then}:]
      The object \emph{M}s.
    \end{itemize}
  \end{itemize}

  Where, \emph{C}\('\) mirror \emph{C} but are not relative to world of evaluation.%
  \footnote{
    Familiar analysis of knowledge:

    \begin{quote}
      S knows P \emph{if and only if}
      \begin{enumerate*}[label=\roman*., ref=(\roman*)]
      \item
        \label{K:jtb:t}
        P is true,
      \item
        \label{K:jtb:b}
        S believes P, and
      \item
        \label{K:jtb:j}
        S' belief is justified.
      \end{enumerate*}%
    \end{quote}

    \ref{K:jtb:t}, \ref{K:jtb:b} and \ref{K:jtb:j} conditions.
    If these conditions obtain, then S knows P.

    \citeauthor{Gettier:1963aa}'s (\citeyear{Gettier:1963aa}) counterexamples are \scen{1} which show the analysis is not sufficient.
  }

  Further, condition on whether the object has the disposition.

  \begin{itemize}
  \item
    \begin{itemize}
    \item[\emph{If}:]
      Object has disposition.
    \item[\emph{Then}:]
      For any case:
      \begin{itemize}
      \item[\emph{If}:]
        Conditions \emph{C}\('\) obtain.
      \item[\emph{Then}:]
        The object \emph{M}s.
      \end{itemize}
    \end{itemize}
  \end{itemize}

  So, example.
  Ball is bouncy.
  Drop, then bounces.
  Drop, does not bounce.
  So, not bouncy.

  Here, circumstances.
  Counterexamples to circumstances, but add in various other things.%
  \footnote{
    For example, Lewis, intrinsic properties.

  \begin{quote}
    An object \emph{x} is disposed to \emph{M} when \emph{C} iff \emph{x} has an intrinsic property \emph{B} such that, if it were the case that \emph{C}, and if \emph{x} were to retain \emph{B} for a sufficient time, then \emph{C} and \emph{B} would jointly cause \emph{x} to \emph{M}.%
    \mbox{ }\hfill\mbox{(\citeauthor[\S1.4]{Choi:2021wg})}
  \end{quote}
  }

  Any counterexample which is a counterexample to the specified conditions.%
  \footnote{
    As \citeauthor{Bonevac:2011tz} stress:
    \begin{quote}
    Counterexamples must be deployed as counterexamples to specific proposals.
    The example of a glass packed in styrofoam can perhaps show that fragile cannot be analysed as would break if struck, but it shows nothing about a proposed analysis of fragile as would break if struck when unwrapped, and certainly shows nothing about any proposed analysis of a different dispositional term, such as irascible.%
    \mbox{ }\hfill\mbox{(\citeyear[1144]{Bonevac:2011tz})}
  \end{quote}
  }

  \autoref{idea:tR-law} is neutral.
  Event from some collection.
  This is necessary.
  For, an agent is not reasoning unless an event is taking place.
  Difficulty is characterising collection.
  Though, this is not something we have any interest in doing.
\end{note}

\subsection{\illu{3}}
\label{sec:illu3-1}

\paragraph*{Selection tasks}
\nocite{Wason:1968aa}
\nocite{Wason:1971aa}

\begin{note}
  Wason selection task and other paradoxes.

  \begin{quote}
    The subjects (students) were presented with an array of cards and told that every card had a letter on one side and a number on the other side, and that either would be face upwards.
    They were then instructed to decide which cards they would need to turn over in order to determine whether the experimenter was lying in uttering the following statement:
    \emph{if a card has a vowel on one side then it has an even number on the other side}.%
    \mbox{ }\hfill\mbox{(\citeyear[145--146]{Wason:1966aa})}
  \end{quote}

  An example task is given in \autoref{fig:sectionTask}.%
  \footnote{
    The appropriate response is to turn over the third and fourth cards.

    Keep in mind, the conditional is false \emph{if and only if} a card has a vowel on one side and does not have an even number on the other side.
  }

  \begin{figure}[H]
    \centering
    \begin{tikzpicture}[
      cardnode/.style={
        rectangle,
        minimum width=10mm,
        minimum height=14mm,
        align=center,
        rounded corners,
        font = {\Large\sffamily},
        very thick,
      },
      node distance=5mm,
      ]

      \node[cardnode, draw] (1) {2};
      \node[cardnode, draw, right = of 1] (2) {N};
      \node[cardnode, draw, right = of 2] (3) {E};
      \node[cardnode, draw, right = of 3] (4) {1};
    \end{tikzpicture}
    \caption{A selection task}
    \label{fig:sectionTask}
  \end{figure}

  \citeauthor{Wason:1966aa} observes the results are consistent with the following hypothesis:
  \begin{quote}
    Subjects assume implicitly that a conditional statement has, not two truth values, but three: true, false and `irrelevant'.
    Vowels with even numbers verify, vowels with odd numbers falsify and consonants with any number are irrelevant.%
    \mbox{ }\hfill\mbox{(\citeyear[146]{Wason:1966aa})}
  \end{quote}

  As \citeauthor{Johnson-Laird:1969aa} summarise, Wason \textquote{argues that adults do not treat the conditional in a truth-functional manner: they consider it to be irrelevant when its antecedent is false} (\citeyear[367]{Johnson-Laird:1969aa}).
  Indeed, \citeauthor{Johnson-Laird:1969aa} additional results regarding how the (material) conditional is expressed.%
  \footnote{
    \citeauthor{Johnson-Laird:1969aa} also find the (material) conditional as expressed by the conditional is understood in line with \citeauthor{Wason:1966aa}'s hypothesis (and likewise if the contrapositive is expressed in conditional form).
    This time with detailed results of \emph{twenty four} University College London students!
    Specifically, 19 of the 24 responded as excepted given \citeauthor{Wason:1966aa}'s hypothesis.
    (\citeyear[369,370]{Johnson-Laird:1969aa}).
  }
\end{note}

\begin{note}
  So, we set up a law, and it applies.

  \begin{quote}
    If reason with conditionals in a truth-functional manner, then under circumstance when evaluating, conclude check:
    \begin{itemize}[noitemsep]
    \item
      Cards where antecedent is true.
    \item
      Cards where consequent is false.
    \end{itemize}
  \end{quote}

  From this, in an instance when someone infers q from p and if p then q, then it is not simply due to the truth functional reasoning regarding p and if p then q.

  The key part is the generalisation.
  It is intuitively clear an agent is not reasoning fails to be of some type if the agent fails to reason to an appropriate proposition-value pair.
  However, at issue is other instances of the agent's reasoning.

  In other words, the takeaway of results from \citeauthor{Wason:1966aa}'s selection task is with respect to conditional reasoning in general.
  Even if there are cases of truth-functional reasoning, something separates this from the reasoning found in selection tasks --- does not extend to all circumstances.
\end{note}

\paragraph*{Other}

\begin{note}
  \citeauthor{Allais:1979aa} paradox (\cite{Allais:1979aa}).
  The Ellsberg paradox (\cite{Ellsberg:1961aa}), \citeauthor{Makinson:1965aa}'s Paradox of the Preface (\citeyear{Makinson:1965aa}), \citeauthor{Kyburg:1997aa}'s Lottery Paradox (\citeyear{Kyburg:1997aa}), and so on.%
  \footnote{
    Consider also \citeauthor{Harman:1984aa}'s (\citeyear{Harman:1984aa,Harman:1986ux}) arguments against a strong connexion between logical principles and principles of belief revision.

  \begin{quote}
    Logical principles are not directly rules of belief revision.
    [\dots]
    Logical principles hold universally, without exception, whereas the corresponding principles of belief revision would be at best prima facie principles, which do not always hold.%
    \mbox{ }\hfill\mbox{(\citeyear[107--108]{Harman:1984aa})}
  \end{quote}

  \citeauthor{Harman:1984aa}'s point is that universality, so if hold, then hold in all circumstances.
  And, it is clear this is not the case.
  }
\end{note}

\paragraph*{Power}

\begin{note}
  \begin{illustration}[Textbook]
    \label{illu:tR:powers}
    Student has been studying algebra and has just been introduced to the rule of multiplication for powers (\(a^{n} \cdot a^{m} = a^{n + m}\)).

    A handful of exercises:%
    \footnote{
      From \textcite[31]{Gelfand:1993aa}.
    }

    \begin{quote}
      \begin{enumerate}[label=(\alph*), ref=(\alph*)]
      \item
        \label{mfp:a}
        You know that \(2^{1001} \cdot 2^{n} = 2^{2000}\).
        What is \(n\)?
      \item
        \label{mfp:b}
        You know that \(2^{1001} \cdot 2^{n} = \sfrac{1}{4}\).
        What is \(n\)?
      \item
        \label{mfp:c}
        Which is bigger: \(10^{-3}\) or \(2^{-10}\)?
      \item
        \label{mfp:d}
        You know that \(\sfrac{2^{1000}}{2^{n}} = 2^{501}\).
        What is \(n\)?
      \item
        \label{mfp:e}
        You know that \(\sfrac{2^{1000}}{2^{n}} = \sfrac{1}{16}\).
        What is \(n\)?
      \item
        \label{mfp:f}
        You know that \(4^{100} = 2^{n}\).
        What is \(n\)?
      \item
        \label{mfp:g}
        You know that \(2^{100} \cdot 3^{100} = a^{100}\).
        What is \(a\)?
      \item
        \label{mfp:h}
        You know that \((2^{10})^{15} = 2^{n}\).
        What is \(n\)?
      \end{enumerate}
    \end{quote}

    Student starts work on exercise~\ref{mfp:f}.

    Reasons from the problem to the proposition-value pair, \(\pv{n\prop{ is }200}{\val{True}}\).

    As the event in which the student reasoned to \(\pv{n\prop{ is }200}{\val{True}}\), it seems true that the student was reasoning to \(\pv{n\prop{ is }200}{\val{True}}\) from some \pool{} \(\Phi\) by a type of reasoning that involves the rule of multiplication for powers --- the student did not arbitrarily fix on \(n\) being \(200\).

    Granting that the student was reasoning to \(\pv{n\prop{ is }200}{\val{True}}\) from \(\Phi\), the agent may have stopped to being reasoning about a different problem.
    For example, the student may have stopped working on~\ref{mfp:f} and began to work on~\ref{mfp:a}.
    And, it seems that if the agent was reasoning to \(\pv{n\prop{ is }200}{\val{True}}\) from \(\Phi\), the agent would be reasoning to \(\pv{n\prop{ is }999}{\val{True}}\) from \(\Phi'\).

    Likewise for the other problems.
  \end{illustration}

  Suppose not, then it seems there is some misunderstanding regarding the rule of multiplication for powers.
\end{note}

\begin{note}
  Given available actions, no clear relation to the broad upshots of Wason, Allias, and \citeauthor{Harman:1984aa}.

  For the particular instances, fails to be the case that reasoning is of type, as agent's fail to reason, and hence \autoref{sketch:tR} fails when \(\pvp{\phi}{v}{\Phi}\) is the same as \(\pvp{\phi}{v}{\Phi}\).
  However, it does not follow that failure says anything about other instance of reasoning when the agent does not have the option to reason about a selection task, or Allias gambles.

  However, expected.
  Wason, results concern reasoning in general.
  Compatible with results that agents do, on occasion, consider conditionals as truth-functional, etc.
\end{note}

\section{\tR{2}}
\label{cha:typical:sec:tR}

\begin{note}
  \autoref{idea:tR-law} (\autopageref{idea:tR-law}), lawful.

  Role of \tR{} is to motivate counterexamples to \issueConstraint{}.
  We define \tR{} by narrowing \autoref{idea:tR-law} to particular instances of reasoning.

  Obtain necessary condition from two definitions.
\end{note}

\subsection{\ptR{2}}
\label{sec:ptr0}

\begin{note}
    \begin{definition}[\ptR{2}]
    \label{def:ptR}
    \cenLine{
      \begin{itemize*}[noitemsep, label=\(\circ\)]
      \item
        Agent: \vAgent{}
      \item
        Propositions: \(\phi\), \(\psi\)
      \item
        Values: \(v\), \(v'\)
      \item
        \pool{3}: \(\Phi\), \(\Psi\)
      \item
        \mbox{ }
      \end{itemize*}
    }

    \noindent%
    \cenLine{
      \begin{itemize*}[noitemsep, label=\(\circ\)]
      \item
        Event: \(e\)
      \item
        Type of reasoning: \(T\)
      \item
        \mbox{ }
      \end{itemize*}
    }

    \begin{itemize}
    \item
      \(e\) is an event in which \vAgent{} is \emph{\ptRV{0}} by type \(T\) to \(\pv{\phi}{v}\) from \(\Phi\).
    \end{itemize}

    \emph{If and only if}:

    \begin{itemize}[noitemsep]
    \item
      For every \tI{} \(\pvp{\psi}{v'}{\Psi}\) of \(T\).
      \begin{itemize}[noitemsep]
      \item[\emph{If}:]
        There is some available action \(a\) such that \vAgent{} is reasoning from \(\Psi\) when \vAgent{} does \(a\).
      \item[\emph{Then}:]
        There is some available action \(a'\) such that \vAgent{} is reasoning to \(\pv{\psi}{v'}\) from \(\Psi\) when \vAgent{} does \(a'\).
      \end{itemize}
    \end{itemize}
    \vspace{-\baselineskip}
  \end{definition}
\end{note}

\begin{note}
  \ptR{2}, simple law.

  Key is specification of conditions.
  If reasoning from some \pool{}, then get to a relevant conclusion.

  Here, make use of the progressive, as interest is with event, not with what happens.
  Though, ensure that we have enough from the actual to ensure truth of progressive.
\end{note}

\begin{note}
  The broad idea.

  If reasoning is of type \(T\), then, if the agent has the opportunity to reason from some premises associated with the type, then the agent may reason from the premises by the type of reasoning.
\end{note}

\begin{note}
  Limitation.

  Failure does not establish not of type.%
  \footnote{
    Point is similar to \citeauthor{Chomsky:2015aa}'s distinction between competence and performance.

    \begin{quote}
    Arithmetical competence yields the correct number z for every pair~(x,~y) under addition or multiplication.
    But only a small finite subpart of arithmetical competence can be exhibited without external aids (by calculating in one's head).
    Obviously, that fact does not imply that arithmetical competence is correspondingly limited.%
    \mbox{ }\hfill\mbox{(\citeyear[xii]{Chomsky:2015aa})}
  \end{quote}

    \citeauthor{Chomsky:2015aa} also motivates distinction by errors (\citeyear[2]{Chomsky:2015aa}).
    Unclear how errors relate to progressive.
    Present problems are sufficient, so we ignore.
  }

  For example, illustration.
  Here,~\ref{mfp:b},~\ref{mfp:d}, and~\ref{mfp:e} all involve fractions.
  However, agent is shaky on fractions.
  Rule of multiplication is good, but not with fractions.

  Issue is perhaps with the type.
  Perhaps~\ref{mfp:b},~\ref{mfp:d}, and~\ref{mfp:e} should not correspond to members of relevant type.

  However, with some imagination, available action that leads to difficulties.
  For example, \(2^{43 \cdot 54} \cdot 2^{53!}\).
  Agent needs not more than multiplication and addition, but progressive fails.
  Difficulty of solving multiplication in order to sum.
  And, perhaps, temptation to get a calculator.

  Still, reasoning by type in problems the agent does solve.
\end{note}

\subsection{\rotor{2}}
\label{sec:rotor}

\begin{note}
  To extend \ptR{} to necessary condition of \tR{}, \rotor{}.
\end{note}

\begin{note}
  \begin{definition}[A \rotor{}]
    \label{def:rotor}
    \cenLine{
      \begin{itemize*}[noitemsep, label=\(\circ\)]
      \item
        Agent: \vAgent{}
      \item
        Proposition: \(\phi\)
      \item
        Value: \(v\)
      \item
        \pool{2}: \(\Phi\)
      \item
        \mbox{ }
      \end{itemize*}
    }
    \cenLine{
      \begin{itemize*}[noitemsep, label=\(\circ\)]
      \item
        Event: \(e\)
      \item
        Type of reasoning: \(T\)
      \item
        \mbox{ }
      \end{itemize*}
    }

    \begin{itemize}
    \item
      \(T'\) is a \emph{\tRep{}} of \vAgent{} \tR{} by type \(T\) to \(\pv{\phi}{v}\) from \(\Phi\) in \(e\).
    \end{itemize}

    \emph{If and only if:}

    \begin{itemize}
    \item
      \begin{itemize}
      \item[\emph{If}:]
        \vAgent{} is \tRV{} by type \(T\) to \(\pv{\phi}{v}\) from \(\Phi\) in \(e\).
      \item[\emph{Then}:]
        \vAgent{} is \ptRV{} by type \(T'\) to \(\pv{\phi}{v}\) from \(\Phi\) in \(e\).
      \end{itemize}
    \end{itemize}
    \vspace{-\baselineskip}
  \end{definition}

  For purposes, \rotor{} is abstract characterisation of relation between \tR{} and \ptR{}.

  Necessary condition.

  Leave whatever secures the truth of conditional by which \rotor{} is defined intuitive.
  Still, intuitive that there are instances where true.

  For example, \autoref{illu:tR:powers}.

  \tor{}, particular collection.

  Student starts working on \ref{mfp:a}.
  Then, \ref{mfp:d}.
  Etc.

  So, omit~\ref{mfp:b},~\ref{mfp:d}, and~\ref{mfp:e}.

  So, this type, \rotor{}.

  Here, clearly not sufficient.
  Various ways to get a handful of answers.


  Likewise, selection task.

  Here's the answer.

  In general, require more.
\end{note}

\begin{note}
  Contrapositive is key:

  \begin{itemize}
  \item
    \begin{itemize}
    \item[\emph{If}:]
      \vAgent{} is \emph{not} \ptRV{} by type \(T'\) to \(\pv{\phi}{v}\) from \(\Phi\) in \(e\).
    \item[\emph{Then}:]
      \vAgent{} is \emph{not} \tRV{} by type \(T\) to \(\pv{\phi}{v}\) from \(\Phi\) in \(e\).
    \end{itemize}
  \end{itemize}
\end{note}

\begin{note}
  The converse conditional is also interesting.
  Type \(T'\) is sufficient for type \(T\).
  However, the difficulty with this suggestion is whether the conditional.
  May need more than just extension in order for reasoning to be of type.
  And, only necessary conditions, given abstract about reasoning.
\end{note}

\subsection{\tR{2}}
\label{cha:typical:sec:tR:tRDef}

\begin{note}
  Necessary condition for \tR{}, \rotor{}:

  \begin{proposition}[\tR{2}]
    \label{def:cmptnc}
    \cenLine{
      \begin{itemize*}[noitemsep, label=\(\circ\)]
      \item
        Agent: \vAgent{}
      \item
        Propositions: \(\phi\), \(\psi\)
      \item
        Values: \(v\), \(v'\)
      \item
        \pool{3}: \(\Phi\), \(\Psi\)
      \item
        \mbox{ }
      \end{itemize*}
    }

    \noindent%
    \cenLine{
      \begin{itemize*}[noitemsep, label=\(\circ\)]
      \item
        Event: \(e\)
      \item
        Type of reasoning: \(T\)
      \item
        \mbox{ }
      \end{itemize*}
    }

    \begin{itemize}
    \item
      \(e\) is an event in which \vAgent{} is \emph{\tRV{0}} by type \(T\) to \(\pv{\phi}{v}\) from \(\Phi\).
    \end{itemize}

    \emph{Only if}:

    \begin{itemize}[noitemsep]
    \item
      For any type of reasoning \(T'\):
      \begin{itemize}
      \item[\emph{If}:]
        \(T'\) is a \tRep{} of \vAgent{} \tR{} by type \(T\) to \(\pv{\phi}{v}\) from \(\Phi\) in \(e\).
      \item[\emph{Then}:]
        \vAgent{} is \ptRV{} by type \(T'\) to \(\pv{\phi}{v}\) from \(\Phi\) in \(e\).
      \end{itemize}
    \end{itemize}
    \vspace{-\baselineskip}
  \end{proposition}

  \begin{argument}{def:cmptnc}
    Suppose.
    Consider arbitrary \tRep{}.
    Then, \ptR{}.
  \end{argument}
\end{note}


\begin{note}
  Necessary condition.
  So, it may be the case that there are no options.
  However, if there are, then these are of interest.
\end{note}

\begin{note}
  Slight issue, single instances is sufficient for failure.
  In this case, set \(T'\) to be \(\pvp{\phi}{v}{\Phi}\), and consider a parallel definition which requires failure of progressive to hold for some threshold of \tI{}.

  In part, appeal of \autoref{def:cmptnc} is that necessary condition identified requires non-empty \tRep{} to do any work.
  At issue is whether \autoref{idea:tR-law} holds.
  Whether there is a `lawful' connexion between reasoning of type and other circumstances.
\end{note}

\begin{note}
  \begin{illustration}[Translation]
    Book.
    List of chapters.
    Is the agent reading the chapter titles?

    \begin{itemize}
    \item
      \textquote{ソフト&ウェット} to \textquote{Soft and wet}
    \item
      \textquote{毎日が夏休み} to \textquote{Every day is summer vacation}
    \item
      \textquote{ザ・ワンダー・オブ・ユー(君の奇跡の愛)} to \textquote{The wonder of you (the wonder of your love)}
    \end{itemize}
    However, agent may fail to translate certain chapter titles such as:
    \begin{itemize}
    \item
      \textquote{整形外科医 – 羽伴毅先生}
    \item
      \textquote{清の時代の髪留め}
    \end{itemize}
  \end{illustration}

  Interesting case.

  For, doesn't clearly follow that needs to be the case the agent translates each chapter title.

  And, extending, it need not be the case that agent translates any other.
  For, each other chapter title falls outside sufficient core understanding.
  Still, \textquote{母と子} to \textquote{Mother and child}.
  If not, something is off.
\end{note}

\begin{note}
  \begin{observation}[Trivial \tRep{1}]%
    \label{obs:tR:trivialRep}%
    It may be the case that the only \rotor{} is type which contains \(\pv{\phi}{v}\), etc.
  \end{observation}

  \begin{motivation}{obs:tR:trivialRep}
    Exam.
    Consider any other question, then immediately return to present question.%
    \footnote{
      To go further, plausibly get to counterfactuals.

      For example, rather than action to the agent, go via prompting.
      I give question, you get answer.
    }
    Here, \tRep{} is just the question.
  \end{motivation}
\end{note}


\section{\tC{2}}
\label{cha:typical:sec:tC}

\begin{note}
  With reasoning in hand, expand our understanding of typicality to concluding:

  \begin{definition}[\tC{2}]
    \label{def:tC}
    \cenLine{
      \begin{itemize*}[noitemsep, label=\(\circ\)]
      \item
        Agent: \vAgent{}
      \item
        Propositions: \(\phi\), \(\psi\)
      \item
        Values: \(v\), \(v'\)
      \item
        \pool{3}: \(\Phi\), \(\Psi\)
      \item
        \mbox{ }
      \end{itemize*}
    }

    \noindent%
    \cenLine{
      \begin{itemize*}[noitemsep, label=\(\circ\)]
      \item
        Event: \(e\)
      \item
        Type of reasoning: \(T\)
      \item
        \mbox{ }
      \end{itemize*}
    }

    \begin{itemize}
    \item
      \(e\) is an event in which \vAgent{} is \emph{\tCV{0}} (by type \(T\)) \(\pv{\phi}{v}\) from \(\Phi\).
    \end{itemize}

    \emph{Only if}:

    \begin{itemize}[noitemsep]
    \item
      For the \tRep{0} \(T'\) of \(T\) with respect to \(e\):
      \begin{itemize}[noitemsep]
      \item
        For every pairing \(\pvp{\psi}{v'}{\Psi}\) in \(T'\).
        \begin{itemize}[noitemsep]
        \item[\emph{If}:]
          There is some available action \(a\) such that \vAgent{} is reasoning from \(\Psi\) when \vAgent{} does \(a\).
        \item[\emph{Then}:]
          There is some available action \(a'\) such that \vAgent{} is \underline{concluding} \(\pv{\phi}{v}\) from \(\Psi\) when \vAgent{} does \(a'\).
        \end{itemize}
      \end{itemize}
    \end{itemize}
    \vspace{-\baselineskip}
  \end{definition}

  Motivation for this is simple.
  \tR{} just looks at instance of reasoning.
  This is okay, but less interesting.
  For, if fail to do something of same type, then what is really going on?
\end{note}

\subsection{\tC{2} and \fc{1}}
\label{cha:typical:tC:fc}

\begin{note}
  Link \tC{} to \fc{1}.

  \begin{proposition}[\tC{2} and \fc{1}]
    \label{prop:tC-and-fc}
    \cenLine{
      \begin{itemize*}[noitemsep, label=\(\circ\)]
      \item
        Agent: \vAgent{}
      \item
        Propositions: \(\phi\), \(\psi\)
      \item
        Values: \(v\), \(v'\)
      \item
        \pool{3}: \(\Phi\), \(\Psi\)
      \item
        \mbox{ }
      \end{itemize*}
    }

    \noindent%
    \cenLine{
      \begin{itemize*}[noitemsep, label=\(\circ\)]
      \item
        Event: \(e\)
      \item
        Type of reasoning: \(T\)
      \item
        \mbox{ }
      \end{itemize*}
    }

    \begin{itemize}
    \item[\emph{If}:]
      Clauses~\ref{prop:tC-and-fc:A:tC},~\ref{prop:tC-and-fc:A:rep},~\ref{prop:tC-and-fc:A:tI},~and~\ref{prop:tC-and-fc:A:novel} jointly hold:
      \begin{enumerate}[label=\arabic*., ref=(\arabic*)]
      \item
        \label{prop:tC-and-fc:A:tC}
        \(e\) is an event \vAgent{} is \tCV{} (by type \(T\)) \(\pv{\phi}{v}\) from \(\Phi\).
      \item
        \label{prop:tC-and-fc:A:rep}
        \(T'\) is representative of \(T\) with respect to \(e\).
      \item
        \label{prop:tC-and-fc:A:tI}
        \(\pvp{\psi}{v'}{\Psi}\) \tI{} of \(T'\)
      \item
        \label{prop:tC-and-fc:A:novel}
        No novel information.
      \end{enumerate}
    \item[\emph{Then}:]
      \(\pv{\phi}{v}\) is a \fc{} from \(\Phi\).
    \end{itemize}
    \vspace{-\baselineskip}
  \end{proposition}

  \begin{argument}{prop:tC-and-fc}
    Assume conditions.
    Get \fc{} by def.
  \end{argument}

  Mentioned strengthening \fc{1}.
  If do so, then need to strengthen \tC{}.
  Though, this seems entirely reasonable.
\end{note}


\section*{Summary}
% \label{cha:typical:sec:summ}

\begin{note}
  \tRN{2}.

  Extension account of \tor{}.
  Due to abstracting over theories.

  Then, necessary condition on \tR{}.
\end{note}

\begin{note}
  Only motivated \tR{} by intuition.
  Have not argued that this intuition is correct.
  \rotr{2}.
  Key piece, and intuitive.
\end{note}


% \section[\citeauthor{Carroll:1895uj}]{\citeauthor{Carroll:1895uj}\hfill(Optional)}

% \nocite{Black:1951aa}

% \begin{note}
%   The point here is that with Carroll, generality that goes beyond any single instance.
%   Must apply to all instances, to be valid.
%   But, cannot hope to cover all instances in a single move.
% \end{note}

% \begin{note}
%   A difficulty found on a reading of \citeauthor{Carroll:1895uj}'s \citetitle{Carroll:1895uj}.
% \end{note}

% \begin{note}
%   \begin{quote}
%     ``Plenty of blank leaves, I see!'' the Tortoise cheerily remarked.
%     ``We shall need them \emph{all}!''
%     (Achilles shuddered.)
%     ``Now write as I dictate:---

%     \begin{enumerate}[label=(\emph{\Alph*}), ref=\emph{\Alph*}]
%     \item
%       \label{AatT:a}
%       Things that are equal to the same are equal to each other.
%     \item
%       \label{AatT:b}
%       The two sides of this Triangle are things that are equal to the same.
%     \item
%       \label{AatT:c}
%       If~\ref{AatT:a} and~\ref{AatT:b} are true,~\ref{AatT:z} must be true.
%       \setcounter{enumi}{25}
%     \item
%       \label{AatT:z}
%       The two sides of this Triangle are equal to each other.''
%     \end{enumerate}

%     ``You should call it~\ref{AatT:d}, not~\ref{AatT:z},'' said Achilles.
%     ``It comes \emph{next} to the other three.
%     If you accept~\ref{AatT:a} and~\ref{AatT:b} and~\ref{AatT:c}, you \emph{must} accept~\ref{AatT:z}.''

%     ``And why \emph{must} I?''

%     ``Because it follows \emph{logically} from them.
%     If~\ref{AatT:a} and~\ref{AatT:b} and~\ref{AatT:c} are true,~\ref{AatT:z} \emph{must} be true.
%     You don't dispute \emph{that}, I imagine?''

%     ``If~\ref{AatT:a} and~\ref{AatT:b} and~\ref{AatT:c} are true,~\ref{AatT:z} \emph{must} be true,'' the Tortoise thoughtfully repeated.
%     ``That's \emph{another} Hypothetical, isn't it?
%     And, if I failed to see its truth, I might accept~\ref{AatT:a} and~\ref{AatT:b} and~\ref{AatT:c}, and \emph{still} not accept~\ref{AatT:z}, mightn't I ?''

%     \mbox{}\hfill\(\vdots\)\hfill\mbox{}

%     ``Then Logic would take you by the throat, and force you to do it!''
%     Achilles triumphantly replied.
%     ``Logic would tell you 'You ca'n't help yourself.''%
%     \mbox{ }\hfill\mbox{(\citeyear[279--280]{Carroll:1895uj})}
%   \end{quote}

%   The Tortoise has written down three premises,~\ref{AatT:a},~\ref{AatT:b}, and~\ref{AatT:c}.
%   Achilles holds that~\ref{AatT:z} follows from~\ref{AatT:a},~\ref{AatT:b}, and~\ref{AatT:c}.
%   The Tortoise observes they have the possibility of refraining to accept~\ref{AatT:z} follows from~\ref{AatT:a},~\ref{AatT:b}, and~\ref{AatT:c}.
%   And (initially), the Tortoise does not accept~\ref{AatT:z} follows from~\ref{AatT:a},~\ref{AatT:b}, and~\ref{AatT:c}.
%   Achilles requests the Tortoise accepts that~\ref{AatT:z} follows from~\ref{AatT:a},~\ref{AatT:b}, and~\ref{AatT:c}, and the Tortoise complies.
%   Specifically, the Tortoise grants:

%   \begin{quote}
%     \begin{enumerate}[label=(\emph{\Alph*}), ref=\emph{\Alph*}]
%       \setcounter{enumi}{3}
%     \item
%       \label{AatT:d}
%       If~\ref{AatT:a} and~\ref{AatT:b} and~\ref{AatT:c} are true,~\ref{AatT:z} must be true.%
%       \mbox{ }\hfill\mbox{(\citeyear[279]{Carroll:1895uj})}
%     \end{enumerate}
%   \end{quote}

%   But, does not accept~\ref{AatT:z} follows from~\ref{AatT:a},~\ref{AatT:b},~\ref{AatT:c}, and~\ref{AatT:d}.
% \end{note}

% \begin{note}
%   Modus ponens.

%   \begin{quote}
%     From \(\phi\) and \emph{if} \(\phi\) then \(\psi\), infer \(\psi\).
%   \end{quote}

%   Modus ponens is general.
%   For \emph{any} \(\phi\), \(\psi\).

%   Now, there is a difference between \emph{modus ponens} and conditional.

%   However, take any instance.
%   Then, if \(P\), \(P \rightarrow Q\), \(Q\) must be true.
%   But, then this means that the conditional is true.

%   Consequence of the deduction theorem.

%   Likewise, deduction theorem goes the other way.

%   However, going from \(P\), \(P \rightarrow Q\) to \(Q\) need not be an instance of \emph{modus ponens}.
% \end{note}

% \begin{note}
%   Well, this is a headache.
%   \citeauthor{Carroll:1895uj} is talking about a specific A, B, and Z.
%   There is no clear generality.
% \end{note}

% \begin{note}
%   So, consider at issue is modus ponens.
%   For any specific instance accept, there is a further instance.
%   For, \(A, (A \rightarrow B) \vDash B\).
%   Then, \(\vDash (A \land (A \rightarrow B) \rightarrow B)\).
%   However, now, \(A \land (A \rightarrow B), (A \land (A \rightarrow B) \rightarrow B) \vDash B\).
%   And, so on.

%   The general pattern, get conditional, but then this gives a new instance of modus ponens, which must be true in order for modus ponens to be valid rule of inference.

%   \citeauthor{Carroll:1895uj}, by contrast, starts with \(A \vDash B\).
%   This is different.
%   However, rather than focusing on a single rule of inference, the puzzle turns on what validity amounts to.

%   Validity is a general thing, with specific instances.
%   However, grant any particular instance of validity without employing validity in general.
% \end{note}

% \begin{note}
%   \begin{quote}
%     My paradox \dots turns on the fact that, in a Hypothetical, the \emph{truth} of the Protasis, the \emph{truth} of the Apodosis, and the \emph{validity of the sequence}, are 3 distinct Propositions.

%     \mbox{}\hfill\(\vdots\)\hfill\mbox{}

%     Suppose I say ``I grant~\ref{AatT:a} and~\ref{AatT:b} and~\ref{AatT:c}, but I do \emph{not} grant that I am thereby \emph{obliged} to grant~\ref{AatT:z}.''
%     Surely, my granting~\ref{AatT:z} must \emph{wait} until I have been made to see the validity of this sequence: i.e.\ in order to grant~\ref{AatT:z}, I must grant~\ref{AatT:a},~\ref{AatT:b},~\ref{AatT:c}, and~\ref{AatT:d}! And so on.%
%     \mbox{ }\hfill\mbox{(\citeyear[472]{Carroll:1977wl})}
%   \end{quote}

%   My interpretation of the point \citeauthor{Carroll:1895uj} makes in this passage is that the truth of A B and the truth of C is distinct from the validity of A B C.
%   Granting is substantial, not merely moving.
%   But, in order to grant, this means granting all other cases.

%   So, the paradox is that, on the one hand, don't need validity for any specific true things.
%   But, on the other hand, only of interest if via validity.

%   The Tortoise is slowly working through each instance, but this has no hope of getting the Tortoise to general validity.
%   So, how does the Tortoise ever make it there?
% \end{note}

% \begin{note}
%   This point differs from received interpretation.

%   \citeauthor{Wieland:2013vf} (\citeyear{Wieland:2013vf}) characterises the general understanding of \textcite{Carroll:1895uj} in terms of two lessons:
%   \begin{quote}
%     [T]he negative lesson is that if you add ever more premises to an argument \dots, then you will never demonstrate that its conclusion follows logically.\newline
%     \mbox{ }\hfill\mbox{(\citeyear[984]{Wieland:2013vf})}
%   \end{quote}

%   Parallel, static answers, still option for concluding otherwise.

%   \begin{quote}
%     [T]he positive lesson is that rules of inference, rather than premises of the form `if premises such and such are true, then the conclusion is true', will do the job.\newline
%     \mbox{ }\hfill\mbox{(\citeyear[984]{Wieland:2013vf})}
%   \end{quote}

%   \begin{quote}
%     [\citeauthor{Carroll:1895uj}] simply lacked any distinct conception of a deduction as opposed to the assertion (``granting'' of) a hypothetical proposition.
%     \dots
%     Any attempt by Carroll to tackle the question of inference was bound to begin in confusion and end in constipation-all those premises piling up, but no motion.
%   \end{quote}
% \end{note}

% \paragraph{The Dichotomy}

% \begin{note}
%   Achilles and the Tortoise, Zeno's argument.

%   Surely, right?

%   Two ways to understand.
%   Does the Tortoise move at all, or does the Tortoise arrive at the end?
%   I mean, as formulated by Zeno, it's about catching up, no matter how much one moves.

%   It is different from Zeno's Dichotomy paradox.


%   If so, then we should expect the Tortoise to be making some movement.
%   Adding rules of inference is of no help, because the problem is not movement, it's about how to move so much in a single step.
% \end{note}

% \begin{note}
%   \color{red}
%   Something about logic forcing.
%   The Tortoise hasn't arrived.

%   Nothing hangs on validity.
%   Same issue with testimony.
%   `A'.
%   Why?
%   Testified A, so A.
%   Okay, but another instance of testimony.
%   Testified(Testified A, so A), so Testified A, so A.
% \end{note}

% \begin{note}
%   \begin{quote}
%     But if we who wish to represent his belief in Q as based on P are to write in our notebook everything his having that belief on that basis consists in then when we have written only P and Q we will not have written enough.
%     Someone can believe P and believe Q and still not believe Q on the basis of P whatever the relations between the propositions P and Q happen to be.
%     He might believe Q for some reason completely unconnected with P, or perhaps for no reason at all (if that is possible).%
%     \mbox{ }\hfill\mbox{(\citeyear[185]{Stroud:1979aa})}
%   \end{quote}
%   However, the moral drawn is narrow
%   \begin{quote}
%     The moral is that for every proposition or set of propositions the belief or acceptance of which is involved in someone's believing one proposition on the basis of another there must be something else, not simply a further proposition accepted, that is responsible for the one belief's being based on the other.%
%     \mbox{ }\hfill\mbox{(\citeyear[187]{Stroud:1979aa})}
%   \end{quote}

%   Even if we grant each individual is \ros{}, rather than an instance of the material conditional, \emph{logic} hasn't done anything.
% \end{note}

% \paragraph{General and specific: Contrast}

% \begin{note}
%   Use \citeauthor{Carroll:1895uj} to illustrate this point.

%   However, given the worry, various other things may be understood this way.

%   Hume, constant conjunction.
%   Part of the problem is identifying cause.
%   We get the famous line about observing.
%   However, Hume goes on.
%   It's not only no observation, but no generality.

%   Right, so more narrow than Hume.
%   Because, with Hume, at issue is whether we have grounds for this general thing.
%   With Carroll, it's whether we even really get to the general thing.
% \end{note}


%%% Local Variables:
%%% mode: latex
%%% TeX-master: "master"
%%% End:


% \begin{note}
%   \begin{quote}
%     Let me ask this: what has the expression of a rule—say a sign-post—got to do with my actions?
%     What sort of connexion is there here?%
%     ---%
%     Well, perhaps this one:
%     I have been trained to react to this sign in a particular way, and now I do so react to it.

%     But that is only to give a causal connexion; to tell how it has come about that we now go by the sign-post; not what this going-by-the sign really consists in.
%     On the contrary; I have further indicated that a person goes by a sign-post only in so far as there exists a regular use of sign-posts, a custom.%
%     \mbox{ }\hfill\mbox{(\citeyear[\S198]{Wittgenstein:1958aa})}
%   \end{quote}

%   Regular use of sign-posts, custom.

%   Ugh, this is ambiguous.
% \end{note}


% %
%   \(^{,}\)
%   \footnote{
%     \citeauthor{Hlobil:2014tq}'s ``Inferential Moorean Phenomenon'':
%   \begin{quote}
%     \begin{enumerate}
%     \item[(IMP)]
%       It is either impossible or seriously irrational to infer \emph{P} from \emph{Q} and to judge, at the same time, that the inference from \emph{Q} to \emph{P} is not a good inference.
%     \end{enumerate}
%     \dots
%     By the ``goodness'' of an inference I mean the feature that makes the relevant inference permissible. Thus, if the inference under consideration is an inductive inference, the relevant kind of goodness is not deductive validity.%
%     \mbox{ }\hfill\mbox{(\citeyear[\S1]{Hlobil:2014tq})}
%   \end{quote}
%   Though, this really isn't more basic given the interest in \tR{}.
%   For, the puzzle is what it is to `infer'.

%   Rationality isn't part of the picture.
%   And, this is a significant drawback of \citeauthor{Hlobil:2014tq}'s approach.
% }


% \subsection{Types and explanation}
% \label{cha:typical:sec:tor:bkgd}

% \begin{note}
%   There is a related, stronger claim, that generality derives from rule following.

%   For this, \citeauthor{Boghossian:2008vf}:

%   \begin{quote}
%     [O]ur internalization of general epistemic rules---like Modus Ponens and Induction---explain and rationalize why we form the beliefs that we form.
%     And that seems intuitively correct.

%     As in the case of our linguistic and conceptual abilities, our ability to form rational beliefs is \emph{productive}: on the basis of finite learning, we are able to form rational beliefs under a potential infinity of novel circumstances.
%     The only plausible explanation for this is that we have, somehow, internalized a rule that tells us, in some general way, what it would be rational to believe under varying epistemic circumstances.%
%     \mbox{ }\hfill\mbox{(\citeyear[483]{Boghossian:2008vf})}
%   \end{quote}

%   Strictly, \citeauthor{Boghossian:2008vf}, rules \textquote{represent our conception of how it would be most rational for a thinker to form beliefs under different epistemic circumstances} (\citeyear[473]{Boghossian:2008vf}).

%   The difference in approach is clearest with \citeauthor{Boghossian:2008vf}'s account of modus ponens:%
%   \footnote{
%     \citeauthor{Boghossian:2008vf} notes the rule is distinct from modus pones as found in textbooks.
%     Remarks: \textquote{It is actually quite mysterious what the logic textbook rule is supposed to be} (\citeyear[472,fn.2]{Boghossian:2008vf})
%     I don't think there is any mystery about the rule in most logic textbooks.
%     Instead, the mystery is the way in which logic relates to reasoning.
%     (Cf.~\cite{Harman:1986ux,MacFarlane:2004aa,Steinberger:2022aa}, etc.)
%     % Issue for the presentation.
%     % Literature is full of issues.
%     % The most well known, Gricean pragmatics.
%     % Though, also McGee, McFarlane, biscuit conditonals, the miners paradox, etc.
%   }

%   \begin{quote}
%     (Modus Ponens):
%     If you are rationally permitted to believe both that \emph{p} and that `If \emph{p}, then \emph{q}', then, you are prima facie rationally permitted to believe that \emph{q}.%
%     \mbox{ }\hfill\mbox{(\citeyear[472]{Boghossian:2008vf})}
%   \end{quote}

%   Here, we have permissions.
%   What the agent is allowed to do.
%   However, this is distinct from what the agent does.
% \end{note}

% \begin{note}
%   \tR{} is distinct.
%   Whether came to \emph{q} from \emph{p} , if \emph{p} then \emph{q}.

%   Rationality is not part of our understanding.
%   Rather, generality.%
%   \footnote{
%     Observe, ~\cite{Kolodny:2005aa} is of no interest here.
%     Why be rational is distinct from whether there is some generality.
%   }
% \end{note}

% \begin{note}
%   Likewise, means-end reasoning is distinct from \citeauthor{Broome:2013aa}'s

%   \begin{quote}
%     \emph{End to Means Transmission}.
%     ((\emph{S} requires of \emph{N} that \emph{p}) \& necessarily \newline (\emph{p} \(\supset\) \emph{q}) \& \emph{q} is a means to \emph{p}) \(\supset\) (\emph{S} requires of \emph{N} that \emph{q}).%
%     \mbox{ }\hfill\mbox{(\citeyear[126]{Broome:2013aa})}
%   \end{quote}

%   \emph{S} is some source, such as morality.
%   \emph{N} is a person. (\citeyear[117]{Broome:2013aa})

%   Instead, the significantly weaker idea that the agent has reasoned from some end to a means to that end.
% \end{note}

% \begin{note}
%   On my understanding, this is, in part, the role of \citeauthor{Boghossian:2014aa}'s Taking Condition.

%   Way in which \dots

%   Indeed, \citeauthor{Boghossian:2014aa} highlights how condition allow to draw distinction between deductive and inductive.
%   With taking, get generality.

%   Indeed, \textcite{Boghossian:2014aa} is structured so that Taking is a generalisation of rule.
% \end{note}

% \begin{note}
%   However, \tor{} does not need to amount to a rule.
%   Rather, \tR{} only requires the rough phenomenon that \citeauthor{Boghossian:2008vf} argues rule following is the only plausible explanation of.%
%   \footnote{
%     Our interest with \tor{1} is independent of the worries about rule following raised by~\textcite{Kripke:1982aa}, to the extent that the worries raised by~\citeauthor{Kripke:1982aa} concern \emph{which} rule an agent is following, rather than \emph{whether} the agent is following a rule.
%     At interest is not whether the \tor{} corresponds to plus or quus, but whether the agent's reasoning is of some type.
%   }
% \end{note}

% \begin{note}
%   Same for modus ponens.

%   \citeauthor{Davies:2004aa} discussing~\textcite{Wright:2004aa} with respect to~\citeauthor{Moore:1959aa}'s proof of an external world (\citeyear{Moore:1959aa}):

%   \begin{quote}
%     Moore's argument can be set out as follows:
%     \begin{quote}
%       \begin{enumerate}[label=MOORE (\Roman*), ref=MOORE (\Roman*)]
%       \item
%         \label{MoorePoEW:1}
%         I am having an experience as of one hand [here] and another [here].
%       \item
%         \label{MoorePoEW:2}
%         I have hands.

%         If I have hands then an external world exists.
%       \end{enumerate}

%       Therefore:

%       \begin{enumerate}[label=MOORE (\Roman*), ref=MOORE (\Roman*), resume]
%       \item
%         \label{MoorePoEW:3}
%         An external world exists.
%       \end{enumerate}
%     \end{quote}

%     [\dots] the key question at this point in Wright's account is whether the support for~\ref{MoorePoEW:2} is transmitted to~\ref{MoorePoEW:3} across the modus ponens inference in which the conditional premise is supported by an elementary piece of philosophical theorising.\newline
%     \mbox{ }\hfill\mbox{(\citeyear[215]{Davies:2004aa})}
%   \end{quote}
% \end{note}

% \paragraph*{Preferences}

% \begin{note}
%   Type of reasoning isn't basic EU, etc.
% \end{note}

% \begin{note}
%   The situations, as presented by \citeauthor{Allais:1979aa}:%
%   \footnote{
%     The units in \citeauthor{Allais:1979aa}'s situations are French francs at 1952 prices (\citeauthor[138, fn.94]{Allais:1979aa}).\newline
%     100 French francs in 1952 is worth about the same as 50 US Dollars in 2022.
%   }
%   \begin{quote}
%     \begin{enumerate}[label=(\arabic*), ref=(\arabic*)]
%     \item
%       \emph{Do you prefer Situation A to Situation B}?

%       Situation A:
%         \begin{enumerate}[label=--]
%         \item
%           \emph{certainty} of receiving 100 million
%         \end{enumerate}
%         Situation B:
%         \begin{enumerate}[label=--]
%         \item
%           \emph{a} 10\% \emph{chance} of winning 500 million,
%         \item
%           \emph{an} 89\% \emph{chance} of winning 100 million,
%         \item
%           \emph{a} 1\% \emph{chance} of winning nothing.
%         \end{enumerate}
%       \item
%       \emph{Do you prefer Situation C to Situation D}?

%       Situation C:
%         \begin{enumerate}[label=--]
%         \item
%           \emph{a} 11\% \emph{chance} of winning 100 million,
%         \item
%           \emph{an} 89\% \emph{chance} of winning nothing.
%         \end{enumerate}
%         Situation D:
%         \begin{enumerate}[label=--]
%         \item
%           \emph{a} 10\% \emph{chance} of winning 500 million,
%         \item
%           \emph{a} 90\% \emph{chance} of winning nothing.%
%           \mbox{ }\hfill\mbox{(\citeyear[89]{Allais:1979aa})}
%         \end{enumerate}
%     \end{enumerate}
%   \end{quote}

%   \begin{quote}
%     The preference \(\text{A} > \text{B}\) should entail \(\text{C} > \text{D}\).%
%     \footnote{
%       First, write out expected utility for each situation.

%       \smallskip
%       \mbox{ }\hfill%
%       \EU{Sit.\ A} = \(\Util{100\text{m}}\)%
%       \hfill%
%       \EU{Sit.\ B} = \(.10 \cdot \Util{500\text{m}} + .89 \cdot \Util{100\text{m}} + .1 \cdot \Util{0\text{m}}\)%
%       \hfill\mbox{ }

%       \mbox{ }\hfill%
%       \EU{Sit.\ C} = \(.11 \cdot \Util{100\text{m}} + .89 \cdot \Util{0\text{m}}\)%
%       \hfill%
%       \EU{Sit.\ D} = \(.10 \cdot \Util{500\text{m}} + .90 \cdot \Util{0\text{m}}\)%
%       \hfill\mbox{ }
%       \smallskip

%       Suppose Situation A is preferred to Situation B.
%       Hence, \(\EU{Sit.\ A} > \EU{Sit.\ B}\).
%       Expanding, we obtain:
%       %
%       \[
%         \Util{100\text{m}} > .10 \cdot \Util{500\text{m}} + .89 \cdot \Util{100\text{m}} + .1 \cdot \Util{0\text{m}}
%       \]

%       Now, consider subtracting \(.89 \cdot \Util{100\text{m}}\) from each side of the inequality and adding \(.89 \cdot \Util{0\text{m}}\):
%       %
%       \begin{align*}
%         \Util{100\text{m}} - .89 \cdot \Util{100\text{m}} &> .10 \cdot \Util{500\text{m}} + .01 \cdot \Util{0\text{m}} \\
%          .11 \cdot \Util{100\text{m}} + .89 \cdot \Util{0\text{m}} &> .10 \cdot \Util{500\text{m}} + .90 \cdot \Util{0\text{m}}
%       \end{align*}

%       The left and right side of the inequality are \EU{Sit.\ C} and \EU{Sit.\ D}, respectively.
%       Therefore, it should be the case that Situation C is preferred to Situation D.
%     }
%   \end{quote}
%   However, this is not the case.
%   \citeauthor{Allais:1979aa} highlights pattern to the contrary:
%   \textquote{\emph{[T]he pattern for most highly prudent persons [\dots] who are considered generally as rational, is the pairing \(\text{A} > \text{B}\) and \(\text{C} < \text{D}\).}}
%   (\citeyear[89]{Allais:1979aa})

%   So, we have a something lawlike.%

%   Two things here.
%   \begin{itemize}
%   \item
%     Entailment between preferences.
%   \item
%     Circumstance in which conflicting preferences.
%   \end{itemize}

%   Both work, but the latter is of interest.%
%   \footnote{
%     \color{red}
%     The former \dots
%   }

%   For, if axioms, then preferences.
% \end{note}


\begin{note}
  Or, whether properly based.%
  \footnote{
    \citeauthor{Schaffer:2010vq}'s (\citeyear{Schaffer:2010vq}) Debasing demon.

    The debasing demon \textquote{throws her victims into the belief state on an improper basis, while leaving them with the impression as if they had proceeded properly}. (\citeyear[231]{Schaffer:2010vq})

    (However, see \textcite{Bondy:2018tk} for ways in which the \citeauthor{Schaffer:2010vq}'s demon fails.)
  }
\end{note}
