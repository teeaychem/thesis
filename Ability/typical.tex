\chapter{\tCV{3}}
\label{cha:typical}
\label{idea:tC}

\nocite{Wilson:1994aa}
\nocite{Goodman:1983aa}

\begin{note}
  Intuitively, conclusions are often of some type.
  For example, an agent concludes by modus ponens, arithmetic, or the categorical imperative, etc.
  Colloquially, we say an agent is \emph{\typeAdj{}} concluding some \prop{0}-\val{0} pair from some \pool{}.

  This chapter takes the idea of an agent \tCV{} for granted and highlights a feature of an agent \tCV{}.

  The key idea of an agent \tCV{} is to provide sufficient conditions for \fc{1} which do not need to be made prior to the \agents{} conclusion of \(\pv{\phi}{v}\) from \(\Phi\).
  For, though we have provided a few detailed examples of \fc{1}, these rely on the relevant (foregone-)conclusion being either the conclusion in progress (\autoref{obs:var:factor:fc}, \autopageref{obs:var:factor:fc}) or a necessary sub-conclusion of the conclusion in progress (\autoref{obs:LucasRequ}, \autopageref{obs:LucasRequ}).
\end{note}


\begin{note}
  We term the feature of interest a `\tpro{}' of a type of reasoning.
  Roughly, the \tpro{} of a \torN{} is a collection of descriptions and \prop{0}-\val{0}-\pool{0} pairings.
  And, if the agent is concluding by some \torNa{}, then for every description and associated \prop{0}-\val{0}-\pool{0} pairing in the \tpro{} of the type, for every event which satisfies the description (of which there must be at least one) the agent is concluding the relevant \prop{0}-\val{0} pair from the \pool{0}.
  Conversely, if there is no \eiw{} the description is true of, or the agent is not concluding the relevant \prop{0}-\val{0} pair from the \pool{0} under some event the description is true of, the agent is not concluding by the \torNa{}.
\end{note}

\begin{note}
  To give a brief example we return to in additional detail later, \citeauthor{Wason:1966aa} (\citeyear{Wason:1966aa}) argued the failure of agents to complete selection tasks in a particular way shows agents do generally reason about natural language conditionals as material conditionals.

  Key to \citeauthor{Wason:1966aa}s argument is the idea that whenever an agent is engaging in a selection task the agent is reasoning about the conditional included in the selection task as a material conditional \emph{only if} the agent selects appropriate cards.
  And, agents often do not select appropriate cards.

  Expressed in terms of \tpro{1}:
  The relevant descriptions capture engagement in a selection task.
  And, the associated \prop{0}-\val{0}-\pool{0} pairings capture expected results of selection tasks if agents reason by the material conditional.
  Failure to \emph{make} the relevant conclusion indicates the agent was \emph{making} the relevant conclusion.
  So, by the idea of a \tpro{} the agent was not reasoning by the material conditional.
\end{note}


\begin{note}
  The role of a \tprof{0} in the overall argument is to provide sufficient conditions for a \requ{}.
  The first section of this chapter introduces the idea of a \tprof{0}, and the second section uses the idea of a \tpro{0} to link an agent \tCN{} to \fc{1}.
\end{note}


\section{\tprof{3}}
\label{cha:typical:int}



\subsection{\torN{3}}
\label{cha:typical:tCDef:ToRdef}


\begin{note}
  Our overall interest is agent is concluding.
  A conclusion is often the result reasoning.
  And, some instances of reasoning are, intuitively, of a type.
  For example, \(T\) in the schema `type of reasoning \(T\)' may be substituted with `modus ponens', `means-end reasoning', `arithmetic', `consequentialism', and so on.

  We define a \torN{0} extensionally via \prop{0}-\val{0}-\pool{0} pairings:

  \begin{rdefinition}{def:tor}{A \torN{0}}
    \mbox{ }
    \vspace{-\baselineskip}
    \begin{itemize}
    \item
      \(T\) is a \torN{}.
    \end{itemize}

    \emph{If and only if}:

    \begin{itemize}
    \item
      \(T\) is a collection of \prop{0}-\val{0}-\pool{0} pairings.
    \end{itemize}
    \vspace{-\baselineskip}
  \end{rdefinition}

  \noindent%
  \autoref{def:tor} is motivated by the (minimal) assumptions we have made about conclusions.

  First, we assumed a conclusion is an \evalion{} of some \prop{0} having some \val{0} from some \pool{0}.
  So, for any reasoning associated with a conclusion there is always some \prop{0}-\val{0}-\pool{0} pairing associated with the reasoning --- the \prop{0} and \val{0} which are part of the \evalion{}, and the \pool{0} the \evalion{} is made from.
  Hence, for any \emph{intuitive} \torN{} one may associate the \prop{0}-\val{0}-\pool{0} pairs from any instance of the \torNa{} to obtain a corresponding \torN{} as defined.

  Second, we have made no other assumptions about what a conclusion amounts to.
  So, it is not possible to provide an account of a \torN{0} in any finer-grain without appealing to additional resources.%
  \footnote{
    Note, it need not be the case that the agent is following a rule as advocated for by, e.g.\ \textcite{Boghossian:2008vf,Boghossian:2012vb} and \textcite{Broome:2002aa,Broome:2013aa}.
    Though, a \torNa{} may be understood as the extension of some rule.
  }
\end{note}


\begin{note}
  An extensional definition of a \torN{0} may fail to capture an intuitive understanding of a \torN{0}.

  In particular, it may be possible for an agent to conclude \(\pv{\phi}{v}\) from \(\Phi\) by two sufficiently distinct ways.
  Or, a little more abstractly, two \torN{1} may be intensionally distinct by extensionally equivalent.%
  \footnote{
    Note, however, both the \prop{0}-\val{0} pair and \pool{} are fixed.
    So, though both \autoref{illu:gist:roots:F} and \autoref{illu:gist:roots:QF} involve a conclusion of \pv{\rootsCon{}}{\valI{True}}, these \scen{1} do not clearly amount to the same \torN{} by \autoref{def:tor}.
    For, \autoref{illu:gist:roots:F} involves an application factorisation, \autoref{illu:gist:roots:QF} involves an application of the quadratic formula, and details regarding factorisation and the quadratic formula are plausibly included in the relevant \pool{1}.
  }

  Still, a more ideal definition of a \torN{0} may be substituted in place of \autoref{def:tor} so long as the definition returns a collection of \prop{0}-\val{0}-\pool{0} pairings.
\end{note}


\begin{note}
  For a brief example of a \torN{}, consider factorisation (as in~\autoref{illu:gist:roots:F}).

  The following \prop{}-\val{}-\pool{} pairings are plausibly in the type `factorisation', where the ellipses are expanded to include \prop{0}-\val{0} pairs which capture the \agents{} understanding of factorisation:
  \begin{center}
    \begin{tabular}{R{.45\textwidth} L{.45\textwidth}}
      \pv{\propM{\rootsCon{}}}{\valI{True}} & \pv{\propM{\rootsConEq{}}}{\valI{True}}, \dots \\
      \pv{\propM{x \in \{4,2\}}}{\valI{True}} & \pv{\propM{x^{2} - 4x + 8 = 0}}{\valI{True}}, \dots \\
      \pv{\propM{x \in \{70,-2\}}}{\valI{True}} & \pv{\propM{x^{2} - 68x - 140 = 0}}{\valI{True}}, \dots \\
    \end{tabular}
  \end{center}
  %
  Likewise, the following \prop{0}-\val{}-\pool{} pairings are plausibly \emph{not} in the type `factorisation', where the ellipses may be expanded arbitrarily:

  \begin{center}
    \begin{tabular}{R{.45\textwidth} L{.45\textwidth}}
      \pv{\propM{\rootsConBad{}}}{\valI{True}} & \pv{\propM{\rootsConEq{}}}{\valI{True}}, \dots \\
      \pv{\propI{n is 2000}}{\valI{True}} & \pv{\propM{4^{100} = 2^{n}}}{\valI{True}}, \dots \\
      \pv{\propM{x \in \{70,-2\}}}{\valI{Want}} & \pv{\propM{x^{2} - 68x - 140 = 0}}{\valI{True}}, \dots \\
    \end{tabular}
  \end{center}

  \noindent%
  For, the first \prop{0}-\val{0}-\pool{0} pairing is not the case, the second pairing is unrelated to factorisation, and the third pairing concerns a desire, rather than what is.
\end{note}


\begin{note}
  Our characterisation of \pool{1} as capturing understanding of factorisation (\autoref{illu:gist:roots:F}), and so on,%
  \footnote{
    E.g., natural langauge algebraic problems (\autoref{scen:fc:chick}, \autopageref{scen:fc:chick}), the Lucas numbers (\autoref{scen:LucasNums}, \autopageref{scen:LucasNums}).
  }
  is designed to implicitly capture the agent reasoning by some \torNa{}.

  \begin{proposition}[Factorisation: \torNa{}, \pool{}]%
    \label{prop:p-t}%
    Given \(\Phi\) is a \pool{} which includes \vAgent{}' understanding of factorisation:
    \begin{itenum}
    \item[\emph{If}:]
      \(\ed{\flat}\) is an \eiw{0} \vAgent{} is concluding \(\pv{\phi}{\valI{True}}\) from \(\Phi\).
    \item[\emph{Then}:]
      \(\ed{\flat}\) is an \eiw{0} \vAgent{} is \tCV{} \(\pv{\phi}{\valI{True}}\) from some \pool{} \(\Phi\) by factorisation.
    \end{itenum}
    \vspace{-\baselineskip}
  \end{proposition}


  \begin{argument}{prop:p-t}
    Suppose \(\ed{\flat}\) is an \eiw{0} \vAgent{} concludes \(\pv{\phi}{\valI{True}}\) from \(\Phi\).

    Now, factorisation as understood in the context of quadratic equations just is rewriting an expression of the form \rootsConEqGenV{a}{b}{c} as the product of some factors of the form \(\rootsFacV{i}{j}{q}{r}\) and then identifying values of \(x\) such that \(\rootsFacL{i}{q} = 0\) and \(\rootsFacR{j}{r} = 0\).
    Hence, in order for the agent to have applied their understanding of factorisation, the agent must be somewhere in the process of rewriting some expression as the product of some factors and identifying the relevant values of \(x\).
    Therefore, \(\ed{\flat}\) is an \eiw{0} \vAgent{} is \tCV{} \(\pv{\phi}{\valI{True}}\) from some \pool{} \(\Phi\) by factorisation.
    \smallskip

    \noindent%
    Conversely, the agent may be concluding \(\pv{\phi}{\valI{True}}\) by some other \torNa{}, but then the agent is concluding \(\pv{\phi}{\valI{True}}\) from some \pool{} other than \(\Phi\).
  \end{argument}

  \noindent%
  For example, in \autoref{illu:gist:roots:F} the agent to concludes \pv{\propM{\rootsCon{}}}{\valI{True}} from some \pool{} which includes the \agents{} understanding of factorisation.
  In other words, the agent \tCV[concluded]{} \pv{\propM{\rootsCon{}}}{\valI{True}} from the relevant \pool{} \(\Phi\) by the \torNa{} `factorisation'.
  Or, a little more colloquially, the agent factored.%
  \footnote{
    \autoref{def:tpro} does not, strictly, require a fixed \pool{}, though the \illu{1} which follow build on the intuitive connexion between a \torN{} an a \pool{} which includes an \agents{} understanding of some thing.
  }
\end{note}



\subsection{\tprof{3}}
\label{sec:tpros}


\begin{note}
  With the definition of a \torN{0} in hand we define \emph{\tpro{}} of a \torN{}:

  \begin{rdefinition}{def:tpro}{\tprof{3}}%
    \vspace{-\baselineskip}
    \begin{itemize}
    \item
      A collection \(\tproS{}\) of descriptions associated with \prop{0}-\val{0}-\pool{0} pairings is a \emph{\tpro{}} with respect to a description \(\edo{}\) an agent \vAgent{}, \prop{0}-\val{0} pair \(\pv{\phi}{v}\), \pool{} \(\Phi\), and \torNa{} \(T\).
    \end{itemize}

    \emph{If and only if}:

    \begin{itemize}
    \item
      Clauses \ref{def:tpro:A} and \ref{def:tpro:B} hold:

      \begin{enumerate}[label=\Alph*., ref=\Alph*]
      \item
        \label{def:tpro:A}
        \(\edo{}\) entails: \vAgent{} is \tCV{} \(\pv{\phi}{v}\) from \(\Phi\) by type \(T\).
      \item
        \label{def:tpro:B}
        For any (maybe possible) event \(\edn{}\):
        \begin{itenum}
        \item[\emph{If}:]
          \(\edo{}\) is true of \(\edn{}\).
        \item[\emph{Then}:]
          For every description \(\edo{\ast}\) and associated \prop{0}-\val{} pair \(\pv{\psi}{v'}\) and \pool{0} \(\Psi\) in \(\tproS{}\):
          \begin{itemize}
          \item
            \(\edo{\ast}\) is true of a (maybe possible) event \(\edn{+}\).
          \item
            For any (maybe possible) event \(\edn{\times}\) that \(\edo{\ast}\) is true of:
            \begin{itemize}
            \item
              \(\edn{\times}\) is an \eiw{0} \vAgent{} is concluding \(\pv{\psi}{v'}\) from \(\Psi\).
            \end{itemize}
          \end{itemize}
        \end{itenum}
      \end{enumerate}
    \end{itemize}
    \vspace{-\baselineskip}
  \end{rdefinition}

  \noindent%
  \autoref{def:tpro} defines a property of some collection of descriptions associated with \prop{0}-\val{0}-\pool{0} pairings relative to a description of an event, a \torNa{}, \prop{0}-\val{0} pair, and \pool{}.%
  \footnote{
    In other words, \autoref{def:tpro} does not define a collection of descriptions and associated pairings.
    Rather, \autoref{def:tpro} defines when a term applies to any such collection, given some additional constraints.
  }
  The description entails the agent is \tCV{} the \prop{0}-\val{0} pair from the \pool{} by type \(T\), and a \tpro{} captures a necessary condition of the description being true of any event:
  Each description in the \tpro{} is true of some \eiw{} the agent is concluding the associated \prop{0}-\val{0} pair from the associated \pool{}.%
  \footnote{
    We do not require each \prop{0}-\val{0} pair in the \tpro{} is included in the relevant \torNa{}.
    For, while the constraint is intuitive --- and indeed every application of \autoref{def:tpro} we discuss is compatible with the constraint --- we have no need for the constraint to hold.
  }
\end{note}

\begin{note}
  Our interest with \tpro{} is when the relevant description is true of some event.
  \autoref{prop:tpro-switch} captures the key feature:%
  \footnote{
    It is not possible to strengthen \autoref{prop:tpro-switch} to an if and \emph{only if}, as \tpro{1} are defined by a necessary condition on an agent \tCV{}.
  }

  \begin{rproposition}{prop:tpro-switch}{Shadows}
    \vspace{-\baselineskip}
    \begin{itenum}
    \item[\emph{If}:]
      Conditions \ref{prop:tpro-switch:a} and \ref{prop:tpro-switch:b} hold:
      \begin{enumerate}
      \item
        \label{prop:tpro-switch:a}
        \(\ed{}\) is an \eiw{0} \vAgent{} is \tCV{} \(\pv{\phi}{v}\) from \(\Phi\).
      \item
        \label{prop:tpro-switch:b}
        \(\tproS{}\) is a \tpro{} with respect to \(\edo{}\), \vAgent{}, \(\pv{\phi}{v}\), \(\Phi\), and \(T\).
      \end{enumerate}
    \item[\emph{Then}:]
      For every description \(\edo{\sharp}\) and associated \prop{0}-\val{0}-\pool{0} pair \(\pv{\psi}{v'}\) and \(\Psi\) in \(\tproS{}\):
      \begin{itemize}
      \item
        There is some event \(\edn{+}\) such that \(\edo{\sharp}\) is true of \(\edn{\sharp}\).
      \item
        For every event \(\edn{\times}\) such that \(\edo{\sharp}\) is true of \(\edn{\times}\):
        \begin{itemize}
        \item
          \(\edn{\times}\) is an \eiw{0} \vAgent{} is concluding \(\pv{\psi}{v'}\) from \(\Psi\).
        \end{itemize}
      \end{itemize}
    \end{itenum}
    \vspace{-\baselineskip}
  \end{rproposition}

  \begin{argument}{prop:tpro-switch}
    Immediate by rearranging \autoref{def:tpro}.
  \end{argument}

  \autoref{prop:tpro-switch} highlights a consequence of an agent \tCV{}, given the idea of a \tpro{}.

  Intuitively, \autoref{def:tpro} helps highlight the way \autoref{def:tpro} expresses the idea that an agent is \tCV{} \(\pv{\phi}{v}\) from \(\Phi\) by type of reasoning \(T\) only if some `law' holds.
  Where, `law' is understood in the colloquial sense of a universally quantified material conditional.%
  \footnote{
    For example, consider \citeauthor{Helmholtz:1977aa}'s characterisation of laws of nature:%
    \begin{quote}
      \nocite{Wilson:2006aa}
      Every law of nature asserts that upon preconditions alike in a certain respect, there always follow consequences that are alike in a certain other respect.%
      \mbox{ }\hfill\mbox{(\citeyear[122]{Helmholtz:1977aa})}
    \end{quote}
    %
    Or the law of large numbers:
    %
    \begin{quote}
      Things of every kind of nature are subject to a universal law which one may well call \emph{the Law of Large Numbers}.
      It consists in that if one observes large numbers of events of the same nature [\dots] one finds that the proportions of occurrence are almost constant [\dots]%
      \mbox{ }\hfill\mbox{(\citeauthor{Seneta:2013aa}'s (\citeyear[9--10]{Seneta:2013aa}) translation of (\cite[7]{Poisson:1837aa}))}
    \end{quote}
    %
    And \citeauthor{Hempel:1965aa}'s Deductive-Nomological account of scientific explanation, \citeauthor{Boole:1854aa}'s laws of thought, etc.
  }
  Or, as stating the agent is disposed to conclude in a certain way.%
  \footnote{
    For additional notes on this suggestion, see \autoref{cha:tc2-dispositions} (\autopageref{cha:tc2-dispositions}).

    Alternatively, a \tpro{} may be through to capture the modal robustness of an agent \tCV{}.
    In line with, for example, sensitivity (\cite{Nozick:1981aa}) and safety (\cite{Sosa:1999aa}) conditions on knowledge.
    Or, the intuition that a skilfully made paper aeroplane flies in a variety of conditions, etc.
  }

  The role of a \tpro{} is to collect together some descriptions which must be true of some event \emph{if} an agent is \tCV{}.

  Note, in particular, \autoref{def:tpro} does not assume the relevant definition is true of any particular event, and so allows a \tpro{} to be identified even if the agent is not \tCV{}.
\end{note}


\begin{note}
  I have little to say in general about when some description and associated \prop{0}-\val{0} pairing is included in a \tpro{}.
  Part of the overall argument of this document as stated rests on particular descriptions and \prop{0}-\val{0}-\pool{0} pairings being included in a \tpro{}, as we are in search of failures of \issueInclusion{}.
  Still, the way we obtain failures of \issueInclusion{} is compatible with a variety of different perspectives on which descriptions and associated \prop{0}-\val{0} pairings are included in a \tpro{}.

  Some \illu{1} and an application follow.
  The application is designed to highlight considerations I think are sufficient (though not necessary) to establish some description and \prop{0}-\val{0}-\pool{0} pairing is part of a \tpro{} by showing, rather than telling.
\end{note}



\subsection{\illu{3}}
\label{sec:illu3-1}

\begin{note}
  We give two \illu{1} of \autoref{def:tpro} and then apply \autoref{def:tpro} to \autoref{illu:gist:roots:F}.

  The first \illu{0} considers selection tasks, and more broadly links \tprof{1} to negative results concerning reasoning found in the literature.

  The second \illu{0} considers multiplication by powers, and highlights our interest with a \tprof{}.
\end{note}



\paragraph*{Selection tasks}
\nocite{Wason:1968aa}
\nocite{Wason:1971aa}
\label{par:selection-tasks}


\begin{note}
  \citeauthor{Wason:1966aa} details their initial section task as follows:
  %
  \begin{quote}
    The subjects (students) were presented with an array of cards and told that every card had a letter on one side and a number on the other side, and that either would be face upwards.
    They were then instructed to decide which cards they would need to turn over in order to determine whether the experimenter was lying in uttering the following statement:
    \emph{if a card has a vowel on one side then it has an even number on the other side}.%
    \mbox{ }\hfill\mbox{(\citeyear[145--146]{Wason:1966aa})}
  \end{quote}
  %
  \autoref{fig:sectionTask} is an example task.
  %
  \begin{figure}[H]
    \centering
    \begin{tikzpicture}[
      cardnode/.style={
        rectangle,
        minimum width=10mm,
        minimum height=14mm,
        align=center,
        rounded corners,
        font = {\Large\sffamily},
        very thick,
      },
      node distance=5mm,
      ]

      \node[cardnode, draw] (1) {2};
      \node[cardnode, draw, right = of 1] (2) {N};
      \node[cardnode, draw, right = of 2] (3) {E};
      \node[cardnode, draw, right = of 3] (4) {1};
    \end{tikzpicture}
    \caption{A selection task}
    \label{fig:sectionTask}
  \end{figure}

  \citeauthor{Wason:1966aa} observes the results are consistent with the following hypothesis:%
  \footnote{
    \citeauthor{Wason:1966aa} does not provide a detailed summary of the results.
    However, \citeauthor{Johnson-Laird:1969aa} detail results of \emph{twenty-four} University College London students!
    Specifically, 19 of the 24 responded as excepted given \citeauthor{Wason:1966aa}'s hypothesis.
    (\citeyear[369--370]{Johnson-Laird:1969aa}).
  }
  \begin{quote}
    Subjects assume implicitly that a conditional statement has, not two truth values, but three: true, false and `irrelevant'.
    Vowels with even numbers verify, vowels with odd numbers falsify and consonants with any number are irrelevant.%
    \mbox{ }\hfill\mbox{(\citeyear[146]{Wason:1966aa})}
  \end{quote}
\end{note}

\begin{note}
  Selection tasks may be understood in connexion with \autoref{def:tpro}.
  For example:
  %
  \begin{itemize}
  \item
    The relevant \torN{} \(T\) is `reasoning via the material conditional'.%
    \footnote{
      \citeauthor{Wason:1966aa}'s hypothesis concerns a conditional statement having two truth values, but does not specify what these values are.
      Still, it is clear by \citeauthor{Wason:1966aa}'s discussion these capture the material conditional.
    }
  \item
    The relevant \tpro{} \(\tproS{}\) includes descriptions where an agent is doing a selection task with an associated \prop{0}-\val{0}-\pool{0} pairings which capture appropriate \evalN{1} of what do.
    E.g.:
    %
    \begin{center}
      \begin{tabular}{R{.40\textwidth} L{.40\textwidth}}
        \prop{2}-\val{0} pair & \prop{2}-\val{0} pairs in \pool{0} \\
        \hline
        \pv{C\propI{ needs to be turned over}}{\valI{True}} & \pv{C\propI{ has a vowel}}{\valI{True}}, \dots \\
        \pv{C\propI{ needs to be turned over}}{\valI{True}} & \pv{C\propI{ has an odd number}}{\valI{True}}, \dots \\
        \pv{C\propI{ needs to be turned over}}{\valI{False}} & \pv{C\propI{ has consonant}}{\valI{True}}, \dots \\
        \pv{C\propI{ needs to be turned over}}{\valI{False}} & \pv{C\propI{ has an even number}}{\valI{True}}, \dots \\
      \end{tabular}
    \end{center}
  \end{itemize}
  %
  Where `\(C\)' represent the relevant card.
\end{note}


\begin{note}
  \citeauthor{Wason:1966aa} observes most agents do not succeed at selection tasks.
  Hence, \citeauthor{Wason:1966aa} suggests agents do not --- in general --- reason via the material conditional.
\end{note}


\begin{note}
  Note, our reconstruction of \citeauthor{Wason:1966aa} does not state agents do not reason with the material conditional.
  Instead, the relevant \torN{} is intended to capture the idea that \emph{in general} agents reason with conditionals via the material conditional.
  Rejecting agents `in general' reason with conditionals via the material conditional is compatible with agents \emph{sometimes} reasoning with the material conditional.%
  \footnote{
    For example, various definitions and ideas in this document are given with an implicit expectation that any instance \emph{if} \dots \emph{then} \dots construction is understood in line with the material conditional.
  }

  So, we understand there to be a difference between `reasoning via the material conditional' and `reasoning via the material conditional when implicitly or explicitly called on by the situation'.%
  \footnote{
    Likewise, \citeauthor{Wason:1966aa}'s hypothesis of three truth values does not single out any particular \torN{}, as \citeauthor{Wason:1966aa} does not give a complete account of the truth values.
  }
\end{note}


\begin{note}
  Various other observations may be seen to parallel to our treatment of selection tasks via \autoref{def:tpro}.
  In particular, consider \citeauthor{Harman:1984aa}'s (\citeyear{Harman:1984aa,Harman:1986ux}) arguments against a strong connexion between logical principles and principles of belief revision.
  \citeauthor{Harman:1984aa} summarises:
  %
  \begin{quote}
    Logical principles are not directly rules of belief revision.
    [\dots]
    Logical principles hold universally, without exception, whereas the corresponding principles of belief revision would be at best prima facie principles, which do not always hold.%
    \mbox{ }\hfill\mbox{(\citeyear[107--108]{Harman:1984aa})}
  \end{quote}
    %
  Likewise, consider the \citeauthor{Allais:1979aa} paradox (\cite{Allais:1979aa}),
  the Ellsberg paradox (\cite{Ellsberg:1961aa}), \citeauthor{Makinson:1965aa}'s Paradox of the Preface (\citeyear{Makinson:1965aa}), \citeauthor{Kyburg:1997aa}'s Lottery Paradox (\citeyear{Kyburg:1997aa}), \citeauthor{Quinn:1990aa}'s puzzle of the self-torturer (\citeyear{Quinn:1990aa}), \citeauthor{Bratman:1981aa}'s arguments against the desire-belief model of practical reasoning (\citeyear{Bratman:1981aa,Bratman:1987aa}), and so on.

  Common to each observation mentioned is the idea that an agent \agents{} reasoning has some property in a particular instance just in case the agent reasons in a particular way in other circumstances --- and that the agent does not reason in the expected way in some relevant circumstance.
\end{note}



\paragraph*{Powers}

\begin{note}
  Selection tasks are events which happen and suggest an \agents{} reasoning is \emph{not} of some particular type.
  However, our interest with \tCN{} primarily concerns events where the agent \emph{is} \tCV{} by some type.
\end{note}

\begin{note}
  \begin{scenario}[Powerful multiplication]%
    \label{illu:tR:powers}%
    A student has been studying algebra and has just been introduced to the rule of multiplication for powers (\(a^{n} \cdot a^{m} = a^{n + m}\)).

    At hand are a handful of exercises (from \cite[32]{Gelfand:1993aa}):
    %
    \begin{quote}
      \begin{enumerate}[label=(\alph*), ref=(\alph*)]
      \item
        \label{mfp:a}
        You know that \(2^{1001} \cdot 2^{n} = 2^{2000}\).
        What is \(n\)?
      \item
        \label{mfp:b}
        You know that \(2^{1001} \cdot 2^{n} = \sfrac{1}{4}\).
        What is \(n\)?
      \item
        \label{mfp:c}
        Which is bigger: \(10^{-3}\) or \(2^{-10}\)?
      \item
        \label{mfp:d}
        You know that \(\sfrac{2^{1000}}{2^{n}} = 2^{501}\).
        What is \(n\)?
      \item
        \label{mfp:e}
        You know that \(\sfrac{2^{1000}}{2^{n}} = \sfrac{1}{16}\).
        What is \(n\)?
      \item
        \label{mfp:f}
        You know that \(4^{100} = 2^{n}\).
        What is \(n\)?
      \item
        \label{mfp:g}
        You know that \(2^{100} \cdot 3^{100} = a^{100}\).
        What is \(a\)?
      \item
        \label{mfp:h}
        You know that \((2^{10})^{15} = 2^{n}\).
        What is \(n\)?
      \end{enumerate}
    \end{quote}
    %
    The student starts work on exercise~\ref{mfp:f}, and is concluding \pv{\propI{n is 200}}{\valI{True}}.
  \end{scenario}

  \noindent%
  Intuitively, the agent is concluding by their understanding of multiplication for powers.

  Further, it is plausible exercises \ref{mfp:a} through to \ref{mfp:h} were chosen by \citeauthor{Gelfand:1993aa} to test basic competence with multiplication for powers.
  So, given \autoref{def:tpro}, a \tpro{} \tproS{} of multiplication for powers includes \eiw{1} the agent begins to work on exercises \ref{mfp:a} through to \ref{mfp:h} and the relevant \prop{0}-\val{0}-\pool{0} pairings capture the appropriate answer to the exercise the agent is working on.

  In turn, by \autoref{prop:tpro-switch}, if it is not possible for the agent to be concluding, e.g., \pv{\propI{a is 6}}{\valI{True}} from a \pool{} which includes \(2^{100} \cdot 3^{100} = a^{100}\) when working on~\ref{mfp:f}, then the agent is not concluding \emph{by} an understanding of multiplication for powers.%
  \footnote{
    This may be resisted.
    For example, exercises~\ref{mfp:b},~\ref{mfp:d}, and~\ref{mfp:e} all involve fractions.
    And, the agent may be shaky on fractions, but good with multiplication by powers.
    Hence, the relevant collection \prop{0}-\val{0}-\pool{0} pairings may only include pairings which correspond to exercises \ref{mfp:a}, \ref{mfp:c}, \ref{mfp:f}, \ref{mfp:g}, and \ref{mfp:h}.

    For broader motivation for such restrictions, consider, e.g., \citeauthor{Chomsky:2015aa}'s distinction between competence and performance    (\citeyear[xii]{Chomsky:2015aa}).
    Note, any restriction still amounts to a \tpro{} by \autoref{def:tpro}.
  }
\end{note}



\paragraph*{An application}


\begin{note}
  \begin{application}[A \tpro{} of \autoref{illu:gist:roots:F}]%
    \label{app:sc1-typ}%
    Given:
    %
    \begin{itemize}
    \item
      \(\edn{\flat}\) covers Step~\ref{illu:gist:roots:F:factor} of the \agents{} reasoning in \autoref{illu:gist:roots:F}.
    \item
      \(\Phi\) includes the \agents{} understanding of factorisation prior to \(\edn{}\).
    \end{itemize}
    %
    There is a reading of \autoref{illu:gist:roots:F} such that the following conditional holds:
    %
    \begin{itenum}
    \item[\emph{If}:]
      If \(\ed{\flat}\) is an \eiw{0} the agent is \tCV{} \pv{\propI{\rootsCon{}}}{\valI{True}} from \(\Phi\) by factorisation.
    \item[\emph{Then}:]
      There is a \tpro{} \(\mathbb{P}\) which consists of:
      \begin{itemize}
      \item
        \prop{2}-\val{0}-\pool{0} pairings, for each \(n\) in \(\rootsConAltSet{}\), such that:
        \begin{itemize}
        \item
          The \pool{} contains an equation of the form \rootsConEqGen{}.
        \item
          The \prop{0}-\val{0} pair is of the form \pv{\propI{\rootsConGen{}}}{\valI{True}}, where \(m \times (m - 1) = n\).
        \end{itemize}
      \item
        Descriptions \(\edo{\ast}\) such that:
        \(\edo{\ast}\) is true of the outcome of an action done by \vAgent{} in \(\edn{\flat}\).
      \end{itemize}
    \end{itenum}
    \vspace{-\baselineskip}
  \end{application}

  \noindent%
  For quick intuition, consider \autoref{app:sc1-typ} as a variant of \autoref{illu:tR:powers} where the relevant exercises are generated by the agent (rather than being given to the agent).

  Still, part of the motivation for \autoref{illu:tR:powers} was the idea that the exercises are designed to test basic competence with multiplication for powers.
  Hence, some work is required to show the agent is concluding \pv{\propM{\rootsCon{}}}{\valI{True}} from from a \pool{} which includes \rootsConEq{} and the \agents{} understanding of factorisation by factorisation \emph{only if} the agent would be concluding \pv{\propI{\rootsConGen{}}}{\valI{True}} from some \pool{} if the agent did some action.

  The details of \autoref{app:sc1-typ} highlight the way the instances of \(n\) amount to \emph{structurally similar} quadratic equations.
  In other words, each instance of \(n\) amounts to a quadratic equation which may be solved by reasoning which parallels what the agent is doing when concluding \pv{\propI{\rootsCon{}}}{\valI{True}} from \(\Phi\) by factorisation.

  Indeed, it not enough to highlight, e.g.\ \rootsConEqGenV{20}{}{1} is a quadratic equation and \rootsConGen{} is an expression of the relevant factors.
  For, \rootsConEqGenV{15}{}{1} is likewise a quadratic equation, but it seems unreasonable to expect the agent to be concluding \(x \in \{\sfrac{(\sqrt{61} - 1)}{30}, -\sfrac{(1 + \sqrt{61})}{30}\}\).%
  \footnote{
    Likewise, as each instance of \(n\) amount to a structurally similar quadratic equation, there a plausibly other \tpro{1} which include quadratic equations of a different structure (e.g.\ \(x^{2} - 1\)).
    However, our interest with \tpro{1} is limited to a necessary condition on an agent \tCV{} (cf.\ \autoref{prop:tpro-switch}) and so there is no need for the \tpro{} to include all relevant description and \prop{0}-\val{0}-\pool{0} pairings.
  }

  \begin{dets}{app:sc1-typ}
    We first argue there relevant \prop{0}-\val{0}-\pool{0} pairings are part of some \tpro{}.
    Then, we argue a \tpro{} in which each description is true of and event which happens by the agent doing some action in \(\edn{\flat}\) is compatible with \autoref{illu:gist:roots:F}.
    \medskip

    \noindent%
    To see the relevant \prop{0}-\val{0}-\pool{0} pairings are part of some \tpro{}, suppose \(\ed{\flat}\) is an \eiw{0} the agent is \tCV{} \pv{\propI{\rootsCon{}}}{\valI{True}} from \(\Phi\) by factorisation.

    Now, factorisation is a \torNa{}.
    Specifically, given a quadratic equation is of the form \rootsConEqGenV{a}{b}{c}, factorisation involves finding \(i\), \(j\), \(q\), and \(r\) such that \(\rootsFacV{i}{j}{q}{r}\).
    In this respect, there must be some generality to the \agents{} reasoning --- assignments other than \(a \coloneq 6\), \(b \coloneq 1\), \(c \coloneq 1\) such that it is possible for the agent to conclude \(\rootsConEqGenV{a}{b}{c} = \rootsFacV{i}{j}{q}{r}\).

    At issue is:
    \begin{itemize}
    \item
      Which values of \(a\), \(b\), and \(c\) are such that it must be possible for the agent to conclude \(\rootsConEqGenV{a}{b}{c} = \rootsFacV{i}{j}{q}{r}\), if the agent is concluding by factorisation.%
    \footnote{
      Step~\ref{illu:gist:roots:F:factor} of the \agents{} reasoning in \autoref{illu:gist:roots:F}.
    }
    \item
      Whether the agent goes on to identify the values of \(x\) such that \(\rootsFacL{i}{q} = 0\) or \(\rootsFacR{j}{r} = 0\).%
    \footnote{
      Steps~\ref{illu:gist:roots:F:zero} to \ref{illu:gist:roots:F:factor:done} of the \agents{} reasoning in \autoref{illu:gist:roots:F}.
    }
    \end{itemize}
    \smallskip

    \noindent%
    To see the values of \(a\), \(b\), and \(c\) in \(\mathbb{P}\) must be possible, observe the relation between \(a\), \(b\), \(c\), \(i\), \(j\), \(q\), and \(r\) in the variant quadratic equations structurally parallel those of the equation the agent reasons about in Step~\ref{illu:gist:roots:F:factor}.
    In particular:
    %
    \begin{itemize}
    \item
      \(c = q \times r\).
      And, for each \(n\), \(c \coloneq -1\).
      So, \(\rootsFacV{i}{j}{q}{r} =\rootsFacV{i}{j}{1}{1}\).
    \item
      Given \(\rootsFacV{i}{j}{1}{1}\), \(b = (i - j)\).
      And, for each \(n\), \(b \coloneq 1\).
      So, \(j = (i - 1)\).
    \item
      \(a = i \times j\).
      So, given \(j = (i - 1)\), \(a = i \times (i - 1)\).
    \end{itemize}
    %
    In short, the values of \(a\), \(b\), and \(c\) determine the values of \(i\), \(j\), \(q\), and \(r\) in a certain way.
    Hence, the quadratic equations included in \(\mathbb{P}\) amount to minimal variations of the quadratic equation in Step~\ref{illu:gist:roots:F:factor}.
    I.e., there is no structural difference between figuring out \rootsConEqExV{6}{3}{2} and, e.g., \rootsConEqExV{20}{5}{4}.
    \smallskip

    \noindent%
    So, figuring out \(\rootsConEqGenV{n}{}{1} = \rootsFacV{i}{(i - 1)}{1}{1}\) for each \(n\) in \(\rootsConAltSet{}\) must be possible for the agent.
    A few additional steps remain for the agent to conclude \pv{\rootsConGenV{i}{(i - 1)}}{\valI{True}}.
    Still, by assumption the agent is \tCV{} \pv{\propI{\rootsCon{}}}{\valI{True}} from \(\Phi\) by factorisation.
    And, the remaining reasoning amounts to slight variations of steps \ref{illu:gist:roots:F:zero} to \ref{illu:gist:roots:F:factor:done} of \autoref{illu:gist:roots:F}.
    Similar reasoning applies, and I leave this as an exercise for a motivated reader.
    \medskip

    \noindent%
    The argument has not yet specified the relevant descriptions in the \tpro{} \(\mathbb{P}\).
    Still, \autoref{illu:gist:roots:F} only characterises the \agents{} reasoning.
    So, the agent may have plenty of time to work on a different problem, and substituting \(n\) for \(6\) in \rootsConEq{}.
    So, I take it to be plausible a description which captures the agent substituting \(n\) for \(6\) in \rootsConEq{} and then working on the substitution is plausible.
  \end{dets}
\end{note}



\section{\tCV{3} and \fc{1}}
\label{sec:tcv3-requ1}


\begin{note}
  The link between and agent \tCV{} and \fc{1} is as follows:

  In some cases a \tpro{} of a \torNa{} will be the result of some action \(a\) the agent may do (e.g.\ switching to a different exercise in \autoref{illu:tR:powers}).
  So, given the agent \evals{} \(\psi'\) as having value \(v''\) prior to doing \(a\), for each \prop{0}-\val{0} pair \(\pv{\psi'}{v''}\) in the relevant \pool{}, the \prop{0}-\val{0} pair must be a \fc{} from the \pool{}.

  In detail:

  \begin{proposition}[\typeAdj{2} \fc{1}]%
    \label{prop:tCV-fc}%
    \vspace{-\baselineskip}
    \begin{itenum}
    \item[\emph{If}:]
      Conditions \ref{prop:tCV-fc:tC} and \ref{prop:tCV-fc:pro} hold:
      \begin{enumerate}[label=\arabic*., ref=\arabic*]
      \item
        \label{prop:tCV-fc:tC}
        \(\ed{\flat}\) is an \eiw{0} \vAgent{} is \tCV{} \(\pv{\phi}{v}\) from \(\Phi\) by \torNa{} \(T\).
      \item
        \label{prop:tCV-fc:pro}
        \(\tproS{}\) is a projection with respect to \(\edo{\flat}\), \vAgent{}, \(\pv{\phi}{v}\), \(\Phi\), and \torNa{} \(T\) such that:
        \begin{itemize}
        \item
          For some description \(\edo{\sharp}\) and \prop{0}-\val{0}-\pool{0} pair \(\pv{\psi}{v'}\) and \(\Psi\) in \(\tproS{}\):
          \begin{enumerate}[label=\alph*., ref=\theenumi\alph*]
          \item
            \label{prop:tCV-fc:e:act:i}
            \(\edo{\sharp}\) is satisfied by an event \(\edn{\sharp}\) which is the outcome of an action \(a\) done by \vAgent{} in \(\edn{\flat}\).
          \item
            \label{prop:tCV-fc:e:act:ii}
            For each \(\pv{\psi'}{v''}\) in \(\Psi\):
            \vAgent{} \evals{} \(\psi'\) as having value \(v''\) prior to doing \(a\).
          \end{enumerate}
        \end{itemize}
      \end{enumerate}
    \item[\emph{Then}:]
      \(\pv{\psi}{v'}\) is a \fc{0} from \(\Psi\) for \vAgent{} through \(\ed{\flat}\).
    \end{itenum}
    \vspace{-\baselineskip}
  \end{proposition}

  \begin{argument}{prop:tCV-fc}
    Suppose conditions \ref{prop:tCV-fc:tC} and \ref{prop:tCV-fc:pro} hold.

    Our task is to show \autoref{def:fc} (\autopageref{def:fc}) holds.
    Specifically, we show clauses \ref{def:fc:ai}, \ref{def:fc:act} and \ref{def:fc:result} of \autoref{def:fc} are satisfied with respect to \(\ed{\flat}\), \(\pv{\psi}{v'}\) and \(\Psi\).
    For, \(\ed{\flat}\) is a trivial partition of \(\ed{\flat}\) into sub-events.
    So, in turn:

    \begin{itemize}
    \item
      Clause~\ref{def:fc:ai} is satisfied as Sub-condition~\ref{prop:tCV-fc:e:act:i} \(\edn{\sharp}\) is the outcome of an action \(a\) done by \vAgent{} in \(\ed{\flat}\).
      In other words, \(\ed{\flat}\) is an \eiw{0} the agent may do \(a\).
    \item
      Clause~\ref{def:fc:act} is satisfied as conditions \ref{prop:tCV-fc:tC} and \ref{prop:tCV-fc:pro} entail \(\ed{\sharp}\) is an \eiw{0} \vAgent{} is concluding \(\pv{\psi}{v'}\) from \(\Psi\).
      I.e., the \eiw[\(\edn{\sharp}\)]{0} the agent does \(a\) is an \eiw{0} the agent is concluding \(\pv{\psi}{v'}\) from \(\Psi\).
    \item
      Finally, Sub-condition~\ref{prop:tCV-fc:e:act:ii} is easily seen as equivalent to Clause~\ref{def:fc:result}.
    \end{itemize}
    %
    So, \(\pv{\psi}{v'}\) is a \fc{0} from \(\Psi\) for \vAgent{} through \(\ed{\flat}\).
  \end{argument}
\end{note}


\begin{note}
  For example, consider \autoref{illu:tR:powers}.

  The agent is concluding \pv{\propI{n is 200}}{\valI{True}} by an understanding of multiplication for powers.
  So, Condition \ref{prop:tCV-fc:tC} is satisfied.

  Further, it is plausible the agent is concluding \pv{\propI{n is 200}}{\valI{True}} by an understanding of multiplication for powers \emph{only if} the agent would be concluding \pv{\propI{a is 6}}{\valI{True}} from \pv{\propM{2^{100} \cdot 3^{100} = a^{100}}}{\valI{True}} were the agent to abandon \ref{mfp:g} and work on~\ref{mfp:f}.

  So, we plausibly have a \tpro{} where the relevant descriptions are satisfied by an event which is the outcome of an action done by the agent while concluding \pv{\propI{n is 200}}{\valI{True}}.
  So, Condition \ref{prop:tCV-fc:pro} is also satisfied.
\end{note}


\begin{note}
  \begin{application}[\fc{3} of \autoref{illu:gist:roots:F} via a \tpro{}]%
    \label{app:sc1-fc}%
    There is a reading of \autoref{illu:gist:roots:F} such that the following conditional holds:
    \begin{itenum}
    \item[\emph{If}:]
      If the agent is \tCV{} \pv{\propI{\rootsCon{}}}{\valI{True}} from \(\Phi\) by factorisation.
    \item[\emph{Then}:]
      For \(n\) in  \(\rootsConAltSet{}\) and \(m\) such that \(m \times (m - 1) = n\):
      \begin{itemize}
      \item
        \pv{\propM{\rootsConGen{}}}{\valI{True}} is a \fc{} from a \pool{} which includes \pv{\propM{\rootsConEqV{n}}}{\valI{True}}.
      \end{itemize}
    \end{itenum}
    \vspace{-\baselineskip}
  \end{application}

  \begin{dets}{app:sc1-fc}
    Pair (the details of) \autoref{app:sc1-typ} with \autoref{prop:tCV-fc}.
  \end{dets}
\end{note}


\section*{Summary}

\begin{note}
  This chapter introduced the idea of an agent \tCV{} some \prop{0}-\val{0} pair from some \pool{} of premises and focused on the idea of a \tprof{} (\autoref{def:tpro}).

  The idea of a \tpro{} is motivated by common sense (cf.\ \autoref{illu:tR:powers}), through also motivated by arguments which concern an agent not reasoning by some \torN{} such as selection tasks (\cite{Wason:1966aa}) and various observations regarding decision theory (\cite{Allais:1979aa}, \cite{Ellsberg:1961aa}, \cite{Quinn:1990aa}), logical principles (\cite{Makinson:1965aa}, \cite{Kyburg:1997aa}, \cite{Harman:1984aa}), and practical reasoning (\cite{Bratman:1981aa,Bratman:1987aa}).
\end{note}


\begin{note}
  The key result of this chapter (\autoref{prop:tCV-fc}) linked an agent \tCV{} some \prop{0}-\val{0} pair from some \pool{} of premises to various \prop{0}-\val{0}-\pool{0} pairing being \fc{1} via the idea of a \tpro{}.
  Specifically, \autoref{prop:tCV-fc} provides sufficient conditions for a \prop{0}-\val{0} pair being a \fc{} from some \pool{} given an agent is \tCV{}.
\end{note}







%%% Local Variables:
%%% mode: latex
%%% TeX-master: "master"
%%% TeX-engine: luatex
%%% End:
