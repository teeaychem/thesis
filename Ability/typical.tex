\chapter{\tR{2}}
\label{cha:typical}

\begin{note}
  This chapter, \tR{}.

  Intuitively, \tR{} is reasoning which has some generality.
  Think, modus ponens.
\end{note}

\begin{note}
  Function in relation to counterexamples to \issueConstraint{} is conclusions.
  Relevant conclusion is the result of \tR{} --- or, \tR[concluding]{}.
  That is, `part' of the relevant phenomena.

  Two goals.

  Motivate \tR{}.
  Motivate a necessary condition on \tR{}.
\end{note}

\begin{note}
  No guarantee that this holds.
  Only constraint on conclusion is pairing.
  Likewise, no constraint on reasoning.

  Hence, generality.
\end{note}

\begin{note}
  The primary thing with this chapter is link between \tor{} and \tR{}.

  This link will be somewhat non-standard, but will is suited for purpose.
\end{note}

\begin{note}
  Two important things.

  \begin{itemize}
  \item
    motivation for \tR{}.
  \item
    the way in which we (partially) define \tR{}.
  \end{itemize}
\end{note}

\begin{note}
  \begin{enumerate}[label=]
  \item
    \TOCLine{cha:typical:sec:g-s}
  \item
    \TOCLine{cha:typical:sec:tR}
  \end{enumerate}
\end{note}

\section{Specific instances of general reasoning}
\label{cha:typical:sec:g-s}

\begin{note}
  We begin with three \scen{0}.
  Each \scen{0} involves, either implicitly or explicitly, a specific instance of some type of reasoning.%
  \footnote{
    Not designed to use `type' in the sense of type-token distinction.
    Common characteristic.
    Though, type distinct from instances.
  }
\end{note}

\begin{note}
  \begin{scenario}[We Are Introduced]
    ``I wonder if you've got such a thing as a balloon about you?''

    \noindent%
    ``A balloon?''

    \noindent%
    ``Yes, I just said to myself coming along:
    `I wonder if Christopher Robin has such a thing as a balloon about him?'
    I just said it to myself, thinking of balloons, and wondering.''

    \noindent%
    ``What do you want a balloon for?'' you said.

    \noindent%
    Winnie-the-Pooh looked round to see that nobody was listening, put his paw to his mouth, and said in a deep whisper:
    ``\emph{Honey!}''

    \noindent%
    ``But you don't get honey with balloons!''

    \noindent%
    ``I do,'' said Pooh.%
    \mbox{ }\hfill\mbox{(\cite[12]{Milne:2009aa})}
  \end{scenario}

  \begin{scenario}[The Cicada and the Fox]
    A cicada was singing on top of a tall tree.
    The fox wanted to eat the cicada, so she came up with a trick.
    She stood in front of the tree and marvelled at the cicada's beautiful song.
    The fox then asked the cicada to come down and show himself, since the fox wanted to see how such a tiny creature could be endowed with such a sonorous voice.
    But the cicada saw through the fox's trick.
    He tore a leaf from the tree and let it fall to the ground.
    Thinking it was the cicada, the fox pounced and the cicada then said,
    `Hey, you must be crazy to think I would come down from here! I've been on my guard against foxes ever since I saw the wings of a cicada in the spoor of a fox.'%
    \mbox{ }\hfill\mbox{(\cite[136]{Aesop:2002aa})}
    % The Sheep and the Injured Wolf, 85--86
    % The Mouse, the Fox, and the Grapes 238
  \end{scenario}

  \begin{scenario}[Humpty Dumpty]
    `I'm afraid I ca'n't quite remember it,'
    Alice said, very politely.

    \noindent%
    `In that case we start afresh,'
    said Humpty Dumpty,
    `and it's my turn to choose a subject---'
    (`He talks about it just as if it was a game!' thought Alice.)
    `So here's a question for you. How old did you say you were?'

    \noindent%
    Alice made a short calculation, and said
    `Seven years and six months.'

    \noindent%
    `Wrong!'
    Humpty Dumpty exclaimed triumphantly.
    `You never said a word like it!'

    \noindent%
    `I thought you meant ``How old \emph{are} you?'''
    Alice explained.

    \noindent%
    `If I'd meant that, I'd have said it,'
    said Humpty Dumpty.

    \noindent%
    Alice didn't want to begin another argument, so she said nothing.\newline
    \mbox{ }\hfill\mbox{(\cite[188]{Carroll:2009aa})}
  \end{scenario}
\end{note}

\begin{note}
  Common is means-end reasoning, and reasoning about means-end reasoning.

  Pooh, end is honey, balloon is a means.
  Robin is confused about how balloon is a means to end.

  End of eating.
  Complementing means.
  Whether or not cicada recognises, understands end, and that anything proposed is a means to that end.

  End, answering question.
  Calculation, means.

  Humpty, something went wrong.
  Not a good means.

  Then, no more arguing.
  And, saying nothing.
\end{note}

\begin{note}
  Pollock.

  \begin{quote}
    Means-end reasoning is concerned with finding the means for achieving goals.
    [\dots]
    To achieve a goal, we consider an action that would achieve it under some specified circumstances and then try to find a way of putting ourselves in those circumstances in order to achieve the goal by performing the action.%
    \mbox{ }\hfill\mbox{(\cite[60]{Pollock:2002aa})}
  \end{quote}
\end{note}

\begin{note}
  So, we have a common pattern.
  However, in what sense in any \scen{0} an instance of that pattern.

  For example, arbitrary.

  However, each \scen{0} suggests otherwise.

  With Pooh, Pooh has a particular idea, and maintains interest in getting a ballooon, and, goes on to perform the means.

  With the cicada, on guard against foxes.
  If the fox had offered something for the cicada to collect, may have chosen this as a means, and cicada recognise this as a means to the end.

  Alice, attempting to answer Humpty's question, and keeps quiet.

  Of course, could be accidental.
  These \scen{1} just happened.
\end{note}

\begin{note}
  Davidson

  For \citeauthor{Davidson:1963aa}, primary reason.

    \begin{quote}
      \emph{R} is a primary reason why an agent performed the action \emph{A} under the description \emph{d} only if \emph{R} consists of a pro attitude of the agent toward actions with a certain property, and a belief of the agent that \emph{A}, under the description \emph{d}, has that property.\newline
      \mbox{ }\hfill\mbox{(\citeyear[687]{Davidson:1963aa})}
    \end{quote}

  \begin{quote}
    1. For us to understand how a reason of any kind rationalizes an action it is necessary and sufficient that we see, at least in essential outline, how to construct a primary reason.
    2. The primary reason for an action is its cause.%
    \mbox{ }\hfill\mbox{(\citeyear[686]{Davidson:1963aa})}
  \end{quote}

  Now, Davidson despairs over the second part.
  Have cases in which primary reason, but does not cause in the right way.

  Here, we have action, letting go of the rope.
  But, same for reasoning.
  Believe p, p then q.
  However, mention of q is so nerve that q.
  (The game is lost).

  Well, with the cicada.
  Tore a leaf from the tree, but in the same way as the rope.

  This is a question about whether the agent's reasoning no.
  Ugh, the point with Davidson is that we have everything but the link in between.
  We can always get everything but what happens.

  Okay, and it goes the other way.
  So, we've got this person doing a whole bunch of logic, but then they see the game.

  So, there are problems either way.
  This kind of Deviance, no clear answer.

  However, limits on any description.

  Look, this \emph{doesn't} undermine the analysis.
  It is still the case that means-end reasoning.

  However, collection does.
  Here, things like Wason and Allias.
  We don't reason in certain ways.
\end{note}

\begin{note}
  Wason selection task and other paradoxes.

  \begin{quote}
    Wason, however, argues that adults do not treat the conditional in a truth-functional manner: they consider it to be irrelevant when its antecedent is false.
  \end{quote}

  Then, How Implication Is Understood.

  With respect to individuals, different forms.
  Reasoning is not truth functional.

  \begin{quote}
    This experiment showed that the way in which implication is expressed exerts a decisive influence upon what it is understood to denote.
    371.
  \end{quote}

  Even if disjunction works out, need to factor in expression.

  Allias paradox.

  This isn't EU.

  Interesting, as decision theory covers a specific domain.
\end{note}

\begin{note}
  So, we set up a law, and it applies.
  If it fails to apply, then the law is bad.

  Problem with analysis of any other circumstance.

  Allias is more interesting.
\end{note}

\begin{note}
  Ugh, we get, that agent's do not reason in this way.
\end{note}

\begin{note}
  \citeauthor{Hempel:1965aa} on Deductive-Nomological explanations:

  \begin{quote}
    [A Deductive-Nomological] explanation answers the question
    `\emph{Why} did the explanandum-phenomenon occur?'
    by showing that the phenomenon resulted from certain particular circumstances, specified in \(C_{1}, C_{2}, \dots C_{k}\), in accordance with the laws \(L_{1}, L_{2}, \dots L_{\psi}\).
    By pointing this out, the argument shows that, given the particular circumstances and the laws in question, the occurrence of the phenomenon \emph{was to be expected}; and it is in this sense that the explanation enables us to \emph{understand why} the phenomenon occurred.\newline
    \mbox{ }\hfill\mbox{(\citeyear[337]{Hempel:1965aa})}
  \end{quote}
\end{note}


\begin{note}
  The point is, something general, the we get a law.
\end{note}

\begin{note}
  Example with Moore.
  If we don't think the law holds, then we have no guesses about the specific instances.

  Now, this gets difficult.
  It's not necessarily the case that the particulars of some event are going to make it the case that general is true (though taking arguably does this).
  However, if particulars fail, then law fails.
  And, there's no clear path to maintaining particular.
\end{note}

\begin{note}
  So, this is plausible.
  The goal for the remainder of the chapter is to capture this in a specific way.
\end{note}

\begin{note}
  {
    \color{red}
    Harman is another instance of this strategy.
  }

  Similar to \citeauthor{Harman:1984aa,Harman:1986ux}.
  However, different.
  \citeauthor{Harman:1986ux} raises problem regarding the \tor{}.
  At issue here is, regardless of the type, still get a problem.

  However, same principle employed.

  \begin{quote}
    Logical principles are not directly rules of belief revision.
    \dots
    Logical principles hold universally, without exception, whereas the corresponding principles of belief revision would be at best prima facie principles, which do not always hold.%
    \mbox{ }\hfill\mbox{(\citeyear[107--108]{Harman:1984aa})}
  \end{quote}
  Not only belief, not \emph{particularly} about anything.

  Part of the insight \citeauthor{Harman:1986ux} offers is that there is no quick link between general and specific.
\end{note}


\newpage

\begin{note}
  Our goal is to construct a plausible law-like constraint.
\end{note}

\section{\tR{2}}
\label{cha:typical:sec:tR}

% \section{Types, instances, and representatives of reasoning}
% \label{cha:typical:sec:tor}

\begin{note}
  Out goal is to construct a necessary condition for an instance of reasoning to be an instance of a type of reasoning.
\end{note}

\begin{note}
  Break this down into three steps.

  \begin{itemize}
  \item
    \tor{}.
  \item
    \rotor{}.
  \item
    \tR{}.
  \end{itemize}
\end{note}


\subsection{\tor{3}}
\label{cha:typical:sec:tor:def}

\begin{note}
  What is it about the events?

  This is difficult.
  Indeed, it is particularly difficult as we aim to be theory neutral.
  Nothing about the process.

  Inputs and outputs.
  Some relation holds between inputs and outputs.

  So, here, we have belief, desire, and desire.

  Indeed, have to get an extension.

  So,inputs and outputs, this is okay.
  Now, the problem is linking back to instances of reasoning.
\end{note}

\begin{note}
  \begin{restatable}[\tor{2}]{definition}{defToR}
    \label{def:tor}
    \mbox{ }
    \vspace{-\baselineskip}
    \begin{itemize}
    \item
      \(T\) is (the extension of) a \emph{\tor{0}}.
    \end{itemize}

    \emph{If and only if}:

    \begin{itemize}
    \item
      \(T\) is a collection of proposition-value-\poP{} pairings.
    \end{itemize}
    \vspace{-\baselineskip}
  \end{restatable}

  So, if we have a type, then it must have some extension.

  Extensional characterisation.

  Neutral generality.
\end{note}

\begin{note}
  Of interest is not with specific types, but generality.
  Specifying premises and conclusions, rather than whatever reasoning amounts to.
  This does not assume priority.
  Point is only that will at the very least have premises and conclusions.
\end{note}

\subsubsection{\tR{2}: Sketch}
\label{sec:idea}

\begin{note}
  \begin{sketch}[\tR{2}]
    \label{sketch:tR}
    \cenLine{
      \begin{itemize*}[noitemsep, label=\(\circ\)]
      \item
        Agent: \vAgent{}
      \item
        Propositions: \(\phi\), \(\psi\)
      \item
        Values: \(v\), \(v'\)
      \item
        \poP{3}: \(\Phi\), \(\Psi\)
      \item
        \mbox{ }
      \end{itemize*}
    }

    \noindent%
    \cenLine{
      \begin{itemize*}[noitemsep, label=\(\circ\)]
      \item
        Event: \(e\)
      \item
        Event description: \(d\)
      \item
        Type of reasoning: \(T\)
      \item
        \mbox{ }
      \end{itemize*}
    }

    \begin{itemize}
    \item
      \(e\) under description \(d\) is an event in which \vAgent{} is reasoning to \(\pv{\phi}{v}\) from \(\Phi\) by \emph{\tR{0}} of type \(T\).
    \end{itemize}

    \emph{Only if}:

    \begin{itemize}[noitemsep]
      \item
        For every pairing \(\pvp{\psi}{v'}{\Psi}\) in \(T\).
        \begin{itemize}[noitemsep]
        \item[\emph{If}:]
          \begin{enumerate}[label=\alph*., ref=(\alph*), series=tRSketch]
          \item
            There is some available action \(a\) such that \vAgent{} would be reasoning from \(\Psi\) after performing \(a\).
          \end{enumerate}
        \item[\emph{Then}:]
          \begin{enumerate}[label=\alph*., ref=(\alph*), resume*=tRSketch]
          \item
            There is some available action \(a'\) such that \vAgent{} would be reasoning to \(\pv{\phi}{v}\) from \(\Psi\) after performing \(a'\).
          \end{enumerate}
        \end{itemize}
      \end{itemize}
    \vspace{-\baselineskip}
  \end{sketch}

  \autoref{sketch:tR}.
  The broad idea.

  If reasoning is of type \(T\), then, if the agent has the opportunity to reason from some premises, then it must also be the case that agent reasons from those premises by the same type of reasoning.
\end{note}

\begin{note}
  \begin{illustration}[Textbook]
    Student has been studying \dots something.

    Test understanding.

    Problem exercises.
  \end{illustration}

  Available action.
  Different problem.
  Then, as present reasoning of type, apply same reasoning to this other problem.

  Suppose not.
  Then, there's something about this instance of reasoning which does not generalise to other instances of the problem.
\end{note}

\begin{note}
  We shortly turn to difficulty with this proposal, and then a fix.

  However, briefly clarify points:

  \begin{itemize}
  \item
    Progressive.
    As with \fc{1}, progressive, as interruptions.
    At issue is whether there is sufficient substance to the event to make progressive true.
    It does not need to be the case that it would be the case the agent reasons to \(\pv{\psi}{v'}\) from \(\Psi\).

    Fail to make the progressive true.

    Need sufficient for it to be the case that reasoning applies to instances of same type.

    I.e. \ counterexamples.
  \item
    Limited to extension.
  \end{itemize}
\end{note}

\begin{note}
  Same as Wason and Allias.

  In these cases, observed problems.
  However, don't need to be observed.
\end{note}

\paragraph{Problems}
\label{cha:typical:sec:tor:g-s:three-problems}
\nocite{Wilson:1994aa}

\begin{note}
  \autoref{sketch:tR}.
  Motivated the basic idea.
  In this section, at least three ways in which progressive fails but may still be the case that reasoning is of type.
\end{note}

\begin{note}
  For example, mathematics.
\end{note}

\begin{note}
  Similar, though different from \citeauthor{Chomsky:2015aa}'s distinction between competence and performance.

  \begin{quote}
    Arithmetical competence yields the correct number z for every pair (x, y) under addition or multiplication.
    But only a small finite subpart of arithmetical competence can be exhibited without external aids (by calculating in one's head).
    Obviously, that fact does not imply that arithmetical competence is correspondingly limited.\newline
    \mbox{ }\hfill\mbox{(\citeyear[xii]{Chomsky:2015aa})}
  \end{quote}

  Agent may not, strictly, have arithmetical competence, as understood by \citeauthor{Chomsky:2015aa}.
  For, \emph{arbitrary} pairs.%
  \footnote{
    \citeauthor{Boghossian:2008vf}, following \citeauthor{Kripke:1982aa}, strengthens.
    Rules out any dispositional account.
    However, no assumption against dispositions.
  }

  At issue here is not errors.
  Though, \citeauthor{Chomsky:2015aa} also motivates by errors (\citeyear[2]{Chomsky:2015aa})
  Rather, it's that there's no way for an agent to do this.
\end{note}

\begin{note}
  This applies even if restrict.

  For example, fine with multiplication.
  However, from some instance, the only thing the agent reasons to is:
  `Heck this, I'm getting a calculator'.
\end{note}

\begin{note}
  And, deviance.
\end{note}

\begin{note}
  Of course, types are extensionally defined.
  So, in principle there is no problem with restricting the relevant type.

  However, this doesn't really help.
  For, then there's different types for different instances.

  Suppose have studied for a test.
  Suppose cheerios had been in the cupboard.
  And, suppose tiger isn't present.
\end{note}

\subsection{\rotor{3}}
\label{sec:represntative}

\begin{note}
  \tor{} is general.
  Narrow down.
  In particular, useful for when the agent is limited.
\end{note}

\begin{note}
  Problems from relations to other instances of type.
  Want to ignore.
  So, solution in search of in restricted.

  Solution, whatever it amounts to in detail, forms a conditional.

  If related to these, then sufficient to say of general type.
  Idea of a representative.
\end{note}

\begin{note}
  \begin{restatable}[\rotor{2}]{definition}{defToRRep}
    \label{def:rotor}
    \cenLine{
      \begin{itemize*}[noitemsep, label=\(\circ\)]
      \item
        Agent: \vAgent{}
      \item
        Event: \(e\)
      \item
        Event description: \(d\)
      \item
        Types of reasoning: \(T\), \(T'\)
      \item
        \mbox{ }
      \end{itemize*}
    }

    \begin{itemize}
    \item
      \(T'\) is \emph{representative} of \(T\) for \vAgent{} with respect to an event \(e\) and description \(d\).
    \end{itemize}

    \emph{If and only if}:

    \begin{itemize}
    \item

      \begin{itemize}
      \item[\emph{If}:]
        \begin{enumerate}[label=\alph*., ref=(\alph*), series=tRDef]
        \item
          \(e\) is an event such that \vAgent{}' reasoning is of type \(T'\)
        \end{enumerate}
      \item[\emph{Then}:]
        \begin{enumerate}[label=\alph*., ref=(\alph*), resume*=tRDef]
        \item
          \(e\) is an event such that \vAgent{}' reasoning is of type \(T\).
        \end{enumerate}
      \end{itemize}
    \end{itemize}
    \vspace{-\baselineskip}
  \end{restatable}
\end{note}

\begin{note}
  So, given a representative, avoid difficult cases.
  Allow as many imperfections%
  \footnote{
    I.e.\ counterexamples.
  }
  as one wishes.
  The difficulty is maintaining the conditional.

  I do not have any interest in saying what this amounts to.

  Indeed, I have no particular interest in whether or not reasoning is of any \emph{particular} type.
  Rather, whether the reasoning is of \emph{some} type.
\end{note}

\begin{note}
  Omit a natural constraint.
  Intuitively, everything of type \(T'\) is also of type \(T\).

  I doubt there are any compelling counterexamples.
  However, there is also no need to add this.
  All that matters is that the conditional is true.
\end{note}

\begin{note}
  Now, whether the conditional holds.
  For, it's always going to be the case that wriggle in some doubt.

  However, this is beyond our interest.

  Indeed, any representative of a type is a type.

  And, so long as not trivial, \emph{some} generality.
  This is what we want to capture.
\end{note}

\subsection{\tR{2}}
\label{sec:tr3}

\begin{note}
  \begin{idea}[\tR{2}]
    \label{def:cmptnc}
    \cenLine{
      \begin{itemize*}[noitemsep, label=\(\circ\)]
      \item
        Agent: \vAgent{}
      \item
        Propositions: \(\phi\), \(\psi\)
      \item
        Values: \(v\), \(v'\)
      \item
        \poP{3}: \(\Phi\), \(\Psi\)
      \item
        \mbox{ }
      \end{itemize*}
    }

    \noindent%
    \cenLine{
      \begin{itemize*}[noitemsep, label=\(\circ\)]
      \item
        Event: \(e\)
      \item
        Event description: \(d\)
      \item
        Type of reasoning: \(T\)
      \item
        \mbox{ }
      \end{itemize*}
    }

    \begin{itemize}
    \item
      \(e\) under description \(d\) is an event in which \vAgent{} is reasoning to \(\pv{\phi}{v}\) from \(\Phi\) by \emph{\tR{0}} of type \(T\).
    \end{itemize}

    \emph{Only if}:

    \begin{itemize}[noitemsep]
    \item
      For some representative \(T'\) of \(T\) with respect to \(e\) under \(d\):
      \begin{itemize}[noitemsep]
      \item
        For every pairing \(\pvp{\psi}{v'}{\Psi}\) in \(T'\).
        \begin{itemize}[noitemsep]
        \item[\emph{If}:]
          \begin{enumerate}[label=\alph*., ref=(\alph*), series=tRSketch]
          \item
            There is some available action \(a\) such that \vAgent{} would be reasoning from \(\Psi\) after performing \(a\).
          \end{enumerate}
        \item[\emph{Then}:]
          \begin{enumerate}[label=\alph*., ref=(\alph*), resume*=tRSketch]
          \item
            There is some available action \(a'\) such that \vAgent{} would be reasoning to \(\pv{\phi}{v}\) from \(\Psi\) after performing \(a'\).
          \end{enumerate}
        \end{itemize}
      \end{itemize}
    \end{itemize}
    \vspace{-\baselineskip}
  \end{idea}

  So, in short, pair descriptions of the event, and make sure that we have \pevent{1}.

  Intuitively, other things, choice of action.
  And, premises and conclusion.
  So, we get it to be the case that different action, different result.
  If this fails, then bad.
  Doing something which doesn't correspond to the type.
\end{note}

\begin{note}
  The work here is done by representative.

  Note, the definition fails if done in terms of type.
  For, it may then be the case that satisfies necessary condition for representative, but does not satisfy condition for type.
  Hence, no way for the conditional for representative to be true.
\end{note}


\begin{note}
  So, progressive.
  This means that there's enough from description to get an event in which agent reasons to \(\pv{\phi}{v}\) from \(\Phi\).

  However, it is not the case that there is enough from description to get an event in which agent reasons to \(\pv{\psi}{v'}\) from \(\Psi\).

  Well, so, only if \(\bigvee\pvp{\psi}{v'}{\Psi}\).
\end{note}

\begin{note}
  So, look, we have assumed that for each representative, there is some instance of representative such that no \pevent{}.

  What I want to say is that, if deny each instance of representative, then not of type T.
  For, then, by definition of representative, yeah.
  This part should be clear.

  The difficult part is getting that there needs to be a \pevent{} for each instance of representative.

  Needs to be a \pevent{}.
  Find some action such that progressive is true.

  Suppose no action, but still of type T.
  Then, representative.
  
\end{note}

\begin{note}
  Two problems?

  \begin{itemize}
  \item
    Not enough information to get T
  \item
    Alternative type.
  \end{itemize}
\end{note}


\begin{note}
  \color{blue}
  So, absence of something which makes it the case that \pevent{}.
  
\end{note}

\begin{note}
  Generality, but nothing in description gets generality.

  In what way does this work?

  Well, progressive.
  Nothing to make the progressive true.
\end{note}

\begin{note}
  Second part.
  Not at loss for options.

  Look, instead, different type.
\end{note}


\begin{note}
  Intuitively, hold fixed the way in which the agent is reasoning.
  Then \dots

  Intuition is that necessary condition on \tR{} is reasoning in the same way from different premises.

    Other proposition-value-\poP{} pairings, reasoning if and only if applied proposition-value-\poP{} pairings.

  In more plain terms, reasoning is specific instance of general way of reasoning.
\end{note}

\begin{note}
  \color{blue}

  So, what is the idea, exactly?

  Well, we've got representatives.
  Just need one representative to hold.
  For, then get type, by the ways \rotor{1} are defined.
  What must be the case in order for this representative to hold?

  Well, suppose there's no potential event.
  Then, it is not the case that the agent would get to the same reasoning.

  There are two things here.

  First, it could be that this falls due to problems as outlined.
  There is something outside the description of the event, so that the representative fails to apply.

  Second, it could be that the agent isn't performing type of reasoning.
  If the agent isn't performing type of reasoning, then something fails to extend.
\end{note}

\begin{note}
  Proof by contraposition.

  Suppose fails.
  Then, for all representatives, there is some pairing, and no \pevent{}.

  So, by assumption, reasoning to \(\pv{\phi}{v}\) from \(\Phi\), so we know this has got to be some distinct pair.
  Okay, but, no \pevent{}.

  So, this means, that \(\pv{\phi}{v}\) from \(\Phi\) alone is not enough to be of type.
  In order for it to be the case, something must be said about this other pair.

  But, no \pevent{}.
  If no \pevent{}, then not of type.

  So, why this is the case is unspecified.
  Motivation is in terms of generality.
  If there is something which distinguishes reasoning from type, and find something the agent should


  So, type, generality.
  However, it seems this amounts to possibility of reasoning.
  But, restrict, and possibility doesn't hold.
\end{note}

\begin{note}
  \color{red}
  So, focusing on representatives.
  The point is that isolated failure is not clearly sufficient.
  It's not the case that, in general, perhaps, that \pevent{} for some proposition-value-\poP{} pair is necessary.

  With representatives, it means there's a problem across anything that would be sufficient.

  There is some systematic failing.

  Though, motivation by initial case.
  Maybe, maybe not.

  The key idea I want to press is that, failure to make the progressive true.
  So, if there's no action, then remove this from representative.
  Hence, at issue \emph{is} the failure of the progressive.

  There's nothing to be said with respect to the event or parts, in particular the agent, that extends present reasoning to other instance of the type.

  The core thing here is the failure of the progressive, and hence option to act.
  So, breaking down the definition of a \pevent{}.
\end{note}

\begin{note}
  The upshot, then, is some systematic failure.
  We're not pointing to any specific instance of type.
  Instead, we've identified across a wide range.
  Unless, of course, you think that one case is sufficient.
\end{note}

\begin{note}
  Well, this is not a proof, rather, this is getting at the idea the definition captures.
\end{note}

\subsection{\illu{3} of \tR{}}
\label{sec:illu3}

\begin{note}
  Three \scen{1}.
  For each \scen{0}, identify type, outline representative, and consider.
\end{note}


\begin{note}
  \begin{illustration}[Tic-tac-toe]
    Game.
  \end{illustration}

  Two types here.
  Type is reasoning as a competent tic-tac-toe player.

  So, in this case, don't leave open sure win unless forced.

  Second, no defeat.
  Here, follow the algorithm.

  If either thing goes wrong, then something is up.
\end{note}

\begin{note}
  \begin{illustration}[Translation]
    Reading an newspaper.

    Headlines.

    Request, translation.

    Okay, and other headlines.
  \end{illustration}

  Keep request fixed, still translate different headline.

  Also, not a function, as multiple distinct translations.

  And, style of translation.

  \begin{center}
    \begin{tabular}[h]{p{.55\linewidth}|p{.4\linewidth}}
      \multicolumn{2}{c}{\emph{Translation}} \\
      これはペンです & This is a pen. \\
      このものは万年筆だ & This is a pen. \\
      この問題は難しいですね & This is a difficult problem, huh. \\
      この問題は難しいですね & This problem is kind of tough. \\
    \end{tabular}
  \end{center}
\end{note}


\subsubsection{Edge cases}
\label{sec:not-tr0}

\begin{note}
  Single instance.

  So, working on an exam.
  Here, in order to get some type, need it to be the case that single representative is sufficient.

  Well, we haven't specified the way in which the conditional is true.
  However, we have to go from specific to general in some way.

  Okay, so relative to description.

  So, if build in.
  And, it's compatible that there isn't anything further to be said.
  It may just be basic that this is general reasoning.
\end{note}


\subsubsection{Alternatives}
\label{sec:alternatives}

\begin{note}
  Definition, necessary condition.

  Nearby definitions add.

  For example, rather than action to the agent, go via prompting.
  I give question, you get answer.

  This provides a nice account of the student taking a test.

  However, counterfactuals.
\end{note}

\paragraph{Error}

\begin{note}
  Representatives also allow error.
  Careful, though.
  Ah, I'll have this above.
\end{note}

\subsection{\tC{2}}
\label{cha:typical:sec:sTR}

\begin{note}
  With reasoning in hand, expand our understanding of typicality to concluding.

  \begin{note}
  \begin{restatable}[\tC{2}]{definition}{defTRConcluding}
    Same but with addition.
  \end{restatable}

  Motivation for this is simple.
  \tR{} just looks at instance of reasoning.
  This is okay, but less interesting.
  For, if fail to do something of same type, then what is really going on?

  So, do things of type, and do not deviate from type.
\end{note}

\end{note}

% \begin{note}
%   \begin{restatable}[\sTR{2}]{definition}{defSTR}
%     \sTR{} just in case:

%     \begin{itemize}
%     \item
%       \tR{}.
%     \item
%       Every representative contains some proposition-value-\poP{} pair distinct from \(\pvp{\phi}{v}{\Phi}\).
%     \end{itemize}
%   \end{restatable}

%   Motivation for this is simple.
%   \tR{} just looks at instance of reasoning.
%   This is okay, but less interesting.
%   For, if fail to do something of same type, then what is really going on?

%   So, do things of type, and do not deviate from type.
% \end{note}

% \begin{note}
%   \begin{proposition}
%     \label{prop:sTRStronger}
%     An instance of reasoning being a instance of \sTR{} is strictly stronger than the instance being \tR{}.
%   \end{proposition}

%   \begin{argument}{prop:sTRStronger}
%     Taking an exam.

%     If the agent started reasoning about any other instance, then the agent would immediately switch back to question on the exam.
%     So, not reasoning to, as no perfection.
%   \end{argument}

%   This is interesting.
%   For, extend this, and not a \fc{}.

%   This may seem surprising.
%   However, it is the result of focusing on specific event.
% \end{note}

\begin{note}
  \begin{proposition}
    Cases where:

    If event, and \tC{}, \fc{}.
  \end{proposition}

  Here, the interest is with respect to filling in a number after filling in a previous number.
  Doing some reasoning.
  This reasoning is \tR{}.
  So, repeat previous.
\end{note}

\begin{note}
  Proposition relies on the existence of an event.
  However, existence of an event is of little interest.
  If continues, then event as desired.
\end{note}

\begin{note}
  proposition is narrow, as only applies to illustration.

  However, the motivating idea is general.
\end{note}

\section{Summary}
\label{cha:typical:sec:summ}

\begin{note}
  Typical reasoning.

  Extension account of \tor{}.
  Due to abstracting over theories.

  Then, necessary condition on \tR{}.
\end{note}



\newpage

% \section[\citeauthor{Carroll:1895uj}]{\citeauthor{Carroll:1895uj}\hfill(Optional)}

% \nocite{Black:1951aa}

% \begin{note}
%   The point here is that with Carroll, generality that goes beyond any single instance.
%   Must apply to all instances, to be valid.
%   But, cannot hope to cover all instances in a single move.
% \end{note}

% \begin{note}
%   A difficulty found on a reading of \citeauthor{Carroll:1895uj}'s \citetitle{Carroll:1895uj}.
% \end{note}

% \begin{note}
%   \begin{quote}
%     ``Plenty of blank leaves, I see!'' the Tortoise cheerily remarked.
%     ``We shall need them \emph{all}!''
%     (Achilles shuddered.)
%     ``Now write as I dictate:---

%     \begin{enumerate}[label=(\emph{\Alph*}), ref=\emph{\Alph*}]
%     \item
%       \label{AatT:a}
%       Things that are equal to the same are equal to each other.
%     \item
%       \label{AatT:b}
%       The two sides of this Triangle are things that are equal to the same.
%     \item
%       \label{AatT:c}
%       If~\ref{AatT:a} and~\ref{AatT:b} are true,~\ref{AatT:z} must be true.
%       \setcounter{enumi}{25}
%     \item
%       \label{AatT:z}
%       The two sides of this Triangle are equal to each other.''
%     \end{enumerate}

%     ``You should call it~\ref{AatT:d}, not~\ref{AatT:z},'' said Achilles.
%     ``It comes \emph{next} to the other three.
%     If you accept~\ref{AatT:a} and~\ref{AatT:b} and~\ref{AatT:c}, you \emph{must} accept~\ref{AatT:z}.''

%     ``And why \emph{must} I?''

%     ``Because it follows \emph{logically} from them.
%     If~\ref{AatT:a} and~\ref{AatT:b} and~\ref{AatT:c} are true,~\ref{AatT:z} \emph{must} be true.
%     You don't dispute \emph{that}, I imagine?''

%     ``If~\ref{AatT:a} and~\ref{AatT:b} and~\ref{AatT:c} are true,~\ref{AatT:z} \emph{must} be true,'' the Tortoise thoughtfully repeated.
%     ``That's \emph{another} Hypothetical, isn't it?
%     And, if I failed to see its truth, I might accept~\ref{AatT:a} and~\ref{AatT:b} and~\ref{AatT:c}, and \emph{still} not accept~\ref{AatT:z}, mightn't I ?''

%     \mbox{}\hfill\(\vdots\)\hfill\mbox{}

%     ``Then Logic would take you by the throat, and force you to do it!''
%     Achilles triumphantly replied.
%     ``Logic would tell you 'You ca'n't help yourself.''%
%     \mbox{ }\hfill\mbox{(\citeyear[279--280]{Carroll:1895uj})}
%   \end{quote}

%   The Tortoise has written down three premises,~\ref{AatT:a},~\ref{AatT:b}, and~\ref{AatT:c}.
%   Achilles holds that~\ref{AatT:z} follows from~\ref{AatT:a},~\ref{AatT:b}, and~\ref{AatT:c}.
%   The Tortoise observes they have the possibility of refraining to accept~\ref{AatT:z} follows from~\ref{AatT:a},~\ref{AatT:b}, and~\ref{AatT:c}.
%   And (initially), the Tortoise does not accept~\ref{AatT:z} follows from~\ref{AatT:a},~\ref{AatT:b}, and~\ref{AatT:c}.
%   Achilles requests the Tortoise accepts that~\ref{AatT:z} follows from~\ref{AatT:a},~\ref{AatT:b}, and~\ref{AatT:c}, and the Tortoise complies.
%   Specifically, the Tortoise grants:

%   \begin{quote}
%     \begin{enumerate}[label=(\emph{\Alph*}), ref=\emph{\Alph*}]
%       \setcounter{enumi}{3}
%     \item
%       \label{AatT:d}
%       If~\ref{AatT:a} and~\ref{AatT:b} and~\ref{AatT:c} are true,~\ref{AatT:z} must be true.%
%       \mbox{ }\hfill\mbox{(\citeyear[279]{Carroll:1895uj})}
%     \end{enumerate}
%   \end{quote}

%   But, does not accept~\ref{AatT:z} follows from~\ref{AatT:a},~\ref{AatT:b},~\ref{AatT:c}, and~\ref{AatT:d}.
% \end{note}

% \begin{note}
%   Modus ponens.

%   \begin{quote}
%     From \(\phi\) and \emph{if} \(\phi\) then \(\psi\), infer \(\psi\).
%   \end{quote}

%   Modus ponens is general.
%   For \emph{any} \(\phi\), \(\psi\).

%   Now, there is a difference between \emph{modus ponens} and conditional.

%   However, take any instance.
%   Then, if \(P\), \(P \rightarrow Q\), \(Q\) must be true.
%   But, then this means that the conditional is true.

%   Consequence of the deduction theorem.

%   Likewise, deduction theorem goes the other way.

%   However, going from \(P\), \(P \rightarrow Q\) to \(Q\) need not be an instance of \emph{modus ponens}.
% \end{note}

% \begin{note}
%   Well, this is a headache.
%   \citeauthor{Carroll:1895uj} is talking about a specific A, B, and Z.
%   There is no clear generality.
% \end{note}

% \begin{note}
%   So, consider at issue is modus ponens.
%   For any specific instance accept, there is a further instance.
%   For, \(A, (A \rightarrow B) \vDash B\).
%   Then, \(\vDash (A \land (A \rightarrow B) \rightarrow B)\).
%   However, now, \(A \land (A \rightarrow B), (A \land (A \rightarrow B) \rightarrow B) \vDash B\).
%   And, so on.

%   The general pattern, get conditional, but then this gives a new instance of modus ponens, which must be true in order for modus ponens to be valid rule of inference.

%   \citeauthor{Carroll:1895uj}, by contrast, starts with \(A \vDash B\).
%   This is different.
%   However, rather than focusing on a single rule of inference, the puzzle turns on what validity amounts to.

%   Validity is a general thing, with specific instances.
%   However, grant any particular instance of validity without employing validity in general.
% \end{note}

% \begin{note}
%   \begin{quote}
%     My paradox \dots turns on the fact that, in a Hypothetical, the \emph{truth} of the Protasis, the \emph{truth} of the Apodosis, and the \emph{validity of the sequence}, are 3 distinct Propositions.

%     \mbox{}\hfill\(\vdots\)\hfill\mbox{}

%     Suppose I say ``I grant~\ref{AatT:a} and~\ref{AatT:b} and~\ref{AatT:c}, but I do \emph{not} grant that I am thereby \emph{obliged} to grant~\ref{AatT:z}.''
%     Surely, my granting~\ref{AatT:z} must \emph{wait} until I have been made to see the validity of this sequence: i.e.\ in order to grant~\ref{AatT:z}, I must grant~\ref{AatT:a},~\ref{AatT:b},~\ref{AatT:c}, and~\ref{AatT:d}! And so on.%
%     \mbox{ }\hfill\mbox{(\citeyear[472]{Carroll:1977wl})}
%   \end{quote}

%   My interpretation of the point \citeauthor{Carroll:1895uj} makes in this passage is that the truth of A B and the truth of C is distinct from the validity of A B C.
%   Granting is substantial, not merely moving.
%   But, in order to grant, this means granting all other cases.

%   So, the paradox is that, on the one hand, don't need validity for any specific true things.
%   But, on the other hand, only of interest if via validity.

%   The Tortoise is slowly working through each instance, but this has no hope of getting the Tortoise to general validity.
%   So, how does the Tortoise ever make it there?
% \end{note}

% \begin{note}
%   This point differs from received interpretation.

%   \citeauthor{Wieland:2013vf} (\citeyear{Wieland:2013vf}) characterises the general understanding of \textcite{Carroll:1895uj} in terms of two lessons:
%   \begin{quote}
%     [T]he negative lesson is that if you add ever more premises to an argument \dots, then you will never demonstrate that its conclusion follows logically.\newline
%     \mbox{ }\hfill\mbox{(\citeyear[984]{Wieland:2013vf})}
%   \end{quote}

%   Parallel, static answers, still option for concluding otherwise.

%   \begin{quote}
%     [T]he positive lesson is that rules of inference, rather than premises of the form `if premises such and such are true, then the conclusion is true', will do the job.\newline
%     \mbox{ }\hfill\mbox{(\citeyear[984]{Wieland:2013vf})}
%   \end{quote}

%   \begin{quote}
%     [\citeauthor{Carroll:1895uj}] simply lacked any distinct conception of a deduction as opposed to the assertion (``granting'' of) a hypothetical proposition.
%     \dots
%     Any attempt by Carroll to tackle the question of inference was bound to begin in confusion and end in constipation-all those premises piling up, but no motion.
%   \end{quote}
% \end{note}

% \paragraph{The Dichotomy}

% \begin{note}
%   Achilles and the Tortoise, Zeno's argument.

%   Surely, right?

%   Two ways to understand.
%   Does the Tortoise move at all, or does the Tortoise arrive at the end?
%   I mean, as formulated by Zeno, it's about catching up, no matter how much one moves.

%   It is different from Zeno's Dichotomy paradox.


%   If so, then we should expect the Tortoise to be making some movement.
%   Adding rules of inference is of no help, because the problem is not movement, it's about how to move so much in a single step.
% \end{note}

% \begin{note}
%   \color{red}
%   Something about logic forcing.
%   The Tortoise hasn't arrived.

%   Nothing hangs on validity.
%   Same issue with testimony.
%   `A'.
%   Why?
%   Testified A, so A.
%   Okay, but another instance of testimony.
%   Testified(Testified A, so A), so Testified A, so A.
% \end{note}

% \begin{note}
%   \begin{quote}
%     But if we who wish to represent his belief in Q as based on P are to write in our notebook everything his having that belief on that basis consists in then when we have written only P and Q we will not have written enough.
%     Someone can believe P and believe Q and still not believe Q on the basis of P whatever the relations between the propositions P and Q happen to be.
%     He might believe Q for some reason completely unconnected with P, or perhaps for no reason at all (if that is possible).%
%     \mbox{ }\hfill\mbox{(\citeyear[185]{Stroud:1979aa})}
%   \end{quote}
%   However, the moral drawn is narrow
%   \begin{quote}
%     The moral is that for every proposition or set of propositions the belief or acceptance of which is involved in someone's believing one proposition on the basis of another there must be something else, not simply a further proposition accepted, that is responsible for the one belief's being based on the other.%
%     \mbox{ }\hfill\mbox{(\citeyear[187]{Stroud:1979aa})}
%   \end{quote}

%   Even if we grant each individual is \ros{}, rather than an instance of the material conditional, \emph{logic} hasn't done anything.
% \end{note}

% \paragraph{General and specific: Contrast}

% \begin{note}
%   Use \citeauthor{Carroll:1895uj} to illustrate this point.

%   However, given the worry, various other things may be understood this way.

%   Hume, constant conjunction.
%   Part of the problem is identifying cause.
%   We get the famous line about observing.
%   However, Hume goes on.
%   It's not only no observation, but no generality.

%   Right, so more narrow than Hume.
%   Because, with Hume, at issue is whether we have grounds for this general thing.
%   With Carroll, it's whether we even really get to the general thing.
% \end{note}


%%% Local Variables:
%%% mode: latex
%%% TeX-master: "master"
%%% End:


% \begin{note}
%   \begin{quote}
%     Let me ask this: what has the expression of a rule—say a sign-post—got to do with my actions?
%     What sort of connexion is there here?%
%     ---%
%     Well, perhaps this one:
%     I have been trained to react to this sign in a particular way, and now I do so react to it.

%     But that is only to give a causal connexion; to tell how it has come about that we now go by the sign-post; not what this going-by-the sign really consists in.
%     On the contrary; I have further indicated that a person goes by a sign-post only in so far as there exists a regular use of sign-posts, a custom.%
%     \mbox{ }\hfill\mbox{(\citeyear[\S198]{Wittgenstein:1958aa})}
%   \end{quote}

%   Regular use of sign-posts, custom.

%   Ugh, this is ambiguous.
% \end{note}


% %
%   \(^{,}\)
%   \footnote{
%     \citeauthor{Hlobil:2014tq}'s ``Inferential Moorean Phenomenon'':
%   \begin{quote}
%     \begin{enumerate}
%     \item[(IMP)]
%       It is either impossible or seriously irrational to infer \emph{P} from \emph{Q} and to judge, at the same time, that the inference from \emph{Q} to \emph{P} is not a good inference.
%     \end{enumerate}
%     \dots
%     By the ``goodness'' of an inference I mean the feature that makes the relevant inference permissible. Thus, if the inference under consideration is an inductive inference, the relevant kind of goodness is not deductive validity.%
%     \mbox{ }\hfill\mbox{(\citeyear[\S1]{Hlobil:2014tq})}
%   \end{quote}
%   Though, this really isn't more basic given the interest in \tR{}.
%   For, the puzzle is what it is to `infer'.

%   Rationality isn't part of the picture.
%   And, this is a significant drawback of \citeauthor{Hlobil:2014tq}'s approach.
% }


% \subsection{Types and explanation}
% \label{cha:typical:sec:tor:bkgd}

% \begin{note}
%   There is a related, stronger claim, that generality derives from rule following.

%   For this, \citeauthor{Boghossian:2008vf}:

%   \begin{quote}
%     [O]ur internalization of general epistemic rules---like Modus Ponens and Induction---explain and rationalize why we form the beliefs that we form.
%     And that seems intuitively correct.

%     As in the case of our linguistic and conceptual abilities, our ability to form rational beliefs is \emph{productive}: on the basis of finite learning, we are able to form rational beliefs under a potential infinity of novel circumstances.
%     The only plausible explanation for this is that we have, somehow, internalized a rule that tells us, in some general way, what it would be rational to believe under varying epistemic circumstances.%
%     \mbox{ }\hfill\mbox{(\citeyear[483]{Boghossian:2008vf})}
%   \end{quote}

%   Strictly, \citeauthor{Boghossian:2008vf}, rules \textquote{represent our conception of how it would be most rational for a thinker to form beliefs under different epistemic circumstances} (\citeyear[473]{Boghossian:2008vf}).

%   The difference in approach is clearest with \citeauthor{Boghossian:2008vf}'s account of modus ponens:%
%   \footnote{
%     \citeauthor{Boghossian:2008vf} notes the rule is distinct from modus pones as found in textbooks.
%     Remarks: \textquote{It is actually quite mysterious what the logic textbook rule is supposed to be} (\citeyear[472,fn.2]{Boghossian:2008vf})
%     I don't think there is any mystery about the rule in most logic textbooks.
%     Instead, the mystery is the way in which logic relates to reasoning.
%     (Cf.~\cite{Harman:1986ux,MacFarlane:2004aa,Steinberger:2022aa}, etc.)
%     % Issue for the presentation.
%     % Literature is full of issues.
%     % The most well known, Gricean pragmatics.
%     % Though, also McGee, McFarlane, biscuit conditonals, the miners paradox, etc.
%   }

%   \begin{quote}
%     (Modus Ponens):
%     If you are rationally permitted to believe both that \emph{p} and that `If \emph{p}, then \emph{q}', then, you are prima facie rationally permitted to believe that \emph{q}.%
%     \mbox{ }\hfill\mbox{(\citeyear[472]{Boghossian:2008vf})}
%   \end{quote}

%   Here, we have permissions.
%   What the agent is allowed to do.
%   However, this is distinct from what the agent does.
% \end{note}

% \begin{note}
%   \tR{} is distinct.
%   Whether came to \emph{q} from \emph{p} , if \emph{p} then \emph{q}.

%   Rationality is not part of our understanding.
%   Rather, generality.%
%   \footnote{
%     Observe, ~\cite{Kolodny:2005aa} is of no interest here.
%     Why be rational is distinct from whether there is some generality.
%   }
% \end{note}

% \begin{note}
%   Likewise, means-end reasoning is distinct from \citeauthor{Broome:2013aa}'s

%   \begin{quote}
%     \emph{End to Means Transmission}.
%     ((\emph{S} requires of \emph{N} that \emph{p}) \& necessarily \newline (\emph{p} \(\supset\) \emph{q}) \& \emph{q} is a means to \emph{p}) \(\supset\) (\emph{S} requires of \emph{N} that \emph{q}).%
%     \mbox{ }\hfill\mbox{(\citeyear[126]{Broome:2013aa})}
%   \end{quote}

%   \emph{S} is some source, such as morality.
%   \emph{N} is a person. (\citeyear[117]{Broome:2013aa})

%   Instead, the significantly weaker idea that the agent has reasoned from some end to a means to that end.
% \end{note}


% \begin{note}
%   On my understanding, this is, in part, the role of \citeauthor{Boghossian:2014aa}'s Taking Condition.

%   Way in which \dots

%   Indeed, \citeauthor{Boghossian:2014aa} highlights how condition allow to draw distinction between deductive and inductive.
%   With taking, get generality.

%   Indeed, \textcite{Boghossian:2014aa} is structured so that Taking is a generalisation of rule.
% \end{note}

% \begin{note}
%   However, \tor{} does not need to amount to a rule.
%   Rather, \tR{} only requires the rough phenomenon that \citeauthor{Boghossian:2008vf} argues rule following is the only plausible explanation of.%
%   \footnote{
%     Our interest with \tor{1} is independent of the worries about rule following raised by~\textcite{Kripke:1982aa}, to the extent that the worries raised by~\citeauthor{Kripke:1982aa} concern \emph{which} rule an agent is following, rather than \emph{whether} the agent is following a rule.
%     At interest is not whether the \tor{} corresponds to plus or quus, but whether the agent's reasoning is of some type.
%   }
% \end{note}

% \begin{note}
%   Same for modus ponens.

%   \citeauthor{Davies:2004aa} discussing~\textcite{Wright:2004aa} with respect to~\citeauthor{Moore:1959aa}'s proof of an external world (\citeyear{Moore:1959aa}):

%   \begin{quote}
%     Moore's argument can be set out as follows:
%     \begin{quote}
%       \begin{enumerate}[label=MOORE (\Roman*), ref=MOORE (\Roman*)]
%       \item
%         \label{MoorePoEW:1}
%         I am having an experience as of one hand [here] and another [here].
%       \item
%         \label{MoorePoEW:2}
%         I have hands.

%         If I have hands then an external world exists.
%       \end{enumerate}

%       Therefore:

%       \begin{enumerate}[label=MOORE (\Roman*), ref=MOORE (\Roman*), resume]
%       \item
%         \label{MoorePoEW:3}
%         An external world exists.
%       \end{enumerate}
%     \end{quote}

%     [\dots] the key question at this point in Wright's account is whether the support for~\ref{MoorePoEW:2} is transmitted to~\ref{MoorePoEW:3} across the modus ponens inference in which the conditional premise is supported by an elementary piece of philosophical theorising.\newline
%     \mbox{ }\hfill\mbox{(\citeyear[215]{Davies:2004aa})}
%   \end{quote}
% \end{note}
