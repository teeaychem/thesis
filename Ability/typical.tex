\chapter{\sR{2}}
\label{cha:typical}

\section{\sR{2}}
\label{sec:gr-sr}

{
  \color{red}
  Somewhere, that this is not a necessary condition for concluding.
}

\begin{note}

  \begin{definition}[\sR{2}]
    \label{def:cmptnc}
    The reasoning of \(e\) is \emph{\sR{0}}, just in case, instance of a type of reasoning.
  \end{definition}

  In more plain terms, reasoning is specific instance of general way of reasoning.%
  \footnote{
    Not designed to use `type' in the sense of type-token distinction.
    Common characteristic.
    Though, type distinct from instances.
  }

  \begin{itemize}
  \item
    2 + 2 = 4 Addition
  \item
    P, if P Q, Q Modus ponens
  \item
    Board, checkmate Chess
  \item
    これはペンです This is a pen Language
  \item
    P(A | B), PB(B) Bayesian conditioning
  \end{itemize}

  \autoref{def:cmptnc} does not specify what the type reasoning is.
  Focus is not on the classification, but that reasoning may be classified.

  For each of the examples, alternatives.
  \begin{itemize}
  \item
    Jeffrey conditionalization
  \item
    \emph{Modus profusus}%
    \footnote{
      \textquote{for any \(p\), \(q\), and \(r\): \((p \land q) \rightarrow r\)} (\cite[317]{Turri:2010aa})
    }
  \end{itemize}
\end{note}



\paragraph{Rule following}

\begin{note}
  So, this relationship may be understood in terms of rule following.

  For this, \citeauthor{Boghossian:2008vf}:

  \begin{quote}
    [O]ur internalization of general epistemic rules---like Modus Ponens and Induction---explain and rationalize why we form the beliefs that we form.
    And that seems intuitively correct.

    As in the case of our linguistic and conceptual abilities, our ability to form rational beliefs is \emph{productive}: on the basis of finite learning, we are able to form rational beliefs under a potential infinity of novel circumstances.
    The only plausible explanation for this is that we have, somehow, internalized a rule that tells us, in some general way, what it would be rational to believe under varying epistemic circumstances.%
    \mbox{ }\hfill\mbox{(\citeyear[483]{Boghossian:2008vf})}
  \end{quote}

  However, \sR{} does not require rule following.
  Rather, \sR{} only requires the rough phenomenon that \citeauthor{Boghossian:2008vf} argues rule following is the only plausible explanation of.
\end{note}

\begin{note}
  Similar, though different from \citeauthor{Chomsky:2015aa}'s distinction between competence and performance.

  Similarity is generality.
  Difference is idealisation.

  On our understanding, competence does not involve idealisation.

  \begin{quote}
    Arithmetical competence yields the correct number z for every pair (x, y) under addition or multiplication.
    But only a small finite subpart of arithmetical competence can be exhibited without external aids (by calculating in one's head).
    Obviously, that fact does not imply that arithmetical competence is correspondingly limited.\newline
    \mbox{ }\hfill\mbox{(\citeyear[xii]{Chomsky:2015aa})}
  \end{quote}

  Agent may not, strictly, have arithmetical competence, as understood by \citeauthor{Chomsky:2015aa}.
  For, \emph{arbitrary} pairs.

  Whatever the generality of \sR{0} on our understanding, it may fail for arbitrary pairs.
  No guarantees about idealisation.
  Still, general.
  Many instance of multiplying, and, when multiplying, reasoning is, typically, at least, \sR{}.
\end{note}

\begin{note}
  Primary upshot is that \sR{} does not depend on resolutions to rule following paradox.

  To illustrate.
  Strictly, \sR{} does not depend on addition.
  Compatible that \sR{} is smaddition(?)

  Though, does depend on basic assumption of generality.
  To the extent worried about smaddition(?) due to no generality, then sure.
\end{note}

\begin{note}
  \begin{quote}
    Let me ask this: what has the expression of a rule—say a sign-post—got to do with my actions?
    What sort of connexion is there here?%
    ---%
    Well, perhaps this one:
    I have been trained to react to this sign in a particular way, and now I do so react to it.

    But that is only to give a causal connexion; to tell how it has come about that we now go by the sign-post; not what this going-by-the sign really consists in.
    On the contrary; I have further indicated that a person goes by a sign-post only in so far as there exists a regular use of sign-posts, a custom.%
    \mbox{ }\hfill\mbox{(\citeyear[\S198]{Wittgenstein:1958aa})}
  \end{quote}

  Regular use of sign-posts, custom.
\end{note}

\begin{note}
  \sR{} is compatible with dispositionalism.

  \begin{quote}
    But a capacity to grasp infinitely long propositions—the inputs in the rulefollowing case—does not follow from our nature as thinking beings, and certainly not from our nature as physical beings.
    In fact, it seems pretty clear that we do not have that capacity and could not have it, no matter how liberally we apply the notion of idealization.
  \end{quote}
\end{note}

\begin{note}
  Some additional motivation from the literature.
\end{note}

\paragraph{Particularism}

\begin{note}
  Opposing view is particularlism.
  There's nothing general.

  Some motivation from the Wason selection task.
  However, Fodor, etc.
\end{note}

\section{\citeauthor{Carroll:1895uj}}

\nocite{Black:1951aa}

\begin{note}
  The point here is that with Carroll, generality that goes beyond any single instance.
  Must apply to all instances, to be valid.
  But, cannot hope to cover all instances in a single move.
\end{note}

\begin{note}
  A difficulty found on a reading of \citeauthor{Carroll:1895uj}'s \citetitle{Carroll:1895uj}.
\end{note}

\begin{note}
  \begin{quote}
    ``Plenty of blank leaves, I see!'' the Tortoise cheerily remarked.
    ``We shall need them \emph{all}!''
    (Achilles shuddered.)
    ``Now write as I dictate:---

    \begin{enumerate}[label=(\emph{\Alph*}), ref=\emph{\Alph*}]
    \item
      \label{AatT:a}
      Things that are equal to the same are equal to each other.
    \item
      \label{AatT:b}
      The two sides of this Triangle are things that are equal to the same.
    \item
      \label{AatT:c}
      If~\ref{AatT:a} and~\ref{AatT:b} are true,~\ref{AatT:z} must be true.
      \setcounter{enumi}{25}
    \item
      \label{AatT:z}
      The two sides of this Triangle are equal to each other.''
    \end{enumerate}

    ``You should call it~\ref{AatT:d}, not~\ref{AatT:z},'' said Achilles.
    ``It comes \emph{next} to the other three.
    If you accept~\ref{AatT:a} and~\ref{AatT:b} and~\ref{AatT:c}, you \emph{must} accept~\ref{AatT:z}.''

    ``And why \emph{must} I?''

    ``Because it follows \emph{logically} from them.
    If~\ref{AatT:a} and~\ref{AatT:b} and~\ref{AatT:c} are true,~\ref{AatT:z} \emph{must} be true.
    You don't dispute \emph{that}, I imagine?''

    ``If~\ref{AatT:a} and~\ref{AatT:b} and~\ref{AatT:c} are true,~\ref{AatT:z} \emph{must} be true,'' the Tortoise thoughtfully repeated.
    ``That's \emph{another} Hypothetical, isn't it?
    And, if I failed to see its truth, I might accept~\ref{AatT:a} and~\ref{AatT:b} and~\ref{AatT:c}, and \emph{still} not accept~\ref{AatT:z}, mightn't I ?''

    \mbox{}\hfill\(\vdots\)\hfill\mbox{}

    ``Then Logic would take you by the throat, and force you to do it!''
    Achilles triumphantly replied.
    ``Logic would tell you 'You ca'n't help yourself.''%
    \mbox{ }\hfill\mbox{(\citeyear[279--280]{Carroll:1895uj})}
  \end{quote}

  The Tortoise has written down three premises,~\ref{AatT:a},~\ref{AatT:b}, and~\ref{AatT:c}.
  Achilles holds that~\ref{AatT:z} follows from~\ref{AatT:a},~\ref{AatT:b}, and~\ref{AatT:c}.
  The Tortoise observes they have the possibility of refraining to accept~\ref{AatT:z} follows from~\ref{AatT:a},~\ref{AatT:b}, and~\ref{AatT:c}.
  And (initially), the Tortoise does not accept~\ref{AatT:z} follows from~\ref{AatT:a},~\ref{AatT:b}, and~\ref{AatT:c}.
  Achilles requests the Tortoise accepts that~\ref{AatT:z} follows from~\ref{AatT:a},~\ref{AatT:b}, and~\ref{AatT:c}, and the Tortoise complies.
  Specifically, the Tortoise grants:

  \begin{quote}
    \begin{enumerate}[label=(\emph{\Alph*}), ref=\emph{\Alph*}]
      \setcounter{enumi}{3}
    \item
      \label{AatT:d}
      If~\ref{AatT:a} and~\ref{AatT:b} and~\ref{AatT:c} are true,~\ref{AatT:z} must be true.%
      \mbox{ }\hfill\mbox{(\citeyear[279]{Carroll:1895uj})}
    \end{enumerate}
  \end{quote}

  But, does not accept~\ref{AatT:z} follows from~\ref{AatT:a},~\ref{AatT:b},~\ref{AatT:c}, and~\ref{AatT:d}.
\end{note}

\begin{note}
  Modus ponens.

  \begin{quote}
    From \(\phi\) and \emph{if} \(\phi\) then \(\psi\), infer \(\psi\).
  \end{quote}

  Modus ponens is general.
  For \emph{any} \(\phi\), \(\psi\).

  Now, there is a difference between \emph{modus ponens} and conditional.

  However, take any instance.
  Then, if \(P\), \(P \rightarrow Q\), \(Q\) must be true.
  But, then this means that the conditional is true.

  Consequence of the deduction theorem.

  Likewise, deduction theorem goes the other way.

  However, going from \(P\), \(P \rightarrow Q\) to \(Q\) need not be an instance of \emph{modus ponens}.
\end{note}

\begin{note}
  Well, this is a headache.
  \citeauthor{Carroll:1895uj} is talking about a specific A, B, and Z.
  There is no clear generality.
\end{note}

\begin{note}
  So, consider at issue is modus ponens.
  For any specific instance accept, there is a further instance.
  For, \(A, (A \rightarrow B) \vDash B\).
  Then, \(\vDash (A \land (A \rightarrow B) \rightarrow B)\).
  However, now, \(A \land (A \rightarrow B), (A \land (A \rightarrow B) \rightarrow B) \vDash B\).
  And, so on.

  The general pattern, get conditional, but then this gives a new instance of modus ponens, which must be true in order for modus ponens to be valid rule of inference.

  \citeauthor{Carroll:1895uj}, by contrast, starts with \(A \vDash B\).
  This is different.
  However, rather than focusing on a single rule of inference, the puzzle turns on what validity amounts to.

  Validity is a general thing, with specific instances.
  However, grant any particular instance of validity without employing validity in general.
\end{note}

\begin{note}
  \begin{quote}
    My paradox \dots turns on the fact that, in a Hypothetical, the \emph{truth} of the Protasis, the \emph{truth} of the Apodosis, and the \emph{validity of the sequence}, are 3 distinct Propositions.

    \mbox{}\hfill\(\vdots\)\hfill\mbox{}

    Suppose I say ``I grant~\ref{AatT:a} and~\ref{AatT:b} and~\ref{AatT:c}, but I do \emph{not} grant that I am thereby \emph{obliged} to grant~\ref{AatT:z}.''
    Surely, my granting~\ref{AatT:z} must \emph{wait} until I have been made to see the validity of this sequence: i.e.\ in order to grant~\ref{AatT:z}, I must grant~\ref{AatT:a},~\ref{AatT:b},~\ref{AatT:c}, and~\ref{AatT:d}! And so on.%
    \mbox{ }\hfill\mbox{(\citeyear[472]{Carroll:1977wl})}
  \end{quote}

  My interpretation of the point \citeauthor{Carroll:1895uj} makes in this passage is that the truth of A B and the truth of C is distinct from the validity of A B C.
  Granting is substantial, not merely moving.
  But, in order to grant, this means granting all other cases.

  So, the paradox is that, on the one hand, don't need validity for any specific true things.
  But, on the other hand, only of interest if via validity.

  The Tortoise is slowly working through each instance, but this has no hope of getting the Tortoise to general validity.
  So, how does the Tortoise ever make it there?
\end{note}

\begin{note}
  This point differs from received interpretation.

  \citeauthor{Wieland:2013vf} (\citeyear{Wieland:2013vf}) characterises the general understanding of \textcite{Carroll:1895uj} in terms of two lessons:
  \begin{quote}
    [T]he negative lesson is that if you add ever more premises to an argument \dots, then you will never demonstrate that its conclusion follows logically.\newline
    \mbox{ }\hfill\mbox{(\citeyear[984]{Wieland:2013vf})}
  \end{quote}

  Parallel, static answers, still option for concluding otherwise.

  \begin{quote}
    [T]he positive lesson is that rules of inference, rather than premises of the form `if premises such and such are true, then the conclusion is true', will do the job.\newline
    \mbox{ }\hfill\mbox{(\citeyear[984]{Wieland:2013vf})}
  \end{quote}

  \begin{quote}
    [\citeauthor{Carroll:1895uj}] simply lacked any distinct conception of a deduction as opposed to the assertion (``granting'' of) a hypothetical proposition.
    \dots
    Any attempt by Carroll to tackle the question of inference was bound to begin in confusion and end in constipation-all those premises piling up, but no motion.
  \end{quote}

  Achilles and the Tortoise, Zeno's argument.

  Surely, right?

  Two ways to understand.
  Does the Tortoise move at all, or does the Tortoise arrive at the end?
  I mean, as formulated by Zeno, it's about catching up, no matter how much one moves.

  It is different from Zeno's Dichotomy paradox.


  If so, then we should expect the Tortoise to be making some movement.
  Adding rules of inference is of no help, because the problem is not movement, it's about how to move so much in a single step.
\end{note}


%%% Local Variables:
%%% mode: latex
%%% TeX-master: "master"
%%% End:
