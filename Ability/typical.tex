\chapter{\tR{2}}
\label{cha:typical}

\begin{note}
  This chapter, \tR{}.

  Intuitively, \tR{} is reasoning which has some generality.
  Think, modus ponens.
\end{note}

\begin{note}
  Function in relation to counterexamples to \issueConstraint{} is conclusions.
  Relevant conclusion is the result of \tR{} --- or, \tR[concluding]{}.
  That is, `part' of the relevant phenomena.

  Two goals.

  Motivate \tR{}.
  Motivate a necessary condition on \tR{}.
\end{note}

\begin{note}
  No guarantee that this holds.
  Only constraint on conclusion is pairing.
  Likewise, no constraint on reasoning.

  Hence, generality.
\end{note}

\begin{note}
  The primary thing with this chapter is link between \tor{} and \tR{}.

  This link will be somewhat non-standard, but will is suited for purpose.
\end{note}

\begin{note}
  Two important things.

  \begin{itemize}
  \item
    motivation for \tR{}.
  \item
    the way in which we (partially) define \tR{}.
  \end{itemize}
\end{note}

\begin{note}
  \begin{enumerate}[label=]
  \item
    \TOCLine{cha:typical:sec:g-s}
  \item
    \TOCLine{cha:typical:sec:tR}
  \item
    \TOCLine{cha:typical:sec:tC}
  \end{enumerate}
\end{note}

\section{Specific instances of general reasoning}
\label{cha:typical:sec:g-s}

\begin{note}
  We begin with three \scen{0}.
  Each \scen{0} involves, either implicitly or explicitly, a specific instance of some type of reasoning.%
  \footnote{
    Not designed to use `type' in the sense of type-token distinction.
    Common characteristic.
    Though, type distinct from instances.
  }
\end{note}

\begin{note}
  \begin{scenario}[We Are Introduced]
    ``I wonder if you've got such a thing as a balloon about you?''

    \noindent%
    ``A balloon?''

    \noindent%
    ``Yes, I just said to myself coming along:
    `I wonder if Christopher Robin has such a thing as a balloon about him?'
    I just said it to myself, thinking of balloons, and wondering.''

    \noindent%
    ``What do you want a balloon for?'' you said.

    \noindent%
    Winnie-the-Pooh looked round to see that nobody was listening, put his paw to his mouth, and said in a deep whisper:
    ``\emph{Honey!}''

    \noindent%
    ``But you don't get honey with balloons!''

    \noindent%
    ``I do,'' said Pooh.%
    \mbox{ }\hfill\mbox{(\cite[12]{Milne:2009aa})}
  \end{scenario}

  \begin{scenario}[The Cicada and the Fox]
    A cicada was singing on top of a tall tree.
    The fox wanted to eat the cicada, so she came up with a trick.
    She stood in front of the tree and marvelled at the cicada's beautiful song.
    The fox then asked the cicada to come down and show himself, since the fox wanted to see how such a tiny creature could be endowed with such a sonorous voice.
    But the cicada saw through the fox's trick.
    He tore a leaf from the tree and let it fall to the ground.
    Thinking it was the cicada, the fox pounced and the cicada then said,
    `Hey, you must be crazy to think I would come down from here! I've been on my guard against foxes ever since I saw the wings of a cicada in the spoor of a fox.'%
    \mbox{ }\hfill\mbox{(\cite[136]{Aesop:2002aa})}
    % The Sheep and the Injured Wolf, 85--86
    % The Mouse, the Fox, and the Grapes 238
  \end{scenario}

  \begin{scenario}[Humpty Dumpty]
    `I'm afraid I ca'n't quite remember it,'
    Alice said, very politely.

    \noindent%
    `In that case we start afresh,'
    said Humpty Dumpty,
    `and it's my turn to choose a subject---'
    (`He talks about it just as if it was a game!' thought Alice.)
    `So here's a question for you. How old did you say you were?'

    \noindent%
    Alice made a short calculation, and said
    `Seven years and six months.'

    \noindent%
    `Wrong!'
    Humpty Dumpty exclaimed triumphantly.
    `You never said a word like it!'

    \noindent%
    `I thought you meant ``How old \emph{are} you?'''
    Alice explained.

    \noindent%
    `If I'd meant that, I'd have said it,'
    said Humpty Dumpty.

    \noindent%
    Alice didn't want to begin another argument, so she said nothing.\newline
    \mbox{ }\hfill\mbox{(\cite[188]{Carroll:2009aa})}
  \end{scenario}
\end{note}

\begin{note}
  Common is means-end reasoning, and reasoning about means-end reasoning.

  Pooh, end is honey, balloon is a means.
  Robin is confused about how balloon is a means to end.

  End of eating.
  Complementing means.
  Whether or not cicada recognises, understands end, and that anything proposed is a means to that end.

  End, answering question.
  Calculation, means.

  Humpty, something went wrong.
  Not a good means.

  Then, no more arguing.
  And, saying nothing.
\end{note}

\begin{note}
  Pollock.

  \begin{quote}
    Means-end reasoning is concerned with finding the means for achieving goals.
    [\dots]
    To achieve a goal, we consider an action that would achieve it under some specified circumstances and then try to find a way of putting ourselves in those circumstances in order to achieve the goal by performing the action.%
    \mbox{ }\hfill\mbox{(\cite[60]{Pollock:2002aa})}
  \end{quote}

  So, we have a common pattern.
\end{note}

\begin{note}
  However, in what sense in any \scen{0} an instance of that pattern.

  For example, arbitrary.

  However, each \scen{0} suggests otherwise.

  With Pooh, Pooh has a particular idea, and maintains interest in getting a ballooon, and, goes on to perform the means.

  With the cicada, on guard against foxes.
  If the fox had offered something for the cicada to collect, may have chosen this as a means, and cicada recognise this as a means to the end.

  Alice, attempting to answer Humpty's question, and keeps quiet.

  Of course, could be accidental.
  Perhaps these \scen{1} just happened.
\end{note}

\begin{note}
  Familiar worry is deviant causal chains.%
  \footnote{
    A well know example is \citeauthor{Davidson:1973vd}'s climber~(\citeyear[79]{Davidson:1973vd}).

    Believe p, p then q.
    However, mention of q is so nerve that q.
    (The game is lost).
  }

  However, we have not said anything about what reasoning is, and we intended to remain neutral.

  Similarly, whether justified.%
  \footnote{
    \citeauthor{Kripke:1982aa}.
  }

  Or, properly based.%
  \footnote{
    \citeauthor{Schaffer:2010vq}'s (\citeyear{Schaffer:2010vq}) Debasing demon.

    The debasing demon \textquote{throws her victims into the belief state on an improper basis, while leaving them with the impression as if they had proceeded properly} (\citeyear[231]{Schaffer:2010vq})

    \citeauthor{Schaffer:2010vq} really is the way to think about things.
    For, \citeauthor{Schaffer:2010vq}'s entire point is that one may get arbitrary conclusions.
    There's nothing too important regarding basing here.
    And, the other paper is really puzzling in terms of what it offers.
    Parallel, knowledge, truth, therefore any account of knowledge is incompatible with scepticism.

    However, as \textcite{Bondy:2018tk} highlight, it is not clear that this raises problems for accounts of the basing relation.
    In short, beliefs may be formed arbitrarily, but it is not necessarily the case (and is indeed, not the case for significant accounts) that accounts of the basing relation consider arbitrary.

    Though, this is not to say basing relation avoids worries.
    Still, at best, basing is distinct from the way in which belief comes about.
  }

  However, we are not interested in whether or not justified or properly based.
  Only whether the agent's reasoning is of some type.
\end{note}

\subsection{Types and laws}

\begin{note}
  Our interest is in giving a necessary condition.

  Idea pursue, \tRN{} is lawful:%
  \footnote{
    Follow \citeauthor{Hempel:1965aa}'s Deductive-Nomological account of scientific explanation:

    \begin{quote}
      [A Deductive-Nomological] explanation answers the question
      `\emph{Why} did the explanandum-phenomenon occur?'
      by showing that the phenomenon resulted from certain particular circumstances, specified in \(C_{1}, C_{2}, \dots C_{k}\), in accordance with the laws \(L_{1}, L_{2}, \dots L_{\gamma}\).
      By pointing this out, the argument shows that, given the particular circumstances and the laws in question, the occurrence of the phenomenon \emph{was to be expected}; and it is in this sense that the explanation enables us to \emph{understand why} the phenomenon occurred.%
      \mbox{ }\hfill\mbox{(\citeyear[337]{Hempel:1965aa})}
    \end{quote}

    Distinct between understanding and whether or not phenomenon occurred.
    So, much stronger.
    However, safe to take the weaker.
    \qWhyV{} is target, and eventual trade off will be between instances in which agent concludes via \tR{} and \ros{} without \wit{}.
  }

  \begin{idea}[\tRN{2} is lawful]
    \label{idea:tR-law}
    \cenLine{
      \begin{itemize*}[noitemsep, label=\(\circ\)]
      \item
        Agent: \vAgent{}
      \item
        Propositions: \(\phi\), \(\psi\)
      \item
        Values: \(v\), \(v'\)
      \item
        \pool{3}: \(\Phi\), \(\Psi\)
      \item
        Event: \(e\)
      \item
        \mbox{ }
      \end{itemize*}
    }

    \begin{itemize}
    \item
      Event \(e\) is such that \vAgent{} is reasoning to \(\pv{\phi}{v}\) from \(\Phi\) by some type.
    \end{itemize}

    \emph{Only if}:

    \begin{itemize}
    \item
      For every event \(e'\) in some collection of events \(E\):
      \begin{itemize}
      \item[\emph{If}:]
        \(e'\) were to occur.
      \item[\emph{Then}:]
        \(e'\) would be an event in which \vAgent{} is reasoning to \(\pv{\psi}{v'}\) from \(\Psi\), for some proposition-value-\pool{} pair \(\pvp{\psi}{v'}{\Psi}\).
      \end{itemize}
    \end{itemize}
  \end{idea}

  \autoref{idea:tR-law} may seem strong.
  Need the agent to do the reasoning.
  Perhaps it is sufficient that the agent does not conflict with reasoning of the type.
  However, free to limit the relevant collection of events.

  In other words, if circumstances, and not reasoning of type, then reasoning is not of type in initial circumstance.

  Motivation in terms of abstractness.
  We have not said anything about what reasoning is with the exception that to proposition-value pair from \pool{}.
  In this case, sufficient to specify circumstance in which \pool{} and failure of proposition-value pair.

  Before developing this idea, handful of examples.
\end{note}

\subsubsection{\illu{3}}
\label{sec:illu3-1}

\paragraph{The selection task}
\nocite{Wason:1968aa}
\nocite{wason1971natural}

\begin{note}
  Wason selection task and other paradoxes.

  \begin{quote}
    The subjects (students) were presented with an array of cards and told that every card had a letter on one side and a number on the other side, and that either would be face upwards.
    They were then instructed to decide which cards they would need to turn over in order to determine whether the experimenter was lying in uttering the following statement:
    \emph{if a card has a vowel on one side then it has an even number on the other side}.%
    \mbox{ }\hfill\mbox{(\citeyear[145--146]{Wason:1966aa})}
  \end{quote}

  \begin{center}
    An \illu{0} of the selection task.

    [A] [9] [B] [8] [C] [7], etc.
  \end{center}

  \citeauthor{Wason:1966aa} observes the results are consistent with the following hypothesis:
  \begin{quote}
    Subjects assume implicitly that a conditional statement has, not two truth values, but three: true, false and `irrelevant'.
    Vowels with even numbers verify, vowels with odd numbers falsify and consonants with any number are irrelevant.%
    \mbox{ }\hfill\mbox{(\citeyear[146]{Wason:1966aa})}
  \end{quote}

  As \citeauthor{Johnson-Laird:1969aa} summarise, Wason \textquote{argues that adults do not treat the conditional in a truth-functional manner: they consider it to be irrelevant when its antecedent is false} (\citeyear[367]{Johnson-Laird:1969aa}).
  Indeed, \citeauthor{Johnson-Laird:1969aa} additional results regarding how the (material) conditional is expressed.%
  \footnote{
    \citeauthor{Johnson-Laird:1969aa} also find the (material) conditional as expressed by the conditional is understood in line with \citeauthor{Wason:1966aa}'s hypothesis (and likewise if the contrapositive is expressed in conditional form).
    This time with detailed results of \emph{twenty four} University College London students!
    Specifically, 19 of the 24 responded as excepted given \citeauthor{Wason:1966aa}'s hypothesis.
    (\citeyear[369,370]{Johnson-Laird:1969aa}).
  }
\end{note}

\begin{note}
  So, we set up a law, and it applies.

  \begin{quote}
    If reason with conditionals in a truth-functional manner, then under circumstance when evaluating, check for any possible occurrences of p and not-q.
  \end{quote}

  From this, in an instance when someone infers q from p and if p then q, then it is not simply due to the truth functional reasoning regarding p and if p then q.

  The key part is the generalisation.
  It is intuitively clear an agent is not reasoning fails to be of some type if the agent fails to reason to an appropriate proposition-value pair.
  However, at issue is other instances of the agent's reasoning.

  In other words, the takeaway of results from \citeauthor{Wason:1966aa}'s selection task is with respect to conditional reasoning in general.
  Even if there are cases of truth-functional reasoning, something separates this from the reasoning found in selection tasks --- does not extend to all circumstances.
\end{note}

\paragraph{Allias}

\begin{note}
  Allias paradox.

  Type of reasoning isn't basic EU, etc.
\end{note}

\begin{note}
  The situations, as presented by \citeauthor{Allais:1979aa}:%
  \footnote{
    All the examples which follow were given in French francs at 1952 prices. 100 French francs in 1952 were worth about the same as 10 dollars in 1975.

    \citeauthor[138, fn.94]{Allais:1979aa}

    That's \$50ish today.
  }
  \begin{quote}
    \begin{enumerate}[label=(\arabic*), ref=(\arabic*)]
    \item
      \emph{Do you prefer Situation A to Situation B}?

      Situation A:
        \begin{enumerate}[label=--]
        \item
          \emph{certainty} of receiving 100 million
        \end{enumerate}
        Situation B:
        \begin{enumerate}[label=--]
        \item
          \emph{a} 10\% \emph{chance} of winning 500 million,
        \item
          \emph{an} 89\% \emph{chance} of winning 100 million,
        \item
          \emph{a} 1\% \emph{chance} of winning nothing.
        \end{enumerate}
      \item
      \emph{Do you prefer Situation C to Situation D}?

      Situation C:
        \begin{enumerate}[label=--]
        \item
          \emph{a} 11\% \emph{chance} of winning 100 million,
        \item
          \emph{an} 89\% \emph{chance} of winning nothing.
        \end{enumerate}
        Situation D:
        \begin{enumerate}[label=--]
        \item
          \emph{a} 10\% \emph{chance} of winning 500 million,
        \item
          \emph{a} 90\% \emph{chance} of winning nothing.%
          \mbox{ }\hfill\mbox{(\citeyear[89]{Allais:1979aa})}
        \end{enumerate}
    \end{enumerate}
  \end{quote}

  \begin{quote}
    The preference \(\text{A} > \text{B}\) should entail \(\text{C} > \text{D}\).%
    \footnote{
      \EU{Sit.\ A} = \(\Util{100\text{m}}\).

      \EU{Sit.\ B} = \(.10 \cdot \Util{500\text{m}} + .89 \cdot \Util{100\text{m}} + .1 \cdot \Util{0\text{m}}\).

      \EU{Sit.\ C} = \(.11 \cdot \Util{100\text{m}} + .89 \cdot \Util{0\text{m}}\).

      \EU{Sit.\ D} = \(.10 \cdot \Util{500\text{m}} + .90 \cdot \Util{0\text{m}}\).

      Suppose Situation A is preferred to Situation B.
      Hence, \(\EU{Sit.\ A} > \EU{Sit.\ B}\).
      Expanding, we obtain:
      %
      \[
        \Util{100\text{m}} > .10 \cdot \Util{500\text{m}} + .89 \cdot \Util{100\text{m}} + .1 \cdot \Util{0\text{m}}
      \]

      Now, consider subtracting \(.89 \cdot \Util{100\text{m}}\) from each side of the inequality and adding \(.89 \cdot \Util{0\text{m}}\):
      %
      \begin{align*}
        \Util{100\text{m}} - .89 \cdot \Util{100\text{m}} &> .10 \cdot \Util{500\text{m}} + .01 \cdot \Util{0\text{m}} \\
         .11 \cdot \Util{100\text{m}} + .89 \cdot \Util{0\text{m}} &> .10 \cdot \Util{500\text{m}} + .90 \cdot \Util{0\text{m}}
      \end{align*}

      The left and right side of the inequality are \EU{Sit.\ C} and \EU{Sit.\ D}, respectively.
      Therefore, it should be the case that Situation C is preferred to Situation D.
    }
  \end{quote}
  However, this is not the case.
  \citeauthor{Allais:1979aa} highlights pattern to the contrary:
  \textquote{\emph{[T]he pattern for most highly prudent persons [\dots] who are considered generally as rational, is the pairing \(\text{A} > \text{B}\) and \(\text{C} < \text{D}\).}}
  (\citeyear[89]{Allais:1979aa})

  So, we have a something lawlike.%

  Two things here.
  \begin{itemize}
  \item
    Entailment between preferences.
  \item
    Circumstance in which conflicting preferences.
  \end{itemize}

  Both work, but the latter is of interest.%
  \footnote{
    \color{red}
    The former \dots
  }

  For, if axioms, then preferences.
\end{note}

\paragraph{Belief revision}

\begin{note}
  {
    \color{red}
    Harman is another instance of this strategy.
  }

  Similar to \citeauthor{Harman:1984aa,Harman:1986ux}.
  However, different.
  \citeauthor{Harman:1986ux} raises problem regarding the \tor{}.
  At issue here is, regardless of the type, still get a problem.

  However, same principle employed.

  \begin{quote}
    Logical principles are not directly rules of belief revision.
    \dots
    Logical principles hold universally, without exception, whereas the corresponding principles of belief revision would be at best prima facie principles, which do not always hold.%
    \mbox{ }\hfill\mbox{(\citeyear[107--108]{Harman:1984aa})}
  \end{quote}
  \citeauthor{Harman:1984aa}'s point is that universality, so if hold, then hold in all circumstances.
  And, it is clear this is not the case.
\end{note}

\paragraph{Other}

\begin{note}
  The Ellsberg paradox (\cite{Ellsberg:1961aa}).

  Consider also \citeauthor{Kyburg:1997aa}'s Lottery Paradox (\citeyear{Kyburg:1997aa}) or \citeauthor{Makinson:1965aa}'s Paradox of the Preface (\citeyear{Makinson:1965aa}).

  Here, belief and conjunction.

  Gettier.
  Here, type of reasoning present in knowledge gets restricted.

  Wright, with transmission failure.
  Characterising a type of reasoning.
  Then, troubles.
  For, there's that problem of triviality.
\end{note}

\subsection{Summary}

\begin{note}
  So, this is plausible.
  The goal for the remainder of the chapter is to capture this in a specific way.

  Our goal is to construct a lawful constraint.
\end{note}

\section{\tR{2}}
\label{cha:typical:sec:tR}

% \section{Types, instances, and representatives of reasoning}
% \label{cha:typical:sec:tor}

\begin{note}
  Following \autoref{idea:tR-law} (\autopageref{idea:tR-law}), our goal is to construct a `lawful' condition for an instance of reasoning to be an instance of a \torN{}.

  To do so, need two things:

  \begin{itemize}
  \item
    Abstract definition of a \tor{} which is compatible with assumptions made about reasoning.
    Link the definition of a \tor{} to instances of reasoning.
  \end{itemize}

  Define \tor{}.
  Incorporate link between \torN{} and instances of reasoning by a (partial) definition of \tRN{}.
\end{note}

\begin{note}
  Break down into four sections.

  \begin{enumerate}[label=]
  \item
    \TOCLine{cha:typical:sec:tR:ToRdef}

    Define \tor{}.
  \item
    \TOCLine{cha:typical:sec:tR:tRSketch}

    Sketch of how the definition of a \tor{} relates to instances of reasoning.
    And, problems.
  \item
    \TOCLine{cha:typical:sec:tR:RoToR}

    Definition of a \rotorN{}.
  \item
    \TOCLine{cha:typical:sec:tR:tRDef}

    Definition of \tR{}, lawful.
  \end{enumerate}
\end{note}


\subsection{\tor{3}}
\label{cha:typical:sec:tR:ToRdef}

\begin{note}
  We begin by defining the (extension of) a \tor{}:

  \begin{definition}[\tor{2}]
    \label{def:tor}
    \mbox{ }
    \vspace{-\baselineskip}
    \begin{itemize}
    \item
      \(T\) is (the extension of) a \emph{\tor{0}}.
    \end{itemize}

    \emph{If and only if}:

    \begin{itemize}
    \item
      \(T\) is a collection of proposition-value-\pool{} pairings.
    \end{itemize}
    \vspace{-\baselineskip}
  \end{definition}

  The definition of a \tor{} is motivated as follows:
  \begin{itemize}
  \item
    We have only assumed reasoning amounts to reasoning is to some proposition-value pair \(\pv{\phi}{v}\) from some \pool{} \(\Phi\).
    Therefore, nothing other than proposition-value pairs, \pool{1}, and events in which an agent reasons.
  \item
    Not events, as it may be the case that agent's reasoning is of type, even though there is no event in which the agent reasons.
    Reasoning, progressive.
    Need not be the case that agent reasons.
  \item
    Intuition that \tor{} is not restricted to any extant event.
    For example, language comprehension.
    Novel sentences, even those which are never produced.
  \end{itemize}
\end{note}

\begin{note}
  To \illu{1} with the basic case of modus ponens:

  \begin{illustration}[Modus ponens]
    Collection of proposition-value-\pool{} pairs as follows:

    \begin{center}
      \begin{tabular}[h]{p{.55\linewidth}|p{.4\linewidth}}
        \multicolumn{2}{c}{\emph{In collection}} \\
        \(P\), if \(P\) then \(Q\) & \(Q\) \\
        \(R\), if \(R\) then \(S\) & \(S\) \\
        \dots & \dots \\
      \end{tabular}
    \end{center}

    Important is what fails to be in the collection.

    \begin{center}
      \begin{tabular}[h]{p{.55\linewidth}|p{.4\linewidth}}
        \multicolumn{2}{c}{\emph{Not in collection}} \\
        \(P\), if (\(P\text{ or }Q\)) then \(R\) & \(R\) \\
        \(S\), if \(S\) then \(T\) & \emph{not}-\(T\) \\
        \dots & \dots \\
      \end{tabular}
    \end{center}
  \end{illustration}
\end{note}

\begin{note}
  Difficulty.
  As extension, no way to distinguish between two \tor{1} with the same extension.

  For example, modus ponens and hypothetical syllogism.

  However, our goal is a necessary condition.
  In line with idea, failure.
  So, if some circumstance, then neither modus ponens nor hypothetical syllogism.
\end{note}

% \begin{note}
%   With definition of \tor{} in hand, we fix a piece of terminology:
%   \begin{definition}[\tI{} of a \tor{}]
%     \label{def:typeInstance}
%     \cenLine{
%       \begin{itemize*}[noitemsep, label=\(\circ\)]
%       \item
%         Proposition: \(\phi\)
%       \item
%         Value: \(v\)
%       \item
%         \pool{2}: \(\Phi\)
%       \item
%         Type of reasoning: \(T\)
%       \item
%         \mbox{ }
%       \end{itemize*}
%     }

%     \begin{itemize}
%     \item
%       \(\pvp{\phi}{v}{\Phi}\) is an \emph{\tI{0}} of \tor{} \(T\).
%     \end{itemize}

%     \emph{If and only if}:

%     \begin{itemize}
%     \item
%       \(\pvp{\phi}{v}{\Phi}\) is a included in \(T\).
%     \end{itemize}
%     \vspace{-\baselineskip}
%   \end{definition}

%   Do not use `instance' in order to identify events in which an agent reasons.
% \end{note}

\begin{note}
  With the definition of a \tor{} in hand, turn to a sketch of \tR{}.
  Following the sketch and \illu{0}, highlight the problem for the sketch.
  Then, work toward definition.
\end{note}

\subsection{\tRN{2}: Sketch}
\label{cha:typical:sec:tR:tRSketch}

\begin{note}
  With the definition of a \tor{} we now sketch an elaboration of \autoref{idea:tR-law}:

  \begin{sketch}[\tRN{2}]
    \label{sketch:tR}
    \cenLine{
      \begin{itemize*}[noitemsep, label=\(\circ\)]
      \item
        Agent: \vAgent{}
      \item
        Propositions: \(\phi\), \(\psi\)
      \item
        Values: \(v\), \(v'\)
      \item
        \pool{3}: \(\Phi\), \(\Psi\)
      \item
        \mbox{ }
      \end{itemize*}
    }

    \noindent%
    \cenLine{
      \begin{itemize*}[noitemsep, label=\(\circ\)]
      \item
        Event: \(e\)
      \item
        Type of reasoning: \(T\)
      \item
        \mbox{ }
      \end{itemize*}
    }

    \begin{itemize}
    \item
      \(e\) is an event in which \vAgent{} is \emph{\tRVN{0}} (by type \(T\)) to \(\pv{\phi}{v}\) from \(\Phi\).
    \end{itemize}

    \emph{Only if}:

    \begin{itemize}[noitemsep]
      \item
        For every \tI{} \(\pvp{\psi}{v'}{\Psi}\) of \(T\).
        \begin{itemize}[noitemsep]
        \item[\emph{If}:]
          \begin{enumerate}[label=\alph*., ref=(\alph*), series=tRSketch]
            \label{sketch:tR:action}
          \item
            There is some available action \(a\) such that \vAgent{} is reasoning from \(\Psi\) after performing \(a\).
          \end{enumerate}
        \item[\emph{Then}:]
          \begin{enumerate}[label=\alph*., ref=(\alph*), resume*=tRSketch]
          \item
            \label{sketch:tR:prog}
            There is some available action \(a'\) such that \vAgent{} is reasoning to \(\pv{\psi}{v'}\) from \(\Psi\) after performing \(a'\).
          \end{enumerate}
        \end{itemize}
      \end{itemize}
    \vspace{-\baselineskip}
  \end{sketch}

  The broad idea.

  If reasoning is of type \(T\), then, if the agent has the opportunity to reason from some premises associated with the type, then the agent may reason from the premises by the type of reasoning.

  Motivation for this restriction is progressive.
  Core idea is that whatever it is which secures the truth of progressive, it is securing reasoning of type \(T\).
  Therefore, this same thing extends, at least, to other \pool{1} the agent may reason from.
  And, the progressive allows for the possibility that the agent is interrupted, etc.

  However, wish to keep this aspect of the event fixed.
  Hence, avoid counterfactual conditional.
  Keep event fixed, and consider actions agent may perform.
  Both~\ref{sketch:tR:action} and~\ref{sketch:tR:prog} may be considered subjunctive conditionals, as the agent does not perform action, but to evaluate these we only need to consider the way in which the event develops given the action is performed, as opposed to the way in which things may otherwise have been.

  As a consequence, \autoref{sketch:tR} is restricted.
  \tI{2} of type that are relevant are limited by event.
  Hence, \autoref{sketch:tR} only amounts to a necessary condition.

  For example, selection task.
  Good with beer and age, but fail with vowels and even numbers.
  Still, no task given only involves beer and age.
  \autoref{sketch:tR} does not apply.

  Still, our interest is only with respect to necessary condition, and even so restricted, \autoref{sketch:tR} is informative.
\end{note}

\begin{note}
  \begin{illustration}[Textbook]
    Student has been studying algebra and has just been introduced to the rule of multiplication for powers (\(a^{n} \cdot a^{m} = a^{n + m}\)).

    A handful of exercises:%
    \footnote{
      From \textcite[31]{Gelfand:1993aa}.
    }

    \begin{quote}
      \begin{enumerate}[label=(\alph*), ref=(\alph*)]
      \item
        \label{mfp:a}
        You know that \(2^{1001} \cdot 2^{n} = 2^{2000}\).
        What is \(n\)?
      \item
        \label{mfp:b}
        You know that \(2^{1001} \cdot 2^{n} = \sfrac{1}{4}\).
        What is \(n\)?
      \item
        \label{mfp:c}
        Which is bigger: \(10^{-3}\) or \(2^{-10}\)?
      \item
        \label{mfp:d}
        You know that \(\sfrac{2^{1000}}{2^{n}} = 2^{501}\).
        What is \(n\)?
      \item
        \label{mfp:e}
        You know that \(\sfrac{2^{1000}}{2^{n}} = \sfrac{1}{16}\).
        What is \(n\)?
      \item
        \label{mfp:f}
        You know that \(4^{100} = 2^{n}\).
        What is \(n\)?
      \item
        \label{mfp:g}
        You know that \(2^{100} \cdot 3^{100} = a^{100}\).
        What is \(a\)?
      \item
        \label{mfp:h}
        You know that \((2^{10})^{15} = 2^{n}\).
        What is \(n\)?
      \end{enumerate}
    \end{quote}

    Student starts work on exercise~\ref{mfp:f}.

    Reasons from the problem to the proposition-value pair, \(\pv{n\text{ is }200}{\text{true}}\).

    As the event in which the student reasoned to \(\pv{n\text{ is }200}{\text{true}}\), it seems true that the student was reasoning to \(\pv{n\text{ is }200}{\text{true}}\) from some \pool{} \(\Phi\) by a type of reasoning that involves the rule of multiplication for powers --- the student did not arbitrarily fix on \(n\) being \(200\).

    Granting that the student was reasoning to \(\pv{n\text{ is }200}{\text{true}}\) from \(\Phi\), the agent may have stopped to being reasoning about a different problem.
    For example, the student may have stopped working on~\ref{mfp:f} and began to work on~\ref{mfp:a}.
    And, it seems that if the agent was reasoning to \(\pv{n\text{ is }200}{\text{true}}\) from \(\Phi\), the agent would be reasoning to \(\pv{n\text{ is }999}{\text{true}}\) from \(\Phi'\).

    Likewise for the other problems.
  \end{illustration}

  Suppose not, then it seems there is some misunderstanding regarding the rule of multiplication for powers.
\end{note}

\begin{note}
  Given available actions, no clear relation to the broad upshots of Wason, Allias, and \citeauthor{Harman:1984aa}.

  For the particular instances, fails to be the case that reasoning is of type, as agent's fail to reason, and hence \autoref{sketch:tR} fails when \(\pvp{\phi}{v}{\Phi}\) is the same as \(\pvp{\phi}{v}{\Phi}\).
  However, it does not follow that failure says anything about other instance of reasoning when the agent does not have the option to reason about a selection task, or Allias gambles.

  However, expected.
  Wason, Allias, results concern reasoning in general.
  Compatible with results that agents do, on occasion, consider conditionals as truth-functional and have nice preferences.
\end{note}

\begin{note}
  The sketch of \tRN{} does some work, but is not quite right.
\end{note}


\subsubsection{Problems for the sketch}
\label{cha:typical:sec:tor:g-s:three-problems}
\nocite{Wilson:1994aa}

\begin{note}
  The \hyperref[sketch:tR]{sketch of \tRN{}} captures idea.
  However, too stringent.
  Plausibly cases in which action, but progressive fails and reasoning is of type.
\end{note}

\begin{note}
  For example, illustration.
  Here,~\ref{mfp:b},~\ref{mfp:d}, and~\ref{mfp:e} all involve fractions.
  However, agent is shaky on fractions.
  Rule of multiplication is good, but not with fractions.

  Issue is perhaps with the type.
  Perhaps~\ref{mfp:b},~\ref{mfp:d}, and~\ref{mfp:e} should not correspond to members of relevant type.

  However, with some imagination, available action that leads to difficulties.
  For example, \(2^{432 \cdot 5543} \cdot 2^{543!}\).
  Agent needs not more than multiplication and addition, but progressive fails.
  Difficulty of solving multiplication in order to sum.
  And, perhaps, temptation to get a calculator.

  Still, reasoning by type in problems the agent does solve.
\end{note}

\begin{note}
  Point is similar to \citeauthor{Chomsky:2015aa}'s distinction between competence and performance.%
  \footnote{
    \citeauthor{Chomsky:2015aa} also motivates distinction by errors (\citeyear[2]{Chomsky:2015aa}).
    Unclear how errors relate to progressive.
    Present problems are sufficient, so we ignore.
  }

  \begin{quote}
    Arithmetical competence yields the correct number z for every pair (x, y) under addition or multiplication.
    But only a small finite subpart of arithmetical competence can be exhibited without external aids (by calculating in one's head).
    Obviously, that fact does not imply that arithmetical competence is correspondingly limited.\newline
    \mbox{ }\hfill\mbox{(\citeyear[xii]{Chomsky:2015aa})}
  \end{quote}
\end{note}

\subsection{\rotor{3}}

\label{cha:typical:sec:tR:RoToR}

\begin{note}
  \tor{}, sketch, problems.

  Certain \tI{1} are such that progressive fails, yet, still of type.

  Solution is simple.

  \tor{3} are existentially defined.
  And, sketch fails as quantifies over all instances of type with available action.
  Therefore, to solve problem, ignore problematic \tI{1} of the type.

  With respect to \illu{1}, if student fails to solve certain problems, reasoning is not of type, even if failure to solve other problems is irrelevant.

  To help capture this is, introduce the \rotor{}.
\end{note}

\begin{note}
  \begin{definition}[\rotor{2}]
    \label{def:rotor}
    \cenLine{
      \begin{itemize*}[noitemsep, label=\(\circ\)]
      \item
        Agent: \vAgent{}
      \item
        Event: \(e\)
      \item
        Types of reasoning: \(T\), \(T'\)
      \item
        \mbox{ }
      \end{itemize*}
    }

    \begin{itemize}
    \item
      \(T'\) is a \emph{\tRep{0}} of \(T\) for \vAgent{} with respect to an event \(e\).
    \end{itemize}

    \emph{If and only if}:

    \begin{itemize}
    \item

      \begin{itemize}
      \item[\emph{If}:]
        \begin{enumerate}[label=\alph*., ref=(\alph*), series=tRDef]
        \item
          \(e\) is an event such that \vAgent{}' reasoning is of type \(T\)
        \end{enumerate}
      \item[\emph{Then}:]
        \begin{enumerate}[label=\alph*., ref=(\alph*), resume*=tRDef]
        \item
          \(e\) is an event such that \vAgent{}' reasoning is of type \(T'\).
        \end{enumerate}
      \end{itemize}
    \end{itemize}
    \vspace{-\baselineskip}
  \end{definition}

  The idea of a \tRep{}:
  For reasoning to be of type \(T\) with respect to some event it is necessary that the reasoning is of type \(T'\).

  Type \(T'\) is just a collection of proposition-value-\pool{} pairs.
  Hence, \(T'\) is, for present purposes, the result of omitting difficult \tI{1} of \(T\).%
  \footnote{
    Hence, a natural constraint on an \tRep{} \(T'\) of \(T\) is that any \tI{} of \(T'\) is also an \tI{} of \(T\).
    However, we have no use for such a constraint.
  }

  Alternatively phrased, \rotor{} is just type relative to the specified event.

  Contrapositive is key:

  If not of type \(T'\), then not of type \(T\).

  So, with respect to \illu{1}, restrict to a handful of relevant problems.
\end{note}

\begin{note}
  The converse conditional is also interesting.
  Type \(T'\) is sufficient for type \(T\).
  However, the difficulty with this suggestion is whether the conditional.
  May need more than just extension in order for reasoning to be of type.
  And, only necessary conditions, given abstract about reasoning.
\end{note}

\subsection{\tR{2}}
\label{cha:typical:sec:tR:tRDef}

\begin{note}
  With the definition of a \rotor{} in hand, we refine the \hyperref[sketch:tR]{sketch of \tRN{}} to definition.
\end{note}

\begin{note}
  \begin{definition}[\tR{2}]
    \label{def:cmptnc}
    \cenLine{
      \begin{itemize*}[noitemsep, label=\(\circ\)]
      \item
        Agent: \vAgent{}
      \item
        Propositions: \(\phi\), \(\psi\)
      \item
        Values: \(v\), \(v'\)
      \item
        \pool{3}: \(\Phi\), \(\Psi\)
      \item
        \mbox{ }
      \end{itemize*}
    }

    \noindent%
    \cenLine{
      \begin{itemize*}[noitemsep, label=\(\circ\)]
      \item
        Event: \(e\)
      \item
        Type of reasoning: \(T\)
      \item
        \mbox{ }
      \end{itemize*}
    }

    \begin{itemize}
    \item
      \(e\) is an event in which \vAgent{} is \emph{\tRV{0}} (by type \(T\)) to \(\pv{\phi}{v}\) from \(\Phi\).
    \end{itemize}

    \emph{Only if}:

    \begin{itemize}[noitemsep]
    \item
      For every \tRep{0} \(T'\) of \(T\) with respect to \(e\):
      \begin{itemize}[noitemsep]
      \item
        For every \tI{} \(\pvp{\psi}{v'}{\Psi}\) of \(T'\).
        \begin{itemize}[noitemsep]
        \item[\emph{If}:]
          \begin{enumerate}[label=\alph*., ref=(\alph*), series=tRSketch]
          \item
            There is some available action \(a\) such that \vAgent{} is reasoning from \(\Psi\) after performing \(a\).
          \end{enumerate}
        \item[\emph{Then}:]
          \begin{enumerate}[label=\alph*., ref=(\alph*), resume*=tRSketch]
          \item
            There is some available action \(a'\) such that \vAgent{} is reasoning to \(\pv{\phi}{v}\) from \(\Psi\) after performing \(a'\).
          \end{enumerate}
        \end{itemize}
      \end{itemize}
    \end{itemize}
    \vspace{-\baselineskip}
  \end{definition}

  The difference between \autoref{def:cmptnc} and \autoref{sketch:tR} is simple.
  Rather than the the \itc{} holding with respect to \emph{all} \tI{} of some type (as in~\autoref{sketch:tR}), the \itc{} is only required to hold for those \tI{1} of any \emph{\tRep{0}} of the type.

  Given this, of interest in \rotor{1}.
  In particular, which proposition-value-\pool{} pairs are \tI{1} of the \tRep{1}.

  Still, keep in mind motivating idea.
  Something about present event which secures the truth of the progressive.

  So, proposition-value-\pool{} pairs are \tI{1} of the \tRep{1} just in case \itc{} must be true.
\end{note}

\begin{note}
  Slight issue, single instances is sufficient for failure.
  In this case, set \(T'\) to be \(\pvp{\phi}{v}{\Phi}\), and consider a parallel definition which requires failure of progressive to hold for some threshold of \tI{}.

  In part, appeal of \autoref{def:cmptnc} is that necessary condition identified requires non-empty \tRep{} to do any work.
  At issue is whether \autoref{idea:tR-law} holds.
  Whether there is a `lawful' connexion between reasoning of type and other circumstances.
\end{note}

\subsection{\illu{3} of \tR{}}
\label{sec:illu3}

\begin{note}
  Revisit \scen{0} from above.

  Intuitively, would solve problems, and that's fine.
  Observed problem with this, as any problem of same type.
  However, understand the collection of problems as a \tRep{0}.
  It does not need to be the case that all instances.
  So, omit~\ref{mfp:b},~\ref{mfp:d}, and~\ref{mfp:e}.
\end{note}

\begin{note}
  Three additional \illu{1}.
\end{note}

\begin{note}
  \begin{illustration}[Tic-tac-toe]
    So, type of reasoning, forcing a draw.

    At this point, variety of options.
    So, could go anywhere.

    Regardless, continues to a draw.
  \end{illustration}

  In this case, \rotor{} plausible coincides with type.
  For, available actions restrict instances of type to available moves, and failure of the progressive after any move means that some part of the strategy is not understood.
\end{note}

\begin{note}
  \begin{illustration}[Translation]
    Book.
    List of chapters.
    Is the agent reading the chapter titles?
  \end{illustration}

  Well, available action such that the agent translates.
  For example, \textquote{ソフト&ウェット} to \textquote{Soft and wet}, \textquote{毎日が夏休み} to \textquote{Every day is summer vacation}, and \textquote{ザ・ワンダー・オブ・ュー(君の奇跡の愛)} to \textquote{The wonder of you (the marvel of you love)}.

  However, agent may fail to translate certain chapter titles such as \textquote{整形外科医 – 羽伴毅先生}%
  \footnote{
    Some kind of doctor, and their name\dots
    Oh, Dr. Wu Tomoki, Orthopedic Surgeon.
  }
  or \textquote{清の時代の髪留め}%
  \footnote{
    Hair clip of the age of pure?
    Ah! \textquote{清の時代} translates to \textquote{Qing Dynasty}.
  }%
  .

  Interesting case.

  For, doesn't clearly follow that needs to be the case the agent translates each chapter title.

  And, extending, it need not be the case that agent translates any other.
  For, each other chapter title falls outside sufficient core understanding.
  Still, \textquote{母と子} to \textquote{Mother and child}.
  If not, something is off.
\end{note}

\begin{note}
  \begin{illustration}[Focus]
    Exam.
    Consider any other question, then immediately return to present question.%
    \footnote{
      To go further, plausibly get to counterfactuals.

      For example, rather than action to the agent, go via prompting.
      I give question, you get answer.
    }
  \end{illustration}

  Here, \tRep{} is just the question.
  \autoref{def:cmptnc} does not interesting work.
\end{note}

\section{\tC{2}}
\label{cha:typical:sec:tC}

\begin{note}
  With reasoning in hand, expand our understanding of typicality to concluding:

  \begin{definition}[\tC{2}]
    \label{def:tC}
    \cenLine{
      \begin{itemize*}[noitemsep, label=\(\circ\)]
      \item
        Agent: \vAgent{}
      \item
        Propositions: \(\phi\), \(\psi\)
      \item
        Values: \(v\), \(v'\)
      \item
        \pool{3}: \(\Phi\), \(\Psi\)
      \item
        \mbox{ }
      \end{itemize*}
    }

    \noindent%
    \cenLine{
      \begin{itemize*}[noitemsep, label=\(\circ\)]
      \item
        Event: \(e\)
      \item
        Type of reasoning: \(T\)
      \item
        \mbox{ }
      \end{itemize*}
    }

    \begin{itemize}
    \item
      \(e\) is an event in which \vAgent{} is \emph{\tCV{0}} (by type \(T\)) \(\pv{\phi}{v}\) from \(\Phi\).
    \end{itemize}

    \emph{Only if}:

    \begin{itemize}[noitemsep]
    \item
      For the \tRep{0} \(T'\) of \(T\) with respect to \(e\):
      \begin{itemize}[noitemsep]
      \item
        For every pairing \(\pvp{\psi}{v'}{\Psi}\) in \(T'\).
        \begin{itemize}[noitemsep]
        \item[\emph{If}:]
          \begin{enumerate}[label=\alph*., ref=(\alph*), series=tRSketch]
          \item
            There is some available action \(a\) such that \vAgent{} is reasoning from \(\Psi\) after performing \(a\).
          \end{enumerate}
        \item[\emph{Then}:]
          \begin{enumerate}[label=\alph*., ref=(\alph*), resume*=tRSketch]
          \item
            There is some available action \(a'\) such that \vAgent{} is concluding \(\pv{\phi}{v}\) from \(\Psi\) after performing \(a'\).
          \end{enumerate}
        \end{itemize}
      \end{itemize}
    \end{itemize}
    \vspace{-\baselineskip}
  \end{definition}

  Motivation for this is simple.
  \tR{} just looks at instance of reasoning.
  This is okay, but less interesting.
  For, if fail to do something of same type, then what is really going on?
\end{note}

\subsection{\tR{2} and \fc{1}}
\label{sec:fc3}

\begin{note}
  Extend definition of \tC{} to cover \fc{1}.
  Interest here is additional step.

  Intuitively, if fails to be a \fc{} due to conflict, then this will not amount to a \tRep{}.
  However, if no conflict, then by \fc{}, action, and hence \tRep{}.

  What we capture is additional circumstances.

  Naturally, the idea generalises further.
\end{note}

\begin{note}
  \begin{definition}[\vtC{2}]
    \label{def:vtC}
    \cenLine{
      \begin{itemize*}[noitemsep, label=\(\circ\)]
      \item
        Agent: \vAgent{}
      \item
        Propositions: \(\phi\), \(\psi\)
      \item
        Values: \(v\), \(v'\)
      \item
        \pool{3}: \(\Phi\), \(\Psi\)
      \item
        \mbox{ }
      \end{itemize*}
    }

    \noindent%
    \cenLine{
      \begin{itemize*}[noitemsep, label=\(\circ\)]
      \item
        Event: \(e\)
      \item
        Type of reasoning: \(T\)
      \item
        \mbox{ }
      \end{itemize*}
    }

    \begin{itemize}
    \item
      \(e\) is an event in which \vAgent{} is \emph{\vtCV{0}} (by type \(T\)) \(\pv{\phi}{v}\) from \(\Phi\).
    \end{itemize}

    \emph{Only if}:

    \begin{itemize}[noitemsep]
    \item
      For the \tRep{0} \(T'\) of \(T\) with respect to \(e\):
      \begin{itemize}[noitemsep]
      \item
        For every pairing \(\pvp{\psi}{v'}{\Psi}\) in \(T'\).
        \begin{itemize}[noitemsep]
        \item[\emph{If}:]
          \begin{enumerate}[label=\alph*., ref=(\alph*), series=tRSketch]
          \item
            There is some available action \(a\) such that \vAgent{} is reasoning from \(\Psi\) after performing \(a\).
          \end{enumerate}
        \item[\emph{Then}:]
          \begin{enumerate}[label=\alph*., ref=(\alph*), resume*=tRSketch]
          \item
            For any proposition \(\phi'\), value \(v'\), and action \(a'\) such that \vAgent{} is concluding \(\pv{\phi'}{v'}\) from \(\Phi\) after \(a'\) is done:
            \begin{itemize}
            \item
              Either:
              \begin{enumerate}[label=\arabic*., ref=(\arabic*)]
              \item
                \(\phi'\) is \(\phi\) and \(v'\) is \(v\).%
                \footnote{
                  In other words, \vAgent{} is concluding \(\pv{\phi}{v}\) from \(\Phi\) after doing \(a'\).
                }
              \item
                Throughout event in which \vAgent{} concludes \(\pv{\phi'}{v'}\) from \(\Phi\), there is some action \(b\) such that \vAgent{} is concluding \(\pv{\phi}{v}\) from \(\Phi\) after \(b\) is done.
              \end{enumerate}
            \end{itemize}
          \end{enumerate}
        \end{itemize}
      \end{itemize}
    \end{itemize}
    \vspace{-\baselineskip}
  \end{definition}
\end{note}






\section{Summary}
\label{cha:typical:sec:summ}

\begin{note}
  \tRN{2}.

  Extension account of \tor{}.
  Due to abstracting over theories.

  Then, necessary condition on \tR{}.
\end{note}

% \newpage

% \section[\citeauthor{Carroll:1895uj}]{\citeauthor{Carroll:1895uj}\hfill(Optional)}

% \nocite{Black:1951aa}

% \begin{note}
%   The point here is that with Carroll, generality that goes beyond any single instance.
%   Must apply to all instances, to be valid.
%   But, cannot hope to cover all instances in a single move.
% \end{note}

% \begin{note}
%   A difficulty found on a reading of \citeauthor{Carroll:1895uj}'s \citetitle{Carroll:1895uj}.
% \end{note}

% \begin{note}
%   \begin{quote}
%     ``Plenty of blank leaves, I see!'' the Tortoise cheerily remarked.
%     ``We shall need them \emph{all}!''
%     (Achilles shuddered.)
%     ``Now write as I dictate:---

%     \begin{enumerate}[label=(\emph{\Alph*}), ref=\emph{\Alph*}]
%     \item
%       \label{AatT:a}
%       Things that are equal to the same are equal to each other.
%     \item
%       \label{AatT:b}
%       The two sides of this Triangle are things that are equal to the same.
%     \item
%       \label{AatT:c}
%       If~\ref{AatT:a} and~\ref{AatT:b} are true,~\ref{AatT:z} must be true.
%       \setcounter{enumi}{25}
%     \item
%       \label{AatT:z}
%       The two sides of this Triangle are equal to each other.''
%     \end{enumerate}

%     ``You should call it~\ref{AatT:d}, not~\ref{AatT:z},'' said Achilles.
%     ``It comes \emph{next} to the other three.
%     If you accept~\ref{AatT:a} and~\ref{AatT:b} and~\ref{AatT:c}, you \emph{must} accept~\ref{AatT:z}.''

%     ``And why \emph{must} I?''

%     ``Because it follows \emph{logically} from them.
%     If~\ref{AatT:a} and~\ref{AatT:b} and~\ref{AatT:c} are true,~\ref{AatT:z} \emph{must} be true.
%     You don't dispute \emph{that}, I imagine?''

%     ``If~\ref{AatT:a} and~\ref{AatT:b} and~\ref{AatT:c} are true,~\ref{AatT:z} \emph{must} be true,'' the Tortoise thoughtfully repeated.
%     ``That's \emph{another} Hypothetical, isn't it?
%     And, if I failed to see its truth, I might accept~\ref{AatT:a} and~\ref{AatT:b} and~\ref{AatT:c}, and \emph{still} not accept~\ref{AatT:z}, mightn't I ?''

%     \mbox{}\hfill\(\vdots\)\hfill\mbox{}

%     ``Then Logic would take you by the throat, and force you to do it!''
%     Achilles triumphantly replied.
%     ``Logic would tell you 'You ca'n't help yourself.''%
%     \mbox{ }\hfill\mbox{(\citeyear[279--280]{Carroll:1895uj})}
%   \end{quote}

%   The Tortoise has written down three premises,~\ref{AatT:a},~\ref{AatT:b}, and~\ref{AatT:c}.
%   Achilles holds that~\ref{AatT:z} follows from~\ref{AatT:a},~\ref{AatT:b}, and~\ref{AatT:c}.
%   The Tortoise observes they have the possibility of refraining to accept~\ref{AatT:z} follows from~\ref{AatT:a},~\ref{AatT:b}, and~\ref{AatT:c}.
%   And (initially), the Tortoise does not accept~\ref{AatT:z} follows from~\ref{AatT:a},~\ref{AatT:b}, and~\ref{AatT:c}.
%   Achilles requests the Tortoise accepts that~\ref{AatT:z} follows from~\ref{AatT:a},~\ref{AatT:b}, and~\ref{AatT:c}, and the Tortoise complies.
%   Specifically, the Tortoise grants:

%   \begin{quote}
%     \begin{enumerate}[label=(\emph{\Alph*}), ref=\emph{\Alph*}]
%       \setcounter{enumi}{3}
%     \item
%       \label{AatT:d}
%       If~\ref{AatT:a} and~\ref{AatT:b} and~\ref{AatT:c} are true,~\ref{AatT:z} must be true.%
%       \mbox{ }\hfill\mbox{(\citeyear[279]{Carroll:1895uj})}
%     \end{enumerate}
%   \end{quote}

%   But, does not accept~\ref{AatT:z} follows from~\ref{AatT:a},~\ref{AatT:b},~\ref{AatT:c}, and~\ref{AatT:d}.
% \end{note}

% \begin{note}
%   Modus ponens.

%   \begin{quote}
%     From \(\phi\) and \emph{if} \(\phi\) then \(\psi\), infer \(\psi\).
%   \end{quote}

%   Modus ponens is general.
%   For \emph{any} \(\phi\), \(\psi\).

%   Now, there is a difference between \emph{modus ponens} and conditional.

%   However, take any instance.
%   Then, if \(P\), \(P \rightarrow Q\), \(Q\) must be true.
%   But, then this means that the conditional is true.

%   Consequence of the deduction theorem.

%   Likewise, deduction theorem goes the other way.

%   However, going from \(P\), \(P \rightarrow Q\) to \(Q\) need not be an instance of \emph{modus ponens}.
% \end{note}

% \begin{note}
%   Well, this is a headache.
%   \citeauthor{Carroll:1895uj} is talking about a specific A, B, and Z.
%   There is no clear generality.
% \end{note}

% \begin{note}
%   So, consider at issue is modus ponens.
%   For any specific instance accept, there is a further instance.
%   For, \(A, (A \rightarrow B) \vDash B\).
%   Then, \(\vDash (A \land (A \rightarrow B) \rightarrow B)\).
%   However, now, \(A \land (A \rightarrow B), (A \land (A \rightarrow B) \rightarrow B) \vDash B\).
%   And, so on.

%   The general pattern, get conditional, but then this gives a new instance of modus ponens, which must be true in order for modus ponens to be valid rule of inference.

%   \citeauthor{Carroll:1895uj}, by contrast, starts with \(A \vDash B\).
%   This is different.
%   However, rather than focusing on a single rule of inference, the puzzle turns on what validity amounts to.

%   Validity is a general thing, with specific instances.
%   However, grant any particular instance of validity without employing validity in general.
% \end{note}

% \begin{note}
%   \begin{quote}
%     My paradox \dots turns on the fact that, in a Hypothetical, the \emph{truth} of the Protasis, the \emph{truth} of the Apodosis, and the \emph{validity of the sequence}, are 3 distinct Propositions.

%     \mbox{}\hfill\(\vdots\)\hfill\mbox{}

%     Suppose I say ``I grant~\ref{AatT:a} and~\ref{AatT:b} and~\ref{AatT:c}, but I do \emph{not} grant that I am thereby \emph{obliged} to grant~\ref{AatT:z}.''
%     Surely, my granting~\ref{AatT:z} must \emph{wait} until I have been made to see the validity of this sequence: i.e.\ in order to grant~\ref{AatT:z}, I must grant~\ref{AatT:a},~\ref{AatT:b},~\ref{AatT:c}, and~\ref{AatT:d}! And so on.%
%     \mbox{ }\hfill\mbox{(\citeyear[472]{Carroll:1977wl})}
%   \end{quote}

%   My interpretation of the point \citeauthor{Carroll:1895uj} makes in this passage is that the truth of A B and the truth of C is distinct from the validity of A B C.
%   Granting is substantial, not merely moving.
%   But, in order to grant, this means granting all other cases.

%   So, the paradox is that, on the one hand, don't need validity for any specific true things.
%   But, on the other hand, only of interest if via validity.

%   The Tortoise is slowly working through each instance, but this has no hope of getting the Tortoise to general validity.
%   So, how does the Tortoise ever make it there?
% \end{note}

% \begin{note}
%   This point differs from received interpretation.

%   \citeauthor{Wieland:2013vf} (\citeyear{Wieland:2013vf}) characterises the general understanding of \textcite{Carroll:1895uj} in terms of two lessons:
%   \begin{quote}
%     [T]he negative lesson is that if you add ever more premises to an argument \dots, then you will never demonstrate that its conclusion follows logically.\newline
%     \mbox{ }\hfill\mbox{(\citeyear[984]{Wieland:2013vf})}
%   \end{quote}

%   Parallel, static answers, still option for concluding otherwise.

%   \begin{quote}
%     [T]he positive lesson is that rules of inference, rather than premises of the form `if premises such and such are true, then the conclusion is true', will do the job.\newline
%     \mbox{ }\hfill\mbox{(\citeyear[984]{Wieland:2013vf})}
%   \end{quote}

%   \begin{quote}
%     [\citeauthor{Carroll:1895uj}] simply lacked any distinct conception of a deduction as opposed to the assertion (``granting'' of) a hypothetical proposition.
%     \dots
%     Any attempt by Carroll to tackle the question of inference was bound to begin in confusion and end in constipation-all those premises piling up, but no motion.
%   \end{quote}
% \end{note}

% \paragraph{The Dichotomy}

% \begin{note}
%   Achilles and the Tortoise, Zeno's argument.

%   Surely, right?

%   Two ways to understand.
%   Does the Tortoise move at all, or does the Tortoise arrive at the end?
%   I mean, as formulated by Zeno, it's about catching up, no matter how much one moves.

%   It is different from Zeno's Dichotomy paradox.


%   If so, then we should expect the Tortoise to be making some movement.
%   Adding rules of inference is of no help, because the problem is not movement, it's about how to move so much in a single step.
% \end{note}

% \begin{note}
%   \color{red}
%   Something about logic forcing.
%   The Tortoise hasn't arrived.

%   Nothing hangs on validity.
%   Same issue with testimony.
%   `A'.
%   Why?
%   Testified A, so A.
%   Okay, but another instance of testimony.
%   Testified(Testified A, so A), so Testified A, so A.
% \end{note}

% \begin{note}
%   \begin{quote}
%     But if we who wish to represent his belief in Q as based on P are to write in our notebook everything his having that belief on that basis consists in then when we have written only P and Q we will not have written enough.
%     Someone can believe P and believe Q and still not believe Q on the basis of P whatever the relations between the propositions P and Q happen to be.
%     He might believe Q for some reason completely unconnected with P, or perhaps for no reason at all (if that is possible).%
%     \mbox{ }\hfill\mbox{(\citeyear[185]{Stroud:1979aa})}
%   \end{quote}
%   However, the moral drawn is narrow
%   \begin{quote}
%     The moral is that for every proposition or set of propositions the belief or acceptance of which is involved in someone's believing one proposition on the basis of another there must be something else, not simply a further proposition accepted, that is responsible for the one belief's being based on the other.%
%     \mbox{ }\hfill\mbox{(\citeyear[187]{Stroud:1979aa})}
%   \end{quote}

%   Even if we grant each individual is \ros{}, rather than an instance of the material conditional, \emph{logic} hasn't done anything.
% \end{note}

% \paragraph{General and specific: Contrast}

% \begin{note}
%   Use \citeauthor{Carroll:1895uj} to illustrate this point.

%   However, given the worry, various other things may be understood this way.

%   Hume, constant conjunction.
%   Part of the problem is identifying cause.
%   We get the famous line about observing.
%   However, Hume goes on.
%   It's not only no observation, but no generality.

%   Right, so more narrow than Hume.
%   Because, with Hume, at issue is whether we have grounds for this general thing.
%   With Carroll, it's whether we even really get to the general thing.
% \end{note}


%%% Local Variables:
%%% mode: latex
%%% TeX-master: "master"
%%% End:


% \begin{note}
%   \begin{quote}
%     Let me ask this: what has the expression of a rule—say a sign-post—got to do with my actions?
%     What sort of connexion is there here?%
%     ---%
%     Well, perhaps this one:
%     I have been trained to react to this sign in a particular way, and now I do so react to it.

%     But that is only to give a causal connexion; to tell how it has come about that we now go by the sign-post; not what this going-by-the sign really consists in.
%     On the contrary; I have further indicated that a person goes by a sign-post only in so far as there exists a regular use of sign-posts, a custom.%
%     \mbox{ }\hfill\mbox{(\citeyear[\S198]{Wittgenstein:1958aa})}
%   \end{quote}

%   Regular use of sign-posts, custom.

%   Ugh, this is ambiguous.
% \end{note}


% %
%   \(^{,}\)
%   \footnote{
%     \citeauthor{Hlobil:2014tq}'s ``Inferential Moorean Phenomenon'':
%   \begin{quote}
%     \begin{enumerate}
%     \item[(IMP)]
%       It is either impossible or seriously irrational to infer \emph{P} from \emph{Q} and to judge, at the same time, that the inference from \emph{Q} to \emph{P} is not a good inference.
%     \end{enumerate}
%     \dots
%     By the ``goodness'' of an inference I mean the feature that makes the relevant inference permissible. Thus, if the inference under consideration is an inductive inference, the relevant kind of goodness is not deductive validity.%
%     \mbox{ }\hfill\mbox{(\citeyear[\S1]{Hlobil:2014tq})}
%   \end{quote}
%   Though, this really isn't more basic given the interest in \tR{}.
%   For, the puzzle is what it is to `infer'.

%   Rationality isn't part of the picture.
%   And, this is a significant drawback of \citeauthor{Hlobil:2014tq}'s approach.
% }


% \subsection{Types and explanation}
% \label{cha:typical:sec:tor:bkgd}

% \begin{note}
%   There is a related, stronger claim, that generality derives from rule following.

%   For this, \citeauthor{Boghossian:2008vf}:

%   \begin{quote}
%     [O]ur internalization of general epistemic rules---like Modus Ponens and Induction---explain and rationalize why we form the beliefs that we form.
%     And that seems intuitively correct.

%     As in the case of our linguistic and conceptual abilities, our ability to form rational beliefs is \emph{productive}: on the basis of finite learning, we are able to form rational beliefs under a potential infinity of novel circumstances.
%     The only plausible explanation for this is that we have, somehow, internalized a rule that tells us, in some general way, what it would be rational to believe under varying epistemic circumstances.%
%     \mbox{ }\hfill\mbox{(\citeyear[483]{Boghossian:2008vf})}
%   \end{quote}

%   Strictly, \citeauthor{Boghossian:2008vf}, rules \textquote{represent our conception of how it would be most rational for a thinker to form beliefs under different epistemic circumstances} (\citeyear[473]{Boghossian:2008vf}).

%   The difference in approach is clearest with \citeauthor{Boghossian:2008vf}'s account of modus ponens:%
%   \footnote{
%     \citeauthor{Boghossian:2008vf} notes the rule is distinct from modus pones as found in textbooks.
%     Remarks: \textquote{It is actually quite mysterious what the logic textbook rule is supposed to be} (\citeyear[472,fn.2]{Boghossian:2008vf})
%     I don't think there is any mystery about the rule in most logic textbooks.
%     Instead, the mystery is the way in which logic relates to reasoning.
%     (Cf.~\cite{Harman:1986ux,MacFarlane:2004aa,Steinberger:2022aa}, etc.)
%     % Issue for the presentation.
%     % Literature is full of issues.
%     % The most well known, Gricean pragmatics.
%     % Though, also McGee, McFarlane, biscuit conditonals, the miners paradox, etc.
%   }

%   \begin{quote}
%     (Modus Ponens):
%     If you are rationally permitted to believe both that \emph{p} and that `If \emph{p}, then \emph{q}', then, you are prima facie rationally permitted to believe that \emph{q}.%
%     \mbox{ }\hfill\mbox{(\citeyear[472]{Boghossian:2008vf})}
%   \end{quote}

%   Here, we have permissions.
%   What the agent is allowed to do.
%   However, this is distinct from what the agent does.
% \end{note}

% \begin{note}
%   \tR{} is distinct.
%   Whether came to \emph{q} from \emph{p} , if \emph{p} then \emph{q}.

%   Rationality is not part of our understanding.
%   Rather, generality.%
%   \footnote{
%     Observe, ~\cite{Kolodny:2005aa} is of no interest here.
%     Why be rational is distinct from whether there is some generality.
%   }
% \end{note}

% \begin{note}
%   Likewise, means-end reasoning is distinct from \citeauthor{Broome:2013aa}'s

%   \begin{quote}
%     \emph{End to Means Transmission}.
%     ((\emph{S} requires of \emph{N} that \emph{p}) \& necessarily \newline (\emph{p} \(\supset\) \emph{q}) \& \emph{q} is a means to \emph{p}) \(\supset\) (\emph{S} requires of \emph{N} that \emph{q}).%
%     \mbox{ }\hfill\mbox{(\citeyear[126]{Broome:2013aa})}
%   \end{quote}

%   \emph{S} is some source, such as morality.
%   \emph{N} is a person. (\citeyear[117]{Broome:2013aa})

%   Instead, the significantly weaker idea that the agent has reasoned from some end to a means to that end.
% \end{note}


% \begin{note}
%   On my understanding, this is, in part, the role of \citeauthor{Boghossian:2014aa}'s Taking Condition.

%   Way in which \dots

%   Indeed, \citeauthor{Boghossian:2014aa} highlights how condition allow to draw distinction between deductive and inductive.
%   With taking, get generality.

%   Indeed, \textcite{Boghossian:2014aa} is structured so that Taking is a generalisation of rule.
% \end{note}

% \begin{note}
%   However, \tor{} does not need to amount to a rule.
%   Rather, \tR{} only requires the rough phenomenon that \citeauthor{Boghossian:2008vf} argues rule following is the only plausible explanation of.%
%   \footnote{
%     Our interest with \tor{1} is independent of the worries about rule following raised by~\textcite{Kripke:1982aa}, to the extent that the worries raised by~\citeauthor{Kripke:1982aa} concern \emph{which} rule an agent is following, rather than \emph{whether} the agent is following a rule.
%     At interest is not whether the \tor{} corresponds to plus or quus, but whether the agent's reasoning is of some type.
%   }
% \end{note}

% \begin{note}
%   Same for modus ponens.

%   \citeauthor{Davies:2004aa} discussing~\textcite{Wright:2004aa} with respect to~\citeauthor{Moore:1959aa}'s proof of an external world (\citeyear{Moore:1959aa}):

%   \begin{quote}
%     Moore's argument can be set out as follows:
%     \begin{quote}
%       \begin{enumerate}[label=MOORE (\Roman*), ref=MOORE (\Roman*)]
%       \item
%         \label{MoorePoEW:1}
%         I am having an experience as of one hand [here] and another [here].
%       \item
%         \label{MoorePoEW:2}
%         I have hands.

%         If I have hands then an external world exists.
%       \end{enumerate}

%       Therefore:

%       \begin{enumerate}[label=MOORE (\Roman*), ref=MOORE (\Roman*), resume]
%       \item
%         \label{MoorePoEW:3}
%         An external world exists.
%       \end{enumerate}
%     \end{quote}

%     [\dots] the key question at this point in Wright's account is whether the support for~\ref{MoorePoEW:2} is transmitted to~\ref{MoorePoEW:3} across the modus ponens inference in which the conditional premise is supported by an elementary piece of philosophical theorising.\newline
%     \mbox{ }\hfill\mbox{(\citeyear[215]{Davies:2004aa})}
%   \end{quote}
% \end{note}
