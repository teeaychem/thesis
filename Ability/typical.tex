\chapter{\tR{2}}
\label{cha:typical}

\begin{note}
  This chapter, \tR{}.

  Intuitively, \tR{} is reasoning which has some generality.
  Think, modus ponens.
\end{note}

\begin{note}
  Function in relation to counterexamples to \issueConstraint{} is conclusions.
  Relevant conclusion is the result of \tR{} --- or, \tR[concluding]{}.
\end{note}

\begin{note}
  No guarantee that this holds.
  Only constraint on conclusion is pairing.
  Likewise, no constraint on reasoning.

  Hence, generality.
\end{note}

\begin{note}
  Still, two important things.

  \begin{itemize}
  \item
    motivation for \tR{}.
  \item
    the way in which we (partially) define \tR{}.
  \end{itemize}
\end{note}

\begin{note}
  \begin{enumerate}[label=]
  \item
    \TOCLine{cha:typical:sec:tor}
  \item
    \TOCLine{cha:typical:sec:tR}
  \item
    \TOCLine{cha:typical:sec:sTR}
  \end{enumerate}
\end{note}

\section{Types of reasoning}
\label{cha:typical:sec:tor}

{
  \color{red}
  Somewhere, that this is not a necessary condition for concluding.
}

\subsection{Definition}
\label{cha:typical:sec:tor:def}

\begin{note}
  \begin{restatable}[\tor{2}]{definition}{defToR}
    \label{def:type-r}
    \mbox{ }
    \vspace{-\baselineskip}
    \begin{itemize}
    \item
      \(T\) is a \emph{\tor{0}}.
    \end{itemize}

    \emph{If and only if}

    \begin{itemize}
    \item
      \(T\) is a collection of proposition-value-\poP{} pairings.
    \end{itemize}
    \vspace{-\baselineskip}
  \end{restatable}

  Extensional characterisation.

  Generality.
\end{note}

\subsection{\illu{3}}
\label{cha:typical:sec:tor:illus}

\begin{note}
  Handful of illustrations of \tor{1}.
  In each case, premises on the left, and conclusion on the right.
  In all cases, the value True is left implicit.

  \begin{illustration}[\emph{Modus ponens}]
    \begin{tabular}[h]{p{.55\linewidth}|p{.4\linewidth}}
      P, if P Q & Q \\
      R, if R S & S \\
    \end{tabular}
  \end{illustration}

  \emph{Modus ponens} is natural.

  However, doesn't capture rule of inference.
  For, it may be the case that reform if P, Q to not-P or Q, and then disjunctive syllogism.

  However, could be \emph{modus profusus}%
  \footnote{
    \textquote{for any \(p\), \(q\), and \(r\): \((p \land q) \rightarrow r\)} (\cite[317]{Turri:2010aa})
    }


  \begin{illustration}[Applied addition]
    \begin{tabular}[h]{p{.55\linewidth}|p{.4\linewidth}}
      2 apples in left hand, 2 apples in right hand & 4 apples in hands \\
      5 pears in left hand, 7 cherries in right hand & 12 fruits in hands \\
      3 grapes in left hand, 4 grapes in right hand & 7 grapes in hands \\
      \(\hdots\) & \(\hdots\)
    \end{tabular}
    Applied addition.
  \end{illustration}


  \begin{illustration}[Bayesian Conditioning]
    \begin{tabular}[h]{p{.55\linewidth}|p{.4\linewidth}}
      \(\pr[B]{A} = .75\), just learnt \(B\)  & \(\pr[][B]{A} = .75\) \\
      \(\pr[C]{D} = .65\), just learnt  \(C\) & \(\pr[][C]{D} = .65\) \\
      \(\pr[B]{A} = .55\), just learnt \(B\) & \(\pr[][B]{\text{not-}A} = .45\)
    \end{tabular}
  \end{illustration}

  As with \emph{modus ponens}, Bayesian conditioning isn't too important.
  For example, instances are compatible with Jeffrey conditionalization.
\end{note}

\begin{note}
  \autoref{def:type-r} does not specify what the type is.
  Focus is not on the classification, but that reasoning may be classified.
\end{note}

\begin{note}
  Of interest is not with specific types, but generality.
  Specifying premises and conclusions, rather than whatever reasoning amounts to.
  This does not assume priority.
  Point is only that will at the very least have premises and conclusions.
\end{note}

\subsection{Background}
\label{cha:typical:sec:tor:bkgd}

\paragraph{Rule following}

\begin{note}
  So, this relationship may be understood in terms of rule following.

  For this, \citeauthor{Boghossian:2008vf}:

  \begin{quote}
    [O]ur internalization of general epistemic rules---like Modus Ponens and Induction---explain and rationalize why we form the beliefs that we form.
    And that seems intuitively correct.

    As in the case of our linguistic and conceptual abilities, our ability to form rational beliefs is \emph{productive}: on the basis of finite learning, we are able to form rational beliefs under a potential infinity of novel circumstances.
    The only plausible explanation for this is that we have, somehow, internalized a rule that tells us, in some general way, what it would be rational to believe under varying epistemic circumstances.%
    \mbox{ }\hfill\mbox{(\citeyear[483]{Boghossian:2008vf})}
  \end{quote}
\end{note}

\begin{note}
  On my understanding, this is, in part, the role of \citeauthor{Boghossian:2014aa}'s Taking Condition.

  Way in which \dots

  Indeed, \citeauthor{Boghossian:2014aa} highlights how condition allow to draw distinction between deductive and inductive.
  With taking, get generality.

  Indeed, \textcite{Boghossian:2014aa} is structured so that Taking is a generalisation of rule.
\end{note}

\begin{note}
  However, \tor{} does not need to amount to a rule.
  Rather, \tR{} only requires the rough phenomenon that \citeauthor{Boghossian:2008vf} argues rule following is the only plausible explanation of.%
  \footnote{
    Our interest with \tor{1} is independent of the worries about rule following raised by~\textcite{Kripke:1982aa}, to the extent that the worries raised by~\citeauthor{Kripke:1982aa} concern \emph{which} rule an agent is following, rather than \emph{whether} the agent is following a rule.
    At interest is not whether the \tor{} corresponds to plus or quus, but whether the agent's reasoning is of some type.
  }
\end{note}

\begin{note}
  \citeauthor{Hlobil:2014tq}'s ``Inferential Moorean Phenomenon'':
  \begin{quote}
    \begin{enumerate}
    \item[(IMP)]
      It is either impossible or seriously irrational to infer \emph{P} from \emph{Q} and to judge, at the same time, that the inference from \emph{Q} to \emph{P} is not a good inference.
    \end{enumerate}
    \dots
    By the ``goodness'' of an inference I mean the feature that makes the relevant inference permissible. Thus, if the inference under consideration is an inductive inference, the relevant kind of goodness is not deductive validity.%
    \mbox{ }\hfill\mbox{(\citeyear[\S1]{Hlobil:2014tq})}
  \end{quote}

  Though, this really isn't more basic given the interest in \tR{}.
  For, the puzzle is what it is to `infer'.

  Rationality isn't part of the picture.
  And, this is a significant drawback of \citeauthor{Hlobil:2014tq}'s approach.
\end{note}

\subsection{\tor{3} and an instances of reasoning}
\label{cha:typical:sec:tor:g-s}

\begin{note}
  So, if \tor{} is rule following, link between type and specific instance may amount to something like the taking condition.

  However, extensional approach.
  Something about other instances of the type.

  Our interest is not with what it fundamentally is, but in relation to other instances of general.
  For this, even if does not capture, plausible.

  \begin{suggestion}
    Instance of reasoning is type of reasoning only if would reason according to type.
  \end{suggestion}

  I.e., if rule, then with some adjustments, keep fixed rule and change premises.
  That rule, leads to conclusion.

  If not, then there is something about the reasoning which does not generalise.
\end{note}

\paragraph{The link}

\begin{note}
  \tor{2}.
  However, no clear link.

  Two basic issues.
  First, scope of type. (C).
  Second, application of type (H).
\end{note}

\paragraph{Scope}

\begin{note}
  Similar, though different from \citeauthor{Chomsky:2015aa}'s distinction between competence and performance.

  \begin{quote}
    Arithmetical competence yields the correct number z for every pair (x, y) under addition or multiplication.
    But only a small finite subpart of arithmetical competence can be exhibited without external aids (by calculating in one's head).
    Obviously, that fact does not imply that arithmetical competence is correspondingly limited.\newline
    \mbox{ }\hfill\mbox{(\citeyear[xii]{Chomsky:2015aa})}
  \end{quote}

  Agent may not, strictly, have arithmetical competence, as understood by \citeauthor{Chomsky:2015aa}.
  For, \emph{arbitrary} pairs.
\end{note}

\begin{note}
  Some motivation from the Wason selection task.
  For, it looks as though there's no clear type.
  Participants do well in some cases, and badly in others.
  So, seems relevant type can't be modus ponens.

  However, nothing hangs on the type being modus ponens.
  What's of interest is generality.
  Surprising that good with certain instances, and not good with others.
  So, possible to reify the relevant type.

  (\cite{Fodor:2000aa}) is a nice paper highlighting the way this may be done.
\end{note}

\paragraph{Application}

{
  \cite{Wilson:1994aa}, somewhere.
}

\begin{note}
  Difficulty with linking \tor{} with an instance of reasoning.

  For, there is no direct correspondence.

  Simple example, \citeauthor{Harman:1986ux}.
  Modus ponens, but equally modus tollens.
  Or, clutter.

  \begin{quote}
    logical principles are not directly rules of \emph{belief revision}.
    They are not particularly about belief [or the other mental states and acts that constitute reasoning] at all. (Harman 1984: 107)
  \end{quote}
  Not only belief, not \emph{particularly} about anything.

  Part of the insight \citeauthor{Harman:1986ux} offers is that there is no quick link between general and specific.
\end{note}

\begin{note}
  Harman, whether it makes sense for the agent.

  Though, more broadly, whether the agent has the option.

  Then, there's the tigger case (thanks Ray!).

  Though, to event.
  Only thing that matters is the presence of a tigger.

  Wow, gotta hide the medicine.
\end{note}

\begin{note}
  To link type to instance, some care.
  Next section.
\end{note}

\section{\tR{2}}
\label{cha:typical:sec:tR}

\begin{note}
  Previous section, type of reasoning.

  In this section, instances of reasoning which fall under type.

  We will not provide anything of much insight.

  Necessary condition.
\end{note}

\subsection{Definitions}
\label{cha:typical:sec:tR:defs}

\subsubsection{Representatives of a type of reasoning}
\label{sec:represntative}

\begin{note}
  \tor{} is general.
  Narrow down.
  In particular, useful for when the agent is limited.
\end{note}

\begin{note}
  \begin{restatable}[Representative of a type of reasoning]{definition}{defToRRep}
    \label{def:type-r}
    \cenLine{
      \begin{itemize*}[noitemsep, label=\(\circ\)]
      \item
        Agent: \vAgent{}
      \item
        Types of reasoning: \(T\), \(T'\)
      \item
        \mbox{ }
      \end{itemize*}
    }

    \begin{itemize}
    \item
      \(T'\) is representative \(T\) for \vAgent{} with respect to description \(d\).
    \end{itemize}

    \emph{If and only if}:

    \begin{itemize}
    \item
      For any event \(e\) under description \(d\):
      \begin{itemize}
      \item[\emph{If}:]
        \begin{enumerate}
        \item
          \(e\) is an event such that \vAgent{}' reasoning is of type \(T'\)
        \end{enumerate}
      \item[\emph{Then}:]
        \begin{enumerate}
        \item
          \(e\) is an event such that \vAgent{}' reasoning is of type \(T\).
        \end{enumerate}
      \end{itemize}
    \end{itemize}
    \vspace{-\baselineskip}
  \end{restatable}

  Role of this definition is to maintain extensional approach while avoiding scope and application.

  Link instance of reasoning to type under restrictions.
\end{note}

\begin{note}
  Given a representative, avoid difficult cases.
  Allow as many imperfections%
  \footnote{
    I.e.\ counterexamples.
  }
  as one wishes.
  The difficulty is maintaining the conditional.

  I do not have any interest in saying what this amounts to.

  Indeed, I have no particular interest in whether or not reasoning is of any \emph{particular} type.
  Rather, whether the reasoning is of \emph{some} type.
\end{note}

\begin{note}
  Omit a natural constraint.
  Intuitively, everything of type \(T'\) is also of type \(T\).

  I doubt there are any compelling counterexamples.
  However, there is also no need to add this.
  All that matters is that the conditional is true.
\end{note}

\begin{note}
  Now, whether the conditional holds.
  For, it's always going to be the case that wriggle in some doubt.

  However, this is beyond our interest.

  Indeed, any representative of a type is a type.

  And, so long as not trivial, \emph{some} generality.
  This is what we want to capture.
\end{note}

\subsubsection{\tR{2}}
\label{sec:tr3}

\begin{note}
  {
    \color{red}
    Partial definition!
  }
  \begin{definition}[\tR{2}]
    \label{def:cmptnc}
    \cenLine{
      \begin{itemize*}[noitemsep, label=\(\circ\)]
      \item
        Agent: \vAgent{}
      \item
        Propositions: \(\phi\), \(\psi\)
      \item
        Values: \(v\), \(v'\)
      \item
        \poP{3}: \(\Phi\), \(\Psi\)
      \item
        \mbox{ }
      \end{itemize*}
    }

    \noindent%
    \cenLine{
      \begin{itemize*}[noitemsep, label=\(\circ\)]
      \item
        Event: \(e\)
      \item
        Event description: \(d\)
      \item
        Type of reasoning: \(T\)
      \item
        \mbox{ }
      \end{itemize*}
    }

    \begin{itemize}
    \item
      \(e\) under description \(d\) is an event in which \vAgent{} is reasoning to \(\pv{\phi}{v}\) from \(\Phi\) by \emph{\tR{0}} of type \(T\).
    \end{itemize}

    \emph{Only if}:

    For some representative \(T'\) of \(T\):

    For every pairing in the collection \(T'\),

    Interval from action to current.

    \pevent{}.
  \end{definition}

  So, in short, pair descriptions of the event, and make sure that we have \pevent{1}.

  Intuitively, other things, choice of action.
  And, premises and conclusion.
  So, we get it to be the case that different action, different result.
  If this fails, then bad.
  Doing something which doesn't correspond to the type.
\end{note}

\begin{note}
  The work here is done by representative.
  Still, as observed above, type.
\end{note}


\begin{note}
  Intuitively, hold fixed the way in which the agent is reasoning.
  Then \dots

  Intuition is that necessary condition on \tR{} is reasoning in the same way from different premises.

    Other proposition-value-\poP{} pairings, reasoning if and only if applied proposition-value-\poP{} pairings.

  In more plain terms, reasoning is specific instance of general way of reasoning.%
  \footnote{
    Not designed to use `type' in the sense of type-token distinction.
    Common characteristic.
    Though, type distinct from instances.
  }
\end{note}

\subsection{\illu{3} of \tR{}}
\label{sec:illu3}

\begin{note}
  Three \scen{1}.
  For each \scen{0}, identify type, outline representative, and consider.
\end{note}


\begin{note}
  \begin{illustration}[Textbook]
    Student has been studying \dots something.

    Test understanding.

    Problem exercises.
  \end{illustration}
\end{note}

\begin{note}
  \begin{illustration}[Translation]
    Reading an newspaper.

    Headlines.

    Request, translation.

    Okay, and other headlines.
  \end{illustration}

  Keep request fixed, still translate different headline.

  Also, not a function, as multiple distinct translations.

  And, style of translation.

  \begin{center}
    \begin{tabular}[h]{p{.55\linewidth}|p{.4\linewidth}}
      \multicolumn{2}{c}{\emph{Translation}} \\
      これはペンです & This is a pen. \\
      このものは万年筆だ & This is a pen. \\
      この問題は難しいですね & This is a difficult problem, huh. \\
      この問題は難しいですね & This problem is kind of tough. \\
    \end{tabular}
  \end{center}
\end{note}

\begin{note}
  \begin{illustration}[Tic-tac-toe]
    Game.
  \end{illustration}

  Two types here.
  Type is reasoning as a competent tic-tac-toe player.

  So, in this case, don't leave open sure win unless forced.

  Second, no defeat.
  Here, follow the algorithm.

  If either thing goes wrong, then something is up.
\end{note}

\subsubsection{Edge cases}
\label{sec:not-tr0}

\begin{note}
  Single instance.

  So, working on an exam.
  Here, in order to get some type, need it to be the case that single representative is sufficient.

  Well, we haven't specified the way in which the conditional is true.
  However, we have to go from specific to general in some way.

  Okay, so relative to description.

  So, if build in.
  And, it's compatible that there isn't anything further to be said.
  It may just be basic that this is general reasoning.
\end{note}


\subsubsection{Alternatives}
\label{sec:alternatives}

\begin{note}
  Definition, necessary condition.

  Nearby definitions add.

  For example, rather than action to the agent, go via prompting.
  I give question, you get answer.

  This provides a nice account of the student taking a test.

  However, counterfactuals.
\end{note}

\subsection{Background}
\label{sec:background}



\begin{note}
  \tR{} is compatible with dispositionalism.

  % \begin{quote}
  %   But a capacity to grasp infinitely long propositions—the inputs in the rule following case—does not follow from our nature as thinking beings, and certainly not from our nature as physical beings.
  %   In fact, it seems pretty clear that we do not have that capacity and could not have it, no matter how liberally we apply the notion of idealization.
  % \end{quote}
\end{note}

\section{\sTR{2}}
\label{cha:typical:sec:sTR}


{
  \color{red}
  Change this so that there are \pevent{1}.
}

\begin{note}
  \begin{restatable}[\sTR{2}]{definition}{defSTR}
    \sTR{} just in case, for all representatives, \pevent{} in which reasons, no \pevent{} in which block.
  \end{restatable}

  Motivation for this is simple.
  \tR{} just looks at instance of reasoning.
  This is okay, but less interesting.
  For, if fail to do something of same type, then what is really going on?

  So, do things of type, and do not deviate from type.
\end{note}

\begin{note}
  \begin{proposition}
    \label{prop:sTRStronger}
    An instance of reasoning being a instance of \sTR{} is strictly stronger than the instance being \tR{}.
  \end{proposition}

  \begin{argument}{prop:sTRStronger}
    Taking an exam.

    If the agent started reasoning about any other instance, then the agent would immediately switch back to question on the exam.
    So, not reasoning to, as no perfection.
  \end{argument}

  This is interesting.
  For, extend this, and not a \fc{}.

  This may seem surprising.
  However, it is the result of focusing on specific event.
\end{note}

\begin{note}
  \begin{proposition}
    Cases where:

    If event, and \sTR{}, \fc{}.
  \end{proposition}

  So, if \tR{}, this means type of reasoning.

  Here, the interest is with respect to filling in a number after filling in a previous number.
  Doing some reasoning.
  This reasoning is \tR{}.
  So, repeat previous.
\end{note}

\begin{note}
  Proposition relies on the existence of an event.
  However, existence of an event is of little interest.
  If continues, then event as desired.
\end{note}

\begin{note}
  proposition is narrow, as only applies to illustration.

  However, the motivating idea is general.
\end{note}

\section{\citeauthor{Carroll:1895uj}}

\nocite{Black:1951aa}

\begin{note}
  The point here is that with Carroll, generality that goes beyond any single instance.
  Must apply to all instances, to be valid.
  But, cannot hope to cover all instances in a single move.
\end{note}

\begin{note}
  A difficulty found on a reading of \citeauthor{Carroll:1895uj}'s \citetitle{Carroll:1895uj}.
\end{note}

\begin{note}
  \begin{quote}
    ``Plenty of blank leaves, I see!'' the Tortoise cheerily remarked.
    ``We shall need them \emph{all}!''
    (Achilles shuddered.)
    ``Now write as I dictate:---

    \begin{enumerate}[label=(\emph{\Alph*}), ref=\emph{\Alph*}]
    \item
      \label{AatT:a}
      Things that are equal to the same are equal to each other.
    \item
      \label{AatT:b}
      The two sides of this Triangle are things that are equal to the same.
    \item
      \label{AatT:c}
      If~\ref{AatT:a} and~\ref{AatT:b} are true,~\ref{AatT:z} must be true.
      \setcounter{enumi}{25}
    \item
      \label{AatT:z}
      The two sides of this Triangle are equal to each other.''
    \end{enumerate}

    ``You should call it~\ref{AatT:d}, not~\ref{AatT:z},'' said Achilles.
    ``It comes \emph{next} to the other three.
    If you accept~\ref{AatT:a} and~\ref{AatT:b} and~\ref{AatT:c}, you \emph{must} accept~\ref{AatT:z}.''

    ``And why \emph{must} I?''

    ``Because it follows \emph{logically} from them.
    If~\ref{AatT:a} and~\ref{AatT:b} and~\ref{AatT:c} are true,~\ref{AatT:z} \emph{must} be true.
    You don't dispute \emph{that}, I imagine?''

    ``If~\ref{AatT:a} and~\ref{AatT:b} and~\ref{AatT:c} are true,~\ref{AatT:z} \emph{must} be true,'' the Tortoise thoughtfully repeated.
    ``That's \emph{another} Hypothetical, isn't it?
    And, if I failed to see its truth, I might accept~\ref{AatT:a} and~\ref{AatT:b} and~\ref{AatT:c}, and \emph{still} not accept~\ref{AatT:z}, mightn't I ?''

    \mbox{}\hfill\(\vdots\)\hfill\mbox{}

    ``Then Logic would take you by the throat, and force you to do it!''
    Achilles triumphantly replied.
    ``Logic would tell you 'You ca'n't help yourself.''%
    \mbox{ }\hfill\mbox{(\citeyear[279--280]{Carroll:1895uj})}
  \end{quote}

  The Tortoise has written down three premises,~\ref{AatT:a},~\ref{AatT:b}, and~\ref{AatT:c}.
  Achilles holds that~\ref{AatT:z} follows from~\ref{AatT:a},~\ref{AatT:b}, and~\ref{AatT:c}.
  The Tortoise observes they have the possibility of refraining to accept~\ref{AatT:z} follows from~\ref{AatT:a},~\ref{AatT:b}, and~\ref{AatT:c}.
  And (initially), the Tortoise does not accept~\ref{AatT:z} follows from~\ref{AatT:a},~\ref{AatT:b}, and~\ref{AatT:c}.
  Achilles requests the Tortoise accepts that~\ref{AatT:z} follows from~\ref{AatT:a},~\ref{AatT:b}, and~\ref{AatT:c}, and the Tortoise complies.
  Specifically, the Tortoise grants:

  \begin{quote}
    \begin{enumerate}[label=(\emph{\Alph*}), ref=\emph{\Alph*}]
      \setcounter{enumi}{3}
    \item
      \label{AatT:d}
      If~\ref{AatT:a} and~\ref{AatT:b} and~\ref{AatT:c} are true,~\ref{AatT:z} must be true.%
      \mbox{ }\hfill\mbox{(\citeyear[279]{Carroll:1895uj})}
    \end{enumerate}
  \end{quote}

  But, does not accept~\ref{AatT:z} follows from~\ref{AatT:a},~\ref{AatT:b},~\ref{AatT:c}, and~\ref{AatT:d}.
\end{note}

\begin{note}
  Modus ponens.

  \begin{quote}
    From \(\phi\) and \emph{if} \(\phi\) then \(\psi\), infer \(\psi\).
  \end{quote}

  Modus ponens is general.
  For \emph{any} \(\phi\), \(\psi\).

  Now, there is a difference between \emph{modus ponens} and conditional.

  However, take any instance.
  Then, if \(P\), \(P \rightarrow Q\), \(Q\) must be true.
  But, then this means that the conditional is true.

  Consequence of the deduction theorem.

  Likewise, deduction theorem goes the other way.

  However, going from \(P\), \(P \rightarrow Q\) to \(Q\) need not be an instance of \emph{modus ponens}.
\end{note}

\begin{note}
  Well, this is a headache.
  \citeauthor{Carroll:1895uj} is talking about a specific A, B, and Z.
  There is no clear generality.
\end{note}

\begin{note}
  So, consider at issue is modus ponens.
  For any specific instance accept, there is a further instance.
  For, \(A, (A \rightarrow B) \vDash B\).
  Then, \(\vDash (A \land (A \rightarrow B) \rightarrow B)\).
  However, now, \(A \land (A \rightarrow B), (A \land (A \rightarrow B) \rightarrow B) \vDash B\).
  And, so on.

  The general pattern, get conditional, but then this gives a new instance of modus ponens, which must be true in order for modus ponens to be valid rule of inference.

  \citeauthor{Carroll:1895uj}, by contrast, starts with \(A \vDash B\).
  This is different.
  However, rather than focusing on a single rule of inference, the puzzle turns on what validity amounts to.

  Validity is a general thing, with specific instances.
  However, grant any particular instance of validity without employing validity in general.
\end{note}

\begin{note}
  \begin{quote}
    My paradox \dots turns on the fact that, in a Hypothetical, the \emph{truth} of the Protasis, the \emph{truth} of the Apodosis, and the \emph{validity of the sequence}, are 3 distinct Propositions.

    \mbox{}\hfill\(\vdots\)\hfill\mbox{}

    Suppose I say ``I grant~\ref{AatT:a} and~\ref{AatT:b} and~\ref{AatT:c}, but I do \emph{not} grant that I am thereby \emph{obliged} to grant~\ref{AatT:z}.''
    Surely, my granting~\ref{AatT:z} must \emph{wait} until I have been made to see the validity of this sequence: i.e.\ in order to grant~\ref{AatT:z}, I must grant~\ref{AatT:a},~\ref{AatT:b},~\ref{AatT:c}, and~\ref{AatT:d}! And so on.%
    \mbox{ }\hfill\mbox{(\citeyear[472]{Carroll:1977wl})}
  \end{quote}

  My interpretation of the point \citeauthor{Carroll:1895uj} makes in this passage is that the truth of A B and the truth of C is distinct from the validity of A B C.
  Granting is substantial, not merely moving.
  But, in order to grant, this means granting all other cases.

  So, the paradox is that, on the one hand, don't need validity for any specific true things.
  But, on the other hand, only of interest if via validity.

  The Tortoise is slowly working through each instance, but this has no hope of getting the Tortoise to general validity.
  So, how does the Tortoise ever make it there?
\end{note}

\begin{note}
  This point differs from received interpretation.

  \citeauthor{Wieland:2013vf} (\citeyear{Wieland:2013vf}) characterises the general understanding of \textcite{Carroll:1895uj} in terms of two lessons:
  \begin{quote}
    [T]he negative lesson is that if you add ever more premises to an argument \dots, then you will never demonstrate that its conclusion follows logically.\newline
    \mbox{ }\hfill\mbox{(\citeyear[984]{Wieland:2013vf})}
  \end{quote}

  Parallel, static answers, still option for concluding otherwise.

  \begin{quote}
    [T]he positive lesson is that rules of inference, rather than premises of the form `if premises such and such are true, then the conclusion is true', will do the job.\newline
    \mbox{ }\hfill\mbox{(\citeyear[984]{Wieland:2013vf})}
  \end{quote}

  \begin{quote}
    [\citeauthor{Carroll:1895uj}] simply lacked any distinct conception of a deduction as opposed to the assertion (``granting'' of) a hypothetical proposition.
    \dots
    Any attempt by Carroll to tackle the question of inference was bound to begin in confusion and end in constipation-all those premises piling up, but no motion.
  \end{quote}
\end{note}

\paragraph{The Dichotomy}

\begin{note}
  Achilles and the Tortoise, Zeno's argument.

  Surely, right?

  Two ways to understand.
  Does the Tortoise move at all, or does the Tortoise arrive at the end?
  I mean, as formulated by Zeno, it's about catching up, no matter how much one moves.

  It is different from Zeno's Dichotomy paradox.


  If so, then we should expect the Tortoise to be making some movement.
  Adding rules of inference is of no help, because the problem is not movement, it's about how to move so much in a single step.
\end{note}

\begin{note}
  \color{red}
  Something about logic forcing.
  The Tortoise hasn't arrived.

  Nothing hangs on validity.
  Same issue with testimony.
  `A'.
  Why?
  Testified A, so A.
  Okay, but another instance of testimony.
  Testified(Testified A, so A), so Testified A, so A.
\end{note}

\begin{note}
  \begin{quote}
    But if we who wish to represent his belief in Q as based on P are to write in our notebook everything his having that belief on that basis consists in then when we have written only P and Q we will not have written enough.
    Someone can believe P and believe Q and still not believe Q on the basis of P whatever the relations between the propositions P and Q happen to be.
    He might believe Q for some reason completely unconnected with P, or perhaps for no reason at all (if that is possible).%
    \mbox{ }\hfill\mbox{(\citeyear[185]{Stroud:1979aa})}
  \end{quote}
  However, the moral drawn is narrow
  \begin{quote}
    The moral is that for every proposition or set of propositions the belief or acceptance of which is involved in someone's believing one proposition on the basis of another there must be something else, not simply a further proposition accepted, that is responsible for the one belief's being based on the other.%
    \mbox{ }\hfill\mbox{(\citeyear[187]{Stroud:1979aa})}
  \end{quote}

  Even if we grant each individual is \ros{}, rather than an instance of the material conditional, \emph{logic} hasn't done anything.
\end{note}

\paragraph{General and specific: Contrast}

\begin{note}
  Use \citeauthor{Carroll:1895uj} to illustrate this point.

  However, given the worry, various other things may be understood this way.

  Hume, constant conjunction.
  Part of the problem is identifying cause.
  We get the famous line about observing.
  However, Hume goes on.
  It's not only no observation, but no generality.

  Right, so more narrow than Hume.
  Because, with Hume, at issue is whether we have grounds for this general thing.
  With Carroll, it's whether we even really get to the general thing.
\end{note}


%%% Local Variables:
%%% mode: latex
%%% TeX-master: "master"
%%% End:


% \begin{note}
%   \begin{quote}
%     Let me ask this: what has the expression of a rule—say a sign-post—got to do with my actions?
%     What sort of connexion is there here?%
%     ---%
%     Well, perhaps this one:
%     I have been trained to react to this sign in a particular way, and now I do so react to it.

%     But that is only to give a causal connexion; to tell how it has come about that we now go by the sign-post; not what this going-by-the sign really consists in.
%     On the contrary; I have further indicated that a person goes by a sign-post only in so far as there exists a regular use of sign-posts, a custom.%
%     \mbox{ }\hfill\mbox{(\citeyear[\S198]{Wittgenstein:1958aa})}
%   \end{quote}

%   Regular use of sign-posts, custom.

%   Ugh, this is ambiguous.
% \end{note}
