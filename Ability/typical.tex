\chapter{\sR{2}}
\label{cha:typical}

\section{Type of reasoning}
\label{sec:gr-sr}

{
  \color{red}
  Somewhere, that this is not a necessary condition for concluding.
}

\begin{note}
  \begin{definition}[Type of reasoning]
    \label{def:type-r}
    \cenLine{
      \begin{itemize*}[noitemsep, label=\(\circ\)]
      \item
        Agent: \vAgent{}
      \item
        Type of reasoning: \(T\)
      \item
        \mbox{ }
      \end{itemize*}
    }

    \begin{itemize}
    \item
      \(T\) is a \emph{type of reasoning} for \vAgent{}.
    \end{itemize}

    \emph{If and only if}

    \begin{itemize}
    \item
      \(T\) is a collection of proposition-value-premises pairings.
    \end{itemize}
  \end{definition}

  Extensional characterisation of some general form of reasoning.
\end{note}

\begin{note}
  Handful of illustrations of types of reasoning.
  In each case, premises on the left, and conclusion on the right.
  In all cases, the value True is left implicit.


  \begin{illustration}[\emph{Modus ponens}]
    \begin{tabular}[h]{p{.55\linewidth}|p{.4\linewidth}}
      P, if P Q & Q \\
      R, if R S & S \\
    \end{tabular}
  \end{illustration}

  \emph{Modus ponens} is natural.
  However, could be \emph{modus profusus}%
  \footnote{
    \textquote{for any \(p\), \(q\), and \(r\): \((p \land q) \rightarrow r\)} (\cite[317]{Turri:2010aa})
    }


  \begin{illustration}[Applied addition]
    \begin{tabular}[h]{p{.55\linewidth}|p{.4\linewidth}}
      2 apples in left hand, 2 apples in right hand & 4 apples in hands \\
      5 pears in left hand, 7 cherries in right hand & 12 fruits in hands \\
      3 grapes in left hand, 4 grapes in right hand & 7 grapes in hands \\
      \(\hdots\) & \(\hdots\)
    \end{tabular}
    Applied addition.
  \end{illustration}


  \begin{illustration}[Bayesian Conditioning]
    \begin{tabular}[h]{p{.55\linewidth}|p{.4\linewidth}}
      \(\pr[B]{A} = .75\), just learnt \(B\)  & \(\pr[][B]{A} = .75\) \\
      \(\pr[C]{D} = .65\), just learnt  \(C\) & \(\pr[][C]{D} = .65\) \\
      \(\pr[B]{A} = .55\), just learnt \(B\) & \(\pr[][B]{\text{not-}A} = .45\)
    \end{tabular}
  \end{illustration}

  As with \emph{modus ponens}, Bayesian conditioning isn't too important.
  For example, instances are compatible with Jeffrey conditionalization.
\end{note}

\begin{note}
  \autoref{def:type-r} does not specify what the type is.
  Focus is not on the classification, but that reasoning may be classified.
\end{note}

\begin{note}
  Of interest is not with specific types, but generality.
  Specifying premises and conclusions, rather than whatever reasoning amounts to.
  This does not assume priority.
  Point is only that will at the very least have premises and conclusions.
\end{note}


\section{\sR{2}}
\label{sec:sr2-1}

\begin{note}
  \begin{definition}[\sR{2}]
    \label{def:cmptnc}
    \cenLine{
      \begin{itemize*}[noitemsep, label=\(\circ\)]
      \item
        Agent: \vAgent{}
      \item
        Event: \(e\)
      \item
        Propositions: \(\phi\), \(\psi\)
      \item
        Values: \(v\), \(v'\)
      \item
        \poP{3}: \(\Phi\), \(\Psi\)
      \item
        Type of reasoning: \(T\)
      \item
        \mbox{ }
      \end{itemize*}
    }

    Grating \(e\) is an event in which \vAgent{} is reasoning to \(\pv{\phi}{v}\) from \(\Phi\):

    \begin{itemize}
    \item
      \vAgent{}' reasoning to \(\pv{\phi}{v}\) from \(\Phi\) in \(e\) is \emph{\sR{0}} of type \(T\) (to \(\pv{\phi}{v}\) from \(\Phi\) in \(e\)).
    \end{itemize}

    \emph{If and only if}

    \begin{itemize}
    \item
      \begin{itemize}
      \item[\emph{If}:]
        \begin{enumerate}[label=\alph*., ref=(\alph*)]
        \item
          \vAgent{} would be reasoning to \(\pv{\psi}{v'}\), were \vAgent{} reasoning from \(\Psi\).%
          \footnote{
            More explicitly:
            \begin{itemize}
            \item
              \vAgent{} would be reasoning to \(\pv{\psi}{v'}\) in \(e\), were \vAgent{} were reasoning from \(\Psi\) in \(e\) --- instead of reasoning to \(\pv{\phi}{v}\) from \(\Phi\).
            \end{itemize}
          }
        \end{enumerate}
      \item[\emph{Then}:]
        \begin{enumerate}[label=\alph*., ref=(\alph*), resume]
        \item
          \(\pvp{\psi}{v'}{\Psi}\) is in \(T\).
        \end{enumerate}
      \end{itemize}
    \end{itemize}

    Intuition is that whether or not reasoning is \sR{} depends on whether the agent would have obtained an appropriate conclusion from different set of premises.

    Other proposition-value-premises pairings, reasoning if and only if applied proposition-value-premises pairings.
  \end{definition}

  In more plain terms, reasoning is specific instance of general way of reasoning.%
  \footnote{
    Not designed to use `type' in the sense of type-token distinction.
    Common characteristic.
    Though, type distinct from instances.
  }
\end{note}

\begin{note}
  \begin{center}
    \begin{tabular}[h]{p{.55\linewidth}|p{.4\linewidth}}
      \multicolumn{2}{c}{\emph{Translation}} \\
      これはペンです & This is a pen \\
      この問題は難しいですね & This is a difficult problem \\
      この問題は難しいですね & This problem is kind of tough \\
    \end{tabular}
  \end{center}

  Translation is an interesting case.
  Highlights two things.

  First, type of reasoning does not amount to a function.
  For, multiple translations.
  Second, for the agent.
  Agent may have a particular style of translation.
  Need not be the case that agent would perform all permissible translations.
\end{note}

\begin{note}
  \begin{center}
    \begin{tabular}[h]{p{.55\linewidth}|p{.4\linewidth}}
      \multicolumn{2}{c}{\emph{Chess}} \\
      Board & Win! \\
      Same board with the colours swapped & Loss! \\
    \end{tabular}
  \end{center}
\end{note}

{
  \color{red}

  The remainder of this section are general notes to support/clarify \sR{}.

  In short, think of reasoning in terms of rule following.

  However, \sR{} does not necessarily amount to rule following.

  In particular, \sR{} is independent of the worries about rule following raised by ~\textcite{Kripke:1982aa}, to the extent that the worries concern \emph{which} rule an agent is following.
  However, if the worries extend to \emph{whether} an agent is following a rule, then the worries raise issues for \sR{}.
  However, if this is the case, then there is no clear sense in which an agent's reasoning falls under any general description.

  A section on \textcite{Carroll:1895uj} is planned where I highlight an interpretation of the tale in terms of validity as general reasoning.
  In short, the Tortoise wants to see whether the argument Achilles proposes is valid, not merely whether it has true premises and a true conclusion.
  Each instance of the material conditional captures an instance of validity, but this does not get the Tortoise close to an understanding of validity in general.
  For, given each instance of validity, there is a further instance which the Tortoise has not yet accepted.

  An interesting thing about this interpretation is that the usual suggestion of a distinction between premises and rules of inference is of no help, because rather than framing the tale in terms of validity, one may focus on the validity of modus ponens, and the same issue arises.
  No number of specific instance gets the validity of modus ponens.
  Hence, what guarantee does the Tortoise have that the inference, as proposed by Achilles, would amount to a valid inference?

  Yet, it is intuitively possible to infer via a valid argument, and in particular, via modus ponens.
}

\section{Background}
\label{sec:background}

\subsection{Rule following}

\paragraph{Boghossian}

\begin{note}
  So, this relationship may be understood in terms of rule following.

  For this, \citeauthor{Boghossian:2008vf}:

  \begin{quote}
    [O]ur internalization of general epistemic rules---like Modus Ponens and Induction---explain and rationalize why we form the beliefs that we form.
    And that seems intuitively correct.

    As in the case of our linguistic and conceptual abilities, our ability to form rational beliefs is \emph{productive}: on the basis of finite learning, we are able to form rational beliefs under a potential infinity of novel circumstances.
    The only plausible explanation for this is that we have, somehow, internalized a rule that tells us, in some general way, what it would be rational to believe under varying epistemic circumstances.%
    \mbox{ }\hfill\mbox{(\citeyear[483]{Boghossian:2008vf})}
  \end{quote}
\end{note}

\begin{note}
  On my understanding, this is, in part, the role of \citeauthor{Boghossian:2014aa}'s Taking Condition.

  Way in which \dots

  Indeed, \citeauthor{Boghossian:2014aa} highlights how condition allow to draw distinction between deductive and inductive.
  With taking, get generality.

  Indeed, \textcite{Boghossian:2014aa} is structured so that Taking is a generalisation of rule.
\end{note}

\begin{note}
  \citeauthor{Hlobil:2014tq}'s ``Inferential Moorean Phenomenon'':
  \begin{quote}
    \begin{enumerate}
    \item[(IMP)]
      It is either impossible or seriously irrational to infer \emph{P} from \emph{Q} and to judge, at the same time, that the inference from \emph{Q} to \emph{P} is not a good inference.
    \end{enumerate}
    \dots
    By the ``goodness'' of an inference I mean the feature that makes the relevant inference permissible. Thus, if the inference under consideration is an inductive inference, the relevant kind of goodness is not deductive validity.%
    \mbox{ }\hfill\mbox{(\citeyear[\S1]{Hlobil:2014tq})}
  \end{quote}

  Though, this really isn't more basic given the interest in \sR{}.
  For, the puzzle is what it is to `infer'.

  Rationality isn't part of the picture.
  And, this is a significant drawback of \citeauthor{Hlobil:2014tq}'s approach.
\end{note}


\begin{note}
  However, \sR{} does not require rule following.
  Rather, \sR{} only requires the rough phenomenon that \citeauthor{Boghossian:2008vf} argues rule following is the only plausible explanation of.
\end{note}


\paragraph{Chomsky}

\begin{note}
  Similar, though different from \citeauthor{Chomsky:2015aa}'s distinction between competence and performance.

  Similarity is generality.
  Difference is idealisation.

  On our understanding, competence does not involve idealisation.

  \begin{quote}
    Arithmetical competence yields the correct number z for every pair (x, y) under addition or multiplication.
    But only a small finite subpart of arithmetical competence can be exhibited without external aids (by calculating in one's head).
    Obviously, that fact does not imply that arithmetical competence is correspondingly limited.\newline
    \mbox{ }\hfill\mbox{(\citeyear[xii]{Chomsky:2015aa})}
  \end{quote}

  Agent may not, strictly, have arithmetical competence, as understood by \citeauthor{Chomsky:2015aa}.
  For, \emph{arbitrary} pairs.

  Whatever the generality of \sR{0} on our understanding, it may fail for arbitrary pairs.
  No guarantees about idealisation.
  Still, general.
  Many instance of multiplying, and, when multiplying, reasoning is, typically, at least, \sR{}.
\end{note}

\begin{note}
  Primary upshot is that \sR{} does not depend on resolutions to rule following paradox.

  To illustrate.
  Strictly, \sR{} does not depend on addition.
  Compatible that \sR{} is smaddition(?)

  Though, does depend on basic assumption of generality.
  To the extent worried about smaddition(?) due to no generality, then sure.
\end{note}

\begin{note}
  \begin{quote}
    Let me ask this: what has the expression of a rule—say a sign-post—got to do with my actions?
    What sort of connexion is there here?%
    ---%
    Well, perhaps this one:
    I have been trained to react to this sign in a particular way, and now I do so react to it.

    But that is only to give a causal connexion; to tell how it has come about that we now go by the sign-post; not what this going-by-the sign really consists in.
    On the contrary; I have further indicated that a person goes by a sign-post only in so far as there exists a regular use of sign-posts, a custom.%
    \mbox{ }\hfill\mbox{(\citeyear[\S198]{Wittgenstein:1958aa})}
  \end{quote}

  Regular use of sign-posts, custom.
\end{note}

\begin{note}
  \sR{} is compatible with dispositionalism.

  % \begin{quote}
  %   But a capacity to grasp infinitely long propositions—the inputs in the rule following case—does not follow from our nature as thinking beings, and certainly not from our nature as physical beings.
  %   In fact, it seems pretty clear that we do not have that capacity and could not have it, no matter how liberally we apply the notion of idealization.
  % \end{quote}
\end{note}

\begin{note}
  Some additional motivation from the literature\dots
\end{note}

\paragraph{Particularism}

\begin{note}
  Opposing view is particularlism.
  There's nothing general.

  Some motivation from the Wason selection task.
  For, it looks as though there's no clear type.
  Participants do well in some cases, and badly in others.
  So, seems relevant type can't be modus ponens.

  However, nothing hangs on the type being modus ponens.
  What's of interest is generality.
  Surprising that good with certain instances, and not good with others.
  So, possible to reify the relevant type.

  (\cite{Fodor:2000aa}) is a nice paper highlighting the way this may be done.
\end{note}

\section{\citeauthor{Carroll:1895uj}}

\nocite{Black:1951aa}

\begin{note}
  The point here is that with Carroll, generality that goes beyond any single instance.
  Must apply to all instances, to be valid.
  But, cannot hope to cover all instances in a single move.
\end{note}

\begin{note}
  A difficulty found on a reading of \citeauthor{Carroll:1895uj}'s \citetitle{Carroll:1895uj}.
\end{note}

\begin{note}
  \begin{quote}
    ``Plenty of blank leaves, I see!'' the Tortoise cheerily remarked.
    ``We shall need them \emph{all}!''
    (Achilles shuddered.)
    ``Now write as I dictate:---

    \begin{enumerate}[label=(\emph{\Alph*}), ref=\emph{\Alph*}]
    \item
      \label{AatT:a}
      Things that are equal to the same are equal to each other.
    \item
      \label{AatT:b}
      The two sides of this Triangle are things that are equal to the same.
    \item
      \label{AatT:c}
      If~\ref{AatT:a} and~\ref{AatT:b} are true,~\ref{AatT:z} must be true.
      \setcounter{enumi}{25}
    \item
      \label{AatT:z}
      The two sides of this Triangle are equal to each other.''
    \end{enumerate}

    ``You should call it~\ref{AatT:d}, not~\ref{AatT:z},'' said Achilles.
    ``It comes \emph{next} to the other three.
    If you accept~\ref{AatT:a} and~\ref{AatT:b} and~\ref{AatT:c}, you \emph{must} accept~\ref{AatT:z}.''

    ``And why \emph{must} I?''

    ``Because it follows \emph{logically} from them.
    If~\ref{AatT:a} and~\ref{AatT:b} and~\ref{AatT:c} are true,~\ref{AatT:z} \emph{must} be true.
    You don't dispute \emph{that}, I imagine?''

    ``If~\ref{AatT:a} and~\ref{AatT:b} and~\ref{AatT:c} are true,~\ref{AatT:z} \emph{must} be true,'' the Tortoise thoughtfully repeated.
    ``That's \emph{another} Hypothetical, isn't it?
    And, if I failed to see its truth, I might accept~\ref{AatT:a} and~\ref{AatT:b} and~\ref{AatT:c}, and \emph{still} not accept~\ref{AatT:z}, mightn't I ?''

    \mbox{}\hfill\(\vdots\)\hfill\mbox{}

    ``Then Logic would take you by the throat, and force you to do it!''
    Achilles triumphantly replied.
    ``Logic would tell you 'You ca'n't help yourself.''%
    \mbox{ }\hfill\mbox{(\citeyear[279--280]{Carroll:1895uj})}
  \end{quote}

  The Tortoise has written down three premises,~\ref{AatT:a},~\ref{AatT:b}, and~\ref{AatT:c}.
  Achilles holds that~\ref{AatT:z} follows from~\ref{AatT:a},~\ref{AatT:b}, and~\ref{AatT:c}.
  The Tortoise observes they have the possibility of refraining to accept~\ref{AatT:z} follows from~\ref{AatT:a},~\ref{AatT:b}, and~\ref{AatT:c}.
  And (initially), the Tortoise does not accept~\ref{AatT:z} follows from~\ref{AatT:a},~\ref{AatT:b}, and~\ref{AatT:c}.
  Achilles requests the Tortoise accepts that~\ref{AatT:z} follows from~\ref{AatT:a},~\ref{AatT:b}, and~\ref{AatT:c}, and the Tortoise complies.
  Specifically, the Tortoise grants:

  \begin{quote}
    \begin{enumerate}[label=(\emph{\Alph*}), ref=\emph{\Alph*}]
      \setcounter{enumi}{3}
    \item
      \label{AatT:d}
      If~\ref{AatT:a} and~\ref{AatT:b} and~\ref{AatT:c} are true,~\ref{AatT:z} must be true.%
      \mbox{ }\hfill\mbox{(\citeyear[279]{Carroll:1895uj})}
    \end{enumerate}
  \end{quote}

  But, does not accept~\ref{AatT:z} follows from~\ref{AatT:a},~\ref{AatT:b},~\ref{AatT:c}, and~\ref{AatT:d}.
\end{note}

\begin{note}
  Modus ponens.

  \begin{quote}
    From \(\phi\) and \emph{if} \(\phi\) then \(\psi\), infer \(\psi\).
  \end{quote}

  Modus ponens is general.
  For \emph{any} \(\phi\), \(\psi\).

  Now, there is a difference between \emph{modus ponens} and conditional.

  However, take any instance.
  Then, if \(P\), \(P \rightarrow Q\), \(Q\) must be true.
  But, then this means that the conditional is true.

  Consequence of the deduction theorem.

  Likewise, deduction theorem goes the other way.

  However, going from \(P\), \(P \rightarrow Q\) to \(Q\) need not be an instance of \emph{modus ponens}.
\end{note}

\begin{note}
  Well, this is a headache.
  \citeauthor{Carroll:1895uj} is talking about a specific A, B, and Z.
  There is no clear generality.
\end{note}

\begin{note}
  So, consider at issue is modus ponens.
  For any specific instance accept, there is a further instance.
  For, \(A, (A \rightarrow B) \vDash B\).
  Then, \(\vDash (A \land (A \rightarrow B) \rightarrow B)\).
  However, now, \(A \land (A \rightarrow B), (A \land (A \rightarrow B) \rightarrow B) \vDash B\).
  And, so on.

  The general pattern, get conditional, but then this gives a new instance of modus ponens, which must be true in order for modus ponens to be valid rule of inference.

  \citeauthor{Carroll:1895uj}, by contrast, starts with \(A \vDash B\).
  This is different.
  However, rather than focusing on a single rule of inference, the puzzle turns on what validity amounts to.

  Validity is a general thing, with specific instances.
  However, grant any particular instance of validity without employing validity in general.
\end{note}

\begin{note}
  \begin{quote}
    My paradox \dots turns on the fact that, in a Hypothetical, the \emph{truth} of the Protasis, the \emph{truth} of the Apodosis, and the \emph{validity of the sequence}, are 3 distinct Propositions.

    \mbox{}\hfill\(\vdots\)\hfill\mbox{}

    Suppose I say ``I grant~\ref{AatT:a} and~\ref{AatT:b} and~\ref{AatT:c}, but I do \emph{not} grant that I am thereby \emph{obliged} to grant~\ref{AatT:z}.''
    Surely, my granting~\ref{AatT:z} must \emph{wait} until I have been made to see the validity of this sequence: i.e.\ in order to grant~\ref{AatT:z}, I must grant~\ref{AatT:a},~\ref{AatT:b},~\ref{AatT:c}, and~\ref{AatT:d}! And so on.%
    \mbox{ }\hfill\mbox{(\citeyear[472]{Carroll:1977wl})}
  \end{quote}

  My interpretation of the point \citeauthor{Carroll:1895uj} makes in this passage is that the truth of A B and the truth of C is distinct from the validity of A B C.
  Granting is substantial, not merely moving.
  But, in order to grant, this means granting all other cases.

  So, the paradox is that, on the one hand, don't need validity for any specific true things.
  But, on the other hand, only of interest if via validity.

  The Tortoise is slowly working through each instance, but this has no hope of getting the Tortoise to general validity.
  So, how does the Tortoise ever make it there?
\end{note}

\begin{note}
  This point differs from received interpretation.

  \citeauthor{Wieland:2013vf} (\citeyear{Wieland:2013vf}) characterises the general understanding of \textcite{Carroll:1895uj} in terms of two lessons:
  \begin{quote}
    [T]he negative lesson is that if you add ever more premises to an argument \dots, then you will never demonstrate that its conclusion follows logically.\newline
    \mbox{ }\hfill\mbox{(\citeyear[984]{Wieland:2013vf})}
  \end{quote}

  Parallel, static answers, still option for concluding otherwise.

  \begin{quote}
    [T]he positive lesson is that rules of inference, rather than premises of the form `if premises such and such are true, then the conclusion is true', will do the job.\newline
    \mbox{ }\hfill\mbox{(\citeyear[984]{Wieland:2013vf})}
  \end{quote}

  \begin{quote}
    [\citeauthor{Carroll:1895uj}] simply lacked any distinct conception of a deduction as opposed to the assertion (``granting'' of) a hypothetical proposition.
    \dots
    Any attempt by Carroll to tackle the question of inference was bound to begin in confusion and end in constipation-all those premises piling up, but no motion.
  \end{quote}

  Achilles and the Tortoise, Zeno's argument.

  Surely, right?

  Two ways to understand.
  Does the Tortoise move at all, or does the Tortoise arrive at the end?
  I mean, as formulated by Zeno, it's about catching up, no matter how much one moves.

  It is different from Zeno's Dichotomy paradox.


  If so, then we should expect the Tortoise to be making some movement.
  Adding rules of inference is of no help, because the problem is not movement, it's about how to move so much in a single step.
\end{note}

\begin{note}
  \color{red}
  Something about logic forcing.
  The Tortoise hasn't arrived.

  Nothing hangs on validity.
  Same issue with testimony.
  `A'.
  Why?
  Testified A, so A.
  Okay, but another instance of testimony.
  Testified(Testified A, so A), so Testified A, so A.
\end{note}

\begin{note}
  \begin{quote}
    But if we who wish to represent his belief in Q as based on P are to write in our notebook everything his having that belief on that basis consists in then when we have written only P and Q we will not have written enough.
    Someone can believe P and believe Q and still not believe Q on the basis of P whatever the relations between the propositions P and Q happen to be.
    He might believe Q for some reason completely unconnected with P, or perhaps for no reason at all (if that is possible).%
    \mbox{ }\hfill\mbox{(\citeyear[185]{Stroud:1979aa})}
  \end{quote}
  However, the moral drawn is narrow
  \begin{quote}
    The moral is that for every proposition or set of propositions the belief or acceptance of which is involved in someone's believing one proposition on the basis of another there must be something else, not simply a further proposition accepted, that is responsible for the one belief's being based on the other.%
    \mbox{ }\hfill\mbox{(\citeyear[187]{Stroud:1979aa})}
  \end{quote}

  Even if we grant each individual is \ros{}, rather than an instance of the material conditional, \emph{logic} hasn't done anything.
\end{note}


%%% Local Variables:
%%% mode: latex
%%% TeX-master: "master"
%%% End:
