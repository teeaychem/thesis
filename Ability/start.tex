\chapter{Introduction}
\label{cha:introduction}


\begin{note}
  This document shows a constraint on (partial) explanations of \eiw{1} an agent make a conclusion from some \pool{} of premises does not hold.
  Here the constraint is introduced in a relaxed way and we indicate the kind of argument to follow.

  In particular, The style of the document proper is somewhat precise.
  So, here, hopefully, enough is said for you to decide whether to carefully work through some ideas.
\end{note}


\begin{note}
  When an agent makes a conclusion from some \pool{} of premises, some relation holds between the conclusion and the \pool{} of premises from the \agpe{}.
  For, at least, the agent obtains the conclusion \emph{from} the \pool{}.
  In addition to `from' we talk about `\fingf{0}' relations.
  Two key assumptions are made:
  %
  \begin{enumerate}[label=\Alph*., ref=(\Alph*)]
  \item
    \label{start:A}
    If an agent is in the process of concluding some conclusion \(C\) from some \pool{} \(P\), then \(C\) \fof{} \(P\), from the \agpe{} (while the agent is concluding).
  \item
    \label{start:B}
    It \emph{may} be \(C\) \fof{} \(P\), from the \agpe{}, though the agent has not been in the process of concluding \(C\) from \(P\).%
    \footnote{
      \ref{start:A} already entails it may be that \(C\) \fof{} \(P\), from the \agpe{}, though the agent has not concluded \(C\) from \(P\).
    }
  \end{enumerate}
\end{note}

\begin{note}
  In support of \ref{start:A}, when an agent is in the process of concluding \(C\) from \(P\) some thing ensures it is the case the agent is concluding \(C\) from \(P\) rather than \(C'\) from \(P\), \(C\) from \(P'\), etc.
  So, for example, when an agent is concluding \(43 \times 12 = 516\) from their understanding of arithmetic, \(43 \times 12 = 516\) \fof{} some \pool{} of premises associated with their understanding of arithmetic, from the \agpe{}.

  Likewise, nothing about the agent's understanding of arithmetic changed when the agent began concluding, so in support of \ref{start:B}, it seems \(43 \times 12  = 516\) followed from the relevant \pool{}, from the \agpe{}, (maybe long) before the agent began concluding.%
  \footnote{
    By contrast, if an agent is choosing between different flavours of ice cream, it need not be the case that there is a \fingfr{} between choice of chocolate and some \pool{} --- the eventual choice may be spontaneous.

    If you're unsure about this idea in general, then it is fine to narrow attention to cases like arithmetic where an agent has some well-understood method of obtaining a conclusion from a \pool{} of premises, and needs only apply the method to obtain the conclusion.

    All the important cases in the document have this form.
  }
\end{note}


\begin{note}
  Given our example \eiw{} an agent concludes \(43 \times 12 = 516\) from some \pool{} associated with the agent's understanding of arithmetic one may ask:
  Why was the event an \eiw{} the agent concludes \(43 \times 12 = 516\) from that \pool{} (rather than an \eiw{} some other thing happened)?%
  \footnote{
    Stated a little awkwardly so as not to assume agency had any role.
  }

  In response one may observe the agent has a fair understanding of arithmetic, and they're on point today.
  So, when the agent started to think about \(43 \times 12\) they started to apply their understanding of arithmetic.
  And, they kept going until they figured out \(43 \times 12 = 516\).

  In short, the agent was concluding \(43 \times 12 = 516\) from the relevant \pool{}.
  Some things that were \emph{while} the event was progressing (partially) explain why the event happened.
\end{note}


\begin{note}
  Now, given \ref{start:A}, when an agent is concluding \(C\) from \(P\) it is the case \(C\) \fof{} \(P\).
  So, as \(C\) \fingf{0} \(P\) is closely connected to the agent concluding \(C\) from \(P\), \(C\) \fingf{0} \(P\) may (partially) explain why the \eiw{} the agent concludes \(C\) from \(P\) happens.
  In particular, the main document argues \(C\) \fingf{0} \(P\) \emph{does} so (partially) explain, so long as the agent is concluding \(C\) from \(P\) during the event.

  Taking (partial) explanation by \(C\) \fingf{0} \(P\) for granted, our attention turns to \ref{start:B}.
  For, given \ref{start:B} there may be other \fingfr{1} present when the agent is concluding \(C\) from \(P\).
  Perhaps other \fingfr{1} also (partially) explain why the agent concludes \(C\) from \(P\)?
  Or, perhaps, the following constraint holds:

  \begin{constraint}{consFirst}{A first pass}
    \(C'\) \fingf{} \(P'\) (partially) explains why an \eiw{0} an agent concludes \(C\) from \(P\) happened \emph{only if} \(C'\) is \(C\) and \(P'\) is \(P\).
  \end{constraint}

  \noindent%
  In other words, no \fingfr{} \emph{other than} \(C\) \fingf{} \(P\) may (partially) explain why an \eiw{0} an agent concludes \(C\) from \(P\) happened.
\end{note}


\begin{note}
  I think \autoref{consFirst} makes some sense.

  For example, consider \(3 \times 3 = 9\) \fingf{} some \pool{} associated with the agent's understanding of arithmetic.
  Above, the agent concludes \(43 \times 12 = 516\) rather than \(3 \times 3 = 9\), and in principle it seems the agent could conclude \(43 \times 12 = 516\) without having figured out \(3 \times 3 = 9\).
  So, nothing about what happened seems to rest on \(3 \times 3 = 9\) \fingf{} any \pool{} associated with the agent's understanding of arithmetic and hence this \fingfr{} appears irrelevant to (partially) explaining the relevant event.
\end{note}


\begin{note}
  Still, I don't think \autoref{consFirst} makes complete sense.

  For, an \eiw{0} an agent concludes \(C\) from \(P\) may include a sub-event in which the agent concludes \(C'\) from \(P'\).
  And in this respect, \(C'\) \fingf{} \(P'\) may (partially) explain why the \eiw{0} an agent concludes \(C\) from \(P\) happened --- both \fingfr{1} were `witnessed' during the event.

  So, I'd like to consider a weaker constraint:

  \begin{constraint}{consSecond}{A second pass}
    \(C'\) \fingf{} \(P'\) (partially) explains why an \eiw{0} an agent concludes \(C\) from \(P\) happened \emph{only if} the agent has concluded or is concluding \(C'\) from \(P'\).
  \end{constraint}

  \noindent%
  Given \autoref{consSecond}, a \emph{\fingfr{}} requires some extant \emph{from relation} to be (partially) explanatory.
\end{note}


\begin{note}
  \autoref{consFirst} entails \autoref{consSecond}, covers sub-conclusions, and more.
  Perhaps \autoref{consSecond} is too weak!
  Still, if \autoref{consSecond} fails to hold, then any weaker constraint (such as \autoref{consFirst}) also fails to hold.
  Hence, an argument against \autoref{consSecond} is an argument against any weaker constraint.
\end{note}

\begin{note}
  This document shows \autoref{consSecond} fails to hold.

  Part of this document amounts to making the idea of a \fingfr{} precise, and motivating \fingfr{1} as (partial) explains of why an \eiw{} an agent concludes happens.

  The other part of this document amounts to showing the precise variant of \autoref{consSecond} fails to hold.
  Specifically, a deductive argument establishes \emph{if} certain conditions hold \emph{then} \autoref{consSecond} fails to hold, together with supporting argumentation suggests the relevant conditions are common.

  For those still tentative, \autoref{cha:intro} contains more details and a sketch of the overall argument.
\end{note}


%%% Local Variables:
%%% mode: latex
%%% TeX-master: "master"
%%% TeX-engine: luatex
%%% End:
