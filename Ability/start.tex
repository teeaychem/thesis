\chapter{Introduction}
\label{cha:introduction}


\begin{note}
  This document shows a constraint on (partial) explanations of \eiw{0} an agent draws a conclusion from some \pool{} of premises does not hold.
  This introduction introduces the constraint in a relaxed manner.

  The main document is quite precise.
  Here, hopefully, enough is said for you to decide whether you would like to work through some ideas very carefully.
\end{note}


\begin{note}
  When an \eiw{0} an agent draws a conclusion from some \pool{} of premises some relation holds between conclusion and the \pool{} of premises from the \agpe{}.

  We talk about `\fingfr{1}' and two key assumptions are made:
  %
  \begin{enumerate}[label=\Alph*., ref=(\Alph*)]
  \item
    \label{start:A}
    If an agent is in the process of concluding some conclusion \(C\) from some \pool{} \(P\), then \(C\) \fof{} \(P\), from the \agpe{}.
  \item
    \label{start:B}
    It \emph{may} be \(C'\) \fof{} \(P'\), from the \agpe{}, though the agent has not been in the process of concluding \(C'\) from \(P'\).%
    \footnote{
      \ref{start:A} already entails it may be \(C'\) \fof{} \(P'\), from the \agpe{}, though the agent has not concluded \(C'\) from \(P'\).
    }
  \end{enumerate}
\end{note}

\begin{note}
  In support of \ref{start:A} and \ref{start:B}, when an agent is in the process of concluding \(C\) from \(P\) something ensures it is the case the agent is concluding \(C\) from \(P\) rather than concluding \(C'\) from \(P\) or concluding \(C\) from \(P'\).

  So, for example, when an agent is concluding \(43 \times 12 = 516\) from their understanding of arithmetic, \(43 \times 12 = 516\) \fof{} some \pool{} of premises associated with the agent's understanding of arithmetic, from the \agpe{}.

  Likewise, nothing about the agent's understanding of arithmetic changed when the agent began concluding, so it seems \(43 \times 12  = 516\) followed from the \pool{} of premises associated with the agent's understanding of arithmetic, from the \agpe{}, before the agent began concluding.%
  \footnote{
    By contrast, if an agent is choosing between different flavours of ice cream, it need not be the case that there is a \fingfr{} between chocolate and some \pool{}.
    For, it may not be the case the agent is concluding chocolate chip --- the eventual choice may be spontaneous.

    If you're unsure about this idea in general, then it is fine to narrow attention to cases like arithmetic where an agent has some well-understood method of obtaining a conclusion from a \pool{} of premises, and needs only apply the method to obtain the conclusion.

    All the important cases in the document have this form.
  }
\end{note}


\begin{note}
  Given an \eiw{} an agent concludes \(43 \times 12 = 516\) we make ask:
  Why was the event an \eiw{} the agent concluded \(43 \times 12 = 516\)?%
  \footnote{
    Stated a little awkwardly so as not to assume agency had any role.
  }

  And, in response on may observe the agent has a fair understanding of arithmetic, and they're on point today.
  So, when the agent started to think about \(43 \times 12\) they started to apply their understanding of arithmetic.
  And, they kept going until they figured out \(43 \times 12 = 516\).

  In short, the agent was concluding.
  And, what happened while an event was in progress (partially) explains why an event happened.
\end{note}


% \begin{note}
%   Why did the agent capture the rook?

%   Well, as it's the endgame there are very few moves available to the agent, so they started an exhaustive search for checkmate and found a case by capturing the rook.
%   And, before this action, the agent concluded to take the rook.

%   Alternatively, it may be there is no answer of this kind to why the agent concluded to capture the rook.
%   For, it may be the case the agent was bored and only looked for a piece they could move --- so, if the agent's line of sight had wandered a little differently they would have moved the bishop.
% \end{note}


\begin{note}
  Given \ref{start:A}, when an agent is concluding \(C\) from \(P\) it is the case \(C\) \fof{} \(P\).
  So, \(C\) \fingfr{1} \(P\) may (partially) explain why an agent concluded \(C\) from \(P\).

  Further, given \ref{start:B} there may be other \fingfr{} present when the agent is concluding \(C\) from \(P\).
  So, perhaps other \fingfr{1} may (partially) explain why an agent concluded \(C\) from \(P\)?
  Or, perhaps the following constraint holds:

  \begin{constraint}{consFirst}{A first pass}
    \(C'\) \fingfr{} \(P'\) (partially) explains why an \eiw{0} an agent concluded \(C\) from \(P\) happened \emph{only if} \(C'\) is \(C\) and \(P'\) is \(P\).
  \end{constraint}

  \noindent%
  In other words, no \fingfr{} other than \(C\) \fingf{} \(P\) may (partially) explain why an \eiw{0} an agent concluded \(C\) from \(P\) happened.
\end{note}


\begin{note}
  I think \autoref{consFirst} makes some sense.
  For, in what connexion could \(C'\) \fingfr{} \(P'\) have to an \eiw{0} an agent concluded \(C\) from \(P\), given \(C\) is different from \(C'\) or \(P\) is different from \(P'\)?

  For example, consider \(3 \times 3 = 9\) \fingf{} some \pool{} associated with the agent's understanding of arithmetic.
  The agent concludes \(43 \times 12 = 516\) rather than \(3 \times 3 = 9\), and in principle it seems the agent could conclude \(43 \times 12 = 516\) without ever figuring out \(3 \times 3 = 9\).

  Nothing about what happened seems to rest on \(3 \times 3 = 9\) \fingf{} any \pool{} associated with the agent's understanding of arithmetic.
\end{note}


\begin{note}
  Still, I don't think \autoref{consFirst} makes complete sense.

  For example, an \eiw{0} an agent concluded \(C\) from \(P\) may include a sub-event in which the agent concludes \(C'\) from \(P'\).
  And, in this respect \(C'\) \fingfr{} \(P'\) may (partially) explain why the \eiw{0} an agent concluded \(C\) from \(P\) happened.

  Whether or not you agree, I'd like to suggest a weaker constraint:

  \begin{constraint}{consSecond}{A second pass}
    \(C'\) \fingfr{} \(P'\) (partially) explains why an \eiw{0} an agent concluded \(C\) from \(P\) happened \emph{only if} the agent has concluded or is concluding \(C'\) from \(P'\).
  \end{constraint}

  \noindent%
  \autoref{consSecond} says \(C'\) \fingfr{} \(P'\) (partially) explains why an \eiw{0} an agent concluded \(C\) from \(P\) happened \emph{only if} the agent has obtained or is obtaining \(C'\) from \(P'\).
  In other words, a \fingfr{} requires a accompanying from relation to be (partially) explanatory.
\end{note}


\begin{note}
  \autoref{consFirst} entails \autoref{consSecond}, and \autoref{consSecond} covers sub-conclusions and more.
  \autoref{consSecond} may even be too lax.
  Still, the scope of \autoref{consSecond} has an upshot.
  For, if \autoref{consSecond} fails to hold, then any weaker constraint (such as \autoref{consFirst}) also fails to hold.
\end{note}

\begin{note}
  This document shows \autoref{consSecond} fails to hold.

  Part of this document amounts to making the idea of \fofr{1} precise, and further motivating the idea that \fofr{1} may explain why.
  And, part of this document amounts to showing the precise variant of \autoref{consSecond} fails to hold.
\end{note}

% \begin{note}
%   Document is centred around failure of the constraint.
%   Still, focus of the document is on the argument.

%   What I mean by this is that I don't take the failure of the constraint to be a significant contribution.
%   Instead, the argument.
%   Even if you think something is clear, still need to show it is the case.
%   And, in this case I think the details are worth working through.
%   Indeed, though tailored to a particular result, I have attempted to keep things quite general.
%   Rather than failure reducing to a single idea, failure is due to the way a collection of ideas interact.
%   These ideas may be separated and sent separate ways.
% \end{note}

% \begin{note}
%   \autoref{consFirst} is a first pass.

%   A more careful statement of \autoref{consFirst} both specifies the senses of `why' and `how' present in \autoref{consFirst} and clarifies (at least) sufficient conditions for a conclusion to happen (as used to show the failure of \autoref{consFirst}).
%   In reverse order we touch on a few key details:
%   \begin{itemize}
%   \item
%     Conclusions are more-or-less those identified by natural language.

%     For example, one concludes \(43 \times 12 = 516\) from their understanding of arithmetic, a winning move in a game of chess from their understanding of chess and the state of a game of chess, and what the time is from their impression of a clock and their understanding of the way clocks display the time.%
%     \footnote{
%       Further examples are present throughout the document.
%       Some detailed examples are given in \autoref{cha:intro}.
%       And, for the interested reader \autoref{cha:clar} reduces \eiw{0} an agent concludes to a collection of definitions and assumptions.
%     }
%   \item
%     How \eiw{0} an agent concludes happen is fairly unconstrained.
%     Somehow an agent moves from their understanding of chess and the state of a game of chess to a winning move.
%     And, however this happens answers how.
%     Of importance is only that whatever happened happened.
%   \item
%     (Partial) explanations of why an \eiw[\(e\)]{0} an agent concludes happened (partially) establish why \(e\) was an \eiw{0} the agent concludes, rather than \(e\) being \eiw{0} some other thing happened.

%     For example, why was \(e\) an \eiw{0} the agent concluded \(43 \times 12 = 516\) from their understanding of arithmetic as opposed to an \eiw{0} the agent concluded \(43 \times 12 = 516\) by using a calculator, or an \eiw{} the agent concluded \(43 \times 12 = 506\).%
%     \footnote{
%       Chapters \ref{cha:intro} and \ref{cha:events-progress} significantly expand on the (partial) explanations of why at issue.
%     }
%   \end{itemize}

%   \noindent
%   On some days I find the failure of \autoref{consFirst} surprising, and on other days the failure of \autoref{consFirst} seems obvious.
%   Either way, what follows is a careful argument for the failure of \autoref{consFirst}.
% \end{note}

% \begin{note}
%   Specifically, I establish that so long as various conditions obtain, then \autoref{consFirst} fails.
%   Further, I motivate each of the conditions as either pre-theoretically useful or an instance of a commonly accepted idea.

%   That is, I provide a deductive argument against \autoref{consFirst} with strong supporting (though non-deductive) argumentation for the soundness of the relevant premises.
%   As such, significant work has been put into obtaining an suitable balance between premises which are stated with sufficient precision and contain sufficient information to entail further results while holding sufficiently close to an intuitive understanding of the relevant phenomenon in order to be seen as sound.
% \end{note}







% \begin{note}
%   In form, \autoref{consFirst} is a constraint on (partial) explanations of why some thing came to be in terms of how that thing came to be.
  

  
%   Constraints of this kind  \citeauthor{Davidson:1963aa}.
%   Rationalisations are a relation between a reason and an action such that the reason explains the action by giving the agent's reason for doing the action.
%   And, \citeauthor{Davidson:1963aa} argues rationalisations are causal explanations.

%   So, an action is explained by giving the agent's reason for doing the action only if there is some causal trace from the agent's reason for the action to the action.
% \end{note}


% \begin{note}
%   Failure of constraint is compatible with maintaining the \citeauthor{Davidson:1963aa}ian position of rationalisations being causal explanations.

%   Though I think there are two ways to purse this contrast with \citeauthor{Davidson:1963aa}.

%   The first is to consider differences in the target of the constraint.
%   Rationalisations target an agent's reason for doing some action.
%   By contrast, the constraint we consider targets a premise-conclusion relation.

%   Second is to re-examine rationalisations and similar general constraints.
%   Perhaps way failures are identified extends to or approximates other failures.%
%   \footnote{
%     In the case of \citeauthor{Davidson:1963aa} the reason included in the rationalisation is the \emph{agent's} reason.
%     And, I think the way this attribution is understood may make a difference.
%     Still, this line of inquiry is for an other time.
%   }
% \end{note}





%%% Local Variables:
%%% mode: latex
%%% TeX-master: "master"
%%% TeX-engine: luatex
%%% End:
