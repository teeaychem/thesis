\chapter{Two ways of concluding that \(\phi\) has value \(v\)}
\label{cha:reasoning-two-ways}

\section{Introduction}
\label{sec:reasoning-two-ways:intro}



\begin{note}
  In the introduction, outlined two ways of claiming support, and how they relate to two key ideas.
  Here, develop in more detail.
  In the following chapter, then consider ideas in more detail.
\end{note}


\subsection{Initial \illu{1}}

\begin{note}[What we're going to look at]

  {\color{red} footnote}\nolinebreak
  \footnote{
    Here, as with other examples, focus on existential, as this is relevant.
    However, question about semantic counterpart.
    Every model, or there does not exist a model.
    In contrast to existential, pointing to some specific thing won't do.
    Still, may extend to composite properties of any model.
    Either \(\phi\) or \(\lnot\phi\).
    So, \(\psi\), or trivial.
  }

  \begin{enumerate}[label=\named{\(\exists\mathord{\vdash}{,}\top\)}, ref=\named{\(\exists\mathord{\vdash}{,}\top\)}]
  \item\label{ill:Eproof:def} A syntactic proof of a formula (using a sound first-order system) is sufficient to establish the formula is a (syntactic) theorem of first-order logic.
  \end{enumerate}
\end{note}

\begin{note}[Memory]
  \begin{illustration}\label{ill:ad:proof:mem}
    \mbox{}
    \vspace{-\baselineskip}
    \begin{enumerate}
    \item\label{ill:Eproof:mem} I remember having created a syntactic proof of \formula{\forall x Px \rightarrow \lnot \exists x \lnot P x} (using a sound first-order system).
    \item\label{ill:Eproof:exP} So, there exists a syntactic proof of \formula{\forall x Px \rightarrow \lnot \exists x \lnot P x} (using a sound first-order system)
    \item\label{ill:Eproof:thm} Hence, by \ref{ill:Eproof:def}, \formula{\forall x Px \rightarrow \lnot \exists x \lnot P x} is a theorem of first-order logic.
    \end{enumerate}
    \vspace{-\baselineskip}
  \end{illustration}

  \autoref{ill:ad:proof:mem} seems a straightforward case of claiming support.\nolinebreak
    \footnote{
      Looking ahead, one may concerned that \nI{} rules out the agent claiming support in the way outlined by \ref{ill:ad:proof:mem}.

      For, \ref{nI:background} is be satisfied by taking \(\phi\) to be the proof, and \(\psi\) to be theorem-hood.
      For sure, requires collapsing steps~\ref{ill:ad:proof:eve} and~\ref{ill:ad:proof:eve:app} into a complex conditional so \nI{} does not apply to the specific reasoning.
      Still, at issue is whether the spirit of \nI{} applies, rather than the letter --- reasoning with the derived conditional doesn't seem to change much.

      So, at issue is whether \ref{nI:inclusion:position} and~\ref{nI:inclusion:bound} hold with respect to the reasoning.
      In short, is the agent confident their claimed support for their memory of the proof to be \mom{} if they are not in position to claim support for theorem-hood without appealing to their memory?

      This may be the case.
      One may only trust their memory of proving if they consider themselves to be in position to (re)create the proof, as failure would suggest that failed to create an adequate proof in the past --- e.g.\ the agent may have created a proof as a student and is now (re)considering whether the formula is a theorem as an expert in the field.

      Of course, this may also not be the case, but the worry is that the reasoning seems fine regardless of how additional details are added, so long as the additional details do not conflict with the claimed support.

      So, our attention turns to~\ref{nI:inclusion:bound}.
      In short, is the agent is confident that they would be \nmom{} when claiming support for theorem-hood if their memory is \nmom{}?

      Here, any worry eases.
      Memory of creating the proof seems quite independent of whether the agent would be successful if they were to attempt to (re)create the proof.
      For, the memory is (merely) of the existence of the proof, rather than the details of the syntactic system, and so on.

      In particular, it seems clear that the agent would not need to assume that the formula is a theorem prior to attempting to recreate the proof.
    }

  First, the agent has claimed support for \ref{ill:Eproof:def} by their familiarity with systems of first order logic.

  Second, the agent remembers having created a syntactic proof of the relevant formula.
  And, it seems sufficient, generally speaking, to claim support for some proposition by appealing to memory, hence the agent claims support that there was some event which culminated in a syntactic proof of the formula.\nolinebreak
  \footnote{
    It may be more natural to say `I remember creating\dots' or `I remember proving\dots'.
    The particular phrasing is chosen to remove any ambiguity about whether the agent \emph{finished} the activity creating or proving.
  }
  Of course, the agent may have misremembered, but there seems no issue with the agent expecting that appeal to their memory is \nmom{}.

  Following, this allows the agent to claim support that a syntactic proof of the formula exists.
  As before, the agent may have been \mom{} about whether what they created really was a syntactic proof of the formula
  And, as before it seems the agent may expect that they were not \mom{}.

  Hence, finally, the agent claims support that the formula is a (syntactic) theorem of first-order logic.

  To concisely summarise, we may say that the agent claimed support for the formula being a (syntactic) theorem of first-order logic \emph{because} of their understanding of syntactic theorem-hood and their memory of proving the formula.

  For sure,~\autoref{ill:ad:proof:mem} is designed to be as straightforward as possible.
  Of interest is not whether the agent claims support, but how the role the agent gives to their memory in claiming support.

  The agent appeals to their memory to establish that there exists a syntactic proof of the formula, and then combines the existence of a syntactic proof with~\ref{ill:Eproof:def} to claim support that the formula is a theorem.
  Hence, the agent's memory is directly involved in their claimed support for the formula being a theorem.
\end{note}


\begin{note}
  \begin{illustration}\label{ill:ad:proof:eve}
    \mbox{}
    \vspace{-\baselineskip}
    \begin{enumerate}
    \item I remember having created a syntactic proof of \formula{\forall x Px \rightarrow \lnot \exists x \lnot P x} (using a sound first-order system).
    \item\label{ill:ad:proof:eve:app} In creating the syntactic proof I appealed to various aspects of some sound first-order system.
    \item\label{ill:ad:proof:eve:pos} As I created a proof, those various aspects of the sound first-order system are sufficient to ensure there exists a proof.
    \item Hence, by \ref{ill:Eproof:def}, \formula{\forall x Px \rightarrow \lnot \exists x \lnot P x} is a theorem of first-order logic.
    \end{enumerate}
    \vspace{-\baselineskip}
  \end{illustration}

  As with~\autoref{ill:ad:proof:mem}, the agent's memory has a role in~\autoref{ill:ad:proof:eve}, but the role is quite different.
  Above, the agent claimed support for the formula being a theorem primarily \emph{because} they remembered creating a proof.
  By contrast, here the agent claims support for the formula being a theorem primarily because of the properties of some sound first-order system.

  Step~\ref{ill:ad:proof:eve:app} appeals to various aspects of some sound first-order system and, in turn, step~\ref{ill:ad:proof:eve:app} observes that those aspects are sufficient to ensure a proof exists.
  The agent claims support for the existence of a proof by appeal to the various aspects of some first-order system they appealed to when constructing the proof, rather than their memory of constructing the proof.
\end{note}

\begin{note}
  To help clarify, let's fix a particular syntactic proof using the Fitch-style proof system of~\textcite[557--560]{Barwise:1999tu}:

  \begin{figure}[H]
    \centering
    \begin{quote}
      \fitchprf{}{
        \subproof{\pline[1.]{\forall x P x}}{
          \subproof{\pline[2.]{\exists x \lnot Px}}{
            \boxedsubproof[3.]{a}{\lnot Pa}{
              \pline[4.]{Pa}[\lalle{1}] \\
              \pline[5.]{\bot}[\lfalsei{3}{4}]
            }
            \pline[6.]{\bot}[\lexie{2}{3--5}]
          }
          \pline[7.]{\lnot \exists x \lnot Px}[\lnoti{2--6}]
        }
        \pline[8.]{\forall x Px \rightarrow \lnot \exists x \lnot Px}[\lifi{1--7}]
      }
    \end{quote}
    \caption{A syntactic proof}\label{fig:syntx-prf}
  \end{figure}

  The proof consists of single instances of five introduction or elimination rules.
  Each rule is part of the Fitch-style proof system, and the specific application of the rules constitute the proof.
\end{note}


\begin{note}[Before\dots]
  Before returning to~\autoref{ill:ad:proof:eve}, let us observe that with the proof in hand one may claim support that a proof of the formula exists via the contents of~\autoref{fig:syntx-prf}.

  Broadly stated:

  \begin{enumerate}
  \item The proof is constructed from a sound first-order proof system.
  \item And, the particular application of some rules of the system to formulae is such that the proof begins with no assumptions and the last line of the proof is not part of any assumption made during the course of the proof.
  \end{enumerate}
\end{note}

\begin{note}
  Note, appeal to creation of the proof involves appeal to various aspects of the Fitch-style proof system.

  The object itself is mute to whether or not it is a proof.

  For example, adding `\formula{Ba}' as an assumption would void the proof, but you would need to observe that the appeal to existential elimination on line 6 requires that `\formula{a}' does not appear in the proof prior to its introduction on line 3 in order to claim support that the proof is void.

  Indeed, the proof consists of eight steps, each step is permitted by the first-order system, the proof begins with no assumptions, the last line of the proof is not part of any assumption made during the course of the proof and the proof, and so on.

  Sparing the details, claimed support that~\autoref{fig:syntx-prf} is a syntactic proof of \formula{\forall x Px \rightarrow \lnot \exists x \lnot P x} from the creation of~\autoref{fig:syntx-prf} is a matter of claiming support for each step of the creation.

  Indeed, to spare the details in general, let us instead talk of some collection of propositions and steps of reasoning.
  Claiming support that a proof exists from the some creation in the way under discussion is an instance of reasoning from details of the creation to the conclusion that a proof exists.
  Hence, as an instance of reasoning involves certain premises and steps of reasoning.
  And, whatever these turn out to be, the proceed from the creation of the proof rather than from some other source such as memory, testimony, and so on.
\end{note}

\begin{note}
  In other words, one may claim support that a proof of \formula{\forall x Px \rightarrow \lnot \exists x \lnot P x} exists (primarily) \emph{because} of their reasoning from some collection of premises and steps of reasoning concerning the creation to the existence of a proof of \formula{\forall x Px \rightarrow \lnot \exists x \lnot P x}.
\end{note}

\begin{note}[Return to \ref{ill:ad:proof:eve}]
  Now let us return to the reasoning of~\autoref{ill:ad:proof:eve}, and in particular steps~\ref{ill:ad:proof:eve:app} and~\ref{ill:ad:proof:eve:pos}:
  \begin{quote}
    \begin{enumerate}
      \setcounter{enumi}{1}
    \item In creating the syntactic proof I appealed to various aspects of some sound first-order system.
    \item As I created a proof, those various aspects of the sound first-order system are sufficient to ensure there exists a proof.
    \end{enumerate}
  \end{quote}
  Given that the agent remembers having created a syntactic proof, the `various aspects of some sound first-order system' of step~\ref{ill:ad:proof:eve} may be taken as those aspects of the first-order system that were appealed to in the premises and steps of reasoning when the agent created the proof.
  And step \ref{ill:ad:proof:eve}, in turn, appeals to how those various aspects of some sound first-order system were sufficient for the agent to claim support that a proof exists by the reasoning that occurred.

  In short, the agent remembers creating a syntactic proof and claiming support that a proof exists from the creation.
  The instance of claiming support involved reasoning from premises via steps to the relevant conclusion.
  Hence, it is possible to claim support for the conclusion by those premises and steps of reasoning.
  So, in~\ref{ill:ad:proof:eve} the agent observes that those premises and steps of reasoning are sufficient to claim support by way of their memory, and in turn appeals to those premises and steps of reasoning to claim support for the relevant conclusion.
\end{note}

\begin{note}
  {
    \color{red}
    Propositional support.
    (If I talk about this, it should be after the definitions.)
  }
\end{note}

\begin{note}
  Generalising, the way in which the agent claims support in~\autoref{ill:ad:proof:eve} is of interest because the agent appeals to premises and steps of reasoning that are not `part' of their present reasoning.
  The role of memory in the \illu{0} is (merely) a way for the agent to recognise that there are such premises and steps of reasoning.
  And, in the definitions that follow, we will abstract from any particular way in which the recognises that relevant premises and steps of reasoning are available.

  Still, even though memory is contingent, we may briefly observe that the way in which the agent claim support in~\autoref{ill:ad:proof:eve} is compatible with \ESU{}.
  For, \ESU{} requires that an agent may claim support for some conclusion from premises and steps of reasoning only if the agent has witnessed reasoning to the conclusion from those premises via those steps of reasoning.
  So, if the initial instance of claiming support conformed to \ESU{} then the agent will have witnessed reasoning from those steps and premises to the conclusion --- the instance of claiming support in~\autoref{ill:ad:proof:eve} does not involve such witnessing, but the agent's memory would be about how the relevant premises and steps were used to claim support.

  Of course, the way in which the agent claim support in~\autoref{ill:ad:proof:eve} is incompatible with a strengthened variant of \ESU{} which requires the agent to use any premises and steps they appeal to in the \emph{present} instance of reasoning, but the point for the moment is that the way in which the agent claims support in~\autoref{ill:ad:proof:eve} does not already require what we are arguing against: \ESU{}.
\end{note}

\subsection{Definitions}

\begin{note}
  With a somewhat detailed pair of contrasting \illu{1} in hand, we now turn to fixing a pair of definitions which capture the general way in which the agent claims support in the respective illustrations.

  The two ways will be termed `\adA{}' and `\adB{}', respectively.
\end{note}

\begin{note}
  \defADA*
\end{note}

\begin{note}
  \adA{} does not outline a specific way of reasoning.
  Rather, captures the role of claimed support for \(\phi\) having value \(v\) in some instance of reasoning when the agent claims support for \(\psi\) having value \(v'\).

  Intuitive idea is that claimed support for \(\phi\) having value \(v\)~\ref{def:adA:phi} provides agent with resource to claim support for \(\psi\) having value \(v'\) to~\ref{def:adA:psi}.
\end{note}

\begin{note}
  Applied to the two sketches seen, claim support by existence of proof, or by specific ability to \emph{V} that \(\phi\).
  Key thing is that claimed support for existence of proof or the specific ability to \emph{V} that \(\phi\) rather than something else.

  \phantlabel{abstract-adA}
  Indeed, noting and abstracting from the role of conditionals in these two \illu{1}, basic (abstract) instance of \adA{}:

  {
    \small
    \begin{enumerate}[label=\arabic*., ref=\arabic*]
    \item\label{def:adA:ex:C:Cp} I have concluded that \(\phi\) has value \(v\).
    \item\label{def:adA:ex:C:p} So, \(\phi\) has value \(v\). \hfill(From~\ref{def:adA:ex:C:Cp})
    \item\label{def:adA:ex:C:Cps} Likewise, I have concluded that \(\psi\) has value \(v'\) when \(\phi\) has value \(v\).
    \item\label{def:adA:ex:C:ps} So, \(\psi\) has value \(v'\) when \(\phi\) has value \(v\). \hfill(From~\ref{def:adA:ex:C:Cps})
    \item\label{def:adA:ex:C:T} If \(\psi\) has value \(v'\) when \(\phi\) has value \(v\) and \(\phi\) has value \(v\), then it must be the case that \(\psi\) has value \(v'\). \hfill (From understanding of `if\dots then\dots')
    \item\label{def:adA:ex:C:s} Hence, \(\psi\) has value \(v'\).\newline
      \mbox{}\hfill (From \ref{def:adA:ex:C:p},~\ref{def:adA:ex:C:ps}~and~\ref{def:adA:ex:C:T})
    \item Therefore, I conclude that \(\psi\) has value \(v'\). \hfill (From \ref{def:adA:ex:C:Cp} -- \ref{def:adA:ex:C:s})
    \end{enumerate}
  }
  The reasoning is a verbose because claimed support is not necessarily factive
  \nolinebreak
  \footnote{
    It may be that an agent has claimed support for \(\phi\) having value \(v\) while \(\phi\) has value \(v'\).
  }
  yet the agent has claimed support about \(\phi\) having value \(v\) and how that relates to \(\psi\) having value \(v'\), rather than how claimed support for \(\phi\) having \(v\) relates to claimed support for \(\psi\) having value \(v'\),
  (Consider parallel reasoning with knowledge, rather than (mere) claimed support.\nolinebreak
  \footnote{The parallel reasoning in full:
    \begin{enumerate}[label=\arabic*., ref=\arabic*]
    \item\label{def:adA:ex:K:Kp} I know that \(\phi\) has value \(v\).
    \item\label{def:adA:ex:K:p} So, \(\phi\) has value \(v\). \hfill (From~\ref{def:adA:ex:K:Kp})
    \item\label{def:adA:ex:K:Kps} I know that \(\psi\) has value \(v'\) when \(\phi\) has value \(v\).
    \item\label{def:adA:ex:K:ps} So, \(\psi\) has value \(v'\) when \(\phi\) has value \(v\). \hfill(From~\ref{def:adA:ex:K:Kps})
    \item\label{def:adA:ex:K:T} If \(\psi\) has value \(v'\) when \(\phi\) has value \(v\) and \(\phi\) has value \(v\), then it must be the case that \(\psi\) has value \(v'\). \hfill (From understanding of `if\dots then\dots')
    \item\label{def:adA:ex:K:s} Hence, \(\psi\) has value \(v'\). \hfill (From \ref{def:adA:ex:C:p},~\ref{def:adA:ex:C:ps}~and~\ref{def:adA:ex:C:T})
    \item So, I know that \(\psi\) has value \(v'\) as \(\psi\) having value \(v'\) follows from~(\ref{def:adA:ex:K:Kp}) and~(\ref{def:adA:ex:K:Kps}).
      \mbox{}\hfill (From \ref{def:adA:ex:K:Kp} -- \ref{def:adA:ex:K:s})
    \end{enumerate}
  }%
  )
  Still, the reasoning is a clear instance of claiming support for \(\psi\) having value \(v'\) by \adA{} from \(\phi\) having value \(v\) as the agent claims support for \(\psi\) having value \(v'\) by appealing to their claimed support for \(\phi\) having value \(v\) to satisfy the antecedent of a conditional.

  Still, \adA{} need not involve a conditional.
  Consider, for example, claiming support that they have claimed support for a contradiction from claimed support that \(\phi\) and not-\(\phi\) are both true.
  It seems plausible that so claiming need only involve a reflection on what \(\phi\) and not-\(\phi\) amounts to.
\end{note}

\begin{note}
  \defADB*
\end{note}

\begin{note}
  With \adA{} \(\phi\) having value \(v\).
  \adB{} does not involve the agent claiming support for \(\phi\) having value \(v'\) by \(\phi\) having value \(v\).
  Instead, some (distinct) collection of premises \(\rho_{1},\dots,\rho_{k}\) with respective values and steps \(\delta_{1},\dots,\delta_{m}\).

  The key difference between \adA{} and \adB{}:
  \begin{itemize}
  \item \adA{} involves the agent appealing to \(\phi\) in order to claim support for \(\psi\), while
  \item \adB{} does not involve the agent appealing to \(\psi\) to claim support for \(\psi\).
    Instead, the role of \(\phi\) is to highlight \(\rho_{1},\dots,\rho_{k}\) and the agent appeals to propositions \(\rho_{1},\dots,\rho_{k}\) to claim support for \(\psi\).
  \end{itemize}

  For the definition to be satisfied, \(\phi\) needs only be involved to the extent that it provides the link.
  Hence, \(\phi\) is not irrelevant.
  Still, the agent does not appeal to \(\phi\).
\end{note}

\begin{note}[How appeal?]
  \autoref{def:adB} does not state relation the agent has to premises.
  Issue here is that we only need a way in which \autoref{def:adB}.
  So, no reason to narrow definition.
  And, certainly no point in motivating restriction.

  Assume:
  \begin{itemize}
  \item Agent has already claiming support for \(\rho_{1},\dots,\rho_{n}\).
  \end{itemize}

  This is in the background of the \illu{1}.

  Stronger, not explicitly ruled out:

  \begin{enumerate}
  \item Possibility \(\rho_{1},\dots,\rho_{n}\) secured by \(\phi\), and this is sufficient.
  \end{enumerate}

  This seems unintuitive.
  However, there is no point to ruling this out.
\end{note}

\begin{note}
  Only seen \adB{} with respect to proof \illu{0}.
  Remembered proving \(\phi\), that secures the possibility, but claiming support from the details of the proof itself.

  Intuitively, applies to ability in the same way.
  Premises and steps of reasoning work in the same way as components of a proof.
  However, we will delay details until we've seen a few more \illu{1}.
\end{note}

\begin{note}
  Broad distinction, agent may claim support by appeal to some thing, but it is also possible to break that thing down in to parts or elements such that the agent may claim support by appeal to those parts or elements (and how they compose).

  `Break down' is metaphorical.

  In some cases, the thing itself, in other cases, more basic stuff that must be the case in order for the thing to be the case.

  Break down does the work.
  Agent will typically recognise.
  Break down is not required.

  In this sense, break down is more fundamental.

  `Because\dots'

  Unifying feature is that \adA{} allows claim support for \adB{}, so not clear that need to go via \adB{}.
  Indeed, unclear, given \ESU{}, that may claim support by \adB{}.
  We will only push this question with respect to ability, though.
\end{note}

\begin{note}[\ESU{}]
  We noted above that the reasoning of~\ref{ill:ad:proof:eve} was compatible with \ESU{}.
  The reasoning of~\ref{ill:ad:proof:eve} is an instance of \adB{}.
  Hence, there are instances of \adB{} which are compatible with \ESU{}.

  Argue that there are instances which are not compatible.
\end{note}


\subsection{Additional illustrations}

\begin{note}

  \begin{illustration}
    \mbox{}
    \vspace{-\baselineskip}
    \begin{itemize}
    \item If bag are overweight then they can't be taken on the flight.
    \item Machine reads\dots
    \item Bag can't be taken on the flight.
    \end{itemize}
  \end{illustration}
  Contents of the bag are overweight.

  Combined weight of the items versus the combination of the individual weights.

  Compare, filling the bag and weighing it, versus summing the weight of the items as you fill the bag.

  Now, seems possible to fill the bag and weight it, then appeal to the sum of the items.

  So, this is a little more subtle.
  The bag has been weighed, and the distinction is between the weight of the contents of the bag, and the combined weight of the items that make up the contents of the bag.

  This is particularly interesting.
  Because, it seems clear that something is strange if someone talks about the weight of the contents of the bag without recognising that this is a function of the combined weight of all the individual elements of the bag.
  However, no idea what the contents of the bag are.

  So, claiming support from what is has been observed, the combined weight, rather than what must be the case in order to have made the observation.
\end{note}

\begin{note}[Fire alarm]
  \begin{illustration}
    \mbox{}
    \vspace{-\baselineskip}
    \begin{itemize}
    \item Fire alarm is ringing.
    \item Fire in the building.
    \item Should leave by the nearest exit.
    \end{itemize}
  \end{illustration}
  So, claiming for getting out of the building.
  Fire alarm.
  Or, fire, fire alarm has picked this up.

  So, difference between that there is a fire in the building, and \emph{the} fire in the building.

  The point here is that, okay, you need to go from alarm to fire, that's all fine, but fire itself is sufficient to claim support.
\end{note}

\begin{note}

  \begin{illustration}\label{ill:ad:factorial}
    \mbox{}
    \vspace{-\baselineskip}
    \begin{itemize}
    \item It is possible to write recursive functions in C.
    \item It is possible to write a recursive implementation of the factorial function in C.
    \end{itemize}
  \end{illustration}
  With proofs, abstract objects.

  Consider programming.

  Recursive implementation of factorial in C (chosen to make the implication clear).

  So, \adA{} is just the fact, so to speak.
  But, \adB{} points to the key step of calling function.
  Of course, this is just recursion, but appeal here is to the concept, so to speak, rather than the truth of the statement.

  Don't need to understand details.
  Go by form, so to speak.

  Claiming support by logical relation, rather than the states of affairs that ensure those logical relations hold up.

  Or, the definition is such that\dots
\end{note}

\begin{note}[Existentials]
  In a sense, the point here is that \adA{} cases of interest mean that there's something more.
  This is viewed in terms of some complex of more basic things existing.
  And, \adB{} follows the reference.
\end{note}

\subsection{The distinctions are (sufficiently) exhaustive}
\label{sec:ar-wr-are}

\begin{note}[Style of argument]
  Well, with respect to claiming support for \(\psi\) such that \(\phi\) is involved.

  Either \adA{} or \adB{}.
  \adA{} seems sufficiently clear, so:
  Transform this to: If not \adA{} then \adB{}.
  Equivalent.
\end{note}

\begin{note}[Idea]
  So, \(\phi\) is involved, but isn't \adA{}.
  Hence, agent claims support by something else.
  If \(\phi\) isn't related to that stuff in any way, then completely redundant.
  Note, from agent's perspective, rather than possibility of revising without.
  So, seems it can only be about how those other things relate to the conclusion.

  Okay, so idea is that if no \adA{} then \(\phi\) isn't part of claiming support.
  If other stuff without \(\phi\) then redundant.
  So, if \(\phi\) is involved, about how the other stuff relates.
\end{note}


%%% Local Variables:
%%% mode: latex
%%% TeX-master: "master"
%%% End: