\chapter{\fc{3} and \qWhy{}}
\label{cha:fc-why}

\begin{note}
  We're almost done.

  \qzS{}.
  Answers to \qzS{}.
  \fc{1}.
\end{note}

\begin{note}
  Final part of the argument.
  Answers to \qzS{}, \qWhy{}.

  Corollary, in some cases, \fc{} answers \qWhy{}.
\end{note}

\section{Difficulty}
\label{cha:fc-why:sec:difficulty}

\begin{note}
  However, not quite.

  Our goal is to provide an argument against\dots

  This is still not complete.
  Need to show the relation of support between \(\pv{\psi}{v'}\) and \(\Psi\) answers \qWhy{}.

  To highlight the difficulty.

  Consider knowledge.

  Here, we've argued that \(\pv{\chi}{v''}\) holds from agent's point of view.
  However, what matters, when concluding, is the relation of support between \(\pv{\chi}{v''}\) and \(\pv{\phi}{v}\).

  Whether or not \(\pv{\chi}{v''}\) in part, answers \qWhy{}.
\end{note}

\begin{note}
  Now, this is fairly delicate.

  For this objection, need to deny content also answers why.

  So, here, the question is why the agent concluded.
  This specific act.
  And, I think there's a plausible case to be made that \(\pv{\chi}{v''}\) isn't right, but rather the relation of support between \(\pv{\chi}{v''}\) and other premises, and conclusion.

  To elaborate.
  Conclusion is \(\pv{\phi}{v}\).
  Now, \(\pv{\chi}{v''}\) is in part an answer to why \(\pv{\phi}{v}\) is the case from the agent's perspective.
  However, not, in part, an answer to why agent concluded, from agent's perspective.
  For, concluding, what matters is relation of support between pool of premises containing \(\pv{\chi}{v''}\) and \(\pv{\phi}{v}\).

  Whether or not you agree, this is the worry.

  So, functions as a premise.
\end{note}

\begin{note}
  So, here's the idea.
  In cases of concluding:
  For any proposition-value pair which is answers from agent's perspective, strictly,
  if it relation of support between pool of premises containing and conclusion.
\end{note}

\begin{note}
  Applied to the results so far, relation of support between \fc{} and \(\pv{\phi}{v}\).
\end{note}

\section{Proposition}
\label{cha:fc-why:sec:proposition}

\begin{note}
  Proposition for this chapter.

  \begin{proposition}
    \label{prop:fc-answers-why}
    If \requ{}, then \fc{} (directly) answers \qWhy{}.
  \end{proposition}
\end{note}

\section{Argument}
\label{cha:fc-why:sec:argument}

\begin{note}
  Argument for~\autoref{prop:fc-answers-why} is abstract.

  To clarify, provide a concrete example.
\end{note}

\subsection{Abstract}
\label{cha:fc-why:sec:argument:abstract}

\begin{note}
  \color{red}
  Question is not about whether \(\phi\) has value \(v\).
  Question is about whether the agent would perform the act.

  So, embedding, answers why performed the act.

  Relation of support between pool of premises and conclusion.

  So, in this case, the relation is between the \fc{} and the conclusion.

  However, this can't be right.
  What is at issue is not the conclusion, but the act of concluding.

  This is the basic idea.
\end{note}

\paragraph{\qzS{} is about the act}

\begin{note}
  First observation is that \qzS{} is about the act of concluding.

  So, in order for \fc{} to function as a premise, the relevant conclusion is something to do with the act.
\end{note}

\begin{note}
  At issue with \qzS{} is \emph{not}:

  If agent were to fail to hold that you would conclude \(\pv{\psi}{v'}\) from \(\Psi\) as \emph{a premise}.
\end{note}

\begin{note}
  Highlight this with an example.

  Ability.
  General ability.
  So, get something.
  All of this may be the case, but \requ{} is about concluding.
  Again, observation of a \deadEnd{}.
  So, something which gets potential event doesn't really matter.
\end{note}

\begin{note}
  Note, the tension is not about whether \(\phi\) has value \(v\).
  Instead, the tension is about whether the agent would have a certain property if they were to conclude \(\phi\) has value \(v\).
  Property of having claimed \support{}.
  Expanded, property of holding that any independent check is satisfied.
  Any other reasoning about whether \(\phi\) has value \(v\) would conclude \(\phi\) has value \(v\).
\end{note}

\paragraph{As a premise\dots}

\begin{note}
  In order for the potential event to function as a premise, the relevant conclusion involves concluding.
  So, conclude is that the agent has the option to conclude.

  Tentative way of understanding this.
  \begin{enumerate}
  \item
    \label{tentative:option}
    Option to conclude \(\pv{\phi}{v}\) from \(\Phi^{+}\).
  \end{enumerate}
  Note, \(\Phi^{+}\) to allow for additional premises, such as \ref{tentative:option}.
\end{note}

\begin{note}
  Now, how to get to \(\pv{\phi}{v}\) from this is a further issue.

  Though, I don't think it is implausible.

  For example.

  Witnessed reasoning.
  Pair with conclusion.

  In this case, conclude \(\pv{\phi}{v}\) from \(\Phi^{+}\), where \(\Phi^{+}\) includes the option.

  So, in this respect, intermediate conclusion.
  Concluding \(\pv{\phi}{v}\) from \(\Phi\) in part involves concluding that \qzS{} has a positive answer.
  I mean, further, then, the relation of support is embedded twice.
\end{note}

\begin{note}
  Focus is on~\ref{tentative:option}.
\end{note}

\begin{note}[Splitting up reasoning]
  Now, consider the \requ{}, but here it's whether conclude \(\pv{\phi}{v}\) from \(\Phi\).

  So, then, it's clear that the agent must have witnessed reasoning, I think.

  But, what is the status of this reasoning?

  I mean, here, we've got an act.
  Suitably related to concluding.
  So, \requ{} holds for this.

  {
    \color{red}
    This complicates things too much.
    It's much simpler to observe that the \requ{} applies to the option to conclude in the same way as it applies to the conclusion.

    This is the way in which \requ{1} are sort of recursive.
  }

  (Okay, this is kind of cool.)

  If concluded, and so \requ{} still holds.

  So, weaker.
  Well, weaker, as only require witnessed reasoning.
  However, I think just repeat the observation.
  Look, technically, but hollow.
  If grant \requ{}, then a variant is going to hold, as the witnessed reasoning needs to be good enough to stand in for concluding.
  And, at issue with a \requ{} is whether such reasoning is good enough.
\end{note}

\paragraph{Ah!}

\begin{note}
  So, here we have the problem.
  Still have issue of \requ{}.

  Really, it's about the reasoning.

  If you grant there are cases in which \qzS{} matters, and hold to inclusion, then an additional \requ{}.
\end{note}

\begin{note}
  Viewed all of this from the perspective of positive answers.
  And, as we're looking for a `counterexample', positive answers are of interest.

  Still, negative answers.
  Whether this makes sense.
\end{note}


%%% Local Variables:
%%% mode: latex
%%% TeX-master: "master"
%%% End:
