\chapter{\qWhy{} and \qHow{} an agent concludes}
\label{cha:intro}


\begin{note}
  Our interest is understanding the way an event in which an agent concludes happens.

  \begin{rscenario}{illu:gist:roots:F}{Quadratic roots --- Factors}%
    % \autoref{illu:gist:roots:F}
    An agent reasons as follows:
    %
    \begin{enumerate}[label=\arabic*., ref=\arabic*]
    \item
      \label{illu:gist:roots:F:eq}
      For some \(x \in \mathbb{R}\), \(2x^{2} - x - 1 = 0\).
    \item
      \label{illu:gist:roots:F:factor}
      \((2x + 1)(x - 1) = 0\).
    \item
      \label{illu:gist:roots:F:zero}
      Either \((2x + 1) = 0\) or \((x - 1) = 0\).
    \item
      \label{illu:gist:roots:F:case:a}
      If \((x - 1) = 0\), then \(x = 1\).
    \item
      \label{illu:gist:roots:F:case:b}
      If \((2x + 1) = 0\), then \(x = -\sfrac{1}{2}\).
    \item
      \label{illu:gist:roots:F:factor:done}
      Either \(x = 1\) or \(x = -\sfrac{1}{2}\).
    \end{enumerate}
    %
    The agent concludes:
    \rootsCon{}.
  \end{rscenario}

  \noindent%
  Intuitively, the agent concludes \propI{\rootsCon{}} from their understanding of factorisation.
  And, as the agent concluded via factorisation, the \agents{} understanding of the quadratic formula seems irrelevant understanding to the way in which \autoref{illu:gist:roots:F} happened.

  Likewise, consider the following \scen{0}:

  \begin{scenario}[Quadratic roots --- Formula]%
    \label{illu:gist:roots:QF}%
    An agent reasons as follows:
    %
    \begin{enumerate}[label=\arabic*., ref=\arabic*]
    \item
      \label{illu:gist:roots:QF:eq}
      For some \(x \in \mathbb{R}\), \(2x^{2} - x - 1 = 0\).
    \item
      \label{illu:gist:roots:QF:qf}
      The quadratic formula states \(x = \sfrac{-b \pm \sqrt{b^{2} - 4ac}}{2a}\).
    \item
      \label{illu:gist:roots:QF:subs}
      Let \(a \coloneq 2\), \(b \coloneq -1\), and \(c \coloneq -1\).%
      \footnote{
        \(a\) is the coefficient of the \(x^{2}\) term, \(b\) is the coefficient of the \(x\) term, and \(c\) is the constant.
      }
    \item
      \label{illu:gist:roots:QF:qf-subs}
      \(x = \sfrac{-(-1) \pm \sqrt{(-1)^{2} - 4(2)(-1)}}{2(2)}\).
    \item
      \label{illu:gist:roots:QF:qf:1}
      \(x = \sfrac{(1 \pm 3)}{4}\).
    \item
      \label{illu:gist:roots:QF:qf:done}
      Either \(x = 1\) or \(x = -\sfrac{1}{2}\).
    \end{enumerate}
    %
    The agent concludes:
    \rootsCon{}.
  \end{scenario}

  \noindent%
  Just as the quadratic formula seems irrelevant to understanding \autoref{illu:gist:roots:F}, the \agents{} understanding of factorisation seems irrelevant to understanding \autoref{illu:gist:roots:QF}.
\end{note}



\paragraph*{Abstractions}


\begin{note}
  This document is about understanding the way an event in which an agent concludes happens.
  Specifically, our interest is with where a plausible pre-theoretical constraint holds with respect to any account of \scen{1} such as \autoref{illu:gist:roots:F} and \autoref{illu:gist:roots:QF}.

  We introduce the contain below, but to talk about the constraint is sufficient detail, a few first introduce a few abstractions.
  These abstractions are introduced with respect to \scen{1} \ref{illu:gist:roots:F} and \ref{illu:gist:roots:QF} and are then applied to an additional \scen{} before moving on.
\end{note}


\begin{note}
  The conclusion of \scen{1}~\ref{illu:gist:roots:F}~and~\ref{illu:gist:roots:QF} --- \propM{\rootsCon{}} --- corresponds to a way things may be.
  If the truths of mathematics are necessary, then things are always this way.
  Even so, \propM{\rootsCon{}} captures a way things may be in the same way as \propI{Foxes change the colour of their fur to reflect their mood} captures a way things may be.
  For ease we refer to ways things may be as \emph{\prop{1}}.

  Continuing the abstraction, we say an \agents{} concludes a \prop{0} has a \emph{\val{0}}.
  Where, a \val{0} captures the \agpe{} on the way things may be.%
  \footnote{
    If it helps, think of a \prop{0}-\val{0}-\pool{0} pair as a propositional attitude, where a \prop{0} corresponds to a proposition and a \val{0} to an attitude.
    Though, things get a little clumsy if expressed in terms of propositional attitudes.
    Additional details are given in \autoref{cha:clar}.
  }
  For example, in \scen{1}~\ref{illu:gist:roots:F}~and~\ref{illu:gist:roots:QF} the agent concludes \propM{\rootsCon{}} has value \valI{True}.%
  \footnote{
    Our interest is primarily with the \val{} \valI{True}.
    However, we place no restrictions on possible \val{1}.
    Hence, an agent may conclude \propI{Foxes change the colour of their fur to reflect their mood} has \val{} \valI{Desirable}.
    I.e., the agent concludes it is desirable that Foxes change the colour of their fur to reflect their mood.
  }
  Though, the agent may have concluded \propM{\rootsCon{}} has \val{} \valI{False} or \propM{\rootsConBad{}} has \val{} \valI{True}, etc.

  At some point in a conclusion, a \prop{} comes to have some \val{}, such as Step~\ref{illu:gist:roots:F:factor:done} in \autoref{illu:gist:roots:F}.
  Here we say the agent \emph{\evals{}} the \prop{} to have the \val{}.
\end{note}


\begin{note}
  Further, an agent concludes \emph{from} some thing.
  In \autoref{illu:gist:roots:F} the agent concluded from their understanding of factorisation, and in \autoref{illu:gist:roots:QF} the agent concluded from their understanding of the quadratic formula.

  The \agents{} understanding of factorisation or the quadratic formula is the way some things are from the \agpe{}.
  Hence, in general we understand an agent to conclude some \prop{0}-\val{0} pair from some \emph{\pool{}} of \prop{}-\val{} pairs --- `\pool{}' has a few less characters than `collection' and sounds nicer.

  What is included in a \pool{} is difficult to say in detail, but broad characterisations may be given.
  For example, with \autoref{illu:gist:roots:F}, the \pool{} captures the \agents{} understanding of factorisation but does not capture the \agents{} understanding of the quadratic formula.
  And, with \autoref{illu:gist:roots:QF}, the \pool{} captures the \agents{} understanding of the quadratic formula but does not capture the \agents{} understanding of factorisation.
\end{note}


\begin{note}
  Two connexions between \prop{1}, \val{1}, and \pool{1} are of particular interest given an event in which agent concludes:
  %
  \begin{enumerate}
  \item
    The \agents{} conclusion of the \prop{0}-\val{0} pair \emph{from} the \pool{0}.
  \item
    The \prop{0} having the \val{0} \emph{following from} the \pool{0}, from the \agpe{}.
  \end{enumerate}
  %
  `From' is about what happened independent of the \agpe{}.
  `Follows from' is about connections between various \prop{1}-\val{1} dependent on the \agpe{}.

  For example, in \autoref{illu:gist:roots:F} the agent concludes \propM{\rootsCon{}} has value \valI{True} from some \pool{} which captures their understanding of factorisation.
  Hence, the agent concluded \propM{\rootsCon{}} has value \valI{True} from the \pool{} and \propM{\rootsCon{}} having value \valI{True} follows from the \pool{} from the \agpe{}.
\end{note}


\begin{note}
  A \emph{\ros{0}} abstractly captures when a \prop{0}-\val{0} pair `follows from' some \pool{}.
  Hence, a \ros{} between some \prop{0}, \val{0} and \pool{0} holds when an agent \eval{} the \prop{} as having the \val{0} from the \pool{0}.
  So, in \autoref{illu:gist:roots:F} a \ros{} holds between \propM{\rootsCon{}}, \valI{True} and the relevant \pool{} when the agent \evals{} \propM{\rootsCon{}} as having \val{0} \valI{True}.

  In this respect, \ros{0} help capture where an agent is \emph{concluding}.
  For example, consider \autoref{illu:gist:roots:F}.
  It seems clear at the second step of reasoning the agent is concluding \propM{\rootsCon{}} has \val{} \valI{True}.
  Something ensures the agent is concluding.
  And, we use a \ros{} between \propM{\rootsCon{}}, \valI{True} and a \pool{} which captures the \agents{} understanding of factorisation to captures whatever it is that ensures the agent is concluding.

  Alternatively one may say that agent has the ability, the know-how, skill, disposition, etc.\ to conclude \propM{\rootsCon{}} has value \valI{True} from the relevant \pool{} and is executing the know-how or manifesting the disposition at the second step of reasoning.
  Still, \ros{1} are designed to abstract from such specific details.

  Further, a \ros{0} may hold between some \prop{0}, \val{0}, and \pool{} regardless of whether the \prop{0}-\val{0} pair has been concluded `from' the \pool{}.
  In this respect, a \ros{} between \propM{\rootsCon{}}, \valI{True} and a \pool{} which captures the \agents{} understanding of \emph{the quadratic equation} may hold at the second step of reasoning in \autoref{illu:gist:roots:F}.
  For, intuitively, the agent may abandon factorisation and do the reasoning outlined in \autoref{illu:gist:roots:QF}.
  Though, it need not be the case that, given \(y = \sqrt{ 2 + \sqrt{2 + \cdots}}\), a \ros{} holds between \propM{y = 2}, \valI{True}, and the \agents{} understanding of factorisation (even though it is clear \(y^{2} - y - 2 = 0\) given a little insight).%
  \footnote{
    Specifically, we assume a \ros{} between some \prop{0}, \val{0}, and \pool{0} holds when there is some action may immediately do such that the agent is concluding the \prop{0}-\val{0} pair from the \pool{} when the agent does the action.
  }
\end{note}


\begin{note}
  In short, a \ros{} is an abstraction which relates a \prop{0}, \val{0}, and \pool{0} and captures that the \prop{} having the \val{} `follows from' the \pool{} from the \agpe{}.
  Where, in particular, a \ros{} may hold prior to a conclusion and is such that whatever secures the applicability of the abstraction also ensures the agent may the option to be, or is, concluding the \prop{0} has value \val{0} from the \pool{0}.

  This is all quite abstract.
  However, the abstraction has a purpose.
  Our interest is understanding the way an event in which an agent concludes happens, though we do not have interest in any particular theory about the way an event in which an agent concludes happens.
  Rather, our interest is with a pre-theoretical constraint on such understanding.
  And, abstracting to \prop{1}, \val{1}, \pool{1}, and \ros{1} allows a theory neutral account of some key aspects of conclusions.

  Given this is an introduction, the characterisation of the abstractions is brief.
  Other chapters in this document expand on these abstractions in detail.%
  \footnote{
    \autoref{cha:clar} focuses on \prop{1}, \val{1} and \pool{1}.
    And, \autoref{cha:ros} focues on \ros{1}.
  }
\end{note}



\paragraph*{An additional \scen{0}}


\begin{note}
  \begin{rscenario}{scen:countS}{Countersign}%
    \indent The captain mumbled, ``I come from Miran.''

    The man returned the gambit, grimly.
    ``Miran is early this year.''

    The captain said, ``No earlier than last year.''

    But the man did not step aside.
    He said, ``Who are you?''

    ``Aren't you Fox?''

    ``Do you always answer by asking?''

    The captain took an imperceptibly longer breath, and then said calmly,
    ``I am Han Pritcher, Captain of the Fleet, and member of the Democratic Underground Party.
    Will you let me in?''%
    \mbox{ }\hfill\mbox{(\cite[70]{Asimov:1945aa})}%
    \newline
  \end{rscenario}

  \noindent%
  Applying \prop{1}, \val{1}, and \pool{1}, we say that in the event described by \autoref{scen:countS} Fox concludes \propI{\signConA{}} has value \valI{True} from some \pool{} \(\Phi\).
  And, when Fox \evals{} \propI{\signConA{}} as having \valI{True}, a \ros{} holds between \propI{\signConA{}}, \valI{True}, and \(\Phi\).%
  \footnote{
    Pritcher tells Fox the are a Party member at the end of the \scen{}.
    Still, Fox has already drawn the conclusion when Fox asks `Who are you?' via the sequence of countersigning.
  }

  Further, after Pritcher's opening gambit `I come from Miran', Fox (sub-)concludes \propI{What Pritcher said was a sign} has \val{0} \valI{True} and some \pool{} \(\Psi\).
  And, of interest is with the \ros{} between \propI{What Pritcher said was a sign}, \valI{True} and some \pool{} \(\Psi\).

  In contrast to \scen{1} \ref{illu:gist:roots:F} and \ref{illu:gist:roots:QF}, the highlighted \ros{} is distinct from the \ros{} which characterises Fox's final conclusion.
  The \ros{} did not hold prior to Pritcher saying `I come from Miran'.
  And, the \ros{} does not ensure Fox is concluding \propI{\signConA{}} has value \valI{True} from \(\Phi\).

  Still, the \ros{} is tightly connected to \emph{whether} Fox is concluding \propI{\signConA{}} has value \valI{True} from \(\Phi\) by the second conditional.
  For, the following conditional plausibly holds, from \agpe{Fox's}:
  %
  \begin{itenum}
  \item[\emph{If}:]
    They didn't open with `I come from Miran'.
  \item[\emph{Then}:]
    They're not engaging in countersign.
  \end{itenum}
  %
  And, if Fox doesn't think Pritcher is engaging in countersign, Fox does not conclude Pritcher a fellow Party member by countersign.
  Hence, phrased in terms of \ros{1}, we have:
  \begin{itenum}
  \item[\emph{If}:]
    A \ros{} between \propI{What Pritcher said was a sign}, \valI{True} and some \pool{} \(\Psi\) fails to hold.
  \item[\emph{Then}:]
    Fox does not conclude \propI{\signConA{}} has value \valI{True} from some \pool{} \(\Phi\).
  \end{itenum}
  %
  Hence, the same abstractions applied to \scen{1}~\ref{illu:gist:roots:F}~and~\ref{illu:gist:roots:QF} apply to \autoref{scen:countS} and may be used to capture parts of the way \autoref{scen:countS} happened.
\end{note}



\section*{\qWhy{}, \qHow{} and \issueInclusion{}}
\label{cha:intro:why-how}


\begin{note}
  \phantlabel{how-and-why-first-mention}%
  We distinguish two questions about the way an event in which an agent concludes happens.
  `\qWhy{}' and `\qHow{}':

  \begin{question}{questionWhy}{\qWhy{}}
    Given \(e\) is an event in which \vAgent{} concludes \(\phi\) has \val{0} \(v\) from \(\Phi\):
    \begin{itemize}
    \item
      Which \ros{1} explain \emph{why} \(e\) is an event in which \vAgent{} concludes \(\phi\) has \val{0} \(v\) from \(\Phi\) (rather than some other event)?
    \end{itemize}
    \vspace{-1.5\baselineskip}
  \end{question}

  \begin{question}{questionHow}{\qHow{}}
    \label{q:how}
    Given \(e\) is an event in which \vAgent{} concludes \(\phi\) has \val{0} \(v\) from \(\Phi\):
    \begin{itemize}
    \item
      Which past or present events explain \emph{how} \(e\) is an event in which \vAgent{} concludes \(\phi\) has \val{0} \(v\) (rather than some other event)?
    \end{itemize}
    \vspace{-1.5\baselineskip}
  \end{question}
\end{note}


\begin{note}
  \qWhy{} fixes an event, and queries why the event is an event in which an agent concludes \(\phi\) has value \(v\) from \(\Phi\) rather than an event in which something incompatible with the conclusion \(\phi\) has \val{0} \(v\) for \(\Phi\) happens.
  For example, an event in which the agent concludes \(\pv{\phi'}{v}\) from \(\Phi\), \(\pv{\phi}{v}\) from \(\Phi'\), takes their dog for a walk, runs a bath, etc.

  \qHow{}, by contrast, queries how the event is an event in which an agent concludes \(\phi\) has \val{0} \(v\) from \(\Phi\) came about.
\end{note}

\begin{note}
  We briefly discuss the sense of `why' at issue in \qWhy{}.
  Then, we turn to the primary focus of this document:
  A constraint on answers to \qWhy{} by answers to \qHow{}.
\end{note}


\paragraph*{The sense of `why' at issue in \qWhy{}}


\begin{note}
  The sense of `why' at issue in \qWhy{} is roughly%
  \footnote{
    Some subtle details are expanded on in \autoref{cha:events-progress}.
  }
  captured by the following idea:

  \begin{idea}[`Why']%
    \label{idea:why}%
    For a given event \(e\), some thing \(t\) answers why \(e\) happened just in case:
  %
    \begin{enumerate}[label=\Alph*., ref=\Alph*]
    \item
      \label{idea:why:result}
      There is some event \(e'\) such that \(e\) is a result of \(e'\), and:
    \item
      \label{idea:why:favour}
      \(e'\) favours \(e\).
    \item
      \label{idea:why:feat}
      \(e'\) favours \(e\) happening only if \(t\) is the case during \(e'\).
    \end{enumerate}
    \vspace{-\baselineskip}
  \end{idea}

  \noindent%
  In short, some thing \(t\) answers `why' an event \(e\) happened when \(t\) being the case follows from \(e\) being favoured over some other event.
\end{note}


\begin{note}
  We briefly motivate \autoref{idea:why} with two examples independent of conclusions.
  We then return to \scen{1}~\ref{illu:gist:roots:F},~\ref{illu:gist:roots:QF}, and \ref{scen:countS}.
\end{note}


\begin{note}
  Consider an event \(e\) in which a deck of cards is shuffled, and \mainCard{} is \mainCardPos{}.
  Our interest is with why \mainCard{} is \mainCardPos{}.

  As the deck of cards was shuffled, there is a collection of events in which the cards are being shuffled.
  Some event \(e'\) satisfies Clause~\ref{idea:why:result} of \autoref{idea:why}.
  For, \mainCard{} was shuffled to \mainCardPosX{}, hence there is some event in which \mainCard{} being shuffled to \mainCardPos{}, a results in \(e\).

  Clause~\ref{idea:why:favour} is not immediately satisfied --- setting causal determinism aside.%
  \footnote{
    \nocite{Hoefer:2023aa}%
    Indeed, if causal determinism holds --- any event is the result of antecedent events together with the laws of nature --- then there is always some answer to `why' the event happened.
    For, the antecedent events and laws of nature ensure no other event may have happened.
  }
  For, during a random shuffle \mainCard{} being in \mainCardPos{} was not favoured over \mainCard{} being in any other position of the deck.
  For example, if the cards are randomly shuffled, then, intuitively, nothing favours \mainCard{} being in \mainCardPos{} after the shuffle.
  Hence, though Clause~\ref{idea:why:result} of \autoref{idea:why} is satisfied, Clause~\ref{idea:why:favour} is not satisfied.

  Still, if the shuffler is engaging in sleight of hand clauses \ref{idea:why:result}, \ref{idea:why:favour}, and \ref{idea:why:feat} are satisfied.

  The shuffle happens as a result of the agent shuffling.
  As the agent is engaging in sleight of hand, \mainCard{} being in \mainCardPos{} is favoured over \mainCard{} being in any other position.
  And, only if sleight of hand.%
  \footnote{
    One may, of course, expand on the details of the sleight --- such as what a break and shuffle is (\cite[cf.][189--190]{Hilliard:1994aa}) or the shufflers' specific method for a break and shuffle.
  }
\end{note}


\begin{note}
  The idea of a shuffler is engaging in sleight of hand may suggest determinism.
  However, some thing may answer `why' an event happened without the thing determining that the event happens.
  For example, consider rolling a biased die.%
  \footnote{
    It is hard to bias a coin flip. (\cite{Gelman:2002ww})
  }
  As biased die need not guarantee a roll which lands on 6.
  However, if an agent lands a 6 with a die biased toward 6 then the bias of the die favours an event in which any agent rolls a 6.
  Hence, so long as happens as a result of bias, then clauses \ref{idea:why:result}, \ref{idea:why:favour}, and \ref{idea:why:feat} of \autoref{idea:why} are satisfied by an event \(e'\) in which the die is rolling and the bias influences the roll.
  In particular:
  Clause~\ref{idea:why:result} is satisfied as the die landing on 6 is the result of the die rolling.
  Clause~\ref{idea:why:favour} is satisfied as the die is biased toward landing on 6, by assumption.
  And, Clause~\ref{idea:why:feat} is satisfied as the die is biased toward landing on 6 only if the die has bias.

  Clause~\ref{idea:why:feat} is important.
  For, consider the same die dropped from a short height to ensure it lands 6 without rolling.
  Clauses~\ref{idea:why:result} and~\ref{idea:why:favour} are seen to be satisfied by parallel reasoning.
  However, a die without bias also lands 6 if dropped from a short height to ensure it lands 6.
  Hence, the bias of the die does not explain why the die dropped from a short height to ensure it lands 6 without rolling lands 6, given~\autoref{idea:why}.

  (Though details of the drop do explain why the die lands 6.)
\end{note}


\begin{note}
  Now, \qWhy{}, specifically, seeks answers to why \(e\) is an event in which an agent concludes \(\phi\) has \val{0} \(v\) from \(\Phi\) in terms of \ros{1}.
  Hence, given \autoref{idea:why}, our interest is with whether there is some event \(e'\) such that:
  \begin{enumerate*}[label=]
  \item \(e\) is the result of \(e'\),
  \item \(e'\) favours \(e\), and
  \item \(e'\) favours \(e\) happening only if a \ros{} holds during \(e'\).
  \end{enumerate*}
\end{note}


\begin{note}
  Consider an event corresponding to Step~\ref{illu:gist:roots:F:factor} in \autoref{illu:gist:roots:F}.
  Clause~\ref{idea:why:result} is satisfied as the eventual conclusion is a result of figuring out \((2x + 1)(x - 1) = 0\).

  Clause~\ref{idea:why:favour} is satisfied as the agent figuring out \((2x + 1)(x - 1) = 0\) favours an event in which the agent concludes \propI{\rootsCon{}} has \val{0} \valI{True} from the \pool{} over an event in which the agent fails to conclude \propI{\rootsCon{}} has \val{0} \valI{True} from the \pool{}.
  In short, an event in which the agent concludes \propI{\rootsCon{}} has \val{0} \valI{True} when the agent figures out \((2x + 1)(x - 1) = 0\) is in progress.

  Finally, Clause~\ref{idea:why:feat} is satisfied as the \ros{} between \propI{\rootsCon{}}, \valI{True} and the \pool{} captures \emph{that} the agent is concluding \propM{\rootsCon{}}, \valI{True} from the relevant \pool{}.
  Indeed, \ros{1} were introduced above to abstractly capture whatever makes it the case that the agent is concluding \propM{\rootsCon{}}, \valI{True} from the relevant \pool{}.
  So, the \ros{} captures what it is for the agent to be concluding \propI{\rootsCon{}} has \val{0} \valI{True} from the \pool{}.
\end{note}


\begin{note}
  In short, the relevant \ros{0} answers \qWhy{} as the \ros{} captures what it is for the conclusion to in progress, and hence be favoured over some other event.

  Parallel reasoning applies to the other (non-final) steps of \scen{1}~\ref{illu:gist:roots:F} and \ref{illu:gist:roots:QF}.
\end{note}


\begin{note}
  Our discussion of \autoref{scen:countS} focused on a \ros{} between \propI{What Pritcher said was a sign}, \valI{True} and some \pool{} \(\Psi\).
  In contrast to the \ros{1} from \scen{1}~\ref{illu:gist:roots:F} and \ref{illu:gist:roots:QF}, this \ros{} does not hold prior to Fox concluding \propI{What Pritcher said was a sign} has value \valI{True} from some \pool{} \(\Phi\).
  So, the reasoning applied to Step~\ref{illu:gist:roots:F:factor} in \autoref{illu:gist:roots:F} does not apply to the between \propI{What Pritcher said was a sign}, \valI{True} and \(\Psi\).

  Still, Fox concludes \propI{\signConA{}} has value \valI{True} from \(\Phi\) as a result of countersign.
  Hence, we observed \emph{if} a \ros{} between \propI{What Pritcher said was a sign}, \valI{True} and some \pool{} \(\Psi\) failed to hold \emph{then} Fox does not conclude \propI{\signConA{}} has value \valI{True} from \(\Phi\).
  For, the relevant \ros{} captures Fox observing the initial step of the countersign.

  In short, given \(e'\) is the event in which Fox concludes \propI{What Pritcher said was a sign} has \val{0} \valI{True} from \(\Psi\) and \(e\) is the event in which Fox concludes \propI{\signConA{}} has value \valI{True} from \(\Phi\), and the \ros{} between \propI{What Pritcher said was a sign}, \valI{True} and \(\Psi\):

  Clause~\ref{idea:why:result} is satisfied as Fox concludes \propI{\signConA{}} has value \valI{True} from \(\Phi\) by countersign as a result of Fox observing Pritcher's sign.

  Clause~\ref{idea:why:favour} is satisfied as Fox recognising Pritcher's sign leads Fox to continue the countersign.

  And, Clause~\ref{idea:why:feat} is satisfied as the \ros{} between \propI{What Pritcher said was a sign}, \valI{True} and \(\Psi\) captures Fox's observation that Pritcher is engaging in countersign.

\end{note}




\paragraph*{A constraint on answers to \qWhy{} in terms of answers to \qHow{}}


\begin{note}
  Consider the following constraint on answers to \qWhy{} in terms of answers to \qHow{}:

  \begin{constraint}{consInclusion}{\issueInclusion{}}
    \mbox{ }
    \vspace{-\baselineskip}
    \begin{itenum}
    \item[\emph{If}:]
      A \ros{} between \(\psi\), \(v'\), and \(\Psi\) answers \qWhy{}.
    \item[\emph{Then}:]
      An event in which the agent concludes \(\psi\) has value \(v'\) from \(\Psi\) answers \qHow{}.
    \end{itenum}
    \vspace{-\baselineskip}
  \end{constraint}

  \noindent%
  In short, \issueInclusion{} states that for any \ros{} between a \prop{0}, \val{0}, and \pool{} which grants some understanding of the way an agent concluded \(\phi\) has value \(v\) from \(\Phi\), there is an event such that the agent concludes the \prop{} has the \val{} from the \pool{} past or present to the \agents{} conclusion of \(\phi\) has value \(v\) from \(\Phi\).%
  \footnote{
    Note, \qHow{} does not explicitly require the relevant event to be the event in which the agent concludes \(\phi\) has value \(v\) from \(\Phi\).
    Hence, a previous event in which the agent concludes \(\psi\) has value \(v\) from \(\Psi\) may answer \qHow{}.
    \issueInclusion{} is a constraint, and hence is compatible with a more stringent constraint.
    Still, we have no interest in motivating a more stringent constraint.
  }
\end{note}


\begin{note}
  I think \issueInclusion{} is intuitive.
\end{note}


\begin{note}
  For broad motivation which focuses on the sense of `why' and `how', consider sleight of hand as discussed above.
  The sleight of hand intuitively explains why \mainCard{} is \mainCardPos{}, but the sleight of hand also explains \emph{how} \mainCard{} is \mainCardPos{} after the shuffle.
  For, the shuffler had to engage in sleight of hand in order for \mainCard{} to be in \mainCardPos{} over any other position.
\end{note}


\begin{note}
  For narrow motivation, we observe \issueInclusion{} is compatible with the \ros{1} highlighted with respect to \scen{1}~\ref{illu:gist:roots:F},~\ref{illu:gist:roots:QF}, and \ref{scen:countS}.

  With \scen{1}~\ref{illu:gist:roots:F}~and~\ref{illu:gist:roots:QF} the \ros{} of interest holds between \propM{\rootsCon{}}, \valI{True} and some \pool{} which captures the \agents{} understanding of factorisation or the quadratic formula.
  And, the agent concludes \propM{\rootsCon{}} has value \valI{True} from a \pool{} which captures the \agents{} understanding of factorisation or the quadratic formula.

  And, with \autoref{scen:countS} the \ros{} of interest holds between \propI{What Pritcher said was a sign}, \valI{True} and some \pool{} \(\Psi\).
  Fox has concluded \propI{What Pritcher said was a sign} has value \valI{True} from \(\Psi\) when Fox concludes \propI{\signConA{}} has value \valI{True} from some \pool{} \(\Phi\).
\end{note}


\begin{note}
  To get a feel for what \issueInclusion{} rules out, consider \autoref{illu:gist:roots:F} and a \ros{} between \propM{\rootsCon{}}, \valI{True} and some \pool{} which captures the \agents{} understanding of the quadratic formula.
  We suggested it may be the case a \ros{} between \propM{\rootsCon{}}, \valI{True} and some \pool{} which captures the \agents{} understanding of the quadratic formula answers `why' the agent concluded \propM{\rootsCon{}} has value \valI{True} from their understanding of factorisation.
  \issueInclusion{} entails a \ros{} between \propM{\rootsCon{}}, \valI{True} and some \pool{} which captures the \agents{} understanding of the quadratic formula \emph{does not} answer `why' the agent concluded \propM{\rootsCon{}} has value \valI{True} from their understanding of factorisation.

  And, this seems intuitive.
  For, it seems implausible that the \agents{} understanding of the quadratic formula favoured an event in which the agent concluded \propM{\rootsCon{}} has value \valI{True} from their understanding of factorisation over any other event, as the agent did not conclude by the quadratic formula.

  
\end{note}


\begin{note}
  I think \issueInclusion{} is intuitive.
  Still, we have only considered a few \scen{1} and \issueInclusion{} is a general constraint.
  Hence, if there is some doubt regarding \issueInclusion{} then further argument is required.
  However, I am not aware of any arguments for \issueInclusion{}.%
  \footnote{
    The construction of \qWhy{}, \qHow{}, and \issueInclusion{} is somewhat idiosyncratic, and so it is no surprise there are no direct argument for \issueInclusion{}.
    However, I take the idea captured by \issueInclusion{} to be intuitive, and I am not aware of any argument for this idea either.
  }
  And, no argument for \issueInclusion{} will not be given.
\end{note}



\section*{Counterexamples to \issueInclusion{}}


\begin{note}
  The goal is to provide resources to identify counterexamples to \issueInclusion{}, given a general understanding of the way an event in which an agent concludes happens.
  For ease of reference, we term the basic resources and the way they combine a `recipe' for counterexamples to \issueInclusion{}.
  Specifically, the recipe identifies \scen{1} with the following features:
  %
  \begin{enumerate}
  \item
    An event in which an agent concludes \(\phi\) has value \(v\) from \(\Phi\).
  \item
    A \ros{} between \(\psi\), \(v'\), and \(\Psi\) such that clauses \ref{reciF:qWhy}, \ref{reciF:distinct}, and \ref{reciF:nHow} hold:
    %
    \begin{enumerate}[label=\Alph*., ref=\Alph*]
    \item
      \label{reciF:qWhy}
      The \ros{} between \(\psi\), \(v'\), and \(\Psi\) answers \qWhy{}.
    \item
      \label{reciF:distinct}
      Either \(\psi\) is distinct from \(\phi\), \(v'\) is distinct from \(v\), or \(\Psi\) is distinct from \(\Phi\).
    \item
      \label{reciF:nHow}
      There is no prior (or present) event in which the agent concludes \(\psi\) has value \(v'\) from \(\Psi\).
    \end{enumerate}
  \end{enumerate}
  %
    The importance of clauses \ref{reciF:qWhy}, \ref{reciF:distinct}, and \ref{reciF:nHow} follow from the following observation:

    \begin{observation}[Failures to \issueInclusion{}]%
    \label{obs:iIceRestriction}%
    \vspace{-\baselineskip}
    \begin{itenum}
    \item[\emph{If}:]
      A \ros{} between \(\psi\), \(v'\), and \(\Psi\) is a for sure a counterexample to \issueInclusion{}.
    \item[\emph{Then}:]
      Conditions \ref{obs:iIceRestriction:why}, \ref{obs:iIceRestriction:distinct}, and \ref{obs:iIceRestriction:prior} are true of the \ros{} between \(\psi\), \(v'\), and \(\Psi\):
      \begin{enumerate}[label=\arabic*., ref=\arabic*]
      \item
        \label{obs:iIceRestriction:why}
        The \ros{} between \(\psi\), \(v'\), and \(\Psi\) answers \qWhy{}.
      \item
        \label{obs:iIceRestriction:distinct}
        Either \(\psi\) is distinct from \(\phi\), \(v'\) is distinct from \(v\), or \(\Psi\) is distinct from \(\Phi\).
      \item
        \label{obs:iIceRestriction:prior}
        There is no prior (or present) event in which the agent concludes \(\psi\) has value \(v'\) from \(\Psi\).
      \end{enumerate}
    \end{itenum}
    \vspace{-1.5\baselineskip}
  \end{observation}
  %
  \begin{motivation}{obs:iIceRestriction}
    Suppose a \ros{} between \(\psi\), \(v'\), and \(\Psi\) is a counterexample to \issueInclusion{}.
    We observe each condition holds.

    \begin{enumerate}
    \item
      \issueInclusion{} constrains answers to \qWhy{} in terms of answers to \qHow{}.
    So, for the \ros{} between \(\psi\), \(v'\), and \(\Psi\) to be a counterexample to \issueInclusion{}, the \ros{} between \(\psi\), \(v'\), and \(\Psi\) must answer \qWhy{}.
  \item
    \qWhy{} applies to events in which an agent concludes \(\phi\) has value \(v\) from \(\Phi\).

    Now, if \(\psi\) is the same as \(\phi\), \(v'\) is the same as \(v\), and \(\Psi\) is the same as \(\Phi\) then the event in which the agent concludes \(\phi\) has value \(v\) from \(\Phi\) is an event in which the agent concludes \(\psi\) has value \(v'\) from \(\Psi\).
    And, the event in which the agent concludes \(\phi\) has value \(v\) from \(\Phi\) answers \qHow{}.
    Hence, an event in which the agent concludes \(\psi\) has value \(v'\) from \(\Psi\) answers \qHow{}.
  \item
    Finally, \qHow{} is compatible with \scen{1} where some prior event in which the agent concludes \(\psi\) has value \(v'\) from \(\Psi\) answers \qHow{}.
    Whether or not there are \scen{1} with this feature is something we set aside.
    Still, a counterexample to \issueInclusion{} is not guarantee if there is some prior event in which the agent concludes \(\psi\) has value \(v'\) from \(\Psi\).
  \end{enumerate}
  \vspace{-\baselineskip}
  \end{motivation}
\end{note}


\begin{note}
  Still, the recipe is about showing \issueInclusion{} fails to hold \emph{given} appropriate \scen{1}, rather than constructing \scen{1}.
  So, strictly speaking, the thesis is:
  %
  \begin{quote}
    \emph{If} such-and-such features hold of a \scen{} \emph{then} \issueInclusion{} fails to hold.
  \end{quote}
  %
  In other words, the goal of this document is to show that any \scen{0} which satisfies a collection of features entails \issueInclusion{} does not hold.
  This means assumptions, definitions, and propositions.

  Of some interest, then, is whether \issueInclusion{} actually fails --- whether there really are events where a \ros{} between \(\psi\), \(v'\), and \(\Psi\) answers \qWhy{} without an event in which the agent concludes \(\psi\) has value \(v'\) from \(\Psi\).

  Various \scen{1} which the recipe plausibly applies to will be given, are designed to be the sort of \scen{1} that plausibly happen to anyone multiple times on any given day.%
  \footnote{
    Though a complete argument that such \scen{1} exist is beyond the scope of this thesis.
    For example, a corollary of such an argument would proof of an external world.

    And, it may be the case that the \scen{1} identified by the recipe are understood to exist, and \issueInclusion{} fails to hold given this understanding, though it turns out that what's really going on in each \scen{0} identified is compatible with \issueInclusion{} ---- scepticism is quite powerful.
  }
  And, I take this to be mostly on par with showing \issueInclusion{} fails to hold.
\end{note}


\begin{note}
  The broad idea of the recipe is as follows:

  A \ros{} answers \qWhy{} when the \ros{} explains why an event in which an agent concludes \(\phi\) has \val{0} \(v\) from \(\Phi\) happens, rather than some other event.
  Given \autoref{idea:why}, a \ros{} answers \qWhy{} just in case the \ros{} holding is a key part of some event which both:
  \begin{itemize}
  \item
    Favours the \agents{} conclusion of \(\phi\) having \val{0} \(v\) from \(\Phi\).
  \item
    Results in the \agents{} conclusion of \(\phi\) having \val{0} \(v\) from \(\Phi\).
  \end{itemize}
\end{note}


\begin{note}
  Consider an event in which an agent concludes some \prop{0} \(\phi\) has \val{} \(v\) from some \pool{} \(\Phi\).
  Further, suppose there is an event in which the agent is \emph{concluding} \(\phi\) has \val{} \(v\) from \(\Phi\).
  And, suppose the agent concludes \(\phi\) has \val{} \(v\) from \(\Phi\) as a result of whatever happened when the agent was concluding \(\phi\) has \val{} \(v\) from \(\Phi\).

  As motivated above, a \ros{} between \(\phi\), \(v\), and \(\Phi\) helps capture \emph{that} the agent is concluding \(\phi\) has \val{0} \(v\) from \(\Phi\), and in turns answers \qWhy{}.
  %
  \begin{itemize}
  \item
    For example, the event captured by Step~\ref{illu:gist:roots:F:factor} in \autoref{illu:gist:roots:F}.
    When the agent figures out \((2x + 1)(x - 1) = 0\), the agent is concluding \propI{\rootsCon{}} has \val{0} \valI{True} from their understanding of factorisation.
    And, the eventual conclusion is a result of figuring out \((2x + 1)(x - 1) = 0\).
    So, a \ros{} between \propI{\rootsCon{}}, \valI{True}, and a \pool{} which captures the \agents{} understanding of factorisation answers \qWhy{}, as the \agents{} eventual conclusion is favoured over any event in which the agent does not conclude \propI{\rootsCon{}} has \val{0} \valI{True} from the relevant \pool{}.
  \end{itemize}
  %
  Still, it may be the case the agent is concluding \(\phi\) has \val{0} \(v\) from \(\Phi\) \emph{only if} a \ros{} between \(\psi\), \(v'\), and \(\Psi\) (also) holds.
  Hence, if the \ros{} between \(\psi\), \(v'\), and \(\Psi\) fails to hold, the agent is not concluding \(\phi\) has \val{0} \(v\) from \(\Phi\).
  Yet, as the agent \emph{is} concluding \(\phi\) has \val{0} \(v\) from \(\Phi\), the \ros{} between \(\psi\), \(v'\), and \(\Psi\) favours an event in which the agent concludes \(\phi\) has \val{0} \(v\) from \(\Phi\) over some other event.
  %
  \begin{itemize}
  \item
    For example, consider Step~\ref{illu:gist:roots:F:factor} again.
    I doubt possible the agent is concluding \propI{\rootsCon{}} has \val{0} \valI{True} from their understanding of factorisation \emph{and} a \ros{} fails to hold between \propI{\rootsSimp{}}, \val{0} \valI{True} from the \agents{} understanding of factorisation.
    For, if \propI{\rootsSimp{}} having \val{0} \valI{True} plausibly follows from the \agents{} understanding of factorisation.
  \end{itemize}
  %
  In other words, a \ros{} between \(\psi\), \(v'\), and \(\Psi\) holding is part of what it is for the agent to be concluding \(\phi\) has \val{0} \(v\) from \(\Phi\).
  And, as the event in which the agent is concluding \(\phi\) has \val{0} \(v\) from \(\Phi\) favours a conclusion of \(\phi\) having \val{0} \(v\) from \(\Phi\), and the conclusion is the result of the concluding, the \ros{} between \(\psi\), \(v'\), and \(\Psi\) answers \qWhy{}.
  The only remaining task is to ensure the agent has not concluded \(\psi\) has \val{0} \(v'\) from \(\Psi\), but instances are not too difficult to find.
  %
  \begin{itemize}
  \item
    For example, the agent may have yet concluded \propI{\rootsSimp{}} has \val{0} \valI{True} from their understanding of factorisation.
    And, if the agent has, consider \propI{\rootsSimpX{}}, \propI{\rootsSimpY{}}, and so on\dots
  \end{itemize}
  %
  This, is more-or-less it.

  Sometimes an agent concludes as a result of concluding in a particular way (e.g.\ by factorisation), and if the agent is concluding in a particular way, then various distinct \ros{1} holding are part of what it is to conclude in that particular way, and hence these answer \qWhy{}.
  Put otherwise, sometimes an agent concludes as a result of concluding in a particular way and the agent is concluding in a particular way \emph{only if} various distinct \ros{1} holding are part of what it is to conclude in that particular way.
  Hence, given \autoref{idea:why}, those \ros{} answer \qWhy{}.

  I think \issueInclusion{} is intuitive, though I also think this argument is basic enough to highlight that the intuition isn't quite right.

  Most of the work to be done is setting things up so that a precise argument is given which does not lead to the various objects that may be raised against this sketch.
\end{note}


\begin{note}
  Note, \ros{3} are an abstraction.
  And, with minor adjustments the above argument may be made in terms of ability, know-how, disposition etc.
  These concepts all involve something which extends beyond any instance.
  For example, an ability to use the quadratic formula extends beyond a single use of the formula, the know-how to countersign extends beyond a particular sequences of signing, and the disposition to factor involves factoring under a variety of stimulus conditions.
  We work with \ros{1} as details about ability, know-how, or dispositions are largely irrelevant for the argument proper.
\end{note}

\section*{Conclusion}
\label{sec:conclusion}

\begin{note}
  Our interest is understanding the way an event in which an agent concludes happens.
  Considered a few \scen{1}.
  Abstractions.
  Two questions.
  Plausible constraint.
  Motivated constraint by intuition and analogy.
  Sketched argument to follow against constraint.
\end{note}


\newpage

\paragraph*{Structure}


\begin{note}
  Before constructing counterexamples to \issueInclusion{} we detail key ideas introduced in this introduction.

  The details are important.
  For, without a sufficiently detailed account of the phenomena of interest, any recipe proposed with fail to identify \scen{1}.
  And, without a detailed account of \qWhy{} and \qHow{} it is unclear whether any identified \scen{1} is a counterexample to \issueInclusion{}.

  Start with events.
  Clarify \qWhy{}.

  Following, conclusions.
  Little of interest here.

  Key idea with respect to \scen{1}~\ref{illu:gist:roots:F}~and~\ref{illu:gist:roots:QF} is \ros{} prior to conclusion.
  To capture instances, \fc{1}.

  With \fc{1}, \ros{1}.

  With \ros{1}, return to \qWhy{}, \qHow{}, and \issueInclusion{}.

  Counterexamples.
  \requ{2}.
  \fc{3}, \ros{1}, fails to favour without \fc{1}.

  To motivate the existence of \requ{1}, \tC{}.

  Link these together.

  This is it.

  Counter-samples.

  Nothing assumes counterexamples exist.
  Instead, when these combine to form a \scen{}.

  Close with some additional stuff.
\end{note}



%%% Local Variables:
%%% mode: latex
%%% TeX-master: "master"
%%% TeX-engine: luatex
%%% End:
