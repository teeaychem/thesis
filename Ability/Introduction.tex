\chapter{Introduction}
\label{cha:introduction}

\begin{note}
  \begin{itemize}
  \item
    Initial \scen{0}, intuitions, questions, relation between answers, motivation for positive answer.
  \end{itemize}
\end{note}

\section{Concluding: Why? and How?}
\label{sec:overview:issue}

\subsection{Initial \scen{0}}

\begin{note}
  \begin{scenario}[Multiplication]
    \label{illu:gist:calc}
    An agent enters `\(23 \times 15\)' into a calculator and presses the button marked `\(=\)'.
    The calculator displays `\(345\)'.

    \mbox{ }

    The agent observes they have the option to calculate \(23 \times 15\) via their understanding of arithmetic.
    And, \emph{if} the calculator is trustworthy, then they would not fail to conclude \(23 \times 15 = 345\) via their understanding of arithmetic.

    \mbox{ }

    The agent concludes \(23 \times 15 = 345\).
  \end{scenario}

  Intuitively, the agent concluded \(23 \times 15 = 345\) from the calculator.
  Personifying a little, we may say the agent has concluded \(23 \times 15 = 345\) from the testimony of the calculator.%
  \footnote{
    \autoref{illu:gist:calc} may be recast in terms of an agent providing the solution.
  }

  Still, as noted by the agent, they had the option of concluding \(23 \times 15 = 345\) through their own understanding of arithmetic, and hence without the use of the calculator.

  Still, it is intuitive that an agent concludes some proposition has some value%
  \footnote{In \scen{0} the proposition is `\(23 \times 15 = 345\)' and the value is `true'.
    In isolation, `\(23 \times 15 = 345\)' describes some possible state of affairs, and assigning the value `true' indicates the possible state of affairs is the actual state of affairs.
    Still, the agent may have concluded `\(23 \times 15 = 345\)' is `desired', `impossible', `probable', and so on.
    When speaking generally, we will keep explicit which value a proposition is paired with, though when describing specific \scen{0} or examples, we will leave the associated value implicit.
  }
  from some pool of premises only if the agent reasoned from those premises to the proposition-value pair.

  In other words, the agent may have concluded \(23 \times 15 = 345\) via their understanding of arithmetic, but as the agent did not calculate \(23 \times 15\) themselves, they did not conclude \(23 \times 15 = 345\) from whatever premises would be involved when reasoning via their understanding of arithmetic.

  Indeed, given the agent's understanding of arithmetic, it seems clear that prior to using the calculator the agent knew \emph{whether} \(23 \times 15 = 345\).
  Though, knowing whether \(23 \times 15 = 345\) is not knowing \(23 \times 15 = 345\).
  For example, I expect --- though one of the following equalities does not hold --- you know whether \(345 \times 11 = 3,795\), whether \(3,795 \div 5 = 760\), and whether \(760 \div 8 = 95\).

  Rephrasing things a little, and keeping track of truth, we may say \(23 \times 15 = 345\) was a \fc{} for the agent in \autoref{illu:gist:calc}.
  And, for you, \(345 \times 11 = 3,795\), \(3,795 \div 5 \ne 760\), and \(760 \div 8 = 95\) are \fc{1}.

  Instead, it seems the agent concluded \(23 \times 15 = 345\) by use of the calculator, regardless of whether \(23 \times 15 = 345\) was a \fc{1}.
  And, likewise, if you have concluded \(3,795 \div 5 \ne 760\) from the paragraph above, it was due to my testimony, and not your understanding of arithmetic.
\end{note}

\begin{note}[Summary of basic intuitions]
  So, intuitively, in \autoref{illu:gist:calc} the agent concluded \(23 \times 15 = 345\) from the testimony of the calculator.
  And, intuitively, \(23 \times 15 = 345\) being a \fc{0} had no significant role in the agent concluding \(23 \times 15 = 345\) from the testimony of the calculator.
\end{note}

\subsection{Two questions: Why? and How?}

\begin{note}[Not just concluding]
  So far we have spoken about intuitions of the basic form:

  \begin{itemize}
  \item
    Agent \(A\) concluded some proposition \(\phi\) has some value \(v\) from some pool of premises \(\Phi\).
  \end{itemize}

  I take intuitions of this basic form to be readily available in a variety of \scen{1}, and I also take those intuitions expressed in regards to \autoref{illu:gist:calc} to be fairly clear.

  \phantlabel{how-and-why-first-mention}
  Still, concluding is something an agent does, and in this respect there are (at least) two distinct questions intuitions of the above form may answer:%

  \begin{restatable}[\qWhy{}]{question}{questionWhyBasic}
    \label{q:why}
    Which proposition-value-premises pairings an involved in explaining \emph{why} the agent concluded \(\phi\) has value \(v\)?
  \end{restatable}

  \begin{restatable}[\qHow{}]{question}{questionHowBasic}
    \label{q:how}
    Which proposition-value-premises pairings an involved in explaining \emph{how} the agent concluded \(\phi\) has value \(v\)?
  \end{restatable}

  In basic form, focus is on what the agent did, or alternatively whether some action way performed.
  Did the agent conclude \(\phi\) has value \(v\)?
  Or, did the agent conclude \(\phi\) has value \(v\) from the pool of premises \(\Phi\)?
  Or, perhaps, did the agent fail to conclude \(\phi\) has value \(v\)?

  `How?' and `why?' by contrast, take for granted the agent concluded \(\phi\) has value \(v\) and consider, respectively, how and why this action was performed.

  How do intuitions regarding whether or not an agent performed an action relation to how and why the agent performed the action?
\end{note}

\begin{note}
  \color{red}
  Designed to allow intermediate conclusions.
\end{note}

\begin{note}
  \qWhy{} and \qHow{} are distinct questions.
  Particular instances of broader `why?' and `how?'.
  And, generally speaking, `why?' and `how?' may have distinct answers.%
  \footnote{
    From the agent's perspective.
    Avoid explanations which are distinct from the agent's point of view.

    For example, it may be that what `really' explains why the agent arrived at the post-office was to take their dog for a walk.
    {
      \color{red}
      Here, there's an example from Russell\dots
    }
    Likewise, may have actually been a slight jog.
  }

  For example, consider asking why and how some agent arrived at some location.
  `By walking answers how the agent arrived at the post office, but does not answer why the agent arrived at the post office.
  Likewise, `to post a letter' intuitively answers why the agent arrived at the post office, but does not answer how the agent arrived at the post office.%
  \footnote{
    Of course, things get complex.
    Action, motivating reason, belief-desire pair.
    Hence, desire to post a letter is part of how.

    Preface with `intuitively'.
    The point is that may refine intuition regarding the agent concluding \(23 \times 15 = 345\).
  }

  So, the intuitions expressed with respect to \autoref{illu:gist:calc} may answer `how?' but not `why?', or `why?' but not `how?'.

  Still, I suspect this is not the case.
  In the case of \autoref{illu:gist:calc} both have the same rough answer:

  \begin{itemize}
  \item
    The pairing of testimony of the calculator with \(23 \times 15 = 345\), is, in part, \emph{both} how \emph{and} why the agent concluded \(23 \times 15 = 345\).
  \end{itemize}
  That premises associated with the agent's understanding of arithmetic do not answer either `how?' or `why?' is implicit by omission.

  Again, the agent used the testimony of the calculator to conclude \(23 \times 15 = 345\), and the agent appealed to the testimony of the calculator to conclude \(23 \times 15 = 345\).
  The agent did not use their understanding of arithmetic to conclude \(23 \times 15 = 345\), and the agent did not appeal to their understanding of arithmetic to conclude \(23 \times 15 = 345\).
\end{note}

\begin{note}
  \color{red}
  Pairing works in two different ways.
  For why, it's the pairing by the agent, from the agent's perspective.
  Perspective is included in pairing.
  For how, it's the pairing by the action of the agent.
  So, in this sense, the agent, rather than the agent's perspective, is key.

  Note, this help things be compatible with \citeauthor{Davidson:1963aa}.
\end{note}

\subsection{Relationship between \qWhy{} and \qHow{}}

\begin{note}
  Our observation that the testimony of the calculator seems to answer both `why?' and `how?' the agent concluded \(23 \times 15 = 345\) suggests, even if --- as a single \scen{0} --- only slightly, the following basic idea:%
  \footnote{
    The observation also suggests the converse holds --- an answer, in part, to `how?' is also, in part, an answer to `why?' --- though I think the converse faces some immediate difficulties.
    For, it seems answers to `how?' may include details that are irrelevant to `why?'.
    For example, typing digits and operators into the calculator answers, in part, how the agent concluded \(23 \times 15 = 345\) but these actions seems irrelevant to why the agent concluded \(23 \times 15 = 345\).
    Rather, an answer to `why?' seems to be limited to the calculator providing testimony that \(23 \times 15 = 345\), regardless of whether it was the agent who used the calculator, or whether the agent observed someone else using the calculator.
  }

  \phantlabel{how-and-why-relation-first-mention}
  \begin{restatable}[\issueInclusion{}]{issue}{issueInclusionFirst}
    \label{issue:why-inc-in-how}
    Some proposition-value-premises pairing is, in part, an answer to \qWhy{} only if that proposition-value-premises pairing is (also), in part, an answer to \qHow{}.
  \end{restatable}

  In other words, in order for premise to, in part, answer \emph{why} an agent concluded \(\phi\) has value \(v\), that premise must also, in part, answer \emph{how} the agent concluded \(\phi\) has value \(v\).

  With respect to \autoref{illu:gist:calc}, the testimony of the calculator satisfies the constraint imposed by the idea, while the agent's understanding of arithmetic does not.
  Specifically, the testimony of the calculator was part of how the agent concluded \(23 \times 15 = 345\), and so the testimony of the calculator may be, in part, an answer to why the agent concluded \(23 \times 15 = 345\).
  However, the agent's understanding of arithmetic was \emph{not} part of how the agent concluded \(23 \times 15 = 345\), and so the agent's understanding of arithmetic \emph{may not}, in part, an answer to why the agent concluded \(23 \times 15 = 345\).

  In addition, the basic idea may be taken to capture some explanatory significance and we may even say:
  The agent's understanding of arithmetic is not, in part, an answer to why the agent concluded \(23 \times 15 = 345\) \emph{because} the agent's understanding of arithmetic was \emph{not} part of how the agent concluded \(23 \times 15 = 345\).
\end{note}

\subsection{Motivation from \citeauthor{Davidson:1963aa}}

\begin{note}
  The basic idea, rather than intuitions regarding specific \scen{1} is our interest.
  Roughly, at least.%
  \footnote{
    We will shortly refine our understanding of `why?' and `how?' to focus on support between premises and conclusions, and will motivate a slightly weaker idea with respect to support.
  }

  Additional \scen{1} may provide additional motivation for the basic idea.
  Though, I think the basic idea is sufficiently intuitive independently of individual \scen{1}.
  Instead, observe the basic idea may be motivated not only by \scen{1}, but also by theories.
  Perhaps the most prominent is \citeauthor{Davidson:1963aa}' causal theory of action.

  \citeauthor{Davidson:1963aa} opens \citetitle{Davidson:1963aa} with the following question:

  \begin{quote}
    What is the relation between a reason and an action when the reason explains the action by giving the agent's reason for doing what he did?
    We may call such explanations \emph{rationalizations}, and say that the reason \emph{rationalizes} the action.%
    \mbox{}\hfill\mbox{(\citeyear[685]{Davidson:1963aa})}
  \end{quote}

  As noted, concluding is an action, and hence \qWhy{} is a closely related to \citeauthor{Davidson:1963aa}' question.

  Though, only closely.
  Reasons.%
  \footnote{
    With a little more care, we make further distinguish between normative and explanatory reasons.

    Normative: `reasons which show a given action, attitude, activity or outcome good, right, appropriate or called for'
    Explanatory: `the reasons why things happen, or why things are the way they are'
    \citeyear[410]{Hieronymi:2011aa}

    \citeauthor{Hieronymi:2011aa} also distinguishes `motivating' reasons,

    Motivating: `psychological facts which explain action'
    (\citeyear[411--412]{Hieronymi:2011aa})

    This way of dividing reasons is difficult.
    \citeauthor{Hieronymi:2011aa}'s account of motivating reasons follows~\textcite{Smith:1994wo}, but~\citeauthor{Smith:1994wo} distinguishes motivating reasons from normative reasons.

    \begin{quote}
      The distinctive feature of a motivating reason to \(\phi\) is that, in virtue of having such a reason, an agent is in a state that is \emph{explanatory} of her \(\phi\)-ing, at least other things being equal --- other things must be equal because an agent may have a motivating reason to \(\phi\) without that reason's being overriding.%
      \mbox{}\hfill\mbox{(\citeyear{Smith:1994wo})}
    \end{quote}

    See also \citeauthor{Broome:2013aa}:
    \begin{quote}
      Sometimes the explanation of why a person does something has a particular character:
      roughly, it involves the person's rationality in a distinctive way that I shall not try to describe.
      Then we say the person does what she does for a reason.
      We might say ‘The reason for which Hannibal used elephants was to terrorize the Romans'.
      The reason for which a person does something is called a ‘motivating reason'.
      In general, a motivating reason is whatever explains or helps to explain what a person does in the distinctive way that involves her rationality.

      \mbox{}\hfill\(\vdots\)\hfill\mbox{}

      Whereas motivating reasons explain or help to explain why a person does something, normative reasons explain or help to explain why a person ought to do something, or to believe something, or to hope for something, or to like something, or in general to F, where ‘F' stands for a verb phrase.%
      \mbox{}\hfill\mbox{(\citeyear[46--47]{Broome:2013aa})}
    \end{quote}

    In this way of looking at things, relation of support concerns explanatory reasons.

    \textcite{Hieronymi:2011aa} for a general overview.
  }
  \citeauthor{Davidson:1963aa} distinguishes between a reason and the agent's reason.
  And, the question \citeauthor{Davidson:1963aa} asks concerns reasons, rather than the agent's reasons.%
  \footnote{
    For \citeauthor{Davidson:1963aa}, primary reason.

    \begin{quote}
      \emph{R} is a primary reason why an agent performed the action \emph{A} under the description \emph{d} only if \emph{R} consists of a pro attitude of the agent toward actions with a certain property, and a belief of the agent that \emph{A}, under the description \emph{d}, has that property.\newline
      \mbox{ }\hfill\mbox{(\citeyear[687]{Davidson:1963aa})}
    \end{quote}
    Now, primary reason is not from the agent's perspective.
    Pro attitude, we state something about the agent's perspective.
  }

  By contrast, \qWhy{} is from the agent's perspective.
  If expressed in terms of reasons, \qWhy{} asks:

  \begin{quote}
    \color{red}
    \qWhy{} in terms of reasons.
  \end{quote}

  Still, answers to \qWhy{} then get a reason.%
  \footnote{
    Strictly, the variant.
  }

  \citeauthor{Davidson:1963aa} argues, in short, for the following answer to the relation between a reason and the rationalisation of an action:

  \begin{quote}
    \begin{enumerate}[label=\arabic*]
      [R]ationalization is a species of ordinary causal explanation.\newline
      \mbox{ }\hfill\mbox{(\citeyear[685]{Davidson:1963aa})}
    \end{enumerate}
  \end{quote}

  Still, primary reason for every step of reasoning.
  Here, we get a causal trace.
  No need to look for any relation of support other than premises of reasoning.

  So, \citeauthor{Davidson:1963aa} moves to reasons, and to action.
  Plausible that the relation between reason and action, is mirrored, from perspective of agent.%
  \footnote{
    This is delicate.
  On my understanding of \citeauthor{Davidson:1963aa}, there's a tight link between then content of some state and the causal relations that arise from the state.

  So, go from content to state, and then proceed from here.

  This, I think, is correct.
  And, the problem of deviant causal chains highlights this.
  For, \citeauthor{Davidson:1963aa} recognises there's a problem with the link between content and the causal relations which hold between the states.

  \begin{quote}
    Beliefs and desires that would rationalize an action if they caused it in the right way—through a course of practical reasoning, as we might try saying---may cause it in other ways.%
    \mbox{ }\hfill\mbox{(\citeyear[79]{Davidson:1973vd})}
  \end{quote}
  }

  In other words, there's no significant shift between first and third person with respect to relations between.

  This doesn't follow immediately, but natural.
  In part, charm of the proposal.
  And, primary reason by specifying the contents of the mental state.

  But, resist this.
  From agent's perspective, more than relation, but in terms of reasons, remains causal relation between mental states.
  Translation gets more difficult.%
  \footnote{
    I'm somewhat inclined to think this is the case.
    It's all causal, in some respect, but the change in perspective makes things very difficult.
  }

  Implicit in this quick argument is the idea that a causal explanation answers `how?'.
  Note, however, that we did not appeal to the converse.

  Causal theories of action seem to motivate the basic idea, though I do not think the basic idea (directly, at least) motivates causal theories of action (or, specifically, concluding).
  In other words, for our purposes, answers to `how?' need not be causal explanations, though they may be.

  More broadly, I take the basic idea to capture a pre-theoretical constraint on classes of theories.
  There are theories that agree with the basic idea, such as \citeauthor{Davidson:1963aa}' causal theory of action (when the action is concluding) and there \emph{may be} theories which do not agree with the basic idea --- though I do not know of any specific theories that are explicitly of this kind.
\end{note}

\subsection{Questioning the intuitive relationship}

\begin{note}
  The basic idea is more-or-less the basic issue of this document.

  Both intuitions, such as those regarding \autoref{illu:gist:calc}, and theories, such as \citeauthor{Davidson:1963aa} causal theory of action, provide motivation for the basic issue.

  Our goal is to motivate the following, basic, contrary idea:

  \begin{itemize}
  \item
    There are cases in which something is, in part, an answer to `why?' and that something is \emph{not} (also), in part, an answer to `how?'.
  \end{itemize}

  The basic contrary idea is the negation of the basic idea.
  For, the basic idea states, roughly, answers to `why?' are always included in answers to `how?' while the basic contrary idea states that there are cases in which something that answers `why?' does not also answer `how?'.

  The basic contrary idea, then, has the form of an existential.
  We will not motivate the idea that there is always something which answers `why?' but does not also answer `how?'.
  And, in particular, it may be the case that the intuitions observed with respect to \autoref{illu:gist:calc} are correct.
\end{note}

\begin{note}
  Now, all this has been said without giving attention to the conditional observed by the agent in \autoref{illu:gist:calc}:

  \begin{itemize}
  \item
    If the calculator is trustworthy, then the agent would not fail to conclude \(23 \times 15 = 345\) via their understanding of arithmetic.
  \end{itemize}

  Both natural, and somewhat surprising.

  Consider the contraposition.

  \begin{itemize}
  \item
    If the agent were to fail to conclude \(23 \times 15 = 345\) via their understanding of arithmetic, then the calculator is not trustworthy.
  \end{itemize}

  \begin{itemize}
  \item
    Is it possible, from the agent's perspective, fail to conclude \(23 \times 15 = 345\) via their understanding of arithmetic?
  \end{itemize}

  If possible, then difficulty.
  For, testimony, so must be, but at the same time, possible that it is not.
  If not possible, then it doesn't seem that observing \(23 \times 15 = 345\) via the testimony of the calculator is sufficient.
  For, by the previous observation, difficulty with the testimony of the calculator.

  \begin{itemize}
  \item
    Testimony of the calculator \emph{only if} \(23 \times 15 = 345\) is a \fc{0} given the agent's understanding of arithmetic.
  \end{itemize}
\end{note}

\begin{note}
  Important:

  Multiple ways to conclude.
  So, have a check.

  Differs to, for example, concluding {\color{red} ???} from a scientific calculator.
  {\color{red} ???} goes beyond typical understanding of arithmetic.
  Parallel pair of conditionals does not hold.

  Or, alternatively, testimony that {\color{red} ???}.
  Beyond understanding.
\end{note}

\begin{note}
  I am not sure what to make of \ref{illu:gist:calc}.
  Understanding of arithmetic is a partial check.
  However, testimony.

  Unsure because status of a premise.

  Basic contrary idea only requires some instances.

  Argument against this intuition.
  Type of cases, premises are fixed.
  Check on own reasoning.

  First, expand on intuition.
  Then, introduce type of \scen{0} of interest.
\end{note}

\subsection{Key things going forward}

\begin{note}
  Three things of interest.

  \begin{enumerate}
  \item
    \scen{1} like \autoref{illu:gist:calc} in which an agent concludes some proposition has some value.
  \item
    Relation between Why and how and agent concludes.
    Basic idea, and basic contrary idea.
  \item
    Idea of a \fc{} and how \fc{1} may be involved in concluding.
  \end{enumerate}
\end{note}


\begin{note}
  Follow part, foundations.
  Following, turn to details.

{\color{red} Place somewhere?}.%
  \footnote{
    Roughly, if it were the agent failed to conclude \(23 \times 15 = 345\) in \autoref{illu:gist:calc}, then there would be conflict between the agent's understanding of arithmetic and the testimony of the calculator.
    Expressed differently, there would be conflict between the agent's failure to conclude \(23 \times 15 = 345\) by their understanding of arithmetic, and a premises involved in concluding \(23 \times 15 = 345\) via the calculator.
    I.e. supposing the agent concludes \(23 \times 15 \ne 345\), then from the agent's perspective the calculator is not a source of testimony.
    In the \scen{1} of interest, this hypothetical --- or in some cases possible --- conflict will strictly be between the agent's reasoning from pools of premises to conclusions.
  }
\end{note}



%%% Local Variables:
%%% mode: latex
%%% TeX-master: "master"
%%% End: