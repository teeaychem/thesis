\chapter{\qWhy{} and \qHow{} an agent concludes}
\label{cha:intro}


\begin{note}
  Our interest is understanding the way an event in which an agent concludes happens.

  \begin{scenario}[Quadratic roots --- Factors]%
    \label{illu:gist:roots:F}%
    An agent reasons as follows:
    %
    \begin{enumerate}[label=\arabic*., ref=(\arabic*)]
    \item
      \label{illu:gist:roots:F:eq}
      For some \(x \in \mathbb{R}\), \(2x^{2} - x - 1 = 0\).
    \item
      \label{illu:gist:roots:F:factor}
      \((2x + 1)(x - 1) = 0\).% \hfill \ref{illu:gist:roots:F:eq}, Factoring
    \item
      \label{illu:gist:roots:F:zero}
      Either \((2x + 1) = 0\) or \((x - 1) = 0\).% \hfill \ref{illu:gist:roots:F:factor}, Arithmetic
    \item
      \label{illu:gist:roots:F:case:a}
      If \((x - 1) = 0\), then \(x = 1\).% \hfill \ref{illu:gist:roots:F:factor}, \ref{illu:gist:roots:F:zero}, Arithmetic
    \item
      \label{illu:gist:roots:F:case:b}
      If \((2x + 1) = 0\), then \(x = -\sfrac{1}{2}\).% \hfill \ref{illu:gist:roots:F:factor}, \ref{illu:gist:roots:F:zero}, Arithmetic
    \item
      \label{illu:gist:roots:F:factor:done}
      Either \(x = 1\) or \(x = -\sfrac{1}{2}\).% \hfill \ref{illu:gist:roots:F:zero}, \ref{illu:gist:roots:F:case:a}, \ref{illu:gist:roots:F:case:b}, Replacement
    \end{enumerate}
    %
    The agent concludes:
    \rootsCon{}.%
    \footnote{
      A little more precisely \rootsConFull{}.
    }
  \end{scenario}

  \noindent%
  Intuitively, the agent concludes \propI{\rootsCon{}} from their understanding of factorisation.
  In particular, the \agents{} understanding of the quadratic formula, it seems, is irrelevant to the \agents{} reasoning.
  For example, consider the following \scen{0}:

  \begin{scenario}[Quadratic roots --- Formula]%
    \label{illu:gist:roots:QF}%
    An agent reasons as follows:
    %
    \begin{enumerate}[label=\arabic*., ref=(\arabic*)]
    \item
      \label{illu:gist:roots:QF:eq}
      For some \(x \in \mathbb{R}\), \(2x^{2} - x - 1 = 0\).
    \item
      \label{illu:gist:roots:QF:qf}
      The quadratic formula states \(x = \sfrac{-b \pm \sqrt{b^{2} - 4ac}}{2a}\).% \hfill Memory
    \item
      \label{illu:gist:roots:QF:subs}
      Let \(a \coloneq 2\), \(b \coloneq -1\), and \(c \coloneq -1\).% \hfill \ref{illu:gist:roots:QF:eq}, How to use the quadratic formula%
      \footnote{
        \(a\) is the coefficient of the \(x^{2}\) term, \(b\) is the coefficient of the x term, and \(c\) is the constant.
      }
    \item
      \label{illu:gist:roots:QF:qf-subs}
      \(x = \sfrac{-(-1) \pm \sqrt{(-1)^{2} - 4(2)(-1)}}{2(2)}\).% \hfill \ref{illu:gist:roots:QF:qf}, \ref{illu:gist:roots:QF:subs}, Substitution
    \item
      \label{illu:gist:roots:QF:qf:1}
      \(x = \sfrac{(1 \pm 3)}{4}\).% \hfill \ref{illu:gist:roots:QF:qf-subs}, Simplification
    \item
      \label{illu:gist:roots:QF:qf:done}
      Either \(x = 1\) or \(x = -\sfrac{1}{2}\).% \hfill \ref{illu:gist:roots:QF:qf:1}, Expansion, Simplification
    \end{enumerate}
    %
    The agent concludes:
    \rootsCon{}.
  \end{scenario}

  \noindent%
  Just as the quadratic formula seems irrelevant to the \agents{} reasoning in \autoref{illu:gist:roots:F}, it seems the \agents{} understanding of factorisation is irrelevant to the \agents{} reasoning in \autoref{illu:gist:roots:QF}.
\end{note}



\paragraph*{Abstractions}


\begin{note}
  Abstracting a little, the conclusion of \scen{1}~\ref{illu:gist:roots:F}~and~\ref{illu:gist:roots:QF} --- \propM{\rootsCon{}} --- corresponds to a way things may be.
  If the truths of mathematics are necessary, then things are always this way.
  Even so, \propM{\rootsCon{}} captures a way things may be in the same way as \propI{Foxes change the colour of their fur to reflect their mood} captures a way things may be.
  For ease we refer to ways things may be as \emph{\prop{1}}.

  Continuing the abstraction, we say an \agents{} concludes a \prop{0} has a \emph{\val{0}}.
  Where, a \val{0} captures the \agpe{} on the way things may be.
  For example, in \scen{1}~\ref{illu:gist:roots:F}~and~\ref{illu:gist:roots:QF} the agent concludes \propM{\rootsCon{}} has value \valI{True}.%
  \footnote{
    Our interest is primarily with the \val{} \valI{True}.
    However, we place no restrictions on possible \val{1}.
    Hence, an agent may conclude \propI{Foxes change the colour of their fur to reflect their mood} has \val{} \valI{Desirable}.
    I.e., the agent concludes it is desirable that Foxes change the colour of their fur to reflect their mood.
  }
  Though, the agent may have concluded \propM{\rootsCon{}} has \val{} \valI{False} or \propM{\rootsConBad{}} has \val{} \valI{True}, etc.
\end{note}


\begin{note}
  Further, an agent concludes \emph{from} some thing.
  For our interests, an agent concludes a \prop{}-\val{} pair from some way things are from the \agpe{}.
  In other words, some collection of \prop{}-\val{} pairs.
  Shortening, we term the collection of \prop{}-\val{} pairs an agent concludes a \prop{}-\val{} pair from a \emph{\pool{}}.

  In general, what the relevant \pool{} includes is difficult to say.
  Though, we may highlight what is or is not captured by the relevant \pool{}.
  For example, with respect to \autoref{illu:gist:roots:F}, the \pool{} captures the \agents{} understanding of factorisation but does not capture the \agents{} understanding of the quadratic formula.
  And, with respect to \autoref{illu:gist:roots:QF}, the \pool{} captures the \agents{} understanding of the quadratic formula but does not capture the \agents{} understanding of factorisation.
\end{note}


\begin{note}
  When an agent concludes a \prop{1} has some \val{1} from \pool{1}, relations hold between the \prop{1}, \val{1}, and \pool{1}.
  Two relations are of particular interest:
  First, the agent has concluded the \prop{1} has the \val{1} \emph{from} the \pool{1}, from \agpe{our}.
  Second, that the \prop{1} has the \val{1} \emph{follows from} the \pool{1}, from the \agpe{}.

  `From' is about what happened independent of the \agpe{}.
  `Follows from' is about connections between various \prop{1}-\val{1} dependent on the \agpe{}.
  Rather than say that the \prop{1} has the \val{1} \emph{follows from} the \pool{1} (from the \agpe{}) we say a \ros{} between the \prop{0}, \val{0}, and \pool{0} (from the \agpe{}).

  For example, in \autoref{illu:gist:roots:F} the agent concludes \propM{\rootsCon{}} has value \valI{True} from some \pool{} which captures their understanding of factorisation.
  Hence, a \ros{} between \propM{\rootsCon{}}, \valI{True} and some \pool{} which captures the \agents{} understanding of factorisation holds, from \agpe{their}.
\end{note}


\begin{note}
  As \ros{1} abstractly capture when a \prop{0}-\val{0} pair `follows from' some \pool{}, a \ros{0} may hold between some \prop{0}, \val{0}, and \pool{} regardless of whether the \prop{0}-\val{0} pair has `been concluded from' the \pool{}.

  Consider \autoref{illu:gist:roots:F}.
  It seems clear at the second step of reasoning the agent is \emph{concluding} \propM{\rootsCon{}} has \val{} \valI{True}.
  % In other words, it seems clear an event in which the agent concludes \propM{\rootsCon{}} has \val{} \valI{True} is in progress.
  % In turn, it seems the agent's understanding of factorisation is what ensures it the case an event in which the agent concludes \propM{\rootsCon{}} has \val{} \valI{True} is in progress.
  Something ensures the agent is concluding.
  And, we use a \ros{} between \propM{\rootsCon{}}, \valI{True} and a \pool{} which captures the \agents{} understanding of factorisation to captures whatever it is that ensures the agent is concluding.

  Alternatively one may say that agent has the ability, the know-how, skill, disposition, etc.\ to conclude \propM{\rootsCon{}} has value \valI{True} from the relevant \pool{} and is executing the know-how or manifesting the disposition at the second step of reasoning.
  Still, \ros{0} are designed to abstract from such specific details.
\end{note}


\begin{note}
  \label{rosFirst}
  Roughly, we assume a \ros{} between some \prop{0} \(\phi\), \val{0} \(v\) and \pool{0} \(\Phi\) holds when either:
  %
  \begin{itemize}
  \item
    The agent has concluded \(\phi\) has value \(v\) from \(\Phi\).
  \item
    There is some action may immediately do such that the agent is concluding \(\phi\) has value \(v\) from \(\Phi\) when the agent does the action.
  \end{itemize}
  %
  In this respect, a \ros{} between \propM{\rootsCon{}}, \valI{True} and a \pool{} which captures the \agents{} understanding of \emph{the quadratic equation} may hold at the second step of reasoning in \autoref{illu:gist:roots:F} as the agent may abandon factorisation and do the reasoning outlined in \autoref{illu:gist:roots:QF}.
  And, it need not be the case that, given \(y = \sqrt{ 2 + \sqrt{2 + \cdots}}\), a \ros{} holds between \propM{y = 2}, \valI{True}, and the \agents{} understanding of factorisation (even though it is clear \(y^{2} - y - 2 = 0\) given a little insight).
\end{note}


\begin{note}
  In short, a \ros{} is an abstraction which relates a \prop{0}, \val{0}, and \pool{0} and captures that the \prop{} having the \val{} `follows from' the \pool{} from the \agpe{}.
  Where, in particular, a \ros{} may hold prior to a conclusion and is such that whatever secures the applicability of the abstraction also ensures the agent may the option to be, or is, concluding the \prop{0} has value \val{0} from the \pool{0}.

  This is all quite abstract.
  However, the abstraction has a purpose.
  Our interest is understanding the way an event in which an agent concludes happens, though we do not have interest in any particular theory about the way an event in which an agent concludes happens.
  Rather, our interest is with a pre-theoretical constraint on such understanding.
  And, abstracting to \prop{1}, \val{1}, \pool{1}, and \ros{1} allows a theory neutral account of some key aspects of conclusions.

  Given this is an introduction, the characterisation of the abstractions is brief.
  Other chapters in this document expand on these abstractions in greater detail.%
  \footnote{
    \autoref{cha:clar} focuses on \prop{1}, \val{1} and \pool{1}.
    And, \autoref{cha:ros} focues on \ros{1}.
  }
\end{note}



\paragraph*{An additional \scen{0}}


\begin{note}
  \begin{rscenario}{scen:countS}{Countersign}%
    \indent The captain mumbled, ``I come from Miran.''

    The man returned the gambit, grimly.
    ``Miran is early this year.''

    The captain said, ``No earlier than last year.''

    But the man did not step aside.
    He said, ``Who are you?''

    ``Aren't you Fox?''

    ``Do you always answer by asking?''

    The captain took an imperceptibly longer breath, and then said calmly,
    ``I am Han Pritcher, Captain of the Fleet, and member of the Democratic Underground Party.
    Will you let me in?''%
    \mbox{ }\hfill\mbox{(\cite[70]{Asimov:1945aa})}%
    \newline
  \end{rscenario}

  \noindent%
  In the event described by \autoref{scen:countS} Fox concludes \propI{The person I'm talking to is a member of the Democratic Underground Party} has value \valI{True}.%
  \footnote{
    Pritcher tells Fox the are a fellow member of the party at the end of the \scen{}.
    Still, Fox has already drawn the conclusion when Fox asks `Who are you?' via the sequence of countersigning.
  }

  Consider Pritcher's opening gambit `I come from Miran'.
  The following conditional plausibly holds, from \agpe{Fox's}:
  %
  \begin{itenum}
  \item[\emph{If}:]
    They didn't open with `I come from Miran'.
  \item[\emph{Then}:]
    They're not engaging in countersign.
  \end{itenum}
  %
  And, if Fox doesn't think Pritcher is engaging in countersign, it seems clear Fox does not conclude Pritcher a fellow member of the party.
  Hence, phrased in terms of \ros{1}, we have:
  \begin{itenum}
  \item[\emph{If}:]
    A \ros{} between \propI{What they said was a sign}, \valI{True} and some \pool{} \(\Psi\) fails to hold.
  \item[\emph{Then}:]
    Fox does not conclude \propI{The person I'm talking to is a member of the Democratic Underground Party} has value \valI{True} from some \pool{} \(\Phi\).
  \end{itenum}
  %
  In contrast to \scen{1} \ref{illu:gist:roots:F} and \ref{illu:gist:roots:QF}, the highlighted \ros{} \propI{What they said was a sign}, \valI{True} and \(\Psi\) is distinct from the \ros{} which characterises Fox's final conclusion.
  Further, the \ros{} did not hold prior to Pritcher saying `I come from Miran'.
  And, the \ros{} does not ensure Fox is concluding \propI{The person I'm talking to is a member of the Democratic Underground Party} has value \valI{True} from \(\Phi\).

  Still, the \ros{} between \propI{What they said was a sign}, \valI{True} and \(\Psi\) is tightly connected to \emph{whether} Fox is concluding \propI{The person I'm talking to is a member of the Democratic Underground Party} has value \valI{True} from \(\Phi\) by the second conditional.%
  \footnote{
    If Pritcher does not sign, Fox does not countersign, and the \scen{} goes differently.
  }

  Hence, the same abstractions applied to \scen{1}~\ref{illu:gist:roots:F}~and~\ref{illu:gist:roots:QF} apply to \autoref{scen:countS}.
\end{note}



\section*{\qWhy{}, \qHow{} and \issueInclusion{}}
\label{cha:intro:why-how}


\begin{note}
  \phantlabel{how-and-why-first-mention}%
  With respect to understanding the way an event in which an agent concludes happens we distinguish two questions:
  `\qWhy{}' and `\qHow{}'.

  \begin{question}{questionWhy}{\qWhy{}}
    Given \(e\) is an event in which \vAgent{} concludes \(\phi\) has value \(v\):
    \begin{itemize}
    \item
      Which \ros{} between \prop{1}, \val{1}, and \pool{1} explain \emph{why} \(e\) is an event in which \vAgent{} concludes \(\phi\) has value \(v\) (rather than some other event)?
    \end{itemize}
    \vspace{-1.5\baselineskip}
  \end{question}

  \begin{question}{questionHow}{\qHow{}}
    \label{q:how}
    Given \(e\) is an event in which \vAgent{} concludes \(\phi\) has value \(v\):
    \begin{itemize}
    \item
      Which past or present events explain \emph{how} \(e\) is an event in which \vAgent{} concludes \(\phi\) has value \(v\) (rather than some other event)?
    \end{itemize}
    \vspace{-1.5\baselineskip}
  \end{question}
\end{note}


\begin{note}
  \qWhy{} fixes an event, and queries why the event is an event in which an agent concludes \(\phi\) has value \(v\) from \(\Phi\) rather than an event in which something incompatible with the conclusion \(\phi\) has \val{0} \(v\) for \(\Phi\) happens.
  For example, an event in which the agent concludes \(\pv{\phi'}{v}\) from \(\Phi\), \(\pv{\phi}{v}\) from \(\Phi'\), takes their dog for a walk, runs a bath, etc.

  \qHow{}, by contrast, queries how the event is an event in which an agent concludes \(\phi\) has \val{0} \(v\) from \(\Phi\) came about.
\end{note}


\paragraph*{The sense of `why' at issue in \qWhy{}}


\begin{note}
  To get a feel for the sense of `why' at issue in \qWhy{}, consider an event in which a deck of cards is shuffled.

  If the cards are randomly shuffled, then, intuitively, nothing explains \emph{why} \mainCard{} is \mainCardPos{} after the shuffle.
  For, if cards are shuffled randomly, then nothing favours \mainCard{} being at any particular position in the pack after the shuffle.

  However, if the shuffler is engaging in sleight of hand there may be an explanation why \mainCard{} is \mainCardPos{} after the shuffle.
  At a high level on may say: The shuffler selected \mainCard{} and then performed a `break and shuffle'.
  And, at a lower level, one may expand on the details of a break and shuffle (\cite[cf.][189--190]{Hilliard:1994aa}), and lower still one may detail the break and shuffle the shuffler performed.
  The key to any such explanation is that \mainCard{} was selected, and a break and shuffle ensures the selected card is at the \mainCardPos{} after the shuffle.%
  \footnote{
    A similar observation applies to coin flips.
    For, intuitively nothing explains why a unbiased coin landed heads rather than tails.
    Though, it is hard to bias a coin flip. (\cite{Gelman:2002ww})
  }
\end{note}

\begin{note}
  Still, the shuffle explains \emph{how} \mainCard{} is \mainCardPos{} after the shuffle, whether random or sleighted.
  For, there is a trace the position of \mainCard{} as it moves through the pack during the shuffle.
\end{note}


\begin{note}
  \nocite{Hoefer:2023aa}
  Generalising, there is no answer to `why' any random event happened, given the sense of `why' present in \qWhy{}.
  And, anything one may highlight which favours an event \(e\) happening over any other event \(e'\) answers `why' \(e\) happened, given the sense of `why' present in \qWhy{}.

  In this respect, if causal determinism holds --- any event is the result of antecedent events together with the laws of nature --- then there is always some answer to `why' the event happened.
  For, the antecedent events and laws of nature ensure no other event may have happened.
\end{note}


\begin{note}
  The idea of a shuffler is engaging in sleight of hand suggests determinism.
  However, some thing may answer `why' an event happened without the thing determining that the event happens.

  For example, consider rolling a die.
  As with a random shuffle, intuitively nothing answers `why' an agent rolls a 6 with an unbiased die  --- setting causal determinism aside.
  However, if an agent rolls a 6 with a die biased toward 6 then, the bias of the die favours an event in which any agent rolls a 6.
  Though, an agent may still roll of 1,2,4, etc. with given die with bias towards a 6.
\end{note}


\begin{note}
  Now, \qWhy{}, specifically, seeks answers to an event in which an agent concludes in the form of \ros{1}.

  In this respect, for a \ros{} to answer \qWhy{}, a \ros{} holds prior to agent's conclusion, and favours the conclusion over some other event.

  In the case of \scen{1}~\ref{illu:gist:roots:F}~and~\ref{illu:gist:roots:QF} we suggested a \ros{} between \propM{\rootsCon{}}, \valI{True} and some \pool{} which captures the \agents{} understanding of factorisation or the quadratic formula holds prior to the \agents{} conclusion and the \ros{} ensures the agent is concluding \propM{\rootsCon{}}, \valI{True} from the relevant \pool{}.
  In this respect, as the \ros{} ensures the agent is concluding, the \ros{} answers \qWhy{}.
  For, an event in which an agent is concluding favours an event in which an agent concludes.

  Likewise, in the case of \autoref{scen:countS} we considered a \ros{} between \propI{What Pritcher said was a sign}, \valI{True} and some \pool{} \(\Psi\) with respect to Fox and noted the if the \ros{} fails to hold, Fox does not conclude Pritcher is a member of The Party.
  In this respect, the \ros{} favours an event in which Fox concludes \propI{The person I'm talking to is a party member} has value \valI{True} from some \pool{} \(\Phi\).

  Of course, that an agent is concluding is no guarantee that an agent concludes.
  However, in parallel, an agent performing sleight of hand does not guarantee the agent performs slight of hand, and the bias of a biased die does not guarantee the die lands in line with the bias.
  Answers to \qWhy{} do not need to `completely' answer \qWhy{}.

  And, it need not be the case that a \ros{} which holds between \(\phi\), \(v\), and \(\Phi\) answers \qWhy{}.
  For, example, a \ros{} between \propI{What Pritcher said was a sign}, \valI{True} and some \pool{} \(\Psi\) fails to answer \qWhy{} Fox concluded \propI{What Pritcher said was a sign} has value \valI{True} from \(\Psi\) as the \ros{} did not hold prior to Fox's (intermediate) conclusion.
  Though, as noted above, the \ros{} favours Fox's (overall) conclusion that \propI{The person I'm talking to is a party member} has value \valI{True} from \(\Phi\).
\end{note}



\paragraph*{A constraint on answers to \qWhy{} in terms of answers to \qHow{}}


\begin{note}
  Consider the following constraint on answers to \qWhy{} in terms of answers to \qHow{}:

  \begin{constraint}{consInclusion}{\issueInclusion{}}
    \mbox{ }
    \vspace{-\baselineskip}
    \begin{itenum}
    \item[\emph{If}:]
      A \ros{} between \(\psi\), \(v'\), and \(\Psi\) answers \qWhy{}.
    \item[\emph{Then}:]
      An event in which the agent concludes \(\psi\) has value \(v'\) from \(\Psi\) answers \qHow{}.
    \end{itenum}
    \vspace{-\baselineskip}
  \end{constraint}

  \noindent%
  In short, \issueInclusion{} states that for any \ros{} between a \prop{0}, \val{0}, and \pool{} which grants some understanding of the way an agent concluded \(\phi\) has value \(v\), there is an event such that the agent concludes the \prop{} has the \val{} from the \pool{}.%
  \footnote{
    Note, \qHow{} does not explicitly require the relevant event to be the event in which the agent concludes \(\phi\) has value \(v\) from \(\Phi\).
    Hence, a previous event in which the agent concludes \(\psi\) has value \(v\) from \(\Psi\) may answer \qHow{}.
    \issueInclusion{} is a constraint, and hence is compatible with a more stringent constraint.
    Still, we have no interest in motivating a more stringent constraint.
  }

  I think \issueInclusion{} is intuitive.

  For example, \issueInclusion{} is compatible with the \ros{1} highlighted with respect to \scen{1}~\ref{illu:gist:roots:F},~\ref{illu:gist:roots:QF}, and \autoref{scen:countS}.

  With respect to \scen{1}~\ref{illu:gist:roots:F}~and~\ref{illu:gist:roots:QF} the \ros{} of interest holds between \propM{\rootsCon{}}, \valI{True} and some \pool{} which captures the \agents{} understanding of factorisation or the quadratic formula.
  And, the agent concludes \propM{\rootsCon{}} has value \valI{True} from \pool{} which captures the \agents{} understanding of factorisation or the quadratic formula.

  And, with respect to \autoref{scen:countS} the \ros{} of interest holds between \propI{What Pritcher said was a sign}, \valI{True} and some \pool{} \(\Psi\).
  Fox has concluded \propI{What Pritcher said was a sign} has value \valI{True} from \(\Psi\) when Fox concludes \propI{The person I'm talking to is a party member} has value \valI{True} from some \pool{} \(\Phi\).
\end{note}


\begin{note}
  To get a feel for what \issueInclusion{} rules out, consider \autoref{illu:gist:roots:F} and a \ros{} which holds between \propM{\rootsCon{}}, \valI{True} and some \pool{} which captures the \agents{} understanding of the quadratic formula.
  We suggested above the agent may abandon factorisation and conclude \propM{\rootsCon{}} has value \valI{True} via the quadratic formula.
  Hence, it is may be the case a \ros{} which holds between \propM{\rootsCon{}}, \valI{True} and some \pool{} which captures the \agents{} understanding of the quadratic formula answers `why' the agent concluded \propM{\rootsCon{}} has value \valI{True} from their understanding of factorisation.
  However, this seems bad.
  The agent concluded by factorisation and the quadratic formula was irrelevant to the reasoning the agent did, even if it was reasoning the agent had the option to do.

  More explicitly, it seems implausible that the \agents{} understanding of the quadratic formula favoured an event in which the agent concluded \propM{\rootsCon{}} has value \valI{True} from their understanding of factorisation over any other event.

  To draw an analogy, if a die's bias towards \(6\) takes effect as the die is rolling and the die lands \(6\), it seems the die bias explain (in part) why the die landed \(6\).
  However, if a biased die is dropped from a short height to ensure it does not roll and lands \(6\), then the die's bias is irrelevant to explaining why (in part) the die landed \(6\).
\end{note}


\begin{note}
  I think \issueInclusion{} is intuitive.
  Still, we have only considered a few \scen{1} and \issueInclusion{} is a general constraint.
  Hence, if there is some doubt regarding \issueInclusion{} then further argument is required.
  However, I am not aware of any arguments for \issueInclusion{}.%
  \footnote{
    The construction of \qWhy{}, \qHow{}, and \issueInclusion{} is somewhat idiosyncratic, and so it is no surprise there are no direct argument for \issueInclusion{}.
    However, I take the idea captured by \issueInclusion{} to be intuitive, and I am not aware of any argument for this idea either.

    {
      \color{blue}
      The appendix on rationalisations.
    }
  }
  And, no argument for \issueInclusion{} will not be given.
\end{note}



\section*{Counterexamples to \issueInclusion{}}


\begin{note}
  The goal is to provide a recipe for constructing counterexamples to \issueInclusion{}.

  By `recipe', I mean an description of some features which hold of \scen{1} and an account of the way in which these combine which entails violations of \issueInclusion{}.
  With the recipe in hand, we will construct a few counter samples, but our interest is with the way in which \issueInclusion{} fails, rather than the failure of \issueInclusion{}.

  In other words, the goal of this document is to show that any \scen{0} which satisfies a collection of features entails \issueInclusion{} does not hold.
  This means definitions, and propositions which extract features from definitions and propositions which link definitions together.%
  % \footnote{
  %   More carefully put, the goal is to establish a collection of features \emph{more-or-less} deductively entails \issueInclusion{} does not hold.
  %   Some propositions (I think three) fall short of being deductive.
  %   Though, there is an important distinction between definitions and propositions and the motivation for definitions and propositions.
  %   I do not claim the relevant motivation is (more-or-less) deductive.
  % }

  The recipe identifies \scen{1} with the following features:
  \begin{enumerate}
  \item
    An event in which an agent concludes \(\phi\) has value \(v\) from \(\Phi\).
  \item
    A \ros{} between \(\psi\), \(v'\), and \(\Psi\) such that clauses \ref{reciF:qWhy}, \ref{reciF:distinct}, and \ref{reciF:nHow} hold:
    \begin{enumerate}[label=\Alph*., ref=\Alph*]
    \item
      \label{reciF:qWhy}
      The \ros{} between \(\psi\), \(v'\), and \(\Psi\) answers \qWhy{}.
    \item
      \label{reciF:distinct}
      Either \(\psi\) is distinct from \(\phi\), \(v'\) is distinct from \(v\), or \(\Psi\) is distinct from \(\Phi\).
    \item
      \label{reciF:nHow}
      There is no prior (or present) event in which the agent concludes \(\psi\) has value \(v'\) from \(\Psi\).
    \end{enumerate}
  \end{enumerate}
\end{note}


\begin{note}
  The importance of clauses \ref{reciF:qWhy}, \ref{reciF:distinct}, and \ref{reciF:nHow} follow from the following observation:

    \begin{observation}[Failures to \issueInclusion{}]%
    \label{obs:iIceRestriction}%
    \vspace{-\baselineskip}
    \begin{itenum}
    \item[\emph{If}:]
      A \ros{} between \(\psi\), \(v'\), and \(\Psi\) is a for sure a counterexample to \issueInclusion{}.
    \item[\emph{Then}:]
      Conditions \ref{obs:iIceRestriction:why}, \ref{obs:iIceRestriction:distinct}, and \ref{obs:iIceRestriction:prior} are true of the \ros{} between \(\psi\), \(v'\), and \(\Psi\):
      \begin{enumerate}[label=\arabic*., ref=\arabic*]
      \item
        \label{obs:iIceRestriction:why}
        The \ros{} between \(\psi\), \(v'\), and \(\Psi\) answers \qWhy{}.
      \item
        \label{obs:iIceRestriction:distinct}
        Either \(\psi\) is distinct from \(\phi\), \(v'\) is distinct from \(v\), or \(\Psi\) is distinct from \(\Phi\).
      \item
        \label{obs:iIceRestriction:prior}
        There is no prior (or present) event in which the agent concludes \(\psi\) has value \(v'\) from \(\Psi\).
      \end{enumerate}
    \end{itenum}
    \vspace{-1.5\baselineskip}
  \end{observation}
  %
  \begin{motivation}{obs:iIceRestriction}
    Suppose a \ros{} between \(\psi\), \(v'\), and \(\Psi\) is a counterexample to \issueInclusion{}.
    We observe each condition holds.

    \begin{enumerate}
    \item
      \issueInclusion{} constrains answers to \qWhy{} in terms of answers to \qHow{}.
    So, for the \ros{} between \(\psi\), \(v'\), and \(\Psi\) to be a counterexample to \issueInclusion{}, the \ros{} between \(\psi\), \(v'\), and \(\Psi\) must answer \qWhy{}.
  \item
    \qWhy{} applies to events in which an agent concludes \(\phi\) has value \(v\) from \(\Phi\).

    Now, if \(\psi\) is the same as \(\phi\), \(v'\) is the same as \(v\), and \(\Psi\) is the same as \(\Phi\) then the event in which the agent concludes \(\phi\) has value \(v\) from \(\Phi\) is an event in which the agent concludes \(\psi\) has value \(v'\) from \(\Psi\).
    And, the event in which the agent concludes \(\phi\) has value \(v\) from \(\Phi\) answers \qHow{}.
    Hence, an event in which the agent concludes \(\psi\) has value \(v'\) from \(\Psi\) answers \qHow{}.
  \item
    Finally, \qHow{} is compatible with \scen{1} where some prior event in which the agent concludes \(\psi\) has value \(v'\) from \(\Psi\) answers \qHow{}.
    Whether or not there are \scen{1} with this feature is something we set aside.
    Still, a counterexample to \issueInclusion{} is not guarantee if there is some prior event in which the agent concludes \(\psi\) has value \(v'\) from \(\Psi\).
  \end{enumerate}
  \vspace{-\baselineskip}
  \end{motivation}
\end{note}


\begin{note}
  The broad idea of the recipe is simple.
  A \ros{} answers \qWhy{} just in case it favours an event in which the agent concludes over any other event.
  In particular, \autoref{illu:gist:roots:F}.

  \agents{} understanding of factorisation.
  The \agents{} understanding of factorisation is not limited to \ros{}.
  Rather, various instances of factorisation.
  Without these, it is not clear the agent is concluding by factorisation.
  The agent may be 

  Hence, \ros{1} answer \qWhy{}.
  However, it need not be the case that the agent has concluded.
  So, get clauses \ref{reciF:qWhy}, \ref{reciF:distinct}, and \ref{reciF:nHow}.

  Sleight of hand.
  Distinction between sleight of hand and a bad shuffle.
  May be the case that agent tenses hand so that bottom card stays in position and then place on top to finish the shuffle.

  Sleight of hand, then other cards.

  Phrased a little differently, ability, know-how, disposition etc.
  Know-how favours conclusion prior to conclusion.
  And, know-how is connected to other conclusions the agent may do.
\end{note}


\begin{note}
  Still, \issueInclusion{} as a pre-theoretical constraint.
  The construction of the recipe does not assume too much about what a conclusion amounts to.
  A specific account of know-how may generate the kind of \scen{1} of interest with ease.
  However, the recipe does not rely on an account of know-how, dispositions, etc.

  Things do get technical.
  No particular theory of reasoning, but general constraints in order to capture phenomenon (such as \prop{1}, \val{1}, \pool{1}, and \ros{1}) and further definitions.
\end{note}

\paragraph*{Structure}


\begin{note}
  Before constructing counterexamples to \issueInclusion{} we detail key ideas introduced in this introduction.

  The details are important.
  For, without a sufficiently detailed account of the phenomena of interest, any recipe proposed with fail to identify \scen{1}.
  And, without a detailed account of \qWhy{} and \qHow{} it is unclear whether any identified \scen{1} is a counterexample to \issueInclusion{}.

  Start with events.
  Clarify \qWhy{}.

  Following, conclusions.
  Little of interest here.

  Key idea with respect to \scen{1}~\ref{illu:gist:roots:F}~and~\ref{illu:gist:roots:QF} is \ros{} prior to conclusion.
  To capture instances, \fc{1}.

  With \fc{1}, \ros{1}.

  With \ros{1}, return to \qWhy{}, \qHow{}, and \issueInclusion{}.

  Counterexamples.
  \requ{2}.
  \fc{3}, \ros{1}, fails to favour without \fc{1}.

  To motivate the existence of \requ{1}, \tC{}.

  Link these together.

  This is it.

  Counter-samples.

  Nothing assumes counterexamples exist.
  Instead, when these combine to form a \scen{}.

  Close with some additional stuff.
\end{note}



\section*{Notes}


\begin{note}
  Quick notes:

  First, I hold explanatory reasons are factive, and I think that answers to \qWhy{} reduce to psychological facts of an agent, specifically psychological facts which hold of the agent when they conclude.%
  % \footnote{
  %   This is in slight tension with views of reasons explanation advanced by~(\cite{Dancy:2000aa}) and~(\cite{Alvarez:2013aa}), in which the state of affairs may be an agent's reason (and not just the evaluation of some state of affairs).
  %   Though, the argument will not depend on assuming that the relations reduce to psychological facts.

  %   See~(\cite[413--418]{Hieronymi:2011aa}),~(\cite[3--5]{DOro:2013vh}), and~(\cite[\S2]{Alvarez:2017aa}) for more.
  % }

  Given what I hold and think, from a very broad perspective, \issueInclusion{} is a restriction on which psychological facts of an agent matter when an agent concludes:

  If \issueInclusion{} holds, the only facts which matter are those which secure the \ros{} between \(\phi\), \(v\), and \(\Phi\).

  And, if \issueInclusion{} fails, facts other than those which secure the \ros{} between \(\phi\), \(v\), and \(\Phi\) matter, specifically, facts which secure some \ros{} between \(\phi\), \(v\), and \(\Phi\).
\end{note}

% \begin{note}
%   As seen above with respect to \autoref{scen:animalism}, it follows from \issueInclusion{} that the content of Snowball's explanation is irrelevant.
%   For, the birds do not understand Snowball's explanation, hence there is no event in which the birds reason from the content to \propI{Four legs good, two legs bad} has value \valI{True}.
%   And, I think this is correct.

%   However, parallel reasoning entails the agent's understanding of arithmetic is irrelevant with respect to \autoref{scen:calc}.
%   And, I do not think this entailment holds.
%   I think it may be the case that a relation between \propM{\gistCalcEq{}}, \valI{True} and the agent's understanding of arithmetic \emph{may} answer \qWhy{}.
%   Whether this is the case will depend on whether some additional details hold of~\autoref{scen:calc}.
%   Still, as the details matter, the entailment, and hence \issueInclusion{}, is not right.
% \end{note}

\begin{note}
  Still, there is distinction between psychological facts and the exercise of agency.
  To illustrate, there a senses in which the following two questions are distinct:

  \begin{itemize}
  \item
    \emph{Why} is \(e\) is an event in which \vAgent{} concludes \(\phi\) has value \(v\)?
  \item
    \emph{Why} is it the case \vAgent{} concludes \(\phi\) has value \(v\) in \(e\)?
  \end{itemize}
  %
  Consider \citeauthor{Davidson:1973vd}'s climber:
  %
  \begin{quote}
    A climber might want to rid himself of the weight and danger of holding another man on a rope, and he might know that by loosening his hold on the rope he could rid himself of the weight and danger.
    This belief and want might so unnerve him as to cause him to loosen his hold, and yet it might be the case that he never chose to loosen his hold, nor did he do it intentionally.%
    \mbox{ }\hfill\mbox{(\citeyear[79]{Davidson:1973vd})}
  \end{quote}
  %
  The \agents{} belief and want are psychological facts about the agent, and answer why \emph{the event} is an event in which the agent loosens their hold on the rope.
  However, the \agents{} belief and want do not answer why \emph{the agent} loosens their hold on the rope.
  Indeed, there seems no answer to why the agent loosens their hold on the rope --- the act was not an exercise of agency.

  \citeauthor{Davidson:1973vd}'s climber does not conclude.
  Still, if an agent may loosen their grip without exercising their agency, it is seems plausible that a \ros{} between \(\psi\), \(v'\) and \(\Psi\) which answers why is \emph{\(e\) is an event} in which \vAgent{} concludes \(\phi\) has value \(v\) may fail to answer why is it the case the agent concludes \(\phi\) has value \(v\) in \(e\).

  In short, it is not clear that any counterexample to \issueInclusion{} relates to an exercise of agency.
  Whether this matters is for you to decide.
  On some days I think this matters a great deal, and on other days I doubt there is any coherent idea of what an exercise of agency amounts to.

  We set concerns about agency aside for the argument against \issueInclusion{}, and then motivate that relevant counterexamples either involve agency or have nearby neighbours which involve agency.
\end{note}


%   \begin{quote}
%     Sometimes the explanation of why a person does something has a particular character:
%     roughly, it involves the person's rationality in a distinctive way that I shall not try to describe.
%     Then we say the person does what she does for a reason.
%     We might say ‘The reason for which Hannibal used elephants was to terrorize the Romans'.
%     The reason for which a person does something is called a ‘motivating reason'.
%     In general, a motivating reason is whatever explains or helps to explain what a person does in the distinctive way that involves her rationality.
%     \mbox{}\hfill\mbox{(\citeyear[46--47]{Broome:2013aa})}
%   \end{quote}
% \end{note}

% \begin{note}
%   \color{red}
%   \begin{quote}
%     \emph{R} is a primary reason why an agent performed the action \emph{A} under the description \emph{d} only if \emph{R} consists of a pro attitude of the agent toward actions with a certain property, and a belief of the agent that \emph{A}, under the description \emph{d}, has that property.\newline
%     \mbox{ }\hfill\mbox{(\citeyear[687]{Davidson:1963aa})}
%   \end{quote}

%   We have distinguished \qWhy{} from pro-attitudes.
%   However, fill in whatever motivation.
%   What matters is the belief.
%   This is the relevant proposition-value pair.

%   If \citeauthor{Davidson:1963aa}, then granting restriction, seems we don't need to look beyond the proposition-value pair.
% \end{note}


%%% Local Variables:
%%% mode: latex
%%% TeX-master: "master"
%%% TeX-engine: luatex
%%% End:
