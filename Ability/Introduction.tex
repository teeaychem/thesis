\chapter{\qWhy{} and \qHow{} an agent concludes}
\label{cha:intro}


\begin{note}
  Our interest is understanding the way an \eiw{0} an agent concludes happens.
  For example, consider:

  \begin{rscenario}{illu:gist:roots:F}{Quadratic roots --- Factors}%
    % \autoref{illu:gist:roots:F}
    An agent reasons as follows:
    %
    \begin{enumerate}[label=\arabic*., ref=\arabic*]
    \item
      \label{illu:gist:roots:F:eq}
      For some \(x \in \mathbb{R}\), \rootsConEq{}.
    \item
      \label{illu:gist:roots:F:factor}
      \rootsConEqFac{}.
    \item
      \label{illu:gist:roots:F:zero}
      Either \(\rootsConEqFacL{} = 0\) or \(\rootsConEqFacR{} = 0\).
    \item
      \label{illu:gist:roots:F:case:a}
      If \(\rootsConEqFacL{} = 0\), then \(x = \rootsConEqFacLx{}\).
    \item
      \label{illu:gist:roots:F:case:b}
      If \(\rootsConEqFacR{} = 0\), then \(x = \rootsConEqFacRx{}\).
    \item
      \label{illu:gist:roots:F:factor:done}
      Either \(x = \rootsConEqFacLx{}\) or \(x = \rootsConEqFacRx{}\).
    \end{enumerate}
    %
    The agent concludes:
    \rootsCon{}.
  \end{rscenario}

  \noindent%
  Intuitively, the agent concludes \propM{\rootsCon{}} from their understanding of factorisation.
  And, as the agent concluded via factorisation, the \agents{} understanding of the quadratic formula seems irrelevant understanding to the way \autoref{illu:gist:roots:F} happened.
  Consider, by contrast:

  \begin{rscenario}{illu:gist:roots:QF}{Quadratic roots --- Formula}%
    An agent reasons as follows:
    %
    \begin{enumerate}[label=\arabic*., ref=\arabic*]
    \item
      \label{illu:gist:roots:QF:eq}
      For some \(x \in \mathbb{R}\), \rootsConEq{}.
    \item
      \label{illu:gist:roots:QF:qf}
      The quadratic formula states \(x = \rootsQEq{}\).
    \item
      \label{illu:gist:roots:QF:subs}
      Set \(a \coloneq \rootsQa{}\), \(b \coloneq \rootsQb{}\), and \(c \coloneq \rootsQc{}\).%
      \footnote{
        \(a\) is the coefficient of the \(x^{2}\) term, \(b\) is the coefficient of the \(x\) term, and \(c\) is the constant.
      }
    \item
      \label{illu:gist:roots:QF:qf-subs}
      \(x = \rootsQEqFill{\rootsQa{}}{\rootsQb{}}{\rootsQc{}}\).
    \item
      \label{illu:gist:roots:QF:qf:1}
      \(x = \sfrac{(- 1 \pm 5)}{12}\).
    \item
      \label{illu:gist:roots:QF:qf:done}
      Either \(x = \rootsConEqFacLx{}\) or \(x = \rootsConEqFacRx{}\).
    \end{enumerate}
    %
    The agent concludes:
    \rootsCon{}.
  \end{rscenario}

  \noindent%
  Just as the quadratic formula seems irrelevant to understanding \autoref{illu:gist:roots:F}, the \agents{} understanding of factorisation seems irrelevant to understanding \autoref{illu:gist:roots:QF}.
\end{note}


\begin{note}
  This document is about understanding the way an \eiw{0} an agent concludes happens.
  In particular, our interest is with whether a constraint on answers to \emph{why} an event happened in terms of answers to \emph{how} the event happened holds.
\end{note}


\section{Abstractions}


\begin{note}
  To help state and reason about why, how, and a constraint on answers to why in terms of answers to \emph{how} we introduce some abstractions.
  These abstractions are stated with the help of \scen{1} \ref{illu:gist:roots:F} and \ref{illu:gist:roots:QF} and are then applied an additional \scen{}.
\end{note}


\begin{note}
  The conclusion of \scen{1}~\ref{illu:gist:roots:F}~and~\ref{illu:gist:roots:QF} --- \propM{\rootsCon{}} --- is about the way things may be.
  We refer to ways things may be as \emph{\prop{1}}, and write these in a \textsf{sans-serif} font and use lower case Greek letters for arbitrary \prop{1}.
  Other examples of \prop{1} include \propI{The soup is sweet}, \propI{Foxes like pineapples}, \propI{The time is 17:30}, and so on.

  A conclusion is an \eval{} of a \prop{0}, where, an \eval{} captures the \agpe{} on the way things may be.
  For example, in \scen{1}~\ref{illu:gist:roots:F}~and~\ref{illu:gist:roots:QF} the agent concludes \propM{\rootsCon{}} has \emph{\val{0}} \valI{True}.
  As with \prop{1} we write \val{1} in a \textsf{sans-serif} font and use variations of \(v\) for arbitrary \val{1}.%
  \footnote{
    Our interest is mostly with the \val{} \valI{True}.
    However, we place no restrictions on possible \val{1}.
    E.g., an agent may conclude \propI{Foxes like pineapples} has \val{} \valI{Desirable}.
  }
  For example, the agent may have concluded \propM{\rootsCon{}} has \val{} \valI{False} or \propM{\rootsConBad{}} has \val{} \valI{True}, and so on.
\end{note}


\begin{note}
  Further, an agent concludes a \prop{0} has some \val{0} \emph{from} some way things are from the \agpe{}.
  So, we understand an agent to conclude some \prop{0}-\val{0} pair from some \emph{\pool{}} of \prop{}-\val{} pairs.
  We use upper case Greek letters for \pool{1}.

  For example, in \autoref{illu:gist:roots:F} the agent concluded \propM{\rootsCon{}} has \valI{True} from their understanding of factorisation.
  The \agents{} understanding of factorisation is just the way various things relating to quadratic equations are, and hence there is some \pool{} \(\Phi\) of \prop{0}-\val{0} pairs captures this understanding.

  What is included in a \pool{} is difficult to say in detail, but broad characterisations may be given.
  For example, with \autoref{illu:gist:roots:F}, the \pool{} includes the \agents{} understanding of factorisation but does not include the \agents{} understanding of the quadratic formula, as the \agents{} understanding of the quadratic formula is irrelevant to factoring.
  In addition, the relevant \pool{} also contains \rootsConEq{} \evaled{} as \valI{True}.
  Still, we will often limit our description of a \pool{} to something about the way an agent concluded --- e.g.\ by saying \(\Phi\) includes the \agents{} understanding of factorisation.%
  \footnote{
    `Understanding' is assumed factive --- i.e.\ correct, through maybe not exhaustive.
  }
\end{note}


\begin{note}
  So, an event in which an agent concludes involves an agent \evaling{} some \prop{0} \(\phi\) to have \val{0} \(v\) from some \pool{} \(\Phi\).
  To help readability, we write \(\pv{\phi}{v}\) to associate a \prop{0} and a \val{0}.
  So, e.g., we say in \autoref{illu:gist:roots:F} the agent concluded \pv{\propM{\rootsCon{}}}{\valI{True}} a \pool{} which includes their understanding of factorisation.
\end{note}


\begin{note}
  Two relations between \prop{1}, \val{1}, and \pool{1} are of particular interest given an \eiw{0} agent concludes \(\pv{\phi}{v}\) from \(\Phi\):
  %
  \begin{enumerate}
  \item
    The \agents{} conclusion of \(\pv{\phi}{v}\) \emph{from} \(\Phi\).
  \item
    That \(\pv{\phi}{v}\) \emph{\fof{0}} \(\Phi\) from the \agpe{}.
  \end{enumerate}
  %
  `From' is about what happened.
  `Follows from' is about connections between various \prop{1}-\val{1} pair and \pool{} from the \agpe{}.

  For example, in \autoref{illu:gist:roots:F} the agent concludes \pv{\propM{\rootsCon{}}}{\valI{True}} from some \pool{} which includes their understanding of factorisation.
  Hence, the agent concluded \pv{\propM{\rootsCon{}}}{\valI{True}} \emph{from} the \pool{} and \pv{\propM{\rootsCon{}}}{\valI{True}} \emph{\fof{}} the \pool{} from the \agpe{}.

  In general, \fofb{some \prop{0}-\val{0} pair}{some \pool{}} from an \agpe{} when the agent concludes the \prop{0}-\val{0} pair from the \pool{}.
\end{note}


\begin{note}
  It may also be the case \fofb{some \prop{0}-\val{0} pair}{some \pool{}} without there being an event where the agent has concluded the \prop{0}-\val{0} pair from the \pool{}.
  `\fingf{2}' is about the way things are from the \agpe{}, and is not limited to what the agent has established.

  In particular, it may be the case \fofb{some \prop{0}-\val{0} pair}{some \pool{}} when, or just prior to, an agent \emph{concluding} the \prop{0}-\val{0} pair from the \pool{}.

  For example, consider \autoref{illu:gist:roots:F}.
  At the second step of the \agents{} reasoning it seems the agent is concluding \pv{\propM{\rootsCon{}}}{\valI{True}} from a \pool{} --- an event where the agent concludes \pv{\propM{\rootsCon{}}}{\valI{True}} from the \pool{} is in progress.
  Hence, as the agent is concluding \pv{\propM{\rootsCon{}}}{\valI{True}} from a \pool{}, it seems \pv{\propM{\rootsCon{}}}{\valI{True}} already \fof{} the \pool{} from the \agpe{}.%
  \footnote{
    One may go further and say that agent has the ability, the know-how, skill, disposition, etc.\ to conclude \pv{\propM{\rootsCon{}}}{\valI{True}} from the relevant \pool{} and is executing the know-how or manifesting the disposition at the second step of reasoning.
    Each idea, on a plausible reading, suggests \pv{\propM{\rootsCon{}}}{\valI{True}} already \fof{} the \pool{} from the \agpe{}.
    However, our interest is limited to the relation.
  }
  Nothing about the way things are changes when the agent concludes \pv{\propM{\rootsCon{}}}{\valI{True}} from the \pool{}.
  Rather, the agent comes to recognise something already in their field of view.

  Whether \fofb{a \prop{0}-\val{0} pair}{some \pool{}} depends on whether:%
  \footnote{
    Detailed assumptions are given in \autoref{cha:ros}.
  }
  %
  \begin{itemize}
  \item
    The agent has \evaled{} every \prop{0} in the \pool{} with the respective \val{0}.
  \item
    The agent may conclude the \prop{0}-\val{0} pair from the \pool{}.
  \end{itemize}
  %
  For example, suppose the time is 17:47.
  If an agent has not looked at a clock in some time it is plausible \pv{\propI{It is 17:47}}{\valI{True}} does not \fof{0} any \pool{} for the agent.
  For, the agent only be willing to conclude \pv{\propI{It is 17:47}}{\valI{True}} from the testimony of a clock.

  Likewise, it need not be the case that, given \(y = \sqrt{ 2 + \sqrt{2 + \cdots}}\), \fofb{\pv{\propM{y = 2}}{\valI{True}}}{the \agents{} understanding of factorisation} holds.
  The \fofr{} takes the \agpe{}, and the \agents{} grasp on factorisation and related concepts may be insufficient to solve all factorisation problems.
  An \agpe{} is limited by what the agent may perceive.
\end{note}


\begin{note}
  This is all quite abstract.
  However, the abstraction has a purpose.
  Our interest is understanding the way an \eiw{0} an agent concludes happens.
  Still, our interest is with a (more-or-less) pre-theoretical constraint on such understanding.
  And, abstracting to \prop{1}, \val{1}, \pool{1}, and \fofr{1} allows a theory neutral account of some key aspects of conclusions.

  Given this is an introduction, the characterisation of the abstractions is brief.
  Other chapters in this document expand on these abstractions in detail.%
  \footnote{
    \autoref{cha:clar} focuses on \prop{1}, \val{1} and \pool{1}.
    And, \autoref{cha:ros} focuses on \fofr{1}.
  }
\end{note}



\paragraph*{An additional \scen{0}}


\begin{note}
  \begin{rscenario}{scen:countS}{Countersign}%
    \indent The captain mumbled, ``I come from Miran.''

    The man returned the gambit, grimly.
    ``Miran is early this year.''

    The captain said, ``No earlier than last year.''

    But the man did not step aside.
    He said, ``Who are you?''

    ``Aren't you Fox?''

    ``Do you always answer by asking?''

    The captain took an imperceptibly longer breath, and then said calmly,
    ``I am Han Pritcher, Captain of the Fleet, and member of the Democratic Underground Party.
    Will you let me in?''%
    \mbox{ }\hfill\mbox{(\cite[70]{Asimov:1945aa})}%
    \newline
  \end{rscenario}

  \noindent%
  \autoref{scen:countS} is an instance of `countersign'.
  Both Fox and Pritcher are members of The Party, and verify this to each other by countersigning --- completing a pre-determined exchange (``I come from Miran.'', ``Miran is early this year.'', \dots).

  \noindent%
  Applying the abstractions introduced, \autoref{scen:countS} is an \eiw{} Fox concludes \pv{\propI{\signConA{}}}{\valI{True}} from some \pool{} \(\Phi\) which includes recognition of the countersigning that took place.
  And, when Fox \evals{} \propI{\signConA{}} as \valI{True}, \fofb{\pv{\propI{\signConA{}}}{\valI{True}}}{\(\Phi\)} from \agpe{Fox's}.%
  \footnote{
    Pritcher tells Fox they are a Party member at the close of the \scen{}.
    Still, Fox has already drawn the conclusion when Fox asks `Who are you?' via the sequence of countersigning.
  }

  Likewise, after Pritcher's opening `I come from Miran', Fox (sub-)concludes \pv{\propI{\signConB{}}}{\valI{True}} and some \pool{} \(\Psi\) which includes Pritcher's initial sign.
\end{note}


\begin{note}
  In contrast to \scen{1} \ref{illu:gist:roots:F} and \ref{illu:gist:roots:QF}, the relevant \fofr{} do not hold prior to the event.
  For, both conclusions are a result of (some part of) the exchange.

  Still, \fingfb{\pv{\propI{\signConB{}}}{\valI{True}}}{\(\Psi\)} is connected to \emph{whether} Fox is concluding \pv{\propI{\signConA{}}}{\valI{True}} from \(\Phi\).
  For, the following conditional plausibly holds, from \agpe{Fox's}:

  \begin{itenum}
  \item[\emph{If}:]
    They didn't open with `I come from Miran'.
  \item[\emph{Then}:]
    They're not engaging in countersign.
  \end{itenum}

  \noindent%
  Hence, phrased in terms of \fofr{1}, we have:
  %
  \begin{itenum}
  \item[\emph{If}:]
    It not the case \fofb{\pv{\propI{\signConB{}}}{\valI{True}}}{\(\Psi\)}.
  \item[\emph{Then}:]
    Fox does not conclude \pv{\propI{\signConA{}}}{\valI{True}} from \(\Phi\).
  \end{itenum}
  %
  In this respect, \fofb{\pv{\propI{\signConB{}}}{\valI{True}}}{\(\Psi\)} is relevant to understanding \autoref{scen:countS} similar to the way, e.g., \fingfb{\pv{\propM{\rootsCon{}}}{\valI{True}}}{some \pool{} which includes the \agents{} understanding of factorisation} is relevant to understanding \autoref{illu:gist:roots:F}.
  In \autoref{illu:gist:roots:F} the \fingfr{} is part of what it is for the agent to be concluding \pv{\propM{\rootsCon{}}}{\valI{True}} from the relevant \pool{}.
  And, in \autoref{scen:countS} a prior \fingfr{0} is required for Fox's overall conclusion.
\end{note}



\section{\qWhy{}, \qHow{}, and \issueInclusion{}}
\label{cha:intro:why-how}


\begin{note}
  \phantlabel{how-and-why-first-mention}%
  \qWhy{} and \qHow{} query the way an \eiw{0} an agent concludes happens:

  \begin{question}{questionWhy}{\qWhy{}}
    Given \(e\) is an \eiw{0} \vAgent{} concludes \(\pv{\phi}{v}\) from \(\Phi\):
    \begin{itemize}
    \item
      Which \fofr{1} are included in (partial) explanations of \emph{why} \(e\) is an \eiw{0} \vAgent{} concludes \(\pv{\phi}{v}\) from \(\Phi\) (rather than an \eiw{0} some other thing happens)?
    \end{itemize}
    \vspace{-1\baselineskip}
  \end{question}

  \begin{question}{questionHow}{\qHow{}}
    Given \(e\) is an \eiw{0} \vAgent{} concludes \(\pv{\phi}{v}\) from \(\Phi\):
    \begin{itemize}
    \item
      Which past or present events (partially) explain \emph{how} \(e\) is an \eiw{0} \vAgent{} concludes \(\pv{\phi}{v}\) (rather than an \eiw{0} some other thing happens)?
    \end{itemize}
    \vspace{-1.5\baselineskip}
  \end{question}
\end{note}


\begin{note}
  \qWhy{} fixes an event \(e\), and queries why it is an \eiw{0} some agent concludes \(\pv{\phi}{v}\) from \(\Phi\) rather than an event in which some other thing happens --- why \(e\) is not an \eiw{0} something incompatible with the conclusion \(\pv{\phi}{v}\) for \(\Phi\) happens.
  For example, an \eiw{0} the agent concludes \(\pv{\phi}{v}\) from some other \pool{} \(\Phi'\), takes their dog for a walk, eats some soup, etc.

  \qHow{}, by contrast, queries how \(e\) is an \eiw{0} an agent concludes \(\pv{\phi}{v}\) from \(\Phi\) came about --- what happened during \(e\)?
\end{note}

\begin{note}
  We briefly discuss the senses of `why' and `how' at issue in \qWhy{} and \qHow{}.
  Then, we turn to a constraint on answers to \qWhy{} by answers to \qHow{}.
\end{note}


\subsection{The sense of `why' at issue in \qWhy{}}


\begin{note}
  The sense of `why' at issue in \qWhy{} is roughly%
  \footnote{
    Some subtle details are expanded on in \autoref{cha:events-progress}.
  }
  captured by the following idea:

  \begin{ridea}{idea:why}{`Why'}%
    For a given event \(e\), some thing \(t\) answers why \(e\) happened just in case:
    %
    \begin{itemize}
    \item
      For some event \(e'\):
      \begin{enumerate}[label=\Alph*., ref=\Alph*]
      \item
        \label{idea:why:result}
        \(e\) is a result of \(e'\).
      \item
        \label{idea:why:favour}
        \(e'\) favours \(e\).
      \item
        \label{idea:why:feat}
        \(e'\) favours \(e\) happening only if \(t\) is the case during \(e'\).
      \end{enumerate}
    \end{itemize}
    \vspace{-\baselineskip}
  \end{ridea}

  \noindent%
  In short, some thing \(t\) answers `why' an event \(e\) happened when \(t\) being the case follows from \(e\) being favoured over some other event.
\end{note}


\begin{note}
  We first motivate \autoref{idea:why} independently of conclusions and then return to \scen{1}~\ref{illu:gist:roots:F},~\ref{illu:gist:roots:QF}, and \ref{scen:countS}.
\end{note}


\begin{note}
  Consider an \eiw[\(e\)]{0} a deck of cards is shuffled, and \mainCard{} is \mainCardPos{}.
  Our interest is with why \mainCard{} is \mainCardPos{}.

  As the deck of cards was shuffled, there is a collection of \eiw{1} the cards are being shuffled.
  Some event \(e'\) satisfies Clause~\ref{idea:why:result} of \autoref{idea:why}.
  For, \mainCard{} was shuffled to \mainCardPosX{}, hence there is some \eiw{0} \mainCard{} being shuffled to \mainCardPos{}, a results in \(e\).

  Clause~\ref{idea:why:favour} is not immediately satisfied --- setting causal determinism aside.%
  \footnote{
    \nocite{Hoefer:2023aa}%
    If causal determinism holds --- any event is the result of antecedent events together with the laws of nature --- then there is always some answer to `why' the event happened.
    For, the antecedent events and laws of nature ensure no other event may have happened.
  }
  For, during a random shuffle \mainCard{} being in \mainCardPos{} was not favoured over \mainCard{} being in any other position of the deck.
  For example, if the cards are randomly shuffled, then, intuitively, nothing favours \mainCard{} being in \mainCardPos{} after the shuffle.
  Hence, though Clause~\ref{idea:why:result} of \autoref{idea:why} is satisfied, Clause~\ref{idea:why:favour} is not satisfied.

  Still, if the shuffler is engaging in sleight of hand clauses \ref{idea:why:result}, \ref{idea:why:favour}, and \ref{idea:why:feat} are satisfied.

  The shuffle happens as a result of the agent shuffling.
  As the agent is engaging in sleight of hand, \mainCard{} being in \mainCardPos{} is favoured over \mainCard{} being in any other position.
  And, only if sleight of hand.%
  \footnote{
    One may, of course, expand on the details of the sleight --- such as what a break and shuffle is (\cite[cf.][189--190]{Hilliard:1994aa}) or the shufflers' specific method for a break and shuffle.
  }
\end{note}


\begin{note}
  The idea of a shuffler is engaging in sleight of hand may suggest determinism.
  However, some thing may answer `why' an event happened without the thing determining that the event happens.
  For example, consider rolling a biased die.%
  \footnote{
    It is hard to bias a coin flip. (\cite{Gelman:2002ww})
  }
  As biased die need not guarantee a roll which lands on 6.
  However, if an agent lands a 6 with a die biased toward 6 then the bias of the die favours an \eiw{0} any agent rolls a 6.
  Hence, so long as happens as a result of bias, then clauses \ref{idea:why:result}, \ref{idea:why:favour}, and \ref{idea:why:feat} of \autoref{idea:why} are satisfied by an \eiw[\(e'\)]{0} the die is rolling and the bias influences the roll.
  In particular:
  Clause~\ref{idea:why:result} is satisfied as the die landing on 6 is the result of the die rolling.
  Clause~\ref{idea:why:favour} is satisfied as the die is biased toward landing on 6, by assumption.
  And, Clause~\ref{idea:why:feat} is satisfied as the die is biased toward landing on 6 only if the die has bias.

  Clause~\ref{idea:why:feat} is important.
  For, consider the same die dropped from a short height to ensure it lands 6 without rolling.
  Clauses~\ref{idea:why:result} and~\ref{idea:why:favour} are seen to be satisfied by parallel reasoning.
  However, a die without bias also lands 6 if dropped from a short height to ensure it lands 6.
  Hence, the bias of the die does not explain why the die dropped from a short height to ensure it lands 6 without rolling lands 6, given~\autoref{idea:why}.

  (Though details of the drop do explain why the die lands 6.)
\end{note}


\begin{note}
  Now, \qWhy{}, specifically, seeks answers to why \(e\) is an \eiw{0} an agent concludes \(\pv{\phi}{v}\) from \(\Phi\) in terms of \fofr{1}.
  Hence, given \autoref{idea:why}, our interest is with whether there is some event \(e'\) such that:
  %
  \begin{enumerate*}[label=]
  \item
    \(e\) is the result of \(e'\),
  \item
    \(e'\) favours \(e\), and
  \item
    \(e'\) favours \(e\) happening only if a \fofr{} holds during \(e'\).
  \end{enumerate*}
\end{note}


\begin{note}
  Consider an event corresponding to Step~\ref{illu:gist:roots:F:factor} in \autoref{illu:gist:roots:F}.

  Clause~\ref{idea:why:result} is satisfied as the eventual conclusion is a result of factoring, and figuring out \rootsConEqExV{6}{3}{2} is a key step of factoring.

  And, Clause~\ref{idea:why:favour} is satisfied as an \eiw{0} the agent concludes \pv{\propM{\rootsCon{}}}{\valI{True}} is in progress when the agent figures out \rootsConEqExV{6}{3}{2}.
  Hence, the agent figuring out \rootsConEqExV{6}{3}{2} favours an \eiw{0} the agent concludes \pv{\propM{\rootsCon{}}}{\valI{True}} from the \pool{} over, e.g., an \eiw{0} the agent fails to conclude.

  Finally, Clause~\ref{idea:why:feat} is satisfied as the agent concluding \pv{\propM{\rootsCon{}}}{\valI{True}} from the relevant \pool{} (plausibly) entails \fofb{\pv{\propM{\rootsCon{}}}{\valI{True}}}{the \pool{}}.
\end{note}


\begin{note}
  In short, the relevant \fofr{0} answers \qWhy{} as it is part of what it is for the conclusion to in progress, and so to be favoured over any other thing.

  Parallel reasoning applies to the other (non-final) steps of \scen{1}~\ref{illu:gist:roots:F} and \ref{illu:gist:roots:QF}.
\end{note}


\begin{note}
  Our discussion of \autoref{scen:countS}, by contrast, highlighted \fingfb{\pv{\propI{\signConB{}}}{\valI{True}}}{some \pool{} \(\Psi\)}.
  And, Pritcher had to say something in order for Fox to conclude \pv{\propI{\signConB{}}}{\valI{True}} from \(\Phi\).
  Hence, the reasoning applied to Step~\ref{illu:gist:roots:F:factor} of \autoref{illu:gist:roots:F} does not apply here.

  Still, Fox concludes \pv{\propI{\signConA{}}}{\valI{True}} from \(\Phi\) as a result of countersign.
  And, we observed \emph{if} it is not the case \fofb{\pv{\propI{\signConB{}}}{\valI{True}}}{\(\Psi\)} \emph{then} Fox does not conclude \pv{\propI{\signConA{}}}{\valI{True}} from \(\Phi\).

  So, given \(e'\) is the \eiw{0} Fox concludes \pv{\propI{\signConB{}}}{\valI{True}} from \(\Psi\) and \(e\) is the \eiw{0} Fox concludes \pv{\propI{\signConA{}}}{\valI{True}} from \(\Phi\):

  Clause~\ref{idea:why:result} is satisfied as Fox concludes \pv{\propI{\signConA{}}}{\valI{True}} from \(\Phi\) by countersign as a result of Fox observing Pritcher's sign.
  Clause~\ref{idea:why:favour} is satisfied as Fox recognising Pritcher's sign leads Fox to continue to countersign.
  And, Clause~\ref{idea:why:feat} is satisfied as \fingfb{\pv{\propI{\signConB{}}}{\valI{True}}}{\(\Psi\)} is such that \(\Psi\) includes Fox's observation that Pritcher is engaging in countersign.
\end{note}


\subsection{The sense of `how' at issue in \qHow{}}


\begin{note}
  The sense of `how' at issue in \qHow{} concerns what happened.

  Fr example, given is an \eiw[\(e\)]{0} a deck of cards is shuffled, \mainCard{} is \mainCardPos{}, and the shuffler is engaging in sleight of hand, \qHow{} queries the way the shuffler performed slight of hand, which cards where moved where, etc.

  Likewise, in the case of a biased die \qHow{} queries the way the die landed, when the bias of the die took effect, and so on.

  \scen{3} \ref{illu:gist:roots:F}, \ref{illu:gist:roots:QF}, and \ref{scen:countS} may all be seen as providing answers to \qHow{} in this respect.
  For, each \scen{} provides a description of what happened during an event.
  That is, to understand `how' the agent concluded \pv{\propM{\rootsCon{}}}{\valI{True}} from some \pool{} which includes their understanding of factorisation in \autoref{illu:gist:roots:F} we need an account of the way the event happened, and the description of the event provides this by breaking down the \agents{} reasoning in a series of steps.
\end{note}



\subsection{A constraint on answers to \qWhy{} in terms of answers to \qHow{}}


\begin{note}
  Consider the following constraint on answers to \qWhy{} in terms of answers to \qHow{}:

  \begin{constraint}{consInclusion}{\issueInclusion{}}
    \mbox{ }
    \vspace{-\baselineskip}
    \begin{itenum}
    \item[\emph{If}:]
      \fingfb{\(\pv{\psi}{v'}\)}{\(\Psi\)} answers \qWhy{}.
    \item[\emph{Then}:]
      An \eiw{0} the agent concludes \(\pv{\psi}{v'}\) from \(\Psi\) answers \qHow{}.
    \end{itenum}
    \vspace{-\baselineskip}
  \end{constraint}

  \noindent%
  In short, \issueInclusion{} states that for any \fingfr{} between a \prop{0}, \val{0}, and \pool{} which grants some understanding of the way an agent concluded \(\pv{\phi}{v}\) from \(\Phi\), there is an event such that the agent concludes the \prop{} has the \val{} from the \pool{} past or present to the \agents{} conclusion of \(\pv{\phi}{v}\) from \(\Phi\).%
  \footnote{
    Note, \qHow{} does not explicitly require the relevant event to be the \eiw{0} the agent concludes \(\pv{\phi}{v}\) from \(\Phi\).
    Hence, a previous \eiw{0} the agent concludes \(\pv{\psi}{v'}\) from \(\Psi\) may answer \qHow{}.
    \issueInclusion{} is a constraint, and hence is compatible with a more stringent constraint.
    Still, I have no interest in motivating a more stringent constraint, and we will argue against \issueInclusion{} we obtain a slightly stronger overall result.
  }
\end{note}


\begin{note}
  I think \issueInclusion{} is intuitive.
\end{note}


\begin{note}
  For broad motivation, consider sleight of hand as discussed above.

  The sleight of hand partly explains why \mainCard{} is \mainCardPos{}.
  Likewise, the sleight of hand explains how \mainCard{} is \mainCardPos{} after the shuffle.
  So, what answers how \mainCard{} was in \mainCardPos{} also answers why \mainCard{} is \mainCardPos{}.
  And, it is plausible nothing other than sleight of hand answers why \mainCard{} was in \mainCardPos{}.
  For, without engaging in sleight of hand it seems \mainCard{} being \mainCardPos{} was not favoured over \mainCard{} being in any other position of the deck (or even a shuffle taking place at all).
\end{note}


\begin{note}
  For narrow motivation, we observe \issueInclusion{} is compatible with the \fofr{1} highlighted with respect to \scen{1}~\ref{illu:gist:roots:F},~\ref{illu:gist:roots:QF}, and \ref{scen:countS}.

  With \scen{1}~\ref{illu:gist:roots:F}~and~\ref{illu:gist:roots:QF} our interest is with \fingfb{\pv{\propM{\rootsCon{}}}{\valI{True}}}{some \pool{} which includes the \agents{} understanding of factorisation or the quadratic formula}.
  And, the agent concludes \pv{\propM{\rootsCon{}}}{\valI{True}} from a \pool{} which includes the \agents{} understanding of factorisation or the quadratic formula.

  And, with \autoref{scen:countS} our interest is with \fingfb{\pv{\propI{\signConB{}}}{\valI{True}}}{some \pool{} \(\Psi\)}.
  And, Fox had concluded \pv{\propI{\signConB{}}}{\valI{True}} from \(\Psi\) when Fox concludes \pv{\propI{\signConA{}}}{\valI{True}} from some \pool{} \(\Phi\).
\end{note}


\begin{note}
  To get a feel for what \issueInclusion{} rules out, consider \autoref{illu:gist:roots:F} and \fingfb{\pv{\propM{\rootsCon{}}}{\valI{True}}}{some \pool{} \(\Phi'\) which includes the \agents{} understanding of the quadratic formula}.
  We suggested it may be the case \fingfb{\pv{\propM{\rootsCon{}}}{\valI{True}}}{\(\Phi'\)} answers `why' the agent concluded \pv{\propM{\rootsCon{}}}{\valI{True}} from \(\Phi'\).
  \issueInclusion{} entails a \fingfb{\pv{\propM{\rootsCon{}}}{\valI{True}}}{\(\Phi'\)} \emph{does not} answer `why' the agent concluded \pv{\propM{\rootsCon{}}}{\valI{True}} from their understanding of factorisation.

  And, this seems sensible.
  For, it seems implausible that the \agents{} understanding of the quadratic formula favoured an \eiw{0} the agent concluded \pv{\propM{\rootsCon{}}}{\valI{True}} from their understanding of factorisation over any other event --- the agent did not conclude by the quadratic formula.
\end{note}


\begin{note}
  Still, we have only considered a few \scen{1} and \issueInclusion{} is a general constraint.
  Hence, if there is some doubt regarding \issueInclusion{} then further argument is required.
  However, I am not aware of any arguments for or against \issueInclusion{}.%
  \footnote{
    The construction of \qWhy{}, \qHow{}, and \issueInclusion{} is somewhat idiosyncratic, and so it is no surprise there are no direct argument for \issueInclusion{}.
    However, I take the idea captured by \issueInclusion{} to be intuitive, and I am not aware of any argument for this idea either.
  }
  Well, that is, except for the argument against \issueInclusion{} developed in this document.%
  \footnote{
    An idea similar to \issueInclusion{} may be argued for with respect to \emph{\rationalisation{1}}.
  Still, \rationalisation{3} explain an action done by an agent by giving the \agents{} reason for doing the action.
  (\cite[685]{Davidson:1963aa})
  Hence, any account of \rationalisation{1} only details the kind of explanation which holds between an \agents{} reason and the action of interest.

  In the case of conclusion of \(\pv{\phi}{v}\) from \(\Phi\), the relevant action is an \evalion{} \(\phi\) having \val{} \(v\).
  Now, it may be the case that \(\Phi\) is the \agents{} reason for \evaling{} \(\phi\) as having \val{} \(v\).
  Hence, an account of \rationalisation{1} may ensure the agent has concluded \(\pv{\phi}{v}\) from \(\Phi\) --- and, indeed, accounts given by \textcite{Davidson:1963aa}, \textcite{Hieronymi:2011aa}, and \textcite{Harman:1973ww} have this implication.
  Though, this is of little interest as we are talking about a case in which the agent concluded \(\pv{\phi}{v}\) from \(\Phi\).

  Further, an account of \rationalisation{1} does not extend to any relation between \(\pv{\psi}{v'}\) and \(\Psi\), where either \(\psi\) is distinct from \(\phi\) or \(v'\) is distinct from \(v\).
  For, the relevant act is an \evalion{} \(\phi\) having \val{} \(v\).
  So, it is not possible to substitute \(\psi\) for \(\phi\) or \(v'\) for \(v\) while preserving the act.

  In other words, part of what makes \fingfr{1} as answers to \qWhy{} somewhat interesting is the possibility that a \fingfr{1} which answers \qWhy{} has no direct connection to the relevant conclusion.
  And, \rationalisation{1}, while interesting in various other ways, are not sufficiently expressive to capture this possibility.

  A similar issue holds with respect to basing relations, where a belief is fixed.
  (cf.\ \cite{Korcz:2021ue}).
  However, as basing relations concern justification rather than explanation (\cite[cf.][35]{Pollock:1999tm}) these fall outside the scope of our interest.
  }
\end{note}



\section{Counterexamples to \issueInclusion{}}


\begin{note}
  The goal is to provide resources to identify counterexamples to \issueInclusion{}, given a general understanding of the way an \eiw{0} an agent concludes happens.
  The following observation characterises what is required for a counterexample:

  \begin{observation}[Failures to \issueInclusion{}]%
    \label{obs:iIceRestriction}%
    \vspace{-\baselineskip}
    \begin{itenum}
    \item[\emph{If}:]
      \issueInclusion{} fails to hold.
    \item[\emph{Then}:]
      There is some event \(e\) such that:
      \begin{enumerate}
      \item
        \(e\) is an \eiw{0} an agent concludes \(\pv{\phi}{v}\) from \(\Phi\).
      \item
        There is some \fofr{} between \(\pv{\psi}{v'}\) and \(\Psi\) such that conditions \ref{reciF:qWhy}, \ref{reciF:distinct}, and \ref{reciF:nHow} hold:
        %
        \begin{enumerate}[label=\Alph*., ref=\Alph*]
        \item
          \label{reciF:qWhy}
          \fingfb{\(\pv{\psi}{v'}\)}{\(\Psi\)} answers \qWhy{}.
        \item
          \label{reciF:distinct}
          Either \(\psi\) is distinct from \(\phi\), \(v'\) is distinct from \(v\), or \(\Psi\) is distinct from \(\Phi\).
        \item
          \label{reciF:nHow}
          There is no prior (or present) \eiw{0} the agent concludes \(\pv{\psi}{v'}\) from \(\Psi\).
        \end{enumerate}
      \end{enumerate}
    \end{itenum}
    \vspace{-1.5\baselineskip}
  \end{observation}
  %
  \begin{motivation}{obs:iIceRestriction}
    Suppose \issueInclusion{} fails to hold.
    We observe each condition holds.

    \begin{enumerate}[label=\arabic*., ref=\arabic*]
    \item
      \qWhy{} concerns an event in which an agent concludes, hence it must be the case that for some \prop{0}-\val{0} pair \(\pv{\phi}{v}\) and \pool{} \(\Phi\) we have an event in which an agent concludes \(\pv{\phi}{v}\) from \(\Phi\).
    \item
      In turn, for some fixed \fofr{} between \(\pv{\psi}{v'}\) and \(\Psi\):
      \begin{enumerate}[label=\Alph*., ref=\Alph*]
      \item
        \issueInclusion{} constrains answers to \qWhy{} in terms of answers to \qHow{}.
        So, in order for \issueInclusion{} to fail, \fingfb{\(\pv{\psi}{v'}\)}{\(\Phi\)} must answer \qWhy{}.
      \item
        If \(\psi\) is the same as \(\phi\), \(v'\) is the same as \(v\), and \(\Psi\) is the same as \(\Phi\) then the \eiw{0} the agent concludes \(\pv{\phi}{v}\) from \(\Phi\) is an \eiw{0} the agent concludes \(\pv{\psi}{v'}\) from \(\Psi\).
        And, the \eiw{0} the agent concludes \(\pv{\phi}{v}\) from \(\Phi\) answers \qHow{}.
        Hence, an \eiw{0} the agent concludes \(\pv{\psi}{v'}\) from \(\Psi\) answers \qHow{}.
        This contradicts the failure of \issueInclusion{}.
        So, at least on of these identifies must fail.
      \item
        Finally, \qHow{} is compatible with \scen{1} where some prior \eiw{0} the agent concludes \(\pv{\psi}{v'}\) from \(\Psi\) answers \qHow{}.
        Whether there are \scen{1} with this feature is something we set aside.
        Still, a counterexample to \issueInclusion{} is not guaranteed if there is some prior \eiw{0} the agent concludes \(\pv{\psi}{v'}\) from \(\Psi\).
      \end{enumerate}
    \end{enumerate}
    \vspace{-\baselineskip}
  \end{motivation}
\end{note}


\begin{note}
  Still, our goal is to show \issueInclusion{} fails to hold for a range of \scen{1}, rather than any specific \scen{1}.
  So, strictly speaking, the thesis is:
  %
  \begin{quote}
    \emph{If} such-and-such features hold of a \scen{} \emph{then} the \scen{} is incompatible with \issueInclusion{}.
  \end{quote}
  %
  In other words, the goal of this document is to show that any \scen{0} which satisfies a collection of features entails \issueInclusion{} does not hold.

  Of some interest, then, is whether \issueInclusion{} actually fails --- whether there really are events where \fingfb{\(\pv{\psi}{v'}\)}{\(\Psi\)} answers \qWhy{} without an \eiw{0} the agent concludes \(\pv{\psi}{v'}\) from \(\Psi\).%
  \footnote{
    Though a complete argument that such \scen{1} exist is beyond the scope of this thesis.
    For example, a corollary of such an argument would proof of an external world.

    And, it may be the case that the \scen{1} identified are understood to exist, and \issueInclusion{} fails to hold given this understanding, though it turns out that what's really going on in each \scen{0} identified is compatible with \issueInclusion{} ---- scepticism is quite powerful.
  }
  A few specific \scen{1} will be given.
  In particular, I argue there is a plausible reading of \autoref{illu:gist:roots:F} for which \issueInclusion{} fails to hold.
\end{note}


\begin{note}
  The broad idea is as follows:

  A \fingfr{} answers \qWhy{} when the \fingfr{} partly explains why an \eiw{0} an agent concludes \(\pv{\phi}{v}\) from \(\Phi\) happens, rather than some other event.
  Given \autoref{idea:why}, a \fingfr{} answers \qWhy{} just in case the \fingfr{} holding is a key part of some event which both:
  \begin{itemize}
  \item
    Favours the \agents{} conclusion of \(\pv{\phi}{v}\) from \(\Phi\).
  \item
    Results in the \agents{} conclusion of \(\pv{\phi}{v}\) from \(\Phi\).
  \end{itemize}
\end{note}


\begin{note}
  Consider an \eiw{0} an agent concludes \(\pv{\phi}{v}\) from some \pool{} \(\Phi\).
  Further, suppose there is an \eiw{0} the agent is \emph{concluding} \(\pv{\phi}{v}\) from \(\Phi\).
  And, suppose the agent concludes \(\pv{\phi}{v}\) from \(\Phi\) as a result of whatever happened when the agent was concluding \(\pv{\phi}{v}\) from \(\Phi\).

  As motivated above, an agent concluding \(\pv{\phi}{v}\) from \(\Phi\) plausibly involves \fingfb{\(\pv{\phi}{v}\)}{\(\Phi\)}, and, in turn, the \fingfr{} answers \qWhy{}.
  %
  \begin{itemize}
  \item
    For example, the event captured by Step~\ref{illu:gist:roots:F:factor} in \autoref{illu:gist:roots:F}.
    When the agent figures out \rootsConEqExV{6}{3}{2}, the agent is concluding \pv{\propM{\rootsCon{}}}{\valI{True}} from their understanding of factorisation.
    And, the eventual conclusion is a result of figuring out \rootsConEqExV{6}{3}{2}.
    So, \fingfb{\pv{\propM{\rootsCon{}}}{\valI{True}}}{a \pool{} which includes the \agents{} understanding of factorisation} answers \qWhy{}, as the \agents{} eventual conclusion is favoured over any \eiw{0} the agent does not conclude \pv{\propM{\rootsCon{}}}{\valI{True}} from the relevant \pool{}.
  \end{itemize}
  %
  Still, it may be the case the agent is concluding \(\pv{\phi}{v}\) from \(\Phi\) \emph{only if} \fofb{\(\pv{\psi}{v'}\)}{\(\Psi\)}.
  In this respect, \fingfb{\(\pv{\psi}{v'}\)}{\(\Psi\)} is part of what it is for an \eiw{0} the agent concludes \(\pv{\phi}{v}\) from \(\Phi\) to be favoured over some other event.
  %
  \begin{itemize}
  \item
    For example, consider Step~\ref{illu:gist:roots:F:factor} again.
    \pv{\propM{\rootsConEqExV{12}{4}{3}}}{\valI{True}} plausibly follows from a sufficient understanding of factorisation.
    So, I doubt it is possible the agent is concluding \pv{\propM{\rootsCon{}}}{\valI{True}} from their understanding of factorisation \emph{unless} \fofb{\pv{\propM{\rootsConEqExV{12}{4}{3}}}{\valI{True}}}{a \pool{} which includes the \agents{} understanding of factorisation}.
  \end{itemize}
  %
  In other words, \fingfb{\(\pv{\psi}{v'}\)}{\(\Psi\)} is part of what it is for the agent to be concluding \(\pv{\phi}{v}\) from \(\Phi\).
  And, as the \eiw{0} the agent is concluding \(\pv{\phi}{v}\) from \(\Phi\) favours a conclusion of \(\pv{\phi}{v}\), and the conclusion is the result of the concluding, \fingfb{\(\pv{\psi}{v'}\)}{\(\Psi\)} answers \qWhy{}.
  The only remaining task is to ensure the agent has not concluded \(\pv{\psi}{v'}\) from \(\Psi\), but instances are not too difficult to find.
  %
  \begin{itemize}
  \item
    For example, the agent may have yet concluded \pv{\propM{\rootsConEqExV{12}{4}{3}}}{\valI{True}} from their understanding of factorisation.
    And, if the agent has, consider \propI{\rootsConEqExV{20}{5}{4}}, and so on\dots
  \end{itemize}
  %
  This, is more-or-less it.

  Sometimes an agent concludes as a result of concluding in a particular way (e.g.\ by factorisation), and if the agent is concluding in a particular way, then various distinct \fingfr{1} are part of what it is to conclude in that particular way, and hence these answer \qWhy{}.
  Put otherwise, sometimes an agent concludes as a result of concluding in a particular way and the agent is concluding in a particular way \emph{only if} various distinct \fingfr{1} are part of what it is to conclude in that particular way.
  Hence, given \autoref{idea:why}, those \fingfr{} answer \qWhy{}.

  I think \issueInclusion{} is intuitive, though I also think this argument is basic enough to highlight that the intuition isn't quite right.

  Most of the work to be done is setting things up so that a precise argument is given which does not lead to the various objects that may be raised against this sketch.
\end{note}


\begin{note}
  With minor adjustments the above argument may be made in terms of ability, know-how, disposition etc.
  These concepts all involve something which extends beyond any instance.
  For example, an ability to use the quadratic formula extends beyond a single use of the formula, the know-how to countersign extends beyond a particular sequences of signing, and the disposition to factor involves factoring under a variety of stimulus conditions.
  We work with \fingfr{1} as details about ability, know-how, or dispositions are irrelevant to the argument.
\end{note}

\section{Summary and overview}
\label{sec:conclusion}

\begin{note}
  Our interest is understanding the way an \eiw{0} an agent concludes happens.
  In particular, our interest is with whether the constraint on answers to \emph{why} an event happened in terms of answers to \emph{how} the event happened holds.
\end{note}


\begin{note}
  This chapter introduced the way we understand \eiw{0} an agent concludes, questions \qWhy{} and \qHow{}, and the constraint \issueInclusion{}.
  We sketched some motivation for \issueInclusion{} and then the general form counterexamples to \issueInclusion{} take.
\end{note}


\begin{note}
  The remainder of this document amounts to working through a handful of ideas fairly carefully.
  Important ideas are applied to \autoref{illu:gist:roots:F} as worked through.
  So, we identify a counterexample to \issueInclusion{} throughout the document.
\end{note}


\begin{note}
  \autoref{cha:events-progress} expands on \autoref{idea:why} to state sufficient conditions for answers to \qWhy{}.

  Specifically, we split \autoref{idea:why} into two ideas:
  An \se{} event and \progEx{}.

  To help keep things simple we have talked about events in which an agent concludes.
  However, when turning to the details, our interest it is useful to distinguish between events and (true) descriptions of events.

  With a distinction between events and descriptions in hand, \(\edn{\flat}\) under description \(\edo{\flat}\) is an \emph{\se{}} of \(\edn{}\) under description \(\edo{}\) just in case:

  \begin{enumerate}[label=\arabic*., ref=\arabic*]
  \item
    \label{sketch:se:r}
    \(\edn{}\) under \(\edo{}\) is an result of \(\edn{\flat}\) under \(\edo{\flat}\).
  \item
    \label{sketch:se:p}
    \(\edn{\flat}\) under \(\edo{\flat}\) is such that \(\edn{}\) under \(\edo{}\) is in progress.
  \end{enumerate}
  %
  \ref{sketch:se:r} follows Clause \ref{idea:why:result} of \autoref{idea:why}, and \ref{sketch:se:p} (we argue) entails clause \ref{idea:why:favour}.%
  \footnote{
    I.e., \(\edn{\flat}\) under \(\edo{\flat}\) favours \(\edn{}\) under \(\edo{}\) due to it being the case \(\edn{}\) is in progress.
  }

  In turn, \progEx{} expands on Clause \ref{idea:why:feat} of \autoref{idea:why}.
  Roughly, \progEx{} states that whatever is captured of \(\edn{\flat}\) by description \(\edo{\flat}\) is a (partial) explanation of `why' an agent concluded \(\pv{\phi}{v}\) from \(\Phi\).

  The remainder of the document amounts to arguing certain descriptions capture \fingfr{1}, and some such \fingfr{1} are incompatible with \issueInclusion{}.

  Chapters~\ref{cha:fcs}, \ref{cha:ros}, and \autoref{cha:requs} establish the way a description of an event captures a \fingfr{}.
  Following, \autoref{cha:requs} outlines a way to obtain \fingfr{1} which may amount to counterexamples to \issueInclusion{}, and \autoref{cha:ces} provides a pair of detailed counterexamples.
\end{note}

\begin{note}
  In a little more detail, chapters~\ref{cha:fcs} and \ref{cha:ros} focus on identifying \fingfr{}.

  Specifically, \autoref{cha:fcs} introduces the idea of \(\pv{\psi}{v'}\) being a \fc{} from \(\Psi\) for an agent.
  Where, roughly, \(\pv{\psi}{v'}\) is a \fc{0} from \(\Psi\) just in case the agent may do some action and be concluding \(\pv{\psi}{v'}\) from \(\Psi\).
  For example, \pv{\rootsCon{}}{\valI{True}} was plausibly a \fc{} from a \pool{} which included the \agents{} understanding of factorisation \emph{prior to} \autoref{illu:gist:roots:F}, as the agent plausibly was concluding \pv{\rootsCon{}}{\valI{True}} from the \pool{} when the agent being their reasoning.

  In turn, \autoref{cha:ros} states:
  %
  \begin{itemize}
  \item
    A conclusion of \(\pv{\phi}{v}\) from \(\Phi\) is sufficient for \(\pv{\phi}{v}\) to \fof{} \(\Phi\) from the \agpe{} when the agent concludes \(\pv{\phi}{v}\) from \(\Phi\).
  \item
    \(\pv{\psi}{v'}\) being a \fc{} from \(\Psi\) is sufficient for \(\pv{\phi}{v}\) to \fof{} \(\Phi\) from the \agpe{}.
  \end{itemize}
  %
  So, if a description \(\edo{\flat}\) of an event \(\edn{\flat}\) entails a \fc{0}, \(\edo{\flat}\) also entails a corresponding \fingfr{}.
  And, if \(\edo{\flat}\) captures a \se{} of an event in which an agent concludes, the \fingfr{} answers \qWhy{}.

  A pair of quick notes may be helpful here:

  \begin{itemize}
  \item
    I expect you to have a somewhat intuitive grasp on the `\fof{}' relation and \fc{1} hopefully identify some instances compatible with your intuitive grasp.
  \item
    When we speak of \fc{1} our interest is with whatever it is that makes it the case \(\pv{\psi}{v'}\) is a \fc{} from \(\Psi\), rather than the (foregone-)conclusion of \(\pv{\phi}{v}\) from \(\Phi\).
    That is, whatever it is that makes it the case \(\pv{\psi}{v'}\) is a \fc{} from \(\Psi\) amounts to something which makes it the case that \(\pv{\psi}{v'}\) `follows from' \(\Psi\), from the \agpe{}.
  \end{itemize}
\end{note}


\begin{note}
  \autoref{cha:requs} details the way \fc{1}, \fingfr{1}, and \progEx{} interact to provide sufficient conditions for an answer to \qWhy{}.

  Here, we identify a few answers to \qWhy{} which are compatible with \issueInclusion{}.

  So, taken as a unit, chapters \autoref{cha:intro} to \autoref{cha:requs} detail a way to find answers to \qWhy{}, and the answers we find are compatible with \issueInclusion{}.
\end{note}


\begin{note}
  \autoref{cha:typical} introduces the idea an agent \tCV{} to help identify answers to \qWhy{} which are not compatible with \issueInclusion{}.

  The basic idea of an agent \tCV{} \(\pv{\phi}{v}\) from \(\Phi\) is that there is some generality to the \agents{} reasoning.
  As noted above, with respect to \autoref{illu:gist:roots:F}, \pv{\propM{\rootsConEqExV{12}{4}{3}}}{\valI{True}} plausibly follows from a sufficient understanding of factorisation.
  In turn, the agent may do some action any be concluding \pv{\propM{\rootsConEqExV{12}{4}{3}}}{\valI{True}} from a \pool{} which captures their understanding of factorisation.
  By the ideas of chapters~\ref{cha:fcs} and \ref{cha:ros} this identifies a \fingfr{}, and so long as the agent has not concluded \pv{\propM{\rootsConEqExV{12}{4}{3}}}{\valI{True}} from the relevant \pool{}, \issueInclusion{} fails to hold.
\end{note}


\begin{note}
  Finally, \autoref{cha:ces} details counterexamples to \issueInclusion{}.
  In particular, we highlight in detail the way ideas as applied to \autoref{illu:gist:roots:F} show \issueInclusion{} fails to hold.
  And, with some luck you, yourself, create a counterexample to \issueInclusion{}.
\end{note}

\begin{note}
  The argument is more about the way a handful of ideas interact and apply to various \scen{1} rather than any single idea.

  To help separate the main line of argument about the interaction between various ideas from detailed arguments and applications of ideas the argument is built from `definitions', `ideas', `assumptions', and `propositions'.

  Definitions, ideas, and assumptions are stated and motivation follows in an unstructured way, while propositions are always accompanied by a corresponding argument.

  In addition, `applications' and `observations' are included.
  Applications are used to highlight the way a particular definition or idea applies to a \scen{} and a handful of applications are used to highlight a failure of \issueInclusion{} (with respect to \autoref{illu:gist:roots:F}) while observations amount to notes which may be helpful but can be safely ignored.
  Like propositions, applications and observations come with marked motivation.

  You are encouraged to read the statement of a proposition, application, or observation and then move on if you'd prefer not to work through the details.

  Also, sometimes sections are marked as `optional'.
  This is meant sincerely.
  Nothing in a section marked as optional is mentioned in the main line of argument.
\end{note}

\begin{note}
  Though we only provide a few explicit counterexamples to \issueInclusion{}, these counterexamples are obtained by applying general ideas developed to specific \scen{1}.
  So, by the close of this document I hope to have given you the resources to find other failures of \issueInclusion{}.
\end{note}



\paragraph*{A note on reading this document}

\begin{note}
  The argument of this document is fairly tightly connected.
  Ideas introduced and definitions made in earlier chapters are often used to introduce further ideas or make additional definitions in later chapter.
  And, propositions argued for often build on prior propositions.

  If you are reading this as a PDF every reference to something stated in this document is hyperlinked.
  For example, clicking on --- \autoref{illu:gist:roots:F} --- takes you to the statement of \autoref{illu:gist:roots:F}, and clicking on --- \qWhy{} --- takes you to the statement of \qWhy{}.

  I have also tried to include some page references where I think they may be helpful for anyone reading without access to hyperlinks (or a preference not to use them).

  And, in addition, \autoref{cha:re} collects together a few definitions, ideas, and propositions referred back to a number of times after being introduced.
  So, if you have a print-out I recommend setting \autoref{cha:re} to one side for easy reference, and if you have a PDF opening a second copy of this document in a separate window to \autoref{cha:re} will hopefully ease the reading experience.
\end{note}



%%% Local Variables:
%%% mode: latex
%%% TeX-master: "master"
%%% TeX-engine: luatex
%%% End:
