\chapter{Introduction}
\label{cha:introduction}

\section{History}

\begin{note}
  Started with idea about premises.

  Distinction between representation and content of representation.

  Fix something like a representation, but no content.

  One way to think about this, inchoate.
\end{note}

\begin{note}
  Key idea is something missing.
  Whatever relation typically holds between perspective of agent and action, weaken this.
\end{note}

\begin{note}
  Idea is hard to argue for.
\end{note}

\begin{note}
  Different approach.
  Relation holds in cases where agent has not done the think which typically characterises the relation.
\end{note}

\section{Starting}
\nocite{Brown:2004us}

\begin{note}
  Start with a instance of reasoning.

  Logic problem.

  General ability.
  Instance of general ability.
  Specific ability.
  So \dots

  As an argument, somewhat difficult.
  Specific only if general.
  So, no going from premises if don't already have the conclusion.

  Warrant transmission failure.
  Looks like this works to me.

  However, as an instance of reasoning, seems okay.
  Drawing out the consequences of general ability.

  No warrant transmission.
  Link between general and specific.
  No warrant for general without warrant for specific.

  This falls short of an argument.
  Just highlights a possible way out.
\end{note}

\begin{note}
  Still, something less clear.
  Opportunity.
  Specific ability.

  Test whether really have the specific ability.

  But, also, specific and specific.
\end{note}

\begin{note}
  More general argument.
  Cases in which reasoning applies equally to premise-conclusion pairing have no witnessed from.

  This is the main focus of this document.
\end{note}

%%% Local Variables:
%%% mode: latex
%%% TeX-master: "master"
%%% End:
