\chapter{Positive answers to \qzS{}}
\label{cha:zSpA}

\nocite{Scriven:1962vq}
\nocite{Woodward:2021ue}
\nocite{Perry:1979vc}
\nocite{Perry:1986aa}

\begin{note}
  \autoref{cha:zS} introduced a question, \qzS{}, and argued that there are cases in which either a positive or negative answer to the question answers, in part, why an agent concluded \(\phi\) has value \(v\).

  In this chapter, we argue for a proposition regarding position answers to \qzS{}.
  Specifically, the proposition highlights a necessary condition on positive answers to \qzS{}.

  The proposition is as follows:

  \begin{restatable}[Potential events]{proposition}{propPotentialEvents}
    \label{prop:PWEs}
    For an agent \vAgent{}, when concluding \(\pv{\phi}{v}\) from \(\Phi\):

    \begin{itemize}
    \item[]
      For any thing \(\qzSaV{}\):
      \begin{enumerate}[label=\alph*., ref=(\alph*)]
      \item
        \label{prop:PWEs:a}
        \(\qzSaV{}\) is a positive answer to \qzS{}, from \vAgent{}' perspective.
      \end{enumerate}
      \begin{itemize}
      \item[\emph{Only if}]
        For any proposition-value-premises pairing \(\pvp{\psi}{v'}{\Psi}\):
        \begin{itemize}
        \item[\emph{If}]
          \begin{enumerate}[label=\alph*., ref=(\alph*), resume]
          \item
            \label{prop:PWEs:b}
            \(\pvp{\psi}{v'}{\Psi}\) is a \requ{} of \vAgent{} concluding \(\pv{\phi}{v}\) from \(\Phi\).
          \end{enumerate}
        \item[\emph{then}]
          \begin{enumerate}[label=\alph*., ref=(\alph*), resume]
          \item
            \label{prop:PWEs:c}
            \(\qzSaV{}\) involves, in part, there being a potential event in which \vAgent{} concludes \(\pv{\psi}{v'}\) from \(\Psi\).
          \end{enumerate}
        \end{itemize}
      \end{itemize}
    \end{itemize}
    \vspace{-\baselineskip}
  \end{restatable}
\end{note}

\begin{note}
  The consequent here, potential event.
  Term this a \fc{}.

  \begin{restatable}[Foregone-conclusions]{definition}{definitionForegoneC}
    For any proposition-value-premises pairing \(\pvp{\psi}{v'}{\Psi}\):

    From \vAgent{}' perspective:
    \begin{itemize}
    \item
      \(\pv{\phi}{v}\) is a \emph{\fc{0}} from some pool of premises \(\Phi\)
    \end{itemize}
    \emph{If and only if}
    \begin{itemize}
    \item
      If the agent were to reason, would conclude.
    \end{itemize}
  \end{restatable}

  Note, \fc{} are more general.
  If \requ{}, then get \fc{}.
  However, \fc{} in general, is not limited to a \requ{}.
\end{note}

\begin{note}
  \autoref{cha:zSpA:sec:positive-answers-qzs}, clarify the proposition.
  And, offer \emph{a} simple argument.

  \autoref{prop:PWEs} key going forward.
  However, second focus of this chapter is argument.
  Simple argument, flaw.

  \autoref{cha:zSpA:sec:difficulty}, difficulty.
  Motivate the difficulty.
  \autoref{cha:zSpA:sec:the-argument} provide \emph{the} argument for \autoref{prop:PWEs}.

  Following, \autoref{cha:zSpA:sec:interpretation}, interpretation.

  \autoref{cha:zSpA:sec:misc-issues}, some miscellaneous issues.

  Finally, \autoref{cha:zSpA:sec:ptivity-question}, difficulty raised in \autoref{cha:zSpA:sec:difficulty} to \qzS{} itself.
\end{note}

\begin{note}
  In following chapter, expand on this with \fc{1}, and relations of support.
\end{note}

\begin{note}
  Consider a difficulty.
  Refined argument.

  Difficulty is optional.
  However, clarifies function of the proposition.
  In other words, that objection is part of the interest.
  Distinction raised by objection, and where proposition falls is the other part.
\end{note}

\begin{note}
  Details matters.

  Key point of interest is propositions.
  Arguments matter.

  Present first proposition.
  Then second.
  Argument for second.
  Expand on second.

  Argument for the first in detail.
  Here, make use of ideas from the argument for the second.
\end{note}

\section{Outline}
\label{cha:zSpA:sec:outline}

\begin{note}
  \begin{itemize}
  \item
    The point is, \requ{1} for any general ability, and these are also \requ{1} for main pairing.
    (%
    Note --- or perhaps emphasise --- here, that the problem is \emph{not} recursive.
    Instead, the problem is about the spread.%
    )
  \end{itemize}
\end{note}

\section{Potential events}
\label{cha:zSpA:sec:positive-answers-qzs}

\begin{note}

  When concluding \(\pv{\phi}{v}\) from \(\Phi\), an instance of \qzS{} has a positive answer only if, for any \requ{} \(\pvp{\psi}{v'}{\Psi}\) of concluding \(\pv{\phi}{v}\) from \(\Phi\), from the agent's perspective, the agent would conclude \(\pv{\psi}{v'}\) from \(\Psi\).

  Expanding, the following \hyperref[prop:PWEs]{proposition}:

  \propPotentialEvents*

  \autoref{prop:PWEs}, takes an arbitrary positive answers to \qzS{}, \(\qzSaV{}\), and expresses a \emph{necessary} condition on \(\qzSaV{}\).%
  \footnote{
    Otherwise expressed, \autoref{prop:PWEs} has the following form:

    \begin{quote}
      For an agent \vAgent{}, when concluding \(\pv{\phi}{v}\) from \(\Phi\):
      \begin{quote}
        For any thing \(\qzSaV{}\), (\hyperref[prop:PWEs:a]{a} \emph{only if} (for any \(\pvp{\psi}{v'}{\Psi}\), \emph{if} \hyperref[prop:PWEs:b]{b} \emph{then} \hyperref[prop:PWEs:c]{c}))
      \end{quote}
    \end{quote}
    Alternatively, \autoref{prop:PWEs} may be reformulated in terms of a single conditional by shifting the internal quantifier to scope over the both conditionals, and appealing to \emph{import-export}:
    \begin{quote}
      \begin{quote}
        For any thing \(\qzSaV{}\), and for any \(\pvp{\psi}{v'}{\Psi}\), ((\hyperref[prop:PWEs:a]{a} and \hyperref[prop:PWEs:b]{b}) \emph{only if} \hyperref[prop:PWEs:c]{c})
      \end{quote}
    \end{quote}
  }

  Hence, \autoref{prop:PWEs} states that in order for some thing \(\qzSaV{}\) to be a positive answer to an instance of \qzS{}, \(\qzSaV{}\) involves, in part, a potential event in which \vAgent{} concludes \(\pv{\psi}{v'}\) from \(\Psi\).
\end{note}

\begin{note}
  Corollary, potential event is part of why.
\end{note}

\begin{note}
  Two minor points of clarification, and then why \autoref{prop:PWEs} is difficult.
  Following, argument for \autoref{prop:PWEs}.

  Two minor points of clarification are:
  Necessary condition.
  Potential event.
\end{note}

\begin{note}
  First, the form of \autoref{prop:PWEs} as a necessary condition has syntactic and dialectical motivation.

  The syntactic motivation is straightforward:
  \begin{shiftpar}
    \ref{prop:PWEs:b} and \ref{prop:PWEs:c} are within the scope of quantification over all proposition-value-premises pairings, with \ref{prop:PWEs:b} serving to restrict instances of \ref{prop:PWEs:c} to pairings which are \requ{}.
    And, we have observed that, for an agent, there may be more than one \requ{} of concluding \(\pv{\phi}{v}\) from \(\Phi\).
    Therefore, it is not possible for the potential event of \ref{prop:PWEs:c} to, in general, be identified with \(\qzSaV{}\).%
    \footnote{
      For example, suppose two \requ{1}.
      \(\pvp{\psi}{v'}{\Psi}\) and \(\pvp{\theta}{v''}{\Theta}\).
      A potential event in which \vAgent{} concludes \(\pv{\psi}{v'}\) from \(\Psi\) (trivially) involves a potential event in which \vAgent{} concludes \(\pv{\psi}{v'}\) from \(\Psi\).
      However, such a potential event need not involve the agent concluding \(\pv{\theta}{v''}\) from \(\Theta\).
    }
  \end{shiftpar}

  The dialectical motivation is likewise straightforward:

  \begin{shiftpar}
    We will have no interest in whether or not the collection of potential events for every \requ{} is sufficient for \(\qzSaV{}\) is a positive answer to \qzS{}.
    I suspect such a collection would be sufficient, but nothing will depend on this.
  \end{shiftpar}
\end{note}

\begin{note}
  Second, what is meant by a `potential event'.
  Potential is subjunctive, an instance of a possible event.
  However, interested in answers to \qzS{}.
  Constraints on the event.
  `Potential' is a restriction of possible which satisfies constraints.
\end{note}

\subsection{Arguments}
\label{cha:zSpA:sec:arguments}

\begin{note}
  Two arguments for~\autoref{prop:PWEs}.
\end{note}

\begin{note}
  Recall \qzS{}:
  \begin{quote}
    \questionZS*
  \end{quote}
\end{note}

\begin{note}
  First, direct.
  Second, indirect.
\end{note}

\subsubsection{The simple argument}
\label{cha:zSpA:sec:simple-argument}

\begin{note}
  The simple argument for~\autoref{prop:PWEs} highlights that \qzS{} asks whether a conditional holds, and observes it is not possible for the conditional to hold without~\autoref{prop:PWEs} holding.
\end{note}

\begin{note}[Simple argument for~\autoref{prop:PWEs}]
  The simple argument for~\autoref{prop:PWEs} is direct.
  Take some arbitrary thing \(\qzSaV{}\), some \(\pvp{\psi}{v'}{\Psi}\), assume \requ{}, potential event.

  In full:

  Consider an agent when concluding \(\pv{\phi}{v}\) from \(\Phi\).
  Positive answer, and let \(\qzSaV{}\) be positive answer.

  Further, assume \(\pvp{\psi}{v'}{\Psi}\) is a \requ{}.
  { \color{red} Seen, a and b just express \requ{}\dots}

  From \qzS{} \ref{question:zs:may-fail}, it follows that from the agent's perspective, the agent would conclude \(\pv{\psi}{v'}\) from \(\Psi\), if \vAgent{} were to attempt to conclude \(\pv{\psi}{v'}\) from \(\Psi\).

  Agent has the option, from \ref{question:zs:option}.
  So, there is a potential event.
  And, given \ref{question:zs:may-fail}, concludes.
\end{note}

\begin{note}[Alternatively, by contradiction]
  No potential event.
  Then, focusing only on \(X\), we have the possibility of \(X\) and there not being a potential witnessing event.
  So, we then have \(X\) and failure to conclude.
  But then \ref{question:zs:may-fail} still.
\end{note}


\subsubsection{The indirect argument}
\label{cha:zSpA:sec:the-argument}

\begin{note}
  Second argument views things from a different perspective.

  Core idea is that nothing other than potential event does the work.
\end{note}

\begin{note}[A \deadEnd{}]
  Piece of terminology for the argument.

  \begin{definition}[A \deadEnd{0}]
    \label{def:dead-end}
    For an agent \vAgent{}:

    \begin{itemize}
    \item
      \vAgent{}' epistemic state is a \emph{\deadEnd{0}} if:
      \begin{itemize}
      \item
        No conclusion without revision, from \vAgent{}' perspective.
      \end{itemize}
    \end{itemize}
    \vspace{-\baselineskip}
  \end{definition}

  Typical instance of dead end is conflicting proposition-value pairings.

  Qualifier, from \vAgent{}' perspective.
  Some difficulty.
  \deadEnd{} prevents agent from concluding.

  In this respect, \deadEnd{} does not need to be genuine.
  For example, further reasoning, so not conflicting.
  Understand as distinct epistemic state, as no longer block.

  Conversely, don't need proposition-value pairs to be in genuine conflict.
\end{note}


\begin{note}
  \begin{argument}
    If premise, alone, then, doesn't show that the agent's present epistemic state is \emph{not} a \deadEnd{}.
    From agent's perspective, hold the premise is true, and still fail to conclude \(\pv{\psi}{v'}\) from \(\Psi\).

    This is clear.

    Not merely premise, but that it's really there.

    More simple example.
    (Here, maybe cite Harman.)
    Know where car is parked, from perspective, and still find car is missing.

    Indeed, here, parallels to knowledge are interesting.
    In the case of knowledge, not only perspective, but also state of the world answers.
    For, need the truth connexion.

    Contrast with belief.

    Important point.
    The agent could be wrong.
    This is all still from the agent's perspective.
  \end{argument}

  Important, this is still from the agent's perspective.
  So, there's no guarantee that there really is a potential witnessing event.%
  \footnote{
    This works even with reduction to knowledge that, as there's no guarantee that the agent really has the knowledge (that).
  }
  What we get is the way in which \qzS{} receives a positive answer from the agent's perspective, and that is, in part, in terms of potential witnessing event, or something that depends on such an event.
\end{note}

\begin{note}
  Illustration.
  From perspective, reasoning is exercise of ability.
\end{note}

\begin{note}
  {
    \color{green}
    Parallel to \citeauthor{Carroll:1895uj}.
    Can't add as premises.
    The difficulty with \citeauthor{Carroll:1895uj} is that the issue is unclear.
    Don't see why adding as a premise does no work, other than acceptance of the Tortoise to form a hypothetical.
    Here, hopefully clear.
  }
\end{note}


\begin{note}
  \color{red}

  ~\cite{Besson:2018wz} in here somewhere.
\end{note}

\begin{note}
  \color{red}
  Point here is role of rule of inference is key.
  And,~\autoref{prop:PWEs} is observing this.
\end{note}

\begin{note}
  Similar to \citeauthor{Carroll:1895uj}.
  \begin{quote}
    Logic would take you by the throat, and \emph{force} you to do it!%
    \mbox{ }\hfill\mbox{(\citeyear[280]{Carroll:1895uj})}
  \end{quote}
  Looking at something static.
  Achilles fails to convey this to the Tortoise, arguably through some fault of Achilles' own.

  In parallel, we could stack up additional passives in the same way, but there's little interest in doing so.
  The point is the base \requ{} is not satisfied.
\end{note}

\begin{note}
  So, with \citeauthor{Carroll:1895uj}, we get a rule of inference, great.

  \citeauthor{Wieland:2013vf} characterises the general understanding of \textcite{Carroll:1895uj} in terms of two lessons:
  \begin{quote}
    [T]he negative lesson is that if you add ever more premises to an argument \dots, then you will never demonstrate that its conclusion follows logically.%
    \mbox{ }\hfill\mbox{(\citeyear[984]{Wieland:2013vf})}
  \end{quote}

  Parallel, static answers, still option for concluding otherwise.

  \begin{quote}
    [T]he positive lesson is that rules of inference, rather than premises of the form `if premises such and such are true, then the conclusion is true', will do the job.%
    \mbox{ }\hfill\mbox{(\citeyear[984]{Wieland:2013vf})}
  \end{quote}

  Parallel, the dynamic status of a rule.
\end{note}

\begin{note}
  Similar, but a little different.
\end{note}

\begin{note}
  No regress.

  Following \citeauthor{Wieland:2013vf}:

  \begin{quote}
    \begin{itemize}[noitemsep]
    \item[IR]
      For any item x of a certain type, S \(\varphi\)-s x only if
      \begin{enumerate}[label=(\roman*),noitemsep]
      \item
        there is a new item y of that same type, and
      \item
        S \(\varphi\)-s y.%
        \mbox{ }\hfill\mbox{(\citeyear[996]{Wieland:2013vf})}
      \end{enumerate}
    \end{itemize}
  \end{quote}

  Here, opposite direction.
  But, somewhat similar.

  Here, the pattern is introduced by a \requ{}.
  However, subjunctive.
  And, from agent's perspective.
  So, there is no requirement of recursion.
\end{note}

\subsection{Corollary}
\label{cha:zSpA:sec:corollary}

\begin{note}
  \begin{corollary}
    \label{corr:prop:PWEs}
    Positive answer to \qzS{} only if, for any \requ{}, there is a potential event from agent's perspective.
  \end{corollary}
\end{note}

\section{Interpretation}
\label{cha:zSpA:sec:interpretation}

\begin{note}
  Two instances where subjunctives of this form have a role.

  Dispositions/ability.

  Doxastic justification.
\end{note}

\begin{note}
  Important to observe here that with dispositions and ability, the subjunctive analysis is an analysis.
  So, in principle possible to provide a distinct analysis.
  This is surely the case, and I can probably find some example.

  By contrast, in the case of positive answers to \qzS{}, the subjunctive is `built in' to the question.
\end{note}

\subsubsection{Dispositions and ability}
\label{sec:dispositions}

\begin{note}[Parallel between dispositions and ability]
  Consider \citeauthor{Choi:2021wg}'s characterisation of the Simple Conditional Analysis of dispositions:
  \begin{quote}
    An object is disposed to \emph{M} when \emph{C} iff it would \emph{M} if it were the case that \emph{C}.\nolinebreak
    \mbox{}\hfill\mbox{(\citeyear{Choi:2021wg})}
  \end{quote}
  For example, an object is disposed to dissolve when it is placed in water iff the object would dissolve if it were the case that it is placed in water.

  The Simple Conditional Analysis may be challenged, but for our purposes it is adequate.
  We are interested in the broad form of the truth condition, and various more refined analyses share the same broad form.
  Note, in particular, that it being the case that \emph{C} and \emph{M} happening describes an event.
  Given appropriate conditions; salt dissolves, glass breaks, and I mumble when I am tired.
  The key idea is that the property of being disposed to \emph{M} when \emph{C} is analysed in terms of the (possible) event of \emph{M} happening when \emph{C}.

  The parallel to ability is established by noting that ability may also be analysed in terms of a (possible) event, as we have seen.
  In particular, by incorporating volition in the analysans of the Simple Conditional Analysis.
  To illustrate, \citeauthor{Mandelkern:2017aa} trace the Conditional Analysis of ability  to \textcite{Hume:1748tp} and \textcite{Moore:1912te}, among others:
  \begin{quote}
    S can \(\phi\) iff S would \(\phi\) if S tried to \(\phi\)\nolinebreak
    \mbox{}\hfill\mbox{(\citeyear[Cf.][308]{Mandelkern:2017aa})}
  \end{quote}
  Compare to the Simple Conditional Analysis of dispositions:
  The object is some agent \emph{S}, \emph{C} is `S tried to \(\phi\)' and \emph{M} is `S \(\phi\)s' --- it is volition alone which distinguishes the analyses.
  For example, I have the ability to demonstrate that a rectangle with dimensions \(19\text{cm}\) by \(7\text{cm}\) has area \(133\text{cm}^{2}\) only if I would demonstrate that a rectangle with dimensions \(19\text{cm}\) by \(7\text{cm}\) has area \(133\text{cm}^{2}\) if it were the case that I tried that a rectangle with dimensions \(19\text{cm}\) by \(7\text{cm}\) has area \(133\text{cm}^{2}\).
\end{note}


\subsubsection{Doxastic justification}
\label{cha:fcs:sec:dox-just}

\begin{note}
  \citeauthor{Turri:2010aa}

  \begin{quote}
    Necessarily, for all S, \emph{p}, and \emph{t}, if \emph{p} is propositionally justified for S at \emph{t}, then \emph{p} is propositionally justified for S at \emph{t} because S currently possesses at least one means of coming to believe \emph{p} such that, were S to believe \emph{p} in one of those ways, S's belief would thereby be doxastically justified.%
    \mbox{ }\hfill\mbox{(\citeyear[316]{Turri:2010aa})}
  \end{quote}

  Key is that doxastic justification depends on what the agent does.

  \citeauthor{Turri:2010aa}'s focus is on how reasons are used.
  What the agent does.

  Seen with example.

  \begin{quote}
    \begin{enumerate}[label=(P\arabic*)]
      \setcounter{enumi}{4}
    \item
      The Spurs will win if they play the Pistons.
    \item
      The Spurs will play the Pistons.
    \end{enumerate}

    \mbox{}\hfill\(\vdots\)\hfill\mbox{}

    \begin{enumerate}[label=(P\arabic*), resume]
    \item
      Therefore, the Spurs will win.%
    \mbox{ }\hfill\mbox{(\citeyear[317]{Turri:2010aa})}
    \end{enumerate}
  \end{quote}

  Rather than \emph{modus ponens}, `\emph{modus profusus}'.
  Conclude \(r\) from \(p\) and \(q\).
  (\citeyear[317]{Turri:2010aa})

  \begin{quote}
    The way in which the subject performs, the manner in which she makes use of her reasons, fundamentally determines whether her belief is doxastically justified.
    Poor utilization of even the best reasons for believing \emph{p} will prevent you from justifiedly believing or knowing that \emph{p}.%
    \mbox{ }\hfill\mbox{(\citeyear[316]{Turri:2010aa})}
  \end{quote}

  Variant of ~\cite{Prior:1960wh}'s `tonk' connective.
  Though, difference is between connective and rule.
  \(p\) tonk \(q\) would not be propositionally justified.
\end{note}

\begin{note}
  \citeauthor{Turri:2010aa} is similar to \citeauthor{Goldman:1979ui}

  Begin with justification.

  \begin{quote}
    \begin{enumerate}[label=(\arabic*)]
      \setcounter{enumi}{10}
    \item
      Person \emph{S} is \emph{ex ante} justified in believing \emph{p} at \emph{t} if and only if there is a reliable belief-forming operation available to \emph{S} which is such that if \emph{S} applied that operation to this total cognitive state at \emph{t}, \emph{S} would believe \emph{p} at \emph{t}-plus-delta (for a suitably small delta) and that belief would be \emph{ex post} justified.
    \end{enumerate}
  \end{quote}

  Where, sufficient condition for belief would be \emph{ex post} justified:
  \begin{quote}
    \begin{enumerate}[label=(\arabic*)]
      \setcounter{enumi}{4}
    \item
      If S's believing \emph{p} at \emph{t} results from a reliable cognitive belief-forming process (or set of processes), then S's belief in \emph{p} at \emph{t} is justified.%
      \mbox{ }\hfill\mbox{(\citeyear[13]{Goldman:1979ui})}
    \end{enumerate}
  \end{quote}
  Roughly, at least.
  \citeauthor{Goldman:1979ui} refines this a fair bit, but this isn't important.

  Availability of a reliable belief-forming operation!

  Relation here is brittle.
  Account of justification, apply to concluding.
  Well, then all we get is that before concluding, would make sense to conclude only if available.
  Running something like the \citeauthor{Carroll:1895uj} regress, not some state.
  But, this only tells us about suitability to conclude.

  Still, key point is process.

  Another useful thing to highlight is the suitably small delta.
  With \requ{}, this is captured in terms of the option.
\end{note}

\begin{note}
  Significant difference is in the case of justification, we're not interested in the agent's perspective.
  Hence, these accounts are understood in terms of the agent having the ability, roughly.

  With \qzS{}, we're interested in the agent's perspective, and there is no guarantee that the agent really has the ability.
\end{note}

\section{\ptivity{2} and \ptivityQ{2}}
\label{cha:zSpA:sec:difficulty}

\begin{note}
  The goal of this section is to highlight what we get from \autoref{prop:PWEs}.

  To do so, we present and motivate interest in a principle we term `\ptivity{}'.

  Highlight how it may lead to a failure of the simple argument for \autoref{prop:PWEs}.

  Could see this as a potential objection.
  However, no additional observations need to be made.
  So, clarification.
  Still, walk though exactly why.
\end{note}

\begin{note}[Basic idea]
  Distinguish between:
  \begin{itemize}
  \item
    \(\phi\) having value \(v\) from agent's perspective.
  \end{itemize}
  And:
    \begin{itemize}
  \item
    \(\phi\) \(\{\) seems, being believed to have, imagining, misrepresented \dots \(\}\) to have value \(v\) from agent's perspective.
  \end{itemize}

  In the second case, some qualifier.
\end{note}

\begin{note}
    Motivating reasons are psychological states of the agent.
  \citeauthor{Dancy:2000aa}%
  \footnote{
    Neither \citeauthor{Dancy:2000aa} nor \citeauthor{Collins:1997wn} give instances.
    Indeed, there are no citations in \textcite{Collins:1997wn}.
  }
  \begin{quote}
    The statement
    \begin{enumerate}[label=(\arabic*), ref=(\arabic*), nosep]
    \item%
      \emph{A}'s reason for \(\phi\)-ing was that \emph{p}
    \end{enumerate}
    can only be true if
    \begin{enumerate}[label=(\arabic*), ref=(\arabic*), resume, nosep]
    \item
      \emph{A} believed that \emph{p}.
    \end{enumerate}
    Therefore
    \begin{enumerate}[label=(\arabic*), ref=(\arabic*), resume, nosep]
    \item
      \emph{A}'s reason for \(\phi\)-ing was that \emph{A} believed that \emph{p}.%
      \mbox{ }\hfill\mbox{(\citeyear[102]{Dancy:2000aa})}
    \end{enumerate}
  \end{quote}
  As \citeauthor{Dancy:2000aa} observes, this argument is not obviously valid.
  However, not too interested in the argument.

  \begin{quote}
    Its conclusion is not that /Vs reason for <p-ing was his believing that p, but that his reason was that he believed that p.
    \dots
    [the] reason [the argument] `discovers' is not itself a psychological state of the agent but the `fact' that the agent was in such a state.
  \end{quote}

   \citeauthor{Dancy:2000aa} highlights misunderstanding of conclusion.

  \begin{quote}
    \begin{enumerate}[label=(\arabic*\(^{\ast}\)), ref=(\arabic*\(^{\ast}\))]
      \setcounter{enumi}{2}
    \item
      \emph{A}'s reason for \(\phi\)-ing was \emph{A}'s believing that \emph{p}.%
      \mbox{ }\hfill\mbox{(\citeyear[102]{Dancy:2000aa})}
    \end{enumerate}
  \end{quote}
\end{note}

\begin{note}
  As expressed by \citeauthor{Collins:1997wn} (\citeyear{Collins:1997wn}):
  \begin{quote}
    The perspective of the agent, when rightly interpreted, is not a call for introspectible deteterminants of action.
    It is a reminder that it is objective circumstances \emph{as apprehended by the agent} that are relevant.
    The perspective is not the subject matter.
    An agent makes statements about the objective circumstances as he understands them.
    This qualification: `as he understands them' is not a shift to the mental realm.%
    \mbox{ }\hfill\mbox{(\citeyear[120]{Collins:1997wn})}
  \end{quote}
\end{note}

\begin{note}
  \phantlabel{dancy-to-return}
  Interest here is with the distinction.

  The general argument \citeauthor{Dancy:2000aa} makes is more delicate.
  Question about whether \(\phi\) having value \(v\) explains, from the agent's perspective.
  Distinct from Question about whether \(\phi\) having value \(v\) from the agent's perspective, explains.

  For the moment, our interest is whether there is a difference with respect to positive answers to \qzS{}.
  If no difference, then broader issues concern statement of the question, rather than answers to the question.
  \hyperref[return-to-dancy]{Return to this} in \autoref{cha:zSpA:sec:ptivity-question}.
\end{note}


\begin{note}
  Build up slowly.

  First, \autoref{cha:zSpA:sec:ptivity}, basic principle, indirect link to \autoref{prop:PWEs}.
  Second, \autoref{cha:zSpA:sec:ptivityq}, expand on the basic principle, direct link to \autoref{prop:PWEs}.

  Then, \autoref{cha:zSpA:sec:revisiting-arguments}, revisit arguments.
\end{note}


\begin{note}
  Argument is sound.

  However, highlight an important aspect of the argument.

  Positive answers to \qzS{} require \ptivity{} to hold.

  This is what \ptivity{} is.

  \ptivity{} does not hold in general.

  It holds in the simple argument because \qzS{}.

  Make this explicit with a variant argument.
\end{note}

\begin{note}
  At issue is whether some qualifier such that `\emph{\ptivity{0}}' fails to hold:%
  \footnote{
    The term `\emph{\ptivity{0}}' is designed to highlight the parallel between \emph{\ptivity{0}} and \factivity{}.

    \factivity{2}, \(\mathbb{A}\) \(\pv{\phi}{v}\) only if \(\phi\).

    For example, know \(\pv{\phi}{v}\) only if \(\phi\) or realise \(\pv{\phi}{v}\) only if \(\phi\).

    Fact is necessary in order to have attitude.
    From attitude, get fact.

    \ptivity{2}: drop perspective.
    Perspective is necessary in order for qualifier.
  }
\end{note}

\subsection{\ptivity{2}}
\label{cha:zSpA:sec:ptivity}

\subsubsection{The principle}

\begin{note}
  \ptivity{2}, agent, proposition, value, qualifier.
  Given qualifier, also holds from agent's perspective.

  \begin{absolutelynopagebreak}
    \begin{principle}[\ptivity{2}]
      \label{def:perspectivity}
      Given an agent \vAgent{}, proposition \(\phi\), value \(v\), and qualifier \ptivityQV{}:

      \begin{itemize}
      \item[]
        \ptivityQV{} satisfies \ptivity{} \emph{only if}:
      \item[]
        \begin{enumerate}[noitemsep]
        \item[\emph{If}]
          \begin{enumerate}[label=\alph*., ref=(\alph*)]
          \item
            \(\phi\) is \ptivityQV{}'d to have value \(v\) from \vAgent{}' perspective.
          \end{enumerate}
        \item[\emph{then}]
          \begin{enumerate}[label=\alph*., ref=(\alph*), resume]
          \item
            \(\phi\) has value \(v\) from \vAgent{}' perspective.
          \end{enumerate}
        \end{enumerate}
      \end{itemize}
      \vspace{-\baselineskip}
    \end{principle}
  \end{absolutelynopagebreak}

  Note, whether or not \ptivity{} holds with respect to some qualifier \ptivityQV{} is always with respect to some agent, proposition, and value.
  Hence, \ptivity{} allows for the possibility of \ptivity{} to hold or fail depending on the relevant agent, proposition, and value.

  For example, it is possible for \ptivity{} to hold with respect to \ptivityQV{} for some agent \vAgent{}, value \(v\), and proposition \(\phi\) but failing with respect to \vAgent{}, \(v\), and some other proposition \(\phi'\).

  In this respect, \ptivity{} differs from general principles such as \factivity{}, which quantify over agents, propositions, and values.

  The motivation is dialectical.
  Our broad interest is with the arguments for~\autoref{prop:PWEs}.
  Here, we have a fixed agent, proposition, and value.
  And, at issue is whether there may be \emph{some} qualifier such that (positive) answers to~\autoref{prop:PWEs} may be qualified by a qualifier for which \ptivity{} fails.

  So, rather than considering when a qualifier holds for arbitrary agents, propositions, and values, we may restrict our attention to a specific choice of agent, proposition, and value.

  Still, it is straightforward to define \ptivity{} with respect to a qualifier.%
  \footnote{
    \ptivity{2} holds with respect to the qualifier \ptivityQV{} just in case:
    \begin{itemize}
    \item[]
      For any agent \vAgent{}, proposition \(\phi\), and value \(v\), \ptivity{} holds with respect to \ptivityQV{}.
    \end{itemize}
  }
\end{note}


\subsubsection{Examples}
\label{cha:zSpA:sec:examples-of-ptivity-y-n}

\begin{note}
  Two positive instances and two negative instances.

  Two negative instances are straightforward.
  \scen{0} such that distinction is important.

  Positive examples are a little different.
  First example, argue that \ptivity{} holds for knowledge as qualifier.
  Second, sketch an argument that \ptivity{} holds for belief as a qualifier.
\end{note}

\paragraph{Positive}

\subparagraph{Knowledge}

\begin{note}[Knowledge]
  Though we have defined \ptivity{} with respect to an agent, proposition and value, there is a straightforward argument that \ptivity{} holds for knowledge when used as a qualifier, regardless of the agent, proposition, or value.

  The argument rests on the \factivity{} of knowledge:

  \begin{principle}[Factivity of knowledge]
    \label{principle:factivity}
    \vAgent{} knows \(\phi\) has value \(v\) \emph{only if} \(\phi\) has value \(v\).
  \end{principle}

  Specifically, we argue \factivity{} entails:

  \begin{principle}[Factivity of knowledge, relative to perspective]
    \label{principle:factivity-perspective}
    \vAgent{} knows \(\phi\) has value \(v\), from \vAgent{} perspective \emph{only if} \(\phi\) has value \(v\), from \vAgent{}' perspective.
  \end{principle}
\end{note}

\begin{note}[Argument]
  The argument for~\autoref{principle:factivity-perspective} is simple.

  Suppose an agent knows \(\phi\) has value \(v\), from the agent's perspective.
  Then, from the agent's perspective the agent knows \(\phi\) has value \(v\) only if \(\phi\) has value \(v\).
  Therefore, \(\phi\) must have value \(v\) from the agent's perspective.
  For, else the agent does not know that \(\phi\) has value \(v\), from the agent's perspective.
\end{note}

\begin{note}[Elaboration]
  What is being claimed is perhaps clearer with the following principle which directly follows from \factivity{}:%
  \footnote{
    Suppose \(\phi\) has value \(v\) \emph{only if} \(\psi\) has value \(v'\), and further suppose \vAgent{} knows \(\phi\) has value \(v\).
    Then, by \factivity{}, \(\phi\) has value \(v\).
    Hence, by the initial assumption, \(\psi\) must have value \(v'\).
  }

  \begin{principle}[Two-step factivity]
    \mbox{ }
    \vspace{-\baselineskip}
    \label{principle:tsf}
    \begin{itemize}[noitemsep]
    \item[\emph{If}]
      \begin{itemize}
      \item[]
        \(\phi\) has value \(v\) \emph{only if} \(\psi\) has value \(v'\),
      \end{itemize}
    \item[\emph{then}]
      \begin{itemize}
      \item[]
        [\vAgent{} knows \(\phi\) has value \(v\)] \emph{only if} \(\psi\) has value \(v'\).
      \end{itemize}
    \end{itemize}
    \vspace{-\baselineskip}
  \end{principle}

  Suppose an agent knows \(\phi\) has value \(v\), from the agent's perspective.

  Now, if \(\phi\) has value \(v\) \emph{only if} \(\psi\) has value \(v'\), from the agent's perspective, then \(\psi\) has value \(v'\) from the agent's perspective.
  For, applying the same reasoning as above, \(\phi\) has value \(v\) from the agent's perspective, and \(\psi\) having value \(v'\) follows directly from both the conditional and the antecedndent of a conditional holding from the agent's perspective.%
  \footnote{
    Give or take the agent recognising both the condition and the antecedent hold from their perspective and obtain the consequent.
  }

  However, if it is not the case that \(\phi\) has value \(v\) \emph{only if} \(\psi\) has value \(v'\), from the agent's perspective, then \(\psi\) need not have value \(v'\) from the agent's perspective.

  For, from the agent's perspective, there is no link between \(\phi\) having value \(v\) and \(\psi\) having value \(v'\).
\end{note}

\begin{note}[Caution]
  Note, we have not argued that if an agent knows that \(\phi\) has value \(v\) from the agent's perspective, then \(\phi\) has value \(v\).
  \emph{This} is a far stronger principle.
  \factivity{2}, as stated above, talks about whether an agent knows \(\phi\) has value \(v\) independently of any (particular) agent's perspective.
  Our observation is merely that when internalised to an perspective, \ptivity{} holds for `knows' as a qualifier.
\end{note}

\subparagraph{Belief}

\begin{note}
  We now turn to belief.

  Idea here is that understanding under which \ptivity{} holds with respect to belief.
\end{note}

\begin{note}[Looking back]
  Considered belief and agent's perspective in~\autoref{cha:introduction:sec:agents-perspective}.

  So, if you think belief and agent's perspective coincide, then question is whether belief obeys the following:
  \begin{quote}
    Agent believes that they believe \(\phi\) has value \(v\), then the agent believes that \(\phi\) has value \(v\).
  \end{quote}
  Equivalent to:
  \begin{quote}
    Agent believes \(\phi\) has value \(v\), from their perspective, then \(\phi\) has value \(v\) from their perspective.
  \end{quote}

  This is how I understand the arguments of \citeauthor{Dancy:2000aa} and \citeauthor{Collins:1997wn}.

  Here, take the agent's perspective as basic, and develop motivation based on \citeauthor{Moore:1993wk}an sentences.
\end{note}

\begin{note}
  Bad for to conflicting things to hold from agent's perspective.
\end{note}

\begin{note}
  Consider the following \citeauthor{Moore:1993wk}an sentence:%
  \footnote{
    The original sentence \citeauthor{Moore:1993wk} considers is:

    \begin{quote}
      Though I don't believe it's raining, yet as a matter of fact it really is raining.%
      \mbox{ }\hfill\mbox{(\citeyear[207]{Moore:1993wk})}
    \end{quote}
    There are two difficulties with appealing to \citeauthor{Moore:1993wk}'s sentence.

    First, we are interested in belief, rather than the absence of belief.
    Second, the relevant qualifier is plausibly `I don't believe' and the simplistic way in which \ptivity{} is stated does not allow for \(\phi\) not having value \(v\) from the agents perspective to follow from a \emph{intuitively} negative qualifier applied to \(\phi\) having value \(v\), from the agent's perspective.

    There is also some complexity.
    For, `as a matter of fact it really' is, likewise, a qualifier, and though I think it is intuitive that \ptivity{} holds, it is more straightforward to ignore this.
  }

  \begin{enumerate}[label=\emph{M}., ref=(\emph{M})]
  \item
    \label{ptivity:holds:moore-sent}
    I believe it is not raining, though it is raining.
  \end{enumerate}

  Intuitively, problematic.
  For \citeauthor{Moore:1993wk} absurd to assert.%
  \footnote{
    Though, focus of paradox is that it should be absurd to say them.
    (\citeyear[Cf.][208]{Moore:1993wk})
  }

  Likewise, absurd to hold from agent's perspective.

  \begin{enumerate}[label=\emph{M\('\)}., ref=(\emph{M\('\)})]
  \item
    \label{ptivity:holds:moore-sent:pers}
    I believe it is not raining, though it is raining, from my perspective.
  \end{enumerate}


  As~\ref{ptivity:holds:moore-sent} is a conjunction, let us consider the left and right conjuncts individually:

  \begin{enumerate}[label=\emph{M\('\)\textsubscript{l}}., ref=(\emph{M\('\)\textsubscript{l}})]
  \item
    \label{ptivity:holds:moore-sent:pers:l}
    I believe it is not raining, from my perspective.
  \end{enumerate}

  \begin{enumerate}[label=\emph{M\('\)\textsubscript{r}}., ref=(\emph{M\('\)\textsubscript{r}})]
  \item
    \label{ptivity:holds:moore-sent:pers:r}
    It is raining, from my perspective.
  \end{enumerate}

  Observe, there is no immediate conflict between~\ref{ptivity:holds:moore-sent:pers:l} and~\ref{ptivity:holds:moore-sent:pers:l}.
  Still, if \ptivity{} holds with respect to belief, then~\ref{ptivity:holds:moore-sent:pers:l} entails the following:

  \begin{enumerate}[label=\emph{M\('\)\textsubscript{l\('\)}}., ref=(\emph{M\('\)\textsubscript{l\('\)}})]
  \item
    \label{ptivity:holds:moore-sent:pers:lp}
    It is not raining, from my perspective.
  \end{enumerate}

  Now, \ref{ptivity:holds:moore-sent:pers:lp} and \ref{ptivity:holds:moore-sent:pers:r} seem to be in clear conflict.
  Indeed, a sufficient (and, I think, plausible) principle we may appeal to here is the principle that it is not possible for \(\phi\) and some proposition \(\phi'\) which conflicts with \(\phi\) to both be true from an agent's perspective.

  Hence, \ptivity{} holding with respect to belief offers an account of why~\ref{ptivity:holds:moore-sent:pers} is absurd.

  In short, because:
  \begin{enumerate*}[label=(\roman*)]
  \item
    \ref{ptivity:holds:moore-sent:pers} entails~\ref{ptivity:holds:moore-sent:pers:r} and~\ref{ptivity:holds:moore-sent:pers:lp},
  \item
    \ref{ptivity:holds:moore-sent:pers:r} and~\ref{ptivity:holds:moore-sent:pers:lp} are incompatible, and therefore
  \item
    there is no plausible interpretation of \ref{ptivity:holds:moore-sent:pers}.
  \end{enumerate*}
\end{note}

\begin{note}
  There are a few things to note about this argument.

  First, the argument inconclusive.

  We have provided an explanation for the absurdity of~\ref{ptivity:holds:moore-sent:pers}, but~\ref{ptivity:holds:moore-sent:pers} is a single instance of belief, and we have provided no motivation for thinking what holds of~\ref{ptivity:holds:moore-sent:pers} generalises to other instances of belief.%
  \footnote{
    Indeed, it is not clear to me that \ptivity{} holds with respect to belief in general.
    For example, consider again Goodman's example from~\autoref{fn:belief-is-difficult} on page~\pageref{fn:belief-is-difficult}.
  }

  Second, the argument is not valid.
  For, we have not argued that the only way to obtain the absurdity of~\ref{ptivity:holds:moore-sent:pers} is via \ptivity{} holding with respect to belief.

  Still, in response to both of these points, I think the argument as is highlights why one may expect \ptivity{} to hold with respect to belief, and this, I think, is sufficient for present purposes.

  Third, the structure of the argument focuses on eliminating a qualifier.
  This is a slightly more subtle point, but is important to keep in mind.
  Of course, eliminating `belief' as a qualifier from~\ref{ptivity:holds:moore-sent:pers:l} is expected, given our interest in \ptivity{}.
  But, observe arguing in the converse direction is far from straightforward.
  One may hold that belief as a qualifier should be \emph{introduced} to~\ref{ptivity:holds:moore-sent:pers:r}.
  However, while the revised~\ref{ptivity:holds:moore-sent:pers:r} would be in conflict with~\ref{ptivity:holds:moore-sent:pers:l}, one would also need to argue that~\ref{ptivity:holds:moore-sent:pers:l} should not also be qualified.%
  \footnote{
    Explicitly:
    \begin{enumerate}[label=\emph{M\('\)\textsubscript{r+}}., ref=(\emph{M\('\)\textsubscript{r+}})]
    \item
      \label{ptivity:holds:moore-sent:pers:rq}
      I believe it is raining, from my perspective.
    \end{enumerate}
    Is in conflict with~\ref{ptivity:holds:moore-sent:pers:l}, but if belief is introduced in this way, one needs to resists obtaining:
    \begin{enumerate}[label=\emph{M\('\)\textsubscript{l+}}., ref=(\emph{M\('\)\textsubscript{l+}})]
    \item
      \label{ptivity:holds:moore-sent:pers:lq}
      I believe I believe it is not raining, from my perspective.
    \end{enumerate}
    For, any absurdity from conjoining~\ref{ptivity:holds:moore-sent:pers:rq} and~\ref{ptivity:holds:moore-sent:pers:lq} seems equivalent to the absurdity of conjoining~\ref{ptivity:holds:moore-sent:pers:r} and~\ref{ptivity:holds:moore-sent:pers:l}.

    Indeed, \citeauthor{Moore:1993wk} does this with assertion.
    \begin{quote}
      [T]here is a difference between what I \emph{imply} by uttering assertively the words `it's raining' and what you \emph{imply} by uttering the same words at the same time in the same place.
      Namely I \emph{imply} that I believe it's raining and \emph{not} that you do; you imply that you do, and not that I do.%
      \mbox{ }\hfill\mbox{(\citeyear[209--210]{Moore:1993wk})}
    \end{quote}
    So, by asserting `it's raining', I imply I believe it's raining.
    And, this is in conflict with not believing that it's raining.
    However, it is not immediate that the assertion is in conflict with asserting `I don't believe it's raining', as by the same idea the assertion may only imply that I believe that I don't believe it's raining.
  }
  By contrast, \ptivity{} eliminates belief as a qualifier from~\ref{ptivity:holds:moore-sent:pers:l}, and as~\ref{ptivity:holds:moore-sent:pers:r} is \emph{un}qualified, \ptivity{} does not apply to~\ref{ptivity:holds:moore-sent:pers:r}.
\end{note}

\paragraph{Two scenarios in which \ptivity{} fails to hold}

\begin{note}
  Two instances.

  First instance builds on the idea of taking an non-committal such as `seems',
  Second, on an antagonistic qualifier such as `misrepresented'.

  In first case, non-committal.
  In second case, some incompatible proposition holds from agent's perspective.

  However, neither instance depends on the term for the qualifier chosen, or a complete description of the qualifier.
  Instead, two \scen{1} where capture what's going on via a failure of \ptivity{} to hold for whichever qualifier.
\end{note}

\subparagraph{The first \scen{0}}

\begin{note}
  The first scenario is interactive!

  \begin{scenario}[\citeauthor{Zollner:1860vx}]
    You looking at the variation of the \citeauthor{Zollner:1860vx} illusion given in \autoref{fig:zollner-illusion}.
    \begin{figure}[!h]
      \centering
      \def\svgwidth{\columnwidth}
      \input{ZIOh.pdf_tex}
      \caption{A variation of the \citeauthor{Zollner:1860vx} illusion --- crop of ~\textcite{Fibonacci:2007vj}}
      \label{fig:zollner-illusion}
    \end{figure}
  \end{scenario}

  \autoref{fig:zollner-illusion} contains eight long black lines which are crossed with short black lines.
  Alternating, between the long lines from left to right. the short lines are horizontal and then almost vertical.

  To me, at least, the long black lines do not appear parallel to one another.
  And, I suspect the same it true for you.

  Further, as~\autoref{fig:zollner-illusion} is titled `a variation of the \citeauthor{Zollner:1860vx} illusion' I guess you might think the long black lines really are parallel to one another.
  Though, to keep this interactive, there is a slight twist.
  In the \citeauthor{Zollner:1860vx} illusion the long black lines are, indeed, parallel to one another.
  However, as~\autoref{fig:zollner-illusion} is titled a variation, the long black lines may not be parallel to one another.

  So, I would guess the following is true of you:

  \begin{itemize}[noitemsep]
  \item
    It \emph{seems} long black lines are not parallel, from your perspective.
  \item
    The long black lines are neither parallel nor not parallel, from your perspective.
  \end{itemize}

  Here, `seems' is being used as a qualifier for which \ptivity{} does not hold.
  Indeed, if \ptivity{} were to hold for `seems' then it would follow that the long black lines are not parallel, from your perspective.
  But, given what has been typed above I am guessing you are ambivalent about whether the long black lines are parallel to one another.

  Likewise, the following is true of me:

  \begin{itemize}[noitemsep]
  \item
    It \emph{seems} long black lines are not parallel, from my perspective.
  \item
    The long black lines are parallel, from my perspective.%
    \footnote{
      Unfortunately, the variation is limited to a simple crop of~\textcite{Fibonacci:2007vj}'s image.
    }
  \end{itemize}

  The long black lines are parallel, from my perspective, as I've analysed the image using tools.
  However, the illusion continues to hold, and it remains the case that it \emph{seems} the long black lines are not parallel, from my perspective.

  Again, the failure of \ptivity{} for the qualifier `seems' is important for distinguishing between how the long black lines are (parallel to one another) and how they appear to me (not parallel to one another).
\end{note}

\subparagraph{The second \scen{0}}

\begin{note}
  The second \scen{0} is a variation on ideas common to external-world scepticism.%
  \footnote{
    It is unclear to me whether the following \scen{0} is compatible with \hyperlink{cite.Descartes:1996vp}{Cartesian} scepticism.

    For, consider \citeauthor{Lichtenberg:1991tf}'s remark:

    \begin{quote}
      \emph{Es denkt}, sollte man sagen, so wie man sagt: \emph{es blitzt}.
      Zu sagen \emph{cogito}, ist schon zu viel, so bald man es durch \emph{Ich denke} \"{U}bersetzt.
      \mbox{ }\hfill\mbox{(\citeyear[412]{Lichtenberg:1991tf}/K76)}
    \end{quote}

    As translated by \citeauthor{Zoller:1992ud}'s:

    \begin{quote}
      One should say, \emph{it thinks}, just as one says, \emph{it lightens}.
      It is already saying too much to say \emph{cogito}, as soon as one translates it as \emph{I think}.
      \mbox{ }\hfill\mbox{(\citeyear[418]{Zoller:1992ud})}
    \end{quote}

    In other words, it is unclear whether \hyperlink{cite.Descartes:1996vp}{Cartesian} scepticism is compatible with a subject to attribute the qualified perspective to.
    See also \textcite{Perry:1986aa}.
  }
  \nocite{Harman:1973ww}

  In the \scen{0} the agent on interest, Cypher, is taking to Agent Smith.
  The Matrix is a simulation of the world as it was in 1999, and both Cypher and Agent Smith are eating at a restaurant `inside' the Matrix.
  However, Cypher has experienced the (real) world `outside' of the matrix.
\end{note}

\begin{note}
  \begin{scenario}[Cypher at the restaurant]
    \mbox{ }%
    \vspace{-\baselineskip}
    \begin{quote}
      {
        \ttfamily

        A fork stabs the cube of meat and we follow it up to the face \\
        of Cypher.

        \qquad\qquad CYPHER

        \qquad You know, I know that this steak \\
        \indent\qquad doesn't exist. I know when I \\
        \indent\qquad put it in my mouth, the Matrix is \\
        \indent\qquad telling my brain that it is juicy \\
        \indent\qquad and delicious. After nine years, \\
        \indent\qquad do you know what I've realized?

        He shoves it in, eyes rolling up, savoring the tender beef \\
         melting in his mouth.

        \qquad\qquad CYPHER

        \qquad Ignorance is bliss.}%
      \mbox{ }\hfill\mbox{(\citeyear[330--331]{Wachowski:2000uh})}
    \end{quote}%
    \vspace{-\baselineskip}
  \end{scenario}

  With this \scen{}, our interest is with what hold with respect the following proposition when assigned value true, from Cypher's perspective:

  \begin{itemize}
  \item
    There is a steak.
  \end{itemize}

  From the first statement made by Cypher, the steak does not exist, so the proposition is not true from Cypher's perspective.
  However, from the second statement, it seems the proposition is true.
  For, Cypher \emph{knows} what will happen when Cypher puts it in his mouth.
  And, surely, Cypher cannot put a non-existing thing in his mouth.

  So, naively, the two statements are in conflict with one another.
  The difficulty is resolved with a simple qualifier for which \ptivity{} fails to hold.
  For example, consider `misrepresents':

  \begin{itemize}[noitemsep]
  \item
    It is not the case there is a steak, from Cypher's perspective.
  \item
    Cypher misrepresents that there is a steak, from Cypher's perspective.
  \end{itemize}
  Some term other than `misrepresents' may work equally well.
  Though, whichever term is chosen, the point remains the same:
  Cypher is aware that they are in the Matrix, and therefore from Cypher's perspective, Cypher's thoughts regarding objects `in' the Matrix are qualified, from Cypher's perspective.
\end{note}

\paragraph{Summary}

\begin{note}
  We have provided two examples where \ptivity{} arguably holds for a qualifier, and two examples involving \scen{1} where a qualifier for which \ptivity{} fails is important for capturing the details of the respective \scen{0}.

  Hence, as \ptivity{} may hold or fail to hold, there is room to question whether an agent's perspective may be qualified and whether \ptivity{} holds for the agent's perspective when qualified.

  We link \ptivity{} to~\autoref{prop:PWEs} in~\autoref{cha:zSpA:sec:ptivity-prop}, below.
\end{note}

\begin{note}[Other examples]
  Still, before doing so, and with instances where \ptivity{} holds and fails in hand, the general worry being raised may be highlighted by considering a final case in which it is not clear whether the agent's perspective is qualified, and if \ptivity{} holds given qualification.%
  \footnote{
    We present here a toy \scen{0}.
    For a less toy scenario, consider `\emph{evidence of}' as a qualifier in the phrase `evidence of evidence is evidence'.
    In other words, is it the case that \(\mathbb{E}\) is \emph{evidence of} evidence for \(\pv{\phi}{v}\) (from an agent's perspective) entails \(\mathbb{E}\) is evidence for \(\pv{\phi}{v}\) (from an agent's perspective).
    See, e.g.\ \textcite{Tal:2017uw} for discussion of the complexity involved in answering this.
  }

  \begin{scenario}[Soup]
    \label{scen:soup}
    An agent is eating some soup at a restaurant.
    After eating some of the soup, the agent remarks:
    \begin{quote}
      `The chef added too much salt to this soup.'
    \end{quote}
  \end{scenario}

  Initially, we may not that taste is subjective.
  Therefore the following holds of~\autoref{scen:soup}:
  \begin{enumerate}[label=\arabic*, ref=(\arabic*)]
  \item
    \label{chef:1}
    The chef added too much salt to this soup, from agent's perspective.
  \end{enumerate}

  However, if taste is subjective, then whether or not the chef added too much salt to the soup should be qualified by something like `for the agent's tastes'.
  Hence, strictly, \ref{chef:1} is \emph{not} the case, but \ref{chef:2} is the case:

  \begin{enumerate}[label=\arabic*, ref=(\arabic*), resume]
  \item
    \label{chef:2}
    For the agent's tastes, the chef added too much salt to this soup, from agent's perspective.
  \end{enumerate}

  Where, `for the agent's tastes' is a qualifier for which \ptivity{} does not hold, as we have no clear indication on whether or not the chef added too much salt apart from the agent's perspective.

  Still, to complicate matters further, the agent may understand the chef was not aware of the agent's tastes when creating the soup.
  And, perhaps is considering the right amount of salt for the average customer.
  So, \ref{chef:3}, rather than \ref{chef:2}, captures the content of the agent's remark:

  \begin{enumerate}[label=\arabic*, ref=(\arabic*), resume]
  \item
    \label{chef:3}
    For the average customer, the chef added too much salt to this soup, from agent's perspective.
  \end{enumerate}

  Now, \ref{chef:3} begins with the phrase `for the average customer'.
  Interpreted as a qualifier, \ptivity{} seems to fail---the average customer's tastes are subjective.
  However, the phrase need not function as a qualifier, and it is the case that (from the agent's perspective) the chef added too much salt to this soup for the average customer.
\end{note}

\begin{note}
  So, some care, is need to provide a clear statement of which propositions have which values from the agent's perspective, which proposition-value pairs are qualified, and a statement of whether or not \ptivity{} holds for any qualification.
\end{note}

\subsubsection{Relation between \ptivity{} and~\autoref{prop:PWEs}}
\label{cha:zSpA:sec:ptivity-prop}

\begin{note}
  We have stated \ptivity{} and seen through a handful of examples how \ptivity{} hold or fail to hold.

  We now highlight the relation between \ptivity{} and~\autoref{prop:PWEs}.
\end{note}

\begin{note}[\ptivity{} \& \qzS{}]
  The agent is fixed by the relevant instance of \qzS{}, the value is `True', and the proposition of interest is:

  \begin{itemize}
  \item[]
    There is a potential event in which agent concludes \(\pv{\psi}{v'}\) from \(\Psi\).
  \end{itemize}

  Where the relevant \(\psi\), \(v'\), and \(\Phi\) are likewise specified by the instance of \qzS{}.

  In order for~\autoref{prop:PWEs}, it must be the case that the relevant instance of \qzS{} is, from the relevant agent's perspective, answered by:

  \begin{enumerate}[label=A., ref=A]
  \item
    \label{ptivity:qzS:ans:good}
    It is true that there is a potential event in which agent concludes \(\pv{\psi}{v'}\) from \(\Psi\).
  \end{enumerate}

  However, introduce a qualifier, so that the relevant instance of \qzS{} is, from the relevant agent's perspective, answered by:

  \begin{enumerate}[label=A\('\)., ref=A\('\)]
  \item
    \label{ptivity:qzS:ans:bad}
    It is \ptivityQV{}'d to be true that there is a potential event in which agent concludes \(\pv{\psi}{v'}\) from \(\Psi\).
  \end{enumerate}

  Where, \ptivity{} fails for \ptivityQV{}, and therefore~\ref{ptivity:qzS:ans:good} need not hold, from the agent's perspective.

  Worry, slightly indirect.
  \ptivity{} fails, possibility.
  However, even if \ptivity{} holds, issue of question.
\end{note}

\paragraph{Summary}

\begin{note}
  \ptivity{}.
  Whether qualify how things are from the agent's perspective.

  Distinction, given distinction possibility to qualify perspective in answers to \qzS{}.

  This is the key thought.
  However, a little more complex.
\end{note}


\subsection{\ptivityQ{2}}
\label{cha:zSpA:sec:ptivityq}

\begin{note}
  In the previous section we introduced \ptivity{}, provided a handful of examples where \ptivity{} holds and fails to hold, and observed how \ptivity{} is indirectly related to the arguments for~\autoref{prop:PWEs}.

  In the present section to present a variation of \ptivity{} that is directly related to the arguments for~\autoref{prop:PWEs}.
  This variant of \ptivity{} ties whether or not \ptivity{} holds to a question.
  We term the variant `\ptivityQ{}'.

  We begin by stating \ptivityQ{}, and how \ptivityQ{} directly relates to the arguments for~\autoref{prop:PWEs}.
  Following, we observe that \ptivity{} holding does not entail \ptivityQ{} holds.
\end{note}

\begin{note}
  In this section we present a refined version of \ptivity{} with involves reference to a question.

  Failure instances adapt, but success instances do not.
  Hence, observing \ptivity{} holds for any qualifier suitable for a positive answer to \qzS{} isn't quite sufficient.%
  \footnote{
    Dialectic choice.
    Break down \autoref{prop:PWEs} and \autoref{corr:prop:PWEs} into distinct propositions.

    \autoref{prop:PWEs} concerns answers to \qzS{}.

    Strictly, relation to question isn't important.
    Matters is \(\phi\) has value \(v\).
    \autoref{corr:prop:PWEs}.
  }
\end{note}

\subsubsection{The principle}

\begin{note}[The principle]
  \ptivityQ{2} is a straightforward variation of \ptivity{}, in which the move from a proposition-value pair holding under some qualifier (from an agent's perspective) to the proposition-value pair holding from the agent's perspective is tied to a fixed question.
  The principle is as follows:

  \begin{principle}[\ptivityQ{2}]
    \label{def:perspectivityQ}
    For an agent \vAgent{}, proposition \(\phi\), value \(v\), and question \emph{Q}:

    A qualifier \ptivityQV{} satisfies \ptivityQ{} \emph{only if}:

    \begin{enumerate}[noitemsep]
    \item[\emph{If}]
      \begin{enumerate}[label=\alph*., ref=(\alph*)]
      \item
        \(\phi\) is \ptivityQV{}'d to have value \(v\) answers \emph{Q}, from \vAgent{}' perspective .
      \end{enumerate}
    \item[\emph{then}]
      \begin{enumerate}[label=\alph*., ref=(\alph*), resume]
      \item
        \(\phi\) has value \(v\) answers \emph{Q}, from \vAgent{}' perspective.
      \end{enumerate}
    \end{enumerate}
    \vspace{-\baselineskip}
  \end{principle}
\end{note}

\begin{note}
  Now, applies to \autoref{prop:PWEs}.

  Quantifies over agent's and instance of \qzS{}.

  Hence, fix agent, relevant proposition, value, and question:

  Agent.

  Question, \qzS{}.
\end{note}

\paragraph{No entailment}

\begin{note}[Proposition]
  So, \ptivityQ{}.
  Now observe the following proposition:

  \begin{proposition}
    \label{prop:ptivity-ne-ptivityQ}
    Given an agent \vAgent{}, proposition \(\phi\), value \(v\), question \emph{Q}, and qualifier \ptivityQV{}:

    \begin{itemize}
    \item
      \ptivity{2} holding with respect to \vAgent{}, \(\phi\), \(v\), and \ptivityQV{}.
    \end{itemize}

    Does \emph{not} entail:

    \begin{itemize}
    \item
      \ptivityQ{2} holds with respect to \vAgent{}, \(\phi\), \(v\), \ptivityQV{}, and \emph{Q}.
    \end{itemize}
    \vspace{-\baselineskip}
  \end{proposition}

  With respect to the arguments for \autoref{prop:PWEs}, it is insufficient to observe that \ptivity{} holds for any qualifier.
\end{note}

\begin{note}[Argument for proposition]
  The argument for \autoref{prop:ptivity-ne-ptivityQ} builds on a \scen{0} from \citeauthor{Hyman:1999tm} (\citeyear{Hyman:1999tm}):

  \begin{quote}
    [S]uppose Ruth reasons as follows:

    \begin{enumerate}[label=, noitemsep]
    \item
      I believe that property is theft
    \item
      People who believe that property is theft should join the Workers’ Party
    \item
      So I should join.
    \end{enumerate}

    And so she joins.%
    \mbox{ }\hfill\mbox{(\citeyear[444]{Hyman:1999tm})}
  \end{quote}

  \citeauthor{Hyman:1999tm} observes:

  \begin{quote}
    The fact that she believes that property is theft is among her reasons, but the fact, or the supposed fact, that property is theft is not.%
    \mbox{ }\hfill\mbox{(\citeyear[444]{Hyman:1999tm})}
  \end{quote}

  Here, question is `Should I join'.

  If you think \ptivity{} holds for belief, then \scen{} in which \ptivity{} holds but \ptivityQ{} fails to hold.%
  \footnote{
    Counterexample to the arguments of \citeauthor{Collins:1997wn}.

    However, delicate.

    \begin{quote}
      We could introduce psychological matters if we mean that they are the things that make his situation and his course of action intelligible to an agent.
      But if objective circumstances are what make his own action intelligible to the agent then we do not depart from the agent's perspective in putting forward objective circumstances in the context of reason-giving.%
      \mbox{ }\hfill\mbox{(\citeyear[120]{Collins:1997wn})}
    \end{quote}

    Carefully stated, \citeauthor{Collins:1997wn} goes for \ptivity{} with respect to belief, but not for  \ptivityQ{}.

    From the preface:
    \begin{quote}
      Wherever an agent correctly adduces a belief that an objective circumstance obtains in explaining his action, a de-psychologizing restatement that merely makes the objective claim must be ascribable to the agent.%
      \mbox{ }\hfill\mbox{(\citeyear[120]{Collins:1997wn})}
    \end{quote}

    Here, objective circumstances are believing (or knowing).
    So, \citeauthor{Collins:1997wn}, not possible to weaken to belief that belief, or belief that knows.
  }

  For, qualified from agent's perspective, as qualified is sufficient to answer.
  And, as sufficient, unqualified does not answer.

  If you don't think \ptivity{} holds for belief, and specifically fails in this instance, the consider substituting `belief' from `knows' through the \scen{0}.
\end{note}


\begin{note}
  So, given proposition concerns \qzS{}, and failure of entailment from~\autoref{prop:ptivity-ne-ptivityQ}, this is what we're interested in.
\end{note}


\begin{note}
  Belief.

  Arguments by \citeauthor{Collins:1997wn} and \citeauthor{Dancy:2000aa}.
\end{note}

\begin{note}
  Even if \ptivity{} holds, not clear that holds with respect to a question.
\end{note}


\begin{note}
  More generally, fails for factive attitudes.
  Substitution into \citeauthor{Hyman:1999tm}'s scenario.

  Basically, scenario is such that question asks about qualifier.

  Looking ahead, key point.
\end{note}


\begin{note}
  \citeauthor{Collins:1997wn}, may hold from agent's perspective, and \citeauthor{Hyman:1999tm}, agent's perspective may be irrelevant to why.
\end{note}


\begin{note}
  Similar, though distinct.
  ~\cite{Donnellan:1966wt}.

  Attributive and referential.

  With attributive, nothing in mind.
  Similar, here the agent gives up what the time actually is.
  However, for~\citeauthor{Donnellan:1966wt}, the way things are still matter.

  Well, this is not clear.

  So, said something true, even though not referring to anything.
  Of course, from agent's perspective.
\end{note}

\begin{note}
  Interest with \ptivity{} is with respect to \autoref{prop:PWEs}.
  So\dots
\end{note}

\subsection{Revisiting the arguments}
\label{cha:zSpA:sec:revisiting-arguments}

\begin{note}
  Issue is whether what answers \qzS{} is potential event from agent's perspective, or  weaker, independent of whether there is a potential event.

  Regardless of whether there is, what matters is thought.
  Silent on the \emph{content} of the thought.

  Some qualifier, with hold the content.

  This is a difficult objection.
  \qzS{} interpreted.
  So, it seems simple argument deals with this.

  Really need content to answer, though.

  Dependent on the agent's perspective, rather than the content of the perspective.

  So, whether it is possible for the agent to substitute in some qualifier that does not lead to potential event from agent's perspective.

  If qualifiers exist, then problem.
  For, conclusion holds with these qualifiers.

  Though, this is rather difficult, because, simple argument just observes that the content matters.

  From the agent's perspective, and aware of this.
  So, don't get potential event.
  Rather, perspective that there is a potential event.

  Clearest formulation:

  Rather than \ref{prop:PWEs:c} involving potential event:
  {
    \color{red}
    \(\qzSaV{}\) involves, in part, \vAgent{}' perspective that there is a potential event in which \vAgent{} concludes \(\pv{\psi}{v'}\) from \(\Psi\).
  }

  And, hold from perspective potential event, while no opinion on whether there really is a potential event.
\end{note}


\begin{note}
  Walk through things carefully with \ptivityQ{} in mind.
\end{note}

\subsection{Summary}
\label{sec:summary}

\begin{note}
  Simple argument.
  \ptivity{}.
  \deadEnd{1}.
  \ptivity{}.

  So, things look good.
\end{note}

\begin{note}
  Remaining sections, more general.

  First, a few ways of understanding.

  Second, pair of less interesting objections.
\end{note}

\begin{note}
  Question about the question.
  Address in final section, though the summary is easy to state.
  This objection because~\autoref{prop:PWEs}.
  Consequence of \qzS{}.
  Weakening the question might be possible, but nothing more to say.

  In other words,~\autoref{cha:zS} already addresses this.
\end{note}


\newpage

\section{Misc.\ issues}
\label{cha:zSpA:sec:misc-issues}

\subparagraph{Concluding?}

\begin{note}
  By no means clear, however.

  Suitable for \emph{concluding}?
\end{note}

\begin{note}
  However, nothing really depends on concluding here.
  It's to simplify things.
\end{note}


\newpage

\begin{note}
  \color{blue}
  Point is potential event in which concludes.

  \autoref{prop:PWEs} may seem straightforward.
  \qzS{} asks whether the agent would conclude.
  So, from agent's perspective, must be the case would conclude.
  And, would conclude only if there is a potential event in which the agent concludes.

  Parallel with more standard case of reason explanation.

  Davidson.
  Pro-attitude.
  Belief.
  So, from agent's perspective, contents of belief.
\end{note}

\section{The question?}
\label{cha:zSpA:sec:ptivity-question}

\begin{note}
  Distinction, applied to the question.

  Worry about whether this should be weakened.
\end{note}

\begin{note}
  \phantlabel{return-to-dancy}
  \citeauthor{Dancy:2000aa}, \hyperref[dancy-to-return]{noted above}.

  Distinction between psychologism and non-psychologism.

  \begin{quote}
    [P]sychologism holds that reasons are mental states such as “an agent’s believing (or wanting, or knowing) something”, and it is easy to move from the claim that someone’s reason is \emph{his believing something} (a mental state) to the claim that his reason is \emph{that he believes} something (a psychological fact).%
    \mbox{ }\hfill\mbox{(\citeauthor[\S3]{Alvarez:2017vr})}
  \end{quote}

  Non-psychologism denies this, all sorts of things.

  Now, everything so far is independent of this distinction.
  Qualified, from the agent's perspective, so psychologism.
  Here, though, reason is understood independent of the agent's perspective.

  Difficulty is if reason holds in both cases.
  But, stayed away from any claims about reasons.

  So, gap.

  \citeauthor{Hieronymi:2011aa} terms this `Dancy's gap'.


  However, moral realism.
  Require Dancy's gap to be closed.

  \begin{quote}
    [I]n cases where Angus punches his boss, believing mistakenly that he has been fired, it seems quite wrong to say he so acts because he has been fired.
    In such a case we surely must retreat to a psychologised explanation if we are to have a credible motivating reason explanation at all.%
    \mbox{ }\hfill\mbox{(\citeyear[\S6]{Lenman:2011wy})}%
    \footnote{
      See also~\textcite[\S2]{Alvarez:2017vr}.
    }
  \end{quote}

  As I understand this, \citeauthor{Dancy:2000aa} is here distinguishing between what is the case from the agent's perspective.
  So, in this case, do not explain via firing.
  For, still mediated by the agent's perspective.

  However, pressure is whether content of agent's perspective explains.

  Issue is moral realism and causation.

  Following the book, mistaken.
  Then, nothing.
  Must be the agent's perspective, rather than content of the agent's perspective.

  Or, something that is not the case.

  Not the distinction interested in.

  Different perspective.

  Agent's perspective, so this is a different explanation from third person perspective.
  From third person, include agent's perspective.

  So, if mind the gap, instances of \qzS{}, and argument goes through, then it seems there's a problem.
\end{note}

\begin{note}
  Example:

  \citeauthor{Hieronymi:2011aa}
  \begin{quote}
    It seems appropriate that in the explanation of her action the agent’s activities should, so to speak, ‘stand in’ for those (purported) facts that she takes to be reason-giving.
    Her taking them to be reasons explains her action.
    Thus, she is accountable for her actions---she, not the facts that call for action, brings her action to be.%
    \mbox{ }\hfill\mbox{(\citeyear[425]{Hieronymi:2011aa})}
  \end{quote}

  Taking, qualification.
  But, repeat the argument from before.%
  \footnote{
    \citeauthor{DOro:2013vh} also \citeauthor{Alvarez:2010to}, disjunctivism.
    \begin{quote}
      [O]ur actions are to be explained by \emph{what} we believe in veridical cases and by facts about our believing what we believe when our beliefs are false.%
      \mbox{ }\hfill\mbox{(\citeyear[30]{DOro:2013vh})}
    \end{quote}
    Similar issue, as proposition applies generally.
    No distinction, and so mistakes are still a problem.

    In this case, things are even more difficult.
    And, to be honest, I don't care.
  }

  So, must reduce to question.
\end{note}

\begin{note}
  However, this won't really affect anything going forward.
  For, the point is that we get a relation of support.
  That's the goal.
  Now, if we press these worries from moral realism, then the same is going to hold with relations of support.
  For, need a genuine relation of support.
  But, always possible for the agent to be mistaken about this.
\end{note}

\begin{note}
  Considered \ptivity{}.

  Given \qzS{}, \ptivity{} holds with respect to answers.

  However, is this the right understanding of \qzS{}?
\end{note}

\begin{note}
  Well, \emph{potential} events.
  This might push the worry a little more.
  Might seem as though committed to some kind of modal realism.
  Or, y'know, it's just hard to make sense of whatever this is.
\end{note}

\begin{note}
  Raise similar issue.

  Further, hypothetical.
  Don't take the hypothetical.
  What really matters is perspective.
\end{note}

\begin{note}
  Break \qzS{} away from \qWhy{}.
\end{note}

\begin{note}
  Second option, weaken the question.
  So, what we're interested in is whether the agent thinks they're going to conclude.

  So, now, we're going the weaker question.
  Now, because this is weaker, it doesn't matter, from the agent's perspective `properly' understood, what would happen.

  Weaker question, then need this distinction.
  Hence, surely ask what's going to happen.
  Hence, best thing we get here is failure for the present question to have any relation to \qWhy{}.

  This, argued for in \autoref{cha:zS}.
  Sufficient motivation.
\end{note}

\begin{note}
  This is difficult.
  The first idea is to consider negative cases.
  So, for example, lost keys.
  Here, things look straightforward.
  It really is a problem for the agent that they \emph{might} conclude otherwise.

  So, this suggests to me that in cases where the same question seems to hold, what matters to the agent is that the \emph{would not} conclude otherwise.

  However, I think this falls short of a suitable response.

  For, this is just switching perspective.
  If two different perspectives on the same thing, and these conflict, this doesn't favour either perspective.
\end{note}

\begin{note}
  So, abstract from content.
\end{note}

\begin{note}
  Well, motivation regarding truth isn't so good.
\end{note}

\begin{note}
  Look, in contrast, here this really is not possibility of a stronger answer.
  The thing about making the distinction with respect to \qzS{} is that we have something to work with.
  When asking whether \qzS{} makes sense, whether there could be a weaker question, there is nothing else to make sense with.

  Before, the distinction was interesting because it had the potential to show that~\autoref{prop:PWEs} does not hold.
  Here, there's no possibility of anything like this.
\end{note}

\begin{note}
  So, no pursuing this any further.
\end{note}

\section{Reflections}
\label{sec:reflections}

\begin{note}
  Something being a \requ{} does a lot of work.

  However, just comes down to the idea that, whether or not would conclude matters.

  The argument is fairly straightforward.

  Consequences might lead to reconsider relation between \qzS{} and \qWhy{}.

  But, I don't think there is sufficient motivation.
\end{note}

%%% Local Variables:
%%% mode: latex
%%% TeX-master: "master"
%%% End:
