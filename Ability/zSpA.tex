\chapter{Positive answers to \qzS{}}
\label{cha:positive-answers}

\nocite{Scriven:1962vq}
\nocite{Woodward:2021ue}
\nocite{Perry:1979vc}
\nocite{Perry:1986aa}

\begin{note}
  Focus of this chapter is a proposition on positive answers to \qzS{}.

  Potential event.

  In following chapter, expand on this with \fc{1}, and relations of support.
\end{note}

\begin{note}
  Proposition is fairly straightforward.
  Being with a simple argument.
  Consider a difficulty.
  Refined argument.

  Difficulty is optional.
  However, clarifies function of the proposition.
  In other words, that objection is part of the interest.
  Distinction raised by objection, and where proposition falls is the other part.
\end{note}

\begin{note}
  Details matters.

  Key point of interest is propositions.
  Arguments matter.

  Present first proposition.
  Then second.
  Argument for second.
  Expand on second.

  Argument for the first in detail.
  Here, make use of ideas from the argument for the second.
\end{note}

\section{Outline}
\label{sec:outline}

\begin{note}
  \begin{itemize}
  \item
    The point is, \requ{1} for any general ability, and these are also \requ{1} for main pairing.
    (%
    Note --- or perhaps emphasise --- here, that the problem is \emph{not} recursive.
    Instead, the problem is about the spread.%
    )
  \end{itemize}
\end{note}

\section{Potential events}
\label{sec:positive-answers-qzs}

\begin{note}

  When concluding \(\pv{\phi}{v}\) from \(\Phi\), an instance of \qzS{} has a positive answer only if, for any \requ{} \(\pvp{\psi}{v'}{\Psi}\) of concluding \(\pv{\phi}{v}\) from \(\Phi\), from the agent's perspective, the agent would conclude \(\pv{\psi}{v'}\) from \(\Psi\).

  Expanding, the following \hyperref[prop:PWEs]{proposition}:

  \begin{restatable}[Potential events]{proposition}{propPotentialEvents}
    \label{prop:PWEs}
    For an agent \vAgent{}, when concluding \(\pv{\phi}{v}\) from \(\Phi\):

    \begin{itemize}
    \item[]
      For any thing \(\qzSaV{}\):
      \begin{enumerate}[label=\alph*., ref=(\alph*)]
      \item
        \label{prop:PWEs:a}
        \(\qzSaV{}\) is a positive answer to \qzS{}.
      \end{enumerate}
      \begin{itemize}
      \item[\emph{Only if}]
        For any proposition-value-premises pairing \(\pvp{\psi}{v'}{\Psi}\):
        \begin{itemize}
        \item[\emph{If}]
          \begin{enumerate}[label=\alph*., ref=(\alph*), resume]
          \item
            \label{prop:PWEs:b}
            \(\pvp{\psi}{v'}{\Psi}\) is a \requ{} of \vAgent{} concluding \(\pv{\phi}{v}\) from \(\Phi\).
          \end{enumerate}
        \item[\emph{then}]
          \begin{enumerate}[label=\alph*., ref=(\alph*), resume]
          \item
            \label{prop:PWEs:c}
            \(\qzSaV{}\) involves, in part, there being a potential event in which \vAgent{} concludes \(\pv{\psi}{v'}\) from \(\Psi\), from \vAgent{}' perspective.
          \end{enumerate}
        \end{itemize}
      \end{itemize}
    \end{itemize}
  \end{restatable}

  \autoref{prop:PWEs}, takes an arbitrary positive answers to \qzS{}, \(\qzSaV{}\), and expresses a \emph{necessary} condition on \(\qzSaV{}\).%
  \footnote{
    Otherwise expressed, \autoref{prop:PWEs} has the following form:

    \begin{quote}
      For an agent \vAgent{}, when concluding \(\pv{\phi}{v}\) from \(\Phi\):
      \begin{quote}
        For any thing \(\qzSaV{}\), (\hyperref[prop:PWEs:a]{a} \emph{only if} (for any \(\pvp{\psi}{v'}{\Psi}\), \emph{if} \hyperref[prop:PWEs:b]{b} \emph{then} \hyperref[prop:PWEs:c]{c}))
      \end{quote}
    \end{quote}
    Alternatively, \autoref{prop:PWEs} may be reformulated in terms of a single conditional by shifting the internal quantifier to scope over the both conditionals, and appealing to \emph{import-export}:
    \begin{quote}
      \begin{quote}
        For any thing \(\qzSaV{}\), and for any \(\pvp{\psi}{v'}{\Psi}\), ((\hyperref[prop:PWEs:a]{a} and \hyperref[prop:PWEs:b]{b}) \emph{only if} \hyperref[prop:PWEs:c]{c})
      \end{quote}
    \end{quote}
  }

  Hence, \autoref{prop:PWEs} states that in order for some thing \(\qzSaV{}\) to be a positive answer to an instance of \qzS{}, \(\qzSaV{}\) involves, in part, a potential event in which \vAgent{} concludes \(\pv{\psi}{v'}\) from \(\Psi\).
\end{note}

\begin{note}
  Corollary, potential event is part of why.
\end{note}

\begin{note}
  Two minor points of clarification, and then why \autoref{prop:PWEs} is difficult.
  Following, argument for \autoref{prop:PWEs}.

  Two minor points of clarification are:
  Necessary condition.
  Potential event.
\end{note}

\begin{note}
  First, the form of \autoref{prop:PWEs} as a necessary condition has syntactic and dialectical motivation.

  The syntactic motivation is straightforward:
  \begin{shiftpar}
    \ref{prop:PWEs:b} and \ref{prop:PWEs:c} are within the scope of quantification over all proposition-value-premises pairings, with \ref{prop:PWEs:b} serving to restrict instances of \ref{prop:PWEs:c} to pairings which are \requ{}.
    And, we have observed that, for an agent, there may be more than one \requ{} of concluding \(\pv{\phi}{v}\) from \(\Phi\).
    Therefore, it is not possible for the potential event of \ref{prop:PWEs:c} to, in general, be identified with \(\qzSaV{}\).%
    \footnote{
      For example, suppose two \requ{1}.
      \(\pvp{\psi}{v'}{\Psi}\) and \(\pvp{\theta}{v''}{\Theta}\).
      A potential event in which \vAgent{} concludes \(\pv{\psi}{v'}\) from \(\Psi\) (trivially) involves a potential event in which \vAgent{} concludes \(\pv{\psi}{v'}\) from \(\Psi\).
      However, such a potential event need not involve the agent concluding \(\pv{\theta}{v''}\) from \(\Theta\).
    }
  \end{shiftpar}

  The dialectical motivation is likewise straightforward:

  \begin{shiftpar}
    We will have no interest in whether or not the collection of potential events for every \requ{} is sufficient for \(\qzSaV{}\) is a positive answer to \qzS{}.
    I suspect such a collection would be sufficient, but nothing will depend on this.
  \end{shiftpar}
\end{note}

\begin{note}
  Second, what is meant by a `potential event'.
  Potential is subjunctive, an instance of a possible event.
  However, interested in answers to \qzS{}.
  Constraints on the event.
  `Potential' is a restriction of possible which satisfies constraints.
\end{note}

\subsection{A simple argument}
\label{cha:zSpA:sec:simple-argument}

\begin{note}
  The simple argument for~\autoref{prop:PWEs} highlights that \qzS{} asks whether a conditional holds, and observes it is not possible for the conditional to hold without~\autoref{prop:PWEs} holding.

  Recall \qzS{}:
  \begin{quote}
    \questionZS*
  \end{quote}
\end{note}

\begin{note}[Simple argument for~\autoref{prop:PWEs}]
  The simple argument for~\autoref{prop:PWEs} is direct.
  Take some arbitrary thing \(\qzSaV{}\), some \(\pvp{\psi}{v'}{\Psi}\), assume \requ{}, potential event.

  In full:


  Consider an agent when concluding \(\pv{\phi}{v}\) from \(\Phi\).
  Positive answer, and let \(\qzSaV{}\) be positive answer.

  Further, assume \(\pvp{\psi}{v'}{\Psi}\) is a \requ{}.
  { \color{red} Seen, a and b just express \requ{}\dots}

  From \qzS{} \ref{question:zs:may-fail}, it follows that from the agent's perspective, the agent would conclude \(\pv{\psi}{v'}\) from \(\Psi\), if \vAgent{} were to attempt to conclude \(\pv{\psi}{v'}\) from \(\Psi\).

  Agent has the option, from \ref{question:zs:option}.
  So, there is a potential event.
  And, given \ref{question:zs:may-fail}, concludes.
\end{note}

\begin{note}[Alternatively, by contradiction]
  No potential event.
  Then, focusing only on \(X\), we have the possibility of \(X\) and there not being a potential witnessing event.
  So, we then have \(X\) and failure to conclude.
  But then \ref{question:zs:may-fail} still.
\end{note}

\section{Difficulty}

\begin{note}
  Flaw.
  From the agent's perspective, and aware of this.
  So, don't get potential event.
  Rather, perspective that there is a potential event.

  Clearest formulation:

  Rather than \ref{prop:PWEs:c} involving potential event:
  {
    \color{red}
    \(\qzSaV{}\) involves, in part, \vAgent{}' perspective that there is a potential event in which \vAgent{} concludes \(\pv{\psi}{v'}\) from \(\Psi\).
  }

  And, hold from perspective potential event, while no opinion on whether there really is a potential event.
\end{note}

\paragraph{Sketch of the difficulty}

\begin{note}
  Knowledge and belief.

  From perspective of agent, it cannot be the case that knows \(\phi\) has value \(v\) while \(\phi\) having value \(v\) is not also part of \qWhy{} the agent knows \(\phi\) has value \(v\).

  Knowledge is factive.

  Know \(\phi\) has value \(v\).
  \(\phi\) has value \(v\).

  From perspective of agent, if consider only state of knowing \(\phi\) has value \(v\), then:
  Either state also gives \(\phi\) has value \(v\).
  Or, unclear whether they really know \(\phi\) has value \(v\).

  The same does not hold with belief.
  Not factive.
  Possibility of error.

  Agent recognises error, and, given this recognition, reflexively appeals to their perspective, rather than the content of their perspective.

  So, possible just the believing.
\end{note}

\begin{note}
  \color{red}
  For a footnote:

  The horse race case from~\textcite{Hawthorne:2016wv}.
  Is belief really this strong?
  But, set this aside.
\end{note}

\begin{note}
  Agent's perspective.
  Distinguish.

  \begin{enumerate}
  \item
    \(\phi\) has value \(v\), from the agent's perspective.
  \item
    \(\phi\) is perceived to have value \(v\), from the agent's perspective.%
    \footnote{
      Or, stated a little more carefully, where \vAgent{} stands for the agent:
      \vAgent{} perceives \(\phi\) to have value \(v\), from \vAgent{}' perspective.
    }
  \end{enumerate}
\end{note}

\paragraph{Dancy and Collins}

\begin{note}
  Now, this argument is incomplete.
  Agent makes the distinction, but this doesn't show \(\phi\) having value \(v\) doesn't matter from the agent's perspective given belief that \(\phi\) has value \(v\).

  Make this distinction from the third person perspective, but whether it really is possible from the first person perspective.

  And, \citeauthor{Dancy:2000aa} suggests links to~\citeauthor{Moore:1993wk}'s paradox (though \citeauthor{Collins:1997wn} does not explicitly consider).
\end{note}

\begin{note}
  Motivating reasons are psychological states of the agent.
  \citeauthor{Dancy:2000aa}%
  \footnote{
    Neither \citeauthor{Dancy:2000aa} nor \citeauthor{Collins:1997wn} give instances.
    Indeed, there are no citations in \textcite{Collins:1997wn}.
  }

  \begin{quote}
    The statement
    \begin{enumerate}[label=(\arabic*), ref=(\arabic*), nosep]
    \item%
      \emph{A}'s reason for \(\phi\)-ing was that \emph{p}
    \end{enumerate}
    can only be true if
    \begin{enumerate}[label=(\arabic*), ref=(\arabic*), resume, nosep]
    \item
      \emph{A} believed that \emph{p}.
    \end{enumerate}
    Therefore
    \begin{enumerate}[label=(\arabic*), ref=(\arabic*), resume, nosep]
    \item
      \emph{A}'s reason for \(\phi\)-ing was that \emph{A} believed that \emph{p}.%
      \mbox{ }\hfill\mbox{(\citeyear[102]{Dancy:2000aa})}
    \end{enumerate}
  \end{quote}

  As \citeauthor{Dancy:2000aa} observes, this argument is not obviously valid.
  \citeauthor{Dancy:2000aa} doesn't suggest details.
  However.
  Invalid: Shift in subject matter.
  Not clearly valid, going from (2) to (3).

  \begin{quote}
    Its conclusion is not that /Vs reason for <p-ing was his believing that p, but that his reason was that he believed that p.
    \dots
    [the] reason [the argument] ‘discovers’ is not itself a psychological state of the agent but the ‘fact’ that the agent was in such a state.
  \end{quote}

  This argument doesn't do anything.

  \citeauthor{Dancy:2000aa} highlights misunderstanding of conclusion.

  \begin{quote}
    \begin{enumerate}[label=(\arabic*\(^{\ast}\)), ref=(\arabic*\(^{\ast}\))]
      \setcounter{enumi}{2}
    \item
      \emph{A}'s reason for \(\phi\)-ing was \emph{A}'s believing that \emph{p}.%
      \mbox{ }\hfill\mbox{(\citeyear[102]{Dancy:2000aa})}
    \end{enumerate}
  \end{quote}

  As expressed by \citeauthor{Collins:1997wn} (\citeyear{Collins:1997wn}):
  \begin{quote}
    The perspective of the agent, when rightly interpreted, is not a call for introspectible deteterminants of action.
    It is a reminder that it is objective circumstances \emph{as apprehended by the agent} that are relevant.
    The perspective is not the subject matter.
    An agent makes statements about the objective circumstances as he understands them.
    This qualification: `as he understands them' is not a shift to the mental realm.%
    \mbox{ }\hfill\mbox{(\citeyear[120]{Collins:1997wn})}
  \end{quote}
\end{note}

\begin{note}
  \citeauthor{Dancy:2000aa}, three objections.
  First two pair of objections \citeauthor{Dancy:2000aa} raises concern a normative constraint on reasons.
  Roughly, \citeauthor{Dancy:2000aa} argues that believing \emph{p} is never (or rarely) a good reason to \(\phi\).
  Rather, it is \emph{p} --- what is believed --- that is a good reason.
  (\citeyear[107]{Dancy:2000aa})

  No opinion on this pair.
  Normative is beyond the scope of this document.

  Independent on whether the pair of objections succeed, difficulty is qualification `\emph{rarely}'.
\end{note}

\begin{note}
  Third objection is the argument of \citeauthor{Collins:1997wn} (\citeyear{Collins:1997wn}).

  \citeauthor{Dancy:2000aa} summarises:
  \begin{quote}
    No explanation that obliterated that endorsement would be the correct explanation of the action, since it would fail to give the agent’s perspective on things, and hence fail to capture the light in which the action was done.
    \citeyear[108]{Dancy:2000aa}
  \end{quote}
\end{note}

\begin{note}
  However, delicate.
  \begin{quote}
    We could introduce psychological matters if we mean that they are the things that make his situation and his course of action intelligible to an agent.
    But if objective circumstances are what make his own action intelligible to the agent then we do not depart from the agent’s perspective in putting forward objective circumstances in the context of reason-giving.%
    \mbox{ }\hfill\mbox{(\citeyear[120]{Collins:1997wn})}
  \end{quote}

  In short, \citeauthor{Collins:1997wn}' argument rests on sufficiently strong attitude.%
  \footnote{
    For \citeauthor{Collins:1997wn}, belief is such an attitude.
    From the preface:
    \begin{quote}
      Wherever an agent correctly adduces a belief that an objective circumstance obtains in explaining his action, a de-psychologizing restatement that merely makes the objective claim must be ascribable to the agent.%
      \mbox{ }\hfill\mbox{(\citeyear[120]{Collins:1997wn})}
    \end{quote}
  }

  However, our question is whether something weaker than belief.
\end{note}

\paragraph{\ptivity{2}}

\begin{note}
  To move from this to what we want in general, need `\emph{\ptivity{0}}'%
  \footnote{
    The term is a play on factivity.

    Factivity, know \(\pv{\phi}{v}\) only if \(\phi\).
    \ptivity{2}: drop perspective.

    Unlike factivity, converse seems straightforwardly true.
  }

  \begin{principle}[\ptivity{2}]
    \label{def:perspectivity}
    For an agent \vAgent{}, proposition \(\phi\), and value \(v\):

    Qualifier on relation between \vAgent{}'s perspective \(\mathbb{P}\) satisfies \ptivity{} only if:

    \begin{enumerate}[noitemsep]
    \item[\emph{If}]
      \begin{enumerate}[label=\alph*., ref=(\alph*)]
      \item
        \(\phi\) is \(\mathbb{P}\)'d to have value \(v\) from \vAgent{}' perspective.
      \end{enumerate}
    \item[\emph{then}]
      \begin{enumerate}[label=\alph*., ref=(\alph*), resume]
      \item
        \(\phi\) has value \(v\) from \vAgent{}' perspective.
      \end{enumerate}
    \end{enumerate}
    \vspace{-\baselineskip}
  \end{principle}
\end{note}

\paragraph{\ptivity{2} does not hold in general}

\begin{note}
  Clear examples.

  Consider looking at a visual illusion.
  And, consider:
  My perspective is X
  Illusion is working.

  To illustrate, consider the variation of the \citeauthor{Zollner:1860vx} illusion given in \autoref{fig:zollner-illusion}.

  \begin{enumerate}
  \item
    The two lines are not parallel.
  \item
    The illusion is working.
  \end{enumerate}

  \begin{figure}
    \centering
    \def\svgwidth{\columnwidth}
    \input{ZIOh.pdf_tex}
    \caption{A variation of the \citeauthor{Zollner:1860vx} illusion --- modification of ~\textcite{Fibonacci:2007vj}}
    \label{fig:zollner-illusion}
  \end{figure}
\end{note}

\begin{note}
  Of course, really should be:

  \begin{enumerate}
  \item
    From my perspective, the two lines are not parallel.
  \item
    The illusion is working.
  \end{enumerate}

  However, if \ptivity{} holds, then, the two lines are parallel.
\end{note}


\begin{note}
  So, doesn't hold in general.
\end{note}

\begin{note}
  Failure of \ptivity{} doesn't do much, illusions are difficult.
  But, means can't appeal to \ptivity{} in general.
\end{note}

\begin{note}[Detailed example]
  Distinct between representation and what is represented.

  So, clock.

  Look at the clock.
  Time.
  Late.
  Start rushing.

  Is it 9:15a, or perspective?

  Clocks can be wrong.
  Feels the clock may be wrong.
  Looks around apartment for their watch (they lost it).
  Hmpf.

  Doesn't matter.
  Only have the clock to go by.

  Strengthen, doubts about the clock.
  But, do choice.

  From perspective, time in 9:15a.
  But, this can't be quite right.
  Nothing beyond perspective.
  Operative, links to action of rushing around.
  But, perspective is not that the time is 9:15a because I don't do anything other than treat the time as being 9:15a.

  Here, I think it's plausible that from the agent's perspective, 9:15a, but at the same time, the agent's reasoning is sufficiently detached from whether it is 9:15a.
  Perspective matters, but content of perspective does not matter.
\end{note}

\begin{note}
  Point here is failure to replace with any other content.
  Seems just as plausible to adjust the attitude.
\end{note}

\begin{note}[Other examples]
  Why sad?
  They said.
  I think they said.

  Why did the food taste salty.
  Answer, too much salt.
  What answers is, in part, excess salt.
  Well, might be something else.

  {
    \color{red}
    These are hard.
    They don't seem to go either way.
  }
\end{note}

\subsection{Where we are}
\label{sec:where-we-are}

\begin{note}
  \ptivity{2}.

  Some counterexamples.

  In need of an argument that for \(\mathbb{P}\) with positive answer, \ptivity{} holds.
\end{note}

\subparagraph{Options not taken (optional)}

\subparagraph{Natural language}

\begin{note}
  Straightforward expression of whether or not \ptivity{} holds.
\end{note}

\begin{note}
  Cypher, talking to a representative of the \hyperlink{cite.Descartes:1996vp}{Cartesian} demon.%
  \footnote{
    Whether get issue of \ptivity{0} from \citeauthor{Descartes:1996vp} is difficult.

    \citeauthor{Lichtenberg:1991tf} against Descartes.

    \begin{quote}
      \emph{Es denkt}, sollte man sagen, so wie man sagt: \emph{es blitzt}.
      Zu sagen \emph{cogito}, ist schon zu viel, so bald man es durch \emph{Ich denke} \"{U}bersetzt.
      \mbox{ }\hfill\mbox{(\citeyear[412]{Lichtenberg:1991tf}/K76)}
    \end{quote}

    \citeauthor{Zoller:1992ud}'s translation follows:
    \begin{quote}
      One should say, \emph{it thinks}, just as one says, \emph{it lightens}.
      It is already saying too much to say \emph{cogito}, as soon as one translates it as \emph{I think}.
      \mbox{ }\hfill\mbox{(\citeyear[418]{Zoller:1992ud})}
    \end{quote}
  }
  \begin{quote}
    {
    \ttfamily

    A fork stabs the cube of meat and we follow it up to the  \\
    face of Cypher.

    \qquad\qquad CYPHER

    \qquad You know, I know that this steak \\
    \indent\qquad doesn't exist. I know when I \\
    \indent\qquad put it in my mouth, the Matrix is \\
    \indent\qquad telling my brain that it is juicy \\
    \indent\qquad and delicious. After nine years, \\
    \indent\qquad do you know what I've realized?

    He shoves it in, eyes rolling up, savoring the tender beef \\
    melting in his mouth.

    \qquad\qquad CYPHER

    \qquad Ignorance is bliss.}%
    \mbox{ }\hfill\mbox{(\citeyear[330--331]{Wachowski:2000uh})}
  \end{quote}

  Ignorance, no conclusion.

  What is going on with this statement?

  Puzzling.

  I know it doesn't exist.
  Know things about it.
  Reference of it is unclear.
\end{note}

\begin{note}[\citeauthor{Descartes:1996vp}]
  Observe, problem for \citeauthor{Descartes:1996vp}.

  Here, two perspectives.

  First, the agent has shifted their perspective.
  There no longer is an external world.

  Second, the agent has weakened their attitude.
  From their perspective, but only their perspective.
\end{note}

\subparagraph{Concluding?}

\begin{note}
  By no means clear, however.

  Suitable for \emph{concluding}?
\end{note}

\begin{note}
  However, nothing really depends on concluding here.
  It's to simplify things.
\end{note}

\subsection{Argument}
\label{sec:argument}

\begin{note}[A \deadEnd{}]
  Piece of terminology for the argument.

  \begin{definition}[A \deadEnd{0}]
    \label{def:dead-end}
    For an agent \vAgent{}:

    \begin{itemize}
    \item
      \vAgent{}' epistemic state is a \emph{\deadEnd{0}} if:
      \begin{itemize}
      \item
        No conclusion without revision, from \vAgent{}' perspective.
      \end{itemize}
    \end{itemize}
    \vspace{-\baselineskip}
  \end{definition}

  Typical instance of dead end is conflicting proposition-value pairings.

  Qualifier, from \vAgent{}' perspective.
  Some difficulty.
  \deadEnd{} prevents agent from concluding.

  In this respect, \deadEnd{} does not need to be genuine.
  For example, further reasoning, so not conflicting.
  Understand as distinct epistemic state, as no longer block.

  Conversely, don't need proposition-value pairs to be in genuine conflict.
\end{note}


\begin{note}
  \begin{argument}
    If premise, alone, then, doesn't show that the agent's present epistemic state is \emph{not} a \deadEnd{}.
    From agent's perspective, hold the premise is true, and still fail to conclude \(\pv{\psi}{v'}\) from \(\Psi\).

    This is clear.

    Not merely premise, but that it's really there.

    More simple example.
    (Here, maybe cite Harman.)
    Know where car is parked, from perspective, and still find car is missing.

    Indeed, here, parallels to knowledge are interesting.
    In the case of knowledge, not only perspective, but also state of the world answers.
    For, need the truth connexion.

    Contrast with belief.

    Important point.
    The agent could be wrong.
    This is all still from the agent's perspective.
  \end{argument}

  Important, this is still from the agent's perspective.
  So, there's no guarantee that there really is a potential witnessing event.%
  \footnote{
    This works even with reduction to knowledge that, as there's no guarantee that the agent really has the knowledge (that).
  }
  What we get is the way in which \qzS{} receives a positive answer from the agent's perspective, and that is, in part, in terms of potential witnessing event, or something that depends on such an event.
\end{note}

\begin{note}
  Illustration.
  From perspective, reasoning is exercise of ability.
\end{note}

\begin{note}
  {
    \color{green}
    Parallel to \citeauthor{Carroll:1895uj}.
    Can't add as premises.
    The difficulty with \citeauthor{Carroll:1895uj} is that the issue is unclear.
    Don't see why adding as a premise does no work, other than acceptance of the Tortoise to form a hypothetical.
    Here, hopefully clear.
  }
\end{note}


\subsection{Literature}
\label{sec:literature}


\subsubsection{Dispositions}
\label{sec:dispositions}

\begin{note}[Parallel between dispositions and ability]
  Consider \citeauthor{Choi:2021wg}'s characterisation of the Simple Conditional Analysis of dispositions:
  \begin{quote}
    An object is disposed to \emph{M} when \emph{C} iff it would \emph{M} if it were the case that \emph{C}.\nolinebreak
    \mbox{}\hfill\mbox{(\citeyear{Choi:2021wg})}
  \end{quote}
  For example, an object is disposed to dissolve when it is placed in water iff the object would dissolve if it were the case that it is placed in water.

  The Simple Conditional Analysis may be challenged, but for our purposes it is adequate.
  We are interested in the broad form of the truth condition, and various more refined analyses share the same broad form.
  Note, in particular, that it being the case that \emph{C} and \emph{M} happening describes an event.
  Given appropriate conditions; salt dissolves, glass breaks, and I mumble when I am tired.
  The key idea is that the property of being disposed to \emph{M} when \emph{C} is analysed in terms of the (possible) event of \emph{M} happening when \emph{C}.

  The parallel to ability is established by noting that ability may also be analysed in terms of a (possible) event, as we have seen.
  In particular, by incorporating volition in the analysans of the Simple Conditional Analysis.
  To illustrate, \citeauthor{Mandelkern:2017aa} trace the Conditional Analysis of ability  to \textcite{Hume:1748tp} and \textcite{Moore:1912te}, among others:
  \begin{quote}
    S can \(\phi\) iff S would \(\phi\) if S tried to \(\phi\)\nolinebreak
    \mbox{}\hfill\mbox{(\citeyear[Cf.][308]{Mandelkern:2017aa})}
  \end{quote}
  Compare to the Simple Conditional Analysis of dispositions:
  The object is some agent \emph{S}, \emph{C} is `S tried to \(\phi\)' and \emph{M} is `S \(\phi\)s' --- it is volition alone which distinguishes the analyses.
  For example, I have the ability to demonstrate that a rectangle with dimensions \(19\text{cm}\) by \(7\text{cm}\) has area \(133\text{cm}^{2}\) only if I would demonstrate that a rectangle with dimensions \(19\text{cm}\) by \(7\text{cm}\) has area \(133\text{cm}^{2}\) if it were the case that I tried that a rectangle with dimensions \(19\text{cm}\) by \(7\text{cm}\) has area \(133\text{cm}^{2}\).
\end{note}

\subsubsection{Doxastic justification}
\label{cha:fcs:sec:dox-just}

\begin{note}
  \citeauthor{Turri:2010aa}

  \begin{quote}
    Necessarily, for all S, \emph{p}, and \emph{t}, if \emph{p} is propositionally justified for S at \emph{t}, then \emph{p} is propositionally justified for S at \emph{t} because S currently possesses at least one means of coming to believe \emph{p} such that, were S to believe \emph{p} in one of those ways, S's belief would thereby be doxastically justified.%
    \mbox{ }\hfill\mbox{(\citeyear[316]{Turri:2010aa})}
  \end{quote}

  Key is that doxastic justification depends on what the agent does.

  \citeauthor{Turri:2010aa}'s focus is on how reasons are used.
  What the agent does.

  Seen with example.

  \begin{quote}
    \begin{enumerate}[label=(P\arabic*)]
      \setcounter{enumi}{4}
    \item
      The Spurs will win if they play the Pistons.
    \item
      The Spurs will play the Pistons.
    \end{enumerate}

    \mbox{}\hfill\(\vdots\)\hfill\mbox{}

    \begin{enumerate}[label=(P\arabic*), resume]
    \item
      Therefore, the Spurs will win.%
    \mbox{ }\hfill\mbox{(\citeyear[317]{Turri:2010aa})}
    \end{enumerate}
  \end{quote}

  Rather than \emph{modus ponens}, `\emph{modus profusus}'.
  Conclude \(r\) from \(p\) and \(q\).
  (\citeyear[317]{Turri:2010aa})

  \begin{quote}
    The way in which the subject performs, the manner in which she makes use of her reasons, fundamentally determines whether her belief is doxastically justified.
    Poor utilization of even the best reasons for believing \emph{p} will prevent you from justifiedly believing or knowing that \emph{p}.%
    \mbox{ }\hfill\mbox{(\citeyear[316]{Turri:2010aa})}
  \end{quote}

  Variant of ~\cite{Prior:1960wh}'s `tonk' connective.
  Though, difference is between connective and rule.
  \(p\) tonk \(q\) would not be propositionally justified.
\end{note}

\begin{note}
  \citeauthor{Turri:2010aa} is similar to \citeauthor{Goldman:1979ui}

  Begin with justification.

  \begin{quote}
    \begin{enumerate}[label=(\arabic*)]
      \setcounter{enumi}{10}
    \item
      Person \emph{S} is \emph{ex ante} justified in believing \emph{p} at \emph{t} if and only if there is a reliable belief-forming operation available to \emph{S} which is such that if \emph{S} applied that operation to this total cognitive state at \emph{t}, \emph{S} would believe \emph{p} at \emph{t}-plus-delta (for a suitably small delta) and that belief would be \emph{ex post} justified.
    \end{enumerate}
  \end{quote}

  Where, sufficient condition for belief would be \emph{ex post} justified:
  \begin{quote}
    \begin{enumerate}[label=(\arabic*)]
      \setcounter{enumi}{4}
    \item
      If S's believing \emph{p} at \emph{t} results from a reliable cognitive belief-forming process (or set of processes), then S's belief in \emph{p} at \emph{t} is justified.%
      \mbox{ }\hfill\mbox{(\citeyear[13]{Goldman:1979ui})}
    \end{enumerate}
  \end{quote}
  Roughly, at least.
  \citeauthor{Goldman:1979ui} refines this a fair bit, but this isn't important.

  Availability of a reliable belief-forming operation!

  Relation here is brittle.
  Account of justification, apply to concluding.
  Well, then all we get is that before concluding, would make sense to conclude only if available.
  Running something like the \citeauthor{Carroll:1895uj} regress, not some state.
  But, this only tells us about suitability to conclude.

  Still, key point is process.

  Another useful thing to highlight is the suitably small delta.
  With \requ{}, this is captured in terms of the option.
\end{note}

\begin{note}
  Significant difference is in the case of justification, we're not interested in the agent's perspective.
  Hence, these accounts are understood in terms of the agent having the ability, roughly.

  With \qzS{}, we're interested in the agent's perspective, and there is no guarantee that the agent really has the ability.
\end{note}

\newpage

\paragraph{Pressing the worry given subject}

\begin{note}
  Well, \emph{potential} events.
  This might push the worry a little more.
  Might seem as though committed to some kind of modal realism.
  Or, y'know, it's just hard to make sense of whatever this is.
\end{note}

\subsubsection{Given question}
\label{sec:given-question}

\begin{note}
  Given the question, weakening doesn't work, because anything weaker is a \requ{}.
\end{note}

\begin{note}
  Return to the idea of a \requ{}.

  Here, the key feature is failing to conclude

  This is the key link.

  Weaker condition doesn't answer.

  So, by same argument as for second proposition, roughly, whatever qualifications are made, still get this.
\end{note}

\subsubsection{Right question}
\label{sec:right-question}

\begin{note}
  Second option, weaken the question.
  So, what we're interested in is whether the agent thinks they're going to conclude.

  So, now, we're going the weaker question.
  Now, because this is weaker, it doesn't matter, from the agent's perspective `properly' understood, what would happen.

  Weaker question, then need this distinction.
  Hence, surely ask what's going to happen.
  Hence, best thing we get here is failure for the present question to have any relation to \qWhy{}.

  This, argued for in \autoref{cha:zS}.
  Sufficient motivation.
\end{note}

\subsubsection{Subtle}
\label{sec:subtle}


\begin{note}
  Slightly more subtle about this.

  Key question is whether or not the agent would conclude.

  Doesn't really matter whether qualify attitude toward concluding, so long as have relation between whatever the qualified attitude is an actually concluding.
\end{note}

\subsubsection{Answers to \qzS{} deviating from \qWhy{}}
\label{sec:probl-link-prop}

\begin{note}
  Here, causal deviance.
\end{note}

\begin{note}
  Problem is, there's no way to guarantee a link between positive answer to \qzS{} and the agent concluding or not refraining from concluding.
\end{note}

\begin{note}
  Argument relies on tying content to explanation.

  In this respect, there is room for an objection.
  Deviant causal chains.
  Point here is that there are cases where these come apart.

  This isn't only a problem for causal theories of reasoning.
  The point is, some instantiation, and so long as act may be caused by something else, then possibly caused by the instantiation.

  So, possible here.

  Well, hold on.
  What is need is the relevance of the content.
  For this objection to work, need to take a theoretical perspective.
  See, in Davidson's case, the idea is fusing these two things together.
  We answer two different questions with a common thing viewed in two ways.

  Still, I think the objection can be pressed!
  Only \emph{really} an explanation is no deviance.
  To the same extent that potential event matters, it matters to the agent that there is no deviance.

  {
    \color{red}
    Resolution is, if deviance, then no agency.
  }

  I think this makes sense, or at least makes enough sense.
  Answers to `why', on this understanding, are tentative.

  Or, rest on presupposition that agent performed the action.

  So, contingent on showing there is no causal deviance.

  This is different to error.
  With error, thing appealed to isn't the case, but appeal still did work.
  Here, it doesn't matter whether or not the case, no work is done.

  In contrast to more typical instances of the problem, don't need to rule out deviant causal chains.
  Instead, just need one instance to fail to hold.
  One instance of non-deviousness.

  Still a problem for a compatible account which avoids.
  For, here, there can't be any direct link from perspective to reason.

  For example, \citeauthor{Hieronymi:2011aa}

    \begin{quote}
      [W]e explain an event that is an action done for reasons by appealing to the fact that the agent took certain considerations to settle the question of whether to act in some way, therein intended so to act, and successfully executed that intention in action.
    [\emph{T}]\emph{his} complex fact, [\dots] is the reason that rationalizes the action---that explains the action by giving the agent’s reason for acting.%
    \mbox{ }\hfill\mbox{(\citeyear[431]{Hieronymi:2011aa})}
  \end{quote}

  So, here, considerations which settle question, and in so settling question.
  Link between settling the question and acting.

  Following \citeauthor{Hieronymi:2011aa}, no room for deviance.
  Too tight.

  In other words, so long as this fact holds, there is no distinction between settling the question and acting.
  Therefore, no deviance.

  Compatible, I think.
  Question is whether in resolving \qzS{} is sufficiently tied to resolving the question \citeauthor{Hieronymi:2011aa} identifies.
  And, plausibly is.
  This is what the motivation for \qzS{} did.

  Trouble is, for our purposes, need at least sufficient conditions for when this complex fact obtains.
  And, no account of this.

  \citeauthor{Hieronymi:2011aa} notes the gaps.

  Some tension.
  These considerations aren't premises.
\end{note}

\begin{note}
  So, the other option is to embrace deviant causal chains.
  Have the content, but this doesn't work in the way the agent thinks it does.

  Example from Davidson.

  The trouble here is that the content and resulting action match.
  So, things make sense from the agent's point of view.

  Deviant, but maybe not so deviant here.

  Systematic deviance, where content is separated from role of mental state.

  But, I see no motivation for this.

  Solution to causal chains doesn't get round this, because the result is a restricted account.
  So, there's no guaranteed trade-off here.
  Trouble is, it seems hard to see a case where this wouldn't be the case.
\end{note}

\begin{note}
  Similar, though distinct.
  ~\cite{Donnellan:1966wt}.

  Attributive and referential.

  With attributive, nothing in mind.
  Similar, here the agent gives up what the time actually is.
  However, for~\citeauthor{Donnellan:1966wt}, the way things are still matter.

  Well, this is not clear.

  So, said something true, even though not referring to anything.
  Of course, from agent's perspective.
\end{note}

\newpage

\begin{note}
  \color{blue}
  Point is potential event in which concludes.

  \autoref{prop:PWEs} may seem straightforward.
  \qzS{} asks whether the agent would conclude.
  So, from agent's perspective, must be the case would conclude.
  And, would conclude only if there is a potential event in which the agent concludes.

  Parallel with more standard case of reason explanation.

  Davidson.
  Pro-attitude.
  Belief.
  So, from agent's perspective, contents of belief.
\end{note}

\subsection{Objections to the argument}
\label{sec:objections-argument}

\subsubsection{Re-expression}
\label{sec:answers-which-are}

\begin{note}
  \begin{itemize}
  \item
    Somewhat simple.
  \item
    Is there a flaw?
  \item
    Well, point is there's something that isn't a proposition-value pair.
  \item
    Parallel?
  \item
    Knowledge how and knowledge that.
  \item
    Same argument applies to dispute, in a fairly straightforward way.
    Specifically, regarding knowledge that, regardless of knowledge how.
  \end{itemize}
\end{note}

\begin{note}
  Sketch of the argument.

  \begin{enumerate}
  \item
    Potential witnessing event in which agent concludes.
  \item
    \label{pwe-iff-kh}
    This is the case if and only if knows how to conclude (if not knowledge, then insufficient grasp on witnessing event).
  \item
    \label{kw-is-kt}
    Knowledge how is a species of knowledge that.
  \item
    Knowledge that \(\varphi\)
  \item
    Knowledge that \(\varphi\) does not involve event.
  \item
    Equivalent.
  \item
    So, at least possible to answer with something that does not involve event.
  \end{enumerate}

  So, replacement.

  Limitation is option to replace.

  Still, this is enough to highlight a flaw in \autoref{prop:PWEs}.

  In addition, given that agent is not literally answering the question, additional argument that this is how to understand.
\end{note}

\begin{note}
  \dots In outline, depends on how the details are filled in.

  \autoref{pwe-iff-kh} and \autoref{kw-is-kt} in particular.

  Grant \autoref{pwe-iff-kh}, explore \autoref{kw-is-kt}.
\end{note}

\begin{note}
  Not just strong intellectualism, but\dots

  Knowing that is not,~\cite{Stalnaker:2012tp} `justified true belief, painted over with a Gettier-proof coating of some kind.' (\citeyear[754]{Stalnaker:2012tp})

  Instead, ~\citeauthor{Stanley:2011ut} (and~\citeauthor{Stalnaker:2012tp}'s) views are compatible with knowing that involving, at least in part, a potential witnessing event.
  Sparing the details, characterisation by~\citeauthor{Weatherson:2017tb}:%
  \footnote{
    \textcite{Weatherson:2017tb} investigates kind of dispositions involved.
  }

  \nocite{Stanley:2012wg}
  \begin{quote}
    Knowing that \emph{p} is not just a matter of having \emph{p} written in a knowledge box somewhere in the brain; it can in part be constituted by active dispositions.%
    \mbox{ }\hfill\mbox{(\citeyear[8]{Weatherson:2017tb})}
  \end{quote}

  \citeauthor{Stalnaker:2012tp} highlights irony.

  Active disposition.
  Hence, potential witnessing event.%
  \footnote{
    Question whether views such as those of \cite{Stalnaker:2012tp} and \citeauthor{Stanley:2011ut} help with argument here.

    Non-committal.
    Roughly, need two things:
    It to be the case that what matters to the agent is viewing conclusion in terms of general disposition.
    And, that have the disposition, rather than obtaining the conclusion as an instance of the disposition.

    So, still a question about \qzS{} and \qWhy{}.
  }
\end{note}

\begin{note}
  So, finding a theory to make this argument work is not immediate.
  Hence, even if valid, not clear sound.
  Still, this won't deter.
\end{note}

\begin{note}
  Point is, that, even if you get this reduction, still need the potential event.
\end{note}

\begin{note}
  Or, still get a \requ{}.
  For, \deadEnd{}.
\end{note}

\section{Reflections}
\label{sec:reflections}

\begin{note}
  Something being a \requ{} does a lot of work.

  However, just comes down to the idea that, whether or not would conclude matters.

  The argument is fairly straightforward.

  Consequences might lead to reconsider relation between \qzS{} and \qWhy{}.

  But, I don't think there is sufficient motivation.
\end{note}

%%% Local Variables:
%%% mode: latex
%%% TeX-master: "master"
%%% End:
