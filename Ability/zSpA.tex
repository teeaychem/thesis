\chapter{Positive answers to \qzS{}}
\label{cha:positive-answers}

\begin{note}
  Focus of this chapter is two key propositions which expand on positive answers to \qzS{}.

  First, potential event.
  Second, how the potential event functions.

  Combined, potential event answers, but is not a premise of the agent's reasoning.

  In following chapter, expand on this with \fc{1}, and relations of support.
\end{note}

\begin{note}
  Details matters.

  Key point of interest is propositions.
  Arguments matter.

  Present first proposition.
  Then second.
  Argument for second.
  Expand on second.

  Argument for the first in detail.
  Here, make use of ideas from the argument for the second.
\end{note}

\section{Outline}
\label{sec:outline}

\begin{note}
  So, the way in which past reasoning relates is by ensuring that the agent would reach the same conclusion.
  About the agent's reasoning.
  \emph{How} rather than \emph{that}.

  Look, what we are getting is that the agent would conclude.
  If something were to happen, then some action would be performed.
  There's no distinction between the answer and performing the act, roughly.
  Or, better put, the answer \emph{is about present reasoning}.
  Answer states that in present reasoning, would not fail.

  In this respect, \fc{}.

  Perhaps obvious, this is what the question asks.
  But, very important.
  Characterisation of the answer in terms of something forward looking.
  \fc{}.
  It is about the agent's present epistemic state, and in particular what the agent's present epistemic state is capable of.

  In other words, ability.
  What answers is ability, in the sense that ability iff would.

  This is very important to the understanding of \fc{}.

  And, I kind of want to have ability as a gloss, while focusing on \fc{} to avoid going into ability in too much detail.

  So, positive answer, then it's the pairing \emph{being} a \fc{}.
  (I should always use this instance of the copula.)
\end{note}

\begin{note}
  An interesting observation here is that in certain this all arises, to a certain extent, because of general abilities.
  General ability spans multiple different proposition-value-premises pairings.
  Hence, all of these function as \requ{1}, so long as the agent has the option.
\end{note}

\section{Outline}

\begin{note}
  \begin{itemize}
  \item
    The point is, \requ{1} for any general ability, and these are also \requ{1} for main pairing.
    (%
    Note --- or perhaps emphasise --- here, that the problem is \emph{not} recursive.
    Instead, the problem is about the spread.%
    )
  \item
    Here, then, ability is both the problem and the answer.
    What's interesting is the way in which ability functions.
    It's not merely \emph{that} the agent has the ability.
    Instead, it \emph{is} the ability.
  \end{itemize}
\end{note}

\begin{note}
  Somewhere at the end, or perhaps on a speculative chapter:
  Deduction theorem for reasoning.
  And, support, so why not conclude from without witnessing the reasoning.
  This would just be witnessing a foregone-conclusion.
\end{note}

\section{Fragments}

\begin{note}
  A \fc{} is not something which functions as a premise.
  This is a category mistake.

  So, answer to \qzS{} can't be a premise, because, in principle, for any premise, if we have a \requ{}, then any premise which states that the \requ{} is satisfied is subject to the \requ{}.

  I mean, do I get an infinite regress here?
  Even if I do, I don't think it's important.
  The point is, even granting that the agent is correct, hum.
  The \requ{} remains, but that is true in both cases.
  The task isn't getting rid of the \requ{}.
  Rather, the task is to show that the agent would conclude.
  But, now, fails to answer, because think!

  \fc{} answers by pointing to the reasoning.
  Alternative answers by not doing so.
  As you have not pointed to the reasoning, it remains the case that whatever this is, the \requ{} applies.

  This is probably a better way of doing things.
  I have a clearer understanding of pointing to the reasoning.
  And, with respect to the reasoning, it's clear.
  There's no question about whether the agent would conclude, that is what the agent is pointing to.
  The only question is whether there really is such an event.

  By contrast, if we're not pointing to the reasoning, then\dots
  Whatever it is the agent is pointing to, the agent has the option of appealing to this regardless of whether there is a witnessing event.
  Hence, if there is no witnessing event, then useless???

  \emph{The same \requ{} applies}
  Because, failure to conclude, then this thing is useless.

  This is quite subtle.
  The point is, potential witnessing event.
  Without this, without this doing the work, whatever one thinks of, the same question still applies.
  (No recursion!)

  If one does appeal to the potential witnessing event, then one is pointing to the very thing that matters.

  Now, failure to conclude, then something has gone wrong.
  Yes.
  The key observation is that in the failure case, the alternative thing, whatever this happens to be, persists, or at least may persist.
  It's independent of there being a potential witnessing event.

  This is what shows it doesn't work.

  So, looking.
  If mistake, then bad things all around.

  If not a mistake, then still a problem.
  For, independence of potential witnessing event.
  Therefore, failure would prevent from doing work.

  Thing is, right about potential, done.
  Right about alternative, then still a question regarding the potential.
\end{note}

\section{Potential events}
\label{sec:positive-answers-qzs}

\begin{note}

  When concluding \(\pv{\phi}{v}\) from \(\Phi\), an instance of \qzS{} has a positive answer only if, for any \requ{} \(\pvp{\psi}{v'}{\Psi}\) of concluding \(\pv{\phi}{v}\) from \(\Phi\), from the agent's perspective, the agent would conclude \(\pv{\psi}{v'}\) from \(\Psi\).

  Expanding, the following \hyperref[prop:PWEs]{proposition}:

  \begin{proposition}[Potential events]
    \label{prop:PWEs}
    For an agent \vAgent{}, when concluding \(\pv{\phi}{v}\) from \(\Phi\):

    \begin{itemize}
    \item[]
      For any thing \(\qzSaV{}\):
      \begin{enumerate}[label=\alph*., ref=(\alph*)]
      \item
        \label{prop:PWEs:a}
        \(\qzSaV{}\) is a positive answer to \qzS{}.
      \end{enumerate}
      \begin{itemize}
      \item[\emph{Only if}]
        For any proposition-value-premises pairing \(\pvp{\psi}{v'}{\Psi}\):
        \begin{itemize}
        \item[\emph{If}]
          \begin{enumerate}[label=\alph*., ref=(\alph*), resume]
          \item
            \label{prop:PWEs:b}
            \(\pvp{\psi}{v'}{\Psi}\) is a \requ{} of \vAgent{} concluding \(\pv{\phi}{v}\) from \(\Phi\).
          \end{enumerate}
        \item[\emph{then}]
          \begin{enumerate}[label=\alph*., ref=(\alph*), resume]
          \item
            \label{prop:PWEs:c}
            \(\qzSaV{}\) involves, in part, there being a potential event in which \vAgent{} concludes \(\pv{\psi}{v'}\) from \(\Psi\), from \vAgent{}' perspective.
          \end{enumerate}
        \end{itemize}
      \end{itemize}
    \end{itemize}
  \end{proposition}

  {
    \color{red}
    \(\qzSaV{}\) involves, in part, \vAgent{}' judgement that there is a potential event in which \vAgent{} concludes \(\pv{\psi}{v'}\) from \(\Psi\).
  }

  \autoref{prop:PWEs}, takes an arbitrary positive answers to \qzS{}, \(\qzSaV{}\), and expresses a \emph{necessary} condition on \(\qzSaV{}\).%
  \footnote{
    Otherwise expressed, \autoref{prop:PWEs} has the following form:

    \begin{quote}
      For an agent \vAgent{}, when concluding \(\pv{\phi}{v}\) from \(\Phi\):
      \begin{quote}
        For any thing \(\qzSaV{}\), (\hyperref[prop:PWEs:a]{a} \emph{only if} (for any \(\pvp{\psi}{v'}{\Psi}\), \emph{if} \hyperref[prop:PWEs:b]{b} \emph{then} \hyperref[prop:PWEs:c]{c}))
      \end{quote}
    \end{quote}
    Alternatively, \autoref{prop:PWEs} may be reformulated in terms of a single conditional by shifting the internal quantifier to scope over the both conditionals, and appealing to \emph{import-export}:
    \begin{quote}
      \begin{quote}
        For any thing \(\qzSaV{}\), and for any \(\pvp{\psi}{v'}{\Psi}\), ((\hyperref[prop:PWEs:a]{a} and \hyperref[prop:PWEs:b]{b}) \emph{only if} \hyperref[prop:PWEs:c]{c})
      \end{quote}
    \end{quote}
  }

  Hence, \autoref{prop:PWEs} states that in order for some thing \(\qzSaV{}\) to be a positive answer to an instance of \qzS{}, \(\qzSaV{}\) involves, in part, a potential event in which \vAgent{} concludes \(\pv{\psi}{v'}\) from \(\Psi\).
\end{note}

\begin{note}
  Two minor points of clarification, and then why \autoref{prop:PWEs} is difficult.
  Following, argument for \autoref{prop:PWEs}.

  Two minor points of clarification are:
  Necessary condition.
  Potential event.
\end{note}

\begin{note}
  First, the form of \autoref{prop:PWEs} as a necessary condition has syntactic and dialectical motivation.

  The syntactic motivation is straightforward:
  \begin{shiftpar}
    \ref{prop:PWEs:b} and \ref{prop:PWEs:c} are within the scope of quantification over all proposition-value-premises pairings, with \ref{prop:PWEs:b} serving to restrict instances of \ref{prop:PWEs:c} to pairings which are \requ{}.
    And, we have observed that, for an agent, there may be more than one \requ{} of concluding \(\pv{\phi}{v}\) from \(\Phi\).
    Therefore, it is not possible for the potential event of \ref{prop:PWEs:c} to, in general, be identified with \(\qzSaV{}\).%
    \footnote{
      For example, suppose two \requ{1}.
      \(\pvp{\psi}{v'}{\Psi}\) and \(\pvp{\theta}{v''}{\Theta}\).
      A potential event in which \vAgent{} concludes \(\pv{\psi}{v'}\) from \(\Psi\) (trivially) involves a potential event in which \vAgent{} concludes \(\pv{\psi}{v'}\) from \(\Psi\).
      However, such a potential event need not involve the agent concluding \(\pv{\theta}{v''}\) from \(\Theta\).
    }
  \end{shiftpar}

  The dialectical motivation is likewise straightforward:

  \begin{shiftpar}
    We will have no interest in whether or not the collection of potential events for every \requ{} is sufficient for \(\qzSaV{}\) is a positive answer to \qzS{}.
    I suspect such a collection would be sufficient, but nothing will depend on this.
  \end{shiftpar}
\end{note}

\begin{note}
  Second, what is meant by a `potential event'.
  Potential is subjunctive, an instance of a possible event.
  However, interested in answers to \qzS{}.
  Constraints on the event.
  `Potential' is a restriction of possible which satisfies constraints.
\end{note}

\subsection{A simple argument}
\label{cha:zSpA:sec:simple-argument}

\begin{note}
  The simple argument for~\autoref{prop:PWEs} highlights that \qzS{} asks whether a conditional holds, and observes it is not possible for the conditional to hold without~\autoref{prop:PWEs} holding.

  Recall \qzS{}:
  \begin{quote}
    \questionZS*
  \end{quote}
\end{note}

\begin{note}[Simple argument for~\autoref{prop:PWEs}]
  The simple argument for~\autoref{prop:PWEs} is direct.
  Take some arbitrary thing \(\qzSaV{}\), some \(\pvp{\psi}{v'}{\Psi}\), assume \requ{}, potential event.

  In full:


  Consider an agent when concluding \(\pv{\phi}{v}\) from \(\Phi\).
  Positive answer, and let \(\qzSaV{}\) be positive answer.

  Further, assume \(\pvp{\psi}{v'}{\Psi}\) is a \requ{}.
  { \color{red} Seen, a and b just express \requ{}\dots}

  From \qzS{} \ref{question:zs:may-fail}, it follows that from the agent's perspective, the agent would conclude \(\pv{\psi}{v'}\) from \(\Psi\), if \vAgent{} were to attempt to conclude \(\pv{\psi}{v'}\) from \(\Psi\).

  Agent has the option, from \ref{question:zs:option}.
  So, there is a potential event.
  And, given \ref{question:zs:may-fail}, concludes.
\end{note}

\begin{note}[Alternatively, by contradiction]
  No potential event.
  Then, focusing only on \(X\), we have the possibility of \(X\) and there not being a potential witnessing event.
  So, we then have \(X\) and failure to conclude.
  But then \ref{question:zs:may-fail} still.
\end{note}

\begin{note}
  Argument captures the core.
  However, details matter, and expand on this later.
  Get to the important proposition.
\end{note}

\paragraph{The difficulty}

\begin{note}
  For the moment, simple argument.
  However, we will return to this argument in~\autoref{cha:zSpA:sec:return-simple-arg}.

  Looking ahead, subtle flaw.
  From the agent's perspective, and aware of this.
  So, don't get potential event.
  Rather, perspective that there is a potential event.

  Clearest formulation:
  Agent perceives \dots

  And, hold from perspective potential event, while no opinion on whether there really is a potential event.
\end{note}

\begin{note}[Why this matters]
  Why does this matter?

  Broad scope.
  Relation of support.
  Distinction between relation of support, and agent's perspective that there is a relation of support.
\end{note}

\section{The role of the potential event}
\label{sec:no-premise}

\begin{note}
  Stated \ref{prop:PWEs:c}.

  Focus on event.

  This is, delicate.

  From agent's perspective, it is true that there is a potential event in which \dots

  So, functions as a premise.
  In this respect, no problem.
  Agent will witness reasoning.
\end{note}

\begin{note}
  In general?
  No, because here there's the assumption of \(\pvp{\psi}{v'}{\Psi}\) being a \requ{}.

  \begin{proposition}[The role of potential events]
    \label{prop:qzS-ans-event}
    For an agent \vAgent{}, when concluding \(\pv{\phi}{v}\) from \(\Phi\) such that \qzS{} has positive answer:
    \begin{enumerate}
    \item[\emph{If}]
      \begin{enumerate}[label=\alph*., ref=(\alph*)]
      \item \(\pvp{\psi}{v'}{\Psi}\) is a \requ{} of concluding \(\pv{\phi}{v}\) from \(\Phi\).
      \end{enumerate}
    \item[\emph{then}]
      \begin{enumerate}[label=\alph*., ref=(\alph*), resume]
      \item
        Potential event conclude \(\pv{\psi}{v'}\) from \(\Psi\) is not a premise or intermediary step of \vAgent{}' reasoning.
      \end{enumerate}
    \end{enumerate}
  \end{proposition}

  \autoref{prop:qzS-ans-event} builds on~\autoref{prop:PWEs}.%
  \footnote{
    Consider some agent, and assume \qzS{} has a positive answer when the agent is concluding \(\pv{\phi}{v}\) from \(\Phi\).
    Further, assume \(\pvp{\psi}{v'}{\Psi}\) is a \requ{} of concluding \(\pv{\phi}{v}\) from \(\Phi\).
    Then, from~\autoref{prop:PWEs} it follows that, from the agent's point of view there is a potential event in which the agent concludes \(\pv{\psi}{v'}\) from \(\Psi\).
    And,~\autoref{prop:qzS-ans-event} adds that the potential event, from the agent's perspective, is neither a premise nor intermediary step of the agent's reasoning.
  }
\end{note}

\begin{note}
  Proposition is negative.
  Doesn't assign a clear status to the potential event.

  No clear account of the role.
  Reasoning, premises, rules, conclusion.
  But, rule goes from proposition-value pairs to proposition-value pair.

  Still, like a rule.
\end{note}

\begin{note}
  Disposition.

  From agent's perspective, answer, disposition.

  Same problem.
  That disposed also as premise.
  But, clear reading where this isn't a premise.

  Well, maybe in terms of the car breaking down.
  Well, here, non-premise conclusion structure.
  Rather, a statement of how things are.
  Explains in the same way.

  So, with premise an conclusion, it being true explains why it is true.
  With other form of explanation, the event of X explains the event of X.
  Causal explanation.
  \citeauthor{Scriven:1962vq} vs Hempel.
\end{note}


\subsection{Understanding}
\label{sec:understanding}

\paragraph{Making this clear with ability (maybe)}

\begin{note}
  Ability to conclude.
  On the one hand, static perspective on ability.
  On the other hand, dynamic perspective.
\end{note}

\subsection{Argument}
\label{sec:argument-1}

\begin{note}
  Argument for~\autoref{prop:qzS-ans-event}.
\end{note}

\begin{note}[A \deadEnd{}]
  Piece of terminology for the argument.

  \begin{definition}[A \deadEnd{0}]
    \label{def:dead-end}
    For an agent \vAgent{}:

    \begin{itemize}
      \item
        \vAgent{}' epistemic state is a \emph{\deadEnd{0}} if:
        \begin{itemize}
        \item
          No conclusion without revision, from \vAgent{}' perspective.
        \end{itemize}
      \end{itemize}
      \vspace{-\baselineskip}
    \end{definition}

    Typical instance of dead end is conflicting proposition-value pairings.

    Qualifier, from \vAgent{}' perspective.
    Some difficulty.
    \deadEnd{} prevents agent from concluding.

    In this respect, \deadEnd{} does not need to be genuine.
    For example, further reasoning, so not conflicting.
    Understand as distinct epistemic state, as no longer block.

    Conversely, don't need proposition-value pairs to be in genuine conflict.
\end{note}

\begin{note}
  \begin{argument}
    Well, it's an event!
    Wow, this doesn't work, because I don't have the option of making a distinction.
    Yikes.
  \end{argument}
  Still, direct argument.
  \begin{argument}
    {
      \color{green}
      So, another way to do this:
      Construct so that the answer has not yet been given.
      Now, we have \(\Phi\).
      Problem is, adding anything to \(\Phi\), there question still remains.
      For, failure would dead-end anything added.

      In \emph{this} presentation, parallel to \citeauthor{Carroll:1895uj}.
    }
  \end{argument}

  Important, this is still from the agent's perspective.
  So, there's no guarantee that there really is a potential witnessing event.%
  \footnote{
    This works even with reduction to knowledge that, as there's no guarantee that the agent really has the knowledge (that).
  }
  What we get is the way in which \qzS{} receives a positive answer from the agent's perspective, and that is, in part, in terms of potential witnessing event, or something that depends on such an event.
\end{note}

\paragraph{Premises and past conclusions}

\begin{note}[Premises]
  So, as we have seen with testimony, status of a premises blocks a \requ{}.

  Whether the same may hold for this problem.

  It's the case that, part of agent's present epistemic state that they would conclude.

  Problem is, if attempt and fail, then this premise does nothing.
  Their present epistemic state develops into a dead-end.
\end{note}

\begin{note}[Note!]
  This doesn't hold in general, for all premises.

  In particular, premise is past conclusion.

  Consider cases of being somewhat impaired, e.g., via exhaustion.
  Indeed, exhaustion is interesting.
  Basic consistency checks.
  Should be the case that conclude A, but just concluded \emph{not}-A, or something like this\dots

  Denying that past continues to secure in all instances.
  So, just need the potential to revise perspective on previous conclusion.
\end{note}

\begin{note}[Past conclusions and positive answers]
  \begin{itemize}
  \item
    \emph{If} positive answer due to some past conclusion \emph{then} possible for the agent to conclude.
  \end{itemize}
  This conditional is immediate, because \qzS{} is about whether the agent would conclude, given that they have the option.
  \begin{itemize}
  \item
    \emph{If} possible to conclude, \emph{then} fact is insufficient.
  \end{itemize}
  This conditional is also immediate, because if the agent failed to conclude, then the fact that they had concluded wouldn't go anywhere.
\end{note}


\subsection{Literature}
\label{sec:literature}

\subsubsection{\citetitle{Carroll:1895uj}}
\label{sec:carroll}

\begin{note}
  \color{red}

  ~\cite{Besson:2018wz} in here somewhere.
\end{note}

\begin{note}
  \color{red}
  Point here is role of rule of inference is key.
  And,~\autoref{prop:PWEs} is observing this.
\end{note}

\begin{note}
  Similar to \citeauthor{Carroll:1895uj}.
  \begin{quote}
    Logic would take you by the throat, and \emph{force} you to do it!%
    \mbox{ }\hfill\mbox{(\citeyear[280]{Carroll:1895uj})}
  \end{quote}
  Looking at something static.
  Achilles fails to convey this to the Tortoise, arguably through some fault of Achilles' own.

  In parallel, we could stack up additional passives in the same way, but there's little interest in doing so.
  The point is the base \requ{} is not satisfied.
\end{note}

\begin{note}
  So, with \citeauthor{Carroll:1895uj}, we get a rule of inference, great.

  \citeauthor{Wieland:2013vf} characterises the general understanding of \textcite{Carroll:1895uj} in terms of two lessons:
  \begin{quote}
    [T]he negative lesson is that if you add ever more premises to an argument \dots, then you will never demonstrate that its conclusion follows logically.%
    \mbox{ }\hfill\mbox{(\citeyear[984]{Wieland:2013vf})}
  \end{quote}

  Parallel, static answers, still option for concluding otherwise.

  \begin{quote}
    [T]he positive lesson is that rules of inference, rather than premises of the form `if premises such and such are true, then the conclusion is true', will do the job.%
    \mbox{ }\hfill\mbox{(\citeyear[984]{Wieland:2013vf})}
  \end{quote}

  Parallel, the dynamic status of a rule.
\end{note}

\begin{note}
  Similar, but a little different.
\end{note}

\begin{note}
  No regress.

  Following \citeauthor{Wieland:2013vf}:

  \begin{quote}
    \begin{itemize}[noitemsep]
    \item[IR]
      For any item x of a certain type, S \(\varphi\)-s x only if
      \begin{enumerate}[label=(\roman*),noitemsep]
      \item
        there is a new item y of that same type, and
      \item
        S \(\varphi\)-s y.%
        \mbox{ }\hfill\mbox{(\citeyear[996]{Wieland:2013vf})}
      \end{enumerate}
    \end{itemize}
  \end{quote}

  Now, concluding, versus would conclude.
  However, focus is before concluding.
  So, would conclude and would conclude.

  Difficulty is, it's not at all clear this is the case.
  Need to be sure that there is a \requ{} for any \requ{}.
  Yet, from agent's perspective.
\end{note}

\begin{note}
  Interesting thing here is that it's not `just' the rule.

  \emph{And}, important difference is that the agent isn't moving from premises to conclusion.
  Following the standard interpretation, \citeauthor{Carroll:1895uj} gets us that there's a rule in play when agent concludes.
  (Or, more strictly, modus ponens\dots)
  But, this is very different from something similar being active when drawing some other conclusion.

  So, there is a link to \citeauthor{Carroll:1895uj}, but it is somewhat indirect.
  Still, this should soften the conclusion.

  In short, with \citeauthor{Carroll:1895uj} it's the rule.
  Here, it's the ability to employ the rule.
\end{note}


\subsubsection{Dispositions}
\label{sec:dispositions}

\begin{note}[Parallel between dispositions and ability]
  Consider \citeauthor{Choi:2021wg}'s characterisation of the Simple Conditional Analysis of dispositions:
  \begin{quote}
    An object is disposed to \emph{M} when \emph{C} iff it would \emph{M} if it were the case that \emph{C}.\nolinebreak
    \mbox{}\hfill\mbox{(\citeyear{Choi:2021wg})}
  \end{quote}
  For example, an object is disposed to dissolve when it is placed in water iff the object would dissolve if it were the case that it is placed in water.

  The Simple Conditional Analysis may be challenged, but for our purposes it is adequate.
  We are interested in the broad form of the truth condition, and various more refined analyses share the same broad form.
  Note, in particular, that it being the case that \emph{C} and \emph{M} happening describes an event.
  Given appropriate conditions; salt dissolves, glass breaks, and I mumble when I am tired.
  The key idea is that the property of being disposed to \emph{M} when \emph{C} is analysed in terms of the (possible) event of \emph{M} happening when \emph{C}.

  The parallel to ability is established by noting that ability may also be analysed in terms of a (possible) event, as we have seen.
  In particular, by incorporating volition in the analysans of the Simple Conditional Analysis.
  To illustrate, \citeauthor{Mandelkern:2017aa} trace the Conditional Analysis of ability  to \textcite{Hume:1748tp} and \textcite{Moore:1912te}, among others:
  \begin{quote}
    S can \(\phi\) iff S would \(\phi\) if S tried to \(\phi\)\nolinebreak
    \mbox{}\hfill\mbox{(\citeyear[Cf.][308]{Mandelkern:2017aa})}
  \end{quote}
  Compare to the Simple Conditional Analysis of dispositions:
  The object is some agent \emph{S}, \emph{C} is `S tried to \(\phi\)' and \emph{M} is `S \(\phi\)s' --- it is volition alone which distinguishes the analyses.
  For example, I have the ability to demonstrate that a rectangle with dimensions \(19\text{cm}\) by \(7\text{cm}\) has area \(133\text{cm}^{2}\) only if I would demonstrate that a rectangle with dimensions \(19\text{cm}\) by \(7\text{cm}\) has area \(133\text{cm}^{2}\) if it were the case that I tried that a rectangle with dimensions \(19\text{cm}\) by \(7\text{cm}\) has area \(133\text{cm}^{2}\).
\end{note}

\subsubsection{Doxastic justification}
\label{cha:fcs:sec:dox-just}

\begin{note}
  \citeauthor{Turri:2010aa}

  \begin{quote}
    Necessarily, for all S, \emph{p}, and \emph{t}, if \emph{p} is propositionally justified for S at \emph{t}, then \emph{p} is propositionally justified for S at \emph{t} because S currently possesses at least one means of coming to believe \emph{p} such that, were S to believe \emph{p} in one of those ways, S's belief would thereby be doxastically justified.%
    \mbox{ }\hfill\mbox{(\citeyear[316]{Turri:2010aa})}
  \end{quote}

  Key is that doxastic justification depends on what the agent does.

  \citeauthor{Turri:2010aa}'s focus is on how reasons are used.
  What the agent does.

  Seen with example.

  \begin{quote}
    \begin{enumerate}[label=(P\arabic*)]
      \setcounter{enumi}{4}
    \item
      The Spurs will win if they play the Pistons.
    \item
      The Spurs will play the Pistons.
    \end{enumerate}

    \mbox{}\hfill\(\vdots\)\hfill\mbox{}

    \begin{enumerate}[label=(P\arabic*), resume]
    \item
      Therefore, the Spurs will win.%
    \mbox{ }\hfill\mbox{(\citeyear[317]{Turri:2010aa})}
    \end{enumerate}
  \end{quote}

  Rather than \emph{modus ponens}, `\emph{modus profusus}'.
  Conclude \(r\) from \(p\) and \(q\).
  (\citeyear[317]{Turri:2010aa})

  \begin{quote}
    The way in which the subject performs, the manner in which she makes use of her reasons, fundamentally determines whether her belief is doxastically justified.
    Poor utilization of even the best reasons for believing \emph{p} will prevent you from justifiedly believing or knowing that \emph{p}.%
    \mbox{ }\hfill\mbox{(\citeyear[316]{Turri:2010aa})}
  \end{quote}

  Variant of ~\cite{Prior:1960wh}'s `tonk' connective.
  Though, difference is between connective and rule.
  \(p\) tonk \(q\) would not be propositionally justified.
\end{note}

\begin{note}
  \citeauthor{Turri:2010aa} is similar to \citeauthor{Goldman:1979ui}

  Begin with justification.

  \begin{quote}
    \begin{enumerate}[label=(\arabic*)]
      \setcounter{enumi}{10}
    \item
      Person \emph{S} is \emph{ex ante} justified in believing \emph{p} at \emph{t} if and only if there is a reliable belief-forming operation available to \emph{S} which is such that if \emph{S} applied that operation to this total cognitive state at \emph{t}, \emph{S} would believe \emph{p} at \emph{t}-plus-delta (for a suitably small delta) and that belief would be \emph{ex post} justified.
    \end{enumerate}
  \end{quote}

  Where, sufficient condition for belief would be \emph{ex post} justified:
  \begin{quote}
    \begin{enumerate}[label=(\arabic*)]
      \setcounter{enumi}{4}
    \item
      If S's believing \emph{p} at \emph{t} results from a reliable cognitive belief-forming process (or set of processes), then S's belief in \emph{p} at \emph{t} is justified.%
      \mbox{ }\hfill\mbox{(\citeyear[13]{Goldman:1979ui})}
    \end{enumerate}
  \end{quote}
  Roughly, at least.
  \citeauthor{Goldman:1979ui} refines this a fair bit, but this isn't important.

  Availability of a reliable belief-forming operation!

  Relation here is brittle.
  Account of justification, apply to concluding.
  Well, then all we get is that before concluding, would make sense to conclude only if available.
  Running something like the \citeauthor{Carroll:1895uj} regress, not some state.
  But, this only tells us about suitability to conclude.

  Still, key point is process.

  Another useful thing to highlight is the suitably small delta.
  With \requ{}, this is captured in terms of the option.
\end{note}

\begin{note}
  Significant difference is in the case of justification, we're not interested in the agent's perspective.
  Hence, these accounts are understood in terms of the agent having the ability, roughly.

  With \qzS{}, we're interested in the agent's perspective, and there is no guarantee that the agent really has the ability.
\end{note}

\subsubsection{Ryle}

\begin{note}
  Ideas regarding \citeauthor{Ryle:1946tu}'s distinction between knowing \emph{how} and knowing \emph{that} (Cf.~\citeyear{Ryle:1946tu}).

  Now, I confess my understanding of \citeauthor{Ryle:1946tu}'s distinction is limited --- I have not taken whatever opportunities I have had to read through \citeauthor{Ryle:1946tu}'s work.%
  \footnote{
    Though, I understand enough from passing commentary to note that the idea \emph{I} am perusing here does not, strictly, require that knowledge how and knowledge that are distinct kinds of knowledge.
    (See~\textcite{Pavese:2022up} for more!)
  }

  Following analogy from~\textcite{Ryle:2009us}:

  \begin{quote}
    Knowing `\emph{if p, then q}' is, \dots rather like being in possession of a railway ticket.
    It is having a licence or warrant to make a journey from London to Oxford.
    (Knowing a variable hypothetical or `law' is like having a season ticket.)
    As a person can have a ticket without actually travelling with it and without ever being in London or getting to Oxford, so a person can have an inference warrant without actually making any inferences and even without ever acquiring the premisses from which to make them.%
    \mbox{ }\hfill\mbox{(\citeyear[250]{Ryle:2009us})}
  \end{quote}

  Continuing~\citeauthor{Ryle:2009us}'s analogy, in the case of positive answers to \qzS{}:
  What matters is that the agent is currently in possession of the (season) ticket.

  Even if current possession of the (season) ticket is knowledge that, it is present knowledge.
  And, present without being applied.
\end{note}


\section{The argument for~\autoref{prop:PWEs}}
\label{cha:zSpA:sec:return-simple-arg}

\begin{note}
  Return to simple argument from \autoref{cha:zSpA:sec:simple-argument}.

  Simple argument gets the idea.

  In this section, consider and respond to possible objection.

  Roughly:

  Simple argument, get from agent's perspective.
  However, insufficient to show potential event.
  Only get that the agent has some attitude toward potential event.
  And, attitude may be consistent with indifference to whether there really is a potential event.

  Two sub-sections.

  First, motivate the distinction in this quick (counter-)argument.

  Second, argue that~\autoref{prop:PWEs} follows, granting the distinction.
\end{note}

\begin{note}[Why this is important]
  Why this is important.
  Given proposition, potential event.
  Following chapter.
  Relation of support.
  In other words, potential doesn't matter.
  However, qualify.
  So, even granting arguments to follow, not a relation of support proper.
  Hence, failure of overall goal.
\end{note}

\begin{note}
  Unsure on whether the distinction really makes sense.
  This will show.
  Sometimes convinced, other times, hard to see.
  However, possibility is sufficiently important.
  Hence, do best to motivate.
  And, offer response granting.
\end{note}

\subsection{Limitation}
\label{sec:limitation}
\nocite{Perry:1979vc}
\nocite{Perry:1986aa}

\begin{note}
  Agent's perspective.
  Distinguish.

  \begin{enumerate}
  \item
    \(\phi\) has value \(v\), from the agent's perspective.
  \item
    \(\phi\) is perceived to have value \(v\), from the agent's perspective.%
    \footnote{
      Or, stated a little more carefully, where \vAgent{} stands for the agent:
      \vAgent{} perceives \(\phi\) to have value \(v\), from \vAgent{}' perspective.
    }
  \end{enumerate}

  Second, qualifier.
  \emph{perceived}.
  Terminology is arbitrary.
  Introduction to this section, various attitudes.
  Here, perspective already.
  And, that's really what is at issue.

  Omit qualifier `from agent's perspective' and distinction is straightforward.

  Separate, whether \(\phi\) has value \(v\) and whether \(\phi\) has value \(v\) from agent's perspective.
  This is common.
  Whether agent from their perspective has this option.

  For example\dots

  If so, then answer to \qzS{} only in terms of perception.
  Doesn't follow that potential event, from agent's perspective.

  Issue is whether, perceptive without that perspective involving how things are.

  Perception, visual perception, and factive.

  In part, why chosen.

  Visual illusions.

  So, question is whether the same holds here.

  Not exactly the same.
  After inspecting closely, from perspective, illusion does not hold.

  So, seems the agent has a choice.
  Which perspective to adopt.
  And, distinction disappears with choice.

  Whether principle holds.
\end{note}

\begin{note}
  To move from this to what we want in general, need `\emph{\ptivity{0}}'%
  \footnote{
    The term is a play on factivity.

    Factivity, know \(\pv{\phi}{v}\) only if \(\phi\).
    \ptivity{2}: drop perspective.

    Unlike factivity, converse seems straightforwardly true.
  }

  \begin{principle}[\ptivity{2}]
    \label{def:perspectivity}
    For an agent \vAgent{}, proposition \(\phi\), and value \(v\):
    \begin{enumerate}[noitemsep]
    \item[\emph{If}]
      \begin{enumerate}[label=\alph*., ref=(\alph*)]
      \item
        \(\phi\) is perceived to have value \(v\) from \vAgent{}' perspective.
      \end{enumerate}
    \item[\emph{then}]
      \begin{enumerate}[label=\alph*., ref=(\alph*), resume]
      \item
        \(\phi\) has value \(v\) from \vAgent{}' perspective.
      \end{enumerate}
    \end{enumerate}
    \vspace{-\baselineskip}
  \end{principle}
\end{note}

\begin{note}
  Argument for \ptivity{}.

  Agent's perspective, so of course.
  If perspective, then deletion.%
  \footnote{
    \citeauthor{Collins:1997wn} (\citeyear{Collins:1997wn}) presents an argument along these lines.
    \begin{quote}
    The perspective of the agent, when rightly interpreted, is not a call for introspectible deteterminants of action.
    It is a reminder that it is objective circumstances \emph{as apprehended by the agent} that are relevant.
    The perspective is not the subject matter.
    An agent makes statements about the objective circumstances as he understands them.
    This qualification: `as he understands them' is not a shift to the mental realm.%
    \mbox{ }\hfill\mbox{(\citeyear[120]{Collins:1997wn})}
  \end{quote}

  \citeauthor{Dancy:2000aa} summarises:
  \begin{quote}
    No explanation that obliterated that endorsement would be the correct explanation of the action, since it would fail to give the agent’s perspective on things, and hence fail to capture the light in which the action was done.
    \citeyear[108]{Dancy:2000aa}
  \end{quote}

  And, \citeauthor{Dancy:2000aa} suggests links to ~\citeauthor{Moore:1993wk}'s paradox (though \citeauthor{Collins:1997wn} does not explicitly consider).

  However, delicate.
  \begin{quote}
    We could introduce psychological matters if we mean that they are the things that make his situation and his course of action intelligible to an agent.
    But if objective circumstances are what make his own action intelligible to the agent then we do not depart from the agent’s perspective in putting forward objective circumstances in the context of reason-giving.%
    \mbox{ }\hfill\mbox{(\citeyear[120]{Collins:1997wn})}
  \end{quote}
  In short, \citeauthor{Collins:1997wn}' argument rests on belief.
  \begin{quote}
    Wherever an agent correctly adduces a belief that an objective circumstance obtains in explaining his action, a de-psychologizing restatement that merely makes the objective claim must be ascribable to the agent.%
    \mbox{ }\hfill\mbox{(\citeyear[120]{Collins:1997wn})}
  \end{quote}
  Limited to belief, where belief does require.
  For first passage quoted to hold in full generality `As he understands' must involve belief.

  \citeauthor{Collins:1997wn} does not expand too much on what other psychological matters in second passage quote.
  However, our question is whether something weaker than belief.
  }

  Focused on agent recognises \(\phi\) may not have value \(v\).
  However, this doesn't allow the agent to avoid \(\phi\) having value \(v\).

  If this is correct, then \ptivity{}.
  And, the simple argument goes through.

  However, I don't think this is right.
  Perspective, weak.

  This seems in conflict.
  Focus on belief.
  Look at the abstract.
\end{note}

\begin{note}
  Distinct between representation and what is represented.

  So, clock.

  Look at the clock.
  Time.
  Late.
  Start rushing.

  Is it XX:XX, or perspective?

  Clocks can be wrong.
  Feels the clock may be wrong.
  Looks around apartment for their watch (they lost it).
  Hmpf.

  Doesn't matter.
  Only have the clock to go by.

  Strengthen, doubts about the clock.
  But, do choice.

  From perspective, time in 9:15a.
  But, this can't be quite right.
  Nothing beyond perspective.
  Operative, links to action of rushing around.
  But, perspective is not that the time is 9:15a because I don't do anything other than treat the time as being 9:15a.
\end{note}

\begin{note}
  Different to the horse race case from~\textcite{Hawthorne:2016wv}.

  Here, conflict, but still seems perspective.
\end{note}

\begin{note}[\citeauthor{Descartes:1996vp}]
  Observe, problem for \citeauthor{Descartes:1996vp}.

  Here, two perspectives.

  First, the agent has shifted their perspective.
  There no longer is an external world.

  Second, the agent has weakened their attitude.
  From their perspective, but only their perspective.
\end{note}

\begin{note}
  \begin{quote}
    A fork stabs the cube of meat and we FOLLOW it UP TO the face of Cypher.

    CYPHER

    You know, I know that this steak doesn't exist.
    I know when I put it in my mouth, the Matrix is telling my brain that it is juicy and delicious.
    After nine years, do you know what I've realized?

    He shoves it in, eyes rolling up, savoring the tender beef melting in his mouth.

    CYPHER
    Ignorance is bliss.
  \end{quote}

  Perspective, but just perspective.
\end{note}


\begin{note}[Other examples]
  Why sad?
  They said.
  I think they said.

  Why did the food taste salty.
  Answer, too much salt.
  What answers is, in part, excess salt.
  Well, might be something else.

  {
    \color{red}
    These are hard.
    They don't seem to go either way.
  }
\end{note}

\begin{note}
  Similar, though distinct.
  ~\cite{Donnellan:1966wt}.

  Attributive and referential.

  With attributive, nothing in mind.
  Similar, here the agent gives up what the time actually is.
  However, for~\citeauthor{Donnellan:1966wt}, the way things are still matter.
\end{note}


\begin{note}
  Variation on factivity.
\end{note}


\paragraph{Response in two parts}

\begin{note}
  Two things.

  Whether failure is compatible with \qzS{} as stated.

  Whether \qzS{} should be revised.
\end{note}

\begin{note}
  First, argument similar to second key proposition here.
\end{note}

\begin{note}
  Second, \qzS{} is fine.
\end{note}

\newpage

\begin{note}
  First is given, this is what we're interested in.

  Second, this comes from \citeauthor{Descartes:1996vp}.

  Third, reverse \citeauthor{Descartes:1996vp}.
  For \citeauthor{Descartes:1996vp}, how to go from perspective to how things actually are.
\end{note}

\begin{note}
  So, this just goes back to \citeauthor{Descartes:1996vp}.
  Look, recognise perspective, but question remains as to whether it really is the case.

  The presentation of this is difficult, as we're dealing with an agent.

  Already speaking about agent's perspective, as \requ{1}, etc.\ may hold from agent's perspective.
\end{note}

\begin{note}
  Second, recognise this is my perspective.

  \cite{Descartes:1996vp}

  \begin{quote}
    \dots
    But for all that I am a thing which is real and truly exists.
    But what kind of a thing?
    As I have just said --- a thinking thing.%
    \mbox{ }\hfill\mbox{(\citeyear[18]{Descartes:1996vp})}
  \end{quote}

  Also, \citeauthor{Lichtenberg:1991tf} against Descartes.%
  \footnote{
    \citeauthor{Zoller:1992ud}'s translation follows:
    \begin{quote}
      One should say, \emph{it thinks}, just as one says, \emph{it lightens}.
      It is already saying too much to say \emph{cogito}, as soon as one translates it as \emph{I think}.
      \mbox{ }\hfill\mbox{(\citeyear[418]{Zoller:1992ud})}
    \end{quote}
  }
  \begin{quote}
    \emph{Es denkt}, sollte man sagen, so wie man sagt: \emph{es blitzt}.
    Zu sagen \emph{cogito}, ist schon zu viel, so bald man es durch \emph{Ich denke} \"{U}bersetzt.
    \mbox{ }\hfill\mbox{(\citeyear[412]{Lichtenberg:1991tf}/K76)}
  \end{quote}
\end{note}

\begin{note}
  So, if above is correct, then simple argument is enthymematic.
  Without further statement, relies on \ptivity{0}.
\end{note}

\subsection{Less quick argument}
\label{sec:less-quick-argument}

\subsubsection{Given question}
\label{sec:given-question}

\begin{note}
  Given the question, weakening doesn't work, because anything weaker is a \requ{}.
\end{note}

\begin{note}
  Understanding of `why'.

  Why, from the agent's perspective.

  But, this doesn't entail anything.
  Doesn't follow there is a potential witnessing event.

  Only have something psychological.

  So, whether there is a relation of support is dispensable.

  So, contents is not part of the explanation why from our perspective.
  This is what \citeauthor{Hieronymi:2011aa} terms `Dancy's gap'.
  Worry here is whether we have a clear account of why.
  For, whether the content holds isn't clearly relevant.

  As I understand the gap, issue in terms of our answers to why.
  In some cases, the agent's perspective, and in other cases more than this.

  Or, another way of putting thing, in some cases the agent's perspective is only part.

  Different perspective, even the agent.

  Now, the problem.
  Agent recognises from their perspective.

  So, the idea is, only psychological facts matter.
  Agent recognises this.
  Then, this breaks down the role of the content.
  So, there's no easy movement from content to account of why from agent's perspective.
  For, there's no clear factive presence from the agent's point of view.

  But, this can't work.
  Still matters that the agent thinks.
  And, if non-factive, then no positive answer.

  Push this gap from our perspective, and raise an issue about whether there is anything beyond agent's perspective.
  But, this fails when internalised, at least for positive answers.

  Still, question is irrelevant.
  But, the issue here is that we're talking about mental states.
  It's concluding.

  There's no way to ask this question.
  For, then the question is independent of whether the agent would conclude.
  What the agent would do with other proposition value pairs.
  But, if all that matters is what's going on in the present, this fails.

  Only understand this in terms of our perspective, and then, because reasoning, there's no pressing issue.

  But, I don't think this is relevant.
  For, what we are interested in accounting for is concluding.
\end{note}

\subsubsection{Right question}
\label{sec:right-question}

\begin{note}
  Two points here.
  First, if think weaker question is only question of interest, then there is no distinction.
  Here, relation of support isn't what matters, but judgement that there is a relation of support.

  Second, if distinction, then clearly intelligible.

  Switch these, I think there is a clear distinction.
  However, if you insist, then it doesn't really matter.
\end{note}

\subsubsection{Problem of linking proposition to \qWhy{}}
\label{sec:probl-link-prop}

\begin{note}
  Here, causal deviance.
\end{note}

\newpage

\begin{note}
  Point is potential event in which concludes.

  \autoref{prop:PWEs} may seem straightforward.
  \qzS{} asks whether the agent would conclude.
  So, from agent's perspective, must be the case would conclude.
  And, would conclude only if there is a potential event in which the agent concludes.

  Parallel with more standard case of reason explanation.

  Davidson.
  Pro-attitude.
  Belief.
  So, from agent's perspective, contents of belief.
\end{note}

\begin{note}
  Key move, \(\phi\) has value \(v\) from agent's perspective answers why.
  To, \(\phi\) has value \(v\) from agent's perspective.

  First, Includes a statement about the agent's perspective.
  Second, states how things are from agent's perspective.

  Point is, from perspective, X from perspective.
  Then, X from perspective.

  Doesn't follow that from perspective, X regardless of perspective.

  Don't immediately get any further commitment to X beyond it being the case that X from the agent's perspective.

  Disspaernace of perspective when adopting perspective.

  So, for this arugment I end up using belief, but it's really independent of belief, as you can make the relevant role of agent's perspective as weak as you like.

  Point is, perspective can't limit.

  Problem is dropping the agent's perspective.
\end{note}

\begin{note}
  \citeauthor{Moore:1993wk}
  Here, look, there's something bad about I believe \emph{p} but not-\emph{p}.
  Nothing bad about I believe \emph{p} but I do not know \emph{p}.

  This is the distinction I'm after.
\end{note}

\begin{note}[Distinction]
  It is the case that:
  \begin{itemize}
  \item
    from agent's point of view \emph{p}, and
  \item
    \emph{p} from agent's point of view answers, and
  \item
    \emph{p} does not matter.
  \end{itemize}

  Sollopsistic, to a degree.

  Distinction here between belief and knowledge.
  Note, only that belief we have the option, not claiming this holds for all cases.

  Get the converse in other cases.
  \emph{p}, but don't take my word for it, check yourself.
  Or, that's only my point of view, check yourself.

  Highlights the possibility of mistake.

  However, whether this answers why.
  Point is, possibility of mistake.

  Cite explanation, bridge etc.
  However, there's a difference between belief, and this answering the question.

  Similar observation in \textcite[132ish]{Dancy:2000aa}.
  However, \citeauthor{Dancy:2000aa} is quite different, really.
  From the agent's point of view.
  Case of knowledge.
  From our perspective, fine, mental states, etc.
  However, from agent's point of view, different.

  Well, \citeauthor{Dancy:2000aa} ends up arguing that there's really no difference between factive and non-factive.
  To make this argument, first person is factive.
  But, this is what I'm denying.

  \citeauthor{Collins:1997wn}.
  \textcite[108+]{Dancy:2000aa}
  Moore phenomena.
  As I understand things, it is not the case that the agent has the option of dissenting from \emph{p}.
  However, it is the case that the agent has the option for dissenting from \emph{p} as an answer to \qWhy{}.

  From the agent's perspective, yes, but agent's perspective recognises this is what they consider to be the case.

  Difficult in general.
  No account of general case.
  Here, just answers to \qzS{}.
\end{note}

\begin{note}
  So, some parallel to factivity.

  Why factivity?
  Instead, extract content.

  Main task is getting parallel to factivity, then.
\end{note}

\begin{note}[Argument for~\autoref{prop:PWEs}]
  \begin{argument}
    {
      \color{blue}
      So, this leads to a possible objection.
      The only relevant questions to ask are in terms of proposition-value pairs, or something like this.
      I don't think this works, given all the examples, but it's an option.
    }
  \end{argument}
\end{note}

\subsection{Objections to the argument}
\label{sec:objections-argument}

\begin{note}
  Argument relies two things.
  First, apparent factivity.
  Second, content of \qzS{} relating to event.

  I don't see a way around this.
  However, two possible avenues.
  Deny either of these two things.
  Attempt plausible motivation for both, but show how the motivation fails to move.

  Following, a more interesting chance for failure.
  Does raise problems, but not for this argument.
\end{note}

\subsection{A shortcoming}
\label{sec:shortcoming}

\begin{note}
  \color{red}
  Need to shift.
  This isn't really about positive answers to \qzS{}.
  Rather, this is about how \qzS{} relates to why.
\end{note}

\begin{note}
  Argument relies on tying content to explanation.

  In this respect, there is room for an objection.
  Deviant causal chains.
  Point here is that there are cases where these come apart.

  This isn't only a problem for causal theories of reasoning.
  The point is, some instantiation, and so long as act may be caused by something else, then possibly caused by the instantiation.

  So, possible here.

  Well, hold on.
  What is need is the relevance of the content.
  For this objection to work, need to take a theoretical perspective.
  See, in Davidson's case, the idea is fusing these two things together.
  We answer two different questions with a common thing viewed in two ways.

  Still, I think the objection can be pressed!
  Only \emph{really} an explanation is no deviance.
  To the same extent that potential event matters, it matters to the agent that there is no deviance.

  {
    \color{red}
    Resolution is, if deviance, then no agency.
  }

  I think this makes sense, or at least makes enough sense.
  Answers to `why', on this understanding, are tentative.

  Or, rest on presupposition that agent performed the action.

  So, contingent on showing there is no causal deviance.

  This is different to error.
  With error, thing appealed to isn't the case, but appeal still did work.
  Here, it doesn't matter whether or not the case, no work is done.

  In contrast to more typical instances of the problem, don't need to rule out deviant causal chains.
  Instead, just need one instance to fail to hold.
  One instance of non-deviousness.

  Still a problem for a compatible account which avoids.
  For, here, there can't be any direct link from perspective to reason.

  For example, \citeauthor{Hieronymi:2011aa}

    \begin{quote}
      [W]e explain an event that is an action done for reasons by appealing to the fact that the agent took certain considerations to settle the question of whether to act in some way, therein intended so to act, and successfully executed that intention in action.
    [\emph{T}]\emph{his} complex fact, [\dots] is the reason that rationalizes the action---that explains the action by giving the agent’s reason for acting.%
    \mbox{ }\hfill\mbox{(\citeyear[431]{Hieronymi:2011aa})}
  \end{quote}

  So, here, considerations which settle question, and in so settling question.
  Link between settling the question and acting.

  Following \citeauthor{Hieronymi:2011aa}, no room for deviance.
  Too tight.

  In other words, so long as this fact holds, there is no distinction between settling the question and acting.
  Therefore, no deviance.

  Compatible, I think.
  Question is whether in resolving \qzS{} is sufficiently tied to resolving the question \citeauthor{Hieronymi:2011aa} identifies.
  And, plausibly is.
  This is what the motivation for \qzS{} did.

  Trouble is, for our purposes, need at least sufficient conditions for when this complex fact obtains.
  And, no account of this.

  \citeauthor{Hieronymi:2011aa} notes the gaps.

  Some tension.
  These considerations aren't premises.
\end{note}

\subsection{Issues}
\label{sec:issues}

\begin{note}
  Something wrong.
\end{note}

\subsubsection{Other questions}
\label{sec:other-questions}

\nocite{Smith:1988aa,Smith:1987tz}

\begin{note}
  Similar question:

  By assumption, there is no external world, no food, and no excess salt.
  But, from agent's perspective.

  What happened, what was the case.
\end{note}

\begin{note}
  Put an explicit proposition-value pair in the question.
  Why is it true that the food tasted salty.
  It is true that there was too much salt.

  Still, in part, circumstances.
  True.
  Interpreted.

  No account of true without the circumstances having a role.
\end{note}

\begin{note}
  Any other variation without interpreting, don't get a good answer to the question.
\end{note}

\paragraph{No getting carried away}

\begin{note}[Different values]
  Desire, then something else, likely.
\end{note}

\begin{note}[Mental states]
  Question about why believed, after finding belief is mistaken.
\end{note}

\subsection{Rationalisations}
\label{sec:rationalisations}


\subsubsection{Deviant causal chains}
\label{sec:devi-caus-chains}

\begin{note}
  So, the other option is to embrace deviant causal chains.
  Have the content, but this doesn't work in the way the agent thinks it does.

  Example from Davidson.

  The trouble here is that the content and resulting action match.
  So, things make sense from the agent's point of view.

  Deviant, but maybe not so deviant here.

  Systematic deviance, where content is separated from role of mental state.

  But, I see no motivation for this.

  Solution to causal chains doesn't get round this, because the result is a restricted account.
  So, there's no guaranteed trade-off here.
  Trouble is, it seems hard to see a case where this wouldn't be the case.
\end{note}

\subsubsection{Mistakes and vats}

\begin{note}
  Well, it's possible that the agent is wrong.
  This is fine, from the agent's perspective.
\end{note}


\subsubsection{Answers which are not proposition-value pairs, in part}
\label{sec:answers-which-are}

\begin{note}
  \begin{itemize}
  \item
    Somewhat simple.
  \item
    Is there a flaw?
  \item
    Well, point is there's something that isn't a proposition-value pair.
  \item
    Parallel?
  \item
    Knowledge how and knowledge that.
  \item
    Same argument applies to dispute, in a fairly straightforward way.
    Specifically, regarding knowledge that, regardless of knowledge how.
  \end{itemize}
\end{note}

\begin{note}
  Sketch of the argument.

  \begin{enumerate}
  \item
    Potential witnessing event in which agent concludes.
  \item
    \label{pwe-iff-kh}
    This is the case if and only if knows how to conclude (if not knowledge, then insufficient grasp on witnessing event).
  \item
    \label{kw-is-kt}
    Knowledge how is a species of knowledge that.
  \item
    Knowledge that \(\varphi\)
  \item
    Knowledge that \(\varphi\) does not involve event.
  \item
    Equivalent.
  \item
    So, at least possible to answer with something that does not involve event.
  \end{enumerate}

  So, replacement.

  Limitation is option to replace.

  Still, this is enough to highlight a flaw in \autoref{prop:PWEs}.

  In addition, given that agent is not literally answering the question, additional argument that this is how to understand.
\end{note}

\begin{note}
  \dots In outline, depends on how the details are filled in.

  \autoref{pwe-iff-kh} and \autoref{kw-is-kt} in particular.

  Grant \autoref{pwe-iff-kh}, explore \autoref{kw-is-kt}.
\end{note}

\begin{note}
  Not just strong intellectualism, but\dots

  Knowing that is not,~\cite{Stalnaker:2012tp} `justified true belief, painted over with a Gettier-proof coating of some kind.' (\citeyear[754]{Stalnaker:2012tp})

  Instead, ~\citeauthor{Stanley:2011ut} (and~\citeauthor{Stalnaker:2012tp}'s) views are compatible with knowing that involving, at least in part, a potential witnessing event.
  Sparing the details, characterisation by~\citeauthor{Weatherson:2017tb}:%
  \footnote{
    \textcite{Weatherson:2017tb} investigates kind of dispositions involved.
  }

  \nocite{Stanley:2012wg}
  \begin{quote}
    Knowing that \emph{p} is not just a matter of having \emph{p} written in a knowledge box somewhere in the brain; it can in part be constituted by active dispositions.%
    \mbox{ }\hfill\mbox{(\citeyear[8]{Weatherson:2017tb})}
  \end{quote}

  \citeauthor{Stalnaker:2012tp} highlights irony.

  Active disposition.
  Hence, potential witnessing event.%
  \footnote{
    Question whether views such as those of \cite{Stalnaker:2012tp} and \citeauthor{Stanley:2011ut} help with argument here.
    I have no idea.
    Trying to find ways to argue against proposition.
  }
\end{note}

\begin{note}
  So, finding a theory to make this argument work is not immediate.
  Hence, even if valid, not clear sound.
  Still, this won't deter.
\end{note}


%%% Local Variables:
%%% mode: latex
%%% TeX-master: "master"
%%% End:
