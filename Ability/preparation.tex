\begin{note}
  \begin{TOCEnum}
  \item
    \TOCLine{cha:clar}

    Conclusions.
    In particular,

    \begin{itemize}
    \item
      An agent evaluating a proposition has some value.
    \item
      Additional clarification on \pool{} (of premises).
    \item
      The event in which an agent concludes a proposition has some value from some \pool{}.
    \item
      What is it for an agent to be concluding a proposition has some value from some \pool{}.
    \item
      Relation between an event in which an agent concludes and an event in which an agent reasons.
    \end{itemize}
  \item
    \TOCLine{cha:var}

    Relations.

    Clarify by \ros{1}.
    With this, variants to \qWhy{}, \qHow{} and \issueInclusion{} which clarify way counterexamples may be developed.
  \item
    \TOCLine{cha:lit}

    Extend parallels to \issueInclusion{} in terms of reasons to variant of \issueInclusion{} through a number of other accounts.
  \end{TOCEnum}
\end{note}

\begin{note}
  Key terms are defined.
  Though, to avoid excessive detail these definitions are closer to specifications.
  I have attempted say what is needed without saying anything too obvious.

  Assumptions are made when something is added to defined terms that does not already follow or we add something to an undefined term which does not follow or assume something about a term left intuitive that may be objected to.
  In the latter case, we include some references to literature or a short summary of possible objections.

  If something is particularly useful (for the purposes of this document) we isolate it as a `Proposition' and include an argument.
  If something may be helpful but can be ignored we isolate it as an `Observation' and include motivation.
\end{note}

%%% Local Variables:
%%% mode: latex
%%% TeX-master: "master"
%%% End:
