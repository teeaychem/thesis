\chapter{Claiming support (and claimed support)}
\label{sec:abil-access-supp}

\paragraph*{Overview}

\begin{note}
  The present chapter is about claiming support.
  Or, more precisely, claiming support and claimed support --- we are interested in both an activity and the result of that activity.

  The purpose of the following discussion of claiming support with respect to the {\color{red} overall project} is as follows:

  Reasoning with ability, obtaining some conclusion.
  Incompatible with intuitive understanding of part of such reasoning and, in particular, an common idea appealed to in various arguments.

  In order to make such an argument we require some background constraints on such reasoning.
  Hence, in this chapter we work though those background constraints.

  This then provides sufficiently precise context for \autoref{cha:claiming-support-use} in which we state in detail what we arguing against, and what we are arguing for.
  And, provide background for~\autoref{part:tension} in which we argue for a principle restricting claiming support (specifically \autoref{sec:second-conditional}).

  And, as part of providing sufficiently precise context will include a rough template of the considerations which motivate argument.

  In other words, context, and also preview.
\end{note}

\begin{note}
  Two things to keep in mind.

  \begin{enumerate}
  \item These constraints are partial.
  \item Consequence of constraints that matters.
  \end{enumerate}

  Enough to identify certain ways in which an agent may fail to perform the reasoning of interest.

  Benefit of not building in `too much'.

  However, also possible to endorse additional constraints.
  In particular, rule out alternative.
  If so, also a problem.
  We will not give much attention to this.
  Argument rests on plausibility when applied to particular case.
  If agree with constraints and agree with particular case, then this is an indirect argument against stronger constraints.


  Consequence of constraints that matters.
  Preview of considerations that matter, but additional step.

  However, postpone final constraint until argument proper.
  Primarily because for the moment, enough to interpret what we are argument against and for --- consequence isn't required.
  Secondary, this constraint where it matters in the argument.
  Third, surrounding motivation and context within literature.

  Welcome to turn to \autoref{sec:second-conditional} following this section.
\end{note}

\paragraph*{Plan}

\begin{note}
  \begin{enumerate}
  \item `Claimed support' is a term, but with certain pragmatic implications.
  \item Two ideas.
  \item Assumptions from these ideas.
  \item Working through assumptions with respect to particular consequence.
  \item Additional comments.
  \end{enumerate}
\end{note}

\begin{note}
  Roughly speaking, the two ideas:
  \begin{enumerate}
  \item Possible that there are defeaters for claimed support.
  \item Claimed support involves reasoning that defeaters do not obtain.
  \end{enumerate}
\end{note}



\section{`Claiming'}
\label{sec:claiming}

\begin{note}[Introducing support]
  Initial clarification is with respect to claiming support.
  Emphasis on `\emph{claim}'.

  The thesis is not about when and why an agent has \emph{support}.

  There are three primary reasons why we focus on claimed support.

  First, neutral for main thread of argument on what support amounts to.
  Interest is with structure of claim, and background assumption that if success in claiming then structure of support follows structure of claim.

  Second, whether or not an agent has support often seems secondary.
  It may be that any claimed support for a proposition is support for that proposition, but perhaps not.
  \begin{illustration}[A box of flan(nels)]
    \label{illu:flan-nels}
    Suppose `flan' is written on the side of a container.
    I may claim support that the container contains flan.
    And, it may be that the writing on the side of the container is support for the box containing flan.
    However, the straps ensuring the container remains closed is unfortunately placed, and if moved would reveal the side of the container reads `flannels'.
  \end{illustration}

  The unfortunate placing of the straps does not seem to prevent \emph{claiming} support, but I'm not sure whether it is right to say that the writing on the side of the box (straps in place) \emph{does} provide support that the box containing flan.
  So, speaking in terms of claiming support leaves open whether what is claimed reflects on whether an agent has support.\nolinebreak
  \footnote{
    In particular, claiming allows focus on internal constrains, while remaining silent on whether having support is (in part) determined by external factors.
  }
  \(^{,}\)\nolinebreak
  \footnote{
    Distinction between propositional and doxastic support.
    Propositional, support agent has whether or not made a claim.
    Doxastic is successful claim and propositional support.
    So, both require that the agent has support.
    Claimed support is the agentive component of doxastic support.
    Not interested in whether the agent also has propositional support, though more or less assume.
  }
  \(^{,}\)\nolinebreak
  \footnote{
    {
      \color{red}
      English is somewhat difficult.
      It is somewhat unfortunate that `an agent has claimed support for \(\phi\)' may be read `there is support which the agent has claimed for \(\phi\)'.
      Still, this seems to follow more easily from `support claimed'.
      So, `claimed support' emphasises the claim, while `support claimed' emphasises support.
    }
  }

  Third, and following from the second, focusing on claimed support allows us to make no assumption about the relationship between claimed support and support.
  To elaborate, consider enthymematic inferences.
  One may hold that an agent may claim support for some conclusion via enthymematic inference, but hold that the support the agent has is understood from the perspective of the (corresponding) complete inference.\nolinebreak
  \footnote{
    Cf.\ \textcite{Moretti:2019wx}.
  }
  Alternatively, one may hold that the enthymematic argument is an adequate support relation (at least with respect to context in which the inference was made).

  Hence, one may question whether the structure of claimed support follows the structure of support.

  An important consequence of this final point is that we will only be interested in why (and when) an agent claims support for and from ability rather than why (or when) an agent \emph{has} an ability.
\end{note}

\section{Three basic assumptions}
\label{sec:three-basic-assumpt}

\begin{note}
  Three basic assumptions to narrow down the kind of thing claiming support is.
  Following sections then narrow down what type of thing claiming support is.
\end{note}

\subsection{Proposition and evaluation}

\begin{note}[Value proposition]
  Reasoning and claims to support focus.
  Briefly introduce a pair of propositions to clarify claim to support and reasoning.

  \begin{restatable}[Claimed support is for the value of a proposition]{assumption}{assuCSVP}
    \label{assu:CSVP}
    When an agent claims support for some proposition, the agent claims that the proposition has some value.
    Where:
    \begin{itemize}
    \item A proposition is some information. (Mode of presentation) And,
    \item A value is an assessment of that information.
    \end{itemize}
    \vspace{-\baselineskip}
  \end{restatable}
  \autoref{assu:CSVP} fixes terminology.
  To illustrate, when stating the conclusion of the reasoning sketched above we used the proposition that \emph{the area of the rectangle is \(133\text{cm}^{2}\)}.
  The proposition refers to the state of affairs in which the area of the rectangle is \(133\text{cm}^{2}\), and speaking a little more precisely, the agent claimed that the proposition has the value `true' --- though it may be the value turns out to be `false'.
  Or, perhaps if the agent was a little unsure about the accuracy of the ruler, that the proposition has the value `likely', `probable', or some quantitative credence.
  And, some other instance of reasoning may have concluded that the proposition has the value `desirable' --- e.g.\ if the agent was searching for a rectangle of some approximate size.\nolinebreak
  \footnote{
    Nothing in particular hangs on the distinction between different values.
    If you prefer, you may expand the proposition (state of affairs) to include additional factors, and consider only the values `true' and `false'.
    For example, the proposition that \emph{I desire the bath to be warm} is false, as opposed to the proposition that \emph{the bath is warm} is valued undesirable by me.
  }

  Core idea is that claim of support is that things are a certain way.
  Proposition, what the thing is.
  Value, the way it is.
\end{note}

\begin{note}
    \begin{itemize}
  \item \(p\) has the value `true'. \hfill (\emph{p} is true.)
  \item \(p\) has the value `ought to be'. \hfill (\emph{p} ought to be the case.)
  \item \(p\) has the value `desirable'. \hfill (\emph{p} is desirable.)
  \item \(p\) has the value `improbable'. \hfill (\emph{p} is improbable.)\nolinebreak
    \footnote{
      Probability is somewhat interesting.
      The value of the probability of \(p\) is below some threshold.
      E.g.\ the value of the probability of \(p\) is \(0.1\).
      So, on a surface reading the thing that is in a certain way is the probability of \(p\) rather than \(p\).
    }
  \end{itemize}
\end{note}

\begin{note}
In most cases the value will be clear (i.e. that the proposition is true, though sometimes that the proposition is desirable), and so we will talk of claiming support for the proposition.
  A handful of additional examples will be provided when illustrating the next proposition.
\end{note}

\begin{note}
  Nothing hangs on distinction between values.
  Reduce everything to truth and falsity.
  However, we do not assume this, and if you do not think this is the case either, then I would like to not suggest that the assumptions and arguments to follow concern only those propositions which may be evaluated as true or false.
\end{note}

\subsection{Culmination of reasoning}

\begin{note}[Reasoning proposition]
  \begin{restatable}[Claiming support is the result of reasoning]{assumption}{assuCSRR}
    \label{assu:CS-culmination-of-R}
    Claiming support some proposition \(\phi\) having value \(v\) is an instance of reasoning.
    And, claimed support some proposition \(\phi\) having value \(v\) is the culmination of some instance of reasoning.
  \end{restatable}
\end{note}

\begin{note}
  Key thing here is that result of premises and steps of reasoning.

  The role of~\autoref{assu:CS-culmination-of-R} is primarily to ensure that claiming support always guarantee the existence of premises and steps.
  With the exception of some broad constraints to be outlined in further assumptions below, we (will) have little to say about the specifics of what the reasoning consists of.
\end{note}

\begin{note}
  Reasoning, some mental activity.
\end{note}

\begin{note}[Quick examples]
  \begin{itemize}
  \item \(S\) testified that \(p\), so \(p\) is true.
  \item \(p\) would satisfy every member of the group, so \(p\) ought to be the case.
  \item The song is produced by \(S\), so it is desirable that I listen to it.
  \item The device reads \(p\) and is reliable, so \emph{not}-\(p\) is improbable.
  \end{itemize}
\end{note}

\begin{note}
  Instances of reasoning may culminate in other ways, so we are only interested in a specific type of reasoning.
\end{note}

\begin{note}[Claiming support]
  Expand on this below.
  Briefly mention that this falls short of \emph{establishing} that \(\phi\) has value \(v\).
  \emph{Claimed} support means there's always some possible defeater.
\end{note}

\begin{note}[Understanding `having value \(v\)']
  In a deductive case, if the premises are true, then the conclusion is true.
  Means-end reasoning for desire.
  The value is important.
  If it is true that it past 6pm, then it is true the shop is closed.
  Provides value of shop being closed.

  However, if agent desires that it is past 6pm, then it doesn't follow that the agent desires that the shop is closed.
  Question an agent as to why they think their desires conform to truth --- is-ought problem.

  Means-end reasoning.
  It is true that there is cheese at the centre of the maze.
  And, it is desirable that I obtain the cheese at the centre of the maze.
  Further, it is true that I may only obtain the cheese at the centre of the maze by solving the maze.
  Therefore, it is desirable that I solve the maze.
\end{note}

\subsection{`Confined' reasoning}
\label{sec:no-closure}

\begin{note}
  Assumptions \ref{assu:CSVP} and \ref{assu:CS-culmination-of-R} are designed to be general and straightforward.

  However, combined they allow for a subtlety that will become important.
  The subtlety is this:
  From \autoref{assu:CSVP} claimed support is for the value of a proposition, and from \autoref{assu:CS-culmination-of-R}, claimed support is the result of some reasoning.
  In turn, if an agent has information that some proposition \(\phi\) having value \(v\) requires that some other proposition \(\psi\) has value \(v'\), then it is a consequence of the agent's claimed support for \(\phi\) having value \(v\) that \(\psi\) has value \(v'\).
  Yet, as claimed support is the result of some reasoning, it does not immediately follow that the agent has claimed support for \(\psi\) having value \(v'\).
  For, it need not be the case that the agent performs any reasoning about \(\psi\) having value \(v'\).

  In other words, as claimed support it is tied to reasoning, it is not immediate that reasoning about a proposition having some value extends to any (held) consequence of that proposition.

  This subtlety is intuitive, but given the importance it will have, and the observation that it does not directly follow from Assumptions \ref{assu:CSVP} and \ref{assu:CS-culmination-of-R}, we formulate it as an explicit assumption.
\end{note}

\begin{note}
  \begin{restatable}[Confined Reasoning]{assumption}{assuRClosure}
    \label{assu:R-closure}
    Reasoning about some proposition \(\phi\) having value \(v\) does not necessarily apply to any proposition-value pair which follows from \(\phi\) having value \(v\).
  \end{restatable}

  Following from above, suppose \(\psi\) having value \(v'\) follows from \(\phi\) having value \(v\).
  \autoref{assu:R-closure} then denies that the reasoning about \(\phi\) having value \(v\) applies to \(\psi\) having value \(v'\).
\end{note}

\begin{note}
  For example, an agent's reasoning about it being true that it is raining does not (necessarily) apply to it being true that the sheets that they have put out to dry are now getting wet.

  It may that the agent concludes that it is beneficial that it is raining given the recent drought, but it does not (necessarily) follow that the agent concludes that it is beneficial that the sheets are getting wet.
  Likewise, the agent may be confident that it is raining, but it does not confident that the sheets are getting wet as the agent may have forgot they put the sheets out to dry.

  Likewise, an agent's reasoning about it not being true that two friends are playing tic-tac-toe does not (necessarily) apply to it being true that the two friends are playing a competitive game.
  For, the agent's reasoning concerns the mutual interest the friends share in tic-tac-toe, and that interest may not extend to competitive games in general.
\end{note}

\begin{note}
  Applied to claiming support.
  It is not necessarily the case that claimed support for \(\phi\) having value \(v\) is claimed support for \(\psi\) having value \(v'\).\nolinebreak
  \footnote{
    No immediate `closure' principle for claimed support:
    \begin{itemize}
    \item \((\phi \rightarrow \psi) \rightarrow (\text{CS}\phi \rightarrow \text{CS}\psi)\)
    \end{itemize}
  }
\end{note}

\begin{note}
  Two notes on terminology:

  First, `following from' is understood broadly.
  The relevant consequence may be logical, semantic, dialectic, and so on.
  Further, whatever follows need not follow \emph{as a matter of} logic, semantics, dialectics, and so on.
  Rather, what follows may follow as a matter of the agent's perspective.
  Indeed, we will not have particular interest in reasoning from any perspective other than that of the agent who has performed the instance of reasoning.

  Second, `applies' is vague, but I seem no way of clarifying what `applies' amounts to in any significant without expanding on what reasoning is, but such an expansion is beyond what will be required for the following arguments.
  We will, however, provide a handful of additional examples in \autoref{sec:claim-supp-requ} when we \hyperref[independence-and-assu:R-closure]{return} to \autoref{assu:R-closure}.
\end{note}

\begin{note}
  There are two ways in which \autoref{assu:R-closure} may be expanded upon that are of interest.

  \begin{enumerate}
  \item Present reasoning, present consequence.
  \item Past reasoning present consequence.
  \end{enumerate}

  Both.
\end{note}

\subsection{Summary of basic assumptions}
\label{sec:summary}

\begin{note}
  Assumptions \ref{assu:CSVP},~\ref{assu:CS-culmination-of-R}, and~\ref{assu:R-closure} are quite general assumptions.

  Taken in reverse:
  \autoref{assu:R-closure} holds that (in general) reasoning about some proposition having some value does not necessarily apply to anything which follows from that proposition having that value.
  \autoref{assu:CS-culmination-of-R} identifies claiming support as the result of reasoning.
  And, \autoref{assu:CSVP} identifies claiming support with reasoning that some proposition has some value.

  \autoref{assu:R-closure} is indirectly related to claiming support by Assumptions~\ref{assu:CSVP} and~\ref{assu:CS-culmination-of-R}, but Assumptions~\ref{assu:CSVP} and~\ref{assu:CS-culmination-of-R} do not single out anything unique.
  For example, the basic folk-psychological pair of belief and desire (plausibly) involve
  \begin{enumerate*}
  \item a proposition-value pair, which is
  \item the result of some reasoning, such that
  \item the reasoning does not necessarily apply to any following proposition-value pair.
  \end{enumerate*}
  As noted, Assumptions \ref{assu:CSVP},~\ref{assu:CS-culmination-of-R}, and~\ref{assu:R-closure} are basic in the sense that they fix what kind of thing claiming support is.
  In the following section we state two ideas to loosely narrow down what type of thing claiming support is.
\end{note}

\section{Two ideas}
\label{sec:two-ideas}

\begin{note}
  In \ref{sec:three-basic-assumpt} we made three assumptions regarding the kind of thing claiming support it.
  In the present section we introduce two ideas to narrow down the type of thing claiming support is.

  View as upper and lower bound on `strength'.
  Few constraints on suspecting that some proposition is true.
  Sufficient to not be certain that the proposition is false.
  No evidence, justification, and so on.
  Contrast, knowing that some proposition is false.
  Rule out various ways in which the proposition could be true.

  First idea, upper bound in sense that not rule out that proposition may not have value --- i.e.\ compatible that proposition does not have value.
  Second idea, lower bound, in sense that must include reasoning that indicates value even if the proposition does not in fact have value.
  Certain kind of counterfactual robustness, similar to safety or tracking conditions on knowledge.

  However, ideas.
  In \ref{sec:assumpt-from-ideas} revise ideas in to explicit assumptions.
  In particular, arguments will rely only on fairly weak constraints about what indication amounts to.

  Still, as the explicit assumptions made are weak, the ideas presents will hopefully provide sufficient motivation and intuition.
  \end{note}

\subsection{Idea 1}
\label{sec:idea-1}

\begin{note}
  \begin{restatable}[\ideaCSA{-} --- \ideaCSA{}]{idea}{iCSA}
    \label{idea:defs-for-CS}
    Claimed support for \(\phi\) having value \(v\) is always compatible (from the perspective of the agent) with:
    \begin{enumerate}
    \item The reasoning appeals to some proposition \(\psi\) having value \(v'\), while \(\psi\) does not value \(v'\).
    \item \(\phi\) not having value \(v\).
    \end{enumerate}
  \end{restatable}
\end{note}

\begin{note}
  I take it to be intuitive that agents are fallible in many cases of claiming support.
  It is not too difficult to think of ways in which claimed support may be misleading or mistaken.
  As noted above, claiming support for what the time is from glancing at a clock seems sufficient, but clocks may be incorrectly set (misled) or broken (mistake).
  Similarly, a sample of \(1,000\) rolls may mislead me into thinking that a die is unbiased, or an overloaded operator may lead to a mistake in claiming support for the proposition that \(x = 4\) is an expression of equality rather than variable assignment.
\end{note}

\begin{note}
  \phantlabel{first-mention-undercutting-defeater} % first mention of undercutting defeaters
  Slightly different from the possibility of undercutting and rebutting defeaters.

  Following~\citeauthor{Moretti:2018va}:
  \begin{quote}
    A \emph{rebutting} defeater for a belief that \(P\) of \(S\) is, roughly, a reason of \(S\) for believing the negation of \(P\) or for believing some proposition \(Q\) incompatible with \(P\).
    Whereas an \emph{undercutting} defeater for a belief that \(P\) of \(S\) is, roughly, a reason of \(S\) that attacks the connection between S's ground for believing \(P\) and \(P\)[.]\nolinebreak
    \mbox{}\hfill\mbox{(\citeyear{Moretti:2018va})}
  \end{quote}
\end{note}

\begin{note}
  Defeaters are reasons.
  As a plausible consequence, possibility of undercutting and rebutting defeaters.
  However, somewhat stronger.

  Still, may also add defeaters, if this helps.
\end{note}

\begin{note}
  This does not entail that there are defeaters, nor that there are any possible defeaters --- only that defeaters are an epistemic possibility.

  Still, in some cases, this may seem absurd.
  Suppose in front of me are two apples and two pears.
  So, four pieces of fruit.

  There are two responses here.

  First, not an instance of claiming support.

  Second, outline various way in which the claimed support may be \mom{}.
  Those apples and pears may not be real pieces of fruit, they may be replicas.\nolinebreak
  \footnote{
    So beings the process of attempting to quarantine fallibility from infallibility.

    There are two pairs of objects in front of me, and the objects appear to be fruit.
    So, there are four objects in front of me, which appear to be fruit.

    There are two pairs of objects hence there are four objects.

    But I may be hallucinating.

    There appear to be two pairs of objects in front of me, which appear to be fruit.
    So, there appear to be four objects in front of me, which appear to be fruit.

    Whenever there are two pairs of objects, there are four objects.

    Perhaps it is impossible to be \mom{}, but then stronger than claiming support.

    Likewise:

    Any object is identical with itself --- but I doubt one needs to claim support for reflexivity of equality.

    Similarly, it doesn't seem to be the case that I need to claim support for the proposition that my name is Humpty --- no matter what my birth certificate says, I get to decide what my name is.
  }
\end{note}

\begin{note}[Summarising \autoref{idea:defs-for-CS}]
  Limitation on strength of claimed support.

  Plausibly holds with respect to various other attitudes.
  Belief because can't rule out.

  Issue here is that compatible with claimed support being a very weak attitude.
  For example, agent is allowed to assume some arbitrary proposition \(\psi\) has value \(v'\) when claiming support for \(\phi\) having value \(v\).
  Possibility of being \mistaken{} with respect to \(\psi\) has value \(v'\), and hence possibility of being \misled{} with respect to \(\phi\) having value \(v\).

  Still, interest is in something stronger than this.

  The (epistemic) possibility of being \mom{} is as a worry.
  Goal, roughly, is to establish good ground for appealing to \(\phi\) having value \(v\) in further reasoning.
  So, \ideaCSA{} limits the strength of claimed support in reasoning, but also need additional assumptions to ensure sufficiently strong.
\end{note}

\subsection{Idea 2}
\label{sec:idea-2}

\begin{note}[]
  From \ideaCSA{}, (epistemic) possibility that claimed support relies or establishes proposition-value pair that is not the case.

  So, agent is never in a position to rule out epistemic possibility.
  However, should work out however things are.

  \begin{restatable}[\ideaCSB{-} --- \ideaCSB{}]{idea}{ideaEIS}
    \label{idea:eiS}
    Claimed support for \(\phi\) having value \(v\) indicates that \(\phi\) has value \(v\) regardless (from the perspective of the agent) of whether
    \begin{enumerate}
    \item The reasoning appeals to some proposition \(\psi\) having value \(v'\), while \(\psi\) does not value \(v'\).
    \item \(\phi\) has value \(v\).
    \end{enumerate}
  \end{restatable}
  Entertain possibility that \(\phi\) does not have value \(v\) while maintaining claimed support.
\end{note}

\begin{note}
  \ideaCSA{} places limitation on claimed support.

  \ideaCSB{} goes in the opposite direction.
  Claimed support to be a relatively strong attitude.
\end{note}

\subsection{Summary of ideas}
\label{sec:summary-1}

\begin{note}
  Two key ideas: \ideaCSA{} and \ideaCSB{}.
  These provide some characterisation of claimed support.

  We revised \ideaCSA{} to an explicit assumption: \autoref{assu:supp:nfactive}.
  From \autoref{assu:supp:nfactive} claimed support is such that possibility of being \mom{}.
  This limits the strength.

  In turn, \ideaCSB{} pushes for claimed support to be sufficiently strong that it compensates for such possibilities.

  For this paper, significant interest with \ideaCSB{}.
  Different ways of reasoning, and how these deal with the concern about possibility of being \mom{}.

  In particular, how `defeaters' work.
\end{note}

\begin{note}
  Internalist flavour, specifically `mentalism' in the sense of \citeauthor{Feldman:2001uy} where:
  \begin{quote}
    \dots a person's beliefs are justified only by things that are \dots internal to the person's mental life.\nolinebreak
    \mbox{}\hfill\mbox{(\citeyear[233]{Feldman:2001uy})}\nolinebreak
    \footnote{
      See also~\textcite[\S4,9]{Pappas:2017vi}.
    }
  \end{quote}
  Result of reasoning, and indication is part of that reasoning.
  Reasoning is internal to an individual's mental life, and hence internalist in this sense.\nolinebreak
  \footnote{
    \color{red}
    Intuitively, external circumstances may impact the \emph{support} the agent has.
    However, as these are external, it seems external circumstances does not impact \emph{claiming} support.

    This is how you get puzzles for externalism.
    In both cases, it's fine for the agent to claim support, but the external circumstances impact whether the agent \emph{has} support.
    The internalist/externalist divide would seem to affect the conditions on claiming.

    Way to expand on this is reconstructing bootstrapping examples with and without \eiS{}.
    If the agent would only get basic support if reliable, then it's not clear that bootstrapping is a problem.
  }

  Indication of \ideaCSB{} need not amount to justification --- it may, but this will not be important.
  This is also not to require sufficient.
  These two ideas may be developed in various ways.
  Not only indicates from perspective of the agent, but is neither \mom{} and ways in which \mom{} are counterfactually distant.
  However, ideas for the moment.
  Assumptions are what will matter.
\end{note}

\section{Refining ideas through assumptions}
\label{sec:assumpt-from-ideas}

\begin{note}
  Two ideas.
  Possibility, and requirement from this.
  As noted, claiming support is still quite general, and this is by design.
\end{note}

\subsection{\requ{3}}
\label{sec:claim-supp-requ}

\begin{note}
  To develop \ideaCSA{} and \ideaCSB{} into assumptions we start by defining a `\requ{}' of some instance of reasoning.

  Both \ideaCSA{} and \ideaCSB{} concern propositions having values appealed to in reasoning.
  In turn, the role of a \requ{} is to, on the one hand provide a sufficiently clear account of an important proposition-value pairs.

  Of course, one may hold that additional proposition-value pairs, or indeed other factors beyond \requ{1} are important for claiming support.
  Our interest, however, will be limited to \requ{1} (and an extension of \requ{1}).
  For, our interest is with providing sufficient conditions to highlight failures of claiming support.
\end{note}

\subsubsection{Definition of a \requ{0}}
\label{sec:def-of-requ}

\begin{note}
  \begin{restatable}[\requ{3} of reasoning]{definition}{defRequisite}\label{def:requisite}
    \(\psi\) has value \(v'\) is \requ{} of reasoning that concludes that \(\phi\) has value \(v\) (from an agent's perspective) if:
    \begin{enumerate}
    \item\label{def:requi:not-v-poss} \(\psi\) not having value \(v'\) is epistemically possible.
    \item\label{def:requi:counterfactual} If \(\psi\) were not to have value \(v'\) then some step of reasoning would involve appealing to:
      \begin{enumerate}
      \item\label{def:requi:counterfactual:p} something that is not the case, or
      \item\label{def:requi:counterfactual:c} moving to something that is not the case
      \end{enumerate}
      No making that step of reasoning.
    \item\label{def:requi:conseq-persists} That step has consequence of \(\psi\) having value \(v'\) which persists through to conclusion of reasoning.
    \end{enumerate}

    \requ{2} of claimed support if \requ{} of claiming support.
  \end{restatable}
  Intuitively, non-disposable consequence of some step of reasoning in isolation.

  If \requ{} fails, then reasoning fails as a whole.
\end{note}

\begin{note}
  \ref{def:requi:not-v-poss} as \ref{def:requi:counterfactual} is a counterfactual.
  Interest is in the counterfactual obtaining.
  \ref{def:requi:counterfactual} and \ref{def:requi:conseq-persists} may apply in cases of epistemically impossible proposition-value pairs, but it is not clear why such epistemically impossible proposition-value pairs would matter to the agent's reasoning.

  \ref{def:requi:counterfactual} has two components.
  No clear reduction of moving to something that is not the case to moving from something that is not the case.
  For, it may be that step of reasoning is content dependent {\color{red} (ecological)} and no need to make the assumption that being in the right context is always some premise.
  (However, not the case that ignore this possibility.)

  Finally, \ref{def:requi:conseq-persists} ensures that the proposition-value pair remains important.
  In particular, avoids classifying proposition-value pairs appeals to as part of reasoning by cases as \requ{1}.
\end{note}

\begin{note}
  \begin{illustration}
    \label{illu:requ:bank}
    `Where is the nearest bank?'
    Reason that `next to the post office' is an appropriate response.
  \end{illustration}
  However, friend is asking about where the nearest bank in the sense of financial establishment, rather than river bank.
  If appropriate disambiguation, then appealed to something that is not the case.
  Hence, that `river bank' is not the appropriate disambiguation is a \requ{}.

  Note, however, that if agent did not learn about different disambiguations of `bank' then no \requ{}.
\end{note}

\begin{note}
  We will see many more examples of \requ{1} below.
  For the moment, we offer a second, abstract, instance of a \requ{} and then an instance of something which is not a \requ{}.
\end{note}

\begin{note}
  \begin{illustration}
    \label{illu:requ:import-export}
    Suppose we have some conditional `\(\rightarrow\)' that admits of import-export.\nolinebreak
    \footnote{
      For example, the material conditional, but not necessarily the natural language conditional expressed in certain `if \dots then \dots' constructions {\color{red} Vann}.
    }
    I.e.:
    \begin{quote}
      \(\phi \rightarrow (\psi \rightarrow \xi)\) if and only if \((\phi \text{ and } \psi) \rightarrow \xi\)
    \end{quote}
    And, suppose an agent's reasoning has the structure:
    \begin{enumerate}
    \item \(\phi\)
    \item \(\phi \rightarrow (\psi \rightarrow \xi)\)
    \item \(\psi \rightarrow \xi\)
    \end{enumerate}
    Such that possible \(\phi \rightarrow (\psi \rightarrow \xi)\) is not the case.
  \end{illustration}

  Here, \((\phi \text{ and } \psi) \rightarrow \xi\) is a \requ{} of the agent's reasoning.
  For, if \(\phi \rightarrow (\psi \rightarrow \xi)\) then also \((\phi \text{ and } \psi) \rightarrow \xi\).
  Hence, if \((\phi \text{ and } \psi) \rightarrow \xi\) is not the case, then \(\phi \rightarrow (\psi \rightarrow \xi)\) is also not the case.
\end{note}

\begin{note}
  In these two \illu{1}, something that would highlight a fault in reasoning.
  No all things that would highlight faults in reasoning are \requ{1}.
\end{note}

\begin{note}
  \begin{illustration}[Textbook answers]
    \label{illu:textbook-answers}
    Suppose an agent is working through a logic textbook with practice problems at the end of each chapter to test a student's understanding.

    For any given question, the following conditional holds with respect to the agent:
    \begin{itemize}
    \item If my answer to the problem is incorrect, then I {\color{red} appealed to something that is not the case or moved to something that is not the case}.
    \end{itemize}
  \end{illustration}
  Possible that the agent {\color{red} appealed to something that is not the case or moved to something that is not the case}.
  However, implausible that the agent appealed to \emph{not} {\color{red} appealed to something that is not the case or moved to something that is not the case}.

  Key point is that a \requ{} follows from some part of the agent's reasoning, not from things that may also hold if the agent's reasoning is successful.\nolinebreak
  \footnote{
    Well, meta-beliefs with inference.
    While I do not think particularly plausible, there is also a question of whether any such meta-belief is part of the agent's reasoning, \dots
  }

  For example:
  \begin{enumerate}
  \item \(2 + 2 = 4\).
  \item \(1 + 3 = 4\).
  \item Equality is a symmetric and transitive (and reflexive) relation.
  \item So, \(2 + 2 = 1 + 3\).
  \end{enumerate}
  It does not seem that each premise in the reasoning requires an accompanying premise \dots\nolinebreak
  \footnote{
    \textcite{Carroll:1895uj}.
  }
  With respect to \autoref{illu:textbook-answers}, that the reasoning is good is not a consequence of a step of reasoning.
\end{note}

\begin{note}
  Regular conditional and subjunctive.
\end{note}

\begin{note}
  The subjunctive here does, generally speaking, highlight a possible defeater for reasoning.

  Subjunctive because what follows from entertaining possibility.
  This then allows `discovering'.

  Of course, given that subjunctive, questions about how to evaluate.
  However, not counterfactual.
  So, given that possibility, do not need to consider cases where entertaining possibility requires some revision to the reasoning that the agent has performed.

  The final clause seems to explicit a plausible implicit consequence of the first clause.

  For \(\lnot p \leadsto \lnot q\), \(\lnot p \rightarrow \lnot q\), so \(q \rightarrow p\).

  Plausible, but this is not quite right.
  Consider reasoning by cases.
  If not that case, then a problem, but does not have the consequence that the case holds.
  Hence, persists.\nolinebreak
  \footnote{
    This is not to deny a slightly more complex clause.
    Rather, easier this way.
  }
\end{note}

\begin{note}
  Really important here is that being a \requ{} does not imply that the reasoning involves direct appeal.

  {
    \color{red}
    Indeed, this distinguishes from \citeauthor{Sgaravatti:2013wu}, which requires belief in each of premises.
  }
\end{note}

\begin{note}
  Not the case that any thing that is appealed to is a \requ{}.
  For, may add in certain additional things.
  In other words, careful to make sure that the subjunctive part is there.
\end{note}

\begin{note}
  Conclusion of reasoning is typically a \requ{}.

  This is somewhat tricky.
  Intuitively, conclusion is not something that matters.
  However, it would not do to simply exclude.
  For, conclusion may be introduced as a \requ{} by some intermediate step.

  Possible to add the idea of a `sub-argument'.
  Maybe\dots
\end{note}

\subsubsection{Two assumptions relating to \requ{1}}
\label{sec:two-assumpt-relat-to-requ}

\paragraph{\ideaCSA{}}

\begin{note}[Agents are fallible]
  The first assumption applies \ideaCSA{} to the definition of a \requ{}.

  \begin{restatable}[\nfcs{--} -- \nfcs{}]{assumption}{assuNFCS}
    \label{assu:supp:nfactive}
    When claiming support for a proposition and agent is not in a position to rule out the (epistemic) possibility that some premise and conclusion are \requ{1}.
  \end{restatable}

  So, conclusion.
  Consequence, epistemic possibility that conclusion does not hold.
  Obtained from some step of reasoning.
  And, claimed support for some proposition-value pair, and this trivially persists through to the conclusion of reasoning.

  From some premise.
  Possibility that it is not the case, etc.
  Consequence from 2 is that premise is not the case.

  So, assumption with respect to some premise indirectly does the first of \ideaCSA{}.
  And, assumption with respect to conclusion directly does second of \ideaCSA{}.
\end{note}

\begin{note}
  Now explicit assumption, epistemic possibility \dots either any agent, or resource bound\dots
\end{note}

\paragraph{\ideaCSB{}}

\begin{note}
  Now turn to \ideaCSB{}.
  The assumption is minimal:
\end{note}

\begin{note}
  \begin{restatable}[\eiS{-} --- \eiS{}]{assumption}{assuCSRReq}
    \label{assu:supp:independence}
    Claiming support for \(\phi\) having value \(v\) involves:
    \begin{itemize}
    \item Reasoning about whether or not any \requ{} has value, given that \requ{} may not have value.
    \end{itemize}
    \vspace{-\baselineskip}
  \end{restatable}

  \autoref{assu:supp:independence} expresses a necessary condition from \ideaCSB{}.

  The basic idea is that a \requ{} has the possibility to highlight a fault in agent's reasoning.
  So, if an agent has failed to reason about whether a \requ{} has {\color{red} value}, then the agent's reasoning fails to indicate that \(\phi\) has value \(v\) regardless of whether {\color{red} some problem with premise or conclusion}.
  {\color{red} In particular, with the \requ{} not having value.}

  \autoref{assu:supp:independence} does not include any particular constraint on what the reasoning about a \requ{} amounts to {\color{red} other than \requ{} not having value.}
\end{note}

\begin{note}
  From~\autoref{assu:CS-culmination-of-R}, claimed support is the result of reasoning.
  \autoref{assu:supp:independence} provides addition constraints on what such reasoning involves.
  This is quite weak and rather uninformative.
\end{note}

\begin{note}[Strengthen?]
  Of course, one may wish to strengthen this necessary condition by specifying additional constraints on what is involved in reasoning about a \requ{}.

  For example, suppose \(\psi\) having value \(v'\) is a \requ{} of reasoning that concludes with \(\phi\) has value \(v\).
  If the reasoning that concludes with \(\phi\) having \(v\) is an instance of claiming support, then it seems reasoning that amounts to it being nice if \(\psi\) has value \(v'\), or that it is not obvious that \(\psi\) does not have value \(v'\) seems inadequate.

  In short, while \autoref{assu:supp:independence} may be a necessary condition, it does not seem the strongest necessary condition of its kind.

  Still, for our purposes \autoref{assu:supp:independence} will be sufficient.
  Either the lack of reasoning, or that it is not possible to do the required reasoning --- though not necessarily not possible in general (\nI{} for details).
\end{note}

\begin{note}[`Unrecognised' \requ{1}]
  Above, strength.
  Worry that reasoning is too weak
  A few other ways.

  `Unrecognised' \requ{1}.
  Definition of a \requ{} requires that conditions hold from perspective of an agent.
  Possible that there are `unrecognised' \requ{1}.
  And, expand~\autoref{assu:supp:independence} to cover these.

  Or, alternatively have some requirement concerning a search for \requ{1}.

  Former is weaker than the latter, but both extend beyond the agent's perspective.
  Inclined to think that some suggestion along these lines is plausible with respect to \ideaCSB{}.
  However, avoid building in assumptions that are not required.

  This is not, however, to grant arbitrary strengthening.
  Core argument is that one possible strengthening does \emph{not} hold. (\ESU{}).
\end{note}

\begin{note}[Not require claiming support]
  Even so, \ref{assu:supp:nfactive} already places some constraints.

  For, given \nfcs{} it need not be the case that an agent completely rules out possibility of defeater obtaining --- i.e.\ of being \mom{}.
  Appeal to claim support in reasoning that defeaters do not obtain.
  Hence, reasoning does not require ruling out defeaters.

  Also, I think plausible \autoref{assu:supp:independence} does not require claimed support for \requ{1}.
\end{note}

\begin{note}
  Well, this definition and assumption may seem arbitrary.
  That much I concede.
  The key point is that proposition having value is an unavoidable consequence of some step of reasoning.

  However, we are not searching for a complete account of possible defeaters of interest.
  Rather, a clear account of something sufficiently troublesome.

  However complex things get when additional aspects of reasoning are taken into account, what this definition captures seems clear enough, even if part of that clarity is obtained by arbitrary simplification.
\end{note}

\begin{note}
  \eiS{} does not deny that things may need to be a certain way for an agent to claim, or to be in a position to, claim support.
  It may be the case that no agent would be in a position to claim support that the speed of light is constant if the speed of light were not constant.
  Still, in claiming support an agent must expect that possible defeaters do not obtain, e.g.\ that the laws of nature are constant, and that no mistakes have been made when observing relevant phenomena.
\end{note}

\begin{note}
  \color{red}

  no-lose investigations from ~\cite{Weisberg:2012vs}.

  \begin{quote}
    Disconfirmability If you know that a test cannot disconfirm a hypothesis, no result of the test can confirm the hypothesis either.
  \end{quote}

  \begin{quote}
    No Risk, No Gain ‘‘To test a hypothesis we must do something that could result in presumptive evidence against the hypothesis.’’ (Glymour 1980: 115)
  \end{quote}

  \begin{quote}
    No-Lose Investigation (NLI) A no-lose investigation is one where:

1. P is not justified for the agent at t1 ,

2. at t 1 the agent knows that ØP will not be justified for her at t2 , and

3. at t 1 the agent knows that if P is true, P will be justified for her at t2 .
\end{quote}

\begin{quote}
  without any epistemic risk, there is no epistemic gain.
\end{quote}
\end{note}

\subsubsection{Proposition}
\label{sec:proposition}

\begin{note}
  Following is an immediate consequence of \ref{assu:supp:nfactive} and \autoref{assu:supp:independence}:

  \begin{restatable}[Reason about recognised \requ{1}]{proposition}{propRecogniseDefeaters}
    \label{prop:CS-only-if-reason-recognised-defeaters}
    \requ{}, and:
    For any such recognised \requ{} at time of reasoning and does not reason, not an instance of claiming support.
  \end{restatable}
\end{note}

  \begin{note}
  \begin{itemize}
  \item If result of reasoning to \(\phi\) having value \(v\) is such that agent considers that reasoning fails if \(\phi\) does not have value \(v\), then reasoning is not an instance of claiming support.
  \item Not possible that instance of reasoning to \(\phi\) having value \(v\) is claimed support only if \(\phi\) has value \(v\).
  \item Claimed support for \(\phi\) having value \(v\) never requires that \(\phi\) has value \(v\).
  \end{itemize}
\end{note}

\begin{note}
  \color{red}
    \begin{itemize}
    \item Always possibility of \mom{} from \nfcs{}.
    \item This means that the agent has no guarantee that \(\phi\) has value \(v\) --- or better put the agent considers it to be an (epistemic) possibility that their claimed support is \mom{}.
    \item However, if the agent requires that \(\phi\) is the case, then there is no possibility of the claimed support being \emph{mistaken}.
    \item Well, no reasoning against being mistaken with respect to claimed support for this \requ{}.
  \end{itemize}

  Again, it does not seem impossible for an agent to adopt an attitude that recognises the possibility but assumes regardless.
\end{note}

\begin{note}
  \phantlabel{independence-and-assu:R-closure}
  Important consequence of \autoref{assu:R-closure}, block:

  S did not reason about possibility that Q is false.
  If Q is false, then P must also be false.
  Hence, P may be false.
  S did not reason about the possibility that P is false.

  \begin{note}
  Examples:
  \begin{itemize}
  \item Possibility that the trains are on strike.
  \item No Indication of strike, so do not consider live possibility.
  \item Read newspaper.
  \item Newspaper reported strike.
  \item Consequence of possibility is that the newspaper misreported.
  \item Reasoning does not extend to newspaper.
  \end{itemize}

  \begin{itemize}
  \item Out of milk.
  \item Then come to hold that there is milk in the fridge.
  \item Hallucinating.
  \item Does not extend.
  \end{itemize}

  \begin{itemize}
  \item Turing machine reduction.
  \item If possible then also possible.
  \item So, give up.
  \end{itemize}
\end{note}
\end{note}

\subsubsection{Terminology: Being \mistaken{} or \misled{}}
\label{sec:presence}

\begin{note}
  \begin{restatable}[\mistaken{-} and \misled{}]{definition}{defMoM}\label{def:MoM}
    Some instance of claiming support that culminates with \(\phi\) having value \(v\).
    \begin{itemize}
      \item The reasoning is \emph{\mistaken{}} by involving appeal to some proposition \(\psi\) having value \(v'\) which does not have value \(v'\).
    \item The reasoning is \emph{\misled{}} if \(\phi\) does not have value \(v\).
    \end{itemize}
    \vspace{-\baselineskip}
  \end{restatable}
\end{note}

\begin{note}[M\&M \illu{2}]
  To illustrate:

  \begin{illustration}[Clock]
    \label{illu:mom:clock}
    Suppose I glance at the clock on the wall. The clock reads 11:45a, so I claim support that it is 11:45a.
  \end{illustration}

  Two possibilities:
  \begin{enumerate}
  \item Clock is not functioning.
  \item Clock is incorrectly set.
  \end{enumerate}

  If not functioning.
  By claiming support from the time expressed by the clock, I would have been \emph{\misled{}} about what the time actually is.
  For, not functioning.
  However, not necessarily \mistaken{}.
  For, might be that the time is 11:45a.

  If incorrectly set.
  By claiming support from the time expressed by the clock, I would have been \emph{\mistaken{}} about what the time actually is.

  However, not necessarily \misled{}.
  For, claimed support by appeal to a functioning clock.
  Though, despite the clock being broken, it is 11:45a and so the claim to support is not misleading.

  Combining, claimed support for the time from a broken clock expressing the wrong time would be both \misled{} \emph{and} \mistaken{}.\nolinebreak
  \footnote{
    A second \illu{0}:
    Consider a smoke detector, designed to sound an alarm if and only if sufficient levels of smoke are detected.
    Hence, if the alarm sounds, one may claim support there being smoke in the room where the alarm is installed.
    One may be misled; the alarm may have malfunctioned, so no fire.
    Or, one may be mistaken; the same type of alarm may be installed in a different room, wouldn't be a useful indicator.
  }

  {
    \color{red}
    Of course, clocks are typically glanced at, and a glance at a clock is often insufficient to determine whether the clock is incorrectly set or broken.
    Hence, the \emph{possibility} that a clock is incorrectly set or broken --- or more broadly the possibility that claimed support is misleading or mistaken --- does not prevent an agent from claiming support.
    So, ensuring that to-be-claimed support would be \mom{} is not a necessary condition for claiming support.
  }
\end{note}

\subsection{Persistence}
\label{sec:persistence}

\begin{note}
  \autoref{assu:supp:nfactive} and \autoref{assu:supp:independence} are about the activity of claiming support.
  Key thing is a \requ{}.

  However, nothing about the role of claimed support in reasoning.
  Seems plausible that appeal to claimed support.
  And, many cases will concern appeal to claimed support.
  Hence, make this explicit.
\end{note}

\begin{note}[Assumption]
  \begin{restatable}[Claimed support persists]{assumption}{assuCSPersists}
    \label{assu:CS-persists}
    In case of claiming support for \(\phi\) having value \(v\) such that \(\psi\) having value \(v'\) is a \requ{}, then appeal to previously claimed support for \(\psi\) having value \(v'\) is sufficient for reasoning about \(\psi\) having value \(v'\).
  \end{restatable}
\end{note}

\begin{note}[Intuition]
  The content of \autoref{assu:CS-persists} is clearest when viewed from the perspective of the constraints \autoref{assu:supp:independence} places on claiming support:

  Suppose \(\psi\) having value \(v'\) is a \requ{} of claiming support for \(\phi\) having value \(v\) then by \autoref{assu:supp:independence}, the agent must reason about whether \(\psi\) has value \(v'\).
  Still, by definition \(\psi\) has value \(v'\) is a \requ{} of claiming support for \(\psi\) has value \(v'\).
  Therefore, as the agent has claimed support for \(\psi\) having value \(v'\), the agent has reasoned about whether \(\psi\) has value \(v'\).
  And, in some cases an agent reasoning that they have reasoned about whether \(\psi\) has value \(v'\) is sufficient satisfy the constraint of \autoref{assu:supp:independence} that an agent reasons about \requ{1}.
  Hence, appeal to previously claimed support for \(\psi\) having value \(v'\) involves reasoning that the agent has reasoned about whether \(\psi\) has value \(v'\).
\end{note}

\begin{note}
  For example, consider the following reasoning:

  \begin{enumerate}
  \item If \(\psi\) has value \(v'\), then \(\phi\) has value \(v\).
  \item \(\psi\) has value \(v'\).
  \item So, \(\phi\) has value \(v\).
  \end{enumerate}

  \(\psi\) has value \(v'\) is a \requ{}.

  \begin{enumerate}
  \item I have claimed support for \(\psi\) having value \(v'\).
  \end{enumerate}

  Natural expression:

  \begin{enumerate}
  \item If \(\psi\) has value \(v'\), then \(\phi\) has value \(v\).
  \item I have claimed support for \(\psi\) having value \(v'\).
  \item \(\psi\) has value \(v'\)
  \item So, \(\phi\) has value \(v\).
  \end{enumerate}

  However, this formulation suggests.
  \(\psi\) has value \(v'\) from claimed support for \(\psi\) having value \(v'\).
  Strictly, this need not be the case.
  Most, if not all, \illu{1} will have this form.
  However, important is dealing with \requ{}, and not how agent gets to \(\psi\) having value \(v'\).
\end{note}

\begin{note}
  Intuitively, previous reasoning was good, so don't need to go through the details again.
  Note, this does not guarantee that appeal will be successful.

  Following section deals in part with this issue.
\end{note}

\begin{note}
  Strictly speaking, however, \autoref{assu:CS-persists} and following are not required.
  As we will see, restriction is new `defeaters'.
  If \autoref{assu:CS-persists} does not hold, then reasoning about these.
  However, \autoref{assu:CS-persists} is plausible.
  Hence, additional assumptions.
\end{note}



\section{\illu{3}}
\label{sec:CS:illustrations}

\begin{note}
  Collection of assumptions, expanding on two ideas, and propositions building on these.
  Work through how these assumptions combine in some cases.
  Also drawing out intuition.

  To start, familiar case, distinguish from circularity and related idea.
  Then, looking at \autoref{prop:CS-only-if-reason-recognised-defeaters}.
  Pair of simple \illu{1}.
  Then, final \illu{0} to highlight intuition and draw out some tension.
  After these, corollary of \autoref{prop:CS-only-if-reason-recognised-defeaters} which summarises, some discussion, and an idea.
\end{note}

\subsection{\illu{3} with respect to ideas}

\begin{note}[Testimony 1]
  \begin{illustration}[Testimony 1]
    \label{illu:CS:test:basic}
    \mbox{}
    \begin{enumerate}[label=\arabic*., ref=(\arabic*)]
    \item\label{ex:eiS:t:basic:test} \nagent{11} testified that they are trustworthy when speaking on matters regarding their personal character.
    \item Any agent and proposition, agent testified that proposition is the case only if proposition is the case.
    \item \nagent{11} testified that p is the case only if p is the case.
    \item\label{ex:eiS:t:basic:ok} \nagent{11} is trustworthy when speaking on matters regarding their personal character.
    \end{enumerate}
  \end{illustration}

  I take the reasoning of~\autoref{illu:CS:test:basic} to be intuitively problematic.

  I am also somewhat confident that you will have seen some variant of the reasoning in relation to circularity before.

  The goal for the moment is to explain why the two ideas expressed in relation to claiming support (\nfcs{} and \eiS{}) highlight a way in which the reasoning is problematic.
  And, to distinguish the way in which the ideas highlight a problem from a pair of nearby considerations.
\end{note}

\begin{note}
  Observe, however, that the intuitive problem is not that the agent has any reason(ing) to think that \nagent{11} is \emph{not} trustworthy when speaking on matters regarding their personal character.

  Rather, the intuitive problem is that the agent does not have any reason(ing) to think that \nagent{11} \emph{is} trustworthy when speaking on matters regarding their personal character.

  In particular, that that \nagent{11} is not trustworthy when speaking on matters regarding their personal character is simply a possibility.
  It may be the case that \nagent{11} is trustworthy.\nolinebreak
  \footnote{
    \color{red}
    It's not like this suggests that they are not trustworthy.
    Asking for directions.
    These are fine, but addition is not.
  }
\end{note}

\begin{note}
  Following, let us consider why the reasoning of~\autoref{illu:CS:test:basic} is intuitively problematic from the perspective of claiming support.\nolinebreak
  \footnote{
    Of course, the reasoning of \autoref{illu:CS:test:basic} seems problematic without constraints on the purpose.
    However, our interest is with claiming support.
  }

  \ideaCSA{} and \ideaCSB{}.

  The role of \ideaCSA{} is simple: It is possible for the agent's reasoning to be \mom{}.
  For example.

  Possibility of \mistaken{}.
  Well, misheard, speaking sarcastically.
  Possibility of \misled{}.
  Possible that S is not trustworthy.

  It is the possibility of being \misled{} that is of interest.

  Here, \ideaCSB{}.
  In order for the reasoning to be an instance of claiming support, the reasoning should indicate that \nagent{11} is not trustworthy on matters regarding their personal character regardless of whether the reasoning that \nagent{11} is trustworthy on matters regarding their personal character is \mom{}.

  In particular, it should be possible for the agent to entertain the possibility that \nagent{11} is not trustworthy on matters regarding their personal character while maintaining that their reasoning indicates that \nagent{11} is trustworthy on matters regarding their personal character.
\end{note}

\begin{note}[The problem]
  The problem, then, is that it does not seem possible for the agent to entertain the possibility that \nagent{11} is not trustworthy when speaking on matters regarding their personal character while maintaining that their reasoning indicates that \nagent{11} is trustworthy when speaking on matters regarding their personal character.

  For, it may be the case that \nagent{11} is not trustworthy when speaking on matters regarding their personal character.
  And, \nagent{11}'s testimony involves speaking on a matter regarding their personal character.
  Hence, by entertaining the the possibility that \nagent{11} is not trustworthy on matters regarding their personal character, the agent is required to entertain the possibility that \nagent{11}'s statement did not amount to an instance of testimony.
  And, if \nagent{11}'s statement did not amount to an instance of testimony then it would seem the agent lacks a line of reasoning that indicates that \nagent{11} is trustworthy when speaking on matters regarding their personal character.
\end{note}

\begin{note}
  The preceding is only an expression of an intuition.
  \ideaCSA{} and \ideaCSB{} are ideas, and the arguments that follow will rest on the assumptions and propositions drawn from these ideas, rather than the ideas themselves.
  Still, to the extent motivation for those assumptions and propositions rest on these ideas, \autoref{illu:CS:test:basic} highlights the constraints the ideas place on claiming support.

  \begin{itemize}
  \item given \ideaCSA{}, the agent is required to consider the possibility that conclusion of their reasoning is not the case, and
  \item given \ideaCSB{}, the agent is required to hold that their reasoning indicates that the conclusion of their reasoning is not the case regardless of whether the possibility that conclusion of their reasoning is not the case obtains.
  \end{itemize}
  The reasoning of \autoref{illu:CS:test:basic} fails to be an instance of claiming support because the possibility that \nagent{11} is not trustworthy when speaking on matters regarding their personal character is sufficient to undercut the premise that \nagent{11} testified.
\end{note}

\begin{note}
  Now, granting that the above identifies a problem, an immediate question is:
  Is the problem an instance of (vicious) circularity?\nolinebreak
  \footnote{
    In advance of following discussion, variation on~\textcite{Sorensen:1991wh}.

    Claim support for the following proposition from the following sentence:

    \begin{quote}
      Some sentences are typed on a computer.
    \end{quote}

    In line with suggestions of \citeauthor{Sorensen:1991wh}, seems fine.
    No difficulty with ideas as compatible with something else happening.
    But, background that this is so incredibly unlikely.
  }

  Circularity is certainly in the ballpark, but I do not think there is a straightforward reduction.
\end{note}

\begin{note}
  First, the problem was motivated by view the agent's reasoning as an instance of claiming support and applying the two basic ideas of claiming support (\ideaCSA{} and \ideaCSB{}).
  And, in general, it seems that circularity extends beyond the scope of claiming support.

  For example, consider knowledge:
  The reasoning of \autoref{illu:CS:test:basic} seems problematic when view from the perspective of establishing knowledge.
  Yet, from such a viewpoint it seems \ideaCSA{} would not apply --- if the agent had come to know that \nagent{11} is trustworthy when speaking on matters regarding their personal character then it would not be possible (at least relative to the conclusion of their reasoning) that \nagent{11} is not trustworthy.
  Hence, neither \ideaCSA{} nor \ideaCSB{} would apply.

  To be clear, our interest is not that the claiming support explains why the reasoning is problematic.
  Rather, the point is that the sketch given of why the reasoning is problematic when viewed as an instance of claiming support is distinct from circularity, as it seems circularity would extend to cases for which \ideaCSA{} nor \ideaCSB{} would not apply.
\end{note}

\begin{note}
  Second, the term `circularity' suggests that the reasoner has taken the conclusion of the reasoning for granted.

  For example, consider what \citeauthor{Sgaravatti:2013wu} terms the `Justification Account' of circularity.\nolinebreak
  \footnote{
    As \citeauthor{Sgaravatti:2013wu} notes, the Justification Account of circularity is a rewriting of the third type of `epistemic dependence' considered by \citeauthor{Pryor:2004ws}~(\citeyear[359]{Pryor:2004ws}).
    Neither \citeauthor{Pryor:2004ws} nor \citeauthor{Sgaravatti:2013wu} endorse the Justification Account, but I take the spirit of the account to sufficient for interest.
    Still, the considerations which follow also apply to distinguish the {\color{red} problem identified} from \citeauthor{Sgaravatti:2013wu}'s favoured account (\Citeyear[\S3]{Sgaravatti:2013wu}) and the fifth type of `epistemic dependence' considered by \citeauthor{Pryor:2004ws}~(\citeyear[359]{Pryor:2004ws}).
  }

  \begin{quote}
    \begin{enumerate}[label=(JA), ref=(JA)]
    \item\label{sg:JA} An argument is circular if and only if for you to have justification to believe the premisses, it is necessary that you have justification to believe the conclusion.\nolinebreak
      \mbox{}\hfill\mbox{(\Citeyear[754]{Sgaravatti:2013wu})}
    \end{enumerate}
  \end{quote}
  Where `justification to believe' is to be read as in terms of having formed the belief in an epistemically appropriate way as opposed to (merely) possessing sufficient resources to form formed the belief in an epistemically appropriate way.\nolinebreak
  \footnote{
    Or, however you prefer to characterise \citeauthor{Firth:1978vi}'s (\Citeyear{Firth:1978vi}) distinction between doxastic and propositional justification (or warrant).
    See also \citeauthor{Silva:2020aa} (\Citeyear{Silva:2020aa}) --- esp.\ fn.\ 1.
  }
  (\citeauthor[Cf.][754--755]{Sgaravatti:2013wu})

  Observe, \ref{sg:JA} applies to \autoref{illu:CS:test:basic} only if the agent requires a justified belief that \nagent{11} is trustworthy prior to the conclusion of the agent's reasoning.

  Such may be the case, and plausibly is, but to the extent that the instance of claiming support is an instance of forming a justified belief, the problem highlighted by appeal to \ideaCSA{} and \ideaCSB{} relied only on entertaining the possibility that \nagent{11} is not trustworthy.
  Still, that it was necessary for the agent to have claimed support for \nagent{11} being trustworthy in order to claim support for the premises does not follow without additional argument --- no matter how plausible this may be.\nolinebreak
  \footnote{
    Though I have doubts about whether this really is the case.
  }

  % {
  % \color{red}
  % Necessary, as could argue that this is an implicit assumption.
  % Hence, .
  % Given plausibility, possible.

  % Yet, equally, that these two things don't go together regardless of perspective on premise.

  % And, in this sense order of explanation would be reversed.
  % Still a question of why required.
  % }

  %   Admittedly this is a somewhat delicate point.
  %   One may argue that \ref{sg:JA} (or some variant) explains why it is (intuitively) not possible for the agent to entertain possibility that \nagent{11} is not trustworthy given their reasoning.

  %   Indeed, even if it is the case that the agent is required to have claimed support (or a justified belief) for \nagent{11} being trustworthy to claim support for the premises, the problem identified via \ideaCSB{} would remain distinct and would seem at best to motivate such an additional restriction.
  Leaves open the possibility that problem highlighted by \ideaCSA{} and \ideaCSB{} does not reduce.
  Hence, seems that distinguish intuitions from \ideaCSA{} and \ideaCSB{} from intuitions about why or how the agent introduced \nagent{11} testifying as a premise.
\end{note}

\begin{note}
  The basic intuition, really, is that from \ideaCSA{} there is the possibility of being \mom{}.
  In particular, \mistaken{} about testimony.
  So, various ways in which this premise may fail.
  The conclusion not holding is one such way.
  And, if this is the case then whatever considerations the agent has for testimony, those considerations do not extend.
\end{note}

\begin{note}
  Still, rather than toy in the abstract, let's investigate further by granting the agent with a way of claiming support for the initial premise of the reasoning.
  The problem identified by \ideaCSA{} and \ideaCSB{} will remain, but to press the problem of circularity will require stronger assumptions.
\end{note}

\begin{note}[Testimony 2]
  \begin{illustration}[Testimony 2]
    \label{illu:CS:test:with-CS-for-premise}
    \mbox{}
    \begin{enumerate}[label=\arabic*., ref=(\arabic*)]
    \item I was assured that \nagent{11} will be honest with me throughout the meeting.
    \item\label{ex:eiS:tt:test} \nagent{11} testified that they are trustworthy on matters regarding their personal character.
    \item Any agent and proposition, agent testified that proposition is the case only if proposition is the case.
    \item \nagent{11} testified that p is the case only if p is the case.
    \item\label{ex:eiS:tt:ok} \nagent{11} is trustworthy on matters regarding their personal character.
    \end{enumerate}
  \end{illustration}

  \autoref{illu:CS:test:with-CS-for-premise} seems intuitively problematic to a similar degree as \ref{illu:CS:test:basic}.

  The prior assurance does not help, at least given context that the friend was not aware of what \nagent{11} would say.

  However, the prior assurance does provide a clear account of how the agent introduced \nagent{11} testifying as a premise.
\end{note}

\begin{note}
  Problem here remains.
  Now, two instances of claiming support.

  Observe, vouched for particular instance, but not general.
  Assurance went for statements in general, but now have information about particular statement.
  Seems okay in various cases.

  \begin{itemize}
  \item The venue will be crowded.
  \item A squirrel took the birdseed.
  \item \TeX is Turing-complete
  \end{itemize}
  (Observe, in these instances not a \requ{}.)
  Assurance is sufficient.
  In general, so long as appeal to instance of testimony while granting that content of that instance of testimony (and hence the testimony itself) may fail, then there is no tension with the assumptions made regarding claimed support.

  Yet, the reasoning seems to remain problematic with respect to self-attribution of trustworthiness.
\end{note}

\begin{note}
  The issue for \ref{sg:JA} is that it does not seem necessary for the agent to have justification to believe that \nagent{11} is trustworthy on matters regarding their personal character in order to have justification to believe that \nagent{11} would be honest throughout the meeting.

  Instead, it seems that there is a limitation on the scope of the assurance that is sensitive to the content of what \nagent{11} said (or would say).
  In turn, the agent's reasoning is problematic because the reasoning exceeds the relevant limitation.
  And, while it may be the case that would need justification for conclusion to have unlimited justification for the premises, the limitation alone is sufficient identify the problem.

  This is preferred understanding of the problem raised by \ideaCSA{} and \ideaCSB{}.
\end{note}

\begin{note}
  However, granting that the problem arises from some limitation, it is important to keep in mind that the limitation arises from entertaining some possibility as opposed to assuming that the possibility obtains.

  An instance of a limitation arising from assuming that the possibility obtains is the fourth type of dependence between premise and conclusion considered by \citeauthor{Pryor:2004ws}.

  \begin{quote}
    [Type 4] dependence between premise and conclusion is that the conclusion be such that evidence \emph{against it} would (to at least some degree) undermine the kind of justification you purport to have for the premises.\nolinebreak
    \mbox{}\hfill\mbox{(\citeyear[359]{Pryor:2004ws})}
  \end{quote}

  Again, plausible.

  Issue:
  \begin{enumerate}
  \item Evidence undermines the kind of justification the agent purports to have for the premises.
  \end{enumerate}

  Seems to hold for \autoref{illu:CS:test:basic}.
  Would undermine appeal to \nagent{11}'s testimony.
  However, fails for \autoref{illu:CS:test:with-CS-for-premise}.
  Would not undermine the companion's assurance, even if viewed as an instance of testimony.

  Run this through the other instances suggested.

  And, as \citeauthor{Pryor:2004ws} notes, \emph{kind} is important.
  However, it seems kind is not the only problem.
\end{note}

\begin{note}
  Also compatible with \citeauthor{Pryor:2004ws}'s objection to type 4 being sufficient to identify problematic reasoning.
  Details are in the following footnote.\footnote{
  Compatible with \citeauthor{Pryor:2004ws}'s objection to type 4 dependence.

  % \begin{illustration}
    % \mbox{}
    % \vspace{-\baselineskip}
    \begin{quote}
      Suppose you're watching a cat stalk a mouse. Your visual experiences justify you in believing:

      \begin{enumerate}[label=(\arabic*), ref=(\arabic*)]
        \setcounter{enumi}{10}
      \item\label{illu:Pryor:cat:1} The cat sees the mouse.
      \end{enumerate}

      You reason:

      \begin{enumerate}[label=(\arabic*), ref=(\arabic*), resume]
      \item\label{illu:Pryor:cat:2} If the cat sees the mouse, then there are some cases of seeing.
      \item\label{illu:Pryor:cat:3} So there are some cases of seeing.\nolinebreak
        \mbox{}\hfill\mbox{(\citeyear[361]{Pryor:2004ws})}
      \end{enumerate}
    \end{quote}
  % \end{illustration}

  Setting aside whether this is fine.

  Following \citeauthor{Pryor:2004ws}:

  Bad, given proposal, as if no cases of seeing, then the cat is not seeing. (\citeyear[361]{Pryor:2004ws})

  \citeauthor{Pryor:2004ws}'s position is as follows:

  \begin{quote}
    I don't think you need antecedent justification to believe \ref{illu:Pryor:cat:3}, before your experiences can give you justification to believe \ref{illu:Pryor:cat:1}.
    I also think it's plausible that your perceptual justification to believe \ref{illu:Pryor:cat:1} contributes to the credibility of \ref{illu:Pryor:cat:3}.\nolinebreak
    \mbox{}\hfill\mbox{(\citeyear[361]{Pryor:2004ws})}
  \end{quote}

  This is compatible with \ideaCSA{} and \ideaCSB{}, and the assumptions and premises which have followed.
  You need not agree with \citeauthor{Pryor:2004ws}

  No clear trouble.
  The possibility alone isn't going to do enough.
  Still seems possible for 1 to hold up.
  Would count against, but not clear that entertaining possibility raises issue for claimed support for premise.
  Indeed, also compatible with the cat not seeing, but if this is the case then it's not clear why no cases of seeing is any more important than the singular case.
  }
\end{note}

\begin{note}
  Of course, still some question about why entertaining the possibility is sufficient.
  And, why there does not seem to be considerations for certain consequences of claimed support.

  Nothing insightful beyond broad account.

  Given this, the purpose of the assumptions made is to narrow down a clearer problem.
  No reasoning, and as an instance of this when there is the impossibility of such reasoning.

  Examples of the former.
\end{note}

\subsection{\illu{3} with respect to assumptions}
\label{sec:illustrations-wrt-assumption}

\begin{note}
  Previous section.
  \ideaCSA{} and \ideaCSB{}.
  Motivation for assumptions, and effort to clarify in contrast to other intuitions.
  Fell short of an account of the intuition.

  \autoref{assu:supp:nfactive}, \autoref{assu:supp:independence}.
  And, from these \autoref{prop:CS-only-if-reason-recognised-defeaters}.

  Key here is lack of reasoning.
  This allows for the possibility that some variant instance of reasoning would succeed in claiming support.
  Indeed, we will suggest.
\end{note}

\begin{note}
  Two short, then one in some detail.
\end{note}

\subsection{Spot the difference}

\begin{note}[Spot the difference]
  \begin{illustration}[Spot the difference]
    \label{illu:CS:spot-the-diff}
    The agent has been working through a spot-the-difference to pass some time.

    Though the time is not completely passed, the agent examined the two images with what seems sufficient care to claim support that they have found all the differences.
    However, the agent did not keep track of the number of differences.

    The agent announces `I have found all the differences' and their companion responds `All fifteen?'.

    \begin{enumerate}[label=\arabic*., ref=(I\ref{illu:CS:spot-the-diff}.\arabic*)]
      \setcounter{enumi}{-1}
    \item\label{illu:CS:spot-the-diff:info} If I have found all the differences, I have found fifteen differences.
    \end{enumerate}

    The agent then reasons as follows:

    \begin{enumerate}[label=\arabic*., ref=(I\ref{illu:CS:spot-the-diff}.\arabic*), resume]
    \item Exhaustive search.
    \item\label{illu:CS:spot-the-diff:all} I found all the differences.
    % \item\label{illu:CS:spot-the-diff:info} My companion has testified that there are fifteen differences.
    % \item\label{illu:CS:spot-the-diff:cond} If I have found all the differences, I have found fifteen differences.
    \item\label{illu:CS:spot-the-diff:fif} So, I have found fifteen differences. \hfill (From \ref{illu:CS:spot-the-diff:info} and \ref{illu:CS:spot-the-diff:all})
    \end{enumerate}
  \end{illustration}

  Before going further, structure of this.

  The agent performed some reasoning, and concluded that they found all the differences.
  However, that reasoning is mentioned but not stated in the \illu{0}.
  Rather, present is distinct instance of reasoning after being provided with information.
  ``If not 15, then problem''.
  Present reasoning appeals to past reasoning, and draws out consequence of this given new information.
  Important: the present reasoning does not consider possibility that the agent did not find all 15 differences.
  Instead, consequence of conclusion of previous instance of reasoning.
  Still, epistemically possible that the agent did not find 15 differences.
\end{note}

\begin{note}
    Providing additional information about what the agent has claimed support for.
  Recall, \autoref{assu:CSVP}, information rather than states of affairs.
  \nolinebreak
  \footnote{
    Still slight issue.
    Offering a redescription.
    You met Clark Kent, so you met Superman.
    In this case, rather than claiming support for meeting Superman, provided information is seen as an equivalent formulation.
    It is possible to read \autoref{illu:CS:spot-the-diff} in this way, and this might be the most natural interpretation.
    However, it is not the interpretation under which see the problem.
    Rather, problem is where the conditional is explicit.
    Unlike Superman case, proper conditional.
  }
\end{note}

\begin{note}
  Information leads to \requ{}.

  Possibility of not fifteen.
  And, not merely that the agent performed the reasoning, but that the reasoning identified all.
  If not fifteen, then not all, so would involve appeal to something that is not the case.

  And, present reasoning does not include reasoning about \requ{}.
\end{note}

\begin{note}[Avoiding the problem]
  This doesn't rule out some additional reasoning.
  \begin{enumerate}
  \item Exhausted search.
  \end{enumerate}
  Difference here is that the agent is not only appealing to having found.
  In addition, what they recall about reasoning.
  What matters is not that found all but rather that exhaustivity of search.
  This is not specific to 15.
  Up to some \(k\) such that agent is still confident that they performed an exhaustive search.

  Whether you think this is enough is up to you.
  On the one hand, intuitive that this does enough.\nolinebreak
  \footnote{
    Indeed, reasoning framed with all as I think it is much less clear here.
  }

  On the other hand, the agent did not keep track of the number of differences.
  So, may hold that they should go back and count.\nolinebreak
  \footnote{
    Looking ahead, \nI{}.
    Difficulty here is that don't need to go to \(\phi\).
    Indeed, note somewhere that \nI{} really only clearly takes hold when need some sort of factivity in play.
    We'll return to this.
  }

  {
    \color{red} As observed in the footnote above, the trick here is that the agent doesn't really `need' to go to having found all the differences.

    Alternatively, not enough to show that the reasoning is bad.
    E.g.\ then I would have been deceived, some trick, etc.
    Something \emph{I} wouldn't count as a difference.
  }
\end{note}

\begin{note}
  Argued above against circularity.
  Here, additional consideration.

  If the agent were to have had the information first time, then plausibly an instance of circularity.
  And, may think that this is also circularity as must also all must amount to fifteen.
\end{note}

\subsection{Where's Wally}

\begin{note}
  \autoref{illu:CS:spot-the-diff} had something that could be obtained from the reasoning if re-examined.
  Just need to add a counter.

  Now, something that follows if the reasoning was \nmom{}.
\end{note}

\begin{note}
  \begin{illustration}[Where's Wally]
    \label{illu:CS:wheres-wally}
    Searching for Wally.
    On front of book is an image of wally in contrast to a number of other characters.
    Takes not of a number of features.
    Glasses, hat, striped jumper.
    In isolation, necessary but insufficient.
    Combined, sufficient.

    Search through the image.
    I've found Wally.
    Did you spot the cane first?

    The question carries implicit information:
    \begin{enumerate}[label=\arabic*., ref=(I\ref{illu:CS:wheres-wally}.\arabic*)]
      \setcounter{enumi}{-1}
    \item\label{illu:CS:wheres-wally:info} The individual identified is Wally only if the individual is holding a cane.
    \end{enumerate}

    The agent then reasons as follows:

    \begin{enumerate}[label=\arabic*., ref=(I\ref{illu:CS:wheres-wally}.\arabic*), resume]
    \item Collection of features sufficient.
    \item\label{illu:CS:wheres-wally:ante} Had features.
    % \item The individual identified is Wally only if the individual is holding a cane.
    \item The individual was holding a cane. \hfill (From \ref{illu:CS:wheres-wally:info} and \ref{illu:CS:wheres-wally:ante})
    \end{enumerate}
  \end{illustration}

  As with \autoref{illu:CS:spot-the-diff}, the reasoning of \autoref{illu:CS:wheres-wally} is such that the individual was holding the cane is an \requ{} of the claimed support that the individual was Wally, and no reasoning about it.
\end{note}

\begin{note}
  Problem.
  \requ{}.
  Agent didn't notice, so possibility.
  And, agent is appealing to it being Wally.
  Not merely that they identified as Wally.

  It's not clear need to give up claimed support, but does not extend to having a cane.

  In contrast to \ref{illu:CS:spot-the-diff}, does not seem there is anything weaker to fall back on.
  However, no strong claim here.
  Rely on stronger principles about claiming support.
\end{note}

\subsection{A trip to the zoo}

\begin{note}
  \illu{3} \ref{illu:CS:spot-the-diff} and \ref{illu:CS:wheres-wally} looked at reasoning and \requ{1}.
  Final \illu{0} is no different, structurally similar to \autoref{illu:CS:wheres-wally}.
  However, surrounding discussion to clarify details.

  After discussion, a corollary and a conjecture.
\end{note}

\begin{note}
  Zebra.

  Dretske is about knowledge.
  Problem for knowledge as factive.

  Still, don't need factive move.
  Possible not zebra, but vision is sufficient to expect that such a possibility does not obtain.

  Key here is that claiming support is never going to be strong enough to establish knowledge, at least to the extent that knowledge is factive.
\end{note}

\begin{note}
  \begin{illustration}[A trip to the zoo]
    \label{illu:CS:dretske-zebra}
    \mbox{}
    \vspace{-\baselineskip}
  \begin{quote}
    You take your son to the zoo, see several zebras, and, when questioned by your son, tell him they are zebras.
    Do you know they are zebras?
    Well, most of us would have little hesitation in saying that we did know this.
    We know what zebras look like, and, besides, this is the city zoo and the animals are in a pen clearly marked ``Zebras.''
    Yet, something's being a zebra implies that it is not a mule and, in particular, not a mule cleverly disguised by the zoo authorities to look like a zebra.
    Do you know that these animals are not mules cleverly disguised by the zoo authorities to look like zebras?

    \mbox{ }\hfill \(\vdots\) \hfill\mbox{ }

    Did you examine the animals closely enough to detect such a fraud?\linebreak
    \mbox{}\hfill\mbox{(\citeyear[1015--1016]{Dretske:1970to})}
  \end{quote}
  \vspace{-\baselineskip}
  \end{illustration}
\end{note}

\begin{note}
  \citeauthor{Dretske:1970to}'s presentation focuses on knowledge, so let us briefly form a parallel with respect to claiming support:

  \begin{illustration}
    \label{illu:dretske-zebra-var}
    \mbox{}
    `What if those animals are mules cleverly disguised by the zoo authorities to look like zebras?'
    \begin{enumerate}[label=\arabic*., ref=(I\ref{illu:CS:wheres-wally}.\arabic*)]
      \setcounter{enumi}{-1}
    \item If zebra, then not cleverly disguised mule.
    \end{enumerate}
    Reasons as follows:
    \begin{enumerate}[label=\arabic*., ref=(I\ref{illu:CS:wheres-wally}.\arabic*)]
    \item That animal appears to be a zebra.
    \item That animal is a zebra.
    \item That animal is not a cleverly disguised mule.
    \end{enumerate}
  \end{illustration}

  As with \illu{3}~\ref{illu:CS:spot-the-diff} and~\ref{illu:CS:wheres-wally}, no reasoning about the possibility that the animal is a cleverly disguised mule.
  However, given conditional, \requ{} when moving from appearance to animal.

  Introduction of additional consequence, so same structure as~\autoref{illu:CS:wheres-wally}.
  And, plausible that the agent may reason about the possibility that the animal is a cleverly disguised mule in a way sufficient to claim support.\nolinebreak
  \footnote{
    \citeauthor{Dretske:1970to} has a number of suggestions.
    \begin{quote}
    You have some general uniformities on which you rely, regularities to which you give expression by such remarks as, ``That isn't very likely'' or ``Why should the zoo authorities do that?''
    Granted, the hypothesis (if we may call it that) is not very plausible, given what we know about people and zoos.
    But the question here is not whether this alternative is plausible, not whether it is more or less plausible than that there are real zebras in the pen, but whether you know that this alternative hypothesis is false.\nolinebreak
    \mbox{}\hfill\mbox{(\citeyear[1016]{Dretske:1970to})}
  \end{quote}
  }
\end{note}

\begin{note}
  if think that doesn't know, then not too much of an issue.
  However, does know?
  Indeed, \citeauthor{Dretske:1970to}.

  No clear tension.
  Knowledge and claiming support, different.

  However, problem identified.
\end{note}

\begin{note}
  \begin{enumerate}[label=K\Alph*., ref=K\Alph*]
  \item\label{Dretske:No-C:cond:no-k-then-ep} If an agent does not know that \(\phi\) is true, then \(\phi\) being false is an epistemic possibility for that agent.
  \item\label{Dretske:No-C:cond:ep-then-no-k} If \(\phi\) being false is an epistemic possibility for some agent, then that agent does not know that \(\phi\) is true.
  \end{enumerate}
  When taken together, (\ref{Dretske:No-C:cond:no-k-then-ep}) and (\ref{Dretske:No-C:cond:ep-then-no-k}) state that: An agent knows that \(\phi\) is true if and only if \(\phi\) being false is not an epistemic possibility for the agent.
  Still, our interest will primarily be with (\ref{Dretske:No-C:cond:ep-then-no-k}).\nolinebreak
  \footnote{
    \label{fn:factivity-two-readings}
    Contrast to `factivity':
    \begin{itemize}
    \item If \(S\) knows that \(\phi\), then \(\phi\).
    \end{itemize}
    This may be read in at least two different ways.
    \begin{itemize}
    \item First, relation between epistemic state of the agent and state of the world:\newline
      \mbox{} \qquad If \(S\) knows that \(\phi\), then the state of the world is such that \(\phi\) is the case.
    \item Second, how things appear from epistemic state:\newline
      \mbox{} \qquad If \(S\) knows that \(\phi\) then every way the state of the world may be for \(S\) includes \(\phi\).
    \end{itemize}

    The two reading are independent of one another.

    For example, suppose you walked to the shop but the only epistemic possibility entertained by your friend is that you drove to the shop.
    Here, it is not possible for your friend to know that you drove to the shop on the first reading of factivity, but the second reading is not ruled out.

    Conversely, suppose it is the case that you walked to the shop but your friend considers it epistemically possibly that you drove.
    Here, knowing on the second reading of factivity is ruled out, but the first reading is not ruled out.

    Of course, you may endorse both readings of factivity.
    Our focus is on the `weaker' reading as we have made no connexion between claiming support and the state of the word.
    (Perhaps it is of some interest to note that \citeauthor{Dretske:1970to} explicitly denies the second reading, but not the first.)
  }\(^{,}\)\nolinebreak
  \footnote{
    The dogmatism paradox (\cite[39,43--45]{Kripke:2011wv};\cite[148]{Harman:1973ww}) seems to concern the second reading of factivity from~\autoref{fn:factivity-two-readings}, and intuitions concerning evidence.

  Roughly stated, the paradox pairs the following two propositions:
  \begin{enumerate}[label=D\arabic*., ref=(D\arabic*)]
  \item\label{dog:1} If an agent is aware that they know that \(\phi\), then the agent may disregard any evidence against \(\phi\).
  \item\label{dog:2} Rational agents respect their evidence
    (\cite[Cf.][\S2]{Kelly:2016wk})
  \end{enumerate}
  Given~\ref{dog:2}, it seems no agent should not disregard any instance of evidence, even if the antecedent of~\ref{dog:1} is satisfied.

  And, it seems \ref{dog:1} is motivated by factivity.
  For, if the agent is aware that they know that \(\phi\) then the agent knows that \(\phi\).
  And, as knowledge is factive it follows (by second reading) that \(\phi\) is the case.
  In turn, if it is the case that \(\phi\) then any evidence against \(\phi\) is evidence for something that is not the case.
  Hence, the agent may disregard any evidence against \(\phi\).


  Indeed, the second reading of factivity seems required.
  For, it seems an agent is only (apparently) in a position disregard any evidence against \(\phi\) because there knowledge that \(\phi\) guarantees that \(\phi\) is the case.
  If \emph{not}-\(\phi\) is (merely) an epistemic impossibility, and it is not clear why evidence may require an agent to revise what they consider possible.

  Note:
  Neither \citeauthor{Kripke:2011wv} (nor \citeauthor{Harman:1973ww}) make explicit mention of the agent being aware that they know \(\phi\) when formulating the Dogmatism paradox.
      Still, the paradox is clearer with this stated, as it's require addition work to find issue with a permission (to disregard evidence) if an agent is not aware that they have such a permission.

      More generally, I agree with \citeauthor{Zhaoqing:2015vj}'s (\Citeyear{Zhaoqing:2015vj}) proposal to understand the paradox in terms of knowledge attribution rather than of knowledge proper.
  }
  \citeauthor{Dretske:1970to} observes that endorsing (\ref{Dretske:No-C:cond:no-k-then-ep}) and (\ref{Dretske:No-C:cond:ep-then-no-k}) leads to closure.
  The following is a reconstruction.\nolinebreak
  \footnote{
    Specifically, the following passage:
    \begin{quote}
      A slightly more elaborate form of the same argument goes like this:
      If \(S\) does not know whether or not \(Q\) is true, then for all he knows it might be false.
      If \(Q\) is false, however, then \(P\) must also be false.
      Hence, for all \(S\) knows, \(P\) may be false.
      Therefore, \(S\) does not know that \(P\) is true.\nolinebreak
      \mbox{}\hfill\mbox{(\citeyear[1011]{Dretske:1970to})}
    \end{quote}
    Note: (\ref{Dretske:No-C:cond:no-k-then-ep}) is a reformulation of the first conditional of the passage, while a formulation (\ref{Dretske:No-C:cond:ep-then-no-k}) seems required to move from `\(P\) may be false' to `\(S\) does not know that \(P\) is true'.
  }
\end{note}

\begin{note}[Closure argument]
  Let \(S\) be some agent and suppose:
  \begin{enumerate}[label=\arabic*., ref=\arabic*]
  \item\label{Dretske:No-C:k-entail} \(S\) knows that \(\phi\) entails \(\psi\).
  \item\label{Dretske:No-C:dunno-psi} \(S\) does not know that \(\psi\) is true.
  \end{enumerate}
  Consider the following argument:
  \begin{enumerate}[label=\arabic*., ref=\arabic*,resume]
  \item\label{Dretske:No-C:ep-not-psi} \(\psi\) being false is an epistemic possibility for \(S\).%
    \hfill (\ref{Dretske:No-C:cond:no-k-then-ep} \& \ref{Dretske:No-C:dunno-psi})
  \item\label{Dretske:No-C:no-ep-no-entail} \(\phi\) not entailing \(\psi\) is not an epistemic possibility for \(S\)%
    \hfill (\ref{Dretske:No-C:cond:ep-then-no-k} \& \ref{Dretske:No-C:k-entail})
  \item\label{Dretske:No-C:ep-not-psi-and-phi}  \(\phi\) being true while \(\psi\) is false is not an epistemic possibility for \(S\).%
    \hfill (\ref{Dretske:No-C:no-ep-no-entail})
  \item\label{Dretske:No-C:ep-not-phi} \(\phi\) may be false.%
    \hfill (\ref{Dretske:No-C:ep-not-psi} \& \ref{Dretske:No-C:ep-not-psi-and-phi})
  \item\label{Dretske:No-C:not-k-phi} \(S\) does not know that \(\phi\) is true.%
    \hfill (\ref{Dretske:No-C:cond:ep-then-no-k} \& \ref{Dretske:No-C:ep-not-phi})
  \end{enumerate}

  Hence, we have shown that, given (\ref{Dretske:No-C:cond:no-k-then-ep}) and (\ref{Dretske:No-C:cond:ep-then-no-k}), (\ref{Dretske:No-C:k-entail}) and (\ref{Dretske:No-C:dunno-psi}) imply (\ref{Dretske:No-C:not-k-phi}).

  That is to say, we have shown:
  \begin{enumerate}[label=K\Alph*., ref=(K\Alph*)]
    \setcounter{enumi}{2}
  \item\label{K:closure:from-arg} If \(S\) knows that \(\phi\) entails \(\psi\) and \(S\) does not know that \(\psi\) is true, then \(S\) does not know that \(\phi\) is true.%
    \mbox{} \hfill \((K_{S}(\phi \rightarrow \psi) \land \lnot K_{S}\psi) \rightarrow \lnot K_{S}\phi\)
  \end{enumerate}
  And rewriting:\nolinebreak
  \footnote{
    \((\phi \land \lnot\psi) \rightarrow \lnot\xi\) iff \(\phi \rightarrow (\lnot\psi \rightarrow \lnot\xi)\) iff \(\phi \rightarrow (\xi \rightarrow \psi)\).
  }
  \begin{enumerate}[label=K\Alph*\('\)., ref=(K\Alph*\('\))]
    \setcounter{enumi}{2}
  \item\label{K:closure:standard} If \(S\) knows that \(\phi\) entails \(\psi\), then if \(S\) knows that \(\phi\) is true then \(S\) knows that \(\psi\) is true.%
    \mbox{} \hfill \(K_{S}(\phi \rightarrow \psi) \rightarrow (K_{S}\phi \rightarrow K_{S}\psi)\)
  \end{enumerate}
\end{note}

\begin{note}
  Return to \autoref{illu:CS:dretske-zebra}.

  Let us assume you know that:
  \begin{itemize}
  \item If the animals are zebras then the animals are not cleverly disguised mules. And,
  \item The animals are zebras.
  \end{itemize}

  If \ref{K:closure:standard} holds, then you also know that the animals are not cleverly disguised mules.
  However, following \citeauthor{Dretske:1970to}'s intuition, you do not know that the animals are cleverly disguised mules.

  Hence, to accommodate \citeauthor{Dretske:1970to}'s intuition, either (\ref{Dretske:No-C:cond:no-k-then-ep}) or (\ref{Dretske:No-C:cond:ep-then-no-k}) must be rejected.
  \citeauthor{Dretske:1970to} rejects (\ref{Dretske:No-C:cond:ep-then-no-k}).\nolinebreak
  \footnote{
    See below.
  }
\end{note}

\begin{note}
  We now return to claiming support.

  \autoref{assu:supp:nfactive} requires that claiming support for \(\phi\) is compatible with the (epistemic) possibility that the claimed support is \nmom{}.
  And, the claimed support is \misled{} just in case \(\phi\) is not the case.
  Hence~\autoref{assu:supp:nfactive} requires that claimed support for \(\phi\) is compatible with the (epistemic) possibility that \(\phi\) is not the case.

  Therefore, the result of substituting `claimed support' from `knowledge' in (\ref{Dretske:No-C:cond:ep-then-no-k}) conflicts with \autoref{assu:supp:nfactive}.
  And so~\autoref{assu:supp:nfactive} parallels \citeauthor{Dretske:1970to}'s rejection of (\ref{Dretske:No-C:cond:ep-then-no-k}).
  However, \citeauthor{Dretske:1970to}'s rejection of (\ref{Dretske:No-C:cond:ep-then-no-k}) is motivated by a rejection of~\ref{K:closure:standard}.
\end{note}

\begin{note}[Link]
  The refinement of~\ideaCSB{} through \autoref{assu:supp:independence} has lead to a proposition close to~\ref{K:closure:standard}

  \begin{enumerate}[label=P\ref{prop:CS-only-if-reason-recognised-defeaters}\('\).]
  \item If \(\psi\) having value \(v'\) is a \requ{} of claiming support for \(\phi\), then if the agent has claimed support for \(\phi\) having value \(v\) then the agent has reasoned about whether \(\psi\) has value \(v'\).\newline
    \mbox{}\hfill \((\phi \leadsto \psi) \rightarrow (\text{CS}\phi \rightarrow \text{R}\psi)\)
  \end{enumerate}

  Both build on \autoref{def:requisite}.
  A \requ{}.
  This is a complex parallel to knowing that \(\phi\) entails \(\psi\).
  However, somewhat general, and applies to \(K\) also.
  {\color{red} (Strictly speaking, because of contraposition, make this more explicit below)}

  \autoref{assu:supp:independence} then requires reasoning.

  Requiring something with respect to \(\psi\) having value \(v'\) given some state of the agent an principle which relates \(\psi\) having value \(v'\) to that state.

  For sure, appealing, or having claimed support is distinct, so the closure is not that with respect to an \emph{an} epistemic operator such as knowledge.
  However, the tension here is in terms of viewing the rejection of \ref{K:closure:standard} as the endorsement of a `locality constraint'.
  {\color{red} What this means.}
  And, seem to violate such a constraint.
\end{note}

\begin{note}[`Locality constraint']
  \begin{quote}
    To know that \(x\) is \(A\) is to know that \(x\) is \(A\) within a framework of relevant alternatives, \(B\), \(C\), and \(D\).
    This set of contrasts, together with the fact that \(x\) is \(A\), serve to define what it is that is known when one knows that \(x\) is \(A\).
    One cannot change this set of contrasts without changing what a person is said to know when he is said to know that \(x\) is \(A\).\nolinebreak
    \mbox{}\hfill\mbox{(\citeyear[1022]{Dretske:1970to})}
  \end{quote}

  Where:
  \begin{quote}
    A relevant alternative is an alternative that might have been realized in the existing circumstances if the actual state of affairs had not materialized.\nolinebreak
    \footnote{
      \citeauthor{Dretske:1970to} adds:
  \begin{quote}
    \dots alternatives that \emph{might} have been realized in the existing circumstances if the actual state of affairs had not materialized.
    \dots are not relevant alternatives.\nolinebreak
    \mbox{}\hfill\mbox{(\citeyear[fn.6][1021]{Dretske:1970to})}
  \end{quote}
    }
    \nolinebreak
    \mbox{}\hfill\mbox{(\citeyear[1021]{Dretske:1970to})}
  \end{quote}
  Different from a \requ{}.
\end{note}

\begin{note}
  Point is not direct clash.\nolinebreak
  \footnote{Two problems.

    First, \citeauthor{Dretske:1970to} seems to go with no recognition at time, compatible with claiming support at time.
    So, would need instance of appealing to knowledge for some other purpose.
    Restating is not sufficient, assumptions made are compatible with persistence (as noted).

    Second, only get no need to rule out from \citeauthor{Dretske:1970to}, which does not require no reasoning.
  }
  At issue, rather, it the degree to which possibility is compatible with the absence of reasoning.
  That is, if reject closure, is this due to reasoning or due to requirements placed on such reasoning?\nolinebreak
  \footnote{
    (Problem here, but also extends to \nI{}.)
  }

  There is some subtlety, however.
  Kind of closure seems to break for many attitudes.
  This is not at issue, distinguishing feature of claiming support is closure, and the constraints this places on an agent.
  However, reject for stronger attitudes, then why for weaker?
\end{note}

\begin{note}
  Reasoning, but haven't placed constraints.
  So, this allows for something that may be quite weak.
  % I am here only restating what has gone before, but the added context may help.
  No reasoning, then leaves open possibility that does lead to a problem with the reasoning performed.

  Consider again relevant alternatives from internalist perspective.
  It seems, agent determines whether relevant or not will require reasoning.

  Consider:
  \begin{quote}
    `What if those reports about the zoo authorities cleverly disguising animals to look like other animals?
    If there are, could those animals be cleverly disguised mules?'
  \end{quote}
  Perhaps not a relevant alternative if the sense of `might' is non-epistemic, but if epistemic, then seems a problem.
\end{note}

\begin{note}
  To summarise.

  \citeauthor{Dretske:1970to}'s case.
  Rejection of closure.
  Question whether assumptions are okay.
  Argued that these are given understanding of claimed support, and that they plausibly extend to knowledge (and other attitudes intuitively stronger than that of having claimed support).
  For, rejection of closure plausibly amounts to rejection on strength of reasoning, rather than requirement to reason.

  Focus on this point for two reasons:
  First, distinguishing feature of claiming support and following will be about claiming support.
  Second, appeal to similar limitation later.
\end{note}

\newpage

\subsection{A corollary and a conjecture}

\begin{note}
  Talked about `closing principle'.
  Make this precise.
\end{note}

\begin{note}[Generalising point for K example]
  Generalise.
  Any time the agent appeals to a consequence.
  And, this is straightforward, because consequence is only going to hold with the antecedent.

  \begin{corollary}\label{corr:eiS:C:contraposition}
    Suppose an agent has prior claimed support for:
    \begin{enumerate}[label=\arabic*., ref=(\arabic*)]
    \item\label{corr:cond:p} \(\phi\) having value \(v'\). And,
    \item\label{corr:cond:pq} If \(\phi\) has value \(v\) then \(\psi\) has value \(v'\).
  \end{enumerate}
  Such that the agent claimed support for \ref{corr:cond:p} prior to \ref{corr:cond:pq} and as such the reasoning involved in claiming support for \ref{corr:cond:p} did not entertain the possibility of \(\psi\) not having value \(v'\).
  And, further suppose the agent holds that:
  \begin{enumerate}[label=\arabic*., ref=(\arabic*), resume]
    \item The claimed support for \ref{corr:cond:pq} (also) implies that if \(\psi\) does not have value \(v'\) then \(\phi\) does not have value \(v\).
  \end{enumerate}
  Then, if the agent engages in reasoning such that:
    \begin{enumerate}[label=\arabic*., ref=(\arabic*), resume]
    \item The agent does not appeal to anything other premises other than~\ref{corr:cond:p} and ~\ref{corr:cond:pq} from their prior claimed support, some form of conditional detachment applied to ~\ref{corr:cond:p} and~\ref{corr:cond:pq} to conclude that \(\psi\) has value \(v'\).
    \end{enumerate}
    Such that remains possible that \(\psi\) does not have value \(v'\).
    The reasoning is not an instance of claiming support.
  \end{corollary}

  \autoref{corr:eiS:C:contraposition} is a summary of the common problem between \illu{1} \ref{illu:CS:spot-the-diff},~\ref{illu:CS:wheres-wally}, and~\ref{illu:dretske-zebra-var}.
  And, as such captures sufficient conditions for \autoref{prop:CS-only-if-reason-recognised-defeaters}.
\end{note}

\begin{note}
  Observe, from \autoref{prop:CS-only-if-reason-recognised-defeaters} we have that reasoning that \(\psi\) has value \(v'\) is not an instance of claiming support {\color{red} if \requ{} with no reasoning about it}.

  By assumption, no reasoning about \(\psi\) not having value \(v'\) when claiming support for \(\phi\) having value \(v\).

  Therefore, to establish \autoref{corr:eiS:C:contraposition} we need only show that \(\psi\) having value \(v'\) is a \requ{} of the claimed support for \(\phi\) having value \(v\).

  And, the agent has come to consider \(\psi\) having value \(v'\) as a \requ{} of that claimed support as:
  \begin{itemize}
  \item Possible that \(\psi\) does not have value \(v'\), by assumption.
  \item If \(\psi\) does not have value \(v'\) then \(\phi\) does not have value \(v\), and hence the claimed support for \(\phi\) having value \(v\) would be \misled{}.
  \item And, as \(\phi\) having value \(v\) implies \(\psi\) has value \(v'\) it must be the case that \(\psi\) having value \(v'\) persists through to the conclusion of reasoning.
  \end{itemize}
\end{note}

\begin{note}
  Again, intuition is that claimed support for \(\phi\), but novel information about a possible defeater.
  So, it's no good to appeal to prior claimed support --- without additional reasoning about that claimed support --- as the claimed support did not take into account the possible defeater.
\end{note}

\begin{note}
  In general, expect this not to be much of a concern.

  Three considerations.
\end{note}

\begin{note}[Not claiming support]
  First, the obvious, claiming support and may not be the case that agent is claiming support in relevant reasoning.
\end{note}


\begin{note}[Deal with \requ{}]
  Second, in many cases claimed support for \(\phi\) is not going to hold up regardless of whether \(\psi\), so plausible reasoning about \(\psi\) as a \requ{}.

  For example:

  Alarm ringing, so fire.
  If no fire, not alarm ringing.
  However, even if consider the possibility that there isn't really a fire, clear that the alarm is ringing.
\end{note}

\begin{note}
  Strengthening, this will also not apply if the agent has claimed support that \(\phi \rightarrow \psi\) and \(\lnot \phi \rightarrow \psi\) and appeals.
  For, no \requ{} in this case.
\end{note}

\begin{note}[Requires contraposition]
  Finally, contraposition.
  It is not clear that contraposition always holds.

  And, without contraposition, retain claimed support for \(\phi\), and even if \mom{} \(\phi\) provides enough to think that \(\psi\) is the case.
  In particular, a variation of~\ref{corr:eiS:C:contraposition} does not \emph{necessarily} hold with respect to conditional probability.

  \begin{idea}\label{conj:eiS:C:probability}
    Assuming that sufficiently high probability is sufficient for claiming support.

    It is not necessarily the case that an agent may not claim support for \(\psi\) having value \(v'\) by appeal to:
    \begin{enumerate}[label=\arabic*., ref=(\arabic*)]
    \item\label{corr:prob:p} Claimed support for \(\phi\) having value \(v'\) such that no consideration of \(\psi\).
    \end{enumerate}

    Some time after \ref{corr:prob:p}:

    \begin{enumerate}[label=\arabic*., ref=(\arabic*), resume]
  \item\label{corr:prob:pq} Claimed support for \(\psi\) having value \(v'\) when \(\phi\) has value \(v\).
    \end{enumerate}

    If:
    \begin{enumerate}[label=\arabic*., ref=(\arabic*), resume]
    \item The agent does not appeal to any other premises other than claimed support for~\ref{corr:prob:p},~\ref{corr:prob:pq} and some principle regarding conditional probability.
    \end{enumerate}
    \vspace{-\baselineskip}
  \end{idea}

  Follows from the simple observation that conditional probability does not require that \(\phi\) is false when \(\psi\) is false.

  For example, set some threshold \(t\) and consider a probability distribution such that:
  \(P(\phi) > t\), \(P(\psi \mid \phi) > t\).
  It is consistent with such a distribution that \(P(\lnot\phi \mid \lnot\psi) = 0\).\nolinebreak
  \footnote{
    For example, if \(t = .0\), then let \(P(\phi) = .9\), \(P(\psi) = .92\), \(P(\phi \land \psi) = .81\), \(P(\lnot\phi \land \lnot\psi) = 0\).
  }
  And, therefore, from an agent's perspective it need not be the case that \(\phi\) would be false if \(\psi\) were false.
  In other words, it may be that entertaining possibility that \(\psi\) is false is just entertaining a restricted instance of \(\phi\) being true.
  Hence, \(\psi\) being true is not a \requ{} of \(\phi\) being true.

  Still,  \(\psi\) being true may be an `\emph{unrecognised} \requ{}' of \(\phi\) being true.
  However, we have made no assumptions regarding such `unrecognised \requ{1}'.

  {
    \color{red}
    Some care here, though.
    As the issue with possible defeaters is distinct from the probability of such defeaters.
    Indeed, given possibility, probability is typically low.
    Yet, this does not say anything about whether claimed support would hold up if the defeater turned out to be the case.

    The point of this {\color{red} idea} is to highlight an instance of a conditional that does not contrapose, and so does not lead to a \requ{}.
  }
\end{note}

\begin{note}
  These kinds of issues are fundamental to the argument.
  Issue is with a way of claiming support, rather than the possibility of claiming support.
  And, the issue is a result of the two assumptions made regarding claiming support.

  If something stronger or weaker, then the issue does not (necessarily) arise.
  Take it for granted that testimony.
  Knowledge excludes (epistemic) possibilities associated with defeaters.

  So, even if agent fails to claim support, may have done something interesting!

  However, focus is on claiming support.
  Possible defeaters, and some defence against those defeaters obtaining.

  Really important thing from these \illu{1} is that no particular assumptions about reasoning.
  \illu{3} all relied on absence of reasoning.
  So, although somewhat strong, still plausibly weak.
  Mentioned at many times things that seem sufficient.
  However, did not rely on these.

  So, claiming support is this odd thing.
  We will argue for additional proposition.
  \nI{}.
  Again, this will not rely on assumptions about reasoning.
\end{note}

\section{Summary of assumptions}

\begin{note}
  Claiming support.

  Two key ideas:

  \begin{enumerate}
  \item From \nfcs{}, always possible.
  \item From \eiS{}, doesn't depend.
  \end{enumerate}
  Second in part motivated by the first.

  Intuitively, stronger than belief, but weaker than knowledge.

  {
    \color{red}
    Claiming support is strong in the sense that it requires the agent's reasoning to hold up even if things are not how the agent thinks.
    However, strength is mitigated as bar for reasoning about a \requ{} may be set low.
  }

  Consequence of the first is way of establishing that reasoning does not result in claiming support.
  These assumptions are important.
  Argument relies on these assumptions.

  May be that it is possible to revise, but seem to capture something sufficiently interesting.
\end{note}

\begin{note}
  Well, it's a little more complex.
  What the argument really depends on is \ESU{} and \nI{}.
  Specifically \nI{}.

  It's not clear to me that these assumptions are strictly speaking required for \nI{}.
  And, some motivation of \nI{} independently of these.
  Still, this is as far as I got with \nI{}.

  Just so happens that these two assumptions also offer a nice `functional' characterisation.
  Therefore, appeal to these in order to set the stage.
\end{note}

\section{Closing focus on claimed support}

\begin{note}[Closing support]
  To summarise, claim of support.
  Certain kind of independence.
  Only interested in support, and not how this relates to attitudes.
  Somewhat intuitive, but no claims that this is the only understanding of support.

  For the moment, this provides clarity for understanding of support.
  Below, use to argue for failure to claim support.
\end{note}

\begin{note}[Something to emphasise]
  \color{red}
  Something to emphasise here is that this means that there's a way for an agent to claim support without being certain that \(\phi\) is the case.
  I don't have any answers for what this is.
  However, I do take this to be highly intuitive.
\end{note}


\begin{note}[Adequate]
  Kind of reasoning that we, the folk, do.
  Distinction for claiming support is that this is different from whether the agent has support, and we may set issues about whether the agent has support.

  Our interest is what is required for an agent to \emph{claim} support for (premises and) steps of reasoning, rather than what is required for an agent to \emph{have} support for (premises and) steps of reasoning.

  Use support as opposed to justification.
  Initial focus is on epistemic/doxastic attitudes.
  However, practical reasoning.
  For example, means-end.
  Support considered quite general to also include this.
\end{note}

\begin{note}
  Highlight again \phantref{dogmatism-wrt-nI}{below}.
\end{note}

%%% Local Variables:
%%% mode: latex
%%% TeX-master: "master"
%%% End: