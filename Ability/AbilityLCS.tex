\chapter{\LCS{8}}
\label{sec:second-conditional}

\section{Introduction}
\label{sec:introduction}
\label{sec:outline-lcs}

\begin{note}
  To role of the present chapter is primarily to state and clarify an observation regarding the structure of certain instances of claiming support.
  {\color{red} Specifically, about certain \requ{1}.}
  The observation follows from the assumptions made regarding claiming support from \ref{cha:claiming-support}.
  We term this limitation `\LCS{0}'.

  \LCS{} will have an important role in both establishing tension and constraining how the tension may be resolved.

  Still, as \LCS{} is solely an observation regarding \requ{1}, we will also state and clarify a straightforward consequence of \LCS{}, which we term `\FCS{0}'.
\end{note}

\begin{note}
  The primary purpose of \LCS{} and \FCS{} is to establish tension.

  In~\autoref{sec:first-conditional} \ESU{} directly rule out \adB{}.
  In~\ref{sec:second-conditional} \LCS{}, \FCS{} rule out \adA{}.\nolinebreak
  \footnote{
    However, also why \LCS{} is compatible with \adB{}.
  }

  In this respect, only really care about applying \LCS{} to ability.
  And, an option is to argue for the result of applying \LCS{} (and \FCS{}) directly to ability.

  However, consequence of assumptions.
  With every consequence, consideration that the consequence shows against the premises.
  In this respect, independent treatment of \LCS{} (and \FCS{}) does two things.

  \begin{enumerate}
  \item Separates possible rejection of \LCS{} as applied to ability from \LCS{}.
    If you are inclined to reject conclusion drawn, then easier to identify points of failure.
    Move to ability, consequence, and hence assumptions.
  \item Conversely, motivate accepting \LCS{} (\FCS{}) and assumptions without relying on ability.
  \end{enumerate}
\end{note}

\begin{note}
  The structure of this chapter is as follows:
  \begin{enumerate}
  \item Statement of \LCS{}.
  \item Statement of \FCS{}.
  \item Illustration.
  \item Focus on \LCS{}.
    \begin{enumerate}
    \item Argument.
    \item Clarification.
    \end{enumerate}
  \item Focus on \FCS{}.
    \begin{enumerate}
    \item Argument.
    \item Clarification.
    \end{enumerate}
  \item Illustrations.
  \item Summary.
  \end{enumerate}
\end{note}


\paragraph{Literature}


\begin{note}[Literature]
  We will highlight the contrast between \nI{} and similar principles in~\autoref{sec:contr-other-cond}, below.
  In particular, \citeauthor{Wright:2011wn}'s work on transmission failure (\Citeyear{Wright:2003aa,Wright:2011wn}) and \citeauthor{Weisberg:2010to}'s No Feedback condition (\Citeyear{Weisberg:2010to}).

  For now, it may be helpful to highlight that \nI{} does not deny that any agent may claim support for \(\psi\) having value \(v'\).
  Rather,~\ref{nI:going-by-value} (only) denies that the agent may claim support for \(\psi\) in a specific way.
\end{note}

\section{\LCS{0}}
\label{sec:ni-1}

\subsection{Statement of \LCS{}}

\begin{note}
  \begin{restatable}[\LCS{0}  --- \LCS{}]{proposition}{propLCS}
    \label{prop:lcs}
    \label{prem:ni}
    For any agent \(S\), propositions \(\phi\), \(\psi\), and values \(v\), \(v'\).

    \begin{enumerate}[ref=\named{\LCSAcro{}:\arabic*}, series=LCS_counter]
    \item
      \label{nI:background}
      Suppose \(S\)' reasoning such that:
      \begin{enumerate}[label=\alph*., ref=\named{\LCSAcro{}:1\alph*}]
      \item
        \label{nI:background:step}
        Reasoning includes some step \(\delta\) for which \(\phi\) having value \(v\) is the conclusion.
      \item
        \(\phi\) not having value \(v\) is an \ep{} (both prior to and after making step \(\delta\)).
      \end{enumerate}
    \end{enumerate}

    \begin{enumerate}[ref=\named{\LCSAcro{}:\arabic*}, resume*=LCS_counter]
    \item
      \label{nI:inclusion}
      When making step \(\delta\), \(S\) is confident that:
      \begin{enumerate}[label=\alph*., ref=\named{\LCSAcro{}:2\alph*}]
      \item
        \label{nI:inclusion:position}
        If \(S\) is not \misled{} when making step \(\delta\), then it is possible for \(S\), given present context, to reason to \(\psi\) having value \(v'\) (without appealing to \(\phi\) having value \(v\)).
      \item
        \label{nI:inclusion:bound}
        If \(S\) is not \misled{} when making step \(\delta\), then \(S\) would not be \misled{} were \(S\) to reason to \(\psi\) having value \(v'\).
      \end{enumerate}
    \end{enumerate}

    \begin{enumerate}[ref=\named{\LCSAcro{}:\arabic*}, resume*=LCS_counter]
    \item
      \label{nI:going-by-value}
      And:
      \begin{enumerate}[label=\alph*., ref=\named{\LCSAcro{}:3\alph*}]
      \item
      \item
        \(\phi\) having value \(v\) persists through to the conclusion of the instance of reasoning.
        \label{nI:goingbyvalue:psi}
        \(\psi\) not having value \(v'\) is an \ep{} for \(S\) (both prior to and after making step \(\delta\)).
      \end{enumerate}
    \end{enumerate}

    Then:
    \begin{enumerate}[ref=\named{\LCSAcro{}:\arabic*}, resume*=LCS_counter]
    \item
      \label{LCS:requ}
      \(\psi\) having value \(v'\) is a \cprequ{} of step \(\delta\).
    \end{enumerate}
    \vspace{-\baselineskip}
  \end{restatable}
\end{note}

\paragraph{Intuition}

\begin{note}
  Within scope of the paper, the basic idea is that we have identified sufficient conditions for \(\psi\) having value \(v'\) a \cprequ{}.
\end{note}

\subparagraph{Structure}

\begin{note}[What \LCS{} amounts to]
  \LCS{} is an observation about an agent making a step of reasoning leads to a \cprequ{} in the presence of information about other reasoning the agent may perform.

  From a semantic perspective, \LCSBackground{} combine to ensure that \dots

  From a syntactic perspective, the proposition is a conditional whose antecedent consists of three clauses --- \LCSBackground{} ---, and a consequent~\ref{LCS:requ} which states that a certain proposition-value pair is a \cprequ{} of making the step \(\delta\).
\end{note}

\begin{note}[Intuitive idea]
  To bring out the intuitive idea:

  Claiming support.
  So, \ep{}.
  Hence, that \(\phi\) has value \(v\) is not immediate.
  Step \(\delta\), \(\phi\) has value \(v\).
  From some premises, \(\phi\) has value \(v\).
  Then, as conclusion of step is \(\phi\) having value \(v\), it follows that some long as \(\phi\) does have value \(v\), agent is not \misled{}.
  Possible to claim support for \(\psi\) having value \(v'\).
  Hence, \(\phi\) having value \(v'\) is a \requ{}.
  Specifically, a \cprequ{} as without \(\phi\) having value \(v\), nothing in particular follows about \(\psi\) having value \(v'\).

  This is fast.
  When turn to argument proper we will move more slowly.

  Note, however, that the step \(\delta\).
  Here, \(\phi\) having value \(v\).
  Still the case that \(\phi\) having value \(v\) is an \ep{}.
  This kind of step is fine, in general.
  Point is that claiming support requires reasoning about \requ{1} of this kind of step if conclusion of step persists.
  
\end{note}

\begin{note}[Kind of defeater]
  The basic idea behind \nI{} is that given clause~\ref{nI:background}, clause~\ref{nI:inclusion} captures a condition which \emph{undercuts} the agent claiming support as captured by~\ref{nI:going-by-value}.

  Mentioned undercutting {\color{red} \hyperref[first-mention-undercutting-defeater]{above}}

  This doesn't say anything about whether \(\psi\) has value \(v'\).
  Rather, if \eiS{} doesn't hold then the agent doesn't get to claim support {\color{red} due to impossibility of reasoning about a \requ{}}.
\end{note}

\paragraph{Quick notes}

\begin{note}
  Two quick notes.
\end{note}

\begin{note}[First point to stress]
  First, \LCS{} outlines sufficient conditions for a limitation on claiming support to obtain.
  Hence, there may be other limitations on claiming support.
  For example, \ESU{} implies a limitation of claiming support --- an agent may not claim support by appeal to some premises or step of reasoning that they have not used.

  Similarly, there may be other limitations on claiming support with obtain.
\end{note}

\begin{note}[Second point to stress]
  Second~\ref{nI:going-by-value} is a limitation of a \emph{way} of claiming support.
  So, \nI{} does not imply that the agent is not in a position to claim support for \(\psi\), only that one way of claiming support is ruled out when the other clauses of \nI{} hold.
  Hence, \nI{} is compatible with the agent claiming support for \(\psi\) having value \(v'\) --- so long the agent doesn't follow the pattern captured by~\ref{nI:going-by-value}.

  Indeed, I take the primary upshot of \nI{} to be a demand for understanding alternative ways of claiming support when each clause of \nI{} holds and it seems that the agent may claim support for \(\psi\) having value \(v'\).
  And, after arguing for \nI{} our attention will turn to examining how an agent may claim support by \EAS{} when clauses~\ref{nI:background} obtain.

  For now, however, our focus is on arguing for \nI{}.

  It is perhaps also helpful to flag that the following discussion of \nI{} will relate previous details on claiming support (in particular~\autoref{cha:claiming-support}) but will proceed without consideration of ability.
  Our goal is to motivate \nI{} as a general limitation on claiming support, and in turn draw a consequence from such a limitation when applied to ability.
\end{note}

\begin{note}[`Reasoning']
  `Reasoning'.
  No real constraints.
  Present, only interested in what the reasoning is not.
  Past and possible, no reason to restrict.

    And, restriction would be a significant problem as we haven't placed constraints on reasoning that goes in to claiming support.\nolinebreak
  \footnote{
    To be explicit.
    Apply \LCS{} to some case.
    This would only show that if claimed support.
    However, weaken so that \LCS{} no longer applies.

    So, would need to then argue that claiming support by appeal to past reasoning only if past reasoning is instance of claiming support.

    A worry in the background is regress.
    However, even if these worries may be resolved, this would be pointless from a dialectical perspective.
  }

  Still, a natural reading to do is replace reasoning with claimed support.
  Claimed support for \(\phi\), and possible claiming support for \(\psi\).
  Compatible.
  The thing that really matters is that reasoning for \(\phi\) is compatible with epistemic possible of \(\psi\) not having value \(v'\).
\end{note}

\begin{note}
  Does not show that there's no hope for the agent.
  Only that the reasoning fails.

  Two possibilities:
  \begin{itemize}
  \item Different approach to claiming support.
  \item Reasoning that indicates \cprequ{}.
  \end{itemize}

  So, \LCS{} is a limitation, but not a particularly strong limitation.
  However, not too weak either.

  Next:
  Highlighting circumstances such that it is not possible for the agent to reason to \(\psi\) having value \(v'\) which indicates that \(\psi\) having value \(v'\) regardless (from the perspective of the agent) whether \(\psi\) has value \(v'\).
\end{note}


\subsection{A consequence of \LCS{} (\FCS{0})}
\label{sec:fcs}

\begin{note}
  \begin{restatable}[\FCS{0}  --- \FCS{}]{proposition}{propFCS}
    \label{prop:fcs}
    Suppose:

    \begin{enumerate}
    \item
      Clauses of \LCS{} hold.

    \item
      \(S\) requires \(\phi\) having value \(v\) from step \(\delta\) of~\ref{nI:background:step}\nolinebreak
      \footnote{
        \color{red}
        I.e.\ which leads to \(\psi\) having value \(v'\) being a \cprequ{} of step \(\delta\)\dots
      }
      in order for some instance of reasoning (other than mentioned) to indicate that \(\psi\) has value \(v'\).
    \end{enumerate}

    Then:
    \begin{enumerate}[resume]
    \item
      It is not possible for \(S\) to claiming support for \(\phi\) having value \(v\) without witnessing the reasoning mentioned in \ref{nI:inclusion}.
    \end{enumerate}
    \vspace{-\baselineskip}
  \end{restatable}

  Hence, so long as \(\phi\) having \(v\) persists, further reasoning that appeals to \(\phi\) having value \(v'\) as a premise is (also) not an instance of claiming support without witnessing reasoning mentioned in \ref{nI:inclusion}.
\end{note}

\begin{note}
  {
    \color{red}
    This is where part of the problem is.
    Now, key idea here is that there are two ways in which \FCS{} fails to apply.
    Either \LCS{} does not hold.
    Or, this second clause doesn't hold.
  }
\end{note}

\begin{note}
  Undercutting from \LCS{}.
  \FCS{} is highlighting that no way to avoid being undercut given additional constraints on reasoning.
\end{note}

\newpage


\paragraph{Plan}

\begin{note}[Plan]
  To start:
  \begin{enumerate}
  \item Core idea of \nI{}.
  \item Role in overall argument.
  \item Some intuition.
  \item Key points.
  \item Argument
  \item Some connexions to the literature.
  \end{enumerate}
  Then, details.
\end{note}

\begin{note}[Task]
  Two important tasks with respect to arguing for \nI{}:

  \begin{itemize}
  \item \ref{nI:inclusion:position} and~\ref{nI:inclusion:bound}, how these function.
  \item And, why \nI{} holds.
  \end{itemize}

  Then, additional details.
\end{note}

\paragraph{Situating \nI{}}


\begin{note}[Expanding on~\ref{nI:inclusion}]
  Claiming support, so possible that \(\psi\) is not the case.
  Then, claiming support.

  In particular, \ref{nI:background} and~\ref{nI:inclusion} combine to ensure that \(\psi\) would be an \requ{} of the claimed support for \(\phi\).
  Hence:

  \begin{quote}
    \vspace{-\baselineskip}
    \assuCSRReq*
  \end{quote}

  As indicated.
  \(\psi\) having value \(v'\) as a \prequ{} of step getting \(\phi\).
  So, as a \requ{}, from first, need some reasoning to indicate.
  However, as a \prequ{}, antecedent of second is satisfied.
  Hence, reasoning must not appeal to conclusion obtain from step of inference.\nolinebreak
  \footnote{
    Variation on proposition that only looks at lack of reasoning.

    \emph{N} things:
    \begin{itemize}
    \item Less general, requires various clauses.
    \item Stronger, as does not require assumption that agent does not reason.
    \end{itemize}
  }

  Motivated \ref{assu:supp:independence} in \ref{sec:two-assumpt-relat-to-requ}.
  However, it may be help to briefly recap the core idea.

  Recall:
  \begin{quote}
    \vspace{-\baselineskip}
    \ideaEIS*
  \end{quote}
  The key feature that we're interested in which claiming support is that it does not `depend' on appealing to premises for which the agent has not have reasoning to indicate that are the case.

  {\color{red} \dots}
\end{note}


\begin{note}
  Interdependence between claimed support for \(\phi\) having value \(v\) and claiming support for \(\psi\) having value \(v'\) given \ref{nI:inclusion}.
  If \(\phi\) has value \(v\), then claimed support for \(\phi\) is \nmom{}.
  Hence, claimed support for \(\psi\) having value \(v'\) would be \nmom{}.
  So, given background, \(\phi\) having value \(v\) requires \(\psi\) having value \(v'\).

  Problem, as claimed support for \(\psi\) having value \(v'\) --- via claimed support for \(\phi\) having value \(v\) --- `requires' \(\psi\) having value \(v\).

  Though, leaves open claiming support that does not going via the claimed support for \(\phi\) having value \(v\).
\end{note}


\subsection{An \illu{0} with variations}


\begin{note}[A few illustrations]
  Let us now turn to a few illustrations before discussing \nI{} in further detail.

  We'll begin with a somewhat detailed illustration.
  \nI{} identifies a particular way in which an agent may fail to claim support, and the primary goal of the initial demonstration is to highlight why the agent would fail to claim support.
  Hence, the illustration treads a fine line between highlighting a problematic method, but not necessarily a problematic result.
  This is by design.
  And, I will continue to stress that \nI{} concerns a way of claiming support for some proposition, rather than the possibility of claiming support for some proposition.

  Following two illustrations will be variations on the initial.

  Still, it may be helpful to observe how \nI{} relates to an intuitively problematic result.
  Therefore, we will provide an additional, simple, illustration of a failure to claim support.

  The final illustration in the trio will complement the initial par of illustrations by highlighting an instance where~\nI{} does not apply.

  \phantlabel{dogmatism-wrt-nI}
  The reader may note structural similarities between these illustrations and \citeauthor{Kripke:2011wv}'s Dogmatism paradox.
  We will discuss the relation after the illustrations.
\end{note}

\paragraph{First}


\begin{note}[Brief illustration of \nI{}]
  The first illustration considers theories and counterexamples.

  \begin{illustration}
    \label{ill:CE:main}
    Suppose a researcher have constructed a theory of some general phenomenon.

    The theory seems to capture the phenomenon, and the researcher has claimed (inductive) support that the theory is adequate by applying it to various instances of the general phenomenon.
    Even if the theory isn't adequate, the theory has been (seemingly) successful applied to sufficient specific instances of the phenomenon.
    Hence, even if \mom{}.

    However, as the phenomenon is a \emph{general} phenomenon it also makes various predictions about what must happen in all other instances to which the researcher has not (yet, at least) applied the theory to.

    Hence, there is a possible counterexample to the theory associated with each instance the researcher has not (yet) applied the theory to.
    If some particular instance does not conform to the theory, the theory is inadequate.
    Conversely, if the theory is adequate, every particular instance of the phenomenon conforms the theory.
    In other words, if the theory is adequate, then there are no counterexamples to the theory.

    Of course, it may be simple to revise the theory is a counterexample exists, and the fundamental ideas of the theory may remain sound (\cite[Cf.][]{Bonevac:2011tz}).
    And, the theory may have sufficient resources to explain why any apparent counterexample is not a counterexample.
    Yet, it remains the case that the theory would need to be revised in light of a counterexample.

    Now, to summarise, the researcher may claim support for two propositions allow the agent to claim support that there are no counterexamples.

    \begin{enumerate}
    \item The theory is adequate, and
    \item If the theory is adequate, then there are no counterexamples.
    \end{enumerate}

    At issue is whether the researcher may claim support that there are no counterexamples to the theory from the claimed support for the two propositions in the following way:

    \begin{enumerate}
    \item I have claimed support that the theory is adequate.
    \item So, given the claimed support, theory is adequate.
    \item Therefore, as the theory is adequate, given the claimed support, it follows that there are no counterexamples.
    \item Hence, I claim support that there are no counterexamples to the theory.
    \end{enumerate}
    \vspace{-\baselineskip}
  \end{illustration}
\end{note}

\begin{note}[Seems problematic]
  Seems problematic.
  Claimed support that the theory is adequate is qualified by the possibility of counterexamples.
  {
    \color{red}
    Note, agent is, here, only claiming support that there are no counterexamples.
    And, claiming support may be \mom{}.
    So, it does not follow that the agent is ruling out the possibility of counterexamples to the theory.
    Plausible that the agent \emph{may} claim support.
    Problem is the way in which the agent goes about this.
  }

  {
    Even if not convinced about support, this way of claiming support seems problematic.
    Relying on theory being adequate.
    However, if this is the case, then no possible counterexamples.
    Issue is that such counterexamples are possible given the state of your claimed support.
    Hence, claiming support in this way seems to take for granted that there are no counterexamples.
  }

  Problem is that the reasoning only works if there are no counterexamples.
  If there are counterexamples, misled.
  Hence, problem to go from the theory is adequate.
  However, without this step, researcher doesn't get to no counterexamples.

  So, this is \eiS{}.
\end{note}

\begin{note}
  So, relation between theory and counterexample \emph{undercuts} using way of using theory to get no counterexample.

  Now, given that the researcher has claimed support that the theory is adequate, the researcher may \emph{expect} that there are no counterexamples to the theory.
  And, it doesn't follow that the researcher may not claim support.
  Specific way --- reasoning captured by~\ref{nI:going-by-value}
  Plausible that details of the theory provide some way of claiming support.

  Indeed, it seems the researcher is require to take the alternative path --- to show that the proposed counterexample is accounted for by the contents of the theory, regardless of whether the theory is true.

  Fault here is with respect to \eiS{}.
  {
    \color{red}
    Here, conditions of~\ref{nI:inclusion} are satisfied, but we did not explicitly appeal to them.
    Purpose of~\ref{nI:inclusion} is conditions sufficient for this kind of problem to arise.
    So, to do in argument for \nI{} is to develop is why~\ref{nI:inclusion} does something similar.
    Upshot is that \nI{} is general.

    In the third illustration, we'll see why the way of claiming support is okay in some cases.
  }
  Difficult part is to account for why~\ref{nI:inclusion} sets things up and ensures that things don't go too far.
\end{note}

\paragraph{Second}

\begin{note}[Idea main part of \nI{} works]
  As noted above, it is unclear whether or not there may be some way for the researcher to claim that there are no counterexamples to the theory.
  And, if there is some way for the researcher to claim that there are no counterexamples to the theory, one may be inclined to wonder whether there really is a problem with claiming support in the way outlined by~\ref{nI:going-by-value}.

  In other words, one may be wondering whether \eiS{} is a plausible constraint on claiming support.
  We gave a general argument for \eiS{} in~\autoref{cha:claiming-support}.
  However, it may help to see how the issue highlighted relates to an intuitively problematic instance of reasoning, regardless of how support is claimed.

  \begin{illustration}\label{ill:CE:colleague}
    Suppose a colleague has studied the researcher's theory, and they (the colleague) thinks they have found a counterexample.

    The colleague has informed the researcher that they think they have observed a counterexample.

    However, the colleague has not provided the researcher with any further details about the counterexample.

    Now, the conditional of interest may be made more precise:
    \begin{enumerate}
    \item If the theory is adequate, then the colleague has failed to identify a counterexample to the theory.
    \end{enumerate}

    Now, let's replicate the way of claiming support from before.

    \begin{enumerate}
    \item I have claimed support that the theory is adequate.
    \item So, given the claimed support, theory is adequate.
    \item Therefore, as the theory is adequate, given the claimed support, it follows that the colleague has failed to identify a counterexample to the theory.
    \item Hence, I claim support that the colleague has failed to identify a counterexample to the theory.
    \end{enumerate}
    \vspace{-\baselineskip}
  \end{illustration}

  I take this illustration to be intuitively problematic.
  In short, if claim support, then doesn't need to examine counterexample to claim support that it is not a counterexample.

  Possible response is that researcher does claim support, but information colleague impacts claimed support for theory.
  However, this is also puzzling.
  Researcher has no information.
  Hence, if retain confidence, then equally against counterexample.
  And, if does not retain confidence, then down the theory in a way that seems implausible.

  Seems, instead, that claimed support for theory persists, but that this doesn't extend to counterexample.\nolinebreak
  \footnote{
    Inclined to apply this to previous illustration.

    However, there's a difference between two illustrations.
    Here, someone (the colleague) has reason to think there is a counterexample, and this seems a sufficiently important difference to draw any quick conclusions.
    And, as we don't require a resolution to this issue, I won't explore further.
  }

  Perhaps more detail is needed.
  I have some doubts that claiming support is always bad.
  However, clearer that developed in a way such that problem remains.

  Now, seems that the researcher doesn't get to claim support because if counterexample, then theory is bad.
  Hence, requires that counterexample is not true in order to progress.
  But, then, doesn't make the move regardless of whether or not there is a counterexample.

  So, it seems \eiS{} does the work.
\end{note}


\begin{note}
  Undercuts using \(\phi\) for \(\psi\).
  Same problem, failure of \eiS{}.

  For, the agent has already assumed \(\psi\).

  Problem is that the agent doesn't get to claim support for \(\psi\) because fail the \eiS{} thing.
  If \(\psi\) isn't really the case, then reasoning collapses.
  Key thing about our understanding of claimed support is that it holds up even if the agent is \mom{} about the value of the proposition.

  {
    \color{red}
    Note:
    There's possible tension here.
    It seems that if the first illustration is okay, then this (second) illustration should also be okay.
    Maybe.
    But, this is too quick.
    Additional information here.

    Now, still some difficulty, as I think \EAS{} might apply to the first.
    So, shouldn't it apply to this?
    Well, no.
    For, \EAS{} only suggests possibility in some cases.
    Fine to think of this additional information as constraint on appeal via ability.
    For, if the colleague thinks they've found a counterexample, then this suggests a problem with the agent's ability.
  }
\end{note}

\paragraph{Third}

\begin{note}[Variation where \nI{} does not apply]
  \begin{illustration}\label{ill:CE:testimony}
    Suppose the researcher has published a paper containing the details of the theory.

    Our attention now turns to a novice who has read far enough into the paper to understand, at least, the general phenomenon that the theory applies to and that the researcher has claimed inductive support for the theory.
    We'll also assume that the novice does not possess the expertise required to apply the theory.\nolinebreak
    \footnote{
      Though I don't think this assumption is important.
    }

    The novice is thinking about instances of the general phenomenon, and identifies one.

    The conditional of interest is:
    \begin{enumerate}
    \item If theory is adequate, it accounts for this instance of the phenomenon.
    \end{enumerate}

    Of course, the novice also recognises that the theory is inadequate if it  does not account for the particular instance of the phenomenon.
    Still, the novice claims support in the familiar manner.

    \begin{enumerate}
    \item I have claimed support that the theory is adequate (this time by reading a published paper).
    \item So, given the claimed support, the theory is adequate.
    \item Therefore, as the theory is adequate, given the claimed support, it follows that the theory accounts for this instance of the phenomenon.
    \item Hence, I claim support that the theory accounts for this instance of the phenomenon.
    \end{enumerate}
    \vspace{-\baselineskip}
  \end{illustration}

  In contrast to the previous illustrations, it seems the novice may claim support in such a way.

  Possibility of being either \mom{} remains.
  Still, not in position to reason through theory and phenomena.
  Hence, claiming support from something like status of peer review --- or testimony.
  And, not accounting would not show peer review is bad.
\end{note}

\paragraph{Summary of illustration and variations}

\begin{note}
  These three illustrations.
  First, kind of scenario that's the main interest.
  Where claiming support in a certain way seems problematic, even if it not clear that the agent may claim support in some other way.

  To stress the problem, considered a cleaner case, where it seems agent may not claim support, and argued that same problem is a plausible account of why.

  Third illustration, way of claiming support is okay.
  As all instances of \nI{}, and hence the previous two illustrations, focus on particular way of claiming support illustrated that it's okay.
\end{note}

\begin{note}[Intuition]
  In short, \nI{} captures a limitation: An agent is not in a position to claim support for some proposition \(\psi\) when circumstances are such that the claimed support requires (from agent's point of view) that the agent is already in position to claim support for \(\psi\).

  No claiming support for\(\psi\) if failure to establish support for \(\psi\) independently of the value of \(\phi\) would reveal problem with the support claim for \(\phi\).

  Hence, \nI{} focuses on when an agent may claim support for some proposition by noting that (from the agent's perspective) that the value of the proposition is determined by further propositions the agent has claimed support for.

  Some other way of claiming support for \(\psi\).
  However, not merely an alternative path, but an alternative path that must be possible given claimed support for \(\phi\).

  Issue is that given \ref{nI:background} and~\ref{nI:inclusion}, agent expect that they have the resources, and hence expects that \(\psi\) is the case.

  So, that \(\phi\) has value \(v\).
  In doing so, resources to claim support for \(\psi\) has value \(v'\).
  Hence, that \(\psi\) has value \(v'\).
  So, \(\psi\) having value \(v'\) is a requirement on claimed support for \(\phi\) being any good.
  However, no support claimed for \(\psi\) having value \(v'\).

  In cases of reasoning with a conditional, such as the illustrations given, that value of \(\phi\) constrains value of \(\psi\) is in general helpful information, but in these specific cases it does not help the agent claim support for \(\psi\) having value \(v'\) because if \(\psi\) isn't already so constrained, then no appeal to \(\phi\) having value \(v\).

  Similar to other principles, failure because establishing something that needs to be the case in order to be in a position to establish.
\end{note}

\section{\LCS{}}
\label{sec:details-ni}
\label{sec:re-do-ni}

\begin{note}
  Statement of \LCS{}.
  So, task is to show that \(\psi\) is a \cprequ{}.
\end{note}

\subsection{Argument for \LCS{}}
\label{sec:argument-for-lcs}

\paragraph{Argument overview}

\begin{note}
  Key assumption is:

  \assuCSRReq*
\end{note}

\begin{note}
  \begin{enumerate}
  \item \(\psi\) having value \(v'\) is an \cprequ{} of the reasoning for \(\phi\) having value \(v\).
  \item In other words, going to \(\phi\) having value \(v\) includes \(\psi\) having value \(v'\) as a result of previous.
  \end{enumerate}
  Claiming support for \(\psi\) so recognise \(\psi\) might not be the case.
  However, claimed support is such that it requires \(\psi\) is the case.
  Hence, no account of \(\psi\) given this possibility.
\end{note}

\paragraph{Argument outline}

\begin{note}
  Two \requ{2}
\end{note}

\subparagraph{First \requ{}}

\begin{note}
  So, relevant instance of~\autoref{assu:supp:independence}:
  \begin{enumerate}
  \item (\(\text{CS}\phi \vdash \phi\)) \(\rightarrow \psi\)
  \item \(\lnot\psi \leadsto\) \(\lnot\)(\(\text{CS}\phi \vdash \phi\))
  \end{enumerate}

  Point here is that we do get \(\psi\), and it is also the case that \(\psi\) is not a `mere contingency'.
\end{note}

\begin{note}
  Two immediate consequences from~\ref{nI:background}.
  \begin{enumerate}
  \item \(S\) is (given present context) in position to claim support that \(\psi\) has value \(v'\) (without appeal to \(\phi\) having value \(v\))
  \item \(S\) would be \nmom{} were they to claim support.
  \end{enumerate}
\end{note}

\subparagraph{Second \requ{}}

\begin{note}
  \(\psi\) having value \(v'\) is a \requ{}.
  This is a more-or-less immediate consequence of the previous.

  Already have that position for \(\psi\) is a \requ{}.

  Here, \autoref{nI:inclusion:bound} again.
  If \(\psi\) does not have value \(v'\), then would be misled regarding \(\psi\)\nolinebreak
  \footnote{I.e.\ would then would be misled regarding \(\phi\)}

  {
    \color{red} This second \requ{} is important because claiming support for \(\psi\), and I don't want to explicitly mark the assumption that not sure about reasoning to \(\psi\).
  }
\end{note}

\paragraph{Summary}

\begin{note}
  In broad outline.
  \begin{itemize}
  \item Claimed support for \(\phi\).
  \item No reasoning about \(\psi\).
    \begin{itemize}
    \item No claimed support for \(\psi\).
    \item No account of why \(\phi\) regardless of \(\psi\). (I.e.\ weaker than claimed support, if think this is possible).
    \end{itemize}
  \item From inclusion, this means that claiming support for \(\psi\) is an \requ{} of claimed support for \(\phi\).
  \item Consequence of \requ{} is that \(\psi\) is a \requ{} of the move to \(\phi\).
    In sense that:
    \begin{itemize}
    \item Move to \(\phi\) is not compatible with possibility that \(\psi\) is not the case.
    \item Also, as result of this, move to \(\phi\) requires \(\psi\) to be the case.
    \end{itemize}
  \item So, reasoning to \(\psi\) is not compatible with the possibility of \(\psi\) not holding.
  \item And, as way of claiming support needs this move, failure.
  \end{itemize}
\end{note}

\subsection{Details of \LCS{}}
\label{sec:details-of-lcs}

\paragraph{Clarification}

\begin{note}
  Important to note here is that it's the move to \(\phi\) that brings in these \requ{}.

  This move is not compatible with \(\psi\) not having value \(v'\).
\end{note}

\begin{note}
  No problem with reasoning for \(\phi\) having value \(v\).
  \(\psi\) not having value \(v'\) is only an \ep{}.
  And, while it is the case that \ref{nI:inclusion} establishes this link between the reasoning for \(\psi\) having value \(v\) and reasoning for \(\psi\) having value \(v'\), the agent has no reason to think (given only the conditional, at least) that there is any problem with their reasoning for \(\phi\) having value \(v\).
\end{note}

\begin{note}
  Do not need is that the agent would be confident that their reasoning for \(\phi\) is defective in some way.
  Indeed, motivating idea of claiming support is that indicates `regardless'.
\end{note}

\begin{note}
  {
    \color{red}
    For, at issue is the impossibility of getting to \(\psi\) without going to \(\phi\).
    The point here is that the case is set up so that whatever the reasoning is for \(\phi\), this doesn't get the agent to \(\psi\) \emph{unless} the agent moves from \(R\phi\) to \(\phi\).
    And, given that this move is incompatible with \(\lnot\psi\) being the case, it follows that the agent's reasoning is not compatible with \(\lnot\psi\) being the case.
    Hence, it is not an instance of claiming support.

    This leaves \(R\phi\) unharmed.
    \(R\phi\) is fine, then problem is the move to \(\phi\).
    Indeed, this is precisely why I think interpolation is a possibility.

    I'm getting something more out of \(R\phi\), which is fine, there's some kind of amplification going on.
    However, as a result of this, I require that \(\psi\) is the case.
    So, it is not possible that I get an account of \(\psi\) even granting the possibility that \(\psi\) is not the case.

    So, claiming support, all I need is that there is some reasoning about why the reasoning indicates \(\psi\) when granting the possibility of \(\lnot\psi\).
    And, there is no way to get this given that the agent needs to go to \(\phi\).

    This is a significant restriction on the scope of the limitation.
    It needs to be the case that it is not possible for the agent to get to \(\psi\) without getting to \(\phi\).
    And, this is mostly limited to instances where there's some sort of `factive' inference.
  }
\end{note}

\paragraph{Why not appeal to earlier result?}

\begin{note}
  Before, result about \requ{1} from contraposition.
  Why doesn't think work?
  Well, the key thing is that there's no clear regarding \crequ{1}.
  And, only require reasoning about \prequ{1}.
  So, that's why this isn't too interesting.
\end{note}

\paragraph{The clauses}
\label{sec:nI:arg:clauses}

\begin{note}
  Two clauses.

  First,~\ref{nI:inclusion}.
  Second,~\ref{nI:going-by-value}.

  Add further information regarding these.
  Not required for argument.
\end{note}

\subsubsection{~\ref{nI:inclusion}}

\begin{note}[\ref{nI:inclusion}]
  Restated:
  \begin{quote}
    \begin{enumerate}
    \item[\ref{nI:inclusion}]
      \nIClauseInclusion{}
      \begin{enumerate}
      \item[\ref{nI:inclusion:position}] \nIClauseInclusionPosition{}
      \item[\ref{nI:inclusion:bound}] \nIClauseInclusioBound{}
      \end{enumerate}
    \end{enumerate}
  \end{quote}
\end{note}

\begin{note}
  \begin{itemize}
  \item \ref{nI:inclusion:position}.
    \begin{itemize}
    \item \(\psi\) having value \(v'\) is the case, granting claimed support \(\phi\) having value \(v\) is the case goes through.
    \item Not the result of appealing to \(\phi\) having value \(v\).
      \begin{itemize}
      \item So, agent considers it possible to claim support for \(\psi\) having value \(v'\) without \(\phi\).
        Hence, way of removing \requ{} of \(\psi\) from claimed support for \(\phi\).
      \end{itemize}
    \end{itemize}
  \end{itemize}
\end{note}

\begin{note}
  {
    \color{red}
    The point of these parenthesis is that the relevant way of claiming support isn't the one that's going to give rise to the issue.
    Otherwise, it's somewhat trivial.
    Still, I feel there is a much better way of expressing the parenthetical parts of these clauses.
    I mean, the point would be that any instance is always going to be self defeating.
    But, the problem isn't \(\phi\) having value \(v\), rather it's there being some other instance that `forces' \(\psi\).
  }
  {
    \color{blue}
    Well, the interdependence here is really about the way of claiming support for \(\phi\) requiring that agent also claims support for \(\psi\).
    In this sense, \(\psi\) is a presupposition for the method of claim support.

    If agent requires implication, then this should break down, because it won't be the case that the agent needs \(\psi\) to hold up in order for the way in which the agent claims support for \(\phi\) to be okay.

    In this sense, the difficulty is whether these two conditions really capture this idea.
    It's the way of claiming support that's related, rather than \(\phi\) and \(\psi\).

    Could I make this explicit?
    Position to claim for \(\psi\) in same way as \(\phi\).
    And, that's successful only if claiming for \(\psi\) is also good.
  }
\end{note}

\begin{note}[Two clauses]
  The two sub-clauses of~\ref{nI:inclusion} are separated by~\ref{nI:inclusion:position} being a conditional with an antecedent and consequent that describe present circumstances while \ref{nI:inclusion:bound} is a conditional with an antecedent that describe present circumstances and a subjunctive consequent.

  {
    \color{red}
    The key this here is that if claimed support for \(\phi\) is not misled, then \(\psi\) has value \(v'\) --- that's the importance of being in position.
  }

  In turn, each sub-clause is stated from the perspective of possible defeaters, but as a result both sub-clauses carry certain implications if the agent is confident that they have successful claimed support for \(\phi\) having value \(v\).

  First, as stated, \ref{nI:inclusion:position} ensures that the agent considers their claimed support \(\phi\) having value \(v\) is mistaken or misled if they are not in a position to claim support that \(\psi\) has value \(v'\).
  In turn, if the agent is confident that they have claimed support \(\phi\) having value \(v\) then the agent should be equally confident that they are in a position to claim support that \(\psi\) has value \(v'\)

  Second,~\ref{nI:inclusion:bound} ensures that the agent is confident that if they were to claim support for \(\psi\) having value \(v'\) then such claimed support for would be \mom{} if their claimed support for \(\phi\) having value \(v\) is \mom{}.\nolinebreak
  \footnote{
    Note, however, that the subjective element of this contraposed form is restricted to the antecedent of the conditional.
  }
  In turn, if the agent is confident that they have claimed support \(\phi\) having value \(v\) then the agent should be equally confident that claimed support for \(\psi\) having value \(v'\) would not be \mom{} if they were to do so.

  The two sub-clauses are closely related, but are distinct.
  It is possible for either to hold without the other.

  \begin{illustration}
    Suppose taught addition.
    Has been told that multiplication reduces to addition, but has not been informed of the details.
  \end{illustration}
  \ref{nI:inclusion:bound} holds but~\ref{nI:inclusion:position} does not.

  \begin{illustration}
    Suppose calculator from scratch.
    Good with arithmetic, but less good with programming.
  \end{illustration}
  \ref{nI:inclusion:position} holds, but~\ref{nI:inclusion:bound} does not.
  For, a whole bunch of additional stuff.
  However, revise scenario so that~\ref{nI:inclusion:bound} does hold.

  So, distinct.
  \ref{nI:inclusion:position} in a position and~\ref{nI:inclusion:bound}, binds.
\end{note}

{
  \color{red}
  Seen examples above.
}

\begin{note}
  `Confidence'

  Then, each sub-clause separately.

  Finally, link to literature.
\end{note}

\begin{note}[`Confidence']
  Our use of the term `confidence' does not require the agent to have claimed support for the conditional content of \ref{nI:inclusion}.
  Nor does out use of `confidence' imply that the claimed support for \(\phi\) \emph{is} mistaken or misled given the identified conditions.
  We are interested only in what makes sense from the agent's perspective.
  Nearby reformulations of \ref{nI:inclusion} may also be true, but confidence is sufficient to recognise a problem.
  To illustrate: If I am confident that the water is poisoned, then regardless of whether I claimed support for the water being poisoned, I will not drink it.
\end{note}

\paragraph{\ref{nI:inclusion:position}}

\begin{note}[Points that will be covered]
  We will focus on two parts of \ref{nI:inclusion:position}.

  First, on what it is for an agent to be in a position to claim support for some proposition (given present context), as this is an unfortunate source of imprecision.

  Second, we'll clarify the restriction that the agent does not appeal to \(\phi\) having value \(v\) --- though this will be further explored when discussing~\ref{nI:inclusion:bound}.
\end{note}

\begin{note}
  First, an admission.
  Key role here is that get a problem with the claimed support for \(\phi\) having value \(v\).
  Possible that some other condition could do the work.
  However, this is sufficient for the purposes of this paper.
\end{note}

\begin{note}[`(given present context)']
  Imprecise.
  No clear account of what it is to `be in a position' nor what `present context' amounts to.

  Unfortunately, I have no simple characterisation for either.
  Best I have to offer is that the agent is not prevented from claiming support by lack of resources.
  However, it seems to me that isn't really substantially different.
  Resource is broad enough to mean whatever is required for the agent to claim support.
  And, bound to context.

  Instead, walk through considerations with respect to some scenarios.
  Consider issues.
  Argue that don't need to do better than imprecision.
\end{note}

\begin{note}
  Pair of easy cases.
  Clarify to some extent in a position.

  Then, some harder cases.
\end{note}

\begin{note}[Flavours]
  Consider variations on a case where information relating \(\phi\) and \(\psi\) comes from a third party.

  You'll enjoy this flavour of ice cream.

  Claimed support from testimony, roughly.

  Seems fine when discussing things on the train.

  More difficult when tasters are available.
  Here, \ref{nI:inclusion} holds.
\end{note}

\begin{note}
  Difference is that in the first, no way of checking.
  In the latter, there's a way of checking.

  Only issue here is `testimony'.
  With respect to testimony, the response is that in cases of testimony, one can be thought of as appealing to a general truthful property, rather than always truthful.
  So, it's not obvious that testimony is always going to give rise to an instance of \ref{nI:inclusion}.
\end{note}

\begin{note}
  Receiving a letter.
  Unmarked, okay, it's for me.
  Marked, check the address.
\end{note}

\begin{note}[More difficult]
  \dots
\end{note}

\begin{note}[Illustration, testimony]
  To illustrate, consider expert testimony to a layperson.
  Suppose you, the expert, have testified to me, the layperson, that there are exactly five intermediate logics that have the interpolation property.\nolinebreak
  \footnote{Cf.\ \textcite{Maksimova:1977un}}
  From this it follows that there is an intermediate logics that has the interpolation property.
  However, I am quite confident that I would not be in a position to claim support for the latter proposition without your testimony.
  Given that I do not have the expertise involved, any failure by me to claim support that there is a intermediate logic with the interpolation property is uninformative.
  Likewise, given that I am a layperson I'm not in a position to rule out that there aren't intermediate logics with the interpolation property, and therefore I may consider this a potential defeater to your testimony.\nolinebreak
  \footnote{
    Additional example: reports of internal states.

    I have a virus scanner.
    Run this on your pc.
    Also, a test pc.
    Test PC contains a know virus, so if the virus scanner is good, then it will identify infection.
    However, no relation between your PC and my test PC.
    All that would be established is that the scanner is not good for claiming support.
    }

  Still, given \ref{nI:background}, agent may hold that \(\psi\) to have value \(v'\), and may claim support.
  And, may expect to have the resources to claim support for \(\psi\) without appealing to \(\phi\) having value \(v\).
  To illustrate, suppose you and I are both experts.
  You claim to have developed a sound and complete proof system for an logic and presented me with a paper containing the system and a proof.
  Given that I have the paper and the expertise, I am confident that I would be mistaken or misled by your testimony if I am not in a position to claim support that the system is sound and complete by working through the paper.\nolinebreak
  \footnote{
    Here, complexity of understanding of having resources shows.
    For, it may be that the reader learns something new, a lemma etc.\ which could be considered a new resource.
    Likewise, one may think that it's fine to continue to follow testimony given a problematic proof as one is confident that the prover has the resources to revise the proof.
    If so, not clear whether conditional holds, and will depend having resources.
    If proof synthesises resources, then may still hold.
    If proof introduces new information, then conditional does not hold.

    No clear answer for these cases.
    Intend to be compatible with your understanding of resources.
    Will only take a stance on this when applying.
  }
\end{note}

\begin{note}
  Even more difficult
\end{note}

\begin{note}
  Coworker.
  Rely on colleague, as the agent doesn't have access to the file.
  But, access is granted quickly after hearing from the colleague.
\end{note}

\begin{note}
  These cases are harder within the broader context of \nI{}.
  Deny \RBV{}.
  Issue is that in both cases, result seems excessive.

  Well, first thing to do is to check that the agent really is claiming support.
  Fine, it seems, for the agent to stop at claiming support for the other agent, and not going any further.
  See no reason to hold that the agent must claim support.

  Still, this isn't quite satisfactory.
  Doesn't seem that bad, and the above suggestion requires a careful understanding of when an agent is required to claim support.
  So, what if the agent does claim support?

  As noted above, views on testimony can sort this out.
  First, by going for `weak' testimony.
  Second, by breaking \ref{nI:inclusion:bound} is the testimony turns out to be a mistake.
  Look, it's not obvious why it would make sense for the agent to claim support, but the point is that \nI{} wouldn't hold.

  Alternatively, Simple restriction is for first time claiming of support.
  Difficulty is a variation of the expert case.
  It isn't obvious that one gets to claim support for the stuff learnt as a layperson when one develops expertise.
  For example, translation between languages.
  Claim support for simple translation.
  When fluent, seems claimed support is distinct, based on broader understanding of the language.
  Not relying on simplifications in learner's dictionary.

  However, other ways of claiming support may also work.
  Arguing for one such way.
  Besides this, \RBV{} is quite strong.
  And, who knows about other types of reasoning.
  In particular, ways in which reasoning \adA{} might work.
\end{note}

\begin{note}[Uninspiring]
  These responses aren't particularly inspiring.

  However, let's look at this from a different angle.
  What's going to follow from insensitivity to context?
  End up with claiming support that does not depend on whether or not the agent is in a position to deal with defeaters.

  Well, the first option is that these never matter.

  Some kind of built in support.
  This comes up with ~\citeauthor{Pryor:2012tq}'s dogmatism (\cite{Pryor:2000tl,Pryor:2012tq}) and various ideas about entitlement (Wright, Burge, etc.)
  For example, Pryor's dogmatism for perception, just having the experience is good enough.
  Question about these kinds of defeaters.
  Reads to me that these kinds of things mean that \ref{nI:inclusion} will never hold.

  Question is whether this extends to all cases, so that \nI{} is trivial, but before pressing this seems too strong.
  Problems in various cases.
  The red room, but in the corner is a switch, flipped to off, but says it's broken.

  Context makes a difference.
  So, to this extent, looks as though there's going to be difficult cases.
\end{note}

\begin{note}[Following doesn't depend on difficult cases]
  Of course, this isn't a general defence of the clauses.
  Rather, that such difficulties can't be avoided.
  Upshot here is that we aren't really interested in such difficult cases.
\end{note}

\begin{note}[`(without appealing to \(\phi\) having value \(v\))']
  The parenthetical clause `(without appealing to \(\phi\) having value \(v\))' ensures that \ref{nI:inclusion} may only be true when the agent is confident that they are in a position to claim support for \(\psi\) having value \(v'\) independent of whether \(\phi\) has value \(v\).

  In this respect, \ref{nI:inclusion} requires an independent check on the claimed support for \(\phi\) having value \(v\).\nolinebreak
  \footnote{
    Note also that without the parenthetical clause, \nI{} would deny the possibility of any instance of the reasoning described in \ref{nI:going-by-value}.
  }

  Indeed, not being in a position to claim support for \(\psi\) having value \(v'\) (without appealing to \(\phi\) having value \(v\)) as a potential defeater to claimed support for \(\phi\) having value \(v\) is distinct from the potential defeater of \(\psi\) not having value \(v'\).
  For, an agent may consider \(\psi\) not having value \(v'\) is a potential defeater given \(\phi \rightarrow \psi\) while being confident that they could not be in a position to claim support for \(\psi\) having value \(v'\) without claimed support for \(\phi\) having value \(v\).

  \begin{itemize}
  \item Illustration
  \item No restriction on why conditional is true.
    \begin{itemize}
    \item Distinction between two ways in which it might be true.
    \end{itemize}
  \end{itemize}
\end{note}

\begin{note}[Inclusion and Association]
  \color{red}
  The illustrations provided offer some intuition, and it seems these will have to do.
  For example, one may consider `in a position' to mean that the agent does not require any novel resources to claim support.
  However, an agent may need to synthesise more fundamental concepts when following a proof, and it is unclear whether the synthesis is `novel information'.
  Similarly, it is difficult to say what the present context is when an agent may phone a friend as a source of testimony.
  In some cases, corroborating testimony may be sufficient to claim support (another plausible instance of `novel information'), while in other cases at issue may be the agent's own understanding (e.g.\ with respect to cases of Inclusion).
  In defence of this latent ambiguity, the specifics will not matter when arguing for the truth of \nI{}.
  And, I suspect the cases to which we apply \nI{} will be sufficiently clear cut.
\end{note}

\paragraph{\ref{nI:inclusion:bound}}

\begin{note}[Inclusion and Association]
  For \ref{nI:inclusion:bound} we will consider two general ways in which \ref{nI:inclusion:bound} may be true, and provide examples for both.

  The task, then, is to account for why claimed support for some proposition being \nmom{} may imply that claiming support for some other proposition would be \nmom{}.

  We term the ways \incl{} and \asso{}, respectively.
\end{note}


\begin{note}[Inclusion]
  \incl{} is when the same (primary) resources used to claim support for some proposition may be (re)applied to establish a distinct proposition.

  To see why \incl{} leads to instances of~\ref{nI:inclusion:bound} suppose:
  The agent is confident that support claimed for the initial proposition is \nmom{}.
  And, the agent is confident that \incl{} holds with respect to the initial proposition and some other proposition.
  In turn, the agent will use the same resources to claim support for the distinct proposition.
  Therefore, if the agent is confident that the claimed support is \nmom{} for the former proposition then the claim supported must be \nmom{} for the latter proposition, else the agent should not be confident that their claimed support for the former proposition is \nmom{}.

  \begin{illustration}
    Consider claimed support that \(6^{2} \times 6^{3} = 6^{5}\).
    Support has been claimed by understanding basic properties of exponents.
    Hence, an agent may be confident that they are in a position to claim support that \(3^{15} \times 3^{12} = 3^{27}\).
  \end{illustration}
  Indeed, working through problem exercises in a textbook is way of ensuring that one has understood such principles.
  Not that textbooks typically ask for the working out.\nolinebreak
  \footnote{
    Though sometimes.
    For a highly specific example, consider constructing canonical models to prove completeness for various normal modal logics.

    The exercises in textbooks such as~\citetitle{Blackburn:2002aa} require the reader to consider a specific system, so there's no surprise that, e.g., \textbf{K1.1} is sound and complete with respect to the class of frames with a relation function that mirrors a partial function.
    Rather, the task of the exercise is to ensure the reader understands how to reason with canonical models, and if the reader has claimed support with respect to \textbf{K1.1} which is \nmom{} then they should be confident that their claimed support will be \nmom{} when they tackle \textbf{K4.3}.

    See \textcite[210]{Blackburn:2002aa} for the respective exercises.
  }

  {
    \color{red}
    Note here that this is the sort of thing that seems most likely in counterexample type cases.
  }
\end{note}

\begin{note}[\asso{}]
  \asso{} is when claiming support for some proposition ensures the agent is in a position to appeal to some distinct collection of resources for some other proposition.

  \begin{illustration}
    Example here is something like storing a guide in a document.
    Here, the agent has created the document, \(\phi\) is just that the document has all of the info.
    So, if document is good, then contents for \(\psi\).
    This is different, as \(\phi\) is a now a check on the stuff in the document working out.
  \end{illustration}

  Also, ice cream example from above.
\end{note}

\begin{note}[Ways in which \ref{nI:inclusion} may fail to hold]
  Finally, the illustrations given have focused on instances for which \ref{nI:inclusion} holds.
  It is important that there are instances where \ref{nI:inclusion} holds, but it is equally important not to suggest that these are in abundance.
\end{note}

\begin{note}[Failures for~\ref{nI:inclusion:position}]
  There are various ways in which \ref{nI:inclusion} may fail to hold.
  For example, if \(\phi\) is sufficiently general or probabilistic.
  If so, not having the resources to claim support \(\psi\) may not establish much.
\end{note}

\paragraph{\ref{nI:inclusion:position} and~\ref{nI:inclusion:bound} combined}

\begin{note}
  {\color{red}
    Main point is that this is a variant of a \requ{}, in effect.
  }
\end{note}

\begin{note}[Failure for zebra case]
  Here, not obvious that holds for zebra case, as it's not clear there is an alternative.
\end{note}

\begin{note}[Literature]
  {
    \color{red}
    Place this here as it helps clarify why~\ref{nI:inclusion:position} and~\ref{nI:inclusion:bound} seem to work well together.
  }

  Still, relative to a certain way of claiming support.
  It's not the case that this idea holds in general.

  The difference is that we're not dealing with circularity nor justified beliefs.
  The issue here is not that the agent needs a justified belief that \(\psi\) is the case.
  Rather, it's the claimed support is such that \requ{} is a problem.
  Intuitively, the task of the agent is to avoid \requ{}.
  But doing so does not amount to forming a justified belief.
  Instead, it not mattering whether \(\psi\) has value \(v'\).

  Applied to illustrations.
  Don't need justified belief against each counterexample for theory.

  Still, similar with respect to strength of connexion.
\end{note}

\begin{note}[Different example.]
  \color{red}
  Recall \autoref{illu:CS:tfc} --- truth functional completeness.
  Instead, don't go by value.
  Rather, go by interdefinability.
  Indeed, interdefinability plays an important role in getting this condition.
\end{note}

\subsubsection{~\ref{nI:going-by-value}}

\begin{note}
  \ref{nI:going-by-value} serves two purposes.

  First, claiming support for \(\psi\) having value \(v'\).
  Second, claimed support for \(\phi\) having value \(v\) as part of such reasoning, from claimed support.
  And, that this is something that it is not possible for the agent to avoid.
  {
    \color{red} Because here it's not the case that \(\phi\) is an non-important part.
  }

  Some interest.
  Saw about with conditionals and contraposition.
  However, also saw with conditionals without contraposition.

  So, it is possible that the details of the reasoning introduce \(\psi\) having value \(v\) as an \requ{}.
  Still, this is not necessarily the case.

  Hence, that agent is claiming support and claimed support for \(\phi\) having value \(v\) is involved so not provide anything immediate.

  Hence, the task of~\ref{nI:inclusion} is to point to conflict with assumptions \emph{given} the details of this reasoning.
\end{note}

\begin{note}
  `Factive reasoning'
\end{note}

\paragraph{Other things}

\begin{note}
  \color{red} Compare to contraposition proposition.
\end{note}


\hozline

\newpage


\subparagraph{Clarifying important parts of \nI{}}

\begin{note}
    However, same motivating intuition.
  With \autoref{prop:CS-only-if-reason-recognised-defeaters}, no reasoning about \requ{} at time of reasoning, then no reasoning regarding possible defeater.

  With \nI{}.
  Claiming support for \(\psi\) having value \(v'\) is \requ{} of claimed support for \(\phi\) having value \(v\).
  And, as a consequence of claimed support for \(\psi\) having value \(v'\) being a \requ{} it is not possible for the agent to entertain the possibility that \(\psi\) does not have value \(v'\) while appealing to \(\phi\) having value \(v\).
  Yet, assumption that the agent only gets to \(\psi\) having value \(v'\) via \(\phi\) having value \(v\).
  Therefore, conflict with \autoref{idea:eiS} --- Agent needs to have \indicateN{} of why even if \mom{}.
  The problem is that not possible to conclude that \(\psi\) has value \(v'\) from a line of reasoning that entertains the possibility that \(\psi\) does not have value \(v'\).
\end{note}

\begin{note}
  First, \(\psi\) having value \(v'\) is \requ{} of moving from claimed support for \(\phi\) having value \(v\) to \(\phi\) having value \(v\).

  Second, consequence of claimed support for \(\psi\) having value \(v'\) being a \requ{} it is not possible for the agent to entertain the possibility that \(\psi\) does not have value \(v'\) while appealing to \(\phi\) having value \(v\).

  Third, why this consequence prevents claiming support.
\end{note}

\subsubsection{Summary}

\section{\FCS{}}
\label{sec:fcs-1}

\subsection{Argument for \FCS{}}
\label{sec:argument-fcs}

\paragraph{Summary}

\begin{note}[To summarise]
  Look, the problem is that it is impossible for the agent to hold that \(\phi\) has value \(v\) without also holding that \(\psi\) has value \(v'\).

  \emph{Therefore}, when the agent entertains the possibility that \(\psi\) does not have value \(v'\) (as they must do in order to claim support), this includes \(\phi\) not having value \(v\).

  Yet, by assumption, the only way for the agent to conclude some line of reasoning with \(\psi\) having value \(v'\) is by appeal to \(\phi\) having value \(v\).

  Therefore, when entertaining this possibility, the agent \emph{must} fail to reason to \(\psi\) having value \(v'\).
  (Which is not to say that the agent ends up reasoning to some contrary evaluation --- only that \(\psi\) having value \(v'\) is not a possible conclusion.)

  The reason that we're interested in the relationship between the ways of claiming support is because it leads to this consequence.
  It is true that in the way of reasoning outlined claiming support for \(\psi\) having value \(v'\) ends up being a \requ{}.
  However, this is not the main focus.
  Rather, it is what follows from this.
  I.e.\ that \(\psi\) having value \(v'\) is a \requ{}.
\end{note}

\subsection{Observations}
\label{sec:observations-fcs}

\begin{note}[Unused component of argument]
  There is an additional component to the clauses of \nI{} that may be added to strengthen the argument.
  Not only is \(\psi\) not having value \(v'\) a possible defeater, but given \ref{nI:inclusion} it is a possible defeater that the agent is confident that they can claim support about.

  ``One perspective on \ref{nI:inclusion} is that it ensures that claiming support for \(\psi\) having value \(v'\) is a possible and `pressing' defeater for the claimed support that \(\phi\) has value \(v\).

  So, enough to prevent the agent from moving from whatever premises they have to \(\phi\) having value \(v\)\dots''

  {
    \color{red}
     In turn, links to \autoref{assu:supp:independence} to motivate some account of reasoning against recognised defeater.

  I think that there is a successful argument that follows this pattern.
  However, this is not the argument we present.
  }

  Without saying more about reasoning regarding recognised possible defeaters, it is hard to say how important this is.
  Add in assumption that if possible to claim support then no dismissing by `weaker' reasoning.
  Not necessary, as clause of \nI{} is that \(\psi\) only from \(\phi\).
  Instead, suggestions along these lines would suggest that \nI{} stronger constraint would allow final clause to be weaker.
  Upshot for application of \nI{} is that need to identify this link between \(\phi\) and \(\psi\).
\end{note}

\begin{note}[Generalising \nI{}]
  Core question about whether there's a generalisation of \nI{}.

  In particular, one might think that there's a requirement for the agent to witness the relevant reasoning in certain cases.
  I mean, that's the core of \nI{}.
  In some cases, the agent doesn't have the option of skipping this by appealing to claimed support for something.

  However, the difficulty is in finding an expansion which doesn't also prevent the agent from claiming support when they do witness.
  In all cases, it's clear that one may get things wrong.

  The way that \adB{} avoids this is by avoiding strong claims to the specific ability.
  Indeed, principle is the same as witnessing.
  So, there's no plausible way to expand \nI{} to cover the proposals without also denying the relevant instance of witnessing.

  Rather, objections here comes from supporting \ESU{}.
  That this isn't a way to claim support.
\end{note}

\begin{note}[Summary, and testimony]
  Final case to summarise:
  Knowledge via testimony.
  This condition doesn't necessarily apply, as agent may not be in position to claim support for what follows from knowledge claim.

  Two reasons for this.
  First, agent may not be in a position to check.
  E.g.\ missing premise, or layperson, e.g.\ missing steps of reasoning.

  Second, agent may not need to \RBV{0}.
  For, if you've testified, then it follows from your statement.
  I don't need to appeal to me having heard from you.
  Instead, given the additional information that I have, you've already made the claim.
  Even if \(\phi\) doesn't have value, this is still an okay reinterpretation of the testimony you have provided.
  Here, to get the intuition, it's really not clear that I need to endorse that I do have the option to check.
  {
    \color{red}
    This point only really makes sense after the argument has been given.
  }
\end{note}


\section{Illustrations}
\label{sec:illustrations-ni}

\begin{note}[Abstract, so examples]
  Turn to illustrations, and then to how \nI{} applies to \gsi{}.
\end{note}

\begin{note}
  Here, only interest is in support.
  Hence, recognised by the agent that they may be mistaken or misled.
  From this perspective, the issue is not ruling out potential defeaters.
  Similar to knowledge, etc.\ but no requirement that there are no defeaters.
\end{note}

\begin{note}
  May think that this restricts any application of \RBV{} to claimed support for \(\phi\) without value independent.
  This isn't quite right.
  \eiS{} keeps focus on \(\psi\).
  Only committed to \(\psi\) being a problem.
  Potential issue is no worse than any other instance of claim to support --- possibility of being mistaken or misled.
  If \(\psi\) ends up being used, then there's going to be a gap, where agent isn't in position to claim support by value, but unless eventual consequence is in turn used for \(\psi\), no clear problem --- at least not without stronger assumption.
\end{note}

\begin{note}
  \ESU{} is going to require the agent to reason from premises and steps `included' in claimed support for \(\phi\) in order to claim support for \(\psi\).
\end{note}

\begin{note}[Examples]
  Examples are somewhat difficult, due to complexities of state.
\end{note}

\begin{note}
  First, seeing exactly why the theory examples fail.
\end{note}

\begin{note}[Serial number]
  \begin{illustration}
    \label{illu:number-check}
    Genuine, only if serial number \dots (think credit cards).
  \end{illustration}
  No need to reinspect.

  Key idea here is that really relying on the product being genuine.
  Did not check for number when claiming support.
  So, without move to genuine, this breaks down.
\end{note}

\begin{note}[Logician]
  Here, novice logician, so limited to claiming support for sure.
  In principle, proof is stronger.
  However, possibility of \mistaken{}, and as a result \misled{}.
  \begin{illustration}\label{illu:CS:tfc}
    Novice logician.
    \begin{enumerate}
    \item Claimed support that \(\{\land,\lnot\}\) are truth functionally complete.
    \item If \(\{\land,\lnot\}\) are truth functionally complete then \(\{\lor,\lnot\}\) are truth functionally complete.
    \item So, \(\{\land,\lnot\}\) are truth functionally complete.
    \item Hence, \(\{\lor,\lnot\}\) are truth functionally complete.
    \end{enumerate}
  \end{illustration}

  First, reasoning for \(\{\land,\lnot\}\) did not depend on \(\{\lor,\lnot\}\).
  But this is quite complex.
  Not explicit assumption, but perhaps implicit.
  Still, going back through reasoning, it seems this is fair.

  Second, interdefinability.
  Hence, good account of why deals with possibility, as in general what holds for \(\{\land,\lnot\}\) will hold form \(\{\lor,\lnot\}\).

  Indeed, this suggests an alternative way of getting to the conclusion.
\end{note}

\begin{note}[Programming]
  \begin{illustration}
    \label{illu:programming}
    Writing a program to automate some reasoning/processing of data.
  \end{illustration}
  Various test cases.
  In these, possible to do the reasoning oneself.
  Therefore, no appeal to program for these simple cases, at least.
  This is quite similar to the logic illustration in this sense.

  However, interest here as interdependence breaks down in interesting ways.
  For, may break down due to resource constraints.
  E.g.\ available time or complexity of inputs.

  And, after enough time with the program, failure to obtain the same result is not clearly going to indicate a problem with the program.
  Rather, one's reasoning.
  Though, in turn, this may be reversed after enough checking of the reasoning.
\end{note}

\subsection{Variations on earlier examples}

\begin{note}
  Seen \nI{}, and in how builds on ideas which motivate \autoref{prop:CS-only-if-reason-recognised-defeaters}.

  To round of the illustrations, consider variations on the illustrations of~\autoref{cha:claiming-support} which related to \autoref{prop:CS-only-if-reason-recognised-defeaters}.
\end{note}

\begin{note}[Spot the difference]
  Back to \autoref{illu:CS:spot-the-diff}

  Spot the difference, think all.
  Okay, so found seven.
  Well \dots

  This turns out to be a very natural extension.
  And, strengthens the initial by avoiding the need to go to finding all of the differences.
  The reasoning alone doesn't do enough.
\end{note}

\begin{note}[Wally]
  Recall \autoref{illu:CS:wheres-wally}.

  Here, seems to apply.
  For, need it to be the case that Wally.
  However, somewhat less interesting.
  For, at the moment of completing.
  \nI{} will not find fault if, for example, in variation where book was returned to the library.
  Else, turns to be a variation on \autoref{illu:number-check}
\end{note}

\begin{note}[A trip to the zoo]
  Here, there seems no plausible variation.
  For, the interdependency fails given that there's no way to tell if it's a cleverly disguised mule by sight.

  Even in case where appeal to zebra, it need not be the case that there is interdependence.
  And, even if position to claim support by some other method (e.g.\ talking to a zoo keeper), it does not seem that sight does anything to ensure this.
\end{note}

\subsection{Illustrations where \nI{} does not apply}

\begin{note}[Treasure --- failure of interdependence]
  \begin{illustration}
    Claimed treasure only if learnt secret.
  \end{illustration}
  A little more interesting, as here, agent is going to have done something to learn secret when claiming support for treasure, but may not recognise that they've learnt the information.

  Of course, may be wrong treasure.

  However, there's too little information here to establish interdependence.
  That's the key point.

  Useful, as earlier examples may seem to rely on easy checks, but putting pieces together to reveal secret may be quite difficult.
\end{note}

\begin{note}
  The novice instance of the theory.

  And, testimony in general.
  Problem is that interdependence breaks down in these cases.
\end{note}

%%% Local Variables:
%%% mode: latex
%%% TeX-master: "master"
%%% End: