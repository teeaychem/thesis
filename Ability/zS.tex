\chapter{\zSN{2}}
\label{cha:zS}

\begin{note}[Intro, locating]
  Our goal motivate a negative resolution to~\issueConstraint{}.
  And, as sketched in~\autoref{cha:sketch}, \zSN{} particular question.
  From the agent's perspective.

  \autoref{sec:clar:type-of-scen}, \cScen{1}.
  \autoref{cha:fcs}, \fc{1}.

  Link.
  \cScen{1}, the \collateral{} is a \fc{}.

  Goal of the present chapter is foundations to link \fc{1} to concluding in certain \cScen{1}.

  Split into three main sections.
  \begin{enumerate}
  \item
    \autoref{cha:zS:sec:lost-keys}.
    Two \scen{1}, \TNSketch{1} of the core phenomena, and breakdown of the \TNSketch{1}.
  \item
    \autoref{cha:zS:sec:requs}, the idea of \(\pvp{\psi}{v'}{\Psi}\) being (partial) \check{} on concluding \(\pv{\phi}{v}\) from \(\Phi\).
    Term we give is `\requ{}'.
  \item
    \autoref{cha:zS:sec:question}, question.
    Roughly, whether or not check on reasoning is a \fc{0} --- though with some restrictions.
  \end{enumerate}

  Intuitively: \qWhyV{}.
  Though, given the outline, resist.
  Present the \scen{1}.
  May jump ahead.
  Though, we will be careful to keep things straight.
  Account of \scen{1} is general, construct your own.

  Connexion to \qWhyV{} follows in \autoref{cha:zSpA} and \autoref{cha:zSpAwhy}.
  \autoref{cha:zS:sec:question:illu} will provide additional \illu{1} of \requ{} and \qzS{}.
\end{note}

\section{Lost keys and sound rules}
\label{cha:zS:sec:lost-keys}

\begin{note}
  Previous chapter, \fc{1}.
  Potential event in which agent concludes.

  \fc{3} are about what the agent may reason to.

  Now, turn to broader, whether there is something incompatible.
  Alternative conclusion.
\end{note}

\begin{note}
  Use the \illu{} to provide a general introduction.
\end{note}

\begin{note}
  \begin{scenario}[Lost keys]
    \label{illu:lost-key}
    I think I might have lost my keys.
    I usually leave place my keys on the right side of my desk, next to a copy of~\citeauthor{Vickers:1989tr}'s~\citetitle{Vickers:1989tr} which I've been saving for a rainy day.
    And, my keys aren't there.

    I've searched on the desk, under the desk, and beside the desk.
    And, I haven't found my keys.

    Still, I haven't (yet, at least) \emph{concluded} that I've lost my keys.

    For, there might still be some place I haven't looked.
    If I think a little harder a figure out where that place is, I would conclude my keys might be in that place.
    And, my keys aren't lost if they are in that place.
    So, I might conclude that my keys aren't lost, which would conflict with concluding that my keys are lost.
  \end{scenario}

  You may disagree with the tension I see in~\autoref{illu:lost-key}.
  Perhaps it's fine to conclude my keys are lost while allowing for the possibility that they're some place I haven't yet thought of.
  However, there's tension for me.
  `I've lost my keys, but they might be under that book' feels bad to me, and to me the badness extends to `I've lost my keys, but they might in that place I haven't yet considered'.

  Though, my goal is only to convince you that my refusal to conclude I've lost my keys makes sense.
  Your perspective on truth conditions for the sentence `I've lost my keys' may be different, but I think my perspective is at least intelligible.
\end{note}

\begin{note}
  Granting the intelligibility of~\autoref{illu:lost-key}, our interest is with the following sketch:

  \begin{sketch}
    \label{sketch:zS:fail}
    For some agent, the follow two conditions obtain:
    \begin{enumerate}
      \label{skech:zS:fail:requ}
    \item
    There is some \(\pvp{\psi}{v'}{\Psi}\) such that, from the agent's perspective:
    \begin{enumerate}[label=\alph*., ref=(\alph*)]
    \item
      \label{sketch:zS:fail:requ:opportunity}
      Opportunity to reason about whether \(\pv{\psi}{v'}\) follows from \(\Psi\).
    \item
      \label{sketch:zS:fail:requ:conditional}
      If the agent took to opportunity, the agent would conclude \(\pv{\phi}{v}\) from \(\Phi\) \emph{only if} the agent would conclude \(\pv{\psi}{v'}\) from \(\Psi\).
    \end{enumerate}

  \item
    \label{sketch:zS:fail:no-c}
    The agent entertains the possibility of concluding \(\pv{\psi}{v'}\) from \(\Psi\), and therefore (it seems) they do not conclude \(\pv{\phi}{v}\) from \(\Phi\).
  \end{enumerate}
  \vspace{-\baselineskip}
  \end{sketch}

  In short, the agent, from their perspective, thinks the may reason to some other conclusion, and therefore (it seems) the agent does not conclude.

  Filling in the details of the abstract sketch:
  \begin{itemize}[noitemsep]
  \item
    I am the agent.
  \item
    \(\phi\) is the proposition: `I've lost my keys'.
  \item
    \(\psi\) is a some proposition: `My keys are not in location \(l\)'
  \item
    Both \(v\) and \(v'\) are the value: `True'.
    And,
  \item
    The pools of premises \(\Phi\) and \(\Psi\) are left unspecified.
  \end{itemize}
\end{note}

\begin{note}
  \phantlabel{qzS:intro:qualification}
  \ref{sketch:zS:fail:requ:conditional}, qualified.
  `It seems'.

  First, identifying type of \scen{}.
  In this respect, as neutral as possible.
  For a \scen{} to be the case, don't need failure of \issueConstraint{}.
\end{note}

\begin{note}[I think about this\dots]
  \autoref{illu:lost-key} is introduces this phenomena as I seem to encounter the pattern every time I think I've lost something, whether keys, books, or files.
  After some searching I feel I should have accumulated enough evidence to conclude the item is lost.
  However, the item isn't lost (from my perspective, at least), while there remains a place to check, and experience shows I eventually think of a place to check and, more often than not, the item is there.
\end{note}

\begin{note}
  Closely related.
  Replace \ref{sketch:zS:fail:requ:conditional} with:
  \begin{enumerate}[label=\alph*\('\)., ref=(\alph*\('\))]
  \item
    \label{sketch:zS:fail:requ:conditional:var}
    The agent would conclude \(\pv{\phi}{v}\) from \(\Phi\) \emph{only if} the agent would not conclude \(\pv{\psi}{v'}\) from \(\Psi\).
  \end{enumerate}
  And, replace \ref{sketch:zS:fail:no-c} with:
  \begin{enumerate}[label=\arabic*\('\)., ref=(\arabic*\('\))]
    \setcounter{enumi}{1}
  \item
    \label{sketch:zS:fail:no-c:var}
    The agent entertains the possibility of concluding \(\pv{\psi}{v'}\) from \(\Psi\), and therefore (it seems) they do not conclude \(\pv{\phi}{v}\) from \(\Phi\).
  \end{enumerate}

  Given~\ref{sketch:zS:fail:requ:conditional:var}, the relevant instance of `\(\psi\)' would be: `My keys \emph{might be} in location \(l\)'.

  Given~\ref{sketch:zS:fail:requ:conditional:var} and~\ref{sketch:zS:fail:no-c:var}, focus shifts from failing to conclude something to concluding something.

  Plausible variation.
  Concerned about failing to conclude keys are not in location \(l\) or concerned about concluding keys might be in location \(l\).

  Benefit \ref{sketch:zS:fail}, failure to conclude \(\pv{\psi}{v'}\) is also a problem.
  Specific location.
  If keys are lost, then conclude not in location.

  Ties to \cScen{1}, which shortly turn to.
\end{note}

\begin{note}
  \autoref{sketch:zS:fail}, failing to conclude.
  Agent does not conclude \(\pv{\phi}{v}\) from \(\Phi\), in part, because the agent entertains the possibility of failing to conclude \(\pv{\psi}{v'}\) from \(\Psi\).

  Consider the following variation on~\autoref{sketch:zS:fail}:
  \begin{sketch}
    \label{sketch:zS:succeed}
    For some agent, the follow two conditions obtain:
    \begin{enumerate}
    \item
      \label{sketch:zS:succeed:requ}
      There is some \(\pvp{\psi}{v'}{\Psi}\) such that, from the agent's perspective:
      \begin{enumerate}[label=\alph*., ref=(\alph*)]
      \item
        \label{sketch:zS:succeed:requ:opportunity}
        Opportunity to reason about whether \(\pv{\psi}{v'}\) follows from \(\psi\).
      \item
        \label{sketch:zS:succeed:requ:conditional}
        If took opportunity, the agent would conclude \(\pv{\phi}{v}\) from \(\Phi\) \emph{only if} the agent would conclude \(\pv{\psi}{v'}\) from \(\Psi\).
      \end{enumerate}
    \item
      \label{sketch:zS:succeed:no-c}
       The agent \emph{does not} entertain the possibility of concluding \(\pv{\psi}{v'}\) from \(\Psi\), and therefore (it seems) they do conclude \(\pv{\phi}{v}\) from \(\Phi\).
    \end{enumerate}%
    \vspace{-\baselineskip}
  \end{sketch}

  Clause~\ref{skech:zS:fail:requ} is shared between \TNSketch{3}~\ref{sketch:zS:fail} and~\ref{sketch:zS:succeed}.

  The difference between \TNSketch{3}~\ref{sketch:zS:fail} and~\ref{sketch:zS:succeed} is with respect to clause~\ref{sketch:zS:fail:no-c}.
  In clause~\ref{sketch:zS:fail:no-c} of~\autoref{sketch:zS:fail} the agent entertains the possibility of failing to conclude \(\pv{\psi}{v'}\) from \(\Psi\).
  By contrast, in clause~\ref{sketch:zS:succeed:no-c} of~\autoref{sketch:zS:succeed} the agent \emph{does not} entertain the possibility of failing to conclude \(\pv{\psi}{v'}\) from \(\Psi\).
\end{note}

\begin{note}
  If \scen{1} satisfying~\autoref{sketch:zS:succeed}, then close to counterexample to \issueConstraint{}.
  For, does not entertain \emph{because} \fc{}.
  And, from  \fc{} relation of support.
  And, no deviance.

  Though, caution, also need not witnessed reasoning.

  Provide an initial \scen{} which does not fully work as a counterexample.
  Then, develop~\autoref{sketch:zS:succeed} with care.
\end{note}

\begin{note}
  Return to~\autoref{scen:squish}:

  \scenarioPLSquish*

  \scen{0} focused on the non-standard `Squish'-elimination rule of inference and the possibility of concluding `Squish'-elimination is a sound rule of inference.

  Uncommon, but enough to memorise the rule.
  Still, I consider my general understanding of propositional logic more important than memory.
  And, if failed, then would not consider sound.

  Filling in the details of the second abstract sketch:
  \begin{itemize}[noitemsep]
  \item
    I am the agent.
  \item
    \(\phi\) is the proposition: `\((P \rightarrow Q) \rightarrow P, Q \vdash P \land Q\)'.
  \item
    \(\psi\) is a some proposition: `Squish elimination is sound'
  \item
    Both \(v\) and \(v'\) are the value: `True'.
    And,
  \item
    The pools of premises \(\Phi\) and \(\Psi\) are left unspecified.%
    \footnote{
      Note, premises of reasoning.
      Distinct from premises of deduction.
    }
  \end{itemize}

  Still, I do not entertain the possibility of failing to show sound.

  With some luck, uncommon and you will witness the conclusion similar to what I would, if I chosen to reason.%
  \footnote{
    To preserve the integrity of the \illu{0}, the proof on page~\pageref{squish-elimination-proof} was considered \emph{after} concluding the (syntactic) consequence via Squish-elimination.
  }
\end{note}

\begin{note}
  So, as with lost keys.
  I wonder about derived rules of inference.
  Clearly a problem.
  Understanding of propositional logic is good, better than memory.
  But, same understanding, sound.
\end{note}

\subsubsection{Analysis}
\label{cha:zS:sec:lost-keys:analysis}

\begin{note}
  Split these cases into three components.
  \begin{enumerate}[label=\Roman*., ref=(\Roman*)]
  \item
    \label{zS:breakdown:opp}
    Opportunity.

    Interested in concluding \(\pv{\phi}{v}\) from \(\Phi\).
    Reasoning regarding \(\pvp{\psi}{v'}{\Psi}\) is reasoning I have the opportunity to do.
  \item
    \label{zS:breakdown:check}
    \check{2}.

    Whether or not would conclude \(\pv{\psi}{v'}\) from \(\Psi\) if took up. opportunity.
  \item
    \label{zS:breakdown:psat}
    \psat{2}.

    Whether or not would conclude.
  \end{enumerate}

  From the sketches,~\ref{zS:breakdown:opp}~and~\ref{zS:breakdown:check} are the first part,~\ref{zS:breakdown:psat} is the second part.

  \check{2} is key.
  Whether it makes sense to conclude \(\pv{\phi}{v}\) from \(\Phi\).
  Opportunity to \check{0}.

  \psat{2} then concerns whether agent would satisfy \check{0}.

  \cScen{1}.
  \collateral{} reasoning, serves as \check{0}, as sketched.
\end{note}

\begin{note}
  Build this out in two separate parts.

  To begin,~\autoref{cha:zS:sec:requs} will combine~\ref{zS:breakdown:opp}~and~\ref{zS:breakdown:check} to form the idea of a `\requ{}'.
  To end,~\ref{cha:zS:sec:question} will combine the idea of a \requ{} with~\ref{zS:breakdown:psat} in the form of a question `\qzS{}'.
\end{note}

\begin{note}
  Basic point, strengthen the conditional from \cScen{1}.
  Though, impose some limitations\dots

  Intuitively, \fc{} in the good case.
  Though, with some restrictions.
  In \autoref{cha:zS:sec:requs} outline.
\end{note}

\section{\requ{3}}
\label{cha:zS:sec:requs}

%%%% TEMP from question
{
  \color{blue} \requ{} is expressed by a subjunctive conditional as there is no requirement that the agent will attempt to conclude \(\pv{\psi}{v'}\) from \(\Psi\).

  \color{red}
  As this alternative expression makes clear,~\autoref{question:zs} focuses on the agent (and their epistemic state).
  At no point do we consider any variation of the agent's epistemic state.
  Likewise,~\autoref{question:zs} concerns only the agent's perspective on concluding \(\pv{\psi}{v'}\) from \(\Psi\).
  Whether or not the agent would conclude \(\pv{\psi}{v'}\) from \(\Psi\) is irrelevant.
  What matters is whether, from the agent's perspective, there is potential for reasoning about whether \(\pv{\psi}{v'}\) follows from \(\Psi\) to block concluding \(\pv{\phi}{v}\) from \(\Phi\).
}

\begin{note}
  \color{red}
  Basic idea here is that strengthen the opportunity with ability, following \citeauthor{Austin:1961vz}'s way of putting things.
\end{note}

\begin{note}
  \color{red}
  From the analysis, opportunity and subjunctive ability.
  Capture in broader terms with the idea of a \requ{}.
\end{note}

\begin{note}[\requ{3}]
  We begin with the idea of a \requ{}, develops when something is a partial check.
  Familiar from clause~\autoref{sketch:zS:succeed:requ} of~\TNSketch{3}~\ref{sketch:zS:fail} and~\ref{sketch:zS:succeed}.

  \begin{definition}[A \requ{0}]
    \label{idea:requ}
    For an agent \vAgent{}, and proposition-value-premises pairings \(\pvp{\phi}{v}{\Phi}\), \(\pvp{\psi}{v'}{\Psi}\):

    \begin{itemize}
    \item
      \(\pvp{\phi}{v'}{\Psi}\) is a \emph{\requ{}} of concluding \(\pv{\phi}{v}\) from \(\Phi\) when:
      \begin{enumerate}[label=\arabic*., ref=\named{R:\arabic*}]
      \item
        \label{idea:requ:pool}
        \vAgent{} has the opportunity to (attempt to) conclude \(\pv{\psi}{v'}\) from \(\Psi\) such that:
        \begin{enumerate}[label=\roman*., ref=\named{R:1\roman*}]
        \item
          \label{idea:requ:pool:int}
          It is possible for \vAgent{} to conclude \(\pv{\psi}{v'}\) from \(\Psi\) without concluding \(\pv{\phi}{v}\) from \(\Phi\) as an intermediary step.
        \end{enumerate}
      \end{enumerate}

      \begin{enumerate}[label=\arabic*., ref=\named{R:\arabic*}, resume]
      \item
        \label{idea:requ:nPsi-nPhi}
        The following conditional is true:
        \begin{enumerate}
        \item[\emph{If either}:]
          \begin{enumerate}[label=\alph*., ref=\named{R:2\alph*}]
          \item
            \label{idea:requ:nPsi-nPhi:opp}
            No potential event in which conclude \(\pv{\psi}{v'}\) from \(\Phi\).
          \end{enumerate}
        \item[\emph{Or}:]
          \begin{enumerate}[label=\alph*., ref=\named{R:2\alph*}, resume]
          \item
            \label{idea:requ:nPsi-nPhi:link}
            There is some potential event in which \vAgent{} concludes some proposition-value pairing which is incompatible with concluding \(\pv{\psi}{v'}\) from \(\Psi\).
          \end{enumerate}
        \item[\emph{Then}:]
          \begin{enumerate}[label=\alph*., ref=\named{R:2\alph*}, resume]
            \label{idea:requ:nPsi-nPhi:fail}
          \item
            \vAgent{} would not conclude \(\pv{\phi}{v}\) from \(\Phi\).
          \end{enumerate}
        \end{enumerate}
      \end{enumerate}
    \end{itemize}
    \vspace{-\baselineskip}
  \end{definition}

  So, in other words, if \(\pv{\psi}{v'}\) from \(\Psi\) is \emph{not} a \fc{}, then the agent would not conclude \(\pv{\phi}{v}\) from \(\Phi\).

  {
    \color{blue}
    This is the `collective' version of \requ{}.
    So, positive.
    Need one instance of concluding \(\pv{\phi}{v}\) from \(\Psi\).
    And, also, general guarantee that wouldn't conclude something incompatible.
    So, even if instance fails, possibility still remains.

    The benefit of this is that to rule this out, don't need a strong `always conclude \(\pv{\psi}{v'}\) from \(\Psi\)' with respect to \qzS{}.
    Much weaker condition.
    Only that suitable.

    Then, the dynamic between the two clauses is familiar from definition of a \fc{}.
    Indeed, intuitively, if not a \fc{}.
  }

  {
    \color{green}
    The problem I now have is motivating the conditional.
    As stated, the link is intuitive.
    Because, it all happens in some potential event.
    Though, the due to makes a significant difference, right.
    It doesn't mean that failure alone is enough.

    So, this means, huh, that if the negation of the above works out, then plausibly not a \requ{}.
    For, if there is some potential event in which the agent concludes, absence of conclusion should not be the reason why the agent fails to conclude, and, if no conflicting conclusion, then this can't be it either.
    So, sufficient, I think, to avoid the worry.

    Okay, then, in this respect, I get to keep the understanding of a \requ{}, and then focus on weakening \qzS{}.

    So, example of not sure if potential event, information from an informer that something else follows.
    If really not sure about this, then no conclusion.
    In particular, get this relation from an intermediate step.
    So, already have the information, and it's unexpected that it comes to apply.
  }

  {
    \color{red}
    So, in this respect, the potential to conclude \(\pv{\psi}{v'}\) from \(\Psi\) is a potential check on concluding \(\pv{\phi}{v}\) from \(\Phi\).
  }

  With respect to the breakdown of \TNSketch{3}~\ref{sketch:zS:fail} and~\ref{sketch:zS:succeed}, a \requ{} captures both the opportunity for the agent to reason about whether \(\pv{\psi}{v'}\) follows from \(\Psi\) and the idea of whether \(\pv{\psi}{v'}\) follows from \(\Psi\) being a \check{0} on concluding \(\pv{\psi}{v'}\) from \(\Phi\).

  Specifically,~\ref{idea:requ:pool} corresponds to clause~\ref{sketch:zS:succeed:requ:opportunity} of \TNSketch{3}~\ref{sketch:zS:fail} and~\ref{sketch:zS:succeed}, and~\ref{idea:requ:nPsi-nPhi} corresponds to clause~\ref{sketch:zS:succeed:requ:conditional}.

  Still an issue of how the subjunctive relates to concluding \(\pv{\phi}{v}\) from \(\Phi\).
  This, \autoref{cha:zS:sec:question}.
  \requ{} is half, or two thirds of the story.
\end{note}

\begin{note}
  As presented, minor additions.

  Clause~\ref{idea:requ:pool:int} specifies that the opportunity of concluding \(\pv{\psi}{v'}\) from \(\Psi\) does not require the agent to have concluded \(\pv{\phi}{v}\) from \(\Phi\).
  And, the parenthetical notes appended to~\ref{idea:requ:nPsi-nPhi:fail} ensures that the agent not concluding \(\pv{\phi}{v}\) from \(\Phi\) is tied to failing to conclude \(\pv{\psi}{v'}\) from \(\Psi\) after taking the relevant opportunity.
  Both the additions are, arguably, implicit in clause~\ref{sketch:zS:succeed:requ:opportunity} of \TNSketch{3}~\ref{sketch:zS:fail} and~\ref{sketch:zS:succeed}.
  Still, both are sufficiently important to make explicit.

  \ref{idea:requ:pool:int} relies on distinction between reasoning to and concluding.
  \requ{2} designed to allow the reasoning to take place any time before concluding.
  Hence, just before.
  So, \(\pvp{\psi}{v'}{\Psi}\) may be a \requ{} due to reasoning from \(\Phi\).
  However, \ref{idea:requ:nPsi-nPhi:fail}, fail to conclude.
  Intuitively incompatible with having already concluded \(\pv{\phi}{v}\) from \(\Phi\) when reasoning about whether \(\pv{\psi}{v'}\) follows from \(\Psi\).

  Of course, \(\pv{\psi}{v'}\) might follow from \(\Phi\) only if concluded \(\pv{\psi}{v'}\) from \(\Psi\).
  In this case, concluding is quite important.
  Though, technically still a \requ{}.

  \emph{due to}.
  General problem of deviance.
  Subjunctive conditional, without statement, allows for failure to conclude to be unrelated.
  Or, a finkish disposition.%
  \footnote{
    ~\cite{Lewis:1997wg}
    \begin{quote}
      Anything can cause anything; so stimulus \emph{s} itself might chance to be the very thing that would cause the disposition to give response \emph{r} to stimulus \emph{s} to go away.
      If it went away quickly enough, it would not be manifested.
      In this way it could be false that if \emph{x} were to undergo \emph{s}, \emph{x} would give response \emph{r}.
      And yet, so long as s does not come along, \emph{x} retains its disposition.
      Such a disposition, which would straight away vanish if put to the test, is called finkish.%
      \mbox{ }\hfill\mbox{(\citeyear[144]{Lewis:1997wg})}
    \end{quote}
  }
  Don't have a specific account of `due to'.
  Move to the level of theories, and overall goal is to provide a theory independent motivation for rejecting \issueConstraint{}.
  So, some difficulty, may wonder whether the conditional holds.

  This brings us significant difference, where this worry will be mitigated.
\end{note}

\begin{note}
  The idea of \(\pvp{\psi}{v'}{\Psi}\) being a \requ{2} of concluding \(\pv{\phi}{v}\) form \(\Phi\) for some agent is stated without reference to the agent's perspective.
  However, our interest with \requ{1} will be from an agent's perspective.

  Easily embed within.%
  \footnote{
    Recall, same with respect to \fc{1} --- see page~\pageref{fcs-neutral-perspective}.
  }

  Distinction between whether a \requ{} and whether a \requ{} \emph{from the agent's perspective} is clearest with clause~\ref{idea:requ:pool}.
  For example, there may be no other place to look, but from my perspective, somewhere I haven't checked, and so \requ{} of concluding lost keys from my perspective.
  From my perspective, option, but from an abstract standpoint, I have exhausted all the possibilities.

  Though, clause~\ref{idea:requ:nPsi-nPhi} is where interest is.
  For, clause~\ref{idea:requ:nPsi-nPhi} is the idea of concluding \(\pv{\psi}{v'}\) from \(\Psi\) as a \emph{\check{0}} on concluding \(\pv{\phi}{v}\) from \(\Phi\).

  As with opportunity, conditional may be false.
  Basic link, seems sound.
  In \scen{1}, hard to see anything else.

  Important, though, carry from above: \emph{due to}.
  From the agent's perspective.
  Regardless of what would happen, matters for the agent.

  I think true, and no clear objections to this.
  However, take the agent's perspective, and no objections from the agent's perspective.
  Both scenarios, describing me.
\end{note}

\begin{note}
  \color{red}
  Somewhere, highlight that it doesn't matter whether \requ{} follows.
  Wait, what do I want to say?

  Well, in some cases, it may be the case that \(\pv{\phi}{v}\) follows trivially after getting \(\pv{\psi}{v'}\), this doesn't matter.

  It may also be the case that \(\pv{\phi}{v}\) entails \(\pvp{\psi}{v'}{\Psi}\), but this is also fine, so long as there is an alternative way.
\end{note}

\begin{note}
  \color{red}

  clause~\ref{idea:requ:pool} means that, so long as \(\phi\) has value \(v\), the agent has the option of checking whether it makes sense for the agent to conclude \(\pv{\phi}{v}\) from \(\Phi\).

  And clause~\ref{idea:requ:nPsi-nPhi} expresses that concluding \(\pv{\psi}{v'}\) from \(\Psi\) is a check on whether it makes sense for the agent, from their perspective, to conclude \(\pv{\phi}{v}\) from \(\Phi\).
\end{note}

\subsection{Examples}
\label{cha:zS:sec:requs:examples}

\begin{note}
  See two examples of a \requ{}.

  \autoref{illu:lost-key} and~\autoref{scen:squish}.
\end{note}

\begin{note}[Example]
  \autoref{illu:gist:roots} involves an agent concluding either \(x = 1\) or \(x = -\sfrac{1}{2}\) from premise that for some \(x \in \mathbb{R}\), \(2x^{2} - x - 1 = 0\).%
  \footnote{
    Abstractly, \autoref{illu:gist:roots} is a case where the agent would not conclude \(\pv{\phi}{v}\) from \(\Phi\) if the agent failed to conclude \(\pv{\phi}{v}\) from \(\Psi\).
    I.e.\ the relevant conclusion is the same in both proposition-value-premises pairings, the only difference is the relevant pool of premises (and method of reasoning).
  }
  And, when concluding either \(x = 1\) or \(x = -\sfrac{1}{2}\) the agent observes that \emph{if} \(x = 1\) or \(x = -\sfrac{1}{2}\), then they would also be able to observe this via factorisation.

  In other words, if the agent attempted to conclude either \(x = 1\) or \(x = -\sfrac{1}{2}\) via factorisation and failed, the agent would not conclude either \(x = 1\) or \(x = -\sfrac{1}{2}\) via (their application of) the quadratic formula.
\end{note}

\begin{note}[Calculator]
  Likewise, failure of \(\pv{\psi}{v'}\) being a \requ{}.
  Opening \scen{},~\autoref{illu:gist:calc}.

  Emphasis on testimony.

  Example here, failure for the conditional to hold, though plausibly have the option.

  With respect to the \scen{0}, testimony.

  However, more broadly, agent values reasoning over another.

  Abstract from the testimony of a calculator to settings where the dynamic is more intuitive.

  For example, student in a classroom.

  Though, with the broader idea, ways to make calculator easier.

  Under the weather.
  Things are a little foggy.
  Opportunity persists, but conditional doesn't hold.
\end{note}

\begin{note}[Failure but no option]
  \citeauthor{Dretske:1970to}.
  \begin{scenario}[A trip to the zoo]\mbox{ }
    \label{scen:trip-to-zoo}
    \vspace{-\baselineskip}
    \begin{quote}
      You take your son to the zoo, see several zebras, and, when questioned by your son, tell him they are zebras.
      Do you know they are zebras?
      [\dots]
      We know what zebras look like, and, besides, this is the city zoo and the animals are in a pen clearly marked ``Zebras.''
      Yet, something's being a zebra implies that it is not a mule and, in particular, not a mule cleverly disguised by the zoo authorities to look like a zebra.
      Do you know that these animals are not mules cleverly disguised by the zoo authorities to look like zebras?\newline
      \mbox{ }\hfill\mbox{(\citeyear[1015--1016]{Dretske:1970to})}
    \end{quote}
    \vspace{-\baselineskip}
  \end{scenario}

  \autoref{scen:trip-to-zoo} is framed in terms of knowledge, and is designed to raise a problem for conclude of knowledge under known entailment.
  Intuitively, you know the animals in the pen are zebras.
  And, you know the following conditional is true:
  The animals in the pen are zebras \emph{only if} the animals in the pen are not cleverly disguised mules.
  However, you (intuitively) don't know the animals in the pen are not cleverly disguised mules.

  If knowledge is closed under known entailment, then you:
  \begin{enumerate}
  \item \(\phi\) has value \(v\) only if \(\psi\) has value \(v'\)
  \end{enumerate}
  then, if
  \begin{enumerate}
  \item
    \(\phi\) has value \(v\), then
  \end{enumerate}
  \begin{enumerate}
  \item \(\psi\) has value \(v'\)
  \end{enumerate}

  Framed in terms of knowledge, but relation is similar to \requ{}.

  No opportunity!

  There is no premises to distinguish.
  \scen{3} designed to test closure principles provide various examples of this kind.
  In particular, \citeauthor{Wright:2011wn}.
\end{note}

\begin{note}
  Returning to instances of \requ{1}, ability.
  Most interesting case, from my perspective.

  Simple case is Sudoku puzzles, or puzzles in general.

  For, get clusters and hierarchies of \requ{1}{\color{red} ?}

  Though, it is a little tricky.
  For, depends on methods available.
  For example, addition.
  It may be the case in doing some complex addition, need to get the result of something more basic.
  So, example with long addition.
  Failure of a \requ{}.
\end{note}

\subsection{\requ{3} and (partial) checks on reasoning}
\label{cha:zS:sec:requs:checks}

\begin{note}
  Or, what I really want to say is something like, the agent having the option to conclude \(\pv{\psi}{v'}\) from \(\Psi\).
  This is the check.
\end{note}

\begin{note}
  \begin{itemize}
  \item Check.
  \item Check on \emph{reasoning}.
  \item \emph{Partial} check on reasoning.
  \end{itemize}
\end{note}

\subsubsection{Check}

\begin{note}[Check]
  Prominent clause is clause~\ref{idea:requ:nPsi-nPhi}.
  For, clause~\ref{idea:requ:nPsi-nPhi} captures the core idea of failure to conclude \(\pv{\psi}{v'}\) from \(\Psi\) leading to failure to conclude \(\pv{\phi}{v}\) from \(\Phi\).

  Also opportunity.
  As seen with the \citeauthor{Dretske:1970to} case.
  No opportunity, and therefore not a check in the relevant sense.

  Really, opportunity is easy to overlook, but very important.
  Clause~\ref{idea:requ:nPsi-nPhi} is subjunctive.
  Opportunity ensures that concluding from present state.
  Stops the subjunctive from wandering too far.

  Similar sense of check as checking date of birth.
  Fail to be of age, then no purchasing \dots

  Difference sense of check to label on the box.
  In a sense, determines whether or not make the purchase.
  Though, what really matters is whether the shop assistant asks for date of birth.

  Failure.
  This is stronger than should not, or might not.
\end{note}

\begin{note}
  Stated independently of agent's perspective, whether check in this sense is unclear.

  However, some caution.
  Harman style cases where there's some change.
  \requ{} is tied to particular pools of premises.
  At issue is not whether the agent would conclude \(\pv{\phi}{v}\) after failing to conclude, but conclude \(\pv{\phi}{v}\) \emph{from \(\Phi\)}.
  Concluding \(\pv{\phi}{v}\) from \(\Phi'\) may be okay, but \(\Phi\) really is bad.

  Still, from agent's perspective, fine.
  I was confident I'd stop if failed to get validity of squish.

  Worry about what actually happens relies on things that perspective doesn't take into account.
  But, from perspective, fine.

  Note, with Harman style cases, also the possibility that whether something is a \requ{} may change.
\end{note}

\subsubsection{Reasoning}

\begin{note}[Check on reasoning]
  Or, perhaps, concluding.
\end{note}

\begin{note}
  Two senses in which this is check on reasoning.
  First, concerns reasoning to \(\pv{\phi}{v}\) from \(\Phi\).
  Second, given reasoning to \(\pv{\phi}{v}\) from \(\Phi\).
  At the edge of concluding.
  Recall from chapter on \cScen{0}.
\end{note}

\begin{note}
  Equally important, does not expand beyond reasoning.
\end{note}

\begin{note}[Problems of induction]
  Note, however, both~\autoref{illu:lost-key} and~\autoref{scen:squish} (and \cScen{1} in general) focus on reasoning.

  In turn, \TNSketch{3}~\ref{sketch:zS:fail} and~\ref{sketch:zS:succeed} focus on the failure to conclude to some proposition-value pair which would follow from concluding some other proposition-value pair.

  Hence, the sketch does not apply to black ravens.
  I wouldn't conclude all ravens are black if I saw a white raven.

  I may worry about shortly seeing a white raven when concluding all ravens are black, and I may refuse to entertain the possibility that the sun will rise tomorrow when planning to mow the grass.

  However, it's not possible to reason to seeing a white raven, nor is it possible to reason to the sun not rising tomorrow.

  Abstractly, at issue in~\autoref{illu:lost-key} is the possibility of failing to a reason to some proposition-value pair given \emph{present} information, rather than the possibility of failing to a reason to some proposition-value pair given \emph{new} information.

  To the extent that problems of induction arise from receiving new information, what is at issue is distinct.%
  \footnote{
    See~\textcite{Henderson:2020wb} for more on the problem of induction.
  }

  Similar points for external world scepticism.
  Would not conclude that I have hand if disembodied brain in a vat.

  However, conclusion is out of reach.
\end{note}

\subsubsection{Partial}

\begin{note}
  The `converse' of \ref{idea:requ:nPsi-nPhi} doesn't hold.
  In other words, \(\pvp{\psi}{v'}{\Psi}\) being a \requ{} of concluding \(\pv{\phi}{v}\) from \(\Phi\) does \emph{not} require the following conditional to be true:
  \begin{enumerate}[label=]
  \item
    \begin{enumerate}[label=]
      \setcounter{enumi}{1}
    \item
      \label{idea:requ:nPsi-nPhi:conv}
      \begin{enumerate}
      \item[\emph{If}:]
        \begin{enumerate}[label=\alph*., ref=\named{R:b.\alph*}]
        \item
          \vAgent{} were to take the opportunity to reason about whether \(\pv{\psi}{v'}\) follows from \(\Psi\).
        \end{enumerate}
      \item[\emph{And}:]
        \begin{enumerate}[label=\alph*., ref=\named{R:b.\alph*}, resume]
        \item
          \vAgent{} \emph{were to} conclude \(\pv{\psi}{v'}\) from \(\Psi\) prior to concluding \(\phi\) has value \(v\).
        \end{enumerate}
      \item[\emph{Then}:]
        \begin{enumerate}[label=\alph*., ref=\named{R:b.\alph*}, resume]
        \item
          \vAgent{} \emph{would} conclude \(\pv{\phi}{v}\) from \(\Phi\).
        \end{enumerate}
      \end{enumerate}
    \end{enumerate}
  \end{enumerate}

  There may be other \requ{}.
  {
    \color{red}
    Seen in \scen{0}???
  }
  And, there may be checks other than \(\pvp{\psi}{v'}{\Psi}\) being a \requ{}.

  A different, but related check, would consider whether the agent has concluded.
  However, less interesting.
  Consider the squish scenario.
  Have concluded.
  However, what's of interest is how things are.
\end{note}

\begin{note}
  The role of clause~\ref{idea:requ:pool} is to ensure the agent may conclude \(\pv{\psi}{v'}\) from \(\Psi\) independently of concluding \(\pv{\phi}{v}\) from \(\Phi\).

  If \ref{idea:requ:pool:int} were to fail to hold then:
  \begin{itemize}
  \item
    By~\ref{idea:requ:pool:int}, the agent would need to conclude \(\pv{\phi}{v}\) from \(\Phi\) as a sub-conclusion when reasoning from the relevant pool of premises \(\Psi\).
    Hence, it would not be possible to conclude \(\pv{\psi}{v'}\) from \(\Psi\) without first concluding \(\pv{\phi}{v}\) from \(\Phi\).
  \end{itemize}

  Conversely, if \ref{idea:requ:pool:int} holds, the agent may conclude \(\pv{\psi}{v'}\) from \(\Psi\) independently of concluding \(\pv{\phi}{v}\) from \(\Phi\).

  Note, however, \ref{idea:requ:pool:int} does not rule out the possibility of the agent concluding \(\pv{\phi}{v}\) from \(\Phi\) when concluding \(\pv{\psi}{v'}\) from \(\Psi\) or, conversely, concluding \(\pv{\psi}{v'}\) from \(\Psi\) when concluding \(\pv{\phi}{v'}\) from \(\Phi\).

  Having the option which matters.
  Whether or not the option is taken, doesn't matter.
\end{note}

\paragraph{\requ{3} and undercutting defeaters}

\begin{note}
  Shares similarity.
  Focus on the reasoning.
  However, subjunctive.
  \qzS{} is about entertaining the possibility of an undercutting defeater for reasoning.
\end{note}

\begin{note}
  Not about the proposition-value pair.
  Rather, it is about the concluding.
  At interest is not whether \(\phi\) has value \(v\), but whether it makes sense to conclude \(\pv{\phi}{v}\) from \(\Phi\).
  Of course, if the agent has no information about whether \(\phi\) has value \(v\), then this is also part of the picture, but that is a consequence of the base concern.

  With squish.
  I wouldn't conclude.
  However, wouldn't say that entailment doesn't hold.
  For, may only be the case that the derived rule of inference fails to hold.
\end{note}

\begin{note}
  In this respect, failing to conclude \(\pv{\psi}{v'}\) from \(\Phi\) may be described as an undercutting defeater with respect to conclude \(\pv{\phi}{v}\) from \(\Phi\).%
  \footnote{
    We take the following sketch from \textcite{Worsnip:2018aa}:
  \begin{quote}
    Undercutting defeaters, which are easiest to think of in the context of the attitude of belief, are supposed to be considerations that undermine the justification of a belief in a proposition p not necessarily by providing (sufficient) positive evidence to think that p is false, but rather merely by suggesting (perhaps misleadingly) that one’s reasons for believing p are no good, in a way that neutralizes or mitigates their justificatory or evidential force.%
    \mbox{}\hfill\mbox{(\citeyear[29]{Worsnip:2018aa})}
  \end{quote}
  }

  Consider the following illustration provided by \citeauthor{Pollock:1987un}:
  \begin{quote}
    [Undercutting defeaters] attack the connection between the reason and the conclusion rather than attacking the conclusion itself.
    For instance, ``X looks red to me'' is a prima facie reason for me to believe that X is red.
    Suppose I discover that X is illuminated by red lights and illumination by red lights often makes things look red when they are not.
    This is a defeater, but it is not a reason for denying that X is red (red things look red in red light too).
    Instead, this is a reason for denying that X wouldn't look red to me unless it were red.%
    \mbox{}\hfill\mbox{(\citeyear[485]{Pollock:1987un})}
  \end{quote}
  Completing \citeauthor{Pollock:1987un}'s example, it seems that if agent's support for holding that X is red is that `X wouldn't look red to me unless it were red', then the support for X being red provided by appearance is retracted after discovering that X is illuminated by red lights (though it remains possible that X is red).
\end{note}

\begin{note}
  In \citeauthor{Pollock:1987un}'s example, discover that the light is red.
  By parallel, the agent failing to conclude would undercut.
  By contrast, the agent hasn't failed to conclude \(\pv{\psi}{v'}\) from \(\Psi\).
  So, not a direct undercutting defeater.
\end{note}

\begin{note}
  Similarity.

  However, given the focus on reasoning, \requ{} is not a simple instance of an undercutting defeater.
  At least, given \citeauthor{Pollock:1987un}'s definition of an undercutting defeater.

  \citeauthor{Pollock:1987un} defines undercutting defeaters as follows:
  \begin{quote}
    R is an \emph{undercutting defeater} for P as a prima facie reason for S to believe Q if and only if
    \begin{enumerate}[label=(UD\arabic*), ref=(UD\arabic*)]
    \item
      P is a reason for S to believe Q and R is logically consistent with P but (P and R) is not a reason for S to believe Q, and
    \item
      R is a reason for denying that P wouldn't be true unless Q were true.%
      \mbox{}\hfill\mbox{(\citeyear[485]{Pollock:1987un})}
    \end{enumerate}
  \end{quote}
  Intuitively, an undercutting defeater for P as a reason for Q because it the defeater denies that Q must be true in order for P to be true.

  Variation:
  \begin{quote}
    R is \emph{undercutting reasoning} with respect to \emph{S} concluding \(\pv{\phi}{v}\) from \(\Phi\) if and only if
    \begin{enumerate}[label=(UR\arabic*), ref=(UR\arabic*)]
    \item
      The reasoning from \(\Phi\) is sufficient for \emph{S} to conclude \(\pv{\phi}{v}\) and R is reasoning which is logically consistent with reasoning from \(\Phi\) but combination of both reasoning from \(\Phi\) and R is not a sufficient for \emph{S} to conclude \(\pv{\phi}{v}\), and
    \item
      R is sufficient for denying that S wouldn't conclude \(\pv{\phi}{v}\) from \(\Phi\) unless \(\pv{\phi}{v}\) followed from \(\Phi\).%
    \end{enumerate}
    Where, R is reasoning which fails to conclude \(\pv{\psi}{v'}\) from \(\Psi\).
  \end{quote}

  The problem here is with the first clause.
  Logically consistent.

  It is not clear that the reasoning is logically consistent.
  Two instances of reasoning may involve intermediary steps which are logically inconsistent.

  Though, on the other hand, in interesting cases it is logically consistent for the agent to reason to \(\pv{\phi}{v}\) from \(\Phi\) and fail to conclude \(\pv{\psi}{v'}\) from \(\Psi\).
  For, if not logically possible, then no worries about failing to conclude \(\pv{\psi}{v'}\) from \(\Psi\).

  Yet, if problem, then from agent's perspective, instances of reasoning aren't consistent.

  Puzzle here is how logical consistency is understood with respect to \citeauthor{Pollock:1987un}'s definition.
  Independent of the agent's perspective, or from the agent's perspective?

  Both interpretations are compatible with the example.
  But, principle also holds independently of the agent's perspective with respect to the example.
\end{note}

\begin{note}
  In contrast, \requ{} `attacks' the \emph{reasoning} and the conclusion.
\end{note}

\paragraph{Agent's perspective and concluding}

\begin{note}[Abstract motivation]
  We only need to consider what concluding \(\pv{\phi}{v}\) from \(\Phi\) would commit the agent to, so to speak.
  For example, we do not need to consider the consequences of the agent reasoning about some arbitrary proposition-value pair \(\pv{\chi}{v''}\) prior to concluding \(\pv{\phi}{v}\) from \(\Phi\).

  Of course, if the agent would conclude both \(\pv{\chi}{v''}\) and \(\pv{\chi}{\overline{v''}}\) for some \(\chi\), then it seems the agent's epistemic state is in bad shape.
  Still, given that an agent will typically revise their epistemic state upon concluding both \(\pv{\chi}{v''}\) and \(\pv{\chi}{\overline{v''}}\) for some \(\chi\), such concerns may be isolated to a distinct question.

  Second, we avoid --- to some extent --- concerns about over-generating.
  Role is negative resolution to \issueConstraint{}.
  No witnessing.
  A concern is that this overall argument may over-generate.
  Unintended consequences may diminish interest in the negative resolution we hope to motivate, and hence tip favour to a positive answer to~\issueConstraint{}.
  Though, whether or not there are unintended consequences of broadening the scope of~\autoref{question:zs} is unclear to me.
\end{note}

\paragraph{When concluding}

\begin{note}
  Hence, it need not be the case that the agent has the option of concluding \(\pv{\psi}{v'}\) from \(\Psi\) from their epistemic state prior to starting line of reasoning (as the agent has not yet concluded that \(\phi\) has value \(v\)).
\end{note}

\begin{note}[Prior to concluding\dots]
  An important feature of \qzS{} \dots

  Not particularly marked.
  Allow agent to have built up a bunch of stuff in the reasoning.

  Example.

  \begin{scenario}[Velocity]
    \label{ill:velocity}
    Agent is provided with information about how far a car has travelled north as a function of time travelled.

    From this, take the derivative of the function to obtain the (instantaneous) velocity of the car at a handful of points in time.

    And, from the (instantaneous) velocity of the car, the agent calculates the (instantaneous) acceleration of the car at each of the points in time.

    The agent also has information about the speed of the car as a function of time travelled, and the agent may calculate speed by the taking magnitude of the (instantaneous) velocity of the car.
  \end{scenario}

  \autoref{ill:velocity}, two step calculation.
  Velocity, acceleration.
  After the first step, check by taking the magnitude.
  Calculation of velocity is correct only if taking the magnitude matches speed.

  Just before concluding to include cases such as this.
\end{note}

\begin{note}
  Example highlights how `intermediate conclusions' relate.
  Further point of interest:
  Failure to conclude.

  Two ways to view agent's calculation of the velocity of the car.

  First, as a conclusion.
  Same status as the function.

  Or, as temporary.

  Difference in how we understand agent's present epistemic state.

  On first, the agent's present epistemic state is inconsistent.
  Two proposition-value pairs which conflict.
  Not possible for the car to have velocity the agent calculated and acceleration the agent has been informed of.

  May also be that the function and information about acceleration are inconsistent, but may also be that the agent made a mistake in calculating the velocity of the car.

  On second, the agent's present epistemic state may%
  \footnote{
    Don't have complete perspective on agent's present epistemic state.
  }
  consistent.

  For, made a mistake.
  But, proposition-value pair is not part of present epistemic state, so distinguished from function and information about acceleration, which are consistent.

  This is a distinction we have little interest in.
  What matters is failure to conclude speed.
  Result is either revising inconsistent epistemic state, or abandoning intermediary steps of reasoning.
\end{note}

\paragraph{Recursion?}

\begin{note}
  This arises in two respects.
\end{note}

\begin{note}
  \ref{idea:requ:pool}, opportunity constrained \ref{idea:requ:pool:int}.

  Purpose is, not need to conclude \(\pv{\phi}{v}\) from \(\Phi\).

  Implicit assumption that \requ{} applies when concluding.
  And, has option, so \(\Psi\) are available.
  Hence, not yet concluded, so \(\pv{\phi}{v}\) from \(\Phi\) was not involved in any of the premises.

  With this intent, problems may arise.
  For some \(\pv{\psi_{i}}{v_{i}}\) in \(\Psi\), concluded in part from \(\pv{\phi}{v}\) from \(\Phi\).
  Agent retained \(\pv{\psi_{i}}{v_{i}}\) but forgot \(\pv{\phi}{v}\) from \(\Phi\).

  I don't find this particularly interesting.
  About present reasoning.
  Same as with testimony.
  Take the agent's present perspective on how things are to be fixed.

  However, if a worry, then either add clause which quantifies over past reasoning directly, or recursive.

  \begin{enumerate}
  \item
    \label{idea:requ:pool:ind}
    For any proposition-value pair \(\pv{\psi_{i}}{v_{i}}\) in \(\Psi\), \vAgent{} may conclude \(\pv{\psi_{i}}{v_{i}}\) from some pool of premises \(\Psi_{i}\) without concluding \(\pv{\phi}{v}\) from \(\Phi\).
  \end{enumerate}
\end{note}

\subsection{\requP{3}}
\label{sec:requ3-plus}

\begin{note}
  \begin{definition}[A \requP{0}]
    \label{idea:requP}
    For an agent \vAgent{}, and proposition-value-premises pairings \(\pvp{\phi}{v}{\Phi}\) and \(\pvp{\psi}{v'}{\Psi}\):

    \begin{itemize}
    \item
      \(\pvp{\phi}{v'}{\Psi}\) is a \emph{\requP{}} of concluding \(\pv{\phi}{v}\) from \(\Phi\) when:
      \begin{enumerate}[label=\arabic*., ref=\named{R:\arabic*}]
      \item
        \label{idea:requP:pool}
        \vAgent{} has the opportunity to (attempt to) conclude \(\pv{\psi}{v'}\) from \(\Psi\) such that:
        \begin{enumerate}[label=\roman*., ref=\named{R:1\roman*}]
        \item
          \label{idea:requP:pool:int}
          It is possible for \vAgent{} to conclude \(\pv{\psi}{v'}\) from \(\Psi\) without concluding \(\pv{\phi}{v}\) from \(\Phi\) as an intermediary step.
        \end{enumerate}
      \end{enumerate}

      \begin{enumerate}[label=\arabic*., ref=\named{R:\arabic*}, resume]
      \item
        \label{idea:requP:nPsi-nPhi}
        The following conditional is true:
        \begin{enumerate}
        \item[\emph{If}:]
          \begin{enumerate}[label=\alph*., ref=\named{R:2\alph*}]
          \item
            \label{idea:requP:nPsi-nPhi:opp}
            \vAgent{} were to take the opportunity to reason about whether \(\pv{\psi}{v'}\) follows from \(\Psi\).
          \end{enumerate}
        \item[\emph{And}:]
          \begin{enumerate}[label=\alph*., ref=\named{R:2\alph*}, resume]
          \item
            \label{idea:requP:nPsi-nPhi:link}
            \vAgent{} were to fail to conclude \(\pv{\psi}{v'}\) from \(\Psi\) prior to concluding \(\phi\) has value \(v\).
          \end{enumerate}
        \item[\emph{Then}:]
          \begin{enumerate}[label=\alph*., ref=\named{R:2\alph*}, resume]
            \label{idea:requP:nPsi-nPhi:fail}
          \item
            \vAgent{} would not conclude \(\pv{\phi}{v}\) from \(\Phi\) (\emph{due to}~\ref{idea:requP:nPsi-nPhi:opp} and~\ref{idea:requP:nPsi-nPhi:link}).
          \end{enumerate}
        \end{enumerate}
      \end{enumerate}
    \end{itemize}
    \vspace{-\baselineskip}
  \end{definition}
\end{note}

\begin{note}
  This may be a stronger condition.
  Straightforward proposition.
  The problem here is \emph{would}.
  However, weakened by `due to'.
  So, plausible that these are equivalent.
\end{note}

\subsection{Summary}
\label{cha:zS:sec:requs:summary}

\begin{note}
  Two conditions from breakdown of \TNSketch{3}~\ref{sketch:zS:fail} and~\ref{sketch:zS:succeed}.
  Opportunity and \check{}.

  \requ{}.
  Minor clarifications.
  From the agent's perspective.

  Examples.
  In particular, link to \cScen{1}.

  Idea that a \requ{} identifies a (partial) check on reasoning in detail.

  Undercutting defeaters.

  Minor details.
\end{note}

\begin{note}
  Breakdown of \TNSketch{3}~\ref{sketch:zS:fail} and~\ref{sketch:zS:succeed}.
  Third condition.
  \psat{2}.
  Turn to this in follow section.
\end{note}

\section{\zS{}}
\label{cha:zS:sec:question}

\begin{note}[Outline]
  Outline
  \begin{itemize}
  \item
    Start with \qzS{}, motivation
  \item
    Differences to \autoref{sketch:zS:succeed}.
  \item
    Broader argument
  \item
    \scen{1}
  \item
    Visualisation
  \item
    Questions about recursion.
  \item
    Variation and additional motivation.
  \end{itemize}
\end{note}

\begin{note}[The question]
  With the idea of a \requ{} in hand, turn to \zSN{}.
  We begin stating a question.
  Understand \zSN{} in terms of answers to the question.

  \begin{restatable}[\qzS{}]{question}{questionZS}
    \label{question:zs}
    For an agent \vAgent{}, when concluding \(\pv{\phi}{v}\) from \(\Phi\), is it the case that:%
    \footnote{
      As with \qWhy{}, \qHow{}, and \qWhyV{}, we take the agent and instance of concluding to be independently specified for instances of \qzS{} in order to avoid the verbose `\qzS{} with respect to agent \emph{A} concluding \(\pv{\phi}{v}\) from \(\Phi\)'.
    }

    \begin{itemize}
    \item
      From \vAgent{}' perspective:
      \begin{itemize}
      \item
        For any proposition-value-premises pairing \(\pvp{\psi}{v'}{\Psi}\):
        \begin{itemize}
        \item[\emph{If}:]
          \begin{enumerate}[label=\alph*., ref=(\alph*)]
          \item
            \label{question:zs:option}
            \(\pvp{\psi}{v'}{\Psi}\) is a \requ{} of concluding \(\pv{\phi}{v}\) from \(\Phi\).
          \end{enumerate}
        \item[\emph{Then}:]
          \begin{enumerate}[label=\alph*., ref=(\alph*), resume]
          \item
            \label{question:zs:may-fail}
            \vAgent{} would conclude \(\pv{\psi}{v'}\) from \(\Psi\), if \vAgent{} were to attempt to conclude \(\pv{\psi}{v'}\) from \(\Psi\).
          \end{enumerate}
        \end{itemize}
      \end{itemize}
    \end{itemize}
    \vspace{-\baselineskip}
  \end{restatable}

  \qzS{} concerns an agent when concluding \(\pv{\phi}{v}\) from \(\Phi\), and asks --- straightforwardly --- whether the agent would conclude \(\pv{\psi}{v'}\) from \(\Psi\) for any \(\pv{\psi}{v'}\) which is a \requ{} of concluding \(\pv{\phi}{v}\) from \(\Phi\), from the agent's perspective.
\end{note}

\begin{note}[Intuitive motivation]
  {
    \color{blue}
    Intuitive motivation.
  }
  \qzS{} is a question about what holds with respect to \emph{any} proposition-value-premises pairing, from an agent's perspective, when the agent concludes some proposition-value pair \(\pv{\phi}{v}\) from some pool of premises \(\Phi\).

  Key thing, concluding.
  Alternative conclusions.

  Suppose \requ{}.
  Just prior to concluding.
  So, take up option.
  Possible, don't conclude \(\pv{\psi}{v'}\) from \(\Psi\).
\end{note}

\begin{note}[\autoref{sketch:zS:succeed} is \(\exists\), but \qzS{} is \(\forall\)]
  \qzS{} is distinct from~\autoref{sketch:zS:succeed}.

  \begin{quote}
    There is a \requ{} and would conclude.
  \end{quote}

  Weaker, in sense positive answers if no \requ{}.
  Stronger, in sense that quantifies over all \requ{}.

  So following the stronger sense, \qzS{} captures instances of~\autoref{sketch:zS:succeed}.

  It's more natural.
  Would not reason to a different conclusion.
  Shift focus from \requ{} to concluding \(\pv{\phi}{v}\) from \(\Phi\).
\end{note}

\begin{note}[\qzS{} \(\forall\)]
  Note in particular \qzS{} asks whether a quantified conditional holds \emph{from the agent's perspective}.
  Hence, in order for \qzS{} to have a positive answer, one only needs to consider whether \(\pvp{\psi}{v'}{\Psi}\) is a \requ{} of concluding \(\pv{\phi}{v}\) from \(\Phi\) from the agent's perspective.%
  \footnote{
    If \requ{3} were from the agent's perspective, then of interest would be whether \(\pvp{\psi}{v'}{\Psi}\) is a \requ{} of concluding \(\pv{\phi}{v}\) from \(\Phi\) from the agent's perspective, from the agent's perspective.

    Of course, this is only an issue of phrasing.
    The intent of \qzS{} is for~\ref{question:zs:may-fail} to be true from the agent's perspective whenever~\ref{question:zs:option} is true from the agent's perspective.
  }
  And, likewise, whether, from the agent's perspective, the would conclude \(\pv{\psi}{v'}\) from \(\Psi\) if they were to make an attempt.

  In negative paraphrase, \qzS{} asks whether it is the case that:%
  \footnote{
    Re-expressed with a negated conditional, \qzS{} asks the following question:
    \begin{restatable}[\nqzS{}]{question}{questionZSNeg}
      \label{question:zs:neg}
      For an agent \vAgent{}, when concluding \(\pv{\phi}{v}\) from \(\Phi\), is it the case that:

      \begin{itemize}
      \item
        From \vAgent{}' perspective:
        \begin{itemize}
        \item
          There some proposition-value-premises pairing \(\pvp{\psi}{v'}{\Psi}\):
          \begin{itemize}
          \item[\emph{Both}:]
            \begin{enumerate}[label=\alph*., ref=(\alph*)]
            \item
              \(\pvp{\psi}{v'}{\Psi}\) is a \requ{} of concluding \(\pv{\phi}{v}\) from \(\Phi\).
            \end{enumerate}
          \item[\emph{And}:]
            \begin{enumerate}[label=\alph*., ref=(\alph*), resume]
            \item
              \label{question:zs:may-fail}
              \vAgent{} may fail conclude \(\pv{\psi}{v'}\) from \(\Psi\), if \vAgent{} were to attempt to conclude \(\pv{\psi}{v'}\) from \(\Psi\).
            \end{enumerate}
          \end{itemize}
        \end{itemize}
      \end{itemize}
      \vspace{-\baselineskip}
    \end{restatable}
  }
  \begin{itemize}
  \item
    From the agent's perspective:
    \begin{itemize}
    \item
      There is some \requ{} --- \(\pvp{\psi}{v'}{\Psi}\) --- of concluding \(\pv{\phi}{v}\) from \(\Phi\) such that the agent \emph{may} fail to conclude \(\pv{\psi}{v'}\) from \(\Psi\).
    \end{itemize}
  \end{itemize}

  The negative paraphrase is, at least to me, softer than~\qzS{}.
  If find entertaining whether or not some \(\pvp{\psi}{v'}{\Psi}\) pairing may raise an issue with respect to concluding \(\pv{\phi}{v}\) from \(\Phi\) (by being a \requ{}) to be less of a cognitive burden that entertaining whether all proposition-value-premises pairings satisfy the \ref{question:zs:option}-\ref{question:zs:may-fail} conditional of \qzS{}.

  Still, the \ref{question:zs:option}-\ref{question:zs:may-fail} conditional of \qzS{} is stated from the agent's perspective.
  Therefore, if, from my perspective, either:
  \begin{enumerate}[label=\roman*., noitemsep]
  \item
    \(\pvp{\psi}{v'}{\Psi}\) is not a \requ{} of concluding \(\pv{\phi}{v}\) from \(\Phi\), or
  \item
    \(\pvp{\psi}{v'}{\Psi}\) is a \requ{} and I would conclude \(\pv{\psi}{v'}\) from \(\Psi\).
  \end{enumerate}
  Then, from my perspective the \ref{question:zs:option}-\ref{question:zs:may-fail} conditional is true.
  And, the same holds for any agent.

  Though, the link between my reasoning about the quantified conditional, and whether the quantified conditional is true, from my perspective is weak.
  A positive answer to \qzS{} does not require the agent to entertain, recognise, believe, or have any `substantial' attitude toward the truth of the quantified conditional.

  Though, in the \scen{1} of interest, \qzS{} itself is of interest because the agent entertains, recognises, believes, or has some `substantial' attitude toward \(\pvp{\psi}{v'}{\Psi}\) being a \requ{} of concluding \(\pv{\phi}{v}\) from \(\Phi\).
  What matters, in short, is that the agent doesn't have `substantial' attitude toward the possibility of failing to conclude \(\pv{\psi}{v'}\) from \(\Psi\), for some \(\pvp{\psi}{v'}{\Psi}\) which is a \requ{} of concluding \(\pv{\phi}{v}\) from \(\Phi\).
\end{note}

\begin{note}[Not our perspective]
  Second difference:

  In place of qualifying our understanding of the \scen{} with `it seems'%
  \footnote{
    Recall, Clause~\ref{sketch:zS:succeed:no-c} of \autoref{sketch:zS:succeed} reads {\color{blue} \dots}
  }%
  , \qzS{} asks whether the quantified conditional is true from the agent's perspective.

  So, for a positive answer to \qzS{} we only need to consider how things are from the agent's perspective.

  At issue is whether---roughly---, from the agent's perspective, they would conclude \(\pv{\psi}{v'}\) from \(\Psi\) for any \(\pvp{\psi}{v'}{\Psi}\) pairing such that if the agent were to fail to conclude \(\pv{\psi}{v'}\) from \(\Psi\), they would not conclude \(\pv{\phi}{v}\) from \(\Phi\).

  Leaves open key link.

  \begin{idea}[\qzS{} and \qWhyV{}]
    \label{prop:qzS-answers-why}
    There are cases in which:

    \begin{itemize}
    \item[\(\pm\)]
      A positive answer to \qzS{} answers, in part, why an agent concludes or does not conclude \(\pv{\phi}{v}\) from \(\Phi\).%
      \footnote{
        There's a difference between answering `no' and failing to answer.
    But, the point I'm arguing for works given this distinction.
    There's no real different.
    I mean, the conditional is either true or false.
    But, it's possible that the falsity of the conditional has a role where the truth of the conditional does not.
      }
    \end{itemize}

    Specifically:
    \begin{enumerate}[label=]
    \item
      \begin{itemize}
      \item[\(+\)]
        A positive answer to~\qzS{} answers, in part, why an agent does conclude \(\pv{\phi}{v}\) from \(\Phi\).
      \item[\(-\)]
        Absence of a positive answer to~\qzS{} answers, in part, why an agent does not conclude \(\pv{\phi}{v}\) from \(\Phi\).
      \end{itemize}
    \end{enumerate}
  \end{idea}

  Need two things.

  Support.
  Concluding.

  \autoref{cha:zSpA} will tie positive answers to \fc{1}.
  Hence, \support{} via \autoref{cha:fcs}.

  \autoref{cha:zSpAwhy} will work through the move to failure to conclude in detail.
  Though,~\autoref{cha:clarification:sec:dis-and-dev} sketched a broad outline.

  Two reasons.

  First, identifying type of \scen{}.
  As mentioned \hyperref[qzS:intro:qualification]{above}, neutral.
  In this respect, as neutral as possible.
  For a \scen{} to be the case, don't need failure of \issueConstraint{}.
  Additional cases in \autoref{cha:zS:sec:question:illu}.

  Second, leverage the agent's perspective.
  This is not essential, but will help clarify a key point of the argument.
  Delayed until {\color{blue} ???}.

  Though, stress again, \autoref{prop:qzS-answers-why} is an existential.
\end{note}

\subsection{Scenarios}
\label{cha:zS:sec:question:scenarios}

\begin{note}
  \autoref{cha:zS:sec:question:illu} will contain additional \illu{1} of positive and negative answers to~\autoref{question:zs}, and in particular \illu{1}.
\end{note}

\begin{note}
  Hence, \autoref{illu:lost-key}, interesting instance of negative answer.
  And,~\autoref{scen:squish}, interesting instance of positive answer.

  In both cases, I identified a \requ{}.

  Of course, \qzS{} is stronger, so raises issue of further \requ{3} with respect to~\autoref{scen:squish}.
  In particular, other rules of logic.
  Etc.
  Witnessed conjunction, but not the soundness of conjunction.
  Sure, and various others.
  Though, important to keep in mind \requ{}.
\end{note}

\begin{note}[\autoref{illu:sketch:prop-logic}]
  Likewise, \autoref{illu:sketch:prop-logic} involves an agent concluding some sentence \(A\) is a syntactic theorem of propositional logic via a formula derivation.%
  \footnote{
    Abstractly, \autoref{illu:gist:roots} is a case where the agent would not conclude \(\pv{\phi}{v}\) from \(\Phi\) if the agent failed to conclude \(\pv{\psi}{v}\) from \(\Psi\), where \(\phi\) is distinct from \(\psi\) and \(\Phi\) is distinct from \(\Psi\).
    Soundness (and completeness) relates syntactic and semantic theorems of propositional logic, but these are distinct, as may be observed by considering, for example, a logic which is incomplete, or an unsound proof system.
  }

  And, when concluding \(A\) is a syntactic theorem, the agent observes that \(A\) is a syntactic theorem only if \(A\) is also a semantic theorem (from soundness).

  In other words, if the agent attempt to show \(A\) is true under an arbitrary valuation and failed, the agent would not conclude \(A\) is a syntactic theorem.

  Only if semantic proof.
  Syntactic proofs, at least in my experience, may be out of reach.
  However, semantic proofs, often straightforward.
\end{note}

\begin{note}
  Similarly, \autoref{illu:gist:roots}.
  Factorisation isn't too difficult.

  \autoref{illu:sketch:math}.
  \(345 \div 15 = 23\) \emph{only if} \(23 \times 15 = 345\).

  Agent has the opportunity, and current result looks good.
\end{note}

\begin{note}
  Absence of \requ{}.
  \autoref{scen:trip-to-zoo}.

  Same with initial \scen{0}.
  Testimony.
\end{note}

\subsection[Visualisation]{Visualisation of what is at issue when asking \qzS{}}

\begin{note}[Naming]
  Our choice of the term `\zgb{0}' is metaphorical.
  \zgb{2} is a family of flower plants which, typically, have the appearance of a single stem with no branches.
  If one starts just before the flower and works back down the stem, one will not find a branch which, if taken, would lead to a different flower.
  In metaphor, if one starts with an agent's perspective prior to the agent concluding \(\pv{\phi}{v}\) from \(\Phi\) and~\autoref{question:zs} has a negative answer with respect to \(\pvp{\phi}{v}{\Phi}\), then one will not find a branch which leads to a different conclusion.

  I have some doubts as to whether or not this metaphor really works, but some term is required.
  `Palm-tree-support', or `Arecaceae-support' would also work.
\end{note}

\begin{figure}[h]
  \centering
  \begin{tikzpicture}
    \node (origin) at (0,0) {};
    \node (psiSplit) at (1,0) {};
    \node (phiSplit) at (4,0) {};
    %
    \node[anchor=west] (Phi) at  (0,0)  {\(\Phi\)};

    \node[anchor=west] (psiV) at  (6,-1)  {\(\pvp{\psi}{v'}{\Psi}\)};
    \node[anchor=west] (psiNv) at (6,-2) {\(\pvp{\psi}{\{\overline{v'}, ?\}}{\Psi}\)};
    % \node[anchor=west] (psiQ) at (6,-3) {\(\pvp{\psi}{?}{\Psi}\)};
    %
    % \node[anchor=west] (psiVPhiV) at (9,-1) {\(\pv{\phi}{v}\)};
    \node[anchor=west] (psiNvPhiU) at (10,-2) {\(\pv{\phi}{\{\overline{v},?\}}\)};
    % \node[anchor=west] (psiQPhiU) at (9,-3) {\(\pv{\phi}{\{\overline{v},?\}}\)};
    %
    \node[anchor=west] (phiQ) at (10,1) {\(\pv{\phi}{\{\overline{v}, ?\}}\)};
    % \node[anchor=west] (phiNv) at (10,2) {\(\pv{\phi}{\overline{v}}\)};;
    \node[anchor=west] (phiV) at (10,0) {\(\pv{\phi}{v}\)};
    %
    \draw[-]  (Phi) -- (phiV);
    %
    % \path[-,dotted] (phiSplit) edge [out=0, in=180] (phiNv);
    \path[-,dotted] (phiSplit) edge [out=0, in=180] (phiQ);
    %
    \path[-, dashed] (psiSplit) edge [out=0, in=180] (psiV);
    \path[-, dotted] (psiSplit) edge [out=0, in=180] (psiNv);
    % \path[-, dotted] (psiSplit) edge [out=0, in=180] (psiQ);
    %
    \draw[-,dashed] (psiV) edge [out=0, in=180] (phiV);
    \draw[-, dotted] (psiNv) edge (psiNvPhiU);
    % \draw[-, dotted] (psiQ) edge (psiQPhiU);
    \end{tikzpicture}
    \caption{Visualisation of when \qzS{} has a positive answer.}
    \label{fig:csN:illu:overview}
  \end{figure}

\begin{note}[Figure]
  \autoref{fig:csN:illu:overview} provides a rough visualisation of when \qzS{} has a positive answer.

  The flat line captures the agent's reasoning, which concludes with \(\pv{\phi}{v}\).
  In concluding \(\pv{\phi}{v}\) the agent rules out two possibilities with respect to \(\phi\).
  First, that \(\phi\) does not have value \(v\), indicated by \(\pv{\phi}{\overline{v}}\).
  Second, that the agent does not assign any value to \(v\), indicated by \(\pv{\phi}{?}\).
  Prior to concluding \(\pv{\phi}{v}\), the agent's reasoning may have branched to either alternative path, but as the agent has concluded \(\pv{\phi}{v}\), neither path is viable, and hence both paths are represented with a dashed line.

  So far, we have seen only that the agent has concluded \(\pv{\phi}{v}\).

  We now consider some proposition-value-premises pairing \(\pv{\psi}{v'}{\Psi}\) such that if the agent were to fail to conclude \(\pv{\psi}{v'}\) from \(\Phi\), the agent would not conclude \(\pv{\phi}{v}\) from \(\Phi\).

  Intuitively, the dotted arrows from the various combinations of \(\psi\) and \(\{v',\overline{v'},?\}\) read, from top to bottom:
  \begin{itemize}
  \item If \(\phi\) has value \(v\) then the agent may conclude \(\pv{\psi}{v'}\) from \(\Psi\), and:
  \item If the agent concludes \(\psi\) has some value \(\overline{v'}\) from \(\Psi\), then the agent either concludes \(\phi\) has some value other than \(v\), or the agent fails to reach a conclusion regarding \(\phi\) from \(\Phi\).
    Both options are combined via the shorthand \(\pv{\phi}{\{\overline{v},?\}}\).
  \item
    And, likewise if the agent fails to conclude \(\pv{\psi}{v'}\) from \(\Psi\).
  \end{itemize}

  With respect to concluding, observe that prior to ruling out alternative branches with respect to \(\pv{\phi}{\{\overline{v},?\}}\), the agent may have reasoned about whether \(\psi\) has value \(v\).
  And, from the agent's perspective, \(\phi\) has value \(v\) only if \(\psi\) has value \(v'\).
  If \(\psi\) does not have value \(v'\), then either \(\phi\) does not have value \(v\), or the agent's reasoning would not conclude with a value for \(\phi\), indicated by \(\pv{\phi}{\{\overline{v},?\}}\).

  Hence, prior to concluding \(\pv{\phi}{v}\), the agent has concluded \(\pv{\psi}{v'}\).
\end{note}

\begin{note}
  Broadly, then, we may say that an agent has {\color{red} particular kind of conclusion} for \(\pv{\phi}{v}\) just in case when concluding \(\pv{\phi}{v}\) it is not the case that the agent's reasoning would have branched to a different conclusion with respect to \(\phi\).

  However, the visualisation of~\autoref{fig:csN:illu:overview} and this broad statement of {\color{red} positive answer to \qzS{}} are a little too broad.
  For, we are only interested in proposition-value pairs guaranteed by \(\phi\) having value \(v\).
  {\color{red} positive answer to \qzS{}} is not global with respect to all proposition-value pairs that the agent may have reasoned about, but local to those guaranteed by the proposition.
\end{note}

\subsection{Recursion}
\label{cha:zS:sec:question:recursion}

\begin{note}
  With this observation in hand, wonder whether \qzS{} should be recursive.

  Two ways in which recursive.
  First, foundations.
  Second, other instances of concluding.
\end{note}

\begin{note}
  Foundations splits into two.
  When doing the reasoning.
  Now, at present.

  When doing the reasoning.

  Ideally, an agent concluding \(\pv{\phi}{v}\) is an instance of XXX  just in case the agent would not have reasoned to a different conclusion if they were to reason first about any other proposition-value pair.

  If interested, define property.
  However, placing this as a requirement would be difficult.
  Would need to dramatically enlarge \scen{1}.
  And, no need to do this.

  In this respect, what is at issue when asking \qzS{} is distinct from issues raised with respect to failure of warrant transmission, which often focus on how (or why) the agent obtained some premise of their present instance of reasoning (i.e.\ how the agent obtained some member of \(\Phi\) when concluding \(\pv{\phi}{v}\) from \(\Phi\)).%
  \footnote{
    \citeauthor{Wright:2011wn}'s (\citeyear{Wright:2011wn}) revised template:
    \begin{quote}
      Where A entails B, a rational claim to warrant for A is not transmissible to B if there is some proposition C such that:
      \begin{enumerate}[label=(\roman*), noitemsep]
      \item
        The process/state of accomplishing the relevant putative warrant for A is subjectively compatible with C’s holding: things could be with one in all respects exactly as they subjectively are yet C be true
      \item
        C is incompatible (not necessarily with A but) with some presupposition of the cognitive project of obtaining a warrant for A in the relevant fashion, and
      \item
        Not-B entails C%
      \mbox{ }\hfill\mbox{(\citeyear[93]{Wright:2011wn})}
      \end{enumerate}
    \end{quote}
    Difficulty with the process, however, of interest is transmission of warrant from A to B.
    Hence, the agent has accomplished the relevant putative warrant for A.

    Or, consider the fourth type of dependence between premise and conclusion considered (but not endorsed) by \textcite{Pryor:2004ws}:

  \begin{quote}
    [Type 4] dependence between premise and conclusion is that the conclusion be such that evidence \emph{against it} would (to at least some degree) undermine the kind of justification you purport to have for the premises.%
    \mbox{}\hfill\mbox{(\citeyear[359]{Pryor:2004ws})}
  \end{quote}
  Premises!

  \nocite{Weisberg:2012vs}
  Closer to \citeauthor{Weisberg:2010to}'s (\citeyear{Weisberg:2010to}) account of bootstrapping.
  However, implicit circularity.
  And, circularity is not at issue with \qzS{}.
  In particular, \requ{}, and in particular~\ref{idea:requ:pool:int}.
  }

  At present, turns on \requ{}.
  Whether repeating the reasoning matters.

  Recall, squish.
  \requ{}.
  Memory may have some role, but present reasoning.
\end{note}

\begin{note}
  Other instances of concluding.
  That of a \requ{}.
  \qzS{} doesn't extend, though \qzS{} could be asked.

  Broad motivation support this.
  And, again, \requ{0}.
  Fail to get \requ{}, then fail to get \(\pv{\phi}{v}\) from \(\Phi\).
  So, fail to get \(\pv{\chi}{v''}\) from \(X\), then extends.

  Though, suspect in most cases, no \requ{} from the agent's perspective.
  In part, because details of how the agent would conclude.
  Interesting cases arise from steps of reasoning.
  Like squish.

  In pursuit of ideal, push the details.
  In turn, do the reasoning.
  However, far from ideal.
  And, still,~\autoref{scen:squish} remains.
  From my perspective, would not fail to conclude.
\end{note}

\subsection{\zSXp{} and additional motivation for \qzS{}}
\label{cha:zS:sec:question:zS}

\begin{note}
  So, if agent concludes, then from perspective, no other conclusion.
  Conclude when \support{} from agent's perspective.
  \zSN{} as a \emph{kind} of \support{}.

  \begin{definition}[\zS{}]
    \label{idea:zS}
    For an agent \vAgent{}, and proposition-value-premises pairing \(\pvp{\phi}{v}{\Phi}\):
    \begin{enumerate}[label=]
    \item
      \begin{enumerate}[label=]
      \item
        \zS{} held between \(\pv{\phi}{v}\) and \(\Phi\) when \vAgent{} concluded \(\pvp{\phi}{v}{\Phi}\) from \(\Phi\).
      \end{enumerate}
    \item \emph{If and only if}:
    \item
      \begin{enumerate}[label=]
      \item
        \qzS{} had a \emph{positive} answer when \vAgent{} concluded \(\pv{\phi}{v}\) from \(\Phi\).%
        \footnote{
          Or, alternatively, \nqzS{} had a \emph{negative} answer when \vAgent{} concluded \(\pv{\phi}{v}\) from \(\Phi\).
        }
      \end{enumerate}
    \item \emph{If and only if}:
      \begin{enumerate}[label=]
      \item
        When concluding \(\pv{\phi}{v}\) from \(\Phi\), from \vAgent{}' perspective:
        \begin{itemize}
        \item
          For any proposition-value-premises pairing \(\pvp{\psi}{v'}{\Psi}\):
          \begin{itemize}
          \item[\emph{If}:]
            \begin{enumerate}[label=\alph*., ref=(\alph*)]
            \item
              \(\pvp{\psi}{v'}{\Psi}\) is a \requ{} of concluding \(\pv{\phi}{v}\) from \(\Phi\).
            \end{enumerate}
          \item[\emph{Then}:]
            \begin{enumerate}[label=\alph*., ref=(\alph*), resume]
            \item
              \vAgent{} would conclude \(\pv{\psi}{v'}\) from \(\Psi\), if \vAgent{} were to attempt to conclude \(\pv{\psi}{v'}\) from \(\Psi\).
            \end{enumerate}
          \end{itemize}
        \end{itemize}
      \end{enumerate}
    \end{enumerate}
    \vspace{-\baselineskip}
  \end{definition}

  \autoref{idea:zS} identifies \zS{} holding between \(\pv{\phi}{v}\) and \(\Phi\) when an agent concludes \(\pv{\phi}{v}\) from \(\Phi\) in terms of positive answers to \qzS{} at the time of concluding.
  The second equivalence simple restates what a positive answer to \qzS{} amounts to at the time of concluding.

  Our interest is with \qzS{} rather than the type of \support{} that would follow given a positive answer to \qzS{}.
  Hence, \zS{} is only of passing interest.

  However, \zS{} suggests additional motivation for \qzS{}.
  For, \zS{} is identified with positive answers to \qzS{}.
  And, in this respect, \zS{} is solely concerned with the agent's perspective when concluding \(\pv{\phi}{v}\) from \(\Phi\).

  However, we may drop the qualification `from \vAgent{}' perspective' from the definition of \zS{}, independently motivate the resulting kind of \support{}, and in turn understand \qzS{} in terms whether \emph{that} kind of \support{} holds between \(\pv{\phi}{v}\) and \(\Phi\), from the agent's perspective.

  Breaking this suggestion down into a series of steps:
  \begin{enumerate}[label=, noitemsep]
  \item
    First, we will define the suggested variant of \zS{}.
  \item
    Then, we will provide a `lazy' argument for variant of \zS{}.
  \item
    Finally, we will motivate, from an agent's perspective, concern regarding the variant type of \support{} when concluding \(\pv{\phi}{v}\) from \(\Phi\).
  \end{enumerate}
\end{note}

\subsubsection{\zSX{}}

\begin{note}
  We term the variant type of \support{} `\zSX{}'.
  The definition is as follows:
  \begin{definition}[\zSX{}]
    \label{idea:zSX}
    For an agent \vAgent{}, and proposition-value-premises pairing \(\pvp{\phi}{v}{\Phi}\):
    \begin{enumerate}[label=]
    \item
      \begin{enumerate}[label=]
      \item
        \zSX{} held between \(\pv{\phi}{v}\) and \(\Phi\) when \vAgent{} concluded \(\pvp{\phi}{v}{\Phi}\) from \(\Phi\) for \vAgent{}.
      \end{enumerate}
    \item \emph{If and only if}:
    \item
      \begin{enumerate}[label=]
      \item
        When \vAgent{} was concluding \(\pv{\phi}{v}\) from \(\Phi\):
        \begin{itemize}
        \item
          For any proposition-value-premises pairing \(\pvp{\psi}{v'}{\Psi}\):
          \begin{enumerate}[label=]
          \item[\emph{If}:]
            \begin{enumerate}[label=\alph*., ref=(\alph*)]
            \item
              \label{question:zs:option}
              \(\pvp{\psi}{v'}{\Psi}\) was a \requ{} of concluding \(\pv{\phi}{v}\) from \(\Phi\).
            \end{enumerate}
          \item[\emph{Then}:]
            \begin{enumerate}[label=\alph*., ref=(\alph*), resume]
            \item
              \label{question:zs:may-fail}
              \vAgent{} would have concluded \(\pv{\psi}{v'}\) from \(\Psi\), if \vAgent{} were to have attempted to conclude \(\pv{\psi}{v'}\) from \(\Psi\).
            \end{enumerate}
          \end{enumerate}
        \end{itemize}
      \end{enumerate}
    \end{enumerate}
    \vspace{-\baselineskip}
  \end{definition}
\end{note}

\begin{note}
  \color{red}
  Strip out agent's perspective, and plausible account of support, relative to an agent.

  \zS{} and \qzS{}, then, internalise this.

  Relative to agent.

  distinction between classical and intuitionistic mathematicians.
\end{note}

\subsubsection{Additional motivation}

\begin{note}
  \zSX{}.

  Turn to lazy argument for interest in \zSX{}.
  Turns on the observation that \zSX{} requires that the agent would not reason to a different conclusion.
  Provide an argument that, in general, not possible for an agent to reason to conflicting conclusion.

  This differs from motivation given.
  In \scen{1}, considered the agent's perspective, didn't matter why.
  Here, lazy motivation.
\end{note}

\begin{note}
  Sketch of a lazy argument.
  Three premises:
  \begin{enumerate}
  \item
    \label{lazy:evidence-constraint}
    If evidence for \(\pv{\phi}{v}\), then no evidence for conflicting \(\pv{\psi}{v'}\).
  \item
    \label{lazy:evidence}
    Support only if evidence.
  \item
    \label{lazy:reason}
    Support to \(\pv{\phi}{v}\) from \(\Phi\) only if possible to conclude from \(\pv{\phi}{v}\) to \(\Phi\) from present perspective.
  \end{enumerate}

  So, from first, evidence doesn't support conflicting.
  Second, evidence determines support.
  Third, evidence for agent means reasoning.


  So, from \requ{}.
  If not possible to reason to \(\pv{\psi}{v'}\) from \(\Psi\), then not possible to reason from \(\pv{\phi}{v}\) to \(\Phi\).
  Here, possibility, quantifies over opportunities.
  Then, via \ref{lazy:reason}, weaken the consequent.
  If not possible to reason to \(\pv{\psi}{v'}\) from \(\Psi\), then not \support{} from \(\pv{\phi}{v}\) to \(\Phi\).
  If support to \(\pv{\phi}{v}\) from \(\Phi\), then possible to conclude \(\pv{\psi}{v'}\) from \(\Psi\).
  But, agent also has the opportunity.
  So, agent would conclude.

  So, premise seems fine.

  \citeauthor{Way:2016vq} (\citeyear{Way:2016vq}) terms the `reasoning constraint':

  \begin{quote}
    \begin{enumerate}
    \item[RC] Reasons for you to \(\varphi\) must be considerations from which you could reason to \(\varphi\)-ing.%
      \mbox{ }\hfill\mbox{(\citeyear[806]{Way:2016vq})}
    \end{enumerate}
  \end{quote}
  As \citeauthor{Way:2016vq} notes, wide support. (\citeyear[806]{Way:2016vq})

  For example,%
  \footnote{Mentioned by \textcite{Way:2016vq}.}
  we have the following statement from~\citeauthor{Williams:1979wi} (\citeyear{Williams:1979wi}):
  \begin{quote}
    \begin{enumerate}[label=(\roman*)]
      \setcounter{enumi}{3}
    \item internal reason statements can be discovered in deliberative reasoning.%
      \mbox{ }\hfill\mbox{(\citeyear[19]{Williams:1979wi})}
    \end{enumerate}
  \end{quote}
\end{note}

\begin{note}
  The difficulty with this argument is premise.
  It's the almost-converse, of basic premise regarding support.
  Problem is, difficulty if the agent concludes.
  For, with first premise get support.
  So, this can't be right.
  Well, no coherent notion of support.

  Or, stick to reasons.
  But, well, responding to reasons, maybe.
\end{note}

\begin{note}
  So, granting the lazy argument, we get stronger instance of \zS{}.
  Now, \qzS{} internalises this.
  Hence, motivation for \qzS{} need not fall on \scen{}.

  Lazy argument is lazy.
  I have done nothing to persuade you other than cite a handful reference which provide isolated support for each of the premises.

  Though, the point is broader.
  Plausible that support only if reasoning is unique.
  Internalise.
  \qzS{}.
  Any argument for the core idea will work.
  And, internalisation step seems good.
\end{note}

\begin{note}
  One upshot here is that instances where \qzS{} matters are common.
  Allowed, to avoid overstepping, \support{} to fail this constraint.
  Though, plausible that \qzS{} just is asking whether the agent has \support{}.
\end{note}

\subsection{Positive answers and ability}
\label{sec:posit-answ-abil}

\begin{note}
  Presented \scen{1}.

  Broad argument for positive answers in certain cases.
\end{note}

\begin{note}
  An interesting observation here is that in certain this all arises, to a certain extent, because of general abilities.
  General ability spans multiple different proposition-value-premises pairings.
  Hence, all of these function as \requ{1}, so long as the agent has the option.

  General ability spans multiple different proposition-value-premises pairings.
  Hence, all of these function as \requ{1}, so long as the agent has the option.

  \begin{itemize}
  \item
    General and specific abilities.
  \item
    Answers to why, then.
    Note, here, that opportunity is interesting.
    The whole conjunction of all instance of the general ability is plausibly not a \requ{}.
    However, all that's needed is the \emph{individual} instances, and for these to raise a problem.
  \item
    The point is, \requ{1} for any general ability, and these are also \requ{1} for main pairing.
    (%
    Note --- or perhaps emphasise --- here, that the problem is \emph{not} recursive.
    Instead, the problem is about the spread.%
    )
  \item
    Here, then, ability is both the problem and the answer.
    What's interesting is the way in which ability functions.
    It's not merely \emph{that} the agent has the ability.
    Instead, it \emph{is} the ability.
  \end{itemize}
\end{note}

\begin{note}
  So, the way in which past reasoning relates is by ensuring that the agent would reach the same conclusion.
  About the agent's reasoning.
  \emph{How} rather than \emph{that}.

  Look, what we are getting is that the agent would conclude.
  If something were to happen, then some action would be performed.
  There's no distinction between the answer and performing the act, roughly.
  Or, better put, the answer \emph{is about present reasoning}.
  Answer states that in present reasoning, would not fail.

  It is about the agent's present epistemic state, and in particular what the agent's present epistemic state is capable of.

  In other words, ability.
  What answers is ability, in the sense that ability iff would.

  This is very important to the understanding of \fc{}.

  And, I kind of want to have ability as a gloss, while focusing on \fc{} to avoid going into ability in too much detail.

  So, positive answer, then it's the pairing \emph{being} a \fc{}.
  (I should always use this instance of the copula.)
\end{note}

\section{Scraps}
\label{sec:scraps}

\begin{note}
  \emph{However}, caution.
  For, as we have seen with testimony, it may be the case that status of a premises blocks a \requ{}.
  And, the argument given relies on the existence of a \requ{}.
  So, it may be the case that past reasoning blocks a \requ{}.
  Still, here, only need to deny this.
  Not saying that in every case agent's present reasoning is given priority.
  (Indeed, consider cases of being somewhat impaired, e.g., via exhaustion.
  Indeed, exhaustion is interesting.
  Basic consistency checks.
  Should be the case that conclude A, but just concluded \emph{not}-A, or something like this\dots)
  Rather, denying that past continues to secure in all instances.
  So, just need the potential to revise perspective on any previous conclusion.
\end{note}



%%% Local Variables:
%%% mode: latex
%%% TeX-master: "master"
%%% End:
