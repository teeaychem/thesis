\chapter{\zSN{2}}
\label{cha:zS}

\begin{note}[Intro, locating]
  Started with \cScen{1}.
  Previous chapter, \fc{1}.
  Now turn to \zSN{}.
  {
    \color{red}
    Roughly, whether or not check on reasoning is a \fc{0} --- though with some restrictions.
    Doesn't hold for all \cScen{1}.
    However, at least some.
  }

  Our goal motivate a negative resolution to~\issueConstraint{}.
  And, as sketched in~\autoref{cha:sketch}, \zSN{} particular question.
  From the agent's perspective.

  Intuitively:
  Positive answer answers why.

  However, for the present chapter, motivate only whether the question has positive answers.
  Connexion to \qWhyV{} follows in \autoref{cha:zSpA} and \autoref{cha:zSpAwhy}.
  \autoref{cha:zS:sec:question:illu} will provide additional \illu{1} of \zS{}.
  Key idea, \autoref{cha:clarification:sec:dis-and-dev}, though don't need to think about the relation for this chapter.
\end{note}

\begin{note}
  In order to establish tension we narrow our attention to when concluding \(\pv{\phi}{v}\) concluding \(\pv{\phi}{v}\) involves the agent establishing a particular property with respect to \(\pv{\phi}{v}\).
  We term the property `\zSN{0}', or `\zS{}' for short.
\end{note}

\begin{note}
  \begin{itemize}
  \item
    \requ{3}.
    Link between \cScen{1} and instance of reasoning.
  \item
    With this, \qzS{}, a question.
  \end{itemize}
\end{note}

\section{Lost keys and logic proofs}
\label{sec:lost-keys}

\begin{note}
  Previous chapter, \fc{1}.
  Potential event in which agent concludes.

  \fc{3} are about what the agent may reason to.

  Now, turn to broader, whether there is something incompatible.
  Alternative conclusion.
\end{note}

\begin{note}
  Use the \illu{} to provide a general introduction.
\end{note}

\begin{note}
  \begin{scenario}[Lost keys]
    \label{illu:lost-key}
    I think I might have lost my keys.
    I usually leave place my keys on the right side of my desk, next to a copy of~\citeauthor{Vickers:1989tr}'s~\citetitle{Vickers:1989tr} which I've been saving for a rainy day.
    And, my keys aren't there.

    I've searched on the desk, under the desk, and beside the desk.
    And, I haven't found my keys.

    Still, I haven't (yet, at least) \emph{concluded} that I've lost my keys.

    For, there might still be some place I haven't looked.
    If I think a little harder a figure out where that place is, I would conclude my keys might be in that place.
    And, if my keys are in that place, then they're not lost.
    So, I might conclude that my keys aren't lost, which would conflict with concluding that my keys are lost.
  \end{scenario}

  You may disagree with the tension I see in~\autoref{illu:lost-key}.
  Perhaps it's fine to conclude my keys are lost while allowing for the possibility that they're some place I haven't yet thought of.
  However, there's tension for me.
  `I've lost my keys, but they might be under that book' feels bad to me, and to me the badness extends to `I've lost my keys, but they might in that place I haven't yet considered'.

  Though, my goal is only to convince you that my refusal to conclude I've lost my keys makes sense.
  Your perspective on truth conditions for the sentence `I've lost my keys' may be different, but I think my perspective is at least intelligible.
\end{note}

\begin{note}
  Granting the intelligibility of~\autoref{illu:lost-key}, our interest is with the following sketch:
  \begin{quote}
    For the agent's perspective, there is some \(\pvp{\psi}{v'}{\Psi}\) such that from their perspective:
    The agent would only conclude \(\pv{\phi}{v}\) from \(\Phi\), if they would conclude \(\pv{\psi}{v'}\) from \(\Psi\).

    And, as the agent entertains the possibility of failing to conclude \(\pv{\psi}{v'}\) from \(\Psi\), they do not conclude \(\pv{\phi}{v}\) from \(\Phi\).
  \end{quote}

  In short, the agent, from their perspective, thinks the may reason to some other conclusion, and therefore does not conclude.

  Filling in the details of the abstract sketch:
  \begin{itemize}[noitemsep]
  \item
    I am the agent.
  \item
    \(\phi\) is the proposition: `I've lost my keys'.
  \item
    \(\psi\) is a some proposition: `My keys are not in location \(l\)'
  \item
    Both \(v\) and \(v'\) are the value: `True'.
    And,
  \item
    The pools of premises \(\Phi\) and \(\Psi\) are left unspecified.
  \end{itemize}

  \autoref{illu:lost-key} is chosen to introduce this phenomena as I seem to encounter the pattern every time I think I've lost something, whether keys, books, or files.
  After some searching I feel I should have accumulated enough evidence to conclude the item is lost.
  However, the item isn't lost (from my perspective, at least), while there remains a place to check, and experience shows I eventually think of a place to check and, more often than not, the item is there.
\end{note}

\begin{note}
  Note, however, that the \scen{} focuses on reasoning.
  And, in particular, the failure to reason to some proposition-value pair which would follow from concluding some other proposition-value pair.

  Hence, the sketch does not apply to black ravens.
  I wouldn't conclude all ravens are black if I saw a white raven.
  However, it's not possible to reason to seeing a white raven.%
  \footnote{
    Abstractly, at issue in~\autoref{illu:lost-key} is the possibility of failing to a reason to some proposition-value pair given \emph{present} information, rather than the possibility of failing to a reason to some proposition-value pair given \emph{new} information.

    To the extent that problems of induction arise from receiving new information, what is at issue is distinct.
    (See~\textcite{Henderson:2020wb} for more on the problem of induction.)
  }
\end{note}

\begin{note}
  Sketch, failing to conclude.
  Consider variation:
  \begin{quote}
    For the agent's perspective, there is some \(\pvp{\psi}{v'}{\Psi}\) such that from their perspective:
    The agent would only conclude \(\pv{\phi}{v}\) from \(\Phi\), if they would conclude \(\pv{\psi}{v'}\) from \(\Psi\).

    And, as the agent \emph{does not} entertain the possibility of failing to conclude \(\pv{\psi}{v'}\) from \(\Psi\), the agent concludes \(\pv{\phi}{v}\) from \(\Phi\).
  \end{quote}
\end{note}

\begin{note}
  Return to~\autoref{scen:squish}:

  \scenarioPLSquish*

  \scen{0} focused on the non-standard `Squish'-elimination rule of inference and the possibility of concluding `Squish'-elimination is a sound rule of inference.

  Uncommon, but enough to memorise the rule.
  Still, I consider my general understanding of propositional logic more important than memory.
  And, if failed, then would not consider sound.

  Filling in the details of the second abstract sketch:
  \begin{itemize}[noitemsep]
  \item
    I am the agent.
  \item
    \(\phi\) is the proposition: `\((P \rightarrow Q) \rightarrow P, Q \vdash P \land Q\)'.
  \item
    \(\psi\) is a some proposition: `Squish elimination is sound'
  \item
    Both \(v\) and \(v'\) are the value: `True'.
    And,
  \item
    The pools of premises \(\Phi\) and \(\Psi\) are left unspecified.%
    \footnote{
      Note, premises of reasoning.
      Distinct from premises of deduction.
    }
  \end{itemize}

  Still, I do not entertain the possibility of failing to show sound.

  With some luck, uncommon and you will witness the conclusion similar to what I would, if I chosen to reason.%
  \footnote{
    To preserve the integrity of the \illu{0}, the proof on page~\pageref{squish-elimination-proof} was considered \emph{after} concluding the (syntactic) consequence via Squish-elimination.
  }
\end{note}

\begin{note}
  So, as with lost keys.
  I wonder about derived rules of inference.
  Clearly a problem.
  Understanding of propositional logic is good, better than memory.
  But, same understanding, sound.
\end{note}

\paragraph{Analysis}

\begin{note}
  Split these cases into three components.
  First, opportunity.
  Second, subjunctive ability.
  Third, present ability.

  So, have the opportunity to do some reasoning, about the keys, or about the rule of inference.
  Following terminology here, opportunity doesn't really mean anything.

  So, this gets strengthened.
  Subjunctive.
  If wouldn't fail to conclude, then general block.

  Now, the opportunity is present, so the hypothetical failure is brought to fact.

  Need an actuality entailment to hold!

  This is neat!
\end{note}

\begin{note}[Entailment]
  \textcite{Alxatib:2019wf} provides the following account of actuality entailments.
  \begin{quote}
    Actuality Entailments (AEs) are inferences from premises that appear to be modal, like (1a), but their content is that the modality is effectuated in the evaluation world --- (1b).

    \begin{enumerate}[label=(\arabic*)]
    \item
      \begin{enumerate}[label=\alph*.]
      \item Pierre a dû \hspace{26pt} prendre le \hspace{3.5pt} train \newline
        Pierre had.to.\textsc{pfv} take \hspace{14pt} the train\newline
        \hspace{-4pt} ‘Pierre had to take the train'
      \item \emph{Inference}: Pierre took the train.\nolinebreak
    \mbox{}\hfill\mbox{(\citeyear[701]{Alxatib:2019wf})}
      \end{enumerate}
    \end{enumerate}
  \end{quote}

  Note, the reading of `had' in (1a) is `unambiguously deontic' (\citeyear[703]{Alxatib:2019wf}).
  English paraphrase does not carry the same entailment, but does seem to carry a corresponding implicature.

  Follows \textcite{Bhatt:1999wq} (implication, to avoid commitment to entailment).

  \begin{quote}
    We show that the English ability modal was able to is ambiguous between two readings which can be paraphrased as `managed to' and `had the ability to'.

    \dots

    The two readings associated with be able to allow different interpretive possibilities for indefinite/bare plural subjects.

    \begin{enumerate}[label=(\arabic*), ref=(\arabic*)]
      \setcounter{enumi}{317}
    \item A fireman was/Firemen were able to eat five apples.
      \begin{enumerate}[label=\alph*., ref=\alph*.]
      \item Yesterday at the apple eating contest, a fireman was/firemen were able to eat five apples. (Past episodic, actuality implication, existentially interpreted subject)
      \item In those days, a fireman were/firemen were able to eat five apples in an hour (Generic, no actuality implication, generically interpreted subject)\nolinebreak
        \mbox{}\hfill\mbox{(\citeauthor[172--173]{Bhatt:1999wq})}
      \end{enumerate}
    \end{enumerate}
  \end{quote}

  \citeauthor{Alxatib:2019wf} highlights the difficulty with actuality entailments:

  \begin{quote}
    AEs are surprising; if we assume that modals attribute their propositional argument to potentially non-actual worlds, something must be special to AE-licensers that leads to the inference of actuality.

    Whatever that special feature is, its effect is complex, and it interacts in nontrivial ways with other phenomena of theoretical interest.\nolinebreak
    \mbox{}\hfill\mbox{(\citeyear[701]{Alxatib:2019wf})}
  \end{quote}

  {
    \color{red}
     \citeauthor{Bhatt:1999ud} goes further, arguing that actuality entailments are cannot be due to ability attribution \citeyear[\S4.2]{Bhatt:1999ud}.
  \cite{Hacquard:2006to} argues against this.
  \citeauthor{Hacquard:2006to} also (seems to) link to events.

  \textcite{Werner:2011tp} suggests that there's a hierarchy.
  Potentive entailment applies to opportunities.
  So, potential entails ability, but need not be the case that ability requires opportunity.
  }
\end{note}

\paragraph{Undercutting defeater}

\begin{note}
  Undercutting defeater.%
  \footnote{
    We take the following sketch from \textcite{Worsnip:2018aa}:
  \begin{quote}
    Undercutting defeaters, which are easiest to think of in the context of the attitude of belief, are supposed to be considerations that undermine the justification of a belief in a proposition p not necessarily by providing (sufficient) positive evidence to think that p is false, but rather merely by suggesting (perhaps misleadingly) that one’s reasons for believing p are no good, in a way that neutralizes or mitigates their justificatory or evidential force.%
    \mbox{}\hfill\mbox{(\citeyear[29]{Worsnip:2018aa})}
  \end{quote}
  }

  Consider the following illustration provided by \citeauthor{Pollock:1987un}%
  \footnote{
    \citeauthor{Pollock:1987un} defines undercutting defeaters as follows:
    \begin{quote}
      R is an \emph{undercutting defeater} for P as a prima facie reason for S to believe Q if and only if
      \begin{enumerate}[label=(UD\arabic*), ref=(UD\arabic*)]
      \item
        P is a reason for S to believe Q and R is logically consistent with P but (P and R) is not a reason for S to believe Q, and
      \item
        R is a reason for denying that P wouldn't be true unless Q were true.%
        \mbox{}\hfill\mbox{(\citeyear[485]{Pollock:1987un})}
      \end{enumerate}
    \end{quote}
    Intuitively, an undercutting defeater for P as a reason for Q because it the defeater denies that Q must be true in order for P to be true.

    Variation:
    \begin{quote}
      R is \emph{undercutting reasoning} with respect to \emph{S} concluding \(\pv{\phi}{v}\) from \(\Phi\) if and only if
      \begin{enumerate}[label=(UR\arabic*), ref=(UR\arabic*)]
      \item
        \(\Phi\) is sufficient for \emph{S} to conclude \(\pv{\phi}{v}\) and R is reasoning which is logically consistent with reasoning from \(\Phi\) but combination of both reasoning from \(\Phi\) and R is not a sufficient for \emph{S} to conclude \(\pv{\phi}{v}\), and
      \item
        R is sufficient for denying that S wouldn't conclude \(\pv{\phi}{v}\) from \(\Phi\) unless \(\pv{\phi}{v}\) followed from \(\Phi\).%
      \end{enumerate}
    \end{quote}
    The first part of this clause establishes that P is a reason, but compatible with other reasons.
    Here, in the case of reasoning, build in that the agent has reasoned, ready to conclude, and from understanding of concluding, implicit that it's problematic.
  }%
  :
  \begin{quote}
    [Undercutting defeaters] attack the connection between the reason and the conclusion rather than attacking the conclusion itself.
    For instance, ``X looks red to me'' is a prima facie reason for me to believe that X is red.
    Suppose I discover that X is illuminated by red lights and illumination by red lights often makes things look red when they are not.
    This is a defeater, but it is not a reason for denying that X is red (red things look red in red light too).
    Instead, this is a reason for denying that X wouldn't look red to me unless it were red.%
    \mbox{}\hfill\mbox{(\citeyear[485]{Pollock:1987un})}
  \end{quote}
  Completing \citeauthor{Pollock:1987un}'s example, it seems that if agent's support for holding that X is red is that `X wouldn't look red to me unless it were red', then the support for X being red provided by appearance is retracted after discovering that X is illuminated by red lights (though it remains possible that X is red).
\end{note}

\begin{note}
  In \citeauthor{Pollock:1987un}'s example, discover that the light is red.
  By parallel, the agent failing to conclude would undercut.
  By contrast, the agent hasn't failed to conclude \(\pv{\psi}{v'}\) from \(\Psi\).
  So, not a direct undercutting defeater.
\end{note}

\begin{note}
  In contrast, \requ{} `attacks' the \emph{reasoning} and the conclusion.
\end{note}

\begin{note}
  Basic point, strengthen the conditional from \cScen{1}.
  Though, impose some limitations\dots

  Intuitively, \fc{} in the good case.
  Though, with some restrictions.
  In \autoref{cha:zS:sec:requs} outline.
\end{note}

\section{\requ{3}}
\label{cha:zS:sec:requs}

\begin{note}
  \color{red}
  From the analysis, opportunity and subjunctive ability.
  Capture in broader terms with the idea of a \requ{}.
\end{note}

\begin{note}[\requ{3}]
  We begin with the idea of a \requ{}, develops when something is a partial check.

  \begin{idea}[A \requ{0}]
    \label{idea:requ}
    For an agent \vAgent{}, and proposition-value-premises pairings \(\pvp{\phi}{v}{\Phi}\) and \(\pvp{\psi}{v'}{\Psi}\):

    \begin{itemize}
    \item
      \(\pvp{\phi}{v'}{\Psi}\) is a \emph{\requ{}} of concluding \(\pv{\phi}{v}\) from \(\Phi\) if:
      \begin{enumerate}[label=\alph*., ref=\named{R:\alph*}]
      \item
        \label{idea:requ:pool}
        \vAgent{} has the opportunity to (attempt to) conclude \(\pv{\psi}{v'}\) from \(\Psi\) such that:
        \begin{enumerate}[label=\roman*., ref=\named{R:a.\roman*}]
        \item
          \label{idea:requ:pool:int}
          It is possible for \vAgent{} to conclude \(\pv{\psi}{v'}\) from \(\Psi\) without concluding \(\pv{\phi}{v}\) from \(\Phi\) as an intermediary step.
        \end{enumerate}
      \end{enumerate}

      \begin{enumerate}[label=\alph*., ref=\named{R:\alph*}, resume]
      \item
        \label{idea:requ:nPsi-nPhi}
        It is the case that:
        \begin{enumerate}
        \item[\emph{If}:]
          \begin{enumerate}[label=\roman*., ref=\named{R:b.\roman*}]
          \item
            \vAgent{} were to fail to conclude \(\pv{\psi}{v'}\) from \(\Psi\) prior to concluding \(\phi\) has value \(v\).
          \end{enumerate}
        \item[\emph{Then}:]
          \begin{enumerate}[label=\roman*., ref=\named{R:b.\roman*}, resume]
          \item
            \vAgent{} would not conclude \(\pv{\phi}{v}\) from \(\Phi\).
          \end{enumerate}
        \end{enumerate}
      \end{enumerate}
    \end{itemize}
    \vspace{-\baselineskip}
  \end{idea}

  With respect to \scen{1}, \requ{} is half the story.
  Tells us how \(\pvp{\psi}{v'}{\Psi}\) would relate to concluding \(\pv{\phi}{v}\) from \(\Phi\).
  \emph{If} the agent were fail, then would not conclude.

  Still an issue of how the subjunctive relates to concluding \(\pv{\phi}{v}\) from \(\Phi\).
  This, \autoref{cha:zS:sec:question}.
\end{note}

\begin{note}
  The idea of \(\pvp{\psi}{v'}{\Psi}\) being a \requ{2} of concluding \(\pv{\phi}{v}\) form \(\Phi\) for some agent is stated without reference to the agent's perspective.
  However, our interest with \requ{1} will be from an agent's perspective.

  Distinction between whether a \requ{} and whether a \requ{} \emph{from the agent's perspective} is clearest with clause~\ref{idea:requ:pool}.
  For example, there may be no other place to look, but from my perspective, somewhere I haven't checked, and so \requ{} of concluding lost keys from my perspective.
  From my perspective, option, but from an abstract standpoint, I have exhausted all the possibilities.

  Clause~\ref{idea:requ:nPsi-nPhi} is less clear.
  However, deviance.

  Clause~\ref{idea:requ:nPsi-nPhi} is where interest is.
  For, 



  Easily embed within.%
  \footnote{
    Recall, same with respect to \fc{1} --- see page~\pageref{fcs-neutral-perspective}.
  }
\end{note}

\begin{note}
  \color{red}
  Somewhere, highlight that it doesn't matter whether \requ{} follows.
  Wait, what do I want to say?

  Well, in some cases, it may be the case that \(\pv{\phi}{v}\) follows trivially after getting \(\pv{\psi}{v'}\), this doesn't matter.

  It may also be the case that \(\pv{\phi}{v}\) entails \(\pvp{\psi}{v'}{\Psi}\), but this is also fine, so long as there is an alternative way.
\end{note}

\begin{note}
  \color{red}

  clause~\ref{idea:requ:pool} means that, so long as \(\phi\) has value \(v\), the agent has the option of checking whether it makes sense for the agent to conclude \(\pv{\phi}{v}\) from \(\Phi\).

  And clause~\ref{idea:requ:nPsi-nPhi} expresses that concluding \(\pv{\psi}{v'}\) from \(\Psi\) is a check on whether it makes sense for the agent, from their perspective, to conclude \(\pv{\phi}{v}\) from \(\Phi\).

\end{note}

\begin{note}
  Clear link to \fc{1}.
  For, \fc{1} option.
  If \fc{1}, then second part fails to hold.

  Both \requ{}, and \fc{}.
  This will be \qzS{}.
\end{note}

\begin{note}
  Start with some examples.
\end{note}

\begin{note}[Example]
  \autoref{illu:gist:roots} involves an agent concluding either \(x = 1\) or \(x = -\sfrac{1}{2}\) from premise that for some \(x \in \mathbb{R}\), \(2x^{2} - x - 1 = 0\).%
  \footnote{
    Abstractly, \autoref{illu:gist:roots} is a case where the agent would not conclude \(\pv{\phi}{v}\) from \(\Phi\) if the agent failed to conclude \(\pv{\phi}{v}\) from \(\Psi\).
    I.e.\ the relevant conclusion is the same in both proposition-value-premises pairings, the only difference is the relevant pool of premises (and method of reasoning).
  }
  And, when concluding either \(x = 1\) or \(x = -\sfrac{1}{2}\) the agent observes that \emph{if} \(x = 1\) or \(x = -\sfrac{1}{2}\), then they would also be able to observe this via factorisation.

  In other words, if the agent attempted to conclude either \(x = 1\) or \(x = -\sfrac{1}{2}\) via factorisation and failed, the agent would not conclude either \(x = 1\) or \(x = -\sfrac{1}{2}\) via (their application of) the quadratic formula.
\end{note}

\begin{note}
  \autoref{illu:gist:calc}.
  Option, but testimony.
\end{note}

\begin{note}
  Variant, with being quite tired.
\end{note}

\begin{note}
  \citeauthor{Dretske:1970to}.
  Failure, but no option.
\end{note}

\begin{note}
  Cases where both hold.
\end{note}

\paragraph{Ability}

\begin{note}
  Most interesting case, from my perspective.

  Simple case is Sudoku puzzles, or puzzles in general.

  For, get clusters and hierarchies of \requ{1}{\color{red} ?}

  Though, it is a little tricky.
  For, depends on methods available.
  For example, addition.
  It may be the case in doing some complex addition, need to get the result of something more basic.
  So, example with long addition.
  Failure of a \requ{}.
\end{note}

\paragraph{Details}

\begin{note}
  key clause is clause~\ref{idea:requ:nPsi-nPhi}.
  For, clause~\ref{idea:requ:nPsi-nPhi} captures the core idea of failure to conclude \(\pv{\psi}{v'}\) from \(\Psi\) leading to failure to conclude \(\pv{\phi}{v}\) from \(\Phi\).

  The role of clause~\ref{idea:requ:pool} is to ensure the agent may conclude \(\pv{\psi}{v'}\) from \(\Psi\) independently of concluding \(\pv{\phi}{v}\) from \(\Phi\).

  If \ref{idea:requ:pool:int} were to fail to hold then:
  \begin{itemize}
  \item
    By~\ref{idea:requ:pool:int}, the agent would need to conclude \(\pv{\phi}{v}\) from \(\Phi\) as a sub-conclusion when reasoning from the relevant pool of premises \(\Psi\).
    Hence, it would not be possible to conclude \(\pv{\psi}{v'}\) from \(\Psi\) without first concluding \(\pv{\phi}{v}\) from \(\Phi\).
  \end{itemize}

  Conversely, if \ref{idea:requ:pool:int} holds, the agent may conclude \(\pv{\psi}{v'}\) from \(\Psi\) independently of concluding \(\pv{\phi}{v}\) from \(\Phi\).

  Note, however, \ref{idea:requ:pool:int} does not rule out the possibility of the agent concluding \(\pv{\phi}{v}\) from \(\Phi\) when concluding \(\pv{\psi}{v'}\) from \(\Psi\) or, conversely, concluding \(\pv{\psi}{v'}\) from \(\Psi\) when concluding \(\pv{\phi}{v'}\) from \(\Phi\).

  Having the option which matters.
  Whether or not the option is taken, doesn't matter.
\end{note}

\begin{note}[\requ{2}: Partial check]
  Intuitively, concluding \(\pv{\psi}{v'}\) from \(\Psi\) would serve as a partial check on whether the agent may reason to a conclusion other than \(\pv{\phi}{v}\), captured by~\ref{idea:requ:nPsi-nPhi}.

  Concluding \(\pv{\psi}{v'}\) from \(\Psi\) is a check.
  For, if the agent were to fail to conclude \(\pv{\psi}{v'}\) from \(\Psi\) then, the agent would not conclude \(\pv{\phi}{v}\) from \(\Phi\), from the agent's perspective.
  Hence, contraposing, the agent would conclude \(\pv{\phi}{v}\) from \(\Phi\) only if the agent would conclude \(\pv{\psi}{v'}\) from \(\Psi\), from the agent's perspective.

  However, the check is partial, as it need not be the case that the agent would conclude \(\pv{\psi}{v'}\) from \(\Psi\) only if the agent \(\pv{\phi}{v}\) from \(\Phi\).
  Therefore, failing to conclude \(\pv{\psi}{v'}\) from \(\Psi\) may block concluding \(\pv{\phi}{v}\) (from the perspective of the agent) though concluding \(\pv{\psi}{v'}\) from \(\Psi\) need not ensure that the agent would conclude \(\pv{\phi}{v}\).

  Combining these two ideas, intuitively, \(\pv{\psi}{v'}\) is a \requ{} of concluding \(\pv{\phi}{v}\) just in case there is some pool of premises \(\Psi\) such that determining whether the agent would conclude \(\pv{\psi}{v'}\) is an independent partial check on whether the agent may reason to a conclusion other than \(\pv{\phi}{v}\).
\end{note}

\paragraph{Agent's perspective and concluding}

\begin{note}[Abstract motivation]
  We only need to consider what concluding \(\pv{\phi}{v}\) from \(\Phi\) would commit the agent to, so to speak.
  For example, we do not need to consider the consequences of the agent reasoning about some arbitrary proposition-value pair \(\pv{\chi}{v''}\) prior to concluding \(\pv{\phi}{v}\) from \(\Phi\).

  Of course, if the agent would conclude both \(\pv{\chi}{v''}\) and \(\pv{\chi}{\overline{v''}}\) for some \(\chi\), then it seems the agent's epistemic state is in bad shape.
  Still, given that an agent will typically revise their epistemic state upon concluding both \(\pv{\chi}{v''}\) and \(\pv{\chi}{\overline{v''}}\) for some \(\chi\), such concerns may be isolated to a distinct question.

  Second, we avoid --- to some extent --- concerns about over-generating.
  Role is negative resolution to \issueConstraint{}.
  No witnessing.
  A concern is that this overall argument may over-generate.
  Unintended consequences may diminish interest in the negative resolution we hope to motivate, and hence tip favour to a positive answer to~\issueConstraint{}.
  Though, whether or not there are unintended consequences of broadening the scope of~\autoref{question:zs} is unclear to me.
\end{note}

\paragraph{Recursion?}

\begin{note}
  \ref{idea:requ:pool}, option constrained \ref{idea:requ:pool:int}.

  Purpose is, not need to conclude \(\pv{\phi}{v}\) from \(\Phi\).

  Implicit assumption that \requ{} applies when concluding.
  And, has option, so \(\Psi\) are available.
  Hence, not yet concluded, so \(\pv{\phi}{v}\) from \(\Phi\) was not involved in any of the premises.

  With this intent, problems may arise.
  For some \(\pv{\psi_{i}}{v_{i}}\) in \(\Psi\), concluded in part from \(\pv{\phi}{v}\) from \(\Phi\).
  Agent retained \(\pv{\psi_{i}}{v_{i}}\) but forgot \(\pv{\phi}{v}\) from \(\Phi\).

  I don't find this particularly interesting.
  About present reasoning.
  Same as with testimony.
  Take the agent's present perspective on how things are to be fixed.

  However, if a worry, then either add clause which quantifies over past reasoning directly, or recursive.

  \begin{enumerate}
  \item
    \label{idea:requ:pool:ind}
    For any proposition-value pair \(\pv{\psi_{i}}{v_{i}}\) in \(\Psi\), \vAgent{} may conclude \(\pv{\psi_{i}}{v_{i}}\) from some pool of premises \(\Psi_{i}\) without concluding \(\pv{\phi}{v}\) from \(\Phi\).
  \end{enumerate}
\end{note}


\subsubsection{Summary}

\begin{note}
  Reasoning, \support{}, would not reason to a different conclusion.

  Specifically, \requ{} of some conclusion.
  So long as conclusion, then it is possible to reason about whether \(\psi\) has value \(v'\), and unless conclude \(\psi\) has value \(v'\), would not conclude \(\phi\) has value \(v\).

  Intuitively, \requ{} as an independent check on the reasoning.
  If don't hold \(\psi\) from premises, then question about whether \(\phi\).

  Concluding, then, may be weaker than having support.
  Restricted to whether conclusion of reasoning would introduce a \requ{}.
  And, may be further restricted without impact to the tension we will develop to whether the conclusion would `clearly' introduce a \requ{}.
\end{note}

\section{\zS{}}
\label{cha:zS:sec:question}

\begin{note}
  With the idea of a \requ{} in hand, turn to \zSN{}.
\end{note}

\begin{note}
  \begin{restatable}[\qzS{}]{question}{questionZS}
    \label{question:zs}
    For an agent \vAgent{}, when concluding \(\pv{\phi}{v}\) from \(\Phi\), is it the case that:%
    \footnote{
      As with \qWhy{}, \qHow{}, and \qWhyV{}, we take the agent and instance of concluding to be independently specified for instances of \qzS{} in order to avoid the verbose `\qzS{} with respect to agent \emph{A} concluding \(\pv{\phi}{v}\) from \(\Phi\)'.
    }

    \begin{itemize}
    \item
      From \vAgent{}' perspective:
      \begin{itemize}
      \item
        For any proposition-value-premises pairing \(\pvp{\psi}{v'}{\Psi}\):
        \begin{itemize}
        \item[\emph{If}:]
          \begin{enumerate}[label=\alph*., ref=(\alph*)]
          \item
            \label{question:zs:option}
            \(\pvp{\psi}{v'}{\Psi}\) is a \requ{} of concluding \(\pv{\phi}{v}\) from \(\Phi\).
          \end{enumerate}
        \item[\emph{Then}:]
          \begin{enumerate}[label=\alph*., ref=(\alph*), resume]
          \item
            \label{question:zs:may-fail}
            \vAgent{} would conclude \(\pv{\psi}{v'}\) from \(\Psi\), if \vAgent{} were to attempt to conclude \(\pv{\psi}{v'}\) from \(\Psi\).
          \end{enumerate}
        \end{itemize}
      \end{itemize}
    \end{itemize}
    \vspace{-\baselineskip}
  \end{restatable}

  Basic idea here is that strengthen the opportunity with ability, following \citeauthor{Austin:1961vz}'s way of putting things.
\end{note}

\begin{note}
  \qzS{} is a question about what holds with respect to \emph{any} proposition-value-premises pairing, from an agent's perspective, when the agent concludes some proposition-value pair \(\pv{\phi}{v}\) from some pool of premises \(\Phi\).

  Key thing, concluding.
  Alternative conclusions.

  Suppose antecedents.
  Just prior to concluding.
  So, take up option.
  Possible, don't conclude \(\pv{\psi}{v'}\) from \(\Psi\).
\end{note}

\begin{note}
  \qzS{} concerns an agent when concluding \(\pv{\phi}{v}\) from \(\Phi\), but just prior to having concluding \(\pv{\phi}{v}\) from \(\Phi\).
  And, in paraphrase, asks whether there is some proposition-value-premises pairing \(\pvp{\psi}{v'}{\Psi}\) such that, from the agent's perspective:
  \begin{itemize}
  \item
    The agent \emph{may} fail to conclude \(\pv{\psi}{v'}\) from \(\Psi\), and hence \emph{possible} not concluding \(\pv{\phi}{v}\) from \(\Phi\).
  \end{itemize}

  This paraphrase combines and makes implicit various aspects of { \color{blue} a \requ{} }, and~\ref{question:zs:may-fail} to form a single simple statement which is assessed from the agent's perspective.
\end{note}

\begin{note}[Quantified conditional]
  Whether there is such a \(\pvp{\psi}{v'}{\Psi}\).

  Clauses~{ \color{blue} XXX } of~\autoref{question:zs} from a quantified conditional, so there are three distinct ways in which~\ref{question:zs} may receive a positive answer.
  \begin{itemize}
  \item
    For any proposition-value-premises pairing \(\pvp{\psi}{v'}{\Psi}\), either:
    \begin{enumerate}[label=\alph*\('\).]
    \item
      It is not the case that the agent has the option of concluding \(\pv{\psi}{v'}\) from \(\Psi\), given the agent's reasoning from \(\Phi\) to \(\pv{\phi}{v}\).
    \item
      The agent may conclude \(\pv{\phi}{v}\) from \(\Phi\), (even) if they were to attempt and fail to conclude \(\pv{\psi}{v'}\) from \(\Psi\).
    \item
      The agent would conclude \(\pv{\psi}{v'}\) from \(\Psi\), if they attempted to do so.
    \end{enumerate}
  \end{itemize}
\end{note}

\paragraph{Negated conditional}

\begin{note}
  Consider \qzS{} re-expressed with a negated conditional:
  \begin{restatable}[\nqzS{}]{question}{questionZSNeg}
    \label{question:zs:neg}
    For an agent \vAgent{}, when concluding \(\pv{\phi}{v}\) from \(\Phi\), is it the case that:

    \begin{itemize}
    \item
      From \vAgent{}' perspective:
      \begin{itemize}
      \item
        There some proposition-value-premises pairing \(\pvp{\psi}{v'}{\Psi}\):
        \begin{itemize}
        \item[\emph{Both}:]
          \begin{enumerate}[label=\alph*., ref=(\alph*)]
          \item
            \(\pvp{\psi}{v'}{\Psi}\) is a \requ{} of concluding \(\pv{\phi}{v}\) from \(\Phi\).
          \end{enumerate}
        \item[\emph{And}:]
          \begin{enumerate}[label=\alph*., ref=(\alph*), resume]
          \item
            \label{question:zs:may-fail}
            \vAgent{} would \emph{not} conclude \(\pv{\psi}{v'}\) from \(\Psi\), if \vAgent{} were to attempt to conclude \(\pv{\psi}{v'}\) from \(\Psi\).
          \end{enumerate}
        \end{itemize}
      \end{itemize}
    \end{itemize}
    \vspace{-\baselineskip}
  \end{restatable}
\end{note}

\subsection[Visualisation]{Visualisation of what is at issue when asking \qzS{}}

\begin{figure}[h]
  \centering
  \begin{tikzpicture}
    \node (origin) at (0,0) {};
    \node (psiSplit) at (1,0) {};
    \node (phiSplit) at (4,0) {};
    %
    \node[anchor=west] (psiV) at  (6,-1)  {\(\pvp{\psi}{v'}{\Psi}\)};
    \node[anchor=west] (psiNv) at (6,-2) {\(\pvp{\psi}{\{\overline{v'}, ?\}}{\Psi}\)};
    % \node[anchor=west] (psiQ) at (6,-3) {\(\pvp{\psi}{?}{\Psi}\)};
    %
    % \node[anchor=west] (psiVPhiV) at (9,-1) {\(\pv{\phi}{v}\)};
    \node[anchor=west] (psiNvPhiU) at (9,-2) {\(\pv{\phi}{\{\overline{v},?\}}\)};
    % \node[anchor=west] (psiQPhiU) at (9,-3) {\(\pv{\phi}{\{\overline{v},?\}}\)};
    %
    \node[anchor=west] (phiQ) at (10,1) {\(\pv{\phi}{\{\overline{v}, ?\}}\)};
    % \node[anchor=west] (phiNv) at (10,2) {\(\pv{\phi}{\overline{v}}\)};;
    \node[anchor=west] (phiV) at (10,0) {\(\pv{\phi}{v}\)};
    %
    \draw[-]  (origin) -- (phiV);
    %
    % \path[-,dotted] (phiSplit) edge [out=0, in=180] (phiNv);
    \path[-,dotted] (phiSplit) edge [out=0, in=180] (phiQ);
    %
    \path[-, dashed] (psiSplit) edge [out=0, in=180] (psiV);
    \path[-, dotted] (psiSplit) edge [out=0, in=180] (psiNv);
    % \path[-, dotted] (psiSplit) edge [out=0, in=180] (psiQ);
    %
    \draw[-,dashed] (psiV) edge [out=0, in=180] (phiV);
    \draw[-, dotted] (psiNv) edge (psiNvPhiU);
    % \draw[-, dotted] (psiQ) edge (psiQPhiU);
    \end{tikzpicture}
    \caption{Sketch of when \qzS{} has a positive answer.}
    \label{fig:csN:illu:overview}
  \end{figure}

\begin{note}[Figure]
  \autoref{fig:csN:illu:overview} provides a rough visualisation of~\qzS{}.

  The flat line captures the agent's reasoning, which concludes with \(\pv{\phi}{v}\).
  In concluding \(\pv{\phi}{v}\) the agent rules out two possibilities with respect to \(\phi\).
  First, that \(\phi\) does not have value \(v\), indicated by \(\pv{\phi}{\overline{v}}\).
  Second, that the agent does not assign any value to \(v\), indicated by \(\pv{\phi}{?}\).
  Prior to concluding \(\pv{\phi}{v}\), the agent's reasoning may have branched to either alternative path, but as the agent has concluded \(\pv{\phi}{v}\), neither path is viable, and hence both paths are represented with a dashed line.

  So far, we have seen only that the agent has concluded \(\pv{\phi}{v}\).

  We now consider some proposition-value-premises pairing \(\pv{\psi}{v'}{\Psi}\) such that if the agent were to fail to conclude \(\pv{\psi}{v'}\) from \(\Phi\), the agent would not conclude \(\pv{\phi}{v}\) from \(\Phi\).

  Intuitively, the dotted arrows from the various combinations of \(\psi\) and \(\{v',\overline{v'},?\}\) read, from top to bottom:
  \begin{itemize}
  \item If \(\phi\) has value \(v\) then the agent may conclude \(\pv{\psi}{v'}\) from \(\Psi\), and:
  \item If the agent concludes \(\psi\) has some value \(\overline{v'}\) from \(\Psi\), then the agent either concludes \(\phi\) has some value other than \(v\), or the agent fails to reach a conclusion regarding \(\phi\) from \(\Phi\).
    Both options are combined via the shorthand \(\pv{\phi}{\{\overline{v},?\}}\).
  \item
    And, likewise if the agent fails to conclude \(\pv{\psi}{v'}\) from \(\Psi\).
  \end{itemize}

  With respect to concluding, observe that prior to ruling out alternative branches with respect to \(\pv{\phi}{\{\overline{v},?\}}\), the agent may have reasoned about whether \(\psi\) has value \(v\).
  And, from the agent's perspective, \(\phi\) has value \(v\) only if \(\psi\) has value \(v'\).
  If \(\psi\) does not have value \(v'\), then either \(\phi\) does not have value \(v\), or the agent's reasoning would not conclude with a value for \(\phi\), indicated by \(\pv{\phi}{\{\overline{v},?\}}\).

  Hence, prior to concluding \(\pv{\phi}{v}\), the agent has concluded \(\pv{\psi}{v'}\).
\end{note}

\begin{note}
  Broadly, then, we may say that an agent has {\color{red} particular kind of conclusion} for \(\pv{\phi}{v}\) just in case when concluding \(\pv{\phi}{v}\) it is not the case that the agent's reasoning would have branched to a different conclusion with respect to \(\phi\).

  However, the visualisation of~\autoref{fig:csN:illu:overview} and this broad statement of {\color{red} positive answer to \qzS{}} are a little too broad.
  For, we are only interested in proposition-value pairs guaranteed by \(\phi\) having value \(v\).
  {\color{red} positive answer to \qzS{}} is not global with respect to all proposition-value pairs that the agent may have reasoned about, but local to those guaranteed by the proposition.
\end{note}

\paragraph{\zSN{2}}

\begin{note}[Naming]
  Our choice of the term `\zgb{0}' is metaphorical.
  \zgb{2} is a family of flower plants which, typically, have the appearance of a single stem with no branches.
  If one starts just before the flower and works back down the stem, one will not find a branch which, if taken, would lead to a different flower.
  In metaphor, if one starts with an agent's perspective prior to the agent concluding \(\pv{\phi}{v}\) from \(\Phi\) and~\autoref{question:zs} has a negative answer with respect to \(\pvp{\phi}{v}{\Phi}\), then one will not find a branch which leads to a different conclusion.

  I have some doubts as to whether or not this metaphor really works, but some term is required.
  `Palm-tree-support', or `Arecaceae-support' would also work.
\end{note}

\begin{note}
  So, if agent concludes, then from perspective, no other conclusion.
  Conclude when \support{} from agent's perspective.
  \zSN{} as a \emph{kind} of \support{}.

  \begin{definition}[\izS{}]
    \label{idea:zS}
    For an agent \vAgent{}:
    \begin{enumerate}[label=]
    \item
      \begin{itemize}
      \item
        \zS{} held between \(\pv{\phi}{v}\) and \(\Phi\) when \vAgent{} concluded \(\pvp{\phi}{v}{\Phi}\) from \(\Phi\), from \vAgent{}' perspective.
      \end{itemize}
    \item \emph{If and only if}:
    \item
      \begin{itemize}[]
      \item
        \qzS{} had a \emph{positive} answer when \vAgent{} concluded \(\pv{\phi}{v}\) from \(\Phi\).%
        \footnote{
          Or, alternatively, \nqzS{} had a \emph{negative} answer when \vAgent{} concluded \(\pv{\phi}{v}\) from \(\Phi\).
        }
      \end{itemize}
    \end{enumerate}
    \vspace{-\baselineskip}
  \end{definition}

  Interest is with \qzS{} rather than the type of \support{} that would follow given a positive answer to \qzS{}.
  So, only indirect interest in \izS{}.
  Though, helpful for clarification.
\end{note}

\paragraph{Scenarios/quantification}

\begin{note}
  Quantification is general, but the core of \qzS{} is a conditional, and the conjunction of clauses which form the antecedent of the conditional place significant constraints on the proposition-value-premises pairings of interest.
  For, { \color{blue} not a \requ{} }, the conditional will be true.

  Our interest is with cases where { \color{blue} \requ{} } \(\pvp{\psi}{v'}{\Psi}\)-pairing.
\end{note}

\begin{note}[\autoref{illu:sketch:prop-logic}]
  Likewise, \autoref{illu:sketch:prop-logic} involves an agent concluding some sentence \(A\) is a syntactic theorem of propositional logic via a formula derivation.%
  \footnote{
    Abstractly, \autoref{illu:gist:roots} is a case where the agent would not conclude \(\pv{\phi}{v}\) from \(\Phi\) if the agent failed to conclude \(\pv{\psi}{v}\) from \(\Psi\), where \(\phi\) is distinct from \(\psi\) and \(\Phi\) is distinct from \(\Psi\).
    Soundness (and completeness) relates syntactic and semantic theorems of propositional logic, but these are distinct, as may be observed by considering, for example, a logic which is incomplete, or an unsound proof system.
  }

  And, when concluding \(A\) is a syntactic theorem, the agent observes that \(A\) is a syntactic theorem only if \(A\) is also a semantic theorem (from soundness).

  In other words, if the agent attempt to show \(A\) is true under an arbitrary valuation and failed, the agent would not conclude \(A\) is a syntactic theorem.
\end{note}

\begin{note}
  \autoref{cha:zS:sec:question:illu} will contain additional \illu{1} of positive and negative answers to~\autoref{question:zs}, and in particular \illu{1}.

  However, our motivation for considering~\autoref{question:zs} is the relation between concluding (or failing to conclude)  \(\pv{\phi}{v}\) from \(\Phi\) and concluding (or failing to conclude) \(\pv{\psi}{v'}\) from \(\Psi\).

  First link the kind of case we are interested in asking~\autoref{question:zs} to \cScen{1}, and make this link explicit by revisiting a pair of \scen{1}.

  Second, highlight why.

  Third, features in additional detail.
\end{note}

\begin{note}[\cScen{1}, examples]
  {
    \color{red}
    In such cases,~\autoref{question:zs} focuses on whether the agent, from their perspective, would conclude \(\pv{\psi}{v'}\) from \(\Psi\).

    We are already familiar with cases of this kind.
    Indeed, as case of this kind is a \cScen{0}.
    For, \cScen{1}, \dots.

    For the relevant conditionals in \cScen{1}, is it the case, from the agent's perspective, that they would conclude.
  }

  To illustrate, consider again \autoref{illu:gist:roots} and \autoref{illu:sketch:prop-logic}:
\end{note}


\begin{note}[In general]
  Generally speaking, the proposition-value-premises pairing present in a \cScen{0} are just is what is required for \requ{}.
  Hence, when~\autoref{question:zs} is paired with an \cScen{0},~\autoref{question:zs} asks whether the agent, from their perspective, would conclude the relevant proposition-value pair from the relevant pool of premises.
\end{note}

\section{\emph{Why}}
\label{cha:zS:section:qzs-and-why}

\begin{note}
  \color{red}
  This, there's no immediate answer to.
  Here, intuitive motivation.
  But, want answers to \qWhyV{}, not just `why'.
\end{note}

\begin{note}
  Intuitively, positive answer in \scen{1}.

  Some care has been taken.

  \autoref{illu:sketch:prop-logic}.
  Only if semantic proof.
  Syntactic proofs, at least in my experience, may be out of reach.
  However, semantic proofs, often straightforward.

  Converse may hold, but more challenging than \autoref{illu:sketch:prop-logic}.

  Similarly, \autoref{illu:gist:roots}.
  Factorisation isn't too difficult.

  \autoref{illu:sketch:math}.
  \(345 \div 15 = 23\) \emph{only if} \(23 \times 15 = 345\).

  Agent has the opportunity, and current result looks good.
\end{note}

\begin{note}
  Now, answers, in part, why.
\end{note}

\begin{note}
  Rough understanding.
  In terms of broader argument, \emph{why}.

  Idea is that agent's perspective regarding \(\pvp{\psi}{v'}{\Psi}\) in part explains why.

  Preferred \illu{0} concerns whether one has or has not lost their keys.
\end{note}

\begin{note}[Motivating \illu{0}]
  {
    \color{red}
    Interest in terms of explaining why and agent did or didn't conclude.
  }

  
\end{note}

\begin{note}
  Similar points for examples given.

  Check via factorising, check via a semantic proof.
\end{note}

\begin{note}
  Here, important, \issueConstraint{} asks about whether a relation of support is part of why an agent \emph{concludes} (only if agent has witnessed).

  With~\autoref{illu:lost-key},~\autoref{illu:gist:roots}, and~\autoref{illu:sketch:prop-logic}, failure to conclude!
\end{note}

\begin{note}
  Still, \emph{negative} answers to~\autoref{question:zs}.

  Interest, what about \emph{positive} answers?

  Positive answer only if from agent's perspective, would conclude.
  No witnessing.
  Relation of support is part of why.

  Lack of support explains, in part, why agent does \emph{not} conclude.
  Conversely, presence of support explains, in part, why agent \emph{does} conclude.
\end{note}

\paragraph{Basic principle}

\begin{note}
  \begin{idea}[\autoref{question:zs} and \emph{Why}]
    \label{prop:qzS-answers-why}
    There are cases in which:

    \begin{itemize}
    \item[\(\pm\)]
      Answers to \autoref{question:zs} answer, in part, why an agent concludes or does not conclude \(\pv{\phi}{v}\) from \(\Phi\).
    \end{itemize}

    Specifically:
    \begin{itemize}
    \item[\(-\)]
      A negative answer to~\autoref{question:zs} answers, in part, why an agent does not conclude \(\pv{\phi}{v}\) from \(\Phi\).
    \item[\(+\)]
      A positive answer to~\autoref{question:zs} answers, in part, why an agent does conclude \(\pv{\phi}{v}\) from \(\Phi\).
    \end{itemize}
  \end{idea}

  Motivation for \autoref{prop:qzS-answers-why} is by cases.
  See additional cases in \autoref{cha:zS:sec:question:illu}.

  Weak point.
  \autoref{prop:qzS-answers-why} is central to overall argument.
  Hence, something else which captures cases.
  Something else does not involve whether or not the agent would conclude.

  I do not think so.
  But, generalising from exhaustion.
  Have not exhausted every possibility, but I've exhausted myself.

  Preferable, I think, to hold that \qzS{} is not relevant.
  \autoref{prop:qzS-answers-why} is an existential.
  Generally speaking, would be good.
  Only trouble when \(\pvp{\psi}{v'}{\Psi}\) is such that the agent has not witnessed reasoning to \(\pv{\psi}{v'}\) from \(\Psi\).
  So, if \qzS{} only applies in such cases, then no problem.

  However, already seen, \autoref{illu:lost-key}.
  Absence of reasoning.

  So, narrow to no cases where a positive answer, but consider \cScen{1}.
\end{note}

\begin{note}
  Run this through \scen{1} listed.
\end{note}

\begin{note}
  What's needed is positive answer only if support.

  Here, maybe illustrate with general ability.
  Got X.
  General ability.
  So, specific ability to Y.
\end{note}

\begin{note}
  {
    \color{red}
    There's a difference between answering `no' and failing to answer.
    But, the point I'm arguing for works given this distinction.
    There's no real different.
    I mean, the conditional is either true or false.
    But, it's possible that the falsity of the conditional has a role where the truth of the conditional does not.
  }
\end{note}

\subsection{Deviance}
\label{sec:deviance}

\begin{note}
  Here, causal deviance.
\end{note}

\begin{note}
  Problem is, there's no way to guarantee a link between positive answer to \qzS{} and the agent concluding or not refraining from concluding.
\end{note}

\begin{note}
  Argument relies on tying content to explanation.

  In this respect, there is room for an objection.
  Deviant causal chains.
  Point here is that there are cases where these come apart.

  This isn't only a problem for causal theories of reasoning.
  The point is, some instantiation, and so long as act may be caused by something else, then possibly caused by the instantiation.

  So, possible here.

  Well, hold on.
  What is need is the relevance of the content.
  For this objection to work, need to take a theoretical perspective.
  See, in Davidson's case, the idea is fusing these two things together.
  We answer two different questions with a common thing viewed in two ways.

  Still, I think the objection can be pressed!
  Only \emph{really} an explanation is no deviance.
  To the same extent that potential event matters, it matters to the agent that there is no deviance.

  {
    \color{red}
    Resolution is, if deviance, then no agency.
  }

  I think this makes sense, or at least makes enough sense.
  Answers to `why', on this understanding, are tentative.

  Or, rest on presupposition that agent performed the action.

  So, contingent on showing there is no causal deviance.

  This is different to error.
  With error, thing appealed to isn't the case, but appeal still did work.
  Here, it doesn't matter whether or not the case, no work is done.

  In contrast to more typical instances of the problem, don't need to rule out deviant causal chains.
  Instead, just need one instance to fail to hold.
  One instance of non-deviousness.

  Still a problem for a compatible account which avoids.
  For, here, there can't be any direct link from perspective to reason.

  For example, \citeauthor{Hieronymi:2011aa}

    \begin{quote}
      [W]e explain an event that is an action done for reasons by appealing to the fact that the agent took certain considerations to settle the question of whether to act in some way, therein intended so to act, and successfully executed that intention in action.
    [\emph{T}]\emph{his} complex fact, [\dots] is the reason that rationalizes the action---that explains the action by giving the agent's reason for acting.%
    \mbox{ }\hfill\mbox{(\citeyear[431]{Hieronymi:2011aa})}
  \end{quote}

  So, here, considerations which settle question, and in so settling question.
  Link between settling the question and acting.

  Following \citeauthor{Hieronymi:2011aa}, no room for deviance.
  Too tight.

  In other words, so long as this fact holds, there is no distinction between settling the question and acting.
  Therefore, no deviance.

  Compatible, I think.
  Question is whether in resolving \qzS{} is sufficiently tied to resolving the question \citeauthor{Hieronymi:2011aa} identifies.
  And, plausibly is.
  This is what the motivation for \qzS{} did.

  Trouble is, for our purposes, need at least sufficient conditions for when this complex fact obtains.
  And, no account of this.

  \citeauthor{Hieronymi:2011aa} notes the gaps.

  Some tension.
  These considerations aren't premises.
\end{note}

\begin{note}
  So, the other option is to embrace deviant causal chains.
  Have the content, but this doesn't work in the way the agent thinks it does.

  Example from Davidson.

  The trouble here is that the content and resulting action match.
  So, things make sense from the agent's perspective.

  Deviant, but maybe not so deviant here.

  Systematic deviance, where content is separated from role of mental state.

  But, I see no motivation for this.

  Solution to causal chains doesn't get round this, because the result is a restricted account.
  So, there's no guaranteed trade-off here.
  Trouble is, it seems hard to see a case where this wouldn't be the case.
\end{note}


\subsection{Details}
\label{cha:zS:sec:details}

\subsubsection{Clauses, and the idea of a \requ{0}}
\label{cha:zS:sec:clauses-idea-requ1}

\paragraph{The components of \qzS{}}



\paragraph{General}

\begin{note}
  Constraints placed on \(\pvp{\psi}{v'}{\Psi}\).
  From reasoning involved in process of concluding \(\pv{\phi}{v}\) from \(\Phi\).
  Would lead to failure.

  Conditionals.

  Involved in concluding \(\pv{\phi}{v}\) from \(\Phi\).
  First, enough to break.
  Second, reasoning makes this proposition-value-premises pairing available.

  Pair of additional features

  Second highlights why \(\pvp{\psi}{v'}{\Psi}\) is of interest.
  However, in this respect, not strictly required.
  Given universal, will also include these.

  First,
  Don't need \(\phi\) to have value \(v\).
  Also, implicit, no revision.
  Built up various things in reasoning, and given all of this\dots


  And, maybe reasoning offers something new.
  Though, not the case that \(\pvp{\psi}{v'}{\Psi}\) only from something new.
  Might be the case that negative answer because go off on wrong reasoning.
\end{note}

\paragraph{Option}

\begin{note}
  Hence, it need not be the case that the agent has the option of concluding \(\pv{\psi}{v'}\) from \(\Psi\) from their epistemic state prior to starting line of reasoning (as the agent has not yet concluded that \(\phi\) has value \(v\)).
\end{note}


\paragraph{Would not conclude}

\begin{note}
  Noted failure.
\end{note}

%%%% TEMP from question
\footnote{
 { \color{blue} \requ{} } is expressed by a subjunctive conditional as there is no requirement that the agent will attempt to conclude \(\pv{\psi}{v'}\) from \(\Psi\).

  \color{red}
  As this alternative expression makes clear,~\autoref{question:zs} focuses on the agent (and their epistemic state).
  At no point do we consider any variation of the agent's epistemic state.
  Likewise,~\autoref{question:zs} concerns only the agent's perspective on concluding \(\pv{\psi}{v'}\) from \(\Psi\).
  Whether or not the agent would conclude \(\pv{\psi}{v'}\) from \(\Psi\) is irrelevant.
  What matters is whether, from the agent's perspective, there is potential for reasoning about whether \(\pv{\psi}{v'}\) follows from \(\Psi\) to block concluding \(\pv{\phi}{v}\) from \(\Phi\).
}

\paragraph*{Minor clarifications}

\begin{note}[Importance of \csN{}]
  First, agent's reasoning.
  At issue is whether the agent may reason to a different conclusion.
  There's nothing that would lead me elsewhere.

  Second, agent's reasoning.
  Independent of whether \(\phi\) has value \(v\), \(\psi\) has value \(v'\), or any of the premises.
  Need not be the case that satisfaction amounts to anything substantial.
  No clause for justification, etc.

  Third, competence, rather than performance.
\end{note}

\subsubsection{Prior to concluding\dots}

\begin{note}[Prior to concluding\dots]
  An important feature of \qzS{} \dots

  Not particularly marked.
  Allow agent to have built up a bunch of stuff in the reasoning.

  Example.

  \begin{illustration}[Velocity]
    \label{ill:velocity}
    Agent is provided with information about how far a car has travelled north as a function of time travelled.

    From this, take the derivative of the function to obtain the (instantaneous) velocity of the car at a handful of points in time.

    And, from the (instantaneous) velocity of the car, the agent calculates the (instantaneous) acceleration of the car at each of the points in time.

    The agent also has information about the speed of the car as a function of time travelled, and the agent may calculate speed by the taking magnitude of the (instantaneous) velocity of the car.
  \end{illustration}

  \autoref{ill:velocity}, two step calculation.
  Velocity, acceleration.
  After the first step, check by taking the magnitude.
  Calculation of velocity is correct only if taking the magnitude matches speed.

  Just before concluding to include cases such as this.

  Note, \cScen{}.
\end{note}

\begin{note}
  Example highlights how `intermediate conclusions' relate.
  Further point of interest:
  Failure to conclude.

  Two ways to view agent's calculation of the velocity of the car.

  First, as a conclusion.
  Same status as the function.

  Or, as temporary.

  Difference in how we understand agent's present epistemic state.

  On first, the agent's present epistemic state is inconsistent.
  Two proposition-value pairs which conflict.
  Not possible for the car to have velocity the agent calculated and acceleration the agent has been informed of.

  May also be that the function and information about acceleration are inconsistent, but may also be that the agent made a mistake in calculating the velocity of the car.

  On second, the agent's present epistemic state may%
  \footnote{
    Don't have complete perspective on agent's present epistemic state.
  }
  consistent.

  For, made a mistake.
  But, proposition-value pair is not part of present epistemic state, so distinguished from function and information about acceleration, which are consistent.

  This is a distinction we have little interest in.
  What matters is failure to conclude speed.
  Result is either revising inconsistent epistemic state, or abandoning intermediary steps of reasoning.
\end{note}

\section{Notes}
\label{cha:zS:sec:notes}

\paragraph*{When}

\begin{note}
  \emph{When} concluding \(\pv{\phi}{v}\) from \(\Phi\) in order to keep things simple.
  A variant of the question may be asked if the agent has (already) concluded \(\pv{\phi}{v}\) from \(\Phi\).
  Here, rather than asking whether the agent would not conclude \(\pv{\phi}{v}\) from \(\Phi\), we may ask whether the agent would revise their conclusion of \(\pv{\phi}{v}\) from \(\Phi\).
\end{note}

\paragraph*{Whether the agent may conclude \(\phi\) has value \(v\), regardless of \(\Phi\)}

\begin{note}
  Not about the proposition-value pair.
  Rather, it is about the concluding.
  At interest is not whether \(\phi\) has value \(v\), but whether it makes sense to conclude \(\pv{\phi}{v}\) from \(\Phi\).
  Of course, if the agent has no information about whether \(\phi\) has value \(v\), then this is also part of the picture, but that is a consequence of the base concern.
\end{note}

\paragraph*{Introduced by \(\pv{\phi}{v}\) from \(\Phi\)}

\begin{note}[Proposition-value-premise pairing introduced by \(\pv{\phi}{v}\) from \(\Phi\)]
  This restriction may seem arbitrary, and to some degree I think it is.
  Ideally, an agent concluding \(\pv{\phi}{v}\) is an instance of \csN{} just in case the agent would not have reasoned to a different conclusion if they were to reason first about any other proposition-value pair.
  However, the advantage of focusing on some proposition-value pair `required' by \(\phi\) having value \(v\) is a significant constraint on the range of proposition-value pairs an agent needs to consider in order to \csN{}.

  In general, it may not be clear which proposition-value pairs may lead an agent to fail to conclude \(\phi\) has value \(v\), but so long the proposition-value pair of interest is given by \(\phi\) having value \(v\), an exhaustive search over all other proposition-value pairs may be avoided.

  Indeed, we will say that an agent has \emph{\support{}} for \(\phi\) having value \(v\) just in case they would not have reasoned otherwise, and reserve \emph{\claiming{}} \support{} for the weaker notion.
\end{note}


\paragraph*{Inductive, abductive, etc.\ reasoning}

\begin{note}
  Narrow, but not too narrow.
\end{note}

\begin{note}
  This doesn't rule out inductive or abductive reasoning.
  Consider standard induction.
  Here, there may be novel information, but this is not available from the agent's present epistemic state, and \qzS{} only concerns the agent's present epistemic state.
  Perhaps the possibility alone would prevent conclusion.
  However, it seems most conclude in recognition of such possibility.
  Instead, what one would need is considerations against uniformity principle.

  Same for any bridge between probabilistic and full.
  Toss a coin \(n\) times, conclude it is fair.
  Possible to toss \(m\) more times, not fair.
  However, \(n\) is sufficient, then no problem.
  It is true that there is something more you could do, but this would require acquiring new information.
\end{note}



\paragraph*{Fragility}

\begin{note}
  Kind of fragility.
  If concludes \(\pv{\phi}{v}\) from \(\Phi\), and does not have \zS{}, then from agent's perspective, possibility of revision.

  Indeed, may break down into two components.

  First, possibility of different conclusion.
  Agent's epistemic state is potentially unstable.

  Second, isolation of potential instability to \(\pv{\phi}{v}\) from \(\Phi\).
\end{note}


\paragraph*{Normative?}

\begin{note}[Just a property]
  There's no kind of normative evaluation here.
  We do not hold any conclusion for which this fails is bad.
  Nor do we hold that any conclusion for which this holds is good.
  Indeed, \zS{} is narrow, far too narrow for a general evaluation.

  Indeed, whether or not \zS{} doesn't tell us anything about the relationship between \(\pv{\phi}{v}\) and \(\Phi\) in general, as relative to an agent's epistemic state.
  May be that there is some \(\pvp{\psi}{v'}{\Psi}\), but only due to some quirk of the agent.
\end{note}

\paragraph*{The agent concluding \(\pv{\psi}{v'}\) from \(\Psi\)}

\begin{note}
  From the perspective of the agent.
  It doesn't matter whether the agent really has the option.
  Indeed, this perspective is important for fragility.
\end{note}

\section{Scraps}
\label{sec:scraps}

\begin{note}
  \emph{However}, caution.
  For, as we have seen with testimony, it may be the case that status of a premises blocks a \requ{}.
  And, the argument given relies on the existence of a \requ{}.
  So, it may be the case that past reasoning blocks a \requ{}.
  Still, here, only need to deny this.
  Not saying that in every case agent's present reasoning is given priority.
  (Indeed, consider cases of being somewhat impaired, e.g., via exhaustion.
  Indeed, exhaustion is interesting.
  Basic consistency checks.
  Should be the case that conclude A, but just concluded \emph{not}-A, or something like this\dots)
  Rather, denying that past continues to secure in all instances.
  So, just need the potential to revise perspective on any previous conclusion.
\end{note}

\begin{note}
  An interesting observation here is that in certain this all arises, to a certain extent, because of general abilities.
  General ability spans multiple different proposition-value-premises pairings.
  Hence, all of these function as \requ{1}, so long as the agent has the option.

  General ability spans multiple different proposition-value-premises pairings.
  Hence, all of these function as \requ{1}, so long as the agent has the option.

  \begin{itemize}
  \item
    General and specific abilities.
  \item
    Answers to why, then.
    Note, here, that opportunity is interesting.
    The whole conjunction of all instance of the general ability is plausibly not a \requ{}.
    However, all that's needed is the \emph{individual} instances, and for these to raise a problem.
  \item
    The point is, \requ{1} for any general ability, and these are also \requ{1} for main pairing.
    (%
    Note --- or perhaps emphasise --- here, that the problem is \emph{not} recursive.
    Instead, the problem is about the spread.%
    )
  \item
    Here, then, ability is both the problem and the answer.
    What's interesting is the way in which ability functions.
    It's not merely \emph{that} the agent has the ability.
    Instead, it \emph{is} the ability.
  \end{itemize}
\end{note}

\begin{note}
  So, the way in which past reasoning relates is by ensuring that the agent would reach the same conclusion.
  About the agent's reasoning.
  \emph{How} rather than \emph{that}.

  Look, what we are getting is that the agent would conclude.
  If something were to happen, then some action would be performed.
  There's no distinction between the answer and performing the act, roughly.
  Or, better put, the answer \emph{is about present reasoning}.
  Answer states that in present reasoning, would not fail.


  It is about the agent's present epistemic state, and in particular what the agent's present epistemic state is capable of.

  In other words, ability.
  What answers is ability, in the sense that ability iff would.

  This is very important to the understanding of \fc{}.

  And, I kind of want to have ability as a gloss, while focusing on \fc{} to avoid going into ability in too much detail.

  So, positive answer, then it's the pairing \emph{being} a \fc{}.
  (I should always use this instance of the copula.)
\end{note}

\paragraph{Details}

\begin{note}
  Now, two basic propositions follow.

  \begin{proposition}[When a agent has \zS{}]
    Agent and proposition-value-premises pairing.

    \begin{itemize}
    \item
      Agent \emph{has} \zS{} for \(\pv{\phi}{v}\) after concluding \(\pv{\phi}{v}\) from \(\Phi\).
    \end{itemize}

    \emph{if and only if}

    \begin{itemize}
    \item
      Either:
      \begin{enumerate}[label=(\alph*), ref=\alph*]
      \item
        There is no proposition-value-premises pairing \(\pvp{\psi}{v'}{\Psi}\) for which the relevant conditions are met.
      \item
        There is some proposition-value-premises pairing \(\pvp{\psi}{v'}{\Psi}\) but, from the agent's perspective, the agent would not fail to conclude \(\pv{\psi}{v'}\) from \(\Psi\).
      \end{enumerate}
    \end{itemize}
  \end{proposition}

  Second, when an agent does not have \zS{}.

  \begin{proposition}[When a agent does not have \zS{}]
    Agent and proposition-value-premises pairing.
    \begin{itemize}
    \item
      Agent \emph{does not} have \zS{} for \(\pv{\phi}{v}\) after concluding \(\pv{\phi}{v}\) from \(\Phi\).
    \end{itemize}

    \emph{if and only if}

    \begin{itemize}
    \item
      There is some proposition-value-premises pairing \(\pvp{\psi}{v'}{\Psi}\) such that, from the agent's perspective, the agent may fail to conclude \(\pv{\psi}{v'}\) from \(\Psi\).
    \end{itemize}
  \end{proposition}
\end{note}

%%% Local Variables:
%%% mode: latex
%%% TeX-master: "master"
%%% End:
