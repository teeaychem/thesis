\chapter{\zSN{2}}
\label{cha:zS}

\begin{note}[Intro, locating]
  Now turn to \zSN{}.

  Our goal motivate a negative resolution to~\issueConstraint{}.
  And, as sketched in~\autoref{cha:sketch}, \zSN{} has key role in developing tension.

  Question.
  Positive answer answers why.
  Positive answer only if either concluded or \fc{}.
  Concluded or \fc{} answers why.
  \fc{} answers why only if relation of support answers why.
  Relation of support answers why only if proposition-value-premises pairing answers why.

  But, \fc{}, so proposition-value-premises pairing is not an answer to how, as the agent has not witnessed reasoning.

  Now, it is consistent that positive answer only if agent has concluded, hence has witnessed reasoning, and hence is, in part, an answer to how.

  Figuring out instances where this does not hold will wait until later.


  In contrast to sketch given, being with focus on motivation.

  Motivate \zSN{} via plain language question, similar to initial \qWhy{} and \qHow{} from~\autoref{cha:introduction}.
  Follow similar pattern.
  Just as \qWhy{} and \qHow{} were developed in~\autoref{cha:clarification}, \zSN{}, is developed in a similar way.
  Here, like with \qWhy{}, specifically, two things reduce to the same.
\end{note}

\begin{note}[Map]
  Focus is a question.

  Certain kind of support \zSN{}, or \zS{}.

  Understanding of \zS{}.
\end{note}

\begin{note}
  In order to establish tension we narrow our attention to when concluding \(\pv{\phi}{v}\) concluding \(\pv{\phi}{v}\) involves the agent establishing a particular property with respect to \(\pv{\phi}{v}\).
  We term the property `\zSN{0}', or `\zS{}' for short.

  Positive resolution only requires existence of cases.
  Hence, existence of cases with this property.
  This will be sufficient.
  Any case of concluding which involves \csVImp{} will also be an instance of concluding.
\end{note}

\paragraph{Naming}

\begin{note}[Naming]
  Our choice of the term `\zgb{0}' is metaphorical.
  \zgb{2} is a family of flower plants which, typically, have the appearance of a single stem with no branches.
  If one starts just before the flower and works back down the stem, one will not find a branch which, if taken, would lead to a different flower.
  In comparison, if one starts with an agent's epistemic state prior to the agent concluding \(\pv{\phi}{v}\) from \(\Phi\) and~\autoref{question:zs} has a negative answer with respect to \(\pvp{\phi}{v}{\Phi}\), then one will not find a branch which leads to a different conclusion.

  I have some doubts as to whether or not this metaphor really works, but some term is required.
  `Palm-tree-support', or `Arecaceae-support' would also work.
\end{note}

\begin{note}[The token `\qzS{}']
  As {\color{red} noted}, we {\color{red} will} associate \zS{} with positive answers to~\autoref{question:zs}.
  So, in anticipation of this connexion we will use the token `\qzS{}' to name and refer to \autoref{question:zs}.
  Though, to be clear, \zS{} will concern an agent \emph{after} concluding \(\pv{\phi}{v}\) from \(\Phi\) while \qzS{} concerns the agent \emph{when} concluding \(\pv{\phi}{v}\) from \(\Phi\).
  So, strictly speaking an instance of \zS{} is not equivalent with a positive answer to some instance of \qzS{}.%
  \footnote{
    Whether, when concluding \(\pv{\phi}{v}\) from \(\Phi\), it is the case that the following conditional which quantifies over proposition-value-premises pairings:

    \begin{itemize}
    \item
      For all \(\pvp{\psi}{v'}{\Psi}\) [\emph{if}~\ref{question:zs:subjunctive} and~\ref{question:zs:option} hold, \emph{then} \ref{question:zs:may-fail} holds].
    \end{itemize}
    Where `hold(s)' expands to `hold(s) from the agent's perspective'.
  }
\end{note}

\section{\zS{}}
\label{cha:zS:sec:question}

\begin{note}
  A question about an agent's epistemic state when concluding \(\pv{\phi}{v}\) from \(\Phi\).
  Goal here is to get why involved.
\end{note}

\subsection{A question}
\label{cha:zS:sec:the-question}

\begin{note}
  We begin with the question.

  \begin{restatable}[\qzS{}]{question}{questionZS}
    \label{question:zs}
    For an agent \vAgent{}, when concluding \(\pv{\phi}{v}\) from \(\Phi\), is it the case that:

    \begin{itemize}
    \item
      From \vAgent{}' perspective:
      \begin{itemize}
      \item
        For any proposition-value-premises pairing \(\pvp{\psi}{v'}{\Psi}\):
        \begin{itemize}
        \item[\emph{If}:]
          \begin{enumerate}[label=\alph*., ref=(\alph*)]
          \item
            \label{question:zs:option}
            \vAgent{} has the option of concluding \(\pv{\psi}{v'}\) from \(\Psi\), given the agent's reasoning from \(\Phi\) to \(\pv{\phi}{v}\).
          \end{enumerate}
        \item[\emph{and}:]
          \begin{enumerate}[label=\alph*., ref=(\alph*), resume]
          \item
            \label{question:zs:subjunctive}
            \vAgent{} would not conclude \(\pv{\phi}{v}\) from \(\Phi\), if \vAgent{} were to attempt and fail to conclude \(\pv{\psi}{v'}\) from \(\Psi\).%
          \end{enumerate}
        \item[\emph{then}:]
          \begin{enumerate}[label=\alph*., ref=(\alph*), resume]
          \item
            \label{question:zs:may-fail}
            \vAgent{} would conclude \(\pv{\psi}{v'}\) from \(\Psi\), if \vAgent{} were to attempt to conclude \(\pv{\psi}{v'}\) from \(\Psi\).
          \end{enumerate}
        \end{itemize}
      \end{itemize}
    \end{itemize}
    \vspace{-\baselineskip}
  \end{restatable}
\end{note}

\begin{note}
  \qzS{} is delicate.

  Implicit is link between present conclusion and potential conclusions, from the agent's perspective.

  Uncertainty about whether or not one has the option to conclude something blocks present conclusion.

  So, positive answer, for any potential conclusion, would conclude in the present.

  In negative answers, potential conclusion, but either no or negative opinion on concluding.

  This doesn't directly tell us anything about whether or not the agent concludes \(\pv{\phi}{v}\) from \(\Phi\).
  Agent may continue, or may not.

  Cover basics of \qzS{} and then link to \qWhy{}.
  Specifically, in \autoref{cha:zS:section:qzs-and-why}.

  Most of the focus here is on connecting \qzS{} with \qWhy{}.
\end{note}

\paragraph{Cases}

\begin{note}
  \qzS{} concerns an agent when concluding \(\pv{\phi}{v}\) from \(\Phi\), but just prior to having concluding \(\pv{\phi}{v}\) from \(\Phi\).
  And, in paraphrase, asks whether there is some proposition-value-premises pairing \(\pvp{\psi}{v'}{\Psi}\) such that for the agent and from their perspective:
  \begin{itemize}
  \item
    The agent \emph{may} fail to conclude \(\pv{\psi}{v'}\) from \(\Psi\), and hence \emph{may} refrain from concluding \(\pv{\phi}{v}\) from \(\Phi\).
  \end{itemize}

  This paraphrase combines and makes implicit various aspects of clauses~\ref{question:zs:option},~\ref{question:zs:subjunctive}, and~\ref{question:zs:may-fail} to form a single simple statement which is assessed from the agent's perspective.

  We step through~\ref{question:zs:may-fail},~\ref{question:zs:option}, and then~\ref{question:zs:subjunctive}:

  \begin{itemize}
  \item
    \ref{question:zs:may-fail}, the question, would the agent conclude \(\pv{\psi}{v'}\) from \(\Psi\) for any such \(\pvp{\psi}{v'}{\Psi}\).

    the question is whether the agent may \emph{refraining} to conclude \(\pv{\phi}{v}\) from \(v\), while \autoref{question:zs} (in particular~\ref{question:zs:may-fail}) asks whether the \emph{would} conclude \(\pv{\phi}{v}\) from \(v\).

  Still, given the broader context,%
  \footnote{
    In general the agent may also get sidetracked and so on, but the broader context \autoref{question:zs} is that the agent is (in the process of) concluding \(\pv{\phi}{v}\) from \(\Phi\), and likewise for the paraphrase.
  }
  these are equivalent questions.
  The difference is how negative and positive answers are characterised.
  \autoref{question:zs} has a positive answer just in case the agent would conclude, and the paraphrase has a positive answer just in case the agent would \emph{not} conclude.

  \emph{At issue} is whether there exists some \(\pvp{\psi}{v'}{\Psi}\).
\item
    \ref{question:zs:option} is implicit in the reading of `may' --- the agent may fail to conclude \(\pv{\psi}{v'}\) from \(\Psi\) only if the agent has the option of concluding \(\pv{\psi}{v'}\) from \(\Psi\).

    In~\autoref{question:zs}, ensuring that the antecedent of the `would' conditional is always interpreted relative to the agent's present epistemic state.%
    \footnote{
      Basic understanding of subjunctive conditionals: Lewis, Stalnaker, Veltman, etc.
    }
    In the paraphrase the three clauses are combined into a single statement with an implicit temporal ordering, so the relevant subjunctive antecedents are already in place.%
    \footnote{
      Possible to omit \ref{question:zs:option} if rewrite \ref{question:zs:subjunctive}.
    }

  \item
    The relationship between failing to conclude \(\pv{\psi}{v'}\) from \(\Psi\) and refraining to conclude \(\pv{\phi}{v}\) from \(\Phi\) captured by~\ref{question:zs:subjunctive} is replaced in the paraphrase by `hence'.
    Implicit is that failing to conclude \(\pv{\psi}{v'}\) from \(\Psi\) explains why the agent would refrain from concluding \(\pv{\phi}{v}\) from \(\Phi\).

  \end{itemize}
\end{note}

\begin{note}[Quantified conditional]
  Whether there is such a \(\pvp{\psi}{v'}{\Psi}\).

  Clauses~\ref{question:zs:option},~\ref{question:zs:subjunctive}, and~\ref{question:zs:may-fail} of~\autoref{question:zs} from a quantified conditional, so there are three distinct ways in which~\ref{question:zs} may receive a positive answer.
  \begin{itemize}
  \item
    For any proposition-value-premises pairing \(\pvp{\psi}{v'}{\Psi}\), either:
    \begin{enumerate}[label=\alph*\('\).]
    \item
      It is not the case that the agent has the option of concluding \(\pv{\psi}{v'}\) from \(\Psi\), given the agent's reasoning from \(\Phi\) to \(\pv{\phi}{v}\).
    \item
      The agent may conclude \(\pv{\phi}{v}\) from \(\Phi\), (even) if they were to attempt and fail to conclude \(\pv{\psi}{v'}\) from \(\Psi\).
    \item
      The agent would conclude \(\pv{\psi}{v'}\) from \(\Psi\), if they attempted to do so.
    \end{enumerate}
  \end{itemize}

  In other words, negative answer only if from the agent's perspective there is some \(\pvp{\psi}{v'}{\Psi}\) such that the agent may fail to conclude \(\pv{\psi}{v'}\) from \(\Psi\), and if the agent were to fail they would not conclude \(\pv{\phi}{v}\) from \(\Phi\).%
  \footnote{
    Same for the paraphrase.
    Existential.
  }
\end{note}

\begin{note}
  Our interest is primarily in cases where both~\ref{question:zs:option} and~\ref{question:zs:subjunctive} hold for some \(\pvp{\psi}{v'}{\Psi}\)-pairing.

  \autoref{cha:zS:sec:question:illu} will contain additional \illu{1} of positive and negative answers to~\autoref{question:zs}, and in particular \illu{1} where~\ref{question:zs:option} and~\ref{question:zs:subjunctive} fail.

  However, our motivation for considering~\autoref{question:zs} is the relation between concluding (or failing to conclude)  \(\pv{\phi}{v}\) from \(\Phi\) and concluding (or failing to conclude) \(\pv{\psi}{v'}\) from \(\Psi\).

  First link the kind of case we are interested in asking~\autoref{question:zs} to \cScen{1}, and make this link explicit by revisiting a pair of \scen{1}.

  Second, highlight why.

  Third, features in additional detail.

  Following, in~\autoref{cha:zS:sec:zS}, kind of support: \zS{}.
  Examples of \zS{} will also provide additional examples of \scen{0}.
\end{note}

\begin{note}[\cScen{1}, examples]
  {
    \color{red}
    In such cases,~\autoref{question:zs} focuses on whether the agent, from their perspective, would conclude \(\pv{\psi}{v'}\) from \(\Psi\).

    We are already familiar with cases of this kind.
    Indeed, as case of this kind is a \cScen{0}.
    For, \cScen{1}, \dots.

    For the relevant conditionals in \cScen{1}, is it the case, from the agent's perspective, that they would conclude.
  }

  To illustrate, consider again \autoref{illu:gist:roots} and \autoref{illu:sketch:prop-logic}:
\end{note}

\begin{note}[\autoref{illu:gist:roots}]
  \autoref{illu:gist:roots} involves an agent concluding either \(x = 1\) or \(x = -\sfrac{1}{2}\) from premise that for some \(x \in \mathbb{R}\), \(2x^{2} - x - 1 = 0\).%
  \footnote{
    Abstractly, \autoref{illu:gist:roots} is a case where the agent would not conclude \(\pv{\phi}{v}\) from \(\Phi\) if the agent failed to conclude \(\pv{\phi}{v}\) from \(\Psi\).
    I.e.\ the relevant conclusion is the same in both proposition-value-premises pairings, the only difference is the relevant pools of premises (and method of reasoning).
  }
  And, when concluding either \(x = 1\) or \(x = -\sfrac{1}{2}\) the agent observes that \emph{if} \(x = 1\) or \(x = -\sfrac{1}{2}\), then they would also be able to observe this via factorisation.

  In other words, if the agent attempted to conclude either \(x = 1\) or \(x = -\sfrac{1}{2}\) via factorisation and failed, the agent would not conclude either \(x = 1\) or \(x = -\sfrac{1}{2}\) via (their application of) the quadratic formula.
\end{note}

\begin{note}[\autoref{illu:sketch:prop-logic}]
  Likewise, \autoref{illu:sketch:prop-logic} involves an agent concluding some sentence \(A\) is a syntactic theorem of propositional logic via a formula derivation.%
  \footnote{
    Abstractly, \autoref{illu:gist:roots} is a case where the agent would not conclude \(\pv{\phi}{v}\) from \(\Phi\) if the agent failed to conclude \(\pv{\psi}{v}\) from \(\Psi\), where \(\phi\) is distinct from \(\psi\) and \(\Phi\) is distinct from \(\Psi\).
    Soundness (and completeness) relates syntactic and semantic theorems of propositional logic, but these are distinct, as may be observed by considering, for example, a logic which is incomplete, or an unsound proof system.
  }

  And, when concluding \(A\) is a syntactic theorem, the agent observes that \(A\) is a syntactic theorem only if \(A\) is also a semantic theorem (from soundness).

  In other words, if the agent attempt to show \(A\) is true under an arbitrary valuation and failed, the agent would not conclude \(A\) is a syntactic theorem.
\end{note}

\begin{note}[In general]
  Generally speaking, the proposition-value-premises pairing present in a \cScen{0} just is what is required for both~\ref{question:zs:option} and~\ref{question:zs:subjunctive} to hold.
  Hence, when~\autoref{question:zs} is paired with an \cScen{0},~\autoref{question:zs} asks whether the agent, from their perspective, would conclude the relevant proposition-value pair from the relevant pool of premises.
\end{note}


\begin{note}
  With example, failure to conclude because 


  Broader, failure to conclude because no conclusion.

  `\emph{Unless}'
  {
    \color{red}
    Strong understanding.
    In short, always question regarding anything weaker.
    However, we will argue for this.
  }
\end{note}

\subsection[Visualisation]{Visualisation of what is at issue when asking \qzS{}}

\begin{figure}[h]
  \centering
  \begin{tikzpicture}
    \node (origin) at (0,0) {};
    \node (psiSplit) at (1,0) {};
    \node (phiSplit) at (4,0) {};
    %
    \node[anchor=west] (psiV) at  (6,-1)  {\(\pvp{\psi}{v'}{\Psi}\)};
    \node[anchor=west] (psiNv) at (6,-2) {\(\pvp{\psi}{\overline{v'}}{\Psi}\)};
    \node[anchor=west] (psiQ) at (6,-3) {\(\pvp{\psi}{?}{\Psi}\)};
    %
    % \node[anchor=west] (psiVPhiV) at (9,-1) {\(\pv{\phi}{v}\)};
    \node[anchor=west] (psiNvPhiU) at (9,-2) {\(\pv{\phi}{\{\overline{v},?\}}\)};
    \node[anchor=west] (psiQPhiU) at (9,-3) {\(\pv{\phi}{\{\overline{v},?\}}\)};
    %
    \node[anchor=west] (phiQ) at (10,1) {\(\pv{\phi}{?}\)};
    \node[anchor=west] (phiNv) at (10,2) {\(\pv{\phi}{\overline{v}}\)};;
    \node[anchor=west] (phiV) at (10,0) {\(\pv{\phi}{v}\)};
    %
    \draw[-]  (origin) -- (phiV);
    %
    \path[-,dashed] (phiSplit) edge [out=0, in=180] (phiNv);
    \path[-,dashed] (phiSplit) edge [out=0, in=180] (phiQ);
    %
    \path[-.] (psiSplit) edge [out=0, in=180] (psiV);
    \path[-, dashed] (psiSplit) edge [out=0, in=180] (psiNv);
    \path[-, dashed] (psiSplit) edge [out=0, in=180] (psiQ);
    %
    \draw[<-,dotted] (psiV) edge [out=0, in=180] (phiV);
    \draw[->, dotted] (psiNv) edge (psiNvPhiU);
    \draw[->, dotted] (psiQ) edge (psiQPhiU);
    \end{tikzpicture}
    \caption{Sketch of when \qzS{} has a negative answer.}
    \label{fig:csN:illu:overview}
  \end{figure}

\begin{note}[Figure]
  \autoref{fig:csN:illu:overview} provides a rough visualisation of~\qzS{}.

  The flat line captures the agent's reasoning, which concludes with \(\pv{\phi}{v}\).
  In concluding \(\pv{\phi}{v}\) the agent rules out two possibilities with respect to \(\phi\).
  First, that \(\phi\) does not have value \(v\), indicated by \(\pv{\phi}{\overline{v}}\).
  Second, that the agent does not assign any value to \(v\), indicated by \(\pv{\phi}{?}\).
  Prior to concluding \(\pv{\phi}{v}\), the agent's reasoning may have branched to either alternative path, but as the agent has concluded \(\pv{\phi}{v}\), neither path is viable, and hence both paths are represented with a dashed line.

  So far, we have seen only that the agent has concluded \(\pv{\phi}{v}\).

  We now consider some proposition-value-premises pairing \(\pv{\psi}{v'}{\Psi}\) such that if the agent were to fail to conclude \(\pv{\psi}{v'}\) from \(\Phi\), the agent would not conclude \(\pv{\phi}{v}\) from \(\Phi\).

  Intuitively, the dotted arrows from the various combinations of \(\psi\) and \(\{v',\overline{v'},?\}\) read, from top to bottom:
  \begin{itemize}
  \item If \(\phi\) has value \(v\) then the agent may conclude \(\pv{\psi}{v'}\) from \(\Psi\), and:
  \item If the agent concludes \(\psi\) has some value \(\overline{v'}\) from \(\Psi\), then the agent either concludes \(\phi\) has some value other than \(v\), or the agent fails to reach a conclusion regarding \(\phi\) from \(\Phi\).
    Both options are combined via the shorthand \(\pv{\phi}{\{\overline{v},?\}}\).
  \item
    And, likewise if the agent fails to conclude \(\pv{\psi}{v'}\) from \(\Psi\).
  \end{itemize}

  With respect to concluding, observe that prior to ruling out alternative branches with respect to \(\pv{\phi}{\{\overline{v},?\}}\), the agent may have reasoned about whether \(\psi\) has value \(v\).
  And, from the agent's perspective, \(\phi\) has value \(v\) only if \(\psi\) has value \(v'\).
  If \(\psi\) does not have value \(v'\), then either \(\phi\) does not have value \(v\), or the agent's reasoning would not conclude with a value for \(\phi\), indicated by \(\pv{\phi}{\{\overline{v},?\}}\).

  Hence, prior to concluding \(\pv{\phi}{v}\), the agent has concluded \(\pv{\psi}{v'}\).
\end{note}

\begin{note}
  Broadly, then, we may say that an agent has {\color{red} particular kind of conclusion} for \(\pv{\phi}{v}\) just in case when concluding \(\pv{\phi}{v}\) it is not the case that the agent's reasoning would have branched to a different conclusion with respect to \(\phi\).

  However, the visualisation of~\autoref{fig:csN:illu:overview} and this broad statement of {\color{red} positive answer to \qzS{}} are a little too broad.
  For, we are only interested in proposition-value pairs guaranteed by \(\phi\) having value \(v\).
  {\color{red} positive answer to \qzS{}} is not global with respect to all proposition-value pairs that the agent may have reasoned about, but local to those guaranteed by the proposition.
\end{note}

\section{\emph{Why}}
\label{cha:zS:section:qzs-and-why}

\begin{note}
  Intuitively, positive answer in \scen{1}.

  Some care has been taken.

  \autoref{illu:sketch:prop-logic}.
  Only if semantic proof.
  Syntactic proofs, at least in my experience, may be out of reach.
  However, semantic proofs, often straightforward.

  Converse may hold, but more challenging than \autoref{illu:sketch:prop-logic}.

  Similarly, \autoref{illu:gist:roots}.
  Factorisation isn't too difficult.

  \autoref{illu:sketch:math}.
  \(345 \div 15 = 23\) \emph{only if} \(23 \times 15 = 345\).

  Agent has the opportunity, and current result looks good.
\end{note}

\begin{note}
  Now, answers, in part, why.
\end{note}

\begin{note}
  Rough understanding.
  In terms of broader argument, \emph{why}.

  Idea is that agent's perspective regarding \(\pvp{\psi}{v'}{\Psi}\) in part explains why.

  Preferred \illu{0} concerns whether one has or has not lost their keys.
\end{note}

\begin{note}[Motivating \illu{0}]
  {
    \color{red}
    Interest in terms of explaining why and agent did or didn't conclude.
  }

  \begin{illustration}[Lost keys]
    \label{illu:lost-key}
    Tempting as it may be to conclude that a pair of keys are lost after some searching, if the keys really are lost then there aren't in a handful of places you haven't yet thought to look. And, until you have concluded that the keys really aren't in those places, and that there is no-where else to look, the keys aren't really lost.
  \end{illustration}

  For the agent's perspective, there is some \(\pvp{\psi}{v'}{\Psi}\), and this explains why the agent does not conclude they have lost their keys.
\end{note}

\begin{note}
  Similar points for examples given.

  Check via factorising, check via a semantic proof.
\end{note}

\begin{note}
  Here, important, \issueConstraint{} asks about whether a relation of support is part of why an agent \emph{concludes} (only if agent has witnessed).

  With~\autoref{illu:lost-key},~\autoref{illu:gist:roots}, and~\autoref{illu:sketch:prop-logic}, failure to conclude!
\end{note}

\begin{note}
  Still, \emph{negative} answers to~\autoref{question:zs}.

  Interest, what about \emph{positive} answers?

  Positive answer only if from agent's perspective, would conclude.
  No witnessing.
  Relation of support is part of why.

  Lack of support explains, in part, why agent does \emph{not} conclude.
  Conversely, presence of support explains, in part, why agent \emph{does} conclude.
\end{note}

\paragraph{Basic principle}

\begin{note}
  \begin{idea}[\autoref{question:zs} and \emph{Why}]
    \label{prop:qzS-answers-why}
    There are cases in which:

    \begin{itemize}
    \item[\(\pm\)]
      Answers to \autoref{question:zs} answer, in part, why an agent concludes or does not conclude \(\pv{\phi}{v}\) from \(\Phi\).
    \end{itemize}

    Specifically:
    \begin{itemize}
    \item[\(-\)]
      A negative answer to~\autoref{question:zs} answers, in part, why an agent does not conclude \(\pv{\phi}{v}\) from \(\Phi\).
    \item[\(+\)]
      A positive answer to~\autoref{question:zs} answers, in part, why an agent does conclude \(\pv{\phi}{v}\) from \(\Phi\).
    \end{itemize}
  \end{idea}

  Motivation for \autoref{prop:qzS-answers-why} is by cases.
  See additional cases in \autoref{cha:zS:sec:question:illu}.

  Weak point.
  \autoref{prop:qzS-answers-why} is central to overall argument.
  Hence, something else which captures cases.
  Something else does not involve whether or not the agent would conclude.

  I do not think so.
  But, generalising from exhaustion.
  Have not exhausted every possibility, but I've exhausted myself.

  Preferable, I think, to hold that \qzS{} is not relevant.
  \autoref{prop:qzS-answers-why} is an existential.
  Generally speaking, would be good.
  Only trouble when \(\pvp{\psi}{v'}{\Psi}\) is such that the agent has not witnessed reasoning to \(\pv{\psi}{v'}\) from \(\Psi\).
  So, if \qzS{} only applies in such cases, then no problem.

  However, already seen, \autoref{illu:lost-key}.
  Absence of reasoning.

  So, narrow to no cases where a positive answer, but consider \cScen{1}.
\end{note}

\begin{note}
  Run this through \scen{1} listed.
\end{note}

\begin{note}
  \fc{}!
  Or, forgone conclusion, but this is different from a \fc{}.
\end{note}

\begin{note}
  What's needed is positive answer only if support.

  Here, maybe illustrate with general ability.
  Got X.
  General ability.
  So, specific ability to Y.
\end{note}

\begin{note}
  {
    \color{red}
    There's a difference between answering `no' and failing to answer.
    But, the point I'm arguing for works given this distinction.
    There's no real different.
    I mean, the conditional is either true or false.
    But, it's possible that the falsity of the conditional has a role where the truth of the conditional does not.
  }
\end{note}

\subsection{Deviance}
\label{sec:deviance}

\begin{note}
  Here, causal deviance.
\end{note}

\begin{note}
  Problem is, there's no way to guarantee a link between positive answer to \qzS{} and the agent concluding or not refraining from concluding.
\end{note}

\begin{note}
  Argument relies on tying content to explanation.

  In this respect, there is room for an objection.
  Deviant causal chains.
  Point here is that there are cases where these come apart.

  This isn't only a problem for causal theories of reasoning.
  The point is, some instantiation, and so long as act may be caused by something else, then possibly caused by the instantiation.

  So, possible here.

  Well, hold on.
  What is need is the relevance of the content.
  For this objection to work, need to take a theoretical perspective.
  See, in Davidson's case, the idea is fusing these two things together.
  We answer two different questions with a common thing viewed in two ways.

  Still, I think the objection can be pressed!
  Only \emph{really} an explanation is no deviance.
  To the same extent that potential event matters, it matters to the agent that there is no deviance.

  {
    \color{red}
    Resolution is, if deviance, then no agency.
  }

  I think this makes sense, or at least makes enough sense.
  Answers to `why', on this understanding, are tentative.

  Or, rest on presupposition that agent performed the action.

  So, contingent on showing there is no causal deviance.

  This is different to error.
  With error, thing appealed to isn't the case, but appeal still did work.
  Here, it doesn't matter whether or not the case, no work is done.

  In contrast to more typical instances of the problem, don't need to rule out deviant causal chains.
  Instead, just need one instance to fail to hold.
  One instance of non-deviousness.

  Still a problem for a compatible account which avoids.
  For, here, there can't be any direct link from perspective to reason.

  For example, \citeauthor{Hieronymi:2011aa}

    \begin{quote}
      [W]e explain an event that is an action done for reasons by appealing to the fact that the agent took certain considerations to settle the question of whether to act in some way, therein intended so to act, and successfully executed that intention in action.
    [\emph{T}]\emph{his} complex fact, [\dots] is the reason that rationalizes the action---that explains the action by giving the agent's reason for acting.%
    \mbox{ }\hfill\mbox{(\citeyear[431]{Hieronymi:2011aa})}
  \end{quote}

  So, here, considerations which settle question, and in so settling question.
  Link between settling the question and acting.

  Following \citeauthor{Hieronymi:2011aa}, no room for deviance.
  Too tight.

  In other words, so long as this fact holds, there is no distinction between settling the question and acting.
  Therefore, no deviance.

  Compatible, I think.
  Question is whether in resolving \qzS{} is sufficiently tied to resolving the question \citeauthor{Hieronymi:2011aa} identifies.
  And, plausibly is.
  This is what the motivation for \qzS{} did.

  Trouble is, for our purposes, need at least sufficient conditions for when this complex fact obtains.
  And, no account of this.

  \citeauthor{Hieronymi:2011aa} notes the gaps.

  Some tension.
  These considerations aren't premises.
\end{note}

\begin{note}
  So, the other option is to embrace deviant causal chains.
  Have the content, but this doesn't work in the way the agent thinks it does.

  Example from Davidson.

  The trouble here is that the content and resulting action match.
  So, things make sense from the agent's perspective.

  Deviant, but maybe not so deviant here.

  Systematic deviance, where content is separated from role of mental state.

  But, I see no motivation for this.

  Solution to causal chains doesn't get round this, because the result is a restricted account.
  So, there's no guaranteed trade-off here.
  Trouble is, it seems hard to see a case where this wouldn't be the case.
\end{note}


\subsection{Details}
\label{cha:zS:sec:details}

\subsubsection{Clauses~\ref{question:zs:option},~\ref{question:zs:subjunctive}, and the idea of a \requ{0}}
\label{cha:zS:sec:clauses-idea-requ1}

\paragraph{The components of \qzS{}}

\begin{note}
  The primary clauses of interest are clauses~\ref{question:zs:option} and~\ref{question:zs:subjunctive}.

  Intuitively, clause~\ref{question:zs:option} means that, so long as \(\phi\) has value \(v\), the agent has the option of checking whether it makes sense for the agent to conclude \(\pv{\phi}{v}\) from \(\Phi\).
\end{note}
  And, clause~\ref{question:zs:subjunctive} expresses that concluding \(\pv{\psi}{v'}\) from \(\Psi\) is a check on whether it makes sense for the agent, from their perspective, to conclude \(\pv{\phi}{v}\) from \(\Phi\).

\paragraph{General}

\begin{note}
  Constraints placed on \(\pvp{\psi}{v'}{\Psi}\).
  From reasoning involved in process of concluding \(\pv{\phi}{v}\) from \(\Phi\).
  Would lead to failure.

  Conditionals.

  Involved in concluding \(\pv{\phi}{v}\) from \(\Phi\).
  First, enough to break.
  Second, reasoning makes this proposition-value-premises pairing available.

  Pair of additional features

  Second highlights why \(\pvp{\psi}{v'}{\Psi}\) is of interest.
  However, in this respect, not strictly required.
  Given universal, will also include these.

  First,
  Don't need \(\phi\) to have value \(v\).
  Also, implicit, no revision.
  Built up various things in reasoning, and given all of this\dots


  And, maybe reasoning offers something new.
  Though, not the case that \(\pvp{\psi}{v'}{\Psi}\) only from something new.
  Might be the case that negative answer because go off on wrong reasoning.
\end{note}

\paragraph{Option}

\begin{note}
  Hence, it need not be the case that the agent has the option of concluding \(\pv{\psi}{v'}\) from \(\Psi\) from their epistemic state prior to starting line of reasoning (as the agent has not yet concluded that \(\phi\) has value \(v\)).
\end{note}


\paragraph{Would not conclude}

\begin{note}
  Noted failure.
\end{note}

{
  \color{red}
  Note, \ref{question:zs:may-fail} is delicate.
  For, the combination of \ref{question:zs:subjunctive} and \ref{question:zs:option} suggest there is a way of concluding \(\pv{\psi}{v'}\) from \(\Psi\).
  Hence, \ref{question:zs:may-fail} may be read in reference to this.
  However, \ref{question:zs:may-fail} is intended to allow other ways of concluding \(\pv{\psi}{v'}\) from \(\Psi\).
  What matters is that the agent has not concluded \(\pv{\psi}{v'}\) from \(\Psi\), the agent has the option, and the agent may fail.%
  \footnote{
    This is important for witnessing, but also motivated by different methods.
    A different way to putting this is that concluding is two place relation.
    Between premises and conclusion.
    Concluding is not a three place relation between premises, conclusion, and method.
    I should really have this stated as an assumption.

    Still, there is a variant where method comes into play, as I have this via ability.
  }
}

%%%% TEMP from question
\footnote{
  Clause~\ref{question:zs:subjunctive} is expressed by a subjunctive conditional as there is no requirement that the agent will attempt to conclude \(\pv{\psi}{v'}\) from \(\Psi\).

  \color{red}
  As this alternative expression makes clear,~\autoref{question:zs} focuses on the agent (and their epistemic state).
  At no point do we consider any variation of the agent's epistemic state.
  Likewise,~\autoref{question:zs} concerns only the agent's perspective on concluding \(\pv{\psi}{v'}\) from \(\Psi\).
  Whether or not the agent would conclude \(\pv{\psi}{v'}\) from \(\Psi\) is irrelevant.
  What matters is whether, from the agent's perspective, there is potential for reasoning about whether \(\pv{\psi}{v'}\) follows from \(\Psi\) to block concluding \(\pv{\phi}{v}\) from \(\Phi\).
}

\paragraph*{Minor clarifications}

\begin{note}[Importance of \csN{}]
  First, agent's reasoning.
  At issue is whether the agent may reason to a different conclusion.
  There's nothing that would lead me elsewhere.

  Second, agent's reasoning.
  Independent of whether \(\phi\) has value \(v\), \(\psi\) has value \(v'\), or any of the premises.
  Need not be the case that satisfaction amounts to anything substantial.
  No clause for justification, etc.

  Third, competence, rather than performance.
\end{note}

\subsection{\requ{3}}

\begin{note}
  We begin by refining the relevant \(\pvp{\psi}{v'}{\Psi}\) proposition-value-premises pairings of interest from~\qzS{}.
  We term such proposition-value-premises pairings `\requ{1}' of concluding \(\pv{\phi}{v}\) from \(\Phi\).
\end{note}

\begin{note}[Notion of a \requ{}]
  \begin{idea}[\iRequ{}]
    \label{idea:requ}
    \(\pvp{\phi}{v'}{\Psi}\) is a \emph{\requ{}} of concluding \(\pv{\phi}{v}\) from \(\Phi\), with respect to an agent \vAgent{}'s epistemic state if:
    \begin{enumerate}
    \item
      \label{idea:requ:main}
      From the perspective of \vAgent{}' epistemic state, \(\phi\) has value \(v\) only if:
      \begin{enumerate}[label=\alph*., ref=\named{R:\alph*}]
      \item
        \label{idea:requ:pool}
        \vAgent{} has the option of concluding \(\pv{\psi}{v'}\) from \(\Psi\) where:
        \begin{enumerate}[label=\roman*., ref=\named{R:a.\roman*}, series=csIdeaCounter]
        \item
          \label{idea:requ:pool:int}
          \vAgent{} may conclude \(\pv{\psi}{v'}\) from \(\Psi\) without concluding \(\pv{\phi}{v}\) from \(\Phi\) as an intermediary step.
        \item
          \label{idea:requ:pool:ind}
          For any proposition-value pair \(\pv{\psi_{i}}{v_{i}}\) in \(\Psi\), \vAgent{} either has concluded or may conclude \(\pv{\psi_{i}}{v_{i}}\) without concluding \(\pv{\phi}{v}\) from \(\Phi\).
        \end{enumerate}
      \item
        \label{idea:requ:nPsi-nPhi}
        If \vAgent{} were to fail to conclude \(\pv{\psi}{v'}\) from \(\Psi\) prior to reasoning about whether \(\phi\) has value \(v\) given \(\Phi\), \vAgent{} would not conclude \(\pv{\phi}{v}\) from \(\Phi\).
      \end{enumerate}
    \end{enumerate}
    \vspace{-\baselineskip}
  \end{idea}

  With the key clause linking~\autoref{idea:requ} to~\qzS{} is clause~\ref{idea:requ:nPsi-nPhi}.
  For, clause~\ref{idea:requ:nPsi-nPhi} captures the core idea of failure to conclude \(\pv{\psi}{v'}\) from \(\Psi\) leading to failure to conclude \(\pv{\phi}{v}\) from \(\Phi\).

  The role of clause~\ref{idea:requ:pool} is explicitly state various properties \(\pv{\psi}{v'}{\Psi}\) must have in order for any failure to conclude \(\pv{\psi}{v'}\) from \(\Psi\) is relevant to concluding \(\pv{\phi}{v}\) from \(\Phi\).%
  \footnote{
    Indeed, we take \ref{idea:requ:pool:int} and~\ref{idea:requ:pool:ind} to be more-or-less implicit constraints on \(\pvp{\psi}{v'}{\Psi}\) in the statement of \qzS{}.
  }
  In particular, \ref{idea:requ:pool:int} and~\ref{idea:requ:pool:ind} are required to ensure the agent may conclude \(\pv{\psi}{v'}\) from \(\Psi\) independently of concluding \(\pv{\phi}{v}\) from \(\Phi\).

    For, if \ref{idea:requ:pool:int} and \ref{idea:requ:pool:ind} were to fail to hold then:
  \begin{itemize}
  \item
    By~\ref{idea:requ:pool:int}, the agent would need to conclude \(\pv{\phi}{v}\) from \(\Phi\) as a sub-conclusion when reasoning from the relevant pool of premises \(\Psi\).
    Hence, it would not be possible to conclude \(\pv{\psi}{v'}\) from \(\Psi\) without first concluding \(\pv{\phi}{v}\) from \(\Phi\).
  \item
    And, likewise, by~\ref{idea:requ:pool:ind}, the agent need to have already concluded \(\pv{\phi}{v}\) from \(\Phi\) in order to appeal to some of the proposition-value pairs in the relevant pool of premises \(\Psi\).
  \end{itemize}

  Conversely, if both~\ref{idea:requ:pool:int} and~\ref{idea:requ:pool:ind} hold, the agent may conclude \(\pv{\psi}{v'}\) from \(\Psi\) independently of concluding \(\pv{\phi}{v}\) from \(\Phi\).

  Note, however, neither~\ref{idea:requ:pool:int} nor~\ref{idea:requ:pool:ind} rule out the possibility of the agent concluding \(\pv{\phi}{v}\) from \(\Phi\) when concluding \(\pv{\psi}{v'}\) from \(\Psi\) or, conversely, concluding \(\pv{\psi}{v'}\) from \(\Psi\) when concluding \(\pv{\phi}{v'}\) from \(\Phi\).
  There may be an interesting variant of the notion of a \requ{} with such a constraint in place, but such a constraint is not of interest with respect to \qzS{}.
  For, at issue is only whether the agent may at interest is only failure to conclude \(\pv{\psi}{v'}\) from \(\Psi\), and both~\ref{idea:requ:pool:int} and~\ref{idea:requ:pool:ind} ensure lack of concluding \(\pv{\phi}{v}\) from \(\Phi\) will not prevent the agent from reaching a conclusion regarding whether \(\psi\) has value \(v\) given \(\Psi\).
\end{note}

\begin{note}
  \color{red}
  Has the option.
\end{note}

\begin{note}[\requ{2}: Partial check]
  Intuitively, concluding \(\pv{\psi}{v'}\) from \(\Psi\) would serve as a partial check on whether the agent may reason to a conclusion other than \(\pv{\phi}{v}\), captured by~\ref{idea:requ:nPsi-nPhi}.

  Concluding \(\pv{\psi}{v'}\) from \(\Psi\) is a check.
  For, if the agent were to fail to conclude \(\pv{\psi}{v'}\) from \(\Psi\) then, the agent would not conclude \(\pv{\phi}{v}\) from \(\Phi\), from the agent's perspective.
  Hence, contraposing, the agent would conclude \(\pv{\phi}{v}\) from \(\Phi\) only if the agent would conclude \(\pv{\psi}{v'}\) from \(\Psi\), from the agent's perspective.

  However, the check is partial, as it need not be the case that the agent would conclude \(\pv{\psi}{v'}\) from \(\Psi\) only if the agent \(\pv{\phi}{v}\) from \(\Phi\).
  Therefore, failing to conclude \(\pv{\psi}{v'}\) from \(\Psi\) may block concluding \(\pv{\phi}{v}\) (from the perspective of the agent) though concluding \(\pv{\psi}{v'}\) from \(\Psi\) need not ensure that the agent would conclude \(\pv{\phi}{v}\).

  Combining these two ideas, intuitively, \(\pv{\psi}{v'}\) is a \requ{} of concluding \(\pv{\phi}{v}\) just in case there is some pool of premises \(\Psi\) such that determining whether the agent would conclude \(\pv{\psi}{v'}\) is an independent partial check on whether the agent may reason to a conclusion other than \(\pv{\phi}{v}\).
\end{note}


\subsubsection{Abstract}

\begin{note}[Abstract motivation]
  First, the combination of~{\color{red} ???} and~{\color{red} ???} keeps the complexity of resolving~\qzS{} relatively low.
  We only need to consider what \(\phi\) having value \(v\) would commit the agent to, so to speak.
  For example, we do not need to consider the consequences of the agent reasoning about some arbitrary proposition-value pair \(\pv{\chi}{v''}\) prior to concluding \(\pv{\phi}{v}\) from \(\Phi\).

  Of course, if the agent would conclude both \(\pv{\chi}{v''}\) and \(\pv{\chi}{\overline{v''}}\) for some \(\chi\), then it seems the agent's epistemic state is in bad shape.
  Still, given that an agent will typically revise their epistemic state upon concluding both \(\pv{\chi}{v''}\) and \(\pv{\chi}{\overline{v''}}\) for some \(\chi\), such concerns may be isolated to a distinct question.

  Second, we avoid --- to some extent --- concerns about over-generating.
  Our overall argument will put~\autoref{question:zs} to work in motivating a negative resolution to~{\color{red} issue:Main}.
  Primarily by drawing consequences from what we will argue is an equivalent characterisation of negative resolutions to~\autoref{question:zs}.
  A concern is that this overall argument may over-generate.
  Given that a negative resolution to~{\color{red} issue:Main} seems by no means clear, unintended consequences of our interest in~\autoref{question:zs} may diminish interest in the tension we hope to motivate, and hence tip favour to a positive answer to~{\color{red} issue:Main}.
  Whether or not there are unintended consequences of broadening the scope of~\autoref{question:zs} is unclear to me.
  Still, without need to investigate, we may ignore any such consequences that may arise.
\end{note}

\subsection*{Narrowing \requ{1}}

\begin{note}[Expanding pool constraints]
  To~\ref{idea:requ:pool} of~\autoref{idea:requ} the following clause may also be added:
  \begin{enumerate}[label=]
  \item
    \begin{enumerate}[label=]
    \item
      \begin{enumerate}[label=\roman*., ref=(\roman*), resume*=csIdeaCounter]
        \setcounter{enumiii}{3}
      \item
        \label{idea:requ:pool:method}
        Concluding \(\pv{\psi}{v'}\) from \(\Psi\) involves the same general method the agent would use to conclude \(\pv{\phi}{v}\) from \(\Phi\).
      \end{enumerate}
    \end{enumerate}
  \end{enumerate}
  We omit~\autoref{idea:requ:pool:method} from the idea of \csN{} for two (related) reasons.
  First, it is not clear what `the same general method' amounts to in detail.
  Second, avoiding questions about method affords flexibility when providing \illu{1} of \zS{}.
  However,~\autoref{idea:requ:pool:method} may be imposed with no loss to the role of \zS{} in the overall argument.
  However, always a check on whether one has the general ability.
\end{note}

\begin{note}
  Reasoning, \support{}, would not reason to a different conclusion.

  Specifically, \requ{} of some conclusion.
  So long as conclusion, then it is possible to reason about whether \(\psi\) has value \(v'\), and unless conclude \(\psi\) has value \(v'\), would not conclude \(\phi\) has value \(v\).

  Intuitively, \requ{} as an independent check on the reasoning.
  If don't hold \(\psi\) from premises, then question about whether \(\phi\).

  Claiming support, necessary condition is satisfying all \requ{1}.
  Claiming support, then, is weaker than having support.
  Restricted to whether conclusion of reasoning would introduce a \requ{}.
  And, may be further restricted without impact to the tension we will develop to whether the conclusion would `clearly' introduce a \requ{}.
\end{note}

\subsubsection{Prior to concluding\dots}

\begin{note}[Prior to concluding\dots]
  An important feature of \qzS{} \dots

  Not particularly marked.
  Allow agent to have built up a bunch of stuff in the reasoning.

  Example.

  \begin{illustration}[Velocity]
    \label{ill:velocity}
    Agent is provided with information about how far a car has travelled north as a function of time travelled.

    From this, take the derivative of the function to obtain the (instantaneous) velocity of the car at a handful of points in time.

    And, from the (instantaneous) velocity of the car, the agent calculates the (instantaneous) acceleration of the car at each of the points in time.

    The agent also has information about the speed of the car as a function of time travelled, and the agent may calculate speed by the taking magnitude of the (instantaneous) velocity of the car.
  \end{illustration}

  

  \autoref{ill:velocity}, two step calculation.
  Velocity, acceleration.
  After the first step, check by taking the magnitude.
  Calculation of velocity is correct only if taking the magnitude matches speed.

  Just before concluding to include cases such as this.

  Note, \cScen{}.
\end{note}

\begin{note}
  Example highlights how `intermediate conclusions' relate.
  Further point of interest:
  Failure to conclude.

  Two ways to view agent's calculation of the velocity of the car.

  First, as a conclusion.
  Same status as the function.

  Or, as temporary.

  Difference in how we understand agent's present epistemic state.

  On first, the agent's present epistemic state is inconsistent.
  Two proposition-value pairs which conflict.
  Not possible for the car to have velocity the agent calculated and acceleration the agent has been informed of.

  May also be that the function and information about acceleration are inconsistent, but may also be that the agent made a mistake in calculating the velocity of the car.

  On second, the agent's present epistemic state may%
  \footnote{
    Don't have complete perspective on agent's present epistemic state.
  }
  consistent.

  For, made a mistake.
  But, proposition-value pair is not part of present epistemic state, so distinguished from function and information about acceleration, which are consistent.

  This is a distinction we have little interest in.
  What matters is failure to conclude speed.
  Result is either revising inconsistent epistemic state, or abandoning intermediary steps of reasoning.
\end{note}


\section{\zS{}}
\label{cha:zS:sec:zS}

\begin{note}[Answers to \qzS{}]
  For ease of reference, define.
  \begin{definition}[\izS{}]
    \label{idea:zS}
    For an agent \vAgent{}, after concluding \(\pv{\phi}{v}\) from \(\Phi\):
    \begin{itemize}
    \item
      \vAgent{} has \zS{} with respect to \(\pvp{\phi}{v}{\Phi}\).
    \end{itemize}

    \emph{if and only if}

    \begin{itemize}
    \item
      \qzS{} had a \emph{positive} answer when \vAgent{} was concluding \(\pv{\phi}{v}\) from \(\Phi\).
    \end{itemize}
  \end{definition}

  Slight problem here with going for after concluding.
  Possible, it seems, for relation of support to no longer hold.
  Likewise for \zS{}.
  For ease, assume agent has not retracted conclusion.
\end{note}

\begin{note}
  Now, two basic propositions follow.

  \begin{proposition}[When a agent has \zS{}]
    Agent and proposition-value-premises pairing.

    \begin{itemize}
    \item
      Agent \emph{has} \zS{} for \(\pv{\phi}{v}\) after concluding \(\pv{\phi}{v}\) from \(\Phi\).
    \end{itemize}

    \emph{if and only if}

    \begin{itemize}
    \item
      Either:
      \begin{enumerate}[label=(\alph*), ref=\alph*]
      \item
        There is no proposition-value-premises pairing \(\pvp{\psi}{v'}{\Psi}\) for which the relevant conditions are met.
      \item
        There is some proposition-value-premises pairing \(\pvp{\psi}{v'}{\Psi}\) but, from the agent's perspective, the agent would not fail to conclude \(\pv{\psi}{v'}\) from \(\Psi\).
      \end{enumerate}
    \end{itemize}
  \end{proposition}

  Second, when an agent does not have \zS{}.

  \begin{proposition}[When a agent does not have \zS{}]
    Agent and proposition-value-premises pairing.
    \begin{itemize}
    \item
      Agent \emph{does not} have \zS{} for \(\pv{\phi}{v}\) after concluding \(\pv{\phi}{v}\) from \(\Phi\).
    \end{itemize}

    \emph{if and only if}

    \begin{itemize}
    \item
      There is some proposition-value-premises pairing \(\pvp{\psi}{v'}{\Psi}\) such that, from the agent's perspective, the agent may fail to conclude \(\pv{\psi}{v'}\) from \(\Psi\).
    \end{itemize}
  \end{proposition}
\end{note}

\begin{note}
  Note, concluding is neither safe nor sensitive (assumption), so neither is \zS{}.
\end{note}

\section{Notes}
\label{cha:zS:sec:notes}

\paragraph*{When}

\begin{note}
  \emph{When} concluding \(\pv{\phi}{v}\) from \(\Phi\) in order to keep things simple.
  A variant of the question may be asked if the agent has (already) concluded \(\pv{\phi}{v}\) from \(\Phi\).
  Here, rather than asking whether the agent would not conclude \(\pv{\phi}{v}\) from \(\Phi\), we may ask whether the agent would revise their conclusion of \(\pv{\phi}{v}\) from \(\Phi\).
\end{note}

\paragraph*{Whether the agent may conclude \(\phi\) has value \(v\), regardless of \(\Phi\)}

\begin{note}
  Not about the proposition-value pair.
  Rather, it is about the concluding.
  At interest is not whether \(\phi\) has value \(v\), but whether it makes sense to conclude \(\pv{\phi}{v}\) from \(\Phi\).
  Of course, if the agent has no information about whether \(\phi\) has value \(v\), then this is also part of the picture, but that is a consequence of the base concern.
\end{note}

\paragraph*{Introduced by \(\pv{\phi}{v}\) from \(\Phi\)}

\begin{note}[Proposition-value-premise pairing introduced by \(\pv{\phi}{v}\) from \(\Phi\)]
  This restriction may seem arbitrary, and to some degree I think it is.
  Ideally, an agent concluding \(\pv{\phi}{v}\) is an instance of \csN{} just in case the agent would not have reasoned to a different conclusion if they were to reason first about any other proposition-value pair.
  However, the advantage of focusing on some proposition-value pair `required' by \(\phi\) having value \(v\) is a significant constraint on the range of proposition-value pairs an agent needs to consider in order to \csN{}.

  In general, it may not be clear which proposition-value pairs may lead an agent to fail to conclude \(\phi\) has value \(v\), but so long the proposition-value pair of interest is given by \(\phi\) having value \(v\), an exhaustive search over all other proposition-value pairs may be avoided.

  Indeed, we will say that an agent has \emph{\support{}} for \(\phi\) having value \(v\) just in case they would not have reasoned otherwise, and reserve \emph{\claiming{}} \support{} for the weaker notion.
\end{note}


\paragraph*{Inductive, abductive, etc.\ reasoning}

\begin{note}
  Narrow, but not too narrow.
\end{note}

\begin{note}
  This doesn't rule out inductive or abductive reasoning.
  Consider standard induction.
  Here, there may be novel information, but this is not available from the agent's present epistemic state, and \qzS{} only concerns the agent's present epistemic state.
  Perhaps the possibility alone would prevent conclusion.
  However, it seems most conclude in recognition of such possibility.
  Instead, what one would need is considerations against uniformity principle.

  Same for any bridge between probabilistic and full.
  Toss a coin \(n\) times, conclude it is fair.
  Possible to toss \(m\) more times, not fair.
  However, \(n\) is sufficient, then no problem.
  It is true that there is something more you could do, but this would require acquiring new information.
\end{note}



\paragraph*{Fragility}

\begin{note}
  Kind of fragility.
  If concludes \(\pv{\phi}{v}\) from \(\Phi\), and does not have \zS{}, then from agent's perspective, possibility of revision.

  Indeed, may break down into two components.

  First, possibility of different conclusion.
  Agent's epistemic state is potentially unstable.

  Second, isolation of potential instability to \(\pv{\phi}{v}\) from \(\Phi\).
\end{note}


\paragraph*{Normative?}

\begin{note}[Just a property]
  There's no kind of normative evaluation here.
  We do not hold any conclusion for which this fails is bad.
  Nor do we hold that any conclusion for which this holds is good.
  Indeed, \zS{} is narrow, far too narrow for a general evaluation.

  Indeed, whether or not \zS{} doesn't tell us anything about the relationship between \(\pv{\phi}{v}\) and \(\Phi\) in general, as relative to an agent's epistemic state.
  May be that there is some \(\pvp{\psi}{v'}{\Psi}\), but only due to some quirk of the agent.
\end{note}

\paragraph*{The agent concluding \(\pv{\psi}{v'}\) from \(\Psi\)}

\begin{note}
  From the perspective of the agent.
  It doesn't matter whether the agent really has the option.
  Indeed, this perspective is important for fragility.
\end{note}

\section{Scraps}
\label{sec:scraps}

\begin{note}
  \emph{However}, caution.
  For, as we have seen with testimony, it may be the case that status of a premises blocks a \requ{}.
  And, the argument given relies on the existence of a \requ{}.
  So, it may be the case that past reasoning blocks a \requ{}.
  Still, here, only need to deny this.
  Not saying that in every case agent's present reasoning is given priority.
  (Indeed, consider cases of being somewhat impaired, e.g., via exhaustion.
  Indeed, exhaustion is interesting.
  Basic consistency checks.
  Should be the case that conclude A, but just concluded \emph{not}-A, or something like this\dots)
  Rather, denying that past continues to secure in all instances.
  So, just need the potential to revise perspective on any previous conclusion.
\end{note}

\begin{note}
  An interesting observation here is that in certain this all arises, to a certain extent, because of general abilities.
  General ability spans multiple different proposition-value-premises pairings.
  Hence, all of these function as \requ{1}, so long as the agent has the option.

  General ability spans multiple different proposition-value-premises pairings.
  Hence, all of these function as \requ{1}, so long as the agent has the option.

  \begin{itemize}
  \item
    General and specific abilities.
  \item
    Answers to why, then.
    Note, here, that opportunity is interesting.
    The whole conjunction of all instance of the general ability is plausibly not a \requ{}.
    However, all that's needed is the \emph{individual} instances, and for these to raise a problem.
  \item
    The point is, \requ{1} for any general ability, and these are also \requ{1} for main pairing.
    (%
    Note --- or perhaps emphasise --- here, that the problem is \emph{not} recursive.
    Instead, the problem is about the spread.%
    )
  \item
    Here, then, ability is both the problem and the answer.
    What's interesting is the way in which ability functions.
    It's not merely \emph{that} the agent has the ability.
    Instead, it \emph{is} the ability.
  \end{itemize}
\end{note}

\begin{note}
  So, the way in which past reasoning relates is by ensuring that the agent would reach the same conclusion.
  About the agent's reasoning.
  \emph{How} rather than \emph{that}.

  Look, what we are getting is that the agent would conclude.
  If something were to happen, then some action would be performed.
  There's no distinction between the answer and performing the act, roughly.
  Or, better put, the answer \emph{is about present reasoning}.
  Answer states that in present reasoning, would not fail.


  It is about the agent's present epistemic state, and in particular what the agent's present epistemic state is capable of.

  In other words, ability.
  What answers is ability, in the sense that ability iff would.

  This is very important to the understanding of \fc{}.

  And, I kind of want to have ability as a gloss, while focusing on \fc{} to avoid going into ability in too much detail.

  So, positive answer, then it's the pairing \emph{being} a \fc{}.
  (I should always use this instance of the copula.)
\end{note}

%%% Local Variables:
%%% mode: latex
%%% TeX-master: "master"
%%% End:
