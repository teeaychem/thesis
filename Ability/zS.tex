\chapter{\requ{3} \curb{0} concluding}
\label{cha:zS}

\begin{note}[Intro, locating]
  Our goal motivate a negative resolution to~\issueConstraint{}.

  \autoref{cha:fcs}, \fc{1}.

  Goal of the present chapter is foundations to link \fc{1} to concluding.

  Split into three main sections.
  \begin{enumerate}[label=]
  \item
    \TOCLine{cha:zS:sec:lost-keys}.

    Two \scen{1}, \TNSketch{1} of the core phenomena, and breakdown of the \TNSketch{1}.
  \item
    \TOCLine{cha:zS:sec:curbs}

    The idea of \(\pvp{\psi}{v'}{\Psi}\) being (partial) \check{} on concluding \(\pv{\phi}{v}\) from \(\Phi\).
    Term we give is `\curb{}'.
  \item
    \TOCLine{cha:zS:sec:requs}

    Connecting the idea of a \curb{} to \ros{}, and hence establishing the structure for counterexamples to \issueConstraint{}.
  \end{enumerate}

  Intuitively: \qWhyV{}.
  Though, given the outline, resist.
  Present the \scen{1}.
  May jump ahead.
  Though, we will be careful to keep things straight.
  Account of \scen{1} is general, construct your own.
\end{note}

\section{Lost keys and sound rules}
\label{cha:zS:sec:lost-keys}

\begin{note}
  Previous chapter, \fc{1}.
  Potential event in which agent concludes.

  \fc{3} are about what the agent may reason to.

  Now, turn to broader, whether there is something incompatible.
  Alternative conclusion.
\end{note}

\begin{note}
  Use the \illu{} to provide a general introduction.
\end{note}

\begin{note}
  \begin{scenario}[Lost keys]
    \label{illu:lost-key}
    I think I might have lost my keys.
    I usually leave place my keys on the right side of my desk, next to a copy of~\citeauthor{Vickers:1989tr}'s~\citetitle{Vickers:1989tr} which I've been saving for a rainy day.
    And, my keys aren't there.

    I've searched on the desk, under the desk, and beside the desk.
    And, I haven't found my keys.

    Still, I haven't (yet, at least) \emph{concluded} that I've lost my keys.

    For, there might still be some place I haven't looked.
    If I think a little harder a figure out where that place is, I would conclude my keys might be in that place.
    And, my keys aren't lost if they are in that place.
    So, I might conclude that my keys aren't lost, which would conflict with concluding that my keys are lost.
  \end{scenario}

  You may disagree with the tension I see in~\autoref{illu:lost-key}.
  Perhaps it's fine to conclude my keys are lost while allowing for the possibility that they're some place I haven't yet thought of.
  However, there's tension for me.
  `I've lost my keys, but they might be under that book' feels bad to me, and to me the badness extends to `I've lost my keys, but they might in that place I haven't yet considered'.

  Though, my goal is only to convince you that my refusal to conclude I've lost my keys makes sense.
  The way in which you think about the truth conditions for the sentence `I've lost my keys' may be different, but I expect my thoughts are intelligible.
\end{note}

\begin{note}
  Granting the intelligibility of~\autoref{illu:lost-key}, our interest is with the following sketch:

  \begin{sketch}
    \label{sketch:zS:fail}
    For some agent, the follow two conditions obtain:
    \begin{enumerate}
      \label{skech:zS:fail:curb}
    \item
    There is some \(\pvp{\psi}{v'}{\Psi}\) such that, from the \agpe{}:
    \begin{enumerate}[label=\alph*., ref=(\alph*)]
    \item
      \label{sketch:zS:fail:curb:opportunity}
      Opportunity to reason about whether \(\pv{\psi}{v'}\) follows from \(\Psi\).
    \item
      \label{sketch:zS:fail:curb:conditional}
      If the agent took to opportunity, the agent would conclude \(\pv{\phi}{v}\) from \(\Phi\) \emph{only if} the agent would conclude \(\pv{\psi}{v'}\) from \(\Psi\).
    \end{enumerate}

  \item
    \label{sketch:zS:fail:no-c}
    The agent entertains the possibility of concluding \(\pv{\psi}{v'}\) from \(\Psi\), and so the agent do not conclude \(\pv{\phi}{v}\) from \(\Phi\).
  \end{enumerate}
  \vspace{-\baselineskip}
  \end{sketch}

  In short, the agent, from their perspective, thinks the may reason to some other conclusion, and therefore (it seems) the agent does not conclude.

  Filling in the details of the abstract sketch:
  \begin{itemize}[noitemsep]
  \item
    I am the agent.
  \item
    \(\phi\) is the proposition: `I've lost my keys'.
  \item
    \(\psi\) is a some proposition: `My keys are not in location \(l\)'
  \item
    Both \(v\) and \(v'\) are the value: `True'.
    And,
  \item
    The pools of premises \(\Phi\) and \(\Psi\) are left unspecified.
  \end{itemize}
\end{note}

\begin{note}
  \phantlabel{qzS:intro:qualification}
  \ref{sketch:zS:fail:curb:conditional}, qualified.
  `It seems'.

  First, identifying type of \scen{}.
  In this respect, as neutral as possible.
  For a \scen{} to be the case, don't need failure of \issueConstraint{}.
\end{note}

\begin{note}[I think about this\dots]
  \autoref{illu:lost-key} is introduces this phenomena as I seem to encounter the pattern every time I think I've lost something, whether keys, books, or files.
  After some searching I feel I should have accumulated enough evidence to conclude the item is lost.
  However, the item isn't lost (from my perspective, at least), while there remains a place to check, and experience shows I eventually think of a place to check and, more often than not, the item is there.
\end{note}

\begin{note}
  Closely related.
  Replace \ref{sketch:zS:fail:curb:conditional} with:
  \begin{enumerate}[label=\alph*\('\)., ref=(\alph*\('\))]
  \item
    \label{sketch:zS:fail:curb:conditional:var}
    The agent would conclude \(\pv{\phi}{v}\) from \(\Phi\) \emph{only if} the agent would not conclude \(\pv{\psi}{v'}\) from \(\Psi\).
  \end{enumerate}
  And, replace \ref{sketch:zS:fail:no-c} with:
  \begin{enumerate}[label=\arabic*\('\)., ref=(\arabic*\('\))]
    \setcounter{enumi}{1}
  \item
    \label{sketch:zS:fail:no-c:var}
    The agent entertains the possibility of concluding \(\pv{\psi}{v'}\) from \(\Psi\), and therefore (it seems) they do not conclude \(\pv{\phi}{v}\) from \(\Phi\).
  \end{enumerate}

  Given~\ref{sketch:zS:fail:curb:conditional:var}, the relevant instance of `\(\psi\)' would be: `My keys \emph{might be} in location \(l\)'.

  Given~\ref{sketch:zS:fail:curb:conditional:var} and~\ref{sketch:zS:fail:no-c:var}, focus shifts from failing to conclude something to concluding something.

  Plausible variation.
  Concerned about failing to conclude keys are not in location \(l\) or concerned about concluding keys might be in location \(l\).

  Benefit \ref{sketch:zS:fail}, failure to conclude \(\pv{\psi}{v'}\) is also a problem.
  Specific location.
  If keys are lost, then conclude not in location.
\end{note}

\begin{note}
  \autoref{sketch:zS:fail}, failing to conclude.
  Agent does not conclude \(\pv{\phi}{v}\) from \(\Phi\), in part, because the agent entertains the possibility of failing to conclude \(\pv{\psi}{v'}\) from \(\Psi\).

  Consider the following variation on~\autoref{sketch:zS:fail}:
  \begin{sketch}
    \label{sketch:zS:succeed}
    For some agent, the follow two conditions obtain:
    \begin{enumerate}
    \item
      \label{sketch:zS:succeed:curb}
      There is some \(\pvp{\psi}{v'}{\Psi}\) such that, from the \agpe{}:
      \begin{enumerate}[label=\alph*., ref=(\alph*)]
      \item
        \label{sketch:zS:succeed:curb:opportunity}
        Opportunity to reason about whether \(\pv{\psi}{v'}\) follows from \(\psi\).
      \item
        \label{sketch:zS:succeed:curb:conditional}
        If the agent took to opportunity, the agent would conclude \(\pv{\phi}{v}\) from \(\Phi\) \emph{only if} the agent would conclude \(\pv{\psi}{v'}\) from \(\Psi\).
      \end{enumerate}
    \item
      \label{sketch:zS:succeed:no-c}
      The agent \emph{does not} entertain the possibility of concluding \(\pv{\psi}{v'}\) from \(\Psi\), and so they conclude \(\pv{\phi}{v}\) from \(\Phi\).
    \end{enumerate}%
    \vspace{-\baselineskip}
  \end{sketch}

  Clause~\ref{skech:zS:fail:curb} is shared between \TNSketch{3}~\ref{sketch:zS:fail} and~\ref{sketch:zS:succeed}.

  The difference between \TNSketch{3}~\ref{sketch:zS:fail} and~\ref{sketch:zS:succeed} is with respect to clause~\ref{sketch:zS:fail:no-c}.
  In clause~\ref{sketch:zS:fail:no-c} of~\autoref{sketch:zS:fail} the agent entertains the possibility of failing to conclude \(\pv{\psi}{v'}\) from \(\Psi\).
  By contrast, in clause~\ref{sketch:zS:succeed:no-c} of~\autoref{sketch:zS:succeed} the agent \emph{does not} entertain the possibility of failing to conclude \(\pv{\psi}{v'}\) from \(\Psi\).
\end{note}

\begin{note}
  If \scen{1} satisfying~\autoref{sketch:zS:succeed}, then close to counterexample to \issueConstraint{}.
  For, does not entertain \emph{because} \fc{}.
  And, from  \fc{} relation of support.
  And, no deviance.

  Though, caution, also need not witnessed reasoning.

  Provide an initial \scen{} which does not fully work as a counterexample.
  Then, develop~\autoref{sketch:zS:succeed} with care.
\end{note}

\begin{note}
  Return to~\autoref{scen:squish}:

  \scenarioPLSquish*

  \scen{0} focused on the non-standard `Squish'-elimination rule of inference and the possibility of concluding `Squish'-elimination is a sound rule of inference.

  Uncommon, but enough to memorise the rule.
  Still, I consider my general understanding of propositional logic more important than memory.
  And, if failed, then would not consider sound.

  Filling in the details of the second abstract sketch:
  \begin{itemize}[noitemsep]
  \item
    I am the agent.
  \item
    \(\phi\) is the proposition: `\((P \rightarrow Q) \rightarrow P, Q \vdash P \land Q\)'.
  \item
    \(\psi\) is a some proposition: `Squish elimination is sound'
  \item
    Both \(v\) and \(v'\) are the value: `True'.
    And,
  \item
    The pools of premises \(\Phi\) and \(\Psi\) are left unspecified.%
    \footnote{
      Note, premises of reasoning.
      Distinct from premises of deduction.
    }
  \end{itemize}

  Still, I do not entertain the possibility of failing to show sound.

  With some luck, uncommon and you will witness the conclusion similar to what I would, if I chosen to reason.%
  \footnote{
    To preserve the integrity of the \illu{0}, the proof on page~\pageref{squish-elimination-proof} was considered \emph{after} concluding the (syntactic) consequence via Squish-elimination.
  }
\end{note}

\begin{note}
  So, as with lost keys.
  I wonder about derived rules of inference.
  Clearly a problem.
  Understanding of propositional logic is good, better than memory.
  But, same understanding, sound.
\end{note}

\subsection{Analysis}
\label{cha:zS:sec:lost-keys:analysis}

\begin{note}
  So, the way to understand these cases is in terms of whether the event is an event in which the agent is concluding.

  Then, the way \ros{1} from \fc{1} work is by explaining why the overall event is an event of concluding, rather than the agent simply pairing \(\pv{\phi}{v}\).
\end{note}

\begin{note}
  Split these cases into three components.
  \begin{enumerate}[label=\Roman*., ref=(\Roman*)]
  \item
    \label{zS:breakdown:opp}
    Opportunity.

    Interested in concluding \(\pv{\phi}{v}\) from \(\Phi\).
    Reasoning regarding \(\pvp{\psi}{v'}{\Psi}\) is reasoning I have the opportunity to do.
  \item
    \label{zS:breakdown:check}
    \check{2}.

    Whether or not would concluding \(\pv{\psi}{v'}\) from \(\Psi\).
  \item
    \label{zS:breakdown:psat}
    \psat{2}.

    Feedback.
  \end{enumerate}

  From the sketches,~\ref{zS:breakdown:opp}~and~\ref{zS:breakdown:check} are the first part,~\ref{zS:breakdown:psat} is the second part.

  \check{2} is key.
  Whether it makes sense to conclude \(\pv{\phi}{v}\) from \(\Phi\).
  Opportunity to \check{0}.

  \psat{2} then concerns whether agent would satisfy \check{0}.
\end{note}

\begin{note}
  Build this out in two separate parts.

  To begin,~\autoref{cha:zS:sec:curbs} will combine~\ref{zS:breakdown:opp}~and~\ref{zS:breakdown:check} to form the idea of a `\curb{}'.
  To end,~\ref{cha:zS:sec:requs} will combine the idea of a \curb{} with~\ref{zS:breakdown:psat} in the form the idea of a \requ{}.
\end{note}

\section{\curb{3}}
\label{cha:zS:sec:curbs}

%%%% TEMP from question

\begin{note}
  \color{red}
  Basic idea here is that strengthen the opportunity with ability, following \citeauthor{Austin:1961vz}'s way of putting things.
\end{note}

\begin{note}
  \color{red}
  From the analysis, opportunity and subjunctive ability.
  Capture in broader terms with the idea of a \curb{}.
\end{note}

\begin{note}[\curb{3}]
  We begin with the idea of a \curb{}, develops when something is a partial check.
  Familiar from clause~\autoref{sketch:zS:succeed:curb} of~\TNSketch{3}~\ref{sketch:zS:fail} and~\ref{sketch:zS:succeed}.

  \begin{definition}[A \curb{0}]
    \label{def:curb}
    For an agent \vAgent{}, and proposition-value-premises pairings \(\pvp{\phi}{v}{\Phi}\), \(\pvp{\psi}{v'}{\Psi}\):

    \begin{itemize}
    \item
      \(\pvp{\phi}{v'}{\Psi}\) is a \emph{\curb{}} of concluding \(\pv{\phi}{v}\) from \(\Phi\)
    \end{itemize}

    \emph{If and only if}

    \begin{itemize}
    \item
        \begin{enumerate}
        \item[\emph{If}:]
          Either~\ref{def:curb:opp} or~\ref{def:curb:link} is true:

          \begin{enumerate}[label=\alph*., ref=(\alph*)]
          \item
            \label{def:curb:opp}
            There is no \pevent{} in which \vAgent{} concludes \(\pv{\psi}{v'}\) from \(\Psi\).
          \item
            \label{def:curb:link}
            There is some \pevent{} in which \vAgent{} concludes \(\pv{\chi}{v''}\) from \(X\) such that \(\pv{\chi}{v''}\) is incompatible with concluding \(\pv{\psi}{v'}\) from \(\Psi\).
          \end{enumerate}
        \item[\emph{Then}:]
          \begin{enumerate}[label=\alph*., ref=(\alph*), resume]
          \item
            \label{def:curb:fail}
            \vAgent{} would not be concluding \(\pv{\phi}{v}\) from \(\Phi\) (\emph{due to}~\ref{def:curb:opp} or~\ref{def:curb:link}).
          \end{enumerate}
      \end{enumerate}
    \end{itemize}
    \vspace{-\baselineskip}
  \end{definition}

  So, \curb{1} are defined with respect to \emph{concluding}.
  In order for the event in which the agent concludes to be in progress, then event must not diverge.

  With respect to the breakdown of \TNSketch{3}~\ref{sketch:zS:fail} and~\ref{sketch:zS:succeed}, a \curb{} captures both the opportunity for the agent to reason about whether \(\pv{\psi}{v'}\) follows from \(\Psi\) and the idea of whether \(\pv{\psi}{v'}\) follows from \(\Psi\) being a \check{0} on concluding \(\pv{\psi}{v'}\) from \(\Phi\).

  At issue is what may occur, which is captured in terms of \pevent{1}.
\end{note}

\begin{note}
  Still an issue of how the subjunctive relates to concluding \(\pv{\phi}{v}\) from \(\Phi\).
  This, \autoref{cha:zS:sec:question}.
  \curb{} is half, or two thirds of the story.
\end{note}

\begin{note}
  The parenthetical `due to' appended to~\ref{def:curb:fail} ensures that the agent not concluding \(\pv{\phi}{v}\) from \(\Phi\) is tied to failing to conclude \(\pv{\psi}{v'}\) from \(\Psi\) after taking the relevant opportunity.%
  \footnote{
    General problem of deviance.
    Subjunctive conditional, without statement, allows for failure to conclude to be unrelated.
    Or, a finkish disposition (\cite[cf.][144]{Lewis:1997wg}).
    % ~\cite{Lewis:1997wg}
    % \begin{quote}
    %   Anything can cause anything; so stimulus \emph{s} itself might chance to be the very thing that would cause the disposition to give response \emph{r} to stimulus \emph{s} to go away.
    %   If it went away quickly enough, it would not be manifested.
    %   In this way it could be false that if \emph{x} were to undergo \emph{s}, \emph{x} would give response \emph{r}.
    %   And yet, so long as s does not come along, \emph{x} retains its disposition.
    %   Such a disposition, which would straight away vanish if put to the test, is called finkish.%
    %   \mbox{ }\hfill\mbox{(\citeyear[144]{Lewis:1997wg})}
    % \end{quote}
  }
  Don't have a specific account of `due to'.
  Move to the level of theories, and overall goal is to provide a theory independent motivation for rejecting \issueConstraint{}.
  So, some difficulty, may wonder whether the conditional holds.
\end{note}

\begin{note}
  \begin{proposition}
    \(\pvp{\phi}{v}{\Phi}\) is not necessarily a \curb{0} of concluding \(\pv{\phi}{v}\) from \(\Phi\).
    \begin{argument}
      This is trivial.

      For:

      \begin{itemize}
      \item
        \ref{def:curb:opp}, then there is no development of the current event in which the agent concludes \(\pv{\phi}{v}\) from \(\Phi\). Hence, the event is not an instance of concluding.
      \item
        \ref{def:curb:link} is where the difficulty lies.
        For, suppose so \pevent{}.
        At issue is whether the event in progress is such that the agent concludes something incompatible.
        And, in general this may not be the case.
        For example, various cases where overlook something obvious.
      \end{itemize}
    \end{argument}
  \end{proposition}
\end{note}




\subsection{\curb{3} and \fc{1}}

\begin{note}
  \begin{proposition}[\curb{3} and \fc{1}]
        For an agent \vAgent{}, and proposition-value-premises pairings \(\pvp{\phi}{v}{\Phi}\), \(\pvp{\psi}{v'}{\Psi}\):

    \begin{itemize}
    \item
      \(\pvp{\phi}{v'}{\Psi}\) is a \emph{\curb{}} of concluding \(\pv{\phi}{v}\) from \(\Phi\)
    \end{itemize}

    \emph{If and only if}

    \begin{itemize}
    \item
        \begin{enumerate}
        \item[\emph{If}:]
          \(\pv{\psi}{v'}\) is not a \fc{} from \(\Psi\).
        \item[\emph{Then}:]
          \begin{enumerate}[label=\alph*., ref=(\alph*), resume]
          \item
            \label{def:curb:fail}
            \vAgent{} would not be concluding \(\pv{\phi}{v}\) from \(\Phi\).
          \end{enumerate}
      \end{enumerate}
    \end{itemize}
    \begin{argument}
      The conditional by which \curb{1} are defined is of the form `\emph{If}~(\ref{def:curb:opp} \emph{or} \ref{def:curb:link}) \emph{then}~\ref{def:curb:fail}'.
      And, `\ref{def:curb:opp}~\emph{or}~\ref{def:curb:link}' is equivalent to \emph{not}-(\emph{not}-\ref{def:curb:opp}~and~\emph{not}-\ref{def:curb:link}).
      Hence, the conditional is equivalent to `\emph{If}~\emph{not}-(\emph{not}-\ref{def:curb:opp}~\emph{and}~\emph{not}-\ref{def:curb:link})~\emph{then}~\ref{def:curb:fail}'.

    And, `(\emph{not}-\ref{def:curb:opp}~and~\emph{not}-\ref{def:curb:link})' is equivalent to pair of conditions for \(\pvp{\psi}{v'}{\Psi}\) being a \fc{}.
    \end{argument}
  \end{proposition}
\end{note}

\begin{note}
  The idea of \(\pvp{\psi}{v'}{\Psi}\) being a \curb{2} of concluding \(\pv{\phi}{v}\) form \(\Phi\) for some agent is stated without reference to the \agpe{}.
  However, our interest with \curb{1} will be from an \agpe{}.

  Easily embed within.%
  \footnote{
    Recall, same with respect to \fc{1} --- see page~\pageref{fcs-neutral-perspective}.
  }
\end{note}

% \begin{note}
%   In some cases, it may be the case that \(\pv{\phi}{v}\) follows trivially after getting \(\pv{\psi}{v'}\), this doesn't matter.

%   It may also be the case that \(\pv{\phi}{v}\) entails \(\pvp{\psi}{v'}{\Psi}\), but this is also fine, so long as there is an alternative way.
% \end{note}

\begin{note}
  \color{red}
  The definition of a \curb{} expresses the idea that concluding \(\pv{\psi}{v'}\) from \(\Psi\) is a check on whether it makes sense for the agent, from the \agpe{}, to conclude \(\pv{\phi}{v}\) from \(\Phi\).
\end{note}

\subsection{Examples}
\label{cha:zS:sec:curbs:examples}

\begin{note}
  Seen two examples of a \curb{}.

  \autoref{illu:lost-key} and~\autoref{scen:squish}.
\end{note}

\begin{note}[Calculator]
  Likewise, failure of \(\pv{\psi}{v'}\) being a \curb{}.
  Opening \scen{},~\autoref{illu:gist:calc}.

  Emphasis on testimony.

  Example here, failure for the conditional to hold, though plausibly have the option.

  With respect to the \scen{0}, testimony.

  However, more broadly, agent values reasoning over another.

  Abstract from the testimony of a calculator to settings where the dynamic is more intuitive.

  For example, student in a classroom.

  Though, with the broader idea, ways to make calculator easier.

  Under the weather.
  Things are a little foggy.
  Some act, which may be such that concluding.
  However, no guarantee the act is an act of concluding.
\end{note}

\begin{note}[Failure but no option]
  \citeauthor{Dretske:1970to}.
  \begin{scenario}[A trip to the zoo]\mbox{ }
    \label{scen:trip-to-zoo}
    \vspace{-\baselineskip}
    \begin{quote}
      You take your son to the zoo, see several zebras, and, when questioned by your son, tell him they are zebras.
      Do you know they are zebras?
      [\dots]
      We know what zebras look like, and, besides, this is the city zoo and the animals are in a pen clearly marked ``Zebras.''
      Yet, something's being a zebra implies that it is not a mule and, in particular, not a mule cleverly disguised by the zoo authorities to look like a zebra.
      Do you know that these animals are not mules cleverly disguised by the zoo authorities to look like zebras?\newline
      \mbox{ }\hfill\mbox{(\citeyear[1015--1016]{Dretske:1970to})}
    \end{quote}
    \vspace{-\baselineskip}
  \end{scenario}

  \autoref{scen:trip-to-zoo} is framed in terms of knowledge, and is designed to raise a problem for conclude of knowledge under known entailment.
  Intuitively, you know the animals in the pen are zebras.
  And, you know the following conditional is true:
  The animals in the pen are zebras \emph{only if} the animals in the pen are not cleverly disguised mules.
  However, you (intuitively) don't know the animals in the pen are not cleverly disguised mules.

  If knowledge is closed under known entailment, then you:
  \begin{enumerate}
  \item \(\phi\) has value \(v\) only if \(\psi\) has value \(v'\)
  \end{enumerate}
  then, if
  \begin{enumerate}
  \item
    \(\phi\) has value \(v\), then
  \end{enumerate}
  \begin{enumerate}
  \item \(\psi\) has value \(v'\)
  \end{enumerate}

  Framed in terms of knowledge, but relation is similar to \curb{}.

  No opportunity!

  There is no premises to distinguish.
  \scen{3} designed to test closure principles provide various examples of this kind.
  In particular, \citeauthor{Wright:2011wn}.
\end{note}

\begin{note}
  Returning to instances of \curb{1}, ability.
  Most interesting case, from my perspective.

  Simple case is Sudoku puzzles, or puzzles in general.
\end{note}

\subsection{\curb{3} as checks on concluding}
\label{cha:zS:sec:curbs:checks}

\begin{note}
  Strictly:

  \curb{3} as \emph{partial} checks on \emph{whether or not an agent is} concluding.

  Core is from \ref{def:curb:opp} and~\ref{def:curb:link}.
  The idea is, that we're given an event, in which the agent is doing some reasoning.
  And, if either of the conditions obtain, then the event is interrupted.

  \begin{itemize}
  \item
    \ref{def:curb:opp}.

    Here, given concluding, then also conclude.
    Hence, if no conclusion, then fail.

  \item
    \ref{def:curb:link}.

    Likewise, but the difficulty is strengthened to something which conflicts.
  \end{itemize}
\end{note}

\begin{note}
  Or, what I really want to say is something like, the agent having the option to conclude \(\pv{\psi}{v'}\) from \(\Psi\).
  This is the check.
\end{note}

\begin{note}
  \begin{itemize}
  \item Check.
  \item Check on \emph{concluding}.
  \item \emph{Partial} check on concluding.
  \end{itemize}
\end{note}

\subsubsection{Check}

\begin{note}[Check]
  Conditional by which \curb{1} are defined captures the core idea of failure to conclude \(\pv{\psi}{v'}\) from \(\Psi\) leading to failure to conclude \(\pv{\phi}{v}\) from \(\Phi\).

  Broadly, suggestion that \(\pvp{\psi}{v'}{\Psi}\) is not a \fc{}.

  In particular, \ref{def:curb:link}.
  Would not conclude something incompatible.

  Also opportunity.
  As seen with the \citeauthor{Dretske:1970to} case.
  No opportunity, and therefore not a check in the relevant sense.

  Really, opportunity is easy to overlook, but very important.
  Conditional is subjunctive.
  Opportunity ensures that concluding from present state.
  Stops the subjunctive from wandering too far.

  Similar sense of check as checking date of birth.
  Fail to be of age, then no purchasing \dots

  Difference sense of check to label on the box.
  In a sense, determines whether or not make the purchase.
  Though, what really matters is whether the shop assistant asks for date of birth.

  Failure.
  This is stronger than should not, or might not.
\end{note}

\begin{note}
  Stated independently of \agpe{}, whether check in this sense is unclear.

  However, some caution.
  Harman style cases where there's some change.
  \curb{} is tied to particular pools of premises.
  At issue is not whether the agent would conclude \(\pv{\phi}{v}\) after failing to conclude, but conclude \(\pv{\phi}{v}\) \emph{from \(\Phi\)}.
  Concluding \(\pv{\phi}{v}\) from \(\Phi'\) may be okay, but \(\Phi\) really is bad.

  Still, from the \agpe{}, fine.
  I was confident I'd stop if failed to get validity of squish.

  Worry about what actually happens relies on things that perspective doesn't take into account.
  But, from perspective, fine.

  Note, with Harman style cases, also the possibility that whether something is a \curb{} may change.
\end{note}

\subsubsection{Concluding}

\begin{note}
  Check on concluding in the sense that event in progress, and whether that event is an event of concluding.

  {
    \color{red}
    Two senses in which this is check on concluding.
    First, concerns reasoning to \(\pv{\phi}{v}\) from \(\Phi\).
    Second, given reasoning to \(\pv{\phi}{v}\) from \(\Phi\).
  }
\end{note}

\begin{note}[Problems of induction]
  Note, however, both~\autoref{illu:lost-key} and~\autoref{scen:squish} focus on concluding.

  In turn, \TNSketch{3}~\ref{sketch:zS:fail} and~\ref{sketch:zS:succeed} focus on the failure to conclude to some proposition-value pair which would follow from concluding some other proposition-value pair.

  Hence, the sketch does not apply to black ravens.
  I wouldn't conclude all ravens are black if I saw a white raven.

  I may worry about shortly seeing a white raven when concluding all ravens are black, and I may refuse to entertain the possibility that the sun will rise tomorrow when planning to mow the grass.

  However, it's not possible to reason to seeing a white raven, nor is it possible to reason to the sun not rising tomorrow.

  Abstractly, at issue in~\autoref{illu:lost-key} is the possibility of failing to a reason to some proposition-value pair given \emph{present} information, rather than the possibility of failing to a reason to some proposition-value pair given \emph{new} information.

  To the extent that problems of induction arise from receiving new information, what is at issue is distinct.%
  \footnote{
    See~\textcite{Henderson:2020wb} for more on the problem of induction.
  }

  Similar points for external world scepticism.
  Would not conclude that I have hand if disembodied brain in a vat.

  However, conclusion is out of reach.
\end{note}

\subsubsection{Partial}

\begin{note}
  The `converse' to the conditional by which \curb{1} are defined does not hold.
  Naturally, as \fc{}\dots

  Still, the more significant issue is that there may be no \emph{unique} \curb{0} of concluding \(\pv{\phi}{v}\) from \(\Phi\).
  For example, both \(\pvp{\psi}{v'}{\Psi}\) and \(\pvp{\chi}{v''}{X}\) may be \curb{0} of concluding \(\pv{\phi}{v}\) from \(\Phi\).

  {
    \color{red}
    Seen in \scen{0}???
  }
  And, there may be checks other than \(\pvp{\psi}{v'}{\Psi}\) being a \curb{}.

  A different, but related check, would consider whether the agent has concluded.
  However, less interesting.
  Consider the squish scenario.
  Have concluded.
  However, what's of interest is how things are.
\end{note}

% \begin{note}
%   The role of clause~\ref{def:curb:pool} is to ensure the agent may conclude \(\pv{\psi}{v'}\) from \(\Psi\) independently of concluding \(\pv{\phi}{v}\) from \(\Phi\).

%   If \ref{def:curb:pool:int} were to fail to hold then:
%   \begin{itemize}
%   \item
%     By~\ref{def:curb:pool:int}, the agent would need to conclude \(\pv{\phi}{v}\) from \(\Phi\) as a sub-conclusion when concluding \(\pv{\psi}{v'}\) from the relevant pool of premises \(\Psi\).
%     Hence, it would not be possible to conclude \(\pv{\psi}{v'}\) from \(\Psi\) without first concluding \(\pv{\phi}{v}\) from \(\Phi\).
%   \end{itemize}

%   Conversely, if \ref{def:curb:pool:int} holds, the agent may conclude \(\pv{\psi}{v'}\) from \(\Psi\) independently of concluding \(\pv{\phi}{v}\) from \(\Phi\).

%   Note, however, \ref{def:curb:pool:int} does not rule out the possibility of the agent concluding \(\pv{\phi}{v}\) from \(\Phi\) when concluding \(\pv{\psi}{v'}\) from \(\Psi\) or, conversely, concluding \(\pv{\psi}{v'}\) from \(\Psi\) when concluding \(\pv{\phi}{v'}\) from \(\Phi\).

%   Having the option which matters.
%   Whether or not the option is taken, doesn't matter.
% \end{note}

\paragraph{When concluding}

\begin{note}
  Hence, it need not be the case that the agent has the option of concluding \(\pv{\psi}{v'}\) from \(\Psi\) from their epistemic state prior to starting line of reasoning (as the agent has not yet concluded that \(\phi\) has value \(v\)).
\end{note}

\begin{note}[Prior to concluding\dots]
  An important feature of \qzS{} \dots

  Not particularly marked.
  Allow agent to have built up a bunch of stuff while reasoning.

  Example.

  \begin{scenario}[Velocity]
    \label{ill:velocity}
    Agent is provided with information about how far a car has travelled north as a function of time travelled.

    From this, take the derivative of the function to obtain the (instantaneous) velocity of the car at a handful of points in time.

    And, from the (instantaneous) velocity of the car, the agent calculates the (instantaneous) acceleration of the car at each of the points in time.

    The agent also has information about the speed of the car as a function of time travelled, and the agent may calculate speed by the taking magnitude of the (instantaneous) velocity of the car.
  \end{scenario}

  \autoref{ill:velocity}, two step calculation.
  Velocity, acceleration.
  After the first step, check by taking the magnitude.
  Calculation of velocity is correct only if taking the magnitude matches speed.

  Just before concluding to include cases such as this.
\end{note}

\begin{note}
  Example highlights how `intermediate conclusions' relate.
  Further point of interest:
  Failure to conclude.

  Two ways to view agent's calculation of the velocity of the car.

  First, as a conclusion.
  Same status as the function.

  Or, as temporary.

  Difference in how we understand agent's present epistemic state.

  On first, the agent's present epistemic state is inconsistent.
  Two proposition-value pairs which conflict.
  Not possible for the car to have velocity the agent calculated and acceleration the agent has been informed of.

  May also be that the function and information about acceleration are inconsistent, but may also be that the agent made a mistake in calculating the velocity of the car.

  On second, the agent's present epistemic state may be consistent.

  For, made a mistake.
  But, proposition-value pair is not part of present epistemic state, so distinguished from function and information about acceleration, which are consistent.

  This is a distinction we have little interest in.
  What matters is failure to conclude speed.
  Result is either revising inconsistent epistemic state, or abandoning intermediary steps of reasoning.
\end{note}

\subsection{Summary}
\label{cha:zS:sec:curbs:summary}

\begin{note}
  Two conditions from breakdown of \TNSketch{3}~\ref{sketch:zS:fail} and~\ref{sketch:zS:succeed}.
  Opportunity and \check{}.

  \curb{}.
  Minor clarifications.
  From the \agpe{}.

  Examples.

  Idea that a \curb{} identifies a (partial) check on reasoning in detail.

  Undercutting defeaters.

  Minor details.
\end{note}

\begin{note}
  Breakdown of \TNSketch{3}~\ref{sketch:zS:fail} and~\ref{sketch:zS:succeed}.
  Third condition.
  \psat{2}.
  Turn to this in follow section.
\end{note}

\subsection{\curb{3} and concluding}
\label{cha:zS:sec:question}

\begin{note}[The question]
  With the idea of a \curb{} in hand \dots

  \begin{restatable}[\curb{3} and concluding]{proposition}{propCurbCing}
    \label{prop:sCing}
    For an agent \vAgent{}, etc.\dots

    \begin{itemize}
    \item
      Event is instance of concluding \(\pv{\phi}{v}\) from \(\Phi\).
    \end{itemize}

    \emph{If and only if}

    \begin{itemize}
    \item
      For any proposition-value-premises pairing \(\pvp{\psi}{v'}{\Psi}\):
      \begin{itemize}
      \item[\emph{If}:]
        \begin{enumerate}[label=\alph*., ref=(\alph*)]
        \item
          \label{question:zs:option}
          \(\pvp{\psi}{v'}{\Psi}\) is a \curb{} of concluding \(\pv{\phi}{v}\) from \(\Phi\).
        \end{enumerate}
      \item[\emph{Then}:]
        \begin{enumerate}[label=\alph*., ref=(\alph*), resume]
        \item
          \label{question:zs:may-fail}
          \(\pv{\psi}{v'}\) is a \fc{} from \(\Psi\).
        \end{enumerate}
      \end{itemize}
    \end{itemize}
    \begin{argument}
      Trivial.
      For, \curb{}.
      If either condition obtains, then not concluding.
      Hence, neither condition obtains.
      Therefore, by the other proposition which links to \fc{}, this means \fc{}.
    \end{argument}
  \end{restatable}
\end{note}

\begin{figure}[h]
  \centering
  \begin{tikzpicture}
    \node (origin) at (0,0) {};
    \node (psiSplit) at (1,0) {};
    \node (phiSplit) at (4,0) {};
    %
    \node[anchor=west] (Phi) at  (0,0)  {\(\Phi\)};

    \node[anchor=west] (psiV) at  (6,-1)  {\(\pvp{\psi}{v'}{\Psi}\)};
    \node[anchor=west] (psiNv) at (6,-2) {\(\pvp{\psi}{\{\overline{v'}, ?\}}{\Psi}\)};
    % \node[anchor=west] (psiQ) at (6,-3) {\(\pvp{\psi}{?}{\Psi}\)};
    %
    % \node[anchor=west] (psiVPhiV) at (9,-1) {\(\pv{\phi}{v}\)};
    \node[anchor=west] (psiNvPhiU) at (10,-2) {\(\pv{\phi}{\{\overline{v},?\}}\)};
    % \node[anchor=west] (psiQPhiU) at (9,-3) {\(\pv{\phi}{\{\overline{v},?\}}\)};
    %
    \node[anchor=west] (phiQ) at (10,1) {\(\pv{\phi}{\{\overline{v}, ?\}}\)};
    % \node[anchor=west] (phiNv) at (10,2) {\(\pv{\phi}{\overline{v}}\)};;
    \node[anchor=west] (phiV) at (10,0) {\(\pv{\phi}{v}\)};
    %
    \draw[-]  (Phi) -- (phiV);
    %
    % \path[-,dotted] (phiSplit) edge [out=0, in=180] (phiNv);
    \path[-,dotted] (phiSplit) edge [out=0, in=180] (phiQ);
    %
    \path[-, dashed] (psiSplit) edge [out=0, in=180] (psiV);
    \path[-, dotted] (psiSplit) edge [out=0, in=180] (psiNv);
    % \path[-, dotted] (psiSplit) edge [out=0, in=180] (psiQ);
    %
    \draw[-,dashed] (psiV) edge [out=0, in=180] (phiV);
    \draw[-, dotted] (psiNv) edge (psiNvPhiU);
    % \draw[-, dotted] (psiQ) edge (psiQPhiU);
    \end{tikzpicture}
    \caption{Visualisation of a conclusion given a \curb{}.}
    \label{fig:csN:illu:overview}
  \end{figure}

\begin{note}[Figure]
  \autoref{fig:csN:illu:overview} provides a rough visualisation of when \qzS{} has a positive answer.

  The flat line captures the agent's reasoning, which concludes with \(\pv{\phi}{v}\).
  In concluding \(\pv{\phi}{v}\) the agent rules out two possibilities with respect to \(\phi\).
  First, that \(\phi\) does not have value \(v\), indicated by \(\pv{\phi}{\overline{v}}\).
  Second, that the agent does not assign any value to \(v\), indicated by \(\pv{\phi}{?}\).
  Prior to concluding \(\pv{\phi}{v}\), the agent's reasoning may have branched to either alternative path, but as the agent has concluded \(\pv{\phi}{v}\), neither path is viable, and hence both paths are represented with a dashed line.

  So far, we have seen only that the agent has concluded \(\pv{\phi}{v}\).

  We now consider some proposition-value-premises pairing \(\pv{\psi}{v'}{\Psi}\) such that if the agent were to fail to conclude \(\pv{\psi}{v'}\) from \(\Phi\), the agent would not conclude \(\pv{\phi}{v}\) from \(\Phi\).

  Intuitively, the dotted arrows from the various combinations of \(\psi\) and \(\{v',\overline{v'},?\}\) read, from top to bottom:
  \begin{itemize}
  \item If \(\phi\) has value \(v\) then the agent may conclude \(\pv{\psi}{v'}\) from \(\Psi\), and:
  \item If the agent concludes \(\psi\) has some value \(\overline{v'}\) from \(\Psi\), then the agent either concludes \(\phi\) has some value other than \(v\), or the agent fails to reach a conclusion regarding \(\phi\) from \(\Phi\).
    Both options are combined via the shorthand \(\pv{\phi}{\{\overline{v},?\}}\).
  \item
    And, likewise if the agent fails to conclude \(\pv{\psi}{v'}\) from \(\Psi\).
  \end{itemize}

  With respect to concluding, observe that prior to ruling out alternative branches with respect to \(\pv{\phi}{\{\overline{v},?\}}\), the agent may have reasoned about whether \(\psi\) has value \(v\).
  And, from the \agpe{}, \(\phi\) has value \(v\) only if \(\psi\) has value \(v'\).
  If \(\psi\) does not have value \(v'\), then either \(\phi\) does not have value \(v\), or the agent's reasoning would not conclude with a value for \(\phi\), indicated by \(\pv{\phi}{\{\overline{v},?\}}\).

  Hence, prior to concluding \(\pv{\phi}{v}\), the agent has concluded \(\pv{\psi}{v'}\).
\end{note}

\paragraph{Odd sands}

\begin{note}
  \autoref{prop:sCing} may seem odd.
  For, suppose an agent concludes \(\pv{\phi}{v}\) from \(\Phi\).
  Then, it seems that throughout the event, then agent was concluding.
  However, if there was some \curb{}, then would not have been concluding.
  Hence, it seems if the agent concludes \(\pv{\phi}{v}\) from \(\Phi\), then the agent \sCe{} \(\pv{\phi}{v}\) from \(\Phi\).

  However, this seems unintuitive.
  For, there are various cases in which it seems an agent may have concluded otherwise.

  There are two ways to approach this:
  \begin{itemize}
  \item
    Deny that the event was an event in which the agent concluded.

    Specifically, narrow the relevant event.
    For, we have some initial event, which develops in two a larger event, which finishes with an event in which the agent concludes \(\pv{\phi}{v}\).
    So long as final sub-event, then the agent concludes.
    The issue only arises from the apparent link to earlier (sub-)events.
    But, if otherwise, then motivation to reject these as (sub-)events of an event in which the agent concludes.
  \item
    Deny the premise.

    If event in which an agent concludes, it need not be the case that the agent was concluding throughout all the relevant sub-events.

    Then, no issue is resolved.
  \item
    Shift perspective.

    What it true after the fact need not be true as things are developing.
    This is \citeauthor{Boylan:2020aa}'s approach.
  \end{itemize}

  I am inclined to consider the former.
  When speak on concluded, just picking out an event.
  Naturally event to some default event.
  However, flexible.

  So, here \citeauthor{Boylan:2020aa} and darts.
  I was able to, sure, but only after things hand developed so far.

  Additionally, consider picture.
  Drew a dog.
  However, not the case that drawing a dog throughout associated event.
  Started out as a doodle.

  Ran 10k.
  However, started out, plan was to run 5k.
  Interesting consequence here is that not running 5k.
  For, supposing extension came about by choice, something changed, and hence no force to finish at 5k.

  Running 10k includes running 5k.
  But, in a more basic case, running.
  Extends to drawing.

  In the case of concluding, have reasoning.
\end{note}

\begin{note}[\autoref{sketch:zS:succeed} is \(\exists\), but \qzS{} is \(\forall\)]
  \sCing{} is distinct from~\autoref{sketch:zS:succeed}.

  \begin{quote}
    There is a \curb{} and would conclude.
  \end{quote}

  Weaker, in sense holds if no \curb{}.
  Stronger, in sense that quantifies over all \curb{}.
\end{note}


\begin{note}
  Limits.

  Is going to be the case that there's a limit on what may be a \curb{}.
  For, instance of concluding.
  Hence, must be such that, if \(\pvp{\psi}{v'}{\Psi}\) is a \curb{} then would conclude \(\pv{\psi}{v'}\) from \(\Psi\) and conclude \(\pv{\phi}{v}\) from \(\Phi\).

  Hence, it is not the case that \curb{} iff incompatible with conclusion of \(\pv{\phi}{v}\) from \(\Phi\).

  No, this is another instance where the due to is important.
  For, in this case, it's not due to the failure to conclude, it's due to the resources required.
\end{note}




\newpage

\subsection{A missing link}
\label{cha:zS:sec:missing-link}

\begin{note}
  \autoref{prop:sCing} is important.
  For, what matters for the conclusion.
  If agent concludes \(\pv{\phi}{v}\) from \(\Phi\), then concluding \(\pv{\phi}{v}\) from \(\Phi\).

  So, suppose, \curb{}.
  Then, need it to be the case that \(\pv{\psi}{v'}\) from \(\Psi\) is a \fc{}.

  In this respect, \fc{} explains, in part, and in some sense, why the agent concludes \(\pv{\phi}{v}\) from \(\Phi\).

  However, that \(\pv{\psi}{v'}\) from \(\Psi\) is a \fc{} does not answer, in part, \qWhyVoP{}.
  For, no \ros{}.
  \ros{} from \agpe{}.
  Only get this from \(\pv{\psi}{v'}\) from \(\Psi\) being a \fc{}, from \agpe{}.

  In this respect, \curb{1} are compatible with \issueConstraint{}!
\end{note}

\section{\requ{3}}
\label{cha:zS:sec:requs}

\begin{note}
  For, need not be the case that \ros{} answers \qWhy{}.

  However, \curb{} from \agpe{}.
  This we term a \requ{}.

  \begin{definition}[\requ{2}]
    \label{def:requ}
    For an agent \vAgent{}, proposition-value pairs \(\pv{\phi}{v}\), \(\pv{\psi}{v'}\) and \poP{1} \(\Phi\), \(\Psi\):

    \begin{itemize}
    \item
      \(\pvp{\psi}{v'}{\Psi}\) is a \requ{} of concluding \(\pv{\phi}{v}\) from \(\Phi\).
    \end{itemize}

    \emph{If and only if}

    \begin{itemize}
    \item
      \(\pvp{\psi}{v'}{\Psi}\) is a \curb{} on concluding \(\pv{\phi}{v}\) from \(\Phi\), from \agpe{\vAgent{}'}.
    \end{itemize}
    \vspace{-\baselineskip}
  \end{definition}

  `\requ{2}' in the sense of `deemed necessary'.
\end{note}

\begin{note}
  \begin{proposition}
    If known \requ{}, then \curb{}.
    \begin{argument}
      For, straightforward by factivity.
    \end{argument}
  \end{proposition}

  Knowledge is doing some work here.
  For, it may be the case that agent would conclude regardless.
  However, this is very odd.
  Would need deviance.

  Still, strictly don't need this.
  Only need something weaker.

  \begin{proposition}[Feedback]
    If \requ{}, then \fc{}, from \agpe{} is \curb{}.
    \begin{argument}
      Typically, at least.

      For, if \requ{}, then from \agpe{}, concluding only if \fc{}.
      Hence, if not \fc{} from \agpe{}, then the agent is not concluding \(\pv{\phi}{v}\), from \agpe{their}.
      So, stop.

      As abort, \curb{}, for then does not develop such that agent concludes \(\pv{\phi}{v}\) from \(\Phi\) --- the agent stopped.
    \end{argument}
  \end{proposition}

  This is what we're really interested in.
  Some self reflection, and concerns about the way in which the present event may develop influence the development of the event.

  At least to me, familiar phenomenon.
  Doing a proof, and from \agpe{my}, concluding.
  However, unexpected observation, and suddenly it is by no means clear.
  And, as I don't see a way to get a \fc{}, then a \curb{}.
\end{note}

\subsection{Scenarios}
\label{cha:zS:sec:question:scenarios}

\begin{note}
  \autoref{cha:zS:sec:question:illu} contains additional \illu{1} of instances and failures of \sCing{}.
\end{note}

\begin{note}
  Hence, \autoref{illu:lost-key}, interesting instance of negative answer.
  And,~\autoref{scen:squish}, interesting instance of positive answer.

  In both cases, I identified a \curb{}.

  Of course, \qzS{} is stronger, so raises issue of further \curb{3} with respect to~\autoref{scen:squish}.
  In particular, other rules of logic.
  Etc.
  Witnessed conjunction, but not the soundness of conjunction.
  Sure, and various others.
  Though, important to keep in mind \curb{}.
\end{note}

\begin{note}[\autoref{illu:sketch:prop-logic}]
  Likewise, \autoref{illu:sketch:prop-logic} involves an agent concluding some sentence \(A\) is a syntactic theorem of propositional logic via a formula derivation.

  And, when concluding \(A\) is a syntactic theorem, the agent observes that \(A\) is a syntactic theorem only if \(A\) is also a semantic theorem (from soundness).

  In other words, if the agent attempt to show \(A\) is true under an arbitrary valuation and failed, the agent would not conclude \(A\) is a syntactic theorem.

  Only if semantic proof.
  Syntactic proofs, at least in my experience, may be out of reach.
  However, semantic proofs, often straightforward.
\end{note}

\begin{note}
  Absence of \requ{}.
  \autoref{scen:trip-to-zoo}.

  Same with initial \scen{0}.
  Testimony.
\end{note}

\begin{note}
  Here, highlight initial scenario.
  For, there, testimony.
  Overrides agent's own reasoning, plausibly.
\end{note}

\paragraph{\requ{3} and undercutting defeaters}

\begin{note}
  Shares similarity.
  Focus on concluding.
  However, subjunctive.
  \requ{3} concern entertaining proposition-value-premises pairings as an undercutting defeater for concluding.
\end{note}

\begin{note}
  Not about the proposition-value pair.
  Rather, it is about the concluding.
  At interest is not whether \(\phi\) has value \(v\), but whether it makes sense to conclude \(\pv{\phi}{v}\) from \(\Phi\).
  Of course, if the agent has no information about whether \(\phi\) has value \(v\), then this is also part of the picture, but that is a consequence of the base concern.

  With squish.
  I wouldn't conclude.
  However, wouldn't say that entailment doesn't hold.
  For, may only be the case that the derived rule of inference fails to hold.
\end{note}

\begin{note}
  In this respect, failing to conclude \(\pv{\psi}{v'}\) from \(\Phi\) may be described as an undercutting defeater with respect to conclude \(\pv{\phi}{v}\) from \(\Phi\).%
  \footnote{
    We borrow the following sketch from \textcite{Worsnip:2018aa}:
  \begin{quote}
    Undercutting defeaters, which are easiest to think of in the context of the attitude of belief, are supposed to be considerations that undermine the justification of a belief in a proposition p not necessarily by providing (sufficient) positive evidence to think that p is false, but rather merely by suggesting (perhaps misleadingly) that one’s reasons for believing p are no good, in a way that neutralizes or mitigates their justificatory or evidential force.%
    \mbox{}\hfill\mbox{(\citeyear[29]{Worsnip:2018aa})}
  \end{quote}
  }

  Consider the following illustration provided by \citeauthor{Pollock:1987un}:
  \begin{quote}
    [Undercutting defeaters] attack the connection between the reason and the conclusion rather than attacking the conclusion itself.
    For instance, ``X looks red to me'' is a prima facie reason for me to believe that X is red.
    Suppose I discover that X is illuminated by red lights and illumination by red lights often makes things look red when they are not.
    This is a defeater, but it is not a reason for denying that X is red (red things look red in red light too).
    Instead, this is a reason for denying that X wouldn't look red to me unless it were red.%
    \mbox{}\hfill\mbox{(\citeyear[485]{Pollock:1987un})}
  \end{quote}
  Completing \citeauthor{Pollock:1987un}'s example, it seems that if agent's support for holding that X is red is that `X wouldn't look red to me unless it were red', then the support for X being red provided by appearance is retracted after discovering that X is illuminated by red lights (though it remains possible that X is red).
\end{note}

\begin{note}
  In \citeauthor{Pollock:1987un}'s example, discover that the light is red.
  By parallel, the agent failing to conclude would undercut.
\end{note}

\begin{note}
  Similarity.

  However, given the focus on concluding, \curb{} is not a simple instance of an undercutting defeater.
  At least, given \citeauthor{Pollock:1987un}'s definition of an undercutting defeater.

  \citeauthor{Pollock:1987un} defines undercutting defeaters as follows:
  \begin{quote}
    R is an \emph{undercutting defeater} for P as a prima facie reason for S to believe Q if and only if
    \begin{enumerate}[label=(UD\arabic*), ref=(UD\arabic*)]
    \item
      \label{pollock:ud:1}
      P is a reason for S to believe Q and R is logically consistent with P but (P and R) is not a reason for S to believe Q, and
    \item
      \label{pollock:ud:2}
      R is a reason for denying that P wouldn't be true unless Q were true.%
      \mbox{}\hfill\mbox{(\citeyear[485]{Pollock:1987un})}
    \end{enumerate}
  \end{quote}
  Intuitively, an undercutting defeater for P as a reason for Q because it the defeater denies that Q must be true in order for P to be true.%
  \footnote{
    \citeauthor{Wright:2011wn}'s (\citeyear{Wright:2011wn}) revised template:
    \begin{quote}
      Where A entails B, a rational claim to warrant for A is not transmissible to B if there is some proposition C such that:
      \begin{enumerate}[label=(\roman*), noitemsep]
      \item
        The process/state of accomplishing the relevant putative warrant for A is subjectively compatible with C’s holding: things could be with one in all respects exactly as they subjectively are yet C be true
      \item
        C is incompatible (not necessarily with A but) with some presupposition of the cognitive project of obtaining a warrant for A in the relevant fashion, and
      \item
        Not-B entails C%
      \mbox{ }\hfill\mbox{(\citeyear[93]{Wright:2011wn})}
      \end{enumerate}
    \end{quote}
    Difficulty with the process, however, of interest is transmission of warrant from A to B.
    Hence, the agent has accomplished the relevant putative warrant for A.

    Or, consider the fourth type of dependence between premise and conclusion considered (but not endorsed) by \textcite{Pryor:2004ws}:

  \begin{quote}
    [Type 4] dependence between premise and conclusion is that the conclusion be such that evidence \emph{against it} would (to at least some degree) undermine the kind of justification you purport to have for the premises.%
    \mbox{}\hfill\mbox{(\citeyear[359]{Pryor:2004ws})}
  \end{quote}
  Premises!

  \nocite{Weisberg:2012vs}
  Closer to \citeauthor{Weisberg:2010to}'s (\citeyear{Weisberg:2010to}) account of bootstrapping.
  However, implicit circularity.
  And, circularity is not at issue with \qzS{}.
  In particular, \curb{}.
  }
\end{note}

\begin{note}
  Difficulty with a clear parallel


  \ref{pollock:ud:1}.
  For, P \emph{is a} reason.

  In the case of \curb{1}, failure to be concluding.
  For, event may develop such that agent does not conclude.

  \ref{pollock:ud:1}.

  Some other way for P to be true.
\end{note}

\begin{note}
  However, ties things too closely to presentation, rather than substance of ideas.
  Could be said that agent is concluding, but there is an additional way in which the event may develop.

  Variation:
  \begin{quote}
    R is \emph{undercutting reasoning} with respect to \emph{S} concluding \(\pv{\phi}{v}\) from \(\Phi\) if and only if
    \begin{enumerate}[label=(UR\arabic*), ref=(UR\arabic*)]
    \item
      The event may develop such that \emph{S} to concludes \(\pv{\phi}{v}\) from \(\Phi\) and R is a \pevent{} which is logically consistent with reasoning from \(\Phi\) but combination of both reasoning from \(\Phi\) and R is not a sufficient for \emph{S} to be concluding \(\pv{\phi}{v}\), and
    \item
      R is sufficient for denying that S wouldn't conclude \(\pv{\phi}{v}\) from \(\Phi\) unless \(\pv{\phi}{v}\) followed from \(\Phi\).%
    \end{enumerate}
    Where, R is reasoning which fails to conclude \(\pv{\psi}{v'}\) from \(\Psi\).
  \end{quote}

  {
    \color{red}
    So, the emphasis here is on whether there's really anything to be said for the conclusion.
    However, if this is the case, then it seems the agent simply doesn't conclude.
  }

  The problem here is with the first clause.
  Logically consistent.

  It is not clear that the reasoning is logically consistent.
  Two instances of reasoning may involve intermediary steps which are logically inconsistent.

  Though, on the other hand, in interesting cases it is logically consistent for the agent to reason to \(\pv{\phi}{v}\) from \(\Phi\) and fail to conclude \(\pv{\psi}{v'}\) from \(\Psi\).
  For, if not logically possible, then no worries about failing to conclude \(\pv{\psi}{v'}\) from \(\Psi\).

  Yet, if problem, then from the \agpe{}, instances of reasoning aren't consistent.

  Puzzle here is how logical consistency is understood with respect to \citeauthor{Pollock:1987un}'s definition.
  Independent of the \agpe{}, or from the \agpe{}?

  Both interpretations are compatible with the example.
  But, principle also holds independently of the \agpe{} with respect to the example.
\end{note}

\begin{note}
  In contrast, \curb{} `attacks' the \emph{reasoning} and the conclusion.
\end{note}

\section{Counterexample found}
\label{sec:counterexample-found}

\begin{note}
  Feedback proposition.
  And, just need it to be the case that no \wit{} for \fc{}.

  A final difficulty.
  \fc{} from \agpe{}.
  However, embed.
  \ros{} does not answer \qWhyVoP{}.
  Rather, premise.

  \begin{proposition}
    No embedding.
    \begin{argument}
      The worry is whether or not the agent is concluding.
      If embed, then the same worry applies.
    \end{argument}
  \end{proposition}
\end{note}

\begin{note}
  \qWhyV{} to \qWhyVnP{}.
  Feedback bridges the gap.
\end{note}

\begin{note}
  Non-factive.

  Well, depends on \requ{}.
  However, plausible that the agent only \emph{concludes} if know \fc{}.
  Else, agent does not conclude \(\pv{\phi}{v}\) from \(\Phi\).

  At issue here is not whether the agent concludes \(\pv{\phi}{v}\), that much is set.
  However, it's the way in which the conclusion comes about.
  The function of a \poP{} is to capture some substantial event.
\end{note}


\section{Scraps}
\label{sec:scraps}

\begin{note}
  \emph{However}, caution.
  For, as we have seen with testimony, it may be the case that status of a premises blocks a \curb{}.
  And, the argument given relies on the existence of a \curb{}.
  So, it may be the case that past reasoning blocks a \curb{}.
  Still, here, only need to deny this.
  Not saying that in every case agent's present reasoning is given priority.
  (Indeed, consider cases of being somewhat impaired, e.g., via exhaustion.
  Indeed, exhaustion is interesting.
  Basic consistency checks.
  Should be the case that conclude A, but just concluded \emph{not}-A, or something like this\dots)
  Rather, denying that past continues to secure in all instances.
  So, just need the potential to revise perspective on any previous conclusion.
\end{note}



%%% Local Variables:
%%% mode: latex
%%% TeX-master: "master"
%%% End:
