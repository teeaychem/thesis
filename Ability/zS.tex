\chapter{\zSN{2}}
\label{cha:zS}

\begin{note}[Intro, locating]
  Now turn to \zSN{}.

  Our goal motivate a positive resolution to~\issueConstraint{}.
  And, as sketched in~\autoref{cha:outline}, \zSN{} has key role in developing tension.

  Question.
  Positive answer answers why.
  Positive answer only if either concluded or \fc{}.
  Concluded or \fc{} answers why.
  \fc{} answers why only if relation of support answers why.
  Relation of support answers why only if proposition-value-premises pairing answers why.

  But, \fc{}, so proposition-value-premises pairing is not an answer to how, as the agent has not witnessed reasoning.

  Now, it is consistent that positive answer only if agent has concluded, hence has witnessed reasoning, and hence is, in part, an answer to how.

  Figuring out instances where this does not hold will wait until later.


  In contrast to sketch given, being with focus on motivation.

  Motivate \zSN{} via plain language question, similar to initial \qWhy{} and \qHow{} from~\autoref{cha:introduction}.
  Follow similar pattern.
  Just as \qWhy{} and \qHow{} were developed in~\autoref{cha:clarification}, \zSN{}, is developed in a similar way.
  Here, like with \qWhy{}, specifically, two things reduce to the same.
\end{note}

\begin{note}[Map]
  Start with a question.

  Certain kind of support \zSN{}, or \zS{}.

  Understanding of \zS{}.

  Argue negative answer \zS{} if and only if \zetaS{}.
\end{note}

\begin{note}
  In order to establish tension we narrow our attention to when concluding \(\pv{\phi}{v}\) concluding \(\pv{\phi}{v}\) involves the agent establishing a particular property with respect to \(\pv{\phi}{v}\).
  We term the property `\zSN{0}', or `\zS{}' for short.

  Positive resolution only requires existence of cases.
  Hence, existence of cases with this property.
  This will be sufficient.
  Any case of concluding which involves \csVImp{} will also be an instance of concluding.

  As sketched, tension by {\color{red} \dots}.

  For the moment, however, we focus on providing a clear account of \csN{}.
  Tension delayed until \dots
  Indeed, following \csN{}, revise resolutions to ~{\color{red} issue:Main}.
  And, additional building blocks for tension via two types of concluding.
\end{note}

\paragraph{Naming}

\begin{note}[Naming]
  Our choice of the term `\zgb{0}' is metaphorical.
  \zgb{2} is a family of flower plants which, typically, have the appearance of a single stem with no branches.
  If one starts just before the flower and works back down the stem, one will not find a branch which, if taken, would lead to a different flower.
  In comparison, if one starts with an agent's epistemic state prior to the agent concluding \(\pv{\phi}{v}\) from \(\Phi\) and~\autoref{question:zs} has a negative answer with respect to \(\pvp{\phi}{v}{\Phi}\), then one will not find a branch which leads to a different conclusion.

  I have some doubts as to whether or not this metaphor really works, but some term is required.
  `Palm-tree-support', or `Arecaceae-support' would also work.
\end{note}

\begin{note}[The token `\qzS{}']
  As {\color{red} noted}, we {\color{red} will} associate \zS{} with positive answers to~\autoref{question:zs}.
  So, in anticipation of this connexion we will use the token `\qzS{}' to name and refer to \autoref{question:zs}.
  Though, to be clear, \zS{} will concern an agent \emph{after} concluding \(\pv{\phi}{v}\) from \(\Phi\) while \qzS{} concerns the agent \emph{when} concluding \(\pv{\phi}{v}\) from \(\Phi\).
  So, strictly speaking an instance of \zS{} is not equivalent with a positive answer to some instance of \qzS{}.%
  \footnote{
    Whether, when concluding \(\pv{\phi}{v}\) from \(\Phi\), it is the case that the following conditional which quantifies over proposition-value-premises pairings:

    \begin{itemize}
    \item
      For all \(\pvp{\psi}{v'}{\Psi}\) [\emph{if}~\ref{question:zs:subjunctive} and~\ref{question:zs:option} hold, \emph{then} \ref{question:zs:may-fail} holds].
    \end{itemize}
    Where `hold(s)' expands to `hold(s) from the agent's perspective'.
  }
\end{note}

\section{\zS{}}
\label{cha:zS:sec:question}

\begin{note}
  A question about an agent's epistemic state when concluding \(\pv{\phi}{v}\) from \(\Phi\).
  Goal here is to get why involved.
\end{note}

\subsection{A question}
\label{cha:zS:sec:the-question}

\begin{note}
  We begin with the question.

  \begin{restatable}[\qzS{}]{question}{questionZS}
    \label{question:zs}
    For an agent \vAgent{}, when concluding \(\pv{\phi}{v}\) from \(\Phi\), is it the case that:

    \begin{itemize}
    \item
      From \vAgent{}' perspective:
      \begin{itemize}
      \item
        For any proposition-value-premises pairing \(\pvp{\psi}{v'}{\Psi}\):
        \begin{itemize}
        \item[\emph{If}]
          \begin{enumerate}[label=\alph*., ref=(\alph*)]
          \item
            \label{question:zs:option}
            \vAgent{} has the option of concluding \(\pv{\psi}{v'}\) from \(\Psi\), given the agent's reasoning from \(\Phi\) to \(\pv{\phi}{v}\).
          \end{enumerate}
        \item[\emph{and}]
          \begin{enumerate}[label=\alph*., ref=(\alph*), resume]
          \item
            \label{question:zs:subjunctive}
            \vAgent{} would not conclude \(\pv{\phi}{v}\) from \(\Phi\), if \vAgent{} were to attempt and fail to conclude \(\pv{\psi}{v'}\) from \(\Psi\).%
          \end{enumerate}
        \item[\emph{then}]
          \begin{enumerate}[label=\alph*., ref=(\alph*), resume]
          \item
            \label{question:zs:may-fail}
            \vAgent{} would conclude \(\pv{\psi}{v'}\) from \(\Psi\), if \vAgent{} were to attempt to conclude \(\pv{\psi}{v'}\) from \(\Psi\).
          \end{enumerate}
        \end{itemize}
      \end{itemize}
    \end{itemize}
    \vspace{-\baselineskip}
  \end{restatable}
\end{note}

\begin{note}
  \qzS{} is delicate.

  Implicit is link between present conclusion and potential conclusions, from the agent's perspective.

  Uncertainty about whether or not one has the option to conclude something blocks present conclusion.

  So, positive answer, for any potential conclusion, would conclude in the present.

  In negative answers, potential conclusion, but either no or negative opinion on concluding.

  This doesn't directly tell us anything about whether or not the agent concludes \(\pv{\phi}{v}\) from \(\Phi\).
  Agent may continue, or may not.

  Cover basics of \qzS{} and then link to \qWhy{}.
  Specifically, in \autoref{cha:zS:section:qzs-and-why}.

  Most of the focus here is on connecting \qzS{} with \qWhy{}.
\end{note}

\paragraph{Cases}

\begin{note}
  \autoref{question:zs} concerns an agent when concluding \(\pv{\phi}{v}\) from \(\Phi\), but just prior to having concluding \(\pv{\phi}{v}\) from \(\Phi\).
  And, in paraphrase, asks whether there is some proposition-value-premises pairing \(\pvp{\psi}{v'}{\Psi}\) such that for the agent and from their perspective:
  \begin{itemize}
  \item
    The agent \emph{may} fail to conclude \(\pv{\psi}{v'}\) from \(\Psi\), and hence \emph{may} refrain from concluding \(\pv{\phi}{v}\) from \(\Phi\).
  \end{itemize}

  This paraphrase combines and makes implicit various aspects of clauses~\ref{question:zs:option},~\ref{question:zs:subjunctive}, and~\ref{question:zs:may-fail} to form a single simple statement which is assessed from the agent's point of view.

  We step through~\ref{question:zs:may-fail},~\ref{question:zs:option}, and then~\ref{question:zs:subjunctive}:

  \begin{itemize}
  \item
    \ref{question:zs:may-fail}, the question, would the agent conclude \(\pv{\psi}{v'}\) from \(\Psi\) for any such \(\pvp{\psi}{v'}{\Psi}\).

    the question is whether the agent may \emph{refraining} to conclude \(\pv{\phi}{v}\) from \(v\), while \autoref{question:zs} (in particular~\ref{question:zs:may-fail}) asks whether the \emph{would} conclude \(\pv{\phi}{v}\) from \(v\).

  Still, given the broader context,%
  \footnote{
    In general the agent may also get sidetracked and so on, but the broader context \autoref{question:zs} is that the agent is (in the process of) concluding \(\pv{\phi}{v}\) from \(\Phi\), and likewise for the paraphrase.
  }
  these are equivalent questions.
  The difference is how negative and positive answers are characterised.
  \autoref{question:zs} has a positive answer just in case the agent would conclude, and the paraphrase has a positive answer just in case the agent would \emph{not} conclude.

  \emph{At issue} is whether there exists some \(\pvp{\psi}{v'}{\Psi}\).
\item
    \ref{question:zs:option} is implicit in the reading of `may' --- the agent may fail to conclude \(\pv{\psi}{v'}\) from \(\Psi\) only if the agent has the option of concluding \(\pv{\psi}{v'}\) from \(\Psi\).

    In~\autoref{question:zs}, ensuring that the antecedent of the `would' conditional is always interpreted relative to the agent's present epistemic state.%
    \footnote{
      Basic understanding of subjunctive conditionals: Lewis, Stalnaker, Veltman, etc.
    }
    In the paraphrase the three clauses are combined into a single statement with an implicit temporal ordering, so the relevant subjunctive antecedents are already in place.%
    \footnote{
      Possible to omit \ref{question:zs:option} if rewrite \ref{question:zs:subjunctive}.
    }

  \item
    The relationship between failing to conclude \(\pv{\psi}{v'}\) from \(\Psi\) and refraining to conclude \(\pv{\phi}{v}\) from \(\Phi\) captured by~\ref{question:zs:subjunctive} is replaced in the paraphrase by `hence'.
    Implicit is that failing to conclude \(\pv{\psi}{v'}\) from \(\Psi\) explains why the agent would refrain from concluding \(\pv{\phi}{v}\) from \(\Phi\).

  \end{itemize}
\end{note}

\begin{note}[Quantified conditional]
  Whether there is such a \(\pvp{\psi}{v'}{\Psi}\).

  Clauses~\ref{question:zs:option},~\ref{question:zs:subjunctive}, and~\ref{question:zs:may-fail} of~\autoref{question:zs} from a quantified conditional, so there are three distinct ways in which~\ref{question:zs} may receive a positive answer.
  \begin{itemize}
  \item
    For any proposition-value-premises pairing \(\pvp{\psi}{v'}{\Psi}\), either:
    \begin{enumerate}[label=\alph*\('\).]
    \item
      It is not the case that the agent has the option of concluding \(\pv{\psi}{v'}\) from \(\Psi\), given the agent's reasoning from \(\Phi\) to \(\pv{\phi}{v}\).
    \item
      The agent may conclude \(\pv{\phi}{v}\) from \(\Phi\), (even) if they were to attempt and fail to conclude \(\pv{\psi}{v'}\) from \(\Psi\).
    \item
      The agent would conclude \(\pv{\psi}{v'}\) from \(\Psi\), if they attempted to do so.
    \end{enumerate}
  \end{itemize}

  In other words, negative answer only if from the agent's perspective there is some \(\pvp{\psi}{v'}{\Psi}\) such that the agent may fail to conclude \(\pv{\psi}{v'}\) from \(\Psi\), and if the agent were to fail they would not conclude \(\pv{\phi}{v}\) from \(\Phi\).%
  \footnote{
    Same for the paraphrase.
    Existential.
  }
\end{note}

\begin{note}
  Our interest is primarily in cases where both~\ref{question:zs:option} and~\ref{question:zs:subjunctive} hold for some \(\pvp{\psi}{v'}{\Psi}\)-pairing.

  \autoref{cha:zS:sec:question:illu} will contain additional \illu{1} of positive and negative answers to~\autoref{question:zs}, and in particular \illu{1} where~\ref{question:zs:option} and~\ref{question:zs:subjunctive} fail.

  However, our motivation for considering~\autoref{question:zs} is the relation between concluding (or failing to conclude)  \(\pv{\phi}{v}\) from \(\Phi\) and concluding (or failing to conclude) \(\pv{\psi}{v'}\) from \(\Psi\).

  First link the kind of case we are interested in asking~\autoref{question:zs} to \cScen{1}, and make this link explicit by revisiting a pair of \scen{1}.

  Second, highlight why.

  Third, features in additional detail.

  Following, in~\autoref{cha:zS:sec:zS}, kind of support: \zS{}.
  Examples of \zS{} will also provide additional examples of \scen{0}.
\end{note}

\begin{note}[\cScen{1}, examples]
  {
    \color{red}
    In such cases,~\autoref{question:zs} focuses on whether the agent, from their perspective, would conclude \(\pv{\psi}{v'}\) from \(\Psi\).

    We are already familiar with cases of this kind.
    Indeed, as case of this kind is a \cScen{0}.
    For, \cScen{1}, \dots.

    For the relevant conditionals in \cScen{1}, is it the case, from the agent's perspective, that they would conclude.
  }

  To illustrate, consider again \autoref{illu:gist:roots} and \autoref{illu:sketch:prop-logic}:
\end{note}

\begin{note}[\autoref{illu:gist:roots}]
  \autoref{illu:gist:roots} involves an agent concluding either \(x = 1\) or \(x = -\sfrac{1}{2}\) from premise that for some \(x \in \mathbb{R}\), \(2x^{2} - x - 1 = 0\).%
  \footnote{
    Abstractly, \autoref{illu:gist:roots} is a case where the agent would not conclude \(\pv{\phi}{v}\) from \(\Phi\) if the agent failed to conclude \(\pv{\phi}{v}\) from \(\Psi\).
    I.e.\ the relevant conclusion is the same in both proposition-value-premises pairings, the only difference is the relevant pools of premises (and method of reasoning).
  }
  And, when concluding either \(x = 1\) or \(x = -\sfrac{1}{2}\) the agent observes that \emph{if} \(x = 1\) or \(x = -\sfrac{1}{2}\), then they would also be able to observe this via factorisation.

  In other words, if the agent attempted to conclude either \(x = 1\) or \(x = -\sfrac{1}{2}\) via factorisation and failed, the agent would not conclude either \(x = 1\) or \(x = -\sfrac{1}{2}\) via (their application of) the quadratic formula.
\end{note}

\begin{note}[\autoref{illu:sketch:prop-logic}]
  Likewise, \autoref{illu:sketch:prop-logic} involves an agent concluding some sentence \(A\) is a syntactic theorem of propositional logic via a formula derivation.%
  \footnote{
    Abstractly, \autoref{illu:gist:roots} is a case where the agent would not conclude \(\pv{\phi}{v}\) from \(\Phi\) if the agent failed to conclude \(\pv{\psi}{v}\) from \(\Psi\), where \(\phi\) is distinct from \(\psi\) and \(\Phi\) is distinct from \(\Psi\).
    Soundness (and completeness) relates syntactic and semantic theorems of propositional logic, but these are distinct, as may be observed by considering, for example, a logic which is incomplete, or an unsound proof system.
  }

  And, when concluding \(A\) is a syntactic theorem, the agent observes that \(A\) is a syntactic theorem only if \(A\) is also a semantic theorem (from soundness).

  In other words, if the agent attempt to show \(A\) is true under an arbitrary valuation and failed, the agent would not conclude \(A\) is a syntactic theorem.
\end{note}

\begin{note}[In general]
  Generally speaking, the proposition-value-premises pairing present in a \cScen{0} just is what is required for both~\ref{question:zs:option} and~\ref{question:zs:subjunctive} to hold.
  Hence, when~\autoref{question:zs} is paired with an \cScen{0},~\autoref{question:zs} asks whether the agent, from their perspective, would conclude the relevant proposition-value pair from the relevant pool of premises.
\end{note}


\begin{note}
  With example, failure to conclude because 


  Broader, failure to conclude because no conclusion.

  `\emph{Unless}'
  {
    \color{red}
    Strong understanding.
    In short, always question regarding anything weaker.
    However, we will argue for this.
  }
\end{note}

\subsection[Visualisation]{Visualisation of what is at issue when asking \qzS{}}

\begin{figure}[h]
  \centering
  \begin{tikzpicture}
    \node (origin) at (0,0) {};
    \node (psiSplit) at (1,0) {};
    \node (phiSplit) at (4,0) {};
    %
    \node[anchor=west] (psiV) at  (6,-1)  {\(\pvp{\psi}{v'}{\Psi}\)};
    \node[anchor=west] (psiNv) at (6,-2) {\(\pvp{\psi}{\overline{v'}}{\Psi}\)};
    \node[anchor=west] (psiQ) at (6,-3) {\(\pvp{\psi}{?}{\Psi}\)};
    %
    % \node[anchor=west] (psiVPhiV) at (9,-1) {\(\pv{\phi}{v}\)};
    \node[anchor=west] (psiNvPhiU) at (9,-2) {\(\pv{\phi}{\{\overline{v},?\}}\)};
    \node[anchor=west] (psiQPhiU) at (9,-3) {\(\pv{\phi}{\{\overline{v},?\}}\)};
    %
    \node[anchor=west] (phiQ) at (10,1) {\(\pv{\phi}{?}\)};
    \node[anchor=west] (phiNv) at (10,2) {\(\pv{\phi}{\overline{v}}\)};;
    \node[anchor=west] (phiV) at (10,0) {\(\pv{\phi}{v}\)};
    %
    \draw[-]  (origin) -- (phiV);
    %
    \path[-,dashed] (phiSplit) edge [out=0, in=180] (phiNv);
    \path[-,dashed] (phiSplit) edge [out=0, in=180] (phiQ);
    %
    \path[-.] (psiSplit) edge [out=0, in=180] (psiV);
    \path[-, dashed] (psiSplit) edge [out=0, in=180] (psiNv);
    \path[-, dashed] (psiSplit) edge [out=0, in=180] (psiQ);
    %
    \draw[<-,dotted] (psiV) edge [out=0, in=180] (phiV);
    \draw[->, dotted] (psiNv) edge (psiNvPhiU);
    \draw[->, dotted] (psiQ) edge (psiQPhiU);
    \end{tikzpicture}
    \caption{Sketch of when \qzS{} has a negative answer.}
    \label{fig:csN:illu:overview}
  \end{figure}

\begin{note}[Figure]
  \autoref{fig:csN:illu:overview} provides a rough visualisation of~\qzS{}.

  The flat line captures the agent's reasoning, which concludes with \(\pv{\phi}{v}\).
  In concluding \(\pv{\phi}{v}\) the agent rules out two possibilities with respect to \(\phi\).
  First, that \(\phi\) does not have value \(v\), indicated by \(\pv{\phi}{\overline{v}}\).
  Second, that the agent does not assign any value to \(v\), indicated by \(\pv{\phi}{?}\).
  Prior to concluding \(\pv{\phi}{v}\), the agent's reasoning may have branched to either alternative path, but as the agent has concluded \(\pv{\phi}{v}\), neither path is viable, and hence both paths are represented with a dashed line.

  So far, we have seen only that the agent has concluded \(\pv{\phi}{v}\).

  We now consider some proposition-value-premises pairing \(\pv{\psi}{v'}{\Psi}\) such that if the agent were to fail to conclude \(\pv{\psi}{v'}\) from \(\Phi\), the agent would not conclude \(\pv{\phi}{v}\) from \(\Phi\).

  Intuitively, the dotted arrows from the various combinations of \(\psi\) and \(\{v',\overline{v'},?\}\) read, from top to bottom:
  \begin{itemize}
  \item If \(\phi\) has value \(v\) then the agent may conclude \(\pv{\psi}{v'}\) from \(\Psi\), and:
  \item If the agent concludes \(\psi\) has some value \(\overline{v'}\) from \(\Psi\), then the agent either concludes \(\phi\) has some value other than \(v\), or the agent fails to reach a conclusion regarding \(\phi\) from \(\Phi\).
    Both options are combined via the shorthand \(\pv{\phi}{\{\overline{v},?\}}\).
  \item
    And, likewise if the agent fails to conclude \(\pv{\psi}{v'}\) from \(\Psi\).
  \end{itemize}

  With respect to concluding, observe that prior to ruling out alternative branches with respect to \(\pv{\phi}{\{\overline{v},?\}}\), the agent may have reasoned about whether \(\psi\) has value \(v\).
  And, from the agent's perspective, \(\phi\) has value \(v\) only if \(\psi\) has value \(v'\).
  If \(\psi\) does not have value \(v'\), then either \(\phi\) does not have value \(v\), or the agent's reasoning would not conclude with a value for \(\phi\), indicated by \(\pv{\phi}{\{\overline{v},?\}}\).

  Hence, prior to concluding \(\pv{\phi}{v}\), the agent has concluded \(\pv{\psi}{v'}\).
\end{note}

\begin{note}
  Broadly, then, we may say that an agent has {\color{red} particular kind of conclusion} for \(\pv{\phi}{v}\) just in case when concluding \(\pv{\phi}{v}\) it is not the case that the agent's reasoning would have branched to a different conclusion with respect to \(\phi\).

  However, the visualisation of~\autoref{fig:csN:illu:overview} and this broad statement of {\color{red} positive answer to \qzS{}} are a little too broad.
  For, we are only interested in proposition-value pairs guaranteed by \(\phi\) having value \(v\).
  {\color{red} positive answer to \qzS{}} is not global with respect to all proposition-value pairs that the agent may have reasoned about, but local to those guaranteed by the proposition.
\end{note}

\section{\emph{Why}}
\label{cha:zS:section:qzs-and-why}

\begin{note}
  Intuitively, positive answer in \scen{1}.

  Some care has been taken.

  \autoref{illu:sketch:prop-logic}.
  Only if semantic proof.
  Syntactic proofs, at least in my experience, may be out of reach.
  However, semantic proofs, often straightforward.

  Converse may hold, but more challenging than \autoref{illu:sketch:prop-logic}.

  Similarly, \autoref{illu:gist:roots}.
  Factorisation isn't too difficult.

  \autoref{illu:sketch:math}.
  \(345 \div 15 = 23\) \emph{only if} \(23 \times 15 = 345\).

  Agent has the opportunity, and current result looks good.
\end{note}

\begin{note}
  Now, answers, in part, why.
\end{note}

\begin{note}
  Rough understanding.
  In terms of broader argument, \emph{why}.

  Idea is that agent's perspective regarding \(\pvp{\psi}{v'}{\Psi}\) in part explains why.

  Preferred \illu{0} concerns whether one has or has not lost their keys.
\end{note}

\begin{note}[Motivating \illu{0}]
  {
    \color{red}
    Interest in terms of explaining why and agent did or didn't conclude.
  }

  \begin{illustration}[Lost keys]
    \label{illu:lost-key}
    Tempting as it may be to conclude that a pair of keys are lost after some searching, if the keys really are lost then there aren't in a handful of places you haven't yet thought to look. And, until you have concluded that the keys really aren't in those places, and that there is no-where else to look, the keys aren't really lost.
  \end{illustration}

  For the agent's perspective, there is some \(\pvp{\psi}{v'}{\Psi}\), and this explains why the agent does not conclude they have lost their keys.
\end{note}

\begin{note}
  Similar points for examples given.

  Check via factorising, check via a semantic proof.
\end{note}

\begin{note}
  Here, important, \issueConstraint{} asks about whether a relation of support is part of why an agent \emph{concludes} (only if agent has witnessed).

  With~\autoref{illu:lost-key},~\autoref{illu:gist:roots}, and~\autoref{illu:sketch:prop-logic}, failure to conclude!
\end{note}

\begin{note}
  Still, \emph{negative} answers to~\autoref{question:zs}.

  Interest, what about \emph{positive} answers?

  Positive answer only if from agent's perspective, would conclude.
  No witnessing.
  Relation of support is part of why.

  Lack of support explains, in part, why agent does \emph{not} conclude.
  Conversely, presence of support explains, in part, why agent \emph{does} conclude.
\end{note}

\paragraph{Basic principle}

\begin{note}
  \begin{idea}[\autoref{question:zs} and \emph{Why}]
    \label{prop:qzS-answers-why}
    In some cases:

    \begin{itemize}
    \item[\(\pm\)]
      Answers to \autoref{question:zs} answer, in part, why an agent concludes or fails to conclude \(\pv{\phi}{v}\) from \(\Phi\).
    \end{itemize}

    Specifically:
    \begin{itemize}
    \item[\(-\)]
      A negative answer to~\autoref{question:zs} answers, in part, why an agent does not conclude \(\pv{\phi}{v}\) from \(\Phi\).
    \item[\(+\)]
      A positive answer to~\autoref{question:zs} answers, in part, why an agent does conclude \(\pv{\phi}{v}\) from \(\Phi\), and hence is, in part, an answer to both \qWhy{} and \qWhyV{}.
    \end{itemize}
  \end{idea}

  Argument for \autoref{prop:qzS-answers-why} is by cases.
  See additional cases in \autoref{cha:zS:sec:question:illu}.

  Weak point.
  \autoref{prop:qzS-answers-why} is central to overall argument.
  Hence, something else which captures cases.
  Something else does not involve whether or not the agent would conclude.

  I do not think so.
  But, generalising from exhaustion.
  Have no exhausted every possibility, but I've exhausted myself.

  Preferable, I think, to hold that \qzS{} is not relevant.
  \autoref{prop:qzS-answers-why} is an existential.
  Generally speaking, would be good.
  Only trouble when \(\pvp{\psi}{v'}{\Psi}\) is such that the agent has not witnessed reasoning to \(\pv{\psi}{v'}\) from \(\Psi\).
  So, if \qzS{} only applies in such cases, then no problem.

  However, already seen, \autoref{illu:lost-key}.
  Absence of reasoning.

  So, narrow to no cases where a positive answer, but consider \cScen{1}.
\end{note}

\begin{note}
  Run this through \scen{1} listed.
\end{note}

\begin{note}
  \fc{}!
  Or, forgone conclusion, but this is different from a \fc{}.
\end{note}

\begin{note}
  What's needed is positive answer only if support.

  Here, maybe illustrate with general ability.
  Got X.
  General ability.
  So, specific ability to Y.
\end{note}

\begin{note}
  {
    \color{red}
    There's a difference between answering `no' and failing to answer.
    But, the point I'm arguing for works given this distinction.
    There's no real different.
    I mean, the conditional is either true or false.
    But, it's possible that the falsity of the conditional has a role where the truth of the conditional does not.
  }
\end{note}

\subsection{Deviance}
\label{sec:deviance}

\begin{note}
  Here, causal deviance.
\end{note}

\begin{note}
  Problem is, there's no way to guarantee a link between positive answer to \qzS{} and the agent concluding or not refraining from concluding.
\end{note}

\begin{note}
  Argument relies on tying content to explanation.

  In this respect, there is room for an objection.
  Deviant causal chains.
  Point here is that there are cases where these come apart.

  This isn't only a problem for causal theories of reasoning.
  The point is, some instantiation, and so long as act may be caused by something else, then possibly caused by the instantiation.

  So, possible here.

  Well, hold on.
  What is need is the relevance of the content.
  For this objection to work, need to take a theoretical perspective.
  See, in Davidson's case, the idea is fusing these two things together.
  We answer two different questions with a common thing viewed in two ways.

  Still, I think the objection can be pressed!
  Only \emph{really} an explanation is no deviance.
  To the same extent that potential event matters, it matters to the agent that there is no deviance.

  {
    \color{red}
    Resolution is, if deviance, then no agency.
  }

  I think this makes sense, or at least makes enough sense.
  Answers to `why', on this understanding, are tentative.

  Or, rest on presupposition that agent performed the action.

  So, contingent on showing there is no causal deviance.

  This is different to error.
  With error, thing appealed to isn't the case, but appeal still did work.
  Here, it doesn't matter whether or not the case, no work is done.

  In contrast to more typical instances of the problem, don't need to rule out deviant causal chains.
  Instead, just need one instance to fail to hold.
  One instance of non-deviousness.

  Still a problem for a compatible account which avoids.
  For, here, there can't be any direct link from perspective to reason.

  For example, \citeauthor{Hieronymi:2011aa}

    \begin{quote}
      [W]e explain an event that is an action done for reasons by appealing to the fact that the agent took certain considerations to settle the question of whether to act in some way, therein intended so to act, and successfully executed that intention in action.
    [\emph{T}]\emph{his} complex fact, [\dots] is the reason that rationalizes the action---that explains the action by giving the agent's reason for acting.%
    \mbox{ }\hfill\mbox{(\citeyear[431]{Hieronymi:2011aa})}
  \end{quote}

  So, here, considerations which settle question, and in so settling question.
  Link between settling the question and acting.

  Following \citeauthor{Hieronymi:2011aa}, no room for deviance.
  Too tight.

  In other words, so long as this fact holds, there is no distinction between settling the question and acting.
  Therefore, no deviance.

  Compatible, I think.
  Question is whether in resolving \qzS{} is sufficiently tied to resolving the question \citeauthor{Hieronymi:2011aa} identifies.
  And, plausibly is.
  This is what the motivation for \qzS{} did.

  Trouble is, for our purposes, need at least sufficient conditions for when this complex fact obtains.
  And, no account of this.

  \citeauthor{Hieronymi:2011aa} notes the gaps.

  Some tension.
  These considerations aren't premises.
\end{note}

\begin{note}
  So, the other option is to embrace deviant causal chains.
  Have the content, but this doesn't work in the way the agent thinks it does.

  Example from Davidson.

  The trouble here is that the content and resulting action match.
  So, things make sense from the agent's point of view.

  Deviant, but maybe not so deviant here.

  Systematic deviance, where content is separated from role of mental state.

  But, I see no motivation for this.

  Solution to causal chains doesn't get round this, because the result is a restricted account.
  So, there's no guaranteed trade-off here.
  Trouble is, it seems hard to see a case where this wouldn't be the case.
\end{note}


\subsection{Details}
\label{cha:zS:sec:details}

\subsubsection{Clauses~\ref{question:zs:option},~\ref{question:zs:subjunctive}, and the idea of a \requ{0}}
\label{cha:zS:sec:clauses-idea-requ1}

\paragraph{The components of \qzS{}}

\begin{note}
  The primary clauses of interest are clauses~\ref{question:zs:option} and~\ref{question:zs:subjunctive}.

  Intuitively, clause~\ref{question:zs:option} means that, so long as \(\phi\) has value \(v\), the agent has the option of checking whether it makes sense for the agent to conclude \(\pv{\phi}{v}\) from \(\Phi\).
\end{note}
  And, clause~\ref{question:zs:subjunctive} expresses that concluding \(\pv{\psi}{v'}\) from \(\Psi\) is a check on whether it makes sense for the agent, from their perspective, to conclude \(\pv{\phi}{v}\) from \(\Phi\).

\paragraph{General}

\begin{note}
  Constraints placed on \(\pvp{\psi}{v'}{\Psi}\).
  From reasoning involved in process of concluding \(\pv{\phi}{v}\) from \(\Phi\).
  Would lead to failure.

  Conditionals.

  Involved in concluding \(\pv{\phi}{v}\) from \(\Phi\).
  First, enough to break.
  Second, reasoning makes this proposition-value-premises pairing available.

  Pair of additional features

  Second highlights why \(\pvp{\psi}{v'}{\Psi}\) is of interest.
  However, in this respect, not strictly required.
  Given universal, will also include these.

  First,
  Don't need \(\phi\) to have value \(v\).
  Also, implicit, no revision.
  Built up various things in reasoning, and given all of this\dots


  And, maybe reasoning offers something new.
  Though, not the case that \(\pvp{\psi}{v'}{\Psi}\) only from something new.
  Might be the case that negative answer because go off on wrong reasoning.
\end{note}

\paragraph{Option}

\begin{note}
  Hence, it need not be the case that the agent has the option of concluding \(\pv{\psi}{v'}\) from \(\Psi\) from their epistemic state prior to starting line of reasoning (as the agent has not yet concluded that \(\phi\) has value \(v\)).
\end{note}


\paragraph{Would not conclude}

\begin{note}
  Noted failure.
\end{note}

{
  \color{red}
  Note, \ref{question:zs:may-fail} is delicate.
  For, the combination of \ref{question:zs:subjunctive} and \ref{question:zs:option} suggest there is a way of concluding \(\pv{\psi}{v'}\) from \(\Psi\).
  Hence, \ref{question:zs:may-fail} may be read in reference to this.
  However, \ref{question:zs:may-fail} is intended to allow other ways of concluding \(\pv{\psi}{v'}\) from \(\Psi\).
  What matters is that the agent has not concluded \(\pv{\psi}{v'}\) from \(\Psi\), the agent has the option, and the agent may fail.%
  \footnote{
    This is important for witnessing, but also motivated by different methods.
    A different way to putting this is that concluding is two place relation.
    Between premises and conclusion.
    Concluding is not a three place relation between premises, conclusion, and method.
    I should really have this stated as an assumption.

    Still, there is a variant where method comes into play, as I have this via ability.
  }
}

%%%% TEMP from question
\footnote{
  Clause~\ref{question:zs:subjunctive} is expressed by a subjunctive conditional as there is no requirement that the agent will attempt to conclude \(\pv{\psi}{v'}\) from \(\Psi\).

  \color{red}
  As this alternative expression makes clear,~\autoref{question:zs} focuses on the agent (and their epistemic state).
  At no point do we consider any variation of the agent's epistemic state.
  Likewise,~\autoref{question:zs} concerns only the agent's perspective on concluding \(\pv{\psi}{v'}\) from \(\Psi\).
  Whether or not the agent would conclude \(\pv{\psi}{v'}\) from \(\Psi\) is irrelevant.
  What matters is whether, from the agent's perspective, there is potential for reasoning about whether \(\pv{\psi}{v'}\) follows from \(\Psi\) to block concluding \(\pv{\phi}{v}\) from \(\Phi\).
}

\paragraph*{Minor clarifications}

\begin{note}[Importance of \csN{}]
  First, agent's reasoning.
  At issue is whether the agent may reason to a different conclusion.
  There's nothing that would lead me elsewhere.

  Second, agent's reasoning.
  Independent of whether \(\phi\) has value \(v\), \(\psi\) has value \(v'\), or any of the premises.
  Need not be the case that satisfaction amounts to anything substantial.
  No clause for justification, etc.

  Third, competence, rather than performance.
\end{note}

\subsection{\requ{3}}

\begin{note}
  We begin by refining the relevant \(\pvp{\psi}{v'}{\Psi}\) proposition-value-premises pairings of interest from~\qzS{}.
  We term such proposition-value-premises pairings `\requ{1}' of concluding \(\pv{\phi}{v}\) from \(\Phi\).
\end{note}

\begin{note}[Notion of a \requ{}]
  \begin{idea}[\iRequ{}]
    \label{idea:requ}
    \(\pvp{\phi}{v'}{\Psi}\) is a \emph{\requ{}} of concluding \(\pv{\phi}{v}\) from \(\Phi\), with respect to an agent \vAgent{}'s epistemic state if:
    \begin{enumerate}
    \item
      \label{idea:requ:main}
      From the perspective of \vAgent{}' epistemic state, \(\phi\) has value \(v\) only if:
      \begin{enumerate}[label=\alph*., ref=\named{R:\alph*}]
      \item
        \label{idea:requ:pool}
        \vAgent{} has the option of concluding \(\pv{\psi}{v'}\) from \(\Psi\) where:
        \begin{enumerate}[label=\roman*., ref=\named{R:a.\roman*}, series=csIdeaCounter]
        \item
          \label{idea:requ:pool:int}
          \vAgent{} may conclude \(\pv{\psi}{v'}\) from \(\Psi\) without concluding \(\pv{\phi}{v}\) from \(\Phi\) as an intermediary step.
        \item
          \label{idea:requ:pool:ind}
          For any proposition-value pair \(\pv{\psi_{i}}{v_{i}}\) in \(\Psi\), \vAgent{} either has concluded or may conclude \(\pv{\psi_{i}}{v_{i}}\) without concluding \(\pv{\phi}{v}\) from \(\Phi\).
        \end{enumerate}
      \item
        \label{idea:requ:nPsi-nPhi}
        If \vAgent{} were to fail to conclude \(\pv{\psi}{v'}\) from \(\Psi\) prior to reasoning about whether \(\phi\) has value \(v\) given \(\Phi\), \vAgent{} would not conclude \(\pv{\phi}{v}\) from \(\Phi\).
      \end{enumerate}
    \end{enumerate}
    \vspace{-\baselineskip}
  \end{idea}

  With the key clause linking~\autoref{idea:requ} to~\qzS{} is clause~\ref{idea:requ:nPsi-nPhi}.
  For, clause~\ref{idea:requ:nPsi-nPhi} captures the core idea of failure to conclude \(\pv{\psi}{v'}\) from \(\Psi\) leading to failure to conclude \(\pv{\phi}{v}\) from \(\Phi\).

  The role of clause~\ref{idea:requ:pool} is explicitly state various properties \(\pv{\psi}{v'}{\Psi}\) must have in order for any failure to conclude \(\pv{\psi}{v'}\) from \(\Psi\) is relevant to concluding \(\pv{\phi}{v}\) from \(\Phi\).%
  \footnote{
    Indeed, we take \ref{idea:requ:pool:int} and~\ref{idea:requ:pool:ind} to be more-or-less implicit constraints on \(\pvp{\psi}{v'}{\Psi}\) in the statement of \qzS{}.
  }
  In particular, \ref{idea:requ:pool:int} and~\ref{idea:requ:pool:ind} are required to ensure the agent may conclude \(\pv{\psi}{v'}\) from \(\Psi\) independently of concluding \(\pv{\phi}{v}\) from \(\Phi\).

    For, if \ref{idea:requ:pool:int} and \ref{idea:requ:pool:ind} were to fail to hold then:
  \begin{itemize}
  \item
    By~\ref{idea:requ:pool:int}, the agent would need to conclude \(\pv{\phi}{v}\) from \(\Phi\) as a sub-conclusion when reasoning from the relevant pool of premises \(\Psi\).
    Hence, it would not be possible to conclude \(\pv{\psi}{v'}\) from \(\Psi\) without first concluding \(\pv{\phi}{v}\) from \(\Phi\).
  \item
    And, likewise, by~\ref{idea:requ:pool:ind}, the agent need to have already concluded \(\pv{\phi}{v}\) from \(\Phi\) in order to appeal to some of the proposition-value pairs in the relevant pool of premises \(\Psi\).
  \end{itemize}

  Conversely, if both~\ref{idea:requ:pool:int} and~\ref{idea:requ:pool:ind} hold, the agent may conclude \(\pv{\psi}{v'}\) from \(\Psi\) independently of concluding \(\pv{\phi}{v}\) from \(\Phi\).

  Note, however, neither~\ref{idea:requ:pool:int} nor~\ref{idea:requ:pool:ind} rule out the possibility of the agent concluding \(\pv{\phi}{v}\) from \(\Phi\) when concluding \(\pv{\psi}{v'}\) from \(\Psi\) or, conversely, concluding \(\pv{\psi}{v'}\) from \(\Psi\) when concluding \(\pv{\phi}{v'}\) from \(\Phi\).
  There may be an interesting variant of the notion of a \requ{} with such a constraint in place, but such a constraint is not of interest with respect to \qzS{}.
  For, at issue is only whether the agent may at interest is only failure to conclude \(\pv{\psi}{v'}\) from \(\Psi\), and both~\ref{idea:requ:pool:int} and~\ref{idea:requ:pool:ind} ensure lack of concluding \(\pv{\phi}{v}\) from \(\Phi\) will not prevent the agent from reaching a conclusion regarding whether \(\psi\) has value \(v\) given \(\Psi\).
\end{note}

\begin{note}
  \color{red}
  Has the option.
\end{note}

\begin{note}[\requ{2}: Partial check]
  Intuitively, concluding \(\pv{\psi}{v'}\) from \(\Psi\) would serve as a partial check on whether the agent may reason to a conclusion other than \(\pv{\phi}{v}\), captured by~\ref{idea:requ:nPsi-nPhi}.

  Concluding \(\pv{\psi}{v'}\) from \(\Psi\) is a check.
  For, if the agent were to fail to conclude \(\pv{\psi}{v'}\) from \(\Psi\) then, from the perspective of the agent's epistemic state, the agent would not conclude \(\pv{\phi}{v}\) from \(\Phi\).
  Hence, contraposing, the agent would conclude \(\pv{\phi}{v}\) from \(\Phi\) only if the agent would conclude \(\pv{\psi}{v'}\) from \(\Psi\).
  However, the check is partial, as it need not be the case that the agent would conclude \(\pv{\psi}{v'}\) from \(\Psi\) only if the agent \(\pv{\phi}{v}\) from \(\Phi\).
  Therefore, failing to conclude \(\pv{\psi}{v'}\) from \(\Psi\) may block concluding \(\pv{\phi}{v}\) (from the perspective of the agent's epistemic state) though concluding \(\pv{\psi}{v'}\) from \(\Psi\) need not ensure that the agent would conclude \(\pv{\phi}{v}\).

  Now, \ref{idea:requ:nPsi-nPhi} contains a slight subtlety.
  For, from~\autoref{assu:conc:d-free}, an agent may conclude various proposition-value pairs from some instance of reasoning without explicit recognition.
  Therefore,~\ref{idea:requ:nPsi-nPhi} does not state that the agent may fail to conclude \(\pv{\phi}{v}\) from \(\Phi\).
  Rather, \ref{idea:requ:nPsi-nPhi} holds that from the perspective of the agent's epistemic state, the agent may fail to conclude \(\pv{\phi}{v}\) from \(\Phi\).
  Again, we tread a fine line between the role of an agent's epistemic state and the role of the agent's \stance{}.
  The role of an agent's epistemic state determines whether \(\pv{\psi}{v'}\) is a \requ{} of concluding \(\pv{\phi}{v}\) from \(\Phi\).
  And, an agent's epistemic state may determine whether an agent concludes \(\pv{\chi}{v''}\) when concluding \(\pv{\phi}{v}\).%
  \footnote{
    Recall the above discussion of \(\pv{\phi}{v}\) \indicatePr{} \(\pv{\chi}{v'}\) in relation to~\autoref{assu:conc:d-free}.
  }
  Therefore, the agent's epistemic state --- the agent's perspective on how things are --- is key.
  However, the agent's \stance{} is unimportant.
  Whether an agent has concluded \(\pv{\phi}{v}\), or whether \(\pv{\psi}{v'}\) is a \requ{} is not a question of whether the agent recognises the have concluded \(\pv{\phi}{v}\) or recognises \(\pv{\psi}{v'}\) is a \requ{}.

  Combining these two ideas, intuitively, \(\pv{\psi}{v'}\) is a \requ{} of concluding \(\pv{\phi}{v}\) just in case there is some pool of premises \(\Psi\) such that determining whether the agent would conclude \(\pv{\psi}{v'}\) is an independent partial check on whether the agent may reason to a conclusion other than \(\pv{\phi}{v}\).
\end{note}


\subsubsection{Abstract}

\begin{note}[Abstract motivation]
  First, the combination of~{\color{red} ???} and~{\color{red} ???} keeps the complexity of resolving~\autoref{question:zs} relatively low.
  We only need to consider what \(\phi\) having value \(v\) would commit the agent to, so to speak.
  For example, we do not need to consider the consequences of the agent reasoning about some arbitrary proposition-value pair \(\pv{\chi}{v''}\) prior to concluding \(\pv{\phi}{v}\) from \(\Phi\).

  Of course, if the agent would conclude both \(\pv{\chi}{v''}\) and \(\pv{\chi}{\overline{v''}}\) for some \(\chi\), then it seems the agent's epistemic state is in bad shape.
  Still, given that an agent will typically revise their epistemic state upon concluding both \(\pv{\chi}{v''}\) and \(\pv{\chi}{\overline{v''}}\) for some \(\chi\), such concerns may be isolated to a distinct question.

  Second, we avoid --- to some extent --- concerns about over-generating.
  Our overall argument will put~\autoref{question:zs} to work in motivating a negative resolution to~{\color{red} issue:Main}.
  Primarily by drawing consequences from what we will argue is an equivalent characterisation of negative resolutions to~\autoref{question:zs}.
  A concern is that this overall argument may over-generate.
  Given that a negative resolution to~{\color{red} issue:Main} seems by no means clear, unintended consequences of our interest in~\autoref{question:zs} may diminish interest in the tension we hope to motivate, and hence tip favour to a positive answer to~{\color{red} issue:Main}.
  Whether or not there are unintended consequences of broadening the scope of~\autoref{question:zs} is unclear to me.
  Still, without need to investigate, we may ignore any such consequences that may arise.
\end{note}

\subsection*{Narrowing \requ{1}}

\begin{note}[Expanding pool constraints]
  To~\ref{idea:requ:pool} of~\autoref{idea:requ} the following clause may also be added:
  \begin{enumerate}[label=]
  \item
    \begin{enumerate}[label=]
    \item
      \begin{enumerate}[label=\roman*., ref=(\roman*), resume*=csIdeaCounter]
        \setcounter{enumiii}{3}
      \item
        \label{idea:requ:pool:method}
        Concluding \(\pv{\psi}{v'}\) from \(\Psi\) involves the same general method the agent would use to conclude \(\pv{\phi}{v}\) from \(\Phi\).
      \end{enumerate}
    \end{enumerate}
  \end{enumerate}
  We omit~\autoref{idea:requ:pool:method} from the idea of \csN{} for two (related) reasons.
  First, it is not clear what `the same general method' amounts to in detail.
  Second, avoiding questions about method affords flexibility when providing \illu{1} of \zS{}.
  However,~\autoref{idea:requ:pool:method} may be imposed with no loss to the role of \zS{} in the overall argument.
  However, always a check on whether one has the general ability.
\end{note}

\begin{note}
  Reasoning, \support{}, would not reason to a different conclusion.

  Specifically, \requ{} of some conclusion.
  So long as conclusion, then it is possible to reason about whether \(\psi\) has value \(v'\), and unless conclude \(\psi\) has value \(v'\), would not conclude \(\phi\) has value \(v\).

  Intuitively, \requ{} as an independent check on the reasoning.
  If don't hold \(\psi\) from premises, then question about whether \(\phi\).

  Claiming support, necessary condition is satisfying all \requ{1}.
  Claiming support, then, is weaker than having support.
  Restricted to whether conclusion of reasoning would introduce a \requ{}.
  And, may be further restricted without impact to the tension we will develop to whether the conclusion would `clearly' introduce a \requ{}.
\end{note}

\subsubsection{Prior to concluding\dots}

\begin{note}[Prior to concluding\dots]
  An important feature of \qzS{} \dots

  Not particularly marked.
  Allow agent to have built up a bunch of stuff in the reasoning.

  Example.

  \begin{illustration}[Velocity]
    \label{ill:velocity}
    Agent is provided with information about how far a car has travelled north as a function of time travelled.

    From this, take the derivative of the function to obtain the (instantaneous) velocity of the car at a handful of points in time.

    And, from the (instantaneous) velocity of the car, the agent calculates the (instantaneous) acceleration of the car at each of the points in time.

    The agent also has information about the speed of the car as a function of time travelled, and the agent may calculate speed by the taking magnitude of the (instantaneous) velocity of the car.
  \end{illustration}

  

  \autoref{ill:velocity}, two step calculation.
  Velocity, acceleration.
  After the first step, check by taking the magnitude.
  Calculation of velocity is correct only if taking the magnitude matches speed.

  Just before concluding to include cases such as this.

  Note, \cScen{}.
\end{note}

\begin{note}
  Example highlights how `intermediate conclusions' relate.
  Further point of interest:
  Failure to conclude.

  Two ways to view agent's calculation of the velocity of the car.

  First, as a conclusion.
  Same status as the function.

  Or, as temporary.

  Difference in how we understand agent's present epistemic state.

  On first, the agent's present epistemic state is inconsistent.
  Two proposition-value pairs which conflict.
  Not possible for the car to have velocity the agent calculated and acceleration the agent has been informed of.

  May also be that the function and information about acceleration are inconsistent, but may also be that the agent made a mistake in calculating the velocity of the car.

  On second, the agent's present epistemic state may%
  \footnote{
    Don't have complete perspective on agent's present epistemic state.
  }
  consistent.

  For, made a mistake.
  But, proposition-value pair is not part of present epistemic state, so distinguished from function and information about acceleration, which are consistent.

  This is a distinction we have little interest in.
  What matters is failure to conclude speed.
  Result is either revising inconsistent epistemic state, or abandoning intermediary steps of reasoning.
\end{note}


\section{\zS{}}
\label{cha:zS:sec:zS}

\begin{note}[Answers to \qzS{}]
  For ease of reference, define.
  \begin{definition}[\izS{}]
    \label{idea:zS}
    For an agent \vAgent{}, after concluding \(\pv{\phi}{v}\) from \(\Phi\):
    \begin{itemize}
    \item
      \vAgent{} has \zS{} with respect to \(\pvp{\phi}{v}{\Phi}\).
    \end{itemize}

    \emph{if and only if}

    \begin{itemize}
    \item
      \qzS{} had a \emph{positive} answer when \vAgent{} was concluding \(\pv{\phi}{v}\) from \(\Phi\).
    \end{itemize}
  \end{definition}

  Slight problem here with going for after concluding.
  Possible, it seems, for relation of support to no longer hold.
  Likewise for \zS{}.
  For ease, assume agent has not retracted conclusion.
\end{note}

\begin{note}
  Now, two basic propositions follow.

  \begin{proposition}[When a agent has \zS{}]
    Agent and proposition-value-premises pairing.

    \begin{itemize}
    \item
      Agent \emph{has} \zS{} for \(\pv{\phi}{v}\) after concluding \(\pv{\phi}{v}\) from \(\Phi\).
    \end{itemize}

    \emph{if and only if}

    \begin{itemize}
    \item
      Either:
      \begin{enumerate}[label=(\alph*), ref=\alph*]
      \item
        There is no proposition-value-premises pairing \(\pvp{\psi}{v'}{\Psi}\) for which the relevant conditions are met.
      \item
        There is some proposition-value-premises pairing \(\pvp{\psi}{v'}{\Psi}\) but, from the agent's perspective, the agent would not fail to conclude \(\pv{\psi}{v'}\) from \(\Psi\).
      \end{enumerate}
    \end{itemize}
  \end{proposition}

  Second, when an agent does not have \zS{}.

  \begin{proposition}[When a agent does not have \zS{}]
    Agent and proposition-value-premises pairing.
    \begin{itemize}
    \item
      Agent \emph{does not} have \zS{} for \(\pv{\phi}{v}\) after concluding \(\pv{\phi}{v}\) from \(\Phi\).
    \end{itemize}

    \emph{if and only if}

    \begin{itemize}
    \item
      There is some proposition-value-premises pairing \(\pvp{\psi}{v'}{\Psi}\) such that, from the agent's perspective, the agent may fail to conclude \(\pv{\psi}{v'}\) from \(\Psi\).
    \end{itemize}
  \end{proposition}
\end{note}

\begin{note}
  Note, concluding is neither safe nor sensitive (assumption), so neither is \zS{}.
\end{note}

\section{Additional \illu{1}}
\label{cha:zS:sec:question:illu}

\begin{note}
  Some \illu{1}.
  Two broad sections.

  \begin{itemize}
  \item
    First,~\autoref{cha:zS:sec:question:illu:basic} provides a collection of \illu{1} in which an agent (plausibly) would or would not have \zS{} if they were to conclude some proposition-value pair.

    This section is further subdivided:
    \begin{itemize}
    \item
      \autoref{cha:zS:sec:question:illu:basic:does-not-have} provides a pair of \illu{1} in which an agent (plausibly) would not have \zS{} if they made the relevant conclusion.
    \item
      \autoref{cha:zS:sec:question:illu:basic:has} provides a handful of \illu{1} in which an agent (plausibly) would have \zS{} if they made the relevant conclusion.
      In particular, this section provides discussion of when an agent would (plausibly) have \zS{} due to the absence of any \requ{1}.
    \end{itemize}
  \item
    Second, \autoref{cha:zS:sec:question:illu:carroll} provides an involved \illu{1} which links \qzS{} to \citeauthor{Carroll:1895uj}'s tale about the Tortoise and Achilles.

    These \illu{1} as a whole are optional, but this section is particularly so.
    The discussion of \citeauthor{Carroll:1895uj}'s tale is involved, and focuses in roughly equal parts on providing further clarification of \zS{} and providing a novel (at least to my awareness) \emph{interpretation} of \citeauthor{Carroll:1895uj}'s tale.%
    \footnote{
      I doubt very much that the interpretation given is what \citeauthor{Carroll:1895uj} had in mind.
      On the other hand, it's not clear that \citeauthor{Carroll:1895uj} had any particular interpretation in mind (\cite[Cf.][]{Thomson:2010tt}).
    }
  \end{itemize}
\end{note}

\subsection{Basic \illu{1}}
\label{cha:zS:sec:question:illu:basic}

\subsubsection{The agent (plausibly) does not have \zS{}}
\label{cha:zS:sec:question:illu:basic:does-not-have}

\begin{note}[Simple negative answers]
  Variant on lost keys, where agent considers plausible they may reason from some other premise.
  {
    \color{red}
    \begin{illustration}
      \begin{enumerate}
      \item
        A search for `Measurement Theory' via the LCC `H61 .R593' returned no results.
      \item
        The library does not have a copy of `Measurement Theory'.
      \end{enumerate}
    \end{illustration}

    So, if conclude, then option of concluding the library does not use DDC indexing.
    May fail to conclude.
  }

  A more direct variant is spot the difference without a clear statement of how many differences are in the picture.
  Continue to search.
\end{note}

\begin{note}[A copper kettle]
  A further \illu{0} builds on a story as told by~\citeauthor{Freud:1960wx}.
  \begin{illustration}[A copper kettle]
    \label{illu:kettle}
    \mbox{ }
    \vspace{-\baselineskip}
    \begin{quote}
      `A.\ borrowed a copper kettle from B.\ and after he had returned it was sued by B.\ because the kettle now had a big hole in it which made it unusable.
      His defence was:
      ``%
      First, I never borrowed a kettle from B.\ at all;
      secondly, the kettle had a hole in it already when I got it from him;
      and thirdly, I gave him back the kettle undamaged.%
      '''\newline
      \mbox{ }\hfill\mbox{(\citeyear[62]{Freud:1960wx})}
    \end{quote}
    An agent listens to A.'s defence, but does not conclude A.\ has provided testimony.
  \end{illustration}

  The agent's failure to conclude A.\ has provided testimony may be understood in terms of a \requ{}.
  For, A.\ has provided testimony only if what A.\ has said is true.
  And, what A.\ has said is true only if the three points of A.'s defence are jointly consistent.
  Putting these observations together, we have the following conditional:
  \begin{itemize}
  \item
    A.\ has provided testimony \emph{only if} if the three points of A.'s defence are jointly consistent.
  \end{itemize}
  Hence, failure for the agent to conclude the consequent would prevent the agent from concluding the antecedent.

  Here, then, clause \ref{idea:requ:nPsi-nPhi} of \iRequ{} is satisfied.

  Likewise, there are only three points, and checking for consistency does not require the agent to establish whether the points are (actually) true.
  Hence, clause \ref{idea:requ:pool} of \iRequ{} is also satisfied.

  Sp, before concluding A.\ has provided testimony, the agent reasons about whether the three points of A.'s defence are jointly consistent.
  After, the agent does not conclude A.\ has provided testimony.%
  \footnote{
    Any pair of points are jointly inconsistent.
    For example, consider the first and third:
    If A.\ never borrowed the kettle from B, then it is not possible for A.\ to have returned the kettle to B.
  }

  The story as told by \citeauthor{Freud:1960wx} is comical, but the \requ{0} identified is fairly general.
  In many cases one may only accept a story if the details add up, and imbalance would lead to rejection.
\end{note}


\subsubsection{The agent (plausibly) has \zS{}}
\label{cha:zS:sec:question:illu:basic:has}

\paragraph{Already}

\begin{note}
  And, more commonplace examples involve the gradual accumulation of proposition-value pairs.

  \begin{illustration}
    \nagent{16} \emph{said} they're coming to the party, but you know from \nagent{17} that \nagent{16} is coming to the party only if \nagent{18} is coming to the party.
    Without further information, reasoning about whether \nagent{18} is coming to the party might prevent you from taking \nagent{16} at the word.
    However, you have already have conformation from \nagent{18} that they are coming to the party.
  \end{illustration}
\end{note}

\begin{note}
  In this case, \requ{}, but concluded.
\end{note}

\begin{note}
  The logic instance is interesting here.
  Only if option of concluding via syntactic derivation.

  Well, in this case, agent collapses both at once.
\end{note}

\paragraph{No option}

\begin{note}
  Failures of clause~\ref{idea:requ:pool} of \iRequ{} seem common.

  Indeed, any agent will have some limitations on the proposition-value pairs they have the option of concluding from some pool of premises.

  For non-idealised agents, simple examples involve lack of ability or information.

  For example, I do not think I have the ability to provide syntactic derivations of arbitrary propositional tautologies, even when limited to small number of propositional atoms, and so there is no corresponding contrast to the key conditional of \autoref{illu:sketch:prop-logic}.%
  \footnote{
    \begin{quote}
      The construction is a proof of \(A\) \emph{only if} \(A\) is true given an arbitrary valuation.
    \end{quote}
  }
  Similarly, a skilled ecologist may have a good understanding of tracking, and so have the option of drawing various conclusions from a set of animal footprints, but without the set of footprints available as premises, lack the option to draw any conclusions.

  The second option holds equally for idealised agents that lack omnipotence, at least.
  However, even granting omnipotence clause~\ref{idea:requ:pool} fails for various instances of \(\pvp{\psi}{v'}{\Psi}\).
  For, recall that \requ{1} are from an agent's point of view.
  And, generally speaking, there is no requirement that \(\pv{\psi}{v'}\) follows from \(\Psi\).
  Hence, a for a non-idealised agent \(\pv{\psi}{v'}{\Psi}\) may be a \requ{} of concluding \(\pv{\phi}{v}\) from \(\Phi\), in part because the agent's present epistemic state is not sound.
  And, so long as an idealised agent's present epistemic state is always sound, the agent will not have the option for any unsound \(\pvp{\psi}{v'}{\Psi}\) instance.
\end{note}

\begin{note}
  The following is a simple \illu{0}.

  \begin{illustration}[Testimony as a layperson]
    \label{illu:testimony-layperson}
    An agent is informed that there are exactly five intermediate logics that have the interpolation property.\nolinebreak
  \footnote{Cf.\ \textcite{Maksimova:1977un}}

    The agent does not have the means to query the proof.

    The agent concludes there are exactly five intermediate logics that have the interpolation property.
  \end{illustration}

  The agent does not have the means to query the proof that there are exactly five intermediate logics that have the interpolation property, hence clause~\ref{idea:requ:pool} fails.
\end{note}

\begin{note}
  Note, in cases such as \autoref{illu:testimony-layperson}, clause~\ref{idea:requ:nPsi-nPhi} of \iRequ{} almost trivially fails.
  For, clause~\ref{idea:requ:nPsi-nPhi} is read with respect to the agent's present epistemic state.
  Therefore, if the agent does not have the option of concluding the theorem is true, the agent is bound to fail to conclude the theorem is true prior to reasoning about whether the statement is testimony.
  Hence, it seems straightforward that failure to conclude has no bearing on whether the agent would conclude the statement is testimony.
\end{note}

\paragraph{No relation}

\begin{note}
  Finally, clause~\ref{idea:requ:nPsi-nPhi} of \iRequ{} may fail to hold.
  In such cases \qzS{} may have a positive answer, because there is no \(\pvp{\psi}{v'}{\Psi}\) such that the agent would not conclude \(\pv{\phi}{v}\) from \(\Phi\) if the agent failed to conclude \(\pv{\psi}{v'}\) from \(\Psi\).

  Generally speaking, \(\pvp{\psi}{v'}{\Psi}\) not being a \requ{} of concluding \(\pv{\phi}{v}\) from \(\Phi\) is this way is fairly uninteresting.
  For, clause~\ref{idea:requ:nPsi-nPhi} will fail to hold when \(\pvp{\psi}{v'}{\Psi}\) and \(\pvp{\phi}{v}{\Phi}\) are unrelated (from the agent's point of view).
  For example, an agent may have the option of concluding \(9^{6} \times 3^{2} = 3^{14}\), but failing to do so would not prevent the agent from concluding their dog would like a walk from the behaviour it is exhibiting.

  Still, we offer a slightly more subtle \illu{0}:
\end{note}

\begin{note}[No entanglement]
  \begin{illustration}[Multiplying Turing Machines]
    \label{illu:turing-add-mult}
    An agent regards following statement as testimony:

    \begin{itemize}
    \item
      If you manage to construct a Turing Machine that performs addition, then you will have no trouble constructing a Turing Machine that performs multiplication.
    \end{itemize}

    The agent then constructs a Turing Machine that performs addition.

    And, given the statement they received as testimony, the agent considers it to be the case that they will have no trouble constructing a Turing Machine that performs multiplication.
  \end{illustration}

  Now, given the testimony, the agent \emph{may} think that they have constructed a Turing Machine that performs addition \emph{only if} they are able to construct a Turing Machine that performs multiplication.
  However, such a conditional does not follow directly from the statement received as testimony.

  There is (without further context) no indication that failure to construct a Turing Machine that performs multiplication would reveal a problem with the Turing Machine that performs addition.
  Strictly, the statement received as testimony only states the agent will have no trouble constructing such a Turing Machine.
  Hence, even if the agent does have trouble constructing such a machine, it is not clear this trouble would indicate a problem with their machine for addition.
  Instead, this may only suggest the content of the statement they received as testimony is not, in fact, true.

  So, while the agent \emph{may} think clause~\ref{idea:requ:nPsi-nPhi} of \iRequ{} is satisfied, the agent may also think clause~\ref{idea:requ:nPsi-nPhi} of \iRequ{} is not satisfied.%
  \footnote{
    \autoref{illu:turing-add-mult} is also interesting with respect to the agent concluding the statement is testimony.

    For, is it the case that constructing a TM that adds and having trouble constructing a TM that multiplies is a \requ{} of concluding the information is testimony?

    I am not sure.

    On one hand, it seems plausible the agent has the option of concluding whether the conjunction holds given their present epistemic state.

    On the other hand, it also seems plausible that the agent may conclude they will not have trouble constructing a TM that multiplies \emph{even if} they were to have trouble to constructing a TM that multiplies on attempting.
    For, the agent may only be interested in concluding the future facing `will' if they have not yet attempted to construct a TM that multiplies.
    %
    % In general, I doubt that \qzS{} has a positive answer in cases of testimony as the constraint a positive answer imposes seems too stringent.
    % However, here I think it is plausible  \qzS{} does have a positive answer.
  }
\end{note}

\begin{note}[Maybe?]
  Would not conclude external world if failed to conclude have hands.

  {
    \color{red}
    Might contrast in an interesting way with something from Wright.
  }
\end{note}

\paragraph{Same time}

\begin{note}[Importance of at same time]
  Propositional logic.
  These premises allow to conclude two things.
  Then, conclusion that \(\phi \land \psi\) is simultaneously a conclusion that \(\phi\) and that \(\psi\).

  Or, apples in a bag.
  Five.
  Well, could do at least four, three, etc.
  Conclude at the same time.
\end{note}

\begin{note}
  For example, counterexample for some formula of propositional logic.
  Constructed a truth table.
  Identified a line.
  If counterexample, then line makes any tautology of propositional logic true.
  And, do not need to appeal to the line being a counterexample to the relevant formula to do so.
  reason from line to any recognised tautology.
  Conclude, tautology would be true.
\end{note}

\paragraph{Caution}

\begin{note}
  So, with \citeauthor{Dretske:1970to}, things work out.
  Wouldn't reason to a different conclusion.
  However, this isn't to say that the agent gets knowledge.
  I'm inclined to say the agent does know, but there's no immediate link between \qzS{} and knowledge.

  Maybe also include Harman here, noting the possibility of checking the car.
\end{note}

\begin{note}
  Here, close to \citeauthor{Dretske:1970to}'s zebra case?
\end{note}

\begin{note}
  \begin{illustration}
    Computer, not turning on.
    Broken.
    Check that it's plugged in.
  \end{illustration}

  Well, this is not about reasoning.
\end{note}

\begin{note}
  Consider a familiar situation from \citeauthor{Harman:1986ux}:

  \begin{quote}
    Mary believes, prior to looking in the cupboard, that if she looks in the cupboard, then she will see a box of Cheerios.
    Mary then looks in the cupboard and does not see a box of Cheerios.
    Hence, Mary abandons her belief about what she would see if she looks in the cupboard.\nolinebreak
    \mbox{}\hfill\mbox{(\citeyear[Cf.][Chs.1\&2]{Harman:1986ux})}
  \end{quote}

  What this a \requ{}?
  No.
  Needed to look in the cupboard!

  However, \requ{} may fail to hold.

  So, book.
  Two problems.
  One is difficult.
  Think about other.
  Clear counterexample.
  No longer book.
  Hence, no longer \requ{}.
\end{note}

\subsubsection[The Tortoise and Achilles]{A dialogue between the Tortoise and Achilles}
\label{cha:zS:sec:question:illu:carroll}

\begin{note}
  In this section, relate \qzS{} to an interpretation of a dialogue between a Tortoise and Achilles, as told by \textcite{Carroll:1895uj}.

  Two things here.
  First, how \qzS{} relates to the puzzle.
  Second, how this is compatible with some other lessons drawn.

  Start with the key points of \textcite{Carroll:1895uj} and our interpretation.
  Provide \illu{0}.
  Relate to \citeauthor{Boghossian:2008vf}'s suggestion regarding rule primativism.
\end{note}

\begin{note}
  \color{red}
  Adding to all the other interpretations of this paper\dots
\end{note}

\begin{note}
  \begin{quote}
    ``Plenty of blank leaves, I see!'' the Tortoise cheerily remarked.
    ``We shall need them \emph{all}!''
    (Achilles shuddered.)
    ``Now write as I dictate:---

    \begin{enumerate}[label=(\emph{\Alph*}), ref=\emph{\Alph*}]
    \item
      \label{AatT:a}
      Things that are equal to the same are equal to each other.
    \item
      \label{AatT:b}
      The two sides of this Triangle are things that are equal to the same.
    \item
      \label{AatT:c}
      If~\ref{AatT:a} and~\ref{AatT:b} are true,~\ref{AatT:z} must be true.
      \setcounter{enumi}{25}
    \item
      \label{AatT:z}
      The two sides of this Triangle are equal to each other.''
    \end{enumerate}

    ``You should call it~\ref{AatT:d}, not~\ref{AatT:z},'' said Achilles.
    ``It comes \emph{next} to the other three.
    If you accept~\ref{AatT:a} and~\ref{AatT:b} and~\ref{AatT:c}, you \emph{must} accept~\ref{AatT:z}.''

    ``And why \emph{must} I?''

    ``Because it follows \emph{logically} from them.
    If~\ref{AatT:a} and~\ref{AatT:b} and~\ref{AatT:c} are true,~\ref{AatT:z} \emph{must} be true.
    You don't dispute \emph{that}, I imagine?''

    ``If~\ref{AatT:a} and~\ref{AatT:b} and~\ref{AatT:c} are true,~\ref{AatT:z} \emph{must} be true,'' the Tortoise thoughtfully repeated.
    ``That's \emph{another} Hypothetical, isn't it?
    And, if I failed to see its truth, I might accept~\ref{AatT:a} and~\ref{AatT:b} and~\ref{AatT:c}, and \emph{still} not accept~\ref{AatT:z}, mightn't I ?''

    \mbox{}\hfill\(\vdots\)\hfill\mbox{}

    ``Then Logic would take you by the throat, and force you to do it!''
    Achilles triumphantly replied.
    ``Logic would tell you 'You ca'n't help yourself.''%
    \mbox{ }\hfill\mbox{(\citeyear[279--280]{Carroll:1895uj})}
  \end{quote}

  The Tortoise has written down three premises,~\ref{AatT:a},~\ref{AatT:b}, and~\ref{AatT:c}.
  Achilles holds that~\ref{AatT:z} follows from~\ref{AatT:a},~\ref{AatT:b}, and~\ref{AatT:c}.
  The Tortoise observes they have the possibility of refraining to accept~\ref{AatT:z} follows from~\ref{AatT:a},~\ref{AatT:b}, and~\ref{AatT:c}.
  And (initially), the Tortoise does not accept~\ref{AatT:z} follows from~\ref{AatT:a},~\ref{AatT:b}, and~\ref{AatT:c}.
  Achilles requests the Tortoise accepts that~\ref{AatT:z} follows from~\ref{AatT:a},~\ref{AatT:b}, and~\ref{AatT:c}, and the Tortoise complies.
  Specifically, the Tortoise grants:

  \begin{quote}
    \begin{enumerate}[label=(\emph{\Alph*}), ref=\emph{\Alph*}]
      \setcounter{enumi}{3}
    \item
      \label{AatT:d}
      If~\ref{AatT:a} and~\ref{AatT:b} and~\ref{AatT:c} are true,~\ref{AatT:z} must be true.%
      \mbox{ }\hfill\mbox{(\citeyear[279]{Carroll:1895uj})}
    \end{enumerate}
  \end{quote}

  But, does not accept~\ref{AatT:z} follows from~\ref{AatT:a},~\ref{AatT:b},~\ref{AatT:c}, and~\ref{AatT:d}.

  General pattern.
  Pool of premises and conclusion.
  The Tortoise does not accept the conclusion follows from the pool of premises.
  Achilles requests the proposition \emph{that} the conclusion follows from the pool of premises is added to the pool of premises.
  The Tortoise does not accept the conclusion follows from the expanded pool of premises.

  Indeed,~\ref{AatT:c} in the passage is granted because the Tortoise (initially) did not accept~\ref{AatT:z} follows from~\ref{AatT:a} and~\ref{AatT:b}.
  (\citeyear[279]{Carroll:1895uj})
\end{note}

\begin{note}
  Problem of priority.
  \begin{quote}
    My paradox \dots turns on the fact that, in a Hypothetical, the \emph{truth} of the Protasis, the \emph{truth} of the Apodosis, and the \emph{validity of the sequence}, are 3 distinct Propositions.

    \mbox{}\hfill\(\vdots\)\hfill\mbox{}

    Suppose I say ``I grant~\ref{AatT:a} and~\ref{AatT:b} and~\ref{AatT:c}, but I do \emph{not} grant that I am thereby \emph{obliged} to grant~\ref{AatT:z}.''
    Surely, my granting~\ref{AatT:z} must \emph{wait} until I have been made to see the validity of this sequence: i.e.\ in order to grant~\ref{AatT:z}, I must grant~\ref{AatT:a},~\ref{AatT:b},~\ref{AatT:c}, and~\ref{AatT:d}! And so on.%
    \mbox{ }\hfill\mbox{(\citeyear[472]{Carroll:1977wl})}
  \end{quote}

  So, granting that \emph{C} or \emph{D} is true does not amount to granting that it's okay to move form the premises to the conclusion.
  What makes it okay to go from premises to conclusion?

  How distinct?
  \citeauthor{Carroll:1977wl} has priority.
  Valid only if implication is true.

  So, why not say both ways?
  Both at the same time.%
  \footnote{
    Applies to any logic for which the deduction theorem holds.
  }

  Well, rule is general.
  Here, get lots of instances of the rule.
  So, at least one way of understanding the puzzle is in terms of the relation between a general rule and particular instances of the rule.%
  \footnote{
    Or, no distinction between premises and rules.
    See, for example,~\textcite{Smiley:1995wk}.
    \begin{quote}
      Any attempt by Carroll to tackle the question of inference was bound to begin in confusion and end in constipation---all those premises piling up, but no motion.\newline
      \mbox{ }\hfill\mbox{(\citeyear[727]{Smiley:1995wk})}
    \end{quote}
  }

  The Tortoise has not (yet) accepted something of a general form, then no persuasion by requesting the Tortoise to accept specific instances of the specific form.

  Failure of Achilles is to provide the Tortoise with motivation to adopt the general rule.

  In this respect, variation on suggestion --- e.g.\ \textcite[21--22,33]{Thomson:2010tt} and~\textcite[573]{Wisdom:1974uc} ---  that the infinite regress \citeauthor{Carroll:1895uj} noted is a `red herring' and the task of Achilles is to clarify to the Tortoise \emph{that} they are under logically necessity to move from~\ref{AatT:a} and~\ref{AatT:b} to~\ref{AatT:z}.
\end{note}

\begin{note}
  However, equally, all these instances of the rule.
  Follow \textcite{Wisdom:1974uc}, go from the particulars.
  But, on the interpretation pressed here, this does nothing for the general problem.

  May think implicit in \citeauthor{Carroll:1895uj}'s paper, what other resource does Achilles have?
\end{note}


\begin{note}
  So, wrong to think that need to cover each individual case in order to get full.
  However, granting this, doubts about particular cases may block adopting rule.

  This is what \qzS{} picks up on.
\end{note}

\begin{note}
  Simple example, testimony.%
  \footnote{
    In keeping with \citeauthor{Carroll:1895uj}'s interest in \emph{modus ponens}, the similar reasoning may be constructed with apparent counterexamples to \emph{modus ponens}.
    For example, consider \textcite{McGee:1985tz}.
  }

  \begin{illustration}[Dodgson's testimony I]
    An agent reasons as follows:
    \begin{enumerate}[label=\arabic*., ref=(\arabic*)]
    \item
      \label{testimony:state}
      Charles Dodgson has testified to me that Lewis Carroll wrote \emph{Alice in Wonderland}.
    \item
      \label{testimony:result}
      I know Lewis Carroll wrote \emph{Alice in Wonderland}.
    \end{enumerate}
    But does not conclude~\ref{testimony:result} from~\ref{testimony:state}.
  \end{illustration}

  Going from~\ref{testimony:state} to~\ref{testimony:result}, general form:

  \begin{enumerate}[label=\(\gamma\)., ref=(\(\gamma\))]
  \item
    \label{testimony:general}
    If someone has testified to me that that \(\phi\) has value \(v\), then I know \(\phi\) has value \(v\).
  \end{enumerate}

  In other words, \emph{if} the agent were to conclude~\ref{testimony:result} from~\ref{testimony:state}, the agent would be committed to proposition of general form, and in turn to concluding \(\phi\) has value \(v\) from received testimony that \(\phi\) has value \(v\).

  This means the agent applies to other instances of received testimony.
  Though, the agent may fail to conclude \(\phi\) has value \(v\) from received testimony that \(\phi\) has value \(v\).

  Now, does the agent consider it the case that there may be some proposition-value pair such that the agent may fail to conclude?

  At a cursory glance,~\ref{testimony:general} seems about as good as conditional detachment, though I think an agent may consider this to be the case.
  And, the agent may even have a particular \(\pvp{\psi}{v'}{\Psi}\) in mind.

  To give a unrealistic but clear example, consider expanding previous \illu{0}.
  The agent recalls Charles Dodgson said more\dots

  \begin{illustration}[Dodgson's testimony II]
    The agent reasons as follows:
    \begin{enumerate}[label=\arabic*\('\)., ref=(\arabic*\('\))]
    \item
      \label{testimony:v:state}
      Charles Dodgson has testified to me that Lewis Carroll wrote \emph{Alice in Wonderland} and I don't know Lewis Carroll wrote \emph{Alice in Wonderland}.
    \end{enumerate}

    Given~\ref{testimony:general},~\ref{testimony:v:result} may be obtained from~\ref{testimony:v:state}.

    \begin{enumerate}[label=\arabic*\('\)., ref=(\arabic*\('\)), resume]
    \item
      \label{testimony:v:result}
      I know that Lewis Carroll wrote \emph{Alice in Wonderland} and I don't know Lewis Carroll wrote \emph{Alice in Wonderland}.
    \end{enumerate}

    Distribution of knowledge over conjunction.

    \begin{enumerate}[label=\arabic*\('\)., ref=(\arabic*\('\)), resume]
    \item
      \label{testimony:v:result:dist}
      I know that Lewis Carroll wrote \emph{Alice in Wonderland} and I know that I don't know Lewis Carroll wrote \emph{Alice in Wonderland}.
    \end{enumerate}

    Take right conjunct, and factivity of knowledge, get~\ref{testimony:v:right:fact}:

    \begin{enumerate}[label=\arabic*\('\)., ref=(\arabic*\('\)), resume]
    \item
      \label{testimony:v:right:fact}
      I don't know Lewis Carroll wrote \emph{Alice in Wonderland}.
    \end{enumerate}

    Combine left conjunct of~\ref{testimony:v:result:dist} with~\ref{testimony:v:right:fact} to obtain~\ref{testimony:v:bad}:

    \begin{enumerate}[label=\arabic*\('\)., ref=(\arabic*\('\)), resume]
    \item
      \label{testimony:v:bad}
      I know that Lewis Carroll wrote \emph{Alice in Wonderland} and I don't know that Lewis Carroll wrote \emph{Alice in Wonderland}.
    \end{enumerate}
    However, as~\ref{testimony:v:bad} is clearly a contradiction, the agent does not conclude~\ref{testimony:v:bad} from~\ref{testimony:v:state}.
  \end{illustration}

  Hence, reject~\ref{testimony:general}, general inference.
  For, if conclude then also this variant reasoning.

  Of course, various ways around this problem.
  But, without solution in hand, still enough to block acceptance of general.%
  \footnote{
    More interesting case is the surprise exam paradox.
    Arguably Same property of coming to know something after testimony.
    (Cf.~\textcite{Chow:1998vw} and~\textcite{Gerbrandy:2007vm})
  }


  (
  So, this can be re-formed with testimony.
  Get a different conclusion.
  It's possible to conclude that this is a bad conditional.

  Now, this is interesting.
  Because, typically, this has been taking to present a problem of infinite regress.
  Note, the formulation of \qzS{} is silent on this.
  We're not interested in whether there's some reasoning for the conditional, which does seem difficult, but whether there's something that could lead to a problem.
  )

  Alternatively, consider \citeauthor{Harman:1986ux} here, where logic might do other things.
\end{note}

% \begin{note}
%   \color{red}
%   Distinction this is pulling on is syntax versus semantics, roughly.
%   Well, I think this is one way of viewing what's going on._
%   The semantic rule of \emph{modus ponens} might be fine, but it's not clear syntax lines up.
%   Still, I don't think it's worth the added time and complexity to make this point.
% \end{note}

\begin{note}
  So, fails to conclude because unclear about what else would follow.

  Key, agent does not withhold because need something positive prior, but because unsure about what else happens if they do conclude.
\end{note}

\begin{note}
  Important, the agent may also conclude.
  Here, we've got a very cautious agent.
  But, \qzS{} is all about the agent's perspective.

  So, may go for testimony.
  Likewise, may conclude keys are lost even if there is some other place.

  So, three things.

  First, alternative interpretation of \citeauthor{Carroll:1895uj}'s paradox.
  Second, seen how \qzS{} may apply to reasoning involving testimony.
  Third, and most important, no regress.
\end{note}

\paragraph{\textcite{Boghossian:2008vf}}

\begin{note}
  Interpretation is different, but related, from \textcite{Boghossian:2008vf}.

  \begin{quote}
    The Carrollian argument, \dots is meant to raise a problem for the \emph{justification} of our rules of inference---How can we justify our belief that Modus Ponens, for example, is a good rule of inference?\newline
    \mbox{ }\hfill\mbox{(\citeyear[493]{Boghossian:2008vf})}
  \end{quote}

  Considered whether, descriptively, the agent may wonder about some problems happening.

  May think related, no worries only if justification.
  However, this is problematic.

  What~\textcite{Boghossian:2008vf} motivates (though does not explicitly endorse) is possibility that any motivation for general rule requires appeal to general rule.

  A kind of primitivism about rule following.
  \begin{quote}
    [W]e would have to take as primitive a \emph{general (often conditional) content serving as the reason for which one believes something}, without this being mediated by inference of any kind.%
    \mbox{ }\hfill\mbox{(\citeyear[500]{Boghossian:2008vf})}
  \end{quote}

  More basic than justification, but also, if primitive then no justification.

  Important is that the interpretation outlined, and the role of \qzS{} is compatible with this primitivism.
  Not saying that the agent needs specific ability.
  However, am holding that worries about bad cases are sufficient.

  In failure cases, not about what is required for the agent to conclude, but why an agent may fail to conclude.
  Two different cases.
  First, agent may have a general worry, in this case, there's no particular \(\pvp{\psi}{v'}{\Psi}\) at issue, and so no particular \(\pvp{\psi}{v'}{\Psi}\) is part of why the agent does not conclude.
  Second, agent may have a specific worry.
  In this case, there is some particular \(\pvp{\psi}{v'}{\Psi}\).
  Lack of support between \(\pv{\psi}{v'}\) and \(\Psi\) is part of why agent does not conclude.

  Parallel in converse cases.
\end{note}

\section{Notes}
\label{cha:zS:sec:notes}

\paragraph*{When}

\begin{note}
  \emph{When} concluding \(\pv{\phi}{v}\) from \(\Phi\) in order to keep things simple.
  A variant of the question may be asked if the agent has (already) concluded \(\pv{\phi}{v}\) from \(\Phi\).
  Here, rather than asking whether the agent would not conclude \(\pv{\phi}{v}\) from \(\Phi\), we may ask whether the agent would revise their conclusion of \(\pv{\phi}{v}\) from \(\Phi\).
\end{note}

\paragraph*{Whether the agent may conclude \(\phi\) has value \(v\), regardless of \(\Phi\)}

\begin{note}
  Not about the proposition-value pair.
  Rather, it is about the concluding.
  At interest is not whether \(\phi\) has value \(v\), but whether it makes sense to conclude \(\pv{\phi}{v}\) from \(\Phi\).
  Of course, if the agent has no information about whether \(\phi\) has value \(v\), then this is also part of the picture, but that is a consequence of the base concern.
\end{note}

\paragraph*{Introduced by \(\pv{\phi}{v}\) from \(\Phi\)}

\begin{note}[Proposition-value-premise pairing introduced by \(\pv{\phi}{v}\) from \(\Phi\)]
  This restriction may seem arbitrary, and to some degree I think it is.
  Ideally, an agent concluding \(\pv{\phi}{v}\) is an instance of \csN{} just in case the agent would not have reasoned to a different conclusion if they were to reason first about any other proposition-value pair.
  However, the advantage of focusing on some proposition-value pair `required' by \(\phi\) having value \(v\) is a significant constraint on the range of proposition-value pairs an agent needs to consider in order to \csN{}.

  In general, it may not be clear which proposition-value pairs may lead an agent to fail to conclude \(\phi\) has value \(v\), but so long the proposition-value pair of interest is given by \(\phi\) having value \(v\), an exhaustive search over all other proposition-value pairs may be avoided.

  Indeed, we will say that an agent has \emph{\support{}} for \(\phi\) having value \(v\) just in case they would not have reasoned otherwise, and reserve \emph{\claiming{}} \support{} for the weaker notion.
\end{note}


\paragraph*{Inductive, abductive, etc.\ reasoning}

\begin{note}
  Narrow, but not too narrow.
\end{note}

\begin{note}
  This doesn't rule out inductive or abductive reasoning.
  Consider standard induction.
  Here, there may be novel information, but this is not available from the agent's present epistemic state, and \qzS{} only concerns the agent's present epistemic state.
  Perhaps the possibility alone would prevent conclusion.
  However, it seems most conclude in recognition of such possibility.
  Instead, what one would need is considerations against uniformity principle.

  Same for any bridge between probabilistic and full.
  Toss a coin \(n\) times, conclude it is fair.
  Possible to toss \(m\) more times, not fair.
  However, \(n\) is sufficient, then no problem.
  It is true that there is something more you could do, but this would require acquiring new information.
\end{note}



\paragraph*{Fragility}

\begin{note}
  Kind of fragility.
  If concludes \(\pv{\phi}{v}\) from \(\Phi\), and does not have \zS{}, then from agent's perspective, possibility of revision.

  Indeed, may break down into two components.

  First, possibility of different conclusion.
  Agent's epistemic state is potentially unstable.

  Second, isolation of potential instability to \(\pv{\phi}{v}\) from \(\Phi\).
\end{note}


\paragraph*{Normative?}

\begin{note}[Just a property]
  There's no kind of normative evaluation here.
  We do not hold any conclusion for which this fails is bad.
  Nor do we hold that any conclusion for which this holds is good.
  Indeed, \zS{} is narrow, far too narrow for a general evaluation.

  Indeed, whether or not \zS{} doesn't tell us anything about the relationship between \(\pv{\phi}{v}\) and \(\Phi\) in general, as relative to an agent's epistemic state.
  May be that there is some \(\pvp{\psi}{v'}{\Psi}\), but only due to some quirk of the agent.
\end{note}

\paragraph*{The agent concluding \(\pv{\psi}{v'}\) from \(\Psi\)}

\begin{note}
  From the perspective of the agent.
  It doesn't matter whether the agent really has the option.
  Indeed, this perspective is important for fragility.
\end{note}



%%% Local Variables:
%%% mode: latex
%%% TeX-master: "master"
%%% End:
