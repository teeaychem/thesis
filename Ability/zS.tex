\chapter{\zSN{2}}
\label{cha:zS}

\begin{note}[Intro, locating]
  Now turn to \zSN{}.

  Our goal motivate a positive resolution to~\issueConstraint{}.
  And, as sketched in~\autoref{cha:outline}, \zSN{} has key role in developing tension.

  Question.
  Positive answer answers why.
  Positive answer only if either concluded or \fc{}.
  Concluded or \fc{} answers why.
  \fc{} answers why only if relation of support answers why.
  Relation of support answers why only if proposition-value-premises pairing answers why.

  But, \fc{}, so proposition-value-premises pairing is not an answer to how, as the agent has not witnessed reasoning.

  Now, it is consistent that positive answer only if agent has concluded, hence has witnessed reasoning, and hence is, in part, an answer to how.

  Figuring out instances where this does not hold will wait until later.


  In contrast to sketch given, being with focus on motivation.

  Motivate \zSN{} via plain language question, similar to initial \qWhy{} and \qHow{} from~\autoref{cha:introduction}.
  Follow similar pattern.
  Just as \qWhy{} and \qHow{} were developed in~\autoref{cha:clarification}, \zSN{}, is developed in a similar way.
  Here, like with \qWhy{}, specifically, two things reduce to the same.
\end{note}

\begin{note}[Map]
  Start with a question.

  Certain kind of support \zSN{}, or \zS{}.

  Understanding of \zS{}.

  Argue negative answer \zS{} if and only if \zetaS{}.
\end{note}

\begin{note}
  In order to establish tension we narrow our attention to when concluding \(\pv{\phi}{v}\) concluding \(\pv{\phi}{v}\) involves the agent establishing a particular property with respect to \(\pv{\phi}{v}\).
  We term the property `\zSN{0}', or `\zS{}' for short.

  Positive resolution only requires existence of cases.
  Hence, existence of cases with this property.
  This will be sufficient.
  Any case of concluding which involves \csVImp{} will also be an instance of concluding.

  As sketched, tension by {\color{red} \dots}.

  For the moment, however, we focus on providing a clear account of \csN{}.
  Tension delayed until \dots
  Indeed, following \csN{}, revise resolutions to ~{\color{red} issue:Main}.
  And, additional building blocks for tension via two types of concluding.
\end{note}

\paragraph{Naming}

\begin{note}[Naming]
  Our choice of the term `\zgb{0}' is metaphorical.
  \zgb{2} is a family of flower plants which, typically, have the appearance of a single stem with no branches.
  If one starts just before the flower and works back down the stem, one will not find a branch which, if taken, would lead to a different flower.
  In comparison, if one starts with an agent's epistemic state prior to the agent concluding \(\pv{\phi}{v}\) from \(\Phi\) and~\autoref{question:zs} has a negative answer with respect to \(\pvp{\phi}{v}{\Phi}\), then one will not find a branch which leads to a different conclusion.

  I have some doubts as to whether or not this metaphor really works, but some term is required.
  `Palm-tree-support', or `Arecaceae-support' would also work.
\end{note}

\begin{note}[The token `\qzS{}']
  As {\color{red} noted}, we {\color{red} will} associate \zS{} with positive answers to~\autoref{question:zs}.
  So, in anticipation of this connexion we will use the token `\qzS{}' to name and refer to \autoref{question:zs}.
  Though, to be clear, \zS{} will concern an agent \emph{after} concluding \(\pv{\phi}{v}\) from \(\Phi\) while \qzS{} concerns the agent \emph{when} concluding \(\pv{\phi}{v}\) from \(\Phi\).
  So, an instance of \zS{} should not be identified with a positive answer to some instance of \qzS{}.%
  \footnote{
    Whether, when concluding \(\pv{\phi}{v}\) from \(\Phi\), it is the case that the following conditional which quantifies over proposition-value-premises pairings:

    \begin{itemize}
    \item
      For all \(\pvp{\psi}{v'}{\Psi}\) [\emph{if}~\ref{question:zs:subjunctive} and~\ref{question:zs:option} hold, \emph{then} \ref{question:zs:may-fail} holds].
    \end{itemize}
    Where `hold(s)' expands to `hold(s) from the agent's perspective'.
  }
\end{note}



\begin{note}
  \qzS{}.
  Then \zS{}.
  Then \zetaS{}.
\end{note}

\section{\zS{}}
\label{cha:zS:sec:question}

\begin{note}
  A question about an agent's epistemic state when concluding \(\pv{\phi}{v}\) from \(\Phi\).
  Goal here is to get why involved.
\end{note}

\subsection{A question}
\label{cha:zS:sec:the-question}

\begin{note}
  We begin with the question.

  \begin{restatable}[\qzS{}]{question}{questionZS}
    \label{question:zs}
    For an agent \vAgent{}, when concluding \(\pv{\phi}{v}\) from \(\Phi\), is it the case that:

    \begin{itemize}
    \item
      From \vAgent{}' perspective:
      \begin{itemize}
      \item
        For any proposition-value-premises pairing \(\pvp{\psi}{v'}{\Psi}\):
        \begin{itemize}
        \item[\emph{If}]
          \begin{enumerate}[label=\alph*., ref=(\alph*)]
          \item
            \label{question:zs:option}
            \vAgent{} has the option of concluding \(\pv{\psi}{v'}\) from \(\Psi\), given the agent's reasoning from \(\Phi\) to \(\pv{\phi}{v}\).
          \end{enumerate}
        \item[\emph{and}]
          \begin{enumerate}[label=\alph*., ref=(\alph*), resume]
          \item
            \label{question:zs:subjunctive}
            \vAgent{} would not conclude \(\pv{\phi}{v}\) from \(\Phi\), if \vAgent{} were to attempt and fail to conclude \(\pv{\psi}{v'}\) from \(\Psi\).%
          \end{enumerate}
        \item[\emph{then}]
          \begin{enumerate}[label=\alph*., ref=(\alph*), resume]
          \item
            \label{question:zs:may-fail}
            \vAgent{} would conclude \(\pv{\psi}{v'}\) from \(\Psi\), if \vAgent{} were to attempt to conclude \(\pv{\psi}{v'}\) from \(\Psi\).
          \end{enumerate}
        \end{itemize}
      \end{itemize}
    \end{itemize}
    \vspace{-\baselineskip}
  \end{restatable}
\end{note}

\paragraph{Cases}

\begin{note}
  \autoref{question:zs} concerns an agent when concluding \(\pv{\phi}{v}\) from \(\Phi\), but just prior to having concluding \(\pv{\phi}{v}\) from \(\Phi\).
  And, in short, asks whether there is some proposition-value-premises pairing \(\pvp{\psi}{v'}{\Psi}\) such that the agent, from their perspective:
  \begin{itemize}
  \item
    Has the option of concluding \(\pv{\psi}{v'}\) from \(\Psi\).
  \item
    Would not conclude \(\pv{\phi}{v}\) from \(\Phi\) if they were to attempt and fail to conclude \(\pv{\psi}{v'}\) from \(\Psi\).
  \item
    Might fail to conclude \(\pv{\psi}{v'}\) from \(\Psi\).
  \end{itemize}
  If there is some \(\pvp{\psi}{v'}{\Psi}\), then negative answer to \autoref{question:zs}.
  If there is no such \(\pvp{\psi}{v'}{\Psi}\), the positive answer.

  Note, in particular, clauses~\ref{question:zs:option},~\ref{question:zs:subjunctive}, and~\ref{question:zs:may-fail} from a quantified conditional, so there are three distinct ways in which~\ref{question:zs} may receive a positive answer.
  \begin{itemize}
  \item
    For any proposition-value-premises pairing \(\pvp{\psi}{v'}{\Psi}\), either:
    \begin{enumerate}[label=\alph*\('\).]
    \item
      It is not the case that the agent has the option of concluding \(\pv{\psi}{v'}\) from \(\Psi\), given the agent's reasoning from \(\Phi\) to \(\pv{\phi}{v}\).
    \item
      The agent may conclude \(\pv{\phi}{v}\) from \(\Phi\), (even) if they were to attempt and fail to conclude \(\pv{\psi}{v'}\) from \(\Psi\).
    \item
      The agent would conclude \(\pv{\psi}{v'}\) from \(\Psi\), if they attempted to do so.
    \end{enumerate}
  \end{itemize}

  Third is of interest.
  \cScen{0}.
  In \cScen{1}, there is at least one \(\pvp{\psi}{v'}{\Psi}\)-pairing such that~\ref{question:zs:option} and~\ref{question:zs:subjunctive} hold.

  For the relevant conditionals in \cScen{1}, is it the case, from the agent's perspective, that they would conclude.
\end{note}


\begin{note}
  Not particularly marked.
  Allow agent to have built up a bunch of stuff in the reasoning.

  Example.

  \begin{illustration}[Velocity]
    \label{ill:velocity}
    Agent is provided with information about how far a car has travelled north as a function of time travelled.

    From this, take the derivative of the function to obtain the (instantaneous) velocity of the car at a handful of points in time.

    And, from the (instantaneous) velocity of the car, the agent calculates the (instantaneous) acceleration of the car at each of the points in time.

    The agent also has information about the speed of the car as a function of time travelled, and the agent may calculate speed by the taking magnitude of the (instantaneous) velocity of the car.
  \end{illustration}

  

  \autoref{ill:velocity}, two step calculation.
  Velocity, acceleration.
  After the first step, check by taking the magnitude.
  Calculation of velocity is correct only if taking the magnitude matches speed.

  Just before concluding to include cases such as this.

  Note, \cScen{}.
\end{note}

\begin{note}
  Example highlights how `intermediate conclusions' relate.
  Further point of interest:
  Failure to conclude.

  Two ways to view agent's calculation of the velocity of the car.

  First, as a conclusion.
  Same status as the function.

  Or, as temporary.

  Difference in how we understand agent's present epistemic state.

  On first, the agent's present epistemic state is inconsistent.
  Two proposition-value pairs which conflict.
  Not possible for the car to have velocity the agent calculated and acceleration the agent has been informed of.

  May also be that the function and information about acceleration are inconsistent, but may also be that the agent made a mistake in calculating the velocity of the car.

  On second, the agent's present epistemic state may%
  \footnote{
    Don't have complete perspective on agent's present epistemic state.
  }
  consistent.

  For, made a mistake.
  But, proposition-value pair is not part of present epistemic state, so distinguished from function and information about acceleration, which are consistent.

  This is a distinction we have little interest in.
  What matters is failure to conclude speed.
  Result is either revising inconsistent epistemic state, or abandoning intermediary steps of reasoning.
\end{note}

\begin{note}
  With example, failure to conclude because 


  Broader, failure to conclude because no conclusion.

  `\emph{Unless}'
  {
    \color{red}
    Strong understanding.
    In short, always question regarding anything weaker.
    However, we will argue for this.
  }
\end{note}

\paragraph{Relation to \cScen{1}}

\begin{note}[\cScen{1}]
  So, \cScen{1}.
  Some check on reasoning, \autoref{question:zs} just asks whether, from the agent's perspective, they would satisfy check.

  Two cases.

  \begin{itemize}
  \item
    \autoref{illu:gist:roots}.
    Conclude either \(x = 1\) or \(x = -\sfrac{1}{2}\) from premise that for some \(x \in \mathbb{R}\), \(2x^{2} - x - 1 = 0\).
    \begin{itemize}
    \item
      Is it the case that the agent would conclude either \(x = 1\) or \(x = -\sfrac{1}{2}\) if the agent were to pause just before concluding the same disjunction via the quadratic formula and attempt to conclude the disjunction via factorisation?
    \end{itemize}
  \item
    \autoref{illu:sketch:prop-logic}.
    Soundness and completeness.
    Formal derivation of some sentence \(A\).
    \begin{itemize}
    \item
      Would the agent conclude \(A\) is a (syntactic) theorem if the agent were to pause just before concluding and attempt to conclude \(A\) is a (semantic) theorem?
    \end{itemize}
  % \item
  %   \autoref{illu:sketch:math}.
  %   Conclude \(345 \div 15 = 23\) via understanding of basic arithmetic.
  %   \begin{itemize}
  %   \item
  %     Would the agent conclude \(345 \div 15 = 23\) if the agent were to pause just before concluding and attempt to conclude \(23 \times 15 = 345\)?
  %   \end{itemize}
  \end{itemize}

  \autoref{illu:gist:roots} is interesting, as here different set of premises.
  Contrasts well to velocity.

  \autoref{illu:sketch:prop-logic}, distinct \(\pvp{\psi}{v'}{\Psi}\)-pair.
  
\end{note}

\begin{note}
  \cScen{1} are where positive answer to \autoref{question:zs} involves at least one recognised \(\pvp{\psi}{v'}{\Psi}\).

  Two other possibilities.

  First is trivial case where conditional is true because there are no \(\pvp{\psi}{v'}{\Psi}\)-parings which satisfy~\ref{question:zs:option} and~\ref{question:zs:subjunctive}.

  Second, where the agent does not have any specific \(\pvp{\psi}{v'}{\Psi}\)-pairing in mind.
\end{note}

\paragraph{\emph{Why}}

\begin{note}
  Rough understanding.
  In terms of broader argument, \emph{why}.

  Idea is that agent's perspective regarding \(\pvp{\psi}{v'}{\Psi}\) in part explains why.

  Preferred \illu{0} concerns whether one has or has not lost their keys.
\end{note}

\begin{note}[Motivating \illu{0}]
  {
    \color{red}
    Interest in terms of explaining why and agent did or didn't conclude.
  }

  \begin{illustration}[Lost keys]
    Tempting as it may be to conclude that a pair of keys are lost after some searching, if the keys really are lost then there aren't in a handful of places you haven't yet thought to look. And, until you have concluded that the keys really aren't in those places, and that there is no-where else to look, the keys aren't really lost.
  \end{illustration}

  For the agent's perspective, there is some \(\pvp{\psi}{v'}{\Psi}\), and this explains why the agent does not conclude they have lost their keys.
\end{note}

\begin{note}
  Similar points for examples given.

  Check acceleration, why the agent does not (immediately, at least) conclude 
\end{note}

\paragraph{The components of \qzS{}}

\begin{note}
  The primary clauses of interest are clauses~\ref{question:zs:subjunctive} and \ref{question:zs:option}.

  Intuitively, clause~\ref{question:zs:subjunctive} expresses that concluding \(\pv{\psi}{v'}\) from \(\Psi\) is a check on whether it makes sense for the agent, from their perspective, to conclude \(\pv{\phi}{v}\) from \(\Phi\).
  And,~\ref{question:zs:option} means that, so long as \(\phi\) has value \(v\), the agent has the option of checking whether it makes sense for the agent to conclude \(\pv{\phi}{v}\) from \(\Phi\).
\end{note}

\paragraph{General}

\begin{note}
  Constraints placed on \(\pvp{\psi}{v'}{\Psi}\).
  From reasoning involved in process of concluding \(\pv{\phi}{v}\) from \(\Phi\).
  Would lead to failure.

  Conditionals.

  Involved in concluding \(\pv{\phi}{v}\) from \(\Phi\).
  First, enough to break.
  Second, reasoning makes this proposition-value-premises pairing available.

  Pair of additional features

  Second highlights why \(\pvp{\psi}{v'}{\Psi}\) is of interest.
  However, in this respect, not strictly required.
  Given universal, will also include these.

  First,
  Don't need \(\phi\) to have value \(v\).
  Also, implicit, no revision.
  Built up various things in reasoning, and given all of this\dots


  And, maybe reasoning offers something new.
  Though, not the case that \(\pvp{\psi}{v'}{\Psi}\) only from something new.
  Might be the case that negative answer because go off on wrong reasoning.
\end{note}

\paragraph{Option}

\begin{note}
  Hence, it need not be the case that the agent has the option of concluding \(\pv{\psi}{v'}\) from \(\Psi\) from their epistemic state prior to starting line of reasoning (as the agent has not yet concluded that \(\phi\) has value \(v\)).
\end{note}


\paragraph{Would not conclude}

\begin{note}
  Noted failure.
\end{note}

{
  \color{red}
  Note, \ref{question:zs:may-fail} is delicate.
  For, the combination of \ref{question:zs:subjunctive} and \ref{question:zs:option} suggest there is a way of concluding \(\pv{\psi}{v'}\) from \(\Psi\).
  Hence, \ref{question:zs:may-fail} may be read in reference to this.
  However, \ref{question:zs:may-fail} is intended to allow other ways of concluding \(\pv{\psi}{v'}\) from \(\Psi\).
  What matters is that the agent has not concluded \(\pv{\psi}{v'}\) from \(\Psi\), the agent has the option, and the agent may fail.%
  \footnote{
    This is important for witnessing, but also motivated by different methods.
    A different way to putting this is that concluding is two place relation.
    Between premises and conclusion.
    Concluding is not a three place relation between premises, conclusion, and method.
    I should really have this stated as an assumption.

    Still, there is a variant where method comes into play, as I have this via ability.
  }
}

%%%% TEMP from question
\footnote{
            Clause~\ref{question:zs:subjunctive} is expressed by a subjunctive conditional as there is no requirement that the agent will attempt to conclude \(\pv{\psi}{v'}\) from \(\Psi\).

            \color{red}
            As this alternative expression makes clear,~\autoref{question:zs} focuses on the agent (and their epistemic state).
            At no point do we consider any variation of the agent's epistemic state.
            Likewise,~\autoref{question:zs} concerns only the agent's perspective on concluding \(\pv{\psi}{v'}\) from \(\Psi\).
            Whether or not the agent would conclude \(\pv{\psi}{v'}\) from \(\Psi\) is irrelevant.
            What matters is whether, from the agent's perspective, there is potential for reasoning about whether \(\pv{\psi}{v'}\) follows from \(\Psi\) to block concluding \(\pv{\phi}{v}\) from \(\Phi\).
          }

\paragraph{Abstract}

\begin{note}[Abstract motivation]
  First, the combination of~{\color{red} ???} and~{\color{red} ???} keeps the complexity of resolving~\autoref{question:zs} relatively low.
  We only need to consider what \(\phi\) having value \(v\) would commit the agent to, so to speak.
  For example, we do not need to consider the consequences of the agent reasoning about some arbitrary proposition-value pair \(\pv{\chi}{v''}\) prior to concluding \(\pv{\phi}{v}\) from \(\Phi\).

  Of course, if the agent would conclude both \(\pv{\chi}{v''}\) and \(\pv{\chi}{\overline{v''}}\) for some \(\chi\), then it seems the agent's epistemic state is in bad shape.
  Still, given that an agent will typically revise their epistemic state upon concluding both \(\pv{\chi}{v''}\) and \(\pv{\chi}{\overline{v''}}\) for some \(\chi\), such concerns may be isolated to a distinct question.

  Second, we avoid --- to some extent --- concerns about over-generating.
  Our overall argument will put~\autoref{question:zs} to work in motivating a negative resolution to~{\color{red} issue:Main}.
  Primarily by drawing consequences from what we will argue is an equivalent characterisation of negative resolutions to~\autoref{question:zs}.
  A concern is that this overall argument may over-generate.
  Given that a negative resolution to~{\color{red} issue:Main} seems by no means clear, unintended consequences of our interest in~\autoref{question:zs} may diminish interest in the tension we hope to motivate, and hence tip favour to a positive answer to~{\color{red} issue:Main}.
  Whether or not there are unintended consequences of broadening the scope of~\autoref{question:zs} is unclear to me.
  Still, without need to investigate, we may ignore any such consequences that may arise.
\end{note}

\subsection{A kind of support}

\begin{note}[Answers to \qzS{}]
  For ease of reference, define.
  \begin{definition}[\izS{}]
    \label{idea:zS}
    For an agent \vAgent{}, after concluding \(\pv{\phi}{v}\) from \(\Phi\):
    \begin{itemize}
    \item
      \vAgent{} has \zS{} with respect to \(\pvp{\phi}{v}{\Phi}\).
    \end{itemize}

    \emph{if and only if}

    \begin{itemize}
    \item
      \Autoref{question:zs} had a \emph{negative} answer when \vAgent{} was concluding \(\pv{\phi}{v}\) from \(\Phi\).
    \end{itemize}
  \end{definition}

  Slight problem here with going for after concluding.
  Possible, it seems, for relation of support to no longer hold.
  Likewise for \zS{}.
  For ease, assume agent has not retracted conclusion.
\end{note}

\begin{note}
  Now, two basic propositions follow.
  \begin{proposition}[When a agent has \zS{}]
    Agent and proposition-value-premises pairing.

    \begin{itemize}
    \item
      Agent \emph{has} \zS{} when concluding \(\pv{\phi}{v}\) from \(\Phi\).
    \end{itemize}

    \emph{if and only if}

    \begin{itemize}
    \item
      Either:
      \begin{enumerate}[label=(\alph*), ref=\alph*]
      \item
        There is no proposition-value-premises pairing \(\pvp{\psi}{v'}{\Psi}\) for which the relevant conditions are met.
      \item
        There is some proposition-value-premises pairing \(\pvp{\psi}{v'}{\Psi}\) but, from the agent's perspective, the agent would not fail to conclude \(\pv{\psi}{v'}\) from \(\Psi\).
      \end{enumerate}
    \end{itemize}
  \end{proposition}

  Second, when an agent does not have \zS{}.

  \begin{proposition}[When a agent does not have \zS{}]
    Agent and proposition-value-premises pairing.
    \begin{itemize}
    \item
      Agent \emph{does not} have \zS{} when concluding \(\pv{\phi}{v}\) from \(\Phi\).
    \end{itemize}

    \emph{if and only if}

    \begin{itemize}
    \item
      There is some proposition-value-premises pairing \(\pvp{\psi}{v'}{\Psi}\) such that, from the agent's perspective, the agent may fail to conclude \(\pv{\psi}{v'}\) from \(\Psi\).
    \end{itemize}
  \end{proposition}
\end{note}

\subsection[Visualisation]{Visualisation of what is at issue when asking \qzS{}}

\begin{figure}[h]
  \centering
  \begin{tikzpicture}
    \node (origin) at (0,0) {};
    \node (psiSplit) at (1,0) {};
    \node (phiSplit) at (4,0) {};
    %
    \node[anchor=west] (psiV) at  (6,-1)  {\(\pvp{\psi}{v'}{\Psi}\)};
    \node[anchor=west] (psiNv) at (6,-2) {\(\pvp{\psi}{\overline{v'}}{\Psi}\)};
    \node[anchor=west] (psiQ) at (6,-3) {\(\pvp{\psi}{?}{\Psi}\)};
    %
    % \node[anchor=west] (psiVPhiV) at (9,-1) {\(\pv{\phi}{v}\)};
    \node[anchor=west] (psiNvPhiU) at (9,-2) {\(\pv{\phi}{\{\overline{v},?\}}\)};
    \node[anchor=west] (psiQPhiU) at (9,-3) {\(\pv{\phi}{\{\overline{v},?\}}\)};
    %
    \node[anchor=west] (phiQ) at (10,1) {\(\pv{\phi}{?}\)};
    \node[anchor=west] (phiNv) at (10,2) {\(\pv{\phi}{\overline{v}}\)};;
    \node[anchor=west] (phiV) at (10,0) {\(\pv{\phi}{v}\)};
    %
    \draw[-]  (origin) -- (phiV);
    %
    \path[-,dashed] (phiSplit) edge [out=0, in=180] (phiNv);
    \path[-,dashed] (phiSplit) edge [out=0, in=180] (phiQ);
    %
    \path[-.] (psiSplit) edge [out=0, in=180] (psiV);
    \path[-, dashed] (psiSplit) edge [out=0, in=180] (psiNv);
    \path[-, dashed] (psiSplit) edge [out=0, in=180] (psiQ);
    %
    \draw[<-,dotted] (psiV) edge [out=0, in=180] (phiV);
    \draw[->, dotted] (psiNv) edge (psiNvPhiU);
    \draw[->, dotted] (psiQ) edge (psiQPhiU);
    \end{tikzpicture}
    \caption{Sketch of when an agent has a negative resolution for~\autoref{question:zs}.}
    \label{fig:csN:illu:overview}
  \end{figure}

\begin{note}[Figure]
  \autoref{fig:csN:illu:overview} provides a rough visualisation of~\autoref{question:zs}

  The flat line captures the agent's reasoning, which concludes with \(\pv{\phi}{v}\).
  In concluding \(\pv{\phi}{v}\) the agent rules out two possibilities with respect to \(\phi\).
  First, that \(\phi\) does not have value \(v\), indicated by \(\pv{\phi}{\overline{v}}\).
  Second, that the agent does not assign any value to \(v\), indicated by \(\pv{\phi}{?}\).
  Prior to concluding \(\pv{\phi}{v}\), the agent's reasoning may have branched to either alternative path, but as the agent has concluded \(\pv{\phi}{v}\), neither path is viable, and hence both paths are represented with a dashed line.

  So far, we have seen only that the agent has concluded \(\pv{\phi}{v}\).

  We now consider some proposition-value-premises pairing \(\pv{\psi}{v'}{\Psi}\) such that if the agent were to fail to conclude \(\pv{\psi}{v'}\) from \(\Phi\), the agent would not conclude \(\pv{\phi}{v}\) from \(\Phi\).

  Intuitively, the dotted arrows from the various combinations of \(\psi\) and \(\{v',\overline{v'},?\}\) read, from top to bottom:
  \begin{itemize}
  \item If \(\phi\) has value \(v\) then the agent may conclude \(\pv{\psi}{v'}\) from \(\Psi\), and:
  \item If the agent concludes \(\psi\) has some value \(\overline{v'}\) from \(\Psi\), then the agent either concludes \(\phi\) has some value other than \(v\), or the agent fails to reach a conclusion regarding \(\phi\) from \(\Phi\).
    Both options are combined via the shorthand \(\pv{\phi}{\{\overline{v},?\}}\).
  \item
    And, likewise if the agent fails to conclude \(\pv{\psi}{v'}\) from \(\Psi\).
  \end{itemize}

  With respect to concluding, observe that prior to ruling out alternative branches with respect to \(\pv{\phi}{\{\overline{v},?\}}\), the agent may have reasoned about whether \(\psi\) has value \(v\).
  And, from the agent's perspective, \(\phi\) has value \(v\) only if \(\psi\) has value \(v'\).
  If \(\psi\) does not have value \(v'\), then either \(\phi\) does not have value \(v\), or the agent's reasoning would not conclude with a value for \(\phi\), indicated by \(\pv{\phi}{\{\overline{v},?\}}\).

  Hence, prior to concluding \(\pv{\phi}{v}\), the agent has concluded \(\pv{\psi}{v'}\).
\end{note}

\begin{note}
  Broadly, then, we may say that an agent has {\color{red} particular kind of conclusion} for \(\pv{\phi}{v}\) just in case when concluding \(\pv{\phi}{v}\) it is not the case that the agent's reasoning would have branched to a different conclusion with respect to \(\phi\).

  However, the visualisation of~\autoref{fig:csN:illu:overview} and this broad statement of {\color{red} positive answer to \qzS{}} are a little too broad.
  For, we are only interested in proposition-value pairs guaranteed by \(\phi\) having value \(v\).
  {\color{red} positive answer to \qzS{}} is not global with respect to all proposition-value pairs that the agent may have reasoned about, but local to those guaranteed by the proposition.
\end{note}

\subsection{Additional \illu{1}}
\label{cha:zS:sec:question:illu}

\begin{note}
  Some \illu{1}.
  Three broad sections.

  \begin{itemize}
  \item
    \autoref{cha:zS:sec:question:illu:basic}: Basic \illu{1}.
  \item
    \autoref{cha:zS:sec:question:illu:carroll}: Involved \illu{1} which links \qzS{} to \citeauthor{Carroll:1895uj}'s tale about the Tortoise and Achilles.
  \end{itemize}
\end{note}

\subsubsection{Basic \illu{1}}
\label{cha:zS:sec:question:illu:basic}

\paragraph{\illu{1} of failing to conclude}

\begin{note}[Simple positive answers]
  Variant on lost keys, where agent considers plausible they may reason from some other premise.
  {
    \color{red}
    \begin{illustration}
      \begin{enumerate}
      \item
        A search for `Measurement Theory' via the LCC `H61 .R593' returned no results.
      \item
        The library does not have a copy of `Measurement Theory'.
      \end{enumerate}
    \end{illustration}

    So, if conclude, then option of concluding the library does not use DDC indexing.
    May fail to conclude.
  }

  A more direct variant is spot the difference without a clear statement of how many differences are in the picture.
  Continue to search.


  \begin{illustration}[Anecdotes]
    {
      \color{red}
      Revise this so that some part of the story doesn't add up.
      Then, in contrast to keys, there's some specific \(\pvp{\psi}{v'}{\Psi}\) that's at issue.
    }
    Likewise, a friend's story regarding an event may be entertaining, but there is a distinction between concluding the event actually happened without checking that the details add up, and concluding the event actually happened after checking that the details add up.
  \end{illustration}

  Subtle, theory, observation, possible to account for observation.
  Prior to concluding theory is sound, possible.
  And, if not, would not conclude the theory is sound.
\end{note}


\paragraph{\illu{1} of concluding}

\begin{note}
  \color{red}
  So, might think that this is just a negative condition.
  But, I don't think this is right.

  Two ways of success.
  First, no proposition-value-premises pairing.
  Second, no worries about some possible proposition-value-premises pairing.
\end{note}

\begin{note}[Success]
  Simple examples of a negative answer to~\ref{question:zs}.

  \begin{illustration}
    Fermat's last theorem is true, I am told, and I do not have the means to query the proof.

    Concluding \(\phi\) has value \(v\) from the testimony of experts as a layperson.
  \end{illustration}

  For, as a layperson one has no way of querying whether \(\phi\) has value \(v\).
  Hence, \(\phi\) having value \(v\) does not introduce the possibility of reasoning about some other proposition-value pair and concluding \(\phi\) does not have value \(v\).

  Other examples involve unique sources of information.
  I conclude from the position of the hands on my watch that it is midday.
  The sky is cloudy, and without a second time piece I have no hope of reaching a different conclusion.
\end{note}

\begin{note}
  And, more commonplace examples involve the gradual accumulation of proposition-value pairs.

  \begin{illustration}
    \nagent{16} \emph{said} they're coming to the party, but you know from \nagent{17} that \nagent{16} is coming to the party only if \nagent{18} is coming to the party. Without further information, reasoning about whether \nagent{18} is coming to the party might prevent you from taking \nagent{16} at the word. However, you have already have conformation from \nagent{18} that they are coming to the party.
  \end{illustration}
\end{note}

\begin{note}
  Intuitively, if an agent has the option to conclude \(\pv{\phi}{v}\) from \(\Phi\) and there is some \(\pvp{\psi}{v'}{\Psi}\) as described by~\autoref{question:zs}, then check.
\end{note}

\subparagraph{Caution}

\begin{note}
  So, with \citeauthor{Dretske:1970to}, things work out.
  Wouldn't reason to a different conclusion.
  However, this isn't to say that the agent gets knowledge.
  I'm inclined to say the agent does know, but there's no immediate link between \qzS{} and knowledge.

  Maybe also include Harman here, noting the possibility of checking the car.
\end{note}

\begin{note}
  Here, close to \citeauthor{Dretske:1970to}'s zebra case?
\end{note}

\begin{note}
  \begin{illustration}
    Computer, not turning on.
    Broken.
    Check that it's plugged in.
  \end{illustration}

  Well, this is not about reasoning.
\end{note}

\subsubsection[The Tortoise and Achilles]{A dialogue between the Tortoise and Achilles}
\label{cha:zS:sec:question:illu:carroll}

\begin{note}
  In this section, relate \qzS{} to an interpretation of a dialogue between a Tortoise and Achilles, as told by \textcite{Carroll:1895uj}.

  Two things here.
  First, how \qzS{} relates to the puzzle.
  Second, how this is compatible with some other lessons drawn.

  Start with the key points of \textcite{Carroll:1895uj} and our interpretation.
  Provide \illu{0}.
  Relate to \citeauthor{Boghossian:2008vf}'s suggestion regarding rule primativism.
\end{note}

\begin{note}
  \color{red}
  Adding to all the other interpretations of this paper\dots
\end{note}

\begin{note}
  \begin{quote}
    ``Plenty of blank leaves, I see!'' the Tortoise cheerily remarked.
    ``We shall need them \emph{all}!''
    (Achilles shuddered.)
    ``Now write as I dictate:---

    \begin{enumerate}[label=(\emph{\Alph*}), ref=\emph{\Alph*}]
    \item
      \label{AatT:a}
      Things that are equal to the same are equal to each other.
    \item
      \label{AatT:b}
      The two sides of this Triangle are things that are equal to the same.
    \item
      \label{AatT:c}
      If~\ref{AatT:a} and~\ref{AatT:b} are true,~\ref{AatT:z} must be true.
      \setcounter{enumi}{25}
    \item
      \label{AatT:z}
      The two sides of this Triangle are equal to each other.''
    \end{enumerate}

    ``You should call it~\ref{AatT:d}, not~\ref{AatT:z},'' said Achilles.
    ``It comes \emph{next} to the other three.
    If you accept~\ref{AatT:a} and~\ref{AatT:b} and~\ref{AatT:c}, you \emph{must} accept~\ref{AatT:z}.''

    ``And why \emph{must} I?''

    ``Because it follows \emph{logically} from them.
    If~\ref{AatT:a} and~\ref{AatT:b} and~\ref{AatT:c} are true,~\ref{AatT:z} \emph{must} be true.
    You don't dispute \emph{that}, I imagine?''

    ``If~\ref{AatT:a} and~\ref{AatT:b} and~\ref{AatT:c} are true,~\ref{AatT:z} \emph{must} be true,'' the Tortoise thoughtfully repeated.
    ``That's \emph{another} Hypothetical, isn't it?
    And, if I failed to see its truth, I might accept~\ref{AatT:a} and~\ref{AatT:b} and~\ref{AatT:c}, and \emph{still} not accept~\ref{AatT:z}, mightn't I ?''

    \mbox{}\hfill\(\vdots\)\hfill\mbox{}

    ``Then Logic would take you by the throat, and force you to do it!''
    Achilles triumphantly replied.
    ``Logic would tell you 'You ca'n't help yourself.''%
    \mbox{ }\hfill\mbox{(\Citeyear[279--280]{Carroll:1895uj})}
  \end{quote}

  The Tortoise has written down three premises,~\ref{AatT:a},~\ref{AatT:b}, and~\ref{AatT:c}.
  Achilles holds that~\ref{AatT:z} follows from~\ref{AatT:a},~\ref{AatT:b}, and~\ref{AatT:c}.
  The Tortoise observes they have the possibility of refraining to accept~\ref{AatT:z} follows from~\ref{AatT:a},~\ref{AatT:b}, and~\ref{AatT:c}.
  And (initially), the Tortoise does not accept~\ref{AatT:z} follows from~\ref{AatT:a},~\ref{AatT:b}, and~\ref{AatT:c}.
  Achilles requests the Tortoise accepts that~\ref{AatT:z} follows from~\ref{AatT:a},~\ref{AatT:b}, and~\ref{AatT:c}, and the Tortoise complies.
  Specifically, the Tortoise grants:

  \begin{quote}
    \begin{enumerate}[label=(\emph{\Alph*}), ref=\emph{\Alph*}]
      \setcounter{enumi}{3}
    \item
      \label{AatT:d}
      If~\ref{AatT:a} and~\ref{AatT:b} and~\ref{AatT:c} are true,~\ref{AatT:z} must be true.%
      \mbox{ }\hfill\mbox{(\Citeyear[279]{Carroll:1895uj})}
    \end{enumerate}
  \end{quote}

  But, does not accept~\ref{AatT:z} follows from~\ref{AatT:a},~\ref{AatT:b},~\ref{AatT:c}, and~\ref{AatT:d}.

  General pattern.
  Pool of premises and conclusion.
  The Tortoise does not accept the conclusion follows from the pool of premises.
  Achilles requests the proposition \emph{that} the conclusion follows from the pool of premises is added to the pool of premises.
  The Tortoise does not accept the conclusion follows from the expanded pool of premises.

  Indeed,~\ref{AatT:c} in the passage is granted because the Tortoise (initially) did not accept~\ref{AatT:z} follows from~\ref{AatT:a} and~\ref{AatT:b}.
  (\citeyear[279]{Carroll:1895uj})
\end{note}

\begin{note}
  Problem of priority.
  \begin{quote}
    My paradox \dots turns on the fact that, in a Hypothetical, the \emph{truth} of the Protasis, the \emph{truth} of the Apodosis, and the \emph{validity of the sequence}, are 3 distinct Propositions.

    \mbox{}\hfill\(\vdots\)\hfill\mbox{}

    Suppose I say ``I grant~\ref{AatT:a} and~\ref{AatT:b} and~\ref{AatT:c}, but I do \emph{not} grant that I am thereby \emph{obliged} to grant~\ref{AatT:z}.''
    Surely, my granting~\ref{AatT:z} must \emph{wait} until I have been made to see the validity of this sequence: i.e.\ in order to grant~\ref{AatT:z}, I must grant~\ref{AatT:a},~\ref{AatT:b},~\ref{AatT:c}, and~\ref{AatT:d}! And so on.%
    \mbox{ }\hfill\mbox{(\citeyear[472]{Carroll:1977wl})}
  \end{quote}

  So, granting that \emph{C} or \emph{D} is true does not amount to granting that it's okay to move form the premises to the conclusion.
  What makes it okay to go from premises to conclusion?

  How distinct?
  \citeauthor{Carroll:1977wl} has priority.
  Valid only if implication is true.

  So, why not say both ways?
  Both at the same time.%
  \footnote{
    Applies to any logic for which the deduction theorem holds.
  }

  Well, rule is general.
  Here, get lots of instances of the rule.
  So, at least one way of understanding the puzzle is in terms of the relation between a general rule and particular instances of the rule.%
  \footnote{
    Or, no distinction between premises and rules.
    See, for example,~\textcite{Smiley:1995wk}.
    \begin{quote}
      Any attempt by Carroll to tackle the question of inference was bound to begin in confusion and end in constipation---all those premises piling up, but no motion.\newline
      \mbox{ }\hfill\mbox{(\Citeyear[727]{Smiley:1995wk})}
    \end{quote}
  }

  The Tortoise has not (yet) accepted something of a general form, then no persuasion by requesting the Tortoise to accept specific instances of the specific form.

  Failure of Achilles is to provide the Tortoise with motivation to adopt the general rule.

  In this respect, variation on suggestion --- e.g.\ \textcite[21--22,33]{Thomson:2010tt} and~\textcite[573]{Wisdom:1974uc} ---  that the infinite regress \citeauthor{Carroll:1895uj} noted is a `red herring' and the task of Achilles is to clarify to the Tortoise \emph{that} they are under logically necessity to move from~\ref{AatT:a} and~\ref{AatT:b} to~\ref{AatT:z}.
\end{note}

\begin{note}
  However, equally, all these instances of the rule.
  Follow \textcite{Wisdom:1974uc}, go from the particulars.
  But, on the interpretation pressed here, this does nothing for the general problem.

  May think implicit in \citeauthor{Carroll:1895uj}'s paper, what other resource does Achilles have?
\end{note}


\begin{note}
  So, wrong to think that need to cover each individual case in order to get full.
  However, granting this, doubts about particular cases may block adopting rule.

  This is what \qzS{} picks up on.
\end{note}

\begin{note}
  Simple example, testimony.%
  \footnote{
    In keeping with \citeauthor{Carroll:1895uj}'s interest in \emph{modus ponens}, the similar reasoning may be constructed with apparent counterexamples to \emph{modus ponens}.
    For example, consider \textcite{McGee:1985tz}.
  }

  \begin{illustration}[Dodgson's testimony I]
    An agent reasons as follows:
    \begin{enumerate}[label=\arabic*., ref=(\arabic*)]
    \item
      \label{testimony:state}
      Charles Dodgson has testified to me that Lewis Carroll wrote \emph{Alice in Wonderland}.
    \item
      \label{testimony:result}
      I know Lewis Carroll wrote \emph{Alice in Wonderland}.
    \end{enumerate}
    But does not conclude~\ref{testimony:result} from~\ref{testimony:state}.
  \end{illustration}

  Going from~\ref{testimony:state} to~\ref{testimony:result}, general form:

  \begin{enumerate}[label=\(\gamma\)., ref=(\(\gamma\))]
  \item
    \label{testimony:general}
    If someone has testified to me that that \(\phi\) has value \(v\), then I know \(\phi\) has value \(v\).
  \end{enumerate}

  In other words, \emph{if} the agent were to conclude~\ref{testimony:result} from~\ref{testimony:state}, the agent would be committed to proposition of general form, and in turn to concluding \(\phi\) has value \(v\) from received testimony that \(\phi\) has value \(v\).

  This means the agent applies to other instances of received testimony.
  Though, the agent may fail to conclude \(\phi\) has value \(v\) from received testimony that \(\phi\) has value \(v\).

  Now, does the agent consider it the case that there may be some proposition-value pair such that the agent may fail to conclude?

  At a cursory glance,~\ref{testimony:general} seems about as good as conditional detachment, though I think an agent may consider this to be the case.
  And, the agent may even have a particular \(\pvp{\psi}{v'}{\Psi}\) in mind.

  To give a unrealistic but clear example, consider expanding previous \illu{0}.
  The agent recalls Charles Dodgson said more\dots

  \begin{illustration}[Dodgson's testimony II]
    The agent reasons as follows:
    \begin{enumerate}[label=\arabic*\('\)., ref=(\arabic*\('\))]
    \item
      \label{testimony:v:state}
      Charles Dodgson has testified to me that Lewis Carroll wrote \emph{Alice in Wonderland} and I don't know Lewis Carroll wrote \emph{Alice in Wonderland}.
    \end{enumerate}

    Given~\ref{testimony:general},~\ref{testimony:v:result} may be obtained from~\ref{testimony:v:state}.

    \begin{enumerate}[label=\arabic*\('\)., ref=(\arabic*\('\)), resume]
    \item
      \label{testimony:v:result}
      I know that Lewis Carroll wrote \emph{Alice in Wonderland} and I don't know Lewis Carroll wrote \emph{Alice in Wonderland}.
    \end{enumerate}

    Distribution of knowledge over conjunction.

    \begin{enumerate}[label=\arabic*\('\)., ref=(\arabic*\('\)), resume]
    \item
      \label{testimony:v:result:dist}
      I know that Lewis Carroll wrote \emph{Alice in Wonderland} and I know that I don't know Lewis Carroll wrote \emph{Alice in Wonderland}.
    \end{enumerate}

    Take right conjunct, and factivity of knowledge, get~\ref{testimony:v:right:fact}:

    \begin{enumerate}[label=\arabic*\('\)., ref=(\arabic*\('\)), resume]
    \item
      \label{testimony:v:right:fact}
      I don't know Lewis Carroll wrote \emph{Alice in Wonderland}.
    \end{enumerate}

    Combine left conjunct of~\ref{testimony:v:result:dist} with~\ref{testimony:v:right:fact} to obtain~\ref{testimony:v:bad}:

    \begin{enumerate}[label=\arabic*\('\)., ref=(\arabic*\('\)), resume]
    \item
      \label{testimony:v:bad}
      I know that Lewis Carroll wrote \emph{Alice in Wonderland} and I don't know that Lewis Carroll wrote \emph{Alice in Wonderland}.
    \end{enumerate}
    However, as~\ref{testimony:v:bad} is clearly a contradiction, the agent does not conclude~\ref{testimony:v:bad} from~\ref{testimony:v:state}.
  \end{illustration}

  Hence, reject~\ref{testimony:general}, general inference.
  For, if conclude then also this variant reasoning.

  Of course, various ways around this problem.
  But, without solution in hand, still enough to block acceptance of general.%
  \footnote{
    More interesting case is the surprise exam paradox.
    Arguably Same property of coming to know something after testimony.
    (Cf.~\textcite{Chow:1998vw} and~\textcite{Gerbrandy:2007vm})
  }


  (
  So, this can be re-formed with testimony.
  Get a different conclusion.
  It's possible to conclude that this is a bad conditional.

  Now, this is interesting.
  Because, typically, this has been taking to present a problem of infinite regress.
  Note, the formulation of \qzS{} is silent on this.
  We're not interested in whether there's some reasoning for the conditional, which does seem difficult, but whether there's something that could lead to a problem.
  )

  Alternatively, consider \citeauthor{Harman:1986ux} here, where logic might do other things.
\end{note}

% \begin{note}
%   \color{red}
%   Distinction this is pulling on is syntax versus semantics, roughly.
%   Well, I think this is one way of viewing what's going on._
%   The semantic rule of \emph{modus ponens} might be fine, but it's not clear syntax lines up.
%   Still, I don't think it's worth the added time and complexity to make this point.
% \end{note}

\begin{note}
  So, fails to conclude because unclear about what else would follow.

  Key, agent does not withhold because need something positive prior, but because unsure about what else happens if they do conclude.
\end{note}

\begin{note}
  Important, the agent may also conclude.
  Here, we've got a very cautious agent.
  But, \qzS{} is all about the agent's perspective.

  So, may go for testimony.
  Likewise, may conclude keys are lost even if there is some other place.

  So, three things.

  First, alternative interpretation of \citeauthor{Carroll:1895uj}'s paradox.
  Second, seen how \qzS{} may apply to reasoning involving testimony.
  Third, and most important, no regress.
\end{note}

\paragraph{\textcite{Boghossian:2008vf}}

\begin{note}
  Interpretation is different, but related, from \textcite{Boghossian:2008vf}.

  \begin{quote}
    The Carrollian argument, \dots is meant to raise a problem for the \emph{justification} of our rules of inference---How can we justify our belief that Modus Ponens, for example, is a good rule of inference?\newline
    \mbox{ }\hfill\mbox{(\citeyear[493]{Boghossian:2008vf})}
  \end{quote}

  Considered whether, descriptively, the agent may wonder about some problems happening.

  May think related, no worries only if justification.
  However, this is problematic.

  What~\textcite{Boghossian:2008vf} motivates (though does not explicitly endorse) is possibility that any motivation for general rule requires appeal to general rule.

  A kind of primitivism about rule following.
  \begin{quote}
    [W]e would have to take as primitive a \emph{general (often conditional) content serving as the reason for which one believes something}, without this being mediated by inference of any kind.%
    \mbox{ }\hfill\mbox{(\citeyear[500]{Boghossian:2008vf})}
  \end{quote}

  More basic than justification, but also, if primitive then no justification.

  Important is that the interpretation outlined, and the role of \qzS{} is compatible with this primitivism.
  Not saying that the agent needs specific ability.
  However, am holding that worries about bad cases are sufficient.

  In failure cases, not about what is required for the agent to conclude, but why an agent may fail to conclude.
  Two different cases.
  First, agent may have a general worry, in this case, there's no particular \(\pvp{\psi}{v'}{\Psi}\) at issue, and so no particular \(\pvp{\psi}{v'}{\Psi}\) is part of why the agent does not conclude.
  Second, agent may have a specific worry.
  In this case, there is some particular \(\pvp{\psi}{v'}{\Psi}\).
  Lack of support between \(\pv{\psi}{v'}\) and \(\Psi\) is part of why agent does not conclude.

  Parallel in converse cases.
\end{note}

\section{Refining \zSN{2}: \zetaS{}}

\begin{note}
  We opened this section with a question---\qzS{}.
  We have expanded, in part, on what this question amounts to, and have seen a handful of \illu{1}.

  To ease discussion going forward, we reformulate negative and positive answers to~\qzS{} in terms of whether an agent satisfies a certain property.

  In longform we term this property `\zSN{2}', though with few exceptions we will use the term `\zSN{0}'.

  We also take this opportunity to refine our understanding of the relevant \(\pvp{\psi}{v'}{\Psi}\) proposition-value-premises pairings from~\qzS{}.
\end{note}

\subsection{\requ{3}}

\begin{note}
  We begin by refining the relevant \(\pvp{\psi}{v'}{\Psi}\) proposition-value-premises pairings of interest from~\qzS{}.
  We term such proposition-value-premises pairings `\requ{1}' of concluding \(\pv{\phi}{v}\) from \(\Phi\).
\end{note}

\begin{note}[Notion of a \requ{}]
  \begin{notion}[\requ{3}]
    \label{notion:overview:requ}
    \(\pvp{\phi}{v'}{\Psi}\) is a \requ{} of concluding \(\pv{\phi}{v}\) from \(\Phi\), with respect to an agent \vAgent{}'s epistemic state if:
    \begin{enumerate}
    \item
      \label{notion:overview:requ:main}
      From the perspective of \vAgent{}' epistemic state, \(\phi\) has value \(v\) only if:
      \begin{enumerate}[label=\alph*., ref=\named{R:\alph*}]
      \item
        \label{notion:overview:requ:pool}
        \vAgent{} has the option of concluding \(\pv{\psi}{v'}\) from \(\Psi\) where:
        \begin{enumerate}[label=\roman*., ref=\named{R:a.\roman*}, series=csIdeaCounter]
        \item
          \label{notion:overview:requ:pool:int}
          \vAgent{} may conclude \(\pv{\psi}{v'}\) from \(\Psi\) without concluding \(\pv{\phi}{v}\) from \(\Phi\) as an intermediary step.
        \item
          \label{notion:overview:requ:pool:ind}
          For any proposition-value pair \(\pv{\psi_{i}}{v_{i}}\) in \(\Psi\), \vAgent{} either has concluded or may conclude \(\pv{\psi_{i}}{v_{i}}\) without concluding \(\pv{\phi}{v}\) from \(\Phi\).
        \end{enumerate}
      \item
        \label{notion:overview:requ:nPsi-nPhi}
        If \vAgent{} were to fail to conclude \(\pv{\psi}{v'}\) from \(\Psi\) prior to reasoning about whether \(\phi\) has value \(v\) given \(\Phi\), \vAgent{} would not conclude \(\pv{\phi}{v}\) from \(\Phi\).
      \end{enumerate}
    \end{enumerate}
    \vspace{-\baselineskip}
  \end{notion}

  With the key clause linking~\autoref{notion:overview:requ} to~\qzS{} is clause~\ref{notion:overview:requ:nPsi-nPhi}.
  For, clause~\ref{notion:overview:requ:nPsi-nPhi} captures the core idea of failure to conclude \(\pv{\psi}{v'}\) from \(\Psi\) leading to failure to conclude \(\pv{\phi}{v}\) from \(\Phi\).

  The role of clause~\ref{notion:overview:requ:pool} is explicitly state various properties \(\pv{\psi}{v'}{\Psi}\) must have in order for any failure to conclude \(\pv{\psi}{v'}\) from \(\Psi\) is relevant to concluding \(\pv{\phi}{v}\) from \(\Phi\).%
  \footnote{
    Indeed, we take \ref{notion:overview:requ:pool:int} and~\ref{notion:overview:requ:pool:ind} to be more-or-less implicit constraints on \(\pvp{\psi}{v'}{\Psi}\) in the statement of \qzS{}.
  }
  In particular, \ref{notion:overview:requ:pool:int} and~\ref{notion:overview:requ:pool:ind} are required to ensure the agent may conclude \(\pv{\psi}{v'}\) from \(\Psi\) independently of concluding \(\pv{\phi}{v}\) from \(\Phi\).

    For, if \ref{notion:overview:requ:pool:int} and \ref{notion:overview:requ:pool:ind} were to fail to hold then:
  \begin{itemize}
  \item
    By~\ref{notion:overview:requ:pool:int}, the agent would need to conclude \(\pv{\phi}{v}\) from \(\Phi\) as a sub-conclusion when reasoning from the relevant pool of premises \(\Psi\).
    Hence, it would not be possible to conclude \(\pv{\psi}{v'}\) from \(\Psi\) without first concluding \(\pv{\phi}{v}\) from \(\Phi\).
  \item
    And, likewise, by~\ref{notion:overview:requ:pool:ind}, the agent need to have already concluded \(\pv{\phi}{v}\) from \(\Phi\) in order to appeal to some of the proposition-value pairs in the relevant pool of premises \(\Psi\).
  \end{itemize}

  Conversely, if both~\ref{notion:overview:requ:pool:int} and~\ref{notion:overview:requ:pool:ind} hold, the agent may conclude \(\pv{\psi}{v'}\) from \(\Psi\) independently of concluding \(\pv{\phi}{v}\) from \(\Phi\).

  Note, however, neither~\ref{notion:overview:requ:pool:int} nor~\ref{notion:overview:requ:pool:ind} rule out the possibility of the agent concluding \(\pv{\phi}{v}\) from \(\Phi\) when concluding \(\pv{\psi}{v'}\) from \(\Psi\) or, conversely, concluding \(\pv{\psi}{v'}\) from \(\Psi\) when concluding \(\pv{\phi}{v'}\) from \(\Phi\).
  There may be an interesting variant of the notion of a \requ{} with such a constraint in place, but such a constraint is not of interest with respect to \qzS{}.
  For, at issue is only whether the agent may at interest is only failure to conclude \(\pv{\psi}{v'}\) from \(\Psi\), and both~\ref{notion:overview:requ:pool:int} and~\ref{notion:overview:requ:pool:ind} ensure lack of concluding \(\pv{\phi}{v}\) from \(\Phi\) will not prevent the agent from reaching a conclusion regarding whether \(\psi\) has value \(v\) given \(\Psi\).
\end{note}

\begin{note}
  \color{red}
  Has the option.
\end{note}

\begin{note}[\requ{2}: Partial check]
  Intuitively, concluding \(\pv{\psi}{v'}\) from \(\Psi\) would serve as a partial check on whether the agent may reason to a conclusion other than \(\pv{\phi}{v}\), captured by~\ref{notion:overview:requ:nPsi-nPhi}.

  Concluding \(\pv{\psi}{v'}\) from \(\Psi\) is a check.
  For, if the agent were to fail to conclude \(\pv{\psi}{v'}\) from \(\Psi\) then, from the perspective of the agent's epistemic state, the agent would not conclude \(\pv{\phi}{v}\) from \(\Phi\).
  Hence, contraposing, the agent would conclude \(\pv{\phi}{v}\) from \(\Phi\) only if the agent would conclude \(\pv{\psi}{v'}\) from \(\Psi\).
  However, the check is partial, as it need not be the case that the agent would conclude \(\pv{\psi}{v'}\) from \(\Psi\) only if the agent \(\pv{\phi}{v}\) from \(\Phi\).
  Therefore, failing to conclude \(\pv{\psi}{v'}\) from \(\Psi\) may block concluding \(\pv{\phi}{v}\) (from the perspective of the agent's epistemic state) though concluding \(\pv{\psi}{v'}\) from \(\Psi\) need not ensure that the agent would conclude \(\pv{\phi}{v}\).

  Now, \ref{notion:overview:requ:nPsi-nPhi} contains a slight subtlety.
  For, from~\autoref{assu:conc:d-free}, an agent may conclude various proposition-value pairs from some instance of reasoning without explicit recognition.
  Therefore,~\ref{notion:overview:requ:nPsi-nPhi} does not state that the agent may fail to conclude \(\pv{\phi}{v}\) from \(\Phi\).
  Rather, \ref{notion:overview:requ:nPsi-nPhi} holds that from the perspective of the agent's epistemic state, the agent may fail to conclude \(\pv{\phi}{v}\) from \(\Phi\).
  Again, we tread a fine line between the role of an agent's epistemic state and the role of the agent's \stance{}.
  The role of an agent's epistemic state determines whether \(\pv{\psi}{v'}\) is a \requ{} of concluding \(\pv{\phi}{v}\) from \(\Phi\).
  And, an agent's epistemic state may determine whether an agent concludes \(\pv{\chi}{v''}\) when concluding \(\pv{\phi}{v}\).%
  \footnote{
    Recall the above discussion of \(\pv{\phi}{v}\) \indicatePr{} \(\pv{\chi}{v'}\) in relation to~\autoref{assu:conc:d-free}.
  }
  Therefore, the agent's epistemic state --- the agent's perspective on how things are --- is key.
  However, the agent's \stance{} is unimportant.
  Whether an agent has concluded \(\pv{\phi}{v}\), or whether \(\pv{\psi}{v'}\) is a \requ{} is not a question of whether the agent recognises the have concluded \(\pv{\phi}{v}\) or recognises \(\pv{\psi}{v'}\) is a \requ{}.

  Combining these two ideas, intuitively, \(\pv{\psi}{v'}\) is a \requ{} of concluding \(\pv{\phi}{v}\) just in case there is some pool of premises \(\Psi\) such that determining whether the agent would conclude \(\pv{\psi}{v'}\) is an independent partial check on whether the agent may reason to a conclusion other than \(\pv{\phi}{v}\).
\end{note}

\subsection{\zetaS{}}
\label{sec:zs2}

\begin{note}
  With the notion of a \requ{} in hand, we now state when an agent has \zSN{0} for some proposition-value pair:

  \begin{idea}[\izetaS{}]
    \label{idea:zetaS}
    An agent \vAgent{} has \emph{\izetaS{}} for a proposition-value pair \(\pv{\phi}{v}\) when concluding \(\pv{\phi}{v}\) from some pool of premises \(\Phi\) just in case:
    \begin{itemize}
    \item When concluding \(\pv{\phi}{v}\) from \(\Phi\):
      \begin{enumerate}[label=\arabic*., ref=\named{CS:\arabic*}]
      \item
        \label{idea:zetaS::requ}
        For any proposition-value-premises pairing \(\pvp{\psi}{v'}{\Psi}\) which is a \requ{} of concluding \(\pv{\phi}{v}\) from \(\Phi\) either:
        \begin{enumerate}[label=\alph*., ref=\named{CS:1.\alph*}]
        \item
          \label{idea:zetaS::requ-sat:Past}
          \vAgent{} has concluded \(\pv{\psi}{v'}\) from \(\Psi\).
        \item
          \label{idea:zetaS::requ-sat:Pres}
          In concluding \(\pv{\phi}{v}\), \vAgent{} simultaneously concludes \(\pv{\psi}{v'}\) from \(\Psi\).
        \item
          \label{idea:zetaS::requ-sat:Forgone}
          \(\pvp{\psi}{v'}{\Psi}\) is a \fc{0}.
        \end{enumerate}
      \end{enumerate}
    \end{itemize}
    \vspace{-\baselineskip}
  \end{idea}
\end{note}

\begin{note}
  \begin{proposition}[Equivalence between \zS{} and \zetaS{}]
    \label{prop:qzs-tick-equals-iCS}
    For an agent \vAgent{}, the following are equivalent:
    \begin{enumerate}[label=\arabic*., ref=(\arabic*)]
    \item
      \label{prop:qzs-tick-equals-iCS:qzS}
      \vAgent{} has \zS{} for \(\pv{\phi}{v}\) after concluding \(\pv{\phi}{v}\) from \(\Phi\).
      (\qzS{} had a negative resolution when concluding \(\pv{\phi}{v}\) from \(\Phi\))
    \item
      \label{prop:qzs-tick-equals-iCS:ZS}
      \vAgent{} has \zetaS{} for \(\pv{\phi}{v}\) after concluding \(\pv{\phi}{v}\) from \(\Phi\).
    \end{enumerate}
    \vspace{-\baselineskip}
  \end{proposition}
  In other words, we hold that an agent ruling out failure to conclude \(\pv{\phi}{v}\) from \(\Phi\) for any \requ{} \(\pvp{\psi}{v'}{\Psi}\) is \emph{equivalent}%
  \footnote{
    In the context of a negative resolution to \qzS{}.
  }
  to the agent either
  \begin{enumerate*}[label=(\alph*)]
  \item
    having had concluded \(\pv{\psi}{v'}\) from \(\Psi\), or
  \item
    the agent simultaneously concluding \(\pv{\psi}{v'}\) from \(\Psi\) when concluding \(\pv{\phi}{v}\) from \(\Phi\).
  \item
    \(\pvp{\psi}{v'}{\Psi}\) being a \fc{0}.
  \end{enumerate*}
\end{note}

\paragraph*{Observations}

\paragraph*{Contraposition}

\begin{note}[Contraposition]
  An argument for~\autoref{prop:qzs-tick-equals-iCS} is important from the perspective of the overall argument of this document.

  Suppose we have~\autoref{prop:qzs-tick-equals-iCS}.
  Then, view \izetaS{} as a clarification of \qzS{}.

  Specifically, the left-to-right direction.

  Only negative resolution if \izetaS{}.
  I.e.\ only if concluded, for any \requ{}.

  \begin{itemize}
  \item
    If an agent has concluded \(\pv{\phi}{v}\) from \(\Phi\) \emph{without} having concluded \(\pv{\psi}{v'}\) from some \(\Psi\), where \(\pv{\psi}{v'}\) is a \requ{} of concluding \(\pv{\phi}{v}\) from \(\Phi\), then the agent has not \csVed{} for \(\pv{\phi}{v}\) from \(\Phi\).
  \end{itemize}

  Taking the contraposition, if not \izetaS{}, then no negative resolution.
  If intuitions are unclear, then \izetaS{} offers a way to clarify those intuitions.

  Conversely, fix a negative resolution to \qzS{}, then from left-to-right direction, draw out what must also be the case.%
  \footnote{
    Consider, by analogy, knowledge, and the idea that knowledge is closed under known entailment.
    \begin{quote}
      If \vAgent{} knows both
      \begin{enumerate*}[label=(\roman*)]
      \item \(\phi\), and
      \item \(\phi\) entails \(\psi\),
      \end{enumerate*}
      then \vAgent{} knows \(\psi\).
    \end{quote}
    Observe the same dynamics are present.

    If whether an agent knows both \(\phi\) and \(\phi\) entails \(\psi\) is at issue, then establishing the agent does not know \(\psi\) establishes that either the agent does not know \(\phi\) or the agent does not know \(\psi\).

    Conversely, if an agent knows both \(\phi\) and \(\phi\) entails \(\psi\), then by closure of knowledge under known entailment, the agent must also know \(\psi\).

    Of course, whether knowledge \emph{is} closed under known entailment is unclear, but the same dynamics hold for any similar condition.
    In general, these observations amount to little more than both the closure of knowledge under known entailment and \izetaS{} both having the general form of a conditional \(A \Rightarrow B\), such that:
    \begin{itemize}
    \item
      From \(A \Rightarrow B\) and \(A\), one may infer \(B\), and
    \item
      From \(A \Rightarrow B\) and \emph{not}-\(B\), one may infer \emph{not}-\(A\).
    \end{itemize}
  }

  \begin{itemize}
  \item
    If an agent has \zetaS{} for \(\pv{\phi}{v}\) from \(\Phi\), by concluding \(\pv{\phi}{v}\) from \(\Phi\) then, has concluded \(\pv{\psi}{v'}\) from \(\Psi\), for any \(\pvp{\psi}{v'}{\Psi}\) which is a \requ{} of concluding \(\pv{\phi}{v}\) from \(\Phi\).
  \end{itemize}

  In outline, path to tension.
  Cases in which \qzS{} has negative resolution.
  Draw out as a consequence of~\autoref{prop:qzs-tick-equals-iCS} that the agent has concluded.
  So long as cases in which no witnessing, we will have tension.
  On the one hand, negative resolution, and on the other hand, has not witnessed reasoning.

  Now, if some weaker, that does not require concluding, then lack a way to generate tension.

  So,~\autoref{prop:qzs-tick-equals-iCS} should be treated with some caution.

  Of course, this does not guarantee anything interesting.
  Tension will still depend on such cases.
\end{note}

\begin{note}
  With the importance of~\autoref{prop:qzs-tick-equals-iCS} motivated, we now turn to arguing for~\autoref{prop:qzs-tick-equals-iCS}.

  The argument we provide for~\autoref{prop:qzs-tick-equals-iCS} is somewhat involved, and will go via an intermediary lemma.
  We being by outlining the structure of the argument, before turning to the details.
\end{note}

\paragraph*{Same time}

\begin{note}[Importance of at same time]
  Propositional logic.
  These premises allow to conclude two things.
  Then, conclusion that \(\phi \land \psi\) is simultaneously a conclusion that \(\phi\) and that \(\psi\).

  Or, apples in a bag.
  Five.
  Well, could do at least four, three, etc.
  Conclude at the same time.
\end{note}

\begin{note}
  For example, counterexample for some formula of propositional logic.
  Constructed a truth table.
  Identified a line.
  If counterexample, then line makes any tautology of propositional logic true.
  And, do not need to appeal to the line being a counterexample to the relevant formula to do so.
  reason from line to any recognised tautology.
  Conclude, tautology would be true.
\end{note}

\paragraph*{Inductive, abductive, etc.\ reasoning}

\begin{note}
  Narrow, but not too narrow.
\end{note}

\begin{note}
  This doesn't rule out inductive or abductive reasoning.
  Consider standard induction.
  Here, there may be novel information, but this is not available from the agent's present epistemic state, and \qzS{} only concerns the agent's present epistemic state.
  Perhaps the possibility alone would prevent conclusion.
  However, it seems most conclude in recognition of such possibility.
  Instead, what one would need is considerations against uniformity principle.

  Same for any bridge between probabilistic and full.
  Toss a coin \(n\) times, conclude it is fair.
  Possible to toss \(m\) more times, not fair.
  However, \(n\) is sufficient, then no problem.
  It is true that there is something more you could do, but this would require acquiring new information.
\end{note}

\paragraph*{Fragility}

\begin{note}
  Kind of fragility.
  If concludes \(\pv{\phi}{v}\) from \(\Phi\), and does not have \zS{}, then from agent's perspective, possibility of revision.

  Indeed, may break down into two components.

  First, possibility of different conclusion.
  Agent's epistemic state is potentially unstable.

  Second, isolation of potential instability to \(\pv{\phi}{v}\) from \(\Phi\).
\end{note}

\paragraph*{Whether the agent may conclude \(\phi\) has value \(v\), regardless of \(\Phi\)}

\begin{note}
  Not about the proposition-value pair.
  Rather, it is about the concluding.
  At interest is not whether \(\phi\) has value \(v\), but whether it makes sense to conclude \(\pv{\phi}{v}\) from \(\Phi\).
  Of course, if the agent has no information about whether \(\phi\) has value \(v\), then this is also part of the picture, but that is a consequence of the base concern.
\end{note}

\paragraph*{Normative?}

\begin{note}[Just a property]
  There's no kind of normative evaluation here.
  We do not hold any conclusion for which this fails is bad.
  Nor do we hold that any conclusion for which this holds is good.
  Indeed, \zS{} is narrow, far too narrow for a general evaluation.

  Indeed, whether or not \zS{} doesn't tell us anything about the relationship between \(\pv{\phi}{v}\) and \(\Phi\) in general, as relative to an agent's epistemic state.
  May be that there is some \(\pvp{\psi}{v'}{\Psi}\), but only due to some quirk of the agent.
\end{note}

\paragraph*{The agent concluding \(\pv{\psi}{v'}\) from \(\Psi\)}

\begin{note}
  From the perspective of the agent.
  It doesn't matter whether the agent really has the option.
  Indeed, this perspective is important for fragility.
\end{note}

\subparagraph*{Almost-transitive}

\begin{note}
  Key idea that may be obvious is that \csN{} is almost-transitive.
  If it needs to be the case that I'd conclude \(\pv{\psi}{v'}\) from \(\Psi\) to get \(\pv{\phi}{v}\) from \(\Phi\), and \(\pv{\chi}{v''}\) from \(X\) to get \(\pv{\psi}{v'}\) from \(\Psi\) then, need to get \(\pv{\chi}{v''}\) from \(X\) to get \(\Phi\), and \(\pv{\chi}{v''}\).

  However, this is not quite right.
  For, it may be the case that the agent has the option of getting to \(\pv{\psi}{v'}\) from \(\Psi\) from the different branch.
\end{note}

\begin{note}
  Only about the option of concluding.
  There are various other properties.
  In this respect, \qzS{} is narrow.
\end{note}

\begin{note}
  Most reasoning is short, and composes.
  Further, it needs to be the case that novel proposition-value pair introduces this concern.
  So, some extraneous proposition.
  This is irrelevant.
\end{note}

\begin{note}[Recursion]
   {
    \color{red}
    Here, it is important that we don't go fully recursive.
    For, we're only interested in \requ{1} arising from concluding \(\pv{\phi}{v}\) from \(\Phi\).
    This means that it is no immediate.
    In particular, it may be the case that some of \(\Phi\) remove what would otherwise be a \requ{} of some \requ{}.
    But, then, this is still a \requ{} with respect to the current epistemic state.

    So, I think I actually get the result that this is recursive.
    However, with a slight change.
    For, \ref{idea:zetaS::requ-sat:Past} looks to the past, and instead of \csVImp{} in the past, it only need to be the case that the agent has \csVed{} from present epistemic state.

    The point is, that there may be some pruning without concluding.
    For, something fails to be a \requ{}.
    Likewise, it may be the case that the tree grows, as the agent's epistemic state develops.

    So, we don't get a clear recursive clause.

    This leads to an observation.
    Reasoning about a \requ{} may lead to a revision of the agent's epistemic state.

    This is a somewhat interesting consequence.
    We began with the idea of failing to conclude \(\pv{\phi}{v}\) from \(\Phi\).
    This said nothing about revision.
    However, we see that this raises the possibility of revision.

    From a different perspective, this should come as no surprise.
    If the agent concluded without \csN{}, then this is going to remain a problem for any further conclusions, unless the agent figures out they were mistaken regarding the proposition-value-premises pairing being a \requ{}.
  }
\end{note}

\paragraph*{Introduced by \(\pv{\phi}{v}\) from \(\Phi\)}

\begin{note}[Proposition-value-premise pairing introduced by \(\pv{\phi}{v}\) from \(\Phi\)]
  This restriction may seem arbitrary, and to some degree I think it is.
  Ideally, an agent concluding \(\pv{\phi}{v}\) is an instance of \csN{} just in case the agent would not have reasoned to a different conclusion if they were to reason first about any other proposition-value pair.
  However, the advantage of focusing on some proposition-value pair `required' by \(\phi\) having value \(v\) is a significant constraint on the range of proposition-value pairs an agent needs to consider in order to \csN{}.

  In general, it may not be clear which proposition-value pairs may lead an agent to fail to conclude \(\phi\) has value \(v\), but so long the proposition-value pair of interest is given by \(\phi\) having value \(v\), an exhaustive search over all other proposition-value pairs may be avoided.

  Indeed, we will say that an agent has \emph{\support{}} for \(\phi\) having value \(v\) just in case they would not have reasoned otherwise, and reserve \emph{\claiming{}} \support{} for the weaker notion.
\end{note}

\paragraph*{When}

\begin{note}
  \emph{When} concluding \(\pv{\phi}{v}\) from \(\Phi\) in order to keep things simple.
  A variant of the question may be asked if the agent has (already) concluded \(\pv{\phi}{v}\) from \(\Phi\).
  Here, rather than asking whether the agent would not conclude \(\pv{\phi}{v}\) from \(\Phi\), we may ask whether the agent would revise their conclusion of \(\pv{\phi}{v}\) from \(\Phi\).
\end{note}

\subsection{Equivalence between \qzS{} and \izetaS{}}
%\subsection{An equivalent statement of \zS{}, \zetaS{}}
\label{sec:overview:an-equiv-stat-of-zs}

\begin{note}
  We began this section with the statement of a question ---~\autoref{question:zs}, or \qzS{}  --- concerning whether something is the case when an agent concludes some proposition-value pair \(\pv{\phi}{v}\) from some pool of premises \(\Phi\).
  We then reformulated the question in terms of a property, \zS{}.

  The terminology of \zS{}, or \zSN{0}, is certainly artificial.
  Still, I take the question, and the content of \zS{} to be fairly intuitive.

  Roughly, at issue is whether there is some proposition-value-premises pairing which would prevent an agent from concluding some proposition-value pair from some pool of premises.

  Some technical concepts have been appealed to in order to clarify the question --- in particular with respect to our assumptions regarding concluding, and our focus on a fixed epistemic state --- but the question itself is fairly natural.

  Consider again the \illu{1} regarding a lost pair of keys.
  There does seem to be a difference between an exhaustive consideration of all the places the keys may be before concluding the keys are lost on the one hand, and concluding the keys are lost while expecting some place on hasn't looked will come to mind on the other.
  Or, concluding that something happened based on a friends story by passive acceptance on the one hand, and concluding the thing happened after checking for consistency on the other.

  So, we take \qzS{} to be an intuitive question, and \zS{} to be an intuitive property.
  Non-standard, perhaps, but still intuitive.
\end{note}

\begin{note}
  Our attention now turns to providing an equivalent statement of \zS{}, or in other words necessary and sufficient conditions for negative resolution to \qzS{}.%
  \footnote{
    Indirectly, necessary and sufficient conditions for a positive answer, by negating either side.
  }
  We term this equivalent statement of \zS{}, `\izetaS{}', or `\zetaS{}' for short.

  \zetaS{} will have a key role in our main argument for a negative resolution to~{\color{red} issue:Main}.

  Given the prominent role \zetaS{} will have in our main argument, we will take some additional care in stating \zetaS{}.
  In particular, we will provide an separate characterisation of the relevant \(\pvp{\psi}{v'}{\Psi}\) proposition-value-premises pairings of interest.
  These we will term `\requ{1}' of concluding \(\pv{\phi}{v}\) from \(\Phi\).
  And we will include additional discussion of some subtleties regarding \requ{1} given our assumption regarding concluding from~\autoref{chapter:concluding}.

  The following account of \zetaS{} will then be relatively straightforward.
  We hold that an agent has \zetaS{} for some proposition-value pair \(\pv{\phi}{v}\) with respect to some pool of premises \(\Phi\) just in case the agent has concluded \(\pv{\psi}{v'}\) from \(\Psi\) for any \requ{} \(\pvp{\psi}{v'}{\Psi}\) of concluding \(\pv{\phi}{v}\) from \(\Phi\).

  As noted in the introduction, this leads to a closure condition.
  If an agent has \zetaS{} for some proposition-value pair \(\pv{\phi}{v}\) with respect to some pool of premises \(\Phi\), then it will follow that an agent has concluded \(\pv{\psi}{v'}\) from \(\Psi\) for any \requ{} \(\pvp{\psi}{v'}{\Psi}\) of concluding \(\pv{\phi}{v}\) from \(\Phi\).
  Before arguing that \zetaS{} is equivalent to \zS{}, we will walk through this observation in some detail.
\end{note}

\begin{note}
  Still, it is important to note the (proposed) link between \qzS{}, \zS{}, and \zetaS{}.

  \zetaS{} has the potential to be a strong condition, \emph{if} we show that \zetaS{} applies to some instance of concluding \(\pv{\phi}{v}\) from \(\Phi\) where the agent has not witnessed reasoning from \(\Psi\) to \(\pv{\psi}{v'}\) for some \requ{} \(\pvp{\psi}{v'}{\Psi}\).%
  \footnote{
    I take it this, by itself, is not immediate.
  }
  However, even if \zetaS{} turn out to be a strong condition, it remains motivated by, and --- granting the arguments to follow --- equivalent to a fairly intuitive question.
\end{note}

\begin{note}[Intuition]
  This may seem a trivial point, but I think it is important to keep in mind.
  We have introduced \zS{} via \qzS{}, and it may be easy to grant us authority over what \zS{} is, or what \qzS{} asks.
  However, the interrogative core of \qzS{} is stated in neutral terms.
  So long as you have some intuitive understanding of `concluding', then, granting sufficient information about an agent's epistemic state, you are in position to determine whether \qzS{} has a negative or positive answer, and hence whether and agent has \zS{} for the relevant proposition-value-premises pairing.

  Our argument for the equivalence between \zS{} and \zetaS{} will not further specify the content of \qzS{}.
  Instead, \zetaS{} will provide an alternative characterisation of \zS{} from which we will draw further consequences.
  In short, it will be up to you to evaluate whether the \zetaS{} really is an alternative characterisation of \zS{}.
\end{note}

\paragraph*{The argument for \autoref{prop:qzs-tick-equals-iCS}}

\begin{note}
  We now turn to providing an argument for~\autoref{prop:qzs-tick-equals-iCS}.
  To do so, we argue for an equivalent proposition focusing on \qzS{} and \zetaS{}:

  \begin{proposition}
    \label{prop:qzs-tick-equals-iCS:var}
    For any property \(\chi\):
    \begin{enumerate}[label=\Alph*.]
    \item
      \label{squish:A}
      \squishA{An}{the}
    \end{enumerate}
    \emph{if and only if}:
    \begin{enumerate}[label=\Alph*.,resume]
    \item
      \label{squish:B}
      \squishB{the}.
    \end{enumerate}
    \vspace{-\baselineskip}
  \end{proposition}

  \Autoref{prop:qzs-tick-equals-iCS:var} states that an agent satisfying \izetaS{} is a necessary condition of the agent satisfying an property \(\chi\) which is sufficient to ensure a negative resolution to~\autoref{question:zs}.

  The purpose of arguing for~\autoref{prop:qzs-tick-equals-iCS} via~\autoref{prop:qzs-tick-equals-iCS:var} is option to contrast some property \(\chi\) with \zetaS{}.

  Specifically, if we contrapose the left-to-right direction we have some property \(\chi\) which does not entail satisfaction of \izetaS{}.
  And, we will argue that any such property \(\chi\) is insufficient for a negative resolution to \qzS{}.

  Indeed, the left-to-right direction is the only direction of significant interest with respect to~\autoref{prop:qzs-tick-equals-iCS:var}.
  The right-to-left direction requires little effort.

  This is not to say the argument for the left-to-right direction will be watertight.
  Rather, we will focus on localised tension.
  Still, we are not ready to provide the argument for~\autoref{prop:qzs-tick-equals-iCS:var} just yet.

  First we need to note the equivalence between~\autoref{prop:qzs-tick-equals-iCS} and~\autoref{prop:qzs-tick-equals-iCS:var}.

  Then, with the equivalence in hand, we may turn to arguing for~\autoref{prop:qzs-tick-equals-iCS:var}.
\end{note}

\paragraph*{The equivalence between~\autoref{prop:qzs-tick-equals-iCS} and~\autoref{prop:qzs-tick-equals-iCS:var}}

\begin{note}
  \begin{proposition}
    Equivalence between~\autoref{prop:qzs-tick-equals-iCS} and~\autoref{prop:qzs-tick-equals-iCS:var}.
  \end{proposition}
\end{note}

\begin{note}
  To see the equivalence between~\autoref{prop:qzs-tick-equals-iCS} and~\autoref{prop:qzs-tick-equals-iCS:var} observe that ~\autoref{prop:qzs-tick-equals-iCS} and~\autoref{prop:qzs-tick-equals-iCS:var} entail one another.

  For, assume~\autoref{prop:qzs-tick-equals-iCS} holds.
  Then, an agent satisfying \izetaS{} is both necessary and sufficient for a positive resolution to~\qzS{}.

  Now, take some property \(\chi\) and assume~\ref{squish:A} holds.
  Given~\ref{squish:A} holds, \(\chi\) is sufficient to positively resolve~\qzS{}.
  And, by~\autoref{prop:qzs-tick-equals-iCS}, a positive resolution to \qzS{} entail the agent satisfies \izetaS{}.
  Therefore, whenever the agent satisfies \(\chi\), the agent also satisfies \izetaS{}.
  So,~\ref{squish:B} holds.

  Conversely, take some property \(\chi\) and assume~\ref{squish:B} holds.
  Then, if the agent satisfies \(\chi\), the agent also satisfies \izetaS{}.
  Given~\autoref{prop:qzs-tick-equals-iCS}, satisfying \izetaS{} is sufficient to resolve~\autoref{question:zs}.
  So,~\ref{squish:A} holds.

  Now assume~\autoref{prop:qzs-tick-equals-iCS:var} holds.

  First, suppose the agent has resolved \qzS{}.
  Then, the agent satisfies some property \(\chi\), sufficient to resolve \qzS{}.
  Therefore, by~\autoref{prop:qzs-tick-equals-iCS:var}, we have that the agent also satisfies \izetaS{}.
  Hence, satisfying \izetaS{} is a necessary condition of resolving \qzS{}.

  Second, suppose the agent satisfies \izetaS{}.
  It is immediate that satisfaction of \izetaS{} entails satisfaction of \izetaS{}.
  Therefore, satisfaction of \izetaS{} is sufficient for a positive resolution to~\qzS{}.

  This tells us \izetaS{} is necessary.
  And, as an immediate consequence, \izetaS{} is sufficient.
  For, holds for all sufficient conditions.
  Though, the most straightforward argument for the right-to-left direction involves establishing this directly.
\end{note}

% \begin{note}
%   \footnote{
%     Symbolically, represent this as follows.

%     First, shorthand for \(\pvp{\psi}{v'}{\Psi}\) being a \requ{} of \(\pvp{\phi}{v}{\Phi}\):
%     \begin{itemize}
%     \item \(\pvp{\psi}{v'}{\Psi} \rleadsto \pvp{\phi}{v}{\Phi}\)
%     \end{itemize}
%     Now, quantify:
%     \begin{itemize}
%     \item \(\forall \pvp{\psi}{v'}{\Psi}\colon (\pvp{\psi}{v'}{\Psi} \rleadsto \pvp{\phi}{v}{\Phi})\)
%     \end{itemize}
%     Concludes
%     \begin{itemize}
%     \item \(\mathcal{C}(\pv{\psi}{v'},\psi)\)
%     \end{itemize}
%     Statement that the agent concludes or has concluded \(\pv{\psi}{v'}\) from \(\Psi\) for all \(\pvp{\psi}{v'}{\Psi}\) such that \(\pvp{\psi}{v'}{\Psi}\) is a \requ{} of \(\pvp{\phi}{v}{\Phi}\).
%     \begin{itemize}
%     \item \(\forall \pvp{\psi}{v'}{\Psi}\colon (\pvp{\psi}{v'}{\Psi} \rleadsto \pvp{\phi}{v}{\Phi}) \rightarrow \mathcal{C}(\pv{\psi}{v'},\Psi)\)
%     \end{itemize}
%     Let this be \(\mathcal{c}\).

%     Now, \csN{}.
%     \begin{itemize}
%     \item \(\mathsf{CS}(\pvp{\phi}{v}{\Phi})\)
%     \end{itemize}
%     And, sufficient to resolve question:
%     \begin{itemize}
%     \item \(R_{S}(\chi,?\mathsf{CS}(\pvp{\phi}{v}{\Phi}))\)
%     \end{itemize}
%     So, getting \(\chi\) is sufficient to resolve the question of whether the agent has \csVed{} for \(\pv{\phi}{v}\) from \(\Phi\).

%     First, we get
%     \begin{itemize}
%     \item \(R_{S}(\mathcal{c},?\mathsf{CS}(\pvp{\phi}{v}{\Phi}))\)
%     \end{itemize}
%     Concluding is sufficient.
%     From this, anything that entails \(\mathcal{c}\) is sufficient.

%     \begin{itemize}
%     \item \(\forall \chi((\chi \rightarrow c) \rightarrow R_{S}(\chi,?\mathsf{CS}(\pvp{\phi}{v}{\Phi})))\)
%     \end{itemize}

%     Conversely, if some \(\chi\) is sufficient, \(\chi\) entails \(\mathcal{c}\)

%     \begin{itemize}
%     \item \(\forall \chi(R_{S}(\chi,?\mathsf{CS}(\pvp{\phi}{v}{\Phi})) \rightarrow (\chi \rightarrow c))\)
%     \end{itemize}

%     From the two above:
%     \begin{itemize}
%     \item \(\forall \chi(R(\chi,?\mathsf{CS}(\pvp{\phi}{v}{\Phi})) \leftrightarrow (\chi \rightarrow c))\)
%     \end{itemize}
%   }
%   \end{note}

\paragraph*{The argument for~\autoref{prop:qzs-tick-equals-iCS:var}}

\begin{note}
  Given the equivalence between~\autoref{prop:qzs-tick-equals-iCS} and~\autoref{prop:qzs-tick-equals-iCS:var}, we now turn to arguing for~\autoref{prop:qzs-tick-equals-iCS:var}.

  As suggested above, we split the argument into two parts.
  The right-to-left direction and the left-to-right direction.

  We begin with the right-to-left direction as it is mostly straightforward.
  We then turn to the left-to-right direction, which is significantly more involved.
\end{note}

\paragraph*{Right-to-left}

\begin{note}
  For the right-to-left direction our goal is to establish:

  For any property \(\chi\):
  \begin{quote}
  \begin{enumerate}
    \item[B.]
      \squishB{an}
    \end{enumerate}
    \emph{implies}
    \begin{enumerate}
    \item[A.]
      \squishA{The}{the}.
    \end{enumerate}
  \end{quote}
\end{note}

\begin{note}
  Fix an agent, take some property \(\chi\), and assume \squishB{the}.

  Now, \(\chi\) entails the agent satisfies \izetaS{}, but \(\chi\) may also be \zetaS{}.
  Indeed, \zetaS{} is the minimal property for which satisfaction of \(\chi\) entails \zetaS{}.
  So, though \(\chi\) is arbitrary, we must show that satisfaction of \zetaS{} alone is sufficient to resolve whether the agent has \zS{} for \(\pv{\phi}{v}\) when concluding \(\pv{\phi}{v}\) from \(\Phi\).

  Still, this is relatively straightforward.
  \qzS{} asks whether there is some \requ{} \(\pvp{\psi}{v'}{\Psi}\) such that the agent has not settled whether \(\pv{\psi}{v'}\) follows from \(\Psi\).

  Now, for any such \(\pvp{\psi}{v'}{\Psi}\), satisfying \izetaS{} requires one of the following to conditions holds:
  \begin{itemize}
  \item
    The agent has concluded \(\pv{\psi}{v'}\) from \(\Psi\).
  \item
    When concluding \(\pv{\phi}{v}\) from \(\Phi\) the agent also concludes \(\pv{\psi}{v'}\) from \(\Psi\).
  \end{itemize}
  In both cases, when concluding \(\pv{\psi}{v'}\) from \(\Psi\) for any such \(\pvp{\psi}{v'}{\Psi}\) the agent will have concluded \(\pv{\psi}{v'}\) from \(\Psi\).
  And, given a conclusion of \(\pv{\psi}{v'}\) from \(\Psi\), it seems clear the agent has settled whether \(\pv{\psi}{v'}\) follows from \(\Psi\).
  For, the agent has concluded \(\pv{\psi}{v'}\) from \(\Psi\)!
  Indeed, at question is whether the agent would conclude \(\pv{\psi}{v'}\) from \(\Psi\), and so there is no clearer answer to this question than concluding \(\pv{\psi}{v'}\) from \(\Psi\).

  So, satisfying clauses of \izetaS{} is sufficient for a negative resolve~\qzS{}.
  For, grant that notion of a \requ{0} captures the relevant cases, \izetaS{} requires concluding.
\end{note}

\subsection{Left-to-right}

\begin{note}
  For the left-to-right direction our goal is to establish:

  For any property \(\chi\):
  \begin{quote}
  \begin{enumerate}
    \item[A.]
      \squishA{An}{the}
    \end{enumerate}
    \emph{implies}
    \begin{enumerate}
    \item[B.]
      \squishB{the}.
    \end{enumerate}
  \end{quote}

  To do so, we take some arbitrary property \(\chi\), and then contrapose the conditional.
  In other words, assume \(\chi\) does not entail the agent has satisfied \izetaS{} with the sub-goal of showing that an agent satisfying \(\chi\) is sufficient to resolve whether the agent has \(\zS{}\) for \(\pv{\phi}{v}\) when concluding \(\pv{\phi}{v}\) from \(\Phi\).

  To aid the clarify of the argument we begin by introducing a plausible candidate for \(\chi\).
  We will then use the candidate to develop the core of the argument, and to conclude we will observe how the argument generalises.
\end{note}


\subsubsection{\izetaSm{}, a candidate for \(\chi\)}
\label{overview:sec:iCS-iCSm-limitation-closure}

\begin{note}
  \begin{idea}[\izetaSm{2} --- \izetaSm{}]
    \label{idea:Zsm}
    An agent \vAgent{} has \izetaSm{2} for \(\pv{\phi}{v}\) with respect to some pool of premises \(\Phi\) \emph{only if}:
    \begin{enumerate}[label=\arabic*., ref=\named{\(\zeta^{-}\)S:\arabic*}]
    \item
      \label{idea:Zsm:requ}
      For any proposition-value pair \(\pv{\psi}{v'}\) which is a \requ{} of concluding \(\pv{\phi}{v}\) from \(\Phi\) either:
      \begin{enumerate}[label=\alph*., ref=\named{\(\zeta^{-}\)S:1.\alph*}]
      \item
        \label{idea:Zsm:requ-sat:Past}
        \vAgent{} holds \emph{\vAgent{} would conclude} \(\pv{\psi}{v'}\) from the relevant pool of premises \(\Psi\).
      \item
        \label{idea:Zsm:requ-sat:Pres}
        In concluding \(\pv{\phi}{v}\) \vAgent{} \emph{also} holds \emph{\vAgent{} would conclude} \(\pv{\psi}{v'}\) from the relevant pool of premises \(\Psi\).
      \end{enumerate}
    \end{enumerate}
    \vspace{-\baselineskip}
  \end{idea}
\end{note}

\begin{note}[Difference between \izetaS{} and \izetaSm{}]
  The difference between \izetaS{} and \izetaSm{} is straightforward.

  \izetaS{} requires an agent concludes \(\pv{\psi}{v'}\) from \(\Psi\) for any \requ{} \(\pvp{\psi}{v'}{\Psi}\).
  By contrast, \izetaSm{} requires an agent holds \emph{they would conclude \(\pv{\psi}{v'}\) from \(\Psi\)}.%
  \footnote{
    Expressed differently, the agent concluding \(pv{\psi}{v'}\) from \(\Psi\) is replaced with the agent concluding that they would (conclude \(\pv{\psi}{v'}\) from \(\Psi\)).
    Here, the parentheses indicate that in both~\ref{idea:Zsm:requ-sat:Past} and~\ref{idea:Zsm:requ-sat:Pres} the agent is not required to conclude anything from \(\Psi\) directly.
  }

  The expression of an agent concluding that they would conclude may be somewhat stilted, but expresses a simple idea.
  Rather than concluding \(\pv{\psi}{v'}\) from \(\Psi\), the agent concludes that if they were to reason about whether \(\pv{\psi}{v'}\) follows from \(\Psi\), they would conclude \(\pv{\psi}{v'}\) from \(\Psi\).
  Note, \izetaSm{} does not detail the relevant pool of premises that the agent draw the conclusion from.
\end{note}

\begin{note}[\izetaS{} is (intuitively) stronger than \izetaSm{}]
  In particular, \izetaS{} is intuitively stronger than \izetaSm{}.

  First, observe that~\ref{idea:zetaS::requ-sat:Past} and~\ref{idea:zetaS::requ-sat:Pres} (plausibly) entail ~\ref{idea:Zsm:requ-sat:Past} and~\ref{idea:Zsm:requ-sat:Pres}, respectively.
  For, if an agent has concluded \(\pv{\psi}{v'}\) from \(\Psi\), then \emph{in so doing} the agent has shown that they would conclude \(\pv{\psi}{v'}\) from \(\Psi\).

  Second, observe neither~\ref{idea:Zsm:requ-sat:Past} nor~\ref{idea:Zsm:requ-sat:Pres} (plausibly) entail~\ref{idea:zetaS::requ-sat:Past} nor \ref{idea:zetaS::requ-sat:Pres}, respectively, so long as there are plausible cases in which an agent concludes that they would conclude \(\pv{\phi}{v}\) from \(\Phi\) without concluding \(\pv{\phi}{v}\) from \(\Phi\).

  Indeed, it seems there are plausible cases.

  For, it seems an agent may conclude that they would conclude \(\pv{\phi}{v}\) from \(\Phi\) without concluding \(\pv{\phi}{v}\) from \(\Phi\).
  For example, one may be informed that one would conclude \(\pv{\phi}{v}\) from \(\Phi\) via testimony.
  Hence, the only relevant premises one plausibly requires is that they have been informed they would conclude \(\pv{\phi}{v}\) from \(\Phi\) via testimony, and \(\Phi\) may be arbitrary.
  E.g.\ I tell you that if you looked at the map you would conclude that East Palo Alto is directly north of Palo Alto (shock!) and if you trust the map you may even conclude that East Palo Alto \emph{is} directly north of Palo Alto.
  Still, you do not (obviously) conclude East Palo Alto is directly north of Palo Alto from any premises associated with details of the map.
\end{note}

\begin{note}
  \izetaSm{} as a candidate \(\chi\).
  {
    \color{red}
    Following, arbitrary \(\chi\).
    However, substitute in \izetaSm{} is desired.
  }
\end{note}

\begin{note}
  Now, we have seen how~\ref{idea:Zsm:requ-sat:Past} and~\ref{idea:Zsm:requ-sat:Pres} are (plausibly) \emph{strictly} weaker than~\ref{idea:zetaS::requ-sat:Past} and~\ref{idea:zetaS::requ-sat:Pres}, respectively.
  And, we have noted that both~\ref{idea:Zsm:requ-sat:Past} and~\ref{idea:Zsm:requ-sat:Pres} seem to align with the motivation provided from \csN{}.
  We briefly highlight why the distinction between ~\ref{idea:Zsm:requ-sat:Past} and~\ref{idea:Zsm:requ-sat:Pres} and ~\ref{idea:zetaS::requ-sat:Past} and~\ref{idea:zetaS::requ-sat:Pres}, respectively, will matter for our overall argument.

  Observe, \csN{}%
  \footnote{
    As stated, with~\ref{idea:zetaS::requ-sat:Past} and~\ref{idea:zetaS::requ-sat:Pres} over ~\ref{idea:Zsm:requ-sat:Past} and~\ref{idea:Zsm:requ-sat:Pres}, respectively.
  }
  will lead to tension with a positive resolution to~{\color{red} issue:Main} just in case we manage to find an instance in which an agent \csN{} for \(\pv{\phi}{v}\) from \(\Phi\) without witnessing reasoning from \(\Psi\) to \(\pv{\psi}{v'}\) for some \requ{} \(\pvp{\psi}{v'}{\Psi}\) of concluding \(\pv{\phi}{v}\) from \(\Phi\).
  However, this tension will not follow if the weakened variants of~\ref{idea:zetaS::requ-sat:Past} and~\ref{idea:zetaS::requ-sat:Pres} are adopted.
  For, neither~\ref{idea:Zsm:requ-sat:Past} nor~\ref{idea:Zsm:requ-sat:Pres} would require the agent to conclude \(\pv{\psi}{v'}\) from \(\Psi\).

  Indeed, we will argue for the existence of cases of exactly the kind described.
  Hence, the role of~\ref{idea:zetaS::requ-sat:Past} and~\ref{idea:zetaS::requ-sat:Pres} over~\ref{idea:Zsm:requ-sat:Past} and~\ref{idea:Zsm:requ-sat:Pres} is not merely an issue of motivation, but also crucial to establishing tension.
\end{note}

\paragraph*{Arguing}

\begin{note}
  Fix an agent.
  Take arbitrary \(\chi\), such that \(\chi\) does not imply satisfaction of \izetaS{}.
  (E.g.\ \izetaSm{}, as seen above.)
\end{note}

\begin{note}
  Now, assume the agent is concluding \(\pv{\phi}{v}\) from \(\Phi\).
  There are two cases to consider.
  We state the each case for \(\chi\) generally, and then below state the case with respect to \izetaSm{}.
  \begin{enumerate}[label=\Roman*., ref=\Roman*]
  \item
    \label{iZm:arg:case:I}
    The agent satisfies \(\chi\) when concluding \(\pv{\phi}{v}\) from \(\Phi\).
    \begin{itemize}
    \item
      The agent concludes they would conclude \(\pv{\psi}{v'}\) from \(\Psi\) when concluding \(\pv{\phi}{v}\) from \(\Phi\), for any \(\pvp{\psi}{v'}{\Psi}\) which is a \requ{} of concluding \(\pv{\phi}{v}\) from \(\Phi\).
    \end{itemize}
  \item
    \label{iZm:arg:case:II}
    The agent has already satisfied \(\chi\) prior to concluding \(\pv{\phi}{v}\) from \(\Phi\).
    \begin{itemize}
    \item
      The agent has (already) concluded they would conclude \(\pv{\psi}{v'}\) from \(\Psi\) when concluding \(\pv{\phi}{v}\) from \(\Phi\), for any \(\pvp{\psi}{v'}{\Psi}\) which is a \requ{} of concluding \(\pv{\phi}{v}\) from \(\Phi\).
    \end{itemize}
  \end{enumerate}

  We take each case in turn.
  Further, as \(\chi\) does not imply satisfaction of \izetaS{}, and \izetaS{} is not trivially satisfied, for both cases we will assume the agent does not satisfy \izetaS{}.
\end{note}

\subparagraph*{Case~\ref{iZm:arg:case:I}}

\begin{note}[With \(\chi\)]
  \(\chi\) is sufficient for a negative answer to \qzS{}.

  Key observation.
  In order for \(\chi\) to be sufficient, agent's perspective.

  \begin{proposition}
    \label{prop:chiProp:no-may-fail}
    \(\chi\) must ensure that from the agent's perspective, the agent may not fail to conclude \(\pv{\psi}{v'}\) from \(\Psi\).
    \begin{argument}
      Direct from \qzS{}.
      For, negative answer.
      However, negative answer only if it is not the case that the agent may fail to conclude \(\pv{\psi}{v'}\) from \(\Psi\).
    \end{argument}
  \end{proposition}

  However, from~\ref{question:zs:option}, the agent has the option of concluding \(\pv{\psi}{v'}\) from \(\Psi\).

  By assumption the agent has not yet satisfied \(\chi\), as the agent has not yet concluded \(\pv{\phi}{v}\) from \(\Phi\).
  Hence, the agent has a check on whether they would come to satisfy \(\chi\) when concluding \(\pv{\phi}{v}\) from \(\Phi\).

  For, if the agent reasons about whether \(\pv{\psi}{v'}\) follows from \(\Psi\) and fails to conclude \(\pv{\psi}{v'}\) from \(\Psi\), then it would not be the case that the agent satisfies \(\chi\).

  Rephrase.
  \(\chi\) is sufficient for a negative answer to \qzS{}.
  In order for a negative answer, it may not be the case that the agent may fail to conclude \(\pv{\psi}{v'}\) from \(\Psi\).
  Therefore, the agent not failing, from their perspective, is required to satisfy \(\chi\).
  However, this means that before concluding \(\pv{\phi}{v}\) from \(\Phi\), the agent may establish whether they would come to satisfy \(\chi\).

  So, this means that there is a possibility of branching.
  If the agent reasons about whether \(\pv{\psi}{v'}\) follows from \(\Psi\), then the agent may fail to conclude \(\pv{\phi}{v}\) from \(\Phi\).

  Whether the agent would conclude \(\pv{\phi}{v}\) from \(\Phi\) is at issue.
  And, the agent only gets \(\chi\) when concluding.
  So, whether or not \(\chi\) cannot be sufficient for a negative answer.

  If failing is sufficient for positive answer, then failing is also sufficient to show that the agent does not satisfy \(\chi\).
  Therefore, \(\chi\) cannot be sufficient.
\end{note}

\begin{note}[With \izetaSm{}]
  From \izetaSm{}.
  \(\pvp{\psi}{v'}{\Psi}\) is a \requ{} of concluding concluding.
  Hence, so long as the agent has not already\dots it follows that concluding concluding is not sufficient for a negative answer to \qzS{}.
\end{note}

\begin{note}[Summary]
  Summary.
  \qzS{}, what would happen if the agent first reasoned about whether \(\pv{\psi}{v'}\) follows from \(\Psi\).
  This is the question.
  In order for negative answer, from agent's point of view, would not fail.
  However, question still holds if this is not yet the agent's point of view.
\end{note}

\begin{note}[Observation]
  Observe, as not yet \(\chi\), re-expressed \qzS{} as a question about whether \(\chi\).
  Therefore, the above argument only applies given we are in case~\ref{iZm:arg:case:I}.
  In case~\ref{iZm:arg:case:II}, we assume the agent already satisfies \(\chi\), and hence the argument will be distinct.
\end{note}

\subparagraph*{Case~\ref{iZm:arg:case:II}}

\begin{note}
  {
    \color{red}
    This should be revised, as with the idea of a \fc{0}, I no longer need to worry about the possibility of revision.

    Instead, the basic argument is that if a \requ{0}, then this still applies to whatever \(\chi\) is.
    For anything weaker, check.
  }
\end{note}

\begin{note}
  Now turn to case~\ref{iZm:arg:case:II}.

  With case~\ref{iZm:arg:case:I}, argued that \(\chi\) fails to be sufficient, because \qzS{} applies equally to whether the agent satisfies \(\chi\).
  Possibility of present reasoning branching so the agent does not satisfy \(\chi\).

  Case~\ref{iZm:arg:case:II} requires distinct argument, as by assumption the agent satisfies \(\chi\).
  Again, we have the assumption that the agent has not concluded \(\pv{\psi}{v'}\) from \(\Psi\).
\end{note}

\begin{note}
  Observe, the strategy applied to case~\ref{iZm:arg:case:I} does not extend to case~\ref{iZm:arg:case:II}.
  For, \requ{} of concluding \(\pv{\phi}{v}\) from \(\Phi\).
  Hence, it need not be the case that the agent had the option of concluding \(\pv{\psi}{v'}\) from \(\Psi\) when satisfying \(\chi\).
  In particular, when establishing from their perspective they would not fail to conclude.

  From the perspective of \izetaSm{}, concluded would conclude when the agent did not have the option of concluding.

  For example, consider a case of (apparent) testimony.
  If follow strategy, win game.
  Did not have an understanding of the rules.
  Hence, did not have the option to reason from premises and reach a different conclusion.
  Only after coming to understand rules (or more strictly, adopting the perspective of understanding rules) does the option of evaluating the conditional become available.
\end{note}

\begin{note}
  Our strategy is to split case~\ref{iZm:arg:case:II} into two sub-cases, depending on whether the agent may revise whether or not the agent may revise their satisfaction of \(\chi\).
  Specifically:

  \begin{enumerate}[label=\roman*., ref=\roman*]
  \item
    \label{iZm:arg:case:II:sub:i}
    From the agent's perspective:
    The agent may revise their epistemic state so that the agent does not satisfy \(\chi\), given the agent's current epistemic state.
  \item
    \label{iZm:arg:case:II:sub:ii}
    From the agent's perspective:
    The agent may not revise their epistemic state so that the agent does not satisfy \(\chi\), given the agent's current epistemic state.
  \end{enumerate}

  It may seem only sub-case~\ref{iZm:arg:case:II:sub:ii} is compatible with satisfaction of \(\chi\).

  For, as we have observed in~\autoref{prop:chiProp:no-may-fail}, not the case that the agent may fail.

  Rewriting:

  \begin{enumerate}[label=\roman*\('\)., ref=\roman*\('\)]
  \item
    \label{iZm:arg:case:II:sub:i:var}
    From the agent's perspective:
    The agent may fail to conclude \(\pv{\psi}{v'}\) from \(\Psi\), if the agent were to attempt to conclude \(\pv{\psi}{v'}\) from \(\Psi\).
  \item
    \label{iZm:arg:case:II:sub:ii:var}
    From the agent's perspective:
    The agent may not fail to conclude \(\pv{\psi}{v'}\) from \(\Psi\), if the agent were to attempt to conclude \(\pv{\psi}{v'}\) from \(\Psi\).
  \end{enumerate}

  So, it may appear only sub-case~\ref{iZm:arg:case:II:sub:ii:var} is compatible with the agent satisfying \(\chi\).

  However, it is important to keep in mind the scope of the relevant instance of `may'.
  Given \(\chi\), failure to conclude \(\pv{\psi}{v'}\) from \(\Psi\) may be ruled out from the agent's perspective.
  However, it may also be the case that the agent entertains the possibility of failing to satisfy \(\chi\).
  Hence, as the agent may not satisfy \(\chi\), the agent may fail to conclude \(\pv{\psi}{v'}\) from \(\Psi\).%
  \footnote{
    Of course, if you interpreted \qzS{} in line with sub-case~\ref{iZm:arg:case:II:sub:ii}, you may ignore sub-case~\ref{iZm:arg:case:II:sub:i}.

    Still, I take \qzS{} to be compatible with both sub-cases~\ref{iZm:arg:case:II:sub:i} and~\ref{iZm:arg:case:II:sub:ii}.
  }
\end{note}

\begin{note}[What we will argue]
  Respectively, we will argue:
  \begin{itemize}
  \item
    For sub-case~\ref{iZm:arg:case:II:sub:i}, \(\chi\) is insufficient for a negative answer to \qzS{}.
  \item
    For sub-case~\ref{iZm:arg:case:II:sub:ii}, re-assignment of concluding, hence \(\chi\) (trivially) entails concluding. Hence, conflict with our assumption that \(\chi\) does not entail concluding.
  \end{itemize}

  In other words:
  \begin{itemize}
  \item
    \qzS{} scopes over certain revisions to an agent's epistemic state.
  \item
    If no revision, then the relation between \(\pv{\psi}{v'}\) and \(\Psi\) is sufficient to reduce the relevant instance of concluding to \(R\) and witnessing.
  \end{itemize}

  \(R\) the agent would conclude \(\pv{\psi}{v'}\) from \(\Psi\) if the agent were to reason from \(\Psi\) to \(\pv{\psi}{v'}\), and from the agent's current epistemic state, \(R\) may not fail to hold.

  Here, importance of~\ref{idea:reassignment}.
  Re-assignment.
  The only thing for the agent to do is witness \(R\).
\end{note}

\subparagraph*{\(\pvp{\psi}{v'}{\Psi}\) remains a \requ{0} of concluding \(\pv{\phi}{v}\) from \(\Phi\), given \(\chi\)}

\begin{note}
  We begin with a minor, but important observation.

  \begin{proposition}
    \(\pvp{\psi}{v'}{\Psi}\) remains a \requ{0} of concluding \(\pv{\phi}{v}\) from \(\Phi\), given \(\chi\)
  \end{proposition}
\end{note}

\begin{note}[Still a \requ{}]
  Observe that \(\pvp{\psi}{v'}{\Psi}\) is a \requ{} with respect to concluding \(\pv{\phi}{v}\) from \(\Phi\).

  Even if agent has satisfied \(\chi\), and so from perspective, still holds up.
  Indeed, whether \(\pvp{\psi}{v'}{\Psi}\) is \requ{} is independent of whether the agent satisfies \(\chi\) or has concluded \(\pv{\psi}{v'}\) from \(\Psi\).

  For a \requ{}, what matters is option and failure.
  And, for a negative resolution to \qzS{}, no failure.

  Of course, from agent's perspective they would not fail.
  However, if reason and did fail, problem.

  So, from the present point of view, a \requ{}.
\end{note}

\paragraph{The sub-cases}

\subparagraph*{Sub-case~\ref{iZm:arg:case:II:sub:i}.}

\begin{note}[Distinction between \(\chi\) and concluding]
  There is an important difference between the two cases.
  By assumption, \(\chi\), has not concluded.

  So with \(\chi\), two types of possible epistemic states.
  Concluded, not concluded.
  Further, possibility of either of these epistemic states.

  Point is, with \(\chi\), because no entailment, and assumption, these two types of epistemic state are open.

  From present, the agent only expect to go to one.
  However, this is from the perspective of current epistemic state.

  By contrast, with concluding, the agent already in the relevant epistemic state.
  The agent has concluded \(\pv{\psi}{v'}\) from \(\Psi\).
\end{note}

\begin{note}
  As the agent is not in the relevant type of epistemic state, and has the option to go to the right epistemic state, \qzS{}.
\end{note}

\begin{note}
  At issue is whether the agent may do some reasoning any end up not concluding \(\pv{\phi}{v}\) from \(\Phi\) granted they have not already concluded \(\pv{\psi}{v'}\) from \(\Psi\).
  And, as the agent has not concluded \(\pv{\psi}{v'}\) from \(\Psi\), then regardless of perspective on how reasoning would go, the option is present, and as \(\chi\) does not entail, the agent does not have the result of taking the option.
  Taking the option would give something distinct.
  Option, so figure out whether.
\end{note}

\begin{note}
  Counterpoint.

  Do not need to drop the pen in order to know that it will fall to the ground.

  Or, perhaps, do not need to go and check my car is parked outside in order to know that it is parked outside.

  Difference here is these cases involve acquiring novel information, while \qzS{} involves reasoning with respect to the agent's current epistemic state.
\end{note}

\begin{note}
  Although we have introduced the possibility of revision, it is very narrow.
  Our interest is not with revision in general.
  The idea that an agent would conclude \(\pv{\phi}{v}\) from \(\Phi\) given arbitrary revision is incredibly strong.

  Rather, the revision is quite specific.
  It is because the agent has concluded \emph{that}, and because this is now a \requ{}.
  This does not scope over querying arbitrary premises, nor adopting addition premises.
  Only because this question remains open.
\end{note}


\subparagraph*{Sub-case~\ref{iZm:arg:case:II:sub:ii}.}

\begin{note}
  No revision.

  So, from the agent's perspective, if reason from \(\Psi\), would conclude \(\pv{\psi}{v'}\).
\end{note}

\begin{note}
  Now, by assumption witnessing can't do anything.
  There can be nothing added by witnessing that would matter to concluding \(\pv{\psi}{v'}\) from \(\Psi\).
  If there were, then possibility of revision.

  Okay, so, this means that witnessing doesn't add anything.

  Hence, whether on not X is determined by the agent's present epistemic state.
  This does not mean X, as X may be the result of witnessing.
  However, must have enough.
  So, reduce whether to something independent of witnessing.




  Some X, but then, agent's present epistemic state secures X.







  So, have something in the agent's present epistemic state.
  
\end{note}

\begin{note}
  So, sufficient resources given present epistemic state.

  For, if insufficient, then fail to conclude.

  So, witnessing, putting those resources into action.

  So, take any X.
  If X makes a difference to concluding, then, availability of X + witnessing.

  So, something, X.
  Get this by witnessing.
  However, if this makes a difference, then split into pre-X and witnessing X.
  Given sub-case, no failure.
  So, only pre-X makes a difference.
  Yet, have pre-X.
  Adding in witnessing won't do anything.

  So, then, split.

  Not possible to find some X, such from the agent's point of view, such that whether X is unique to witnessing reasoning from \(\Psi\) to \(\pv{\psi}{v'}\) such that X matters to whether conclude, and X is not composite.

  This is quite strong, but does not generalise easily.
  For, if possibility of branching, revision, etc.\ then there may be some such X.
  Indeed, agent has not concluded, might fail, so conclusion may add something unique.

  But, then, re-assignment.

  Hence, concluded.

  To \illu{1}, deterministic causation.
  Some causal model, a bunch of equations.
  Wouldn't say X has caused Y given application of equations.
  However, nothing more than putting those equations in motion.

  Note, though, that given re-assignment, this is what we get.
  What we're after is this reduction.
  At issue is not the intuitive sense of `concluding', but whether there is some reduction of this kind.

  Observe, this kind of thing does give a reduction.
  However, no clear cases of this happening.
  Too much information required.
  And, deterministic, at least with respect to description at level of premises and conclusions.
  Strong assumptions.

  Further, we have no eliminated role of witnessing.
  As with causation, we have no shown that this relation is specified independently.

  However, we do have a static reduction.
  Nothing of relevance is introduced by the dynamics.
  We might not get a specification without reference to the dynamics, but we don't get anything of relevance from the dynamics themselves.

  This is basically a redescription of the assumption made.
  If there is no possibility of failing, then the dynamics are pre-determined, at least from the agent's perspective.





  Sufficient resources plus witnessing.
  These resources, no possibility of failure.
  So, for anything that does not involve witnessing, we have this.

  In addition, for anything witnessing would get, the agent has a guarantee that they would obtain.

  Not perfect information.
  Maybe exact premises, steps of reasoning, etc.
  Still, all of these are available.
\end{note}



\begin{note}[Edge case]
  \(\chi\), the agent reasons, but does not conclude \(\pv{\psi}{v'}\) from \(\Psi\).
  Nor does the agent fail to conclude \(\pv{\psi}{\overline{v'}}\) from \(\Psi\).

  However, no revision.
  Well, then the agent won't conclude \(\pv{\phi}{v}\) from \(\Phi\).
  If the agent does, then revised epistemic state so that \ref{question:zs:subjunctive} does not hold.
  Hence, \qzS{} would no longer apply.
\end{note}

\begin{note}[Key idea]
  \begin{itemize}
  \item
    \(\chi\) does not entail concluded.
  \item
    Option to conclude.
  \item
    If agent were to reason, to types resulting states.
  \item
    Concluded, failed to conclude.
    Really, various possible resulting states, as reasoning may not terminate either way, and in general there may be various values other than \(v'\) that the agent may conclude.
  \item
    The agent isn't in either type of epistemic state.
  \item
    Now, \(\chi\) states from current perspective, what the result would be.
  \item
    However, it remains the case that these two types of epistemic states are open.
  \item
    Hence, it remains the case that reasoning could lead to either type.
  \item
    So, it remains the case that the agent may fail to conclude \(\pv{\psi}{v'}\) from \(\Psi\).
  \end{itemize}

  The core of the idea is that \qzS{} concerns the dynamics.
  If reason, then what would the result be with respect to concluding \(\pv{\phi}{v}\) from \(\Phi\).
  So, concluding to dynamics.
  Hence, \(\chi\) of the appropriate kind does not settle.
\end{note}

\section{Notes}

\begin{note}[Witnessing?]
  Intuitively, this suggests a negative resolution to~{\color{red} issue:Main}.
  Agent needs to witness reasoning, else, the agent does not have sufficient information about the relevant dynamics.

  Indeed, if negative resolution to~{\color{red} issue:Main}, then I suspect this brief sketch is intuitive.
  Of course, negative answer to \qzS{} is not required for concluding, as \qzS{} requires that this holds for all \requ{1}.
  Still, given what is intuitively required for negative answer to \qzS{}, witnessing.
\end{note}

\begin{note}
  Now, important to keep in mind is that we have no assumed that concluding involves witnessing.
  By assumption, there is some difference, else we are done.
  But, what exactly this distinction amounts to is unclear.
  An account of concluding, beyond scope.
  Further, still have the issue to resolve.
  However, have a clue.
  Concluding, between \(\pv{\phi}{v}\) and \(\Phi\).
  Concluding that, existence of a relation.
  Whatever the relation of concluding is, it does not yet hold, but the agent has a guarantee that it will hold.
\end{note}

\begin{note}
  The point, in brief, is that \qzS{} is about concluding.
  And, so long as the agent has not concluded \(\pv{\psi}{v'}\) from \(\Psi\) when concluding \(\pv{\phi}{v}\), that failure to conclude would lead to failure.
  So, \(\pvp{\psi}{v'}{\Psi}\) remains unsettled.
  If the agent does the reasoning, then would not conclude.
  So, the agent may fail to conclude, because on reasoning about whether, failure to conclude \(\pv{\psi}{v'}\) from \(\Psi\) would lead to a recognised failure of \(\chi\).

  \qzS{} is about whether the agent would conclude.
  If agent hasn't concluded, then question remains.
  Still, I do not think this is immediate.
\end{note}


\begin{note}
  Important to note, that still all relative to the agent's present epistemic state, and what the agent is interested in concluding.
  For, it may be that the agent revises their epistemic state so that some relation does not hold.
  For example, the agent has concluded that some reduction holds.
  Therefore, if they prove this, then they prove something else, and, also prove the something else by other means.
  However, learn this reduction does not hold.
  Now, no longer a \requ{}.
  Indeed, relation fails because the reduction does not hold, or because the agent does not have the means to prove.
\end{note}

\paragraph*{Closing}

\begin{note}[Closure]
  So, this is our motivation.
  What we have is an intuitive idea which leads to this kind of limitation, and hence conclusre condition.
  So, if there are cases of interest, motivation that an agent concludes.
  But, conversely, the condition is strong.
  So, these cases are harder.
\end{note}

\paragraph{More details on \izetaS{}}

\begin{note}
  We've only focused on failure to conclude.
  However, the agent may also conclude something else.
  Possible that there are premises for the agent \(\Psi\) and \(\overline{\Psi}\) such that from \(\Psi\) get \(\pv{\psi}{v'}\) and from \(\overline{\Psi}\), get \(\pv{\psi}{\overline{v'}}\).

  For sure, but this is a different condition.
  You may also want to impose this given the intuitive motivation for \csN{}.
  Indeed, from the perspective of no branching.
  Distinct condition, and we will not impose this.

  Note, also, that so long as distinction between concluding \emph{that} and concluding, then this is also going to be insufficient in isolation.
  For, though the agent may have exhausted other possibilities, this won't get a conclusion.
  And, if not distinct, then a plausible path to negative resolution.
\end{note}

\begin{note}
  \izetaS{} does not require the conclusion to be any good.
  If you want to build this in, sure.
  However, not for us.
  It is a strong assumption, and would have no function in the arguments to follow.
\end{note}

\paragraph*{Minor clarifications}

\begin{note}[Importance of \csN{}]
  \izetaS{} is key.
  Argument for negative resolution to~{\color{red} issue:Main} largely rests on \izetaS{}.
  As we have seen, closure.

  However, briefly note that a few things.

  First, agent's reasoning.
  At issue is whether the agent may reason to a different conclusion.
  There's nothing that would lead me elsewhere.

  Second, agent's reasoning.
  Independent of whether \(\phi\) has value \(v\), \(\psi\) has value \(v'\), or any of the premises.
  Need not be the case that satisfaction amounts to anything substantial.
  No clause for justification, etc.

  Third, competence, rather than performance.
\end{note}

\paragraph*{\emph{Concluding}}

\begin{note}[Key feature]
  We now turn to the key feature of \izetaS{}:
  The requirement that an agent \emph{concludes} \(\pv{\psi}{v'}\) from \(\Psi\) when \(\pvp{\phi}{v'}{\Psi}\) is a \requ{} of concluding \(\pv{\phi}{v}\) from \(\Phi\).

  We begin by noting why this requirement should be treated with caution from a general perspective.
  We then further motivate caution by first relating the requirement to the motivation we have provided for \csN{}, and second by highlighting how the requirement will have a role in developing tension.
  This discussion will involve considering a weaker requirement, and following our motivation of caution we will argue that no weaker requirement will suffice to capture the motivation we have provided for \csN{}.
\end{note}

\paragraph*{Narrowing \requ{1}}

\begin{note}[Expanding pool constraints]
  To~\ref{notion:overview:requ:pool} of~\autoref{notion:overview:requ} the following clause may also be added:
  \begin{enumerate}[label=]
  \item
    \begin{enumerate}[label=]
    \item
      \begin{enumerate}[label=\roman*., ref=(\roman*), resume*=csIdeaCounter]
        \setcounter{enumiii}{3}
      \item
        \label{notion:overview:requ:pool:method}
        Concluding \(\pv{\psi}{v'}\) from \(\Psi\) involves the same general method the agent would use to conclude \(\pv{\phi}{v}\) from \(\Phi\).
      \end{enumerate}
    \end{enumerate}
  \end{enumerate}
  We omit~\autoref{notion:overview:requ:pool:method} from the idea of \csN{} for two (related) reasons.
  First, it is not clear what `the same general method' amounts to in detail.
  Second, avoiding questions about method affords flexibility when providing \illu{1} of \zS{}.
  However,~\autoref{notion:overview:requ:pool:method} may be imposed with no loss to the role of \zS{} in the overall argument.
  However, always a check on whether one has the general ability.
\end{note}

\begin{note}
  Reasoning, \support{}, would not reason to a different conclusion.

  Specifically, \requ{} of some conclusion.
  So long as conclusion, then it is possible to reason about whether \(\psi\) has value \(v'\), and unless conclude \(\psi\) has value \(v'\), would not conclude \(\phi\) has value \(v\).

  Intuitively, \requ{} as an independent check on the reasoning.
  If don't hold \(\psi\) from premises, then question about whether \(\phi\).

  Claiming support, necessary condition is satisfying all \requ{1}.
  Claiming support, then, is weaker than having support.
  Restricted to whether conclusion of reasoning would introduce a \requ{}.
  And, may be further restricted without impact to the tension we will develop to whether the conclusion would `clearly' introduce a \requ{}.
\end{note}

\begin{note}
  Now, consequence from \ideaCS{} and \ESU{},
  If potential witnessing event, then either concluded previously, or witness reasoning.
\end{note}

\begin{note}[Kettle logic]
  Well, it's true that the person must be acquitted, but at the same time, the person is going to have a hard time explaining how, for example, he brewed coffee for a week.

  Still, highlights what the neighbour needs to conclude.
  Did borrow.
  Was not lent damaged
  And, was returned damaged.

  Here, burden of argument.
  However, we're not interested in whether the neighbour would convince, but whether the neighbour would reach a different conclusion if they were first to reason about one of the three before concluding that the kettle was returned damaged.
\end{note}

\subsection{Literature}
\label{sec:zS:literature}

\paragraph{Circularity}

\begin{note}
  \ideaCS{} is not about circular reasoning in the sense that the term `circularity' suggests that the reasoner has taken the conclusion of the reasoning for granted.

  There's nothing in \ideaCS{} that appeals to getting \(\psi\) having value \(v'\) from \(\phi\) having value \(v\).

  However, does identify a problem in the sense that would prevent the agent from getting \(\psi\) having value \(v'\) from \(\phi\) having value \(v\).
\end{note}

\begin{note}[Testimony 1]
  \begin{illustration}[Testimony 1]
    \label{illu:CS:test:basic}
    \mbox{}
    \begin{enumerate}[label=\arabic*., ref=(\arabic*)]
    \item
      \label{ex:eiS:t:basic:test}
      \nagent{11} stated that they are trustworthy when speaking on matters regarding their personal character.
    \item
      \label{ex:eiS:t:basic:ok}
      \nagent{11} is trustworthy when speaking on matters regarding their personal character.
    \end{enumerate}
  \end{illustration}
  This kind of case is intuitively problematic.
  It seems that already need trustworthy.
  However, in order for \csN{} to apply, need for it to be the case that one has some check on whether \nagent{11} is trustworthy.
  And, by reasoning.

  This need not be the case.
  Of course, this does not mean that an agent need \csN{}.
  May be other necessary conditions.
\end{note}

\paragraph{Sgaravatti}

\begin{note}
  For example, consider what \citeauthor{Sgaravatti:2013wu} terms the `Justification Account' of circularity.\nolinebreak
  \footnote{
    As \citeauthor{Sgaravatti:2013wu} notes, the Justification Account of circularity is a rewriting of the third type of `epistemic dependence' considered by \citeauthor{Pryor:2004ws}~(\citeyear[359]{Pryor:2004ws}).
    Neither \citeauthor{Pryor:2004ws} nor \citeauthor{Sgaravatti:2013wu} endorse the Justification Account, but I take the spirit of the account to sufficient for interest.
    Still, the considerations which follow also apply to distinguish the {\color{red} problem identified} from \citeauthor{Sgaravatti:2013wu}'s favoured account (\Citeyear[\S3]{Sgaravatti:2013wu}) and the fifth type of `epistemic dependence' considered by \citeauthor{Pryor:2004ws}~(\citeyear[359]{Pryor:2004ws}).
  }

  \begin{quote}
    \begin{enumerate}[label=(JA), ref=(JA)]
    \item\label{sg:JA} An argument is circular if and only if for you to have justification to believe the premisses, it is necessary that you have justification to believe the conclusion.\nolinebreak
      \mbox{}\hfill\mbox{(\Citeyear[754]{Sgaravatti:2013wu})}
    \end{enumerate}
  \end{quote}
  Where `justification to believe' is to be read as in terms of having formed the belief in an epistemically appropriate way as opposed to (merely) possessing sufficient resources to form formed the belief in an epistemically appropriate way.\nolinebreak
  \footnote{
    Or, however you prefer to characterise \citeauthor{Firth:1978vi}'s (\Citeyear{Firth:1978vi}) distinction between doxastic and propositional justification (or warrant).
    See also \citeauthor{Silva:2020aa} (\Citeyear{Silva:2020aa}) --- esp.\ fn.\ 1.
  }
  (\citeauthor[Cf.][754--755]{Sgaravatti:2013wu})
\end{note}

\begin{note}
  First, reliance on something like justification.

  With \support{}, we arguably have something distinct.
  Have not placed constraints on reasoning.
  Hence, \ideaCS{} applies even when no justification (or any other epistemic attribute) is found.

  Indeed, to the extent that the value \(v\) need not be truth, \ideaS{} and \ideaCS{} are broader.

  Point extends to relation between the premises and the conclusion of a step of reasoning.
  There's some issue with whether there's a clear reduction to premises.

  Now, both these points may be addressed by linking justification to steps of reasoning.
  However, it still remains that get this kind of circularity by placing a constraint on permissible steps of reasoning.
\end{note}

\begin{note}
  Second, having something.
  Contrasts to reasoning in an interesting way.
\end{note}

\begin{note}[\citeauthor{Sgaravatti:2013wu} on necessity]

  \begin{quote}
    For my present purposes it will suffice to say that a good test of A’s being necessary for B (and thus of B’s being sufficient for A) is the satisfaction of two subjunctive conditionals. First, if A did not hold, B would not hold; secondly, if B were to hold, A would hold.%
    \mbox{}\hfill\mbox{(\citeyear[761]{Sgaravatti:2013wu})}
  \end{quote}
  This is very similar to what is captured by a \requ{}.

  Also, points out only a test due to implications.
  For us, \requ{} is not a test.
  And, the implications are embraced.
  Though, differences limit these somewhat.

  For the moment, point with the implications is that this makes \zS{} fairly strong.
\end{note}

\paragraph{Pryor}

\begin{note}[\citeauthor{Pryor:2004ws}'s Type 4]
  An instance of a limitation arising from assuming that the possibility obtains is the fourth type of dependence between premise and conclusion considered by \citeauthor{Pryor:2004ws}.

  \begin{quote}
    [Type 4] dependence between premise and conclusion is that the conclusion be such that evidence \emph{against it} would (to at least some degree) undermine the kind of justification you purport to have for the premises.\nolinebreak
    \mbox{}\hfill\mbox{(\citeyear[359]{Pryor:2004ws})}
  \end{quote}

  Again, plausible.\nolinebreak
  \footnote{
    A variant of \citeauthor{Pryor:2004ws}'s Type 4 dependence is~\citeauthor{Jackson:1984vk}'s account of circularity.
    \begin{quote}
      [I]t may be that a given argument to a given conclusion is such that anyone --- or anyone sane --- who doubted the conclusion would have background beliefs relative to which the evidence for the premises would be no evidence.\space \dots

      Such an argument could be of no use in convincing doubters, and is most properly said to beg the question.\nolinebreak
      \mbox{}\hfill\mbox{(\Citeyear[111-12]{Jackson:1984vk})}
    \end{quote}
    Still, in contrast to \citeauthor{Pryor:2004ws}'s Type 4, \citeauthor{Jackson:1984vk}'s account of circularity is dialectical.
    Indeed, on \citeauthor{Jackson:1984vk}'s account (without additional constraints on when an agent has justification or evidence) it need not be the case that the agent's own justification would be undermined by someone doubting the conclusion.
    In this respect, \ideaCS{} is further distinguished from a proposal such as \citeauthor{Jackson:1984vk}'s as \ideaCS{} makes mention only of the relevant agent's epistemic state and reasoning.
  }
  Further, weaken from justification to any reasoning.
  In this respect, motivated by \ideaS{}, plausibly.
  However, much stronger.
  \ideaS{} is just about entertaining.
  Subjunctive with stronger is less clear.

  Issue:
  \begin{enumerate}
  \item Evidence undermines the kind of justification the agent purports to have for the premises.
  \end{enumerate}

  And, as \citeauthor{Pryor:2004ws} notes, \emph{kind} is important.
  However, it seems kind is not the only problem.
\end{note}

\begin{note}
  \citeauthor{Pryor:2004ws}'s argument that type 4 over-generates is somewhat interesting.
  Details are in the following footnote.\footnote{
  Compatible with \citeauthor{Pryor:2004ws}'s objection to type 4 dependence.

  % \begin{illustration}
    % \mbox{}
    % \vspace{-\baselineskip}
    \begin{quote}
      Suppose you're watching a cat stalk a mouse. Your visual experiences justify you in believing:

      \begin{enumerate}[label=(\arabic*), ref=(\arabic*)]
        \setcounter{enumi}{10}
      \item
        \label{illu:Pryor:cat:1}
        The cat sees the mouse.
      \end{enumerate}

      You reason:

      \begin{enumerate}[label=(\arabic*), ref=(\arabic*), resume]
      \item
        \label{illu:Pryor:cat:2}
        If the cat sees the mouse, then there are some cases of seeing.
      \item
        \label{illu:Pryor:cat:3}
        So there are some cases of seeing.\nolinebreak
        \mbox{}\hfill\mbox{(\citeyear[361]{Pryor:2004ws})}
      \end{enumerate}
    \end{quote}
  % \end{illustration}

  Setting aside whether this is fine.

  Following \citeauthor{Pryor:2004ws}:

  Bad, given proposal, as if no cases of seeing, then the cat is not seeing. (\citeyear[361]{Pryor:2004ws})

  \citeauthor{Pryor:2004ws}'s position is as follows:

  \begin{quote}
    I don't think you need antecedent justification to believe \ref{illu:Pryor:cat:3}, before your experiences can give you justification to believe \ref{illu:Pryor:cat:1}.
    I also think it's plausible that your perceptual justification to believe \ref{illu:Pryor:cat:1} contributes to the credibility of \ref{illu:Pryor:cat:3}.\nolinebreak
    \mbox{}\hfill\mbox{(\citeyear[361]{Pryor:2004ws})}
  \end{quote}

  This may be compatible with \ideaS{} and \ideaCS{}.
  With \ideaCS{}, somewhat trivial, if \ref{illu:Pryor:cat:3} holds throughout \epVW{1}.

  More generally, weaker proposition.
  Hence, it seems \indicateV{1}.
  So there's no issue with the reasoning.
  However, `contributes to the credibility\dots'.
  }
\end{note}

\begin{note}[Issue]
  Somewhat similar to above.
  Here, however, role of novel information is of interest.
  Hence, dynamic.
  And, \csN{} is, in this respect, static.
\end{note}


\paragraph{Others}

\begin{note}
  This also extends to \citeauthor{Wright:2011wn}.
  For, \citeauthor{Wright:2011wn} relies on the idea of doubt.

  The issue here is what is required in order to doubt.
  One may need to revise one's epistemic state.

  Of course, if idea of claiming support is taken generally, then it should be the case that for any \epPW{}, it is possible for the agent to conclude from reasoning that \(\phi\) having value \(v\) holds for any \epVAd{} \world{}.

  So, if satisfy claiming support, then may satisfy doubt idea.
  However, ideal.
  Pointing out the issue here does not require such a general thing as doubt.
\end{note}

\begin{note}
  Instead, as \(\psi\) not having value \(v'\) is an \ep{}, it is possible that \(\psi\) does not have value \(v'\).
  And, if \(\psi\) does not have value \(v'\), then step \(\delta'\) does not apply to how things are.
  Hence, observing that \(\psi\) having value \(v'\) follows in turn from the conclusion of step \(\delta'\) (together with other premises) is uninformative about how things are.
\end{note}

\begin{note}
  \color{red}
  Some of the \citeauthor{Wright:2011wn} cases are interesting.
  Especially the twin cases.
  In fact, especially this idea that situations are identical.
  For, one way of understanding this is that the agent makes a choice between two disjuncts, and it is possible for the agent to make the other choice, and then come to a different conclusion.
\end{note}

\subsection{Summarising}

%%% Local Variables:
%%% mode: latex
%%% TeX-master: "master"
%%% End:
