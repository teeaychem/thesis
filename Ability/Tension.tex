\chapter{Overview of Tension}
\label{sec:tension}

\subsection{Sketch of tension}
\label{sec:overview-tension}

\begin{note}
  Test
\end{note}

\begin{note}[Ability]
  Here, our interest turns to ability.
  I take it as given that there are various instances of concluding \(\pv{\phi}{v}\) from \(\Phi\) for which the reasoning from \(\Phi\) to \(\pv{\phi}{v}\) is an instance of a general ability.

  For example, I conclude \(31 + 53 = 84\) from some premises, and the reasoning falls under my general ability to perform (simple) arithmetic.
  The specifics may differ, but there is sufficient overlap with concluding \(43 + 81 = 123\) and \(91 + 54 = 145\) to consider the reasoning of the same type.
  Indeed, \(532 - 91 = 441\), \(19 * 32 = 608\), and \(126/36 = 3.5\) may also fall under the same (general) ability.
  In other words, in concluding each equation I witness a specific instance of the general ability.

  Likewise, one may have the (general) ability to solve chess problems, complete \(\{ \text{Sudoku}, \text{KenKen}, \text{Nonogram}, \dots\}\) puzzles, or parse sentences in a given language.

  I would not have concluded \(31 + 53 \ne 84\) and you would not have failed to identify the relevant winning strategy of some chess problem.

  However, in each of the examples of ability noted, every other specific instance of the general ability functions as an independent check on whether one has the relevant ability.
  I should make no mistake about \(85 + 21\) and you should make no mistake with the next chess problem.
  Granting, of course, that we do have the relevant abilities.

  Hence, it seems clear that if an agent \csV{} for \(\pv{\phi}{v}\) when concluding \(\pv{\phi}{v}\) from \(\Phi\) and the agent's reasoning is a specific instance of a general ability, then there at least various \requ{1} associated with concluding \(\pv{\phi}{v}\) from \(\Phi\).

  More generally, concluding \(\pv{\phi}{v}\) from \(\Phi\) from reasoning which is the specific instance of a general ability leads to a \cluster{}.
  And, it is perhaps already intuitive that one does not witness reasoning from the premises of at least some proposition-value-premises pairing in the \cluster{}.

  For example, it seems plausible that the configuration the chess board for any given problem forms of a premise of the agent's reasoning, but so long as one has not seen the relevant problem, it seems implausible that one has witnessed reasoning that includes the relevant configuration.

  Of course, it may seem equally implausible that an agent concludes there is some winning strategy for some configuration of a chess board they have not yet seen.
  However, this intuition should be carefully examined.

  With an unopened chess book by a reputable author before you, I expect you have no problem concluding that each of the solutions are correct.
  {
    \color{red}
    Or, that granting ability, conclude you would (also) conclude winning strategy or not for each chess piece.
  }
  Yet, you have not yet seen any of the solutions.
  So, in general, there seems no issue with an agent concluding there is some winning strategy for some configuration of a chess board they have not yet seen.

  Of course, in the case of the book there is a key premise:
  The book is written by a reputable author.
  However, if you wish to conclude there is a winning strategy via your own reasoning, then the above considerations take effect.%
  \footnote{
    And, indeed, may already hold with respect to the key premises.
    For, so long as you hold you have the general ability to reason about chess problems of the relevant kind, you have an independent check on whether the author really is reputable.
  }
\end{note}

\chapter{Abstract tension}
\label{cha:tension-abstract}

\begin{note}[\requCluster{3}]
  We begin with the definition of a \cluster{}.
  \begin{definition}[A \requCluster{1}]
    \label{def:requCluster}
    Some collection of proposition-value-premises pairings \(\mathcal{C} = \{\pvp{\phi_{i}}{v_{i}}{\Phi_{i}}\}_{i}\) is a \emph{\cluster{}} with respect to some agent \vAgent{}('s epistemic state) just in case:
    \begin{itemize}
    \item
      For any \(\pvp{\phi_{i}}{v_{i}}{\Phi_{i}}\) in \(\mathcal{C}\), each \(\pvp{\phi_{j}}{v_{j}}{\Phi_{j}}\) in \(\mathcal{C}\) (such that \(j \ne i\)) is a \requ{} of \(\pvp{\phi_{i}}{v_{i}}{\Phi_{i}}\).
    \end{itemize}
    \vspace{-\baselineskip}
  \end{definition}

  Intuitively, a \cluster{0} is a collection of proposition-value-premises pairings such that every proposition-value-premises pairing is a \requ{} of every other proposition-value-premises pairing in the collection.
\end{note}

\begin{note}
  With the definition of a \cluster{} in hand, we are ready to state our first proposition.

  \begin{proposition}[\cluster{3} and \zetaS{}]
    \label{prop:cluster:csN}
    Suppose \(\mathcal{C}\) is a \requCluster{0} with respect to an agent \vAgent{}('s epistemic state).
    And, let \(\pvp{\phi}{v}{\Phi}\) be some proposition-value-premises pairing in \(\mathcal{C}\).

    \begin{itemize}
    \item
      \vAgent{} has \zetaS{} for \(\pv{\phi}{v}\) when concluding \(\pv{\phi}{v}\) from \(\Phi\) only if \emph{either}:
      \begin{itemize}
      \item
        For any \(\pvp{\psi}{v'}{\Psi}\) in \(\mathcal{C}\):
        \begin{itemize}
        \item \vAgent{} has at some point in the past concluded that \vAgent{} would conclude \(\pv{\psi}{v'}\) from \(\Psi\).
        \item
          When concluding \(\pv{\phi}{v}\) from \(\Phi\), \vAgent{} simultaneously concludes that \vAgent{} would conclude \(\pv{\psi}{v'}\) from \(\Psi\).
        \item
          \fc{2}.
        \end{itemize}
      \end{itemize}
    \end{itemize}
  \end{proposition}

  \autoref{prop:cluster:csN} follows directly from~\izetaS{} and~\autoref{def:requCluster}.

  \begin{argument}
    Suppose \(\mathcal{C}\) is a \requCluster{0} with respect to an agent \vAgent{}('s epistemic state).
    Let \(\pvp{\phi}{v}{\Phi}\) be some proposition-value-premises pairing in \(\mathcal{C}\).
    And, let \(\pvp{\psi}{v'}{\Psi}\) be some proposition-value-premises pairing in \(\mathcal{C}\).

    By~\autoref{def:requCluster} we have that \vAgent{} concluding that \vAgent{} would conclude \(\pv{\psi}{v'}\) from \(\Psi\) is a \requ{} of concluding \(\pv{\phi}{v}\) from \(\Phi\).
    And, by~\izetaS{}, it must be the case that either:
    \begin{itemize}
    \item
      \vAgent{} has concluded that they would conclude \(\pv{\psi}{v'}\) from \(\Psi\), ref{idea:Zs:overview:requ-sat:Past}.
      Or,
    \item
      In concluding \(\pv{\phi}{v}\) \vAgent{} simultaneously concludes \(\pv{\psi}{v'}\) from \(\Psi\), ref{idea:Zs:overview:requ-sat:Pres}.
    \end{itemize}
    Hence, we have established~\ref{prop:cluster:csN}.
  \end{argument}
\end{note}

\paragraph*{Examples}

\begin{note}[Examples of \requCluster{0}]
  Focused primarily on \scen{1} where we don't have a \requCluster{}.
  Lots of \emph{only if} conditionals.
  In some of these \scen{0}, plausible the conditional goes both ways.
  Hence, get a \requCluster{}.

  More generally, fairly mundane.
  Basic abilities.
\end{note}

\begin{note}[Ability]
  Ability to reason in certain ways leads to \cluster{1}.
  However, not interested in ability in general, but rather relatively simple instances of ability concerning specific problem types.

  Arithmetic.
  Sudoku.
  Chess.

  Indeed, the latter pair for \requCluster{1}.
  For, different starting positions.
\end{note}

\begin{note}
  Finally, ability.

  General ability, specific ability.

  Claim support for having some general ability.

  Now, here, simple cases.
  Basic arithmetic.
  Sudoku puzzles.
  Chess problems with winning strategies.

  Roughly the same.
  More broadly:

  Logic problems.

  Crossword.

  Reading novels up to a certain level.
  Here, if you can't read, then the writing is bad.

  Fluency.

  So, specific instances of the general ability.
\end{note}

\subsection{Observations about \requCluster{}}
\label{sec:observ-about-requcl}


\begin{note}
  \begin{proposition}
    \label{prop:cluster:simul}
    Need to do everything in a cluster at the same time.
  \end{proposition}

  \begin{argument}
    Straightforward.
    For, anything is a \requ{} for any other.
    So, only \csV{} at the same time.
  \end{argument}
\end{note}

\begin{note}[No \(\gamma\)]
  Consequence:

  \begin{corollary}
    \label{prop:cluster:no-general}
    No general \(\pvp{\gamma}{v}{\Gamma}\) within cluster.
  \end{corollary}

  \begin{argument}
    Quickly, because of~\ref{prop:cluster:simul}.
    Only \(\gamma\) at same time as others.

    In more detail.
    For, by assumption, \requ{} means that it's possible for the agent to conclude.
    By the each other \requ{} functions as a check on \(\gamma\).
    If haven't figured out each individual, then the general is in question.
  \end{argument}
\end{note}


\begin{note}
  Indeed, narrow interest to \ragCluster{1}.

  \begin{definition}[\ragCluster{3}]
    For any cluster \(\mathcal{C}\), \(\mathcal{C}\) is a \emph{\ragCluster{}} if and only if:
    \begin{enumerate}
    \item
      There is some \(\pvp{\phi_{i}}{v_{i}}{\Phi_{i}}\) and \(\pvp{\phi_{j}}{v_{j}}{\Phi_{j}}\) such that \(\Phi_{i}\) and \(\Phi_{j}\) do not overlap.
    \end{enumerate}
    \vspace{-\baselineskip}
  \end{definition}

  A \ragCluster{}, then, is just a cluster where at least some distinct premises.
  Hence, avoid issue where same premises allow simultaneous conclusion, and fail to establish tension with \ESU{}.

  \begin{proposition}
    With \ragCluster{} concluded previous or violate \ESU{}.
  \end{proposition}

  Now, some caution.
  It may be the case that reason from some non-minimal collection of premises.
  Hence, some care when establishing \ragged{}.
  This means, argue that \cluster{} \emph{and} argue \cluster{} is \ragged{}.
  Again, without any clear bounds on premises, this argument is non-deductive.
  However, plausible in various cases.
\end{note}

\begin{note}[Relative \jag{1}]
  Given importance of \ragged{} and specific proposition-value-premises pairings of \ragged{}, terminology:

  \begin{definition}[Relative \jag{1} of a \ragCluster{}]
    \(\mathcal{C}\) some \ragCluster{}.
    \(\pvp{\psi}{v'}{\Psi}\) is a \emph{\jag{0}} relative to \(\pvp{\phi}{v}{\Phi}\) if \(\Psi\) differs from \(\Phi\).
  \end{definition}
\end{note}

\begin{note}
  \begin{proposition}
    If \ragCluster{} and no prior conclusion for some \jag{}.
    Either:
    \begin{itemize}
    \item
      \ESU{} does not hold in general.
    \item
      No \csVImp{} for any proposition-value-premises pairing in cluster.
    \end{itemize}
      \begin{argument}
    More-or-less immediate from previous.
  \end{argument}
  \end{proposition}
\end{note}

\begin{note}
  Abstract tension, then, follows if there are instances of \ragCluster{1} with no prior conclusion for some \jag{}.
\end{note}

\begin{note}[Pointed cluster]
    \begin{definition}[Pointed cluster]
    For any cluster \(\mathcal{C}\), \(\mathcal{C}\) is a \emph{pointed cluster} if and only if:
    \begin{enumerate}
    \item
      Some conclusions are the same.
    \end{enumerate}
    \vspace{-\baselineskip}
  \end{definition}
\end{note}

\section{(Aside) Circularity}
\label{sec:aside-circularity}

\paragraph{\citeauthor{Sgaravatti:2013wu} on circularity}

{
  \color{red}
  The main interest of this is simultaneous conclusions.
  Here, pointing out that we don't get circularity.
}

\begin{note}
  \izetaS{} is not about circular reasoning in the sense that the term `circularity' suggests that the reasoner has taken the conclusion of the reasoning for granted.

  Indeed, \iRequ{} is constructed in such a way that rules out this possibility.
  Though, not \emph{to} rule out this possibility --- worry is rather\dots

  Of interest because some similarity.
\end{note}

\begin{note}
  Consider what \citeauthor{Sgaravatti:2013wu} terms the `Justification Account' of circularity.\nolinebreak
  \footnote{
    As \citeauthor{Sgaravatti:2013wu} notes, the Justification Account of circularity is a rewriting of the third type of `epistemic dependence' considered by \citeauthor{Pryor:2004ws}~(\citeyear[359]{Pryor:2004ws}).
    Neither \citeauthor{Pryor:2004ws} nor \citeauthor{Sgaravatti:2013wu} endorse the Justification Account, but I take the spirit of the account to sufficient for interest.
    Still, the considerations which follow also apply to distinguish the {\color{red} problem identified} from \citeauthor{Sgaravatti:2013wu}'s favoured account (\citeyear[\S3]{Sgaravatti:2013wu}) and the fifth type of `epistemic dependence' considered by \citeauthor{Pryor:2004ws}~(\citeyear[359]{Pryor:2004ws}).
  }

  \begin{quote}
    \begin{enumerate}[label=(JA), ref=(JA)]
    \item
      \label{sg:JA}
      An argument is circular if and only if for you to have justification to believe the premisses, it is necessary that you have justification to believe the conclusion.%
      \mbox{}\hfill\mbox{(\citeyear[754]{Sgaravatti:2013wu})}
    \end{enumerate}
  \end{quote}
  Where `justification to believe' is to be read as in terms of having formed the belief in an epistemically appropriate way as opposed to (merely) possessing sufficient resources to form formed the belief in an epistemically appropriate way.\nolinebreak
  \footnote{
    Or, however you prefer to characterise \citeauthor{Firth:1978vi}'s (\citeyear{Firth:1978vi}) distinction between doxastic and propositional justification (or warrant).
    See also \citeauthor{Silva:2020aa} (\citeyear{Silva:2020aa}) --- esp.\ fn.\ 1.
  }
  (\citeyear[754--755]{Sgaravatti:2013wu})
\end{note}

\begin{note}[\citeauthor{Sgaravatti:2013wu} on necessity]

  \begin{quote}
    For my present purposes it will suffice to say that a good test of A's being necessary for B (and thus of B's being sufficient for A) is the satisfaction of two subjunctive conditionals.
    First, if A did not hold, B would not hold; secondly, if B were to hold, A would hold.%
    \mbox{}\hfill\mbox{(\citeyear[761]{Sgaravatti:2013wu})}
  \end{quote}
  This is very similar to what is captured by a \requ{}.

  Important difference is the second subjunctive conditional.
  This does not need to hold.

  However, does in various cases.
  And, as we will see, in important case.

  \emph{Still, in general this is not a premise conclusion relation}
  But, in certain cases it is?
  No, in general it can't be, because then would have a failing of a \requ{}.
  For, conclusion would need to be premise and vice-versa.
\end{note}

\chapter{Resolving tension}
\label{cha:overview:resolving-tension}

\begin{note}
  Note, the tension is not about whether \(\phi\) has value \(v\).
  Instead, the tension is about whether the agent would have a certain property if they were to conclude \(\phi\) has value \(v\).
  Property of having claimed \support{}.
  Expanded, property of holding that any independent check is satisfied.
  Any other reasoning about whether \(\phi\) has value \(v\) would conclude \(\phi\) has value \(v\).
\end{note}

\subsubsection{Resolving tension by additional ideas}
\label{sec:resolv-tens-addit}

\begin{note}[Strong closure]
  So, we have weak constraints on concluding.
  Is there a way to keep {\color{red} witnessing} by strengthening closure?

  The idea is that \zS{} relies on the possibility of an independent check.
  However, strong closure leaves open the option for denying independence.

  In certain cases, this seems viable.
  Arithmetic.
  Perhaps this does give everything.

  However, the other cases are more challenging.
  For, in these cases, specific premises.

  Chess.
  Winning strategy from board.
  So, then need to conclude would conclude winning strategy from board.
  Now, idea is that understanding basics already give you this.
  Therefore, as the reasoning from the board requires understanding rules, it also follows that before concluding from board, have already concluded winning strategy from board.

  Stepping back, relevant instance of reasoning requires certain premises.
  Possible obtaining those premises already involves concluding various things.
  If some of those conclusions are that one would not fail to conclude \dots
  Then, there is no space for failure to witness a \requ{1} of the relevant kind.

  For, there will be no `gap' between premises and conclusions of interest.
  Hence, no independent check on whether one would get conclusion from premises.

  This is really strong closure though.
  Intuitively, there is some gap between introduction and understanding.

  And this, I don't see as genuinely viable.
\end{note}

\subsection{Observations}
\label{sec:overview:observations}

\subsubsection{Ability}

\begin{note}
  \begin{figure}[H]
    \centering
    \saMtxInterpreted{}
    \caption{Distinction matrix with \aben{the}}
    \label{fig:saMtxInterpreted:outline}
  \end{figure}
\end{note}

\begin{note}
  Recap.

  Claiming support.
  Constraint.

  Ability.
  In order to be compatible, satisfy constraint.
  Either of three options.
  Basic, ignore this.
  Property. Incompatible with constraint.
  Witness. Compatible.

  Here, display the matrix.
  I think this is the easiest way to visualise what is going on.
\end{note}



%%% Local Variables:
%%% mode: latex
%%% TeX-master: "master"
%%% End:
