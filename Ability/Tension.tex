\chapter{Abstract tension}
\label{cha:tension-abstract}

\begin{note}[\requCluster{3}]
  We begin with the definition of a \cluster{}.
  \begin{definition}[A \requCluster{1}]
    \label{def:requCluster}
    Some collection of proposition-value-premises pairings \(\mathcal{C} = \{\pvp{\phi_{i}}{v_{i}}{\Phi_{i}}\}_{i}\) is a \emph{\cluster{}} with respect to some agent \vAgent{}('s epistemic state) just in case:
    \begin{itemize}
    \item
      For any \(\pvp{\phi_{i}}{v_{i}}{\Phi_{i}}\) in \(\mathcal{C}\), each \(\pvp{\phi_{j}}{v_{j}}{\Phi_{j}}\) in \(\mathcal{C}\) (such that \(j \ne i\)) is a \requ{} of \(\pvp{\phi_{i}}{v_{i}}{\Phi_{i}}\).
    \end{itemize}
    \vspace{-\baselineskip}
  \end{definition}

  Intuitively, a \cluster{0} is a collection of proposition-value-premises pairings such that every proposition-value-premises pairing is a \requ{} of every other proposition-value-premises pairing in the collection.
\end{note}

\paragraph*{Examples}

\begin{note}[Examples of \requCluster{0}]
  Focused primarily on \scen{1} where we don't have a \requCluster{}.
  Lots of \emph{only if} conditionals.
  In some of these \scen{0}, plausible the conditional goes both ways.
  Hence, get a \requCluster{}.

  More generally, fairly mundane.
  Basic abilities.
\end{note}

\begin{note}[Ability]
  Ability to reason in certain ways leads to \cluster{1}.
  However, not interested in ability in general, but rather relatively simple instances of ability concerning specific problem types.

  Arithmetic.
  Sudoku.
  Chess.

  Indeed, the latter pair for \requCluster{1}.
  For, different starting positions.
\end{note}

\begin{note}
  Finally, ability.

  \abgen{2} ability, \abspec{} ability.

  Claim support for having some \abgen{} ability.

  Now, here, simple cases.
  Basic arithmetic.
  Sudoku puzzles.
  Chess problems with winning strategies.

  Roughly the same.
  More broadly:

  Logic problems.

  Crossword.

  Reading novels up to a certain level.
  Here, if you can't read, then the writing is bad.

  Fluency.

  So, \abspec{} instances of the \abgen{} ability.
\end{note}

\begin{note}
  Indeed, narrow interest to \ragCluster{1}.

  \begin{definition}[\ragCluster{3}]
    For any cluster \(\mathcal{C}\), \(\mathcal{C}\) is a \emph{\ragCluster{}} if and only if:
    \begin{enumerate}
    \item
      There is some \(\pvp{\phi_{i}}{v_{i}}{\Phi_{i}}\) and \(\pvp{\phi_{j}}{v_{j}}{\Phi_{j}}\) such that \(\Phi_{i}\) and \(\Phi_{j}\) do not overlap.
    \end{enumerate}
    \vspace{-\baselineskip}
  \end{definition}

  A \ragCluster{}, then, is just a cluster where at least some distinct premises.

  Now, some caution.
  It may be the case that reason from some non-minimal collection of premises.
  Hence, some care when establishing \ragged{}.
  This means, argue that \cluster{} \emph{and} argue \cluster{} is \ragged{}.
\end{note}

\begin{note}[Relative \jag{1}]
  Given importance of \ragged{} and specific proposition-value-premises pairings of \ragged{}, terminology:

  \begin{definition}[Relative \jag{1} of a \ragCluster{}]
    \(\mathcal{C}\) some \ragCluster{}.
    \(\pvp{\psi}{v'}{\Psi}\) is a \emph{\jag{0}} relative to \(\pvp{\phi}{v}{\Phi}\) if \(\Psi\) differs from \(\Phi\).
  \end{definition}
\end{note}

\begin{note}[Pointed cluster]
    \begin{definition}[Pointed cluster]
    For any cluster \(\mathcal{C}\), \(\mathcal{C}\) is a \emph{pointed cluster} if and only if:
    \begin{enumerate}
    \item
      Conclusions are the same.
    \end{enumerate}
    \vspace{-\baselineskip}
  \end{definition}
\end{note}

%%% Local Variables:
%%% mode: latex
%%% TeX-master: "master"
%%% End:
