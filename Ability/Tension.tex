\chapter{Overview of Tension}
\label{sec:tension}

\begin{note}[Intro]
  We have discussed concluding.
  Introduced \csN{}.
  Revised~{\color{red} issue:Main} to \ESU{} and \EAS{}.
  And, two types of reasoning, \adA{} and \adB{}.
  Hints regarding ability.

  Task is now to bring these together develop tension.
\end{note}

\begin{note}[Basic idea of tension]
  Strategy.
  Identify some abstract phenomenon.
  Observe how leads to tension.
  Motivate instances of this phenomenon.
  Observe tension.

  Here, phenomenon is consequence of \csN{}.
  Hence, indirectly relies on assumptions concerning concluding.
  Split, either \ESU{} or \EAS{}.
  Construction involve \itp{} introduced with respect to \adB{}.
  Hence, discussion of \adB{} will both clarify and inform resolution to tension.

  Indeed, \adB{} will offer a more general perspective on the consequences of \csN{}.
\end{note}

\subsection{Sketch of tension}
\label{sec:overview-tension}

\begin{note}[Goal]
  To establish tension between \ESU{} and \csN{} we have the following goal:

  \begin{goal}
    \label{goal:tension}
    There are instances in which concluding \(\pv{\phi}{v}\) from \(\Phi\) seems to be an instance of \csN{} for \(\pv{\phi}{v}\) where:
    \begin{enumerate}[label=\arabic*., ref=\named{G\ref{goal:tension}:\arabic*}]
    \item
      \label{goal:tension:requ}
      There is some \(\pvp{\psi}{v'}{\Psi}\) which is a \requ{} of \(\pvp{\phi}{v}{\Phi}\) such that:
      \begin{enumerate}[label=\alph*., ref=\named{G\ref{goal:tension}:1\alph*}]
      \item
        \label{goal:tension:requ:conclude}
        It is not possible for the agent to conclude that they would conclude \(\pv{\psi}{v'}\) from \(\Psi\) \emph{without} concluding \(\pv{\psi}{v'}\) from \(\Psi\).
      \item
        \label{goal:tension:requ:no-reason}
        The agent does not witness any reasoning from \(\Psi\).
      \end{enumerate}
    \end{enumerate}
    \vspace{-\baselineskip}
  \end{goal}

  If there are instance of the type described by \autoref{goal:tension}, then \ESU{} and \csN{} are in tension.

  For, we will have some proposition-value-premises pairing \(\pvp{\phi}{v}{\Phi}\) such that an agent conclude \(\pv{\phi}{v}\) from \(\Phi\), and in doing so \csV{} for \(\pv{\phi}{v}\).
  And, by \autoref{goal:tension:requ} we have some \requ{} \(\pvp{\psi}{v'}{\Psi}\).

  Now, as \(\pvp{\psi}{v'}{\Psi}\) is a \requ{0} of concluding \(\pv{\phi}{v}\) from \(\Phi\), it must be the case that the agent either has concluded or simultaneously concludes that they would conclude \(\pv{\psi}{v'}\) from \(\Psi\) when concluding \(\pv{\phi}{v}\) from \(\Phi\).

  Further, the \requ{} \(\pvp{\psi}{v'}{\Psi}\) has two key properties.

  First, from~\autoref{goal:tension:requ:conclude}, the agent may only conclude that they would conclude \(\pv{\psi}{v'}\) from \(\Psi\) by concluding \(\pv{\psi}{v'}\) from \(\Psi\).
  Hence, by \ESU{} it must be the case that the agent witnesses reasoning from \(\Psi\) which concludes with \(\pv{\psi}{v'}\).

  Second, from~\autoref{goal:tension:requ:no-reason} the agent does not witness any reasoning from \(\Psi\).
  Hence, the agent does not witness reasoning from \(\Psi\) which concludes with \(\pv{\psi}{v'}\).

  From the above, tension.
  For, if concluding \(\pv{\phi}{v}\) from \(\Phi\) really is an instance of \csN{}, then the agent must have concluded \(\pv{\psi}{v'}\) from \(\Psi\) without witnessing reasoning from \(\Psi\) to \(\pv{\psi}{v'}\).
  However, if \ESU{} holds then it is not possible for the agent to conclude \(\pv{\psi}{v'}\) from \(\Psi\) without witnessing reasoning from \(\Psi\) to \(\pv{\psi}{v'}\).
  So, either the agent has not \csN{} for \(\pv{\phi}{v}\) or \ESU{} does not hold.
  Hence, either \csN{} or \ESU{} must be restricted in some way.

  Of course, the exact nature of the tension, and how it should be resolved, depends in part on the proposition-value-premises pairings which satisfy \autoref{goal:tension}.

  Indeed,~\autoref{goal:tension:requ:conclude} and~\autoref{goal:tension:requ:no-reason} are significant restrictions, and it is not clear from the abstract statement that these correspond to sufficiently interesting phenomenon.
\end{note}

\begin{note}
  Whether or not~\autoref{goal:tension:requ:no-reason} is satisfied will depend on the specifics of any given instance.
  For, we have placed no constraints on which premises an agent appeals to, nor what the contents of those premises may be.

  \autoref{goal:tension:requ:conclude}, by contrast, will follows from a collection of \requ{1} forming a `\cluster{}'.

  In short, a \cluster{} is a collection of proposition-value-premises pairings such that every proposition-value-premises pairing is a \requ{} of every other proposition-value-premises pairing in the collection.

  Intuitively, every proposition-value-premises pairing of the \cluster{0} is an independent check on every other proposition-value-premises pairing of the \cluster{0}.

  Indeed, we will argue that \autoref{goal:tension:requ:conclude} is satisfied whenever \(\pvp{\phi}{v}{\Phi}\) is a member of a \cluster{}.

  Finally, then, we seek some concrete instances.
\end{note}

\begin{note}[Ability]
  Here, our interest turns to ability.
  I take it as given that there are various instances of concluding \(\pv{\phi}{v}\) from \(\Phi\) for which the reasoning from \(\Phi\) to \(\pv{\phi}{v}\) is an instance of a general ability.

  For example, I conclude \(31 + 53 = 84\) from some premises, and the reasoning falls under my general ability to perform (simple) arithmetic.
  The specifics may differ, but there is sufficient overlap with concluding \(43 + 81 = 123\) and \(91 + 54 = 145\) to consider the reasoning of the same type.
  Indeed, \(532 - 91 = 441\), \(19 * 32 = 608\), and \(126/36 = 3.5\) may also fall under the same (general) ability.
  In other words, in concluding each equation I witness a specific instance of the general ability.

  Likewise, one may have the (general) ability to solve chess problems, complete \(\{ \text{Sudoku}, \text{KenKen}, \text{Nonogram}, \dots\}\) puzzles, or parse sentences in a given language.

  So long as you have the (general) ability, I expect each instance of witness the ability is intuitively an instance of \csN{}.
  I would not have concluded \(31 + 53 \ne 84\) and you would not have failed to identify the relevant winning strategy of some chess problem.

  However, in each of the examples of ability noted, every other specific instance of the general ability functions as an independent check on whether one has the relevant ability.
  I should make no mistake about \(85 + 21\) and you should make no mistake with the next chess problem.
  Granting, of course, that we do have the relevant abilities.

  Hence, it seems clear that if an agent \csV{} for \(\pv{\phi}{v}\) when concluding \(\pv{\phi}{v}\) from \(\Phi\) and the agent's reasoning is a specific instance of a general ability, then there at least various \requ{1} associated with concluding \(\pv{\phi}{v}\) from \(\Phi\).

  More generally, concluding \(\pv{\phi}{v}\) from \(\Phi\) from reasoning which is the specific instance of a general ability leads to a \cluster{}.
  And, it is perhaps already intuitive that one does not witness reasoning from the premises of at least some proposition-value-premises pairing in the \cluster{}.

  For example, it seems plausible that the configuration the chess board for any given problem forms of a premise of the agent's reasoning, but so long as one has not seen the relevant problem, it seems implausible that one has witnessed reasoning that includes the relevant configuration.

  Of course, it may seem equally implausible that an agent concludes there is some winning strategy for some configuration of a chess board they have not yet seen.
  However, this intuition should be carefully examined.

  With an unopened chess book by a reputable author before you, I expect you have no problem concluding that each of the solutions are correct.
  {
    \color{red}
    Or, that granting ability, conclude you would (also) conclude winning strategy or not for each chess piece.
  }
  Yet, you have not yet seen any of the solutions.
  So, in general, there seems no issue with an agent concluding there is some winning strategy for some configuration of a chess board they have not yet seen.

  Of course, in the case of the book there is a key premise:
  The book is written by a reputable author.
  However, if you wish to conclude there is a winning strategy via your own reasoning, then the above considerations take effect.%
  \footnote{
    And, indeed, may already hold with respect to the key premises.
    For, so long as you hold you have the general ability to reason about chess problems of the relevant kind, you have an independent check on whether the author really is reputable.
  }
\end{note}

\begin{note}[Moving on]
  So much for the rough outline of how~\autoref{goal:tension} leads tension between \csN{} and \ESU{}.
  Let us turn to the details.

  {
    \color{red}
    \begin{itemize}
    \item
      In~{\color{red} ???} we develop tension in the abstract.
      Our attention will be focused solely on drawing out how \csN{} and \ESU{} are in tension if cases of a certain type exist.
      In particular, we develop the notion of a \cluster{} introduced above through a handful of definitions, and state relevant consequences in a number of propositions.
    \item
      With an abstract understanding of how \csN{} is in conflict with \ESU{} given the existence of certain cases, we then turn to arguing for the existence of such cases in~\autoref{cha:concrete-tension}.
      Follow the sketch given, these cases will center around the relationship between the general ability to reason about certain types of problems and specific instances of the general ability.
    \item
      In~\autoref{sec:tension:adb} abstract a little.
      Relationship between general ability and \itp{1}.
      Broader understanding of what the tension amounts to, and the relationship between \EAS{} and \adB{}.
    \item
      Finally, in~\autoref{cha:overview:resolving-tension} we turn to resolving the tension and a number of latent issues.
    \end{itemize}
  }
\end{note}

\chapter{Abstract tension}
\label{cha:tension-abstract}

\begin{note}[Plan]
  In this section we establish how \csN{} and \ESU{} are in tension from an abstract perspective.
  Our immediate goal is to provide a clear characterisation of the kind of cases which will lead to tension between \csN{} and \ESU{}, if concrete instances of those cases exist.
  In the following section (\ref{cha:concrete-tension}) we will then argue for the existence of such cases.

  We begin by defining a \cluster{1} of \requ{1}, and will then refine the definition of a \cluster{1} to that of a \ragCluster{} to further clarify the kind of cases of interest.
  The two-step definition will allow us to observe exactly what is required of cases for tension.
\end{note}

\begin{note}[\requCluster{3}]
  We begin with the definition of a \cluster{}.
  \begin{definition}[A \requCluster{1}]
    \label{def:requCluster}
    Some collection of proposition-value-premises pairings \(\mathcal{C} = \{\pvp{\phi_{i}}{v_{i}}{\Phi_{i}}\}_{i}\) is a \emph{\cluster{}} with respect to some agent \vAgent{}('s epistemic state) just in case:
    \begin{itemize}
    \item
      For any \(\pvp{\phi_{i}}{v_{i}}{\Phi_{i}}\) in \(\mathcal{C}\), each \(\pvp{\phi_{j}}{v_{j}}{\Phi_{j}}\) in \(\mathcal{C}\) (such that \(j \ne i\)) is a \requ{} of \(\pvp{\phi_{i}}{v_{i}}{\Phi_{i}}\).
    \end{itemize}
    \vspace{-\baselineskip}
  \end{definition}

  Intuitively, a \cluster{0} is a collection of proposition-value-premises pairings such that every proposition-value-premises pairing is a \requ{} of every other proposition-value-premises pairing in the collection.
\end{note}

\begin{note}
  With the definition of a \cluster{} in hand, we are ready to state our first proposition.

  \begin{proposition}[\cluster{3} and \csN{0}]
    \label{prop:cluster:csN}
    Suppose \(\mathcal{C}\) is a \requCluster{0} with respect to an agent \vAgent{}('s epistemic state).
    And, let \(\pvp{\phi}{v}{\Phi}\) be some proposition-value-premises pairing in \(\mathcal{C}\).

    \begin{itemize}
    \item
      \vAgent{} \csV{} for \(\pv{\phi}{v}\) when concluding \(\pv{\phi}{v}\) from \(\Phi\) only if \emph{either}:
      \begin{itemize}
      \item
        For any \(\pvp{\psi}{v'}{\Psi}\) in \(\mathcal{C}\):
        \begin{itemize}
        \item \vAgent{} has at some point in the past concluded that \vAgent{} would conclude \(\pv{\psi}{v'}\) from \(\Psi\).
        \item
          When concluding \(\pv{\phi}{v}\) from \(\Phi\), \vAgent{} simultaneously concludes that \vAgent{} would conclude \(\pv{\psi}{v'}\) from \(\Psi\).
        \end{itemize}
      \end{itemize}
    \end{itemize}
  \end{proposition}

  \Autoref{prop:cluster:csN} follows directly from~\izetaS{} and~\autoref{def:requCluster}.

  \begin{argument}
    Suppose \(\mathcal{C}\) is a \requCluster{0} with respect to an agent \vAgent{}('s epistemic state).
    Let \(\pvp{\phi}{v}{\Phi}\) be some proposition-value-premises pairing in \(\mathcal{C}\).
    And, let \(\pvp{\psi}{v'}{\Psi}\) be some proposition-value-premises pairing in \(\mathcal{C}\).

    By~\autoref{def:requCluster} we have that \vAgent{} concluding that \vAgent{} would conclude \(\pv{\psi}{v'}\) from \(\Psi\) is a \requ{} of concluding \(\pv{\phi}{v}\) from \(\Phi\).
    And, by~\izetaS{}, it must be the case that either:
    \begin{itemize}
    \item
      \vAgent{} has concluded that they would conclude \(\pv{\psi}{v'}\) from \(\Psi\), ref{idea:Zs:overview:requ-sat:Past}.
      Or,
    \item
      In concluding \(\pv{\phi}{v}\) \vAgent{} simultaneously concludes \(\pv{\psi}{v'}\) from \(\Psi\), ref{idea:Zs:overview:requ-sat:Pres}.
    \end{itemize}
    Hence, we have established~\ref{prop:cluster:csN}.
  \end{argument}
\end{note}

\begin{note}
  \begin{proposition}
    \label{prop:cluster:simul}
    Need to do everything in a cluster at the same time.
  \end{proposition}

  \begin{argument}
    Straightforward.
    For, anything is a \requ{} for any other.
    So, only \csV{} at the same time.
  \end{argument}
\end{note}

\begin{note}[No \(\gamma\)]
  Consequence:

  \begin{corollary}
    \label{prop:cluster:no-general}
    No general \(\pvp{\gamma}{v}{\Gamma}\) within cluster.
  \end{corollary}

  \begin{argument}
    Quickly, because of~\ref{prop:cluster:simul}.
    Only \(\gamma\) at same time as others.

    In more detail.
    For, by assumption, \requ{} means that it's possible for the agent to conclude.
    By the each other \requ{} functions as a check on \(\gamma\).
    If haven't figured out each individual, then the general is in question.
  \end{argument}
\end{note}

\begin{note}
  \autoref{prop:cluster:simul} is kind of wild.
  Though, this doesn't prevent reasoning, and then only getting \csN{} after the fact.
  However, this does prevent ruling out conflict when concluding.

  Of course, concluding each and then \csVImp{}, plausible.
  As, have not found an issue.

  Also, may conclude all from premises.
  For, special cases of \cluster{} in which all premises are the same.
  If so, then no clear tension.
  For, agent reasons from premises, well all the same premises.
  So, conclude simultaneously.
\end{note}

\begin{note}
  Indeed, narrow interest to \ragCluster{1}.

  \begin{definition}[\ragCluster{3}]
    For any cluster \(\mathcal{C}\), \(\mathcal{C}\) is a \emph{\ragCluster{}} if and only if:
    \begin{enumerate}
    \item
      There is some \(\pvp{\phi_{i}}{v_{i}}{\Phi_{i}}\) and \(\pvp{\phi_{j}}{v_{j}}{\Phi_{j}}\) such that \(\Phi_{i}\) and \(\Phi_{j}\) do not overlap.
    \end{enumerate}
    \vspace{-\baselineskip}
  \end{definition}

  A \ragCluster{}, then, is just a cluster where at least some distinct premises.
  Hence, avoid issue where same premises allow simultaneous conclusion, and fail to establish tension with \ESU{}.

  \begin{proposition}
    With \ragCluster{} concluded previous or violate \ESU{}.
  \end{proposition}

  Now, some caution.
  It may be the case that reason from some non-minimal collection of premises.
  Hence, some care when establishing \ragged{}.
  This means, argue that \cluster{} \emph{and} argue \cluster{} is \ragged{}.
  Again, without any clear bounds on premises, this argument is non-deductive.
  However, plausible in various cases.
\end{note}

\begin{note}[Relative \jag{1}]
  Given importance of \ragged{} and specific proposition-value-premises pairings of \ragged{}, terminology:

  \begin{definition}[Relative \jag{1} of a \ragCluster{}]
    \(\mathcal{C}\) some \ragCluster{}.
    \(\pvp{\psi}{v'}{\Psi}\) is a \emph{\jag{0}} relative to \(\pvp{\phi}{v}{\Phi}\) if \(\Psi\) differs from \(\Phi\).
  \end{definition}

  \csN{} while presence of some \jag{} when premises are no general.
\end{note}

\begin{note}
  Suppose \ragged{}, no prior conclusion for some \jag{}.
  Then, if \csN{}, violation of \ESU{}.

  For, only \csN{} simultaneously.
  \jag{}, some not from current premises.
  And, no previous, so not from previous premises.
  Hence, \csN{} from premises of the \jag{}.

  Hence, need instances of \ragged{} with no prior conclusion for some \jag{}.
\end{note}

\begin{note}
  \begin{proposition}
    If \ragCluster{} and no prior conclusion for some \jag{}.
    Either:
    \begin{itemize}
    \item
      \ESU{} does not hold in general.
    \item
      No \csVImp{} for any proposition-value-premises pairing in cluster.
    \end{itemize}
      \begin{argument}
    More-or-less immediate from previous.
  \end{argument}
  \end{proposition}
\end{note}

\begin{note}
  Abstract tension, then, follows if there are instances of \ragCluster{1} with no prior conclusion for some \jag{}.
\end{note}

\begin{note}[Pointed cluster]
    \begin{definition}[Pointed cluster]
    For any cluster \(\mathcal{C}\), \(\mathcal{C}\) is a \emph{pointed cluster} if and only if:
    \begin{enumerate}
    \item
      Some conclusions are the same.
    \end{enumerate}
    \vspace{-\baselineskip}
  \end{definition}
\end{note}

\chapter{Concrete tension}
\label{cha:concrete-tension}

\begin{note}
  \color{red}
  The key observation is that ability gives rise to \ragCluster{1}.
  Hence, no \csN{}.

  Why does this lead to tension?
  Well, in certain cases, it seems clear that an agent has the ability, and the ability ensures that the agent would not have failed to conclude.
\end{note}

\begin{note}
  Seen how to develop tension in the abstract.
  Now, concrete.
  Ability, as we have seen.
\end{note}

\begin{note}[Ability]
  Ability to reason in certain ways leads to \cluster{1}.
  However, not interested in ability in general, but rather relatively simple instances of ability concerning specific problem types.

  Arithmetic.
  Sudoku.
  Chess.

  Indeed, the latter pair for \requCluster{1}.
  For, different starting positions.
\end{note}

\begin{note}
  Finally, ability.

  General ability, specific ability.

  Claim support for having some general ability.

  Now, here, simple cases.
  Basic arithmetic.
  Sudoku puzzles.
  Chess problems with winning strategies.

  Roughly the same.
  More broadly:

  Logic problems.

  Crossword.

  Reading novels up to a certain level.
  Here, if you can't read, then the writing is bad.

  Fluency.

  So, specific instances of the general ability.
\end{note}

\begin{note}
  Well, conclude that you have the general ability, but also claim support.
  You don't need to go through specific instances.
  In these cases, fail to be an independent check.
  You not fail to reach the relevant conclusion.

  Would not reason to some incorrect summation.
  Would not fill out the Sudoku incorrectly.
  Would not fail to find a winning strategy.

  Of course, failures of performance, but not failures of competence.

  Intuitively, satisfied all \requ{1}.
\end{note}


\begin{note}
  Key argument, then, is that only satisfy a \requ{} by concluding \(\psi\) from \(\rho\).

  For, some other premise, get \(\psi\) from \(\rho\).
  Well, getting \(\psi\) from \(\rho\) is still a \requ{} for this.
\end{note}

\begin{note}
  Briefly stated.

  Specific instances, these introduce \requ{}.
  However, some reasoning.
  Can't jump to general to get rid of \requ{}, as this is forbidden.
  Further, if reason from some distinct set of premises, then still a \requ{}.

  If independent reasoning gets that specific instance of general ability, then doing the reasoning is still an independent check on this.

  So, the problem here is that need to ensure that would conclude \(\psi\) has value \(v'\) from certain premises.
  If appeal to any distinct premises, then failure to claim support.

  Hence, \ESU{} and \ideaCS{}, then no getting general ability without witnessing reasoning for specific instances.

  Core of the tension.
  Always some independent check with distinct premises with specific instance of general ability.

  So, either, allow to bypass independent check.
  Or, do not require witnessing reasoning from premises to conclusion.
\end{note}

\subsubsection{\adB{}}
\label{sec:tension:adb}

\begin{note}
  Now, these cases of ability.
  Work with \adB{}.
\end{note}

\begin{note}
  Here, it turns out that obtaining the \itp{} is just concluding.
\end{note}

\begin{note}[Conditionals, a point of interest]
  More generally, we have the following result.%
  \footnote{
    So long as we do not add~\autoref{notion:overview:requ:pool:method} to~\autoref{notion:overview:requ:pool} of the notion of a \requ{}.
    If so, then result will be constrained accordingly.
  }
  If appeal to some conditional which links premises to a conclusion, such that agent may reason from premises to conclusion, then the agent has always concluded premises from conclusion.

  This is interesting.
  If agent has concluded from conditional in this way, then in effect the conditional drops out as a premise.

  If \(\Sigma, \phi \rightarrow \psi \vdash \psi\) then \(\Sigma, \phi \vdash \psi\).

  If \(\Sigma, \phi \vdash \phi \rightarrow \psi\) then \(\Sigma, \phi \vdash \psi\).

  The second is close to a restricted form of the deduction theorem.

  Note, this is only when the conditional has a special \requ{}.

  In cases where no checking the conditional, then the elimination does not hold.

  Whether anything of independent interest follows from this, unclear.
  One example, responsibility.
  Then, to another person, not only reasoning with conditional, but also full reasoning.
  No way to distinguish between the two cases.
  Or, from the converse perspective, no need to add any additional clauses to account of responsibility.
  However, issues of this kind are far beyond the scope of this document.
\end{note}

\begin{note}
  Indeed, given the constraints of \itp{1}, conditional will satisfy just in case it is an \itp{}.
  Here, need that \(\pv{\mu}{v}\) is not in \(\Phi\) to ensure that \itp{0} is redundant.
\end{note}

\subsubsection{Thoughts}
\label{sec:thoughts}

\begin{note}[Difficulty]
  Here, concrete instantiation of abstract structure.
  Question is, does the agent really \csN{}?
  Intuitively, it seems agent does.
  Adopt \stance{}, I am confident I would not reason otherwise.
  However, this is from adopting a \stance{}.
  And, whether or not concluded or \csVed{} is independent of \stance{}.
  Instead, question is, granting intuition, does this \stance{} reflect on our theory.

  How close these abstractions map to more commonsense reasoning.
  Motivated in part.
  Whether \csN{} seems natural.

  Seems to me, key question is the conditional link.
  It's identifying these two things.
  Abstracts from specific cases, but gives the general principle.
  So long as \csN{}, then these two things coincide.

  Of course, senses of `concluding'.
  However, difficulty in running `concluding' too close to the agent's \stance{} on what the have concluded.
\end{note}

\subsection{Tension}
\label{sec:overview:tension:subsection}

\subsubsection{A different path to tension}
\label{sec:diff-path-tens}

\begin{note}
  We have developed tension with respect to the definition of a \requ{}.
  However, noted that an additional constraint may be placed on the notion of a \requ{}.
  Same type of reasoning.

  If this is the case, then general ability will not be part of \cluster{}.
  Hence, the argument as given will no go through.

  May seem a compelling alternative.
  There are two issues.

  First, consequence.
  General ability functions as a premise in every case of witnessing a specific instance general ability.
  For, if just specific instance, then part of a \cluster{}.

  One way to resist this is to fine grain specific instances.
  However, this is very puzzling.

  Still, when paired against \EAS{} this may not be so bad.

  Hence, we turn to the more substantial issue.
  \csN{2} for the general ability.

  Need to have \csVed{0}.
  For, else, different conclusion about general ability, and hence no \csVImp{} for specific cases.

  So, \csVImp{} for general ability.
  Here, each specific instance is a \requ{}.
  And, we are back to the original problem.

  So, even if you grating general ability as a premise, still a question as to how this premise is obtained.

  Some wiggle room is left.
  Here, informed that one has the general ability.
  But, then, any attempt to witness the specific ability would not show mistake.
  And, no isolated conclusion.

  These, quite puzzling to defend.
\end{note}


\chapter{Resolving tension}
\label{cha:overview:resolving-tension}

\begin{note}
  Note, the tension is not about whether \(\phi\) has value \(v\).
  Instead, the tension is about whether the agent would have a certain property if they were to conclude \(\phi\) has value \(v\).
  Property of having claimed \support{}.
  Expanded, property of holding that any independent check is satisfied.
  Any other reasoning about whether \(\phi\) has value \(v\) would conclude \(\phi\) has value \(v\).
\end{note}

\begin{note}
  Returning to \EAS{}.
  Specific instances of the general ability.
  In this sense, the instances of \EAS{} we argue for are narrow.
  Need strong sign that the agent has the general ability.

  Further.
  It does not state that an agent having claimed support that they have the ability to reason to some conclusion is \emph{always permissible} to claim support for the conclusion by appealing to some premises that do not form part of the agent's reasoning.
  Instead, it states that \emph{may be permissible} for the agent claim support in a certain way.

  In various respects, these aren't particularly interesting cases.
  However, the goal is to argue that such cases exist.
  Whether these are constrained to the type of cases we consider for the argument is a further question.

  There may be more interesting cases, but given that \ESU{} is incompatible with all such cases, I see no compelling reason to explore such cases without \emph{first} motivating a rejection of \ESU{}.
\end{note}

\begin{note}
  Also suggests that the content of general ability is somewhat interesting.
  For, the content is itself general.
  It is a conclusion that ranges over all specific instances of the general ability.
\end{note}

\begin{note}[Terminology]
  So, the upshot of this is that an agent concludes various things in certain cases.
  In concluding \(\phi\), also conclude \(\psi,\dots\).
  And, in cases of interest, because of generality of the reasoning.

  This is somewhat puzzling.
  Though, I think less puzzling than first appears.
  Concluding \(\phi\) has value \(v\) is nothing special.
  Of course, the agent only explicitly concludes a handful of things, but allowing the generality is nothing that different from equivalences.

  It also doesn't follow that any of the additional properties of the reasoning, if any, are carried over to any \requ{1}.
  Is just about concluding.
  Here, then, various ways to keep the intuition for the positive answer.
  There may be various things that are exclusive to witnessing reasoning from premises to conclusion.
  However, distinct from concluding.

  Still, stronger than being committed.
  Ranges over any implication.
  Conclude no winning strategy, then also conclude various other chess things.
  However, committed, but do not necessarily conclude that X is going to lose the game.

  Concluding is still of interest.
  Or, as noted, `reason', in the weak term.
\end{note}

\subsubsection{Resolving tension by rejecting ideas}
\label{sec:resolv-tens-reject}

\begin{note}
  Now, possible to resolve tension various ways.
  Reject \ideaCS{}, \csN{} is of no interest.
  Reject \ESU{}, not witnessing is ok.
  Reject claiming support for general ability.

  Or, any combination of the above.

  Interest is in rejecting \ESU{}.
  So long as \(\psi\) having value \(v'\) follows from some premises, then the reasoning doesn't matter.
  \csV{2} for \(\psi\) having value \(v'\) from those premises, given the possibility of witnessing reasoning.
\end{note}

\subsubsection{Resolving tension by additional ideas}
\label{sec:resolv-tens-addit}

\begin{note}[Strong closure]
  So, we have weak constraints on concluding.
  Is there a way to keep \ESU{} by strengthening closure?

  The idea is that \csN{} relies on the possibility of an independent check.
  However, strong closure leaves open the option for denying independence.

  In certain cases, this seems viable.
  Arithmetic.
  Perhaps this does give everything.

  However, the other cases are more challenging.
  For, in these cases, specific premises.

  Chess.
  Winning strategy from board.
  So, then need to conclude would conclude winning strategy from board.
  Now, idea is that understanding basics already give you this.
  Therefore, as the reasoning from the board requires understanding rules, it also follows that before concluding from board, have already concluded winning strategy from board.

  Stepping back, relevant instance of reasoning requires certain premises.
  Possible obtaining those premises already involves concluding various things.
  If some of those conclusions are that one would not fail to conclude \dots
  Then, there is no space for \requ{1} of the relevant kind.

  For, there will be no `gap' between premises and conclusions of interest.
  Hence, no independent check on whether one would get conclusion from premises.

  This is really strong closure though.
  Intuitively, there is some gap between introduction and understanding.
  This is in part why \csN{} is of interest.
  A sufficiently strong closure principle would need to rule out possibility of failing to \csN{} while having the possibility to reason to the correct answer.
  And this, I don't see as genuinely viable.

  And, if not viable, then tension arises when an agent goes from `merely' concluding to also \csN{}.
\end{note}

\subsubsection{Resolving tension by terminology}
\label{sec:resolv-tens-term}

\begin{note}[`Concluding']
  Reject some of the assumptions regarding `concluding'.
  Or, more generally, recast \csN{} in terms of something like commitment.

  Argument has been developed with the terminology of `concluding' as this seems natural.

  Even if not concluding, take the result to be sufficiently interesting.
\end{note}


\subsection{Observations}
\label{sec:overview:observations}

\subsubsection{Ability}

\begin{note}
  \begin{figure}[H]
    \centering
    \saMtxInterpreted{}
    \caption{Distinction matrix with \aben{the}}
    \label{fig:saMtxInterpreted:outline}
  \end{figure}
\end{note}

\begin{note}
  Recap.

  Claiming support.
  Constraint.

  Ability.
  In order to be compatible, satisfy constraint.
  Either of three options.
  Basic, ignore this.
  Property. Incompatible with constraint.
  Witness. Compatible.

  Here, display the matrix.
  I think this is the easiest way to visualise what is going on.
\end{note}

\paragraph{Deviant causal chains}

\begin{note}
  Do these really matter in the case of reasoning?
\end{note}

\paragraph{Closure principles}

\begin{note}
  No doubt, already observed.
  This does lead to a closure principle, constrained by what reasoning it is possible for the agent to witness.

  There are two perspectives.
  First, leading to tension.
  Second, no need to reason.
\end{note}

%%% Local Variables:
%%% mode: latex
%%% TeX-master: "master"
%%% End:
