\chapter{\ninf{2}}
\label{cha:embed}

\begin{note}
  \requ{3}.
  \tC{2} motivates \requ{1}
  Still, \requ{1} are given by conditional.
  Issue about agency.

  \requ{3}, something about agent's psychology.
  However, this does not get agency.
  Problem is clear from \citeauthor{Davidson:1973vd}.
  Here, psychological facts and a deviant causal chain.

  So, is it the case that psychological facts which give \requ{} are sufficiently connected to agency agent exerts when agent concludes?
\end{note}

\section{\ninf{2}}
\label{sec:infl}

\begin{note}
  Consider the following \scen{0}:

  \begin{scenario}[Apples]%
    \label{scen:apples}%
    Grey is walking to town.
    Grey notices a bicycle for sale.
    Grey purchases the bicycle, gets on the bicycle, and starts cycling to town.
  \end{scenario}

  \noindent%
  The basic structure of \autoref{scen:apples} is simple:
  \begin{itemize}[noitemsep]
  \item
    An initial event is in progress. \hfill (Grey is walking to town.)
  \item
    Some other event happens. \hfill (Grey notices and purchases the bicycle.)
  \item
    The initial event is not longer in progress. \hfill (Grey is cycling to town.)
  \end{itemize}

  \noindent%
  Our interest is with the detail:

  \begin{itemize}
  \item
    The initial event is not longer in progress due to an action the agent performed.
  \end{itemize}

  \noindent%
  When an agent stops an event in progress by performing some action, we say the agent exerted `\ninf{}' over the initial event.
  Hence, Grey exerted `\ninf{}' over whether or not they were walking to town.

  Still, an agent does not need to exert \ninf{} to have \ninf{}.
  Grey may have continued walking to town after noticing the bicycle for sale.

  We define an agent having \ninf{} over an event as follows:

  \begin{definition}[\ninf{2}]
    \label{def:ninf}
    \cenLine{
      \begin{VAREnum}
      \item
        Agent: \vAgent{}
      \item
        Event: \(e\)
      \item
        Event description: \(\alpha\)
      \item
        \mbox{ }
      \end{VAREnum}
    }

    \begin{itemize}
    \item
      \vAgent{} has \ninf{} over whether or not \(\text{Prog}(e, \alpha)\) is true.
    \end{itemize}

    \emph{If and only if}

    \begin{itemize}
    \item
      Both~\ref{def:ninf:action} and~\ref{def:ninf:prog} are true:
      \begin{enumerate}[label=\alph*., ref=(\alph*)]
      \item
        \label{def:ninf:action}
        There is some action \(b\) that \vAgent{} may (immediately) do.
      \item
        \label{def:ninf:prog}
        \(\text{Prog}(e', \alpha)\) is not true if \vAgent{} does \(b\).
      \end{enumerate}
    \end{itemize}
    \vspace{-\baselineskip}
  \end{definition}

  \noindent%
  Applied to \autoref{scen:apples}:

  \begin{itemize}[noitemsep]
  \item
    The agent is Grey.
  \item
    The event \(e\) spans some period of time which starts when or after Grey set out for town, and includes Grey noticing the bicycle for sale.
  \item
    \(\alpha\) is the description `Grey walks to town'.
  \item
    An instance of \(b\) is the action `Grey purchases and rides the bicycle'.
  \end{itemize}

  \noindent%
  Of course, with additional detail given, other actions may satisfy \(b\) action of \autoref{def:ninf}.
  For example, Grey may have turned back home or raided the local orchard.
\end{note}

\begin{note}
  The following table lists a few additional examples of \ninf{0}:

  \begin{center}
    \bgroup
    \def\arraystretch{1.125}
    \begin{tabular}{R{.45\textwidth} L{.45\textwidth}}
      Event description \(\alpha\) & Action \(b\) \\
      \hline
      Finish a book late at night. & Turn off the lights. \\
      Listen to a (full) speech. & Leave before the speech is over. \\
      Play a game of chess. & Flip the board in frustration.
    \end{tabular}
    \egroup
  \end{center}

  \noindent%
  Having \ninf{} over whether or not event is in progress is common, but there are also various events an agent does not have \ninf{} over.
  Here, event description and background condition which blocks.

    \begin{center}
    \bgroup
    \def\arraystretch{1.125}
    \begin{tabular}{R{.45\textwidth} L{.45\textwidth}}
      Event description \(\alpha\) & Background condition \\
      \hline
      Take a shower. & Has been in the shower for a while. \\
      Play chequers & Has made a few moves. \\
      A bank robbery & Is in prison. \\
    \end{tabular}
    \egroup
  \end{center}

  \noindent%
  We briefly comment on the first two examples:

  \begin{itemize}
  \item
    If you've been in the shower for a while, then whatever happens it remains the case that you were taking a shower.
  \item
    Likewise, if you have made a few moves, then what events it remains the case you were playing chequers.
    The contrast between playing chequers and playing a game of chess is the scope of the event.
    One may play chequers (or chess) without playing a full game of chequers (or chess).
  \end{itemize}
\end{note}


\begin{note}%
  \nocite{Peacocke:2021aa}%
  \ninf{2} is a general phenomenon.
  Our specific interest in with whether an agent has \ninf{} over whether they are \emph{concluding} \(\pv{\phi}{v}\) from \(\Phi\).
  In particular, \requ{1} rely on an agent having \ninf{} over concluding \(\pv{\phi}{v}\) from \(\Phi\).
  Still, before turning to \requ{1} we establish there are cases in which has \ninf{} over concluding \(\pv{\phi}{v}\) from \(\Phi\).

  \begin{proposition}[\ninf{2} over concluding]%
    \label{prop:ninfConcl}%
    There are instances in which an agent exerts \ninf{0} over whether or not they are concluding \(\pv{\phi}{v}\) from \(\Phi\).
  \end{proposition}

  \begin{argument}{prop:ninfConcl}
    Consider an agent \vAgent{}, some proposition-value pair \(\pv{\phi}{v}\), \pool{} \(\Phi\), and event \(e\).

    We make two observations:

    \begin{enumerate}[label=\arabic*., ref=(\arabic*), noitemsep]
    \item
      By \assuPP{2}, in order for \(e\) to be an event in which \vAgent{} is concluding \(\pv{\phi}{v}\) from \(\Phi\), there must be some \progAdj{0} development \(e^{\sharp}\) of \(e\) such that \(e^{\sharp}\) is an event in which \vAgent{} concludes \(\pv{\phi}{v}\) from \(\Phi\).
    \item
      There are often actions which if performed by \vAgent{} would result in an event \(e^{+}\), where \(e^{+}\) is a development of \(e\), and there is no \progAdj{0} development of \(e^{+}\) in which \vAgent{} concludes \(\pv{\phi}{v}\) from \(\Phi\).

      In particular it may be the case that:
      \begin{enumerate}[label=\alph*., ref=(\alph*)]
      \item
        \label{prop:ninfConcl:obs:2:concludes}
        \(e^{+}\) does not develop into an event in which \vAgent{} \emph{concludes \(\pv{\phi}{v}\) from \(\Phi\)}.
      \item
        \label{prop:ninfConcl:obs:2:diffPoll}
        \(e^{+}\) does develop into an event in which \vAgent{} concludes \(\pv{\phi}{v}\) \emph{from \(\Phi\)}.
      \end{enumerate}
    \end{enumerate}

    \noindent%
    We illustrate \ref{prop:ninfConcl:obs:2:concludes} and \ref{prop:ninfConcl:obs:2:diffPoll} turn:

    \begin{itemize}
    \item
      Suppose \vAgent{} is playing a game of chess against an opponent.
      The opponent claims checkmate, and \vAgent{} beings to determine whether or not they are in checkmate.

      Still, \vAgent{} may flip the board, scattering pieces everywhere.
      And, with pieces scattered everywhere, \vAgent{} does not have sufficient resources to determine whether or not they were in checkmate.

      Hence, while determining whether or not they are in checkmate, \vAgent{} has \ninf{} over whether they are concluding they are in checkmate.
    \end{itemize}

    \begin{itemize}
    \item
      Suppose \vAgent{} is working on a homework problem which asks for solutions to the quadratic equation \(2x^{2} - x - 1 = 0\).
      \vAgent{} starts to work on the problem via factorisation, and is making slow, but steady, progress.

      Still, \vAgent{} understands the quadratic formula.
      And, \vAgent{} may use the quadratic formula to solve the quadratic equation.
      However, if \vAgent{} uses the quadratic formula, the relevant \pool{} \(\Phi\) must expand to include the details regarding the quadratic formula.

      Hence, while working on the quadratic equation, \vAgent{} has \ninf{} over whether they are concluding \(x = 1\) or \(x = -\sfrac{1}{2}\) from \(\Phi\).
      For, the agent may turn to concluding \(x = 1\) or \(x = -\sfrac{1}{2}\) from some expanded or variant \pool{} \(\Phi'\).
    \end{itemize}
    %
    Abstracting a little:
    There are cases in which there is nothing which prevents an agent from making unavailable some proposition-value pair necessary for the agent to reach some conclusion.
    And, there are cases in which an agent may chose to appeal to additional proposition-value pairs in order to reach a conclusion.%
    \footnote{
      It does not follow the agent had a choice over how event develops, if the agent does not exert \ninf{}.
      It may be that if the agent continued the event, they would have concluded \(\pv{\phi}{v}\) from \(\Phi\).
      And, may remain the case that \(\pv{\phi}{v}\) from \(\Phi\) is a \fc{}.
    }
  \end{argument}
\end{note}

\section{\ninf{2}, \requ{1}, and agency}
\label{sec:ninf2-requ1-agency}

\begin{note}
  \color{red}

  So, agent doesn't exert \ninf{} given presence of a \requ{}.

  Is this sufficiently tied to agency?

  I don't have an argument.

  However, I think it is plausible.

  For example, consider temptation.

  In particular, BRATMAN.

  Here, prior intention constrains admissible actions.

  Agent resisted temptation.
  But, the prior intention.
  When the agent resists temptation, the agent isn't really doing anything.
  This is the point.
  For, else, the intention is redundant.

  So, back to \requ{1}.
  Built up stuff so that \fc{1}.
  E.g.\ understanding of propositional logic.
  The same things persist.
\end{note}

\begin{note}
  \color{blue}

  Or, y'know, take this as an argument against BRATMAN.
  Whatever, idgaf.              %
\end{note}

\begin{note}
  Plausible that get \ninf{} due to \requ{}.

  Here, return to SQUISH.

  Did everything from a third person perspective.

  However, agent adopts this perspective.

  Look, who know what really happens.
  But, this seems quite reasonable.
\end{note}

\begin{note}
  So, in these \scen{1}, \requ{}, not \fc{}, and agent stops.
\end{note}

\begin{note}
  Issue.
  Something other than \fc{}.
\end{note}


\begin{note}
  Feedback.

  If concern over whether \fc{}, then not \fc{}.

  Think about lost keys.

  Maybe there's somewhere else to look.
  But, then, feedback.
  Given possibility, this blocks.

  Same with SQUISH.
  As soon as doubt \fc{}, then no longer \fc{}.
\end{note}

%%% Local Variables:
%%% mode: latex
%%% TeX-master: "master"
%%% TeX-engine: luatex
%%% End:
