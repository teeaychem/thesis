
\chapter{Leftovers}
\label{cha:leftovers}


\section{\fc{3}}
\label{sec:fc3}


\begin{note}
  To what extent does the substance of the argument given rely on \fc{1}?

  The idea is central.

  However, \qWhy{} and \qWhyV{} only concern \ros{1}.

  Function of \fc{1} is to get \ros{} without \wit{}.
  In turn, appeal to \fc{1} when defining \requ{1} and so on is to have something which gets \ros{}.

  Possible that something other than a \fc{} does this work.
  Indeed, what matters is whatever it is that ensures \(\pv{\phi}{v}\) is a \fc{} from \(\Phi\), rather than \(\pv{\phi}{v}\) \emph{being} a \fc{} from \(\Phi\).
\end{note}



\section{Other agents}
\label{sec:other-agents}


\begin{note}
  Mathematic proofs and transferability.

  To get \requ{1}, focused on \tC{}.
  However, other properties that constrain whether reasoning is of some type.
  If these get \ros{}, then answers to \qWhy{} expand further.
\end{note}

%%% Local Variables:
%%% mode: latex
%%% TeX-master: "master"
%%% TeX-engine: luatex
%%% End:
