\chapter{Embedding \ros{1}}
\label{cha:embed}


\begin{note}
  In this chapter we return to \ros{1} in order to introduce and discuss a useful feature of abstracting to \ros{1} which we term `embedding'.

  Embedding have two functions in this document.
  \begin{itemize}
  \item
    Clarify a concern regarding the counterexamples that we will develop to \issueConstraint{}.
  \item
    Capture additional accounts of events in which an agent concludes form the literature.
  \end{itemize}

  Personally, I find the concern quite concerning, and embedding allows a clear account of what this concern is.
  Still, an understanding of the concern is not required background in order to understand the way in which counterexamples are develop.
  Hence, you may safely skip this chapter.
\end{note}

\begin{note}
  If you decide to read this chapter, it is split into three sections:

  \begin{enumerate}[label=]
  \item
    \TOCLine{cha:var:ros:Emb}

    Introduce distinction, define, and intuition.
  \item
    \TOCLine{sec:wrangling}

    The concern.
  \item
    \TOCLine{cha:var:sec:embedding}

    A handful of accounts from the literature.
  \end{enumerate}
\end{note}

\section{Embedding \ros{1}}
\label{cha:var:ros:Emb}

\begin{note}
  \autoref{cha:var:ros} detailed the three key ideas by which we understand \ros{1} in this document.

  The present section concerns, for some arbitrary proposition-value pair \(\pv{\phi}{v}\) and \poP{} \(\Psi\), the distinction between:

  \begin{enumerate}[label=\arabic*., ref=(\arabic*)]
  \item
    \label{Embed:no}
    A \ros{0} between \(\pv{\psi}{v'}\) and \(\Psi\).
  \item
    \label{Embed:yes}
    A \ros{0} between \(\Phi\) and \(\pv{\phi}{v}\), where:
    \begin{itemize}
    \item
      \(\Phi\) contains the proposition-value-premises pairing:
      \begin{itemize}
      \item
        \(\pv{\text{A \ros{} between }\pv{\psi}{v'}\text{ and }\Psi}{\text{True}}\)
      \end{itemize}
    \end{itemize}
  \end{enumerate}

  \ref{Embed:no} is a \ros{} between \(\pv{\psi}{v'}\) and \(\Psi\).
  Likewise, \ref{Embed:yes} involves a \ros{} between \(\pv{\psi}{v'}\) and \(\Psi\).
  However, the \ros{} between \(\pv{\psi}{v'}\) and \(\Psi\) is itself a premise in a \ros{} between \(\pv{\phi}{v}\) and \(\Phi\).
  In this respect, we will say the \ros{} between \(\pv{\psi}{v'}\) and \(\Psi\) is \emph{embedded} within a \ros{} --- specifically the \ros{} between \(\pv{\phi}{v}\) and \(\Phi\).
\end{note}

\begin{note}
  Embedding of this kind will be:
  \begin{itemize}[noitemsep]
  \item
    Useful for relating the variations of \qWhy{} and \qHow{} to theories of concluding, or related phenomena. And,
  \item
    Important for clarifying a difficult for developing counterexamples to \issueInclusion{}.
  \end{itemize}
\end{note}

\subsection{Definitions}
\label{cha:var:ros:Emb:defs}

\begin{note}
  In full, we define embedding in the following way:%
  \footnote{
    We assume, in general, that if a proposition \(\phi\) includes some other proposition \(\phi'\), then if \(\pv{\phi}{\text{True}} \in \Phi\) then \(\pv{\phi'}{\text{True}} \in \Phi\).
    In other words, if \(\pv{\phi'\text{ and }\phi''}{\text{True}} \in \Phi\) then both \(\pv{\phi'}{\text{True}} \in \Phi\) and \(\pv{\phi''}{\text{True}} \in \Phi\).
    We do not extend this assumption to values other than `True'.
  }

  \begin{definition}[Degree of embedding withing a \ros{}]
    \label{def:embedding:degree}
    For a proposition-value pairs \(\pv{\psi}{v'}\), \(\pv{\phi}{v}\), \poP{} \(\Phi\), and \(i \in \mathbb{N}\):

    \begin{itemize}
    \item
      \(\pv{\psi}{v'}\) has a \emph{degree of embedding \(1\)} with respect to a \ros{} between \(\pv{\phi}{v}\) and \(\Phi\) if and only if \(\pv{\psi}{v'} \in \Phi\).
    \item
      \(\pv{\psi}{v'}\) is has a \emph{degree of embedding \(i\)} with respect to a \ros{} between \(\pv{\phi}{v}\) and \(\Phi\) if and only if:
      \begin{itemize}
      \item
        There exists some \(\pv{\theta}{v''}\) and \(\Theta\) such that:
        \begin{itemize}
        \item
          \(\pv{\psi}{v'} \in \Theta\)
        \item
          \(\pv{\text{A \ros{} between }\pv{\theta}{v''}\text{ and }\Theta}{\text{True}}\) is an \(i - 1\) embedding with respect to the \ros{} between \(\pv{\phi}{v}\) and \(\Phi\).
        \end{itemize}
      \end{itemize}
    \end{itemize}
    \vspace{-\baselineskip}
  \end{definition}

  The cases of interest to us are where \(\pv{\psi}{v'}\) is embedded within in a \ros{} between \(\pv{\phi}{v}\) and \(\Phi\), no matter the degree of embedding:

  \begin{definition}[Embedding within a \ros{}]
    \label{def:embedding}
    For a proposition-value pairs \(\pv{\psi}{v'}\), \(\pv{\phi}{v}\), and a \poP{} \(\Phi\):


    \begin{itemize}
    \item
      \(\pv{\psi}{v'}\) is \emph{embedded} within in a \ros{} between \(\pv{\phi}{v}\) and \(\Phi\)
    \end{itemize}

    \emph{If and only if:}

    \begin{itemize}
    \item
      \(\pv{\psi}{v'}\) is has a degree of embedding \(i\) with respect to the \ros{} between \(\pv{\phi}{v}\) and \(\Phi\), for some \(i \in \mathbb{N}\).
    \end{itemize}
    \vspace{-\baselineskip}
  \end{definition}

  The definition of an embedding covers arbitrary proposition-value pairs.
  However, the cases of embedding of interest to us are where \ros{1} are embedded within a \ros{}.
  A final definition captures when this is the case:

  \begin{definition}[Embedded \ros{1}]
    For a proposition-value pairs \(\pv{\psi}{v'}\), \(\pv{\phi}{v}\), and \poP{1} \(\Phi\), \(\Psi\):

    \begin{itemize}
    \item
      The \ros{} between \(\pv{\psi}{v'}\) and \(\Psi\) is embedded within the \ros{} between \(\pv{\phi}{v}\) and \(\Phi\).
    \end{itemize}

    \emph{If and only if}

    \begin{itemize}[noitemsep]
    \item
      \(\chi\) is the proposition `\(\text{A \ros{} between }\pv{\psi}{v'}\text{ and }\Psi\)'.
    \item
      \(v''\) is the value `True'.
    \item
      \(\pv{\chi}{v''}\) is embedded within a \ros{} between \(\pv{\phi}{v}\) and \(\Phi\).
    \end{itemize}
    \vspace{-\baselineskip}
  \end{definition}
\end{note}

\subsection{Interpretation}
\label{cha:var:ros:Emb:interpretation}

\begin{note}
  Understanding in hand, turn to the function of distinguishing embedded from unembedded \ros{}.

  Simple.

  Whether the \ros{} was a premise from which the agent concluded \(\phi\) has value \(v\).
\end{note}

\begin{note}
  Idea is somewhat familiar from propositional logic.
  Certain kind of equivalence between proof and conditional.
  It is possible to find a corresponding conditional to any proof with a finite number of premises, proof captures derivation of conclusion from premises.

  Corresponding conditional is not a premise, nor any part, of the proof.

  For example, consider a proof from \(P\) and \(P \rightarrow Q\) to \(Q\) by conditional detachment.
  Corresponding conditional is \((P \land (P \rightarrow Q)) \rightarrow Q\).
  However, not part of the proof.

  Intuitive distinction between what a proof and a conditional refer to.
  However, informally there is no difficulty in treating a proof as a premise.
  \(P\), and I have a proof of \(P \rightarrow Q\), therefore \(Q\).
\end{note}

\begin{note}
  Discussion by \citeauthor{Carroll:1895uj} in \citetitle{Carroll:1895uj}.

  \begin{quote}
    My paradox \dots turns on the fact that, in a Hypothetical, the \emph{truth} of the Protasis, the \emph{truth} of the Apodosis, and the \emph{validity of the sequence}, are 3 distinct Propositions.

    \mbox{}\hfill\(\vdots\)\hfill\mbox{}

    Suppose I say ``I grant~\ref{AatT:a} and~\ref{AatT:b} and~\ref{AatT:c}, but I do \emph{not} grant that I am thereby \emph{obliged} to grant~\ref{AatT:z}.''
    Surely, my granting~\ref{AatT:z} must \emph{wait} until I have been made to see the validity of this sequence: i.e.\ in order to grant~\ref{AatT:z}, I must grant~\ref{AatT:a},~\ref{AatT:b},~\ref{AatT:c}, and~\ref{AatT:d}! And so on.%
    \mbox{ }\hfill\mbox{(\citeyear[472]{Carroll:1977wl})}
  \end{quote}

  \citeauthor{Carroll:1895uj} is slightly different.
  For, rather than \ros{}, \citeauthor{Carroll:1895uj} focuses on valid inferences.

  My understanding of \citeauthor{Carroll:1895uj} is in terms of the relation between general and specific.

  Inference from \(A\) and \(A \rightarrow B\) to \(B\).
  Valid inference.
  However, valid inference if and only if holds for all \emph{A} and \emph{B}.
  Point, then, is that given \(A\) and \(A \rightarrow B\) still need validity.
  Even if grant \(B\) is true, this only gives specific instance.
  However, at issue is general.
  Does \(D\) follow from \(C, C \rightarrow D\)?

  How do we get a general rule without already being sure of the specific instances of the general rule.

  Immediate observation is that general rule does not correspond to any specific instance to which the general rule applies.

  Hence, motivates the same idea.
  In some cases, these two things are different.
\end{note}

\begin{note}
  Start to see a difference by considering \(\pv{\phi}{v}\) and \(\Phi\).
  For, \ros{} between \(\pv{\phi}{v}\) and \(\Phi\).
  However, immediate difficulties if the proposition that there is a \ros{} between \(\pv{\phi}{v}\) and \(\Phi\) paired with the value `True' is a member of \(\Phi\).

  Likewise, \ros{} between the single premise and \(\pv{\phi}{v}\) is, intuitively, distinct from \ros{} between \(\pv{\phi}{v}\) and \(\Phi\).

  \ros{} between \(\pv{\phi}{v}\) and \(\Phi\) results from an event in which an agent concludes \(\pv{\phi}{v}\) from \(\Phi\).
  By contrast, \ros{} between \(\pv{\phi}{v}\) and \(\Phi\) results from an event in which an agent concludes \(\pv{\phi}{v}\) from the \ros{} between \(\pv{\phi}{v}\) and \(\Phi\).
\end{note}

\begin{note}
  \citeauthor{Boghossian:2008vf}:
  \begin{quote}
    [W]e would have to take as primitive a \emph{general (often conditional) content serving as the reason for which one believes something}, without this being mediated by inference of any kind.%
    \mbox{ }\hfill\mbox{(\citeyear[500]{Boghossian:2008vf})}
  \end{quote}
\end{note}

\begin{note}
  Significance for arguments is that dependence captured by \qWhyV{} does not amount to the \ros{} being a premise.
\end{note}

\subsection{Carroll notes}

\begin{note}
  Well, to the extent \citeauthor{Carroll:1895uj} seems to be.

  For, break out of the loop at any time.
  There's nothing to prevent the Tortoise from accepting.

  However, something on the Tortoise's mind.
  Generates further hypotheticals.

  \begin{generator}[\citeauthor{Carroll:1895uj}]
    Instance of rule of inference is sound only if hypothetical is true and hypothetical without rule of inference.
  \end{generator}

  So, every time Achilles appeals to instance of rule of inference, generator a new hypothetical, and get this without the rule of inference.
  Look, the Tortoise is really clear that you can get the hypothetical without the rule of inference.

  So, hypothetical is necessary, independent, and (possibly) antecedent.%
  \footnote{
    Here might be the trouble.
    Understanding conditional just is getting rule of inference.
    But, this is not obvious.
    For, other connectives.
    It's not obvious that I get conjunction i/e from understanding conjunction.
    For, could be the case that always translate conjunction to some other connective when applying rules of inference.

    Now, still \emph{modus ponens}.
    But, in principle no need for this.

    Consider a Gentzen system for propositional logic termed \textbf{G3cp} (\cite[\S3.5]{Troelstra:2000ue},\cite[\S3.1]{Negri:2008wy}).

    In \textbf{G3c}(\textbf{p}) the two rules concerning material implication are:

    \begin{quote}
      \mbox{ }\hfill%
      \(
      \AxiomC{\(\Gamma \Rightarrow \Delta, A\)}
      \AxiomC{\(B,\Gamma \Rightarrow \Delta\)}
      \LeftLabel{L\(\rightarrow\)}
      \BinaryInfC{\(A \rightarrow B, \Gamma \Rightarrow \Delta\)}
      \DisplayProof
      \)%
      \hfill
      \(
      \AxiomC{\(A,\Gamma \Rightarrow \Delta,B\)}
      \LeftLabel{R\(\rightarrow\)}
      \UnaryInfC{\(\Gamma \Rightarrow \Delta, A \rightarrow B\)}
      \DisplayProof
      \)%
      \hfill\mbox{ }\newline
      \mbox{ }\hfill\mbox{(\citeyear[77]{Troelstra:2000ue})}
    \end{quote}
    By inspection, neither rule corresponds to \emph{modus ponens} in any direct way.
  }\(^{,}\)%
  \footnote{
    Puzzle about the general lesson from \citeauthor{Carroll:1895uj}.

    \citeauthor{Wieland:2013vf} (\citeyear{Wieland:2013vf}) characterises the general understanding of \textcite{Carroll:1895uj} in terms of two lessons:
    \begin{quote}
      [T]he negative lesson is that if you add ever more premises to an argument \dots, then you will never demonstrate that its conclusion follows logically.\newline
      \mbox{ }\hfill\mbox{(\citeyear[984]{Wieland:2013vf})}
    \end{quote}

    Parallel, static answers, still option for concluding otherwise.

    \begin{quote}
      [T]he positive lesson is that rules of inference, rather than premises of the form `if premises such and such are true, then the conclusion is true', will do the job.\newline
      \mbox{ }\hfill\mbox{(\citeyear[984]{Wieland:2013vf})}
    \end{quote}
    Positive lesson \emph{is} the puzzle.
  }

  \begin{quote}
    My paradox \dots turns on the fact that, in a Hypothetical, the \emph{truth} of the Protasis, the \emph{truth} of the Apodosis, \& the \emph{validity of the sequence}, are 3 distinct Propositions.
    \begin{quote}
      For instance, if I grant

      \begin{enumerate}[label=(\arabic*)]
      \item
        All men are mortal, \& Socrates is a man, but not
      \item
        The sequence “If all men are mortal and if Socrates is a man, then Socrates is mortal” is valid,
      \end{enumerate}

      then I do not grant

      \begin{enumerate}[label=(\arabic*), resume]
      \item Socrates is mortal.
      \end{enumerate}
      Again, if I grant (2), but not (1), I still fail to grant (3).

      Hence, before granting (3), I must grant (1) \& (2)\newline
      \mbox{ }\hfill\mbox{(\citeyear[10--11]{Carroll:2016wl})}
    \end{quote}
  \end{quote}
\end{note}

\section{Wrangling}
\label{sec:wrangling}

\begin{note}
  Here is where embedding is interesting.
  If under constraint of \agpe{}, and \fc{}, then why not embed \fc{} as a premise.

  
\end{note}

\begin{note}
  Following, distinction between \ros{} answering, in part, \qWhyV{} and a \ros{} being embedded in some \ros{}.
  In particular, \ros{} between some \(\pv{\psi}{v'}\) and \(\Psi\) such that distinct from \ros{} between \(\pv{\phi}{v}\) and \(\Phi\) being embedded within the \ros{} between \(\pv{\phi}{v}\) and \(\Phi\) as a premise.
\end{note}

\begin{note}
  Rule out possibility of embedding proposed answers in this way.

  However, this will turn on the way in which dependence holds.
\end{note}

\subsection{Quarantine}

\begin{note}
  Granting \agpe{}, still fail to answer \qWhyV{}.
  Here, we return to the prospect of embedding \ros{1} within \ros{1} as seen in~\autoref{cha:var:ros:Emb}.

  If the \agpe{} is correct, then \fc{} matters.
  Hence, \ros{} matters.
  However, need not follow that conclusion of \(\pv{\phi}{v}\) from \(\Phi\) `directly' depends on \ros{} between \(\pv{\psi}{v'}\) and \(\Psi\).
  For, \ros{} between \(\pv{\psi}{v'}\) and \(\Psi\) may be embedded within the \ros{} between \(\pv{\phi}{v}\) from \(\Phi\).
\end{note}

\begin{note}
  So, embedding.

  Basically, concern about not being a \fc{}.
  Then, adding premise that is a \fc{} doesn't do any work.
\end{note}

\begin{note}
  Still, this isn't quite enough.
  What about some other explanation.

  If there is some general method to quarantine, then avoid worries about the link breaking.
  Deny the proposition holds.
  Separate \agpe{} from what happens.

  However, if no general method to quarantine.
  Then, more difficult.
  Something problematic about agent.

  Whether break the \agpe{} from what matters.

  Task: \fc{} matters from \agpe{}, but does not matter apart.
  Argue that this is not possible.
  To do this, relevant \fc{} does not embed.

  Does not function as an attitude.
\end{note}

\begin{note}
  Therefore, require a \wit{} in order to ensure that the \ros{} exists.
\end{note}

\begin{note}
  Factive and perspective is correct.
  If this is the case, then need a method of reducing to something factive.
  The only plausible way to reduce to something factive is to embed within a \ros{}.
  No embedding in relations of support.
  So, either reject factive or reject perspective.

  This is interesting, because typically the case that motivate factive by observing that it preserves the role of the \agpe{}.
\end{note}

\subsection{\citeauthor{Owens:2006tw}}

\begin{note}
  For example, \citeauthor{Owens:2006tw} argues for a belief expression model of assertion in which the rationality of a belief formed by an agent via testimony is connected to justification of the testifier:

  \begin{quote}
    Trusting an expression of belief by accepting what a speaker says involves entering a state of mind which gets its rationality from the rationality of the belief expressed.
    This state's rationality depends on the speaker's justification for the belief he expresses, not on his justification for the action of expressing it.
    And to hear a speaker as making a sincere assertion, as expressing a belief, is \emph{ceteris paribus} to feel able to tap into \emph{that} justification (whether or not his assertion was directed at you) by accepting what he says.%
    \mbox{}\hfill\mbox{(\citeyear[123]{Owens:2006tw})}
  \end{quote}

  On the view advanced by \citeauthor{Owens:2006tw}, justification.
  View in terms of \support{}.

  \support{} directly.
  Rationality of agent is rationality of speaker.

  However, `depends'.

  Distinction between rationality of state, and relation between rationality of state and rationality of state.

  Inclined to think \citeauthor{Owens:2006tw} is arguing for the former.%
  \footnote{
    \begin{quote}
      If we are to believe what the speaker indicates he believes, either the speaker must justify this belief to us, or we must supply some justification of our own
      \dots
      Neither act can be part of a rationality preserving mechanism for belief.%
      \mbox{ }\hfill\mbox{(\citeyear[123--124]{Owens:2006tw})}
    \end{quote}
  }
  Though, it is not clear to me that embedded isn't a viable option.

  Regardless, distinction that is important.
\end{note}

\section{Embedding in the literature}
\label{cha:var:sec:embedding}

\begin{note}
  So, in initial cases, plausible that constraint in terms of having a \wit{}.

  In other cases, embedding.
\end{note}

\begin{note}
  An initial borderline case is \citeauthor{Boghossian:2014aa}'s taking condition.
  Depending on how `taking' is understood, embedded within \ros{}.
  In particular, distinguished proposition-value pairs from attitudes, and hence this does not amount to reducing the taking condition to a doxastic condition.
\end{note}

\subsection{\textcite{Thomson:1965vv}}

\begin{note}
  \citeauthor{Thomson:1965vv} suggests a an account of reasoning such that an agent reasons from \(\phi\) to \(\psi\) just in case the agent believes that \(\phi\) is a reason for \(\psi\).
  \begin{quote}
    The claim which the 'formula' of p.\ 285\nolinebreak
    \footnote{
      The `formula' in question:
      \begin{quote}
        Now reasoning should surely involve drawing a conclusion from a set of premisses.
        But you can't be said to draw the conclusion that \emph{q} from \emph{p} if for all you know in knowing that \emph{p} it would at best be a matter of luck if \emph{q} as well.
        So to ``reason'' from \emph{p} by itself to \emph{q} isn't really to be reasoning; it's like saying one thing, and then taking a chance on it that something else is also true---like taking a leap in the dark, or more prosaically, like guessing.'
        (From here on I shall refer to this as the `\emph{formula}'.)\nolinebreak
        \mbox{}\hfill\mbox{(\citeyear[285]{Thomson:1965vv})}
      \end{quote}
    }
    above was to support was this:
    suppose \emph{p} does not imply \emph{q}, and suppose a man says `\emph{p}, so \emph{q}';
    then he is not reasoning in saying this unless he believes that \emph{r}, where the conjunction of \emph{p} and \emph{r} implies \emph{q}, and \emph{r} is a suppressed premiss of his reasoning.\par
    But suppose such a man believes that \emph{p} is reason for \emph{q}; would this not be enough?
    `It would if ``\emph{p} is reason for \emph{q}'' were construed as a suppressed premiss of his argument'.
    Then let us so construe it.\newline
    \mbox{}\hfill\mbox{(\citeyear[294]{Thomson:1965vv})}
  \end{quote}
  Causation is absent from \citeauthor{Thomson:1965vv}.
  Does not imply that \citeauthor{Thomson:1965vv}'s proposal is independent of causation, but motivated does not appeal to causation.
\end{note}

\subsection{\textcite{Longino:1978wv}}

\begin{note}
  \citeauthor{Longino:1978wv}'s (\citeyear{Longino:1978wv}) account of inferring seems explicit.
  \begin{quote}
    S infers at t that p from x if and only if
    \begin{enumerate}[label=\arabic*]
    \item
      S at t comes to believe that p, and
    \item
      S's epistemic reason for believing that p at t is x, i.e.,
      \begin{enumerate}[label=\alph*]
      \item
        S takes x to be evidence that p, and
      \item
        S's taking x to be evidence that p causes S to believe that p.\newline
        \mbox{}\hfill\mbox{(\citeyear[22]{Longino:1978wv})}
      \end{enumerate}
    \end{enumerate}
  \end{quote}
  Causation between mental states, but explanatory relation between things.
\end{note}

  % As~\ref{ros:ap:maybe:a} is shorter, we will only use~\ref{ros:ap:maybe:b} when we wish to explicitly consider 

  % Still, the difference in presentation between~\ref{ros:ap:maybe:a} and~\ref{ros:ap:maybe:b} is suggestive:
  % \begin{itemize}
  % \item
  %   \ref{ros:ap:maybe:b} explicitly captures a proposition-value pair.

  %   Hence, \ref{ros:ap:maybe:b} explicitly identifies something which may be a premises from which a conclusion is drawn.
  % \item
  %   By contrast,~\ref{ros:ap:maybe:a} does not explicitly capture a proposition-value pair.

  %   Hence,~\ref{ros:ap:maybe:a} only implicitly identifies something which may be a premises from which a conclusion is draw.

  %   And, if~\ref{ros:ap:maybe:a} and~\ref{ros:ap:maybe:b} are not equivalent,~\ref{ros:ap:maybe:a} may fail to identify something which may be a premises from which a conclusion is draw
  % \end{itemize}

  % Hence, we will use the presentation of \ref{ros:ap:maybe:b} when we wish to highlight the possibility that the relevant \ros{} may function as a premise for some conclusion.
  % And, likewise, we will use the presentation of~\ref{ros:ap:maybe:a} when we wish to highlight the possibility that the relevant \ros{} may not function as a premise.

  % Indeed, we will favour the presentation of~\ref{ros:ap:maybe:a} to remain neutral on whether \ros{}

  % In general, presentation of~\ref{ros:ap:maybe:a}.
  % For, it may not be possible to reduce certain things captured by \ros{} to premise.
  % For example, consider a strong distinction between the role of premises and rules of inference following \citeauthor{Carroll:1895uj}'s \citetitle{Carroll:1895uj}.

  % Make this distinction clearer by a discussion of embedding.


%%% Local Variables:
%%% mode: latex
%%% TeX-master: "master"
%%% End:
