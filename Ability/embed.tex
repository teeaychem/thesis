\chapter{Embedded \ros{1}}
\label{cha:embed}

\begin{note}
  In this chapter we return to \ros{1} in order to introduce and discuss a useful feature of abstracting to \ros{1} which we term `embedding'.

  Embedding have two functions in this document.
  \begin{itemize}
  \item
    Specify a concern regarding the counterexamples that we will develop to \issueConstraint{}.
  \item
    Capture additional accounts of events in which an agent concludes form the literature.
  \end{itemize}
\end{note}

\begin{note}
  If you decide to read this chapter, it is split into three sections:

  \begin{TOCEnum}
  \item
    \TOCLine{cha:var:ros:Emb}

    Introduce distinction, define, and intuition.
  \item
    \TOCLine{sec:wrangling}

    The concern.
  \item
    \TOCLine{cha:var:sec:embedding}

    A handful of accounts from the literature.
  \end{TOCEnum}
\end{note}

\section{Embedding \ros{1}}
\label{cha:var:ros:Emb}

\begin{note}
  The present section concerns, for some arbitrary proposition-value pair \(\pv{\phi}{v}\) and \pool{} \(\Psi\), the distinction between:

  \begin{enumerate}[label=\arabic*., ref=(\arabic*)]
  \item
    \label{Embed:no}
    A \ros{0} between \(\pv{\psi}{v'}\) and \(\Psi\).
  \item
    \label{Embed:yes}
    A \ros{0} between \(\Phi\) and \(\pv{\phi}{v}\), where:
    \begin{itemize}
    \item
      \(\Phi\) contains:
      \begin{itemize}
      \item
        \(\pv{\propI{A \ros{} between }\pv{\psi}{v'}\propI{ and }\Psi}{\valI{True}}\)
      \end{itemize}
    \end{itemize}
  \end{enumerate}

  \ref{Embed:no} is a \ros{} between \(\pv{\psi}{v'}\) and \(\Psi\).
  By contrast, \ref{Embed:yes} is a \ros{0} between \(\pv{\phi}{v}\) and \(\Phi\) which \emph{involves} a \ros{} between \(\pv{\psi}{v'}\) and \(\Psi\)%
  ---%
  the \ros{} between \(\pv{\psi}{v'}\) and \(\Psi\) is occurs in a premise of the \ros{} between \(\pv{\phi}{v}\) and \(\Phi\).

  If \ros{} occurs within some premise of a \ros{} between \(\pv{\phi}{v}\) and \(\Phi\), we say the \ros{} between \(\pv{\psi}{v'}\) and \(\Psi\) is \emph{embedded} within the \ros{} between \(\pv{\phi}{v}\) and \(\Phi\).
\end{note}

\begin{note}
  It need not be the case that an agent has a \wit{0} for a \ros{0} in order for \ros{} to be involved in answering \qWhyV{}.

  For, suppose an agent does not have a \wit{0} for the \ros{} between \(\pv{\psi}{v'}\) and \(\Psi\).
  The upshot of the distinction between~\ref{Embed:no} and~\ref{Embed:yes} is as follows:

  \begin{itemize}
  \item
    If the \ros{0} of \ref{Embed:no} is, in part, an answer to \qWhyV{} then the \ros{0} is a counterexample to \issueConstraint{}.
  \item
    If the \ros{0} of \ref{Embed:yes} is, in part, an answer to \qWhyV{} then the \ros{0} is \emph{not} a counterexample to \issueConstraint{}.
  \end{itemize}

  The difference is \emph{the way in which} the \ros{} functions with respect to the agent pairing \(\phi\) with \(v\).
  Whether the \ros{} functions as a premise when the agent concludes \(\pv{\phi}{v}\), or whether the \ros{} functions in a way that is different to a premise.
\end{note}

\begin{note}
  Embedding of this kind will be:
  \begin{itemize}[noitemsep]
  \item
    Useful for relating the variations of \qWhy{} and \qHow{} to theories of concluding, or related phenomena. And,
  \item
    Important for clarifying a difficult for developing counterexamples to \issueInclusion{}.
  \end{itemize}
\end{note}

\subsection{Definitions}
\label{cha:var:ros:Emb:defs}

\begin{note}
  In full, we define embedding in the following way:%
  \footnote{
    We assume, in general, that if a proposition \(\phi\) includes some other proposition \(\phi'\), then if \(\pv{\phi}{\valI{True}} \in \Phi\) then \(\pv{\phi'}{\valI{True}} \in \Phi\).
    In other words, if \(\pv{\phi'\text{ and }\phi''}{\valI{True}} \in \Phi\) then both \(\pv{\phi'}{\valI{True}} \in \Phi\) and \(\pv{\phi''}{\valI{True}} \in \Phi\).
    We do not extend this assumption to values other than `True'.
  }

  \begin{definition}[Degree of embedding withing a \ros{}]
    \label{def:embedding:degree}
    For a proposition-value pairs \(\pv{\psi}{v'}\), \(\pv{\phi}{v}\), \pool{} \(\Phi\), and \(i \in \mathbb{N}\):

    \begin{itemize}
    \item
      \(\pv{\psi}{v'}\) has a \emph{degree of embedding \(1\)} with respect to a \ros{} between \(\pv{\phi}{v}\) and \(\Phi\) if and only if \(\pv{\psi}{v'} \in \Phi\).
    \item
      \(\pv{\psi}{v'}\) is has a \emph{degree of embedding \(i\)} with respect to a \ros{} between \(\pv{\phi}{v}\) and \(\Phi\) if and only if:
      \begin{itemize}
      \item
        There exists some \(\pv{\theta}{v''}\) and \(\Theta\) such that:
        \begin{itemize}
        \item
          \(\pv{\psi}{v'} \in \Theta\)
        \item
          \(\pv{\propI{A \ros{} between }\pv{\theta}{v''}\propI{ and }\Theta}{\valI{True}}\) has degree of embedding \(i - 1\) with respect to the \ros{} between \(\pv{\phi}{v}\) and \(\Phi\).
        \end{itemize}
      \end{itemize}
    \end{itemize}
    \vspace{-\baselineskip}
  \end{definition}

  The cases of interest to us are where \(\pv{\psi}{v'}\) is embedded within in a \ros{} between \(\pv{\phi}{v}\) and \(\Phi\), no matter the degree of embedding:

  \begin{definition}[Embedding within a \ros{}]
    \label{def:embedding}
    For a proposition-value pairs \(\pv{\psi}{v'}\), \(\pv{\phi}{v}\), and a \pool{} \(\Phi\):


    \begin{itemize}
    \item
      \(\pv{\psi}{v'}\) is \emph{embedded} within in a \ros{} between \(\pv{\phi}{v}\) and \(\Phi\)
    \end{itemize}

    \emph{If and only if:}

    \begin{itemize}
    \item
      \(\pv{\psi}{v'}\) is has a degree of embedding \(i\) with respect to the \ros{} between \(\pv{\phi}{v}\) and \(\Phi\), for some \(i \in \mathbb{N}\).
    \end{itemize}
    \vspace{-\baselineskip}
  \end{definition}

  The definition of an embedding covers arbitrary proposition-value pairs.
  However, the cases of embedding of interest to us are where \ros{1} are embedded within a \ros{}.
  A final definition captures when this is the case:

  \begin{definition}[Embedded \ros{1}]
    For a proposition-value pairs \(\pv{\psi}{v'}\), \(\pv{\phi}{v}\), and \pool{1} \(\Phi\), \(\Psi\):

    \begin{itemize}
    \item
      The \ros{} between \(\pv{\psi}{v'}\) and \(\Psi\) is \emph{embedded} within the \ros{} between \(\pv{\phi}{v}\) and \(\Phi\).
    \end{itemize}

    \emph{If and only if}

    \begin{itemize}
    \item
      For some proposition-value pair \(\pv{\chi}{v''}\) in \(\Phi\):
      \begin{itemize}[noitemsep]
      \item
        \(\chi\) is the proposition: \propI{A \ros{} between }\pv{\psi}{v'}\propI{ and }\Psi.
      \item
        \(v''\) is the value: \valI{True}
      \end{itemize}
    \end{itemize}
    \vspace{-\baselineskip}
  \end{definition}
\end{note}

\begin{note}
  Idea is somewhat familiar from distinction between object- and meta-language with respect to propositional logic.
  Certain kind of equivalence between proof and conditional.
  It is possible to find a corresponding conditional to any proof with a finite number of premises, proof captures derivation of conclusion from premises.

  Corresponding conditional is not a premise, nor any part, of the proof.

  For example, consider a proof from \(P\) and \(P \rightarrow Q\) to \(Q\) by conditional detachment.
  Corresponding conditional is \((P \land (P \rightarrow Q)) \rightarrow Q\).
  However, not part of the proof.

  Intuitive distinction between what a proof and a conditional refer to.
  However, informally there is no difficulty in treating a proof as a premise.
  \(P\), and I have a proof of \(P \rightarrow Q\), therefore \(Q\).
\end{note}


\section{Embedding \requ{1}}
\label{sec:wrangling}

\begin{note}
  Interest with embedding is the following proposition:

  \begin{proposition}
    If embedded, then does not answer \qWhyV{}.
  \end{proposition}

  Distinction between \ros{} answering \qWhyV{} and a \ros{} being embedded in some \ros{}

  Now:
  \begin{itemize}
  \item
    If \requ{}, then why not embed \fc{} as a premise.
  \end{itemize}
\end{note}

\begin{note}
  Problem.
  Plausibly remains a \requ{}.
  For, failure to reason is still a problem.
\end{note}

\subsection{Example}
\label{sec:example}

\begin{note}
  Squish is a great example, as I plausibly rely on my memory of a prior proof.
  Indeed, I am not inclined to think about this for proofs I have no done at some point in the past.

  
\end{note}

\section{Embedding in the literature}
\label{cha:var:sec:embedding}

\begin{note}
  So, in initial cases, plausible that constraint in terms of having a \wit{}.

  In other cases, embedding.
\end{note}

\begin{note}
  An initial borderline case is \citeauthor{Boghossian:2014aa}'s taking condition.
  Depending on how `taking' is understood, embedded within \ros{}.
  In particular, distinguished proposition-value pairs from attitudes, and hence this does not amount to reducing the taking condition to a doxastic condition.
\end{note}

\subsection*{\textcite{Thomson:1965vv}}

\begin{note}
  \citeauthor{Thomson:1965vv} suggests a an account of reasoning such that an agent reasons from \(\phi\) to \(\psi\) just in case the agent believes that \(\phi\) is a reason for \(\psi\).
  \begin{quote}
    The claim which the 'formula' of p.\ 285\nolinebreak
    \footnote{
      The `formula' in question:
      \begin{quote}
        Now reasoning should surely involve drawing a conclusion from a set of premisses.
        But you can't be said to draw the conclusion that \emph{q} from \emph{p} if for all you know in knowing that \emph{p} it would at best be a matter of luck if \emph{q} as well.
        So to ``reason'' from \emph{p} by itself to \emph{q} isn't really to be reasoning; it's like saying one thing, and then taking a chance on it that something else is also true---like taking a leap in the dark, or more prosaically, like guessing.'
        (From here on I shall refer to this as the `\emph{formula}'.)\nolinebreak
        \mbox{}\hfill\mbox{(\citeyear[285]{Thomson:1965vv})}
      \end{quote}
    }
    above was to support was this:
    suppose \emph{p} does not imply \emph{q}, and suppose a man says `\emph{p}, so \emph{q}';
    then he is not reasoning in saying this unless he believes that \emph{r}, where the conjunction of \emph{p} and \emph{r} implies \emph{q}, and \emph{r} is a suppressed premiss of his reasoning.\par
    But suppose such a man believes that \emph{p} is reason for \emph{q}; would this not be enough?
    `It would if ``\emph{p} is reason for \emph{q}'' were construed as a suppressed premiss of his argument'.
    Then let us so construe it.%
    \mbox{}\hfill\mbox{(\citeyear[294]{Thomson:1965vv})}
  \end{quote}
\end{note}

\subsection*{\textcite{Longino:1978wv}}

\begin{note}
  \citeauthor{Longino:1978wv}'s (\citeyear{Longino:1978wv}) account of inferring seems explicit.
  \begin{quote}
    S infers at t that p from x if and only if
    \begin{enumerate}[label=\arabic*]
    \item
      S at t comes to believe that p, and
    \item
      S's epistemic reason for believing that p at t is x, i.e.,
      \begin{enumerate}[label=\alph*]
      \item
        S takes x to be evidence that p, and
      \item
        S's taking x to be evidence that p causes S to believe that p.\newline
        \mbox{}\hfill\mbox{(\citeyear[22]{Longino:1978wv})}
      \end{enumerate}
    \end{enumerate}
  \end{quote}
  Causation between mental states, but explanatory relation between things.
\end{note}

%%% Local Variables:
%%% mode: latex
%%% TeX-master: "master"
%%% End:


% \subsection{\citeauthor{Owens:2006tw}}

% \begin{note}
%   For example, \citeauthor{Owens:2006tw} argues for a belief expression model of assertion in which the rationality of a belief formed by an agent via testimony is connected to justification of the testifier:

%   \begin{quote}
%     Trusting an expression of belief by accepting what a speaker says involves entering a state of mind which gets its rationality from the rationality of the belief expressed.
%     This state's rationality depends on the speaker's justification for the belief he expresses, not on his justification for the action of expressing it.
%     And to hear a speaker as making a sincere assertion, as expressing a belief, is \emph{ceteris paribus} to feel able to tap into \emph{that} justification (whether or not his assertion was directed at you) by accepting what he says.%
%     \mbox{}\hfill\mbox{(\citeyear[123]{Owens:2006tw})}
%   \end{quote}

%   On the view advanced by \citeauthor{Owens:2006tw}, justification.
%   View in terms of \support{}.

%   \support{} directly.
%   Rationality of agent is rationality of speaker.

%   However, `depends'.

%   Distinction between rationality of state, and relation between rationality of state and rationality of state.

%   Inclined to think \citeauthor{Owens:2006tw} is arguing for the former.%
%   \footnote{
%     \begin{quote}
%       If we are to believe what the speaker indicates he believes, either the speaker must justify this belief to us, or we must supply some justification of our own
%       [\dots]
%       Neither act can be part of a rationality preserving mechanism for belief.%
%       \mbox{ }\hfill\mbox{(\citeyear[123--124]{Owens:2006tw})}
%     \end{quote}
%   }
%   Though, it is not clear to me that embedded isn't a viable option.

%   Regardless, distinction that is important.
% \end{note}
