\chapter{\LCS{}, \FCS{}, and \aben{the}}
\label{sec:second-conditional}

\begin{note}
  So, two main tasks.
  \begin{enumerate}
  \item Why tension with \adA{}.
  \item Why tension does not arise for \adB{}.
  \end{enumerate}
\end{note}

\begin{note}[Finding tension, still]
  \color{red}
  We have outlined a type of scenario built primarily on an agent receiving information that the agent has some specific ability so long as the agent has some general ability.
  The agent has support for having the general ability, but there are two ways in which the agent's support for having the general ability may be used to establish support for {\color{red} the result of having the specific ability} --- \AR{} and \WR{}.

  The previous section argued that~\ESU{} constrains how an agent may use the received information.
  If an agent is required to traces support from premises to conclusion through reasoning, then an agent may not appeal to the support for the premises and steps of reasoning that the agent would use to witness the specific ability.

  The (initial) plausibility of~\ESU{}, then, suggests that the agent may only establish support for having the {\color{red} result of the specific ability} from the support they have for the general ability by \AR{}:
  The support the agent has for the general ability is support that it is true that the agent has the general ability.
  In turn, given the information received it is true that the agent has the specific ability, and it is only possible for the agent to have the specific ability if the result of witnessing the specific ability is true.

  The argument of this section is that the sketch of \AR{} given conflicts with a different, but equally plausible, premise.
  The premise concerns the way in which the agent obtains support for having the specific ability from the support for the general ability.
  We state conditional, the proceed to the premise.
  The initial statement of the premise is abstract and after providing a handful of clarifications we then link the premise to the type of scenario of interest.
\end{note}

\section{\LCS{} and ability}
\label{sec:ni-ability}

\begin{note}[Notes]
  Here, there are two different options to explore.
  \begin{enumerate}
  \item Going from general to specific to \(\psi\)
  \item Going from general to \(\psi\)
  \end{enumerate}
  Part of the interest here is that I'm not relying on anything to do with some failure of induction.
  If the agent needs to go general to get specific, then there's a problem.
  However, the second option grants that the first problem may be avoided.

  In general, the issue is that so long as the agent appeals to ability, then they don't have a clear account of why \(\psi\) holds that doesn't rely on \(\psi\) holding.
  And, so failure of \ideaCS{} (in particular).
  Specifically, failure of assumption made.
\end{note}

\begin{note}
  So, the general plan is to observe that if \ESU{} holds, then need to appeal to having the ability.
  And, this involves appeal to moving to having the ability from prior reasoning.
  Without the second step here, the argument breaks down.
  However, it seems unavoidable.
  There's no other reasoning possible for the agent.
  Ability seems required.
\end{note}

\subsection{\LCS{}}
\label{sec:ability-and-lcs}

\begin{note}
  Goal of this section is to show how \LCS{} applies to ability.
\end{note}

\begin{note}[Checking conditions]
  Condition~\ref{nI:background} is provided by the scenario.
  And, scenario also provide information about how the agent claims support for \(\psi\) having value \(v'\) when \(\phi\) has value \(v\).
  Key is that \aben{the} requires that the agent has the specific ability, not (merely) that the agent has claimed support that they have the specific ability.
  Condition~\ref{nI:inclusion} is obtained by reflection on ability.
  So, then, condition~\ref{nI:going-by-value} rules out a way of claiming support for specific ability.
\end{note}


\begin{note}
  \color{red}
  \begin{itemize}
  \item So, get \requ{}.
  \item This much is fine.
  \item Question is whether it is possible for the agent to do anything about the \requ{}.
  \item Arguably no.
  \item One big idea is that the agent has claimed support for general ability on some `core' such that this provides strong indication that general ability extends to all cases.
  \item This much is fine, then problem, however, is that we've still got specific information about what is outside of the core.
  \item So, the probability of any possible defeater is super low.
  \item And, this is enough to hold onto claimed support regardless.
  \item Because, I considered the possibility of those unknown defeaters, and still gathered enough to claim support regardless.
  \item So, this seems to allow claiming support for specific from general.
  \item But at the same time this seems bad.
  \item It has the same feel as the problems with \requ{1}.
  \item Because, all the stuff gathered was without recognition of this possibility.
  \end{itemize}

  \begin{itemize}
  \item I mean, problem is before, got the probability low as unrecognised.
  \item Question is whether this remains the case now recognised.
  \item Well, nothing really follows from probability being low.
  \item In a sense, this should already be the case.
  \item The issue isn't that these possible defeaters a \emph{likely}.
  \item The issue is that the agent should think that support holds regardless of whether they hold.
  \item ``Category mistake''
  \end{itemize}

  Okay, so this kind of works against low probability.
  Hence, argument here is that there's no way to get rid of this \requ{} if agent only relies on claimed support for general ability.
  Of course think it's unlikely, but the worry is not that the defeater is there, rather than it's not clear how the evidence goes against the defeater.

  The redux, then, is that this idea of a `core' doesn't really rely on the probability idea.
  But then this just goes against the initial assumption.
  Of course, this is kind of what \citeauthor{Pryor:2000tl} does.
  However, this seems to conflict with the idea of claimed support.
  If we've got some kind of dogmatist position, then it doesn't seem that the possibility of being \mom{} is such an issue.
  Indeed, the problem here is how to make something like this consistent with that assumption.
\end{note}

\subsection{\adA{}}
\label{sec:lcs-and-adA}

\begin{note}
  Observed \LCS{}.
  The important thing here is that \FCS{}.

  So, goal here is to argue that the additional constraints of \FCS{} also hold with respect to ability.
\end{note}

\subsubsection{\gsi{}, \adA{}, and \FCS{}}
\label{sec:ni-ability:adA}

\begin{note}
  The idea here is simple.
  \begin{itemize}
  \item The thing with \adA{} is that the agent appeals to ability `as a whole', so to speak.
  \item This gives us the relevant \(\phi\) instance for \nI{}.
  \item And, \aben{the} is such that the agent needs to appeal to ability, rather than mere claimed support.
  \item Then, the key focus is \ref{nI:inclusion}.
  \end{itemize}
\end{note}

\begin{note}
  Now, the basic observation is that with \adA{} one moves from general to specific, and from ability to proposition.

  Here, only really interested in \aben{the}.
  However, as we've observed, goes from either general or specific.

  I mean, the basic observation is that the agent doesn't reason about general or specific ability.
  So, reasoning follows from it being the case that agent has attribute, or that there is a witnessing event.

  Ohhhh, the point is that the agent is relying on these conditionals.
  First, to move from general to specific.
  Second, to move from ability to proposition.

  With respect to these conditionals, it's \adA{}, so there's no way to move between these things without using the value of one thing to constrain the value of the other.

  So, instance of \adA{}, generally.
  And, because of the construction of the scenarios, the case of \adA{} we're interested involves appeal to the value of the proposition.
\end{note}

\subsection{\adB{}}
\label{sec:lcs-and-adB}

\begin{note}
  \color{red}
  \begin{itemize}
  \item Key lesson learnt from \LCS{} is that if the agent goes to having the ability, then they bring \(\psi\) with them (due to interdependence of claiming support).
    So, \adB{} needs to avoid going to ability.
  \item Key with \adB{} is that it breaks up this interdependence.
  \item Instead of using the ability as a whole, everything gets broken up into premises and steps.
  \item This is motivated by general understanding of reasoning.
  \item For, general point of reasoning is breaking things down so that the conclusion doesn't follow from any particular step or premise, but rather the combination.
  \end{itemize}

  So, the real thing I need for \adA{} is that ability gets treated as a whole.
  And, this then extends to basic \AR{}.

  Do I still need the \adB{} vs.\ \adA{} distinction?
  Probably, as it helps with motivating the key ideas.
  I mean, yes as there isn't a good distinction between \AR{} and \WR{} alone.
\end{note}

\begin{note}
  \adB{}.
  Two ways of interpolating.
  First, from ability.
  Second, from `ability'.
  {\color{red} Goal is to break \ref{nI:inclusion:bound}}
\end{note}

\begin{note}[Main idea]
  \nI{} is about interdependence of claiming support between \(\phi\) and \(\psi\) undermining a way of claiming support for \(\psi\).

  So, the task is to show that this interdependence need not hold when reasoning \adB{}.

  This may not be immediately obvious.
  Saw \nI{} applied to \adA{}.
  Going \adB{} doesn't necessarily make a difference.
  {
    \color{red}
    Problem is, comes from appealing to ability.
  }

  However, what \adB{} \dots
\end{note}

\subsubsection{\nI{} and \adB{} (excluding \ARB{})}
\label{sec:ni-ability:adB}

\begin{note}
  \begin{itemize}
  \item Have \gsi{} information.
  \item This means that we get a sort of conditional
    \begin{itemize}
    \item so long as premises and steps are available, then witnessing event.
    \end{itemize}
  \item Now, the task is to claim support for premises and steps.
  \item Key idea here is that agent does not appeal to ability.
  \item Instead, agent is appealing to those premises and steps in the same way they would do when witnessing \emph{and this doesn't require appeal to general ability}.
    \begin{itemize}
    \item it's not the case that I go `I can do arithmetic, so \(2 + 2 = 4\)'
    \item Rather, it's understanding \(2,+,=,4\), etc.
    \end{itemize}
  \item So, ability as a whole carries the \cprequ{}, but appeal to distinct components does not.
  \item This is really important to stress.
  \item The witnessing conditional (so to speak) comes from the information, not from ability.
    And therefore don't need to appeal to the ability to get the witnessing event --- only issue is whether it can be `made actual'.
  \item I mean, this is what is kind of puzzling about \EAS{}.
  \item I haven't \emph{used} any of this stuff, but claiming support by it anyway.
  \end{itemize}

  \begin{itemize}
  \item Objection here is that there's still a question about missing steps.
  \item Well, information is that all this stuff is sufficient.
  \item The only issue is whether the thing would really amount to a proof.
  \end{itemize}

  \begin{note}
    Interesting is that the above gets to specific ability.
    It's then not clear that the agent is required to get \(\psi\) from specific by the witnessing kind of thing, but this seems natural.
  \end{note}
\end{note}

\begin{note}[Setting expectations]
  So far we have seen \ESU{} requires \adA{}, and that \nI{} rules out \adA{}.
  The final thing to check, then, is whether \adB{} is compatible with \nI{}.

  Some care to be taken here.
  \adB{} only holds that the agent claims support by appeal to ability --- \AR{} and \WR{}.
  So, it is not obvious that \adB{} alone provides us with enough information to provide a complete defence that the agent does claim support.
  Rather, our goal is to provide an `in principle' defence that \adB{} need not conflict with \nI{}.

  The upshot of this is an avenue for further research.
  If \nI{} and scenarios, then either \adB{} or basic \AR{}.
  We will say more on this in section~\ref{sec:establishing-tension}.

  For now, I hope to have appropriately set expectations with respect to following argument.
\end{note}

\begin{note}[Working through the details]
  \ref{nI:background} \dots



  \ref{nI:inclusion} are satisfied in the scenarios of interest.
  So, question is \ref{nI:going-by-value}.

  Letter, but more importantly the spirit.

  Spirit goes back to \eiS{}.
  Saw about that \eiS{} was central to argument for \nI{}.
  In other words, the question is whether the agent claims support which holds up `even if' \(\phi\) turns out not to be the case.

  In other words, \adB{} does not lead to similar conflict with \eiS{}.

  Hence, the `in principle' defence that \adB{} need not conflict with \nI{} rests on showing that \ref{nI:going-by-value} is not necessarily satisfied when steps of reasoning are \adB{} with respect to ability.

  Focus step is ability to \(\phi\).

  So, agent claims from specific ability.
  Idea has been noted.
  Clearest with respect to \WR{}.
  Appeals to event, in particular the premises and steps of reasoning.
  In turn, reduces to observation that claiming support in this way does not require \RBV{}.
\end{note}

\begin{note}
  So, broad idea is claiming support from same premises and steps they would if they were to witness their ability.

  \adB{}, claim support by appeal to thing, and \WR{} provides sufficient detail about what that thing is.

  Question is whether \(\phi\) needs to be the case.

  Recall, \(\phi\) because moved to it being true that agent has the ability.

  This move doesn't happen with \adB{}.

  Key point is that agent claims support for property or event.
  The agent doesn't move to value.

  So, \gsi{} it's the parts of the general ability.
  And \aben{the} it's the premises and so on.

  Key point is that given background information, these allow the agent to claim support, even if it turns out the agent is \mom{}.
  Information is that that stuff is sufficient to claim support.

  Easiest with \WR{}.
  As, this is just the same as an instance of reasoning.
  The only difference is that the agent isn't clear on what's going on.

  Everything the agent has claimed support for allows them to make this move.
  Even if turns out things aren't right, and \mom{}, the agent seems to have enough, and by \adB{} they don't require that they aren't \mom{}.

  This, to my mind, is the key idea with ability.
  It informs the agent of something they have the ability to do.
  And, that thing functions in just the same way as it would if the agent were to do the thing.

  Claim support by appeal to that reasoning.
  Only going to be truly successful if I have the ability, for sure.
  However, claim support for ability even if \mom{}.
\end{note}

\begin{note}
  Key observation is that \adB{} doesn't go by value.

  However, there is a problem.

  For, it may seems as though the agent \emph{does} go by value because they require the premises, etc.

  This is clearest with the idea that:
  \begin{itemize}
  \item If \(\phi\) isn't the case, then some premise or step isn't part of ability.
  \end{itemize}
  Question about whether this gets a violation of \eiS{}.

  But, point is that agent at present is okay with claiming support that the reference resolves.

  So, this really isn't that problematic.

  Obviously it could break down.

  The point is that the agent at present outlines claim to support even if \mom{}.
\end{note}

\subsection{Objections - to be rearranged}

\begin{note}
  The main objection here is something along the lines of a stronger requirement on claiming support.
  Agent gets to keep claimed support for general or specific ability.
  Possible defeater isn't enough, though information introduces an \requ{}.
  Well, this should prevent appealing even to the premises and steps of reasoning.
  Avoided this because these do not require \requ{} of \(\psi\).
\end{note}

\subsubsection{\requ{3}}
\label{sec:requ1}

\begin{note}
  Objection here is that we've identified failure due to \requ{1}, roughly.
  And, ability to claim support, and this is going to involve some \requ{1}.
  However, no use so no reasoning about.

  Yes, this is somewhat difficult.

  Unsatisfactory response would be to observe that \requ{} is only defined with respect to reasoning performed.
  Unsatisfactory because only necessary conditions, and plausible that there are additional necessary conditions.

  Possible line of response is appeal to ability.
  However, this will lead to an instance of \nI{} all over again.

  Instead, witnessing ability is itself sufficient for claiming support.
  And, as would amount to claiming support, this will involve reasoning about \requ{}.
  Important, \EAS{} does not necessarily hold for anything weaker.
  \nI{} has been stated for reasoning quite broadly, so problem if the agent appeals to claiming support, or any other reasoning with interdependence.
  However, witnessing is not appealing in this way.
\end{note}

\subsubsection{Deny claimed support for ability}

\begin{note}
  So, if witnessing, then premises and steps are good enough to go for the conclusion.

  Main problem is that possibility that the relevant witnessing event is not possible.
  However, claimed support that it is.

  Still, suggestion that given possible defeater, the agent doesn't even get to appeal to claimed support for general/specific.

  Then, wouldn't get the details of the witnessing event.

  So, this is much stronger than {\color{red} Assumption ???}.
  Indeed, something like this would prevent event \ESU{}.
  Find this sufficiently implausible.

  The reasoning seems fine.
\end{note}

\subsubsection{Appeal to premises and steps requires appeal to ability}

\begin{note}
  It's true that the combination implies the ability, and so the combination seems to lead to the same problem.
  We get \(\psi\) as an \requ{} of combining all of the premises and steps.

  However, what we're relying on is appeal to the individual components.
  The thing here is that it seems fine for the agent to witness.
  This doesn't block claiming support.

  Hence, if this is the case then it can't be that the problem is simply what follows from the combination.
  Rather, it must be something about not witnessing.
  However, this returns us to \ESU{}.
  This is the very intuition that we're arguing against.
  Hence, the question is whether this really is something that is the case.
\end{note}

\section{Incompatibility of \nI{}, \gsi{}, and \adA{}}
\label{sec:ni-summary}

\begin{note}[Table]
    \begin{figure}[h]
      \centering
      \saMtxRuledOutLCS{}
      \repeatCaptionPrime{fig:saMtxRuledOut}{Distinction matrix}
  \end{figure}
\end{note}

\section{\nI{} isn't that strong}
\label{sec:ni-isnt-that}

\begin{note}
  Look, \nI{} rules out a way of claiming support quite broadly.
  However, this is because we're focusing on \aben{the}.
  This shouldn't be taken to suggest that there's general tension between \nI{} and \adA{}.
\end{note}

%%% Local Variables:
%%% mode: latex
%%% TeX-master: "master"
%%% End: