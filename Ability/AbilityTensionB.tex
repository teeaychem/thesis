\chapter{\nI{}, \gsi{}, and \aben{the}}
\label{sec:second-conditional}

\begin{note}[Redo of section]
  Seen \ESU{} and \adB{}.
  Now turn to \adA{}.
\end{note}

\begin{note}[Broad sketch of section]
  Introduce a general constraint on claiming support.
  The general constraint will relate to moving from general to specific ability information --- agent is not in a position to claim support for having specific ability from information and claimed support for general ability.
  However, initial statement and motivation apply to all instances of claiming support.
  After statement and motivation, show how the constraint relates to \adA{}.
  If so, agent lacks support for having specific ability, and does not have the option of claiming support for result of specific ability by \adA{}.
\end{note}

\begin{note}
  \color{red}
  \large

  The point of \nI{} isn't to find something that gets me something about ability.
  Given the assumptions made, it's fairly easy to figure out the problem.
  Rather, the point on \nI{} is to generalise the problem.
  The ideas we're appealing to aren't relying on anything in particular about ability.
\end{note}

\section{Overview of \nI{}}
\label{sec:ni-1}

\begin{note}
  We turn to the general limitation on claiming support.
\end{note}

\subsection{Statement of \nI{}}

\begin{note}[\nI{}]

  \begin{restatable}[\nI{-}  --- \nI{}]{proposition}{propNI}\label{prem:ni}
    Let \(\phi\) and \(\psi\) be propositions, \(v\) and \(v'\) values, and \(S\) an agent:
    \begin{enumerate}[ref=\named{\nIacro{}:\arabic*}, series=nI_counter]
    \item\label{nI:claimed-support}
      Suppose it is the case that:
      \begin{enumerate}[label=\alph*., ref=\named{\nIacro{}1:\alph*}]
      \item \nIClauseClaimedSupport{}\nolinebreak
        \footnote{
          The agent has at some time in the past (perhaps a moment ago) claimed support for \(\phi\) having \(v\), and at the present time the agent continues to hold that \(\phi\) has value \(v\).
          In some cases, an agent may revise basis for claimed support.
          E.g.\ this may be the case when appealing to memory, a new source, etc.
        }
        % \item\label{nI:received-info} \nIClauseReceivedInfo{}
      \item\label{nI:psi-is-new} {\color{red} \(S\) did not have information about \(\psi\) when claiming support for \(\phi\).}
      \end{enumerate}
    \end{enumerate}
    And, suppose:
    \begin{enumerate}[ref=\named{\nIacro{}:\arabic*}, resume*=nI_counter]
    \item\label{nI:inclusion} \nIClauseInclusion{}
      \begin{enumerate}[label=\alph*., ref=\named{\nIacro{}2:\alph*}]
      \item\label{nI:inclusion:position} \nIClauseInclusionPosition{}\nolinebreak
        \footnote{
          Agent considers.
          This should be stressed.
          And, in most cases, it is mistaken that's at issue.
          For, failure of \(\psi\) condition doesn't raise any issue in particular for \(\phi\).
        }\(^{,}\)\nolinebreak
        \footnote{
          Also, it may be the case that the statement of \nI{} needs further constraints on what the relation is between the claimed support for \(\phi\) and claiming support for \(\psi\) (without appeal to the value of \(\phi\)).
          I don't see a clear problem at the moment.
          However, I do expand below.
        }
      \item\label{nI:inclusion:bound} \nIClauseInclusioBound{}
      \end{enumerate}
    \end{enumerate}
    Then, for any reasoning such that:
    \begin{enumerate}[ref=\named{\nIacro{}:\arabic*}, resume*=nI_counter]
    \item\label{nI:going-by-value} \nIClauseValue{}
      \begin{enumerate}[label=\alph*., ref=\named{\nIacro{}3:\alph*}]
      \item\label{nI:going-by-value:phi} \nIClauseValuePhi{}
      \item\label{nI:goingbyvalue:psi} \nIClauseValuePsi{}
      \end{enumerate}
    \end{enumerate}
    The result of that reasoning is not an instance of \(S\) claimed support that \(\psi\) has value \(v'\).
    \vspace{-\baselineskip}
  \end{restatable}
\end{note}

\begin{note}[What \nI{} amounts to]
  \nI{} is a restriction on the way in which an agent may claim support for some proposition when certain conditions obtain.
  The proposition consists of a collection of `background' conditions --- clauses~\ref{nI:claimed-support} to~\ref{nI:inclusion} --- and a limitation given those background conditions --- clause~\ref{nI:going-by-value}.

  \ref{nI:claimed-support} and~\ref{nI:inclusion} combine to ensure that if \(\phi\) having value \(v\), then claimed support for \(\psi\) having value \(v'\) would violate \eiS{}.
  {
    \color{red} In particular, \ref{nI:psi-is-new} and~\ref{nI:inclusion} combine to ensure that \(\psi\) would be an expectation of the claimed support for \(\phi\).

    Argument for~\nI{} is similar to argument from~\ref{sec:claim-supp-expect}.
  }
\end{note}

\begin{note}[Expanding on~\ref{nI:inclusion}]
  Indeed,~\ref{nI:inclusion} is point of interest.



  Main thing will be expanding on:

  \begin{quote}
    \vspace{-\baselineskip}
    \ideaEIS*
  \end{quote}

  In particular:

  \begin{quote}
    \vspace{-\baselineskip}
    \assuEIS*
  \end{quote}

  Interdependence between claimed support for \(\phi\) having value \(v\) and claiming support for \(\psi\) having value \(v'\) given \ref{nI:inclusion}.
  If \(\phi\) has value \(v\), then claimed support for \(\phi\) is \nmom{}.
  Hence, claimed support for \(\psi\) having value \(v'\) would be \nmom{}.
  So, given background, \(\phi\) having value \(v\) requires \(\psi\) having value \(v'\).

  Problem, as claimed support for \(\psi\) having value \(v'\) --- via claimed support for \(\phi\) having value \(v\) --- `requires' \(\psi\) having value \(v\).

  {\color{green}
    If~\nIBackground{} hold, then the way of claiming support captured by~\ref{nI:going-by-value} is undercut.

    Specifically, undercutting arises from \ref{nI:inclusion} establishing a form of interdependence between claimed support for \(\phi\) and claiming support for \(\phi\).

    Such that in order for agent to appeal to \(\phi\) having value \(v\) the agent must assume that \(\psi\) has value \(v'\).
    Hence, can't go to \(\phi\) without \(\psi\), which means that can't use \(\phi\) to claim support for \(\psi\).
  }

  Though, leaves open claiming support that does not going via the claimed support for \(\phi\) having value \(v\).

  Two possibilities:
  \begin{itemize}
  \item Different approach to claiming support.
  \item Reasoning that removes expectation.
  \end{itemize}

  So, \nI{} is not particularly strong.
  However, not too weak either.
  \autoref{prop:CS-nai} with respect to second point.
\end{note}

\begin{note}[Intuitive idea]
  To bring out the intuitive idea:

  If agent considers possibility that they're \mom{}, then tension.
  For, \(\psi\), but `requiring' something agent is \mom{} about.
  Problem because if agent is \mom{}, then claimed support involves appeal to something which is no the case.
  Claimed support would be \mom{}.

  Hence, trouble with \autoref{assu:supp:independence}.

  This is fast.
  When turn to argument proper we will move more slowly.
  Further, offer variation on the argument which make use of additional stuff of \ref{nI:going-by-value}.\nolinebreak
  \footnote{
    Preview:
    Worry may be that \(\psi\) having value \(v'\) is not sufficiently involved in the reasoning.
    If from claimed support for \(\phi\), then get that this is \mom{}, and hence claiming support involves appeal to \mom{} premise, which is again sufficient for claimed support to be mistaken.
  }
\end{note}

\begin{note}[Plan]
  To start:
  \begin{enumerate}
  \item Core idea of \nI{}.
  \item Role in overall argument.
  \item Some intuition.
  \item Key points.
  \item Some connexions to the literature.
  \end{enumerate}
  Then, details.
\end{note}

\begin{note}[Kind of defeater]
  The basic idea behind \nI{} is that given clause~\ref{nI:claimed-support}, clause~\ref{nI:inclusion} captures a condition which \emph{undercuts} the agent claiming support as captured by~\ref{nI:going-by-value}.

  Mentioned undercutting {\color{red} \hyperref[first-mention-undercutting-defeater]{above}}
\end{note}

\begin{note}[Quick intuition]
  \emph{Undercutting} defeaters.

  Type II defeaters.

  \begin{quote}
    The second kind of defeater attacks the connection between \(P\) and \(Q\) rather than attacking \(Q\) directly.

    \mbox{}\hfill\(\vdots\)\hfill\mbox{}

    A type II defeater is any reason for believing that \({\sim}(P => Q)\) which is not also a reason for believing that \({\sim}Q\).\nolinebreak
    \mbox{}\hfill\mbox{(\cite[43]{Pollock:1974uk})}
  \end{quote}
  Where `\(=>\)' is the subjunctive conditional. (\Citeyear[42]{Pollock:1974uk})

  With \citeauthor{Pollock:1974uk}'s original formulation, the defeater attacks the link.\nolinebreak
  \footnote{
    See \textcite[196,fn.166]{Pollock:1999tm} for a brief note of the history of undercutting defeaters.
    \textcite{Pollock:1974uk} is more-or-less a direct expansion of discussion in~\textcite{Pollock:1970un}.
  }

  However, generalise.
  That there's a relation of claimed support, evidence, or what have you between \(P\) and \(Q\).

  Generalisation by \citeauthor{Bergmann:2005ws}\nolinebreak
  \footnote{
    I am also inclined to \citeauthor{Worsnip:2018aa}'s sketch of undercutting defeaters, which references~\citeauthor{Bergmann:2005ws}.
    \begin{quote}
      Undercutting defeaters, which are easiest to think of in the context of the attitude of belief, are supposed to be considerations that undermine the justification of a belief in a proposition p not necessarily by providing (sufficient) positive evidence to think that p is false, but rather merely by suggesting (perhaps misleadingly) that one’s reasons for believing p are no good, in a way that neutralizes or mitigates their justificatory or evidential force.\linebreak
      \mbox{}\hfill\mbox{(\Citeyear[29]{Worsnip:2018aa})}
    \end{quote}
  }
  \begin{quote}
    \emph{d} is an \emph{undercutting defeater} for \emph{b} iff \emph{d} is a defeater for \emph{b} which is (or is an epistemically appropriate basis for) the belief that one's actual ground or reason for \emph{b} is not indicative of \emph{b}'s truth.\newline
    \mbox{}\hfill\mbox{(\citeyear[424]{Bergmann:2005ws})}
  \end{quote}

  Similar generalisation, but similar to \citeauthor{Pollock:1974uk} the relation is useful.
  So, generalise the subjunctive conditional.
\end{note}

\begin{note}
  If~\nIBackground{} hold, then the way of claiming support captured by~\ref{nI:going-by-value} is undercut.

  {
    \color{red}
    Specifically, undercutting arises from \ref{nI:inclusion} establishing a form of interdependence between claimed support for \(\phi\) and claiming support for \(\phi\).
  }
  Such that in order for agent to appeal to \(\phi\) having value \(v\) the agent must assume that \(\psi\) has value \(v'\).
  Hence, can't go to \(\phi\) without \(\psi\), which means that can't use \(\phi\) to claim support for \(\psi\).
\end{note}

\begin{note}[\eiS{}]
  Idea of an undercutting defeater is quite general.
  Here, we'll develop a specific account.

  Undercut because failure of \eiS{}.
  This doesn't say anything about whether \(\psi\) has value \(v'\).
  Rather, if \eiS{} doesn't hold then the agent doesn't get to claim support {\color{red} due to lack of reasoning regarding recognised possible defeater}.
\end{note}

\begin{note}[Point of \nI{} is to motivate `independent' of ability]
  Purpose of conditions is to get undercutting of this kind.
  Of course, work is done by \eiS{}.
  And, in some respects, it is easier to reason directly.

  However, for our purposes this won't do.
  Our goal is to apply \nI{} to ability.

  Two problems.
  First, whether one gets this kind of failure.
  \nI{} address this by narrowing down on a small set of conditions that we can check.
  Second, whether the kind of failure matters.
  \nI{} address this by motivating the failure in general.
\end{note}

\subsection{Illustrations for intuition}

\begin{note}[A few illustrations]
  Let us now turn to a few illustrations before discussing \nI{} in further detail.

  We'll begin with a somewhat detailed illustration.
  \nI{} identifies a particular way in which an agent may fail to claim support, and the primary goal of the initial demonstration is to highlight why the agent would fail to claim support.
  Hence, the illustration treads a fine line between highlighting a problematic method, but not necessarily a problematic result.
  This is by design.
  And, I will continue to stress that \nI{} concerns a way of claiming support for some proposition, rather than the possibility of claiming support for some proposition.

  Following two illustrations will be variations on the initial.

  Still, it may be helpful to observe how \nI{} relates to an intuitively problematic result.
  Therefore, we will provide an additional, simple, illustration of a failure to claim support.

  The final illustration in the trio will complement the initial par of illustrations by highlighting an instance where~\nI{} does not apply.

  \phantlabel{dogmatism-wrt-nI}
  The reader may note structural similarities between these illustrations and \citeauthor{Kripke:2011wv}'s Dogmatism paradox.
  We will discuss the relation after the illustrations.
\end{note}

\begin{note}[Brief illustration of \nI{}]
  The first illustration considers theories and counterexamples.

  \begin{illustration}\label{ill:CE:main}
    Suppose a researcher have constructed a theory of some general phenomenon.

    The theory seems to capture the phenomenon, and the researcher has claimed (inductive) support that the theory is adequate by applying it to various instances of the general phenomenon.
    Even if the theory isn't adequate, the theory has been (seemingly) successful applied to sufficient specific instances of the phenomenon.
    Hence, even if \mom{}.

    However, as the phenomenon is a \emph{general} phenomenon it also makes various predictions about what must happen in all other instances to which the researcher has not (yet, at least) applied the theory to.

    Hence, there is a possible counterexample to the theory associated with each instance the researcher has not (yet) applied the theory to.
    If some particular instance does not conform to the theory, the theory is inadequate.
    Conversely, if the theory is adequate, every particular instance of the phenomenon conforms the theory.
    In other words, if the theory is adequate, then there are no counterexamples to the theory.

    Of course, it may be simple to revise the theory is a counterexample exists, and the fundamental ideas of the theory may remain sound (Cf.~\textcite{Bonevac:2011tz}).
    And, the theory may have sufficient resources to explain why any apparent counterexample is not a counterexample.
    Yet, it remains the case that the theory would need to be revised in light of a counterexample.

    Now, to summarise, the researcher may claim support for two propositions allow the agent to claim support that there are no counterexamples.

    \begin{itemize}
    \item The theory is adequate, and
    \item If the theory is adequate, then there are no counterexamples.
    \end{itemize}

    At issue is whether the researcher may claim support that there are no counterexamples to the theory from the claimed support for the two propositions in the following way:

    \begin{itemize}
    \item I have claimed support that the theory is adequate.
    \item So, given the claimed support, theory is adequate.
    \item Therefore, as the theory is adequate, given the claimed support, it follows that there are no counterexamples.
    \item Hence, I claim support that there are no counterexamples to the theory.
    \end{itemize}

    Seems problematic.
    Claimed support that the theory is adequate is qualified by the possibility of counterexamples.
    {
      \color{red}
      Note, agent is, here, only claiming support that there are no counterexamples.
      And, claiming support may be \mom{}.
      So, it does not follow that the agent is ruling out the possibility of counterexamples to the theory.
      Plausible that the agent \emph{may} claim support.
      Problem is the way in which the agent goes about this.
    }

    {
      Even if not convinced about support, this way of claiming support seems problematic.
      Relying on theory being adequate.
      However, if this is the case, then no possible counterexamples.
      Issue is that such counterexamples are possible given the state of your claimed support.
      Hence, claiming support in this way seems to take for granted that there are no counterexamples.
    }

    Problem is that the reasoning only works if there are no counterexamples.
    If there are counterexamples, misled.
    Hence, problem to go from the theory is adequate.
    However, without this step, researcher doesn't get to no counterexamples.

    So, this is \eiS{}.
  \end{illustration}

  So, relation between theory and counterexample \emph{undercuts} using way of using theory to get no counterexample.

  Now, given that the researcher has claimed support that the theory is adequate, the researcher may \emph{expect} that there are no counterexamples to the theory.
  And, it doesn't follow that the researcher may not claim support.
  Specific way --- reasoning captured by~\ref{nI:going-by-value}
  Plausible that details of the theory provide some way of claiming support.

  Indeed, it seems the researcher is require to take the alternative path --- to show that the proposed counterexample is accounted for by the contents of the theory, regardless of whether the theory is true.

  Fault here is with respect to \eiS{}.
  {
    \color{red}
    Here, conditions of~\ref{nI:inclusion} are satisfied, but we did not explicitly appeal to them.
    Purpose of~\ref{nI:inclusion} is conditions sufficient for this kind of problem to arise.
    So, to do in argument for \nI{} is to develop is why~\ref{nI:inclusion} does something similar.
    Upshot is that \nI{} is general.

    In the third illustration, we'll see why the way of claiming support is okay in some cases.
  }
  Difficult part is to account for why~\ref{nI:inclusion} sets things up and ensures that things don't go too far.
\end{note}

\begin{note}[Idea main part of \nI{} works]
  As noted above, it is unclear whether or not there may be some way for the researcher to claim that there are no counterexamples to the theory.
  And, if there is some way for the researcher to claim that there are no counterexamples to the theory, one may be inclined to wonder whether there really is a problem with claiming support in the way outlined by~\ref{nI:going-by-value}.

  In other words, one may be wondering whether \eiS{} is a plausible constraint on claiming support.
  We gave a general argument for \eiS{} in~\autoref{sec:abil-access-supp}.
  However, it may help to see how the issue highlighted relates to an intuitively problematic instance of reasoning, regardless of how support is claimed.

  \begin{illustration}\label{ill:CE:colleague}
    Suppose a colleague has studied the researcher's theory, and they (the colleague) thinks they have found a counterexample.

    The colleague has informed the researcher that they think they have observed a counterexample.

    However, the colleague has not provided the researcher with any further details about the counterexample.

    Now, the conditional of interest may be made more precise:
    \begin{itemize}
    \item If the theory is adequate, then the colleague has failed to identify a counterexample to the theory.
    \end{itemize}

    Now, let's replicate the way of claiming support from before.

    \begin{itemize}
    \item I have claimed support that the theory is adequate.
    \item So, given the claimed support, theory is adequate.
    \item Therefore, as the theory is adequate, given the claimed support, it follows that the colleague has failed to identify a counterexample to the theory.
    \item Hence, I claim support that the colleague has failed to identify a counterexample to the theory.
    \end{itemize}
  \end{illustration}

  I take this illustration to be intuitively problematic.
  In short, if claim support, then doesn't need to examine counterexample to claim support that it is not a counterexample.

  Possible response is that researcher does claim support, but information colleague impacts claimed support for theory.
  However, this is also puzzling.
  Researcher has no information.
  Hence, if retain confidence, then equally against counterexample.
  And, if does not retain confidence, then down the theory in a way that seems implausible.

  Seems, instead, that claimed support for theory persists, but that this doesn't extend to counterexample.\nolinebreak
  \footnote{
    Inclined to apply this to previous illustration.

    However, there's a difference between two illustrations.
    Here, someone (the colleague) has reason to think there is a counterexample, and this seems a sufficiently important difference to draw any quick conclusions.
    And, as we don't require a resolution to this issue, I won't explore further.
  }

  Perhaps more detail is needed.
  I have some doubts that claiming support is always bad.
  However, clearer that developed in a way such that problem remains.

  Now, seems that the researcher doesn't get to claim support because if counterexample, then theory is bad.
  Hence, requires that counterexample is not true in order to progress.
  But, then, doesn't make the move regardless of whether or not there is a counterexample.

  So, it seems \eiS{} does the work.
\end{note}


\begin{note}
  Undercuts using \(\phi\) for \(\psi\).
  Same problem, failure of \eiS{}.

  For, the agent has already assumed \(\psi\).

  Problem is that the agent doesn't get to claim support for \(\psi\) because fail the \eiS{} thing.
  If \(\psi\) isn't really the case, then reasoning collapses.
  Key thing about our understanding of claimed support is that it holds up even if the agent is \mom{} about the value of the proposition.

  {
    \color{red}
    Note:
    There's possible tension here.
    It seems that if the first illustration is okay, then this (second) illustration should also be okay.
    Maybe.
    But, this is too quick.
    Additional information here.

    Now, still some difficulty, as I think \EAS{} might apply to the first.
    So, shouldn't it apply to this?
    Well, no.
    For, \EAS{} only suggests possibility in some cases.
    Fine to think of this additional information as constraint on appeal via ability.
    For, if the colleague thinks they've found a counterexample, then this suggests a problem with the agent's ability.
  }
\end{note}

\begin{note}[Variation where \nI{} does not apply]
  \begin{illustration}\label{ill:CE:testimony}
    Suppose the researcher has published a paper containing the details of the theory.

    Our attention now turns to a novice who has read far enough into the paper to understand, at least, the general phenomenon that the theory applies to and that the researcher has claimed inductive support for the theory.
    We'll also assume that the novice does not possess the expertise required to apply the theory.\nolinebreak
    \footnote{
      Though I don't think this assumption is important.
    }

    The novice is thinking about instances of the general phenomenon, and identifies one.

    The conditional of interest is:
    \begin{itemize}
    \item If theory is adequate, it accounts for this instance of the phenomenon.
    \end{itemize}

    Of course, the novice also recognises that the theory is inadequate if it  does not account for the particular instance of the phenomenon.
    Still, the novice claims support in the familiar manner.

    \begin{itemize}
    \item I have claimed support that the theory is adequate (this time by reading a published paper).
    \item So, given the claimed support, the theory is adequate.
    \item Therefore, as the theory is adequate, given the claimed support, it follows that the theory accounts for this instance of the phenomenon.
    \item Hence, I claim support that the theory accounts for this instance of the phenomenon.
    \end{itemize}
  \end{illustration}

  In contrast to the previous illustrations, it seems the novice may claim support in such a way.

  Possibility of being either \mom{} remains.
  Still, not in position to reason through theory and phenomena.
  Hence, claiming support from something like status of peer review --- or testimony.
  And, not accounting would not show peer review is bad.
\end{note}

\begin{note}
  These three illustrations.
  First, kind of scenario that's the main interest.
  Where claiming support in a certain way seems problematic, even if it not clear that the agent may claim support in some other way.

  To stress the problem, considered a cleaner case, where it seems agent may not claim support, and argued that same problem is a plausible account of why.

  Third illustration, way of claiming support is okay.
  As all instances of \nI{}, and hence the previous two illustrations, focus on particular way of claiming support illustrated that it's okay.
\end{note}

\begin{note}[Intuition]
  In short, \nI{} captures a limitation: An agent is not in a position to claim support for some proposition \(\psi\) when circumstances are such that the claimed support requires (from agent's point of view) that the agent is already in position to claim support for \(\psi\).

  No claiming support for\(\psi\) if failure to establish support for \(\psi\) independently of the value of \(\phi\) would reveal problem with the support claim for \(\phi\).

  Hence, \nI{} focuses on when an agent may claim support for some proposition by noting that (from the agent's perspective) that the value of the proposition is determined by further propositions the agent has claimed support for.

  Some other way of claiming support for \(\psi\).
  However, not merely an alternative path, but an alternative path that must be possible given claimed support for \(\phi\).

  Issue is that given \ref{nI:claimed-support} and~\ref{nI:inclusion}, agent expect that they have the resources, and hence expects that \(\psi\) is the case.

  So, that \(\phi\) has value \(v\).
  In doing so, resources to claim support for \(\psi\) has value \(v'\).
  Hence, that \(\psi\) has value \(v'\).
  So, \(\psi\) having value \(v'\) is a requirement on claimed support for \(\phi\) being any good.
  However, no support claimed for \(\psi\) having value \(v'\).

  In cases of reasoning with a conditional, such as the illustrations given, that value of \(\phi\) constrains value of \(\psi\) is in general helpful information, but in these specific cases it does not help the agent claim support for \(\psi\) having value \(v'\) because if \(\psi\) isn't already so constrained, then no appeal to \(\phi\) having value \(v\).

  Similar to other principles, failure because establishing something that needs to be the case in order to be in a position to establish.
\end{note}

\subsection{Moving on\dots}

\begin{note}[Task]
  Two important tasks with respect to arguing for \nI{}:

  \begin{itemize}
  \item \ref{nI:inclusion} and~\ref{nI:inclusion:position}, how these function.
  \item And, why \nI{} holds.
  \end{itemize}

  Then, additional details.
\end{note}

\begin{note}
  Before continuing I would like to stress two general points.
  Follow up on relation to \citeauthor{Kripke:2011wv}'s Dogmatism paradox.
  Link to similar principles in the literature.
\end{note}

\begin{note}
  Two quick points to stress.
\end{note}

\begin{note}[First point to stress]
  First, \nI{} outlines sufficient conditions for a limitation on claiming support to obtain.
  Hence, there may be other limitations on claiming support.
  For example, \ESU{} implies a limitation of claiming support --- an agent may not claim support by appeal to some premises or step of reasoning that they have not used.

  Similarly, there may be other limitations on claiming support with obtain.
\end{note}

\begin{note}[Second point to stress]
  Second~\ref{nI:going-by-value} is a limitation of a \emph{way} of claiming support.
  So, \nI{} does not imply that the agent is not in a position to claim support for \(\psi\), only that one way of claiming support is ruled out when the other clauses of \nI{} hold.
  Hence, \nI{} is compatible with the agent claiming support for \(\psi\) having value \(v'\) --- so long the agent doesn't follow the pattern captured by~\ref{nI:going-by-value}.

  Indeed, I take the primary upshot of \nI{} to be a demand for understanding alternative ways of claiming support when each clause of \nI{} holds and it seems that the agent may claim support for \(\psi\) having value \(v'\).
  And, after arguing for \nI{} our attention will turn to examining how an agent may claim support by \EAS{} when clauses~\ref{nI:claimed-support} obtain.

  For now, however, our focus is on arguing for \nI{}.

  It is perhaps also helpful to flag that the following discussion of \nI{} will relate previous details on claiming support (in particular section~\ref{sec:abil-access-supp}) but will proceed without consideration of ability.
  Our goal is to motivate \nI{} as a general limitation on claiming support, and in turn draw a consequence from such a limitation when applied to ability.
\end{note}

\begin{note}[Literature]
  We will highlight the contrast between \nI{} and similar principles in~\autoref{sec:contr-other-cond}, below.
  In particular, \citeauthor{Wright:2011wn}'s work on transmission failure (\Citeyear{Wright:2003aa,Wright:2011wn}) and \citeauthor{Weisberg:2010to}'s No Feedback condition (\Citeyear{Weisberg:2010to}).

  For now, it may be helpful to highlight that \nI{} does not deny that any agent may claim support for \(\psi\) having value \(v'\).
  Rather,~\ref{nI:going-by-value} (only) denies that the agent may claim support for \(\psi\) in a specific way.
\end{note}

\newpage

\section{Details of --- and argument for --- \nI{}}
\label{sec:details-ni}
\label{sec:re-do-ni}

\subsection*{Argument outline}

\begin{note}
  Our task is now to demonstrate why~\ref{nI:going-by-value} is the case when \nIBackground{} obtain.

  We begin with a brief statement of how the argument is structured, together with the basic idea behind the argument.
  Then, we turn to the argument proper.
  Finally, we finish with a handful of observations.

  In the following section (\ref{sec:illustrations-ni}), with the argument complete, we will consider a number of additional illustrations of \nI{}.
  Then, in~\ref{sec:contr-other-cond} we will contrast~\nI{} to a pair of similar conditions from the literature.
  And, at last, we will apply~\nI{} to ability in~\autoref{sec:ni-ability}.
  In particular, why~\nI{} applies to ability paired with \adA{} (\S\ref{sec:ni-ability:adA}) but not when ability is paired with \adB{} (\S\ref{sec:ni-ability:adB}).
\end{note}

\begin{note}
  Key proposition for \nI{} is:

  \begin{quote}
    \vspace{-\baselineskip}
    \propCSNai*
  \end{quote}

  For, our goal is to show that move to \(\phi\) brings \(\psi\) with.
  Hence, getting to \(\psi\), even if \(\psi\) is not explicit, means that we get a failure of claiming support.

  To get here, relationship between claimed support for \(\phi\) and claimed support for \(\psi\).

  \begin{enumerate}
  \item Claimed support for \(\psi\) having value \(v'\) is an expectation of the claimed support for \(\phi\) having value \(v\).
  \item Going to \(\phi\) having value \(v\) includes \(\psi\) having value \(v'\) as a result of previous.
  \end{enumerate}
  Hence, \ref{prop:CS-nai} applies.
  More broadly, \eiS{}.
  Claiming support for \(\psi\) so recognise \(\psi\) might not be the case.
  However, claimed support is such that it requires \(\psi\) is the case.
  Hence, no account of \(\psi\) given this possibility.
\end{note}

\begin{note}
  In broad outline.

  \begin{itemize}
  \item Claimed support for \(\phi\).
  \item No reasoning about \(\psi\).
    \begin{itemize}
    \item No claimed support for \(\psi\).
    \item No account of why \(\phi\) regardless of \(\psi\). (I.e.\ weaker than claimed support, if think this is possible).
    \end{itemize}
  \item From inclusion, this means that claiming support for \(\psi\) is an expectation of claimed support for \(\phi\).
  \item Consequence of expectation is that \(\psi\) is a requisite of the move to \(\phi\).
    In sense that:
    \begin{itemize}
    \item Move to \(\phi\) is not compatible with possibility that \(\psi\) is not the case.
    \item Also, as result of this, move to \(\phi\) requires \(\psi\) to be the case.
    \end{itemize}
  \item So, reasoning to \(\psi\) is not compatible with the possibility of \(\psi\) not holding.
  \item And, as way of claiming support needs this move, failure.
  \end{itemize}

  This is somewhat complex.
  So, to summarise.
  The key problem is that reasoning to \(\psi\) is not compatible with the possibility that \(\psi\) does not have value \(v'\).
  And, as a result of this the reasoning is not an instance of claiming support.

  Of course, still an instance of reasoning, and result may be of interest.
  However, no independence.

  {
    \color{red}
    Compare to conditional with contraposition.
  }
\end{note}

\begin{note}
  To do is expand on this quick sketch.

  At issue is not only how these clauses link together, but what the clauses amount to.
  Both these things are important.
  Still, to simplify matters split in to two sections.

  In the first section, walk through the clauses --- in particular \ref{nI:inclusion}.
  In the second, link the clauses together.

  Read these in either order.
  Split is in part so that the first section may expand on some aspects of the clauses which are important for understanding the scope of \nI{}, but are not required to obtain the key consequence.
\end{note}



\begin{note}
  Following outline given.
  Assume clauses hold.

  Main thing is `requisite'.

  So,
  \begin{enumerate}
  \item Preliminaries
  \item Requisite
  \end{enumerate}
  Then, tension.
\end{note}

\section{Two things}
\label{sec:nI:arg:clauses}

\begin{note}
  Two subsections.

  First,~\ref{nI:inclusion}.
  Second,~\ref{nI:going-by-value}.

  Add further information regarding these.
  Not required for argument.
\end{note}

\subsection{~\ref{nI:inclusion}}

\begin{note}[\ref{nI:inclusion}]
  Restated:
  \begin{quote}
    \begin{enumerate}
    \item[\ref{nI:inclusion}]
      \nIClauseInclusion{}
      \begin{enumerate}
      \item[\ref{nI:inclusion:position}] \nIClauseInclusionPosition{}
      \item[\ref{nI:inclusion:bound}] \nIClauseInclusioBound{}
      \end{enumerate}
    \end{enumerate}
  \end{quote}
\end{note}

\begin{note}
  \begin{itemize}
  \item \ref{nI:inclusion:position}.
    \begin{itemize}
    \item \(\psi\) having value \(v'\) is the case, granting claimed support \(\phi\) having value \(v\) is the case goes through.
    \item Not the result of appealing to \(\phi\) having value \(v\).
      \begin{itemize}
      \item So, agent considers it possible to claim support for \(\psi\) having value \(v'\) without \(\phi\).
        Hence, way of removing expectation of \(\psi\) from claimed support for \(\phi\).
      \end{itemize}
    \end{itemize}
  \end{itemize}
\end{note}

\begin{note}
  {
    \color{red}
    The point of these parenthesis is that the relevant way of claiming support isn't the one that's going to give rise to the issue.
    Otherwise, it's somewhat trivial.
    Still, I feel there is a much better way of expressing the parenthetical parts of these clauses.
    I mean, the point would be that any instance is always going to be self defeating.
    But, the problem isn't \(\phi\) having value \(v\), rather it's there being some other instance that `forces' \(\psi\).
  }
  {
    \color{blue}
    Well, the interdependence here is really about the way of claiming support for \(\phi\) requiring that agent also claims support for \(\psi\).
    In this sense, \(\psi\) is a presupposition for the method of claim support.

    If agent requires implication, then this should break down, because it won't be the case that the agent needs \(\psi\) to hold up in order for the way in which the agent claims support for \(\phi\) to be okay.

    In this sense, the difficulty is whether these two conditions really capture this idea.
    It's the way of claiming support that's related, rather than \(\phi\) and \(\psi\).

    Could I make this explicit?
    Position to claim for \(\psi\) in same way as \(\phi\).
    And, that's successful only if claiming for \(\psi\) is also good.
  }
\end{note}

\begin{note}[Two clauses]
  The two sub-clauses of~\ref{nI:inclusion} are separated by~\ref{nI:inclusion:position} being a conditional with an antecedent and consequent that describe present circumstances while \ref{nI:inclusion:bound} is a conditional with an antecedent that describe present circumstances and a subjunctive consequent.

  {
    \color{red}
    The key this here is that if claimed support for \(\phi\) is not misled, then \(\psi\) has value \(v'\) --- that's the importance of being in position.
  }

  In turn, each sub-clause is stated from the perspective of possible defeaters, but as a result both sub-clauses carry certain implications if the agent is confident that they have successful claimed support for \(\phi\) having value \(v\).

  First, as stated, \ref{nI:inclusion:position} ensures that the agent considers their claimed support \(\phi\) having value \(v\) is mistaken or misled if they are not in a position to claim support that \(\psi\) has value \(v'\).
  In turn, if the agent is confident that they have claimed support \(\phi\) having value \(v\) then the agent should be equally confident that they are in a position to claim support that \(\psi\) has value \(v'\)

  Second,~\ref{nI:inclusion:bound} ensures that the agent is confident that if they were to claim support for \(\psi\) having value \(v'\) then such claimed support for would be \mom{} if their claimed support for \(\phi\) having value \(v\) is \mom{}.\nolinebreak
  \footnote{
    Note, however, that the subjective element of this contraposed form is restricted to the antecedent of the conditional.
  }
  In turn, if the agent is confident that they have claimed support \(\phi\) having value \(v\) then the agent should be equally confident that claimed support for \(\psi\) having value \(v'\) would not be \mom{} if they were to do so.

  The two sub-clauses are closely related, but are distinct.
  It is possible for either to hold without the other.

  \begin{illustration}
    Suppose taught addition.
    Has been told that multiplication reduces to addition, but has not been informed of the details.
  \end{illustration}
  \ref{nI:inclusion:bound} holds but~\ref{nI:inclusion:position} does not.

  \begin{illustration}
    Suppose calculator from scratch.
    Good with arithmetic, but less good with programming.
  \end{illustration}
  \ref{nI:inclusion:position} holds, but~\ref{nI:inclusion:bound} does not.
  For, a whole bunch of additional stuff.
  However, revise scenario so that~\ref{nI:inclusion:bound} does hold.

  So, distinct.
  \ref{nI:inclusion:position} in a position and~\ref{nI:inclusion:bound}, binds.
\end{note}

{
  \color{red}
  Seen examples above.
}

\begin{note}
  `Confidence'

  Then, each sub-clause separately.

  Finally, link to literature.
\end{note}

\begin{note}[`Confidence']
  Our use of the term `confidence' does not require the agent to have claimed support for the conditional content of \ref{nI:inclusion}.
  Nor does out use of `confidence' imply that the claimed support for \(\phi\) \emph{is} mistaken or misled given the identified conditions.
  We are interested only in what makes sense from the agent's perspective.
  Nearby reformulations of \ref{nI:inclusion} may also be true, but confidence is sufficient to recognise a problem.
  To illustrate: If I am confident that the water is poisoned, then regardless of whether I claimed support for the water being poisoned, I will not drink it.
\end{note}

\subsubsection{\ref{nI:inclusion:position}}

\begin{note}[Points that will be covered]
  We will focus on two parts of \ref{nI:inclusion:position}.

  First, on what it is for an agent to be in a position to claim support for some proposition (given present context), as this is an unfortunate source of imprecision.

  Second, we'll clarify the restriction that the agent does not appeal to \(\phi\) having value \(v\) --- though this will be further explored when discussing~\ref{nI:inclusion:bound}.
\end{note}

\begin{note}
  First, an admission.
  Key role here is that get a problem with the claimed support for \(\phi\) having value \(v\).
  Possible that some other condition could do the work.
  However, this is sufficient for the purposes of this paper.
\end{note}

\begin{note}[`(given present context)']
  Imprecise.
  No clear account of what it is to `be in a position' nor what `present context' amounts to.

  Unfortunately, I have no simple characterisation for either.
  Best I have to offer is that the agent is not prevented from claiming support by lack of resources.
  However, it seems to me that isn't really substantially different.
  Resource is broad enough to mean whatever is required for the agent to claim support.
  And, bound to context.

  Instead, walk through considerations with respect to some scenarios.
  Consider issues.
  Argue that don't need to do better than imprecision.
\end{note}

\begin{note}
  Pair of easy cases.
  Clarify to some extent in a position.

  Then, some harder cases.
\end{note}

\begin{note}[Flavours]
  Consider variations on a case where information relating \(\phi\) and \(\psi\) comes from a third party.

  You'll enjoy this flavour of ice cream.

  Claimed support from testimony, roughly.

  Seems fine when discussing things on the train.

  More difficult when tasters are available.
  Here, \ref{nI:inclusion} holds.
\end{note}

\begin{note}
  Difference is that in the first, no way of checking.
  In the latter, there's a way of checking.

  Only issue here is `testimony'.
  With respect to testimony, the response is that in cases of testimony, one can be thought of as appealing to a general truthful property, rather than always truthful.
  So, it's not obvious that testimony is always going to give rise to an instance of \ref{nI:inclusion}.
\end{note}

\begin{note}
  Receiving a letter.
  Unmarked, okay, it's for me.
  Marked, check the address.
\end{note}

\begin{note}[More difficult]
  \dots
\end{note}

\begin{note}[Illustration, testimony]
  To illustrate, consider expert testimony to a layperson.
  Suppose you, the expert, have testified to me, the layperson, that there are exactly five intermediate logics that have the interpolation property.\nolinebreak
  \footnote{Cf.\ \textcite{Maksimova:1977un}}
  From this it follows that there is an intermediate logics that has the interpolation property.
  However, I am quite confident that I would not be in a position to claim support for the latter proposition without your testimony.
  Given that I do not have the expertise involved, any failure by me to claim support that there is a intermediate logic with the interpolation property is uninformative.
  Likewise, given that I am a layperson I'm not in a position to rule out that there aren't intermediate logics with the interpolation property, and therefore I may consider this a potential defeater to your testimony.\nolinebreak
  \footnote{
    Additional example: reports of internal states.

    I have a virus scanner.
    Run this on your pc.
    Also, a test pc.
    Test PC contains a know virus, so if the virus scanner is good, then it will identify infection.
    However, no relation between your PC and my test PC.
    All that would be established is that the scanner is not good for claiming support.
    }

  Still, given \ref{nI:claimed-support}, agent may expect \(\psi\) to have value \(v'\), and may claim support.
  And, may expect to have the resources to claim support for \(\psi\) without appealing to \(\phi\) having value \(v\).
  To illustrate, suppose you and I are both experts.
  You claim to have developed a sound and complete proof system for an logic and presented me with a paper containing the system and a proof.
  Given that I have the paper and the expertise, I am confident that I would be mistaken or misled by your testimony if I am not in a position to claim support that the system is sound and complete by working through the paper.\nolinebreak
  \footnote{
    Here, complexity of understanding of having resources shows.
    For, it may be that the reader learns something new, a lemma etc.\ which could be considered a new resource.
    Likewise, one may think that it's fine to continue to follow testimony given a problematic proof as one is confident that the prover has the resources to revise the proof.
    If so, not clear whether conditional holds, and will depend having resources.
    If proof synthesises resources, then may still hold.
    If proof introduces new information, then conditional does not hold.

    No clear answer for these cases.
    Intend to be compatible with your understanding of resources.
    Will only take a stance on this when applying.
  }
\end{note}

\begin{note}
  Even more difficult
\end{note}

\begin{note}
  Coworker.
  Rely on colleague, as the agent doesn't have access to the file.
  But, access is granted quickly after hearing from the colleague.
\end{note}

\begin{note}
  These cases are harder within the broader context of \nI{}.
  Deny \RBV{}.
  Issue is that in both cases, result seems excessive.

  Well, first thing to do is to check that the agent really is claiming support.
  Fine, it seems, for the agent to stop at claiming support for the other agent, and not going any further.
  See no reason to hold that the agent must claim support.

  Still, this isn't quite satisfactory.
  Doesn't seem that bad, and the above suggestion requires a careful understanding of when an agent is required to claim support.
  So, what if the agent does claim support?

  As noted above, views on testimony can sort this out.
  First, by going for `weak' testimony.
  Second, by breaking \ref{nI:inclusion:bound} is the testimony turns out to be a mistake.
  Look, it's not obvious why it would make sense for the agent to claim support, but the point is that \nI{} wouldn't hold.

  Alternatively, Simple restriction is for first time claiming of support.
  Difficulty is a variation of the expert case.
  It isn't obvious that one gets to claim support for the stuff learnt as a layperson when one develops expertise.
  For example, translation between languages.
  Claim support for simple translation.
  When fluent, seems claimed support is distinct, based on broader understanding of the language.
  Not relying on simplifications in learner's dictionary.

  However, other ways of claiming support may also work.
  Arguing for one such way.
  Besides this, \RBV{} is quite strong.
  And, who knows about other types of reasoning.
  In particular, ways in which reasoning \adA{} might work.
\end{note}

\begin{note}[Uninspiring]
  These responses aren't particularly inspiring.

  However, let's look at this from a different angle.
  What's going to follow from insensitivity to context?
  End up with claiming support that does not depend on whether or not the agent is in a position to deal with defeaters.

  Well, the first option is that these never matter.

  Some kind of built in support.
  This comes up with ~\citeauthor{Pryor:2012tq}'s dogmatism (\cite{Pryor:2000tl,Pryor:2012tq}) and various ideas about entitlement (Wright, Burge, etc.)
  For example, Pryor's dogmatism for perception, just having the experience is good enough.
  Question about these kinds of defeaters.
  Reads to me that these kinds of things mean that \ref{nI:inclusion} will never hold.

  Question is whether this extends to all cases, so that \nI{} is trivial, but before pressing this seems too strong.
  Problems in various cases.
  The red room, but in the corner is a switch, flipped to off, but says it's broken.

  Context makes a difference.
  So, to this extent, looks as though there's going to be difficult cases.
\end{note}

\begin{note}[Following doesn't depend on difficult cases]
  Of course, this isn't a general defence of the clauses.
  Rather, that such difficulties can't be avoided.
  Upshot here is that we aren't really interested in such difficult cases.
\end{note}

\begin{note}[`(without appealing to \(\phi\) having value \(v\))']
  The parenthetical clause `(without appealing to \(\phi\) having value \(v\))' ensures that \ref{nI:inclusion} may only be true when the agent is confident that they are in a position to claim support for \(\psi\) having value \(v'\) independent of whether \(\phi\) has value \(v\).

  In this respect, \ref{nI:inclusion} requires an independent check on the claimed support for \(\phi\) having value \(v\).\nolinebreak
  \footnote{
    Note also that without the parenthetical clause, \nI{} would deny the possibility of any instance of the reasoning described in \ref{nI:going-by-value}.
  }

  Indeed, not being in a position to claim support for \(\psi\) having value \(v'\) (without appealing to \(\phi\) having value \(v\)) as a potential defeater to claimed support for \(\phi\) having value \(v\) is distinct from the potential defeater of \(\psi\) not having value \(v'\).
  For, an agent may consider \(\psi\) not having value \(v'\) is a potential defeater given \(\phi \rightarrow \psi\) while being confident that they could not be in a position to claim support for \(\psi\) having value \(v'\) without claimed support for \(\phi\) having value \(v\).

  \begin{itemize}
  \item Illustration
  \item No restriction on why conditional is true.
    \begin{itemize}
    \item Distinction between two ways in which it might be true.
    \end{itemize}
  \end{itemize}
\end{note}

\begin{note}[Inclusion and Association]
  \color{red}
  The illustrations provided offer some intuition, and it seems these will have to do.
  For example, one may consider `in a position' to mean that the agent does not require any novel resources to claim support.
  However, an agent may need to synthesise more fundamental concepts when following a proof, and it is unclear whether the synthesis is `novel information'.
  Similarly, it is difficult to say what the present context is when an agent may phone a friend as a source of testimony.
  In some cases, corroborating testimony may be sufficient to claim support (another plausible instance of `novel information'), while in other cases at issue may be the agent's own understanding (e.g.\ with respect to cases of Inclusion).
  In defence of this latent ambiguity, the specifics will not matter when arguing for the truth of \nI{}.
  And, I suspect the cases to which we apply \nI{} will be sufficiently clear cut.
\end{note}

\subsubsection{\ref{nI:inclusion:bound}}

\begin{note}[Inclusion and Association]
  For \ref{nI:inclusion:bound} we will consider two general ways in which \ref{nI:inclusion:bound} may be true, and provide examples for both.

  The task, then, is to account for why claimed support for some proposition being \nmom{} may imply that claiming support for some other proposition would be \nmom{}.

  We term the ways \incl{} and \asso{}, respectively.
\end{note}


\begin{note}[Inclusion]
  \incl{} is when the same (primary) resources used to claim support for some proposition may be (re)applied to establish a distinct proposition.

  To see why \incl{} leads to instances of~\ref{nI:inclusion:bound} suppose:
  The agent is confident that support claimed for the initial proposition is \nmom{}.
  And, the agent is confident that \incl{} holds with respect to the initial proposition and some other proposition.
  In turn, the agent will use the same resources to claim support for the distinct proposition.
  Therefore, if the agent is confident that the claimed support is \nmom{} for the former proposition then the claim supported must be \nmom{} for the latter proposition, else the agent should not be confident that their claimed support for the former proposition is \nmom{}.

  \begin{illustration}
    Consider claimed support that \(6^{2} \times 6^{3} = 6^{5}\).
    Support has been claimed by understanding basic properties of exponents.
    Hence, an agent may be confident that they are in a position to claim support that \(3^{15} \times 3^{12} = 3^{27}\).
  \end{illustration}
  Indeed, working through problem exercises in a textbook is way of ensuring that one has understood such principles.
  Not that textbooks typically ask for the working out.\nolinebreak
  \footnote{
    Though sometimes.
    For a highly specific example, consider constructing canonical models to prove completeness for various normal modal logics.

    The exercises in textbooks such as~\citetitle{Blackburn:2002aa} require the reader to consider a specific system, so there's no surprise that, e.g., \textbf{K1.1} is sound and complete with respect to the class of frames with a relation function that mirrors a partial function.
    Rather, the task of the exercise is to ensure the reader understands how to reason with canonical models, and if the reader has claimed support with respect to \textbf{K1.1} which is \nmom{} then they should be confident that their claimed support will be \nmom{} when they tackle \textbf{K4.3}.

    See \textcite[210]{Blackburn:2002aa} for the respective exercises.
  }

  {
    \color{red}
    Note here that this is the sort of thing that seems most likely in counterexample type cases.
  }
\end{note}

\begin{note}[\asso{}]
  \asso{} is when claiming support for some proposition ensures the agent is in a position to appeal to some distinct collection of resources for some other proposition.

  \begin{illustration}
    Example here is something like storing a guide in a document.
    Here, the agent has created the document, \(\phi\) is just that the document has all of the info.
    So, if document is good, then contents for \(\psi\).
    This is different, as \(\phi\) is a now a check on the stuff in the document working out.
  \end{illustration}

  Also, ice cream example from above.
\end{note}

\begin{note}[Ways in which \ref{nI:inclusion} may fail to hold]
  Finally, the illustrations given have focused on instances for which \ref{nI:inclusion} holds.
  It is important that there are instances where \ref{nI:inclusion} holds, but it is equally important not to suggest that these are in abundance.
\end{note}

\begin{note}[Failures for~\ref{nI:inclusion:position}]
  There are various ways in which \ref{nI:inclusion} may fail to hold.
  For example, if \(\phi\) is sufficiently general or probabilistic.
  If so, not having the resources to claim support \(\psi\) may not establish much.
\end{note}

\subsubsection{\ref{nI:inclusion:position} and~\ref{nI:inclusion:bound} combined}

\begin{note}[Failure for zebra case]
  Here, not obvious that holds for zebra case, as it's not clear there is an alternative.
\end{note}

\begin{note}[Literature]
  {
    \color{red}
    Place this here as it helps clarify why~\ref{nI:inclusion:position} and~\ref{nI:inclusion:bound} seem to work well together.
  }
  The circularity here is similar to that proposed by~\cite{Sgaravatti:2013wu}

  \begin{quote}
    \begin{itemize}
    \item[(JBA)] An argument A is circular relative to an evidential state E iff in order for a subject S in E to have a justified belief in each one of A’s premisses, it is necessary that S has a justified belief in A’s conclusion\nolinebreak \mbox{}\hfill\mbox{(\Citeyear[759]{Sgaravatti:2013wu})}
    \end{itemize}
  \end{quote}
  Relative to an agent, really.

  Talk about claiming support, rather than evidence.
  And, details of why it's necessary.
  Also, contrast to some of the details.

  Well, interesting is:
  \begin{quote}
    For my present purposes it will suffice to say that a good test of A’s being necessary for B (and thus of B’s being sufficient for A) is the satisfaction of two subjunctive conditionals. First, if A did not hold, B would not hold; secondly, if B were to hold, A would hold.\nolinebreak
    \mbox{}\hfill\mbox{(\Citeyear[761]{Sgaravatti:2013wu})}
  \end{quote}
  This is very similar to \ref{nI:inclusion}.
  A is \(\psi\) and B is \(\phi\).\nolinebreak
  \footnote{
    \ref{nI:inclusion} was developed independently, though this is probably no surprise given how clumsy~\ref{nI:inclusion} is.
  }
  However, need \(A \leadsto B \vdash A \rightarrow B\).

  Still, relative to a certain way of claiming support.
  It's not the case that this idea holds in general.

  The difference is that we're not dealing with circularity nor justified beliefs.
  The issue here is not that the agent needs a justified belief that \(\psi\) is the case.
  Rather, it's the claimed support is such that expectation is a problem.
  Intuitively, the task of the agent is to avoid expectation.
  But doing so does not amount to forming a justified belief.
  Instead, it not mattering whether \(\psi\) has value \(v'\).

  Applied to illustrations.
  Don't need justified belief against each counterexample for theory.

  Still, similar with respect to strength of connexion.
\end{note}

\begin{note}[Different example.]
  \color{red}
  Recall \autoref{illu:CS:tfc} --- truth functional completeness.
  Instead, don't go by value.
  Rather, go by interdefinability.
  Indeed, interdefinability plays an important role in getting this condition.
\end{note}

\subsection{Value}

\begin{note}
  `\RBV{-}'

  More or less factive reasoning.
\end{note}

\section{Linking}
\label{sec:nI:arg:linking}

\subsection{Laying out the clauses}

\subsubsection{\ref{nI:claimed-support}}

\begin{note}
\ref{nI:claimed-support} and \ref{nI:psi-is-new} are simple background conditions.

First, \ref{nI:claimed-support} ensures the agent has claimed support for \(\phi\) having value \(v\).
Hence, agent may appeal to \(\phi\) having value \(v\) by the claimed support for \(\phi\) having value \(v\) in further reasoning, and in particular reasoning which culminates in claiming support for some other proposition.

Second, \ref{nI:psi-is-new} ensures that the agent has not considered whether \(\psi\) has value \(v'\).

As a result, it is possible that \(\psi\) not having value \(v'\) is an unrecognised defeater, and hence it is possible that \(\phi\) having value \(v'\) will come to be an expectation of the agent's claimed support for \(\phi\) having value \(v\).

Note, though, that as the agent has not considered \(\psi\), these observations about \(\psi\) were not made when claimed support for \(\phi\) having value \(v\).

Little else of interest follows from \ref{nI:claimed-support} and \ref{nI:psi-is-new} alone.
For example, the agent may never come to consider \(\psi\).
Or, \(\psi\) may be such that it wouldn't show \mom{}.

I hadn't considered that that author has a habit of misquoting articles, but I am quoting directly from the original article\dots

Rather, possibility for \(\psi\) having value \(v'\) to become an expectation of the claimed support for \(\phi\) having value \(v\).
\end{note}

\subsection{\ref{nI:inclusion}}

\begin{note}
  Important here is that of possible defeater.
  Have that agent didn't consider \(\psi\), but this doesn't establish that possible defeater.
\end{note}

\begin{note}[Core idea]
  The role of~\ref{nI:inclusion} is to capture a relation between an agent's claimed support for \(\phi\) having value \(v\) and claiming support for \(\psi\) having value \(v'\) such that the agent considers their claimed support for \(\phi\) having value \(v\) to depend on the possibility of claiming support for \(\psi\) having value \(v'\).

  Loosely phrased, the agent thinks that the support they've claimed for \(\phi\) isn't really worth much if it not also possible for them to claim support for \(\psi\).
\end{note}



\subsubsection{~\ref{nI:going-by-value}}

\begin{note}
  \ref{nI:going-by-value} serves two purposes.

  First, claiming support for \(\psi\) having value \(v'\).
  Second, claimed support for \(\phi\) having value \(v\) as part of such reasoning, from claimed support.
  And, that this is something that it is not possible for the agent to avoid.
  {
    \color{red} Because here it's not the case that \(\phi\) is an non-important part.
  }

  Some interest.
  Saw about with conditionals and contraposition.
  However, also saw with conditionals without contraposition.

  So, it is possible that the details of the reasoning introduce \(\psi\) having value \(v\) as an expectation.
  Still, this is not necessarily the case.

  Hence, that agent is claiming support and claimed support for \(\phi\) having value \(v\) is involved so not provide anything immediate.

  Hence, the task of~\ref{nI:inclusion} is to point to conflict with assumptions \emph{given} the details of this reasoning.
\end{note}

\subsection{The argument}

\begin{note}
  \color{red}
  Maybe restate sketch.
\end{note}

\begin{note}
  Before starting the argument, recall:
  \begin{quote}
    \vspace{-\baselineskip}
    \propRecogniseDefeaters*
  \end{quote}
  \nI{} is a variation of \autoref{prop:CS-only-if-reason-recognised-defeaters}.

  Three things:
  \begin{itemize}
  \item Less general, requires various clauses.
  \item Stronger, as does not require assumption that agent does not reason.
  \item Does not strictly follow from \requ{}.
  \end{itemize}

  However, same motivating intuition.
  With \autoref{prop:CS-only-if-reason-recognised-defeaters}, no reasoning about \requ{} at time of reasoning, then no reasoning regarding possible defeater.
  Hence, conflict with \autoref{idea:eiS}.

  With \nI{}.
  Claiming support for \(\psi\) having value \(v'\) is \requ{} of claimed support for \(\phi\) having value \(v\).
  And, as a consequence of claimed support for \(\psi\) having value \(v'\) being a \requ{} it is not possible for the agent to entertain the possibility that \(\psi\) does not have value \(v'\) while appealing to \(\phi\) having value \(v\).
  Yet, assumption that only get to \(\psi\) having value \(v'\) from \(\phi\) having value \(v\).
  Therefore, conflict with \autoref{idea:eiS} --- Agent needs to have indication of why even if \mom{}.
  The problem is that not possible to conclude that \(\psi\) has value \(v'\) from a line of reasoning that entertains the possibility that \(\psi\) does not have value \(v'\).
  \requ{} does not work, as this is expectation.
\end{note}

\begin{note}
  First, \(\psi\) having value \(v'\) is \requ{} of moving from claimed support for \(\phi\) having value \(v\) to \(\phi\) having value \(v\).

  Second, consequence of claimed support for \(\psi\) having value \(v'\) being a \requ{} it is not possible for the agent to entertain the possibility that \(\psi\) does not have value \(v'\) while appealing to \(\phi\) having value \(v\).

  Third, why this consequence prevents claiming support.
\end{note}

\subsection{First, \requ{}}

\begin{note}[Key proposition]
  First, showing that claimed support for \(\psi\) having value \(v'\) is a \requ{} of claimed support for \(\phi\) having value \(v\).

  \begin{note}
    Definition of interest.
    \begin{quote}
      \vspace{-\baselineskip}
      \defRequisite*
    \end{quote}
  \end{note}
\end{note}

\begin{note}
  So, relevant instance of~\autoref{assu:supp:independence}:
  \begin{enumerate}
  \item (\(\text{CS}\phi \vdash \phi\)) \(\rightarrow \psi\)
  \item \(\lnot\psi \leadsto\) \(\lnot\)(\(\text{CS}\phi \vdash \phi\))
  \end{enumerate}

  Point here is that we do get \(\psi\), and it is also the case that \(\psi\) is not a `mere contingency'.
\end{note}

\begin{note}
  Two immediate consequences from~\ref{nI:claimed-support}.
  \begin{enumerate}
  \item \(S\) is (given present context) in position to claim support that \(\psi\) has value \(v'\) (without appeal to \(\phi\) having value \(v\))
  \item \(S\) would be \nmom{} were they to claim support.
  \end{enumerate}
\end{note}

\subsection{Second, move to \(\phi\), need \(\psi\)}

\begin{note}
  Basic idea:
  \(\psi\) having value \(v'\) is inherited from claimed support for \(\psi\) having value \(v'\) as a \requ{} of claimed support for \(\phi\) having value \(v\) \emph{when} claimed support for \(\phi\) having value \(v\) is appealed to in order to reason from \(\phi\) having value \(v\).

  In this sense, \(\psi\) having value \(v'\) is an `indirect' \requ{} given background.
  For, \(\phi\) having value \(v\) gets \(\psi\) having value \(v'\).
  And, failure of reasoning if \(\psi\) does not have value \(v'\).
\end{note}

\begin{note}
  Suppose agent has moved to \(\phi\) having value \(v\) from claimed support for \(\phi\) having value \(v\).

  What we get here is that claimed support for \(\phi\) is not \misled{}.
  Also, would not be \mistaken{}.
  For, claimed support is doing the work for \(\phi\) having value \(v\).


  Hence, if \(S\) would not be \mistaken{}, then it must (also) be the case that \(\phi\).
  Note here that we only rely on not being \misled{}.

  State this simply:
  Given that \(S\) has taken claimed support to be enough to get to \(\phi\), then because of relation to claiming support for \(\psi\), also \(\psi\).

  What is important to keep in mind is that it is the move to \(\phi\) that requires perspective on claimed support for \(\phi\), in turn perspective on claimed support for \(\psi\), and hence same move.

  So, moving to \(\phi\) is sufficient to fix value for \(\psi\).
\end{note}

\begin{note}
  Observation:

  \(\psi\) is not a `mere contingency'.

  This move is such that it requires ruling out possibility of \(\psi\) not having value \(v'\), roughly.
\end{note}

\subsection{Entertaining \(\lnot\psi\)}

\begin{note}
  Not possible to hold that \(\phi\) has value \(v\).
  For, seen above.
\end{note}

\begin{note}
  True, claimed support for \(\phi\) having value \(v\).
  However, this doesn't also work for \(\psi\) having value \(v'\) as this claimed support for \(\psi\) having value \(v'\) is \expec{}.

  Right, so if giving up \(\phi\) having value \(v\), then not appealing to claimed support for \(\phi\) having value \(v\).
  Yet, if this is the case then no move to claimed support for \(\psi\) having value \(v'\).

  So, not only is claimed support for \(\psi\) having value \(v'\) a \requ{} and an \expec{}, it is also the case that there's no way to appeal to this to secure that \(\psi\) has value \(v'\) regardless of how things actually are.
\end{note}

\subsection{Summary}

\begin{note}[To summarise]
  Look, the problem is that it is impossible for the agent to hold that \(\phi\) has value \(v\) without also holding that \(\psi\) has value \(v'\).

  \emph{Therefore}, when the agent entertains the possibility that \(\psi\) does not have value \(v'\) (as they must do in order to claim support), this includes \(\phi\) not having value \(v\).

  Yet, by assumption, the only way for the agent to conclude some line of reasoning with \(\psi\) having value \(v'\) is by appeal to \(\phi\) having value \(v\).

  Therefore, when entertaining this possibility, the agent \emph{must} fail to reason to \(\psi\) having value \(v'\).
  (Which is not to say that the agent ends up reasoning to some contrary evaluation --- only that \(\psi\) having value \(v'\) is not a possible conclusion.)

  The reason that we're interested in the relationship between the ways of claiming support is because it leads to this consequence.
  It is true that in the way of reasoning outlined claiming support for \(\psi\) having value \(v'\) ends up being both an expectation and a \requ{}.
  However, this is not the main focus.
  Rather, it is what follows from this.
  I.e.\ that \(\psi\) having value \(v'\) is a \requ{}.
\end{note}

\newpage

\subsection*{A handful of observations}

\begin{note}
  Argument is complete.
  A handful of observations.
\end{note}

\begin{note}
  Perspective on claimed support is key.
  \nfcs{} and \eiS{}.

  May challenge these.
  However, \nI{} is going to apply when both these things hold.
\end{note}

\begin{note}[Unused component of argument]
  There is an additional component to the clauses of \nI{} that may be added to strengthen the argument.
  Not only is \(\psi\) not having value \(v'\) a possible defeater, but given \ref{nI:inclusion} it is a possible defeater that the agent is confident that they can claim support about.

  ``One perspective on \ref{nI:inclusion} is that it ensures that claiming support for \(\psi\) having value \(v'\) is a possible and `pressing' defeater for the claimed support that \(\phi\) has value \(v\).''

  {
    \color{red}
     In turn, links to \autoref{assu:supp:independence} to motivate some account of reasoning against recognised defeater.

  I think that there is a successful argument that follows this pattern.
  However, this is not the argument we present.
  }

  Without saying more about reasoning regarding recognised possible defeaters, it is hard to say how important this is.
  Add in assumption that if possible to claim support then no dismissing by `weaker' reasoning.
  Not necessary, as clause of \nI{} is that \(\psi\) only from \(\phi\).
  Instead, suggestions along these lines would suggest that \nI{} stronger constraint would allow final clause to be weaker.
  Upshot for application of \nI{} is that need to identify this link between \(\phi\) and \(\psi\).
\end{note}


\begin{note}
  It is not a consequence of this argument that in order to claim support for \(\psi\) having value \(v'\) from \(\phi\) having value \(v\) from claimed support that \(\phi\) has value \(v\) that the agent must have an account of why \(\psi\) has value \(v'\) granting that the claimed support for \(\phi\) having value \(v\) is \mom{}.
  It may be the case that the claimed support for \(v\) requires that the claimed support for \(\phi\) is \nmom{}.
  The problem is solely that given the clauses, premises of claimed support for \(\psi\) having value \(v'\) `require' that \(\psi\) has value \(v'\).
  It may be case that \(\psi\) only if CS\(\phi\) is \nmom{} (as relying on \(\phi\)).
  However, CS\(\phi\) holds up against being \mom{}, and so even if requiring \(\phi\) from CS, still have a plausible account of \(\psi\).
  So, CS\(\psi\) needs something stronger than CS\(\phi\) grants, but the agent is still fine to hold on to what comes from CS\(\phi\).
  Though, on a related point, it does seem the considerations presented here do rule out the agent from strengthening claimed support.
  (I.e.\ given CS\(\psi\) I now have an even stronger case for \(\phi\).)
\end{note}

\begin{note}
  Important limitation is that going by \(\phi\).
  This means that \nI{} does not rule out not going by \(\phi\).
  Still, expectation which is a worry.
  And, in any case, if there is a possibility, we don't need to worry about it as we're interested in applying \nI{} to cases where \(\phi\) is made use of.
\end{note}

\begin{note}[Generalising \nI{}]
  Core question about whether there's a generalisation of \nI{}.

  In particular, one might think that there's a requirement for the agent to witness the relevant reasoning in certain cases.
  I mean, that's the core of \nI{}.
  In some cases, the agent doesn't have the option of skipping this by appealing to claimed support for something.

  However, the difficulty is in finding an expansion which doesn't also prevent the agent from claiming support when they do witness.
  In all cases, it's clear that one may get things wrong.

  The way that \adB{} avoids this is by avoiding strong claims to the specific ability.
  Indeed, principle is the same as witnessing.
  So, there's no plausible way to expand \nI{} to cover the proposals without also denying the relevant instance of witnessing.

  Rather, objections here comes from supporting \ESU{}.
  That this isn't a way to claim support.
\end{note}

\begin{note}[Summary, and testimony]
  Final case to summarise:
  Knowledge via testimony.
  This condition doesn't necessarily apply, as agent may not be in position to claim support for what follows from knowledge claim.

  Two reasons for this.
  First, agent may not be in a position to check.
  E.g.\ missing premise, or layperson, e.g.\ missing steps of reasoning.

  Second, agent may not need to \RBV{-}.
  For, if you've testified, then it follows from your statement.
  I don't need to appeal to me having heard from you.
  Instead, given the additional information that I have, you've already made the claim.
  Even if \(\phi\) doesn't have value, this is still an okay reinterpretation of the testimony you have provided.
  Here, to get the intuition, it's really not clear that I need to endorse that I do have the option to check.
  {
    \color{red}
    This point only really makes sense after the argument has been given.
  }
\end{note}

\section{Illustrations of \nI{}}
\label{sec:illustrations-ni}

\begin{note}[Abstract, so examples]
  Turn to illustrations, and then to how \nI{} applies to \gsi{}.
\end{note}

\begin{note}
  Here, only interest is in support.
  Hence, recognised by the agent that they may be mistaken or misled.
  From this perspective, the issue is not ruling out potential defeaters.
  Similar to knowledge, etc.\ but no requirement that there are no defeaters.
\end{note}

\begin{note}
  May think that this restricts any application of \RBV{} to claimed support for \(\phi\) without value independent.
  This isn't quite right.
  \eiS{} keeps focus on \(\psi\).
  Only committed to \(\psi\) being a problem.
  Potential issue is no worse than any other instance of claim to support --- possibility of being mistaken or misled.
  If \(\psi\) ends up being used, then there's going to be a gap, where agent isn't in position to claim support by value, but unless eventual consequence is in turn used for \(\psi\), no clear problem --- at least not without stronger assumption.
\end{note}

\begin{note}
  \ESU{} is going to require the agent to reason from premises and steps `included' in claimed support for \(\phi\) in order to claim support for \(\psi\).
\end{note}

\begin{note}[Examples]
  Examples are somewhat difficult, due to complexities of state.
\end{note}

\begin{note}
  First, seeing exactly why the theory examples fail.
\end{note}

\begin{note}[Serial number]
  \begin{illustration}
    \label{illu:number-check}
    Genuine, only if serial number \dots (think credit cards).
  \end{illustration}
  No need to reinspect.

  Key idea here is that really relying on the product being genuine.
  Did not check for number when claiming support.
  So, without move to genuine, this breaks down.
\end{note}


\begin{note}[Logician]
  Here, novice logician, so limited to claiming support for sure.
  In principle, proof is stronger.
  However, possibility of \mistaken{}, and as a result \misled{}.
  \begin{illustration}\label{illu:CS:tfc}
    Novice logician.
    \begin{enumerate}
    \item Claimed support that \(\{\land,\lnot\}\) are truth functionally complete.
    \item If \(\{\land,\lnot\}\) are truth functionally complete then \(\{\lor,\lnot\}\) are truth functionally complete.
    \item So, \(\{\land,\lnot\}\) are truth functionally complete.
    \item Hence, \(\{\lor,\lnot\}\) are truth functionally complete.
    \end{enumerate}
  \end{illustration}

  First, reasoning for \(\{\land,\lnot\}\) did not depend on \(\{\lor,\lnot\}\).
  But this is quite complex.
  Not explicit assumption, but perhaps implicit.
  Still, going back through reasoning, it seems this is fair.

  Second, interdefinability.
  Hence, good account of why deals with possibility, as in general what holds for \(\{\land,\lnot\}\) will hold form \(\{\lor,\lnot\}\).

  Indeed, this suggests an alternative way of getting to the conclusion.
\end{note}

\begin{note}[Programming]
  \begin{illustration}
    \label{illu:programming}
    Writing a program to automate some reasoning/processing of data.
  \end{illustration}
  Various test cases.
  In these, possible to do the reasoning oneself.
  Therefore, no appeal to program for these simple cases, at least.
  This is quite similar to the logic illustration in this sense.

  However, interest here as interdependence breaks down in interesting ways.
  For, may break down due to resource constraints.
  E.g.\ available time or complexity of inputs.

  And, after enough time with the program, failure to obtain the same result is not clearly going to indicate a problem with the program.
  Rather, one's reasoning.
  Though, in turn, this may be reversed after enough checking of the reasoning.
\end{note}

\subsection{Variations on earlier examples}

\begin{note}
  Seen \nI{}, and in how builds on ideas which motivate \autoref{prop:CS-nai}.

  To round of the illustrations, consider variations on the illustrations of \ref{sec:abil-access-supp} which related to \autoref{prop:CS-nai}.
\end{note}

\begin{note}[Spot the difference]
  Back to \autoref{illu:CS:spot-the-diff}

  Spot the difference, think all.
  Okay, so found seven.
  Well \dots

  This turns out to be a very natural extension.
  And, strengthens the initial by avoiding the need to go to finding all of the differences.
  The reasoning alone doesn't do enough.
\end{note}

\begin{note}[Wally]
  Recall \autoref{illu:CS:wheres-wally}.

  Here, seems to apply.
  For, need it to be the case that Wally.
  However, somewhat less interesting.
  For, at the moment of completing.
  \nI{} will not find fault if, for example, in variation where book was returned to the library.
  Else, turns to be a variation on \autoref{illu:number-check}
\end{note}

\begin{note}[A trip to the zoo]
  Here, there seems no plausible variation.
  For, the interdependency fails given that there's no way to tell if it's a cleverly disguised mule by sight.

  Even in case where appeal to zebra, it need not be the case that there is interdependence.
  And, even if position to claim support by some other method (e.g.\ talking to a zoo keeper), it does not seem that sight does anything to ensure this.
\end{note}

\subsection{Illustrations where \nI{} does not apply}

\begin{note}[Treasure --- failure of interdependence]
  \begin{illustration}
    Claimed treasure only if learnt secret.
  \end{illustration}
  A little more interesting, as here, agent is going to have done something to learn secret when claiming support for treasure, but may not recognise that they've learnt the information.

  Of course, may be wrong treasure.

  However, there's too little information here to establish interdependence.
  That's the key point.

  Useful, as earlier examples may seem to rely on easy checks, but putting pieces together to reveal secret may be quite difficult.
\end{note}

\begin{note}
  The novice instance of the theory.

  And, testimony in general.
  Problem is that interdependence breaks down in these cases.
\end{note}


\newpage

\section{Other notes}
\label{sec:other-notes}

% \begin{note}[Other ways to fail, intuitively]
%   \nI{} captures an instance of trouble, but is only a sufficient condition --- there are others.

%   \begin{illustration}
%     Suppose \nagent{8} has received information from a younger and an older sibling.
%     The older sibling has told \nagent{8} that the dog has escaped.
%     And, the younger sibling has told \nagent{8} that if even the dog has escaped, the younger sibling has no idea whether the dog has actually escaped or not.
%   \end{illustration}

%   Here, \(\phi\) is that the dog has escaped, \(\psi\) is that the younger sibling has no idea whether the dog has actually escaped or not, and both \(v\) and \(v'\) are `true'.

%   It seems that \nagent{8} will have a hard time combing the support claimed for \(\phi\) and \(\psi\) to reason to \(\psi\) being true.

%   For, \nagent{8} claims support for \(\phi\) from the younger siblings testimony, but if \(\psi\) is the case then the younger siblings testimony is unreliable (with respect to whether the dog has run away, at least).
%   Hence, rather than concluding that \(\psi\) is the case, \nagent{8} is tasked with resolving the tension between the support claimed for \(\phi\) by way of the younger sibling, and the support claimed for if \(\phi\) then \(\psi\) from the older sibling.

%   The illustration merely highlights that it's not always straightforward to piece together and antecedent and a conditional.
%   This observation is hardly novel, and similar illustrations may be found in ~\citeauthor{Harman:1986ux}'s~\citetitle{Harman:1986ux}, among others.

%   Still, \nI{} is a sufficient, rather than necessary, condition for an agent failing to claim support (and failing in a certain way) and so that there may be other ways in which instances of \(\phi\) and \(\psi\) lead to trouble is not of particular interest.

%   When discussing \nI{} we will assume that there is no tension between the claimed support for \(\phi\) and the claimed support for if \(\phi\) then \(\psi\).
%   Instead, the difficult for the agent will follow (primarily) from \ref{nI:inclusion} --- a certain kind of relation between claimed support for \(\phi\) and claiming support for \(\psi\).
% \end{note}

\begin{note}[Implication wouldn't necessarily raise the problem]
  {
    \color{red}
    Point here is that couldn't \ref{nI:inclusion} be simplified to a conditional?
    Well, it could, as we've seen.
    The point is that \nI{} is something of a generalisation of this.
  }
  Now, the agent may observe that if both~\ref{nI:claimed-support} and~\(\phi \rightarrow \psi\) hold, then if \(\psi\) does \emph{not} have value \(v'\) then the support claimed in either condition is either \mom{}.

  First, if it is the case that \(\psi\) has value \(v'\) when \(\phi\) has value \(v\) and \(\psi\) does not have value \(v'\), then \(\phi\) does not have value \(v\).
  Hence, the agent's claimed support for \(\phi\) having value \(v\) must be misled.

  Second, if the agent's claimed support \(\phi\) having value \(v\) is not misled then \(\phi\) has value \(v\), but then if \(\psi\) does not have value \(v'\) it is not the case that \(\psi\) has value \(v'\) when \(\phi\) has value \(v\), and hence the agent's claimed support for the relation is misled.

  However, this observation alone is not particularly interesting.
  Take any case in which an agent claims support for receiving testimony regarding some matter the agent has no information regarding.
  And, observe that while the agent is mistaken is claiming support for testimony if what the interlocutor has said is not true, this does not prevent the agent for claiming support for the matter by appeal to the (apparent) testimony of their interlocutor.

  For example, if a logic instructor (unintentionally) misstates a theorem, a student may still claim support for the truth of the theorem by appeal to the instruction they received --- and even if the student reflects that they would be mistaken if the theorem is misstated.

  As stated in \eiS{}, claimed support may be misled or mistaken.
  At most, the observation under discussion in general only requires the agent to expect that \(\psi\) has value \(v'\).

%   Stated in terms of value because this also holds for desires, and for probabilistic statements.
%   Desires, means end is easiest to demonstrate with.
%   Probability, think in terms of conditionalization, or in terms of entailment.
%   Truth of \(\phi\) then probability of \(\psi\) is \(40\%\).
%   If the probability of \(\phi\) is \(70\%\) then the probability of \(\psi\) is \(40\%\) (though the probability of \(\phi \land \psi\) may be \(28\%\)).
\end{note}


\section{Contrast to other conditions}
\label{sec:contr-other-cond}

\begin{note}
  Two conditions.

  First, \citeauthor{Wright:2011wn} on warrant transmission.

  Second, \citeauthor{Weisberg:2010to} on bootstrapping.
\end{note}

\begin{note}
  Use to argue that \nI{} is unique.

  Also, observe some interesting things about \nI{}.

  \citeauthor{Wright:2011wn} by difference in extension.
  \citeauthor{Weisberg:2010to} by difference in intension.

  Respective approaches are motivated by ease of demonstrating the relevant difference in extension and intension.
  \citeauthor{Wright:2011wn}'s template(s) match scenarios fairly well, and so extension.
  \citeauthor{Weisberg:2010to}'s make certain things explicit with work for difference in intension.

  However, will suggest that observations made with respect to \citeauthor{Weisberg:2010to} also extend to \citeauthor{Wright:2011wn}.
\end{note}

\subsection{Wright on warrant transmission (failure)}

\begin{note}[How transmission failure relates]
  Inclined to think these are really the same.

  Note, in particular, \citeauthor{Wright:2000tq} is interested in transmission of \emph{second-order} warrant.
  So, not about whether the agent has warrant, but whether the agent may \emph{claim} to have warrant.
  (\Citeyear[89]{Wright:2011wn})

  In parallel, \nI{} is about claiming support, and not about whether the agent has support.
\end{note}

\begin{note}
  Basic ideas go back (at least) to the Proper Execution Principle of~\textcite{Wright:1991vn}, and in particular the \widt{} of~\textcite{Wright:2000tq} and (\Citeyear{Wright:2003aa}).\nolinebreak
  \footnote{See also~\textcite{Wright:1986ug,Wright:2002uk} and \textcite{Wright:2004uo}.}

  The \widt{} is as follows:
  \phantlabel{widt}
  \begin{quote}
    A body of evidence, \emph{e}, is an information-dependent warrant for a particular proposition P if regarding \emph{e} as warranting P rationally requires certain kinds of collateral information, \emph{I}.
    Some examples of such \emph{e}, P and \emph{I} [\dots] have the feature that elements of the relevant \emph{I} are themselves entailed by P (together perhaps with other warranted premises).
    In that case, any warrant supplied by \emph{e} for P will not be transmissible to those elements of \emph{I}.\nolinebreak
    \mbox{}\hfill\mbox{(\Citeyear[143]{Wright:2000tq})}
  \end{quote}

  The ellipses skip a quick illustration given by~\citeauthor{Wright:2000tq} in favour of the following illustration.
  \begin{quote}
    \vspace{-\baselineskip}
    \begin{illustration}
      You go to the zoo, see several zebras in an enclosure, and opine that these animals are zebras.
      Well, you know what zebras look like, and these animals look just like that.
      Surely you are fully warranted in your belief.
      But if the animals are zebras, then it follows that they are not mules painstakingly and skilfully disguised as zebras.\linebreak
      \mbox{}\hfill\mbox{(\Citeyear[154]{Wright:2000tq})}
      \newline\mbox{ }
    \end{illustration}
  \end{quote}

  Here, the body of evidence, \emph{e}, is what you've seen, the proposition P is that those animals are zebras.
  At issue is whether the warrant for P transmits to the proposition that those animals are not mules <adjectives> disguised to look just like zebras.
  In other words, at issue is whether the proposition that those animals are not mules <adjectives> disguised to look just like zebras is collateral information required for what you've seen to warrant those animals being zebras.

  From a broader perspective, the relevant collateral information is, well, there need be no specific collateral information across all possible ways of filling out the remaining details of the scenario, so let's say the collateral information is that things are as they appear.
  If so, the noted proposition is certainly required.

  More specifically, you're at a zoo, so something looking like a zebra seems sufficient to claim warrant that the thing is a zebra, and hence not a disguised mule.
  As such, \citeauthor{Wright:2000tq} holds there is a problem because, generally speaking, \dots

  \begin{quote}
    \dots\space there are external preconditions for the effectiveness of your\linebreak method---casual observation---whose satisfaction you will very likely, without compromise of the warrant you acquire for those beliefs, have done nothing special to ensure.

    [\dots]

    Can the warrants you acquire licitly be transmitted to the claim that those preconditions \emph{are} met---or at least that they are not frustrated in those specific respects?
    It should seem obvious that they cannot.\linebreak
    \mbox{}\hfill\mbox{(\Citeyear[154]{Wright:2000tq})}
  \end{quote}
  We won't go further into why \citeauthor{Wright:2000tq} thinks the result should seem obvious.
  Rather, the above should give you an idea of the phenomenon \citeauthor{Wright:2000tq} is interested in.
  And, with this, we can begin a comparison with \nI{}.
\end{note}

\begin{note}
  In relation to \nI{}, similar idea of undercutting.\nolinebreak
  \footnote{
    Indeed, \citeauthor{Wright:1991vn} notes that Stephen Yablo suggested the kind of defeat in question might be called undercutting in reference to \citeauthor{Pollock:1987un} (\Citeyear[95,fn.9]{Wright:1991vn}).

    I should perhaps note here that I developed~\nI{} after struggling to apply the ideas of \citeauthor{Wright:2011wn} to the scenarios of interest involving ability.
    And, after developing an initial draft of~\nI{} I took to the literature to see if there were any developed ideas that are either equivalent or imply \nI{}.
    This, quite naturally, led me to \citeauthor{Pollock:1987un}'s distinction between overriding and undercutting defeaters, and some of the references above which use `undercutting' in a broader sense than \citeauthor{Pollock:1987un}'s original formulation.
    Hence, it seemed to me that framing \nI{} in terms of identifying something of an undercutting defeater might be a helpful guide.
    If I had found this footnote earlier, I may have had an easier time developing the initial draft of~\nI{}.
  }
  {
    \color{expand}
    Information-dependence blocks transmission of warrant, but does not suggest anything about the relevant elements of \emph{I}.
  }

  Further, parallels between the first part of this and \eiS{}.
  different, but seem to go for the same idea.
  \eiS{} motivated by fallibility, and the first part amounts to fallibility.

  \citeauthor{Wright:2000tq} views this as a requirement, I haven't made this move.
  Significant part of \citeauthor{Wright:2000tq}\nolinebreak
  \footnote{
    See Pryor.
    This is what dogmatism denies.
  }
  , but I don't think this is the thing to focus on.\nolinebreak
  \footnote{
    You might be inclined to think understanding of claimed support should be strengthened.
    I don't want to take a stance on this, and hence problematic for me to distinguish on this basis.
  }

  {% to delete?
    Still, looking at the \emph{form} of \nI{} and \citeauthor{Wright:2000tq}'s \widt{}, there is a difference.
  }

  Instead
  \nI{} is concerned with the way in which in agent uses claimed support for a pair of propositions to claim support (or warrant) for some other proposition.
  By contrast, the \widt{} is concerned with preconditions for claiming warrant (or support) for a proposition that `might' be used to claim support.

  How agent claims support for \(\psi\) given claimed support for \(\phi\) and an implication from \(\phi\) to \(\psi\).

  \(\phi\) and \(\phi \rightarrow \psi\).
  Whether evidence, or claimed support, for this pair requires some collateral information entailed by \(\psi\).

  So, \nI{} doesn't care too much about \(\phi\) and \(\phi \rightarrow \psi\) whereas the template does.

  This is the thing of interest.

  Note, not possible to apply the template in a different way.

  So, in terms of \eiS{}.
  If going by value, require \(\psi\) to be the case.

  For template, need some warranted proposition P.
  Can't be \(\psi\), as template needs warrant, which we're denying.
  So, need something between \(\phi\) and \(\phi \rightarrow \psi\) and \(\psi\).
  Seems there's no proposition here.

  At issue is whether this difference in form corresponds to a difference in substance.
\end{note}

\begin{note}[The revised template]
  The \widt{} is intuitive, but has some downsides.

  Foremost, we would like additional clarity with respect to collateral information.
  As things stand, it's a little vague as to what constitutes and external preconditions for the effectiveness of a method.
  Further exposition might resolve this problem, but \dots

  More significant the \widt{} has been surpassed.\nolinebreak
  \footnote{
    This is a somewhat subtle issue.

    The revised template is, strictly speaking, a revision of the disjunctive template.
    And, \citeauthor{Wright:2002uk} initially distinguished the two templates:

    The \widt{} was designed to identify failures of transmission following from accumulation of defeasible evidence.

    And, by contrast, the disjunctive template was designed to identify failures of transmission following from by some faculty, such as perception or memory.

    See, for example, \textcite{Wright:2002uk} in which both templates are discussed separately, and \textcite[91]{Wright:2011wn} where the difference in motivation is restated.

    Still, \citeauthor{Wright:2011wn} observes that both templates the `base' for failure of transmission is the same in both cases.
    And, in turn, that the initial formulation of the disjunctive template yields unintuitive results when applied to cases covered by the \widt{} is a significant problem.
    (\Citeyear[91]{Wright:2011wn})

    So, \wrt{} is, strictly speaking, not a revision of the \widt{}, but rather the disjunctive template.
    However, \wrt{} is also designed to apply to the cases covered by the \widt{}, given that \citeauthor{Wright:2011wn}
    holds that both the \widt{} and the disjunctive template capture the same core phenomenon.
    Therefore, we have omitted these turns from the body of the paper.
  }
  (\Citeyear[90]{Wright:2011wn})
  This doesn't prevent a comparison, as such, but it may lead one to consider the comparison disingenuous.
  I don't think this is the case, and I began with the information-dependence as it is the spirit, rather than the letter, of the template which is at issue.

  So, to contrast \citeauthor{Wright:2000tq}'s template and \nI{} in detail, let's switch to \phantlabel{wrt}\citeauthor{Wright:2011wn}'s revised template:

  \begin{quote}
    Where A entails B, a rational claim to warrant for A is not transmissible to B if there is some proposition C such that:
    \begin{enumerate}[label=\roman*., ref=(\roman*)]
    \item\label{WT:i} The process/state of accomplishing the relevant putative warrant for A is subjectively compatible with C's holding: things could be with one in all respects exactly as they subjectively are yet C be true
    \item\label{WT:ii} C is incompatible (not necessarily with A but) with some presupposition of the cognitive project of obtaining a warrant for A in the relevant fashion, and
    \item\label{WT:iii} Not-B entails C\nolinebreak
      \mbox{}\hfill\mbox{(\Citeyear[93]{Wright:2011wn})}
    \end{enumerate}
  \end{quote}
  Where
  \begin{quote}
    A presupposition of a cognitive project is any condition P such that to doubt P (in advance of executing the project) would rationally commit one to doubting the significance, or competence of the\linebreak project, irrespective of its outcome.\nolinebreak
    \mbox{}\hfill\mbox{(\Citeyear[92]{Wright:2011wn})}
  \end{quote}

  In relation to the \widt{} discussed above:
  \citeauthor{Wright:2011wn}'s account of presuppositions of cognitive projects clarifies what collateral information amounts to.
  Cases of transmission failure are going to arise when one attempts to claim warrant for a condition for which doubt toward would undercut any outcome of the project.\nolinebreak
  \footnote{
    What matters is whether the relevant cognitive project has a presupposition of this kind, not whether the agent has done `anything special to ensure such presuppositions are satisfied.
    }

  In turn,~\ref{WT:i} to~\ref{WT:iii} detail how warrant for the relevant proposition depends on such collateral information.
\end{note}

\begin{note}
  The core intuition remains the same:
  Failure of transmission from a fixed proposition to conditions that need to be met in order for the agent to claim warrant for the fixed proposition.
\end{note}

\begin{note}[Applied to a case]
  Let's apply \wrt{} to the illustration used above to check:

  The relevant instances of A and B are, respectively:
  \begin{itemize}
  \item[A.] Those animals are zebras
  \item[B.] Those animals are not mules disguised to look like zebras
  \end{itemize}
  And, we may take C to be not-A. (\Citeyear[90]{Wright:2011wn})

  Now,~\ref{WT:i} to~\ref{WT:iii} are satisfied:

  \begin{itemize}
  \item[{\hyperref[WT:i]{i:}}] The process of accomplishing putative warrant for A is that the animals appear to be zebras, and things could be exactly as they \emph{appear} to be and yet the animals are not zebras.
  \item[{\hyperref[WT:ii]{ii:}}] The animals not being zebras is incompatible with A, of course --- given that C is not-A.\nolinebreak
    \footnote{
      For more details: (\Citeyear[90--96]{Wright:2011wn})
    }
    And, more broadly, the animals not being zebras is incompatible with moving from appearance to fact.
  \item[{\hyperref[WT:iii]{iii:}}] If those animals \emph{are} mules disguised to look like zebras, then those animals are not zebras.
  \end{itemize}
  Hence, \wrt{} identifies a failure of transmission in much the same way as we saw above with respect to the \widt{}.
\end{note}

\begin{note}[Entailment]
  Now, turning to the comparison proper.

  First, we'll map A and B from \wrt{} to \(\phi\) and \(\psi\), respectively, from \nI{}.

  An immediate difference is that \wrt{} requires an entailment between the relevant A and B while \nI{} does not include such a requirement.
  % requires that the agent has claimed support that if \(\phi\) has value \(v\) then \(\psi\) has value \(v'\).
  Still, to keep things simple, we'll assume that the agent has claimed support for an entailment from \(\phi\) having value \(v\) to \(\psi\) having value \(v'\).
  \wrt{} doesn't require that the agent has warrant, or has claimed support, for the entailment between A and B, and hence will apply in the case that the agent has.
    % So, \ref{nI:received-info} may be seen as an instance of \wrt{}'s initial condition with some superfluous detail.

  Likewise, \wrt{} is concerned with whether a claim to warrant for A is transmissible to B in general, and so not assume the agent has claimed warrant for A.
  So, as~\ref{nI:claimed-support} requires that the agent has claimed support for \(\phi\), we may consider~\ref{nI:claimed-support} as superfluous from the perspective of \wrt{}.
\end{note}

\begin{note}[Gist of why these are different]
  We're interested with~\ref{nI:inclusion} and~\ref{nI:going-by-value}.
  And, it seems focus should be on~\ref{nI:inclusion}.
  For,~\ref{nI:going-by-value} is (primarily) about the way in which the agent may go about claiming support for \(\psi\) and \nI{} only limits an agent claiming support in such a way if~\ref{nI:inclusion} holds.

  Hence, it seems to me the question is whether~\ref{WT:i} -- \ref{WT:iii} from \wrt{} and~\ref{nI:inclusion} do different things.
\end{note}

\begin{note}[Two questions]
  Let's break this down into two questions.

  \begin{enumerate}
  \item Whether an instance of~\ref{nI:inclusion} obtaining means that the agent makes a presupposition of the kind identified by \wrt{}.
  \item And, conversely, whether an instance~\ref{nI:inclusion} obtains if the agent has made a presupposition of the kind identifies by \wrt{}.
  \end{enumerate}
\end{note}

\begin{note}[Second question]
  The second question is straightforward to answer in the negative.

  Testimony, sight, whatever.
  Here, doesn't need to be any other way for the agent to claim support for the relevant proposition.
  {
    \color{red}
    To illustrate, zebra.
    Not in a position to claim support by some other way that moving from appearance is bad.
  }
\end{note}

\begin{note}[First question]
  The first question is more involved, and requires some care.

  Consider again the general presupposition that things are as they appear.
  Or, even more generally, that one is not in some sceptical scenario, such as a dream or a vat (cf.~\Citeyear{Wright:2002uk}, \Citeyear[97--98]{Wright:2011wn}).
  The difficult here is that it's easy to trivialise the question if the relevant presupposition is any presupposition.
  For, it seems any cognitive project will require some presupposition.

  Instead, the question is whether~\ref{nI:inclusion} obtaining means there is some \emph{related} presupposition.

  And, it seems this need not the case.

  For,~\ref{nI:inclusion} is, intuitively, about whether an agent is confident they have some way to claim support for \(\psi\), other than appealing to \(\phi\) and an implication from \(\phi\) to \(\psi\).
  But, it doesn't seem to follow that doubt about whether the agent has some other way of claiming support for \(\psi\) prior to claiming support for \(\phi\) and the implication from \(\phi\) to \(\psi\) would undercut claiming support for \(\phi\) or the implication from \(\phi\) to \(\psi\).

  I suspect this point is best argued for by illustrations, and we will consider a handful below.
  Still, it may be helpful to first outline the target of such illustrations in some detail.

  \ref{nI:inclusion} is about whether an agent is confident they have some other way to claim support for \(\psi\), but consists of two parts.
  \ref{nI:inclusion:position} requires that the agent is confident that the support claimed for \(\phi\) and would be \mom{} if the agent is not in a position to claim support for \(\psi\) some other way.
  And,~\ref{nI:inclusion:bound} requires that the agent is confident that the claimed support for \(\phi\) is a guarantee of sorts for claiming support for \(\psi\).

  Now, we're interested in deriving a related presupposition from \ref{nI:inclusion}.
  Still, the presupposition needs to be with respect to claimed support.
  So, as~\ref{nI:inclusion:bound} is a condition which concerns (as yet) unclaimed support, the related should follow from~\ref{nI:inclusion:position} --- in particular, from the possibility of the claimed support for \(\phi\) being \mom{}.

  However, claimed support for \(\phi\) being \mom{} reduces to either the claimed support indicating \(\phi\) has some value it does not have (misled) or the claimed support relies on factors that do not indicate the value of the \(\phi\) (mistaken).

  The point here is that claimed support being \mom{} is a relatively broad phenomenon.
  For example, an instance of inductive support may indicate the value of \(\phi\), and hence not be mistaken, but misled due to constrained sampling.
  Consider, by way of quick illustration, testing a random number generator by sampling its output.
  It may take a significant sample size to identify a bias, and hence bug in the source code.

  Yet, \citeauthor{Wright:2011wn}'s notion of a presupposition requires that doubt about the presupposition is such that doubt about the presupposition, \emph{in advance of following through on the project}, would \emph{require} doubt about the significance or competence of the project --- regardless of its outcome.

  So, suppose an instance of~\ref{nI:inclusion} may obtain because an agent has claimed inductive support for \(\phi\), has claimed support that \(\phi\) entails \(\psi\), and has some independent check on whether \(\psi\) is the case.

  If such an instance of~\ref{nI:inclusion} obtaining means that the agent makes a presupposition of the kind identified by \wrt{}, then the relevant presupposition should concern the nature of the claimed inductive support.

  However, it seems fundamental to claimed inductive support that one may doubt the inductive support is not misled without a requirement that one doubts the significance, or competence, of claiming such inductive support.

  I may doubt that I have obtained a sufficiently large sample to conclude that there are no bugs in the source code of the random number generator without being required to doubt the significance, or competence, of the sample acquired.
\end{note}

\begin{note}
  \color{red}
  The big difference is doubt versus an expectation.
  Expectation is weaker, so applies more generally.
  However, as expectation is weaker it might also be easier to deal with.
  (As kind of seen in the zebra case.)
\end{note}

\begin{note}
  To summarise:
  \ref{nI:inclusion} concerns (an agent's confidence in) a particular kind of relationship holding between claimed support for \(\phi\) and claiming support for \(\psi\) from the perspective of whether the respective instances of support are (or would be) \mom{}.
  This relationship may arise from claimed inductive support for \(\phi\).
  If so, a positive answer to the first question would require a corresponding presupposition with respect to the claimed indicative support for \(\phi\).
  Yet, such a presupposition seems incompatible with the nature of inductive support.
\end{note}

\begin{note}
  Stress, briefly, that this does not indicate anything problematic about \citeauthor{Wright:2011wn}'s template.
  I'm inclined to think the template is sound.
  The issue is whether (at least some) of the instances captured by \nI{} fall outside the scope of \citeauthor{Wright:2011wn}'s template.
  Given that both \nI{} and \citeauthor{Wright:2011wn}'s template are sufficient, there's no tension between the two if it is the case.
\end{note}

\begin{note}
  Let's now return to~\autoref{ill:CE:main} from the start of this section in which we examined a researcher may claim support that there are no counterexamples to a theory they have developed.

  Given that we have already seen how \nI{} applies to both illustrations, and outlined the theoretical difference between \nI{} and \wrt{}, we will focus only on why \wrt{} does not seem to apply to the illustration.
  In particular, why it seems there is no plausible candidate for the required `C proposition' of \wrt{}.
\end{note}

\begin{note}[\autoref{ill:CE:main}]
  \autoref{ill:CE:main} considered a researcher who has claimed inductive support for some theory.
  The instance of reasoning we took interest with was as follows:

  \begin{itemize}
  \item I have claimed support that the theory is adequate.
  \item So, given the claimed support, theory is adequate.
  \item Therefore, as the theory is adequate, given the claimed support, it follows that there are no counterexamples.
  \item Hence, I claim support that there are no counterexamples to the theory.
  \end{itemize}

  As we assumed an entailment from an adequate theory to an absence of counterexamples to the theory, we have the following two instances of A and B with respect to \wrt{}:

  \begin{enumerate}[label=\Alph*., ref=(\Alph*)]
  \item\label{wrt:difference:theory:A} The theory is adequate
  \item\label{wrt:difference:theory:B} There are no counterexamples to the theory.\nolinebreak
    \footnote{
      With respect to~\autoref{ill:CE:colleague}, we would have:
      \begin{enumerate}[label=\Alph*., ref=(\Alph*)]
      \item The colleague has failed to identify a counterexample to the theory.
      \end{enumerate}
    }
  \end{enumerate}

  If we are to identify failure of warrant transmission from~\ref{wrt:difference:theory:A} to~\ref{wrt:difference:theory:B} via \wrt{}, then there must be some proposition C such that (paraphrased):

  \begin{enumerate}[label=\roman*., ref=(\roman*)]
  \item\label{wrt:CE:maini} The process of claiming warrant for the theory being adequate is subjectively compatible with C holding.
  \item\label{wrt:CE:mainii} C is incompatible with either the adequacy, or some presupposition of the cognitive project of claiming warrant for the adequacy, of the theory.
  \item\label{wrt:CE:mainiii} The existence of a counterexample to the theory entails C.
  \end{enumerate}

  Well,~\ref{wrt:CE:maini} seems okay.
  Interested in claimed inductive warrant/support.
  And, claiming inductive support seems subjectively compatible with an entailment from some counterexample holding.
  It seems possible that things could be exactly as they subjectively are, yet the theory is inadequate because there is an unobserved instance of the phenomenon which constitutes a counterexample to the theory.

  So,~\ref{wrt:CE:mainii} and~\ref{wrt:CE:mainiii}.

  Working backwards.

  From~\ref{wrt:CE:mainiii}:
  C needs to be entailed by the existence of a counterexample.

  Paired with~\ref{wrt:CE:mainii}, the existence of a counterexample needs to entail something that is incompatible with either the adequacy, or some presupposition of the cognitive project of claiming warrant for the adequacy, of the theory.

  The problem:
  Claiming inductive warrant.
  Seems compatible with some counterexample holding.
  Applied to various instances of the phenomenon, and the theory holds up.
  Possible that it doesn't hold up under some instance of the phenomenon.

  So,
  Suppose the existence of a counterexample entails something that is incompatible with either the adequacy, or some presupposition of the cognitive project of claiming warrant for the adequacy, of the theory.
  Then, it seems the theory denies the possibility of such a counterexample.

  The difficulty is that we're talking generally about some theory for which the researcher has claimed inductive warrant for.
  I see no reason to think that any theory which fits this broad description will deny the possibility of certain counterexamples.

  There, may be that there are assumptions.
  Theories are built on other theories.
  However, interest is in a counterexample to the theory --- not a counterexample to theoretical foundations.

  Claiming warrant that there are no counterexamples in general seems to be the issue, rather than the specific kind a counterexample that would be required for \wrt{} to apply.

  Indeed, we can revise the relevant B instance given \citeauthor{Wright:2011wn}'s notion of a presupposition:
  \begin{enumerate}[label=\Alph*\('\)., ref=(\Alph*\('\))]
    \setcounter{enumi}{1}
  \item No counterexample consistent with the presuppositions.
  \end{enumerate}

  Evaluation of the reasoning seems the same.
  Indeed, natural assumption that there are no such presuppositions, so the concerns raised in~\autoref{ill:CE:main} remain.
\end{note}

\begin{note}[Looking ahead]
  \color{later}
  Difference is one thing, but also difference with respect to cases of interest.
  So, looking ahead, ability.

  Simple variation on second example.
  Ability to demonstrate that instance of phenomenon is covered by theory/not a counterexample.

  Follows from understanding of the theory.
  Seems just as bad.
  And, \citeauthor{Wright:2011wn} doesn't apply to either.
  Just need a little more work.
  Claiming support for general ability, so we add claimed (inductive) support for theory together with understanding of theory.
  Follows to specific ability as instance of general.

  So, while not focusing on cases involving ability from perspective of motivating \nI{}, differences here still relevant.
\end{note}

\subsection{Weisberg}

\begin{note}
  \color{red}
  Difference to \wnf{} is that there's no clear account of why \(\psi\) is needed, the problem, instead, is that it is not possible for the agent to get rid of \(\psi\).
\end{note}

\begin{note}[Intro to \wnf{}]
  Case.
  Condition.
  Contrast.

  Applies to inductive reasoning.
  \nI{} isn't strictly concerned with inductive reasoning.
  However, application is focused on this, and we have appealed to inductive reasoning extensively when contrasting \nI{} to \citeauthor{Wright:2011wn}'s templates.
\end{note}

\begin{note}[Bootstrapping]
  To illustrate \wnf{}, let's consider a case of bootstrapping introduced by~\textcite{Vogel:2000tl}'s --- here following \citeauthor{Weisberg:2010to}'s presentation:
  \begin{quote}
    \begin{illustration}\label{ill:gas-gauge}
      \emph{The Gas Gauge}. The gas gauge in \nagent{9}'s car is reliable, though she has no evidence about its reliability.
      On one occasion the gauge reads F, leading her to believe that the tank is full, which it is.
      She notes that on this occasion the tank reads F and is full.
      She then repeats this procedure many times on other occasions, eventually coming to believe that the gauge reliably indicates when the tank is full.\nolinebreak
      \mbox{}\hfill\mbox{(\Citeyear[526--527]{Weisberg:2010to})}\linebreak
      \mbox{}
    \end{illustration}
  \end{quote}
  \citeauthor{Vogel:2000tl} argued that kind of reasoning present in~\autoref{ill:gas-gauge} is a problem for reliabilist theories of knowledge, and others have argued the problem may be extended further (see \textcite[\S1]{Weisberg:2010to} for more details).

  However, our interest in the reasoning present in~\autoref{ill:gas-gauge} and \wnf{} is merely that the reasoning is intuitively problematic, \wnf{} is an account of why, and \wnf{} may capture the same phenomenon as \nI{}.
\end{note}

\begin{note}[No feedback]
  \begin{quote}\phantlabel{wnf}
    \textbf{No Feedback} If
    \begin{enumerate*}[label=(\roman*)]
    \item\label{W:NF:i} \(L_{1}-L_{n}\) are inferred from \(P_{1}-P_{m}\), and
    \item\label{W:NF:ii} \(C\) is inferred from \(L_{1}-L_{n}\) (and possibly some of \(P_{1}-P_{m}\)) by an argument whose justificatory power depends on making \(C\) at least \(x\) probable,\nolinebreak
      \footnote{
        There may be some ambiguity here.
        As we will see when examining an illustration below, the arguments justificatory power should be read in terms of depending on \emph{having made} \(C\) at least x probable rather than \emph{establishing that} \(C\) at least \(x\) probable.
        (It is in this sense that \(C\) is being `amplified'.)
        By contrast, the following clause requires that \(P_{1}-P_{m}\) \emph{are making} \(C\) at least \(x\) probable without the help of \(L_{1}-L_{n}\).
      }
      and
    \item\label{W:NF:iii} \(P_{1}-P_{m}\) do not make \(C\) at least \(x\) probable without the help of \(L_{1}-L_{n}\), then the argument for \(C\) is defeated.\linebreak
      \mbox{}\hfill\mbox{(\Citeyear[533--534]{Weisberg:2010to})}
    \end{enumerate*}
  \end{quote}
  Where `\(P\)' stands for a premise(s), and `\(L\)' for a lemma(s). (Cf.~\Citeyear[533]{Weisberg:2010to})

  Again, we have a condition in which an argument would be undercut.
  \wnf{} suggests only that the argument for \(C\) would be defeated, but leaves open the status of \(C\).
\end{note}

\begin{note}[\wnf{} intuition]
  \citeauthor{Weisberg:2010to} motivates with the following intuition.
  \begin{quote}
    The idea is that the amplification of an already amplified signal distorts the original signal, resulting in feedback, and bootstrapping is just ``epistemic feedback''.
    Bootstrapping is an undesirable result of amplifying the output of ampliative inference without restriction.\linebreak
    \mbox{}\hfill\mbox{(\Citeyear[534]{Weisberg:2010to})}
  \end{quote}

  To summarise.
  {
    \color{red}
    So, the point is that there's some amplification applied to \(C\) in order to get \(L\) from \(P\).
    And, this in turn blocks an argument to \(C\).
    For, already included amplification to \(C\).
    
  }
  {
    \color{red}
    \phantlabel{wnf:expectation}
    Note, doesn't rule out \(L_{1}-L_{n}\).
    Here, similar to expecting that defeaters don't hold.
    Or, following \citeauthor{Weisberg:2010to}, drawing conclusions from evidence.
  }
\end{note}

\begin{note}
  After walking through how \wnf{} applies to~\autoref{ill:gas-gauge} we will motivate a connexion between \wnf{} and \nI{}, before arguing that the two are sufficiently distinct.
\end{note}

\begin{note}
  The overall conclusion \nagent{9} draws in~\autoref{ill:gas-gauge} is that the gauge reliably indicates when the gas tank is full.
  Still, this overall conclusion is drawn from repeated instances of reasoning on particular occasions that concludes that the gauge is reliable on that occasion.
  And, the fault identified by \wnf{} concerns the reasoning on particular occasions.
  Intuitively, if \nagent{9} fails to establish the reliability of the gauge on any particular occasion by the particular instances of reasoning, then the conclusions of those particular instances of reasoning are unavailable for \nagent{9} to draw the general conclusion.

  So, to begin let us summarise the pattern to which each particular instances of reasoning conforms:

  \begin{enumerate}
  \item\label{W:GG:i} The gauge is reliable. \hfill (Background assumption)\nolinebreak
    \footnote{
      Have as background assumption because in the original, \nagent{9} skips over this as an explicit step.
      However, following \citeauthor{Weisberg:2010to} possible to reformulate to some level of probability sufficient to go to 3, such that the overall result of argument is to raise probability. (\Citeyear[528]{Weisberg:2010to})
    }
  \item\label{W:GG:v} It is sufficiently likely that the gauge is functioning correctly on this occasion. \hfill \mbox{(From~\ref{W:GG:i}, `Amplification')}
  \item\label{W:GG:ii} The gauge reads full. \hfill (Observation)
  \item\label{W:GG:iii} So, the tank is full. \hfill (From~\ref{W:GG:v} \&~\ref{W:GG:ii})
  \item\label{W:GG:iv} Hence, the gauge is functioning correctly on this occasion. \hfill (From~\ref{W:GG:ii} \&~\ref{W:GG:iii})
  \end{enumerate}

  The `feedback' in this reasoning pattern involves establishing (an instance of) the reliability of the gauge from an assumption that the gauge is reliable.

  From the perspective of \wnf{} we have:
  \begin{itemize}
  \item[P:] The gauge reads full.
  \item[L:] The tank is full.
  \item[C:] The gauge is functioning correctly on this occasion.
  \end{itemize}

  And, each of the clauses of \wrt{} are satisfied, for:
  \begin{itemize}[labelwidth=\widthof{(iii)}]
  \item[{\hyperref[W:NF:i]{i:}}] That the tank is full is inferred from the gauge reading full (together with the background assumption applied to the particular occasion).
  \item[{\hyperref[W:NF:ii]{ii:}}] That the gauge is functioning correctly on this occasion is inferred from the tank being full (and the gauge readings full) by an argument whose justificatory power depends on it being probable the gauge functioning correctly on this occasion.
  \item[{\hyperref[W:NF:iii]{iii:}}] That the gauge reads full does not make it probable the gauge functioning correctly on this occasion without the help of it being the case that the tank is full.
  \end{itemize}

  In short, the reasoning from~\ref{W:GG:i} to~\ref{W:GG:iv} captures the (intuitive) idea that \nagent{9}'s reasoning is flawed because and agent doesn't get to use reasoning that proceeds from an assumption to infer that the assumption holds.

  {
    In terms of \citeauthor{Weisberg:2010to}'s presentation, the agent makes an ampliative inference from \(P_{1}-P_{m}\) to \(L_{1}-L_{n}\), requires certain things to be the case, and, results of amplification inference don't provide one with an argument for source of distortion.
    }

  \nagent{9} requires the gauge functioning correctly on this occasion to infer that the gas tank is full, but observing that the gauge functioning correctly follows given the assumption that the gauge functioning correctly doesn't make it any more likely that the gauge really is functioning correctly.
\end{note}

\begin{note}[Different from \citeauthor{Wright:2011wn}]
    {
    Here, very similar to \citeauthor{Wright:2011wn}'s \wrt{}.
    Difference is with respect to \ref{WT:iii}.
    Not-\(C\) does not necessarily entail something incompatible.

    For, \nagent{9} needs sufficiently reliable.
    And, it doesn't follow from the gauge is not functioning on this occasion that it is not sufficiently likely, nor that it is not possible to move from sufficiently likely to working.

    Still, \wnf{} does seem to fall within the general scope of \widt{}.
  }
\end{note}

\begin{note}[In relation to \nI{}]
  We turn now to the relationship between \citeauthor{Weisberg:2010to}'s \wnf{} and \nI{}.

  Recall, \eiS{}:
  Claimed support indicates the value of a proposition regardless of whether the claimed support is \mom{}.

  The argument for \nI{} rests on \eiS{}, and \citeauthor{Weisberg:2010to}'s \wnf{} may, likewise, be seen to rest on \eiS{}.

  For, it seems that any claimed support for the conclusion of an argument that satisfies the clauses of \wnf{} would violate \eiS{}.

  Consider \wnf{} once again.
  An argument for a relevant instance of \(C\) is defeated because \(C\) being probable to some degree is required in order to obtain additional lemmas used to construct an argument for \(C\).
  Recast, then, the agent may not construct an argument for \(C\) if the agent requires \(C\) to be probable to some degree.
  Or, equivalently, the agent may not construct an argument for \(C\) if it is a requirement for the success of the argument that the agent is not misled about degree to which \(C\) is probable.
  I.e.\ the argument would indicate the value, or probability, of \(C\) regardless of whether the claimed support is misled because the argument is only successful if \(C\) is probable to the relevant degree.

  So, is it the case that \citeauthor{Weisberg:2010to}'s \wnf{} and \nI{} are equivalent accounts of how \eiS{} constrains claiming support, if some (perhaps) superficial details about `probability' or `being in a position to claim support' are either removed or revised?
\end{note}

\begin{note}[Technicality]
  There is an initial difference with respect to scope of application.
  \wnf{} only applies to inductive reasoning (Cf.~\Citeyear[533]{Weisberg:2010to}), while \nI{} makes no such restriction.

  Still, I don't think too much should hang on this difference.
  We have motivated \nI{} primarily with respect to inductive reasoning, and reasoning with \gsi{-} is also, plausibly, an instance of inductive reasoning.
  So, even if there is room for a technicality, it doesn't matter for the cases of interest.
\end{note}

\begin{note}
  \color{red}
  Intuition is that in case of \nI{} the agent needs to make \(\psi\) probable to some degree.
  For, intuitively, needs to be that \(\lnot\psi\) is not an actual defeater.
  I.e.\ the probability of \(\lnot\psi\) needs to be low, and so the probability of \(\psi\) needs to be high.
  If this is right, then it looks as though \nI{} collapses into an instance of \wnf{}.
\end{note}

\begin{note}
  \color{red}
  The problem here is that in order to get \nI{} from \wnf{} we need to assume that any argument relies on making possible defeaters sufficiently improbable and hence that the negation of the defeater is sufficiently probable.

  Ugh.
  This is more complex.
  First, it's not clear that \wnf{} applies to the kind of cases of interest.
  For, need three parts.

  So, would need to be the case that general ability does not make \(\psi\) sufficiently probable.

  Well, it's hard to evaluate this.
  Given the information, it seems it does.
  But, suppose it does not.
  Then, would need it to be the case that general to specific requires \(\psi\).
  This goes back to assumption before.

  However, this then seems problematic.
  For, then run through the novice.
  It seems that here the novice would require that the phenomenon is not a counterexample in order to apply.
  And that seems really odd.

  I mean, there is an issue with the novice as the phenomena does come up as an expectation, my solution is to allow the agent to consider this, as it's testimony, or something, but then the parallel for \wnf{} would be to grant that theory alone is sufficient.
  However, if this is the case, then it seems general alone is going to be sufficient.

  And, \nI{} is also going to apply even when \(P\) alone makes \(C\) sufficiently probable.
  Because, it's all about whether something has been considered, not what difference it makes.
  I mean, that's kind of tough, because without considering \(C\) it's hard to know what the probability distribution is.

  \nI{} is more about the impact of unconsidered information.
\end{note}

\begin{note}[Key difference]
  Barring technicalities, still, a fundamental difference is present.

  As we have seen, \wnf{} holds when an agent makes an inductive inference from some premises to some lemmas which require the relevant conclusion of the argument to be probable to a certain degree.

  We \hyperref[wnf:expectation]{noted} that \wnf{} does not require that the agent has claimed support that the conclusion of the argument is probable to the required degree.
  However, clause~\ref{W:NF:ii} of \wnf{} explicitly states the argument of interest depends on relevant conclusion being probable to the required degree.
  So, the reasoning outlined by \wnf{} involves the agent being committed to the relevant conclusion being probable to the required degree as a part of the instance of reasoning to which \wnf{} applies.
  And, hence, tension with \eiS{}.

  \nI{} is fundamentally different.

  In the case of \nI{},~\ref{nI:inclusion} combines with~\ref{nI:going-by-value} to ensure that when an agent moves from reasoning from claimed support for \(\phi\) having value \(v\) to reasoning from \(\phi\) having value \(v\) the agent assumes that \(\psi\) has value \(v'\).
  In turn, this assumption leads to tension with \eiS{} as any reasoning that proceeds from \(\phi\) having value \(v\) depends on \(\psi\) having value \(v'\).
  Hence, even if the agent reasons from the implication that \(\psi\) has value \(v'\) when \(\phi\) has value \(v\), such reasoning is constrained to a context in which it must already be assumed that \(\psi\) has value \(v'\).
  And, so, it is not possible for the agent to claim support that \(\psi\) has value \(v'\) that indicates that \(\psi\) has value \(v'\) regardless of whether or not the claimed support is \mom{} because the reasoning from \(\phi\) having value \(v\) to \(\psi\) having value \(v'\) fails if \(\psi\) does not have value \(v'\).

  This is, admittedly, complex.
  To simplify, \wnf{} holds when an agent assumes the relevant conclusion (is probable to a certain degree) as part of the reasoning to the relevant conclusion.
  And, in contrast, \nI{} holds when an agent is required to assume that the relevant conclusion (has some value) in order to reason to the relevant conclusion.

  Simplified further, for~\wnf{} the issue is with respect to the reasoning that would be performed by the agent.
  And, by contrast, for~\nI{} the issue is with respect to what the agent must hold in order to perform the relevant reasoning.

  Admittedly this is a somewhat delicate.
  However, this distinction is important as it highlights how \nI{} is concerned with the specific details of the way in which the agent reasons, rather than any pattern of reasoning itself.

  As we saw when contrasting Illustrations~\ref{ill:CE:main},~\ref{ill:CE:colleague}, and~\ref{ill:CE:testimony}, the reasoning identified by~\ref{nI:going-by-value} allows an agent to claim support in certain cases (\ref{ill:CE:testimony}), but not in others (\ref{ill:CE:main} and~\ref{ill:CE:colleague}).

  So, there is a significant difference between the cores of \wnf{} and \nI{}.
  Issues with reasoning, and issues with assumptions prior to reasoning, respectively.
\end{note}

\begin{note}[Generalising to \widt{}]
  Noted above that \wnf{} seems to fall in the scope of \widt{}.
  If so, same kind of problem for \widt{}.

  However, relies on additional background about claiming support for presuppositions.
    And, while it is true that \citeauthor{Wright:2011wn} holds such views, it was instructive to observe that difference regardless.

    No such alternative with relation to \citeauthor{Weisberg:2010to}, as it's built into \wnf{}.
\end{note}

\begin{note}[Hm]
  \color{return}
  This is brief, but it's not clear more needs to be said.
  These are distinct phenomenon, even if they turn out to be extensionally equivalent.
  Conjecture that these wouldn't be, as similar issue with respect to difference between \citeauthor{Wright:2011wn}'s \wrt{} and \nI{} arising from \ref{nI:inclusion}.
  Working through illustrations is significant due to difference in formulation.
  There, difference in extension for difference in intension.
  Here, directly observed difference in intension.

  Broader consequences are of some interest (at least to me).
  However, aren't of interest to core line of argument.
  So, pursuing this will be left for some other time.
\end{note}

% \paragraph{Pryor}

% \begin{note}[Objection wrt.\ \citeauthor{Pryor:2000tl}]
%   I've probably already mentioned this somewhere else, but there's a way of reading \citeauthor{Pryor:2000tl} that conflicts.
%   For, Dogmatism could be read as allowing for \RBV{} in cases where it applies.

%   However, this requires some care.

%   Initial point is that it's not clear whether there's a dogmatist position with respect to claiming support.

%   Fruther, there are two possible ways to approach the issue.
%   First, as above.

%   Second, there's some operator which is dogmatic, and consequences stay within the scope of this operator.
%   So, for example with the zebra, the agent sees that the animal is a zebra, and hence sees that it's not a cleverly disguised mule.
%   On this reading, the agent does not \RBV{}.
% \end{note}

\newpage

\section{\nI{} and ability}
\label{sec:ni-ability}

\begin{note}
  \color{red}
  What I end up using here is:

  \begin{quote}
    \vspace{-\baselineskip}
    \propCSNai*
  \end{quote}
  As this is what ends up blocking any way of removing expectation of \(\psi\) from claimed support for general ability.

  At least when ability is view `as a whole'.

  So, need section on why considerations as unrecognised defeater are distinct from recognised.
\end{note}

\subsection{\nI{} applied to \gsi{} and \adA{}}
\label{sec:ni-ability:adA}

\begin{note}
  The idea here is simple.
  \begin{itemize}
  \item Checking that the conditions for \nI{} hold.
  \item The thing with \adA{} is that the agent appeals to ability `as a whole', so to speak.
  \item This gives us the relevant \(\phi\) instance for \nI{}.
  \item And, \aben{the} is such that the agent needs to appeal to ability, rather than mere claimed support.
  \item Then, the key focus is \ref{nI:inclusion}.
  \end{itemize}

  While we focus on \aben{the}, note that it seems the problem extends to going from general to specific.

  And, while argued for \nI{} independently, also motivate why the application to ability `makes sense'.
\end{note}

\begin{note}[Applying to type of scenario]
  Our attention now turns to how \nI{} applies to the use of \aben{the} in scenarios of interest.

  The focus of our attention is whether an agent may claim support for having a specific ability given the claimed support for having a general ability, given \gsi{}.
\end{note}

\begin{note}
  Now, the basic observation is that with \adA{} one moves from general to specific, and from ability to proposition.

  Here, only really interested in \aben{the}.
  However, as we've observed, goes from either general or specific.

  I mean, the basic observation is that the agent doesn't reason about general or specific ability.
  So, reasoning follows from it being the case that agent has attribute, or that there is a witnessing event.

  Ohhhh, the point is that the agent is relying on these conditionals.
  First, to move from general to specific.
  Second, to move from ability to proposition.

  With respect to these conditionals, it's \adA{}, so there's no way to move between these things without using the value of one thing to constrain the value of the other.

  So, instance of \adA{}, generally.
  And, because of the construction of the scenarios, the case of \adA{} we're interested involves appeal to the value of the proposition.
\end{note}

\newpage

\begin{note}[Checking conditions]
  Conditions \ref{nI:claimed-support} is provided by the scenario.
  And, scenario also provide information about how the agent claims support for \(\psi\) having value \(v'\) when \(\phi\) has value \(v\).
  Key is that \aben{the} requires that the agent has the specific ability, not (merely) that the agent has claimed support that they have the specific ability.
  Condition~\ref{nI:inclusion} is obtained by reflection on ability.
  So, then, condition~\ref{nI:going-by-value} rules out a way of claiming support for specific ability.
\end{note}

\begin{note}
  If argument is successful, then agent will not be in a position to claim support for specific ability.
  This is the antecedent of the relevant use of \aben{the}.
  Pair \nI{} with following supplement.

  \begin{restatable}[]{assumption}{assuDetachToClaim}\label{assu:detach-to-claim}
    An agent must have claimed support for the antecedent of an entailment in order to claim support for the consequent of the entailment via the entailment.\nolinebreak
    \footnote{To clarify, entailment is only about value.
      Think of conditional.

      So, does not follow that there being an entailment is a required part of agent's reasoning.
      \nIm{} is talking about when the agent appeals to an entailment, rather than any understanding of entailment beyond it being the case.
    }
  \end{restatable}
  \nIm{} seems indisputable,\nolinebreak
  \footnote{
    An agent may have some other way of claiming support for the consequent of the entailment.
    However, if the agent is not in a position to claim support for the antecedent, then the agent is not in a position to claim support because there is an entailment from the antecedent to the consequent.\nolinebreak

    For example, that the coin landed heads is entailed by Sam knowing that the coin landed heads.
    Here, entailment from \(K\phi\) to \(\phi\).

    Second, this light being on entails that the printer is out of paper.
    If agent appeals to entailment, again, need the light to be on.
    However, could look in the paper drawer, or modify the wiring so that an alarm sounds.

    However, Taylor is not in a position to claim support for the coin landed heads because Sam knows if Taylor has no idea whether Sam knows --- though Taylor may claim support by looking at the coin.
  }
  and so not in a position to claim support for result of witnessing ability via \AR{}.
\end{note}

\hozline{}

\begin{note}[Important points]
  Two important points:

  The role of~\nI{} is to highlight that the agent is not in a position to obtain support for (specific) ability in a certain way.
  That is,~\nI{} does not state that the agent may not obtain support for (specific) ability some other way.

  Second, so long as agent holds that they have general ability, then committed to truth.

  May take issue with information provided, especially if ideal.
  If informer has information, then they should say.
  In turn, not problem with~\nI{} as the agent would have support (via testimony) for specific ability.
  However, informer may only have the conditional.
\end{note}

\hozline{}


\begin{note}[Finding tension, still]
  We have outlined a type of scenario built primarily on an agent receiving information that the agent has some specific ability so long as the agent has some general ability.
  The agent has support for having the general ability, but there are two ways in which the agent's support for having the general ability may be used to establish support for {\color{red} the result of having the specific ability} --- \AR{} and \WR{}.

  The previous section argued that~\ESU{} constrains how an agent may use the received information.
  If an agent is required to traces support from premises to conclusion through reasoning, then an agent may not appeal to the support for the premises and steps of reasoning that the agent would use to witness the specific ability.

  The (initial) plausibility of~\ESU{}, then, suggests that the agent may only establish support for having the {\color{red} result of the specific ability} from the support they have for the general ability by \AR{}:
  The support the agent has for the general ability is support that it is true that the agent has the general ability.
  In turn, given the information received it is true that the agent has the specific ability, and it is only possible for the agent to have the specific ability if the result of witnessing the specific ability is true.

  The argument of this section is that the sketch of \AR{} given conflicts with a different, but equally plausible, premise.
  The premise concerns the way in which the agent obtains support for having the specific ability from the support for the general ability.
  We state conditional, the proceed to the premise.
  The initial statement of the premise is abstract and after providing a handful of clarifications we then link the premise to the type of scenario of interest.
\end{note}

\begin{note}
  \large
  \begin{itemize}
  \item So, get expectation.
  \item This much is fine.
  \item Question is whether it is possible for the agent to do anything about the expectation.
  \item Arguably no.
  \item One big idea is that the agent has claimed support for general ability on some `core' such that this provides strong indication that general ability extends to all cases.
  \item This much is fine, then problem, however, is that we've still got specific information about what is outside of the core.
  \item So, the probability of any possible defeater is super low.
  \item And, this is enough to hold onto claimed support regardless.
  \item Because, I considered the possibility of those unknown defeaters, and still gathered enough to claim support regardless.
  \item So, this seems to allow claiming support for specific from general.
  \item But at the same time this seems bad.
  \item It has the same feel as the problems with expectations.
  \item Because, all the stuff gathered was without recognition of this possibility.
  \end{itemize}

  \begin{itemize}  \item I mean, problem is before, got the probability low as unrecognised.
  \item Question is whether this remains the case now recognised.
  \item Well, nothing really follows from probability being low.
  \item In a sense, this should already be the case.
  \item The issue isn't that these possible defeaters a \emph{likely}.
  \item The issue is that the agent should think that support holds regardless of whether they hold.
  \item ``Category mistake''
  \end{itemize}

  Okay, so this kind of works against low probability.
  Hence, argument here is that there's no way to get rid of this expectation if agent only relies on claimed support for general ability.
  Of course think it's unlikely, but the worry is not that the defeater is there, rather than it's not clear how the evidence goes against the defeater.

  The redux, then, is that this idea of a `core' doesn't really rely on the probability idea.
  But then this just goes against the initial assumption.
  Of course, this is kind of what \citeauthor{Pryor:2000tl} does.
  However, this seems to conflict with the idea of claimed support.
  If we've got some kind of dogmatist position, then it doesn't seem that the possibility of being \mom{} is such an issue.
  Indeed, the problem here is how to make something like this consistent with that assumption.
\end{note}

\subsection{\nI{} and \adB{} (excluding basic \AR{})}
\label{sec:ni-ability:adB}

\begin{note}
  \large
  \begin{itemize}
  \item Have \gsi{} information.
  \item This means that we get a sort of conditional
    \begin{itemize}
    \item so long as premises and steps are available, then witnessing event.
    \end{itemize}
  \item Now, the task is to claim support for premises and steps.
  \item Key idea here is that agent does not appeal to general ability.
  \item Instead, agent is appealing to those premises and steps in the same way they would do when witnessing \emph{and this doesn't require appeal to general ability}.
    \begin{itemize}
    \item it's not the case that I go `I can do arithmetic, so \(2 + 2 = 4\)'
    \item Rather, it's understanding \(2,+,=,4\), etc.
    \end{itemize}
  \item So, ability as a whole carries the expectation, but appeal to distinct components does not.
  \item This is really important to stress.
  \item The witnessing conditional (so to speak) comes from the information, not from the general ability.
    And therefore don't need to appeal to the general ability to get the witnessing event --- only issue is whether it can be `made actual'.
  \item I mean, this is what is kind of puzzling about \EAS{}.
  \item I haven't \emph{used} any of this stuff, but claiming support by it anyway.
  \end{itemize}

  \begin{itemize}
  \item Objection here is that there's still a question about missing steps.
  \item Well, information is that all this stuff is sufficient.
  \item The only issue is whether the thing would really amount to a proof.
  \end{itemize}

  \begin{note}
    Interesting is that the above gets to specific ability.
    It's then not clear that the agent is required to get \(\psi\) from specific by the witnessing kind of thing, but this seems natural.
  \end{note}
\end{note}

{
  \color{red}
  \begin{itemize}
  \item Key lesson learnt from \nI{} and \adA{} is that if the agent goes to having the ability, then they bring \(\phi\) with them (due to interdependence of claiming support).
    So, \adB{} needs to avoid going to ability.
  \item Key with \adB{} is that it breaks up this interdependence.
  \item Instead of using the ability as a whole, everything gets broken up into premises and steps.
  \item This is motivated by general understanding of reasoning.
  \item For, general point of reasoning is breaking things down so that the conclusion doesn't follow from any particular step or premise, but rather the combination.
  \end{itemize}

  So, the real thing I need for \adA{} is that ability gets treated as a whole.
  And, this then extends to basic \AR{}.

  Do I still need the \adB{} vs.\ \adA{} distinction?
  Probably, as it helps with motivating the key ideas.
  I mean, yes as there isn't a good distinction between \AR{} and \WR{} alone.
}

\begin{note}[Setting expectations]
  So far we have seen \ESU{} requires \adA{}, and that \nI{} rules out \adA{}.
  The final thing to check, then, is whether \adB{} is compatible with \nI{}.

  Some care to be taken here.
  \adB{} only holds that the agent claims support by appeal to ability --- \AR{} and \WR{}.
  So, it is not obvious that \adB{} alone provides us with enough information to provide a complete defence that the agent does claim support.
  Rather, our goal is to provide an `in principle' defence that \adB{} need not conflict with \nI{}.

  The upshot of this is an avenue for further research.
  If \nI{} and scenarios, then either \adB{} or basic \AR{}.
  We will say more on this in section~\ref{sec:establishing-tension}.

  For now, I hope to have appropriately set expectations with respect to following argument.
\end{note}

\begin{note}[Main idea]
  \nI{} is about interdependence of claiming support between \(\phi\) and \(\psi\) undermining a way of claiming support for \(\psi\).

  So, the task is to show that this interdependence need not hold when reasoning \adB{}.

  This may not be immediately obvious.
  Saw \nI{} applied to \adA{}.
  Going \adB{} doesn't necessarily make a difference.
  {
    \color{red}
    Problem is, comes from appealing to ability.
  }

  However, what \adB{} \dots
\end{note}

\begin{note}[Working through the details]
  \ref{nI:claimed-support} \dots



  \ref{nI:inclusion} are satisfied in the scenarios of interest.
  So, question is \ref{nI:going-by-value}.

  Letter, but more importantly the spirit.

  Spirit goes back to \eiS{}.
  Saw about that \eiS{} was central to argument for \nI{}.
  In other words, the question is whether the agent claims support which holds up `even if' \(\phi\) turns out not to be the case.

  In other words, \adB{} does not lead to similar conflict with \eiS{}.

  Hence, the `in principle' defence that \adB{} need not conflict with \nI{} rests on showing that \ref{nI:going-by-value} is not necessarily satisfied when steps of reasoning are \adB{} with respect to ability.

  Focus step is ability to \(\phi\).

  So, agent claims from specific ability.
  Idea has been noted.
  Clearest with respect to \WR{}.
  Appeals to event, in particular the premises and steps of reasoning.
  In turn, reduces to observation that claiming support in this way does not require \RBV{}.
\end{note}

\begin{note}
  So, broad idea is claiming support from same premises and steps they would if they were to witness their ability.

  \adB{}, claim support by appeal to thing, and \WR{} provides sufficient detail about what that thing is.

  Question is whether \(\phi\) needs to be the case.

  Recall, \(\phi\) because moved to it being true that agent has the ability.

  This move doesn't happen with \adB{}.


  Key point is that agent claims support for property or event.
  The agent doesn't move to value.

  So, \gsi{} it's the parts of the general ability.
  And \aben{the} it's the premises and so on.

  Key point is that given background information, these allow the agent to claim support, even if it turns out the agent is \mom{}.
  Information is that that stuff is sufficient to claim support.

  Easiest with \WR{}.
  As, this is just the same as an instance of reasoning.
  The only difference is that the agent isn't clear on what's going on.

  Everything the agent has claimed support for allows them to make this move.
  Even if turns out things aren't right, and \mom{}, the agent seems to have enough, and by \adB{} they don't require that they aren't \mom{}.

  This, to my mind, is the key idea with ability.
  It informs the agent of something they have the ability to do.
  And, that thing functions in just the same way as it would if the agent were to do the thing.

  Claim support by appeal to that reasoning.
  Only going to be truly successful if I have the ability, for sure.
  However, claim support for ability even if \mom{}.
\end{note}

\begin{note}
  Key observation is that \adB{} doesn't go by value.

  However, there is a problem.

  For, it may seems as though the agent \emph{does} go by value because they require the premises, etc.

  This is clearest with the idea that:
  \begin{itemize}
  \item If \(\phi\) isn't the case, then some premise or step isn't part of ability.
  \end{itemize}
  Question about whether this gets a violation of \eiS{}.

  But, point is that agent at present is okay with claiming support that the reference resolves.

  So, this really isn't that problematic.

  Obviously it could break down.

  The point is that the agent at present outlines claim to support even if \mom{}.
\end{note}

\subsection{Objections}

\begin{note}
  The main objection here is something along the lines of a stronger requirement on claiming support.
  Agent gets to keep claimed support for general or specific ability.
  Possible defeater isn't enough, though information introduces an expectation.
  Well, this should prevent appealing even to the premises and steps of reasoning.
  Avoided this because these do not require expectation of \(\psi\).
\end{note}

\subsubsection{\requ{1}}
\label{sec:requ1}

\begin{note}
  Objection here is that we've identified failure due to \requ{1}, roughly.
  And, ability to claim support, and this is going to involve some \requ{1}.
  However, no use so no reasoning about.

  Yes, this is somewhat difficult.

  Unsatisfactory response would be to observe that \requ{} is only defined with respect to reasoning performed.
  Unsatisfactory because only necessary conditions, and plausible that there are additional necessary conditions.

  Possible line of response is appeal to ability.
  However, this will lead to an instance of \nI{} all over again.

  Instead, witnessing ability is itself sufficient for claiming support.
  And, as would amount to claiming support, this will involve reasoning about \requ{}.
  Important, \EAS{} does not necessarily hold for anything weaker.
\end{note}

\subsubsection{Deny claimed support for ability}

\begin{note}
  So, if witnessing, then premises and steps are good enough to go for the conclusion.

  Main problem is that possibility that the relevant witnessing event is not possible.
  However, claimed support that it is.

  Still, suggestion that given possible defeater, the agent doesn't even get to appeal to claimed support for general/specific.

  Then, wouldn't get the details of the witnessing event.

  So, this is much stronger than {\color{red} Assumption ???}.
  Indeed, something like this would prevent event \ESU{}.
  Find this sufficiently implausible.
\end{note}

\subsubsection{Appeal to premises and steps requires appeal to ability}

\begin{note}
  It's true that the combination implies the ability, and so the combination seems to lead to the same problem.
  We get \(\psi\) as an expectation of combining all of the premises and steps.

  However, what we're relying on is appeal to the individual components.
  The thing here is that it seems fine for the agent to witness.
  This doesn't block claiming support.

  Hence, if this is the case then it can't be that the problem is simply what follows from the combination.
  Rather, it must be something about not witnessing.
  However, this returns us to \ESU{}.
  This is the very intuition that we're arguing against.
  Hence, the question is whether this really is something that is the case.
\end{note}

\section{Incompatibility of \nI{}, \gsi{}, and \adA{}}
\label{sec:ni-summary}

\begin{note}[Table]
    \begin{figure}[h]
      \centering
      \saMtxRuledOutLCS{}
      \repeatCaptionPrime{fig:saMtxRuledOut}{Distinction matrix}
  \end{figure}
\end{note}

\section{\nI{} isn't that strong}
\label{sec:ni-isnt-that}

\begin{note}
  Look, \nI{} rules out a way of claiming support quite broadly.
  However, this is because we're focusing on \aben{the}.
  This shouldn't be taken to suggest that there's general tension between \nI{} and \adA{}.
\end{note}

%%% Local Variables:
%%% mode: latex
%%% TeX-master: "master"
%%% End: