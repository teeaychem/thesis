\chapter{Tension sensed}

\begin{note}
  Now, developing the tension is one thing.
  We don't really require much more of an understanding of claiming support, reasoning, and ability than what has been given.

  However, our goal is not only to establish tension.
  Rather, motivate a way out by rejecting \ESU{}.
  To do this, some more things about claiming support and ability need to be said.
  In particular, how reasoning may avoid \ESU{}, and specifically with respect to ability.

  For this reason, we will limit tension to a brief sketch for now.
  The full argument will be given in {\color{red} ???}.
  First, however, we will go to ability and reasoning.

  Though not needed for tension, needed for way out.
  Hence, rather than tension then complexities, complexities then tension.
  For, then, don't need to worry about re-verifying tension in light of complexities.
\end{note}



%%% Local Variables:
%%% mode: latex
%%% TeX-master: "master"
%%% End: