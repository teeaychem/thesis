\chapter{Introduction}
\label{cha:introduction}

% \section{Ability}
% \label{sec:ability-broad-intro}

\begin{note}
  \color{red}
  Some general introduction to the topic, generating some interest for cases in which agent's reasoning with ability.
\end{note}

\begin{note}[Point about ability]
  \begin{enumerate}[label=(E\arabic*), ref=(E\arabic*), series=i_ex]
  \item\label{intro:ex:algebra} If you have the ability to reason with basic algebra and radicals, then you have the ability to verify the following equation (\autoref{fig:into:eq}) it true.
  \end{enumerate}

  \begin{figure}[H]
    \[\frac{(3 + \sqrt{3})^{2} + \sqrt{6}^{2} - (2\sqrt{3})^{2}}{2(3 + \sqrt{3})\sqrt{6}} = \frac{1}{\sqrt{2}}\]
    \caption{A mercifully intimidating equation}
    \label{fig:into:eq}
  \end{figure}

  Suppose you do have the ability to reason with basic algebra and radicals --- developed, at least for me, through the public education of my youth and maintained by some chance of memory.
  Then, the antecedent of the stated conditional is true, and so the consequent of the conditional is true:
  You have the ability to verify the given equation.
  Further, you have the ability to verify the equation only if the equation is true (of course, you also have the ability to verify \emph{whether} the equation is true, but the consequent makes a stronger claim).

  So, if it is true that you have the (general) ability to reason with basic algebra and radicals then it is true that you have the (specific) ability to verify the equation is true.
  And, if it is true that you have the (specific) ability to verify the equation is true, then the equation is true.
  Hence, if it is true that you have the (general) ability to reason with basic algebra and radicals, then the equation is true.

  Of note is that you do not need to witness your (general) ability to reason with basic algebra and radicals to conclude that the equation is true.

  Of course, you may only be \emph{confident} that you have the relevant general ability.
  The same observation applies.
  As \ref{intro:ex:algebra} informs you of an application of a general ability, your confidence that you have the general ability will bound confidence in the result of applying the general ability.

  So, reformed, you do not need to witness you ability to reason with basic algebra and radicals to be confident that the equation is true.
\end{note}

\begin{note}[Focus]
  The focal point of the above is the observation that:
  \begin{enumerate}[label=(O\arabic*), ref=(O\arabic*), series=i_ob]
  \item\label{intro:obs:claim} If an agent is in position to claim support for having some (specific) ability, then the agent is in a position to claim support for the result of witnessing that ability (prior to, or independently of, witnessing that ability).
  \end{enumerate}
  This is true, in part, because:
  \begin{enumerate}[label=(O\arabic*), ref=(O\arabic*), resume*=i_ob]
  \item\label{intro:obs:entail} It would not be possible for an individual to have the (specific) ability (of the kind mentioned in~\ref{intro:obs:claim}) if the result of the witnessing the ability were not true.\nolinebreak
    \footnote{The converse does not hold, some lack the ability to reason with basic algebra and radicals --- I'm confident I lack the ability to read sheet music, though I may have been taught at one point.}
  \end{enumerate}
  Information is valuable in part because don't need to witness ability to be confident.
  Witnessing may be desirable --- check on ability, increased confidence, etc. Or, requirement as part of a test.
  Still, it doesn't seem that you need to verify in order to use the information.

  For example, let me add that:
  \[\frac{1}{\sqrt{2}} = \frac{\sqrt{2}}{2}\]
  I expect you may now be confident that:
  \[\frac{\sqrt{2}}{2} = \frac{(3 + \sqrt{3})^{2} + \sqrt{6}^{2} - (2\sqrt{3})^{2}}{2(3 + \sqrt{3})\sqrt{6}}\]
\end{note}

\begin{note}[Focus and resistance]
  Focal point is about what follows from (specific) ability, when claim support for result
  Question about how important understanding of ability is.

  May think that the observed entailment~\ref{intro:obs:entail} is sufficient.
  So long as the entailment holds, there is no need to puzzle about why the entailment holds.
  Now, something about ability that secures the entailment, but given entailment there is no need to understand what that is in order to use entailment.

  Example, that the light is turned on entails that there is some source of electricity.
  Not to important to understand why this entailment holds --- I have no interest with the physical principles involved.
  What matters to me is that a source of electricity means that I might have the opportunity to charge my phone.

  May hold that \(Kp\) and logical equivalence of \(p\) and \(q\) entails \(Kq\).
  Here, understanding of \(K\) is important.
  Similarly, if the entailment does not hold.

  Proposal, then, is to argue that so long as the entailment observed in~\ref{intro:obs:claim} holds, the details of why entailment holds do not matter.

  If I had told you, then must be true in order for me to make claim.
  All that's required is second observation, and then, no need to witness because even if you don't have the ability I would have implicitly testified that the equation is true.
  So, it doesn't really matter why entailment holds for ability.
  For, confidence in result regardless.
  May still be of some interest, but worse than lights and electric circuit.
  So long as entailment holds, ability itself doesn't seem to matter.

  Likewise, usually discover ability by witnessing.
  I did not recognise I had the ability to factor \(210\) into consecutive primes before having done so.

  Here, receiving information, and I was careful not to provide such implicit testimony that the result is true.
  If you lack the ability to reason with basic algebra and radicals, then the stated conditional is true, even if you lack the ability to verify the given equation.
  So, you may lack the ability to verify the given equation (in part) because the equation is not true.
  Therefore, it seems your confidence that the equation is true is due in part to your confidence that you are able to verify that the equation is true.
  For, if you were not confident in general ability, then lack confidence that you have specific ability, and argument for truth of equation collapses.
  Hence, now appealing to the antecedent to obtain consequent.
  Still some way to go before interest in entailment, but have established in some cases, don't just need the entailment, but also the instantiation of the antecedent by the agent.
\end{note}

\begin{note}[Difficulty]
  Well, the impersonal nature of a paper allows further resistance.
  I don't have any information about who is reading, and it is possible that you do not have the (specific) ability to reason with basic algebra and radicals.
  However, there is someone who has the ability to reason with basic algebra and radicals.
  So, in order to have made a true statement, that someone must also have the ability to verify the given equation.
  And, that someone has the ability to verify the given equation only if the equation is true.
  So, regardless of whether you have the relevant abilities, the equation must be true.

  In short, even if some interest in entailment, it is not clear that \emph{your} ability matters for the entailment.
\end{note}

\begin{note}[`\emph{De dicto}' reading]
  Call the above the `\emph{de dicto}' reading of ability.
  Perhaps a slight misnomer, but the terminology effectively conveys the core idea that reference to an individual (you!) is replaced by an existential quantifier, and it is the satisfaction of the quantifier rather than reference to any particular individual which guarantees the truth of the conditional.
  So, as long as there is some person who has the ability to reason with basic algebra and radicals, then that person has the ability to verify the equation, and so the equation is true.

  Previous that I was careful to avoid didn't require a specific reading, simply observation about entailment.
  The `\emph{de dicto}' reading is more flexible.
  Independent from communication, as you're in a position to infer that the equation is true regardless of the intermediary steps.

  To illustrate, if there is someone who is a spy, then Morsel has composed some poetry.
  And, Morsel has composed some poetry only if there are horses to be spared.
  The antecedent of the first consequent is (almost certainly) true, so it must be true that there are horses to be spared, even if the consequent and intervening steps make no sense to you.
  So long as one may only have the ability to verify some proposition if the proposition is true, then this is all we need to understand about ability.

  A question remains as to whether you also satisfy property of having the ability to reason with basic algebra and radicals, but on the `\emph{de dicto}' reading of ability, this question --- like the question about the details of ability --- is only of derivative interest.
  Likewise, question about why ability has entailment, but nothing particularly interesting about claiming support prior to reasoning, because the `\emph{de dicto}' reading doesn't rest on unwitnessed reasoning.
  For example, may be that only discover ability after successful witnessing by someone, hence, given \emph{de dicto} reading, someone has verified that equation, and so by chain of entailments, reduce information to sundry observation that someone has (already) verified the equation is true.

  Some reasoning with ability is like this.
  Plausible that it's the reasoning you performed.
  And, granting the `\emph{de dicto}' reading, confidence that someone has the ability to reason with basic algebra and radicals is likely stronger than confidence that oneself has the same ability.
  It's a relief --- not only have we avoided any difficult questions about ability, but we've also excused our ability from consideration.

  However, not all reasoning with ability may be captured by the `\emph{de dicto}' reading of ability.
  Some ability statements are directed toward individuals, irrespective of whether some other person may satisfy the property of having the ability.
\end{note}

\begin{note}
  Let me try a second time --- this time I'm addressing you, and none of the other readers.\nolinebreak
  \footnote{They probably don't exist --- though it's unlikely you do either.}
  \begin{enumerate}[label=(E\arabic*), ref=(E\arabic*), resume*=i_ex]
  \item\label{intro:ex:chess} If you are able to reason with the rules of chess, then you have the ability to demonstrate that White cannot prevent Black from occupying c4 on Black's second move given the game state (described in figure~\ref{fig:chess:board:intro}).
\end{enumerate}
\end{note}

\begin{figure}[H]
  \centering
  \mbox{ }
  \hfill
  \begin{subfigure}{.4\textwidth}
    \begin{adjustbox}{minipage=\linewidth,scale=0.7}
      \centering
      \newchessgame[
      setwhite={ka5,pa3,pb4,pc4,pe5,pf6,bg5,bh5},
      addblack={pa6,pb7,pc6,pe6,pf7,kc7,nd7,nd4},
      ]%
      \setchessboard{showmover=false}%
      \chessboard
    \end{adjustbox}
    \caption{
      Game state\newline
      \mbox{ }\newline
    }
    \label{fig:chess:board:intro}
  \end{subfigure}
  \mbox{ }
  \hfill
  \mbox{ }
  \begin{subfigure}{.4\textwidth}
    \begin{adjustbox}{minipage=\linewidth,scale=0.7}
      \centering
      \newchessgame[
      setwhite={ka5,pa3,pb4,pc4,pe5,pf6,bg5,bh5},
      addblack={pa6,pb7,pc6,pe6,pf7,kc7,nd7,nd4},
      ]%
      \setchessboard{showmover=false}%
      \chessboard[
      arrow=latex, linewidth=1pt,
      shortenstart=.8ex, shortenend=.5ex,
      pgfstyle=straightmove,
      strokeopacity=0.4, fillopacity=0.4,
      color=black, pgfstyle=border,
      markfields={c4,a3,a5,g6,c5},
      ]
    \end{adjustbox}
    \caption{Example fields White cannot prevent Black from occupying after two moves.}
    \label{fig:chess:move}
  \end{subfigure}
  \hfill
  \mbox{ }
  \caption{Black to checkmate in four moves.\protect\footnotemark}
  \label{fig:chess:intro}
\end{figure}

\begin{note}

  Have a grasp of the general rules, but perhaps there's more.
  If I'm lucky, then you're like me, and I have the opportunity to say something trivially true.

  Think of me as a sibling.
  I wouldn't flat-out lie to you, but I might (intentionally) mislead you --- especially if I thought you were in an excess of confidence.

  So, assuming you know that I have a good understanding of your ability, whether or not there is a strategy for Black is a function of your confidence in your ability to reason with the rules of chess.
  If you are high confident, then a strategy exists.
  If not confident, then no clear route to whether a strategy exists.

  Return to the reasoning above.
  Call this `\emph{de re}' understanding of ability.

  In contrast to `\emph{de dicto}', interest in how ability works.
  Because, now it's your ability to reason with rules of chess.
  And, given information your ability to demonstrate the existence of a strategy.
  And, why this should allow result granting the possibility of mistake.
\end{note}

\begin{note}[More examples, and narrowing]
  Our interest is with the `\emph{de re}' reading of ability.

%   Two examples given, general phenomena.

%   Handful more.
%   \begin{enumerate}
%   \item You have the ability to recognise that performing that action is a bad idea. (It's a bad idea.)
%   \item If you are able to program this C then speed up. (Speed up.)
%   \item You're able to make that argument only if you're able to defend this premise. (I am able to provide a defence of the premise.)
%   \end{enumerate}
%   Focus on mental activity.
%   \begin{enumerate}
%   \item You have the ability to run a 5k.
%   \item If you are able to hit the center of the dartboard, you win a prize.
%   \item You're able to change lanes only if you're able to raise your speed by 20km/h.
%   \end{enumerate}
%   These cases, either dead-end in ability, or state result of witnessing ability.
%   No clear divide.
%   \begin{enumerate}
%   \item If you are able to answer all of the questions, then pass the test.
%   \end{enumerate}
%   In cases of interest, something independent of ability must be true in order for agent to have ability.
  Returning to~\ref{intro:obs:claim} and~\ref{intro:obs:entail}.

  \gsi{}, and your (specific) ability does the work.
  Observation made above, confidence, and wouldn't have ability if result wasn't (independently) true --- does truth of equation or existence of strategy holds regardless of whether you witness ability that you're confident you have.
  Something more to be said about this observation as it is \emph{your} ability that is granting the result.

  Ability applies to a verb.
  Two verbs used `verify' and `demonstrate' are both factive.
  So, here's an account of why result holds.
  Verify or demonstrate only what is (already) true.

  However, prior observation that ability applies to an action that's of real interest.
  Verb.
  On interpretation of interest, if agent is in position to claim support for (specific) ability, then agent is in a position to witness the ability.
  For example, if you able to reason with basic algebra and radicals, then do not need to learn anything further about algebra or radicals to verify the stated equation.
  Likewise, if you able to reason with the rules of chess, then you do not lack any information required to show that a strategy exists.

  Given our interest is with reasoning, express this by stating that the agent is in a position to claim support for a collection of premises and steps of reasoning whose arrangement would be sufficient to claim support for the result.

  Of course, may take some time, explore a few dead-ends, or you may run out of patience --- no guarantee that you will witness ability.
  Still, there is a path from resources at your disposal to the desired result.

  Third observation
  \begin{enumerate}[label=(O\arabic*), ref=(O\arabic*), resume*=i_ob]
  \item\label{intro:obs:resources} Interpretation of interest, sufficient resources are available to the agent for (specific) ability.
  \end{enumerate}
  The agent is in position to witness their ability.
  Not all ability carries this implication, and it may be difficult to determine whether any isolated does.
  Still, relation between general and specific ability helps clarify.

  XXX

  So, on the one hand, confidence that you have the (specific) ability, and on the other hand confidence that you are in a position to witness the (specific) ability.
  And, witnessing, then support claimed on the basis of available resources.

  Two different ways of understanding `\emph{de re}'.
  Term these \AR{} and \WR{}, respectively.

  Distinction is motivated by verb.
  Ability, so there's something agent may do.
  And, given examples, claim support for result prior to doing whatever it is that the agent is able to do.

  And, if \WR{}/result of verb, then support claimed isn't from ability, but the premises and steps.
  If you were to verify or demonstrate, would not appeal to ability.
  Role of ability in \WR{} is to inform that resources are available.

  \AR{}, by contrast, only required premise.
  Ability nominalises verb, property like any other, and this combined with above observations does the work.

  An analogy may be helpful.
  Think of the distinction between \AR{} and \WR{} as akin to the distinction between the mention and use of a term.

    For example, I may mention the term `questionable', and note that `questionable' contains an instance of each vowel.
  Or, I may use the term `questionable' by stating that it is questionable whether an agent may always appeal to having some (specific) ability when claiming support for the result of witnessing the ability.
  However, when I report that the readers of this thesis found questionable, goes to a use.
  There was some event, in which term was mentioned and used to communicate something.
  To understand implications, need to consider use.

  Question is whether the latter observation holds for ability.
  Does the result follow because claiming support having the ability is sufficient to claim support for conclusion --- in line with \AR{}.
  Or, does the result follow because you are in a position to claim support by witnessing ability and hence claim support from premises and steps.

  So, two interpretations of `\emph{de re}' reading of ability.
  \AR{}, result follows from the agent having a property.
  \WR{}, result follows from event that agent is in a position to witness.
\end{note}

\begin{note}
  What separates the \AR{} and \WR{} interpretations of `\emph{de re}' reading of ability?
  In short, constraints on when an agent may claim support for a proposition.

  Unfortunately, details beyond the scope of an introduction matter.
  

  \WR{} understanding of ability conflicts with simple understanding of when an agent may claim support.

  \begin{enumerate}[label=(O\arabic*), ref=(O\arabic*), resume*=i_ob]
  \item\label{intro:obs:uRa} An agent may claim support for the result of reasoning only if the agent has reasoned from premises to conclusion.
  \end{enumerate}
  If~\ref{intro:obs:uRa}, then \WR{} is ruled out, as the~\ref{intro:obs:uRa} requires the agent to witness the reasoning that they are (confident that) they are able to perform.

  Hence, \AR{}, that the agent has the ability is a premise in any reasoning.
\end{note}

\begin{note}[Interest in~\ref{intro:obs:uRa}]
  Notable in many accounts of the basing relation.
  Also, rationality and so on.
  Requires a step that I will not argue for --- that if successful in claiming support by structure, then inherit support by structure.
  Consider this plausible.
  Talk of claiming so may remain neutral on what support is.
  Doesn't matter whether internalist or externalist from premises and steps, etc.\
  \AR{} and \WR{} are structural.

  Primary argument is conflict.
  \ref{intro:obs:uRa} requires \AR{}, but \AR{} is incompatible with minimal understanding of claiming support.
  Roughly, that agent does not require truth of proposition or existence of support in order to claim support for that proposition.\nolinebreak
  \footnote{
    Note, this does not entail that support is always available.
    If don't own a dog, then there may be no information available for neighbour to claim support that you own a dog.
    Rather, neighbour is not in a position to claim support if recognised by neighbour that X only supports if you already have a dog.
  }
  Internalist?
  No.
  Don't need it to be true that visual perception is reliable to be in a position to claim that something \dots.
  Externalism in part because this is support does not follow from claiming support.

  Conflict isn't so straightforward.

  \begin{enumerate}[label=(O\arabic*), ref=(O\arabic*), resume*=i_ob]
  \item\label{intro:obs:nI} An agent is never in a position to claim support for truth of a proposition \(\psi\) given claimed support for some proposition \(\phi\) by observing that \(\psi\) must be true if \(\psi\) is true when:
    \begin{enumerate}[label=(\alph*), ref=(O\arabic{enumi}\alph*)]
    \item\label{intro:obs:nI:pFalse} Possible \(\psi\) is false
    \item\label{intro:obs:nI:inclusion} Support claimed for \(\phi\) `includes' support claimed for \(\psi\)
    \end{enumerate}
  \end{enumerate}
  `Includes' is a detail.
  Roughly, basic algebra and radicals includes support for verification.
  Likewise, rules of chess includes particular strategy.

  Intuitively, not in a position to claim support for \(\psi\) for else support for \(\phi\) would be problematic.\nolinebreak
  \footnote{
    Details are particularly important here.

    Note, for example, that it seems support claimed for an agent knowing \emph{p} would be problematic were \emph{p} false, but equally my belief that you know \emph{p} seems sufficient to form a belief that \emph{p} --- though \emph{p} may be false given that I only believe that you know \emph{p}.

    The `inclusion' requirement \ref{intro:obs:nI:inclusion} means that the constraint does not immediately apply.
    If I am a layperson, then being in a position to claim support that an expert has testified that \emph{p} does not clearly allow me to claim support for \emph{p} from the resources used to claim that the expert has testified.
  }

  The constraint~\ref{intro:obs:nI} builds on the intuitive idea that an agent is never in a position to claim support for some proposition \(\psi\) on the basis that there would be a problem in having claimed support for some other proposition \(\phi\) were \(\psi\) not the case.

  Restructuring previous:

  \begin{enumerate}[label=(E\arabic*), ref=(E\arabic*), resume*=i_ex]
  \item\label{ex:nI:int:valid} So long as there is no problem in you claiming support for the ability to reason with basic algebra and radicals, then the equation is valid.
  \item\label{ex:nI:int:chess} So long as there is no problem in you claiming support for the ability to reason with the rules of chess, then you a strategy exists.
  \end{enumerate}

  It seems that you may not claim that the equation is valid from~\ref{ex:nI:int:valid} and confidence that you are able to reason with basic algebra and radicals.
  Likewise, it seems you may not claim that a strategy exists because you are confident that you are able to reason with the rules of chess and~\ref{ex:nI:int:chess} requires that a strategy exists unless there is some problem in your claim to support.

  \ref{intro:obs:nI} follows because if agent goes by truth, then they already need to hold that they have support for \(\psi\) prior to moving to the truth of \(\psi\).
  Hence, observing truth is similar to claiming no problem with claimed support for \(\phi\).

  Still, both~\ref{ex:nI:int:valid} and~\ref{ex:nI:int:chess} seem problematic in a way that~\ref{intro:ex:algebra} and~\ref{intro:ex:chess} do not.
  Indeed, both~\ref{intro:ex:algebra} and~\ref{intro:ex:chess} provide information about inclusion.
  Yet, similar feature.
  If \WR{}, then no need to make any claim about whether or not support is problematic, support for result because resources.

  The difficult part is showing that \AR{} violates~\ref{intro:obs:nI}.
\end{note}

\begin{note}
  Goal is to argue that~\ref{intro:obs:nI} follows from a minimal understanding of claiming support.
  If so, ~\ref{intro:obs:uRa} and the minimal understanding of claiming support are in tension.
  For, given that~\ref{intro:obs:uRa} requires \AR{} while \AR{} violates~\ref{intro:obs:nI}.
  Upshot, motivate \WR{} understanding of ability as exception to~\ref{intro:obs:uRa}.

  Alternative remains possible.
  However, with \WR{} well motivated exception to~\ref{intro:obs:uRa}.
  With \AR{}, struggle to find plausible alternative.

  Put it this way:
  \WR{},~\ref{intro:obs:uRa} may be mostly correct.
  If \AR{}, then mostly incorrect about support.

  This is the negative argument.
  \WR{} because alternative is worse.
  Main focus.

  Supplementary is positive argument.
  \WR{} provides novel insights in a number of areas.
\end{note}


\begin{note}
  In \WR{} understanding, agent appeals to premises and steps of reasoning that they would access were they to witness ability.
  Of course, prior to witnessing, agent does not have information about what these are.

  Surprising, because exceptions to~\ref{intro:obs:uRa}.

  Even if no particular interest in details of reasoning with ability, broad consequences given~\ref{intro:obs:uRa}.
\end{note}

\begin{note}[GSI]
  Of course, not all cases generate such interest.
  Consider consequent alone:
  \emph{You have the ability to demonstrate that a strategy exists for Black}.
  In order to claim, strategy must exist.

  Term the type of conditionals `\gsi{}'.
  General, so support for ability, and specific, use of ability to obtain result.
  Interest because confidence from general leads to confidence for specific, and then result, with no other route to result.

  \gsi{} highlight cases in which ability really does something, and if argument is successful then plausible applies to `direct' information about ability, such as in the isolated statement of the consequents or examples given.
\end{note}

\begin{note}[Ideal and non-ideal]
  Ideal agents have no need for ability.
  Non-ideal agents do.
\end{note}

\begin{note}
  Main contributions are:
  Ability, specifically the ability to reason.
  Limitation on claiming support.
  Argument against~\ref{intro:obs:uRa}.
\end{note}

\begin{note}[Nothing comes for free]
  Rests on an understanding of claimed support.
  About claiming support, and limitation of support.
  Basic, clarity, neutral.

  True to the preface paradox, I do not think I have smuggled in strong claim, but I'm not confident to hold that the notion developed is fully general.

  Here, either uninteresting, or room for understanding of ability that does not require \WR{}.
\end{note}

%%% Local Variables:
%%% mode: latex
%%% TeX-master: "master"
%%% End:
