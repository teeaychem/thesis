%%% Local Variables:
%%% TeX-master: "master"
%%% End:

\chapter{Introduction}
\label{cha:introduction}

\section{Ability}
\label{sec:ability-broad-intro}

\begin{note}
  \color{red}
  Some general introduction to the topic, generating some interest for cases in which agent's reasoning with ability.
\end{note}

\begin{note}[Point about ability]
  \begin{enumerate}
  \item\label{intro:AC:algebra} If you have the ability to reason with basic algebra and surds, then you have the ability to verify the following equation (\autoref{fig:into:eq}) it true.
  \end{enumerate}

  \begin{figure}[H]
    \[\frac{(3 + \sqrt{3})^{2} + \sqrt{6}^{2} - (2\sqrt{3})^{2}}{2(3 + \sqrt{3})\sqrt{6}} = \frac{1}{\sqrt{2}}\]
    \caption{A mercifully intimidating equation}
    \label{fig:into:eq}
  \end{figure}

  Suppose you do have the ability to reason with basic algebra and surds --- developed, at least for me, through the public education of my youth and maintained by some chance of memory.
  Then, the antecedent of the stated conditional is true, and so the consequent of the conditional is true:
  You have the ability to verify the given equation.
  Further, you have the ability to verify the equation only if the equation is true (of course, you also have the ability to verify \emph{whether} the equation is true, but the consequent makes a stronger claim).

  So, if it is true that you have the (general) ability to reason with basic algebra and surds then it is true that you have the (specific) ability to verify the equation is true.
  And, if it is true that you have the (specific) ability to verify the equation is true, then the equation is true.
  Hence, if it is true that you have the (general) ability to reason with basic algebra and surds, then the equation is true.

  Of note is that you do not need to witness your (general) ability to reason with basic algebra and surds to conclude that the equation is true.

  Of course, you may only be \emph{confident} that you have the relevant general ability.
  The same observation applies.
  As \ref{intro:AC:algebra} informs you of an application of a general ability, your confidence that you have the general ability will bound confidence in the result of applying the general ability.

  So, you do not need to witness you ability to reason with basic algebra and surds to be confident that the equation is true.
\end{note}

\begin{note}[Focus]
  The focal point of the above is the observation that:
  \begin{enumerate}[label=(O\arabic*)]
  \item If an agent is in position to claim support for having some (specific) ability, then the agent is in a position to claim support for the result of witnessing that ability (prior to, or independently of, witnessing that ability).
  \end{enumerate}
  This is true, in part, because:
  \begin{enumerate}[label=(O\arabic*), resume]
  \item It would not be possible for an individual to have the (specific) ability if the result of the witnessing the ability were not true.\nolinebreak
    \footnote{The converse does not hold, some lack the ability to reason with basic algebra and surds --- I lack the ability to read sheet music, though I may have been taught at one point.}
  \end{enumerate}
  Information is valuable in part because don't need to witness ability to be confident.
  Witnessing may be desirable --- check on ability, increased confidence, etc. Or, requirement as part of a test.
  Still, it doesn't seem that you need to verify in order to use the information.

  For example, let me add that:
  \[\frac{1}{\sqrt{2}} = \frac{\sqrt{2}}{2}\]
  I expect you are now confident that:
  \[\frac{\sqrt{2}}{2} = \frac{(3 + \sqrt{3})^{2} + \sqrt{6}^{2} - (2\sqrt{3})^{2}}{2(3 + \sqrt{3})\sqrt{6}}\]
\end{note}

\begin{note}
  Focal point is about what follows from (specific) ability.
  Question about how important ability is.

  Explained from observation.
  If I had told you, then must be true in order for me to make claim.
  All that's required is second observation, and then, clearly no need to witness because even if you don't have the ability I would have implicitly testified that the equation is true.

  Of course, I was careful not to provide such implicit testimony.
  If you lack the ability to reason with basic algebra and surds, then the stated conditional will remains true even if you lack the ability to verify the given equation, and you may lack the ability to verify the given equation (in part) because the equation is not true.
  If so, then it seems your confidence that the equation is true is due in part to your confidence that you are able to verify that the equation is true.
  For, if you were not confident in general ability, then lack confidence that you have specific ability, and argument for truth of equation collapses.
\end{note}

\begin{note}[Difficulty]
  Well, the impersonal nature of a paper allows further resistance.
  I don't have any information about who is reading, and it is possible that you do have the ability.
  Therefore, in order ensure a true statement it must be that the equation is true.
  I can't be sure that you don't, and hence restrained.
\end{note}

\begin{note}[`\emph{De dicto}' reading]
  Call this the `\emph{de dicto}' reading of ability.
  Perhaps a slight misnomer, but the terminology effectively conveys the core idea that reference to an individual (you!) is replaced by an existential quantifier, and it is the satisfaction of the quantifier rather than reference to any particular individual which guarantees the truth of the conditional.
  So, as long as there is some person who has the ability to reason with basic algebra and surds, then that person has the ability to verify the equation, and so the equation is true.

  Previous that I was careful to avoid didn't require a specific reading, simply observation about entailment.
  `\emph{De dictio}' reading is more flexible.
  Independent from communication, as you're in a position to infer that the equation is true regardless of the intermediary steps.

  To illustrate, if there is someone who is a spy, then Morsel has composed some poetry.
  And, Morsel has composed some poetry only if there are horses to be spared.
  The antecedent of the first consequent is (almost certainly) true, so it must be true that there are horses to be spared, even if the consequent and intervening steps make no sense to you.
  So long as one may only have the ability to verify some proposition if the proposition is true, then this is all we need to understand about ability.

  A question remains as to whether you also satisfy property of having the ability to reason with basic algebra and surds, but on the `\emph{de dicto}' reading of ability, this question --- like the question about the details of ability --- is only of derivative interest.
  Likewise, question about why ability has entailment, but nothing particularly interesting about claiming support prior to reasoning, because the `\emph{de dicto}' reading doesn't rest on unwitnessed reasoning.
  For example, may be that only discover ability after successful witnessing by someone, hence, given \emph{de dicto} reading, someone has verified that equation, and so by chain of entailments, reduce information to sundry observation that someone has (already) verified the equation is true.

  Some reasoning with ability is like this.
  Plausible that it's the reasoning you performed.
  And, granting the `\emph{de dicto}' reading, confidence that someone has the ability to reason with basic algebra and surds is likely stronger than confidence that oneself has the same ability.
  It's a relief --- not only have we avoided any difficult questions about ability, but we've also excused our ability from consideration.

  However, not all reasoning with ability may be captured by the `\emph{de dicto}' reading of ability.
  Some ability statements are directed toward individuals, irrespective of whether some other person may satisfy the property of having the ability.
\end{note}

\begin{note}
  Let me try a second time --- this time I'm addressing you, and none of the other readers.\nolinebreak
  \footnote{They probably don't exist --- though it's unlikely you do either.}
  \begin{enumerate}
  \item\label{intro:AC:chess} If you are able to reason with the rules of chess, then you have the ability to demonstrate that White cannot prevent Black from occupying c4 on Black's second move given the game state (described in figure~\ref{fig:chess:board:intro}).
\end{enumerate}
\end{note}

\begin{figure}[h]
  \centering
  \mbox{ }
  \hfill
  \begin{subfigure}{.4\textwidth}
    \begin{adjustbox}{minipage=\linewidth,scale=0.7}
      \centering
      \newchessgame[
      setwhite={ka5,pa3,pb4,pc4,pe5,pf6,bg5,bh5},
      addblack={pa6,pb7,pc6,pe6,pf7,kc7,nd7,nd4},
      ]%
      \setchessboard{showmover=false}%
      \chessboard
    \end{adjustbox}
    \caption{
      Game state\newline
      \mbox{ }\newline
    }
    \label{fig:chess:board:intro}
  \end{subfigure}
  \mbox{ }
  \hfill
  \mbox{ }
  \begin{subfigure}{.4\textwidth}
    \begin{adjustbox}{minipage=\linewidth,scale=0.7}
      \centering
      \newchessgame[
      setwhite={ka5,pa3,pb4,pc4,pe5,pf6,bg5,bh5},
      addblack={pa6,pb7,pc6,pe6,pf7,kc7,nd7,nd4},
      ]%
      \setchessboard{showmover=false}%
      \chessboard[
      arrow=latex, linewidth=1pt,
      shortenstart=.8ex, shortenend=.5ex,
      pgfstyle=straightmove,
      strokeopacity=0.4, fillopacity=0.4,
      color=black, pgfstyle=border,
      markfields={c4,a3,a5,g6,c5},
      ]
    \end{adjustbox}
    \caption{Example fields White cannot prevent Black from occupying after two moves.}
    \label{fig:chess:move}
  \end{subfigure}
  \hfill
  \mbox{ }
  \caption{Black to checkmate in four moves.\protect\footnotemark}
  \label{fig:chess:intro}
\end{figure}

\begin{note}

  Have a grasp of the general rules, but perhaps there's more.
  If I'm lucky, then you're like me, and I have the opportunity to say something trivially true.

  Think of me as a sibling.
  I wouldn't flat-out lie to you, but I might (intentionally) mislead you --- especially if I thought you were in an excess of confidence.

  So, assuming you know that I have a good understanding of your ability, whether or not there is a strategy for Black is a function of your confidence in your ability to reason with the rules of chess.
  If you are high confident, then a strategy exists.
  If not confident, then no clear route to whether a strategy exists.

  Return to the reasoning above.
  Call this `\emph{de re}' understanding of ability.

  In contrast to `\emph{de dicto}', interest in how ability works.
  Because, now it's your ability to reason with rules of chess.
  And, given information your ability to demonstrate the existence of a strategy.
  And, why this should allow result granting the possibility of mistake.
\end{note}

\begin{note}[More examples, and narrowing]
  Our interest is with the `\emph{de re}' reading of ability.
  
  Two examples given, general phenomena.

  Handful more.
  \begin{enumerate}
  \item You have the ability to recognise that performing that action is a bad idea. (It's a bad idea.)
  \item If you are able to program this C then speed up. (Speed up.)
  \item You're able to make that argument only if you're able to defend this premise. (I am able to provide a defence of the premise.)
  \end{enumerate}
  Focus on mental activity.
  \begin{enumerate}
  \item You have the ability to run a 5k.
  \item If you are able to hit the center of the dartboard, you win a prize.
  \item You're able to change lanes only if you're able to raise your speed by 20km/h.
  \end{enumerate}
  These cases, either dead-end in ability, or state result of witnessing ability.
  No clear divide.
  \begin{enumerate}
  \item If you are able to answer all of the questions, then pass the test.
  \end{enumerate}
  In cases of interest, something independent of ability must be true in order for agent to have ability.
\end{note}

\begin{note}[Ideal and non-ideal]
  Ideal agents have no need for ability.
  Non-ideal agents do.
\end{note}

\begin{note}[GSI]
  Of course, not all cases generate such interest.
  Consider consequent alone:
  \emph{You have the ability to demonstrate that a strategy exists for Black}.
  In order to claim, strategy must exist.

  Term the type of conditionals `\gsi{}'.
  General, so support for ability, and specific, use of ability to obtain result.
  Interest because confidence from general leads to confidence for specific, and then result, with no other route to result.

  \gsi{} highlight cases in which ability really does something, and if argument is successful then plausible applies to `direct' information about ability, such as in the isolated statement of the consequents or examples given.
\end{note}

\begin{note}
  Surface, things are straightforward, but I don't think this is so clear.
  Two different ways of understanding `\emph{de re}'.
  \AR{} and \WR{}.

  \AR{} is somewhat straightforward.
  \WR{} is more interesting.

  Motivation by `mention/use' distinction.
\end{note}

\begin{note}
  What separates?

  Instances of ability are about result of reasoning.

  \WR{} understanding of ability conflicts with simple understanding of when an agent may claim support.

  An agent may claim support for the result of reasoning only if the agent has reasoned from premises to conclusion.

  If simple picture, then \WR{} is ruled out, as the simple picture requires the agent to witness the reasoning that they are (confident that) they are able to perform.

  Hence, \AR{}, that the agent has the ability is a premise in any reasoning.
\end{note}

\begin{note}[Interest in simple picture]
  Notable in many accounts of the basing relation.
  Also, rationality and so on.

  Primary argument is conflict.
  Simple requires \AR{}, but \AR{} is incompatible with minimal understanding of claiming support.
  Roughly, that agent does not require truth of proposition or existence of support in order to claim support for that proposition.\nolinebreak
  \footnote{
    Note, this does not entail that support is always available.
    If don't own a dog, then there may be no information available for neighbour to claim support that you own a dog.
    Rather, neighbour is not in a position to claim support if recognised by neighbour that X only supports if you already have a dog.
  }
  Internalist?
  No.
  Don't need it to be true that visual perception is reliable to be in a position to claim that something \dots.
  Externalism in part because this is support does not follow from claiming support.

  Argue that the simple picture and minimal understanding of claiming support are in tension, given that simple picture requires \AR{}.
  Upshot, motivate \WR{} understanding of ability as exception to simple picture.

  Alternative remains possible.
  However, with \WR{} well motivated exception to simple picture.
  With \AR{}, struggle to find plausible alternative.

  Put it this way:
  \WR{}, the simple picture may be mostly correct.
  If \AR{}, then mostly incorrect about support.

  This is the negative argument.
  \WR{} because alternative is worse.
  Main focus.

  Supplementary is positive argument.
  \WR{} provides novel insights in a number of areas.
\end{note}


\begin{note}
  In \WR{} understanding, agent appeals to premises and steps of reasoning that they would access were they to witness ability.
  Of course, prior to witnessing, agent does not have information about what these are.

  Surprising, because exceptions to simple picture.

  Even if no particular interest in details of reasoning with ability, broad consequences given simple picture.
\end{note}

\begin{note}
  Main contributions are:
  Ability.
  `Simple' limitation.
\end{note}

\begin{note}[Nothing comes for free]
  Rests on an understanding of support.
  About claiming support, and limitation of support.
  Basic, clarity, neutral.

  True to the preface paradox, I do not think I have smuggled in strong claim, but I'm not confident to hold that the notion developed is fully general.

  Here, either uninteresting, or room for understanding of ability that does not require \WR{}.
\end{note}