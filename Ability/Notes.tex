\chapter{Notes}
\label{cha:notes}

\paragraph{Reflection}

\begin{note}[Reflection]
  \begin{quote}
    Reflection states that agents should treat their future selves as experts or, roughly, that an agent’s current credence in any proposition A should equal his or her expected future credence in A.%
    \mbox{ }\hfill\mbox{(\citeyear[59]{Briggs:2009up})}
  \end{quote}
\end{note}

\begin{note}[Difference to reflection]
  Key difference is that in these cases, there's no guarantee that the agent will go through with ability.
  So, it's not necessarily a future self of the agent.
  Though, that's only on a quick surface reading of reflection.

  This is somewhat delicate.
  For, reflection has some strong background assumptions.
  Problem with ability is that agent might witness ability.
  With reflection, we don't consider restrictions on the reasoning the agent would do.

  Now, weakening reflection is difficult.

  One the one hand, can consider all evidence that the agent would reason through.
  If so, then it looks as though ability is going to fall within the scope.
  Problem here, however, because the argument for reflection is in terms of coherence.
  And, it's not clear how to apply conditionalisation to boundedness.
  Dutch books are about coherent credence functions.

  So, it is not clear that there's a way to derive instances of \aben{the} from principles which motivate reflection.
\end{note}

\begin{note}
  Taking a step back, in these kinds of cases it's something like evidence of evidence.
  This is in \textcite[2]{Tal:2017uw}, linking to another paper.

  And, this is kind of similar to what's going on with ability.

  This only works easily from \AR{} perspective.
  Still\dots

  Things get complex here.
  For, if this is the case, then it's not clear that the agent needs to worry about \aben{the}.
  So, the issues arising from the matrix don't really apply.

  Point here is that the agent could go straight for general ability.

  Problem is that agent still needs to get specific ability.

  Same issues with \ESU{} and \nI{} apply here.
  Still get something appealed to but not used.
\end{note}

\paragraph{That theory of testimony}

\begin{note}
  For example, \citeauthor{Owens:2006tw} argues for a belief expression model of assertion in which the rationality of a belief formed by an agent on the basis of testimony depends whatever justification the speaker has for the relevant propositional content.
    \begin{quote}
      Trusting an expression of belief by accepting what a speaker says involves entering a state of mind which gets its rationality from the rationality of the belief expressed.
      This state's rationality depends on the speaker's justification for the belief he expresses, not on his justification for the action of expressing it.
      And to hear a speaker as making a sincere assertion, as expressing a belief, is \emph{ceteris paribus} to feel able to tap into \emph{that} justification (whether or not his assertion was directed at you) by accepting what he says.%
      \mbox{}\hfill\mbox{(\citeyear[123]{Owens:2006tw})}
    \end{quote}
    \color{red} Some more
\end{note}

\paragraph{\zS{}: Probability and norms}

\begin{note}
  The set-up here is whether it makes sense to ask a slightly weaker question than \qzS{}.
  For example, it is likely, or would the agent be violating a norm.
\end{note}

\begin{note}
  Might think that whether the agent would fail to conclude isn't really the issue.
  Instead, from present epistemic state the agent consider it sufficiently likely that the would conclude.

  Here, in particular, we don't run into the \requ{} problem, because it might be the case that the agent would fail.
  So, actually doing the reasoning doesn't work to check this.

  Though, I don't think this is right.
  For, it seems that in the lost keys type of case, it really is the possibility of coming to a distinct conclusion.

  Really, the key idea is that we're dealing with \emph{concluding}.
  This, in turn, places a consistency constraint.
  See~\autoref{concluding-consistency}.
  This doesn't admit of coming to a different conclusion.

  This then applies no matter whether the value \(v\) concerns the proposition being true, or the proposition being sufficiently likely.

  Basic constraint on concluding, and hence what motivates the worry about reasoning to a different conclusion.
\end{note}

\begin{note}
  Press this objection further.
  As the agent has not done the reasoning, then even if \fc{0}, there's still a chance.

  Well, the key response to this is the way I end up understanding these core cases.
  It's ability.
  However, press the issue.
  Well, look, if the understanding of ability is right, then the idea being pressed comes down to the agent revising their present epistemic state.
  And, \emph{this} is compatible with everything said.
  The agent doesn't have the ability.
  Oh no!
\end{note}

\begin{note}
  A different variation on the same idea is that the key parts of \zS{} just express a norm.
  So, in ideal cases, it will be the case that the agent would not conclude, but in standard cases, it is permissible for the agent to fall short of this norm.

  Parallel, norm of knowledge for assertion.
  Something like this.
  Clearly violated, but this doesn't prevent the norm being in effect.

  Norm of only concluding when would not conclude otherwise, this then is in effect, but doesn't prevent an agent from concluding.

  But this seems strange, and stranger than probability.
  For, in these cases, determine whether failing to live up to the norm.

  I think, if there is motivation for \zS{}, then it suggests that this idea doesn't really work.

  And, at least in the case of ability, plausibly satisfy both this and original, stronger, \qzS{}.
\end{note}

\paragraph{Has concluded example}

\begin{note}
  For example, suppose I concluded that Riga is the capital of Latvia by studying a map.
  Now, at present, I am asked by a friend what the capital of Latvia is.
  I takes me a moment, but I recall from memory studying the map and concluding that Riga is the capital of Latvia.
  Hence, I (re-)conclude that Riga is the capital of Latvia, and I do so from the relevant premises involved when studying the map.
\end{note}

\paragraph{Ability}

\begin{note}
  The general idea of \adB{} is perhaps surprising.

  Grant an instance of \adB{}.
  Then, an agent has concluded \(\pv{\phi}{v}\) from \(\Phi\) \emph{via} having concluded \(\pv{\mu}{v}\) for some \itp{0} which states that it is possible for the agent to conclude \(\pv{\phi}{v}\) from \(\Phi\).

  However, if the agent has concluded \(\pv{\mu}{v}\) for some \itp{0} which states that it is possible for the agent to conclude \(\pv{\phi}{v}\) from \(\Phi\), then \(\pv{\mu}{v}\) alone is seem sufficient to conclude \(\pv{\phi}{v}\).
  Indeed, at issue is principle which may be stated quite generally:

  \begin{idea}
    \label{idea:c-from-pc}
    It is permissible for an agent to conclude \(\pv{\phi}{v}\) from the premise that it is possible for them to witness reasoning which concludes \(\pv{\phi}{v}\), given their present epistemic state.
  \end{idea}

  Arguing in detail for~\autoref{idea:c-from-pc} goes beyond present interest.
  I take the idea to be sufficiently intuitive.

  Of course, concluding is not factive, so the possibility of witnessing reasoning which concludes \(\pv{\phi}{v}\) does not guarantee \(\phi\) has value \(v\).
  However, by the same token,~\autoref{idea:c-from-pc} only concerns the agent concluding \(\pv{\phi}{v}\), so it seems one only needs to argue that \(\phi\) must have value \(v\) from the agent's present epistemic state.
  And, if from the agent's present epistemic state it is possible for them to witness reasoning which concludes \(\pv{\phi}{v}\), it seems the agent may not question whether \(\phi\) has value \(v\) without questioning their conclusion that it is possible for them to witness reasoning which concludes \(\pv{\phi}{v}\).

  Still, set motivation aside.
  Our interest with~\autoref{idea:c-from-pc} is with the distinction between \adA{} and \adB{}.
  And, in particular, it seems that if~\autoref{idea:c-from-pc} is granted, then the resources present for any instance of \adB{} will always allow for an instance of \adA{}.

  For, an instance of \adB{} requires an agent has concluded \(\pv{\mu}{v}\), where \(\pv{\mu}{v}\) is an \itp{0}.
  Suppose \(\pv{\mu}{v}\) is an \itp{0} for \(\pv{\phi}{v}\).

  Now, by~\autoref{idea:c-from-pc}, it is permissible for the agent to conclude \(\pv{\phi}{v}\) from \(\pv{\mu}{v}\).
  The agent has concluded it is possible for them to witness reasoning which concludes \(\pv{\phi}{v}\), given their present epistemic state, and therefore the agent may conclude \(\pv{\phi}{v}\).

  Indeed, without any further thought, it seems we have established that any proposed instance of \adB{} must contain sufficient detail to allow the instance to be (at least) reinterpreted as an instance of \adA{}.

  I think this is broadly correct.
  Granting~\autoref{idea:c-from-pc}, then the relevant \itp{0} from any instance of \adB{} will allow the agent to conclude \(\pv{\phi}{v}\) from the \itp{0}.
  And, I grant~\autoref{idea:c-from-pc}.

  Note, however, I have refrained from stating that the agent concluding \(\pv{\phi}{v}\) from the \itp{0} is an instance of \adA{}.
  For, appeal to~\autoref{idea:c-from-pc} does not state that the agent concluding \(\pv{\phi}{v}\) from the \itp{0} is a complete account of the instance of reasoning.
  Expanding, we have not show that the agent concluding \(\pv{\phi}{v}\) from the \itp{0} does not also include the agent concluding some other proposition-value pair \(\pv{\psi}{v'}\) from some other premises.
  And, if the instance does involve the agent concluding \(\pv{\psi}{v'}\) from some other premises, the instance of reasoning will only be an instance of \adA{} if the agent witnesses reasoning from those premises.

  In other words,~\autoref{idea:c-from-pc} does not immediately show that have the option of analysing away any proposed instance of \adB{}.
  For, we have not shown that any analysis under \adA{} provides a complete account of the proposed instance of \adB{}.
\end{note}

\begin{note}
  For the moment we leave this observation as a promissory note.
  Tension will be developed in~\autoref{sec:tension}.
  Still, we have already seen one way of introducing additional structure into some conclusion by focusing on instances of concluding which are also instances of \csN{0}.
  And, an important part of the argument for tension in~\autoref{sec:tension} will be showing that certain instances of \csN{} require \adB{} as no analysis under \adA{} will provide a complete account of the relevant instance of concluding/\csVImp{0}.
\end{note}


%%% Local Variables:
%%% mode: latex
%%% TeX-master: "master"
%%% End:
