\chapter{Notes}
\label{cha:notes}

\paragraph{Reflection}

\begin{note}[Reflection]
  \begin{quote}
    Reflection states that agents should treat their future selves as experts or, roughly, that an agent’s current credence in any proposition A should equal his or her expected future credence in A.\linebreak
    \mbox{}\hfill\mbox{(\Citeyear[59]{Briggs:2009up})}
  \end{quote}
\end{note}

\begin{note}[Difference to reflection]
  Key difference is that in these cases, there's no guarantee that the agent will go through with ability.
  So, it's not necessarily a future self of the agent.
  Though, that's only on a quick surface reading of reflection.

  This is somewhat delicate.
  For, reflection has some strong background assumptions.
  Problem with ability is that agent might witness ability.
  With reflection, we don't consider restrictions on the reasoning the agent would do.

  Now, weakening reflection is difficult.

  One the one hand, can consider all evidence that the agent would reason through.
  If so, then it looks as though ability is going to fall within the scope.
  Problem here, however, because the argument for reflection is in terms of coherence.
  And, it's not clear how to apply conditionalisation to boundedness.
  Dutch books are about coherent credence functions.

  So, it is not clear that there's a way to derive instances of \aben{the} from principles which motivate reflection.
\end{note}

\begin{note}
  Taking a step back, in these kinds of cases it's something like evidence of evidence.
  This is in \textcite[2]{Tal:2017uw}, linking to another paper.

  And, this is kind of similar to what's going on with ability.

  This only works easily from \AR{} perspective.
  Still\dots

  Things get complex here.
  For, if this is the case, then it's not clear that the agent needs to worry about \aben{the}.
  So, the issues arising from the matrix don't really apply.

  Point here is that the agent could go straight for general ability.

  Problem is that agent still needs to get specific ability.

  Same issues with \ESU{} and \nI{} apply here.
  Still get something appealed to but not used.
\end{note}

%%% Local Variables:
%%% mode: latex
%%% TeX-master: "master"
%%% End:
