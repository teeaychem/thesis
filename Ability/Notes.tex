\chapter{Notes}
\label{cha:notes}

\paragraph{Reflection}

\begin{note}[Reflection]
  \begin{quote}
    Reflection states that agents should treat their future selves as experts or, roughly, that an agent’s current credence in any proposition A should equal his or her expected future credence in A.%
    \mbox{ }\hfill\mbox{(\citeyear[59]{Briggs:2009up})}
  \end{quote}
\end{note}

\begin{note}[Difference to reflection]
  Key difference is that in these cases, there's no guarantee that the agent will go through with ability.
  So, it's not necessarily a future self of the agent.
  Though, that's only on a quick surface reading of reflection.

  This is somewhat delicate.
  For, reflection has some strong background assumptions.
  Problem with ability is that agent might witness ability.
  With reflection, we don't consider restrictions on the reasoning the agent would do.

  Now, weakening reflection is difficult.

  One the one hand, can consider all evidence that the agent would reason through.
  If so, then it looks as though ability is going to fall within the scope.
  Problem here, however, because the argument for reflection is in terms of coherence.
  And, it's not clear how to apply conditionalisation to boundedness.
  Dutch books are about coherent credence functions.
\end{note}

\paragraph{\zS{}: Probability and norms}

\begin{note}
  The set-up here is whether it makes sense to ask a slightly weaker question than \qzS{}.
  For example, it is likely, or would the agent be violating a norm.
\end{note}

\begin{note}
  Might think that whether the agent would fail to conclude isn't really the issue.
  Instead, from present epistemic state the agent consider it sufficiently likely that the would conclude.

  Here, in particular, we don't run into the \requ{} problem, because it might be the case that the agent would fail.
  So, actually doing the reasoning doesn't work to check this.

  Though, I don't think this is right.
  For, it seems that in the lost keys type of case, it really is the possibility of coming to a distinct conclusion.

  Really, the key idea is that we're dealing with \emph{concluding}.
  This, in turn, places a consistency constraint.
  Now concluding \(\pv{\phi}{v}\) and \(\pv{\phi}{\overline{v}}\).
  This doesn't admit of coming to a different conclusion.

  This then applies no matter whether the value \(v\) concerns the proposition being true, or the proposition being sufficiently likely.

  Basic constraint on concluding, and hence what motivates the worry about reasoning to a different conclusion.
\end{note}

\subsection{Factivity}
\label{sec:factivity}

\begin{note}
  Delicacy.
  \fc{1} and \ros{} from the \agpe{}.

  Two issues:

  \begin{enumerate}
  \item
    If independent of \agpe{}, then it may be the case that \fc{} without any prior recognition from agent.
  \item
    If dependent on \agpe{}, then it may be the case that not a \fc{}.
  \end{enumerate}

  Two issues are important.
  Without \agpe{}, then unclear that get \ros{} of interest for answer to \qWhyV{}.
  Given \agpe{}, unclear that we get a \ros{}.
\end{note}

\section{\requ{3} and undercutting defeaters}

\begin{note}
  Shares similarity.
  Focus on concluding.
  However, subjunctive.
  \requ{3} concern entertaining proposition-value-premises pairings as an undercutting defeater for concluding.
\end{note}

\begin{note}
  Not about the proposition-value pair.
  Rather, it is about the concluding.
  At interest is not whether \(\phi\) has value \(v\), but whether it makes sense to conclude \(\pv{\phi}{v}\) from \(\Phi\).
  Of course, if the agent has no information about whether \(\phi\) has value \(v\), then this is also part of the picture, but that is a consequence of the base concern.

  With squish.
  I wouldn't conclude.
  However, wouldn't say that entailment doesn't hold.
  For, may only be the case that the derived rule of inference fails to hold.
\end{note}

\begin{note}
  In this respect, failing to conclude \(\pv{\psi}{v'}\) from \(\Phi\) may be described as an undercutting defeater with respect to conclude \(\pv{\phi}{v}\) from \(\Phi\).%
  \footnote{
    We borrow the following sketch from \textcite{Worsnip:2018aa}:
  \begin{quote}
    Undercutting defeaters, which are easiest to think of in the context of the attitude of belief, are supposed to be considerations that undermine the justification of a belief in a proposition p not necessarily by providing (sufficient) positive evidence to think that p is false, but rather merely by suggesting (perhaps misleadingly) that one’s reasons for believing p are no good, in a way that neutralizes or mitigates their justificatory or evidential force.%
    \mbox{}\hfill\mbox{(\citeyear[29]{Worsnip:2018aa})}
  \end{quote}
  }

  Consider the following illustration provided by \citeauthor{Pollock:1987un}:
  \begin{quote}
    [Undercutting defeaters] attack the connection between the reason and the conclusion rather than attacking the conclusion itself.
    For instance, ``X looks red to me'' is a prima facie reason for me to believe that X is red.
    Suppose I discover that X is illuminated by red lights and illumination by red lights often makes things look red when they are not.
    This is a defeater, but it is not a reason for denying that X is red (red things look red in red light too).
    Instead, this is a reason for denying that X wouldn't look red to me unless it were red.%
    \mbox{}\hfill\mbox{(\citeyear[485]{Pollock:1987un})}
  \end{quote}
  Completing \citeauthor{Pollock:1987un}'s example, it seems that if agent's support for holding that X is red is that `X wouldn't look red to me unless it were red', then the support for X being red provided by appearance is retracted after discovering that X is illuminated by red lights (though it remains possible that X is red).
\end{note}

\begin{note}
  In \citeauthor{Pollock:1987un}'s example, discover that the light is red.
  By parallel, the agent failing to conclude would undercut.
\end{note}

\begin{note}
  Similarity.

  However, given the focus on concluding, \curb{} is not a simple instance of an undercutting defeater.
  At least, given \citeauthor{Pollock:1987un}'s definition of an undercutting defeater.

  \citeauthor{Pollock:1987un} defines undercutting defeaters as follows:
  \begin{quote}
    R is an \emph{undercutting defeater} for P as a prima facie reason for S to believe Q if and only if
    \begin{enumerate}[label=(UD\arabic*), ref=(UD\arabic*)]
    \item
      \label{pollock:ud:1}
      P is a reason for S to believe Q and R is logically consistent with P but (P and R) is not a reason for S to believe Q, and
    \item
      \label{pollock:ud:2}
      R is a reason for denying that P wouldn't be true unless Q were true.%
      \mbox{}\hfill\mbox{(\citeyear[485]{Pollock:1987un})}
    \end{enumerate}
  \end{quote}
  Intuitively, an undercutting defeater for P as a reason for Q because it the defeater denies that Q must be true in order for P to be true.%
  \footnote{
    \citeauthor{Wright:2011wn}'s (\citeyear{Wright:2011wn}) revised template:
    \begin{quote}
      Where A entails B, a rational claim to warrant for A is not transmissible to B if there is some proposition C such that:
      \begin{enumerate}[label=(\roman*), noitemsep]
      \item
        The process/state of accomplishing the relevant putative warrant for A is subjectively compatible with C’s holding: things could be with one in all respects exactly as they subjectively are yet C be true
      \item
        C is incompatible (not necessarily with A but) with some presupposition of the cognitive project of obtaining a warrant for A in the relevant fashion, and
      \item
        Not-B entails C%
      \mbox{ }\hfill\mbox{(\citeyear[93]{Wright:2011wn})}
      \end{enumerate}
    \end{quote}
    Difficulty with the process, however, of interest is transmission of warrant from A to B.
    Hence, the agent has accomplished the relevant putative warrant for A.

    Or, consider the fourth type of dependence between premise and conclusion considered (but not endorsed) by \textcite{Pryor:2004ws}:

  \begin{quote}
    [Type 4] dependence between premise and conclusion is that the conclusion be such that evidence \emph{against it} would (to at least some degree) undermine the kind of justification you purport to have for the premises.%
    \mbox{}\hfill\mbox{(\citeyear[359]{Pryor:2004ws})}
  \end{quote}
  Premises!

  \nocite{Weisberg:2012vs}
  Closer to \citeauthor{Weisberg:2010to}'s (\citeyear{Weisberg:2010to}) account of bootstrapping.
  However, implicit circularity.
  And, circularity is not at issue.
  }
\end{note}

\begin{note}
  Difficulty with a clear parallel


  \ref{pollock:ud:1}.
  For, P \emph{is a} reason.

  In the case of \curb{1}, failure to be concluding.
  For, event may develop such that agent does not conclude.

  \ref{pollock:ud:1}.

  Some other way for P to be true.
\end{note}

\begin{note}
  However, ties things too closely to presentation, rather than substance of ideas.
  Could be said that agent is concluding, but there is an additional way in which the event may develop.

  Variation:
  \begin{quote}
    R is \emph{undercutting reasoning} with respect to \emph{S} concluding \(\pv{\phi}{v}\) from \(\Phi\) if and only if
    \begin{enumerate}[label=(UR\arabic*), ref=(UR\arabic*)]
    \item
      The event may develop such that \emph{S} to concludes \(\pv{\phi}{v}\) from \(\Phi\) and R is a \pevent{} which is logically consistent with reasoning from \(\Phi\) but combination of both reasoning from \(\Phi\) and R is not a sufficient for \emph{S} to be concluding \(\pv{\phi}{v}\), and
    \item
      R is sufficient for denying that S wouldn't conclude \(\pv{\phi}{v}\) from \(\Phi\) unless \(\pv{\phi}{v}\) followed from \(\Phi\).%
    \end{enumerate}
    Where, R is reasoning which fails to conclude \(\pv{\psi}{v'}\) from \(\Psi\).
  \end{quote}

  {
    \color{red}
    So, the emphasis here is on whether there's really anything to be said for the conclusion.
    However, if this is the case, then it seems the agent simply doesn't conclude.
  }

  The problem here is with the first clause.
  Logically consistent.

  It is not clear that the reasoning is logically consistent.
  Two instances of reasoning may involve intermediary steps which are logically inconsistent.

  Though, on the other hand, in interesting cases it is logically consistent for the agent to reason to \(\pv{\phi}{v}\) from \(\Phi\) and fail to conclude \(\pv{\psi}{v'}\) from \(\Psi\).
  For, if not logically possible, then no worries about failing to conclude \(\pv{\psi}{v'}\) from \(\Psi\).

  Yet, if problem, then from the \agpe{}, instances of reasoning aren't consistent.

  Puzzle here is how logical consistency is understood with respect to \citeauthor{Pollock:1987un}'s definition.
  Independent of the \agpe{}, or from the \agpe{}?

  Both interpretations are compatible with the example.
  But, principle also holds independently of the \agpe{} with respect to the example.
\end{note}

\begin{note}
  In contrast, \curb{} `attacks' the \emph{reasoning} and the conclusion.
\end{note}


%%% Local Variables:
%%% mode: latex
%%% TeX-master: "master"
%%% TeX-engine: luatex
%%% End:
