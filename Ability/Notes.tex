\chapter{Notes}
\label{cha:notes}

\paragraph{Reflection}

\begin{note}[Reflection]
  \begin{quote}
    Reflection states that agents should treat their future selves as experts or, roughly, that an agent’s current credence in any proposition A should equal his or her expected future credence in A.%
    \mbox{ }\hfill\mbox{(\citeyear[59]{Briggs:2009up})}
  \end{quote}
\end{note}

\begin{note}[Difference to reflection]
  Key difference is that in these cases, there's no guarantee that the agent will go through with ability.
  So, it's not necessarily a future self of the agent.
  Though, that's only on a quick surface reading of reflection.

  This is somewhat delicate.
  For, reflection has some strong background assumptions.
  Problem with ability is that agent might witness ability.
  With reflection, we don't consider restrictions on the reasoning the agent would do.

  Now, weakening reflection is difficult.

  One the one hand, can consider all evidence that the agent would reason through.
  If so, then it looks as though ability is going to fall within the scope.
  Problem here, however, because the argument for reflection is in terms of coherence.
  And, it's not clear how to apply conditionalisation to boundedness.
  Dutch books are about coherent credence functions.

  So, it is not clear that there's a way to derive instances of \aben{the} from principles which motivate reflection.
\end{note}

\paragraph{That theory of testimony}

\begin{note}
  For example, \citeauthor{Owens:2006tw} argues for a belief expression model of assertion in which the rationality of a belief formed by an agent on the basis of testimony depends whatever justification the speaker has for the relevant propositional content.
    \begin{quote}
      Trusting an expression of belief by accepting what a speaker says involves entering a state of mind which gets its rationality from the rationality of the belief expressed.
      This state's rationality depends on the speaker's justification for the belief he expresses, not on his justification for the action of expressing it.
      And to hear a speaker as making a sincere assertion, as expressing a belief, is \emph{ceteris paribus} to feel able to tap into \emph{that} justification (whether or not his assertion was directed at you) by accepting what he says.%
      \mbox{}\hfill\mbox{(\citeyear[123]{Owens:2006tw})}
    \end{quote}
    \color{red}
    Ah, this is different, as there's no concluding.
\end{note}

\paragraph{\zS{}: Probability and norms}

\begin{note}
  The set-up here is whether it makes sense to ask a slightly weaker question than \qzS{}.
  For example, it is likely, or would the agent be violating a norm.
\end{note}

\begin{note}
  Might think that whether the agent would fail to conclude isn't really the issue.
  Instead, from present epistemic state the agent consider it sufficiently likely that the would conclude.

  Here, in particular, we don't run into the \requ{} problem, because it might be the case that the agent would fail.
  So, actually doing the reasoning doesn't work to check this.

  Though, I don't think this is right.
  For, it seems that in the lost keys type of case, it really is the possibility of coming to a distinct conclusion.

  Really, the key idea is that we're dealing with \emph{concluding}.
  This, in turn, places a consistency constraint.
  See~\autoref{concluding-consistency}.
  This doesn't admit of coming to a different conclusion.

  This then applies no matter whether the value \(v\) concerns the proposition being true, or the proposition being sufficiently likely.

  Basic constraint on concluding, and hence what motivates the worry about reasoning to a different conclusion.
\end{note}

\begin{note}
  Press this objection further.
  As the agent has not done the reasoning, then even if \fc{0}, there's still a chance.

  Well, the key response to this is the way I end up understanding these core cases.
  It's ability.
  However, press the issue.
  Well, look, if the understanding of ability is right, then the idea being pressed comes down to the agent revising their present epistemic state.
  And, \emph{this} is compatible with everything said.
  The agent doesn't have the ability.
  Oh no!
\end{note}

\begin{note}
  A different variation on the same idea is that the key parts of \zS{} just express a norm.
  So, in ideal cases, it will be the case that the agent would not conclude, but in standard cases, it is permissible for the agent to fall short of this norm.

  Parallel, norm of knowledge for assertion.
  Something like this.
  Clearly violated, but this doesn't prevent the norm being in effect.

  Norm of only concluding when would not conclude otherwise, this then is in effect, but doesn't prevent an agent from concluding.

  But this seems strange, and stranger than probability.
  For, in these cases, determine whether failing to live up to the norm.

  I think, if there is motivation for \zS{}, then it suggests that this idea doesn't really work.

  And, at least in the case of ability, plausibly satisfy both this and original, stronger, \qzS{}.
\end{note}



%%% Local Variables:
%%% mode: latex
%%% TeX-master: "master"
%%% End:
