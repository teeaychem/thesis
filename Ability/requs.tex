\chapter{\requ{3}}
\label{cha:zS:sec:requs}

\begin{note}
  \autoref{cha:zS:sec:curbs} introduced the idea of some proposition-value-premise pairing being a \curb{} on concluding \(\pv{\phi}{v}\) from \(\Phi\).

  As observed, dependence in line with dependence identified by \qWhyV{}.
  However, no \ros{}.

  In this chapter, \requ{}.
  From the \agpe{}.
  Converse problem.
  The \agpe{} doesn't establish dependence.
\end{note}

\section{\requ{3}}
\label{sec:requ3}

\begin{note}
  \begin{definition}[A \requ{0}]
    \label{def:requ}
    For:

    \begin{itemize*}[noitemsep]
    \item
      An agent: \vAgent{}
    \item
      Propositions: \(\phi\), \(\psi\)
    \item
      Values: \(v\), \(v'\)
    \item
      \poP{3}: \(\Phi\), \(\Psi\)
    \item
      An event: \(e\)
    \end{itemize*}

    \begin{itemize}
    \item
    \(\pvp{\phi}{v'}{\Psi}\) is a \emph{\requ{}} of \(e\) just in case:

    \begin{enumerate}[label=\alph*., ref=(\alph*), series=requDefSeries]
      \item
        \(e\) is an event in which \vAgent{} is concluding \(\pv{\phi}{v}\) from \(\Phi\).
      \end{enumerate}

    \emph{Only if}

    \begin{enumerate}[label=\alph*., ref=(\alph*), resume*=requDefSeries]
    \item
      Both \ref{def:requ:curb-agpe} and \ref{def:requ:feedback} hold:
      \begin{enumerate}[label=\roman*., ref=(\roman*)]
      \item
        \label{def:requ:curb-agpe}
        \(\pvp{\psi}{v'}{\Psi}\) is a \curb{} on concluding \(\pv{\phi}{v}\) from \(\Phi\), from \agpe{\vAgent{}'}.
      \item
        \label{def:requ:feedback}
        \(\pvp{\psi}{v'}{\Psi}\) is a \curb{} on concluding \(\pv{\phi}{v}\) from \(\Phi\).%, due to \ref{def:requ:curb-agpe}.
      \end{enumerate}
    \end{enumerate}
  \end{itemize}
  \vspace{-\baselineskip}
  \end{definition}

  `\requ{2}' in the sense of `deemed necessary'.

  So, understand \requ{}.
  Here, from \agpe{}.
  Event is an event in which the agent is concluding \(\pv{\phi}{v}\) from \(\Phi\) just in case there is a \pevent{} in which the agent concludes \(\pv{\psi}{v'}\) from \(\Psi\).

  From \ref{def:requ:curb-agpe}, if there's no \pevent{} in which conclude \(\pv{\psi}{v'}\) from \(\Psi\) then it is not the case that agent is concluding \(\pv{\phi}{v}\) from \(\Phi\).

  So, cases where \ref{def:requ:curb-agpe} holds are straightforward.
  Consider various cases we have seen so far.

  And, from \ref{def:requ:feedback}, the agent's perspective influences whether or not the event develops such that the agent concludes \(\pv{\phi}{v}\) from \(\Phi\).

  We term this \feedback{}.
\end{note}

\section{Two initial illustrations}
\label{sec:two-init-illustr}

\paragraph{Lost keys}

\begin{note}
  \begin{scenario}[Lost keys]
    \label{illu:lost-key}
    I think I might have lost my keys.
    I usually leave place my keys on the right side of my desk, next to a copy of~\citeauthor{Vickers:1989tr}'s~\citetitle{Vickers:1989tr} which I've been saving for a rainy day.
    And, my keys aren't there.

    I've searched on the desk, under the desk, and beside the desk.
    And, I haven't found my keys.

    Still, I haven't (yet, at least) \emph{concluded} that I've lost my keys.

    For, there might still be some place I haven't looked.
    If I think a little harder a figure out where that place is, I would conclude my keys might be in that place.
    And, my keys aren't lost if they are in that place.
    So, I might conclude that my keys aren't lost, which would conflict with concluding that my keys are lost.
  \end{scenario}

  You may disagree with the tension I see in~\autoref{illu:lost-key}.
  Perhaps it's fine to conclude my keys are lost while allowing for the possibility that they're some place I haven't yet thought of.
  However, there's tension for me.
  `I've lost my keys, but they might be under that book' feels bad to me, and to me the badness extends to `I've lost my keys, but they might in that place I haven't yet considered'.

  Though, my goal is only to convince you that my refusal to conclude I've lost my keys makes sense.
  The way in which you think about the truth conditions for the sentence `I've lost my keys' may be different, but I expect my thoughts are intelligible.
\end{note}

\begin{note}
  Filling in the details of the abstract sketch:
  \begin{itemize}[noitemsep]
  \item
    I am the agent.
  \item
    \(\phi\) is the proposition: `I've lost my keys'.
  \item
    \(\psi\) is a some proposition: `My keys are not in location \(l\)'
  \item
    Both \(v\) and \(v'\) are the value: `True'.
    And,
  \item
    The pools of premises \(\Phi\) and \(\Psi\) are left unspecified.
  \end{itemize}
\end{note}

\paragraph{A trip to the zoo}

\begin{note}[Failure but no option]
  \citeauthor{Dretske:1970to}.
  \begin{scenario}[A trip to the zoo]\mbox{ }
    \label{scen:trip-to-zoo}
    \vspace{-\baselineskip}
    \begin{quote}
      You take your son to the zoo, see several zebras, and, when questioned by your son, tell him they are zebras.
      Do you know they are zebras?
      [\dots]
      We know what zebras look like, and, besides, this is the city zoo and the animals are in a pen clearly marked ``Zebras.''
      Yet, something's being a zebra implies that it is not a mule and, in particular, not a mule cleverly disguised by the zoo authorities to look like a zebra.
      Do you know that these animals are not mules cleverly disguised by the zoo authorities to look like zebras?\newline
      \mbox{ }\hfill\mbox{(\citeyear[1015--1016]{Dretske:1970to})}
    \end{quote}
    \vspace{-\baselineskip}
  \end{scenario}

  \autoref{scen:trip-to-zoo} is framed in terms of knowledge, and is designed to raise a problem for conclude of knowledge under known entailment.
  Intuitively, you know the animals in the pen are zebras.
  And, you know the following conditional is true:
  The animals in the pen are zebras \emph{only if} the animals in the pen are not cleverly disguised mules.
  However, you (intuitively) don't know the animals in the pen are not cleverly disguised mules.

  If knowledge is closed under known entailment, then you:

  \begin{enumerate}
  \item
    \(\phi\) has value \(v\) only if \(\psi\) has value \(v'\)
  \end{enumerate}

  then, if

  \begin{enumerate}
  \item
    \(\phi\) has value \(v\), then
  \end{enumerate}

  \begin{enumerate}
  \item
    \(\psi\) has value \(v'\)
  \end{enumerate}

  So, if closure of knowledge under know entailment, and conclusion is that you know, then any further reasoning, need to conclude.

  So, right now, from the current \poP{}, not counting additional information.

  Intuitively, don't know, and there is no \pevent{}.
  However, this is not due to.

  For, holds regardless of whether or not.
\end{note}

\begin{note}
  Still, knowledge.

  Not concluding unless knows.
  Then, if no \pevent{}, then agent doesn't know.
  {
    \color{red}
    I should come back to this point.
    For, it does seem a weakness.
    One could deny that knowledge ever interacts in this way.
    For example, possibility of no external world.

    However, if this is a concern, then it's not clear to me that there is anything interesting to be said about concluding.
    To the extent have concluded via understanding of rules of Sudoku, know that \pevent{}.
    If worried about lack of external world, then no guarantee that concluded from rules of Sudoku.

    So, really, the point is that it seems strange to break this link.
  }

  Event does not develop unless know.
  Therefore, does not develop unless there is some \pevent{}.

  Because, if do not know then will check.
  And, if check fails then will not conclude.
\end{note}


\paragraph{Sound rules}

\begin{note}
  Here, I want to state that the agent has done the proof a number of times.
  So, the agent knows that there is a \pevent{}.
\end{note}

\begin{note}
  Return to~\autoref{scen:squish}:

  \scenarioPLSquish*

  \scen{0} focused on the non-standard `Squish'-elimination rule of inference and the possibility of concluding `Squish'-elimination is a sound rule of inference.

  Uncommon, but enough to memorise the rule.
  Still, I consider my general understanding of propositional logic more important than memory.
  And, if failed, then would not consider sound.

  Filling in the details of the second abstract sketch:
  \begin{itemize}[noitemsep]
  \item
    I am the agent.
  \item
    \(\phi\) is the proposition: `\((P \rightarrow Q) \rightarrow P, Q \vdash P \land Q\)'.
  \item
    \(\psi\) is a some proposition: `Squish elimination is sound'
  \item
    Both \(v\) and \(v'\) are the value: `True'.
    And,
  \item
    The pools of premises \(\Phi\) and \(\Psi\) are left unspecified.%
    \footnote{
      Note, premises of reasoning.
      Distinct from premises of deduction.
    }
  \end{itemize}

  Still, I do not entertain the possibility of failing to show sound.

  With some luck, uncommon and you will witness the conclusion similar to what I would, if I chosen to reason.%
  \footnote{
    To preserve the integrity of the \illu{0}, the proof on page~\pageref{squish-elimination-proof} was considered \emph{after} concluding the (syntactic) consequence via Squish-elimination.
  }
\end{note}

\section{\feedback{2}}

\begin{note}
  \begin{definition}[\feedback{2}]
    If \curb{} from \agpe{}, then \curb{}.
  \end{definition}

  This is what we're really interested in.
  Some self reflection, and concerns about the way in which the present event may develop influence the development of the event.
\end{note}

\begin{note}
  \feedback{} is a non-trivial.
  It is not only the case that block from the \agpe{}, but actual block.
\end{note}

\subsection{\feedback{2} and \fc{1}}
\label{sec:fc}

\begin{note}
  Noted that \curb{} does not entail \fc{}.
  The same holds for \requ{1}.

  For, don't need to ensure that no other conclusion.

  However, \feedback{0} does interesting things.
  Because, if know conclusion, in any non-trivial sense, then cannot be the case that conclude anything else.
\end{note}


\begin{note}
  \begin{proposition}[\requ{3} and \fc{1}]
    For an agent \vAgent{}, and proposition-value-premises pairings \(\pvp{\phi}{v}{\Phi}\), \(\pvp{\psi}{v'}{\Psi}\):

    \begin{itemize}
    \item
      \(\pvp{\phi}{v'}{\Psi}\) is a \requ{} of \vAgent{} concluding \(\pv{\phi}{v}\) from \(\Phi\)
    \item
      Know \pevent{}.
    \item
      Not arbitrary.
    \end{itemize}

    \emph{Only if}

    \begin{itemize}
    \item
        \begin{enumerate}
        \item[\emph{If}:]
          \(\pv{\psi}{v'}\) is not a \fc{} from \(\Psi\), from \agpe{\vAgent{}'}.
        \item[\emph{Then}:]
          \begin{enumerate}[label=\alph*., ref=(\alph*), resume]
          \item
            \label{def:curb:fail}
            \vAgent{} would not be concluding \(\pv{\phi}{v}\) from \(\Phi\).
          \end{enumerate}
      \end{enumerate}
    \end{itemize}
    \begin{argument}
      \requ{}, so \curb{} from perspective.
      So, then via \feedback{0}, get \curb{}.
    \end{argument}
  \end{proposition}
\end{note}

\begin{note}
  Consequence.

  \begin{proposition}
    If \requ{} and \ros{} fails to hold from \agpe{}, then not concluding.
    \begin{argument}
      Suppose \ros{} fails to hold.

      Therefore, not a \fc{}.

      If not a \fc{}, then not concluding.
    \end{argument}
  \end{proposition}

  So:

  \begin{proposition}
    If \requ{}, then if concluding, \ros{} holds.

    \begin{argument}
      Rewriting.
    \end{argument}
  \end{proposition}
\end{note}


\section{\curb{3} as checks on concluding}
\label{cha:zS:sec:curbs:checks}


\begin{note}[\autoref{illu:sketch:prop-logic}]
  Likewise, \autoref{illu:sketch:prop-logic} involves an agent concluding some sentence \(A\) is a syntactic theorem of propositional logic via a formula derivation.

  And, when concluding \(A\) is a syntactic theorem, the agent observes that \(A\) is a syntactic theorem only if \(A\) is also a semantic theorem (from soundness).

  In other words, if the agent attempt to show \(A\) is true under an arbitrary valuation and failed, the agent would not conclude \(A\) is a syntactic theorem.

  Only if semantic proof.
  Syntactic proofs, at least in my experience, may be out of reach.
  However, semantic proofs, often straightforward.
\end{note}

\begin{note}
  Here, highlight initial scenario.
  For, there, testimony.
  Overrides agent's own reasoning, plausibly.
\end{note}

\section{\requ{3} and undercutting defeaters}

\begin{note}
  Shares similarity.
  Focus on concluding.
  However, subjunctive.
  \requ{3} concern entertaining proposition-value-premises pairings as an undercutting defeater for concluding.
\end{note}

\begin{note}
  Not about the proposition-value pair.
  Rather, it is about the concluding.
  At interest is not whether \(\phi\) has value \(v\), but whether it makes sense to conclude \(\pv{\phi}{v}\) from \(\Phi\).
  Of course, if the agent has no information about whether \(\phi\) has value \(v\), then this is also part of the picture, but that is a consequence of the base concern.

  With squish.
  I wouldn't conclude.
  However, wouldn't say that entailment doesn't hold.
  For, may only be the case that the derived rule of inference fails to hold.
\end{note}

\begin{note}
  In this respect, failing to conclude \(\pv{\psi}{v'}\) from \(\Phi\) may be described as an undercutting defeater with respect to conclude \(\pv{\phi}{v}\) from \(\Phi\).%
  \footnote{
    We borrow the following sketch from \textcite{Worsnip:2018aa}:
  \begin{quote}
    Undercutting defeaters, which are easiest to think of in the context of the attitude of belief, are supposed to be considerations that undermine the justification of a belief in a proposition p not necessarily by providing (sufficient) positive evidence to think that p is false, but rather merely by suggesting (perhaps misleadingly) that one’s reasons for believing p are no good, in a way that neutralizes or mitigates their justificatory or evidential force.%
    \mbox{}\hfill\mbox{(\citeyear[29]{Worsnip:2018aa})}
  \end{quote}
  }

  Consider the following illustration provided by \citeauthor{Pollock:1987un}:
  \begin{quote}
    [Undercutting defeaters] attack the connection between the reason and the conclusion rather than attacking the conclusion itself.
    For instance, ``X looks red to me'' is a prima facie reason for me to believe that X is red.
    Suppose I discover that X is illuminated by red lights and illumination by red lights often makes things look red when they are not.
    This is a defeater, but it is not a reason for denying that X is red (red things look red in red light too).
    Instead, this is a reason for denying that X wouldn't look red to me unless it were red.%
    \mbox{}\hfill\mbox{(\citeyear[485]{Pollock:1987un})}
  \end{quote}
  Completing \citeauthor{Pollock:1987un}'s example, it seems that if agent's support for holding that X is red is that `X wouldn't look red to me unless it were red', then the support for X being red provided by appearance is retracted after discovering that X is illuminated by red lights (though it remains possible that X is red).
\end{note}

\begin{note}
  In \citeauthor{Pollock:1987un}'s example, discover that the light is red.
  By parallel, the agent failing to conclude would undercut.
\end{note}

\begin{note}
  Similarity.

  However, given the focus on concluding, \curb{} is not a simple instance of an undercutting defeater.
  At least, given \citeauthor{Pollock:1987un}'s definition of an undercutting defeater.

  \citeauthor{Pollock:1987un} defines undercutting defeaters as follows:
  \begin{quote}
    R is an \emph{undercutting defeater} for P as a prima facie reason for S to believe Q if and only if
    \begin{enumerate}[label=(UD\arabic*), ref=(UD\arabic*)]
    \item
      \label{pollock:ud:1}
      P is a reason for S to believe Q and R is logically consistent with P but (P and R) is not a reason for S to believe Q, and
    \item
      \label{pollock:ud:2}
      R is a reason for denying that P wouldn't be true unless Q were true.%
      \mbox{}\hfill\mbox{(\citeyear[485]{Pollock:1987un})}
    \end{enumerate}
  \end{quote}
  Intuitively, an undercutting defeater for P as a reason for Q because it the defeater denies that Q must be true in order for P to be true.%
  \footnote{
    \citeauthor{Wright:2011wn}'s (\citeyear{Wright:2011wn}) revised template:
    \begin{quote}
      Where A entails B, a rational claim to warrant for A is not transmissible to B if there is some proposition C such that:
      \begin{enumerate}[label=(\roman*), noitemsep]
      \item
        The process/state of accomplishing the relevant putative warrant for A is subjectively compatible with C’s holding: things could be with one in all respects exactly as they subjectively are yet C be true
      \item
        C is incompatible (not necessarily with A but) with some presupposition of the cognitive project of obtaining a warrant for A in the relevant fashion, and
      \item
        Not-B entails C%
      \mbox{ }\hfill\mbox{(\citeyear[93]{Wright:2011wn})}
      \end{enumerate}
    \end{quote}
    Difficulty with the process, however, of interest is transmission of warrant from A to B.
    Hence, the agent has accomplished the relevant putative warrant for A.

    Or, consider the fourth type of dependence between premise and conclusion considered (but not endorsed) by \textcite{Pryor:2004ws}:

  \begin{quote}
    [Type 4] dependence between premise and conclusion is that the conclusion be such that evidence \emph{against it} would (to at least some degree) undermine the kind of justification you purport to have for the premises.%
    \mbox{}\hfill\mbox{(\citeyear[359]{Pryor:2004ws})}
  \end{quote}
  Premises!

  \nocite{Weisberg:2012vs}
  Closer to \citeauthor{Weisberg:2010to}'s (\citeyear{Weisberg:2010to}) account of bootstrapping.
  However, implicit circularity.
  And, circularity is not at issue.
  }
\end{note}

\begin{note}
  Difficulty with a clear parallel


  \ref{pollock:ud:1}.
  For, P \emph{is a} reason.

  In the case of \curb{1}, failure to be concluding.
  For, event may develop such that agent does not conclude.

  \ref{pollock:ud:1}.

  Some other way for P to be true.
\end{note}

\begin{note}
  However, ties things too closely to presentation, rather than substance of ideas.
  Could be said that agent is concluding, but there is an additional way in which the event may develop.

  Variation:
  \begin{quote}
    R is \emph{undercutting reasoning} with respect to \emph{S} concluding \(\pv{\phi}{v}\) from \(\Phi\) if and only if
    \begin{enumerate}[label=(UR\arabic*), ref=(UR\arabic*)]
    \item
      The event may develop such that \emph{S} to concludes \(\pv{\phi}{v}\) from \(\Phi\) and R is a \pevent{} which is logically consistent with reasoning from \(\Phi\) but combination of both reasoning from \(\Phi\) and R is not a sufficient for \emph{S} to be concluding \(\pv{\phi}{v}\), and
    \item
      R is sufficient for denying that S wouldn't conclude \(\pv{\phi}{v}\) from \(\Phi\) unless \(\pv{\phi}{v}\) followed from \(\Phi\).%
    \end{enumerate}
    Where, R is reasoning which fails to conclude \(\pv{\psi}{v'}\) from \(\Psi\).
  \end{quote}

  {
    \color{red}
    So, the emphasis here is on whether there's really anything to be said for the conclusion.
    However, if this is the case, then it seems the agent simply doesn't conclude.
  }

  The problem here is with the first clause.
  Logically consistent.

  It is not clear that the reasoning is logically consistent.
  Two instances of reasoning may involve intermediary steps which are logically inconsistent.

  Though, on the other hand, in interesting cases it is logically consistent for the agent to reason to \(\pv{\phi}{v}\) from \(\Phi\) and fail to conclude \(\pv{\psi}{v'}\) from \(\Psi\).
  For, if not logically possible, then no worries about failing to conclude \(\pv{\psi}{v'}\) from \(\Psi\).

  Yet, if problem, then from the \agpe{}, instances of reasoning aren't consistent.

  Puzzle here is how logical consistency is understood with respect to \citeauthor{Pollock:1987un}'s definition.
  Independent of the \agpe{}, or from the \agpe{}?

  Both interpretations are compatible with the example.
  But, principle also holds independently of the \agpe{} with respect to the example.
\end{note}

\begin{note}
  In contrast, \curb{} `attacks' the \emph{reasoning} and the conclusion.
\end{note}

%%% Local Variables:
%%% mode: latex
%%% TeX-master: "master"
%%% End:
