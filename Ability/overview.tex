\chapter{Overview}
% \addcontentsline{toc}{chapter}{Overview}

\begin{note}
  This chapter provides a brief overview of the argument which follows.
\end{note}

\section{The road ahead}
\label{sec:road-ahead}

\begin{note}
  \autoref{cha:intro} introduced the way we understand \eiw{1} an agent concludes, questions \qWhy{} and \qHow{}, and the constraint \issueInclusion{}.
  We sketched some motivation for \issueInclusion{} and then the general form counterexamples to \issueInclusion{} take.
\end{note}

\begin{note}
  The remainder of this document amounts to working through a handful of ideas fairly carefully.
  Important ideas are applied to \autoref{illu:gist:roots:F} as worked through.
  So, we identify a counterexample to \issueInclusion{} throughout the document.
\end{note}


\begin{note}
  \autoref{cha:events-progress} expands on \autoref{idea:why} to state sufficient conditions for answers to \qWhy{}.

  Specifically, we split \autoref{idea:why} into two ideas:
  An \se{} event and \progEx{}.

  To help keep things simple we have talked about \eiw{1} an agent concludes.
  However, when turning to the details, it is useful to distinguish between events and (true) descriptions of events.

  With a distinction between events and descriptions in hand, \(\edn{\flat}\) under description \(\edo{\flat}\) is an \emph{\se{}} of \(\edn{}\) under description \(\edo{}\) just in case:

  \begin{enumerate}[label=\alph*., ref=\alph*]
  \item
    \label{sketch:se:r}
    \(\edn{}\) under \(\edo{}\) is a result of \(\edn{\flat}\) under \(\edo{\flat}\).
  \item
    \label{sketch:se:p}
    \(\edn{\flat}\) under \(\edo{\flat}\) is such that \(\edn{}\) under \(\edo{}\) is in progress.
  \end{enumerate}
  %
  \ref{sketch:se:r} follows Clause \ref{idea:why:result} of \autoref{idea:why}, and (I argue) \ref{sketch:se:p} entails clause \ref{idea:why:favour}.%
  \footnote{
    I.e., \(\edn{\flat}\) under \(\edo{\flat}\) favours \(\edn{}\) under \(\edo{}\) due to it being the case \(\edn{}\) is in progress.
  }

  In turn, \progEx{} expands on Clause \ref{idea:why:feat} of \autoref{idea:why}.
  Roughly, \progEx{} states that whatever is captured of \(\edn{\flat}\) by description \(\edo{\flat}\) is a (partial) explanation of `why' an agent concluded \(\pv{\phi}{v}\) from \(\Phi\).
\end{note}

\begin{note}
  The remainder of the document amounts to arguing certain descriptions capture \fingfr{1}, and some such \fingfr{1} are incompatible with \issueInclusion{}.
\end{note}

\begin{note}
  Chapters~\ref{cha:fcs}, \ref{cha:ros}, and \autoref{cha:requs} establish the way a description of an event captures a \fingfr{}.
  Following, \autoref{cha:requs} outlines a way to obtain \fingfr{1} which may amount to counterexamples to \issueInclusion{}, and \autoref{cha:ces} provides a pair of detailed counterexamples.
\end{note}

\begin{note}
  In a little more detail, chapters~\ref{cha:fcs} and \ref{cha:ros} focus on identifying \fingfr{}.

  Specifically, \autoref{cha:fcs} introduces the idea of \(\pv{\psi}{v'}\) being a \fc{} from \(\Psi\) for an agent.
  Where, roughly, \(\pv{\psi}{v'}\) is a \fc{0} from \(\Psi\) just in case the agent may do some action and be concluding \(\pv{\psi}{v'}\) from \(\Psi\).
  For example, \pv{\rootsCon{}}{\valI{True}} was plausibly a \fc{} from a \pool{} which included the \agents{} understanding of factorisation \emph{prior to} \autoref{illu:gist:roots:F}, as the agent plausibly was concluding \pv{\rootsCon{}}{\valI{True}} from the \pool{} when the agent begins their reasoning.

  In turn, \autoref{cha:ros} states:
  %
  \begin{itemize}
  \item
    A conclusion of \(\pv{\phi}{v}\) from \(\Phi\) is sufficient for \(\pv{\phi}{v}\) to \fof{} \(\Phi\) from the \agpe{} when the agent concludes \(\pv{\phi}{v}\) from \(\Phi\).
  \item
    \(\pv{\psi}{v'}\) being a \fc{} from \(\Psi\) is sufficient for \(\pv{\phi}{v}\) to \fof{} \(\Phi\) from the \agpe{}.
  \end{itemize}
  %
  So, if a description \(\edo{\flat}\) of an event \(\edn{\flat}\) entails a \fc{0}, \(\edo{\flat}\) also entails a corresponding \fingfr{}.
  And, if \(\edo{\flat}\) captures a \se{} of an \eiw{0} an agent concludes, the \fingfr{} answers \qWhy{}.

  A pair of quick notes may be helpful here:

  \begin{itemize}
  \item
    I expect you to have a somewhat intuitive grasp on the `\fof{}' relation and \fc{1} hopefully identify some instances compatible with your intuitive grasp.
  \item
    When we speak of \fc{1} our interest is with whatever it is that makes it the case \(\pv{\psi}{v'}\) is a \fc{} from \(\Psi\), rather than the (foregone-)conclusion of \(\pv{\phi}{v}\) from \(\Phi\).
    That is, whatever it is that makes it the case \(\pv{\psi}{v'}\) is a \fc{} from \(\Psi\) amounts to something which makes it the case that \(\pv{\psi}{v'}\) `follows from' \(\Psi\), from the \agpe{}.
  \end{itemize}
\end{note}


\begin{note}
  \autoref{cha:requs} details the way \fc{1}, \fingfr{1}, and \progEx{} interact to provide sufficient conditions for an answer to \qWhy{}.

  Here, we identify a few answers to \qWhy{} which are compatible with \issueInclusion{}.

  So, taken as a unit, chapters \autoref{cha:intro} to \autoref{cha:requs} detail a way to find answers to \qWhy{}, and the answers we find are compatible with \issueInclusion{}.
\end{note}


\begin{note}
  \autoref{cha:typical} introduces the idea an agent \tCV{} to help identify answers to \qWhy{} which are not compatible with \issueInclusion{}.

  The basic idea of an agent \tCV{} \(\pv{\phi}{v}\) from \(\Phi\) is that there is some generality to the \agents{} reasoning.
  As noted above, with respect to \autoref{illu:gist:roots:F}, \pv{\propM{\rootsConEqExV{12}{4}{3}}}{\valI{True}} plausibly follows from a sufficient understanding of factorisation.
  In turn, the agent may do some action any be concluding \pv{\propM{\rootsConEqExV{12}{4}{3}}}{\valI{True}} from a \pool{} which captures their understanding of factorisation.
  By the ideas of chapters~\ref{cha:fcs} and \ref{cha:ros} this identifies a \fingfr{}, and so long as the agent has not concluded \pv{\propM{\rootsConEqExV{12}{4}{3}}}{\valI{True}} from the relevant \pool{}, \issueInclusion{} fails to hold.
\end{note}


\begin{note}
  Finally, \autoref{cha:ces} details counterexamples to \issueInclusion{}.
  In particular, we highlight in detail the way ideas as applied to \autoref{illu:gist:roots:F} show \issueInclusion{} fails to hold.
  And, with some luck you, yourself, create a counterexample to \issueInclusion{}.
\end{note}

\begin{note}
  The argument is more about the way a handful of ideas interact and apply to various \scen{1} rather than any single idea.

  To help separate the main line of argument about the interaction between various ideas from detailed arguments and applications of ideas the argument is built from `definitions', `ideas', `assumptions', and `propositions'.

  Definitions, ideas, and assumptions are stated and motivation follows in an unstructured way, while propositions are always accompanied by a corresponding argument.

  In addition, `applications' and `observations' are included.
  Applications are used to highlight the way a particular definition or idea applies to a \scen{} and a handful of applications are used to highlight a failure of \issueInclusion{} (with respect to \autoref{illu:gist:roots:F}) while observations amount to notes which may be helpful but can be safely ignored.
  Like propositions, applications and observations come with marked motivation.

  You are encouraged to read the statement of a proposition, application, or observation and then move on if you'd prefer not to work through the details.

  Also, sometimes sections are marked as `optional'.
  Nothing in a section marked as optional is involved in the main line of argument.
\end{note}

\begin{note}
  Though we only provide a few explicit counterexamples to \issueInclusion{}, these counterexamples are obtained by applying general ideas developed to specific \scen{1}.
  So, by the close of this document I hope to have given you the resources to find other failures of \issueInclusion{}.
\end{note}



\paragraph*{A note on reading this document}

\begin{note}
  The argument of this document is fairly tightly connected.
  Ideas introduced and definitions made in earlier chapters are often used to introduce further ideas or make additional definitions in later chapters.
  And, propositions argued for often build on prior propositions.

  If you are reading this as a PDF, references to ideas stated in this document are hyperlinked.
  For example, clicking on --- \autoref{illu:gist:roots:F} --- takes you to \autoref{illu:gist:roots:F}, and clicking on --- \qWhy{} --- takes you to the statement of \qWhy{}.

  I have also tried to include some page references where I think they may be helpful.

  In addition, \autoref{cha:re} collects together a few recurring definitions, ideas, and propositions.
  So, if you have a print-out I recommend setting \autoref{cha:re} to one side for easy reference.
  Or, if you have a PDF, opening a second copy of this document in a separate window to \autoref{cha:re} may be helpful.
\end{note}



%%% Local Variables:
%%% mode: latex
%%% TeX-master: "master"
%%% TeX-engine: luatex
%%% End:
