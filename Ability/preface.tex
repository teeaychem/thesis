\chapter{Preface}
\label{cha:preface}

% \section{History}

% \begin{note}
%   Started with idea about premises.

%   Distinction between representation and content of representation.

%   Fix something like a representation, but no content.

%   One way to think about this, inchoate.
% \end{note}

% \begin{note}
%   Key idea is something missing.
%   Whatever relation typically holds between perspective of agent and action, weaken this.
% \end{note}

% \begin{note}
%   Idea is hard to argue for.
% \end{note}

% \begin{note}
%   Different approach.
%   Relation holds in cases where agent has not done the think which typically characterises the relation.
% \end{note}

\section{Starting}

\begin{note}
  Consider instance of ability.

  I have the ability to prove X.
  So, X is true.

  This seems problematic.
  For, suppose X is not true.
  Then, do not have the ability to prove X.
  So, premise is true only if conclusion is true.
  Further, premises is true only if conclusion follows from some other set of available premises.
  This is what ability captures, reason from some P to X.
  \support{2} that have not witnessed.
  Indeed, if and only if.
  Seems ability may reduce to unwitnessed \support{}.

  Is this right?
  Are there cases in which reasoning that hasn't been done matters?
  This is where got stuck.
  Plausible that the answer is no.
  Premises conclusion, that's all that matters.
  Like any deductive argument.
  Break the if and only if.
  Intuitively seems to be the case.

  However, I don't think this is right.
  There are cases.

  Difficulty is showing this is the case.
  Grant premises to avoid.
  Broad ability, so instances, so entailment.
  There is no strict dependence.
  What matters isn't support via ability, but premise of general ability.

  Note, extends to broad ability.
  Prove X as instance of broad ability.
  However, if this is deductive, then fail to prove X shows don't have broad ability.

  Now, broad ability is the worry.
  For, instance of broad ability.
  Hence, this I have reasoned to, depend in part or things I have not reasoned to.

  Generalising, idea is \support{} holds between premises and conclusion, matters to why, even though the agent hasn't concluded.

\end{note}

% \begin{note}
% \nocite{Brown:2004us}
%   \color{red}
%   There's no clear link to \citeauthor{Wright:2011wn}.

%   For, with the cases I'm interested in, the problem arises because the agent has the \abgen{} ability.
%   However, to apply transmission failure, it seems this would raise worries about ability.
%   So, for me:
%   I have \abgen{} ability, and if not an instance of \abgen{} then a problem, but is an instance so get conclusion.
% \end{note}

% \begin{note}
%   More general argument.
%   Cases in which reasoning applies equally to premise-conclusion pairing have no witnessed from.

%   This is the main focus of this document.
% \end{note}


%%% Local Variables:
%%% mode: latex
%%% TeX-master: "master"
%%% End:
