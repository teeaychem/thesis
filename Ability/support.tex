\chapter{\ros{3}}
\label{cha:ros}


\begin{note}
  \ros{3} captures the way in which a \prop{0}-\val{0} pair \(\pv{\phi}{v}\) `follows from' some \pool{} \(\Phi\), from an \agpe{}.
  Where the `following from' relation contrasts with `from' in the sense that  an agent concludes \(\pv{\phi}{v}\) `from' \(\Phi\).

  \begin{itemize}
  \item
    \autoref{scen:calc}.

    \ros{3} between \propM{\gistCalcEq{}} and \pool{1} = calculator.

    Also, \ros{3} between \propM{\gistCalcEq{}} and \pool{1} = understanding of arithmetic.

  \item
    \autoref{scen:animalism}. \ros{2} between \pv{\propI{Four legs good, two legs bad}}{\valI{True}} and \pool{} existence of an explanation by Snowball.

    However, no \ros{} between \pv{\propI{Four legs good, two legs bad}}{\valI{True}} and \pool{} the content of Snowball's explanation. For the birds did not understand.
  \end{itemize}
\end{note}


\begin{note}
  Broad idea.

  In turn, \ros{3} are answers to \qWhy{}, and if \issueConstraint{} holds, for any \ros{} which answers \qWhy{} there is some event where agent concludes.
\end{note}


\begin{note}
  Specifying \ros{} in any significant detail is beyond the scope of this document.
  \issueInclusion{} is understood as a (plausible) constraint on theories about the why and how of reasoning rather than the details of any given theory.
  Instead, we characterise \ros{} by two ideas, and anything which satisfies these two ideas in the details of a given theory may be understood as a \ros{}.
\end{note}


% \begin{note}
%   Subsections develop and discuss each in detail:

%   \begin{TOCEnum}
%   \item
%     \TOCLine{cha:ros:I}

%     The first idea, roughly:
%     If an agent concludes \(\pv{\phi}{v}\) from \(\Phi\) then a \ros{} between \(\pv{\phi}{v}\) and \(\Phi\) holds from \agpe{the \agents{}}.
%   \item
%     \TOCLine{cha:ros:W}

%     A definition for ease of expression:
%     An event in which an agent concludes \(\pv{\phi}{v}\) from \(\Phi\) provides an agent with a \wit{0} for a \ros{}.
%   \item
%     \TOCLine{cha:ros:II}

%     The second idea, roughly:
%     If an agent \(\pv{\phi}{v}\) is a \fc{} from \(\Phi\) then a \ros{} between \(\pv{\phi}{v}\) and \(\Phi\) holds from \agpe{the \agents{}}.
%   \end{TOCEnum}
% \end{note}


\begin{note}
  From a broad perspective, \supportI{} gives a sufficient condition, while \supportII{} denies the sufficient condition amounts to a necessary condition.

  \supportI{}, `follows from' of interest is present when agent concludes `from'.
  \supportII{} denies `follows from' is equivalent to `from' by possibility of `follows from' without corresponding `from'.
\end{note}



\section{\supportI{}}
\label{cha:ros:I}


\begin{note}
  The role of \supportI{} is to characterise when a \ros{} holds between some \prop{0}-\val{0} pair and \pool{} from an \agpe{} \emph{given} an event in which the agent concludes the \prop{0}-\val{0} pair from the \pool{}.
  The characterisation is straightforward:

  \begin{idea}[\supportI{}]
    \label{idea:support}
    \vspace{-\baselineskip}
    \begin{itenum}
    \item[\emph{If}:]
      \(e_{d}\) is an event in which \vAgent{} concludes \(\pv{\phi}{v}\) from \(\Phi\).
    \item[\emph{Then}:]
      When \vAgent{} \eval{1} \(\phi\) as having value \(v\) as a \se{0} of \(e_{d}\):
      \begin{itemize}
      \item
        A \emph{\ros{}} between \(\pv{\phi}{v}\) and \(\Phi\) holds, from \agpe{\vAgent{}'}.
      \end{itemize}
    \end{itenum}
    \vspace{-\baselineskip}
  \end{idea}

  \noindent%
  In short, \supportI{} states that a \ros{} between \(\pv{\phi}{v}\) and \(\Phi\) holds form an \agpe{} when the agent concludes \(\pv{\phi}{v}\) from \(\Phi\).

  The motivation for \supportI{} is straightforward:
  The role of \ros{0} a \ros{} between \(\pv{\phi}{v}\) and \(\Phi\) is to abstractly capture the way in which \(\pv{\phi}{v}\) `follows from' \(\Phi\) from the \agpe{}.
  And, if an agent concludes \(\pv{\phi}{v}\) from \(\Phi\), then \(\pv{\phi}{v}\) `follows from' \(\Phi\) from the \agpe{}.

  In other words, the \ros{} between \(\pv{\phi}{v}\) and \(\Phi\) just captures whatever it is, for the agent, that led to the agent concluding \(\pv{\phi}{v}\) from \(\Phi\).
\end{note}


\begin{note}
  Sub-event.
  For, by \autoref{assu:ConRea} the event in which an agent concludes \(\pv{\phi}{v}\) from \(\Phi\) spans the agent's reasoning to \(\pv{\phi}{v}\) from \(\Phi\).
  And, a \ros{} may only hold as the result of reasoning.
\end{note}


\begin{note}
  Our characterisation of \ros{} by \supportI{} rests on the way in which we understand conclusions.
  In particular, \supportI{} allows for \ros{} to hold between any \prop{0}-\val{0}-\pool{0} pair, so long as the agent concludes the relevant \prop{0} has the specified \val{0} from the given \pool{}, or the \prop{0}-\val{0} pair is a \fc{} from the \pool{}.
  Hence, a \ros{} may hold between, e.g., \pv{\propI{Fish are mammals}}{\valI{True}} and so \pool{} from the \agpe{}.

  If the way in which conclusions are understood is restricted, the \ros{} as characterised are likewise restricted.
  And, that \ros{1} hold between certain \prop{0}-\val{0} pairs and \pool{1} is important for the argument to follow, not that \ros{1} may hold between arbitrary \prop{0}-\val{0} pairs and \pool{1}.
  The benefit of the characterisation given is simplicity.
  For example, I expect the argument to follow is compatible with restricting \ros{} to justified conclusions.
\end{note}



\paragraph*{\ros{3} and \citeauthor{Boghossian:2014aa} Taking Condition \hfill (Optional)}


\begin{note}
  \supportI{} is similar to, but distinct from,~\citeauthor{Boghossian:2014aa}'s Taking Condition:%
  \footnote{
    Strictly,~\citeauthor{Boghossian:2014aa} states the Taking Condition in terms of inferring.
    I.e., the Taking Condition reads: `Inferring necessarily involves the thinker \emph{taking} \dots'
    However, as \citeauthor{Boghossian:2008vf} is interested in conclusions, an event in which an agent infers (in~\citeauthor{Boghossian:2014aa} sense) is an event in which an agent concludes (in our sense).
  }

  \begin{quote}
    (Taking Condition):
    [An event in which an agent concludes] necessarily involves the thinker \emph{taking} his premises to support his conclusion and drawing his conclusion because of that fact.%
    \mbox{}\hfill\mbox{(\citeyear[5]{Boghossian:2014aa})}
  \end{quote}

  \noindent%
  For, `taking' is understood by \citeauthor{Boghossian:2014aa} to be component of the agent's reasoning.
  \citeauthor{Boghossian:2014aa} illustrates the Taking Condition as follows:
  % 
  \begin{quote}
    On waking up one morning I recall that:

    \begin{enumerate}[label=(\arabic*), ref=(\arabic*), series=BogEx]
    \item
      \label{BogEx:1}
      It rained last night.
    \end{enumerate}

    I combine this with my knowledge that

    \begin{enumerate}[label=(\arabic*), ref=(\arabic*), resume*=BogEx]
    \item
      \label{BogEx:2}
      If it rained last night, then the streets are wet.
    \end{enumerate}

    to conclude:

    So,

    \begin{enumerate}[label=(\arabic*), ref=(\arabic*), resume*=BogEx]
    \item
      \label{BogEx:3}
      The streets are wet.
    \end{enumerate}
    This belief then affects my choice of footwear.%

    [\dots M]y inferring from~\ref{BogEx:1} and~\ref{BogEx:2} to~\ref{BogEx:3} must involve my arriving at the judgment that~\ref{BogEx:3} in part \emph{because} I \emph{take} the presumed truth of~\ref{BogEx:1} and~\ref{BogEx:2} to provide support for~\ref{BogEx:3}.%
    \mbox{ }\hfill\mbox{(\citeyear[2,4]{Boghossian:2014aa})}
  \end{quote}
  % 
  Hence, for \citeauthor{Boghossian:2014aa}, the Taking Condition captures something \emph{in addition} to~\ref{BogEx:3} being a conclusion from a \pool{} which includes~\ref{BogEx:1} and~\ref{BogEx:2}.

  In contrast, we do not require that a \ros{} has any particular role \emph{for the agent} in event in which an agent concludes \(\pv{\phi}{v}\) from \(\Phi\).
  If an agent concludes~\ref{BogEx:3} from~\ref{BogEx:1} and~\ref{BogEx:2}, then a \ros{} holds between~\ref{BogEx:3} and \(\{\ref{BogEx:1}, \ref{BogEx:2}\}\).
  However, the \ros{} between~\ref{BogEx:3} and \(\{\ref{BogEx:1}, \ref{BogEx:2}\}\) need not itself have a role in the \agents{} conclusion of~\ref{BogEx:3} from \(\{\ref{BogEx:1}, \ref{BogEx:2}\}\).%
  \footnote{
    Also, about the type of reasoning by which the agent concludes.
    This comes from \textcite{Boghossian:2008vf,Boghossian:2012vb}.
    Rule following, taking gets account of rule.
  }

  To illustrate, consider \citeauthor{Wright:2014tt}'s (\citeyear{Wright:2014tt}) `Simple Proposal':
  %
  \begin{quote}
    % [C]onsider instead the proposal, not that the status of the transition as inferential depends on the thinker's judgments about his reasons, but that it depends on \emph{what his reasons are}.
    % We want his acceptance of the premises to supply his \emph{actual} reasons for accepting the conclusion.
    % [\dots]
    %
    % Call this the Simple Proposal.
    % It says that a thinker infers q from p\(_{1}\) \(\cdots\) p\(_{\text{n}}\) when he accepts each of p\(_{1}\) \(\cdots\) p\(_{\text{n}}\), moves to accept q, and does so for the reason that he accepts p\(_{1}\) \(\cdots\) p\(_{\text{n}}\).%
    [The Simple Proposal] says that a thinker infers q from p\(_{1}\) \(\cdots\) p\(_{\text{n}}\) when he accepts each of p\(_{1}\) \(\cdots\) p\(_{\text{n}}\), moves to accept q, and does so for the reason that he accepts p\(_{1}\) \(\cdots\) p\(_{\text{n}}\).%
    \mbox{}\hfill\mbox{(\citeyear[33]{Wright:2014tt})}
  \end{quote}
  %
  \citeauthor{Wright:2014tt}'s proposal is that the relation between a conclusion and some \pool{} need not be part of what moves the agent to conclude the conclusion from the \pool{}.
  Hence, \citeauthor{Wright:2014tt} denies that reasoning must involve a state which connects premises to conclusions.
  So, \citeauthor{Wright:2014tt} denies \citeauthor{Boghossian:2008vf}'s Taking Condition on inference.
  (\citeyear[Cf.][33-34]{Wright:2014tt})

  \supportI{} is compatible with \citeauthor{Wright:2014tt}'s Simple Proposal as \supportI{} only entails a \ros{} holds between \(\pv{\propM{q}}{\valI{True}}\) and a \pool{} containing \(\pv{\propM{p_{1}}}{\valI{True}}\), \(\cdots\) \(\pv{\propM{p_{n}}}{\valI{True}}\) when the agent concludes.%
  \footnote{
    Still, there is an important between~\supportI{} and \citeauthor{Wright:2014tt}'s Simple Proposal.
    For,~\supportI{} is an entailment, while \citeauthor{Wright:2014tt}'s Simple Proposal is an identity statement.
    Inferring, on the Simple Proposal, is an agent accepting some conclusion for the reason that they accept premises from some \pool{}.
    \supportI{} does not entail that concluding is nothing more than moving to accept \(\pv{\phi}{v}\) as a result of accepting each element of \(\Phi\).
  }\(^{,}\)%
  \footnote{
    There are various other objections to~\citeauthor{Boghossian:2014aa}'s Taking Condition.

    For example,~\citeauthor{Hlobil:2014tq} argues against the Taking Condition as it distracts from what accounts of reasoning out to explain, rather than arguing against the Taking Condition directly.
    Likewise, \citeauthor{McHugh:2016vp} present and summarise various objections to \emph{interest} with the Taking Condition.

    In particular,~\supportI{} is closer to what \citeauthor{McHugh:2016vp} term the `Consequence Condition': \textquote{Inferring q from p entails taking p to support q}.
    (\citeyear[316]{McHugh:2016vp})
    And, as \citeauthor{McHugh:2016vp} observe, the condition is \textquote{consistent with the idea that in inference we take our premises to support our conclusion just in virtue of reasoning from the former to the latter}.
    (\citeyear[316]{McHugh:2016vp})

    \citeauthor{McHugh:2016vp} suggest the arguments they consider against the Taking Condition `put pressure' on the Consequence Condition (\citeyear[327]{McHugh:2016vp}).
    However, these arguments concern interest, rather than whether condition is true.
    And, we, uh, have interest in \ros{1}\dots
  }
\end{note}

% \begin{note}
%   \color{red}
%   Humpty Dumpty arbitrary chooses for `glory' to mean `a nice knock-down argument' (\cite[190]{Carroll:2009aa}).
% \end{note}



\section{\wit{3} for \ros{1}}
\label{cha:ros:W}


\begin{note}
  With \supportI{}, and hence some characterisation of \ros{}, in hand we define a `\wit{0}' for \ros{} between \(\pv{\phi}{v}\) and \(\Phi\).
  In full:

  \begin{definition}[A \wit{0} for a \ros{1}]%
    \label{def:witnessing}%
    \vspace{-\baselineskip}
    \begin{itemize}
    \item
      An event \(e^{-}_{d^{-}}\) is \emph{\wit{0}} for a \ros{} between \(\pv{\phi}{v}\) and \(\Phi\), for \vAgent{} through event \(e_{d}\)
    \end{itemize}

    \emph{If and only if:}

    \begin{itemize}
    \item
      \(e^{-}_{d^{-}}\) is an event in which \vAgent{} concludes \(\pv{\phi}{v}\) from \(\Phi\).
    \item
      \(e^{-}_{d^{-}}\) occurs prior to or at the same time as \(e_{d}\).
    \end{itemize}
    \vspace{-\baselineskip}
  \end{definition}

  \noindent%
  In short, a \wit{} for a \ros{} between \(\pv{\phi}{v}\) and \(\Phi\) is some extant event in which the agent concludes \(\pv{\phi}{v}\) from \(\Phi\).

  And, if an agent has concluded \(\pv{\phi}{v}\) from \(\Phi\), we say the agent `has a \wit{}' for the \ros{} between \(\pv{\phi}{v}\) and \(\Phi\).
\end{note}


\begin{note}
  An important, but trivial, case of \autoref{def:witnessing} is when an agent concludes \(\pv{\phi}{v}\) from \(\Phi\).
  For, if an agent concludes \(\pv{\phi}{v}\) from \(\Phi\) then it is immediate that there is some event in which the agent concludes \(\pv{\phi}{v}\) from \(\Phi\) --- the very same event --- and hence the agent has a \wit{} for the \ros{} between \(\pv{\phi}{v}\) and \(\Phi\).
\end{note}


\section{\supportII{}}
\label{cha:ros:II}


\begin{note}
  With \supportI{} and the definition of a \pwitP{} in hand, we turn to \supportII{}.

  The role of \supportII{} is to characterise when a \ros{} holds between some \prop{0}-\val{0} pair and \pool{} from an \agpe{} \emph{without a} given an event in which the agent concludes the \prop{0}-\val{0} pair from the \pool{}.

  To do so, we use the idea of a \prop{0}-\val{0} pair being a \fc{} from the relevant \pool{}:

  \begin{idea}[\supportII{}]%
    \label{idea:support:possible}%
    \vspace{-\baselineskip}
    \begin{itenum}
    \item[\emph{If}:]
      \(\pv{\phi}{v}\) is a \fc{0} from \(\Phi\) for \vAgent{} throughout \(e_{d}\).
    \item[\emph{Then}:]
      A \ros{} between \(\pv{\phi}{v}\) and \(\Phi\) holds from \agpe{\vAgent{}'} throughout \(e_{d}\).
    \end{itenum}
    \vspace{-\baselineskip}
  \end{idea}

  \noindent%
  A \ros{} is designed to abstractly capture the way in which some \prop{0}-\val{0} pair `follows from' some \pool{} from an \agpe{}.
  Hence, \supportI{} characterised \ros{1} via events in which an agent concludes.
  \supportII{}, in short, states the agent does not need to conclude in order for a \ros{} to hold, so long as the possibility of an event in which the agent concludes is already secured.
\end{note}


\begin{note}
  \noindent%
  Expanded with the definition of a \fc{}, \supportII{} reads:
  \begin{itenum}
  \item[\emph{If}:]
    \begin{itemize}
    \item
      Throughout \(e_{d}\) there is some action \(a\) \vAgent{} may immediately perform such that both \ref{def:fc:act} and \ref{def:fc:result} are true:
      \begin{enumerate}[label=\alph*., ref=(\alph*)]
      \item
        For each \prop{0}-\val{0} pair \(\pv{\phi'}{v'}\) in \(\Phi\), \vAgent{} \evals{} \(\phi'\) as having value \(v'\) prior to doing \(a\).
      \item
        The event \(e^{\sharp}_{d^{\sharp}}\) in which \vAgent{} does \(a\) is an event in which \vAgent{} is concluding \(\pv{\phi}{v}\) from \(\Phi\).
      \end{enumerate}
    \end{itemize}
  \item[\emph{Then}:]
    A \ros{} between \(\pv{\phi}{v}\) and \(\Phi\) holds, from \agpe{\vAgent{}'} throughout \(e_{d}\).
  \end{itenum}

  \noindent%
  In short, something about the \agpe{\agents{} present} already secures anything that is required to hold from the \agpe{} when the agent concludes \(\pv{\phi}{v}\) from \(\Phi\).
\end{note}


\begin{note}
  Recall puzzles from XXX.

  Given the way we have characterised \ros{} we may characterise a puzzle as a problem for which a \ros{} holds between some solution to the puzzle and some \pool{} for which the agent does not (yet) have a \wit{}.

  A good puzzle is when obtaining a \wit{} is enjoyable.
\end{note}


\begin{note}
  Additional example.

  \begin{scenario}[Fibonacci numbers]%
    \label{scen:fc:fib}%
    The Fibonacci numbers are recursively defined as follows:

    \[
      F_{n} = \left\{
        \begin{array}{ll}
          0 & \text{if } n = 0 \\
          1 & \text{if } n = 1 \\
          F_{n-1} + F_{n-2} & \text{if } n > 1 \\
        \end{array}
      \right.
    \]
  \end{scenario}

  \begin{enumerate}[label=C\thescenarioCounter., ref=(C\thescenarioCounter)]
  \item
    \label{scen:fc:fib:c}
    \pv{\propI{The sixth number in the Fibonacci sequence is 5}}{\valI{True}}
  \end{enumerate}
  % 
  Given an understanding of \(f\) and sufficient motivation, action an agent performs such that the agent is concluding \pv{\propI{The six number in the Fibonacci sequence is 5}}{\valI{True}} without novel information.
  Hence, \pv{\propI{The six number in the Fibonacci sequence is 5}}{\valI{True}} is a \fc{} from some \pool{}.

  Though, \pv{\propI{The sixtieth number in the Fibonacci sequence is 1548008755920}}{\valI{True}} need not be \fc{0}.
\end{note}


\begin{note}
  Note, agent is concluding.
  Hence, this rules out cases of guidance.

  Still, a slight gap.
  For, it may be the case that when the agent does the action, something changes.
  For example, has the option to try, and so long as does action, given additional powers.
  Possibility, given the contrasting chess scenarios.

  No real way to avoid this given the basic ideas.
  \emph{Could} be that event is in progress.
  But, this significantly complicates things.

  However, it is fine.%
  \footnote{
    It's not really, and it kind of hurts.
  }
  For, interest is in constructing counterexamples.
  So long as do not rely on features, then counterexamples persist.
\end{note}



% \paragraph*{Denying \supportII{} \hfill (Optional)}
% \label{sec:denying-supportii}


% \begin{note}
%   \supportII{} positive condition for \ros{1} without a \wit{}.
%   This doesn't help me.
%   For, I need \ros{} for other work.
%   It's not merely the case that what I'm getting is a \ros{} without a \wit{}.
% \end{note}

% \begin{note}
%   In addition to \wit{} for a \ros{}, we also define a \pwit{} for a \ros{}:

%   \begin{definition}[A \pwit{} for a \ros{}]
%     \label{def:Pwit}%
%     \vspace{-\baselineskip}
%     \begin{itemize}
%     \item
%       An event \(e^{-}_{d^{-}}\) is \emph{\pwit{0}} for a \ros{} between \(\pv{\phi}{v}\) and \(\Phi\), for \vAgent{} through event \(e_{d}\)
%     \end{itemize}

%     \emph{If and only if:}

%     \begin{itemize}
%     \item
%       \(e^{-}_{d^{-}}\) is an event in which \vAgent{} is concluding \(\pv{\phi}{v}\) from \(\Phi\).
%     \item
%       \(e^{-}_{d^{-}}\) occurs prior to or at the same time as \(e_{d}\).
%     \end{itemize}
%     \vspace{-\baselineskip}
%   \end{definition}

%   \noindent%
%   The definition of a \pwit{} is motivated by \assuPP{}.
%   For, if \(e^{-}_{d^{-}}\) is an event in which an agent is concluding \(\pv{\phi}{v}\) from \(\Phi\) then by \assuPP{}, \(e^{-}_{d^{-}}\) develops into an event \(e^{+}_{d^{+}}\) in which the agent concludes \(\pv{\phi}{v}\) from \(\Phi\).
% \end{note}


% \begin{note}
%   Given \autoref{def:Pwit}, an observation:

%   \begin{observation}%
%     \label{obs:supportIIplus}%
%     If deny \supportII{} then \ros{} either \pwitP{} or something compatible with absence of \wit{}.
%   \end{observation}

%   \begin{argument}{obs:supportIIplus}
%     Suppose all compatible with absence of a \wit{}.
%     Now,

%     Suppose \fc{} and no \ros{}.
%     Then, it must be the case that \pwitP{} for \ros{}.
%     For, there is some action the agent does, the agent may do the action, and the completion of the action concludes, and at this point, a \ros{} by \supportII{}.

%     Hence, it is always the case that \pwitP{}.
%   \end{argument}

%   Our interest is with the initial disjunct.
%   It must be the case that \pwitP{}.
%   We tie \ros{1} to conclusions and to avoid commitment.
%   However, free to strengthen if you like.

%   Key insight here is that you are free to strengthen, so long as compatible.
% \end{note}



\paragraph*{\supportII{} and \citeauthor{Goldman:1979ui}'s account of \emph{ex ante} justification \hfill (Optional)}


\begin{note}
  From a structural perspective, our approach to characterising \ros{1} without a given an event in which the agent concludes is similar to \citeauthor{Goldman:1979ui}'s account of \emph{ex ante} justification in terms of \emph{ex post} justification.%
  \footnote{
    \citeauthor{Goldman:1979ui}'s notions of \emph{ex ante} and \emph{ex post} justification is similar to the distinction between doxastic and propositional justification (see \cite{Firth:1978vi} and \cite[esp.\ fn.1]{Silva:2020aa}).

    Given the parallels between \ros{1} and \emph{ex ante} justification, may think of \ros{1} in line with propositional justification.
  }%
  \(^{,}\)%
  \footnote{
    \citeauthor{Turri:2010aa} (\citeyear{Turri:2010aa}) provides similar account.
    However, \citeauthor{Turri:2010aa} does not hold that this is sufficient.
  }

  Here's a paraphrase of \citeauthor{Goldman:1979ui}'s account of \emph{ex post} justification:
  \begin{quote}
    Person \emph{S} is \emph{ex post} justified in believing \emph{p} when \emph{S} believes \emph{p}, and we say \emph{S}' believing \emph{p} is~justified.%
    \mbox{ }\hfill\mbox{(\citeyear[Cf.][21]{Goldman:1979ui})}
  \end{quote}
  % 
  And here's \citeauthor{Goldman:1979ui}'s account of \emph{ex ante} justification:
  % 
  \begin{quote}
    Person \emph{S} is \emph{ex ante} justified in believing \emph{p} at \emph{t} if and only if there is a reliable belief-forming operation available to \emph{S} which is such that if \emph{S} applied that operation to his total cognitive state at \emph{t}, \emph{S} would believe pat \emph{t}-plus-delta (for a suitably small delta) and that belief would be \emph{ex post} justified.%
    \mbox{ }\hfill\mbox{(\citeyear[21]{Goldman:1979ui})}
  \end{quote}
  %
  In a broad stroke, someone is \emph{ex ante} justified in believing \emph{p} at \emph{t} just in case there is some action the person may immediately do and as a result of doing the action, the person is \emph{ex post} justified in \emph{p}.

  Likewise, if \support{} characterises an \emph{ex post} \ros{0} and \supportII{} characterises an \emph{ex ante} \ros{0} we may say, in a broad stroke, an \emph{ex ante} \ros{} holds from an \agpe{} just in case there is some action the person may immediately do and as a result of doing the action a \emph{ex post} \ros{} holds from the \agpe{}.

  There is, however, a small difference:
  \citeauthor{Goldman:1979ui} defines \emph{ex ante} justification in terms of \emph{ex post} justification, whereas we only provide a sufficient condition for an \emph{ex ante} \ros{} in terms of an \emph{ex post} \ros{}.
\end{note}



\section*{Summary}


\begin{note}
  A \ros{} holding between \(\pv{\phi}{v}\) and \(\Phi\) is a state of affairs.
  Therefore, it is possible for an agent to \eval{} the proposition \propI{A \ros{} between \(\pv{\phi}{v}\) and \(\Phi\)} as \valI{True}, \valI{Possible}, \valI{Desired}, and so on\dots.

  Hence, it is possible for both \ref{Embed:no} and \ref{Embed:yes} to occur:

  \begin{enumerate}[label=\arabic*., ref=(\arabic*)]
  \item
    \label{Embed:no}
    A \ros{0} between \(\pv{\phi}{v}\) and \(\Phi\) holds from \agpe{an \agents{}}.
  \item
    \label{Embed:yes}
    A \ros{0} between \(\pv{\psi}{v'}\) and \(\Psi\) holds from \agpe{an \agents{}}, where \(\Psi\) contains:

    \pv{\propI{A \ros{} between \(\pv{\phi}{v}\) and \(\Phi\) holds from \agpe{my}}}{\valI{True}}
  \end{enumerate}

  Now, in the case of \ref{Embed:yes}, it is not necessarily the case that a A \ros{0} between \(\pv{\phi}{v}\) and \(\Phi\) holds from \agpe{the}.
  For example, confident that a \ros{} holds between XXX and YYY, though I am mistaken.

  When speak of \ros{}, state of affairs.
  Not a state of affairs from \agpe{an \agents{}}.

  Of course, \ref{Embed:no} and \ref{Embed:yes} may occur simultaneously, and \ref{Embed:no} may be the case, in part, due to \ref{Embed:yes} being the case.
  However, as this takes some effort to think about, we will not (directly, at least) consider cases where \ref{Embed:no} is the case due to \ref{Embed:yes} being the case.

  In other words, argument to follow does not rely (directly) rely on extracting identifying \ros{} which hold due to the agent \evaling{} a proposition containing the \ros{} as \valI{True}.
\end{note}


% \begin{note}
%   It need not be the case that an agent has a \wit{0} for a \ros{0} in order for \ros{} to be involved in answering \qWhyV{}.

%   For, suppose an agent does not have a \wit{0} for the \ros{} between \(\pv{\psi}{v'}\) and \(\Psi\).
%   The upshot of the distinction between~\ref{Embed:no} and~\ref{Embed:yes} is as follows:

%   \begin{itemize}
%   \item
%     If the \ros{0} of \ref{Embed:no} is, in part, an answer to \qWhyV{} then the \ros{0} is a counterexample to \issueConstraint{}.
%   \item
%     If the \ros{0} of \ref{Embed:yes} is, in part, an answer to \qWhyV{} then the \ros{0} is \emph{not} a counterexample to \issueConstraint{}.
%   \end{itemize}

%   The difference is \emph{the way in which} the \ros{} functions with respect to the agent pairing \(\phi\) with \(v\).
%   Whether the \ros{} functions as a premise when the agent concludes \(\pv{\phi}{v}\), or whether the \ros{} functions in a way that is different to a premise.
% \end{note}

% \begin{note}
%   \begin{itemize}
%   \item
%     \(\pv{\phi}{v}\) is supported by \(\Phi\), from the \agpe{}.
%   \item
%     \(\pv{\psi}{v'}\) is supported, in part, by [the way in which \(\pv{\phi}{v}\) is supported by \(\Phi\), from the \agpe{}], from the \agpe{}.
%   \item
%     The way in which \(\pv{\psi}{v'}\) is supported, in part, by the \agpe{} on [the way in which \(\pv{\phi}{v}\) is supported by \(\Phi\), from the \agpe{}].
%   \end{itemize}

%   \ros{} of \ref{Embed:no} is a \ros{} which holds from the \agpe{}
%   \ros{} of \ref{Embed:yes} is a \ros{} which holds from the \agpe{}, from the \agpe{}.
%   Expanded a little more carefully, the primary \ros{1} of \ref{Embed:no} and \ref{Embed:yes} are paraphrased as capturing:
%   %   
%   \begin{itemize}
%   \item
%     The way in which \(\pv{\phi}{v}\) is supported by \(\Phi\), from the \agpe{}.
%   \item
%     The way in which \(\pv{\psi}{v'}\) is supported, in part, by [the way in which \(\pv{\phi}{v}\) is supported by \(\Phi\), from the \agpe{}], from the \agpe{}.
%   \end{itemize}

%   When we refer to a \ros{} reference is to a state of affairs from \agpe{our}, as in \ref{Embed:no}.
%   Shorthand, an `\rosNE{}' \ros{}.
%   And, a \ros{} from \agpe{an \agents{}} such as \ref{Embed:no} a `\rosE{}' \ros{}.

% \end{note}

% {
% \color{red}
% To refer to a \ros{} as an \rosE{}, need a second \ros{}.
% Hence, this clears things up.
% }

%   \begin{note}
%     The present section concerns, for some arbitrary proposition-value pair \(\pv{\phi}{v}\) and \pool{} \(\Psi\), the distinction between:
%   %     

%   %     
%     \ref{Embed:no} is a \ros{} between \(\pv{\phi}{v}\) and \(\Phi\).
%     By contrast, \ref{Embed:yes} is a \ros{0} between \(\pv{\psi}{v'}\) and \(\Psi\) which \emph{involves} a \ros{} between \(\pv{\phi}{v}\) and \(\Phi\).

%   %     
%     Throughout this document our interest is with \ros{} that do not occur within some other (relevant) \ros{}.
%     In particular, we have implicitly assumed there are no \ros{} which answer \qWhy{} due to the \ros{} occurring within some other \ros{}.
%     And, the variant of \qWhyV{} introduced in \autoref{cha:var} (on \autopageref{questionWhyV}) explicitly requires this.

%     If a \ros{} between \(\pv{\phi}{v}\) and \(\Phi\) referenced due the \ros{} occurring within some other \ros{} (as in \ref{Embed:yes}), we say the reference is to an `\rosE{0}' \ros{}.
%     Otherwise, we say the reference is to an `\rosNE{0}' \ros{} (as in \ref{Embed:no}).

%     Throughout this document reference to a \ros{} is almost always clearly reference to an \rosNE{0} \ros{}.
%     Hence, we only distinguish between \rosE{0} and \rosNE{0} when some ambiguity may be present.
%   \end{note}

%   \begin{note}
%     Idea is somewhat familiar from distinction between object- and meta-language with respect to propositional logic.
%     Certain kind of equivalence between proof and conditional.
%     It is possible to find a corresponding conditional to any proof with a finite number of premises, proof captures derivation of conclusion from premises.

%     Corresponding conditional is not a premise, nor any part, of the proof.

%     For example, consider a proof from \(P\) and \(P \rightarrow Q\) to \(Q\) by conditional detachment.
%     Corresponding conditional is \((P \land (P \rightarrow Q)) \rightarrow Q\).
%     However, not part of the proof.

%     Intuitive distinction between what a proof and a conditional refer to.
%     However, informally there is no difficulty in treating a proof as a premise.
%     \(P\), and I have a proof of \(P \rightarrow Q\), therefore \(Q\).
%   \end{note}

%   \subsubsection[Definitions]{Definitions \hfill (Optional)}
%   \label{cha:var:ros:Emb:defs}

%   \begin{note}
%     Distinction between an \rosE{0} and \rosNE{0} is intuitive, but imprecise.

%     This section provide definition.

%     To keep things simple the following definition assumes:
%     \begin{itenum}
%     \item[\emph{If}:]
%       \(\phi\) having value \(v\) entails \(\phi'\) has value \(v'\), from the \agpe{}.
%     \item[\emph{Then}:]
%       For any \pool{} \(\Phi\), \pv{\phi'}{v'} is in \(\Phi\) whenever  \(\pv{\phi}{v}\) is in \(\Phi\).
%     \end{itenum}
%     E.g., if \(\pv{\phi'\text{ and }\phi''}{\valI{True}}\) is in \(\Phi\) then both \(\pv{\phi'}{\valI{True}}\) and \(\pv{\phi''}{\valI{True}}\) are in \(\Phi\).

%     \begin{definition}[Degree of a \prop{0}-\val{0} pair within a \ros{}]%
%       \label{def:embedding:degree}%
%       For a proposition-value pairs \(\pv{\psi}{v'}\), \(\pv{\phi}{v}\), \pool{} \(\Phi\), and \(i \in \mathbb{N}\):

%       \begin{itemize}
%       \item
%         \(\pv{\psi}{v'}\) has a \emph{degree \(1\)} with respect to a \ros{} between \(\pv{\phi}{v}\) and \(\Phi\) if and only if \(\pv{\psi}{v'} \in \Phi\).
%       \item
%         \(\pv{\psi}{v'}\) is has a \emph{degree \(i\)} with respect to a \ros{} between \(\pv{\phi}{v}\) and \(\Phi\) if and only if:
%         \begin{itemize}
%         \item
%           There exists some \(\pv{\theta}{v''}\) and \(\Theta\) such that:
%           \begin{itemize}
%           \item
%             \(\pv{\psi}{v'} \in \Theta\)
%           \item
%             \(\pv{\propI{A \ros{} between }\pv{\theta}{v''}\propI{ and }\Theta}{\valI{True}}\) has degree \(i - 1\) with respect to the \ros{} between \(\pv{\phi}{v}\) and \(\Phi\).
%           \end{itemize}
%         \end{itemize}
%       \end{itemize}
%       \vspace{-\baselineskip}
%     \end{definition}

%     The cases of interest to us are where \pv{\propI{A \ros{} between \(\pv{\psi}{v'}\) and \(\Psi\)}}{\valI{True}} has degree \(n\) within the \ros{} between \(\pv{\phi}{v}\) and \(\Phi\).

%     Reference to \ros{} between A \ros{} between \(\pv{\psi}{v'}\) and \(\Psi\) due to the \ros{} having degree \(n\) within the \ros{} between \(\pv{\phi}{v}\) and \(\Phi\).
%     That's reference to a \rosE{}.

%     is \rosE{0} within in a \ros{} between \(\pv{\phi}{v}\) and \(\Phi\), no matter the degree of embedding:

%     \begin{definition}[Embedding within a \ros{}]%
%       \label{def:embedding}%
%       For a proposition-value pairs \(\pv{\psi}{v'}\), \(\pv{\phi}{v}\), and a \pool{} \(\Phi\):

%       \begin{itemize}
%       \item
%         \(\pv{\psi}{v'}\) is \emph{\rosE{0}} within in a \ros{} between \(\pv{\phi}{v}\) and \(\Phi\)
%       \end{itemize}

%       \emph{If and only if:}

%       \begin{itemize}
%       \item
%         \(\pv{\psi}{v'}\) is has a degree of embedding \(i\) with respect to the \ros{} between \(\pv{\phi}{v}\) and \(\Phi\), for some \(i \in \mathbb{N}\).
%       \end{itemize}
%       \vspace{-\baselineskip}
%     \end{definition}

%     The definition of an embedding covers arbitrary proposition-value pairs.
%     However, the cases of embedding of interest to us are where \ros{1} are \rosE{0} within a \ros{}.
%     A final definition captures when this is the case:

%     \begin{definition}[A \prop{0}-\val{0} pair \rosE{0} in a \ros{1}]
%       For a proposition-value pairs \(\pv{\psi}{v'}\), \(\pv{\phi}{v}\), and \pool{1} \(\Phi\), \(\Psi\):

%       \begin{itemize}
%       \item
%         A \ros{} between \(\pv{\psi}{v'}\) and \(\Psi\) is \rosE{0} within the \ros{} between \(\pv{\phi}{v}\) and \(\Phi\).
%       \end{itemize}

%       \emph{If and only if}

%       \begin{itemize}
%       \item
%         For some proposition-value pair \(\pv{\chi}{v''}\) in \(\Phi\):
%         \begin{itemize}[noitemsep]
%         \item
%           \(\chi\) is the proposition: \propI{A \ros{} between \(\pv{\psi}{v'}\) and \(\Psi\)}.
%         \item
%           \(v''\) is the value: \valI{True}
%         \end{itemize}
%       \end{itemize}
%       \vspace{-\baselineskip}
%     \end{definition}
%   \end{note}



%%% Local Variables:
%%% mode: latex
%%% TeX-master: "master"
%%% TeX-engine: luatex
%%% End:
