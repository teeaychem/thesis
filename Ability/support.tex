\chapter{\ros{3}}
\label{cha:ros}

\begin{note}
  \ros{3} captures the way in which a \prop{0}-\val{0} pair \(\pv{\phi}{v}\) `follows from' some \pool{} \(\Phi\), from an \agpe{}.
  Where the `following from' relation contrasts with `from' in the sense that  an agent concludes \(\pv{\phi}{v}\) `from' \(\Phi\).

  \begin{itemize}
  \item
    \autoref{scen:calc}.

    \ros{3} between \propM{\gistCalcEq{}} and \pool{1} = calculator.

    Also, \ros{3} between \propM{\gistCalcEq{}} and \pool{1} = understanding of arithmetic.

  \item
    \autoref{scen:animalism}. \ros{2} between \pv{\propI{Four legs good, two legs bad}}{\valI{True}} and \pool{} existence of an explanation by Snowball.

    However, no \ros{} between \pv{\propI{Four legs good, two legs bad}}{\valI{True}} and \pool{} the content of Snowball's explanation. For the birds did not understand.
  \end{itemize}
\end{note}

\begin{note}
  Broad idea.

  In turn, \ros{3} are answers to \qWhy{}, and if \issueConstraint{} holds, for any \ros{} which answers \qWhy{} there is some event where agent concludes.
\end{note}

\begin{note}
  Specifying \ros{} in any significant detail is beyond the scope of this document.
  \issueInclusion{} is understood as a (plausible) constraint on theories about the why and how of reasoning rather than the details of any given theory.
  Instead, we characterise \ros{} by two ideas, and anything which satisfies these two ideas in the details of a given theory may be understood as a \ros{}.
\end{note}

\begin{note}
  Subsections develop and discuss each in detail:

  \begin{TOCEnum}
  \item
    \TOCLine{cha:ros:I}

    The first idea, roughly:
    An event in which an agent concludes \(\pv{\phi}{v}\) from \(\Phi\) is sufficient for a \ros{} to hold, from \agpe{the \agents{}}.
  \item
    \TOCLine{cha:ros:W}

    A definition for ease of expression:
    An event in which an agent concludes \(\pv{\phi}{v}\) from \(\Phi\) provides an agent with a \wit{0} for a \ros{}.
  \item
    \TOCLine{cha:ros:II}

    The second idea, roughly:
    It is not necessary for an agent to have a \wit{} for a \ros{} to hold between \(\pv{\phi}{v}\) and \(\Phi\) from the \agpe{\agents{}}.
  \end{TOCEnum}
\end{note}

\begin{note}
  So, \supportI{} gives a sufficient condition, while \supportII{} denies the sufficient condition amounts to a necessary condition.

  \supportI{}, `follows from' of interest is present when agent concludes `from'.
  \supportII{} denies `follows from' is equivalent to `from' by possibility of `follows from' without corresponding `from'.
\end{note}

\begin{note}
  Note, \supportII{} does not provide sufficient conditions for a \ros{} to hold without a corresponding \wit{0}.

  We defer the argument for such cases to \autoref{part:ing} (specifically \autoref{cha:fc:fc-sII}).
\end{note}

\section{\supportI{}}
\label{cha:ros:I}

\begin{note}
  \begin{idea}[\supportI{}]
    \label{idea:support}
    \vspace{-\baselineskip}
    \begin{itenum}
    \item[\emph{If}:]
      \(e\) is an event in which \vAgent{} concludes \(\pv{\phi}{v}\) from \(\Phi\).
    \item[\emph{Then}:]
      When \vAgent{} \eval{1} \(\phi\) as having value \(v\) as a sub-event of \(e\):
      \begin{itemize}
      \item
        A \emph{\ros{}} between \(\pv{\phi}{v}\) and \(\Phi\) holds, from \agpe{\vAgent{}'}.
      \end{itemize}
    \end{itenum}
    \vspace{-\baselineskip}
  \end{idea}

  Intuitively, way in which \(\pv{\phi}{v}\) is supported by \(\Phi\), from the \agpe{}.

  In other words, the \ros{} between \(\pv{\phi}{v}\) and \(\Phi\) just captures whatever it is, for the agent, that led to the agent concluding \(\pv{\phi}{v}\) from \(\Phi\).
\end{note}

\begin{note}
  Sub-event.
  For, by \autoref{assu:ConRea} the event in which an agent concludes \(\pv{\phi}{v}\) from \(\Phi\) spans the agent's reasoning to \(\pv{\phi}{v}\) from \(\Phi\).
  And, a \ros{} may only hold as the result of reasoning.
\end{note}

\begin{note}
  A \ros{} holds from an \agpe{}.
  We have no interest in whatever the idea of a \ros{} between \(\pv{\phi}{v}\) and \(\Phi\) simpliciter.

  For example, if an agent concludes \propI{Fish are mammals} is \valI{True} from some \pool{} \(\Phi\), then by \supportI{} a \ros{} holds between \pv{\propI{Fish are mammals}}{\valI{True}} and \(\Phi\).

  {
    \color{red}
    So, this allows us to abbreviate `\ros{} from \agpe{}' to `\ros{}'.
  }
\end{note}

\begin{note}
 \supportI{} is similar to, but distinct from,~\citeauthor{Boghossian:2014aa}'s Taking Condition:%
  \footnote{
    Strictly,~\citeauthor{Boghossian:2014aa} states the Taking Condition in terms of inferring.
    I.e., the Taking Condition reads: `Inferring necessarily involves the thinker \emph{taking} \dots'
    However, as \citeauthor{Boghossian:2008vf} is interested in conclusions, an event in which an agent infers (in~\citeauthor{Boghossian:2014aa} sense) is an event in which an agent concludes (in our sense).
  }

  \begin{quote}
    (Taking Condition):
    [An event in which an agent concludes] necessarily involves the thinker \emph{taking} his premises to support his conclusion and drawing his conclusion because of that fact.%
    \mbox{}\hfill\mbox{(\citeyear[5]{Boghossian:2014aa})}
  \end{quote}

  \noindent%
  For, `taking' is understood by \citeauthor{Boghossian:2014aa} to be component of the agent's reasoning.
  \citeauthor{Boghossian:2014aa} illustrates the Taking Condition as follows:
  % 
  \begin{quote}
    On waking up one morning I recall that:

    \begin{enumerate}[label=(\arabic*), ref=(\arabic*), series=BogEx]
    \item
      \label{BogEx:1}
      It rained last night.
    \end{enumerate}

    I combine this with my knowledge that

    \begin{enumerate}[label=(\arabic*), ref=(\arabic*), resume*=BogEx]
    \item
      \label{BogEx:2}
      If it rained last night, then the streets are wet.
    \end{enumerate}

    to conclude:

    So,

    \begin{enumerate}[label=(\arabic*), ref=(\arabic*), resume*=BogEx]
    \item
      \label{BogEx:3}
      The streets are wet.
    \end{enumerate}
    This belief then affects my choice of footwear.%

    [\dots M]y inferring from~\ref{BogEx:1} and~\ref{BogEx:2} to~\ref{BogEx:3} must involve my arriving at the judgment that~\ref{BogEx:3} in part \emph{because} I \emph{take} the presumed truth of~\ref{BogEx:1} and~\ref{BogEx:2} to provide support for~\ref{BogEx:3}.%
    \mbox{ }\hfill\mbox{(\citeyear[2,4]{Boghossian:2014aa})}
  \end{quote}
  % 
  Hence, for \citeauthor{Boghossian:2014aa}, the Taking Condition captures something \emph{in addition} to~\ref{BogEx:3} being a conclusion from a \pool{} which includes~\ref{BogEx:1} and~\ref{BogEx:2}.

  In contrast, we do not require that a \ros{} has any particular role \emph{for the agent} in event in which an agent concludes \(\pv{\phi}{v}\) from \(\Phi\).
  If an agent concludes~\ref{BogEx:3} from~\ref{BogEx:1} and~\ref{BogEx:2}, then a \ros{} holds between~\ref{BogEx:3} and \(\{\ref{BogEx:1}, \ref{BogEx:2}\}\).
  However, the \ros{} between~\ref{BogEx:3} and \(\{\ref{BogEx:1}, \ref{BogEx:2}\}\) need not itself have a role in the \agents{} conclusion of~\ref{BogEx:3} from \(\{\ref{BogEx:1}, \ref{BogEx:2}\}\).%
  \footnote{
    Also, about the type of reasoning by which the agent concludes.
    This comes from \textcite{Boghossian:2008vf,Boghossian:2012vb}.
    Rule following, taking gets account of rule.
  }

  To illustrate, consider \citeauthor{Wright:2014tt}'s (\citeyear{Wright:2014tt}) `Simple Proposal':
  \begin{quote}
    [C]onsider instead the proposal, not that the status of the transition as inferential depends on the thinker's judgments about his reasons, but that it depends on \emph{what his reasons are}.
    We want his acceptance of the premises to supply his \emph{actual} reasons for accepting the conclusion.
    [\dots]

    Call this the Simple Proposal.
    It says that a thinker infers q from p\(_{1}\) \(\cdots\) p\(_{\text{n}}\) when he accepts each of p\(_{1}\) \(\cdots\) p\(_{\text{n}}\), moves to accept q, and does so for the reason that he accepts p\(_{1}\) \(\cdots\) p\(_{\text{n}}\).%
    \mbox{}\hfill\mbox{(\citeyear[33]{Wright:2014tt})}
  \end{quote}

  \noindent%
  \citeauthor{Wright:2014tt}'s proposal is that the relation between a conclusion and some \pool{} need not be part of what moves the agent to conclude the conclusion from the \pool{}.
  Hence, \citeauthor{Wright:2014tt} denies that reasoning must involve a state which connects premises to conclusions.
  So, \citeauthor{Wright:2014tt} denies \citeauthor{Boghossian:2008vf}'s Taking Condition on inference.
  (\citeyear[Cf.][33-34]{Wright:2014tt})

  \supportI{} is compatible with \citeauthor{Wright:2014tt}'s Simple Proposal as \supportI{} only entails a \ros{} holds between \(\pv{\propM{q}}{\valI{True}}\) and a \pool{} containing \(\pv{\propM{p_{1}}}{\valI{True}}\), \(\cdots\) \(\pv{\propM{p_{n}}}{\valI{True}}\) when the agent concludes.%
  \footnote{
    Still, there is an important between~\supportI{} and \citeauthor{Wright:2014tt}'s Simple Proposal.
    For,~\supportI{} is an entailment, while \citeauthor{Wright:2014tt}'s Simple Proposal is an identity statement.
    Inferring, on the Simple Proposal, is an agent accepting some conclusion for the reason that they accept premises from some \pool{}.
    \supportI{} does not entail that concluding is nothing more than moving to accept \(\pv{\phi}{v}\) as a result of accepting each element of \(\Phi\).
  }\(^{,}\)%
  \footnote{
    There are various other objections to~\citeauthor{Boghossian:2014aa}'s Taking Condition.

    For example,~\citeauthor{Hlobil:2014tq} argues against the Taking Condition as it distracts from what accounts of reasoning out to explain, rather than arguing against the Taking Condition directly.
    Likewise, \citeauthor{McHugh:2016vp} present and summarise various objections to \emph{interest} with the Taking Condition.

    In particular,~\supportI{} is closer to what \citeauthor{McHugh:2016vp} term the `Consequence Condition': \textquote{Inferring q from p entails taking p to support q}.
    (\citeyear[316]{McHugh:2016vp})
    And, as \citeauthor{McHugh:2016vp} observe, the condition is \textquote{consistent with the idea that in inference we take our premises to support our conclusion just in virtue of reasoning from the former to the latter}.
    (\citeyear[316]{McHugh:2016vp})

    \citeauthor{McHugh:2016vp} suggest the arguments they consider against the Taking Condition `put pressure' on the Consequence Condition (\citeyear[327]{McHugh:2016vp}).
    However, these arguments concern interest, rather than whether condition is true.
    And, we have interest in \ros{1}.
  }
\end{note}

\begin{note}
  \color{red}
  Humpty Dumpty arbitrary chooses for `glory' to mean `a nice knock-down argument' (\cite[190]{Carroll:2009aa}).
\end{note}

\section{\wit{3} for \ros{1}}
\label{cha:ros:W}

\begin{note}
  \autoref{cha:ros:I} introduced a sufficient condition for a \ros{} between \(\pv{\phi}{v}\) and \(\Phi\) to hold, from an \agpe{}:
  The agent concluded \(\pv{\phi}{v}\) from \(\Phi\).

  For ease of expression we define an event in which an agent concludes \(\pv{\phi}{v}\) from \(\Phi\) as a `\wit{0}' for \ros{} between \(\pv{\phi}{v}\) and \(\Phi\).
  In full:

  \begin{definition}[A \wit{0} for a \ros{1}]%
    \label{def:witnessing}%
    \vspace{-\baselineskip}
    \begin{itemize}
    \item
      An event \(e\) is \emph{\wit{0}} for a \ros{} between \(\pv{\phi}{v}\) and \(\Phi\), for \vAgent{}.
    \end{itemize}

    \emph{If and only if:}

    \begin{itemize}
    \item
      \(e\) is an event in which \vAgent{} concludes \(\pv{\phi}{v}\) from \(\Phi\).
    \end{itemize}
    \vspace{-\baselineskip}
  \end{definition}

  In general, if an agent has concluded \(\pv{\phi}{v}\) from \(\Phi\), we say the agent `has a \wit{}' for the \ros{} between \(\pv{\phi}{v}\) and \(\Phi\).
\end{note}

\begin{note}
  An important, but trivial, case of \autoref{def:witnessing} is when an agent concludes \(\pv{\phi}{v}\) from \(\Phi\).
  For, if an agent concludes \(\pv{\phi}{v}\) from \(\Phi\) then it is immediate that there is some event in which the agent concludes \(\pv{\phi}{v}\) from \(\Phi\) --- the very same event --- and hence the agent has a \wit{} for the \ros{} between \(\pv{\phi}{v}\) and \(\Phi\).

  % Hence, joining \supportI{} with \autoref{def:witnessing}, we have the following:

  % \begin{proposition}[A conclusion entails witnessed \support{}]
  %   % \label{prop:cws}
  %     If \(e\) is an event in which \vAgent{} concludes \(\pv{\phi}{v}\) from \(\Phi\) then:
  %     \begin{itemize}
  %     \item
  %       When \vAgent{} pairs \(\phi\) with \(v\) as a sub-event of \(e\), a \ros{} between \(\pv{\phi}{v}\) and \(\Phi\) holds, from \agpe{\vAgent{}'}.
  %     \item
  %       There is an event \(e'\) such that \(e'\) is a \wit{} for the \ros{} between \(\pv{\phi}{v}\) and \(\Phi\).
  %     \end{itemize}
  %   \vspace{-\baselineskip}
  % \end{proposition}

  % \begin{argument}{prop:cws}
  %   Immediate for assuming the antecedent and appealing to \supportI{} and \autoref{def:witnessing}, respectively.
  % \end{argument}
\end{note}

\begin{note}
  In addition to \wit{}, \pwit{}.

  \begin{definition}[A \pwit{} for a \ros{}]
    \label{def:Pwit}%
    \vspace{-\baselineskip}
    \begin{itemize}
    \item
      An event \(e\) is \emph{\pwit{0}} for a \ros{} between \(\pv{\phi}{v}\) and \(\Phi\), for \vAgent{}.
    \end{itemize}

    \emph{If and only if:}

    \begin{itemize}
    \item
      \(e\) is an event in which \vAgent{} is concluding \(\pv{\phi}{v}\) from \(\Phi\).
    \end{itemize}
    \vspace{-\baselineskip}
  \end{definition}

  \noindent%
  Motivated by \assuPP{}.
  For, if \(e\) is an event in which \vAgent{} is concluding \(\pv{\phi}{v}\) from \(\Phi\) then by \assuPP{}, \(e\) develops into an event \(e'\) in which \vAgent{} concludes \(\pv{\phi}{v}\) from \(\Phi\).
  And, by \autoref{def:witnessing}, \(e'\) is a \wit{} for a \ros{} between \(\pv{\phi}{v}\) and \(\Phi\).
\end{note}

\newpage

\section{\supportII{}}
\label{cha:ros:II}

\begin{note}
  At the start of this chapter we gave the following paraphrase of \supportII{}:
  %
  \begin{itemize}
  \item
    It is not necessary for an agent to have a \wit{} for a \ros{} to hold between \(\pv{\phi}{v}\) and \(\Phi\) from the \agpe{\agents{}}.
  \end{itemize}
  %
  Taken at face value, this paraphrase suggests a strong idea.
  For, the paraphrase concerns all agents and \prop{0}-\val{0}-\pool{0} pairs.
  Yet, a \ros{} is designed to capture a \prop{0}-\val{0} `following from' a \pool{}, from an \agpe{}.
  And, it is not clear it is possible for an arbitrary \prop{0}-\val{0} to `follow from' a \pool{}, from an \agpe{}.

  Two examples illustrate this point.

  \begin{enumerate}
  \item
    Consider the proposition \propM{0 = 1}, value \valI{True}, and a \pool{} which captures a common understanding of arithmetic.

    It seems implausible \pv{\propM{0 = 1}}{\valI{True}} `follows from' a mathematician's (common) understanding of arithmetic during their normal workday.
  \item
    Consider the proposition \propI{The headline of tomorrow's newspaper contains an even number of vowels}, value \valI{True}, and any \pool{} which includes \prop{0}-\val{0} pairs available to the agent.
  \end{enumerate}

  It seems implausible a \ros{} holds between the relevant \prop{0}-\val{0}-\pool{0} pairs for any agent.
  For the first, consider a mathematician during their normal workday.
  \propI{0 = 1} \emph{conflicts} with a common understanding of arithmetic.

  For the second, consider an agent who has no information about tomorrows newspaper.
  \propI{The headline of tomorrow's newspaper contains an even number of vowel} \emph{exceeds} the information available to the agent.


  Of course, mistakes happen, and one may conclude a conclude some contradiction is true, but there is a distinction between a mistake which happens when an agent concludes, and a mistake present in the agent's understanding of some subject matter.
  %
\end{note}

\begin{note}
  In short, it is intuitively possible that a \ros{} holds without a \wit{} in some situations.
  However, it is not intuitively possible that a \ros{} holds without a \wit{} in all situations.
\end{note}

\begin{note}
  The idea in full is a little more complex.

  \begin{idea}[\supportII{}]%
    \label{idea:support:possible}%
    \vspace{-\baselineskip}
    \begin{itenum}
    \item[\emph{If}:]
      \(\pv{\phi}{v}\) is a \fc{0} from \(\Phi\) for \vAgent{}.
    \item[\emph{Then}:]
      It is possible for \ref{idea:support:possible:ros} and \ref{idea:support:possible:noWit} to hold at the same time:
      \begin{enumerate}[label=\alph*., ref=(\alph*)]
      \item
        \label{idea:support:possible:ros}
        A \ros{} between \(\pv{\phi}{v}\) and \(\Phi\) holds, from \agpe{\vAgent{}'}.
      \item
        \label{idea:support:possible:noWit}
        \vAgent{} doesn't have a (\pwP{}) \wit{} for the \ros{} between \(\pv{\phi}{v}\) and~\(\Phi\).
      \end{enumerate}
    \end{itenum}
    \vspace{-\baselineskip}
  \end{idea}

  \noindent%
  Paraphrased:
  So long as there is an option for the agent to conclude, then it is possible for a \ros{}.

  Working through a little more slowly.
  Action.
  Agent is concluding.
  By \assuPP{}, this entails there is a possible event in which the agent concludes.
  However, the sense of possibility is restricted by the possible event being a development.
  Needs to be the case that prior to the agent's conclusion, there is something which makes it the case that there is a development of the event in which the agent concludes.

  Here, deal with mistakes.
  Working mathematician.
  If working well, then no event when concluding.
  For, always possible to notice mistake.

  Likewise, newspaper, something the agent has no information about.

  The additional constraint, it may be the case that the agent receives information.
  For example, newspaper.
  Friend in the room.
  Action, wonder aloud.
  Friend provides information.
  Agent is concluding, but not without novel information.
  
  


  Still, the antecedent of \supportII{} is a little stronger.








  Expanding the definition of a (\pwP{}) \wit{}, the consequent of \supportII{} states it is possible for a \ros{} to hold between \(\pv{\phi}{v}\) and \(\Phi\) from an \agpe{agent's} though there is no present or prior event in which the agent (\emph{is concluding} or) concludes \(\pv{\phi}{v}\) from \(\Phi\).

  A \ros{} `follows from'.
  \supportII{}, possibility without `from'.
  If focus on the absence of an event in which an agent concludes, then event just prior to conclusion.
  \pwit{} avoids this.

  The role of the antecedent is to limit the scope.
  Possible, so situation in which \ros{} but no (\pwP{}) \wit{}.
\end{note}

\begin{note}
  Possible for \ros{} without \wit{}.
  However, implausible this holds for arbitrary \prop{0}-\val{0}-\pool{0} pairs and agents.

  Antecedent of \supportII{} limits.
\end{note}

\begin{note}
  Two refinements on this basic idea.

  \begin{enumerate}
  \item
    Agent is concluding.
    By \assuPP{}, if there is an event in which the agent is concluding, then there is an event in which the agent concludes.
    And, an event in progress narrows the sense in which there is a possible event in which the agent concludes.
  \item
    Additional restriction on event in which agent concludes.

    For, it may be the case that agent only concludes given some substantial change to psychological facts.
  \end{enumerate}
\end{note}


\section{\ros{3} and \fc{1}}
\label{cha:fc:fc-sII}

\begin{note}
  \supportII{} is an idea about when it is \emph{possible} for a \ros{} to hold from an \agpe{} without the agent having a (\pwP{}) \wit{}.
  We now argue \supportII{} \emph{entails} a \ros{} between \(\pv{\phi}{v}\) and \(\Phi\) holds from an \agpe{} whenever \(\pv{\phi}{v}\) is a \fc{} from \(\Phi\).
  Hence, strengthen consequent.

  May consider this in place of \supportII{}.
  However, there is no need.
  Helps clarify \agpe{my} on these things.
\end{note}

\begin{note}
  First, a we define our understanding of necessary and sufficient conditions.
  With the definition in hand, we state a helper proposition, and then argue for the proposition of interest.
\end{note}

\begin{note}
  We understand necessary and sufficient conditions in line with the \textquote{standard theory}  (\cite[cf.][\S2]{Brennan:2022aa}).
  In this respect, necessary and sufficient conditions are equivalent to the truth of a corresponding material conditional.
  In full:

  \begin{definition}[Necessary and sufficient conditions]
    \label{def:NScon}
    \vspace{-\baselineskip}
    \begin{itemize}
    \item
      \(C\) is a \emph{necessary} condition for attribute \(A\), if and only if:
      \begin{itemize}
      \item
        \emph{If} \vAgent{} has attribute \(A\), \emph{then} condition \(C\) obtains.
      \end{itemize}
    \item
      \(C\) is a \emph{sufficient} condition for attribute \(A\), if and only if:
      \begin{itemize}
      \item
        \emph{If} condition \(C\) obtains, \emph{then} \vAgent{} has attribute \(A\)
      \end{itemize}
    \end{itemize}
    \vspace{-\baselineskip}
  \end{definition}

  \noindent%
  With this definition in hand, a simple proposition follows:

  \begin{proposition}[Attribution]%
    \label{prop:attribution}%
    For any attribute \(A\):
    \begin{itenum}
    \item[\emph{If}:]
      \vAgent{} satisfies all necessary conditions for having attribute \(A\).
    \item[\emph{Then}:]
      \vAgent{} satisfies some sufficient condition for having attribute \(A\).
    \end{itenum}
    \vspace{-\baselineskip}
  \end{proposition}

  \begin{argument}{prop:attribution}
    We argue by contraposition.

    Suppose \vAgent{} it is not the case that \vAgent{} satisfies some sufficient condition for having attribute \(A\).
    In other words, \vAgent{} \vAgent{} fails to satisfy all sufficient condition for having attribute \(A\).

    Then, \vAgent{} does not have attribute \(A\).
    For, it is trivially the case that \vAgent{} has attribute \(A\) is a sufficient condition for \vAgent{} to have attribute \(A\).

    Now, consider the collection of all attributes that hold of \vAgent{}.
    It is immediate that any agent failing to have some attribute which holds of \vAgent{} is a necessary condition for an agent to have attribute \(A\).
    For, by assumption, if all attributes, then the agent does not have attribute \(A\).

    Hence, \vAgent{} fails to satisfy some necessary conditions for having attribute \(A\).
    Equivalently, it is not the case that \vAgent{} satisfies all necessary conditions for having attribute \(A\).
  \end{argument}

  \noindent%
  The details of the argument for \autoref{prop:attribution} rests on \autoref{def:NScon}.
  However, I suspect \autoref{prop:attribution} holds for a range of accounts of what necessary and sufficient conditions are.
  The `insight' is that it is not possible for an agent to have an attribute if they fail to satisfy some sufficient condition, and in turn this means something must necessarily differ between the agent and an agent who does have the attribute.
  Our interest is the contraposition of this insight, but our conditionals are material conditionals, hence do not depend on their content.
\end{note}

\begin{note}
  Importance of \autoref{prop:attribution} is limitation on \ros{}.
  Only sufficient condition (\supportI{}).
  However, \supportII{} limits necessary conditions.
\end{note}

\begin{note}
    \begin{proposition}[\fc{3} and \ros{1}]
    \label{prop:fcs-only-if-pot-support}
    \vspace{-\baselineskip}
    \begin{itenum}
    \item[\emph{If}:]
      \(\pv{\phi}{v}\) is a \fc{0} from \(\Phi\) for \vAgent{} throughout \(e\).
    \item[\emph{Then}:]
      A \ros{0} between \(\pv{\phi}{v}\) and \(\Phi\) holds for \vAgent{} throughout \(e\).
    \end{itenum}
    \vspace{-\baselineskip}
  \end{proposition}

  \begin{argument}{prop:fcs-only-if-pot-support}
    Suppose \(\pv{\phi}{v}\) is a \fc{0} from \(\Phi\), for \vAgent{}.

    If \vAgent{} has a \wit{}, then immediate by \supportI{}.

    Suppose \vAgent{} does not have a \wit{}.

    By assumption, \(\pv{\phi}{v}\) is a \fc{0} from \(\Phi\), for \vAgent{}.


    Therefore, there is some action agent does action concludes.
    By \assuPP{}, there is a possible event in which the agent concludes.
    And, by \supportI{} a \ros{} holds.

    Now, a \ros{} holds, and therefore agent satisfies all necessary conditions.
    For, fail some necessary condition and then no \ros{}.

    So, by \autoref{prop:attribution}, some sufficient condition.

    Now, task is to show the sufficient condition holds of \vAgent{} prior to performing the action.

    Suppose fails.
    Then some condition changes.
    However, this entails a (\pwP{}) \wit{} is necessary.
    Which, \supportII{} denies.
  \end{argument}
\end{note}

% \begin{note}
%   \begin{itemize}
%   \item
%     \(\pv{\phi}{v}\) is supported by \(\Phi\), from the \agpe{}.
%   \item
%     \(\pv{\psi}{v'}\) is supported, in part, by [the way in which \(\pv{\phi}{v}\) is supported by \(\Phi\), from the \agpe{}], from the \agpe{}.
%   \item
%     The way in which \(\pv{\psi}{v'}\) is supported, in part, by the \agpe{} on [the way in which \(\pv{\phi}{v}\) is supported by \(\Phi\), from the \agpe{}].
%   \end{itemize}

%   \ros{} of \ref{Embed:no} is a \ros{} which holds from the \agpe{}
%   \ros{} of \ref{Embed:yes} is a \ros{} which holds from the \agpe{}, from the \agpe{}.
%   Expanded a little more carefully, the primary \ros{1} of \ref{Embed:no} and \ref{Embed:yes} are paraphrased as capturing:
%   %
%   \begin{itemize}
%   \item
%     The way in which \(\pv{\phi}{v}\) is supported by \(\Phi\), from the \agpe{}.
%   \item
%     The way in which \(\pv{\psi}{v'}\) is supported, in part, by [the way in which \(\pv{\phi}{v}\) is supported by \(\Phi\), from the \agpe{}], from the \agpe{}.
%   \end{itemize}

%   When we refer to a \ros{} reference is to a state of affairs from \agpe{our}, as in \ref{Embed:no}.
%   Shorthand, an `\rosNE{}' \ros{}.
%   And, a \ros{} from \agpe{an \agents{}} such as \ref{Embed:no} a `\rosE{}' \ros{}.

% \end{note}

% {
%   \color{red}
%   To refer to a \ros{} as an \rosE{}, need a second \ros{}.
%   Hence, this clears things up.
% }

% \begin{note}
%   The present section concerns, for some arbitrary proposition-value pair \(\pv{\phi}{v}\) and \pool{} \(\Psi\), the distinction between:
%   %

%   %
%   \ref{Embed:no} is a \ros{} between \(\pv{\phi}{v}\) and \(\Phi\).
%   By contrast, \ref{Embed:yes} is a \ros{0} between \(\pv{\psi}{v'}\) and \(\Psi\) which \emph{involves} a \ros{} between \(\pv{\phi}{v}\) and \(\Phi\).

%   %
%   Throughout this document our interest is with \ros{} that do not occur within some other (relevant) \ros{}.
%   In particular, we have implicitly assumed there are no \ros{} which answer \qWhy{} due to the \ros{} occurring within some other \ros{}.
%   And, the variant of \qWhyV{} introduced in \autoref{cha:var} (on \autopageref{questionWhyV}) explicitly requires this.

%   If a \ros{} between \(\pv{\phi}{v}\) and \(\Phi\) referenced due the \ros{} occurring within some other \ros{} (as in \ref{Embed:yes}), we say the reference is to an `\rosE{0}' \ros{}.
%   Otherwise, we say the reference is to an `\rosNE{0}' \ros{} (as in \ref{Embed:no}).

%   Throughout this document reference to a \ros{} is almost always clearly reference to an \rosNE{0} \ros{}.
%   Hence, we only distinguish between \rosE{0} and \rosNE{0} when some ambiguity may be present.
% \end{note}

% \begin{note}
%   Idea is somewhat familiar from distinction between object- and meta-language with respect to propositional logic.
%   Certain kind of equivalence between proof and conditional.
%   It is possible to find a corresponding conditional to any proof with a finite number of premises, proof captures derivation of conclusion from premises.

%   Corresponding conditional is not a premise, nor any part, of the proof.

%   For example, consider a proof from \(P\) and \(P \rightarrow Q\) to \(Q\) by conditional detachment.
%   Corresponding conditional is \((P \land (P \rightarrow Q)) \rightarrow Q\).
%   However, not part of the proof.

%   Intuitive distinction between what a proof and a conditional refer to.
%   However, informally there is no difficulty in treating a proof as a premise.
%   \(P\), and I have a proof of \(P \rightarrow Q\), therefore \(Q\).
% \end{note}

% \subsubsection[Definitions]{Definitions \hfill (Optional)}
% \label{cha:var:ros:Emb:defs}

% \begin{note}
%   Distinction between an \rosE{0} and \rosNE{0} is intuitive, but imprecise.

%   This section provide definition.

%   To keep things simple the following definition assumes:
%     \begin{itenum}
%     \item[\emph{If}:]
%       \(\phi\) having value \(v\) entails \(\phi'\) has value \(v'\), from the \agpe{}.
%     \item[\emph{Then}:]
%       For any \pool{} \(\Phi\), \pv{\phi'}{v'} is in \(\Phi\) whenever  \(\pv{\phi}{v}\) is in \(\Phi\).
%     \end{itenum}
%     E.g., if \(\pv{\phi'\text{ and }\phi''}{\valI{True}}\) is in \(\Phi\) then both \(\pv{\phi'}{\valI{True}}\) and \(\pv{\phi''}{\valI{True}}\) are in \(\Phi\).

%   \begin{definition}[Degree of a \prop{0}-\val{0} pair within a \ros{}]%
%     \label{def:embedding:degree}%
%     For a proposition-value pairs \(\pv{\psi}{v'}\), \(\pv{\phi}{v}\), \pool{} \(\Phi\), and \(i \in \mathbb{N}\):

%     \begin{itemize}
%     \item
%       \(\pv{\psi}{v'}\) has a \emph{degree \(1\)} with respect to a \ros{} between \(\pv{\phi}{v}\) and \(\Phi\) if and only if \(\pv{\psi}{v'} \in \Phi\).
%     \item
%       \(\pv{\psi}{v'}\) is has a \emph{degree \(i\)} with respect to a \ros{} between \(\pv{\phi}{v}\) and \(\Phi\) if and only if:
%       \begin{itemize}
%       \item
%         There exists some \(\pv{\theta}{v''}\) and \(\Theta\) such that:
%         \begin{itemize}
%         \item
%           \(\pv{\psi}{v'} \in \Theta\)
%         \item
%           \(\pv{\propI{A \ros{} between }\pv{\theta}{v''}\propI{ and }\Theta}{\valI{True}}\) has degree \(i - 1\) with respect to the \ros{} between \(\pv{\phi}{v}\) and \(\Phi\).
%         \end{itemize}
%       \end{itemize}
%     \end{itemize}
%     \vspace{-\baselineskip}
%   \end{definition}

%   The cases of interest to us are where \pv{\propI{A \ros{} between \(\pv{\psi}{v'}\) and \(\Psi\)}}{\valI{True}} has degree \(n\) within the \ros{} between \(\pv{\phi}{v}\) and \(\Phi\).

%   Reference to \ros{} between A \ros{} between \(\pv{\psi}{v'}\) and \(\Psi\) due to the \ros{} having degree \(n\) within the \ros{} between \(\pv{\phi}{v}\) and \(\Phi\).
% That's reference to a \rosE{}.

  % is \rosE{0} within in a \ros{} between \(\pv{\phi}{v}\) and \(\Phi\), no matter the degree of embedding:

  % \begin{definition}[Embedding within a \ros{}]%
  %   \label{def:embedding}%
  %   For a proposition-value pairs \(\pv{\psi}{v'}\), \(\pv{\phi}{v}\), and a \pool{} \(\Phi\):

  %   \begin{itemize}
  %   \item
  %     \(\pv{\psi}{v'}\) is \emph{\rosE{0}} within in a \ros{} between \(\pv{\phi}{v}\) and \(\Phi\)
  %   \end{itemize}

  %   \emph{If and only if:}

  %   \begin{itemize}
  %   \item
  %     \(\pv{\psi}{v'}\) is has a degree of embedding \(i\) with respect to the \ros{} between \(\pv{\phi}{v}\) and \(\Phi\), for some \(i \in \mathbb{N}\).
  %   \end{itemize}
  %   \vspace{-\baselineskip}
  % \end{definition}

  % The definition of an embedding covers arbitrary proposition-value pairs.
  % However, the cases of embedding of interest to us are where \ros{1} are \rosE{0} within a \ros{}.
  % A final definition captures when this is the case:

%   \begin{definition}[A \prop{0}-\val{0} pair \rosE{0} in a \ros{1}]
%     For a proposition-value pairs \(\pv{\psi}{v'}\), \(\pv{\phi}{v}\), and \pool{1} \(\Phi\), \(\Psi\):

%     \begin{itemize}
%     \item
%       A \ros{} between \(\pv{\psi}{v'}\) and \(\Psi\) is \rosE{0} within the \ros{} between \(\pv{\phi}{v}\) and \(\Phi\).
%     \end{itemize}

%     \emph{If and only if}

%     \begin{itemize}
%     \item
%       For some proposition-value pair \(\pv{\chi}{v''}\) in \(\Phi\):
%       \begin{itemize}[noitemsep]
%       \item
%         \(\chi\) is the proposition: \propI{A \ros{} between \(\pv{\psi}{v'}\) and \(\Psi\)}.
%       \item
%         \(v''\) is the value: \valI{True}
%       \end{itemize}
%     \end{itemize}
%     \vspace{-\baselineskip}
%   \end{definition}
% \end{note}

\section*{Summary}

\begin{note}
  A \ros{} holding between \(\pv{\phi}{v}\) and \(\Phi\) is a state of affairs.
  Therefore, it is possible for an agent to \eval{} the proposition \propI{A \ros{} between \(\pv{\phi}{v}\) and \(\Phi\)} as \valI{True}, \valI{Possible}, \valI{Desired}, and so on\dots.

  Hence, it is possible for both \ref{Embed:no} and \ref{Embed:yes} to occur:

  \begin{enumerate}[label=\arabic*., ref=(\arabic*)]
  \item
    \label{Embed:no}
    A \ros{0} between \(\pv{\phi}{v}\) and \(\Phi\) holds from \agpe{an \agents{}}.
  \item
    \label{Embed:yes}
    A \ros{0} between \(\pv{\psi}{v'}\) and \(\Psi\) holds from \agpe{an \agents{}}, where \(\Psi\) contains:

      \pv{\propI{A \ros{} between \(\pv{\phi}{v}\) and \(\Phi\) holds from \agpe{my}}}{\valI{True}}
  \end{enumerate}

  Now, in the case of \ref{Embed:yes}, it is not necessarily the case that a A \ros{0} between \(\pv{\phi}{v}\) and \(\Phi\) holds from \agpe{the}.
  For example, confident that a \ros{} holds between XXX and YYY, though I am mistaken.

  When speak of \ros{}, state of affairs.
  Not a state of affairs from \agpe{an \agents{}}.

  Of course, \ref{Embed:no} and \ref{Embed:yes} may occur simultaneously, and \ref{Embed:no} may be the case, in part, due to \ref{Embed:yes} being the case.
  However, as this takes some effort to think about, we will not (directly, at least) consider cases where \ref{Embed:no} is the case due to \ref{Embed:yes} being the case.

  In other words, argument to follow does not rely (directly) rely on extracting identifying \ros{} which hold due to the agent \evaling{} a proposition containing the \ros{} as \valI{True}.
\end{note}


% \begin{note}
%   It need not be the case that an agent has a \wit{0} for a \ros{0} in order for \ros{} to be involved in answering \qWhyV{}.

%   For, suppose an agent does not have a \wit{0} for the \ros{} between \(\pv{\psi}{v'}\) and \(\Psi\).
%   The upshot of the distinction between~\ref{Embed:no} and~\ref{Embed:yes} is as follows:

%   \begin{itemize}
%   \item
%     If the \ros{0} of \ref{Embed:no} is, in part, an answer to \qWhyV{} then the \ros{0} is a counterexample to \issueConstraint{}.
%   \item
%     If the \ros{0} of \ref{Embed:yes} is, in part, an answer to \qWhyV{} then the \ros{0} is \emph{not} a counterexample to \issueConstraint{}.
%   \end{itemize}

%   The difference is \emph{the way in which} the \ros{} functions with respect to the agent pairing \(\phi\) with \(v\).
%   Whether the \ros{} functions as a premise when the agent concludes \(\pv{\phi}{v}\), or whether the \ros{} functions in a way that is different to a premise.
% \end{note}

\begin{note}
  From a structural perspective, our approach to characterising \ros{1} is similar to \citeauthor{Goldman:1979ui}'s account of \citeauthor{Goldman:1979ui}'s account of \emph{ex ante} justification in terms of \emph{ex post} justification.%
  \footnote{
    \citeauthor{Turri:2010aa} (\citeyear{Turri:2010aa}) provides similar account.
    However, \citeauthor{Turri:2010aa} does not hold that this is sufficient.
  }

  Here's a paraphrase of \citeauthor{Goldman:1979ui}'s account of \emph{ex post} justification:
  \begin{quote}
    Person \emph{S} is \emph{ex post} justified in believing \emph{p} when \emph{S} believes \emph{p}, and we say \emph{S}' believing \emph{p} is~justified.%
    \mbox{ }\hfill\mbox{(\citeyear[Cf.][21]{Goldman:1979ui})}
  \end{quote}
  %
  And here's \citeauthor{Goldman:1979ui}'s account of \emph{ex ante} justification:
  %
  \begin{quote}
    Person \emph{S} is \emph{ex ante} justified in believing \emph{p} at \emph{t} if and only if there is a reliable belief-forming operation available to \emph{S} which is such that if \emph{S} applied that operation to his total cognitive state at \emph{t}, \emph{S} would believe pat \emph{t}-plus-delta (for a suitably small delta) and that belief would be \emph{ex post} justified.%
    \mbox{ }\hfill\mbox{(\citeyear[21]{Goldman:1979ui})}
  \end{quote}
  %
  So, we have \dots

  Two immediate differences: Definition, and justification.%
  \footnote{
    \citeauthor{Goldman:1979ui}'s notions of \emph{ex ante} and \emph{ex post} justification is similar to the distinction between doxastic and propositional justification (see \cite{Firth:1978vi} and \cite[esp.\ fn.1]{Silva:2020aa}).

    Given the parallels between \ros{1} and \emph{ex ante} justification, may think of \ros{1} in line with propositional justification.
  }

  However, we have no interest in justification.
  Again, this is not because anything will turn on cases where an agent lacks justification.
  Rather, do not wish to make the distinction.
  Ideas apply to cases in which an agent has justification, and lacks justification.

  {
    \color{red}
    Also, justification.
    This restricts the relevant event.
    It's not the case that the agent suddenly believes.
    By contrast, we have no justification, and hence the role of the agent concluding.
  }
\end{note}



%%% Local Variables:
%%% mode: latex
%%% TeX-master: "master"
%%% TeX-engine: luatex
%%% End:
