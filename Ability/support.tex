\chapter{\ros{3}}
\label{cha:ros}

\begin{note}
  The present section develops and discusses \ros{1} in detail.

  The role of \ros{1} within this document is to capture, in a abstract way, the way in which a relation between \(\pv{\phi}{v}\) and \(\Phi\) holds from an \agpe{} when the agent concludes \(\pv{\phi}{v}\) from \(\Phi\).
\end{note}

\begin{note}
  Our understanding of a `\ros{1}' is given in terms of three ideas, and subsections will develop and discuss each idea in detail:

  \begin{TOCEnum}
  \item
    \TOCLine{cha:ros:I}

    An event in which an agent concludes \(\pv{\phi}{v}\) from \(\Phi\) is sufficient for a \ros{} to hold, for the agent.
  \item
    \TOCLine{cha:ros:W}

    An event in which an agent concludes \(\pv{\phi}{v}\) from \(\Phi\) provides an agent with a \wit{0} for a \ros{}.
  \item
    \TOCLine{cha:ros:II}

    It is possible for a \ros{} to hold, from an \agpe{} without the agent having a \wit{} for the \ros{}.
  \item
    \TOCLine{cha:ros:embed}

    Useful distinction.
  \end{TOCEnum}
\end{note}

\section{\supportI{}}
\label{cha:ros:I}

\begin{note}
  \supportI{} states, roughly, that the event in which an agent concludes \(\pv{\phi}{v}\) from \(\Phi\) is sufficient for a \ros{} to hold between \(\pv{\phi}{v}\) and \(\Phi\).%

  \begin{idea}[\supportI{}]
    \label{idea:support}
    \cenLine{
      \begin{VAREnum}
      \item
        Agent: \vAgent{}
      \item
        Proposition: \(\phi\)
      \item
        Value: \(v\)
      \item
        \pool{2}: \(\Phi\)
      \item
        Event: \(e\)
      \item
        \mbox{ }
      \end{VAREnum}
    }

    For an agent \vAgent{}, a proposition-value pair \(\pv{\phi}{v}\), \pool{} \(\Phi\), and event \(e\):

    \begin{itenum}
    \item[\emph{If}:]
      \(e\) is an event in which \vAgent{} concludes \(\pv{\phi}{v}\) from \(\Phi\).
    \item[\emph{Then}:]
      When \vAgent{} \eval{1} \(\phi\) as having value \(v\) as a sub-event of \(e\):
      \begin{itemize}
      \item
        A \emph{\ros{}} between \(\pv{\phi}{v}\) and \(\Phi\) holds, from \agpe{\vAgent{}'}.
      \end{itemize}
    \end{itenum}
    \vspace{-\baselineskip}
  \end{idea}

  Intuitively, way in which \(\pv{\phi}{v}\) is supported by \(\Phi\), from the \agpe{}.

  In other words, the \ros{} between \(\pv{\phi}{v}\) and \(\Phi\) just captures whatever it is, for the agent, that led to the agent concluding \(\pv{\phi}{v}\) from \(\Phi\).
\end{note}

\begin{note}
  Sub-event.
  For, by \autoref{assu:ConRea} the event in which an agent concludes \(\pv{\phi}{v}\) from \(\Phi\) spans the agent's reasoning to \(\pv{\phi}{v}\) from \(\Phi\).
  And, a \ros{} may only hold as the result of reasoning.
\end{note}

\begin{note}
  A \ros{} holds from an \agpe{}.
  We have no interest in whatever the idea of a \ros{} between \(\pv{\phi}{v}\) and \(\Phi\) simpliciter.

  For example, if an agent concludes \propI{Fish are mammals} is \valI{True} from some \pool{} \(\Phi\), then by \supportI{} a \ros{} holds between \pv{\propI{Fish are mammals}}{\valI{True}} and \(\Phi\).

  {
    \color{red}
    So, this allows us to abbreviate `\ros{} from \agpe{}' to `\ros{}'.
  }
\end{note}

\begin{note}
 \supportI{} is similar to, but distinct from,~\citeauthor{Boghossian:2014aa}'s Taking Condition:

  \begin{quote}
    (Taking Condition):
    Inferring necessarily involves the thinker \emph{taking} his premises to support his conclusion and drawing his conclusion because of that fact.%
    \mbox{}\hfill\mbox{(\citeyear[5]{Boghossian:2014aa})}
  \end{quote}

  \noindent%
  There is an immediate superficial difference in that~\citeauthor{Boghossian:2014aa} states the Taking Condition in terms of inferring.
  However, `an event in which an agent concludes' may be substituted for `inferring' and an important distinction remains.
  For, `taking' is understood by \citeauthor{Boghossian:2014aa} to be component of the agent's reasoning.

  \citeauthor{Boghossian:2014aa} illustrates the Taking Condition as follows:
  % 
  \begin{quote}
    On waking up one morning I recall that:

    \begin{enumerate}[label=(\arabic*), ref=(\arabic*), series=BogEx]
    \item
      \label{BogEx:1}
      It rained last night.
    \end{enumerate}

    I combine this with my knowledge that

    \begin{enumerate}[label=(\arabic*), ref=(\arabic*), resume*=BogEx]
    \item
      \label{BogEx:2}
      If it rained last night, then the streets are wet.
    \end{enumerate}

    to conclude:

    So,

    \begin{enumerate}[label=(\arabic*), ref=(\arabic*), resume*=BogEx]
    \item
      \label{BogEx:3}
      The streets are wet.
    \end{enumerate}
    This belief then affects my choice of footwear.%

    [\dots M]y inferring from~\ref{BogEx:1} and~\ref{BogEx:2} to~\ref{BogEx:3} must involve my arriving at the judgment that~\ref{BogEx:3} in part \emph{because} I \emph{take} the presumed truth of~\ref{BogEx:1} and~\ref{BogEx:2} to provide support for~\ref{BogEx:3}.%
    \mbox{ }\hfill\mbox{(\citeyear[2,4]{Boghossian:2014aa})}
  \end{quote}
  % 
  Hence, for \citeauthor{Boghossian:2014aa}, the Taking Condition captures something \emph{in addition} to~\ref{BogEx:3} being a conclusion from a \pool{} which includes~\ref{BogEx:1} and~\ref{BogEx:2}.

  In contrast, we do not require that a \ros{} has any particular role \emph{for the agent} in event in which an agent concludes \(\pv{\phi}{v}\) from \(\Phi\).
  If an agent concludes~\ref{BogEx:3} from~\ref{BogEx:1} and~\ref{BogEx:2}, then a \ros{} holds between~\ref{BogEx:3} and \(\{\ref{BogEx:1}, \ref{BogEx:2}\}\).
  However, the \ros{} between~\ref{BogEx:3} and \(\{\ref{BogEx:1}, \ref{BogEx:2}\}\) need not itself have a role in the \agents{} conclusion of~\ref{BogEx:3} from \(\{\ref{BogEx:1}, \ref{BogEx:2}\}\).%
  \footnote{
    Also, about the type of reasoning by which the agent concludes.
    This comes from \textcite{Boghossian:2008vf,Boghossian:2012vb}.
    Rule following, taking gets account of rule.
  }

  To illustrate, consider \citeauthor{Wright:2014tt}'s (\citeyear{Wright:2014tt}) `Simple Proposal':
  \begin{quote}
    [C]onsider instead the proposal, not that the status of the transition as inferential depends on the thinker's judgments about his reasons, but that it depends on \emph{what his reasons are}.
    We want his acceptance of the premises to supply his \emph{actual} reasons for accepting the conclusion.
    [\dots]

    Call this the Simple Proposal.
    It says that a thinker infers q from p\(_{1}\) \(\cdots\) p\(_{\text{n}}\) when he accepts each of p\(_{1}\) \(\cdots\) p\(_{\text{n}}\), moves to accept q, and does so for the reason that he accepts p\(_{1}\) \(\cdots\) p\(_{\text{n}}\).%
    \mbox{}\hfill\mbox{(\citeyear[33]{Wright:2014tt})}
  \end{quote}

  \noindent%
  \citeauthor{Wright:2014tt}'s proposal is that the relation between a conclusion and some \pool{} need not be part of what moves the agent to conclude the conclusion from the \pool{}.
  Hence, \citeauthor{Wright:2014tt} denies that reasoning must involve a state which connects premises to conclusions.
  So, \citeauthor{Wright:2014tt} denies \citeauthor{Boghossian:2008vf}'s Taking Condition on inference.
  (\citeyear[Cf.][33-34]{Wright:2014tt})

  \supportI{} is compatible with \citeauthor{Wright:2014tt}'s Simple Proposal as \supportI{} only entails a \ros{} holds between \(\pv{\propM{q}}{\valI{True}}\) and a \pool{} containing \(\pv{\propM{p_{1}}}{\valI{True}}\), \(\cdots\) \(\pv{\propM{p_{n}}}{\valI{True}}\) when the agent concludes.%
  \footnote{
    Still, there is an important between~\supportI{} and \citeauthor{Wright:2014tt}'s Simple Proposal.
    For,~\supportI{} is an entailment, while \citeauthor{Wright:2014tt}'s Simple Proposal is an identity statement.
    Inferring, on the Simple Proposal, is an agent accepting some conclusion for the reason that they accept premises from some \pool{}.
    \supportI{} does not entail that concluding is nothing more than moving to accept \(\pv{\phi}{v}\) as a result of accepting each element of \(\Phi\).
  }\(^{,}\)%
  \footnote{
    There are various other objections to~\citeauthor{Boghossian:2014aa}'s Taking Condition.

    For example,~\citeauthor{Hlobil:2014tq} argues against the Taking Condition as it distracts from what accounts of reasoning out to explain, rather than arguing against the Taking Condition directly.
    Likewise, \citeauthor{McHugh:2016vp} present and summarise various objections to \emph{interest} with the Taking Condition.

    In particular,~\supportI{} is closer to what \citeauthor{McHugh:2016vp} term the `Consequence Condition': \textquote{Inferring q from p entails taking p to support q}.
    (\citeyear[316]{McHugh:2016vp})
    And, as \citeauthor{McHugh:2016vp} observe, the condition is \textquote{consistent with the idea that in inference we take our premises to support our conclusion just in virtue of reasoning from the former to the latter}.
    (\citeyear[316]{McHugh:2016vp})

    \citeauthor{McHugh:2016vp} suggest the arguments they consider against the Taking Condition `put pressure' on the Consequence Condition (\citeyear[327]{McHugh:2016vp}).
    However, these arguments concern interest, rather than whether condition is true.
    And, we have interest in \ros{1}.
  }
\end{note}

\section{\wit{3} for \ros{1}}
\label{cha:ros:W}

\begin{note}
  \autoref{cha:ros:I} introduced a sufficient condition for a \ros{} between \(\pv{\phi}{v}\) and \(\Phi\) to hold, from an \agpe{}:
  The agent concluded \(\pv{\phi}{v}\) from \(\Phi\).

  In general, if an agent has concluded \(\pv{\phi}{v}\) from \(\Phi\), then we will say the agent has a \wit{} for the \ros{} between \(\pv{\phi}{v}\) and \(\Phi\).
  In full:

  \begin{definition}[A \wit{2} for a \ros{0}]
    \label{def:witnessing}
    \cenLine{
      \begin{VAREnum}
      \item
        Agent: \vAgent{}
      \item
        Proposition: \(\phi\)
      \item
        Value: \(v\)
      \item
        \pool{2}: \(\Phi\)
      \item
        Event: \(e\)
      \item
        \mbox{ }
      \end{VAREnum}
    }

    \begin{itemize}
    \item
      \(e\) is \emph{\wit{0}} for \ros{} between \(\pv{\phi}{v}\) and \(\Phi\), for \vAgent{}.
    \end{itemize}

    \emph{If and only if:}

    \begin{itemize}
    \item
      \(e\) is an event in which \vAgent{} concludes \(\pv{\phi}{v}\) from \(\Phi\).
    \end{itemize}
    \vspace{-\baselineskip}
  \end{definition}

  Shorthand, we say that \emph{an agent has} a \wit{0} for the relevant \ros{}.
\end{note}

\begin{note}
  An important, but trivial, case of \autoref{def:witnessing} is when an agent concludes \(\pv{\phi}{v}\) from \(\Phi\).
  For, if an agent concludes \(\pv{\phi}{v}\) from \(\Phi\) then it is immediate that there is some event in which the agent concludes \(\pv{\phi}{v}\) from \(\Phi\) --- the very same event --- and hence the agent has a \wit{} for the \ros{} between \(\pv{\phi}{v}\) and \(\Phi\).

  % Hence, joining \supportI{} with \autoref{def:witnessing}, we have the following:

  % \begin{proposition}[A conclusion entails witnessed \support{}]
  %   % \label{prop:cws}
  %   \cenLine{
  %     \begin{VAREnum}
  %     \item
  %       Agent: \vAgent{}
  %     \item
  %       Proposition: \(\phi\)
  %     \item
  %       Value: \(v\)
  %     \item
  %       \pool{2}: \(\Phi\)
  %     \item
  %       \mbox{ }
  %     \end{VAREnum}
  %   }
  %   \begin{itemize}
  %   \item
  %     If \(e\) is an event in which \vAgent{} concludes \(\pv{\phi}{v}\) from \(\Phi\) then:
  %     \begin{itemize}
  %     \item
  %       When \vAgent{} pairs \(\phi\) with \(v\) as a sub-event of \(e\), a \ros{} between \(\pv{\phi}{v}\) and \(\Phi\) holds, from \agpe{\vAgent{}'}.
  %     \item
  %       There is an event \(e'\) such that \(e'\) is a \wit{} for the \ros{} between \(\pv{\phi}{v}\) and \(\Phi\).
  %     \end{itemize}
  %   \end{itemize}
  %   \vspace{-\baselineskip}
  % \end{proposition}

  % \begin{argument}{prop:cws}
  %   Immediate for assuming the antecedent and appealing to \supportI{} and \autoref{def:witnessing}, respectively.
  % \end{argument}

  % \autoref{prop:cws} is of interest with respect to \qWhy{}, \qHow{}, \issueInclusion{}, and the variations to follow in \autoref{cha:var}.

  % For, our variant to \qWhy{} will involve \ros{1}.
  % Our variant to \qHow{} will involve \wit{1}.
  % And, our variant to \issueInclusion{} will hold that a \ros{} is, in part, an answer to why an agent concluded only if the agent has a \wit{} for the \ros{}.

  % Hence, \autoref{prop:cws} ensures that so long as there is an event in which the agent concludes \(\pv{\phi}{v}\) from \(\Phi\), then an answer to `why' will always have a corresponding answer to `how'.

  % At issue is whether it is always the case that an agent has a \wit{} for a \ros{} which is, in part, an answer to why the agent concluded \(\pv{\phi}{v}\) from \(\Phi\).

  % And, given \autoref{prop:cws} it is immediate that any such \ros{} must be distinct from the \ros{} between \(\pv{\phi}{v}\) and \(\Phi\).
\end{note}

\begin{note}
  Note, when we talk of \wit{1} we talk in terms of `having a \wit{0}'.

  It may be the case that an agent has a \wit{} for a \ros{} between \(\pv{\psi}{v'}\) and \(\Psi\) when the agent concludes \(\pv{\phi}{v}\) from \(\Phi\).

  % However, event \(e\) may be an event in which an agent concludes \(\pv{\phi}{v}\) from \(\Phi\) such that throughout the event \(e\), the agent has a \wit{} for a \ros{} between \(\pv{\psi}{v'}\) and \(\Psi\), such that the event \(e'\) which \wit{1} the \ros{} between \(\pv{\psi}{v'}\) and \(\Psi\) is distinct from \(e\).

  % Hence, our understanding of `having a \wit{0}' allows for the possibility that some \ros{} between \(\pv{\psi}{v'}\) and \(\Psi\), in part, `answers why' an agent concludes \(\pv{\phi}{v}\) from \(\Phi\) though the relevant \wit{0} for the \ros{} between \(\pv{\psi}{v'}\) and \(\Psi\) is distinct.
  % The variant to \issueInclusion{} we develop is compatible with such cases.
  % Still, it is safe to ignore this possibility.
  % We will not directly, at least, consider such cases or take a stand either way in the main argument.%
  \end{note}

\section{\supportII{}}
\label{cha:ros:II}

\begin{note}
  \supportI{} states that a \ros{} holds between \(\pv{\phi}{v}\) and \(\Phi\), for the agent, when an agent concludes \(\pv{\phi}{v}\).
  Roughly, \supportII{} denies the converse of \supportI{}:

  \begin{idea}[\supportII{}]
    \label{idea:support:possible}
    \cenLine{
      \begin{VAREnum}
      \item
        Agent: \vAgent{}
      \item
        Proposition: \(\phi\)
      \item
        Value: \(v\)
      \item
        \pool{2}: \(\Phi\)
      \item
        \mbox{ }
      \end{VAREnum}
    }

    It is possible for \ref{idea:support:possible:ros} and \ref{idea:support:possible:noWit} to hold at the same time:

    \begin{enumerate}[label=\alph*., ref=(\alph*)]
    \item
      \label{idea:support:possible:ros}
      A \ros{} between \(\pv{\phi}{v}\) and \(\Phi\) holds, from \agpe{\vAgent{}'}.
    \item
      \label{idea:support:possible:noWit}
      \vAgent{} does not have a \wit{} for the \ros{} between \(\pv{\phi}{v}\) and \(\Phi\).
      \end{enumerate}
    \vspace{-\baselineskip}
  \end{idea}

  \noindent%
  So, \supportI{} states that an event in which the agent concludes \(\pv{\phi}{v}\) from \(\Phi\) is sufficient for a \ros{} between \(\pv{\phi}{v}\) and \(\Phi\) to hold, for the agent.
  In contrast, \supportII{} denies that an event in which the agent concludes \(\pv{\phi}{v}\) from \(\Phi\) is necessary for a \ros{} between \(\pv{\phi}{v}\) and \(\Phi\) to hold, for the agent.
\end{note}

\begin{note}
  \color{red}
  Intuition for \ros{} given \supportI{}.
  With \supportII{}, less clear.

  Defer details to later.

  However, only sufficient condition for \ros{} is \supportI{}.
  So, no sufficient distinction between the way things are for an agent and way things are if the agent concludes.

  For example, arithmetic.
  Various sums.
  Not all.
  However, \ros{}, in the sense that follows from present grasp on arithmetic.
\end{note}

\begin{note}
  Squinting a little, think of a \ros{} from \supportI{} in terms of justification.

  Distinction between doxastic and propositional justification.%
  \footnote{
    See (\cite{Firth:1978vi}) and (\cite[esp.\ fn.1]{Silva:2020aa}).
  }

  \supportI{} possible something akin to doxastic justification.
  \supportII{} narrows to propositional justification.

  Approach to \ros{1} without \wit{} is similar to \citeauthor{Goldman:1979ui}'s account of \emph{ex ante} justification in terms of \emph{ex post} justification (\citeyear[351--352]{Goldman:1979ui}) and \citeauthor{Turri:2010aa}'s account of propositional justification in terms of possessing means for doxastic justification (\citeyear[320]{Turri:2010aa}).

  However, we have no interest in justification.
  Again, this is not because anything will turn on cases where an agent lacks justification.
  Rather, do not wish to make the distinction.
  Ideas apply to cases in which an agent has justification, and lacks justification.
\end{note}

% \begin{note}
%     Think of \ros{} in terms of doxastic justification.

%   \begin{quote}
%     S's belief that p is doxastically justified (i.e. S's belief is held in an epistemically permissible fashion) if and only if S believes p in the right kind of way, on an epistemically appropriate basis.%
%     \mbox{ }\hfill\mbox{(\citeyear{Bondy:2018tk})}
%   \end{quote}
% \end{note}




\begin{note}
  \supportII{} has a key role in the overall argument.
  For, as indicated, answers to the variant of \qWhy{} will concern \ros{}, and the variant of \qHow{} will concern whether the agent has a \wit{} for the relevant \ros{}.
  \supportII{}, then, allows for the \emph{possibility} that the kind of thing which answers why event is such that an agent concludes is not constrained by how the agent concludes.

  However, \supportII{} should not be of any immediate concern.
  For, there is a significant gap between:

  \begin{itemize}[noitemsep]
  \item
    There being a \ros{} between \(\pv{\psi}{v'}\) and \(\Psi\) from an \agpe{} without the agent having a \wit{} for the \ros{}.
  \item
    The \ros{} between \(\pv{\psi}{v'}\) and \(\Psi\)  answering, in some sense, why the event is such that the agent concluded \(\pv{\phi}{v}\) from \(\Phi\).
  \end{itemize}
\end{note}


\section{\rosE{2} and \rosNE{} \ros{1}}
\label{cha:ros:embed}

{
  \color{red}
  To refer to a \ros{} as an \rosE{}, need a second \ros{}.
  Hence, this clears things up.
}

\begin{note}
  The present section concerns, for some arbitrary proposition-value pair \(\pv{\phi}{v}\) and \pool{} \(\Psi\), the distinction between:
  %
  \begin{enumerate}[label=\arabic*., ref=(\arabic*)]
  \item
    \label{Embed:no}
    A \ros{0} between \(\pv{\phi}{v}\) and \(\Phi\).
  \item
    \label{Embed:yes}
    A \ros{0} between \(\pv{\psi}{v'}\) and \(\Psi\), where:
    \begin{itemize}
    \item
      \(\Psi\) contains:
      \begin{itemize}
      \item
        \pv{\propI{A \ros{} between \(\pv{\phi}{v}\) and \(\Phi\)}}{\valI{True}}
      \end{itemize}
    \end{itemize}
  \end{enumerate}
  %
  \ref{Embed:no} is a \ros{} between \(\pv{\phi}{v}\) and \(\Phi\).
  By contrast, \ref{Embed:yes} is a \ros{0} between \(\pv{\psi}{v'}\) and \(\Psi\) which \emph{involves} a \ros{} between \(\pv{\phi}{v}\) and \(\Phi\).

  Intuitively, the primary \ros{1} of \ref{Embed:no} and \ref{Embed:yes} are paraphrased as capturing:
  %
  \begin{itemize}
  \item
    The way in which \(\pv{\phi}{v}\) is supported by \(\Phi\), from the \agpe{}.
  \item
    The way in which \(\pv{\psi}{v'}\) is supported, in part, by [the way in which \(\pv{\phi}{v}\) is supported by \(\Phi\), from the \agpe{}], from the \agpe{}.
  \end{itemize}
  %
  Throughout this document our interest is with \ros{} that do not occur within some other (relevant) \ros{}.
  In particular, we have implicitly assumed there are no \ros{} which answer \qWhy{} due to the \ros{} occurring within some other \ros{}.
  And, the variant of \qWhyV{} introduced in \autoref{cha:var} (on \autopageref{questionWhyV}) explicitly requires this.

  If a \ros{} between \(\pv{\phi}{v}\) and \(\Phi\) referenced due the \ros{} occurring within some other \ros{} (as in \ref{Embed:yes}), we say the reference is to an `\rosE{0}' \ros{}.
  Otherwise, we say the reference is to an `\rosNE{0}' \ros{} (as in \ref{Embed:no}).

  Throughout this document reference to a \ros{} is almost always clearly reference to an \rosNE{0} \ros{}.
  Hence, we only distinguish between \rosE{0} and \rosNE{0} when some ambiguity may be present.
\end{note}

\begin{note}
  Idea is somewhat familiar from distinction between object- and meta-language with respect to propositional logic.
  Certain kind of equivalence between proof and conditional.
  It is possible to find a corresponding conditional to any proof with a finite number of premises, proof captures derivation of conclusion from premises.

  Corresponding conditional is not a premise, nor any part, of the proof.

  For example, consider a proof from \(P\) and \(P \rightarrow Q\) to \(Q\) by conditional detachment.
  Corresponding conditional is \((P \land (P \rightarrow Q)) \rightarrow Q\).
  However, not part of the proof.

  Intuitive distinction between what a proof and a conditional refer to.
  However, informally there is no difficulty in treating a proof as a premise.
  \(P\), and I have a proof of \(P \rightarrow Q\), therefore \(Q\).
\end{note}

\subsubsection[Definitions]{Definitions \hfill (Optional)}
\label{cha:var:ros:Emb:defs}

\begin{note}
  Distinction between an \rosE{0} and \rosNE{0} is intuitive, but imprecise.

  This section provide definition.

  To keep things simple the following definition assumes:
    \begin{itenum}
    \item[\emph{If}:]
      \(\phi\) having value \(v\) entails \(\phi'\) has value \(v'\), from the \agpe{}.
    \item[\emph{Then}:]
      For any \pool{} \(\Phi\), \pv{\phi'}{v'} is in \(\Phi\) whenever  \(\pv{\phi}{v}\) is in \(\Phi\).
    \end{itenum}
    E.g., if \(\pv{\phi'\text{ and }\phi''}{\valI{True}}\) is in \(\Phi\) then both \(\pv{\phi'}{\valI{True}}\) and \(\pv{\phi''}{\valI{True}}\) are in \(\Phi\).

  \begin{definition}[Degree of a \prop{0}-\val{0} pair within a \ros{}]%
    \label{def:embedding:degree}%
    For a proposition-value pairs \(\pv{\psi}{v'}\), \(\pv{\phi}{v}\), \pool{} \(\Phi\), and \(i \in \mathbb{N}\):

    \begin{itemize}
    \item
      \(\pv{\psi}{v'}\) has a \emph{degree \(1\)} with respect to a \ros{} between \(\pv{\phi}{v}\) and \(\Phi\) if and only if \(\pv{\psi}{v'} \in \Phi\).
    \item
      \(\pv{\psi}{v'}\) is has a \emph{degree \(i\)} with respect to a \ros{} between \(\pv{\phi}{v}\) and \(\Phi\) if and only if:
      \begin{itemize}
      \item
        There exists some \(\pv{\theta}{v''}\) and \(\Theta\) such that:
        \begin{itemize}
        \item
          \(\pv{\psi}{v'} \in \Theta\)
        \item
          \(\pv{\propI{A \ros{} between }\pv{\theta}{v''}\propI{ and }\Theta}{\valI{True}}\) has degree \(i - 1\) with respect to the \ros{} between \(\pv{\phi}{v}\) and \(\Phi\).
        \end{itemize}
      \end{itemize}
    \end{itemize}
    \vspace{-\baselineskip}
  \end{definition}

  The cases of interest to us are where \pv{\propI{A \ros{} between \(\pv{\psi}{v'}\) and \(\Psi\)}}{\valI{True}} has degree \(n\) within the \ros{} between \(\pv{\phi}{v}\) and \(\Phi\).

  Reference to \ros{} between A \ros{} between \(\pv{\psi}{v'}\) and \(\Psi\) due to the \ros{} having degree \(n\) within the \ros{} between \(\pv{\phi}{v}\) and \(\Phi\).
That's reference to a \rosE{}.

  % is \rosE{0} within in a \ros{} between \(\pv{\phi}{v}\) and \(\Phi\), no matter the degree of embedding:

  % \begin{definition}[Embedding within a \ros{}]%
  %   \label{def:embedding}%
  %   For a proposition-value pairs \(\pv{\psi}{v'}\), \(\pv{\phi}{v}\), and a \pool{} \(\Phi\):

  %   \begin{itemize}
  %   \item
  %     \(\pv{\psi}{v'}\) is \emph{\rosE{0}} within in a \ros{} between \(\pv{\phi}{v}\) and \(\Phi\)
  %   \end{itemize}

  %   \emph{If and only if:}

  %   \begin{itemize}
  %   \item
  %     \(\pv{\psi}{v'}\) is has a degree of embedding \(i\) with respect to the \ros{} between \(\pv{\phi}{v}\) and \(\Phi\), for some \(i \in \mathbb{N}\).
  %   \end{itemize}
  %   \vspace{-\baselineskip}
  % \end{definition}

  % The definition of an embedding covers arbitrary proposition-value pairs.
  % However, the cases of embedding of interest to us are where \ros{1} are \rosE{0} within a \ros{}.
  % A final definition captures when this is the case:

%   \begin{definition}[A \prop{0}-\val{0} pair \rosE{0} in a \ros{1}]
%     For a proposition-value pairs \(\pv{\psi}{v'}\), \(\pv{\phi}{v}\), and \pool{1} \(\Phi\), \(\Psi\):

%     \begin{itemize}
%     \item
%       A \ros{} between \(\pv{\psi}{v'}\) and \(\Psi\) is \rosE{0} within the \ros{} between \(\pv{\phi}{v}\) and \(\Phi\).
%     \end{itemize}

%     \emph{If and only if}

%     \begin{itemize}
%     \item
%       For some proposition-value pair \(\pv{\chi}{v''}\) in \(\Phi\):
%       \begin{itemize}[noitemsep]
%       \item
%         \(\chi\) is the proposition: \propI{A \ros{} between \(\pv{\psi}{v'}\) and \(\Psi\)}.
%       \item
%         \(v''\) is the value: \valI{True}
%       \end{itemize}
%     \end{itemize}
%     \vspace{-\baselineskip}
%   \end{definition}
\end{note}

\section*{Summary}
% \label{cha:ros:sum}





% \begin{note}
%   It need not be the case that an agent has a \wit{0} for a \ros{0} in order for \ros{} to be involved in answering \qWhyV{}.

%   For, suppose an agent does not have a \wit{0} for the \ros{} between \(\pv{\psi}{v'}\) and \(\Psi\).
%   The upshot of the distinction between~\ref{Embed:no} and~\ref{Embed:yes} is as follows:

%   \begin{itemize}
%   \item
%     If the \ros{0} of \ref{Embed:no} is, in part, an answer to \qWhyV{} then the \ros{0} is a counterexample to \issueConstraint{}.
%   \item
%     If the \ros{0} of \ref{Embed:yes} is, in part, an answer to \qWhyV{} then the \ros{0} is \emph{not} a counterexample to \issueConstraint{}.
%   \end{itemize}

%   The difference is \emph{the way in which} the \ros{} functions with respect to the agent pairing \(\phi\) with \(v\).
%   Whether the \ros{} functions as a premise when the agent concludes \(\pv{\phi}{v}\), or whether the \ros{} functions in a way that is different to a premise.
% \end{note}


%%% Local Variables:
%%% mode: latex
%%% TeX-master: "master"
%%% TeX-engine: luatex
%%% End:
