\chapter{\fingfr{3}}
\label{cha:ros}


\begin{note}
  \autoref{cha:intro} introduced the idea of a \prop{0}-\val{0} pair \(\pv{\phi}{v}\) `\fingf{}' some \pool{} \(\Phi\) from an \agpe{} --- regardless of whether the agent has concluded \(\pv{\phi}{v}\) `from' \(\Phi\).

  In turn, \fingfr{0} answer \qWhy{}, and if \issueInclusion{} holds, for any \fingfr{} which answers \qWhy{} there is some event where agent concludes.

  Specifying \fingfr{} in any significant detail is beyond the scope of this document.
  For, \issueInclusion{} is understood as a (plausible) constraint on theories about the why and how of reasoning rather than the details of any given theory.
  Instead, we state two sufficient conditions for a \fingfr{}.
\end{note}


\section{\supportI{}}
\label{cha:ros:I}


\begin{note}
  \supportI{} states a \fofb{some \prop{0}-\val{0} pair}{some \pool{}} \emph{given} an \agents{} conclusion of the \prop{0}-\val{0} pair from the \pool{}.

  \begin{ridea}{idea:support}{\supportI{}}
    \vspace{-\baselineskip}
    \begin{itenum}
    \item[\emph{If}:]
      \(\ed{}\) is an \eiw{0} \vAgent{} concludes \(\pv{\phi}{v}\) from \(\Phi\).
    \item[\emph{Then}:]
      When \vAgent{} \eval{1} \(\phi\) as having value \(v\):
      \begin{itemize}
      \item
        \(\pv{\phi}{v}\) \emph{\fof{0}} \(\Phi\) from \agpe{\vAgent{}'}.
      \end{itemize}
    \end{itenum}
    \vspace{-\baselineskip}
  \end{ridea}

  \noindent%
  If an agent concludes \(\pv{\phi}{v}\) from \(\Phi\), then the agent \evals{} \(\phi\) to have \val{0} \(v\) on the basis of the \prop{0}-\val{0} pair included in \(\Phi\).
  In this respect, \(\pv{\phi}{v}\) `\fof{}' \(\Phi\) from the \agpe{}.

  In other words, \fingfb{\(\pv{\phi}{v}\)}{\(\Phi\)} just captures whatever it is, for the agent, that led to the conclusion of \(\pv{\phi}{v}\) from \(\Phi\) by the agent.%
  \footnote{
    The existence of a \fingfr{0} is restricted to the sub-event when the agent \evals{} \(\phi\) as having \val{} \(v\).
    For, we assume a concludes event spans the \agents{} reasoning to \(\pv{\phi}{v}\) from \(\Phi\).
    See \autoref{assu:ConRea} (\autopageref{assu:ConRea}) in \autoref{cha:clar} for further detail.
  }
\end{note}


\begin{note}
  Applied to \scen{1}~\ref{illu:gist:roots:F}~and~\ref{illu:gist:roots:QF} \supportI{} entails \fofb{\pv{\propM{\rootsCon{}}}{\valI{True}}}{a \pool{} which includes the \agents{} understanding of factorisation or the quadratic formula} when the agent concludes.
  And, applied to \autoref{scen:countS} \supportI{} entails a \fofb{\pv{\propI{What Pritcher said was a sign}}{\valI{True}}}{some \pool{}} from \agpe{Fox's} when Fox responds `Miran is early this year'.
\end{note}



\section{\wit{3} for \fingfr{1}}
\label{cha:ros:W}


\begin{note}
  With \supportI{} in hand we define a `\wit{0}' for a \fingfr{}.
  In full:

  \begin{rdefinition}{def:witnessing}{A \wit{0} for a \fingfr{1}}%
    \vspace{-\baselineskip}
    \begin{itemize}
    \item
      An event \(\ed{\ast}\) is a \emph{\wit{0}} for \fingfb{\(\pv{\phi}{v}\)}{\(\Phi\)} for \vAgent{} through an event \(\ed{}\).
    \end{itemize}

    \emph{If and only if:}

    \begin{itemize}
    \item
      \(\ed{\ast}\) is an \eiw{0} \vAgent{} concludes \(\pv{\phi}{v}\) from \(\Phi\).
    \item
      \(\ed{\ast}\) happens before or at the same time as \(\ed{}\).
    \end{itemize}
    \vspace{-\baselineskip}
  \end{rdefinition}

  \noindent%
  In short, a \wit{} for \fingfb{\(\pv{\phi}{v}\)}{\(\Phi\)} is some \eiw{0} the agent concludes \(\pv{\phi}{v}\) from \(\Phi\).
\end{note}


\begin{note}
  A simple case of a \wit{} is when an agent concludes \(\pv{\phi}{v}\) from \(\Phi\).
  For, by \supportI{} \fofb{\(\pv{\phi}{v}\)}{\(\Phi\)} from the \agpe{}, and the event trivially satisfies both requirements of \autoref{def:witnessing}.
\end{note}



\section{\supportII{}}
\label{cha:ros:II}


\begin{note}
  \supportII{} characterises when \fofb{some \prop{0}-\val{0} pair}{some \pool{}} from an \agpe{} \emph{without} a \wit{0} --- without an \eiw{0} the agent concludes the \prop{0}-\val{0} pair from the \pool{}.

  \begin{ridea}{idea:support:possible}{\supportII{}}%
    \vspace{-\baselineskip}
    \begin{itenum}
    \item[\emph{If}:]
      \(\pv{\phi}{v}\) is a \fc{0} from \(\Phi\) for \vAgent{} through~\(\ed{}\).
    \item[\emph{Then}:]
      \fofb{\(\pv{\phi}{v}\)}{\(\Phi\)} from \agpe{\vAgent{}'} through~\(\ed{}\).
    \end{itenum}
    \vspace{-\baselineskip}
  \end{ridea}

  \noindent%
  \supportII{}, states \(\pv{\phi}{v}\) being a \fc{} from \(\Phi\) is sufficient for \(\pv{\phi}{v}\) to \fof{} \(\Phi\) from the \agpe{}.
\end{note}


\begin{note}
  \noindent%
  Expanded with a simplification of \autoref{def:fc} (\autopageref{def:fc}), \supportII{} reads:

  \begin{itenum}
  \item[\emph{If}:]
    \begin{itemize}
    \item
      Through \(\ed{}\) \vAgent{} may do some action \(a\) such that clauses \ref{fc:simp:a} and \ref{fc:simp:b} are true:
      %
      \begin{enumerate}[label=\Alph*., ref=\Alph*]
      \item
        \label{fc:simp:a}
        The \eiw[\(\ed{+}\)]{0} \vAgent{} does \(a\) is an \eiw{0} \vAgent{} is concluding \(\pv{\phi}{v}\) from \(\Phi\).
      \item
        \label{fc:simp:b}
        For each \prop{0}-\val{0} pair \(\pv{\phi'}{v'}\) in \(\Phi\), \vAgent{} \evals{} \(\phi'\) as having value \(v'\) prior to doing \(a\).
      \end{enumerate}
    \end{itemize}
  \item[\emph{Then}:]
    \fofb{\(\pv{\phi}{v}\)}{\(\Phi\)} from \agpe{\vAgent{}'} through~\(\ed{}\).
  \end{itenum}

  \noindent%
  The motivation for \supportII{} is simple:

  If \(\pv{\phi}{v}\) is a \fc{0} from \(\Phi\) for \vAgent{} through~\(\ed{}\) then \(\edo{}\) captures something about \(\edn{}\) which makes it the case the agent may be concluding \(\pv{\phi}{v}\) from \(\Phi\).

  In other words, prior to a conclusion of \(\pv{\phi}{v}\) from \(\Phi\) there is something which ensures \(\pv{\phi}{v}\) `follows from' \(\Phi\) from the \agpe{}.
\end{note}


\begin{note}
  \supportII{} captures the idea that a \fingfr{} between \pv{\propM{\rootsCon{}}}{\valI{True}} and some \pool{} which includes their understanding of factorisation or the quadratic formula holds when the agent is concluding \pv{\propM{\rootsCon{}}}{\valI{True}} from the relevant \pool{} in \scen{1}~\ref{illu:gist:roots:F}~and~\ref{illu:gist:roots:QF}.

  Likewise, \supportII{} states a \fingfr{0} holds between the relevant \prop{0}-\val{0} pair and \pool{} for each of the \fc{1} in \autoref{cha:fcs}.%
  \footnote{
    Perhaps one may say a good puzzle of the kind given in \autoref{scen:fc:chick} is a puzzle where a \fingfr{} already holds between the intended solution and some \pool{} for the agent (after reading the puzzle) and obtaining a \wit{} for the \fingfr{} is enjoyable.
  }

  In \autoref{scen:countS}, by contrast, \supportII{} does not state \fofb{\pv{\propI{What Pritcher said was a sign}}{\valI{True}}}{some \pool{} \(\Psi\)} \emph{prior to} Fox's conclusion of \pv{\propI{What Pritcher said was a sign}}{\valI{True}} from \(\Psi\).
  For, for example, Pritcher may have said `I come from Wilau' and failed to sign and hence \pv{\propI{What Pritcher said was a sign}}{\valI{True}} was not a \fc{} from \(\Psi\).

  And, \supportII{} does not state a \fingfr{} holds between the relevant \prop{0}-\val{0} pair and \pool{} for any absent \fc{1} in \autoref{cha:fcs}.
\end{note}


\begin{note}
  Here's an additional example:

  \begin{scenario}[Fibonacci numbers]%
    \label{scen:fc:fib}%
    The Fibonacci numbers are recursively defined as follows:

    \[
      F_{n} = \left\{
        \begin{array}{ll}
          0 & \text{if } n = 0 \\
          1 & \text{if } n = 1 \\
          F_{n-1} + F_{n-2} & \text{if } n > 1 \\
        \end{array}
      \right.
    \]
  \end{scenario}

  \noindent%
  The conclusion of interest is:
  %
  \begin{enumerate}[label=C\thescenarioCounter., ref=(C\thescenarioCounter)]
  \item
    \label{scen:fc:fib:c}
    \pv{\propI{The sixth number in the Fibonacci sequence is 5}}{\valI{True}}
  \end{enumerate}
  %
  Given an understanding of \(f\) and sufficient motivation, an agent may do some action and be concluding \pv{\propI{The sixth number in the Fibonacci sequence is 5}}{\valI{True}}.
  So, \pv{\propI{The sixth number in the Fibonacci sequence is 5}}{\valI{True}} is a \fc{} from some \pool{}.
  Hence, so long as the action is available \fofb{\pv{\propI{The sixth number in the Fibonacci sequence is 5}}{\valI{True}}}{some \pool{} which includes the \agents{} understanding of \(f\)}.

  Though, \pv{\propI{The sixtieth number in the Fibonacci sequence is 1548008755920}}{\valI{True}} need not be \fc{0}.
  For, even if one uses memos, the \emph{sixtieth} number requires a lot of work, and the agent may not have sufficient resources to make the conclusion.
  So, it need not be the case that a \fingfr{} holds by \supportII{}.
\end{note}


\section{An answer to \qWhy{}}


\begin{note}
  To help understand the role of \supportII{}, we return to \autoref{illu:gist:roots:F} and argue \fingfb{\pv{\propM{\rootsCon{}}}{\valI{True}}}{some \pool{} \(\Phi\) which includes the \agents{} understanding of factorisation} answers \qWhy{}.

  \begin{application}[\qWhy{} and \autoref{illu:gist:roots:F}]%
    \label{appl:qWhyV-s1}%
    With respect to \autoref{illu:gist:roots:F}:
    %
    \begin{itemize}
    \item
      \fingfb{\pv{\propM{\rootsCon{}}}{\valI{True}}}{the \pool{} \(\Phi\) which captures the \agents{} understanding of factorisation} answers \qWhy{} (the agent concluded \pv{\propM{\rootsCon{}}}{\valI{True}} from \(\Phi\)).
    \end{itemize}
    \vspace{-\baselineskip}
  \end{application}

  \begin{dets}{appl:qWhyV-s1}
    Given \autoref{sketch:PE:cROS} (\autopageref{sketch:PE:cROS}), our task is to:
    %
    \begin{itemize}
    \item
      Find some \se{} \(\ed{\flat}\) of \(\ed{}\) such that:
      \begin{itemize}
      \item
        \(\edo{\flat}\) entails is true of \(\edn{\flat}\) that:
        \fofb{\(\pv{\psi}{v'}\)}{\(\Psi\)} through~\(\ed{\flat}\).
      \end{itemize}
    \end{itemize}
    %
    \medskip

    \noindent
    The event of interest \(\edn{\flat}\) covers Step~\ref{illu:gist:roots:F:factor} of the \agents{} reasoning.
    And, the description \(\edo{\flat}\) of interest is:
    `The agent figures out \rootsConEqExV{6}{3}{2} with the aim to identify the factors of \rootsConEq{}'.

    \autoref{obs:se-inst} (\autopageref{obs:se-inst}) argued \(\ed{\flat}\) is a \se{} of \(\ed{}\).

    Further, as \(\ed{\flat}\) is a \se{} of \(\ed{}\) it follows \(\ed{\flat}\) is such that \(\ed{}\) is in progress (Clause~\ref{assu:p:se:prog} of \autoref{def:se}, \autopageref{def:se}).
    So, by \autoref{obs:var:factor:fc} (\autopageref{obs:var:factor:fc}) \pv{\propM{\rootsCon{}}}{\valI{True}} is a \fc{} from \(\Phi\) through \(\ed{\flat}\).
    Hence, \supportII{} (\autopageref{idea:support:possible}) entails \fofb{\pv{\propM{\rootsCon{}}}{\valI{True}}}{\(\Phi\)} through \(\ed{\flat}\).

  This completes the argument, by transitivity of entailment.
  \end{dets}
\end{note}



\section*{Summary}


\begin{note}
  For the remainder of this document only the definition of a \wit{} and \supportII{} are of particular interest.
  \supportI{} ensures our assumptions about \fingfr{} are in line with the way \fingfr{1} were introduced in \autoref{cha:intro} and helps motivate \supportII{}.
\end{note}


\begin{note}
  The final section of this chapter contains a handful of optional notes.
\end{note}



\section[Some notes]{Some notes \hfill (Optional)}


\begin{note}
  This section contains four optional notes.

  The first highlights the way \fingfr{1} of interest may be restricted.

  The second contrasts our understanding of the connexion between \fingfr{1} and an \eiw{0} an agent concludes to \citeauthor{Boghossian:2014aa}'s understanding of the connexion between `support' and an \eiw{0} an agent infers.

  The third draws a parallel between our \supportII{} and \citeauthor{Goldman:1979ui}'s account of \emph{ex ante} justification.

  The fourth states we (safely) ignore the possibility of a \fingfr{} occurring within \fingfr{}.
\end{note}


\subsection*{The way conclusions are understood}


\begin{note}
  Our characterisation of \fingfr{1} by \supportI{} and \supportII{} rests on the way we understand conclusions.
  And, we place no constraints on the conclusions an agent may make.
  (See \autoref{cha:clar} for details).

  In particular, \supportI{} allows for a \fingfr{} to hold between any \prop{0}-\val{0} pair and \pool{0}, so long as the agent concludes the relevant \prop{0}-\val{0} pair from the given \pool{}.
  And, \supportII{} allows for a \fingfr{} to hold between any \prop{0}-\val{0} pair and \pool{0}, so long as the agent may be concluding the relevant \prop{0}-\val{0} pairs from the \pool{}.
  So, a \fingfr{} may hold between, e.g., \pv{\propI{Fish are mammals}}{\valI{True}} and some \pool{} from the \agpe{}.

  Still, \supportI{}, \supportII{}, and indeed the definition of a \fc{} are made with the possibility of restricting the conclusions of interest in mind.
  For example, \supportI{}, \supportII{}, and indeed the definition of a \fc{} are designed to be compatible with restricting interest to `justified', `properly based', or `good' conclusions.

  Going forward attention may be restricted to a particular \emph{type} of \fingfr{} as given by \supportI{} and \supportII{} (e.g.\ the type `good').
  The benefit of the characterisation given is simplicity --- explicitly introducing some restriction would add (significant) complexity.

  In short, of interest is only \emph{some} \fingfr{1} as given by \supportI{} and \supportII{}, rather than \emph{every} \fingfr{} given by \supportI{} and \supportII{}.
\end{note}




\subsection*{\supportI{} and \citeauthor{Boghossian:2014aa} Taking Condition}


\begin{note}
  \supportI{} is similar to, but distinct from,~\citeauthor{Boghossian:2014aa}'s Taking Condition:%
  \footnote{
    Strictly,~\citeauthor{Boghossian:2014aa} states the Taking Condition in terms of inferring.
    I.e., the Taking Condition reads: `Inferring necessarily involves the thinker \emph{taking} \dots'
    However, as \citeauthor{Boghossian:2008vf} is interested in conclusions, an \eiw{0} an agent infers (in~\citeauthor{Boghossian:2014aa} sense) is an \eiw{0} an agent concludes (in our sense).
  }

  \begin{quote}
    [An \eiw{0} an agent concludes] necessarily involves the thinker \emph{taking} his premises to support his conclusion and drawing his conclusion because of that fact.%
    \mbox{}\hfill\mbox{(\citeyear[5]{Boghossian:2014aa})}
  \end{quote}

  \noindent%
  For, `taking' is understood by \citeauthor{Boghossian:2014aa} to be component of the agent's reasoning.
  \citeauthor{Boghossian:2014aa} illustrates the Taking Condition as follows:
  %
  \begin{quote}
    On waking up one morning I recall that:

    \begin{enumerate}[label=(\arabic*), ref=(\arabic*), series=BogEx]
    \item
      \label{BogEx:1}
      It rained last night.
    \end{enumerate}

    I combine this with my knowledge that

    \begin{enumerate}[label=(\arabic*), ref=(\arabic*), resume*=BogEx]
    \item
      \label{BogEx:2}
      If it rained last night, then the streets are wet.
    \end{enumerate}

    to conclude:

    So,

    \begin{enumerate}[label=(\arabic*), ref=(\arabic*), resume*=BogEx]
    \item
      \label{BogEx:3}
      The streets are wet.
    \end{enumerate}

    [\dots M]y inferring from~\ref{BogEx:1} and~\ref{BogEx:2} to~\ref{BogEx:3} must involve my arriving at the judgment that~\ref{BogEx:3} in part \emph{because} I \emph{take} the presumed truth of~\ref{BogEx:1} and~\ref{BogEx:2} to provide support for~\ref{BogEx:3}.%
    \mbox{}\hfill\mbox{(\citeyear[2,4]{Boghossian:2014aa})}
  \end{quote}
  %
  Hence, for \citeauthor{Boghossian:2014aa}, the Taking Condition captures something \emph{in addition} to~\ref{BogEx:3} being a conclusion from a \pool{} which includes~\ref{BogEx:1} and~\ref{BogEx:2}.

  In contrast, we do not require that a \fingfr{} has any particular role \emph{for the agent} in \eiw{0} an agent concludes \(\pv{\phi}{v}\) from \(\Phi\).
  If an agent concludes~\ref{BogEx:3} from~\ref{BogEx:1} and~\ref{BogEx:2}, then a \fingfr{} holds between~\ref{BogEx:3} and \(\{\ref{BogEx:1}, \ref{BogEx:2}, \dots\}\).
  However, \fingfb{\ref{BogEx:3}}{\(\{\ref{BogEx:1}, \ref{BogEx:2}, \dots\}\)} need not itself have a role in the \agents{} conclusion of~\ref{BogEx:3} from \(\{\ref{BogEx:1}, \ref{BogEx:2}\}\).%
  \footnote{
    Also, about the type of reasoning by which the agent concludes.
    This comes from \textcite{Boghossian:2008vf,Boghossian:2012vb}.
    Rule following, taking gets account of rule.
  }

  \label{wrightSimp}%
  To illustrate, consider \citeauthor{Wright:2014tt}'s (\citeyear{Wright:2014tt}) `Simple Proposal':
  %
  \begin{quote}
    % [C]onsider instead the proposal, not that the status of the transition as inferential depends on the thinker's judgments about his reasons, but that it depends on \emph{what his reasons are}.
    % We want his acceptance of the premises to supply his \emph{actual} reasons for accepting the conclusion.
    % [\dots]
    %
    % Call this the Simple Proposal.
    % It says that a thinker infers q from p\(_{1}\) \(\cdots\) p\(_{\text{n}}\) when he accepts each of p\(_{1}\) \(\cdots\) p\(_{\text{n}}\), moves to accept q, and does so for the reason that he accepts p\(_{1}\) \(\cdots\) p\(_{\text{n}}\).%
    [A] thinker infers q from p\(_{1}\) \(\cdots\) p\(_{\text{n}}\) when he accepts each of p\(_{1}\) \(\cdots\) p\(_{\text{n}}\), moves to accept q, and does so for the reason that he accepts p\(_{1}\) \(\cdots\) p\(_{\text{n}}\).%
    \mbox{}\hfill\mbox{(\citeyear[33]{Wright:2014tt})}
  \end{quote}
  %
  \citeauthor{Wright:2014tt}'s proposal is that the relation between a conclusion and some \pool{} need not be part of what moves the agent to conclude the conclusion from the \pool{}.
  Hence, \citeauthor{Wright:2014tt} denies that reasoning must involve a state which connects premises to conclusions.
  So, \citeauthor{Wright:2014tt} denies \citeauthor{Boghossian:2008vf}'s Taking Condition on inference  (\citeyear[Cf.][33-34]{Wright:2014tt}).

  \supportI{} is compatible with \citeauthor{Wright:2014tt}'s Simple Proposal as \supportI{} only entails \fofb{\(\pv{\propM{q}}{\valI{True}}\)}{a \pool{} containing \(\pv{\propM{p_{1}}}{\valI{True}}\), \(\cdots\) \(\pv{\propM{p_{n}}}{\valI{True}}\)}.%
  \footnote{
    Note,~\supportI{} is an entailment, while \citeauthor{Wright:2014tt}'s Simple Proposal is an identity statement while \supportI{} is a sufficient condition.
    In particular, \supportI{} does not entail that concluding is nothing more than moving to accept \(\pv{\phi}{v}\) for the reason of accepting \(\Phi\).
  }\(^{,}\)%
  \footnote{
    There are various other objections to~\citeauthor{Boghossian:2014aa}'s Taking Condition.

    For example,~\citeauthor{Hlobil:2014tq} argues against the Taking Condition as it distracts from what accounts of reasoning out to explain, rather than arguing against the Taking Condition directly.
    Likewise, \citeauthor{McHugh:2016vp} present and summarise various objections to \emph{interest} with the Taking Condition.

    In particular,~\supportI{} is closer to what \citeauthor{McHugh:2016vp} term the `Consequence Condition': \textquote{Inferring q from p entails taking p to support q}.
    (\citeyear[316]{McHugh:2016vp})
    And, as \citeauthor{McHugh:2016vp} observe, the condition is \textquote{consistent with the idea that in inference we take our premises to support our conclusion just in virtue of reasoning from the former to the latter}.
    (\citeyear[316]{McHugh:2016vp})

    \citeauthor{McHugh:2016vp} suggest the arguments they consider against the Taking Condition `put pressure' on the Consequence Condition (\citeyear[327]{McHugh:2016vp}).
    However, these arguments concern interest, rather than whether condition is true.
    And, we, uh, have interest in \fingfr{1}\dots
  }
\end{note}

% \begin{note}
%   \color{red}
%   Humpty Dumpty arbitrary chooses for `glory' to mean `a nice knock-down argument' (\cite[190]{Carroll:2009aa}).
% \end{note}


\subsection*{\supportII{} and \citeauthor{Goldman:1979ui}'s account of \emph{ex ante} justification}


\begin{note}
  From a structural perspective, our approach to characterising \fingfr{1} without a given an \eiw{0} the agent concludes is similar to \citeauthor{Goldman:1979ui}'s account of \emph{ex ante} justification in terms of \emph{ex post} justification.%
  \footnote{
    \citeauthor{Goldman:1979ui}'s notions of \emph{ex ante} and \emph{ex post} justification is similar to the distinction between doxastic and propositional justification (see \cite{Firth:1978vi} and \cite[esp.\ fn.1]{Silva:2020aa}).

    Given the parallels between \fingfr{1} and \emph{ex ante} justification, may think of \fingfr{1} in line with propositional justification.
  }%
  \(^{,}\)%
  \footnote{
    \citeauthor{Turri:2010aa} (\citeyear{Turri:2010aa}) provides similar account.
  }

  Here's a paraphrase of \citeauthor{Goldman:1979ui}'s account of \emph{ex post} justification (\citeyear[cf.][21]{Goldman:1979ui}):
  \begin{quote}
    Person \emph{S} is \emph{ex post} justified in believing \emph{p} when \emph{S} believes \emph{p}, and we say \emph{S}' believing \emph{p} is~justified.
  \end{quote}
  %
  And here is \citeauthor{Goldman:1979ui}'s account of \emph{ex ante} justification:
  %
  \begin{quote}
    Person \emph{S} is \emph{ex ante} justified in believing \emph{p} at \emph{t} if and only if there is a reliable belief-forming operation available to \emph{S} which is such that if \emph{S} applied that operation to his total cognitive state at \emph{t}, \emph{S} would believe pat \emph{t}-plus-delta (for a suitably small delta) and that belief would be \emph{ex post} justified.%
    \mbox{}\hfill\mbox{(\citeyear[21]{Goldman:1979ui})}
  \end{quote}
  %
  In a broad stroke, someone is \emph{ex ante} justified in believing \emph{p} at \emph{t} just in case there is some action the person may immediately do and as a result of doing the action, the person is \emph{ex post} justified in \emph{p}.

  Likewise, if a \supportI{} characterises an \emph{ex post} \fingfr{0} and \supportII{} characterises an \emph{ex ante} \fingfr{0} we may say, in a broad stroke, an \emph{ex ante} \fingfr{} holds from an \agpe{} just in case there is some action the person may immediately do and as a result of doing the action a \emph{ex post} \fingfr{} holds from the \agpe{}.

  There is, however, a small difference:
  \citeauthor{Goldman:1979ui} defines \emph{ex ante} justification in terms of \emph{ex post} justification, whereas we only provide a sufficient condition for an \emph{ex ante} \fingfr{} in terms of an \emph{ex post} \fingfr{}.
\end{note}



\subsection*{\fingfr{3} which contain \fingfr{1}}


\begin{note}
  \fingfb{\(\pv{\phi}{v}\)}{\(\Phi\)} is a way things are.
  Therefore, it is possible for an agent to \eval{} the \prop{} \propI{\fofb{\(\pv{\phi}{v}\)}{\(\Phi\)} from \agpe{my}} as \valI{True}, \valI{Possible}, \valI{Desired}, and so on\dots.

  Hence, it is possible for both \ref{Embed:no} and \ref{Embed:yes} to occur:

  \begin{enumerate}[label=\arabic*., ref=(\arabic*)]
  \item
    \label{Embed:no}
    \fofb{\(\pv{\phi}{v}\)}{\(\Phi\)} from \agpe{an \agents{}}.
  \item
    \label{Embed:yes}
    \fofb{\(\pv{\psi}{v'}\)}{\(\Psi\)} from \agpe{an \agents{}}, where \(\Psi\) contains:

    \pv{\propI{\fofb{\(\pv{\phi}{v}\)}{\(\Phi\)} from \agpe{my}}}{\valI{True}}
  \end{enumerate}

  Now, in the case of \ref{Embed:yes}, it is not necessarily the case that \fofb{\(\pv{\phi}{v}\)}{\(\Phi\)} from \agpe{the}.
  For example, if an agent may think the answer to the puzzle of \autoref{illu:fc:chess:II} is a \fc{}, though in reality the agent isn't particularly good at thinking through chess problems.

  When we speak of \fingfr{} we do not consider \fingfr{} which are embedded within \fingfr{}.
  Hence, we do not consider \fingfr{} which are embedded within \fingfr{} as answers to \qWhy{}.%
  \footnote{
    The only consequence of this is that the counterexamples to \issueInclusion{} we develop do not involve embedded \fingfr{}.
  And, if you like, relevant definitions could be refined to explicitly exclude embedded \fingfr{1}.
  %   Though, \ref{Embed:no} and \ref{Embed:yes} may occur simultaneously, and \ref{Embed:no} may be the case, in part, due to \ref{Embed:yes} being the case.
  %   However, as this takes some effort to think about, we will not (directly, at least) consider cases where \ref{Embed:no} is the case due to \ref{Embed:yes} being the case.
  }
\end{note}






% \begin{note}
%   It need not be the case that an agent has a \wit{0} for a \ros{0} in order for \ros{} to be involved in answering \qWhyV{}.

%   For, suppose an agent does not have a \wit{0} for the \ros{} between \(\pv{\psi}{v'}\) and \(\Psi\).
%   The upshot of the distinction between~\ref{Embed:no} and~\ref{Embed:yes} is as follows:

%   \begin{itemize}
%   \item
%     If the \ros{0} of \ref{Embed:no} is, in part, an answer to \qWhyV{} then the \ros{0} is a counterexample to \issueConstraint{}.
%   \item
%     If the \ros{0} of \ref{Embed:yes} is, in part, an answer to \qWhyV{} then the \ros{0} is \emph{not} a counterexample to \issueConstraint{}.
%   \end{itemize}

%   The difference is \emph{the way} the \ros{} functions with respect to the agent pairing \(\phi\) with \(v\).
%   Whether the \ros{} functions as a premise when the agent concludes \(\pv{\phi}{v}\), or whether the \ros{} functions in a way that is different to a premise.
% \end{note}

% \begin{note}
%   \begin{itemize}
%   \item
%     \(\pv{\phi}{v}\) is supported by \(\Phi\), from the \agpe{}.
%   \item
%     \(\pv{\psi}{v'}\) is supported, in part, by [the way \(\pv{\phi}{v}\) is supported by \(\Phi\), from the \agpe{}], from the \agpe{}.
%   \item
%     The way \(\pv{\psi}{v'}\) is supported, in part, by the \agpe{} on [the way \(\pv{\phi}{v}\) is supported by \(\Phi\), from the \agpe{}].
%   \end{itemize}

%   \ros{} of \ref{Embed:no} is a \ros{} which holds from the \agpe{}
%   \ros{} of \ref{Embed:yes} is a \ros{} which holds from the \agpe{}, from the \agpe{}.
%   Expanded a little more carefully, the primary \ros{1} of \ref{Embed:no} and \ref{Embed:yes} are paraphrased as capturing:
%   \begin{itemize}
%   \item
%     The way \(\pv{\phi}{v}\) is supported by \(\Phi\), from the \agpe{}.
%   \item
%     The way \(\pv{\psi}{v'}\) is supported, in part, by [the way \(\pv{\phi}{v}\) is supported by \(\Phi\), from the \agpe{}], from the \agpe{}.
%   \end{itemize}

%   When we refer to a \ros{} reference is to a state of affairs from \agpe{our}, as in \ref{Embed:no}.
%   Shorthand, an `\rosNE{}' \ros{}.
%   And, a \ros{} from \agpe{an \agents{}} such as \ref{Embed:no} a `\rosE{}' \ros{}.

% \end{note}

% {
% \color{red}
% To refer to a \ros{} as an \rosE{}, need a second \ros{}.
% Hence, this clears things up.
% }

%   \begin{note}
%     The present section concerns, for some arbitrary proposition-value pair \(\pv{\phi}{v}\) and \pool{} \(\Psi\), the distinction between:

%     \ref{Embed:no} is a \ros{} between \(\pv{\phi}{v}\) and \(\Phi\).
%     By contrast, \ref{Embed:yes} is a \ros{0} between \(\pv{\psi}{v'}\) and \(\Psi\) which \emph{involves} a \ros{} between \(\pv{\phi}{v}\) and \(\Phi\).

%     Throughout this document our interest is with \ros{} that do not occur within some other (relevant) \ros{}.
%     In particular, we have implicitly assumed there are no \ros{} which answer \qWhy{} due to the \ros{} occurring within some other \ros{}.
%     And, the variant of \qWhyV{} introduced in \autoref{cha:var} (on \autopageref{questionWhyV}) explicitly requires this.

%     If a \ros{} between \(\pv{\phi}{v}\) and \(\Phi\) referenced due the \ros{} occurring within some other \ros{} (as in \ref{Embed:yes}), we say the reference is to an `\rosE{0}' \ros{}.
%     Otherwise, we say the reference is to an `\rosNE{0}' \ros{} (as in \ref{Embed:no}).

%     Throughout this document reference to a \ros{} is almost always clearly reference to an \rosNE{0} \ros{}.
%     Hence, we only distinguish between \rosE{0} and \rosNE{0} when some ambiguity may be present.
%   \end{note}

%   \begin{note}
%     Idea is somewhat familiar from distinction between object- and meta-language with respect to propositional logic.
%     Certain kind of equivalence between proof and conditional.
%     It is possible to find a corresponding conditional to any proof with a finite number of premises, proof captures derivation of conclusion from premises.

%     Corresponding conditional is not a premise, nor any part, of the proof.

%     For example, consider a proof from \(P\) and \(P \rightarrow Q\) to \(Q\) by conditional detachment.
%     Corresponding conditional is \((P \land (P \rightarrow Q)) \rightarrow Q\).
%     However, not part of the proof.

%     Intuitive distinction between what a proof and a conditional refer to.
%     However, informally there is no difficulty in treating a proof as a premise.
%     \(P\), and I have a proof of \(P \rightarrow Q\), therefore \(Q\).
%   \end{note}

%   \subsubsection[Definitions]{Definitions \hfill (Optional)}
%   \label{cha:var:ros:Emb:defs}

%   \begin{note}
%     Distinction between an \rosE{0} and \rosNE{0} is intuitive, but imprecise.

%     This section provide definition.

%     To keep things simple the following definition assumes:
%     \begin{itenum}
%     \item[\emph{If}:]
%       \(\phi\) having value \(v\) entails \(\phi'\) has value \(v'\), from the \agpe{}.
%     \item[\emph{Then}:]
%       For any \pool{} \(\Phi\), \pv{\phi'}{v'} is in \(\Phi\) whenever  \(\pv{\phi}{v}\) is in \(\Phi\).
%     \end{itenum}
%     E.g., if \(\pv{\phi'\text{ and }\phi''}{\valI{True}}\) is in \(\Phi\) then both \(\pv{\phi'}{\valI{True}}\) and \(\pv{\phi''}{\valI{True}}\) are in \(\Phi\).

%     \begin{definition}[Degree of a \prop{0}-\val{0} pair within a \ros{}]%
%       \label{def:embedding:degree}%
%       For a proposition-value pairs \(\pv{\psi}{v'}\), \(\pv{\phi}{v}\), \pool{} \(\Phi\), and \(i \in \mathbb{N}\):

%       \begin{itemize}
%       \item
%         \(\pv{\psi}{v'}\) has a \emph{degree \(1\)} with respect to a \ros{} between \(\pv{\phi}{v}\) and \(\Phi\) if and only if \(\pv{\psi}{v'} \in \Phi\).
%       \item
%         \(\pv{\psi}{v'}\) is has a \emph{degree \(i\)} with respect to a \ros{} between \(\pv{\phi}{v}\) and \(\Phi\) if and only if:
%         \begin{itemize}
%         \item
%           There exists some \(\pv{\theta}{v''}\) and \(\Theta\) such that:
%           \begin{itemize}
%           \item
%             \(\pv{\psi}{v'} \in \Theta\)
%           \item
%             \(\pv{\propI{A \ros{} between }\pv{\theta}{v''}\propI{ and }\Theta}{\valI{True}}\) has degree \(i - 1\) with respect to the \ros{} between \(\pv{\phi}{v}\) and \(\Phi\).
%           \end{itemize}
%         \end{itemize}
%       \end{itemize}
%       \vspace{-\baselineskip}
%     \end{definition}

%     The cases of interest to us are where \pv{\propI{A \ros{} between \(\pv{\psi}{v'}\) and \(\Psi\)}}{\valI{True}} has degree \(n\) within the \ros{} between \(\pv{\phi}{v}\) and \(\Phi\).

%     Reference to \ros{} between A \ros{} between \(\pv{\psi}{v'}\) and \(\Psi\) due to the \ros{} having degree \(n\) within the \ros{} between \(\pv{\phi}{v}\) and \(\Phi\).
%     That's reference to a \rosE{}.

%     is \rosE{0} within in a \ros{} between \(\pv{\phi}{v}\) and \(\Phi\), no matter the degree of embedding:

%     \begin{definition}[Embedding within a \ros{}]%
%       \label{def:embedding}%
%       For a proposition-value pairs \(\pv{\psi}{v'}\), \(\pv{\phi}{v}\), and a \pool{} \(\Phi\):

%       \begin{itemize}
%       \item
%         \(\pv{\psi}{v'}\) is \emph{\rosE{0}} within in a \ros{} between \(\pv{\phi}{v}\) and \(\Phi\)
%       \end{itemize}

%       \emph{If and only if:}

%       \begin{itemize}
%       \item
%         \(\pv{\psi}{v'}\) is has a degree of embedding \(i\) with respect to the \ros{} between \(\pv{\phi}{v}\) and \(\Phi\), for some \(i \in \mathbb{N}\).
%       \end{itemize}
%       \vspace{-\baselineskip}
%     \end{definition}

%     The definition of an embedding covers arbitrary proposition-value pairs.
%     However, the cases of embedding of interest to us are where \ros{1} are \rosE{0} within a \ros{}.
%     A final definition captures when this is the case:

%     \begin{definition}[A \prop{0}-\val{0} pair \rosE{0} in a \ros{1}]
%       For a proposition-value pairs \(\pv{\psi}{v'}\), \(\pv{\phi}{v}\), and \pool{1} \(\Phi\), \(\Psi\):

%       \begin{itemize}
%       \item
%         A \ros{} between \(\pv{\psi}{v'}\) and \(\Psi\) is \rosE{0} within the \ros{} between \(\pv{\phi}{v}\) and \(\Phi\).
%       \end{itemize}

%       \emph{If and only if}

%       \begin{itemize}
%       \item
%         For some proposition-value pair \(\pv{\chi}{v''}\) in \(\Phi\):
%         \begin{itemize}[noitemsep]
%         \item
%           \(\chi\) is the proposition: \propI{A \ros{} between \(\pv{\psi}{v'}\) and \(\Psi\)}.
%         \item
%           \(v''\) is the value: \valI{True}
%         \end{itemize}
%       \end{itemize}
%       \vspace{-\baselineskip}
%     \end{definition}
%   \end{note}



% \paragraph*{Denying \supportII{} \hfill (Optional)}
% \label{sec:denying-supportii}


% \begin{note}
%   \supportII{} positive condition for \ros{1} without a \wit{}.
%   This doesn't help me.
%   For, I need \ros{} for other work.
%   It's not merely the case that what I'm getting is a \ros{} without a \wit{}.
% \end{note}

% \begin{note}
%   In addition to \wit{} for a \ros{}, we also define a \pwit{} for a \ros{}:

%   \begin{definition}[A \pwit{} for a \ros{}]
%     \label{def:Pwit}%
%     \vspace{-\baselineskip}
%     \begin{itemize}
%     \item
%       An event \(\ed{\ast}\) is \emph{\pwit{0}} for a \ros{} between \(\pv{\phi}{v}\) and \(\Phi\), for \vAgent{} through event \(\ed{}\)
%     \end{itemize}

%     \emph{If and only if:}

%     \begin{itemize}
%     \item
%       \(\ed{\ast}\) is an \eiw{0} \vAgent{} is concluding \(\pv{\phi}{v}\) from \(\Phi\).
%     \item
%       \(\ed{\ast}\) occurs prior to or at the same time as \(\ed{}\).
%     \end{itemize}
%     \vspace{-\baselineskip}
%   \end{definition}

%   \noindent%
%   The definition of a \pwit{} is motivated by \assuPP{}.
%   For, if \(\ed{\ast}\) is an \eiw{0} an agent is concluding \(\pv{\phi}{v}\) from \(\Phi\) then by \assuPP{}, \(\ed{\ast}\) develops into an \eiw[\(e^{+}_{d^{+}}\)]{0} the agent concludes \(\pv{\phi}{v}\) from \(\Phi\).
% \end{note}


% \begin{note}
%   Given \autoref{def:Pwit}, an observation:

%   \begin{observation}%
%     \label{obs:supportIIplus}%
%     If deny \supportII{} then \ros{} either \wit{} or something compatible with absence of \wit{}.
%   \end{observation}

%   \begin{argument}{obs:supportIIplus}
%     Suppose all compatible with absence of a \wit{}.
%     Now,

%     Suppose \fc{} and no \ros{}.
%     Then, it must be the case that \wit{} for \ros{}.
%     For, there is some action the agent does, the agent may do the action, and the completion of the action concludes, and at this point, a \ros{} by \supportII{}.

%     Hence, it is always the case that \wit{}.
%   \end{argument}

%   Our interest is with the initial disjunct.
%   It must be the case that \wit{}.
%   We tie \ros{1} to conclusions and to avoid commitment.
%   However, free to strengthen if you like.

%   Key insight here is that you are free to strengthen, so long as compatible.
% \end{note}



%%% Local Variables:
%%% mode: latex
%%% TeX-master: "master"
%%% TeX-engine: luatex
%%% End:
