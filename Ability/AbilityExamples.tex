\chapter{Cases}
\label{cha:cases}

\section{Type of case}
\label{sec:type-case}

\section{Chess}
\label{sec:chess}

\begin{note}[Information]
\begin{enumerate}
\item\label{chess:claim:1}\label{chess:claim:1:conditional} If you are able to reason with the rules of chess, then you have the ability to demonstrate that White cannot prevent Black from occupying c4 on their (Black's) second move given the game state (described in figure~\ref{fig:chess:board}).
\end{enumerate}
\end{note}


\begin{figure}[h]
  \centering
  \mbox{ }
  \hfill
  \begin{subfigure}{.4\textwidth}
    \begin{adjustbox}{minipage=\linewidth,scale=0.7}
      \centering
      \newchessgame[
      setwhite={ka5,pa3,pb4,pc4,pe5,pf6,bg5,bh5}, %{rc1,kh1,pa2,pb2,ph2,pf6,pg6,nc7,qf7},
      addblack={pa6,pb7,pc6,pe6,pf7,kc7,nd7,nd4}, %{rg2,pb5,pe5,qd6,pa7,pb7,ra8,bc8,kd8,bf8},
      ]%
      \setchessboard{showmover=false}%
      \chessboard
    \end{adjustbox}
    \caption{
      Game state\newline
      \mbox{ }\newline
    }
    \label{fig:chess:board}
  \end{subfigure}
  \mbox{ }
  \hfill
  \mbox{ }
  \begin{subfigure}{.4\textwidth}
    \begin{adjustbox}{minipage=\linewidth,scale=0.7}
      \centering
      \newchessgame[
      setwhite={ka5,pa3,pb4,pc4,pe5,pf6,bg5,bh5}, %{rc1,kh1,pa2,pb2,ph2,pf6,pg6,nc7,qf7},
      addblack={pa6,pb7,pc6,pe6,pf7,kc7,nd7,nd4}, %{rg2,pb5,pe5,qd6,pa7,pb7,ra8,bc8,kd8,bf8},
      ]%
      \setchessboard{showmover=false}%
      \chessboard[
      arrow=latex,
      linewidth=1pt,
      shortenstart=.8ex,
      shortenend=.5ex,
      pgfstyle=straightmove,
      strokeopacity=0.4,
      fillopacity=0.4,
      color=black,
      pgfstyle=border,
      markfields={c4,a3,a5,g6,c5},
      % markmoves={b7-b6,c6-c5,d4-c2,d4-b5,d4-f5,d4-e2,d4-f3,d4-b3,d7-c5,d7-b6,d7-b8,d7-f8,d7-f6,d7-e5,d7-e5,c7-c8,c7-b8,c7-d8,c7-b6,c7-d6}%{f7-g8,f7-e6,f7-d5,f7-c4,f7-b3,f7-e8,c7-d5,c7-b5,c7-a8,c7-e8,g6-g7,a2-a3,b2-b3,c1-a1,c1-b1,c1-d1,c1-e1,c1-f1,c1-g1,h2-h3,h1-g1,c1-c2,c1-c3,c1-c4,c1-c5,c1-c6}
      ]
    \end{adjustbox}
    \caption{Example fields White cannot prevent Black from occupying after two moves.}
    \label{fig:chess:move:example}
  \end{subfigure}
  \hfill
  \mbox{ }
  \caption{Black to checkmate in four moves.\protect\footnotemark}
  \label{fig:chess}
\end{figure}
\footnotetext{
  \citeauthor{Emms:2000aa}' Puzzle 150 (\citeyear[33]{Emms:2000aa}).
  \citeauthor{Emms:2000aa} provides the following solution:
  \begin{quote}
    \variation{1... Nb6!}
    (threatening \variation{2... Nb3\#})
    \variation{2. b5}
    (or \variation{2. Bd1 Nxc4+} \variation{3. Ka4 b5\#})
    \variation{2... c5!}
    \variation{3. bxa6 Nxc4+}
    \variation{4. Ka4 b5\#}
    \textbf{(0-1)}\nolinebreak
    \mbox{}
    \hfill
    (\citeyear[46]{Emms:2000aa})
  \end{quote}
}

\begin{note}[Role of conditional]
  Claim~\ref{chess:claim:1} is distinct from the claim that the agent has the ability to demonstrate a strategy --- the unembedded consequent of claim~\ref{chess:claim:1}
\begin{enumerate}
\item\label{chess:claim:2}\label{chess:claim:1:unconditional} You are able to reason from the game state (described in figure~\ref{fig:chess:board}) to the proposition that White cannot prevent Black from occupying c4 on their (Black's) second move.
\end{enumerate}
\end{note}

\begin{note}[Reasoning sketch]
  With the above, sketch out simple reasoning.
  \begin{enumerate}[label=(S\arabic*), ref=(S\arabic*)]
  \item\label{inf:cond:ant} I have the ability to reason with the rules of chess.
  \item\label{inf:cond} If I have the ability to reason with the rules of chess then I have the ability to demonstrate the existence of a (particular) strategy.
  \end{enumerate}

  As~\ref{inf:cond:ant} is the antecedent of~\ref{inf:cond} and the consequent of~\ref{inf:cond} is novel information,~\ref{inf:cond:ant} and~\ref{inf:cond} combine to provide support for:

  \begin{enumerate}[resume, label=(S\arabic*), ref=(S\arabic*)]
  \item\label{inf:cond:conq} I have the ability to demonstrate the existence of the (particular) strategy that White cannot prevent Black from occupying c4 on their second move.
  \end{enumerate}

  Note, however, that~\ref{inf:cond:conq} is not considered a straightforward application of \emph{modus ponens}.
  Rather,~\ref{inf:cond} captures conditional information about how some ability the agent has extends, if it is indeed the ability to reason with the rules of chess.

  Sub-conclusion~\ref{inf:cond:conq}, is the antecedent of the following quantified conditional:

  \begin{enumerate}[resume, label=(S\arabic*), ref=(S\arabic*)]
  \item\label{inf:gae} For any strategy, an agent has the ability to demonstrate the existence of the strategy only if the strategy exists.
  \end{enumerate}

  Therefore:

  \begin{enumerate}[resume, label=(S\arabic*), ref=(S\arabic*)]
  \item\label{inf:sse} The (particular) strategy (that I am able to demonstrate) exists.\newline
    White cannot prevent Black from occupying c4 on their second move.
  \end{enumerate}
\end{note}

\begin{note}[How premise applies to argument]
  \ref{denied-claim} is a universal claim.
  Applies to all instances of reasoning.
  So, trace the support for general ability and information to the conclusion.
\end{note}

\subsection{Clarifying some terms}
\label{sec:clar-some-terms}

\subsubsection{Ability}
\label{sec:ability-1}

\begin{note}[Ability]
  Specific reading of ability.
  \textcite{Hackl:1998tt} distinguishes:
  {
    \small
    \begin{enumerate}
    \item John is able be a tall (in view of the evidence available).\hfill \emph{epistemic}
    \item John is able to listen to punk rock.\hfill \emph{deontic: ``allowed-to-do''}
    \item John is able to be married, according to the law.\hfill \emph{deontic: ``allowed-to-be''}
    \item John is able to jump higher than Bill.\hfill \emph{ability}
    \item John is able to see Mary from where he is standing.\hfill \emph{opportunity}
    \end{enumerate}
  }
  Only the last two readings seems natural, but all work if `able to' is read as `can'.

  We're interested in what \citeauthor{Hackl:1998tt} terms `opportunity'.
\end{note}

\begin{note}[Specific abilities]
  Applies to specific abilities, as the entailment appeals to a witnessing event.
  Analogue holds with respect to general abilities.

  General example:
  \begin{itemize}
  \item Sam has the ability to reason with the rules of chess.
  \end{itemize}
  Infer that there is a body of rules governing chess.
  If there are no rules governing chess, then the agent does not have the ability to reason with those (non-existing) rules.

  Not an instance of the potentive entailment, as this doesn't follow from what is required from a witnessing event.

  Similarly, it may be that there is an entailment from general to specific.
  \begin{itemize}
  \item If general, then specific.
  \end{itemize}
  For example, understand the general ability distributively.
  Again, not an instance of the potentive entailment, as this doesn't follow from what is required from a witnessing event.
\end{note}

\begin{note}[Paraphrasing (footnote)]
  Various paraphrases are available.
  \begin{itemize}
  \item Corey can see there is a zebra in the pen.
  \end{itemize}
  `Can' introduces the complexity that the agent may be witnessing.
  For example, we observe an excited look appear on Corey's face as they look into the pen.
  `Ah, Corey can see a zebra in the pen.'
  Corey isn't exited because they have the ability to see a zebra in the pen, rather Corey is excited because they are seeing a zebra.
\end{note}

\section{Additional sketches}
\label{sec:additional-sketches}


\newpage

\section{Sketch of cases}
\label{sec:sketch-cases}

\begin{itemize}
\item Agent has some body of support \(S\).
\item \(S\) is such the agent has not reasoned from \(S\) to \(\phi\).
\end{itemize}

Things are somewhat difficult here.
If \(A(\phi)\), then the agent has the ability, and so doesn't need any further support.
The agent only needs to perform some reasoning.
However, that reasoning may be understood as establishing new doxastic support.

So,

\begin{itemize}
\item \(S\) does not provide doxastic support for \(\phi\).
\item Or, \(S\) has not obtained doxastic support for \(\phi\).
\end{itemize}

So, the agent doesn't obtain information that they have the ability from some event witnessing the ability.
Else, the agent would have doxastic support for \(\phi\).
Some independent source of information that the agent has the ability.

\begin{enumerate}
\item\label{abGen:i} Information \(i\), such that \(A(\phi)\).
\end{enumerate}

Important is that the source of information does not also include information that \(\phi\) is the case.
Else, agent doesn't need to use \(A(\phi)\) as a premise.

For example, agent has proved in some formal system that \(\alpha \land \beta\) is a theorem.
Hence, the agent has the ability to prove that \(\alpha \land \beta\) is a theorem of the system.
From this, ability to prove that \(\alpha\) (or \(\beta\)) is a theorem.
However, the agent already has support that \(\alpha\) (or \(\beta\)) is a theorem.
Therefore, the agent does not need to appeal to their ability in order to obtain support.

Similarly, testimony presents a problem.
`You are able to prove \(\alpha\)' seems equivalent to `\(\alpha\) and you are able to prove \(\alpha\)'.
For, the testifier would not be in a position to testify that the agent is able to prove \(\alpha\) if \(\alpha\) is not the case.
Unlike the above case, ability is involved, but ability does not do the work.
Instead, it is about what must be the case in order for testimony to provided.

Suggests a general premise.

\begin{enumerate}
\item\label{abGen:i:g} If information provides (direct) support for \(A(\phi)\), then \(i\) provides support for \(\phi\).
\end{enumerate}

Comes from generalising the latter observation.
For, if the agent receives information, then some long as it provides direct support for \(A(\phi)\), then reason that \(\phi\) must be the case in order to receive the information.

Possibility of:

\begin{enumerate}
\item\label{abGen:i:g:denial} Information \(i\), does not provide (direct) support for \(A(\phi)\).
\end{enumerate}

\ref{abGen:i:g} does not require that the agent does reason to \(\phi\) based on receiving information.
For, if \(\phi\) is not explicitly mentioned, some steps are required for doxastic support.
In turn,~\ref{abGen:i:g:denial} is not required for the agent to appeal to ability.

However, if~\ref{abGen:i:g:denial} is not the case, then agent always has an option other than ability.
That is, receiving the information.
\(A(\phi)\) is never required to obtain \(\phi\).

\begin{enumerate}
\item\label{abGen:i:indirect} Information \(i\), provides (indirect) support for \(A(\phi)\).
\end{enumerate}

Examples suggest that this is the case.

[Examples]

These examples suggest:

\begin{enumerate}
\item\label{abGen:i:conditional} Information \(i\), does provides (conditional) support for \(A(\phi)\).
\end{enumerate}

So long as the agent has support for \(X\), the agent has support for \(A(\phi)\).


So, if~\ref{denied-claim}, then it's the attribute that must do the work.
However, \(i\) does not provide direct support for \(A(\phi)\).
We find conflict with~\nI{}.
The kind of conditional support provided here just is that the agent's support for the general ability would be misleading if it does not also extend to the specific case.
Therefore, it is not possible for the attribute to do the work, because the agent does not obtain support for the attribute.

\section{Case recipe}
\label{sec:case-recipe}

We look for:
\begin{enumerate}
\item Case in which ability does the work.
\item Argument that having is a problem in such a case.
\end{enumerate}

Without the first, then don't need to make the choice.
May be some other reasons.

If having is a problem, this will then conflict with the general principle.

Transmission.

\begin{itemize}
\item Some support for ability.
\item Ability entails fact.
\item Fact.
\end{itemize}

Fact in virtue of the first two.

\begin{enumerate}
\item In order for ability to do work, agent obtains ability without obtaining fact.
\end{enumerate}

\begin{itemize}
\item If agent obtains ability without obtaining fact, then this is due to extending support the agent already has for something other than ability.
\item For, if agent is not extending support for something other than ability, then the agent obtains support directly for ability.
  Yet, it then follows that fact comes in with this as a relevant precondition.
\end{itemize}

\begin{itemize}
\item So, extending support.
\end{itemize}

\begin{itemize}
\item Conditional structure.
\item If X then Y.
\item Given this information, agent is constrained.
\item Either not X or Y.
\end{itemize}

\begin{itemize}
\item This is compatible with not X.
\item For, otherwise this includes support for the antecedent, and hence support for the consequent, and in turn support for the fact.
\end{itemize}

\section{Example scenario}
\label{sec:example-cases}

Question about whether such cases exist.

\begin{itemize}
\item This happens in cases where ability is the result of being provided information about how general ability extends.
\end{itemize}

Here, then, extend general ability to specific ability.

Problem?

\begin{itemize}
\item If we're in this kind of case, then there is something difficult about the ability.
\end{itemize}

\begin{itemize}
\item If extending, then no addition of support.
\end{itemize}

\begin{itemize}
\item \(A(\phi)\), general ability.
\item Without information, no support for \(A(\psi)\) from \(A(\phi)\).
\item Information that \(A(\phi) \leadsto A(\psi)\).
\item Information that \(A(\phi) \leadsto A(\psi)\) does not provide support for \(A(\phi)\).
\item So, holding \(A(\psi)\) is the result of extending support for \(A(\phi)\), as information provides constraints on holding support for \(A(\phi)\), rather than helping with \(A(\psi)\).
\end{itemize}


Suggest no.

\section{Understanding the information provided by the informer}
\label{sec:underst-cond-inform}

\section{Different from reflection}
\label{sec:diff-from-refl}

\begin{note}
  First, dealing with different kinds of agents.
  Second, reflection is about changes.
  For the most part, we've got agent's who have not done all the reasoning they are able to, and there's no new information.

  Possible to reconstruct.
  Something like a small model.
  The agent revisiting the premises will then seem to be new information.

  However, this isn't intuitive.
  Problem is that the role of information that the agent hasn't paid attention to is nullified.
  If agent overlooks something important, there's nothing to be said.

  In short, some kind of reduction might be possible, but there doesn't seem to be a much to gain from doing so.
\end{note}


\section{Notes}
\label{sec:notes-1}

\begin{note}[To add]
    Seems there's some intuitive tension that may be found in the cases.
    For example, agent gets the information, and then finds some note that says such a strategy does not exist.
    If agent does not have support, and the note is somewhat plausible, then it seems things should be easily resolved.
    However, this doesn't seem quite right.
    So, it seems there is something to be said in terms of the agent having support.
  \end{note}
%%% Local Variables:
%%% mode: latex
%%% TeX-master: "master"
%%% End:
