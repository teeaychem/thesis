\chapter{Introduction}
\label{cha:reasoning-introduction}

\begin{note}
  The purpose of this chapter is to introduce two ideas concerning reasoning --- specifically reasoning which concludes that some proposition \(\phi\) has some value \(v\).
  In contrast to {\color{red} chapter ???}, we will begin with a pair of definitions, and the two ideas will be motivated with reference to the definitions.
\end{note}

\begin{note}
  {
    \color{red}
    The two definitions, and the corresponding ideas.
  }
\end{note}

\begin{note}
  In relation to the broad argument, \ESU{} is what we're arguing against, and \EAS{}, what we're arguing for.
  More specifically, tension between \ideaCS{}, \ESU{}, and ability.
\end{note}

\begin{note}
  Tension, well, only \ESU{} really matters.
  And, \EAS{} is just the negation.
  So, \adB{}, this is something of a detour.

  However, the detour serves two purposes.
  First, clarify \ESU{}.
  Second, \EAS{}, our goal is not only to establish tension, but also to offer a resolution to the tension by rejecting \ESU{}.
  In this respect, \adB{} is important to motivate.

  Further, stated prior, and independently.
\end{note}

\section{Ideas \& definitions}
\label{sec:ideas}

\begin{note}
  The two key ideas of this chapter are \ESU{} and \EAS{}.
\end{note}

\begin{note}
  \targetESU*
\end{note}

\begin{note}
  \goalEAS*
\end{note}

\begin{note}
  Argue that \ESU{} when combined with \ideaCS{} leads to tension in certain cases.
  \EAS{} is, roughly, the negation of \ideaCS{}.
  Our interest with \EAS{} is from the perspective of motivating a rejection of \ESU{} as a resolution to the mentioned tension.

  For the primary argument of this document, then, may focus only on \ESU{}.
  Consideration of \EAS{} may be postponed for a second pass after determining whether the tension developed is of interest.

  Still, given that \ESU{} and \EAS{} are paired, deal with them simultaneously.
  Indeed, begin by providing a abstract characterisation of reasoning which satisfies \ESU{} and \EAS{}.
  Term these two types of reasoning `\adA{}' and `\adB{}'.
\end{note}

\begin{note}
  \begin{restatable}[\adA{}]{definition}{defADA}
    \label{AR:adA}
    \label{def:adA}
    \vAgent{} concludes \(\phi\) has value \(v\) by `\adA{}' if:
    \begin{enumerate}[label=\textsf{S:\arabic*}., ref=(\textsf{S}:\arabic*)]%, resume*=adA_counter]
    \item
      \label{def:adA:psi}
      \vAgent{} concludes \(\phi\) has value \(v\) by witnessing reasoning from some  pool of premises \(\chi_{1},\dots,\chi_{k}\) with values \(v_{1},\dots,v_{k}\).
    \end{enumerate}
    \vspace{-\baselineskip}
  \end{restatable}
\end{note}

\begin{note}
  \begin{restatable}[\adB{}]{definition}{defADB}
    \label{AR:adB}
    \label{def:adB}
    Suppose:
    \begin{enumerate}[label=\textsf{I:\arabic*}., ref=(\textsf{I}:\arabic*), series=adB_counter]
    \item
      \label{def:adB:poss}
      \(\mu\) having value \(v\) ensures:
      \begin{enumerate}
      \item
        There is some pool of proposition-value pairs \(\langle \chi_{1},v_{1} \rangle,\dots,\langle \chi_{k},v_{k} \rangle\) (where \(\mu\) is not equivalent to any \(\chi_{i}\)), such that:
      \item
        It is possible for \vAgent{} to conclude \(\phi\) has value \(v\) by witnessing reasoning from \(\langle \chi_{1},v_{1} \rangle,\dots,\langle \chi_{k},v_{k} \rangle\) to \(\langle \phi,v \rangle\).
      \end{enumerate}
    \end{enumerate}
    \vAgent{} concludes \(\phi\) has value \(v\) by `\adB{}' if:
    \begin{enumerate}[label=\textsf{I}:\arabic*., ref=(\textsf{I}:\arabic*), resume*=adB_counter]
    \item
      \label{def:adB:inter}
      \vAgent{} concludes that \(\phi\) has value \(v\) by appeal to \(\chi_{1},\dots,\chi_{k}\) with respective values \(v_{1},\dots,v_{k}\) via the possibility of witnessing the relevant reasoning from \(\mu\) having value \(v\).
    \end{enumerate}
    \vspace{-\baselineskip}
  \end{restatable}

  Here, \(\mu\) is not a premise.
  The agent is concluding \(\phi\) has value \(v\) from \(\chi_{1},\dots,\chi_{k}\) having values \(v_{1},\dots,v_{k}\).
\end{note}

\begin{note}
  So, \ESU{}, this is requiring that we always go with \adA{}.
  And, \EAS{}, some instances of \adB{}.
\end{note}

%%% Local Variables:
%%% mode: latex
%%% TeX-master: "master"
%%% End: