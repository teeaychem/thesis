\chapter{Introduction}
\label{cha:reasoning-introduction}

\begin{note}
  The purpose of this chapter is to introduce two ideas concerning reasoning --- specifically reasoning which concludes that some proposition \(\phi\) has some value \(v\).
  In contrast to {\color{red} chapter ???}, we will begin with a pair of definitions, and the two ideas will be motivated with reference to the definitions.
\end{note}

\begin{note}
  {
    \color{red}
    The two definitions, and the corresponding ideas.
  }
\end{note}

\begin{note}
  In relation to the broad argument, \ESU{} is what we're arguing against, and \EAS{}, what we're arguing for.
  More specifically, tension between \ideaCS{}, \ESU{}, and ability.
\end{note}

\begin{note}
  Tension, well, only \ESU{} really matters.
  And, \EAS{} is just the negation.
  So, \adB{}, this is something of a detour.

  However, the detour serves two purposes.
  First, clarify \ESU{}.
  Second, \EAS{}, our goal is not only to establish tension, but also to offer a resolution to the tension by rejecting \ESU{}.
  In this respect, \adB{} is important to motivate.

  Further, stated prior, and independently.
\end{note}

%%% Local Variables:
%%% mode: latex
%%% TeX-master: "master"
%%% End: