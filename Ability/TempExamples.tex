\chapter{TempExamples}
\label{cha:tempexamples}

% \begin{note}[Illustration of \USE{}]
%   \begin{illustration}
%     \label{ill:rectangle:basic}
%     \mbox{}
%     \vspace{-\baselineskip}
%     \begin{enumerate}
%     \item The length of this rectangle measures \(19\text{cm}\) and the breadth of this rectangle measures \(7\text{cm}\).
%     \item So, the length of this rectangle is \(19\text{cm}\) and the breadth of this rectangle is \(7\text{cm}\).
%     \item
%       It is possible that my measuring device in inaccurate, but I purchased it from a reliable hardware store.
%     \item
%       To calculate the area of a rectangle, one multiples the length of the rectangle by breadth.
%     \item
%       \label{ill:rectangle:basic:reasoning}
%       \(19\text{cm}\) multiplied by \(7\text{cm}\) is \(133\text{cm}^{2}\).
%     \item
%       So, the area of the rectangle is \(133\text{cm}^{2}\).
%     \end{enumerate}
%     \vspace{-\baselineskip}
%   \end{illustration}
% \end{note}

\begin{note}[Gifts]
  \begin{illustration}
    \label{illu:S:gifts}
    \begin{enumerate}
    \item S knows me very well.
    \item Whatever S has gifted me satisfies some desire I have.
    \end{enumerate}
  \end{illustration}
  Perhaps wishful thinking, but fine with respect to \zS{}.

  {
    \color{red}
    Maybe useful to include, as this shows how \zS{} is kind of weak.
    The key thing to think about is whether there is something that could lead the agent to conclude that their friend doesn't know them very well.
  }

  The same basic idea underlying~\autoref{illu:S:gifts} extends to various other premise-conclusion instances.
  Consider, for example:

  \begin{enumerate}
  \item The newspaper reported \(S\) said that \(p\).
  \item \(S\) said that \(p\).
  \end{enumerate}

  It seems plausible that the reasoning from premise to conclusion assumes that the paper has a history of accurate reporting.
  And, and if available to the agent are various premises which suggest that the paper does not have such a history, then the agent may refrain from concluding \(S\) said that \(p\) from the report.
  Still, even if the agent were to relax their epistemic state so that the newspaper lacking such a history is \epVAd{}, the agent may still have sufficient premises to conclude that the newspaper does have the history, those premises may resist questioning, and so on.
  Hence, whether or not an agent has \support{} for the conclusion broadly depends on the agent's epistemic state.
  Further, the agent may also have some other premise from which to conclude \(S\) said that \(p\).
  For example, by having been in the room when \(S\) said that \(p\).
\end{note}

\begin{note}
  Pair of trivial instances.
  Inspection immediately grants.
  \begin{illustration}
    \begin{figure}[h!]
    \mbox{}\hfill
    \begin{subfigure}{0.45\linewidth}
      \begin{enumerate}
      \item That persons' eyes have been closed for a long time and their breathing is slow.
      \item That person is asleep.
      \end{enumerate}
      \caption{}
    \end{subfigure}
    \hfill
    \begin{subfigure}{0.45\linewidth}
      \begin{enumerate}
      \item The die has rolled even in \(5251\) of \(6000\) samples.
      \item The die is biased.
      \end{enumerate}
      \caption{}
    \end{subfigure}
    \hfill\mbox{}
    \caption{}
    \label{fig:ideaS:basic-examples}
  \end{figure}
\end{illustration}
\end{note}

\begin{note}
  Similar to verifying an algorithm may be implemented.
  Break down all of the steps in the algorithm, and then ensure that it is possible to express each of the steps in the programming language of choice.

  \begin{quote}
    \textsc{factorial}(\(n\)):\newline
    \textbf{if} \(n = 1\)\newline
    \mbox{}\indent \textbf{return} \(1\)\newline
    \textbf{else}\newline
    \mbox{}\indent \textbf{return} \(n \times\) \textsc{factorial}(\(n-1\))
  \end{quote}

  Fortran 77 does not support recursion, a function may not call an instance of itself.\nolinebreak
  \footnote{
    This is not to say that one may not compute factorials using Fortran 77.
    It's a Turing complete language.
    However, would require a different (non-recursive) algorithm.
  }
  By contrast, the recursive factorial algorithm may implemented in languages that support recursion, such as Lisp or Python.
\end{note}

% % % Spot the difference % % %


% \begin{note}[Failure of \ptivity{}]
%   {
%     \color{red}
%     I like this example, but I don't think it's worth the effort to motivate this properly.
%   }
%   Distinct between representation and what is represented.

%   So, clock.

%   Look at the clock.
%   Time.
%   Late.
%   Start rushing.

%   Is it 9:15a, or perspective?

%   Clocks can be wrong.
%   Feels the clock may be wrong.
%   Looks around apartment for their watch (they lost it).
%   Hmpf.

%   Doesn't matter.
%   Only have the clock to go by.

%   Strengthen, doubts about the clock.
%   But, do choice.

%   From perspective, time in 9:15a.
%   But, this can't be quite right.
%   Nothing beyond perspective.
%   Operative, links to action of rushing around.
%   But, perspective is not that the time is 9:15a because I don't do anything other than treat the time as being 9:15a.

%   Here, I think it's plausible that from the agent's perspective, 9:15a, but at the same time, the agent's reasoning is sufficiently detached from whether it is 9:15a.
%   Perspective matters, but content of perspective does not matter.
% \end{note}

\paragraph{Has concluded example}

\begin{note}
  For example, suppose I concluded that Riga is the capital of Latvia by studying a map.
  Now, at present, I am asked by a friend what the capital of Latvia is.
  I takes me a moment, but I recall from memory studying the map and concluding that Riga is the capital of Latvia.
  Hence, I (re-)conclude that Riga is the capital of Latvia, and I do so from the relevant premises involved when studying the map.
\end{note}

\paragraph{A trip to the zoo}

\begin{note}[Failure but no option]
  \citeauthor{Dretske:1970to}.
  \begin{scenario}[A trip to the zoo]\mbox{ }
    \label{scen:trip-to-zoo}
    \vspace{-\baselineskip}
    \begin{quote}
      You take your son to the zoo, see several zebras, and, when questioned by your son, tell him they are zebras.
      Do you know they are zebras?
      [\dots]
      We know what zebras look like, and, besides, this is the city zoo and the animals are in a pen clearly marked ``Zebras.''
      Yet, something's being a zebra implies that it is not a mule and, in particular, not a mule cleverly disguised by the zoo authorities to look like a zebra.
      Do you know that these animals are not mules cleverly disguised by the zoo authorities to look like zebras?\newline
      \mbox{ }\hfill\mbox{(\citeyear[1015--1016]{Dretske:1970to})}
    \end{quote}
    \vspace{-\baselineskip}
  \end{scenario}

  \autoref{scen:trip-to-zoo} is framed in terms of knowledge, and is designed to raise a problem for conclude of knowledge under known entailment.
  Intuitively, you know the animals in the pen are zebras.
  And, you know the following conditional is true:
  The animals in the pen are zebras \emph{only if} the animals in the pen are not cleverly disguised mules.
  However, you (intuitively) don't know the animals in the pen are not cleverly disguised mules.

  If knowledge is closed under known entailment, then you:

  \begin{enumerate}
  \item
    \(\phi\) has value \(v\) only if \(\psi\) has value \(v'\)
  \end{enumerate}

  then, if

  \begin{enumerate}
  \item
    \(\phi\) has value \(v\), then
  \end{enumerate}

  \begin{enumerate}
  \item
    \(\psi\) has value \(v'\)
  \end{enumerate}

  So, if closure of knowledge under know entailment, and conclusion is that you know, then any further reasoning, need to conclude.

  So, right now, from the current \poP{}, not counting additional information.

  Intuitively, don't know, and there is no \pevent{}.
  However, this is not due to.

  For, holds regardless of whether or not.
\end{note}

\begin{note}
  Still, knowledge.

  Not concluding unless knows.
  Then, if no \pevent{}, then agent doesn't know.
  {
    \color{red}
    I should come back to this point.
    For, it does seem a weakness.
    One could deny that knowledge ever interacts in this way.
    For example, possibility of no external world.

    However, if this is a concern, then it's not clear to me that there is anything interesting to be said about concluding.
    To the extent have concluded via understanding of rules of Sudoku, know that \pevent{}.
    If worried about lack of external world, then no guarantee that concluded from rules of Sudoku.

    So, really, the point is that it seems strange to break this link.
  }

  Event does not develop unless know.
  Therefore, does not develop unless there is some \pevent{}.

  Because, if do not know then will check.
  And, if check fails then will not conclude.
\end{note}

% \begin{note}
%   \begin{illustration}[Fraction]
%     \label{illu:fc:surds}
%     \[\frac{(3 + \sqrt{3})^{2} + \sqrt{6}^{2} - (2\sqrt{3})^{2}}{2(3 + \sqrt{3})\sqrt{6}} = \frac{1}{\sqrt{2}}\]
%   \end{illustration}

%   Granting knowledge of a handful of equalities, beyond basic addition and subtraction,%
%   \footnote{
%     \(\sfrac{ab}{ac} = \sfrac{b}{c}\),
%     \(\sqrt{a b} = \sqrt{a}\sqrt{b}\), and
%     \((a + b)^{2} = (a^{2} + 2ab + b^{2})\).
%   }
%   whether or not the equation is true a \fc{}, and further the truth of the equation is a \fc{}.%
%   \footnote{
%     \label{illu:fc:surds:fn}
%     First, consider the numerator.
%     Each element of the numerator may be rewritten as follows:
%     \((3 + \sqrt{3})^{2} = 12 + 6\sqrt{3}\), \(\sqrt{6}^{2} = 6\), \((2\sqrt{3})^{2} = 12\).
%     By summing the elements we obtain \(6\sqrt{3} + 6\).
%     Hence, by rewriting, the numerator may be replaced with, \(2(3\sqrt{3} + 3)\).

%     Now consider the denominator.
%     Observe we may cancel multiplication by \(2\) from both the numerator and denominator.
%     Further, observe \(\sqrt{6} = \sqrt{2}\sqrt{3}\).
%     Hence, by distributing we  obtain, \((3\sqrt{3} + \sqrt{3}\sqrt{3})\sqrt{2}\).
%     Likewise, observe \(\sqrt{3}\sqrt{3} = \sqrt{9} = 3\).
%     Hence, by rewriting the denominator reads \((3\sqrt{3} + 3)\sqrt{2}\).
%     As both the numerator and denominator contain \((3\sqrt{3} + 3)\), we may cancel to obtain the desired equality.
%   }
%   Though, if like me you may end up exploring a handful of unsuccessful ideas before stumbling across the path to the solution.
%   In particular, don't conclude any intermediary miscalculations.
%   And, keep going until solution is clear.
% \end{note}

%%% Local Variables:
%%% mode: latex
%%% TeX-master: "master"
%%% End:
