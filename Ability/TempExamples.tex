\chapter{TempExamples}
\label{cha:tempexamples}

% \begin{note}[Illustration of \USE{}]
%   \begin{illustration}
%     \label{ill:rectangle:basic}
%     \mbox{}
%     \vspace{-\baselineskip}
%     \begin{enumerate}
%     \item The length of this rectangle measures \(19\text{cm}\) and the breadth of this rectangle measures \(7\text{cm}\).
%     \item So, the length of this rectangle is \(19\text{cm}\) and the breadth of this rectangle is \(7\text{cm}\).
%     \item
%       It is possible that my measuring device in inaccurate, but I purchased it from a reliable hardware store.
%     \item
%       To calculate the area of a rectangle, one multiples the length of the rectangle by breadth.
%     \item
%       \label{ill:rectangle:basic:reasoning}
%       \(19\text{cm}\) multiplied by \(7\text{cm}\) is \(133\text{cm}^{2}\).
%     \item
%       So, the area of the rectangle is \(133\text{cm}^{2}\).
%     \end{enumerate}
%     \vspace{-\baselineskip}
%   \end{illustration}
% \end{note}

\begin{note}[Gifts]
  \begin{illustration}
    \label{illu:S:gifts}
    \begin{enumerate}
    \item S knows me very well.
    \item Whatever S has gifted me satisfies some desire I have.
    \end{enumerate}
  \end{illustration}
  Perhaps wishful thinking, but fine with respect to \zS{}.

  {
    \color{red}
    Maybe useful to include, as this shows how \zS{} is kind of weak.
    The key thing to think about is whether there is something that could lead the agent to conclude that their friend doesn't know them very well.
  }

  The same basic idea underlying~\autoref{illu:S:gifts} extends to various other premise-conclusion instances.
  Consider, for example:

  \begin{enumerate}
  \item The newspaper reported \(S\) said that \(p\).
  \item \(S\) said that \(p\).
  \end{enumerate}

  It seems plausible that the reasoning from premise to conclusion assumes that the paper has a history of accurate reporting.
  And, and if available to the agent are various premises which suggest that the paper does not have such a history, then the agent may refrain from concluding \(S\) said that \(p\) from the report.
  Still, even if the agent were to relax their epistemic state so that the newspaper lacking such a history is \epVAd{}, the agent may still have sufficient premises to conclude that the newspaper does have the history, those premises may resist questioning, and so on.
  Hence, whether or not an agent has \support{} for the conclusion broadly depends on the agent's epistemic state.
  Further, the agent may also have some other premise from which to conclude \(S\) said that \(p\).
  For example, by having been in the room when \(S\) said that \(p\).
\end{note}

\begin{note}
  Pair of trivial instances.
  Inspection immediately grants.
  \begin{illustration}
    \begin{figure}[h!]
    \mbox{}\hfill
    \begin{subfigure}{0.45\linewidth}
      \begin{enumerate}
      \item That persons' eyes have been closed for a long time and their breathing is slow.
      \item That person is asleep.
      \end{enumerate}
      \caption{}
    \end{subfigure}
    \hfill
    \begin{subfigure}{0.45\linewidth}
      \begin{enumerate}
      \item The die has rolled even in \(5251\) of \(6000\) samples.
      \item The die is biased.
      \end{enumerate}
      \caption{}
    \end{subfigure}
    \hfill\mbox{}
    \caption{}
    \label{fig:ideaS:basic-examples}
  \end{figure}
\end{illustration}
\end{note}

\begin{note}
  Similar to verifying an algorithm may be implemented.
  Break down all of the steps in the algorithm, and then ensure that it is possible to express each of the steps in the programming language of choice.

  \begin{quote}
    \textsc{factorial}(\(n\)):\newline
    \textbf{if} \(n = 1\)\newline
    \mbox{}\indent \textbf{return} \(1\)\newline
    \textbf{else}\newline
    \mbox{}\indent \textbf{return} \(n \times\) \textsc{factorial}(\(n-1\))
  \end{quote}

  Fortran 77 does not support recursion, a function may not call an instance of itself.\nolinebreak
  \footnote{
    This is not to say that one may not compute factorials using Fortran 77.
    It's a Turing complete language.
    However, would require a different (non-recursive) algorithm.
  }
  By contrast, the recursive factorial algorithm may implemented in languages that support recursion, such as Lisp or Python.
\end{note}

% % % Spot the difference % % %


% \begin{note}[Failure of \ptivity{}]
%   {
%     \color{red}
%     I like this example, but I don't think it's worth the effort to motivate this properly.
%   }
%   Distinct between representation and what is represented.

%   So, clock.

%   Look at the clock.
%   Time.
%   Late.
%   Start rushing.

%   Is it 9:15a, or perspective?

%   Clocks can be wrong.
%   Feels the clock may be wrong.
%   Looks around apartment for their watch (they lost it).
%   Hmpf.

%   Doesn't matter.
%   Only have the clock to go by.

%   Strengthen, doubts about the clock.
%   But, do choice.

%   From perspective, time in 9:15a.
%   But, this can't be quite right.
%   Nothing beyond perspective.
%   Operative, links to action of rushing around.
%   But, perspective is not that the time is 9:15a because I don't do anything other than treat the time as being 9:15a.

%   Here, I think it's plausible that from the agent's perspective, 9:15a, but at the same time, the agent's reasoning is sufficiently detached from whether it is 9:15a.
%   Perspective matters, but content of perspective does not matter.
% \end{note}

\paragraph{Has concluded example}

\begin{note}
  For example, suppose I concluded that Riga is the capital of Latvia by studying a map.
  Now, at present, I am asked by a friend what the capital of Latvia is.
  I takes me a moment, but I recall from memory studying the map and concluding that Riga is the capital of Latvia.
  Hence, I (re-)conclude that Riga is the capital of Latvia, and I do so from the relevant premises involved when studying the map.
\end{note}

%%% Local Variables:
%%% mode: latex
%%% TeX-master: "master"
%%% End:
