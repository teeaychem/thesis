\chapter{TempExamples}
\label{cha:tempexamples}

\begin{note}[Gifts]
  \begin{illustration}
    \label{illu:S:gifts}
    \begin{enumerate}
    \item S knows me very well.
    \item Whatever S has gifted me satisfies some desire I have.
    \end{enumerate}
  \end{illustration}
  Perhaps wishful thinking, but fine with respect to \zS{}.

  {
    \color{red}
    Maybe useful to include, as this shows how \zS{} is kind of weak.
    The key thing to think about is whether there is something that could lead the agent to conclude that their friend doesn't know them very well.
  }

  The same basic idea underlying~\autoref{illu:S:gifts} extends to various other premise-conclusion instances.
  Consider, for example:

  \begin{enumerate}
  \item The newspaper reported \(S\) said that \(p\).
  \item \(S\) said that \(p\).
  \end{enumerate}

  It seems plausible that the reasoning from premise to conclusion assumes that the paper has a history of accurate reporting.
  And, and if available to the agent are various premises which suggest that the paper does not have such a history, then the agent may refrain from concluding \(S\) said that \(p\) from the report.
  Still, even if the agent were to relax their epistemic state so that the newspaper lacking such a history is \epVAd{}, the agent may still have sufficient premises to conclude that the newspaper does have the history, those premises may resist questioning, and so on.
  Hence, whether or not an agent has \support{} for the conclusion broadly depends on the agent's epistemic state.
  Further, the agent may also have some other premise from which to conclude \(S\) said that \(p\).
  For example, by having been in the room when \(S\) said that \(p\).
\end{note}

\begin{note}
  Pair of trivial instances.
  Inspection immediately grants.
  \begin{illustration}
    \begin{figure}[h!]
    \mbox{}\hfill
    \begin{subfigure}{0.45\linewidth}
      \begin{enumerate}
      \item
        That persons' eyes have been closed for a long time and their breathing is slow.
      \item
        That person is asleep.
      \end{enumerate}
      \caption{}
    \end{subfigure}
    \hfill
    \begin{subfigure}{0.45\linewidth}
      \begin{enumerate}
      \item
        The die has rolled even in \(5251\) of \(6000\) samples.
      \item
        The die is biased.
      \end{enumerate}
      \caption{}
    \end{subfigure}
    \hfill\mbox{}
    \caption{}
    \label{fig:ideaS:basic-examples}
  \end{figure}
\end{illustration}
\end{note}

% % % Spot the difference % % %

\paragraph{A trip to the zoo}

\begin{note}[Failure but no option]
  \citeauthor{Dretske:1970to}.
  \begin{scenario}[A trip to the zoo]\mbox{ }
    \label{scen:trip-to-zoo}
    \vspace{-\baselineskip}
    \begin{quote}
      You take your son to the zoo, see several zebras, and, when questioned by your son, tell him they are zebras.
      Do you know they are zebras?
      [\dots]
      We know what zebras look like, and, besides, this is the city zoo and the animals are in a pen clearly marked ``Zebras.''
      Yet, something's being a zebra implies that it is not a mule and, in particular, not a mule cleverly disguised by the zoo authorities to look like a zebra.
      Do you know that these animals are not mules cleverly disguised by the zoo authorities to look like zebras?\newline
      \mbox{ }\hfill\mbox{(\citeyear[1015--1016]{Dretske:1970to})}
    \end{quote}
    \vspace{-\baselineskip}
  \end{scenario}

  \autoref{scen:trip-to-zoo} is framed in terms of knowledge, and is designed to raise a problem for conclude of knowledge under known entailment.
  Intuitively, you know the animals in the pen are zebras.
  And, you know the following conditional is true:
  The animals in the pen are zebras \emph{only if} the animals in the pen are not cleverly disguised mules.
  However, you (intuitively) don't know the animals in the pen are not cleverly disguised mules.

  If knowledge is closed under known entailment, then you:

  \begin{enumerate}
  \item
    \(\phi\) has value \(v\) only if \(\psi\) has value \(v'\)
  \end{enumerate}

  then, if

  \begin{enumerate}
  \item
    \(\phi\) has value \(v\), then
  \end{enumerate}

  \begin{enumerate}
  \item
    \(\psi\) has value \(v'\)
  \end{enumerate}

  So, if closure of knowledge under know entailment, and conclusion is that you know, then any further reasoning, need to conclude.

  So, right now, from the current \poP{}, not counting additional information.

  Intuitively, don't know, and there is no \pevent{}.
  However, this is not due to.

  For, holds regardless of whether or not.
\end{note}

\begin{note}
  Still, knowledge.

  Not concluding unless knows.
  Then, if no \pevent{}, then agent doesn't know.
  {
    \color{red}
    I should come back to this point.
    For, it does seem a weakness.
    One could deny that knowledge ever interacts in this way.
    For example, possibility of no external world.

    However, if this is a concern, then it's not clear to me that there is anything interesting to be said about concluding.
    To the extent have concluded via understanding of rules of Sudoku, know that \pevent{}.
    If worried about lack of external world, then no guarantee that concluded from rules of Sudoku.

    So, really, the point is that it seems strange to break this link.
  }

  Event does not develop unless know.
  Therefore, does not develop unless there is some \pevent{}.

  Because, if do not know then will check.
  And, if check fails then will not conclude.
\end{note}

% \begin{note}
%   \begin{illustration}[Fraction]
%     \label{illu:fc:surds}
%     \[\frac{(3 + \sqrt{3})^{2} + \sqrt{6}^{2} - (2\sqrt{3})^{2}}{2(3 + \sqrt{3})\sqrt{6}} = \frac{1}{\sqrt{2}}\]
%   \end{illustration}

%   Granting knowledge of a handful of equalities, beyond basic addition and subtraction,%
%   \footnote{
%     \(\sfrac{ab}{ac} = \sfrac{b}{c}\),
%     \(\sqrt{a b} = \sqrt{a}\sqrt{b}\), and
%     \((a + b)^{2} = (a^{2} + 2ab + b^{2})\).
%   }
%   whether or not the equation is true a \fc{}, and further the truth of the equation is a \fc{}.%
%   \footnote{
%     \label{illu:fc:surds:fn}
%     First, consider the numerator.
%     Each element of the numerator may be rewritten as follows:
%     \((3 + \sqrt{3})^{2} = 12 + 6\sqrt{3}\), \(\sqrt{6}^{2} = 6\), \((2\sqrt{3})^{2} = 12\).
%     By summing the elements we obtain \(6\sqrt{3} + 6\).
%     Hence, by rewriting, the numerator may be replaced with, \(2(3\sqrt{3} + 3)\).

%     Now consider the denominator.
%     Observe we may cancel multiplication by \(2\) from both the numerator and denominator.
%     Further, observe \(\sqrt{6} = \sqrt{2}\sqrt{3}\).
%     Hence, by distributing we  obtain, \((3\sqrt{3} + \sqrt{3}\sqrt{3})\sqrt{2}\).
%     Likewise, observe \(\sqrt{3}\sqrt{3} = \sqrt{9} = 3\).
%     Hence, by rewriting the denominator reads \((3\sqrt{3} + 3)\sqrt{2}\).
%     As both the numerator and denominator contain \((3\sqrt{3} + 3)\), we may cancel to obtain the desired equality.
%   }
%   Though, if like me you may end up exploring a handful of unsuccessful ideas before stumbling across the path to the solution.
%   In particular, don't conclude any intermediary miscalculations.
%   And, keep going until solution is clear.
% \end{note}

\begin{note}[Wally]
  \begin{illustration}[Where's Wally]
    \label{illu:CS:wheres-wally}
    \nagent{15} has a book containing numerous drawings of bustling scenes in which various characters are doing a variety of things.
    And, somewhere in each scene is a character called `Wally', identifiable by a collection of individually necessary and jointly sufficient distinguishing features.
    These features include a red and white striped jumper, blue trousers, short brown wavy hair, and so on.

    \nagent{15} has searched through one particular scene, and has identified a character with a variety of the features.
    Before concluding that the character is Wally, \nagent{15} remembers that there is a picture of Wally On the cover of the book, with all the identifying features present.

    Wally is always wearing a pair of round glasses, but this was not a feature \nagent{15} kept in mind when searching for Wally.
    So, perhaps the character \nagent{15} identified is not wearing round glasses  --- \nagent{15} only recalls the features they identified.
  \end{illustration}

  Interest is with whether \nagent{15} may conclude from the variety of features identified that the character is Wally.

  The difficulty for \nagent{15} is that if \nagent{15} were to check whether the character is wearing a pair of round glasses, and the character is not wearing a pair of round glasses, then \nagent{15} would conclude that the character is not Wally.
  Hence, a \requ{}.
  And, not a \fc{}.

  I'm going to ask whether Wally is usually carrying a cane.

  Here, keep in mind premises.
  Most plausible thing is that go back and check.
  However, this plausibly results in an additional premise.
  There is some information that is missing, and once you add it, you will conclude.
  However, not from present information.
\end{note}

\begin{note}
  Two ways in which this works.

  First, soundness of `Squish'-elimination.

  Second, from basic rules.

  \begin{center}
    \begin{fitch}
      \fa (P \rightarrow Q) \rightarrow P \\
      \fj Q \\
      \fa \fh P & \\
      \fa \fa Q & \textbf{Reit:} 2 \\
      \fa P \rightarrow Q & \(\rightarrow\)\textbf{Intro:} 3--4 \\
      \fa P & \(\rightarrow\)\textbf{Elim:} 1,5 \\
      \fa P \land Q & \(\land\)\textbf{Intro:} 2,6
    \end{fitch}
  \end{center}
\end{note}


\begin{note}[Spot the difference]
  \begin{illustration}[Spot the difference]
    \label{illu:CS:spot-the-diff}
    The agent has been working through a spot-the-difference to pass some time.

    Though the time is not completely passed, the agent examined the two images with what seems sufficient care to claim support that they have found all the differences.
    However, the agent did not keep track of the number of differences.

    The agent announces `I have found all the differences' and their companion responds `All fifteen?'

    \begin{enumerate}[label=\arabic*., ref=(I\ref{illu:CS:spot-the-diff}.\arabic*)]
      \setcounter{enumi}{-1}
    \item
      \label{illu:CS:spot-the-diff:info}
      If I have found all the differences, I have found fifteen differences.
    \end{enumerate}

    The agent then reasons as follows:

    \begin{enumerate}[label=\arabic*., ref=(I\ref{illu:CS:spot-the-diff}.\arabic*), resume]
    \item Exhaustive search.
    \item
      \label{illu:CS:spot-the-diff:all}
      I found all the differences.
    \item
      \label{illu:CS:spot-the-diff:fif}
      So, I have found fifteen differences. \hfill (From \ref{illu:CS:spot-the-diff:info} and \ref{illu:CS:spot-the-diff:all})
    \end{enumerate}
  \end{illustration}

  So, as doing the exhaustive search, get this information.

  Here, plausibly a \wit{}.
  For, proposition is just a state of affairs.
  These are the same state of affairs.

  Ugh, too messy.
\end{note}

\begin{note}
  \begin{illustration}
    A search for `Measurement Theory' via the LCC `H61 .R593' returned no results.

    Consider the possibility that library does not use DDC indexing.

    Checked the indexing system, do not conclude the library does not have a copy of `Measurement Theory'.
  \end{illustration}

  So, this is type B because there is no clear entailment.
  However, incompatible, as then may have searched.
  Indeed, this is just a variant on lost keys.
\end{note}

\subsection{Not a \requ{}}
\label{sec:not-requ}

\begin{note}
  \illu{3}~\ref{illu:lost-key}~\ref{scen:squish} are \illu{1} of \requ{1}.
  The contrast between~\ref{illu:lost-key} and~\ref{scen:squish} is whether the agent is concluding \(\pv{\phi}{v}\) from \(\Phi\) given that \(\pvp{\psi}{v'}{\Psi}\) is a \requ{} of concluding \(\pv{\phi}{v}\) from \(\Phi\).

  Still, \requ{1} are various ways in which \(\pvp{\psi}{v'}{\Psi}\) may fail to be a \requ{0} of concluding \(\pv{\phi}{v}\) from \(\Phi\).
  Consider the following \scen{0}:

  \begin{scenario}[Testimony as a layperson]
    \label{illu:testimony-layperson}
    An agent is informed that there are exactly five intermediate logics that have the interpolation property.%
    \footnote{Cf.\ \textcite{Maksimova:1977un}}
  \end{scenario}

  

  The agent does not have the means to query the proof.

    The agent concludes there are exactly five intermediate logics that have the interpolation property.

  Here, also, logic with syntax and semantics.
\end{note}

\subsubsection{Kettle logic}
\label{sec:failures-1}

{
  \color{blue}
  This isn't a particularly helpful example for \requ{1}, ass there is no general/specific tension present.
}

\begin{note}
  Typically, \(\pv{\phi}{v}\) from \(\Phi\) is not a \requ{}.
  You don't need to know that you are concluding in order to be concluding.
\end{note}

\begin{note}[A copper kettle]
  A further \illu{0} builds on a story as told by~\citeauthor{Freud:1960wx}.
  \begin{illustration}[A copper kettle]
    \label{illu:kettle}
    \mbox{ }
    \vspace{-\baselineskip}
    \begin{quote}
      `A.\ borrowed a copper kettle from B.\ and after he had returned it was sued by B.\ because the kettle now had a big hole in it which made it unusable.
      His defence was:
      ``First, I never borrowed a kettle from B.\ at all;
      secondly, the kettle had a hole in it already when I got it from him;
      and thirdly, I gave him back the kettle undamaged.%
      '''\newline
      \mbox{ }\hfill\mbox{(\citeyear[62]{Freud:1960wx})}
    \end{quote}
    An agent listens to A.'s defence, but does not conclude A.\ has provided testimony.
  \end{illustration}

  The agent's failure to conclude A.\ has provided testimony may be understood in terms of a \requ{}.
  For, A.\ has provided testimony only if what A.\ has said is true.
  And, what A.\ has said is true only if the three points of A.'s defence are jointly consistent.
  Putting these observations together, we have the following conditional:

  \begin{itemize}
  \item
    A.\ has provided testimony \emph{only if} if the three points of A.'s defence are jointly consistent.
  \end{itemize}

  The story as told by \citeauthor{Freud:1960wx} is comical.
  Still, when listening, never borrowed, but got it and gave it back.

  Failure for the agent to conclude the consequent would prevent the agent from concluding the antecedent.

  Likewise, there are only three points, and checking for consistency does not require the agent to establish whether the points are (actually) true.

  \requ{} but not \fc{}.

  Before the agent concludes A.\ has provided testimony, the agent reasons about whether the three points of A.'s defence are jointly consistent.
  After, the agent does not conclude A.\ has provided testimony.%
  \footnote{
    Any pair of points are jointly inconsistent.
    For example, consider the first and third:
    If A.\ never borrowed the kettle from B, then it is not possible for A.\ to have returned the kettle to B.
  }

  The \requ{0} identified is fairly general.
  In many cases one may only accept a story if the details are in harmony, and dissonance leads to rejection.
\end{note}

% \begin{note}
%   \begin{scenario}[A cup of tea]%
%     \label{scen:cup-of-tea}%
%     A moment ago I:
%     \begin{itemize}[noitemsep]
%     \item
%       Boiled some water.
%     \item
%       Placed a tea bag into a cup.
%     \item
%       Poured the water into the cup.
%     \item
%       Let the tea bag rest in the water for a while.
%     \item
%       Removed the tea bag.
%     \end{itemize}
%     \vspace{-\baselineskip}
%   \end{scenario}

%   \autoref{scen:cup-of-tea} captures an event in which I made a cup of tea.

%   Each step is a sub-event.
%   In particular, consider the sub-event in which I pour water into the cup.
%   It seems the truth of the following conditional more is-or-less immediate:
%   \begin{itemize}
%   \item
%     If the water isn't boiled, tea isn't made.
%   \end{itemize}
%   For, tea does not infuse in cold water.%
%   \footnote{
%     Provided I wasn't cold-brewing tea, didn't have time to boil the water after noticing it was cold, etc\dots
%   }
% \end{note}

% \begin{note}
%   There is a clear sense with \autoref{scen:cup-of-tea} captures \emph{how} I made a cup of tea.
%   Indeed, with a little adjustment the presentation of \autoref{scen:cup-of-tea} functions as a series of instructions for how to make a cup of tea.

%   Still, there is also a sense with which \autoref{scen:cup-of-tea} captures \emph{why} I made a cup of tea.
%   For, given the truth of the conditional, the event in which I boiled explains, in part, why some other event did not happen.

%   There is a sense with which the description fails to capture why I made a cup of tea.
%   For, does not capture motivation.
%   However,

%   \begin{itemize}
%   \item
%     If I don't want to make tea, tea isn't made.
%   \end{itemize}

%   Presence of a desire to make tea parallels boiling water.
%   And, build out from this.

%   \begin{itemize}
%   \item
%     If not tired, then I don't want to make tea.
%   \end{itemize}

%   Understand why an event happened by identifying features of the event such that, without those features the event does not happen.
%   Motivation is just one such feature.
% \end{note}

% \begin{note}
% The former parallels our interest with conclusions.
%   Not interested with why an agent wanted to reach a conclusion, but why an agent concluded \(\phi\) has value \(v\) as opposed to any other proposition-value pair, and why the agent concluded \(\phi\) has value \(v\) from \(\Phi\) as opposed to some other pool of premises.
% \end{note}



%%% Local Variables:
%%% mode: latex
%%% TeX-master: "master"
%%% TeX-engine: luatex
%%% End:
