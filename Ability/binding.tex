\chapter{Directions}
\label{cha:binding}

\begin{note}
  \autoref{cha:intro} introduced two questions, \qWhy{} and \qHow{}, and motivated a constraint between answers to \qWhy{} and \qHow{}.

  \autoref{cha:var} introduced \qWhyV{} and \qHowV{}, and a variant constraint.

  Three ingredients.
  \fc{3}, \tC{}, and \requ{1}.

  In this chapter, link \fc{1} and \requ{1} to \qWhyV{}.
\end{note}

\begin{note}
  Sections:
  \begin{TOCEnum}
  \item
    \dots
  \end{TOCEnum}
\end{note}


\section{\requ{3}, \qWhyV{}, and \issueConstraint{}}
\label{sec:comining-ingredients}


\subsection{\requ{3} and \qWhyV{}}

\begin{note}
  \begin{definition}[\rCon{2}]
    \label{def:rCon}
    \cenLine{
      \begin{itemize*}[noitemsep, label=\(\circ\)]
      \item
        Agent: \vAgent{}
      \item
        Propositions: \(\phi\), \(\psi\)
      \item
        Values: \(v\), \(v'\)
      \item
        \pool{3}: \(\Phi\), \(\Psi\)
      \item
        Events: \(e\), \(e^{\flat}\)
      \item
        \mbox{ }
      \end{itemize*}
    }

    \begin{itemize}
    \item
      The \emph{\rCon{0}} hold of \(e\) with respect to \(\langle \vAgent{}, \pvp{\phi}{v}{\Phi}, \pvp{\psi}{v'}{\Psi}, e^{\flat} \rangle\).
    \end{itemize}

    \emph{If and only if}:

    \begin{itemize}
    \item
      Conditions~%
      \ref{def:rCs:concludes},~%
      \ref{def:rCs:concluding},~and~%
      \ref{def:rCs:requ}~%
      jointly hold:
    \begin{enumerate}[label=\arabic*., ref=(\arabic*)]
      \item
        \label{def:rCs:concludes}
        \(e\) is an event in which \vAgent{} concludes \(\pv{\phi}{v}\) from \(\Phi\).
      \item
        \label{def:rCs:concluding}
        \(e^{\flat}\) is a sub-event of \(e\) in which \vAgent{} is concluding \(\pv{\phi}{v}\) from \(\Phi\).
      \item
        \label{def:rCs:requ}
        \(\pvp{\psi}{v'}{\Psi}\) is a \requ{} of \(e^{\flat}\).
      \end{enumerate}
    \end{itemize}
    \vspace{-1.5\baselineskip}
  \end{definition}
\end{note}


\begin{note}
  Selection, link \requ{1} to answers to \qWhyV{}.
  Recall:

  \reQuestion{questionWhyV}
\end{note}

\begin{note}
  \begin{proposition}[\requ{3} and \qWhyV{}]
    \label{prop:requ-WhyV}
    \cenLine{
      \begin{itemize*}[noitemsep, label=\(\circ\)]
      \item
        Agent: \vAgent{}
      \item
        Propositions: \(\phi\), \(\psi\)
      \item
        Values: \(v\), \(v'\)
      \item
        \pool{3}: \(\Phi\), \(\Psi\)
      \item
        Events: \(e\), \(e^{\flat}\)
      \item
        \mbox{ }
      \end{itemize*}
    }

    \begin{itenum}
    \item[\emph{If}:]
      The \rCon{0} hold of \(e\) with respect to \(\langle \vAgent{}, \pvp{\phi}{v}{\Phi}, \pvp{\psi}{v'}{\Psi}, e^{\flat} \rangle\).
    \item[\emph{Then}:]
        \(\pvp{\psi}{v'}{\Psi}\), in part, answers \qWhyV{}.
    \end{itenum}
    \vspace{-\baselineskip}
  \end{proposition}

  Here, two important things are a and b.
  c clause is there to exclude cases in which agent concludes without concluding.

  \begin{argument}{prop:requ-WhyV}
    Suppose \ref{def:rCs:concludes},~\ref{def:rCs:concluding}, and~\ref{def:rCs:requ} hold.


    Clause~\ref{def:rCs:requ}.
    So, if concluding, then \fc{}.
    And, by Clause~\ref{def:rCs:concluding}, concluding.
    So, \fc{}.
    And, so by \autoref{prop:fcs-only-if-pot-support}, a \ros{}.

    Now, consider \(d\) such that \requ{}.

    Suppose \ros{} fails to hold.
    Then, by \autoref{prop:fcs-only-if-pot-support}, not \fc{}.
    Therefore, not concluding.
    So, does not develop.

    So, got the conditional.
    By assumption concluding, hence get answer.
  \end{argument}
\end{note}

\begin{note}
  Key.

  Intuition is that \tC{}.
  Therefore, matters whether or not \fc{}.
  Hence, matters whether or not \ros{}.
\end{note}

\subsection{\requ{3} and \issueConstraint{}}
\label{cha:binding:sec:requ-iC}

\begin{note}
  With \autoref{prop:requ-WhyV} we observe the way in which \tC{0}, \fc{1}, and \requ{1} may come together to provide counterexamples to \issueConstraint{}.

  In short, we need some instance in which an agent concludes \(\pv{\phi}{v}\) from \(\Phi\) such that \(\pvp{\psi}{v'}{\Psi}\) is a \requ{} of the agent concluding \(\pv{\phi}{v}\) from \(\Phi\), yet the agent does not have a \wit{} for the \ros{} between \(\pv{\phi}{v}\) and \(\Psi\).

  In full:

  \begin{proposition}[\requ{3} and \issueConstraint{}]
    \label{prop:requ-WhyV-ces}
    \cenLine{
      \begin{itemize*}[noitemsep, label=\(\circ\)]
      \item
        Agent: \vAgent{}
      \item
        Propositions: \(\phi\), \(\psi\)
      \item
        Values: \(v\), \(v'\)
      \item
        \pool{3}: \(\Phi\), \(\Psi\)
      \item
        Events: \(e\), \(e^{\flat}\)
      \item
        \mbox{ }
      \end{itemize*}
    }

    \begin{itenum}
    \item[\emph{If}:]
      The \rCon{0} hold of \(e\) with respect to \(\langle \vAgent{}, \pvp{\phi}{v}{\Phi}, \pvp{\psi}{v'}{\Psi}, e^{\flat} \rangle\).
    \item[\emph{And}:]
      \label{prop:requ-WhyVCes:noW}
      \vAgent{} does not have a \wit{} for a \ros{} between \(\pv{\phi}{v}\) and \(\Psi\) when \vAgent{} concludes \(\pv{\phi}{v}\) from \(\Phi\).
    \item[\emph{Then}:]
      \(\pvp{\psi}{v'}{\Psi}\) is a counterexample to \issueConstraint{}.
    \end{itenum}
    \vspace{-\baselineskip}
  \end{proposition}

  \begin{argument}{prop:requ-WhyV-ces}
    \ref{def:rCs:concludes},~\ref{def:rCs:concluding},~\ref{def:rCs:requ} parallel the same conditions from \autoref{prop:requ-WhyV}.
    Therefore, by~\autoref{prop:requ-WhyV}, \(\pvp{\psi}{v'}{\Psi}\) answers, in part, \qWhyV{}.

    However, by~\ref{prop:requ-WhyVCes:noW}, the agent does not have a \wit{}.

    Therefore, answer, in part, to \qWhyV{} such that no \wit{}.
  \end{argument}
\end{note}

\begin{note}
  \autoref{prop:requ-WhyV-ces} does not guarantee the existence of counterexamples to \issueConstraint{}.
  And, as we were careful not to presuppose counterexamples to \issueConstraint{} when developing \tC{0}, \fc{1}, and \requ{1}, we have no yet seen any explicit counterexamples to \issueConstraint{}.

  In \autoref{cha:ces} we will provide examples which satisfy the conditions of \autoref{prop:requ-WhyV-ces}.
\end{note}

\begin{note}
  The existence of \requ{1} are compatible with \issueConstraint{}.

  \begin{observation}%
    \label{prop:requ-not-n-ce}%
    \autoref{prop:requ-WhyV} is compatible with \issueConstraint{}.
  \end{observation}

  \begin{motivation}{prop:requ-not-n-ce}
    \requ{3} are about an agent concluding.
    \fc{} when the agent concludes.
    An agent may fail to have a \wit{} for \fc{}.

    However, \issueConstraint{}, \wit{} when the agent concludes.
    Therefore, so long as \wit{} when the agent concludes, fine.

    Two possibilities.
    \begin{enumerate}
    \item
      Agent already has a \wit{}.

      For example, first \scen{0}.
      If concluding sequence, \fc{}.
      Prior to getting \(L_{8}\), \fc{}.

      However, expectation of \wit{} when sequence.
      Indeed, \(L_{9}\) from \(L_{8} + L_{7}\).
    \item
      Agent obtains \wit{} prior to conclusion.

      Consider \autoref{scen:squish}.
      The agent has previously concluded \sqE{} is sound.
      Therefore, the agent has a \wit{} for \ros{}.
    \end{enumerate}
  \end{motivation}

  Note, however, that \autoref{prop:requ-not-n-ce} is weak.
  Compatible in terms of existential.
  Hence we only needed to show a single case in which \wit{} for \requ{}.
  This does not suggest that \requ{1} are not in tension with \issueConstraint{}.
\end{note}

\section{\tC{2}, \qWhyV{}, and \issueConstraint{}}
\label{sec:tpyically-concluding}

\begin{note}
  Here, it's the same proposition, but with conditions sufficient for \requ{}.
\end{note}

\begin{note}
  \begin{definition}[The \tCCon{0}]
    \label{def:tCCon}
        \cenLine{
      \begin{itemize*}[noitemsep, label=\(\circ\)]
      \item
        Agent: \vAgent{}
      \item
        Propositions: \(\phi\), \(\psi\)
      \item
        Values: \(v\), \(v'\)
      \item
        \pool{3}: \(\Phi\), \(\Psi\)
      \item
        Events: \(e\), \(e^{\flat}\)
      \item
        \mbox{ }
      \end{itemize*}
    }

    \begin{itemize}
    \item
      The \emph{\tCCon{0}} hold of \(e\) with respect to \(\langle \vAgent{}, \pvp{\phi}{v}{\Phi}, \pvp{\psi}{v'}{\Psi}, e^{\flat} \rangle\).
    \end{itemize}

    \emph{If and only if}:

    \begin{itemize}
    \item
      Conditions~%
      \ref{def:tCCon:C},~%
      \ref{def:tCCon:tCV},~and~%
      \ref{def:tCCon:sR}~%
      jointly hold:
      \begin{enumerate}[label=\arabic*., ref=(\arabic*)]
      \item
        \label{def:tCCon:C}
        \(e\) is an event in which \vAgent{} concludes \(\pv{\phi}{v}\) from \(\Phi\).
      \item
        \label{def:tCCon:tCV}
        \(e^{\flat}\) is a sub-event of \(e\) in which \vAgent{} is \tCV{} \(\pv{\phi}{v}\) from \(\Phi\).
      \item
        \label{def:tCCon:sR}
        Conditions%
        ~\ref{def:tCCon:sR:rep},%
        ~\ref{def:tCCon:sR:tI}, and%
        ~\ref{def:tCCon:sR:itc} %
        jointly hold:
        \begin{enumerate}[label=\alph*., ref=(\alph*)]
        \item
          \label{def:tCCon:sR:rep}
          \(T'\) is a \tRep{} of \vAgent{} \tCV{} \(\pv{\phi}{v}\) from \(\Phi\) by type \(T\) in \(e^{\flat}\).
        \item
          \label{def:tCCon:sR:tI}
          \(\pvp{\psi}{v'}{\Psi}\) is a \tI{} of \(T'\)
        \item
          \label{def:tCCon:sR:itc}
          The following conditional is true:
          \begin{itemize}
          \item[\emph{If}:]
            There is some available action \(a\) such that \vAgent{} is concluding \(\pv{\psi}{v'}\) from \(\Psi\), when \vAgent{} does \(a\).
          \item[\emph{Then}:]
            There is some available action \(a'\) such that \vAgent{} is concluding \(\pv{\psi}{v'}\) from \(\Psi\) without use of any novel information obtained by doing \(a'\), when \vAgent{} does \(a'\).
          \end{itemize}
        \end{enumerate}
      \end{enumerate}
    \end{itemize}
    \vspace{-1.5\baselineskip}
  \end{definition}

  Same as previous conditions, by now \tCV{} and \requ{} replaced with sufficient conditions for \requ{} given \tCV{}.
\end{note}

\begin{note}
  \begin{proposition}[\tCV{3} and \qWhyV{}]
    \label{prop:tCV-WhyV}
    \cenLine{
      \begin{itemize*}[noitemsep, label=\(\circ\)]
      \item
        Agent: \vAgent{}
      \item
        Propositions: \(\phi\), \(\psi\)
      \item
        Values: \(v\), \(v'\)
      \item
        \pool{3}: \(\Phi\), \(\Psi\)
      \item
        Events: \(e\), \(e^{\flat}\)
      \item
        \mbox{ }
      \end{itemize*}
    }

    \begin{itenum}
    \item[\emph{If}:]
      Typical conditions jointly hold.
    \item[\emph{Then}:]
        \(\pvp{\psi}{v'}{\Psi}\), in part, answers \qWhyV{}.
    \end{itenum}
    \vspace{-\baselineskip}
  \end{proposition}

  Instead of \requ{} directly, strengthen to \tCV{} and sufficient conditions for \requ{} given \tCV{}.

  \begin{argument}{prop:tCV-WhyV}
    By definitions.
  \end{argument}

    \begin{proposition}[\tCV{3} and \issueConstraint{}]
    \label{prop:tCV-WhyV-ces}
    \cenLine{
      \begin{itemize*}[noitemsep, label=\(\circ\)]
      \item
        Agent: \vAgent{}
      \item
        Propositions: \(\phi\), \(\psi\)
      \item
        Values: \(v\), \(v'\)
      \item
        \pool{3}: \(\Phi\), \(\Psi\)
      \item
        Events: \(e\), \(e^{\flat}\)
      \item
        \mbox{ }
      \end{itemize*}
    }

    \begin{itenum}
    \item[\emph{If}:]
      Typical conditions jointly hold.
    \item[\emph{And}:]
      No \wit{} when concludes.
    \item[\emph{Then}:]
      \(\pvp{\psi}{v'}{\Psi}\) is a counterexample to \issueConstraint{}.
    \end{itenum}
    \vspace{-\baselineskip}
  \end{proposition}
\end{note}


\section{X}
\label{sec:x}


\subsubsection{Applied}
\label{sec:consequences}


\begin{note}
  \scen{1} which satisfy conditions from \autoref{cha:requs}.
  \begin{itemize}
  \item
    \autoref{scen:squish}.

    \pv{\propI{\sqE{} is sound}}{\valI{True}} from some \pool{} \(\Psi\).

    Where, previous proof of \pv{\propI{\sqE{} is sound}}{\valI{True}} from \(\Psi\).
  \item
    \autoref{illu:gist:sudoku}.

    Interactive \scen{0}, in which  repeat reasoning to some partial or full solution to Sudoku puzzle.
  \end{itemize}
\end{note}

\begin{note}
  \autoref{scen:squish}

  It seems does explain, in part, why.
  For, rehearsing, no \ros{}, no \fc{}, then stop.

  Likewise with Sudoku.
  Not only that you concluded, but that you would repeat.

  At issue is \tC{}.
\end{note}

\begin{note}
  In both cases, \wit{}.
  Indeed, \wit{} follows from construction of \illu{1}.
  Motivating idea was of repetition.
  Repetition, requires original event.
  And, original event serves as a \wit{}.
\end{note}

\begin{note}
  It seems, also, that explain why I did not conclude lost keys.
  And, why kettle logic is bad.
\end{note}




%%% Local Variables:
%%% mode: latex
%%% TeX-master: "master"
%%% TeX-engine: luatex
%%% End:
