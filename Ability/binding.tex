\chapter{Binding}
\label{cha:binding}

\begin{note}
  \autoref{cha:introduction} introduced two questions, \qWhy{} and \qHow{}, and motivated a constraint between answers to \qWhy{} and \qHow{}.

  \autoref{cha:var:sec:vars} introduced \qWhyV{} and \qHowV{}, and a variant constraint.

  \fc{}, and \requ{}.

  In this chapter, link \fc{1} and \requ{1} to \qWhyV{}.
\end{note}

\begin{note}
  Chapter introduced \requ{1}.
  \requ{3} focus on some concrete stuff: \fc{1}.

  However, \qWhyV{} is stated with respect to \ros{}.
\end{note}


\begin{note}
  \autoref{cha:requs} introduced \requ{1}.
  In this chapter, link \requ{1} to \qWhyV{}.
\end{note}


\section{\requ{3} and \ros{1}}

\begin{note}
  We being by observing a key way in which \requ{1} interact with \ros{1}:

  \begin{proposition}[\requ{3} and \ros{1}]
    \label{prop:requ-ros}
    \begin{itemize*}[noitemsep, label=\(\circ\)]
    \item
      An agent: \vAgent{}
    \item
      Propositions: \(\phi\), \(\psi\)
    \item
      Values: \(v\), \(v'\)
    \item
      \poP{3}: \(\Phi\), \(\Psi\)
    \item
      An event: \(e\)
    \item
      \mbox{ }
    \end{itemize*}

    \begin{itemize}
    \item[\emph{If}:]
      \(\pvp{\psi}{v'}{\Psi}\) \requ{} of \(e\):
    \item[\emph{Then}:]
      \begin{itemize}
      \item[\emph{If}:]
        A \ros{} between \(\pv{\psi}{v'}\) and \(\Psi\) fails to hold from \agpe{\vAgent{}'}.
      \item[\emph{Then}:]
        \(e\) is not an event in which \vAgent{} is concluding \(\pv{\phi}{v}\) from \(\Phi\).
      \end{itemize}
    \end{itemize}
    \vspace{-\baselineskip}
  \end{proposition}

  \begin{argument}{prop:requ-ros}
    Assume \(\pvp{\psi}{v'}{\Psi}\) is a \requ{}. Assume no \ros{}.
    Therefore, not a \fc{}, from \agpe{}.
    Therefore, agent does not know \fc{}. Therefore, agent is not concluding.
  \end{argument}
\end{note}

\begin{note}
  Intuitively, \autoref{prop:requ-ros} states that, whatever \ros{1} amount to in practice, if a \ros{} fails to hold, from an \agpe{}, then no concluding.

  \ros{} capture distinctive relation between conclusion and \poP{}.
  \supportII{}, holds without \wit{}.
  \fc{}, holds.
  Hence, the proposition.
\end{note}

\section{\requ{3} and \qWhyV{}}

\begin{note}
  \questionWhyV*

  So, dependence.

  \begin{proposition}[\requ{3} and \qWhyV{}]
    \label{prop:requ-WhyV}
    \begin{itemize*}[noitemsep, label=\(\circ\)]
    \item
      An agent: \vAgent{}
    \item
      Propositions: \(\phi\), \(\psi\)
    \item
      Values: \(v\), \(v'\)
    \item
      \poP{3}: \(\Phi\), \(\Psi\)
    \item
      Events: \(e\), \(e^{\flat}\)
    \item
      \mbox{ }
    \end{itemize*}

    \begin{itemize}
    \item[\emph{If}:]
      \ref{prop:requ-WhyV:if:concludes},~\ref{prop:requ-WhyV:if:concluding}, and~\ref{prop:requ-WhyV:if:requ} hold:
      \begin{enumerate}[label=\alph*., ref=(\alph*), series=propRequWhyVSeries]
      \item
        \label{prop:requ-WhyV:if:concludes}
        \(e\) is an event in which \vAgent{} concludes \(\pv{\phi}{v}\) from \(\Phi\).
      \item
        \label{prop:requ-WhyV:if:concluding}
        \(e^{\flat}\) sub-event of \(e\) in which \vAgent{} is concluding \(\pv{\phi}{v}\) from \(\Phi\).
      \item
        \label{prop:requ-WhyV:if:requ}
        \(\pvp{\psi}{v'}{\Psi}\) is a \requ{} of \(e^{\flat}\)
      \end{enumerate}
    \item[\emph{Then}:]
      \begin{enumerate}[label=\alph*., ref=(\alph*), resume*=propRequWhyVSeries]
      \item
        \label{prop:requ-WhyV:tn:answer}
        \(\pvp{\psi}{v'}{\Psi}\), in part, answers \qWhyV{}.
      \end{enumerate}
    \end{itemize}
    \vspace{-\baselineskip}
  \end{proposition}

  \begin{argument}{prop:requ-WhyV}
    Goal is to establish conditional.

    \vAgent{} would not have concluded, as not a \fc{0}.
  \end{argument}

  \autoref{prop:requ-ros}.
  Here, concludes, so was concluding, and as concludes, it must be that the agent knew.
\end{note}

\begin{note}
    \begin{itemize}
  \item
    \ros{1} between \(\pv{\psi}{v'}\) and \(\Psi\) when pairing \(\phi\) with \(v\) failed to hold, from \agpe{\vAgent{}'}.
  \item
    \(\pv{\psi}{v'}\) failed to be a \fc{} from \(\Psi\).
  \item
    \(e\) would not have been an event in which \vAgent{} concluded \(\pv{\phi}{v}\) from \(\Phi\).
  \end{itemize}
\end{note}

\subsection{Preservation of \requ{}}
\label{sec:preservation-requ}

\begin{note}
  In cases of interest, the agent knows.
  Hence, entertain a counterfactual.
  If it is not the case that \requ{} is preserved under counterfactual assumption, then it is not clear that the agent would not have concluded.
\end{note}

\begin{note}
  Response, is that it is plausible the \requ{} holds under counterfactual assumption.
  For, \requ{} is a general property.
  Plausible that it does hold under counterfactual assumption.

  Tichy.
\end{note}

\subsection{Consequences}
\label{sec:consequences}

\begin{note}
  Here, take a look at previous \scen{1}.

  However, only those in which an agent concludes.
\end{note}

\begin{note}
  Squish.

  It seems does explain, in part, why.
  For, rehearsing, no \ros{}, no \fc{}, then stop.

  Likewise with Sudoku.
  Not only that you concluded, but that you would repeat.
\end{note}

\begin{note}
  In both cases, \wit{}.
  Indeed, \wit{} follows from construction of \illu{1}.
  Motivating idea was of repetition.
  Repetition, requires original event.
  And, original event serves as a \wit{}.
\end{note}

\begin{note}
  It seems, also, that explain why I did not conclude lost keys.
  And, why kettle logic is bad.
\end{note}

\subsection{Interest}
\label{sec:interest}

\begin{note}
  So, established dependence as captured by conditional.

  Does this really capture why the agent concluded?

  We have seen that dependence may over-generate.
  
\end{note}


\subsection{Objection}
\label{sec:objection}

\subsubsection{Subjunctive troubles}
\label{sec:subjunctive-troubles}

\begin{note}
  Non-factive explanation.

  No, the agent knows.
\end{note}

\begin{note}
  Subjunctive conditional.

  Agent knows, then may be the case that \requ{} fails.

  Maybe, not really.
  For, keep \requ{}, but vary whether or not the agent knows.
\end{note}

\subsubsection{Concluding}
\label{sec:concluding}

\begin{note}
  It is not clear the following suggestion holds.
  \begin{suggestion}
    If \(e\) is event which may develop, then \(e\) is event in which concluding.

    {
      \color{red}
      This is where the problem is.
      I don't think this is obvious.

      We have assumed that if \(e^{sub}\) \emph{is} an event of concluding, then may develop.

      However, this suggestion is the converse.
    }
  \end{suggestion}
\end{note}

\begin{note}
  For, \(e\) is an event in which the agent throws a dart at the centre of the dartboard.
  However, at the initial stage of \(e\), it is not the case that the agent is sure to throw the dart at the centre of the dartboard.

  There is a difference between how the event developed, and how the may have developed.

  So, this idea isn't right in general.

  Problem.
  For, if existential account of progressive, then get explain how things came to be simply by focusing on what happened.

  I do have an answer to this, which is that the role of the \fc{} is what happened.

  Okay, so hang on.
  For an agent to be concluding, it does not need to be the case that the event is guaranteed to develop in a way such that things work out.

  However, it does need to be the case that there is some way things work out.

  And, the way in which \requ{1} work is that failure means there is no way in which things work out.
\end{note}

\begin{note}
  There are ways to approach this:
  \begin{itemize}
  \item
    Deny the premise.

    If event in which an agent concludes, it need not be the case that the agent was concluding throughout all the relevant sub-events.

    Then, no issue is resolved.
  \item
    Deny that the event was an event in which the agent concluded.

    Specifically, narrow the relevant event.
    For, we have some initial event, which develops in two a larger event, which finishes with an event in which the agent concludes \(\pv{\phi}{v}\).
    So long as final sub-event, then the agent concludes.
    The issue only arises from the apparent link to earlier (sub-)events.
    But, if otherwise, then motivation to reject these as (sub-)events of an event in which the agent concludes.
  \item
    Shift perspective.

    What it true after the fact need not be true as things are developing.
    This is \citeauthor{Boylan:2020aa}'s approach.
  \end{itemize}

  I am inclined to consider the former.
  When speak on concluded, just picking out an event.
  Naturally event to some default event.
  However, flexible.

  So, here \citeauthor{Boylan:2020aa} and darts.
  I was able to, sure, but only after things hand developed so far.

  Additionally, consider picture.
  Drew a dog.
  However, not the case that drawing a dog throughout associated event.
  Started out as a doodle.

  Ran 10k.
  However, started out, plan was to run 5k.
  Interesting consequence here is that not running 5k.
  For, supposing extension came about by choice, something changed, and hence no force to finish at 5k.

  Running 10k includes running 5k.
  But, in a more basic case, running.
  Extends to drawing.

  In the case of concluding, have reasoning.
\end{note}

\subsection{Embedding}
\label{sec:embedding}

\begin{note}
  \ros{}.
  However, know \fc{}, therefore, it need not be the case that \ros{} answers.

  This is a delicate objection.

  For, it isn't really an objection.
  From the argument given, things work out.

  Still, a concern.
  Complex, so separate chapter.
\end{note}

%%% Local Variables:
%%% mode: latex
%%% TeX-master: "master"
%%% End:
