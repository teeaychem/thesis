\chapter{Binding}
\label{cha:binding}

\begin{note}
  So, have things in order.
  The remaining thing to do is link concluding to conclusion.

  \begin{idea}
    If \(e^{sub}\) isn't event which may develop, then \(e\) isn't an event in which the agent concludes.
    \begin{argument}
      Contradiction, otherwise.
    \end{argument}
  \end{idea}

  {
    \color{red}
    So, then, fix on a sub-event.
    Here, get a \requ{}.
    Now, we get that the \pevent{} is key.
    For, no \pevent{} then no \fc{}, no \fc{} then no knowledge of \fc{} and no knowledge of \fc{} then no extension of the event.
  }

  The key part here is that, it should be the case that this holds.

  For, if not concluding, then no development.

  If I have a \bCurb{}, then the \bCurb{} has to block the possibility of the event developing such that the agent concludes.

  I think this holds.
  Because, the apparent counterexamples all suggest a restriction on the event.
  Though, again, this can't be right.
  For, \(e\) is an event in which the agent throws a dart at the centre of the dartboard.
  However, at the initial stage of \(e\), it is not the case that the agent is sure to throw the dart at the centre of the dartboard.

  There is a difference between how the event developed, and how the may have developed.

  So, this idea isn't right in general.

  Problem.
  For, if existential account of progressive, then get explain how things came to be simply by focusing on what happened.

  I do have an answer to this, which is that the role of the \fc{} is what happened.

  Okay, so hang on.
  For an agent to be concluding, it does not need to be the case that the event is guaranteed to develop in a way such that things work out.

  However, it does need to be the case that there is some way things work out.

  And, the way in which \curb{1} and \requ{1} work is that failure means there is no way in which things work out.
\end{note}

\section{Counterexample found}
\label{sec:counterexample-found}

\begin{note}
  \autoref{cha:introduction} introduced two questions, \qWhy{} and \qHow{}, and motivated a constraint between answers to \qWhy{} and \qHow{}.

  \autoref{cha:var:sec:vars} introduced variants of \qWhy{} and \qHow{}, and a variant constraint.

  \autoref{cha:clar:sec:literature}, in addition to intuition, constraint seems to often be a theoretical assumption.

  Purpose of variants is to motivate counterexamples to constraint.
  Specifically in terms of answers to \qWhyV{} which are not answers to \qHowV{}.
  In other words, \ros{} such that \ros{} explains, in part, why agent concludes but is such that the agent does not have a \wit{} for the \ros{}.

  In this section we outline in rough form how we will (attempt) to provide counterexamples.

  In short, need:
  An agent, event in which agent concludes \(\pv{\phi}{v}\) from \(\Phi\), and \ros{} between \(\pv{\psi}{v'}\) and \(\Psi\) such that:

  \begin{itemize}
  \item
    The agent does not have a \wit{} for the \ros{} between \(\pv{\psi}{v'}\) and \(\Psi\).
  \item
    The \ros{} between \(\pv{\psi}{v'}\) and \(\Psi\), in part, answers \qWhyV{}.
  \end{itemize}

  Our goal is motivate a general method for generating examples in which some \ros{} for which an agent does not have a \wit{} such that the \ros{} answers \qWhyV{}.
\end{note}


\begin{note}
  Feedback proposition.
  And, just need it to be the case that no \wit{} for \fc{}.

  A final difficulty.
  \fc{} from \agpe{}.
  However, embed.
  \ros{} does not answer \qWhyV{}.
  Rather, premise.

  \begin{proposition}
    No embedding.
    \begin{argument}
      The worry is whether or not the agent is concluding.
      If embed, then the same worry applies.
    \end{argument}
  \end{proposition}
\end{note}

\begin{note}
  Feedback bridges the gap between \agpe{} and \ros{}
\end{note}

\begin{note}
  Non-factive.

  Well, depends on \requ{}.
  However, plausible that the agent only \emph{concludes} if know \fc{}.
  Else, agent does not conclude \(\pv{\phi}{v}\) from \(\Phi\).

  At issue here is not whether the agent concludes \(\pv{\phi}{v}\), that much is set.
  However, it's the way in which the conclusion comes about.
  The function of a \poP{} is to capture some substantial event.
\end{note}


\section{Scraps}
\label{sec:scraps}

\begin{note}
  \emph{However}, caution.
  For, as we have seen with testimony, it may be the case that status of a premises blocks a \curb{}.
  And, the argument given relies on the existence of a \curb{}.
  So, it may be the case that past reasoning blocks a \curb{}.
  Still, here, only need to deny this.
  Not saying that in every case agent's present reasoning is given priority.
  (Indeed, consider cases of being somewhat impaired, e.g., via exhaustion.
  Indeed, exhaustion is interesting.
  Basic consistency checks.
  Should be the case that conclude A, but just concluded \emph{not}-A, or something like this\dots)
  Rather, denying that past continues to secure in all instances.
  So, just need the potential to revise perspective on any previous conclusion.
\end{note}

\section{Features}
\label{sec:features}

\subsection{Factivity}
\label{sec:factivity}

\begin{note}
  Delicacy.
  \fc{1} and \ros{} from the \agpe{}.

  Two issues:

  \begin{enumerate}
  \item
    If independent of \agpe{}, then it may be the case that \fc{} without any prior recognition from agent.
  \item
    If dependent on \agpe{}, then it may be the case that not a \fc{}.
  \end{enumerate}

  Two issues are important.
  Without \agpe{}, then unclear that get \ros{} of interest for answer to \qWhyV{}.
  Given \agpe{}, unclear that we get a \ros{}.
\end{note}

\subsubsection{Perspective alone}

\begin{note}
  It may be the case that that \emph{if-then} conditional holds from the \agpe{} but does not hold independently of the \agpe{}.
  Hence, the sense of dependence captured by \qWhyV{} is not equivalent with the intuitive sense of dependence captured by considering whether or not the \emph{if-then} conditional holds independently of the \agpe{}.

  The observation that the \emph{if-then} conditional may hold from the \agpe{} while failing to hold independently of the \agpe{} is clearest when considering conditionals more general.

  For example, suppose an agent has taken a gamble on a coin landing heads.
  The coin lands heads, and the agent receives a prize.
  From the \agpe{}, if the coin failed to lands heads, then the agent would not have received the prize.
  However, the agent was set to receive the prize for participating in the gamble, regardless of whether the coin landed heads.%
  \footnote{
    The present point is similar to issues raised by \citeauthor{Harman:1973ww} (\citeyear{Harman:1973ww}) regarding the proposed equivalence between reasons for which an agent believes something with reasons the agent would offer if asked to justify their belief.
  As \citeauthor{Harman:1973ww} notes, an agent may offer reasons because they think they will convince their audience, not because the agent is compelled by the reasons, etc.
  (\citeyear[Ch.2]{Harman:1973ww})

  To the extent that \citeauthor{Harman:1973ww}'s point is that what holds from an \agpe{} need not actually be the case, the point in the same.
  However, to the extent that \citeauthor{Harman:1973ww} relies on an under-specification of what holds from an \agpe{} --- i.e.\ the distinction between whether \(\phi\) has value \(v\) from the \agpe{} or whether the agent evaluates as true the proposition that their audience is responsive to \(\phi\) having value \(v\), the point is distinct.
  }

  So, it may be the case that, though from the \agpe{} they would not have concluded \(\pv{\phi}{v}\) from \(\Phi\) if \support{} failed to hold between \(\pv{\psi}{v'}\) and \(\Psi\), the agent would have concluded \(\pv{\phi}{v}\) from \(\Phi\) regardless.
\end{note}

\begin{note}
  Consider \citeauthor{Davidson:1963aa}.
  Account of reason is in terms of attitudes.
  \emph{Not} what is the case from the \agpe{}.
  Attitudes, rather than the contents of attitudes.

  \begin{quote}
    Davidson[ asserts] a demand for a more ordinary form of explanation:
    an explanation which shows, not merely what, from another's point of view, \emph{could} count in favour of acting, but why that person did, in fact, act.%
    \mbox{}\hfill\mbox{(\cite[417]{Hieronymi:2011aa})}
  \end{quote}

  In general, difficult to make the switch.
  \textcite{Dancy:2000aa} argues for contents.
  However, problem of non-factivity.

  \citeauthor{Dancy:2000aa}'s position is difficult.
  On the one hand, reason from the \agpe{}.
  If this is the case, then compatible.
  In this respect, when shift to the \agpe{}, it is not the \agents{} belief, but what the agent believes.
  In short, reason from \agpe{} need not be the same as reason independent of \agpe{}.

  However, understanding from literature is that \citeauthor{Dancy:2000aa} holds that reason from \agpe{} \emph{is} reason.
  This, I find strange.
  And, this is not what we are interested in.
  Though, it is a delicate balance.
  Important thing to keep in mind is that we are two levels deep at this point.
  If dependence holds, along with proposition, then still in terms of \ros{} from \agpe{}.
  So, in dependence, then not pushing as far as \citeauthor{Dancy:2000aa}.
  No general entailment.
\end{note}

\begin{note}
  So, the only thing to do is ensure the \agpe{} is correct.

  However, motivation in similar style to \fc{1}.
  \fc{3} focus on whether agent would conclude.
  Pair is knowing that the agent would not conclude.
  So, in parallel fashion, knowledge.
\end{note}

\begin{note}
  So, avoid failure of factivity for a second time.
  However, problem of deviance.
  General disconnect.
  Agent is right, in general, but way in which they would not conclude is tangential.
  I have no solution to this.
  Grant that deviance does not occur.
\end{note}


\begin{note}
  \fc{3} will do the work:

  \begin{itemize}
  \item
    \ros{1} between \(\pv{\psi}{v'}\) and \(\Psi\) when pairing \(\phi\) with \(v\) failed to hold, from \agpe{\vAgent{}'}.
  \item
    \(\pv{\psi}{v'}\) failed to be a \fc{} from \(\Psi\).
  \item
    \(e\) would not have been an event in which \vAgent{} concluded \(\pv{\phi}{v}\) from \(\Phi\).
  \end{itemize}

  If \fc{} failed to hold, then no conclusion.
  Agent, implicitly, recognises link between \ros{1} and \fc{1}.
  However, given that \ros{1} are something of an abstraction, interest is really in whether \fc{}.

  So, dependence is captured from the \agpe{} and task is to construct \scen{1} such that if no \fc{0} then no conclusion.
\end{note}

\begin{note}
  \qWhyV{}, need the \ros{}.
  \ros{} by \fc{} from \agpe{}.
  Is there really a \ros{}?
\end{note}

\begin{note}
  With the exception of more-or-less instantaneous actions, future may develop in surprising ways.

  For example, plausible that an agent knows when they strike the cue ball in a certain way, a particular red ball will land in a pocket.
  However, not plausible that the agent knows where the cue ball will come to rest after the red ball lands in the pocket.
  Hence, agent does not know their following more, and so on.

  In parallel, an agent may have no guarantee that they will not be interrupted, etc.
  Hence, in most cases it seems implausible that an agent knows they will concluded.
  Yet, to be concluding does not require completion.

  With respect to \fc{}, whether event in which the agent concludes would be in progress.
\end{note}

\begin{note}
  Agent \emph{knows} \(\pv{\psi}{v'}\) from \(\Psi\) is a \fc{}.
\end{note}

\begin{note}
  This does not provide a complete solution to problem of factivity.
  For, what distinguishes one case from the other?

  However, this is nothing unique to cases under consideration, so long as relevant instances of \fc{} are plausibly knowledge.

  Though, this still differs from attitudes.
\end{note}



%%% Local Variables:
%%% mode: latex
%%% TeX-master: "master"
%%% End:
