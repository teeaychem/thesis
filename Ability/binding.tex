\chapter{Counterexamples to \issueConstraint{}}
\label{cha:binding}

\begin{note}
  The present chapter states the way in which the ingredients of \autoref{part:ing} connect with constraint \issueConstraint{} of \autoref{cha:var}.

  State a collections of conditions sufficient to ensure some \prop{0}-\val{0}-\pool{0} pair answers \qWhyV{}.
  Overall goal is counterexample to \issueConstraint{}.
  Obtain counterexample by no \wit{}.
  Collection of conditions does not entail no \wit{}.
  Rather, hint.
  \autoref{cha:ces} counterexamples.

  First collection of conditions focuses on \requ{} and concluding, rather than \tCV{}.
  More general, and avoids some complexity introduced by incorporating \tCV{}.

  The second collection of conditions adjusts the first to focus on \tCV{}.
  The benefit of \tCV{} is sufficient conditions for \requ{}.
\end{note}

\begin{note}
  There are three sections:
  \begin{TOCEnum}
  \item
    \TOCLine{sec:recollection}

    We briefly restate \qWhyV{} and \issueConstraint{}.
  \item
    \TOCLine{sec:comining-ingredients}
  \item
    \TOCLine{sec:tpyically-concluding}
  \end{TOCEnum}
\end{note}

\section{\qWhyV{} and \issueConstraint{}}
\label{sec:recollection}

\begin{note}
  The following sections appeal to the content of both \qWhyV{} and \issueConstraint{} to state and argue for a handful of proposition which relate \fc{1}, \requ{1}, and an agent \tCV{} to \qWhyV{} and \issueConstraint{}.

  For ease of reference, we briefly restate \qWhyV{} and \issueConstraint{}:

  \reQuestion{questionWhyV}

  \reConstraint{consConstraint}
\end{note}








%%% Local Variables:
%%% mode: latex
%%% TeX-master: "master"
%%% TeX-engine: luatex[]
%%% End:
