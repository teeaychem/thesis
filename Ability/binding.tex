\chapter{Mix}
\label{cha:binding}

\begin{note}
  \autoref{cha:introduction} introduced two questions, \qWhy{} and \qHow{}, and motivated a constraint between answers to \qWhy{} and \qHow{}.

  \autoref{cha:var:sec:vars} introduced \qWhyV{} and \qHowV{}, and a variant constraint.

  \fc{}, and \requ{}.

  In this chapter, link \fc{1} and \requ{1} to \qWhyV{}.
\end{note}

\begin{note}
  Chapter introduced \requ{1}.
  \requ{3} focus on some concrete stuff: \fc{1}.

  However, \qWhyV{} is stated with respect to \ros{}.
\end{note}


\begin{note}
  \autoref{cha:requs} introduced \requ{1}.
  In this chapter, link \requ{1} to \qWhyV{}.
\end{note}

\begin{note}
  Three sections:
  \begin{enumerate}[label=]
  \item
    \TOCLine{cha:binding:sec:requRos}
  \item
    \TOCLine{cha:binding:sec:requ-qWhyV}
  \item
    \TOCLine{cha:binding:sec:requ-iC}
  \end{enumerate}
\end{note}

\section{\requ{3} and \ros{1}}
\label{cha:binding:sec:requRos}

\begin{note}
  We being by observing a key way in which \requ{1} interact with \ros{1}:

  \begin{proposition}[\requ{3} and \ros{1}]
    \label{prop:requ-ros}
    \cenLine{
      \begin{itemize*}[noitemsep, label=\(\circ\)]
      \item
        Agent: \vAgent{}
      \item
        Propositions: \(\phi\), \(\psi\)
      \item
        Values: \(v\), \(v'\)
      \item
        \poP{3}: \(\Phi\), \(\Psi\)
      \item
        Event: \(e\)
      \item
        \mbox{ }
      \end{itemize*}
    }

    \begin{itemize}
    \item[\emph{If}:]
      \(\pvp{\psi}{v'}{\Psi}\) \requ{} of \(e\):
    \item[\emph{Then}:]
      \begin{itemize}
      \item[\emph{If}:]
        A \ros{} between \(\pv{\psi}{v'}\) and \(\Psi\) fails to hold, for \vAgent{}, through \(e\).
      \item[\emph{Then}:]
        \(e\) is not an event in which \vAgent{} is concluding \(\pv{\phi}{v}\) from \(\Phi\).
      \end{itemize}
    \end{itemize}
    \vspace{-\baselineskip}
  \end{proposition}

  \begin{argument}{prop:requ-ros}
    Assume \(\pvp{\psi}{v'}{\Psi}\) is a \requ{}.

    Now, suppose a \ros{} fails to hold between \(\pv{\psi}{v'}\) and \(\Psi\) in \(e\).

    By \autoref{prop:fcs-only-if-support}, \(\pv{\phi}{v}\) from \(\Psi\) is not a \fc{} for \vAgent{}.

    Therefore, by the definition of a \requ{}, the agent is not concluding.
  \end{argument}
\end{note}

\begin{note}
  Intuitively, \autoref{prop:requ-ros} states that, whatever \ros{1} amount to in practice, if a \ros{} fails to hold, from an \agpe{}, then no concluding.

  % \ros{} capture distinctive relation between conclusion and \poP{}.
  % \supportII{}, holds without \wit{}.
  % \fc{}, holds.
  % Hence, the proposition.
\end{note}

\section{\requ{3} and \qWhyV{}}
\label{cha:binding:sec:requ-qWhyV}

\begin{note}
  \questionWhyV*

  So, dependence.

  \begin{proposition}[\requ{3} and \qWhyV{}]
    \label{prop:requ-WhyV}
    \cenLine{
      \begin{itemize*}[noitemsep, label=\(\circ\)]
      \item
        Agent: \vAgent{}
      \item
        Propositions: \(\phi\), \(\psi\)
      \item
        Values: \(v\), \(v'\)
      \item
        \poP{3}: \(\Phi\), \(\Psi\)
      \item
        Events: \(e\), \(e^{\flat}\)
      \item
        \mbox{ }
      \end{itemize*}
    }

    \begin{itemize}
    \item[\emph{If}:]
      \ref{prop:requ-WhyV:if:concludes},~\ref{prop:requ-WhyV:if:concluding}, and~\ref{prop:requ-WhyV:if:requ} hold:
      \begin{enumerate}[label=\alph*., ref=(\alph*), series=propRequWhyVSeries]
      \item
        \label{prop:requ-WhyV:if:concludes}
        \(e\) is an event in which \vAgent{} concludes \(\pv{\phi}{v}\) from \(\Phi\).
      \item
        \label{prop:requ-WhyV:if:concluding}
        \(e^{\flat}\) sub-event of \(e\) in which \vAgent{} is concluding \(\pv{\phi}{v}\) from \(\Phi\).
      \item
        \label{prop:requ-WhyV:if:requ}
        \(\pvp{\psi}{v'}{\Psi}\) is a \requ{} of \(e^{\flat}\)
      \end{enumerate}
    \item[\emph{Then}:]
      \begin{enumerate}[label=\alph*., ref=(\alph*), resume*=propRequWhyVSeries]
      \item
        \label{prop:requ-WhyV:tn:answer}
        \(\pvp{\psi}{v'}{\Psi}\), in part, answers \qWhyV{}.
      \end{enumerate}
    \end{itemize}
    \vspace{-\baselineskip}
  \end{proposition}

  \begin{argument}{prop:requ-WhyV}
    Suppose \ref{prop:requ-WhyV:if:concludes},~\ref{prop:requ-WhyV:if:concluding}, and~\ref{prop:requ-WhyV:if:requ} hold.

    Now, suppose a \ros{} fails to hold between \(\pv{\psi}{v'}\) and \(\Psi\) in \(e^{\flat}\).

    \autoref{prop:requ-ros}, \(e^{\flat}\) is not an event in which \vAgent{} is concluding \(\pv{\phi}{v}\) from \(\Phi\).

    Hence, does not develop.

    By assumption, does develop.

    Hence, \ros{} holds.

    This gets us the a condition and conditional.
  \end{argument}
\end{note}

\begin{note}
  Key.

  Intuition is that \sR[concluding]{}.
  Therefore, matters whether or not \fc{}.
  Hence, matters whether or not \ros{}.
\end{note}

\begin{note}
  This is not merely some statement about \sR[concluding]{}.

  Intuitively, if \sR[concluding]{} then something about what secures the agent's concluding as \sR[concluding]{0}.

  Stronger.
  \requ{} is such that this is part of what makes the act an act of concluding.

  Put it this way.
  Concern about \sR[concluding]{0}.
  Then, it may still be the agent is concluding.

  With \requ{}, this is not the case.
\end{note}

\begin{note}
  Before turning to \issueConstraint{}, apply to the pair of \scen{1} which satisfy conditions from \autoref{cha:requs}.
\end{note}

\subsection{Applied}
\label{sec:consequences}


\begin{note}
  \scen{1} which satisfy conditions from \autoref{cha:requs}.
  \begin{itemize}
  \item
    \autoref{scen:squish}.

    \(\pv{\text{\sqE{} is sound}}{\text{True}}\) from some \poP{} \(\Psi\).
    Where, previous proof of \(\pv{\text{\sqE{} is sound}}{\text{True}}\) from \(\Psi\).
  \item
    \autoref{illu:gist:sudoku}.

    Interactive \scen{0}, in which  repeat reasoning to some partial or full solution to Sudoku puzzle.
  \end{itemize}
\end{note}

\begin{note}
  \autoref{scen:squish}

  It seems does explain, in part, why.
  For, rehearsing, no \ros{}, no \fc{}, then stop.

  Likewise with Sudoku.
  Not only that you concluded, but that you would repeat.

  At issue is \sRAdj{}.
\end{note}

\begin{note}
  In both cases, \wit{}.
  Indeed, \wit{} follows from construction of \illu{1}.
  Motivating idea was of repetition.
  Repetition, requires original event.
  And, original event serves as a \wit{}.
\end{note}

\begin{note}
  It seems, also, that explain why I did not conclude lost keys.
  And, why kettle logic is bad.
\end{note}


\section{\requ{3} and \issueConstraint{}}
\label{cha:binding:sec:requ-iC}

\begin{note}
  With \autoref{prop:requ-WhyV} we observe the way in which \sR{0}, \fc{1}, and \requ{1} may come together to provide counterexamples to \issueConstraint{}.

  In short, we need some instance in which an agent concludes \(\pv{\phi}{v}\) from \(\Phi\) such that \(\pvp{\psi}{v'}{\Psi}\) is a \requ{} of the agent concluding \(\pv{\phi}{v}\) from \(\Phi\), yet the agent does not have a \wit{} for the \ros{} between \(\pv{\phi}{v}\) and \(\Psi\).

  In full detail:

  \begin{proposition}
    \label{prop:requ-WhyV-ces}
    \cenLine{
      \begin{itemize*}[noitemsep, label=\(\circ\)]
      \item
        Agent: \vAgent{}
      \item
        Propositions: \(\phi\), \(\psi\)
      \item
        Values: \(v\), \(v'\)
      \item
        \poP{3}: \(\Phi\), \(\Psi\)
      \item
        Events: \(e\), \(e^{\flat}\)
      \item
        \mbox{ }
      \end{itemize*}
    }

    \begin{itemize}
    \item[\emph{If}:]
      \ref{prop:requ-WhyVCes:if:concludes},~\ref{prop:requ-WhyVCes:if:concluding},~\ref{prop:requ-WhyVCes:if:requ},~and~\ref{prop:requ-WhyVCes:noW} hold:
      \begin{enumerate}[label=\alph*., ref=(\alph*), series=propRequWhyVSeries]
      \item
        \label{prop:requ-WhyVCes:if:concludes}
        \(e\) is an event in which \vAgent{} concludes \(\pv{\phi}{v}\) from \(\Phi\).
      \item
        \label{prop:requ-WhyVCes:if:concluding}
        \(e^{\flat}\) sub-event of \(e\) in which \vAgent{} is concluding \(\pv{\phi}{v}\) from \(\Phi\).
      \item
        \label{prop:requ-WhyVCes:if:requ}
        \(\pvp{\psi}{v'}{\Psi}\) is a \requ{} of \(e^{\flat}\)
      \item
        \label{prop:requ-WhyVCes:noW}
        No \wit{}.
      \end{enumerate}
    \item[\emph{Then}:]
      \begin{enumerate}[label=\alph*., ref=(\alph*), resume*=propRequWhyVSeries]
      \item
        \label{prop:requ-WhyVCes:tn:answer}
        \(\pvp{\psi}{v'}{\Psi}\) is a counterexample to \issueConstraint{}.
      \end{enumerate}
    \end{itemize}
    \vspace{-\baselineskip}
  \end{proposition}

  \begin{argument}{prop:requ-WhyV-ces}
    \ref{prop:requ-WhyVCes:if:concludes},~\ref{prop:requ-WhyVCes:if:concluding},~\ref{prop:requ-WhyVCes:if:requ} parallel the same conditions from \autoref{prop:requ-WhyV}.
    Therefore, by~\autoref{prop:requ-WhyV}, \(\pvp{\psi}{v'}{\Psi}\) answers, in part, \qWhyV{}.

    However, by~\ref{prop:requ-WhyVCes:noW}, the agent does not have a \wit{}.

    Therefore, answer, in part, to \qWhyV{} such that no \wit{}.
  \end{argument}

  \autoref{prop:requ-WhyV-ces} does not guarantee the existence of counterexamples to \issueConstraint{}.
  And, as we were careful not to presuppose counterexamples to \issueConstraint{} when developing \sR{0}, \fc{1}, and \requ{1}, we have no yet seen any explicit counterexamples to \issueConstraint{}.

  In \autoref{cha:ces} we will provide examples which satisfy the conditions of \autoref{prop:requ-WhyV-ces}.
\end{note}

\begin{note}
  Though, for preview.
  \autoref{illu:gist:sudoku}.
  Suppose fill in each square, and such that concluding solution.
  Then, other squares are \fc{1}.
\end{note}


\begin{note}
  The remainder of this chapter will briefly consider a handful of issues.
\end{note}


\subsection{Interest}
\label{sec:interest}

\begin{note}
  So, established dependence as captured by conditional.

  Does this really capture why the agent concluded?
\end{note}


\subsection{Considerations}
\label{sec:objection}


\paragraph{Factivity}

\begin{note}
  Non-factive explanation.

  No, \requ{1} require that \(\pv{\psi}{v'}\) from \(\Psi\) \emph{is} a \fc{}.
  Hence, if \(\pvp{\psi}{v'}{\Psi}\) answers \qWhyV{} via \requ{}, then there is a \ros{} between \(\pv{\psi}{v'}\) and \(\Psi\).
\end{note}

\paragraph{Subjunctive}

\begin{note}
  Subjunctive conditional.
  If \ros{} failed to hold, then perhaps it's not the case that \requ{}.

  That is, if it is not the case that \requ{} is preserved under counterfactual assumption, then it is not clear that the agent would not have concluded.
\end{note}

\begin{note}
  Response, is that it is plausible the \requ{} holds under counterfactual assumption.
  For, \requ{} is a general property.
\end{note}

\begin{note}
  Consider, coin toss.
  Smile if heads, frown if tails.
  Lands tails.
  If head, would have smiled.

  It is not the case that revise rule so that frown if heads and smile if tails.

  More general point made by (\cite{Tichy:1976tp}) and (\cite{Veltman:2005tj}).
\end{note}

% \subsubsection{Concluding}
% \label{sec:concluding}

% \begin{note}
%   It is not clear the following suggestion holds.
%   \begin{suggestion}
%     If \(e\) is event which may develop, then \(e\) is event in which concluding.

%     {
%       \color{red}
%       This is where the problem is.
%       I don't think this is obvious.

%       We have assumed that if \(e^{sub}\) \emph{is} an event of concluding, then may develop.

%       However, this suggestion is the converse.
%     }
%   \end{suggestion}
% \end{note}

% \begin{note}
%   For, \(e\) is an event in which the agent throws a dart at the centre of the dartboard.
%   However, at the initial stage of \(e\), it is not the case that the agent is sure to throw the dart at the centre of the dartboard.

%   There is a difference between how the event developed, and how the may have developed.

%   So, this idea isn't right in general.

%   Problem.
%   For, if existential account of progressive, then get explain how things came to be simply by focusing on what happened.

%   I do have an answer to this, which is that the role of the \fc{} is what happened.

%   Okay, so hang on.
%   For an agent to be concluding, it does not need to be the case that the event is guaranteed to develop in a way such that things work out.

%   However, it does need to be the case that there is some way things work out.

%   And, the way in which \requ{1} work is that failure means there is no way in which things work out.
% \end{note}

% \begin{note}
%   There are ways to approach this:
%   \begin{itemize}
%   \item
%     Deny the premise.

%     If event in which an agent concludes, it need not be the case that the agent was concluding throughout all the relevant sub-events.

%     Then, no issue is resolved.
%   \item
%     Deny that the event was an event in which the agent concluded.

%     Specifically, narrow the relevant event.
%     For, we have some initial event, which develops in two a larger event, which finishes with an event in which the agent concludes \(\pv{\phi}{v}\).
%     So long as final sub-event, then the agent concludes.
%     The issue only arises from the apparent link to earlier (sub-)events.
%     But, if otherwise, then motivation to reject these as (sub-)events of an event in which the agent concludes.
%   \item
%     Shift perspective.

%     What it true after the fact need not be true as things are developing.
%     This is \citeauthor{Boylan:2020aa}'s approach.
%   \end{itemize}

%   I am inclined to consider the former.
%   When speak on concluded, just picking out an event.
%   Naturally event to some default event.
%   However, flexible.

%   So, here \citeauthor{Boylan:2020aa} and darts.
%   I was able to, sure, but only after things hand developed so far.

%   Additionally, consider picture.
%   Drew a dog.
%   However, not the case that drawing a dog throughout associated event.
%   Started out as a doodle.

%   Ran 10k.
%   However, started out, plan was to run 5k.
%   Interesting consequence here is that not running 5k.
%   For, supposing extension came about by choice, something changed, and hence no force to finish at 5k.

%   Running 10k includes running 5k.
%   But, in a more basic case, running.
%   Extends to drawing.

%   In the case of concluding, have reasoning.
% \end{note}

\subsection{Embedding}
\label{sec:embedding}

\begin{note}
  \ros{}.
  However, know \fc{}, therefore, it need not be the case that \ros{} answers.

  Complex, so separate chapter.
\end{note}

%%% Local Variables:
%%% mode: latex
%%% TeX-master: "master"
%%% End:
