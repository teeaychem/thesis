\chapter{The ingredients and \issueConstraint{}}
\label{cha:binding}

\begin{note}
  The present chapter states the way in which the ingredients of \autoref{part:ing} connect with constraint \issueConstraint{} of \autoref{cha:var}.

  State a collections of conditions sufficient to ensure some \prop{0}-\val{0}-\pool{0} pair answers \qWhyV{}.
  Overall goal is counterexample to \issueConstraint{}.
  Obtain counterexample by no \wit{}.
  Collection of conditions does not entail no \wit{}.
  Rather, hint.
  \autoref{cha:ces} counterexamples.

  First collection of conditions focuses on \requ{} and concluding, rather than \tCV{}.
  More general, and avoids some complexity introduced by incorporating \tCV{}.

  The second collection of conditions adjusts the first to focus on \tCV{}.
  The benefit of \tCV{} is sufficient conditions for \requ{}.
\end{note}

\begin{note}
  There are three sections:
  \begin{TOCEnum}
  \item
    \TOCLine{sec:recollection}

    We briefly restate \qWhyV{} and \issueConstraint{}.
  \item
    \TOCLine{sec:comining-ingredients}
  \item
    \TOCLine{sec:tpyically-concluding}
  \end{TOCEnum}
\end{note}

\section{\qWhyV{} and \issueConstraint{}}
\label{sec:recollection}

\begin{note}
  The following sections appeal to the content of both \qWhyV{} and \issueConstraint{} to state and argue for a handful of proposition which relate \fc{1}, \requ{1}, and an agent \tCV{} to \qWhyV{} and \issueConstraint{}.

  For ease of reference, we briefly restate \qWhyV{} and \issueConstraint{}:

  \reQuestion{questionWhyV}

  \reConstraint{consConstraint}
\end{note}


\section{\requ{3}, \qWhyV{}, and \issueConstraint{}}
\label{sec:comining-ingredients}

\begin{note}
  Section links \requ{0} to \qWhyV{} and \issueConstraint{}.

  Start by defining a collection of conditions.
  Then, \qWhy{}.
  From this, additional condition to get counterexample to \issueConstraint{}.
\end{note}

\subsection{Conditions}
\label{sec:conditions}

\begin{note}
  Link via a collection of conditions.
  In short, event where agent concludes \(\pv{\phi}{v}\) from \(\Phi\), and sub-event where concluding \(\pv{\phi}{v}\) from \(\Phi\) where \(\pvp{\psi}{v'}{\Psi}\) was a \requ{} of the agent concluding \(\pv{\phi}{v}\) from \(\Phi\).

  \begin{definition}[\rCon{2}]
    \label{def:rCon}
    \cenLine{
      \begin{itemize*}[noitemsep, label=\(\circ\)]
      \item
        Agent: \vAgent{}
      \item
        Propositions: \(\phi\), \(\psi\)
      \item
        Values: \(v\), \(v'\)
      \item
        \pool{3}: \(\Phi\), \(\Psi\)
      \item
        Events: \(e\), \(e^{\flat}\)
      \item
        \mbox{ }
      \end{itemize*}
    }

    \begin{itemize}
    \item
      The \emph{\rCon{0}} hold of \(e\) with respect to \(\langle \vAgent{}, \pvp{\phi}{v}{\Phi}, \pvp{\psi}{v'}{\Psi}, e^{\flat} \rangle\).
    \end{itemize}

    \emph{If and only if}:

    \begin{itemize}
    \item
      Conditions~%
      \ref{def:rCs:C}~and~%
      \ref{def:rCs:Cing}~%
      jointly hold:
      \begin{enumerate}[label=\arabic*., ref=(\arabic*)]
      \item
        \label{def:rCs:C}
        \(e\) is an event in which \vAgent{} concludes \(\pv{\phi}{v}\) from \(\Phi\).
      \item
        \label{def:rCs:Cing}
        There is some sub-event \(e^{\flat}\) of \(e\) such that \ref{def:rCs:Cing:requ} and \ref{def:rCs:Cing:itc} jointly hold:
        \begin{enumerate}[label=\alph*., ref=(\arabic{enumi}\alph*)]
        \item
          \label{def:rCs:Cing:itc}
          The following conditional is true:
          \begin{itenum}
          \item[\emph{If}:]
            \(e^{\flat}\) is not an event in which \vAgent{} is concluding \(\pv{\phi}{v}\) from \(\Phi\).
          \item[\emph{Then}:]
            \(e^{\flat}\) is not, or does not develop into, an event in which \vAgent{} concludes \(\pv{\phi}{v}\) from \(\Phi\).
          \end{itenum}
        \item
          \label{def:rCs:Cing:requ}
          \(\pvp{\psi}{v'}{\Psi}\) is a \requ{} of \(e^{\flat}\) being an event in which \vAgent{} is concluding \(\pv{\phi}{v}\) from \(\Phi\).
        \end{enumerate}
      \end{enumerate}
    \end{itemize}
    \vspace{-1.5\baselineskip}
  \end{definition}

  Condition~\ref{def:rCs:C}, and event in which an agent concludes.
  Hence, \qWhyV{} applies.

  Some sub-event such that conditional of \qWhyV{} holds.

  Condition~\ref{def:rCs:Cing} concerns the sub-event.

  \ros{}.
  Does not develop.

  Sub-condition~\ref{def:rCs:Cing:itc}.
  Does not develop.
  However, agent is not concluding, rather than absence of a \ros{}.

  A strong condition, but an intuitive one.
  If event is not in progress, the event does not happen.

  In particular, getting conclusion from premises.
  It may be the case that agent concludes.
  However, if not concluding, then difficult to say that it's from these premises.

  For example, arithmetic or something.
  Working through a problem.
  Intuitively, conclude from some \pool{} captures understanding of arithmetic.
  If not concluding, then is it the case that conclude from \pool{} captures understanding of arithmetic?

  This is not \agpe{}, but the way things are.

  To link not concluding to \ros{}, Sub-condition~\ref{def:rCs:Cing:requ}.
\end{note}

\subsection{\requ{3} and \qWhyV{}}

\begin{note}
  \begin{proposition}[\requ{3} and \qWhyV{}]
    \label{prop:requ-WhyV}
    \cenLine{
      \begin{itemize*}[noitemsep, label=\(\circ\)]
      \item
        Agent: \vAgent{}
      \item
        Propositions: \(\phi\), \(\psi\)
      \item
        Values: \(v\), \(v'\)
      \item
        \pool{3}: \(\Phi\), \(\Psi\)
      \item
        Events: \(e\), \(e^{\flat}\)
      \item
        \mbox{ }
      \end{itemize*}
    }

    \begin{itenum}
    \item[\emph{If}:]
      The \rCon{0} hold of \(e\) with respect to \(\langle \vAgent{}, \pvp{\phi}{v}{\Phi}, \pvp{\psi}{v'}{\Psi}, e^{\flat} \rangle\).
    \item[\emph{Then}:]
      \(\pvp{\psi}{v'}{\Psi}\), in part, answers \qWhyV{}.
    \end{itenum}
    \vspace{-\baselineskip}
  \end{proposition}

  \begin{argument}{prop:requ-WhyV}
    Suppose the \rCon{0} hold of \(e\) with respect to \(\langle \vAgent{}, \pvp{\phi}{v}{\Phi}, \pvp{\psi}{v'}{\Psi}, e^{\flat} \rangle\).

    From Condition~\ref{def:rCs:C}, concludes.
    So, \qWhyV{} applies.

    Our task is to show the conditional of \qWhyV{} is true.

    From~\ref{def:rCs:Cing} of the \rCon{}, we have some sub-event \(e^{\flat}\) of \(e\) such that \ref{def:rCs:Cing:itc} and \ref{def:rCs:Cing:requ} jointly hold.

    Consider the event \(e^{\flat}\) and assume a \ros{} between \(\pv{\psi}{v'}\) and \(\Psi\) fails to hold for \vAgent{} through \(e^{\flat}\).

    \autoref{prop:fcs-only-if-pot-support} (\autopageref{prop:fcs-only-if-pot-support}) established:

    \begin{itenum}
    \item[\emph{If}:]
      \(\pv{\psi}{v'}\) is a \fc{0} from \(\Psi\) for \vAgent{} throughout \(e^{\sharp}\).
    \item[\emph{Then}:]
      A \ros{0} between \(\pv{\psi}{v'}\) and \(\Psi\) holds for \vAgent{} throughout \(e^{\sharp}\).
    \end{itenum}

    \noindent%
    So, by our assumption \(\pv{\psi}{v'}\) not a \fc{0} from \(\Psi\) for \vAgent{} throughout \(e^{\flat}\).

    Now, by Sub-condition~\ref{def:rCs:Cing:requ} of the \rCon{}, \(\pvp{\psi}{v'}{\Psi}\) is a \requ{} of \(e^{\flat}\) being an event in which \vAgent{} is concluding \(\pv{\phi}{v}\) from \(\Phi\).
    And, by expanding the definition of a \requ{} (\autopageref{def:requ}):

    \begin{itenum}
    \item[\emph{If}:]
      \(e\) is an event in which \vAgent{} is concluding \(\pv{\phi}{v}\) from \(\Phi\).
    \item[\emph{Then}:]
      \(\pv{\psi}{v'}\) from \(\Psi\) is a \fc{} for \vAgent{}.
    \end{itenum}

    \noindent%
    Hence, \(e\) is not an event in which \vAgent{} is concluding \(\pv{\phi}{v}\) from \(\Phi\).

    Therefore, by Sub-condition \ref{def:rCs:Cing:itc}, \(e^{\flat}\) is not, or does not develop into, an event in which \vAgent{} concludes \(\pv{\phi}{v}\) from \(\Phi\).
  \end{argument}
\end{note}

\begin{note}
  Key.

  Intuition is that \tC{}.
  Therefore, matters whether or not \fc{}.
  Hence, matters whether or not \ros{}.
\end{note}

\subsection{\requ{3} and \issueConstraint{}}
\label{cha:binding:sec:requ-iC}

\begin{note}
  With \autoref{prop:requ-WhyV} we observe the way in which the \rCon{0} may provide counterexamples to \issueConstraint{}.

  In short, we need some instance in which an agent concludes \(\pv{\phi}{v}\) from \(\Phi\) such that \(\pvp{\psi}{v'}{\Psi}\) is a \requ{} of the agent concluding \(\pv{\phi}{v}\) from \(\Phi\), yet the agent does not have a \wit{} for the \ros{} between \(\pv{\phi}{v}\) and \(\Psi\).

  In full:

  \begin{proposition}[\requ{3} and \issueConstraint{}]
    \label{prop:requ-WhyV-ces}
    \cenLine{
      \begin{itemize*}[noitemsep, label=\(\circ\)]
      \item
        Agent: \vAgent{}
      \item
        Propositions: \(\phi\), \(\psi\)
      \item
        Values: \(v\), \(v'\)
      \item
        \pool{3}: \(\Phi\), \(\Psi\)
      \item
        Events: \(e\), \(e^{\flat}\)
      \item
        \mbox{ }
      \end{itemize*}
    }

    \begin{itenum}
    \item[\emph{If}:]
      The \rCon{0} hold of \(e\) with respect to \(\langle \vAgent{}, \pvp{\phi}{v}{\Phi}, \pvp{\psi}{v'}{\Psi}, e^{\flat} \rangle\).
    \item[\emph{And}:]
      \label{prop:requ-WhyVCes:noW}
      \vAgent{} does not have a \wit{} for a \ros{} between \(\pv{\phi}{v}\) and \(\Psi\) when \vAgent{} concludes \(\pv{\phi}{v}\) from \(\Phi\).
    \item[\emph{Then}:]
      \(\pvp{\psi}{v'}{\Psi}\) is a counterexample to \issueConstraint{}.
    \end{itenum}
    \vspace{-\baselineskip}
  \end{proposition}

  \begin{argument}{prop:requ-WhyV-ces}
    More-or-less immediate.

    If the \rCon{0} hold of \(e\) with respect to \(\langle \vAgent{}, \pvp{\phi}{v}{\Phi}, \pvp{\psi}{v'}{\Psi}, e^{\flat} \rangle\), then \(\pvp{\psi}{v'}{\Psi}\), in part, answers \qWhyV{}, given~\autoref{prop:requ-WhyV}.
    And, granting \vAgent{} does not have a \wit{} for a \ros{} between \(\pv{\phi}{v}\) and \(\Psi\) when \vAgent{} concludes \(\pv{\phi}{v}\) from \(\Phi\), this immediately contradicts \issueConstraint{}.
  \end{argument}
\end{note}

\begin{note}
  \autoref{prop:requ-WhyV-ces} does not guarantee the existence of counterexamples to \issueConstraint{}.
  And, as we were careful not to presuppose counterexamples to \issueConstraint{} when developing \tC{0}, \fc{1}, and \requ{1}, we have no yet seen any explicit counterexamples to \issueConstraint{}.

  \autoref{cha:ces} provides examples which satisfy the conditions of \autoref{prop:requ-WhyV-ces}.
\end{note}

\begin{note}
  The existence of \requ{1} are compatible with \issueConstraint{}.

  \begin{observation}%
    \label{prop:requ-not-n-ce}%
    \autoref{prop:requ-WhyV} is compatible with \issueConstraint{}.
  \end{observation}

  \begin{motivation}{prop:requ-not-n-ce}
    \requ{3} are about an agent concluding.
    \fc{} when the agent concludes.
    An agent may fail to have a \wit{} for \fc{}.

    However, \issueConstraint{}, \wit{} when the agent concludes.
    Therefore, so long as \wit{} when the agent concludes, fine.

    Two possibilities.
    \begin{enumerate}
    \item
      Agent already has a \wit{}.

      For example, first \scen{0}.
      If concluding sequence, \fc{}.
      Prior to getting \(L_{8}\), \fc{}.

      However, expectation of \wit{} when sequence.
      Indeed, \(L_{9}\) from \(L_{8} + L_{7}\).
    \item
      Agent obtains \wit{} prior to conclusion.

      Consider \autoref{scen:squish}.
      The agent has previously concluded \sqE{} is sound.
      Therefore, the agent has a \wit{} for \ros{}.
    \end{enumerate}
    \vspace{-\baselineskip}
  \end{motivation}

  Note, however, that \autoref{prop:requ-not-n-ce} is weak.
  Compatible in terms of existential.
  Hence we only needed to show a single case in which \wit{} for \requ{}.
  This does not suggest that \requ{1} are not in tension with \issueConstraint{}.
\end{note}

\begin{note}
  \begin{observation}
    \label{obs:denyCing}
    Deny that needed to be the case agent was concluding.
  \end{observation}

  \begin{motivation}{obs:denyCing}
    \qWhyV{} is about developing.
    If it does not need to be the case that the agent is concluding, then it is possible for the agent to conclude.
  \end{motivation}

  In part, this ensures we don't get anything that is clearly redundant.

  In general, possible to deny.
  But, I don't think it is viable in many cases.
  Agent is concluding.
  The conclusion does not `happen'.

  This is hand wavy.
  Idea of an agent \tCV{} addresses this.
\end{note}

\section{\tC{2} and \issueConstraint{}}
\label{sec:tpyically-concluding}

\begin{note}
  Understand \requ{} and counterexamples to \issueConstraint{}.

  Introduced \tCV{} to motivate \requ{}.

  In this section, recast previous section in terms of \tCV{}.
\end{note}

\subsection{\tCon{2}}
\label{cha:binding:tCon}

\begin{note}
  The \tCon{0} parallel the \rCon{0}.
  Difference is third.
  Rather than \requ{}, conditions sufficient for \requ{} in terms of \tCV{}.

  \begin{definition}[The \tCon{0}]
    \label{def:tCon}
        \cenLine{
      \begin{itemize*}[noitemsep, label=\(\circ\)]
      \item
        Agent: \vAgent{}
      \item
        Propositions: \(\phi\), \(\psi\)
      \item
        Values: \(v\), \(v'\)
      \item
        \pool{3}: \(\Phi\), \(\Psi\)
      \item
        Events: \(e\), \(e^{\flat}\)
      \item
        \mbox{ }
      \end{itemize*}
    }

    \begin{itemize}
    \item
      The \emph{\tCon{0}} hold of \(e\) with respect to \(\langle \vAgent{}, \pvp{\phi}{v}{\Phi}, \pvp{\psi}{v'}{\Psi}, e^{\flat} \rangle\).
    \end{itemize}

    \emph{If and only if}:

    \begin{itemize}
    \item
      Conditions~%
      \ref{def:tCon:C}~and~%
      \ref{def:tCon:Cing}~%
      jointly hold:
      \begin{enumerate}[label=\arabic*., ref=(\arabic*)]
      \item
        \label{def:tCon:C}
        \(e\) is an event in which \vAgent{} concludes \(\pv{\phi}{v}\) from \(\Phi\).
      \item
        \label{def:tCon:Cing}
        There is some sub-event \(e^{\flat}\) of \(e\) such that \ref{def:tCon:sR} and \ref{def:tCon:Cing:itc} jointly hold:
        \begin{enumerate}[label=\alph*., ref=(\arabic{enumi}\alph*)]
        \item
          \label{def:tCon:Cing:itc}
          The following conditional is true:
          \begin{itenum}
          \item[\emph{If}:]
            \(e^{\flat}\) is not an event in which \vAgent{} is \tCV{} \(\pv{\phi}{v}\) from \(\Phi\).
          \item[\emph{Then}:]
            \(e^{\flat}\) is not, or does not develop into, an event in which \vAgent{} concludes \(\pv{\phi}{v}\) from \(\Phi\).
          \end{itenum}
        \item
          \label{def:tCon:sR}
          Conditions%
          ~\ref{def:tCon:sR:rep},%
          ~\ref{def:tCon:sR:tI}, and%
          ~\ref{def:tCon:sR:itc} %
          jointly hold:
          \begin{enumerate}[label=\roman*., ref=(\arabic{enumi}\alph{enumii}.\roman*)]
          \item
            \label{def:tCon:sR:rep}
            \(T'\) is a \tRep{} of \vAgent{} \tCV{} \(\pv{\phi}{v}\) from \(\Phi\) by type \(T\) in \(e^{\flat}\).
          \item
            \label{def:tCon:sR:tI}
            \(\pvp{\psi}{v'}{\Psi}\) is a \tI{} of \(T'\)
          \item
            \label{def:tCon:sR:itc}
            The following conditional is true:
            \begin{itemize}
            \item[\emph{If}:]
              There is some available action \(a\) such that \vAgent{} is concluding \(\pv{\psi}{v'}\) from \(\Psi\), when \vAgent{} does \(a\).
            \item[\emph{Then}:]
              There is some available action \(a'\) such that \vAgent{} is concluding \(\pv{\psi}{v'}\) from \(\Psi\) without use of any novel information obtained by doing \(a'\), when \vAgent{} does \(a'\).
            \end{itemize}
          \end{enumerate}
        \end{enumerate}
      \end{enumerate}
    \end{itemize}
    \vspace{-1.5\baselineskip}
  \end{definition}

  \noindent%
  Similar to previous conditions.
  Two key differences.
  \ref{def:tCon:Cing}: \tCV{} rather than concluding.
  \ref{def:tCon:sR}: Rather than \requ{}, sufficient conditions for \requ{} so long as an agent is \tCV{}.

  In short, we slightly strengthen \ref{def:tCon:Cing} to provide a more specific account of the \scen{1} of interest.
\end{note}

\subsection{\tCon{2}, \qWhyV{} and \issueConstraint{}}
\label{sec:tccon2-qwhyv-}

\begin{note}
  \begin{proposition}[\tCV{3} and \qWhyV{}]
    \label{prop:tCV-WhyV}
    \cenLine{
      \begin{itemize*}[noitemsep, label=\(\circ\)]
      \item
        Agent: \vAgent{}
      \item
        Propositions: \(\phi\), \(\psi\)
      \item
        Values: \(v\), \(v'\)
      \item
        \pool{3}: \(\Phi\), \(\Psi\)
      \item
        Events: \(e\), \(e^{\flat}\)
      \item
        \mbox{ }
      \end{itemize*}
    }

    \begin{itenum}
    \item[\emph{If}:]
      The \tCon{0} hold of \(e\) with respect to \(\langle \vAgent{}, \pvp{\phi}{v}{\Phi}, \pvp{\psi}{v'}{\Psi}, e^{\flat} \rangle\).
    \item[\emph{Then}:]
      \(\pvp{\psi}{v'}{\Psi}\), in part, answers \qWhyV{}.
    \end{itenum}
    \vspace{-1.5\baselineskip}
  \end{proposition}

  Instead of \requ{} directly, strengthen to \tCV{} and sufficient conditions for \requ{} given \tCV{}.

  \begin{argument}{prop:tCV-WhyV}
    The argument parallels \autoref{prop:requ-WhyV} with  minor adjustment.

    Rather appeal to Sub-condition~\ref{def:rCs:Cing:requ} of the \rCon{} to observe \(e^{\flat}\) is not an event in which \vAgent{} is concluding \(\pv{\phi}{v}\) from \(\Phi\), appeal to Sub-condition~\ref{def:tCon:sR} of the \tCon{} to observe \(e^{\flat}\) is not an event in which \vAgent{} is \tCV{} \(\pv{\phi}{v}\) from \(\Phi\).
    And, with this observation in hand appeal to Sub-condition~\ref{def:tCon:Cing:itc} to observe \(e^{\flat}\) is not, or does not develop into, an event in which \vAgent{} concludes \(\pv{\phi}{v}\) from \(\Phi\).
  \end{argument}

  \noindent%
  With \autoref{prop:tCV-WhyV} in hand, we observe the relevant parallel to \autoref{prop:requ-WhyV-ces}:

  \begin{proposition}[\tCV{3} and \issueConstraint{}]
    \label{prop:tCV-WhyV-ces}
    \cenLine{
      \begin{itemize*}[noitemsep, label=\(\circ\)]
      \item
        Agent: \vAgent{}
      \item
        Propositions: \(\phi\), \(\psi\)
      \item
        Values: \(v\), \(v'\)
      \item
        \pool{3}: \(\Phi\), \(\Psi\)
      \item
        Events: \(e\), \(e^{\flat}\)
      \item
        \mbox{ }
      \end{itemize*}
    }

    \begin{itenum}
    \item[\emph{If}:]
      The \tCon{0} hold of \(e\) with respect to \(\langle \vAgent{}, \pvp{\phi}{v}{\Phi}, \pvp{\psi}{v'}{\Psi}, e^{\flat} \rangle\).
    \item[\emph{And}:]
      \vAgent{} does not have a \wit{} for a \ros{} between \(\pv{\phi}{v}\) and \(\Psi\) when \vAgent{} concludes \(\pv{\phi}{v}\) from \(\Phi\).
    \item[\emph{Then}:]
      \(\pvp{\psi}{v'}{\Psi}\) is a counterexample to \issueConstraint{}.
    \end{itenum}
    \vspace{-\baselineskip}
  \end{proposition}

  \begin{argument}{prop:tCV-WhyV-ces}
    Follows by adapting the argument for \autoref{prop:requ-WhyV-ces} with \autoref{prop:tCV-WhyV}.
  \end{argument}
\end{note}

\section*{Summary}

\begin{note}
  Smashing definitions.
\end{note}






%%% Local Variables:
%%% mode: latex
%%% TeX-master: "master"
%%% TeX-engine: luatex
%%% End:
