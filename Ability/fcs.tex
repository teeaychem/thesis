\chapter{\fc{3}}
\label{cha:fcs}

\nocite{Ryle:1946tu}

\begin{note}
  Here, \fc{1}.
  Two key things.
  An account of \fc{1}, and connexion between \fc{1} and \ros{1}.

  \begin{itemize}
  \item
    If \fc{}, then relation of support.
  \end{itemize}

  Important idea.
  However, limited.
  Support without witnessing doesn't raise a problem for \issueConstraint{}.
  Rather, support needs to be \emph{why}.

  \autoref{cha:zS} will develop, \check{1} on concluding.

  Main focus is \autoref{cha:sec:fcs-def}.
  What it is for some proposition-value pairing \(\pv{\phi}{v}\) to be a \fc{} from some pool of premises \(\Phi\).
  With the right set-up, \support{} --- focus on \autoref{cha:fcs:sec:fcs-support} --- will be simple.

  Progressive, action which is concluding \(\pv{\phi}{v}\) from \(\Phi\).
  Contrast with ability, `ability to conclude \(\pv{\phi}{v}\) from \(\Phi\)'.
  Argue that there is no clear path.
  Though, implicit upshot is reducing ability to progressive (given assumption).
\end{note}

\begin{note}
  Breakdown
  \begin{itemize}
  \item
    \TOCLine{sec:intuition}
  \item
    \TOCLine{cha:sec:fcs-def}

    Definition of \fc{0}.
  \item
    \TOCLine{cha:fcs:sec:illu}
  \item
    \TOCLine{cha:sec:fcs-def:ability}
  \item
    \TOCLine{cha:fcs:sec:fcs-support}

    support.
  \end{itemize}
\end{note}

\begin{note}
  Support.

  Idea here is something distinguished about concluding, but is independent of whether or not the agent has witnessed concluding.

  \autoref{idea:support} and \autoref{idea:support:possible}.

  Noted, independence does not entail relation of support from \agpe{} without concluding.

  Parallel to propositional and doxastic justification.
\end{note}

\section{Intuition}
\label{sec:intuition}

\begin{note}
  General term `foregone conclusion' is ambiguous.
  \begin{itemize}
    \item
    Inevitable results of reasoning.
  \item
    Conclusion which has been settled in advance of reasoning.
  \end{itemize}

  Interest is with the second meaning.
  Two examples of general use.

  % \begin{quote}
  %   [どうぜ]\dots Expresses an attitude of resignation or carelessness on the part of the speaker, in the sense that regardless of what s/he does, the conclusion or outcome is foregone and cannot be changed by the will or effort of an individual.%
  %   \mbox{ }\hfill\mbox{(\citeyear[332--333]{kurufushamashii:2015un})}
  % \end{quote}
  A clear example of the first meaning is found in~\citeauthor{Machover:1996vu}'s~\citetitle{Machover:1996vu}:
w
  \begin{quote}
    I have omitted its proof, but added a detailed analysis of the meaning of the lemma and the reason why its proof works. When this is understood, the proof itself becomes a mere technicality, almost a foregone conclusion.%
    \mbox{ }\hfill\mbox{(\citeyear[viii]{Machover:1996vu})}
  \end{quote}

  \citeauthor{Machover:1996vu} is discussing a proof, and whether or not it is inevitable that one would complete the proof (conclude that the relevant theorem is true) after understanding the lemma and why it works.
  There is no relevant sense in which the truth of the theorem has been settled in advance of reasoning.
  Though, as the proof is somewhat difficult,~\citeauthor{Machover:1996vu} only states the proof is `almost' a foregone conclusion.%
  \footnote{
    The proof is question is of the G\"{o}del-Rosser First Incompleteness Theorem.
    (\citeyear[Cf.][226]{Machover:1996vu})
  }%
  \(^{,}\)
  \footnote{
    For a similar example without qualification, consider the following from~\textcite{Jacquette:2002up}:
    \begin{quote}
    It is nevertheless important to recognize that Russell's evaluation of such sentences as false is predetermined by his existence presuppositional semantics for the ‘existential' quantifier, and by the fact that his logic permits no alternative means of considering the semantic status of sentences ostensibly containing proper names for nonexistent objects.
    This makes it an altogether philosophically foregone conclusion that sentences like ‘Pegasus is winged,' which many logicians would otherwise consider to be true propositions of mythology, are false.%
    \mbox{ }\hfill\mbox{(\citeyear[6]{Jacquette:2002up})}
  \end{quote}
  }

  For an additional example, consider the following from~\citeauthor{Grice:1957vg}'s~\citetitle{Grice:1957vg}:%
  \footnote{
    Same is found in \textcite[219]{Grice:1989uf}.
  }
  \begin{quote}
    He intends the audience's recognition of his intention to produce that response to be effective in producing that response.
    He does not regard it as a foregone conclusion that his action will produce the intended response, whether or not his intention is recognized.\newline
    \mbox{ }\hfill\mbox{(\citeyear[385]{Grice:1957vg})}
  \end{quote}

  In this case, the term `foregone conclusion' is embedded under negation, to highlight that the agent in question entertains the possibility that the agent's action will not produce the intended response.

  By contrast, the following passage from \textcite{Kadane:1996vu} is an example of the second meaning:

  \begin{quote}
    When can a Bayesian select an hypothesis \emph{H} and design an experiment (or a sequence of experiments) to make certain that, given the experimental outcome(s), the posterior probability of \emph{H} will be greater than its prior probably?
    We discuss an elementary result that establishes sufficient conditions under which this reasoning to a foregone conclusion cannot occur.%
    \mbox{ }\hfill\mbox{(\citeyear[1228]{Kadane:1996vu})}
  \end{quote}

  At issue is whether a Bayesian may chose some hypothesis \emph{H} and then guarantee some increase in probability for \emph{H} by running some experiments.
  So, are there cases in which the Bayesian first chooses a hypothesis \emph{H} and then ensures they reason to an increase in the probability of \emph{H}?
\end{note}

\begin{note}
  Our interest is with the first meaning, though narrow.
  To avoid ambiguity with first meaning, write `\fc{0}' rather than `foregone conclusion'.
  That is, the hyphen signifies when we are speaking about the technical term.
\end{note}

\section{\fc{3}}
\label{cha:sec:fcs-def}

\begin{note}[\fc{2} definition]
  We define \(\pv{\phi}{v}\) being a \emph{\fc{0}} as follows:

  \begin{restatable}[\fc{3}]{definition}{definitionForegoneC}
    \label{def:fc}
    For an agent \vAgent{}, and some proposition-value-premises pairing \(\pvp{\phi}{v}{\Phi}\):

    \begin{itemize}
    \item
      \(\pv{\phi}{v}\) is a \emph{\fc{0}} from \(\Phi\), for \vAgent{}.\newline
      \mbox{ }\hfill(equiv.\ \(\pvp{\phi}{v}{\Phi}\) is a \fc{0} for \vAgent{})
    \end{itemize}
    \emph{If and only if}
    \begin{enumerate}[label=]
    \item
      Both~\ref{def:fc:is-pe-good} and~\ref{def:fc:no-pe-bad} are true:
      \begin{enumerate}[label=\alph*., ref=(\alph*)]
      \item
        \label{def:fc:is-pe-good}
        There is \emph{some} \pevent{} \(p\) in which \vAgent{} concludes \(\pv{\phi}{v}\) from \(\Phi\).
      \item
        \label{def:fc:no-pe-bad}
        There is \emph{no} \pevent{} \(p\) in which \vAgent{} concludes some proposition-value-premises pairing which is incompatible with concluding \(\pv{\phi}{v}\) from \(\Phi\).%
        \footnote{
          Incompatible:
          \(\pv{\chi}{v''}\) from \(X\) where:
        If conclude \(\pv{\chi}{v''}\) from \(X\), then does not conclude \(\pv{\phi}{v}\) from \(\Phi\).
        }
      \end{enumerate}
    \end{enumerate}
    \vspace{-\baselineskip}
  \end{restatable}
\end{note}

\begin{note}[Intuition]
  Significant attention will be given to what is means for there to be a potential event in which an agent performs some action in \autoref{cha:sec:fcs-def:potential-events}, below.
  However, the basic features of \autoref{def:fc} follow from substituting `possible' for `\pevent{}'.
  Given this substitution:

  \begin{itemize}[noitemsep]
  \item
    Clause~\ref{def:fc:is-pe-good} ensures that there is some possibility in which the agent to conclude \(\pv{\phi}{v}\) from \(\Phi\).
  \item
    Clause~\ref{def:fc:no-pe-bad} ensures that there is no possibility in which the agent concludes something incompatible with concluding \(\pv{\phi}{v}\) from \(\Phi\).
    Hence, Clause~\ref{def:fc:no-pe-bad} rules out the agent failing to conclude \(\pv{\phi}{v}\) from \(\Phi\) because some incompatible proposition-value-premises pairing is (also) a \fc{0}.
  \end{itemize}

  Intuitively, and inevitable result of possible reasoning.
  For, there is some possibility in which the agent concludes \(\pv{\phi}{v}\) from \(\Phi\) via Clause~\ref{def:fc:is-pe-good}.
  And, at no point prior to concluding could the agent have concluded some other proposition-value-premises pairing which would prevent the agent from concluding \(\pv{\phi}{v}\) from \(\Phi\) via~\ref{def:fc:no-pe-bad}.

  Further, not merely that agent would not prevent on success, but that there is no way to block success.
\end{note}

\begin{note}
  Still, mere possibility is too general.
  There is a possibility in which I immediately obtain a comprehensive understanding of theoretical computer science and settle whether P is equal to NP (or show that the P versus NP problem is undecidable).
  Hence, \fc{1} are defined with respect to \pevent{1}.

  For the moment we will set \pevent{1} aside, and consider a handful of \illu{1} in \autoref{cha:fcs:sec:illu}.
  In \autoref{cha:sec:fcs-def:potential-events} we will provide a fairly detailed account of \pevent{1}.
  (If you would prefer to skip the \illu{1}, please turn to~\autopageref{cha:sec:fcs-def:potential-events}.)
\end{note}

\begin{note}[Neutral perspective]
  \phantlabel{fcs-neutral-perspective}
  Still, before turning to the \illu{1} and \pevent{1}, a final observation:

  Observe that whether \(\pvp{\phi}{v}{\Phi}\) is a \fc{0} is stated from a \agpe{neutral} --- at issue is whether there is a \pevent{} in which the agent concludes.
  Our interest with \fc{1} will be from an \agpe{}, rather than whether \(\pvp{\phi}{v}{\Phi}\) \emph{is} a \fc{0}.
  But, by defining \fc{1} from a \agpe{neutral}, we straightforwardly understand whether \(\pvp{\phi}{v}{\Phi}\) is a \fc{0} from the \agpe{} by shifting to the \agpe{}.
\end{note}

\subsection{\pevent{3}}
\label{cha:sec:fcs-def:potential-events}

\begin{note}
  \autoref{def:fc} appeals to~\ref{def:fc:is-pe-good} the existence of some \pevent{} and~\ref{def:fc:no-pe-bad} the non-existence of some \pevent{}.

  However, \autoref{def:fc} does not rely on anything more than existential quantification.
  The choice is deliberate.
  We given necessary and sufficient conditions for the \emph{existence} of some \pevent{} in terms of
  \begin{enumerate*}[label=(\roman*)]
  \item
    actions available to the agent, and
  \item
    truth conditions for the progressive.
  \end{enumerate*}
\end{note}

\begin{note}[\pevent{2} definition]
  We define a \pevent{} as follows:
  \begin{restatable}[\pevent{3}]{definition}{definitionPEvent}
    \label{def:potenital-event}
    For an agent \vAgent{} and action description \(\alpha\):
    \begin{itemize}
    \item
      There is a \pevent{} \(p\) in which \vAgent{} \(\alpha\)s
    \end{itemize}
    \emph{if and only if}
    \begin{enumerate}[label=]
    \item
      Both~\ref{def:PE:action} and~\ref{def:PE:prog} are true:
      \begin{enumerate}[label=\alph*., ref=(\alph*)]
      \item
        \label{def:PE:action}
        There is some action \(a\) that \vAgent{} may immediately perform.
      \item
        \label{def:PE:prog}
        \(\text{Prog}(e, \alpha)\) would be true in the event \(e\) of \vAgent{} doing \(a\).
      \end{enumerate}
    \end{enumerate}
    Where \(\text{Prog}(e, \alpha)\) stands for the progressive from of \(\alpha\) when evaluated with respect to \(e\) and \assuPP{} holds for the progressive.%
    \footnote{
      I.e.\ \(\text{Prog}(e, \alpha)\) is true \emph{iff} event \(e\) is an event of \(\alpha\)ing.
      See,~\textcite{Richards:1981wo},~\textcite{Portner:2011vi}, etc.
    }
  \end{restatable}

  In short,~\autoref{def:potenital-event} states that there is a \pevent{} in which an agent performs some action \(\alpha\) just in case there is some action the agent may (immediately) perform which would result in the agent \(\alpha\)ing.
\end{note}

\begin{note}[Division of labour between the clauses]
  The division of labour between clauses~\ref{def:PE:action} and~\ref{def:PE:prog} is, in reverse order:
  \begin{itemize}[noitemsep]
  \item
    Clause~\ref{def:PE:prog} captures a sense of possibility, via the progressive, such that that the agent \(\alpha\)s (but in such a way that how the agent \(\alpha\)s is not necessarily settled by the action).
\item
  Clause~\ref{def:PE:action} distinguishes the existence of a \pevent{} from the agent \(\alpha\)ing by existential quantification over actions, but binds the existence of a \pevent{} to circumstances be restriction to immediate actions.
  \end{itemize}
\end{note}

\begin{note}
  Following section \autoref{cha:sec:fcs-def:ability} will get problems with ability.
\end{note}

\subsection{The progressive}
\label{cha:fcs:sec:progressive}

\begin{note}[Interest with the progressive]
  Our interest with the progressive is due to the delicate sense of possibility required for a sentence stating an event in the progressive to be true.

  \phantlabel{imperfective-paradox:intro}
  Perhaps the clearest example is the `imperfective paradox' (\citeyear[cf.][Ch.3.1]{Dowty:1979vq}).

  \citeauthor{Bach:1986tb} summarises:
  \begin{quote}
    [H]ow can we characterize the meaning of a progressive sentences like \ref{Bach:impP:17} on the basis of the meaning of a simple sentence like \ref{Bach:impP:18} when \ref{Bach:impP:17} can be true of a history without \ref{Bach:impP:18} ever being true?
    \begin{enumerate}[label=(\arabic*), ref=(\arabic*)]
      \setcounter{enumi}{16}
    \item
      \label{Bach:impP:17}
      John was crossing the street.
    \item
      \label{Bach:impP:18}
      John crossed the street.%
      \mbox{ }\hfill\mbox{(\citeyear[12]{Bach:1986tb})}
    \end{enumerate}
  \end{quote}

  No completion is required, and often some surprise.
  Something unexpected happened while John was crossing the street.
  Sense of inertia associated with the agent \(\alpha\)ing.

  Expectation that that John reaching the other side of the street does not reduce to \(\{\text{logical}, \text{metaphysical}, \text{nomic}, \dots\}\) possibility.

  For, suppose John is sitting a multiple choice exam.
  To pass the exam John only needs to chose some number of correct choices.
  It is certainly logically, metaphysically, and nomically possible that John chooses a sufficient number of correct choices.
  However, it does not follow that John is passing the exam.%
  \footnote{
    See also Igal Kvart's example of Mary wiping out the Roman army (\cite[18]{Landman:1992wh}).
  }

  Likewise, there is no simple relation to counterfactuals.
  Consider a scenario in which John is passing the exam without external help.
  Then, a classmate slips John some answers, which John then uses.
  It is no longer true that John is passing the exam without external help.
  And, in the closest possible world where the classmate does not slip John answers, it need not be true that John passes the exam without external help.
  For, if John is surrounded by students of a similar mindset then it is plausible that the in closest possible world a different classmate slips John the same answers.
\end{note}

\begin{note}
  Way the modality functions is tied to the event.

  \citeauthor{Dowty:1979vq} adds:
  \begin{quote}
    Notice, furthermore, that to Say that John was drawing a circle is not the same as saying that John was drawing a triangle, the difference between the two activities obviously having to do with the difference between a circle and a triangle.
    Yet if neither activity necessarily involves the existence of such a figure, just how are the two to be distinguished?%
    \mbox{ }\hfill\mbox{(\citeyear[133]{Dowty:1979vq})}
  \end{quote}

  As \citeauthor{Dowty:1979vq} highlights, event is sufficiently specific to determine some outcome over some other.%
  \footnote{
    Though, the force of \citeauthor{Dowty:1979vq}'s observation is perhaps clearer by substituting `square' for `circle'.
    For, straight line\dots
  }
  So, the truth of the progressive doesn't require completion and doesn't require significant progress toward completion.
\end{note}

\begin{note}
  \autoref{def:potenital-event} relies on important (but common)%
  \footnote{
    See, for example:
    \textcite{Bennett:1972uw},
    \textcite{Dowty:1979vq},
    \textcite{Parsons:1990aa},
    \textcite{Landman:1992wh}, and
    \textcite{Portner:1998um}.

    However,~\assuPP{0} is denied by~\textcite{Szabo:2004ul}.
    \citeauthor{Szabo:2004ul} writes:
    \begin{quote}
      Sometimes we are \emph{doing} things even though there is no real chance that we could get them \emph{done}, and this is true even if we abstract away from the possibility of miraculous intervention.%
      \mbox{ }\hfill\mbox{(\citeyear[40]{Szabo:2004ul})}
    \end{quote}
    To illustrate, \citeauthor{Szabo:2004ul} denies~\ref{Szabo:Arch} is necessarily false:
    \begin{quote}
      \begin{enumerate}[label=(\arabic*), ref=(\arabic*)]
        \setcounter{enumi}{12}
      \item
        \label{Szabo:Arch}
        As the architect was building the cathedral he knew that, although he would be building it for another year or so, he couldn't possibly complete it.%
        \mbox{ }\hfill\mbox{(\citeyear[38]{Szabo:2004ul})}
      \end{enumerate}
    \end{quote}
    Though,~\ref{Szabo:Arch} seems false to me, without some priming.
    And, the only priming on which~\ref{Szabo:Arch} reads true involves interpreting the architect's knowledge from the \agpe{architect's}, allowing a failure of factivity, thus allowing the cathedral to be built.

    Still,~\assuPP{0} is an assumption.
    The goal is not to tie potential to progressive, but to evaluation associated with the progressive granting assumption.
  }
  assumption regarding the progressive.

  \begin{assumption}[\assuPP{2}]
    \label{assu:PP}
    For any event \(e\) and action description \(\alpha\):
    \begin{enumerate}
    \item[\emph{If}:]
      \begin{enumerate}[label=\alph*., ref=(\alph*)]
      \item
        \(e\) is an event of \(\alpha\)ing.%
        \hfill(I.e.\ \(\text{Prog}(\alpha)\) is true of \(e\).)
      \end{enumerate}
    \item[\emph{Then}:]
      \begin{enumerate}[label=\alph*., ref=(\alph*), resume]
      \item
        There is some possible event \(e'\) such that \(e'\) is a development of \(e\) and \(\alpha\) is true of \(e'\).
      \end{enumerate}
    \end{enumerate}
    \vspace{-\baselineskip}
  \end{assumption}

  \assuPP{2}, shift evaluation to some possible event in which something related is true.

  Possible here is arbitrary.
  Important is development.

  Though, in same way it is common to restrict attention to some sense of possibility via an adjective, we may speak instead of( event-)continuative-possibility.

  So, task of an account of the progressive is to narrow the relevant sense of continuative-possibility.
  \assuPP{2} holds that success is a necessary condition on continuative-possibility.
  Applied, in particular, to concluding, \assuPP{0} holds that a agent is concluding \(\pv{\phi}{v}\) from \(\Phi\) only if there is some continuative-possibility in which the agent concludes \(\pv{\phi}{v}\) from \(\Phi\).

  Here, \fc{}.
  concluding \(\pv{\phi}{v}\) from \(\Phi\).
  Concludes \(\pv{\phi}{v}\) from \(\Phi\).
  \fc{2}.
\end{note}

\begin{note}
  Paired with choice, allows complex `incomplete' actions.
  Again, progressive develops.

  \begin{illustration}[Darts]
    There is a \pevent{} in which agent scores 180 at darts just in case there is some action available to the agent, such that if the agent were to perform the action they would be scoring 180 at darts.
  \end{illustration}

  Slightly more interesting.
  Determine the available actions.
  Though, similar, no guarantee.
  Hand is knocked at point of release, still scoring.

  Scoring 180 is a complex action.
  Though, interesting.
  First throws don't matter.

  Again, key idea is that sufficient understanding of progressive.

  And, case of interest:

  \begin{illustration}[Concluding]
    There is a \pevent{} in which agent concludes \(\pv{\phi}{v}\) from \(\Phi\) just in case there is some action available to the agent, such that if the agent were to perform the action they would be concluding \(\pv{\phi}{v}\) from \(\Phi\).
  \end{illustration}

  What is it to be concluding something.
  Like crossing the road, fail to complete.
  Like darts, recover from a bad opening.
\end{note}

\begin{note}
  Intuitive distinction between which actions may and may not perform.

  However,~\ref{def:PE:action} without.
  Allow arbitrary division of actions, what matters is immediate.

  Then, agent doing \(a\) is in progressive, so make sure that doing \(a\) is also instance of \(\alpha\).
\end{note}


\begin{note}
  Still, no full account of the progressive.
  Quite difficult.
  Progressive is familiar, intuitive understanding.
  Work through in sufficient detail to be useful.

  A little on choice.
  Then, highlight issue with ability.
  Then, present and modify \citeauthor{Landman:1992wh}'s (\citeyear{Landman:1992wh}) account of the progressive.
\end{note}

\subparagraph*{Summary}

\begin{note}[Summarising]
  To summarise the preceding:
  We began with the definition of a \fc{} (\autoref{def:fc}).
  Definition of a \fc{} relies of the idea of a \pevent{}.
  And, defined \pevent{} in terms of the truth of the progressive aspect applied to a minimal event.

  The exact details of \pevent{} depends on progressive.

  However, gnarly.
  Before turning to the progressive, consider ability.
  The following section --- \autoref{cha:sec:fcs-def:ability} --- will raise difficulties with this suggestion.%
  \footnote{
    And implicitly suggest that any sense of ability sufficient for purpose may be analysed in terms of progressive.
  }
\end{note}


\section{Illustrations}
\label{cha:fcs:sec:illu}

\begin{note}
  Intuition.
  In particular, proofs.

  Before turning to detailed account of \fc{1} in \autoref{cha:sec:fcs-def}, handful of \illu{1}.

  This section consists of two parts:

  \begin{enumerate}[label=]
  \item
    \TOCLine{cha:fcs:sec:illu:yes}

    Cases in which \(\pv{\phi}{v}\) from \(\Phi\) (plausibly) \emph{is} a \fc{}.
  \item
    \TOCLine{cha:fcs:sec:illu:no}

    Cases in which where \(\pv{\phi}{v}\) from \(\Phi\) is \emph{not} (clearly) a \fc{}.
  \end{enumerate}
\end{note}

\subsection{\fc{3}}
\label{cha:fcs:sec:illu:yes}

\begin{note}[Chess I]
  \begin{illustration}[Chess I]
    \label{illu:fc:chess:I}
    Consider the following game state:

    \mbox{ }\hfill%
    \begin{adjustbox}{minipage=\linewidth,scale=.9}
      \centering
      \newchessgame[
      setwhite={pa2,pb2,pc2,pd3,pf2,pg3,ra1,re1,bd4,kg1,qe5},
      addblack={ra8,pa7,ba6,pb5,rc8,pd5,pf2,kg8,qg4,ph7,ph4},
      ]%
      \setchessboard{showmover=false}%
      \chessboard
    \end{adjustbox}%
    \label{fig:chess:easy}%
    \hfill\mbox{ }

    Is possible for White to checkmate in a single move?%
    \footnote{
      \citeauthor{Emms:2000aa}' Puzzle 113 (\citeyear[33]{Emms:2000aa}).
    }
  \end{illustration}
\end{note}

\begin{note}
  \fc{2} of interest:%
  \footnote{
    May also consider whether is possible for White to checkmate in a single move to be a \fc{}, in the sense that it is possible to decide.
    In this case, reduces to solution.
    However, in general decidable may be \fc{} without either answer being a \fc{}.
    In particular, consider an arbitrary first-order formula.
    First-order logic is decidable, however determining which is a different matter.
  }
  \begin{itemize}
  \item
    It is possible for White to checkmate in a single move.
  \end{itemize}
  Clearest way is to do the reasoning, and then observe would not have made a different conclusion.
\end{note}

\begin{note}
  Consider different pieces.
  If like me, first move (\wmove{Qe8}) does not result in checkmate.
  However, do not conclude that it is not possible.
  Other moves to check (e.g\ \wmove{Qf6}, \wmove{Re4}, \wmove{h4}, etc.).

  At some point, consider moving the queen from e5 to h8, which results in checkmate.

  Simple.
  Key observation is that although not immediate, conclude it is possible for White to checkmate in a single move.
  And, would not have concluded otherwise, or indeed concluded something incompatible which would have prevented (for example, that it is possible for Black's king to move to b8 between the two rooks.)

  Perhaps get bored or distracted, and didn't conclude.
  Remains the case that \fc{}.
\end{note}

\begin{note}
  Following from chess, similar structure.
  \begin{illustration}[Sudoku]
    \label{illu:gist:sudoku}
    % https://tex.stackexchange.com/questions/91422/tikz-sudoku-circle-and-connect-with-lines-some-cells
    Consider the following Sudoku puzzle:%
    \footnote{
      From~\textcite[84]{Coussement:2007up}.
    }
    \vspace{\baselineskip}

    \mbox{ }\hfill%
    \begin{adjustbox}{minipage=0.45\linewidth,scale=1}
      \centering
        \begin{tikzpicture}[scale=.5]
          \begin{scope}
            \draw (0, 0) grid (9, 9);
            \draw[very thick, scale=3] (0, 0) grid (3, 3);
            \setcounter{row}{1}
            % Single entries
            \setrow { }{ }{ }  { }{ }{ }  {1}{ }{ }
            \setrow { }{ }{ }  { }{ }{ }  { }{5}{ }
            \setrow {9}{ }{ }  { }{ }{ }  { }{ }{2}
            \setrow { }{ }{3}  { }{2}{ }  { }{ }{ }
            \setrow { }{ }{ }  {8}{ }{ }  {4}{6}{5}
            \setrow { }{4}{ }  { }{5}{9}  { }{ }{8}
            \setrow { }{8}{7}  {2}{3}{1}  { }{4}{6}
            \setrow {2}{1}{ }  {5}{ }{ }  { }{ }{3}
            \setrow {3}{ }{6}  {4}{ }{8}  { }{ }{ }
        \end{scope}
      \end{tikzpicture}
    \end{adjustbox}%
    \hfill\mbox{ }
    \vspace{\baselineskip}
  \end{illustration}

  In contrast to~\autoref{illu:fc:chess:I}, \autoref{illu:gist:sudoku} involves a number of salient \fc{}.
  Specifically, for every empty cell in the Sudoku grid consider the disjunction:
  \[
    \bigvee\{ \phi \text{ is a \fc{}} \mid \phi \in \text{Choices} \}
  \]
  where \(\text{Choices} = \{ i\text{ is the correct number to place in the cell} \mid 1 \leq i \leq 9 \}\).

  For a particular instance:
  \begin{itemize}
  \item
    1 is the correct number to place in the centre cell of the centre sub-grid is a \fc{}.
  \end{itemize}

  In other words, the solution to the Sudoku puzzle is a \fc{}, and each part of the solution to the Sudoku puzzle is a \fc{}.

  Indeed, the relevant \fc{1} follow from a basic understanding the rules of Sudoku, in same way that the possibility for White to checkmate follows from understanding of the rules of chess in \autoref{illu:fc:chess:I}.

  Still, in contrast to \autoref{illu:fc:chess:I}, there is a interesting chance of error.
  For example, accidentally placing 9 above the 8 in the bottom centre sub-grid before observing the 9 in the same column.
  Or, 4 in the top-right of the top-left sub-grid before realising already placed 4 in top-left of the top-left sub-grid.

  However, errors do not (necessarily) amount to conclusions.
  One may make an error while completing the Sudoku puzzle, but refrain from concluding that the mistaken number is the correct number to place in the cell.
  Indeed, given the possibility of error one may only conclude \(i\) is the correct number to place in cell \(c\) only when all cells have been filled and they have ensured there are no errors.

  Though, if an agent is less cautious and is inclined to immediately conclude that \(i\) is the correct number to place in cell \(c\) then each instance of the disjunction may be false.
  For, prior to attempting the puzzle there is a possibility that the agent conclude \(i\) is the correct number to place in cell \(c\) and there is a possibility that the agent may conclude \(j\) is the correct number to place in cell \(c\), where \(i \ne j\).
  Hence, there is the possibility that the agent may conclude either of two (or more) conflicting proposition-value pairs.
\end{note}

\begin{note}
  Games.
  Clear set of rules, and understanding of rules leads to answers to certain questions being determined.
  Mathematics and logic.
\end{note}

\begin{note}
  \begin{restatable}[Squish elimination]{illustration}{scenarioPLSquish}
    \label{scen:squish}
    I conclude \((P \rightarrow Q) \rightarrow P, Q \vdash P \land Q\) from the both the following syntactic proof and the soundness of the rules of inference:
    \begin{center}
      \begin{fitch}
        \phantlabel{illu:sP:1}\fa (P \rightarrow Q) \rightarrow P \\
        \phantlabel{illu:sP:2}\fj Q \\
        \phantlabel{illu:sP:3}\fa P & Squish\textbf{Elim:} \hyperref[illu:sP:1]{1} \\
        \phantlabel{illu:sP:4}\fa P \land Q & \(\land\)\textbf{Intro:} \hyperref[illu:sP:2]{2},\hyperref[illu:sP:3]{3}
      \end{fitch}
    \end{center}
  \end{restatable}

  The proof consists of two premises and two rules of inference.
  The two rules of inference are of interest.

  The second rule of inference used is standard `\(\land\)' introduction, and applies to lines \hyperref[illu:sP:2]{2} and \hyperref[illu:sP:3]{3}.
  Where the conditional holds is unclear.
  On the one hand, troubled if failed to show that `\(\land\)' introduction is sound.
  However, testimony\dots

  The first rule of inference is non-standard `Squish' elimination applied to line \hyperref[illu:sP:1]{1}.

  \begin{center}
    % For any formula of the form \((\alpha \rightarrow \beta) \rightarrow \alpha\), infer \(\alpha\).
    \begin{fitch}
      \ftag{\scriptsize i}{\fa (\alpha \rightarrow \beta) \rightarrow \alpha} \\
      \ftag{\scriptsize }{\fa \vdots } \\
      \ftag{\scriptsize j}{\fa \alpha } & Squish\textbf{Elim:}\emph{i} \\
    \end{fitch}
  \end{center}

  \(\alpha\) `squishes' \(\beta\).

  \phantlabel{squish-elimination-proof}
  For a quick proof, suppose \((P \rightarrow Q) \rightarrow P\) is true.
  And for contradiction assume \(P\) is false.
  As \(P\) is false, it immediately follows that \(P \rightarrow Q\) is true.
  Therefore, by the initial supposition, \(P\) is true.
  Hence, we have obtained the desired contradiction.
\end{note}

\begin{note}
  Two ways in which this works.

  First, soundness of `Squish'-elimination.

  Second, from basic rules.

  \begin{center}
    \begin{fitch}
      \fa (P \rightarrow Q) \rightarrow P \\
      \fj Q \\
      \fa \fh P & \\
      \fa \fa Q & \textbf{Reit:} 2 \\
      \fa P \rightarrow Q & \(\rightarrow\)\textbf{Intro:} 3--4 \\
      \fa P & \(\rightarrow\)\textbf{Elim:} 1,5 \\
      \fa P \land Q & \(\land\)\textbf{Intro:} 2,6
    \end{fitch}
  \end{center}
\end{note}


\begin{note}
  \begin{illustration}[Fraction]
    \label{illu:fc:surds}
    \[\frac{(3 + \sqrt{3})^{2} + \sqrt{6}^{2} - (2\sqrt{3})^{2}}{2(3 + \sqrt{3})\sqrt{6}} = \frac{1}{\sqrt{2}}\]
  \end{illustration}

  Granting knowledge of a handful of equalities, beyond basic addition and subtraction,%
  \footnote{
    \(\sfrac{ab}{ac} = \sfrac{b}{c}\),
    \(\sqrt{a b} = \sqrt{a}\sqrt{b}\), and
    \((a + b)^{2} = (a^{2} + 2ab + b^{2})\).
  }
  whether or not the equation is true a \fc{}, and further the truth of the equation is a \fc{}.%
  \footnote{
    \label{illu:fc:surds:fn}
    First, consider the numerator.
    Each element of the numerator may be rewritten as follows:
    \((3 + \sqrt{3})^{2} = 12 + 6\sqrt{3}\), \(\sqrt{6}^{2} = 6\), \((2\sqrt{3})^{2} = 12\).
    By summing the elements we obtain \(6\sqrt{3} + 6\).
    Hence, by rewriting, the numerator may be replaced with, \(2(3\sqrt{3} + 3)\).

    Now consider the denominator.
    Observe we may cancel multiplication by \(2\) from both the numerator and denominator.
    Further, observe \(\sqrt{6} = \sqrt{2}\sqrt{3}\).
    Hence, by distributing we  obtain, \((3\sqrt{3} + \sqrt{3}\sqrt{3})\sqrt{2}\).
    Likewise, observe \(\sqrt{3}\sqrt{3} = \sqrt{9} = 3\).
    Hence, by rewriting the denominator reads \((3\sqrt{3} + 3)\sqrt{2}\).
    As both the numerator and denominator contain \((3\sqrt{3} + 3)\), we may cancel to obtain the desired equality.
  }
  Though, if like me you may end up exploring a handful of unsuccessful ideas before stumbling across the path to the solution.
  In particular, don't conclude any intermediary miscalculations.
  And, keep going until solution is clear.
\end{note}

\begin{note}[Propositional logic generalised]
  Second option, general type of \scen{0} in two ways:

  Soundness:

  \begin{quote}
    Make sure all the rules are sound, whether these are basic or derived rules.
  \end{quote}

  Semantic entailment:

  \begin{scenario}[Propositional logic generalised]
    \label{illu:sketch:prop-logic}
    Suppose an agent has a good grasp of propositional logic.
    In particular:
    \begin{itemize}
    \item
      The agent has a good understanding of some formal proof system.
      For example, some Fitch-style system.
    \item
      The agent has a good understanding of some method to construct semantic proofs.
      For example, by constructing truth tables, or reasoning about valuation functions.
    \item
      The agent understands the proof system is sound.
      That is to say, the agent understands there exists a proof of some sentence \(A\) \emph{only if} \(A\) is true given an arbitrary valuation.
    \end{itemize}
    The agent constructs a proof of \(A\).

    Given the agent's understanding of propositional logic, the agent observes:
    \begin{quote}
      The construction is a proof of \(A\) \emph{only if} \(A\) is true given an arbitrary valuation.
    \end{quote}
  \end{scenario}

  Intuitively, if the agent were to reason about whether \(A\) is true given an arbitrary valuation and failed to conclude \(A\) is true given an arbitrary valuation, then the agent would not conclude the construction is a proof of \(A\).

  If were to go for semantic, then by soundness, works out.
\end{note}

\begin{note}
  \begin{illustration}[Modal logic I]
    \label{illu:fc:logic:CR}
    The modal system obtained from adding \(\Diamond\Box p \rightarrow \Box\Diamond p\) as an axiom to \(\mathbf{K}\) is canonical for the Church-Rosser property.

    I.e. the canonical model \(W,R,V\) for \(\mathbf{K} + \Diamond\Box p \rightarrow \Box\Diamond p\) is such that \(\forall s,t,u((Rst \land Rsu) \rightarrow \exists v(Rtv \land Ruv))\).
  \end{illustration}

  \autoref{illu:fc:logic:CR} is a \fc{} for me.
  Though, in contrast to the previous \illu{1}, I think there is a reasonable change that \autoref{illu:fc:logic:CR} is not a \fc{} for you.

  Fairly routine, but two important things.
  First, grasp on the relevant concepts.
  If you are unaware of how to construct canonical models for normal modal logics, then unlikely that you will complete the relevant proof.
  Second, sufficient familiarity with the relevant concepts.
  The proof is mostly straightforward, though some care needs to be taken in showing that the canonical model for \(\mathbf{K} + \Diamond\Box p \rightarrow \Box\Diamond p\) has the Church-Rosser property.
  Proof by contradiction is my preferred way of obtaining the result, but this requires keeping certain facts about the canonical model in mind.%
  \footnote{
    A slightly more interesting variation is showing that \(\mathbf{K} + \Diamond\Box p \rightarrow \Box\Diamond p\) is (strongly) complete with respect to the class of frame which have the Church-Rosser property without detour via a canonical model.
  }

  Similar features as \illu{1} given above.

  In particular, perhaps clearer than \autoref{illu:gist:sudoku} and \autoref{illu:fc:surds} in terms of mistakes.
  For, go down some wrong path, still will not conclude until counterexample.
  And, this is very hard to get.
\end{note}

\begin{note}
  Four \illu{}.
  Share common characteristic.
  Result of deductive reasoning with more-or-less explicit collection of rules.

  These characteristics are not (nor any other shared characteristic) required.
  Clear account of why \(\pv{\phi}{v}\) from \(\Phi\) is a \fc{}.

  What is required is available information, conclusion, no divergence.
\end{note}

\begin{note}[Non-deductive \illu{1}]
  The following is a simple \illu{} involving non-deductive conclusion:
  \begin{illustration}[Sunny days]
    It's mid summer in the Bay Area.
  \end{illustration}
  For me, it is a \fc{} that it will not rain tomorrow.

  Of course, I recognise there is a possibility that it \emph{may} rain tomorrow.
  However, I haven't checked the weather forecast, and with no information to the contrary I see no way of \emph{failing} to conclude that tomorrow will be sunny.
  You may object, and perhaps I am too quick to conclude that it will not rain tomorrow.

  Still, no matter the gravitas with which I consider the possibility of rain, I am sufficiently committed to some uniformity principle that the principle, combined with past experience, lead me to conclude that it will be sunny tomorrow.
  Hence, prior to reasoning, the truth of the proposition is a \fc{}.%
  \footnote{
    Same extends to various skeptical hypotheses.
    Entertain the possibility that there is no external world, but nothing that prevents me from concluding that there is an external world.
    Though, your perspective on such issues may differ.
  }

  Note, whether or not it rains tomorrow has no bearing on whether or not it is a \fc{} (for me) that it will not rain tomorrow.
  What happens in the future has no direct bearing on what I may (or may not) conclude in the present.%
  \footnote{
    Consider~\citeauthor{Russell:1912th}'s chicken\dots (Cf.~\citeyear[63]{Russell:1912th})
  }
\end{note}

\begin{note}[Poppies]
  We conclude the \illu{1} with a slightly more speculative \illu{0}:
  \begin{illustration}[Poppies]
    \mbox{ }
    \vspace{-\baselineskip}
    \begin{quote}
      Was Tarquinius Superbus in seinem Garten mit den Mohnköpfen sprach, verstand der Sohn, aber nicht der Bote.

      [What Tarquinius Superbus said in the garden by means of the poppies, the son understood but the messenger did not].\newline
    \mbox{ }\hfill\mbox{(Cf.~\cite[3]{Kierkegaard:1983ta}, and~\cite[190]{Hamann:1822vp})}
  \end{quote}
  \vspace{-\baselineskip}
  \end{illustration}
  The above quite is from the epigraph to~\citeauthor{Kierkegaard:1983ta}'s \hyperlink{cite.Kierkegaard:1983ta}{Fear and Trembling}.
  \hyperlink{cite.Kierkegaard:1983ta}{H.\ Hong and E.\ Hong} detail the relevant background:

  \begin{quote}
    When the son of Tarquinius Superbus had craftily gotten Gabii in his power, he sent a messenger to his father asking what he should do with the city.
    Tarquinius, not trusting the messenger, gave no reply but took him into the garden, where with his cane he cut off the flowers of the tallest poppies.
    The son understood from this that he should eliminate the leading men of the city.%
    \mbox{ }\hfill\mbox{(\citeyear[339]{Kierkegaard:1983ta})}
  \end{quote}
  That he should eliminate the leading men of the city was a \fc{0} for Superbus' son, but not for the messenger.
  Or, at the very least Superbus \emph{expected} the command to eliminate the leading men of the city to be a \fc{} for his son.
\end{note}

\subsection{Not clearly \fc{1}}
\label{cha:fcs:sec:illu:no}

\begin{note}
  Provided a handful of (plausible) instances of knowing whether which (plausibly) involve \fc{1}.
  A pair of (plausible) instances whether which (plausibly) do not involve \fc{1}.
\end{note}

\begin{note}[ML II]
  \begin{illustration}[Modal logic II]
    \label{illu:fc:ML2}
    The modal system \(\mathbf{GL} = \mathbf{K} + \Box(\Box p \rightarrow p) \rightarrow \Box p\) is weakly complete with respect to the class of finite strict partial orders (that is, the class of finite irreflexive transitive frames).
  \end{illustration}

  \autoref{illu:fc:ML2} is similar in structure to \autoref{illu:fc:logic:CR}.
  Indeed, both proofs involve constructing a canonical model.
  The key distinguishing feature of \autoref{illu:fc:ML2}, however, is the difficulty of establishing the canonical model has the desired properties.
  In particular, the general method I keep in mind for proving the relevant result requires a syntactic proof that \(\vdash_{\mathbf{GL}} \Box p \rightarrow \Box \Box p\).
  And, as I have failed to recall the relevant syntactic on sufficient occasion, I do not consider the result a \fc{0} from my understanding of modal logic.

  Hence, the result (plausibly) fails to be a \fc{} from my understanding of modal logic because there is no guarantee that I would provide a proof if I set out to do so.

  On the other hand, I have completed the relevant proof a sufficient number of times.
  So, the result is a \fc{0} from whatever premises are associated with my memory.
\end{note}

\begin{note}[Chess II]
  Observation that absence of \fc{} due to failure to conclude extends to other cases.
  What follows is a more difficult chess problem.
  \begin{illustration}[Chess II]
    \label{illu:fc:chess:II}
    Consider the following game state:

    \mbox{ }\hfill%
    \begin{adjustbox}{minipage=\linewidth,scale=0.9}
      \centering
      \newchessgame[
      setwhite={ka5,pa3,pb4,pc4,pe5,pf6,bg5,bh5},
      addblack={pa6,pb7,pc6,pe6,pf7,kc7,nd7,nd4},
      ]%
      \setchessboard{showmover=false}%
      \chessboard
    \end{adjustbox}%
    \label{fig:chess:intro}%
    \hfill\mbox{ }

    It is possible for Black to checkmate in four moves?%
    \footnote{
      \citeauthor{Emms:2000aa}' Puzzle 150 (\citeyear[33]{Emms:2000aa}).
    }
  \end{illustration}
  As with \autoref{illu:fc:ML2} it is plausible that I would not conclude that it is possible for Black to checkmate in four moves or conversely.

  Though perhaps the bound is too low.
  If I gave it my all and attempted to work my way though all the possibilities present it may be the case that I conclude either way.
  Still, there are a lot of moves to consider, and I lack any intuition about which is correct.%
  \footnote{
    \citeauthor{Emms:2000aa} provides the following solution:
    \begin{quote}
      \variation{1... Nb6!}
      (threatening \variation{2... Nb3\#})
      \variation{2. b5}
      (or \variation{2. Bd1 Nxc4+} \variation{3. Ka4 b5\#})
      \variation{2... c5!}
      \variation{3. bxa6 Nxc4+}
      \variation{4. Ka4 b5\#}
      \textbf{(0-1)}%
      \mbox{}
      \hfill
      (\citeyear[46]{Emms:2000aa})
    \end{quote}
    My statement above remains true---I don't have sufficient background to parse this solution.
  }
  And, if you think I am doing myself a disservice, then a variant of \autoref{illu:fc:chess:II} may be restated with and increase number of moves.
\end{note}

\begin{note}
  Conflicting conclusions.

  \illu{3} may be obtained by taking a proposition-value pairing which conflicts with \fc{}.
  For example, consider again \autoref{illu:fc:surds}.
  The following is \emph{not} a \fc{}
  \[\frac{(3 + \sqrt{3})^{2} + \sqrt{6}^{2} - (2\sqrt{3})^{2}}{2(3 + \sqrt{3})\sqrt{6}} = \frac{1}{\sqrt{3}}\]
  For, by applying the reasoning outlined in \autoref{illu:fc:surds:fn}, I would conclude the left hand side of the equation is equal to \(\sfrac{1}{\sqrt{2}}\).
\end{note}

\begin{note}
  \begin{illustration}[Knowing whether and belief]
    \citeauthor{Barker:1975un} suggests the following two principles hold with respect to knowing whether:%
  \footnote{
    \citeauthor{Barker:1975un} also, as far as I can tell, endorses the principles.
  }
    \begin{enumerate}[label=(\Alph*), ref=(\Alph*), noitemsep]
    \item
      \label{Barker:1975un:A}
      If \emph{S} knows whether \emph{p} and \emph{S} believes that \emph{p}, then \emph{p}.
    \item
      \label{Barker:1975un:B}
      If \emph{S} knows whether \emph{p} and \emph{S} believes that not-\emph{p}, then not-\emph{p}.\newline
      \mbox{ }\hfill\mbox{(\citeyear[281]{Barker:1975un})}
    \end{enumerate}
  \end{illustration}
  I suggest neither principle is \fc{}, as you may conclude counterexamples exist to both.%
  \footnote{
    For example, consider two agents, \emph{A} and \emph{B} playing chess where each move is timed.
  It's the end game, and \emph{A} believes that \emph{B} has a winning strategy.
  Further, \emph{A} (plausibly) knows whether \emph{B} has a winning strategy.
  For, an observer has determined whether or not \emph{B} has a winning strategy, and \emph{A} is capable of tracing the reasoning of the observer.
  So, if \ref{Barker:1975un:A} holds then \emph{B} has a winning strategy.
  But, the observer knows that \emph{B} \emph{does not} have a winning strategy, and \emph{A}'s belief is mistaken.
  }
\end{note}



\section{\fc{3} and ability}
\label{cha:sec:fcs-def:ability}

\begin{note}
  Alternative suggestion is to say \pevent{} just in case agent `can \(\alpha\)' or `has the ability to \(\alpha\)'.
  Preferable, ability.

  \begin{quote}
    \(\pv{\phi}{v}\) from \(\Phi\) is a \fc{} just in case agent has the ability to conclude \(\pv{\phi}{v}\) from \(\Phi\) (and has the ability to avoid concluding something incompatible).
  \end{quote}
\end{note}


\begin{note}
  \begin{enumerate}[label=]
  \item
    \TOCLine{cha:sec:fcs-def:ability:abil-gener-spec}

    Distinguish general and specific abilities.
  \item
    \TOCLine{cha:sec:fcs-def:ability:past}

    Specific ability and the past.
  \item
    \TOCLine{cha:sec:fcs-def:ability:control-intuition}

    `Control'.
  \end{enumerate}
\end{note}

\subsection{Ability, general and specific}
\label{cha:sec:fcs-def:ability:abil-gener-spec}

\begin{note}
  Particular sense of ability.

  Recall \autoref{illu:fc:ML2}.

  In general, not a \fc{} that \(\mathbf{GL}\) is weakly complete with respect to the class of finite strict partial orders.
  For, method relies on a syntactic proof \(\vdash_{\mathbf{GL}} \Box p \rightarrow \Box \Box p\).

  In this respect, it seems I do not have the ability to prove \(\mathbf{GL}\) is weakly complete with respect to the class of finite strict partial orders.

  However, if I have just (by some luck) completed or (by some studying) rehearsed a syntactic proof \(\vdash_{\mathbf{GL}} \Box p \rightarrow \Box \Box p\), then the relevant theorem is a \fc{}.

  In short, it may be true that \(\pvp{\psi}{v'}{\Psi}\) is a \fc{} for an agent while it is false that the agent has the ability to conclude \(\pv{\psi}{v'}\) from \(\Psi\).
\end{note}

\begin{note}
  Still, while there may not be an \emph{immediate} link, whether or not \(\pvp{\psi}{v'}{\Psi}\) is a \fc{} may still reduce to ability, when ability is appropriately understood.

  \phantlabel{ability-g-s-dist}%
  \nocite{Maier:2018uo}
  For, we may distinguish between `general', `categorical' or `global' abilities and `specific' or `local' abilities.

  Following \textcite[2]{Whittle:2010wr} the distinction is roughly as follows:%
  \footnote{
    Though, see~\textcite[esp.\ \S4]{Kittle:2015tb} and~\textcite[1--2]{Kikkert:2022wp} for additional discussion.%
  }
  \begin{itemize}[noitemsep]
  \item
    General (or global) abilities concern `what an agent is able to do in a large range of circumstances', while
  \item
    Specific (or local) ability concern `what the agent is able to do now, in some particular circumstances'.
  \end{itemize}

  General is just given in terms of specific.
  Not conversely, where specific is general and circumstances permit.%
  \footnote{
    For an example of this approach, see \citeauthor{Austin:1961vz}'s (\citeyear{Austin:1961vz}) discussion of `categorical' abilities and opportunities:

    \begin{quote}
      Consider the case where what we wish to assert is that somebody had the opportunity to do something but lacked the ability---`He could have smashed that lob, if he had been any good at the smash':
      here the \emph{if}-clause, which may of course be suppressed and understood, relates not to opportunity but to ability.
      \dots
      `He could have read \emph{Emma}, if he had had a copy', does seem to assert `categorically' that he had a certain ability, although he lacked the opportunity to exercise it.%
      \mbox{ }\hfill\mbox{(\citeyear[177]{Austin:1961vz})}
    \end{quote}
  }

  Example of what \textcite{Hackl:1998tt} terms `opportunity-can' (\citeyear[14]{Hackl:1998tt}):

  \begin{quote}
    \begin{enumerate}
    \item[(92)]
      \begin{enumerate}[label=\alph*., ref=(\alph*)]
      \item
        \label{Hackl:OC:a}
        A star gazer can see the solar eclipse of this year from the Cayman islands.\newline
        So if you were a star gazer and if you were on the Cayman islands at the right time you would see this year's solar eclipse.
      \item
        \label{Hackl:OC:b}
        John can see Mary from where he is standing.\newline
        So if you were standing in his place, you would see Mary.
      \end{enumerate}
    \end{enumerate}

    [\ref{Hackl:OC:b}] says that whoever is in this situation located at John's position and has normal eyesight and directs his/her gaze towards Mary will succeed in seeing Mary.%
    \mbox{ }\hfill\mbox{(\citeyear[39]{Hackl:1998tt})}
  \end{quote}
  \citeauthor{Hackl:1998tt}'s analysis straightforwardly extends to \ref{Hackl:OC:a}:
  A star gazer who is in the Cayman islands at the right time this year and looks for the solar eclipse will succeed in seeing the solar eclipse.

  So, a tentative proposal is to understand whether or not \(\pvp{\psi}{v'}{\Psi}\) is a \fc{} for an agent in terms of whether or not the agent has the \emph{specific} ability to conclude \(\pv{\psi}{v'}\) from \(\Psi\).

  Hence, we set aside \citeauthor{Austin:1961vz}'s `categorical' ability.
  Likewise we set aside `general' accounts of ability such as~\citeauthor{Carter:2021wd}'s~(\citeyear{Carter:2021wd}) `fallibilist',~\citeauthor{Kikkert:2022wp}'s~(\citeyear{Kikkert:2022wp}) `robust', and \citeauthor{Maier:2013vk}'s (\citeyear{Maier:2013vk}) `general' account, among others.

  Two issues.
  Specific ability and the past.
  `Control'

  Discussion will centre around \textcite{Boylan:2020aa}.
  Clear that specific ability (\citeyear[23, fn.3]{Boylan:2020aa})
\end{note}

\subsection{(Specific) ability and the past}
\label{cha:sec:fcs-def:ability:past}

\begin{note}
  First is specific ability and what actually happens.
  Two entailments.
  First, \BoyPS{} following \textcite{Boylan:2020aa}.
  Second, \BoyPSC{} the converse of \BoyPS{}.

  Combined, had the ability to if and only if did.

  Embedded in the past.
  So, doesn't say too much.
  However, difficulty with this is \fc{} depends on something not happening.
\end{note}

\begin{note}
  The first entailment is termed `\BoyPS{}'.

  \begin{enumerate}[label=]
  \item
    \label{Boylan:Past-Success}
    \BoyPS{}: \(\text{Past}(S\text{ does }\phi) \Rightarrow \text{Past}(S\text{ is able to }\phi)\)%
    \mbox{ }\hfill\mbox{(\citeyear[\S1.1]{Boylan:2020aa})}
  \end{enumerate}

  \citeauthor{Boylan:2020aa} motivates \BoyPS{} in the following way:
  \begin{quote}
    \begin{quote}
      \textbf{Fluky Dartboard}.
      I am a terrible dartplayer.
      I struggle to even hit the board whenever I take a shot.
      However, I take my shot and I flukily hit the bullseye.
    \end{quote}

    Once I have taken the shot and hit the bullseye, I can compellingly argue:

    \begin{enumerate}
      \setcounter{enumi}{2}
    \item
      I hit the bullseye on that throw.\newline
      So, I was able to hit the bullseye on that throw.
    \end{enumerate}

    If you know that I have been successful, you must concede I was able to.%
    \mbox{ }\hfill\mbox{(\citeyear[2]{Boylan:2020aa})}
  \end{quote}

  Intuitions regarding \citeauthor{Boylan:2020aa}'s case may be unclear.
  However, recall we are interested in \emph{specific} ability.
  Therefore, the argument provided is consistent with \citeauthor{Boylan:2020aa} failing to have the \emph{general} ability to hit the bullseye.%
  \footnote{
    \textcite{Bhatt:2008aa} observes:
    \begin{quote}
      \begin{enumerate}[label=(\arabic*)]
        \setcounter{enumi}{314}
      \item
        (from~\cite{Thalberg:1969ta})
        \begin{enumerate}[label=\alph*., ref=(315\alph*)]
        \item
          \label{Bhatt:Thal-a}
          Yesterday, Brown hit three bulls-eyes in a row.
          Before he hit three bulls-eyes, he fired 600 rounds, without coming close to the bullseye; and his subsequent tries were equally wild.
        \item
          \label{Bhatt:Thal-b}
          Brown was able to hit three bulls-eyes in a row.
        \item
          \label{Bhatt:Thal-c}
          Brown had the ability to hit three bulls-eyes in a row.
        \end{enumerate}
      \end{enumerate}
      From~\ref{Bhatt:Thal-a}, we can conclude~\ref{Bhatt:Thal-b} but not~\ref{Bhatt:Thal-c}.
      Brown could have hit the target three times in a row by pure chance and he does not need to have had any ability for~\ref{Bhatt:Thal-b} to be true.%
      \mbox{ }\hfill\mbox{(\citeyear[167]{Bhatt:2008aa})}
    \end{quote}
    Distinction between `was able' and `had the ability'.
    \citeauthor{Boylan:2020aa} only `was able', and so agrees with \citeauthor{Bhatt:2008aa}.

    Still, the distinction between~\ref{Bhatt:Thal-b} but not~\ref{Bhatt:Thal-c} is due to the `specific'/`general' divide.
    And, indeed, \citeauthor{Bhatt:2008aa}'s proposal, \emph{to my understanding}, identifies `was able' with the specific reading of ability and `had the ability' with the general reading of ability.

    Indeed, \citeauthor{Boylan:2020aa} makes a similar observation with respect to \citeauthor{Maier:2018uo}'s (\citeyear{Maier:2018uo}) hybrid (modal/generic) account of ability. (\citeyear[23, fn.3]{Boylan:2020aa})
  }%
  \(^{,}\)%
  \footnote{
    Finding additional instances of \BoyPS{} has difficult.

    \textcite[1]{Boylan:2020aa} mentions \citeauthor{Austin:1961vz}'s  remark that `it follows merely from the premiss that he does it, that he has the ability to do it, according to ordinary English' (\citeyear[175]{Austin:1961vz}).
    However, reasoning patterns are often not made explicit.
    Most instances of `was able to' seem to correspond to the converse entailment, \BoyPSC{}, discussed below.

    Still, a clear instance comes from \textcite{Taylor:2011uh}:
    \begin{quote}
      Consider, then, the R-statement (S):

      \begin{quote}
        Stilpo walks through the Diomean Gate at t\textsubscript{2}
      \end{quote}

      and assume that statement, tenselessly expressed so as to avoid ambiguity in what follows, to be true.

      \hbox to \hsize{\hfil{\vdots}\hfil}

      if S is true, then it follows that Stilpo was able to be walking through the gate at t\textsubscript{2}, that being, in fact, precisely what he was doing.%
      \mbox{ }\hfill\mbox{(\cite[139--143]{Taylor:2011uh})}
    \end{quote}

    More generally, one may consider the following to be instance of \BoyPS{}, in which the cited proof explains(?) why the ability attribution is true:

    \begin{quote}
      Cantor was able to show (by a proof we will not reproduce here) that \([0, 1]\) is equivalent to the power set of the integers, and thus its cardinal number is \(2^{\aleph_{0}}\).\newline
      \mbox{ }\hfill\mbox{(\cite[65]{Partee:1990tu})}
    \end{quote}

    \begin{quote}
      Blok [\hyperlink{cite.Blok:1980th}{16}] was able to give a detailed analysis of frame incompleteness by drawing on algebraic methods.
      In particular, he did so by investigating splittings (a concept from lattice theory) of the lattice of normal modal logics\dots\newline
      \mbox{ }\hfill\mbox{(\cite[74]{Blackburn:2007wa})}
    \end{quote}
  }

  Second is the converse of \BoyPS{}
  \phantlabel{BoyPSC:Start}

  \begin{enumerate}[label=]
  \item
    \label{Boylan:Past-Success:C}
    \BoyPSC{}: \(\text{Past}(S\text{ is able to }\phi) \Rightarrow \text{Past}(S\text{ does }\phi)\)
  \end{enumerate}

  \citeauthor{Bhatt:2008aa}

  \begin{quote}
    The two readings associated with \emph{be able to} allow different interpretive possibilities for indefinite/bare plural subjects.

    \begin{enumerate}[label=(\arabic*), ref=(\arabic*)]
      \setcounter{enumi}{300}
    \item
      A fireman was/Firemen were able to eat five apples.
      \begin{enumerate}[label=\alph*., ref=(301\alph*)]
      \item
        \label{Bhatt:apples:ae}
        Yesterday at the apple eating contest, a fireman was/firemen were able to eat five apples.
        (Past episodic, actuality implication, existentially interpreted subject)
      \item
        In those days, a fireman were/firemen were able to eat five apples in an hour (Generic, no actuality implication, generically interpreted subject)%
        \mbox{}\hfill\mbox{(\citeyear[160]{Bhatt:2008aa})}
      \end{enumerate}
    \end{enumerate}
  \end{quote}

  \ref{Bhatt:apples:ae} `actuality implication'.%
  \footnote{
    Following \textcite{Alxatib:2019wf}:
\begin{quote}
      Actuality Entailments (AEs) are inferences from premises that appear to be modal, like~\ref{Alxatib:a}, but their content is that the modality is effectuated in the evaluation world~---~\ref{Alxatib:b}.

      \begin{enumerate}[label=(\arabic*)]
      \item
        \begin{enumerate}[label=\alph*., ref=(1\alph*)]
        \item
          \label{Alxatib:a}
          Pierre a dû \phantom{to.pfv} prendre le \phantom{e} train \newline
          Pierre had.to.\textsc{pfv} take \phantom{dre} the train\newline
          `Pierre had to take the train'
        \item
          \label{Alxatib:b}
          \emph{Inference}: Pierre took the train.%
          \mbox{}\hfill\mbox{(\citeyear[701]{Alxatib:2019wf})}
        \end{enumerate}
      \end{enumerate}
    \end{quote}

    \citeauthor{Alxatib:2019wf} stresses the reading of `had' in (1a) is `unambiguously deontic' (\citeyear[703]{Alxatib:2019wf}).

    See \textcite{Asher:2012vr}, \textcite{Bhatt:2008aa}, \textcite{Hacquard:2006to,Hacquard:2009ta}, \textcite{Palmer:1977wb}, \textcite{Pinon:2003te}, and~\textcite{Werner:2011tp} for examples and additional discussion of actuality entailments.
  }
  Follows that a fireman/firemen ate five apples.%
  \footnote{
    In contrast, to \BoyPS{}, examples of \BoyPSC{} are plentiful.
    Two examples involving reasoning follow:

    \begin{quote}
      One can then, because of the special ``linear'' nature of the electrical process, calculate the distortion of a very complicated signal, such as Uncle Fred's voice, simply by treating it as a series of gradual ``turnings on'' and ``turnings off'' of the unit step response and adding up their combined causal influence.
      Using his operational calculus, Heaviside was able to calculate the unit step response in very quick order and then solve more complicated cases in the manner suggested.%
      \mbox{ }\hfill\mbox{(\cite[316]{Wilson:1988wx})}
    \end{quote}

     \begin{quote}
       One senses from a reading of Russell how he was able to overlook this point:
       the trouble was his failure to focus upon the distinction between ``propositional functions'' as attributes, or relations-in intension, and ``propositional functions'' as expressions\dots%
      \mbox{ }\hfill\mbox{(\cite[152]{Quine:1967tv})}
    \end{quote}

  }

  % Put these together, and specific ability, embedded under past tense reduces to what happened.

  % \begin{enumerate}[label=]
  % \item
  %   \label{Boylan:Past-Success:IFF}
  %   \BoyPSIFF{}: \(\text{Past}(S\text{ is able to }\phi) \Longleftrightarrow \text{Past}(S\text{ does }\phi)\)
  % \end{enumerate}

  The difficulty with respect to \fc{1} is that concluding \(\pv{\phi}{v}\) from \(\Phi\) doesn't entail that \(\pvp{\phi}{v}{\Phi}\) was a \fc{}.
  For, Clause~\ref{def:fc:no-pe-bad} of \autoref{def:fc}.

  In short, \(\pvp{\phi}{v}{\Phi}\) is a \fc{} only if there is no potential event in which the agent concludes something incompatible with concluding \(\pv{\phi}{v}\) from \(\Phi\).

  For, given \BoyPS{} it is not possible to express Clause~\ref{def:fc:no-pe-bad} as:
  \begin{enumerate}[label=\emph{n}., ref=(\emph{n})]
  \item
    \label{Ability:past:narrow}
    The agent has the ability to \emph{not} [conclude something incompatible with concluding \(\pv{\phi}{v}\) from \(\Phi\)].
  \end{enumerate}
  As, so long as the agent concludes \(\pv{\phi}{v}\) from \(\Phi\), then the agent (plausibly) won't have concluded something incompatible, and hence the ability attribution will be true.

  Hence, negation must scope over the ability attribution, e.g.:

  \begin{enumerate}[label=\emph{w}., ref=(\emph{w})]
  \item
    \label{Ability:past:wide}
    The agent does \emph{not} [have the ability to conclude something incompatible with concluding \(\pv{\phi}{v}\) from \(\Phi\)].
  \end{enumerate}

  Still, \ref{Ability:past:wide} will differ in truth value from \ref{Ability:past:narrow} only if the negated variant of \BoyPS{} does not hold:
  \begin{enumerate}[label=]
  \item
    \label{Boylan:Past-Success:CQC}
    \BoyPSCQC{}: \(\text{Past}(S\text{ does \emph{not} }\phi) \Rightarrow \text{Past}(S\text{ is \emph{not} able to }\phi)\)
  \end{enumerate}
  \BoyPSCQC{} may seems false on first glace.
  However, the entailment may be motivated in parallel to \BoyPS{}.
  For, suppose \citeauthor{Boylan:2020aa} takes the shot and does not hit the bullseye.
  It seems we may then argue that:

  \begin{enumerate}[label=\arabic*\('\).]
    \setcounter{enumi}{2}
  \item
    You didn't hit the bullseye on that throw.\newline
    So, you were not able to to hit the bullseye on that throw.
  \end{enumerate}
  So, it is by no means clear that \BoyPSCQC{} fails for the relevant sense of ability.%
  \footnote{
    Further, observe the same substitution applied to \BoyPSC{} intuitively holds.
    In particular, consider \citeauthor{Bhatt:2008aa}'s \ref{Bhatt:apples:ae} where the firemen \emph{weren't} able to eat five apples.
  }
\end{note}

\begin{note}
  To summarise, immediate issue is with entailments.
  Seems these don't coincide with \fc{}.

  However, \BoyPS{}, \BoyPSC{}, nor \BoyPSCQC{} only concern past.

  Possible to give an indirect account of ability, tying specifically to present.

  Assuming unique sense of specific ability, explore.

  For, it is not immediate that \BoyPS{}, \BoyPSC{}, and \BoyPSCQC{} to hold where `Future' or `Present' is substituted for `Past'.

  Additional motivation required.

  If do, then significant problem.
  If do not, though, remains a general question about relationship.

  And, if distinct senses of specific ability, issue is identifying relevant sense for reduction.
  In particular, though \BoyPS{} and \BoyPSCQC{} are difficult, \BoyPSC{} is robust, and raises issue.

  Following section will focus on \BoyPS{} and idea of control.
\end{note}

\subsection{(Specific) ability and control}
\label{cha:sec:fcs-def:ability:control-intuition}

\begin{note}[Segue]
  \autoref{cha:sec:fcs-def:ability:past} raised concerns about specific ability and what does (or does not) happen.

  In this section we present and focus on idea of `control' common to a various analyses of specific ability.
  And, we will argue that idea of control as captured by the act conditional analysis of ability is incompatible with an account of \(\pvp{\phi}{v}{\Phi}\) being a \fc{1} for an agent in terms of the agent have the ability to conclude \(\pv{\phi}{v}\) from \(\Phi\).%
  \footnote{
    Part of the interest of \textcite{Boylan:2020aa} is combining the validity of \BoyPS{} with the failure of `Present Success'.
    However, the combination isn't of interest to us.
    Though, ensures specific ability.
  }
\end{note}

\begin{note}[\AbControl{}]
  \phantlabel{ability:control}
  \textcite{Mandelkern:2017aa} express the idea of control as follows:%
  \footnote{
    \label{fn:control-accounts}
    A similar account of the control intuition is found in \textcite{Jaster:2020wv}:

  \begin{quote}
    \dots think of the ability to sing a song, to build a shag, to play tennis --- all have an action as their manifestation: the agent controls what is going on and she also controls whether to exercise the ability at all.%
    \mbox{ }\hfill\mbox{(\cite[34]{Jaster:2020wv})}
  \end{quote}

  And, \citeauthor{Boylan:2020aa}'s (\citeyear{Boylan:2020aa}) statement of the control intuition is limited to a specific example:
      \begin{quote}
        Imagine a great wave is rising and I have dashed into the sea with my surfboard.
        You know nothing about me: perhaps I am one of the world’s great surfers; perhaps I am a fool. [\dots]

    When said before the fact, the claim that I can surf that wave is strong it says that surfing that wave is within my control.
    This intuition, call it the \emph{control intuition}\dots\newline
    \mbox{ }\hfill\mbox{(\citeyear[1]{Boylan:2020aa})}
  \end{quote}

  Similar accounts may be also be found in~\textcite{Brown:1988tl},~\textcite{Kikkert:2022wp}, and~\textcite{Horty:1995wu}.
  }
  {
    \newbox\qqBoxA
    \newdimen\qqCornerHgt
    \setbox\qqBoxA=\hbox{$\ulcorner$}
    \global\qqCornerHgt=\ht\qqBoxA
    \newdimen\qqArgHgt
    \def\Quinequote #1{%
      \setbox\qqBoxA=\hbox{$#1$}%
      \qqArgHgt=\ht\qqBoxA%
      \ifnum     \qqArgHgt<\qqCornerHgt \qqArgHgt=0pt%
      \else \advance \qqArgHgt by -\qqCornerHgt%
      \fi \raise\qqArgHgt\hbox{$\ulcorner$} \box\qqBoxA %
      \raise\qqArgHgt\hbox{$\urcorner$}}

    \begin{quote}
      When someone says \(\Quinequote{\text{I [am able to] }\varphi}\), she is assuring her interlocutors that \(\sem[c]{\varphi}\) is within her control in a certain way.\newline
      \mbox{ }\hfill\mbox{(\citeyear[326]{Mandelkern:2017aa})}
    \end{quote}
  }
  For ease of reference we will refer to the idea expressed via `\AbControl{}'.
\end{note}

\begin{note}[Control via \citeauthor{Schwarz:2020aa}]
  Similar to \citeauthor{Mandelkern:2017aa}, \textcite{Schwarz:2020aa} motivates \AbControl{} as follows:

  \begin{quote}
    Suppose Cyril does not know the first 10 digits of \(\pi\).
    Intuitively,~\ref{Schwarz:pi} is then false.

    \begin{enumerate}[label=(\arabic*), ref=(\arabic*)]
      \setcounter{enumi}{2}
    \item
      \label{Schwarz:pi}
      Cyril can recite the first 10 digits of \(\pi\).
    \end{enumerate}

    \dots when we say that someone can recite the first 10 digits of \(\pi\), we don't just mean that no relevant facts preclude them from uttering `three, one, four,' etc.
    Rather, the agent must have a certain kind of intentional control over performing the act under the description of `reciting digits of \(\pi\)'.%
    \mbox{ }\hfill\mbox{(\citeyear[2]{Schwarz:2020aa})}
  \end{quote}
\end{note}

\begin{note}[Control via \citeauthor{Boylan:2020aa}]
  Likewise, \citeauthor{Boylan:2020aa} (\citeyear{Boylan:2020aa}), inspired by~\textcite{Kenny:1976vh} motivates the idea of control with the following scenario:
  \begin{quote}
    \begin{quote}
      \textbf{Unreliable Dartboard}.
      I am a fairly bad dartplayer.
      I regularly hit the bottom half when I aim for the top; and vice versa.
      But I never miss the board entirely.
    \end{quote}

    I am about to take a shot.
    I am skilled enough to know I will hit the board; so I know the following:

    \begin{enumerate}[label=(\arabic*)]
      \setcounter{enumi}{6}
    \item
      I will hit the top half of the board on this throw or I will hit the bottom half of the board on this throw.
    \end{enumerate}

    But it does not seem that I should ascribe myself either of the following abilities here:

    \begin{enumerate}[label=(\arabic*), ref=(\arabic*), resume]
    \item
      I can hit the top on the throw.
    \item
      I can hit the bottom on this throw.
    \end{enumerate}

    Even the disjunction does not seem true:

    \begin{enumerate}[label=(\arabic*), ref=(\arabic*), resume]
    \item
      \label{Boylan:10}
      I can hit the top of the board on this throw or I can hit the bottom of the board on this throw.%
      \mbox{ }\hfill\mbox{(\citeyear[3]{Boylan:2020aa})}
    \end{enumerate}
  \end{quote}

  Intuitively, \citeauthor{Boylan:2020aa} lacks control over where the dart lands on the board, the exercises control over whether the dart lands on the dartboard.
  (\citeyear[\S2,19--20]{Boylan:2020aa})
\end{note}

\begin{note}[\BoyVS{}]
    As \citeauthor{Boylan:2020aa} observes,~\ref{Boylan:10} may be further expanded into a more complex disjunction of regions on the dartboard.
  (\citeyear[4]{Boylan:2020aa})
  For example, intuitively it is not the case that:
  \begin{enumerate}[label=(\arabic*'), resume]
    \setcounter{enumi}{10}
  \item
    I can hit outside the bullseye this throw or I can hit the upper-left-quadrant of the bullseye on this throw or I can hit the lower-right-quadrant of the bullseye on this throw or \dots
  \end{enumerate}

  Indeed, \AbControl{} leads to the \emph{invalidity} of \BoyVS{}:

  \begin{enumerate}[label=]
  \item
    \label{Boylan:Or-Success}
    \BoyVS{}: \(S\text{ will }\phi \lor S\text{ will }\psi \Rightarrow S\text{ is able to }\phi \lor S\text{ is able to }\psi\)\newline
    \mbox{ }\hfill\mbox{(\citeyear[\S1.2]{Boylan:2020aa})}
  \end{enumerate}
\end{note}

\begin{note}[Need to get precise]
  Now, \AbControl{} is an idea, but is under-specified by the motivation provided.
  \citeauthor{Mandelkern:2017aa} hedge with `in a certain way', \citeauthor{Schwarz:2020aa} hedges with `certain kind', and \citeauthor{Boylan:2020aa} does not provide an explicit statement of the idea (see Footnote~\ref{fn:control-accounts}).
  Hence, \(\pvp{\phi}{v}{\Phi}\) being a \fc{1} for an agent may be equivalent to the agent having the (controlled) ability to conclude \(\pv{\phi}{v}\) from \(\Phi\).
  Or, the proposed equivalence may fail.

  So, interest turns to details of the accounts of `is able to' advanced by \textcite{Mandelkern:2017aa} and \textcite{Boylan:2020aa} in order to obtain sufficient clarity on what \AbControl{} amounts to on their understanding.
\end{note}

\begin{note}[ACA]
  We present a generalised account of the `act conditional' analysis of ability, common to \textcite{Boylan:2020aa}, \textcite{Mandelkern:2017aa}, and \textcite{Schwarz:2020aa}.%
  \footnote{
    Though \citeauthor{Schwarz:2020aa} is non-committal with respect to a formal account of ability (\citeyear[cf.][13]{Schwarz:2020aa}), the spirit of \citeauthor{Schwarz:2020aa}'s analysis is sufficiently close to \citeauthor{Boylan:2020aa}'s for the issue to arise:
    `[A]n agent has the ability to \(\phi\) iff there are accessible worlds at which she \(\phi\)s simply by deciding to \(\phi\).' (\citeyear[19]{Schwarz:2020aa})
    Decision to action, but then the decision itself must sufficiently determine the action.
  }

  \[%
    \sem[c,w]{\text{S is able to }\varphi} = 1\text{ iff }\exists A \in \mathcal{A}_{S,c,w,t}\colon \forall v \in f_{c}(\text{S does }A,w),  \sem[c,v]{\varphi(S)} = 1%
  \]

  Where:
  \begin{itemize}
  \item
    \(f_{c}\) is a selection function from proposition-world pairings to set of worlds.
  \item
    \(\mathcal{A}_{S,c,w}\) is the set of actions that are available to \(S\) in context \(c\) and world \(w\).
  \end{itemize}

  So, \(S\text{is able to }\varphi\) is true at some world \(w\) in context \(c\), just in case there is some action available to the agent, such that for every world in which it is true that \(S\text{ tries to}A\) determined by the selection function \(f_{c}\), it is the case that \(S \varphi\text{s}\).%
  \footnote{
    Strictly, both \citeauthor{Mandelkern:2017aa} and \citeauthor{Boylan:2020aa} omit universal quantification over worlds returned by the selection function.
    For \citeauthor{Mandelkern:2017aa}, as discussed below, the selection function returns a unique world, though, as discussed below, this assumption is problematic.
    For \citeauthor{Boylan:2020aa}, universal quantification is implicit by embedding `\(\varphi(S)\)' under a modal `\(\mathcal{W}\)' corresponding to `will'.
    % Issue is `if performs act, then \dots'
    % Restrictor semantics for conditional.
    % \(\sem[c,w,f]{\text{if }\phi,\psi} = 1 \text{ iff } \sem[c,w,f^{\sem[c,w,f]{\phi}}]{\psi} = 1\).
    % With universal, effectively inserting a modal.
    % Complexity of \citeauthor{Boylan:2020aa}'s account is getting the right modal.
    % Simplicity of \citeauthor{Mandelkern:2017aa}'s account is avoiding modal by assuming unique world.
  }

  Paraphrased, the act conditional analysis of ability holds:
  `\(S\) is able to \(\varphi\)' is true just in case there is some action \(A\) available to \(S\) such that if \(S\) tried to \(A\) then S would \(\varphi\).
\end{note}

\begin{note}[Selection functions]
  The primary difference between the analyses of \citeauthor{Mandelkern:2017aa} and \citeauthor{Boylan:2020aa} is the specification of \(f_{c}\), though in practice the difference seems minor:
  \begin{itemize}
  \item
    For \citeauthor{Mandelkern:2017aa},
    \(f_{c}\) is~\citeauthor{Stalnaker:1968vt}'s selection function.
    I.e.\ \(f_{c}(\psi, w) = \{v\}\) where \(v\) is the `closest' world to \(w\) where \(\psi\) is true.
    (\citeyear[Cf.][314]{Mandelkern:2017aa})

    However, the assumption of a unique `closest' world is clearly problematic given \BoyVS{}.
    For, an agent has the ability to throw a dart at the dartboard.
    Hence, in the closest possible world where the agent attempts to throw a dart, the agent succeeds.
    Further, the dart lands at some exact region of the dartboard.
    Hence, as there is only one closest world to consider, `\(S\text{ it able to throw a dart at the dartboard}\)' is strengthened to `\(S\text{ it able to throw a dart at the \emph{exact region of the} dartboard}\)'.%
    \footnote{
      In particular, \citeauthor{Mandelkern:2017aa} do not require that what the agent tries to do and what the agent does satisfy the same description (\citeyear[310,314]{Mandelkern:2017aa}).

      The same problem applies to the `orthodox approach' of~\textcite{Hilpinen:1969vw}, \textcite{Kratzer:1977aa,Kratzer:1981vn}~and~\textcite{Lewis:1976us}.
      See \textcite[\S1.3]{Boylan:2020aa} and \textcite[\S2]{Mandelkern:2017aa} for more on the orthodox approach.
    }

    Still, a \citeauthor{Lewis:1973th}ian approach where the selection function return a set of `closest' worlds resolves this issue.
    For, we may assume that the closest possible worlds determine some inexact region of the dartboard.
  \item
    For \citeauthor{Boylan:2020aa}, rather than selecting `close' worlds, \(f_{c}\) selects all worlds which are identical to \(w\) up until time \(t\) (in which \(S\) does \(A\)).%
    \footnote{
      Strictly, there is more.
      For, Non-classical Strong Kleene account of disjunction.
      (\citeyear[\S5]{Boylan:2020aa})
      Though, I really don't get it.
      Just think of \(\mathcal{W}\) as \(G\).
      `Indeterminate' is just \(F \phi \land F \lnot \phi\).
    }
  \end{itemize}

  For present purposes, the key part of the act conditional analysis for capturing \AbControl{} is that \(S \varphi\)s \emph{follows from} \(S\text{ does }A\) in all worlds captured by the selection function.

  In this respect, that the agent \(\varphi\)s is a \emph{consequence} of performing \(A\).
  In particular, it is not possible for \(A\) to set `\(\varphi\)ing in motion'.
  For, we have seen with progressive and the imperfective paradox, \(\varphi\)ing does not entail an agent \(\varphi\)s.
\end{note}

\begin{note}[Availability]
  What counts as an available action is a more complex issue.
  Thankfully, the details of \citeauthor{Mandelkern:2017aa} and \citeauthor{Boylan:2020aa} may be avoided.%
  \footnote{
    Specifically, \citeauthor{Boylan:2020aa} says little on what makes it the case that an action is available to an agent:
    \begin{quote}
      I think of an agent's available actions as their options.
      And, for simplicity at least, we can typically think of options as a set of tryings.%
      \mbox{ }\hfill\mbox{(\citeyear[14]{Boylan:2020aa})}
    \end{quote}
    In contrast, \citeauthor{Mandelkern:2017aa} consider the issue in detail.
    In short:
    \begin{quote}
      [A]n action counts as practically available only if the agent knows that it is a way of bringing about the prejacent \emph{relative to a given description of her practical situation}.\newline
      \mbox{ }\hfill\mbox{(\citeyear[321]{Mandelkern:2017aa})}
    \end{quote}
    On my understanding, there is still some gap between knowing an action is a way of bringing something about and performing the action.

    Hence, the account allows for the possibility that an agent has the ability and fails.
    For, may fail to perform the relevant action.
      See \citeauthor{Maier:2013vk} the importance of allowing for failure.
  }

  For present purposes, a sufficient understanding of when an action is available to an agent by considering \citeauthor{Boylan:2020aa}'s scenario illustrating the invalidity of \BoyVS{}.
  For, if the agent throws a dart and it hits a certain region of the dartboard, then the agent performed the act of throwing the dart at that region.
  However, throwing the dart at that region could not have been an action available to the agent on pain of \BoyVS{} having a true premise and true conclusion.
  Hence, if an agent \emph{lacks} the ability to \(\varphi\), then it cannot be the case that there is an action \(A\) available in which the agent \(\varphi\)s by doing \(A\).
\end{note}

\begin{note}[Summary of \AbControl{}]
  To summarise, it seems that on an act conditional analysis of ability, \AbControl{} amounts to the availability of some action \(A\) such that the agent \(\varphi\)ing is a consequence of performing \(A\).
\end{note}

\begin{note}[Difficulty with \fc{1}]
  With understanding of \AbControl{} in hand, we now turn to \fc{1}.

  We make the simple observation that \(\pvp{\phi}{v}{\Phi}\) may be a \fc{} without the agent having appropriate control (in the sense of \AbControl{}) over concluding \(\pv{\phi}{v}\) from \(\Phi\).

  Specifically, cases where \(\pvp{\phi}{v}{\Phi}\) is a \fc{} but for any action \(A\), it is either the case that \(A\) is inconsequential or unavailable.

  The particular \fc{} is of little importance, so we take the abstract \(\pvp{\phi}{v}{\Phi}\)-pairing.
  The key to failure of \AbControl{} is the assumption that the agent does \emph{not} have the ability to avoid distraction.%
  \footnote{
    E.g.\ the agent may be interrupted at any time, become bored, or think of something else they would prefer to do.
  }
  In particular, consider relatively simple tasks such as long but simple calculations, simple sudoku puzzles, basic chess problems, or routine proofs.

  Now, suppose \(\pvp{\phi}{v}{\Phi}\) being a \fc{0} for an agent is equivalent to the agent having the ability to conclude \(\pv{\phi}{v}\) from \(\Phi\) (where \AbControl{} holds for the relevant sense of ability).

  Given \AbControl{} it must be the case that concluding \(\pv{\phi}{v}\) from \(\Phi\) is a result of performing some action \(A\).
  However, by assumption, the agent does not have the ability of avoid distraction, and so \(A\) is not an action available to the agent.
  For, if \(A\) were available to the agent, the agent would \emph{have} the ability to avoid distraction.

  Conversely, suppose any available action allows for the possibility of distraction.
  Then, it straightforwardly follows that concluding \(\pv{\phi}{v}\) from \(\Phi\) is \emph{not} a result of performing that action.
  For, if the agent gets distracted, then the do not conclude \(\pv{\phi}{v}\) from \(\Phi\).

  In short, \AbControl{} requires an available action such concluding \(\pv{\phi}{v}\) from \(\Phi\) is a result of performing some action.
  However, \(\pvp{\phi}{v}{\Phi}\) being a \fc{} tolerates the absence of any such action.

  We may express the difference in either of two ways.
  \begin{enumerate}[label=\arabic*.]
  \item
    In order for \(\pvp{\phi}{v}{\Phi}\) to be a \fc{} it need only be the case that there is some action which results in the agent concluding \(\pv{\phi}{v}\) from \(\Phi\) (and no action where the agent does not conclude anything incompatible), though this action does not need to be \emph{available} to the agent.
  \item
    In order for \(\pvp{\phi}{v}{\Phi}\) to be a \fc{} there must be some action available to the agent, but it need not be the case that concluding \(\pv{\phi}{v}\) from \(\Phi\) (and no action where the agent does not conclude anything incompatible), is a \emph{consequence} of performing the action.
  \end{enumerate}
  As indicated by interest in the progressive, I think the second expression is correct, but either is sufficient observe that \(\pvp{\phi}{v}{\Phi}\) being a \fc{} does not require \AbControl{} over concluding \(\pv{\phi}{v}\) from \(\Phi\).
\end{note}

\subsection{Summary}

\begin{note}
  Entertained reducing \fc{1} to ability.
  Specific, rather than general ability.
  However, questions about specific ability.
  Specifically, with what happens, and the past.
  And, \AbControl{}.
\end{note}

\begin{note}
  Naturally, I have not argued that there is no sense of `ability' such that \(\pvp{\phi}{v}{\Phi}\) being a \fc{0} for an agent is equivalent to the agent having the ability to conclude \(\pv{\phi}{v}\) from \(\Phi\).
  However, identifying the (or a) sense of ability suitable for the equivalence is difficult.

  Hence, we pursue an account of \(\pvp{\phi}{v}{\Phi}\) being a \fc{0} in terms of \pevent{1} and the progressive.
\end{note}

\subsection[Independent difficulty]{Independent difficulty \hfill (Optional)}

\begin{note}
  This is the `plan' account of ability.
  It's kind of insane.
  Whether or not ability reduces to action such that choice and secure outcome.
\end{note}

\begin{note}
  This is kind of wild.
  For, actions are kind of huge.
  Similar to that paper with minimalism about intentions.
\end{note}

\begin{note}
  Our direct interest with account finishes with universal.
  However, clear additional problem.
  Co-operation.
\end{note}

\begin{note}
  Uh, think.
  Has the ability to X with my help.
  There's no action in advance.
  For, whatever is chosen, I intervene prior, changing the course.
  Well, the point is, I only help if the agent gives up on whatever they had been planning to do.

  This isn't odd, cooperative activity.
  So, actually, refine example a little.
  For, point is that there's the cooperation condition.
\end{note}



\section{\fc{3} and \support{0}}
\label{cha:fcs:sec:fcs-support}

\begin{note}
  \autoref{cha:sec:fcs-def}, account of \fc{1}.
  Now tie to \support{0}.
\end{note}

\begin{note}
  \begin{proposition}
    \label{prop:fcs-only-if-support}
    For an agent \vAgent{} and proposition-value-premises pairing \(\pvp{\phi}{v}{\Phi}\):
    \begin{enumerate}
    \item[\emph{If}:]
      \begin{enumerate}[label=\alph*., ref=(\alph*.)]
      \item
        \(\pvp{\phi}{v}{\Phi}\) is a \fc{0}, from \agpe{\vAgent{}'}.
      \end{enumerate}
    \item[\emph{then}:]
      \begin{enumerate}[label=\alph*., ref=(\alph*.), resume]
      \item
        \support{2} holds between \(\pv{\phi}{v}\) and \(\Phi\), from \agpe{\vAgent{}'}.
      \end{enumerate}
    \end{enumerate}
    \vspace{-\baselineskip}
  \end{proposition}

  {
    \color{red}
    Why this is somewhat interesting.
  }
  However, before turning to the argument for \autoref{prop:fcs-only-if-support}, it is important to note the limitations of \autoref{prop:fcs-only-if-support} with respect to \issueConstraint{}.
  For, in order to argue against \issueConstraint{}, need some \(\pvp{\psi}{v'}{\Psi}\) such that answers \qWhyV{}.
  Answer \qWhyV{} only with dependence.
  Does not follow from \fc{0} that we get dependence.

  Note, also, qualifications.
  From the \agpe{}.
  Mirrored in both cases.
  Plausible that variant of \autoref{prop:fcs-only-if-support} holds unqualified.
  However, we have said nothing of \support{} independent of \agpe{}.
\end{note}

\begin{note}
  The argument for \autoref{prop:fcs-only-if-support} is {\color{red} mostly immediate for ideas regarding support}.

  \begin{goal}
    If conclude only if \fc{}, then support, in part, answers \qWhyV{}.
  \end{goal}

  So, to get answer to \qWhyV{}, need dependency.
  Here, if not \support{} then not \fc{}.
  If not \fc{} then not conclude.

  This is fine, just need to be careful with the counterfactual.
  Relation between \support{} and \fc{} is plain conditional.
  So, it survives any counterfactual changes.
\end{note}

\begin{note}[Argument]
  Argument is straightforward.
  Possible support, by assumption.
  Contraposition.
  If not support, then no \fc{}.
\end{note}

\paragraph{Potential \ros{1}}

\begin{note}
  Start with the following proposition.
  \begin{proposition}
    \label{prop:fcs-only-if-pot-support}
    For an agent \vAgent{} and proposition-value-premises pairing \(\pvp{\phi}{v}{\Phi}\):
    \begin{enumerate}
    \item[\emph{If}:]
      \begin{enumerate}[label=\alph*., ref=(\alph*.)]
      \item
        \(\pvp{\phi}{v}{\Phi}\) is a \fc{0}, from \agpe{\vAgent{}'}.
      \end{enumerate}
    \item[\emph{then}:]
      \begin{enumerate}[label=\alph*., ref=(\alph*.), resume]
      \item
        (A) potential (relation of) \support{} holds between \(\pv{\phi}{v}\) and \(\Phi\), from \agpe{\vAgent{}'}.
      \end{enumerate}
    \end{enumerate}
    \vspace{-\baselineskip}
  \end{proposition}

  Argument is fairly straightforward:
  \begin{argument}
    Suppose \(\pvp{\phi}{v}{\Phi}\) is a \fc{0}.
    Then, from \agpe{}, \pevent{} in which concludes.
    Now, consider the \pevent{}.
    The culmination of the event, agent concludes.

    So, from~\autoref{idea:support}, a relation of support holds, from the \agpe{}.

    Therefore, in whatever sense event is potential, \support{} between \(\pv{\phi}{v}\) and \(\Phi\) is likewise potential.
  \end{argument}
  From the \agpe{}, there is no difference between witnessed relation of support and potential relation of support.
\end{note}

\begin{note}
  \emph{Potential} relation of support, but it does not follow that there is a relation of support, from the \agpe{}.
\end{note}

\begin{note}
  \begin{proposition}
    \label{prop:pot-support-onlyIf-support}
    For an agent \vAgent{} and proposition-value-premises pairing \(\pvp{\phi}{v}{\Phi}\):
    \begin{enumerate}
    \item[\emph{If}:]
      \begin{enumerate}[label=\alph*., ref=(\alph*.)]
      \item
        A potential \ros{} holds between \(\pv{\phi}{v}\) and \(\Phi\), from  \agpe{\vAgent{}'}.
      \end{enumerate}
    \item[\emph{then}:]
      \begin{enumerate}[label=\alph*., ref=(\alph*.), resume]
      \item
        A \ros{0} holds between \(\pv{\phi}{v}\) and \(\Phi\), from \agpe{\vAgent{}'}.
      \end{enumerate}
    \end{enumerate}
    \vspace{-\baselineskip}
  \end{proposition}

  \begin{argument}
    \autoref{idea:support:possible}.
    It is possible for there to be.
    So, we have everything needed.
    Both necessary and sufficient.
    Hence, form \agpe{}, relation of support.

    So, every necessary property that does not involve witnessing.
    But, then, every necessary property.
    Therefore, sufficient.
    For, if not sufficient, then missing a necessary property.
    Contradiction.

    Slight issue, disjunction of properties.
    But, this doesn't change the argument.
    Disjunction.
  \end{argument}
\end{note}

\begin{note}
  \color{red}
  Worry.
  \support{2} doesn't rely on witnessing.
  Now, if this goes through, then seems \support{} for any conclusion before making the conclusion.
  However, possible for the agent to reason to different conclusions.
  For, some faulty reasoning.
  Toggle the fault.
  Therefore, \support{} for contradictory conclusions.
\end{note}

\section{`Foregone-concluding'}
\label{sec:fc-progressive}

\begin{note}
  So, action such that would be concluding.
  On understanding of the progressive, no matter what happens, still conclude.
  Might need to get `lucky' with action.
  However, start then path to the conclusion.

  This `luck' is limited.
  Nothing conflicting.
\end{note}

\begin{note}
  Worry, jumping to conclusions.
  But, in this case, \pevent{} in which agent concludes that step of reasoning is bad.
  Hence, incompatible.
\end{note}

%%% Local Variables:
%%% mode: latex
%%% TeX-master: "master"
%%% End:
