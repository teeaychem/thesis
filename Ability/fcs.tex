\chapter{\fc{3} and support}
\label{cha:fcs}

\section{Introduction}
\label{cha:fcs:sec:introduction}

\begin{note}
  So, introduced question, \qzS{}.
  Positive answers to the question, \fc{1}.

  Bring together positive answers and support.

  Here, slightly more general.
  Introduce idea of a \fc{} and observe, support follows.
\end{note}

\begin{note}
  Support.

  Idea here is something distinguished about concluding, but is independent of whether or not the agent has witnessed concluding.

  \autoref{idea:support} and \autoref{idea:support:possible}.

  Noted, independence does not entail relation of support from agent's perspective without concluding.

  Parallel to propositional and doxastic justification.
\end{note}

\section{\fc{3}}
\label{sec:fc3-2}

\begin{note}
  The consequent here, potential event.
  Term this a \fc{}.

  \begin{restatable}[Foregone-conclusions]{definition}{definitionForegoneC}
    For any proposition-value-premises pairing \(\pvp{\psi}{v'}{\Psi}\):

    From \vAgent{}' perspective:
    \begin{itemize}
    \item
      \(\pv{\phi}{v}\) is a \emph{\fc{0}} from some pool of premises \(\Phi\)
    \end{itemize}
    \emph{If and only if}
    \begin{itemize}
    \item
      Potential event such that if the agent were to reason, would conclude.
    \end{itemize}
  \end{restatable}

  Note, \fc{} are more general.
  If \requ{}, then get \fc{}.
  However, \fc{} in general, is not limited to a \requ{}.
\end{note}

\paragraph{Knowing how}

\begin{note}[Overlap]
  Overlap between knowing whether and \fc{1}.

  In particular, when knowing how involves reasoning, and suitably restricted.
\end{note}

\begin{note}
  Ideas regarding \citeauthor{Ryle:1946tu}'s distinction between knowing \emph{how} and knowing \emph{that}~(Cf.~\citeyear{Ryle:1946tu}).

  Now, I confess my understanding of \citeauthor{Ryle:1946tu}'s distinction is limited --- I have not taken whatever opportunities I have had to read through \citeauthor{Ryle:1946tu}'s work.%
  \footnote{
    Though, I understand enough from passing commentary to note that the idea \emph{I} am perusing here does not, strictly, require that knowledge how and knowledge that are distinct kinds of knowledge.
    (See~\textcite{Pavese:2022up} for more!)
  }

  Following analogy from~\textcite{Ryle:2009us}:

  \begin{quote}
    Knowing `\emph{if p, then q}' is, \dots rather like being in possession of a railway ticket.
    It is having a licence or warrant to make a journey from London to Oxford.
    (Knowing a variable hypothetical or `law' is like having a season ticket.)
    As a person can have a ticket without actually travelling with it and without ever being in London or getting to Oxford, so a person can have an inference warrant without actually making any inferences and even without ever acquiring the premisses from which to make them.%
    \mbox{ }\hfill\mbox{(\citeyear[250]{Ryle:2009us})}
  \end{quote}

  Continuing~\citeauthor{Ryle:2009us}'s analogy, in the case of positive answers to \qzS{}:
  What matters is that the agent is currently in possession of the (season) ticket.

  Even if current possession of the (season) ticket is knowledge that, it is present knowledge.
  And, present without being applied.
\end{note}

\begin{note}
  \fc{2} is weaker.
  Knowing, factive.
  Though, plausible that these amount to the same thing in various cases.
  Either because \fc{} is determined by knowing how to.
  Or, because knowing is weakened to the agent's perspective.

  Sudoku puzzles.
  Know how to figure out.
  So, know whether any solution is valid.

  Of course, in certain cases, there are shortcuts.
  Two even numbers, then know whether by checking whether the last digit is even or odd.
  And, other cases, contingent shortcut, such as two of the same number in a square for Sudoku.

  So, really, knowing how to.
\end{note}

\begin{note}[Intuitive cases]
  Knowing whether and knowing how to.
  More or less interchangeable.

  Know whether \(x + y = z\).
  Know how to calculate \(x + y\).
  Indeed, for any \(z\), know whether \(x + y = z\).
\end{note}

\begin{note}
  To \illu{0}, questions and answers.

  Do you know whether \(83\) is prime?

  Not off the top of my head.

  Do you know whether \(28 + 55 = 83\)?

  Sure, but give me a moment.

  Do you know whether \dots

  No.

  Of course, might hold that the agent needs to have figured things out.
  But, then we have a plausible reduction.
  Knowing whether, and witnessing whether.
  Common component.

  Now, idea is a little different, as knowledge implies \factivity{}.
  Interest with concluding is that not necessarily factive.
  From the agent's perspective.

  `Determining whether'.
  Or, rather `\fc{0}'.
\end{note}

\begin{note}
  Potential, again.

  \cite{Bengson:2011th}.
  \begin{quote}
    \emph{Pi}.
    Louis, a competent mathematician, knows how to find the n\(^{\text{th}}\) numeral, for any numeral \(n\), in the decimal expansion of \(\pi\).
    He knows the algorithm and knows how to apply it in a given case.
    However, because of principled computational limitations, Louis (like all ordinary human beings) is unable to find the \(10^{46}\) numeral in the decimal expansion of \(\pi\).%
    \mbox{ }\hfill\mbox{(\citeyear[170]{Bengson:2011th})}
  \end{quote}

  Not clearly a \fc{}.
\end{note}

\begin{note}
  Understanding here.
  Intellectualist and anti-intellectualist views.

  Proposal fits well with anti-intellectualist such as~\citeauthor{Habgood-Coote:2019we}'s (\citeyear{Habgood-Coote:2019we}) Interrogative Capacity View.%
  \footnote{
    \begin{quote}
      \emph{The Interrogative Capacity View}.
      For any context c, subject S, and activity V, an utterance of `S knows how to V' (in its practical-knowledge ascribing sense) is true in c iff c has associated with it a set of practically relevant situations {F1, F2, \dots}, and, for all (or at least most) Fi that are members of {F1, F2, \dots}, S has the capacity to activate knowledge of a fine-grained answer to the question, how to V in Fi?, in the process of V-ing.%
      \mbox{ }\hfill\mbox{(\citeyear[92]{Habgood-Coote:2019we})}
    \end{quote}
  }
  But, this is not the focus.
\end{note}

\begin{note}
  Difference between how and that.
  However, reduction, then still get potential event.
\end{note}

\begin{note}
  Note, principle from~\cite{Barker:1975un} does not hold.
  for example, chess game.
  Know whether it is possible to win, and strong belief that it is not possible.
  Belief doesn't really mean anything.
\end{note}

\section{Proposition}
\label{sec:proposition}

\begin{note}
  \begin{proposition}
    \label{prop:fcs-only-if-support}
    \fc{0} \emph{only if} relation of support, from agent's perspective.
  \end{proposition}

  Argument is straightforward.
  Possible support, by assumption.
  Contraposition.
  If not support, then no \fc{}.

  Note, this proposition does not rely on previous results regarding \qzS{}.
  The role of \qzS{} is to provide instances of \fc{1}.
\end{note}

\section{Argument}
\label{cha:fcs:sec:argument}

\paragraph{Potential relations of support}

\begin{note}
  Start with the following proposition.
  \begin{proposition}
    If \fc{}, then potential relation of support.
  \end{proposition}

  Argument is fairly straightforward:
  \begin{argument}
    Suppose \fc{}.
    Then, from agent's perspective, potential event in which concludes.
    Now, consider the potential event.
    The culmination of the event, agent concludes.

    So, from~\autoref{idea:support}, a relation of support holds, from the agent's perspective.

    Therefore, in whatever sense event is potential, relation of support is likewise potential.
  \end{argument}
  From the agent's perspective, there is no difference between witnessed relation of support and potential relation of support.
\end{note}

\begin{note}
  \emph{Potential} relation of support, but it does not follow that there is a relation of support, from the agent's perspective.
\end{note}

\begin{note}
  \begin{proposition}
    Potential relation of support only if relation of support.
  \end{proposition}

  \begin{argument}
    \autoref{idea:support:possible}.
    It is possible for there to be.
    So, we have everything needed.
    Hence, form agent's perspective, relation of support.
  \end{argument}
\end{note}

\begin{note}
  Okay, so with support, the way I have things set up is that support is linked to \qWhy{}.
  It's a very general attitude, and doesn't cover anything in particular.
  So, in order for support to answer \qWhy{}, it need not be the case that the agent directly answers a question by appeal to a relation of support holding.

  So, this much is okay.

  So, positive answer to \qzS{}.
  So, then, potential concluding event.
  So, a premise.
  However, unqualified.
  From the agent's perspective, there \emph{is} a potential concluding event.

  From this it follows that there is a potential relation of support.
  For, have by assumption that support if agent concludes.
  Agent concludes in potential event.
  So, take the state at the end of the event.
  This is just state in which there is a relation of support.

  Okay.

  Argue now that there is no relevant distinction, from the agent's perspective, between a relation of support and a potential relation of support.

  So, the idea.
  Consider support.
  Now, no real assumptions about what this is.
  Support either consists in or entails various properties \(P\).

  Claim is that, for any property \(P\).
  If \(P\) holds prior to witnessing, then \(P\) holds of the agent given potential relation of support.

  This is immediate.
  For, if property didn't hold, then from agent's perspective, there wouldn't be a potential relation of support.

  Now, pair this together with the possibility of a relation of support.

  This means, everything sufficient for relation of support holds, from agent's perspective.
\end{note}



\newpage

\begin{note}
  \color{red}

  Major part of the argument is that witnessed, then relation of support from witnessing makes no difference to the situation, so long as a \requ{}.

  The only thing this distinguishes is whether the thing actually happened, but it doesn't matter for the relation between \(\pv{\psi}{v'}\) and \(\Psi\).
\end{note}

\section{Conclusions, forgone}
\label{sec:fc3-1}

\paragraph{Premises and past conclusions}

\begin{note}[Premises]
  So, as we have seen with testimony, status of a premises blocks a \requ{}.

  Whether the same may hold for this problem.

  It's the case that, part of agent's present epistemic state that they would conclude.

  Problem is, if attempt and fail, then this premise does nothing.
  Their present epistemic state develops into a dead-end.
\end{note}

\begin{note}[Note!]
  This doesn't hold in general, for all premises.

  In particular, premise is past conclusion.

  Consider cases of being somewhat impaired, e.g., via exhaustion.
  Indeed, exhaustion is interesting.
  Basic consistency checks.
  Should be the case that conclude A, but just concluded \emph{not}-A, or something like this\dots

  Denying that past continues to secure in all instances.
  So, just need the potential to revise perspective on previous conclusion.
\end{note}

\section{\fc{3}}
\label{cha:fcs:sec:fc}


\section{\fc{3} and support}
\label{cha:fcs:sec:fc3-support}

\begin{note}
  \begin{proposition}
    For any path, present epistemic state determines availability of path.
  \end{proposition}

  Start.
  Then, continue.
  Started from \(\Phi\), so will conclude.
  Hence, no matter choice made, must have taken the possibility of this choice into account.
  So, it must be the case that determined.

  Hence, if witness, then via some path.

  So, witnessing predetermined path.
  Any instances of concluding by witnessing reduces to witnessing predetermined path.

  Witnessing may provide information about path, but witnessing doesn't contribute given a \requ{}.

  For any X from W,
  present determines whether or not X from agent's point of view, then forgone conclusion.

  In other words, agent's present epistemic state determines.
  Agent may need to witness to figure out how determined, but witnessing does not influence.
\end{note}

\begin{note}[Two worries]
  Two worries.

  First, that even though \fc{0}, the agent would not conclude.
  Either because \(\Phi\) is unavailable, or because no potential witnessing event.
  So, can't remove \fc{0} from account of why.

  However, then \fc{0} does not support.

  If grant that \fc{0} supports, then this seems to work out.
  Further, if require existence, then things that support get very messy.
  Dopeganger cases.
  Reason is I saw A, but it wasn't A, appealing to something that doesn't exist.
  Various other cases like this.

  Difference.
  In these cases, have premise, thing is that the truth value is distinct.
  Here, possibly no premise.

  Well, this is different.
  However, I don't think this is sufficient to reject the idea.
  Just because this distinction doesn't arise in the case of witnessing doesn't really do much.

  Look, a `bad' premise offers no more support for the agent than no premise.

  Second, need \emph{that} \fc{0}.
  However, the point is that this is about the agent's present epistemic state.
  \emph{Without} \fc{0}, the agent would reason.
  This is just the key point reiterated.
  Know whether, \fc{0} just adds information about which.
\end{note}

\subsection{Argument}
\label{sec:argument}

\subsection{Limitations}

\begin{note}
  Important limitation here is that \(\pvp{\psi}{v'}{\Psi}\) is a \requ{}.

  This need not be the case.

  For example, consider discussion of~\autoref{illu:gist:calc}.
  Testimony.
  Failure of a \requ{}.

  It is possible that a prior conclusion of \(\pv{\psi}{v'}\) from \(\Psi\) prevents \(\pvp{\psi}{v'}{\Psi}\) being a \requ{}.

  However, in general, this is a bad response.
  For, this would result in discounting the agent's present epistemic state.

  No doubt, some conclusions are more secure that others.
  However, such a hierarchy is not temporal.
\end{note}

\section{Foregone-concluding}

\begin{note}
  Tentative suggestion.
\end{note}

\begin{note}[Foregone-concluding]
  Pair this with a key idea.

  \begin{restatable}[Foregone-concluding]{idea}{ideaForegoneCing}
    \label{idea:reassignment}
    If \fc{}, then may conclude.
    %\vspace{-\baselineskip}
  \end{restatable}

  Cases where concluding by witnessing reduces to witnessing forgone conclusion.
  \emph{Concluding \(\pv{\psi}{v'}\) from \(\Psi\) is just witnessing \fc{}.}
  So, reduction, in certain cases.
  Further, if forgone conclusion, then conclude.
  At least, in certain cases.
\end{note}

\section{???}

\subsubsection{Ryle}


\section{Scraps}
\label{sec:scraps}

\begin{note}
  \emph{However}, caution.
  For, as we have seen with testimony, it may be the case that status of a premises blocks a \requ{}.
  And, the argument given relies on the existence of a \requ{}.
  So, it may be the case that past reasoning blocks a \requ{}.
  Still, here, only need to deny this.
  Not saying that in every case agent's present reasoning is given priority.
  (Indeed, consider cases of being somewhat impaired, e.g., via exhaustion.
  Indeed, exhaustion is interesting.
  Basic consistency checks.
  Should be the case that conclude A, but just concluded \emph{not}-A, or something like this\dots)
  Rather, denying that past continues to secure in all instances.
  So, just need the potential to revise perspective on any previous conclusion.
\end{note}

\begin{note}
  An interesting observation here is that in certain this all arises, to a certain extent, because of general abilities.
  General ability spans multiple different proposition-value-premises pairings.
  Hence, all of these function as \requ{1}, so long as the agent has the option.

  General ability spans multiple different proposition-value-premises pairings.
  Hence, all of these function as \requ{1}, so long as the agent has the option.

  \begin{itemize}
  \item
    General and specific abilities.
  \item
    Answers to why, then.
    Note, here, that opportunity is interesting.
    The whole conjunction of all instance of the general ability is plausibly not a \requ{}.
    However, all that's needed is the \emph{individual} instances, and for these to raise a problem.
  \item
    The point is, \requ{1} for any general ability, and these are also \requ{1} for main pairing.
    (%
    Note --- or perhaps emphasise --- here, that the problem is \emph{not} recursive.
    Instead, the problem is about the spread.%
    )
  \item
    Here, then, ability is both the problem and the answer.
    What's interesting is the way in which ability functions.
    It's not merely \emph{that} the agent has the ability.
    Instead, it \emph{is} the ability.
  \end{itemize}
\end{note}

\begin{note}
  So, the way in which past reasoning relates is by ensuring that the agent would reach the same conclusion.
  About the agent's reasoning.
  \emph{How} rather than \emph{that}.

  Look, what we are getting is that the agent would conclude.
  If something were to happen, then some action would be performed.
  There's no distinction between the answer and performing the act, roughly.
  Or, better put, the answer \emph{is about present reasoning}.
  Answer states that in present reasoning, would not fail.


  It is about the agent's present epistemic state, and in particular what the agent's present epistemic state is capable of.

  In other words, ability.
  What answers is ability, in the sense that ability iff would.

  This is very important to the understanding of \fc{}.

  And, I kind of want to have ability as a gloss, while focusing on \fc{} to avoid going into ability in too much detail.

  So, positive answer, then it's the pairing \emph{being} a \fc{}.
  (I should always use this instance of the copula.)
\end{note}


%%% Local Variables:
%%% mode: latex
%%% TeX-master: "master"
%%% End:
