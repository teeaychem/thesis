\chapter{\fc{3}}
\label{cha:fcs}

\nocite{Ryle:1946tu}

\begin{note}
  This chapter introduces the core idea of this document:
  Some \prop{0}-\val{0} pair \(\pv{\phi}{v}\) being a \emph{\fc{}} from some \pool{0} \(\Phi\).

  Paraphrased, \(\pv{\phi}{v}\) is a \fc{} from \(\Phi\) for an agent just in case there is a possible event in which the agent to conclude \(\pv{\phi}{v}\) from \(\Phi\) and \evals{} every \prop{} in \(\Phi\) with its corresponding \val{} prior to the event.
\end{note}

\begin{note}
  The initial role of \fc{1} is to help characterise \ros{}.
  In turn, price characterisations of \qWhy{} (and \qHow{}).
  Hence, \fc{1} are compatible with \issueInclusion{}, though are also part of the recipe for counterexamples to \issueInclusion{} we develop.
\end{note}

\begin{note}
  For background intuition, the technical term `\fc{}' associates with (by is not an analysis of) a sense of the common term `foregone conclusion' as an expected result of reasoning.%
  \footnote{
    For example:

    \begin{quote}
      [\dots] Russell's evaluation of such sentences as false is predetermined by his existence presuppositional semantics for the ‘existential' quantifier, and by the fact that his logic permits no alternative means of considering the semantic status of sentences ostensibly containing proper names for nonexistent objects.
      This makes it an altogether philosophically foregone conclusion that sentences like ‘Pegasus is winged,' which many logicians would otherwise consider to be true propositions of mythology, are false.%
      \mbox{ }\hfill\mbox{(\cite[6]{Jacquette:2002up})}
    \end{quote}

    \noindent \citeauthor{Jacquette:2002up} is discussing what follows from~\citeauthor{Russell:1905aa}'s analysis of definite descriptions.
    Specifically, from \citeauthor{Russell:1905aa}'s analysis it follows \propI{Pegasus is winged} is \valI{False}.
    (There are no four-legged winged mammals, etc.)
  }
  This sense of the term `foregone conclusion' contrasts against a sense with which a \prop{0}-\val{0} pair has been decided in advance of reasoning.%
  \footnote{
    For example:
    \begin{quote}
      When can a Bayesian select an hypothesis \emph{H} and design an experiment (or a sequence of experiments) to make certain that, given the experimental outcome(s), the posterior probability of \emph{H} will be greater than its prior probably?
      We discuss an elementary result that establishes sufficient conditions under which this reasoning to a foregone conclusion cannot occur.%
      \mbox{ }\hfill\mbox{(\cite[1228]{Kadane:1996vu})}
    \end{quote}
  }
  And, a sense in which a forgone conclusion is some unavoidable state of affairs.%
  \footnote{
    For example:
    \begin{quote}
      [どうぜ][\dots] Expresses an attitude of resignation or carelessness on the part of the speaker, in the sense that regardless of what s/he does, the conclusion or outcome is foregone and cannot be changed by the will or effort of an individual.%
      \mbox{ }\hfill\mbox{(\cite[332--333]{kurufushamashii:2015un})}
    \end{quote}
    See also \citeauthor{Grice:1957vg}'s discussion of intention recognition (\citeyear[385]{Grice:1957vg}/\citeyear[219]{Grice:1989uf}), and \citeauthor{Machover:1996vu}'s preface of their approach to the G\"{o}del-Rosser First Incompleteness Theorem (\citeyear[viii]{Machover:1996vu}).
  }
\end{note}


\section{Definition}
\label{cha:fcs:def}

\begin{note}[\fc{2} definition]
  \begin{definition}[\fc{3}]%
    \label{def:fc}%
    \vspace{-\baselineskip}
    \begin{itemize}
    \item
      \(\pv{\phi}{v}\) is a \emph{\fc{0}} from \(\Phi\) for \vAgent{} throughout\(e\).
    \end{itemize}

    \emph{If and only if}

    \begin{itemize}
    \item
      Throughout \(e\) there is some action \(a\) \vAgent{} may immediately perform such that both \ref{def:fc:act} and \ref{def:fc:result} are true:
      \begin{enumerate}[label=\alph*., ref=(\alph*), series=fcCounter]
      \item
        \label{def:fc:result}
        For each \prop{0}-\val{0} pair \(\pv{\phi'}{v'}\) in \(\Phi\), \vAgent{} \evals{} \(\phi'\) as having value \(v'\) prior to doing \(a\).

      \item
        \label{def:fc:act}
        The event \(e^{\sharp}\) in which \vAgent{} does \(a\) is an event in which \vAgent{} is concluding \(\pv{\phi}{v}\) from \(\Phi\).
      \end{enumerate}
    \end{itemize}
    \vspace{-\baselineskip}
  \end{definition}

  \noindent%
  The idea of a \fc{} is to capture a possible conclusion via the sense of possibility required for an event to be in progress.
  For, by \assuPP{} if concluding, then event in which agent concludes.

  The restriction to present \evals{} and progressive captures the sense in which the conclusion is `foregone'.
  For, the agent does not require any novel information, and there is something about the agent which secures event in progress when the agent does the relevant action.
  In other words, sufficient to consider attributes of the agent as the are in order to see the conclusion is possible.

  Relative to an event \(e\).
  This is for ease of reference in arguments to follow.
  Intuitively, time \(t\).
  And, \fc{} as defined just in case \fc{} at each point of time \(t\) in the span of event \(e\).
\end{note}

\newpage

\section{Illustrations}
\label{cha:fcs:illu}

\begin{note}
  Still, a \fc{} involve some ingenuity.
  For example, consider the following \scen{}:
  
  \begin{scenario}[\cite[9]{Dudeney:1995aa}]
    \label{scen:fc:chick}%
    Three chickens and one duck sold for as much as two geese; one chicken, two ducks, and three geese were sold together for \$25.00.
    What was the price of each bird in an exact number of dollars?
  \end{scenario}

  \begin{enumerate}[label=C\thescenarioCounter., ref=(C\thescenarioCounter)]
  \item
    \label{scen:fc:chick:c}
    \pv{\propI{The price of a chicken was \$2.00, for a duck \$4.00, and for a goose \$5.00}}{\valI{True}}
  \end{enumerate}
  % 
  Solution to the puzzle follows from a grasp of basic algebra, capacity to recast natural language problems as algebraic problems, and some \emph{je ne sais quoi}.
  The puzzles is fairly simple, but it's not just crunching through equations.
\end{note}


\begin{note}[Chess I]
  \begin{scenario}[\citeauthor{Emms:2000aa}' Puzzle 113 (\citeyear[33]{Emms:2000aa})]%
    \label{illu:fc:chess:I}%
    \mbox{ }\hfill%
    \begin{adjustbox}{minipage=\linewidth,scale=.8}
      \centering
      \newchessgame[
      setwhite={pa2,pb2,pc2,pd3,pf2,pg3,ra1,re1,bd4,kg1,qe5},
      addblack={ra8,pa7,ba6,pb5,rc8,pd5,pf7,kg8,qg4,ph7,ph4},
      ]%
      \setchessboard{showmover=false}%
      \chessboard
    \end{adjustbox}%
    \label{fig:chess:easy}%
    \hfill\mbox{ }

    \begin{center}
      Is possible for White to checkmate in a single move?
    \end{center}
    \vspace{-\baselineskip}
  \end{scenario}
\end{note}

\begin{note}
  The \prop{0}-\val{0} pair of interest is:

  \begin{enumerate}[label=C\thescenarioCounter., ref=(C\thescenarioCounter)]
  \item
    \label{illu:fc:chess:I:c}
    \pv{\propI{It is possible for White to checkmate in a single move}}{\valI{True}}
  \end{enumerate}
  % 
  Whether or not \ref{illu:fc:chess:I:c} is a \fc{} depends on details about the agent.
  Assume the agent has a basic understanding of chess and is motivated to solve \citeauthor{Emms:2000aa}' Puzzle 113.

  Need an action, \pool{}, and argument that when agent does the action the agent is concluding.

  Consider the action described by:

  \begin{center}
    `Begin an attempt to solve \citeauthor{Emms:2000aa}' Puzzle 113 by exhaustive search'.
  \end{center}

  Consider a \pool{} which contains \prop{0}-\val{0} pairs which capture rules of chess and the game state.
  \pool{} prior to reasoning.
  For, the rules of chess are independent, and do not need to start reasoning in order to inspect setup of puzzle.

  Now, argue that agent is concluding.
  Consider two (motivated) conditionals:

  \begin{enumerate}[label=\arabic*., ref=(\arabic*)]
  \item
    \label{illu:fc:chess:I:cond:1}
    If agent picks \wmove{Qh8}, then agent is concluding \ref{illu:fc:chess:I:c}.%
    \smallskip

    By assumption, the agent has a basic understanding of chess and is motivated.
    And, as the agent needs to verify \wmove{Qh8} in checkmate, the agent will.
  \item
    \label{illu:fc:chess:I:cond:2}
    If agent picks a move other than \wmove{Qh8} then after some reasoning the agent picks a novel move.%
    \smallskip

    By parallel reasoning.

    The agent needs only verify the move other than \wmove{Qh8} fails to result in checkmate, and then pick some other move.
    The agent will verify, given their understanding of chess.
    And, the agent will pick some novel move as their strategy is exhaustive search.
  \end{enumerate}
  % 
  Now, whichever move the agent picks, either \ref{illu:fc:chess:I:cond:1} or \ref{illu:fc:chess:I:cond:2} is true.
  And, as there are finitely many moves for the agent to pick, \ref{illu:fc:chess:I:cond:1} will (eventually be true).
  Hence, as an event in which the agent concludes \ref{illu:fc:chess:I:c} is in progress when \ref{illu:fc:chess:I:cond:1} is true, an event in which the agent concludes \ref{illu:fc:chess:I:c} is (also) in progress when \ref{illu:fc:chess:I:cond:2} is true.
\end{note}

\begin{note}
  This argument assumes exhaustive search.
  This need not be the case, but alternative strategies are difficult to describe.
  If you have a basic understanding of chess then I suggest you convince yourself the conclusion is a \fc{} by attempting the puzzle.
  You need only conclude \ref{illu:fc:chess:I:c} and verify that you were concluding \ref{illu:fc:chess:I:c} after you began.
  And, that you did only made use of \prop{0}-\val{0} \eval{} which held prior to the action.
\end{note}

\begin{note}[Chess II]
  Still, a basic understanding of chess does not entail any chess puzzle is a \fc{}.
  For example, consider \citeauthor{Emms:2000aa}' Puzzle 150:

  \begin{scenario}[\citeauthor{Emms:2000aa}' Puzzle 150 (\citeyear[33]{Emms:2000aa})]%
    \label{illu:fc:chess:II}%
    \mbox{ }\hfill%
    \begin{adjustbox}{minipage=\linewidth,scale=0.8}
      \centering
      \newchessgame[
      setwhite={ka5,pa3,pb4,pc4,pe5,pf6,bg5,bh5},
      addblack={pa6,pb7,pc6,pe6,pf7,kc7,nd7,nd4},
      ]%
      \setchessboard{showmover=false}%
      \chessboard
    \end{adjustbox}%
    \label{fig:chess:intro}%
    \hfill\mbox{ }

    \begin{center}
      It is possible for Black to checkmate in four moves?
    \end{center}
    \vspace{-\baselineskip}
  \end{scenario}

  \noindent%
  The conclusion of interest is:
  % 
  \begin{enumerate}[label=C\thescenarioCounter., ref=(C\thescenarioCounter)]
  \item
    \label{illu:fc:chess:II:c}
    \pv{\propI{Black checkmates in four moves}}{\valI{Possible}}
  \end{enumerate}
  % 
  The difference between \autoref{illu:fc:chess:II} and \autoref{illu:fc:chess:I} is the difficulty of the puzzle.%
  \footnote{
    \citeauthor{Emms:2000aa} suggests:
    \textquote{%
      \variation{1... Nb6!}%
      (threatening \variation{2... Nb3\#})%
      \variation{2. b5}%
      (or \variation{2. Bd1 Nxc4+} \variation{3. Ka4 b5\#})%
      \variation{2... c5!}%
      \variation{3. bxa6 Nxc4+}%
      \variation{4. Ka4 b5\#}%
      \textbf{(0-1)}%
    }
    (\citeyear[46]{Emms:2000aa}).
  }
  And, a basic understanding of chess does not imply the capacity to work through a complex sequence of moves in the relevant situation.%
  \footnote{
    I gave up after fifteen minutes or so.
    Hence it seems~\ref{illu:fc:chess:II:c} was not a \fc{} for me.
  }\(^{,}\)%
  \footnote{
    An agent may get guess a suitable sequence of four moves and verify they work, but if an agent is guessing, there is nothing to guarantee there next guess is correct.
    Hence, in this case \ref{illu:fc:chess:II:c} likewise fails to be a \fc{} from some \pool{}.
    And, more generally we define \fc{0} in terms of events in progress in part to rule out possible conclusions of this kind.
  }
  And, if the agent lacks the capacity to work through the relevant complex sequence, then an event in which the agent concludes \ref{illu:fc:chess:II:c} from some \pool{} is not in progress.
  Or, the agent may have the relevant capacity but have no interest in chess puzzles, be too resource constrained.

  Indeed, \(\pv{\phi}{v}\) may fail to be a \fc{} from \(\Phi\) but be a \fc{} from some distinct \pool{} \(\Phi'\) as the agent prefers reasoning from \(\Phi'\) as opposed to \(\Phi\).
  Using a chess engine to solve chess problems seems to defeat the purpose of thinking about the problem, but I almost always prefer to use a calculator for mildly complex arithmetic such as \(4^{4!}\) over my own understanding of arithmetic.
\end{note}


\begin{note}
  In broad structure, the idea with \autoref{illu:fc:chess:I} is some effective method, sufficient information to apply method, and an opportunity to apply.

  For example, consider arithmetic.
  I expect the truth of \(13 \cdot 4 = 52\), \(96 \div 4 = 24\), and \(23 \cdot 15 = 345\) are \fc{1}.
  Likewise, if you have basic understanding of propositional logic, then the validity of various theorems are \fc{1}.
  And, if you enjoy Sudoku puzzles then, so long as you have the puzzle and sufficient time to spare, the solution to any puzzle is likely a \fc{}.
\end{note}

\begin{note}
  In some cases it is possible to easily point to something which prevents an agent from drawing a conclusion.
  For example, consider the following story:

  \begin{scenario}[A copper kettle --- \cite[62]{Freud:1960wx}]%
    \label{illu:kettle}%
    A.\ borrowed a copper kettle from B.\ and after he had returned it was sued by B.\ because the kettle now had a big hole in it which made it unusable.
    His defence was:
    ``%
    First, I never borrowed a kettle from B.\ at all;
    secondly, the kettle had a hole in it already when I got it from him;
    and thirdly, I gave him back the kettle undamaged.%
    ''
  \end{scenario}

  \noindent%
  Consider an agent who has listened to A.'s defence.
  The conclusion of interest is:
  % 
  \begin{enumerate}[label=C\thescenarioCounter., ref=(C\thescenarioCounter)]
  \item
    \label{illu:kettle:c}
    \pv{\propI{A.'s defence is testimony}}{\valI{True}}
  \end{enumerate}
  % 
  Further, suppose the agent is committed to the following principle:%
  \footnote{
    Motivated by the observations that
    \begin{enumerate*}[label=(\alph*), ref=(\alph*)]
    \item A.\ has provided testimony only if what A.\ has said is true.
      And,
    \item what A.\ has said is true only if the three points of A.'s defence are jointly consistent.
    \end{enumerate*}
  }
  % 
  \begin{itemize}
  \item
    A.\ has provided testimony \emph{only if} if the three points of A.'s defence are jointly consistent.
  \end{itemize}
  % 
  Then, \ref{illu:kettle:c} is not a \fc{} from any \pool{}, as it is not possible for the agent to conclude \pv{\propI{The three points of A.'s defence are jointly consistent}}{\valI{True}}.
  Indeed, it seems clear \pv{\propI{The three points of A.'s defence are jointly \emph{in}consistent}}{\valI{True}} is a \fc{} from any relevant \pool{}.
  For, it takes only a moments reflection to observe it is not possible to return a kettle one has not borrowed.
\end{note}

\begin{note}
  Still, nothing prevents two conflicting \prop{0}-\val{0} pair being \fc{1}.
  For example, consistency of testimony, but may also be swayed by A.'s rhetoric.
  So, focus on content, or focus on presentation.
  Different actions, different conclusions.

  A \fc{} only captures possible conclusion.
  Nothing in particular hangs on this.
  We refrain from placing additional constraints on \fc{1} in order to keep the definition of a \fc{} simple.
  If you prefer to strengthen definition with justification, etc. that's okay.
  And, we implicitly assume \fc{1} are sensible.
\end{note}

\begin{note}
  Finally, in some cases a \prop{0}-\val{0} pair may fail to be a \fc{} from a specific \pool{} of premises as an agent requires 

  \begin{scenario}[Stag hunt]%
    \label{fc:sh}%
    Two hunters may hunt either stag or hare.
    It is possible for each hunter to capture hare alone, but to capture a stag requires cooperation.
    Hence, the hunters expect a payoff for hunting hare regardless of what the other does, but they one expect a payoff for hunting hare if they do so together.
    And, as a stag is much larger than a hare, the expected payoff for a hunting stag (together) is higher than hunting hare.
  \end{scenario}

  \noindent%
  The conclusion of interest is:

  \begin{enumerate}[label=C\thescenarioCounter., ref=(C\thescenarioCounter)]
  \item
    \label{fc:sh:c}
    \pv{\propI{Hunt stag}}{\valI{Do}}
  \end{enumerate}
  % 
  \ref{fc:sh:c} is plausibly fails to be a \fc{} for either hunter without information about what the other hunter is inclined to do, as there is not expected payoff for hunting hare alone.%
  \footnote{
    \citeauthor{Skyrms:2004aa} expands:
    \begin{quote}
      It is clear that a pessimist, who always expects the worst, would hunt hare [and] a cautious player, who was so uncertain that he thought the other player was as likely to do one thing as another, would also hunt hare.
      [\dots]
      That is not to say that rational players could not coordinate on the stag hunt equilibrium that gives them both a better payoff, but it is to say that they need a measure of trust to do so.%
      \mbox{ }\hfill\mbox{(\citeyear[3]{Skyrms:2004aa})}
    \end{quote}
    And, we may say \pv{\propI{Hunt hare}}{\valI{Do}} is a \fc{} for the both a pessimist and cautious player.
  }

\end{note}

\section{Notes}

\begin{note}
  \color{red}
  Here, ability versus compulsion.
  Nothing really hangs on this.%
  \footnote{
    \label{fn:fc-ability}
    In particular, I suspect `\fc{3}' is close to an ability modal while `foregone conclusion' is close to a compulsion modal.
    (See (\cite{Mandelkern:2017aa}) for discussion).
    Indeed, our definition of a \fc{} is close to an act conditional analysis of ability.
    However, there is no immediate parallel.
    We return to this observation in \autoref{cha:sec:fcs-def:ability}.
  }
\end{note}

\begin{note}
  % The definition of a \fc{} may be strengthened in various ways.
  % For example:

  % \begin{enumerate}[label=\alph*., ref=(\alph*), resume*=fcCounter]
  % \item
  %   \label{def:fc:alt:c}
  %   For any proposition \(\chi\), value \(v''\), and action \(b\) such that \vAgent{} is concluding \(\pv{\chi'}{v''}\) from \(X\) after \(b\) is done:
  %   \begin{itemize}
  %   \item
  %     Either~\ref{def:fc:extra:1} or~\ref{def:fc:extra:2} is the case:
  %     \begin{enumerate}[label=\arabic*., ref=\arabic*]
  %     \item
  %       \label{def:fc:extra:1}
  %       \(\chi\) is \(\psi\) and \(v''\) is \(v'\).
  %     \item
  %       \label{def:fc:extra:2}
  %       Throughout event in which \vAgent{} concludes \(\pv{\chi}{v''}\) from \(X\), there is some action \(c\) such that:
  %       \begin{itemize}
  %       \item
  %         \vAgent{} may easily and immediately do \(c\).
  %       \item
  %         \vAgent{} is concluding \(\pv{\psi}{v'}\) from \(\Psi\) when \vAgent{} does \(c\).
  %         \begin{itemize}
  %         \item
  %           Without use of any novel inf.\ obtained by doing \(a\) and \(b\).
  %         \end{itemize}
  %       \end{itemize}
  %     \end{enumerate}
  %   \end{itemize}
  % \end{enumerate}

  % \noindent Paraphrased:

  % \begin{itemize}[noitemsep]
  % \item
  %   Sub-clause~\ref{def:fc:extra:1} states \vAgent{} is concluding \(\pv{\psi}{v'}\) from \(\Psi\) after \vAgent{} does \(b\).
  % \item
  %   Sub-clause~\ref{def:fc:extra:2} states \vAgent{} \(\pv{\psi}{v'}\) from \(\Psi\) remains a \fc{} after \vAgent{} does \(b\).
  % \end{itemize}

  % \noindent This is a much stronger condition.
  % Not only is it the case that concluding, but remains the case that concluding is available for any other conclusion.

  % This rules out pairwise conflicts.%
  % \footnote{
  % Though, does not rule out more.
  % }
  We do not add any additional clauses as this is all we need.
  Still, additional clauses may be added.

  No restrictions placed on what an agent may conclude.
  So, no restrictions placed on \fc{1}.

  If \fc{} then enough to conclude.
  And, if the may conclude conflicting things, so be it.
  This is no different from conclusions.

  \fc{3}, like conclusions, are descriptive.
  Do not include considerations about `good' or `bad' conclusions.
\end{note}

\begin{note}
  Still, if you prefer something stronger, that's okay.

  For, an agent may conclude two things and then fail.
  Perhaps, in the words of Achilles, a \fc{} should be such that \textquote{you ca'n't help yourself} (\cite[280]{Carroll:1895uj}).
  Though, what this amounts to\dots

  There's nothing in what follows that strictly requires that definition of a \fc{} is as general as it is.
  Additional restrictions limit instances of \fc{1}.
  And, use is existence of \fc{1}.
  So long as \fc{1} with no \wit{} exist, good.

  If do restrict, though, some things will need to be restructured.
\end{note}

\section*{Summary}

%%% Local Variables:
%%% mode: latex
%%% TeX-master: "master"
%%% End:


% \begin{itemize}
% \item
%   The agent has a good understanding of some formal proof system.
%   For example, some Fitch-style system.
% \item
%   The agent has a good understanding of some method to construct semantic proofs.
%   For example, by constructing truth tables, or reasoning about valuation functions.
% \item
%   The agent understands the proof system is sound.
%   That is to say, the agent understands there exists a proof of some sentence \(A\) \emph{only if} \(A\) is true given an arbitrary valuation.
% \end{itemize}
% The agent constructs the following proof:
% \begin{center}
%   \begin{fitch}
%     \phantlabel{illu:sPF:1}\fa \fh (P \rightarrow Q) \rightarrow P \\
%     \phantlabel{illu:sPF:2}\fa \fa \fh  \lnot P \\
%     \phantlabel{illu:sPF:3}\fa \fa \fa \fh  P \\
%     \phantlabel{illu:sPF:4}\fa \fa \fa \fa  \bot & \(\bot\)\textbf{Intro:}\hyperref[illu:sPF:2]{2},\hyperref[illu:sPF:3]{3}\\
%     \phantlabel{illu:sPF:5}\fa \fa \fa \fa  Q & \(\bot\)\textbf{Elim:}\hyperref[illu:sPF:4]{4}\\
%     \phantlabel{illu:sPF:6}\fa \fa \fa P \rightarrow Q & \(\rightarrow\)\textbf{Intro:}\hyperref[illu:sPF:3]{3}--\hyperref[illu:sPF:5]{5} \\
%     \phantlabel{illu:sPF:7}\fa \fa \fa P & \(\rightarrow\)\textbf{Elim:}\hyperref[illu:sPF:1]{1},\hyperref[illu:sPF:6]{6}\\
%     \phantlabel{illu:sPF:8}\fa \fa \fa \bot & \(\bot\)\textbf{Intro:}\hyperref[illu:sPF:2]{2},\hyperref[illu:sPF:7]{7}\\
%     \phantlabel{illu:sPF:9}\fa \fa \lnot\lnot P & \(\lnot\)\textbf{Intro:}\hyperref[illu:sPF:2]{2},\hyperref[illu:sPF:8]{8}\\
%     \phantlabel{illu:sPF:10}\fa \fa P & \(\lnot\)\textbf{Elim:}\hyperref[illu:sPF:9]{9}\\
%     \phantlabel{illu:sPF:11}\fa ((P \rightarrow Q) \rightarrow P) \rightarrow P & \(\rightarrow\)\textbf{Intro:}\hyperref[illu:sPF:1]{1}--\hyperref[illu:sPF:10]{10} \\
%   \end{fitch}
% \end{center}
% \vspace{-\baselineskip}

% \begin{note}
%   Still, the conclusion need not be easy.

%   \begin{illustration}
%     \[\frac{(3 + \sqrt{3})^{2} + \sqrt{6}^{2} - (2\sqrt{3})^{2}}{2(3 + \sqrt{3})\sqrt{6}} = \frac{1}{\sqrt{2}}\]
%   \end{illustration}

%   I suspect this is a \fc{}.
%   Might need to do some work to recall principles, but it's okay.
% \end{note}


% \begin{note}[ML II]
%   \begin{illustration}[Modal logic II]
%     \label{illu:fc:ML2}
%     The modal system \(\mathbf{GL} = \mathbf{K} + \Box(\Box p \rightarrow p) \rightarrow \Box p\) is weakly complete with respect to the class of finite strict partial orders (that is, the class of finite irreflexive transitive frames).
%   \end{illustration}

%   \autoref{illu:fc:ML2} is similar in structure to \autoref{illu:fc:logic:CR}.
%   Indeed, both proofs involve constructing a canonical model.
%   The key distinguishing feature of \autoref{illu:fc:ML2}, however, is the difficulty of establishing the canonical model has the desired properties.
%   In particular, the general method I keep in mind for proving the relevant result requires a syntactic proof that \(\vdash_{\mathbf{GL}} \Box p \rightarrow \Box \Box p\).
%   And, as I have failed to recall the relevant syntactic on sufficient occasion, I do not consider the result a \fc{0} from my understanding of modal logic.

%   Hence, the result (plausibly) fails to be a \fc{} from my understanding of modal logic because there is no guarantee that I would provide a proof if I set out to do so.%
%   \footnote{
%   On the other hand, I have completed the relevant proof a sufficient number of times.
%   So, the result is a \fc{0} from whatever premises are associated with my memory.
% }
% \end{note}

% \begin{note}
%   \illu{3} may be obtained by taking a proposition-value pairing which conflicts with \fc{}.
%   The following is \emph{not} a \fc{}.%
%   \footnote{
%   However, I suspect the following equation is a \fc{}:
%   \[\frac{(3 + \sqrt{3})^{2} + \sqrt{6}^{2} - (2\sqrt{3})^{2}}{2(3 + \sqrt{3})\sqrt{6}} = \frac{1}{\sqrt{2}}\]
% }

%   \begin{illustration}
%     \[\frac{(3 + \sqrt{3})^{2} + \sqrt{6}^{2} - (2\sqrt{3})^{2}}{2(3 + \sqrt{3})\sqrt{6}} = \frac{1}{\sqrt{3}}\]
%   \end{illustration}
%   For, the equation does not hold.
% \end{note}

% \begin{note}
%   This is no different from conclusion that it's 13:48 by looking at the clock.
%   I recognise there is a possibility that the clock is broken.
%   However, nothing to suggest it is.

%   This highlights qualification.
%   Without it, then so long as wearing watch, time is a \fc{}.
%   For, look at watch.
% \end{note}

%%% Local Variables:
%%% mode: latex
%%% TeX-master: "master"
%%% TeX-engine: luatex
%%% End:
