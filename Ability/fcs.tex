\chapter{\fc{3} and answers to \qzS{}}
\label{cha:foregone-conclusions}

\section{Outline}
\label{sec:outline}

\begin{note}
  Idea is to transform this into a chapter on \fc{}.
  Motivate \fc{} by focusing on problem \requ{1} place on appealing to any `fact' in order to provide a positive answer to \qzS{}.
  This isn't recursive, it's just a clear failure.
  Further, this motivates an understanding \emph{without} relying on the reader to grant a negative answer to the main issue.
  The puzzle doesn't arise \emph{only} because of a commitment to positive answers o \qzS{} when the agent has not witnessed the reasoning.

  \emph{However}, caution.
  For, as we have seen with testimony, it may be the case that status of a premises blocks a \requ{}.
  And, the argument given relies on the existence of a \requ{}.
  So, it may be the case that past reasoning blocks a \requ{}.
  Still, here, only need to deny this.
  Not saying that in every case agent's present reasoning is given priority.
  (Indeed, consider cases of being somewhat impaired, e.g., via exhaustion.
  Indeed, exhaustion is interesting.
  Basic consistency checks.
  Should be the case that conclude A, but just concluded \emph{not}-A, or something like this\dots)
  Rather, denying that past continues to secure in all instances.
  So, just need the potential to revise perspective on any previous conclusion.
\end{note}

\begin{note}[Past conclusions and positive answers]
  \begin{itemize}
  \item
    \emph{If} positive answer due to some past conclusion \emph{then} possible for the agent to conclude.
  \end{itemize}
  This conditional is immediate, because \qzS{} is about whether the agent would conclude, given that they have the option.
  \begin{itemize}
  \item
    \emph{If} possible to conclude, \emph{then} fact is insufficient.
  \end{itemize}
  This conditional is also immediate, because if the agent failed to conclude, then the fact that they had concluded wouldn't go anywhere.
\end{note}

\begin{note}
  The important constraint here is that past conclusion is providing a positive answer.

  Apparent counterexample. (Or maybe useful \illu{}?)

  Exhaustion.
  Some conclusion.
  Notice a \requ{}.
  Have concluded before.
  So, reason about the \requ{}.
  Fail to conclude.
  But, this doesn't highlight anything about previous conclusion.

  However, this is not right.
  It does show that previous conclusion does not function as a positive answer.
  It doesn't show there is a problem with the previous conclusion, but it does show that there is a problem with the role of this as an answer.
\end{note}

\begin{note}
  So, given the above, a further question is whether something has the status of a \requ{} if the agent has previously concluded.

  Surely.

  For, else, always defer to the past.
  No correcting mistakes.
\end{note}

\begin{note}
  So, the way in which past reasoning relates is by ensuring that the agent would reach the same conclusion.
  About the agent's reasoning.
  \emph{How} rather than \emph{that}.

  Look, what we are getting is that the agent would conclude.
  If something were to happen, then some action would be performed.
  There's no distinction between the answer and performing the act, roughly.
  Or, better put, the answer \emph{is about present reasoning}.
  Answer states that in present reasoning, would not fail.

  In this respect, \fc{}.

  Perhaps obvious, this is what the question asks.
  But, very important.
  Characterisation of the answer in terms of something forward looking.
  \fc{}.
  It is about the agent's present epistemic state, and in particular what the agent's present epistemic state is capable of.

  In other words, ability.
  What answers is ability, in the sense that ability iff would.

  This is very important to the understanding of \fc{}.

  And, I kind of want to have ability as a gloss, while focusing on \fc{} to avoid going into ability in too much detail.

  So, positive answer, then it's the pairing \emph{being} a \fc{}.
  (I should always use this instance of the copula.)
\end{note}

\begin{note}
  Then, \fc{} only if support.
  And, \fc{} only if any other \requ{} are also \fc{}.
\end{note}

\begin{note}
  An interesting observation here is that in certain this all arises, to a certain extent, because of general abilities.
  General ability spans multiple different proposition-value-premises pairings.
  Hence, all of these function as \requ{1}, so long as the agent has the option.
\end{note}


\section{Outline}

\begin{note}
  \begin{itemize}
  \item
    Warming up?
    Here, focus on having concluded.
    This is probably helpful, as it covers the most important case in a concrete way.
    The difficulty is that it does not generalise.
    Hence, the more general argument.
    And, helps introduce the idea of a \fc{}.
  \item
    Idea of a \fc{}.
    Two reasons focus is on \fc{}.
    First, neutral on what it is that makes something a \fc{}.
    Second, keeps relation to \qzS{} clear.
  \item
    \fc{} only if support.
    This is a difficult point.
    The problem is there is room to resist.
  \item
    Positive answer \emph{only if} \fc{} or conclude at the same time.
    Here, use the squeezing argument.
    Some X such that X answers but X does not ensure pair is a \fc{}.
  \item
    Corollaries, concluding.
    In particular, have not relied on unwitnessed.
  \item
    Answers to why.
    Here, \requ{1}.
    Still, expand to non-witnessing cases.
  \item
    General and specific abilities.
  \item
    Answers to why, then.
    Note, here, that opportunity is interesting.
    The whole conjunction of all instance of the general ability is plausibly not a \requ{}.
    However, all that's needed is the \emph{individual} instances, and for these to raise a problem.
  \item
    The point is, \requ{1} for any general ability, and these are also \requ{1} for main pairing.
    (%
    Note --- or perhaps emphasise --- here, that the problem is \emph{not} recursive.
    Instead, the problem is about the spread.%
    )
  \item
    Here, then, ability is both the problem and the answer.
    What's interesting is the way in which ability functions.
    It's not merely \emph{that} the agent has the ability.
    Instead, it \emph{is} the ability.
  \end{itemize}
\end{note}

\begin{note}
  Somewhere at the end, or perhaps on a speculative chapter:
  Deduction theorem for reasoning.
  And, support, so why not conclude from without witnessing the reasoning.
  This would just be witnessing a foregone-conclusion.
\end{note}

\section{Old}

{
  \color{red}
  So, I don't want to say that \fc{} only if relation of support.
  This is too strong.
  Just that from the agent's point of view, would not fail to conclude.
  Getting a relation of support requires additional work.
}

\begin{note}[Foregone-conclusions]
  Basic idea of a foregone-conclusion.

  \begin{restatable}[Foregone-conclusions]{definition}{definitionForegoneC}
    \(\pv{\phi}{v}\) is a foregone-conclusion from some pool of premises \(\Phi\) just in case, given the agent's present epistemic, the agent would not fail to conclude \(\pv{\phi}{v}\) from \(\Phi\) were the agent to reason.
  \end{restatable}

  Whether foregone-conclusion takes agent's present epistemic state as a function.
  However, does not need to be the case that the agent recognises foregone-conclusion.

  At most, witnessing provides information about method.

  For any property \(P\) which would follow from any instance of witnessing reasoning \(\pv{\phi}{v}\) from \(\Phi\), the agent's present epistemic state is sufficient to determine \(P\) without witnessing reasoning from \(\Phi\) to \(\pv{\phi}{v}\).

  Suppose \(P\) follows from concluding.
  Forgegone-conclusion.
  So, agent's present epistemic state, agent would not fail.
  However, it then follows that \(P\).

  Here, restricted \(P\) to follow from any.
  Hence, if there are multiple methods, \(P\) may be restricted.

  However, broaden.
\end{note}

\begin{note}[Intuitive cases]
  Knowing whether and knowing how to.
  More or less interchangeable.

  Know whether \(x + y = z\).
  Know how to calculate \(x + y\).
  Indeed, for any \(z\), know whether \(x + y = z\).

  Sudoku puzzles.
  Know how to figure out.
  So, know whether any solution is valid.

  Of course, in certain cases, there are shortcuts.
  Two even numbers, then know whether by checking whether the last digit is even or odd.
  And, other cases, contingent shortcut, such as two of the same number in a square for Sudoku.

  So, really, knowing how to.
\end{note}

\begin{note}[Weaken]
  \fc{2} is weaker.
  Knowing, factive.
  Though, plausible that these amount to the same thing in various cases.
  Either because \fc{} is determined by knowing how to.
  Or, because knowing is weakened to the agent's perspective.
\end{note}

\begin{note}
  \begin{proposition}
    For any path, present epistemic state determines availability of path.
  \end{proposition}

  Start.
  Then, continue.
  Started from \(\Phi\), so will conclude.
  Hence, no matter choice made, must have taken the possibility of this choice into account.
  So, it must be the case that determined.

  Hence, if witness, then via some path.

  So, witnessing predetermined path.
  Any instances of concluding by witnessing reduces to witnessing predetermined path.

  Witnessing may provide information about path, but witnessing doesn't 


  For any X from W,
  present determines whether or not X from agent's point of view, then forgone conclusion.

  In other words, agent's present epistemic state determines.
  Agent may need to witness to figure out how determined, but witnessing does not influence.
\end{note}

\begin{note}[Non-cases]
  Now, \(p, p \rightarrow q \vdash q\) case.
  Well, determines \(q\), if we ignore possibility of revision.
  However, this doesn't tell us about all X.

  This is much stronger.
  There is nothing witnessing adds which is not already determined.

  2 + 2, peano arithmetic.

  More intuitive.
  Though, still question about deriving 2 + 2 = 4 from peano arithmetic.

  Understanding of arithmetic.
  Then, add two numbers.
  Forgone-conclusion.

  % Sudoku puzzle.
  % Solution is a foregone-conclusion.
\end{note}

\begin{note}
  To \illu{0}, questions and answers.

  Do you know whether \(83\) is prime?

  Not off the top of my head.

  Do you know whether \(28 + 55 = 83\)?

  Sure, but give me a moment.

  Do you know whether \dots

  No.

  Of course, might hold that the agent needs to have figured things out.
  But, then we have a plausible reduction.
  Knowing whether, and witnessing whether.
  Common component.

  Now, idea is a little different, as knowledge implies factivity.
  Interest with concluding is that not necessarily factive.
  From the agent's perspective.

  `Determining whether'.
  Or, rather `\fc{0}'.
\end{note}

\begin{note}
  Similar to Goldman, etc.?
  The idea is justification\dots
\end{note}

\begin{note}[Trimming]
  \begin{proposition}
    Basically, there's no role for anything beyond \(\Phi\) in the case of a foregone-conclusion.

    \(\Phi\), in the context of the agent's present epistemic state is sufficient to secure the conclusion, and the possibility of witnessing reasoning.
  \end{proposition}

  \begin{proposition}
    Foregone-conclusion just in case \(\Phi\) supports \(\pv{\phi}{v}\).
    \begin{argument}
      In short, given agent's present epistemic state, there's a guaranteed path from \(\Phi\) to \(\pv{\phi}{v}\).
    \end{argument}
    In other words, if \(\Phi\) does not support \(\pv{\phi}{v}\), then \(\pvp{\phi}{v}{\Phi}\) is not a foregone-conclusion.
  \end{proposition}

  Now, as the agent has not witnessed reasoning, need information that \(\pvp{\phi}{v}{\Phi}\) is a foregone-conclusion in order to recognise this.
  However, with information that \(\pvp{\phi}{v}{\Phi}\) is a foregone-conclusion, the information has no role in supporting \(\pv{\phi}{v}\).
\end{note}

\begin{note}
  So, that \(\pvp{\phi}{v}{\Phi}\) is a \fc{0} provides information, and explains, in part or whole, \emph{how} the agent concludes \(\pv{\phi}{v}\).
  However, \emph{why} is accounted for by \(\Phi\).
\end{note}

\begin{note}
  Why does it matter which?

  \begin{idea}[Reduction]
    If foregone-conclusion, then witness relation established by being a foregone-conclusion.
    Hence, reduction.
  \end{idea}
  Here, reduction.
  Restriction to `some'.
  It is not the case that every conclusion is a foregone-conclusion.

  However, if foregone-conclusions are of interest, then some motivation.

  Still, why of interest?
  What role do foregone-conclusion have?

  In particular.
  Need to get that some conclusion is a foregone-conclusion.
  From some reasoning.
  So, some pool of premises.
  And, that pool of premises is sufficient for conclusion.

  So, source for 2 + 2 = 4 and general ability.
  Already concluded that 2 + 2 = 4.
  Not quite, still going from general ability to 2 + 2 = 4.

  So, it's not clear that reduction is irrelevant.

  However, it is also not clear that this reduction is general.
  Want to show that foundation for reduction is not limited.
\end{note}

\begin{note}[Two worries]
  Two worries.

  First, that even though \fc{0}, the agent would not conclude.
  Either because \(\Phi\) is unavailable, or because no potential witnessing event.
  So, can't remove \fc{0} from account of why.

  However, then \fc{0} does not support.

  If grant that \fc{0} supports, then this seems to work out.
  Further, if require existence, then things that support get very messy.
  Dopeganger cases.
  Reason is I saw A, but it wasn't A, appealing to something that doesn't exist.
  Various other cases like this.

  Difference.
  In these cases, have premise, thing is that the truth value is distinct.
  Here, possibly no premise.

  Well, this is different.
  However, I don't think this is sufficient to reject the idea.
  Just because this distinction doesn't arise in the case of witnessing doesn't really do much.

  Look, a `bad' premise offers no more support for the agent than no premise.

  Second, need \emph{that} \fc{0}.
  However, the point is that this is about the agent's present epistemic state.
  \emph{Without} \fc{0}, the agent would reason.
  This is just the key point reiterated.
  Know whether, \fc{0} just adds information about which.
\end{note}


\paragraph{Foregone-concluding}

\begin{note}[Foregone-concluding]
  Pair this with a key idea.

  \begin{restatable}[Foregone-concluding]{idea}{ideaForegoneCing}
    \label{idea:reassignment}
    If foregone-conclusion, then may conclude.
    %\vspace{-\baselineskip}
  \end{restatable}

  Cases where concluding by witnessing reduces to witnessing forgone conclusion.
  \emph{Concluding \(\pv{\psi}{v'}\) from \(\Psi\) is just witnessing foregone-conclusion.}
  So, reduction, in certain cases.
  Further, if forgone conclusion, then conclude.
  At least, in certain cases.
\end{note}

\begin{note}[???]
  Only argue for a positive resolution to~{\color{red} issue:Main} given~\autoref{idea:reassignment}.

  And, leave~\autoref{idea:reassignment} as an idea.
  Insight into adopting this idea, or something like this.
\end{note}

\begin{note}
  Positive resolution may read easier if something like `in committing'.
  Commit to location from map, sum from arithmetic.
  Indeed, perhaps intuitive sense is just commitment via witnessing reasoning.

  But, we then have a reduction.
  Question is, what work does commitment do, and what work does witnessing do?

  Still, why think this?
  Why not think that concluding leads to commitments.
  Independent consequence.
  In same way that knowledge as basic entails justification, same for commitments.
  Indeed, in same way that relevant justification may be distinctive in the case of knowledge, same for commitment with concluding.
\end{note}

\begin{note}
  Assume motivate.
  What exactly is concluding?
  This will be beyond the scope of this document.
  I hope motivating a difference in extension motivates further questions about what the relation is, and the importance of witnessing in certain cases.
  As we will see, making this argument is by no means straightforward.

  I think there needs to be some instance of witnessing --- concluding does not arise from nowhere.
  Still, if witnessing leads to additional conclusions, then what do we get from concluding?
  What we will get is a general closure condition.

  To do so we narrow things down a little.
  Focus on particular types of concluding, and when concluding is accompanied by an additional property.
  Still, if in these instances no witnessing, then not more generally.
  Additional property, concluding some proposition from some premises.

  Argument, won't directly rely on intuitions about whether agent has concluded.
\end{note}

%%% Local Variables:
%%% mode: latex
%%% TeX-master: "master"
%%% End:
