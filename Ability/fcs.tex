\chapter{\fc{3}}
\label{cha:fcs}

\begin{note}
  \fc{}.
  Answers are \fc{}.
  In straightforward sense, follows.
  Potential event.
\end{note}

\begin{note}
  \color{red}

  Major part of the argument is that witnessed, then relation of support from witnessing makes no difference to the situation, so long as a \requ{}.

  The only thing this distinguishes is whether the thing actually happened, but it doesn't matter for the relation between \(\pv{\psi}{v'}\) and \(\Psi\).
\end{note}

\section{Conclusions, forgone}
\label{sec:fc3-1}

\begin{note}
  Ah, really, here I'm going to quite a combined proposition introduced to sum up the previous chapter.
\end{note}

\section{Outline}
\label{sec:outline}

\begin{note}
  Idea is to transform this into a chapter on \fc{}.
  Motivate \fc{} by focusing on problem \requ{1} place on appealing to any `fact' in order to provide a positive answer to \qzS{}.
  This isn't recursive, it's just a clear failure.
  Further, this motivates an understanding \emph{without} relying on the reader to grant a negative answer to the main issue.
  The puzzle doesn't arise \emph{only} because of a commitment to positive answers o \qzS{} when the agent has not witnessed the reasoning.

  \emph{However}, caution.
  For, as we have seen with testimony, it may be the case that status of a premises blocks a \requ{}.
  And, the argument given relies on the existence of a \requ{}.
  So, it may be the case that past reasoning blocks a \requ{}.
  Still, here, only need to deny this.
  Not saying that in every case agent's present reasoning is given priority.
  (Indeed, consider cases of being somewhat impaired, e.g., via exhaustion.
  Indeed, exhaustion is interesting.
  Basic consistency checks.
  Should be the case that conclude A, but just concluded \emph{not}-A, or something like this\dots)
  Rather, denying that past continues to secure in all instances.
  So, just need the potential to revise perspective on any previous conclusion.
\end{note}

\begin{note}
  The important constraint here is that past conclusion is providing a positive answer.

  Apparent counterexample. (Or maybe useful \illu{}?)

  Exhaustion.
  Some conclusion.
  Notice a \requ{}.
  Have concluded before.
  So, reason about the \requ{}.
  Fail to conclude.
  But, this doesn't highlight anything about previous conclusion.

  However, this is not right.
  It does show that previous conclusion does not function as a positive answer.
  It doesn't show there is a problem with the previous conclusion, but it does show that there is a problem with the role of this as an answer.
\end{note}

\begin{note}
  So, given the above, a further question is whether something has the status of a \requ{} if the agent has previously concluded.

  Surely.

  For, else, always defer to the past.
  No correcting mistakes.
\end{note}

\begin{note}
  So, the way in which past reasoning relates is by ensuring that the agent would reach the same conclusion.
  About the agent's reasoning.
  \emph{How} rather than \emph{that}.

  Look, what we are getting is that the agent would conclude.
  If something were to happen, then some action would be performed.
  There's no distinction between the answer and performing the act, roughly.
  Or, better put, the answer \emph{is about present reasoning}.
  Answer states that in present reasoning, would not fail.

  In this respect, \fc{}.

  Perhaps obvious, this is what the question asks.
  But, very important.
  Characterisation of the answer in terms of something forward looking.
  \fc{}.
  It is about the agent's present epistemic state, and in particular what the agent's present epistemic state is capable of.

  In other words, ability.
  What answers is ability, in the sense that ability iff would.

  This is very important to the understanding of \fc{}.

  And, I kind of want to have ability as a gloss, while focusing on \fc{} to avoid going into ability in too much detail.

  So, positive answer, then it's the pairing \emph{being} a \fc{}.
  (I should always use this instance of the copula.)
\end{note}

\begin{note}
  An interesting observation here is that in certain this all arises, to a certain extent, because of general abilities.
  General ability spans multiple different proposition-value-premises pairings.
  Hence, all of these function as \requ{1}, so long as the agent has the option.
\end{note}

\section{Outline}

\begin{note}
  So, the way in which past reasoning relates is by ensuring that the agent would reach the same conclusion.
  About the agent's reasoning.
  \emph{How} rather than \emph{that}.

  Look, what we are getting is that the agent would conclude.
  If something were to happen, then some action would be performed.
  There's no distinction between the answer and performing the act, roughly.
  Or, better put, the answer \emph{is about present reasoning}.
  Answer states that in present reasoning, would not fail.

  In this respect, \fc{}.

  Perhaps obvious, this is what the question asks.
  But, very important.
  Characterisation of the answer in terms of something forward looking.
  \fc{}.
  It is about the agent's present epistemic state, and in particular what the agent's present epistemic state is capable of.

  In other words, ability.
  What answers is ability, in the sense that ability iff would.

  This is very important to the understanding of \fc{}.

  And, I kind of want to have ability as a gloss, while focusing on \fc{} to avoid going into ability in too much detail.

  So, positive answer, then it's the pairing \emph{being} a \fc{}.
  (I should always use this instance of the copula.)
\end{note}

\begin{note}
  An interesting observation here is that in certain this all arises, to a certain extent, because of general abilities.
  General ability spans multiple different proposition-value-premises pairings.
  Hence, all of these function as \requ{1}, so long as the agent has the option.
\end{note}

\begin{note}
  \begin{itemize}
  \item
    \fc{} only if support.
    This is a difficult point.
    The problem is there is room to resist.
  \item
    General and specific abilities.
  \item
    Answers to why, then.
    Note, here, that opportunity is interesting.
    The whole conjunction of all instance of the general ability is plausibly not a \requ{}.
    However, all that's needed is the \emph{individual} instances, and for these to raise a problem.
  \item
    The point is, \requ{1} for any general ability, and these are also \requ{1} for main pairing.
    (%
    Note --- or perhaps emphasise --- here, that the problem is \emph{not} recursive.
    Instead, the problem is about the spread.%
    )
  \item
    Here, then, ability is both the problem and the answer.
    What's interesting is the way in which ability functions.
    It's not merely \emph{that} the agent has the ability.
    Instead, it \emph{is} the ability.
  \end{itemize}
\end{note}

\begin{note}
  Somewhere at the end, or perhaps on a speculative chapter:
  Deduction theorem for reasoning.
  And, support, so why not conclude from without witnessing the reasoning.
  This would just be witnessing a foregone-conclusion.
\end{note}

\section{Fragments}

\begin{note}
  A \fc{} is not something which functions as a premise.
  This is a category mistake.

  So, answer to \qzS{} can't be a premise, because, in principle, for any premise, if we have a \requ{}, then any premise which states that the \requ{} is satisfied is subject to the \requ{}.

  I mean, do I get an infinite regress here?
  Even if I do, I don't think it's important.
  The point is, even granting that the agent is correct, hum.
  The \requ{} remains, but that is true in both cases.
  The task isn't getting rid of the \requ{}.
  Rather, the task is to show that the agent would conclude.
  But, now, fails to answer, because think!

  \fc{} answers by pointing to the reasoning.
  Alternative answers by not doing so.
  As you have not pointed to the reasoning, it remains the case that whatever this is, the \requ{} applies.

  This is probably a better way of doing things.
  I have a clearer understanding of pointing to the reasoning.
  And, with respect to the reasoning, it's clear.
  There's no question about whether the agent would conclude, that is what the agent is pointing to.
  The only question is whether there really is such an event.

  By contrast, if we're not pointing to the reasoning, then\dots
  Whatever it is the agent is pointing to, the agent has the option of appealing to this regardless of whether there is a witnessing event.
  Hence, if there is no witnessing event, then useless???

  \emph{The same \requ{} applies}
  Because, failure to conclude, then this thing is useless.

  This is quite subtle.
  The point is, potential witnessing event.
  Without this, without this doing the work, whatever one thinks of, the same question still applies.
  (No recursion!)

  If one does appeal to the potential witnessing event, then one is pointing to the very thing that matters.

  Now, failure to conclude, then something has gone wrong.
  Yes.
  The key observation is that in the failure case, the alternative thing, whatever this happens to be, persists, or at least may persist.
  It's independent of there being a potential witnessing event.

  This is what shows it doesn't work.

  So, looking.
  If mistake, then bad things all around.

  If not a mistake, then still a problem.
  For, independence of potential witnessing event.
  Therefore, failure would prevent from doing work.

  Thing is, right about potential, done.
  Right about alternative, then still a question regarding the potential.
\end{note}

\paragraph{Premises and past conclusions}

\begin{note}[Premises]
  So, as we have seen with testimony, status of a premises blocks a \requ{}.

  Whether the same may hold for this problem.

  It's the case that, part of agent's present epistemic state that they would conclude.

  Problem is, if attempt and fail, then this premise does nothing.
  Their present epistemic state develops into a dead-end.
\end{note}

\begin{note}[Note!]
  This doesn't hold in general, for all premises.

  In particular, premise is past conclusion.

  Consider cases of being somewhat impaired, e.g., via exhaustion.
  Indeed, exhaustion is interesting.
  Basic consistency checks.
  Should be the case that conclude A, but just concluded \emph{not}-A, or something like this\dots

  Denying that past continues to secure in all instances.
  So, just need the potential to revise perspective on previous conclusion.
\end{note}

\begin{note}[Past conclusions and positive answers]
  \begin{itemize}
  \item
    \emph{If} positive answer due to some past conclusion \emph{then} possible for the agent to conclude.
  \end{itemize}
  This conditional is immediate, because \qzS{} is about whether the agent would conclude, given that they have the option.
  \begin{itemize}
  \item
    \emph{If} possible to conclude, \emph{then} fact is insufficient.
  \end{itemize}
  This conditional is also immediate, because if the agent failed to conclude, then the fact that they had concluded wouldn't go anywhere.
\end{note}

\begin{note}
  Somewhere at the end, or perhaps on a speculative chapter:
  Deduction theorem for reasoning.
  And, support, so why not conclude from without witnessing the reasoning.
  This would just be witnessing a foregone-conclusion.
\end{note}


\section{\fc{3}}
\label{cha:fcs:sec:fc}

\begin{note}[Foregone-conclusions]
  Basic idea of a foregone-conclusion.

  \begin{restatable}[Foregone-conclusions]{definition}{definitionForegoneC}
    For any proposition-value-premises pairing \(\pvp{\psi}{v'}{\Psi}\):

    From \vAgent{}' perspective:
    \begin{itemize}
    \item
      \(\pv{\phi}{v}\) is a \emph{\fc{0}} from some pool of premises \(\Phi\)
    \end{itemize}
    \emph{If and only if}
    \begin{itemize}
    \item
      There is a potential event in which \vAgent{} concludes \(\pv{\phi}{v}\) from \(\Phi\).

      Given the agent's present epistemic, the agent would not fail to conclude \(\pv{\phi}{v}\) from \(\Phi\) were the agent to reason.
    \end{itemize}
  \end{restatable}

  {
    Potential event doesn't say much.
    What is does clarify is the dynamic nature.

    And, the point is that past reasoning does not entail there is a potential event, in general.
    Past reasoning only entails there was some event.

    So, the key point for the argument given is that past reasoning does not entail.
    If it did, then we wouldn't have a \requ{}.
    Well, hold on.
    I'm a little confused.
    The point is, if the agent were to fail to conclude, the past reasoning wouldn't be of any use.
    So, the fact alone doesn't provide the entailment.
  }

  Whether \fc{} takes agent's present epistemic state as a function.
  However, does not need to be the case that the agent recognises \fc{}.

  At most, witnessing provides information about method.

  For any property \(P\) which would follow from any instance of witnessing reasoning \(\pv{\phi}{v}\) from \(\Phi\), the agent's present epistemic state is sufficient to determine \(P\) without witnessing reasoning from \(\Phi\) to \(\pv{\phi}{v}\).

  Suppose \(P\) follows from concluding.
  \fc{2}.
  So, agent's present epistemic state, agent would not fail.
  However, it then follows that \(P\).

  Here, restricted \(P\) to follow from any.
  Hence, if there are multiple methods, \(P\) may be restricted.

  However, broaden.
\end{note}

\section{\fc{3} and support}
\label{cha:fcs:sec:fc3-support}

\begin{note}
  This is where things get somewhat difficult.

  This is something we need to argue for.

  The term is suggestive, `conclusion', and concluded only if support.
  However, the agent has not concluded, so assumption doesn't do any work.%
  \footnote{
    Variant of this argument in which argue that the agent concludes.
  }
\end{note}

\begin{note}
  \begin{proposition}
    For any path, present epistemic state determines availability of path.
  \end{proposition}

  Start.
  Then, continue.
  Started from \(\Phi\), so will conclude.
  Hence, no matter choice made, must have taken the possibility of this choice into account.
  So, it must be the case that determined.

  Hence, if witness, then via some path.

  So, witnessing predetermined path.
  Any instances of concluding by witnessing reduces to witnessing predetermined path.

  Witnessing may provide information about path, but witnessing doesn't contribute given a \requ{}.

  For any X from W,
  present determines whether or not X from agent's point of view, then forgone conclusion.

  In other words, agent's present epistemic state determines.
  Agent may need to witness to figure out how determined, but witnessing does not influence.
\end{note}

\begin{note}[Two worries]
  Two worries.

  First, that even though \fc{0}, the agent would not conclude.
  Either because \(\Phi\) is unavailable, or because no potential witnessing event.
  So, can't remove \fc{0} from account of why.

  However, then \fc{0} does not support.

  If grant that \fc{0} supports, then this seems to work out.
  Further, if require existence, then things that support get very messy.
  Dopeganger cases.
  Reason is I saw A, but it wasn't A, appealing to something that doesn't exist.
  Various other cases like this.

  Difference.
  In these cases, have premise, thing is that the truth value is distinct.
  Here, possibly no premise.

  Well, this is different.
  However, I don't think this is sufficient to reject the idea.
  Just because this distinction doesn't arise in the case of witnessing doesn't really do much.

  Look, a `bad' premise offers no more support for the agent than no premise.

  Second, need \emph{that} \fc{0}.
  However, the point is that this is about the agent's present epistemic state.
  \emph{Without} \fc{0}, the agent would reason.
  This is just the key point reiterated.
  Know whether, \fc{0} just adds information about which.
\end{note}

\paragraph{Knowing whether}

\begin{note}
  \color{red}
  There is scope for an objection here.
  Argued that answer to \qzS{} is not a proposition-value pair.
  But, knowing how.
  If strong intellectualism, then proposition-value pair answers to knowing how.
  And, apply the same to \qzS{}.

  For this to work, need, following terminology of \textcite{Glick:2011vd}, \emph{strong} intellectualism.

  But, this is also fine, no?
  Because, the point is that answering from the agent's perspective.
  Indeed, the same argument holds for knowledge from the agent's perspective.
  Regardless of what produces the answer, knowledge is factive.
  So, it's not only the mental state, but also the state of affairs.

  Contrast this with belief.
  Here, answering why believe.
  This is just a mental state, and is non-factive.
  So, don't need to involve a state of affairs in answer.
\end{note}

\begin{note}[Intuitive cases]
  Knowing whether and knowing how to.
  More or less interchangeable.

  Know whether \(x + y = z\).
  Know how to calculate \(x + y\).
  Indeed, for any \(z\), know whether \(x + y = z\).
\end{note}

\begin{note}
  To \illu{0}, questions and answers.

  Do you know whether \(83\) is prime?

  Not off the top of my head.

  Do you know whether \(28 + 55 = 83\)?

  Sure, but give me a moment.

  Do you know whether \dots

  No.

  Of course, might hold that the agent needs to have figured things out.
  But, then we have a plausible reduction.
  Knowing whether, and witnessing whether.
  Common component.

  Now, idea is a little different, as knowledge implies \factivity{}.
  Interest with concluding is that not necessarily factive.
  From the agent's perspective.

  `Determining whether'.
  Or, rather `\fc{0}'.
\end{note}

\begin{note}
  Relation here is tentative.

  ~\cite{Bengson:2011th}.
  \begin{quote}
    \emph{Pi}.
    Louis, a competent mathematician, knows how to find the n\(^{\text{th}}\) numeral, for any numeral \(n\), in the decimal expansion of \(\pi\).
    He knows the algorithm and knows how to apply it in a given case.
    However, because of principled computational limitations, Louis (like all ordinary human beings) is unable to find the \(10^{46}\) numeral in the decimal expansion of \(\pi\).%
    \mbox{ }\hfill\mbox{(\citeyear[170]{Bengson:2011th})}
  \end{quote}

  Not clearly a \fc{}.
\end{note}

\begin{note}
  Understanding here.
  Intellectualist and anti-intellectualist views.

  Proposal fits well with anti-intellectualist such as~\citeauthor{Habgood-Coote:2019we}'s (\citeyear{Habgood-Coote:2019we}) Interrogative Capacity View.%
  \footnote{
    \begin{quote}
      \emph{The Interrogative Capacity View}.
      For any context c, subject S, and activity V, an utterance of `S knows how to V' (in its practical-knowledge ascribing sense) is true in c iff c has associated with it a set of practically relevant situations {F1, F2, \dots}, and, for all (or at least most) Fi that are members of {F1, F2, \dots}, S has the capacity to activate knowledge of a fine-grained answer to the question, how to V in Fi?, in the process of V-ing.%
      \mbox{ }\hfill\mbox{(\citeyear[92]{Habgood-Coote:2019we})}
    \end{quote}
  }
  But, this is not the focus.
  
\end{note}

\begin{note}
  Note, principle from~\cite{Barker:1975un} does not hold.
  for example, chess game.
  Know whether it is possible to win, and strong belief that it is not possible.
  Belief doesn't really mean anything.
\end{note}


\begin{note}
  \fc{2} is weaker.
  Knowing, factive.
  Though, plausible that these amount to the same thing in various cases.
  Either because \fc{} is determined by knowing how to.
  Or, because knowing is weakened to the agent's perspective.

  Sudoku puzzles.
  Know how to figure out.
  So, know whether any solution is valid.

  Of course, in certain cases, there are shortcuts.
  Two even numbers, then know whether by checking whether the last digit is even or odd.
  And, other cases, contingent shortcut, such as two of the same number in a square for Sudoku.

  So, really, knowing how to.
\end{note}

\begin{note}[Trimming]
  \begin{proposition}
    Basically, there's no role for anything beyond \(\Phi\) in the case of a foregone-conclusion.

    \(\Phi\), in the context of the agent's present epistemic state is sufficient to secure the conclusion, and the possibility of witnessing reasoning.
  \end{proposition}

  \begin{proposition}
    Foregone-conclusion just in case \(\Phi\) supports \(\pv{\phi}{v}\).
    \begin{argument}
      In short, given agent's present epistemic state, there's a guaranteed path from \(\Phi\) to \(\pv{\phi}{v}\).
    \end{argument}
    In other words, if \(\Phi\) does not support \(\pv{\phi}{v}\), then \(\pvp{\phi}{v}{\Phi}\) is not a foregone-conclusion.
  \end{proposition}

  Now, as the agent has not witnessed reasoning, need information that \(\pvp{\phi}{v}{\Phi}\) is a foregone-conclusion in order to recognise this.
  However, with information that \(\pvp{\phi}{v}{\Phi}\) is a foregone-conclusion, the information has no role in supporting \(\pv{\phi}{v}\).
\end{note}

\begin{note}
  So, that \(\pvp{\phi}{v}{\Phi}\) is a \fc{0} provides information, and explains, in part or whole, \emph{how} the agent concludes \(\pv{\phi}{v}\).
  However, \emph{why} is accounted for by \(\Phi\).
\end{note}

\section{Foregone-concluding}

\begin{note}[Foregone-concluding]
  Pair this with a key idea.

  \begin{restatable}[Foregone-concluding]{idea}{ideaForegoneCing}
    \label{idea:reassignment}
    If foregone-conclusion, then may conclude.
    %\vspace{-\baselineskip}
  \end{restatable}

  Cases where concluding by witnessing reduces to witnessing forgone conclusion.
  \emph{Concluding \(\pv{\psi}{v'}\) from \(\Psi\) is just witnessing foregone-conclusion.}
  So, reduction, in certain cases.
  Further, if forgone conclusion, then conclude.
  At least, in certain cases.
\end{note}

\begin{note}[???]
  Only argue for a positive resolution to~{\color{red} issue:Main} given~\autoref{idea:reassignment}.

  And, leave~\autoref{idea:reassignment} as an idea.
  Insight into adopting this idea, or something like this.
\end{note}

\section{Interlude}
\label{cha:fcs:sec:interlude}

\subsection{Limitations}

\begin{note}
  Important limitation here is that \(\pvp{\psi}{v'}{\Psi}\) is a \requ{}.

  This need not be the case.

  For example, consider discussion of~\autoref{illu:gist:calc}.
  Testimony.
  Failure of a \requ{}.

  It is possible that a prior conclusion of \(\pv{\psi}{v'}\) from \(\Psi\) prevents \(\pvp{\psi}{v'}{\Psi}\) being a \requ{}.

  However, in general, this is a bad response.
  For, this would result in discounting the agent's present epistemic state.

  No doubt, some conclusions are more secure that others.
  However, such a hierarchy is not temporal.
\end{note}


\newpage

\section{???}

\subsubsection{\citetitle{Carroll:1895uj}}
\label{sec:carroll}

\begin{note}
  \color{red}

  ~\cite{Besson:2018wz} in here somewhere.
\end{note}

\begin{note}
  \color{red}
  Point here is role of rule of inference is key.
  And,~\autoref{prop:PWEs} is observing this.
\end{note}

\begin{note}
  Similar to \citeauthor{Carroll:1895uj}.
  \begin{quote}
    Logic would take you by the throat, and \emph{force} you to do it!%
    \mbox{ }\hfill\mbox{(\citeyear[280]{Carroll:1895uj})}
  \end{quote}
  Looking at something static.
  Achilles fails to convey this to the Tortoise, arguably through some fault of Achilles' own.

  In parallel, we could stack up additional passives in the same way, but there's little interest in doing so.
  The point is the base \requ{} is not satisfied.
\end{note}

\begin{note}
  So, with \citeauthor{Carroll:1895uj}, we get a rule of inference, great.

  \citeauthor{Wieland:2013vf} characterises the general understanding of \textcite{Carroll:1895uj} in terms of two lessons:
  \begin{quote}
    [T]he negative lesson is that if you add ever more premises to an argument \dots, then you will never demonstrate that its conclusion follows logically.%
    \mbox{ }\hfill\mbox{(\citeyear[984]{Wieland:2013vf})}
  \end{quote}

  Parallel, static answers, still option for concluding otherwise.

  \begin{quote}
    [T]he positive lesson is that rules of inference, rather than premises of the form `if premises such and such are true, then the conclusion is true', will do the job.%
    \mbox{ }\hfill\mbox{(\citeyear[984]{Wieland:2013vf})}
  \end{quote}

  Parallel, the dynamic status of a rule.
\end{note}

\begin{note}
  Similar, but a little different.
\end{note}

\begin{note}
  No regress.

  Following \citeauthor{Wieland:2013vf}:

  \begin{quote}
    \begin{itemize}[noitemsep]
    \item[IR]
      For any item x of a certain type, S \(\varphi\)-s x only if
      \begin{enumerate}[label=(\roman*),noitemsep]
      \item
        there is a new item y of that same type, and
      \item
        S \(\varphi\)-s y.%
        \mbox{ }\hfill\mbox{(\citeyear[996]{Wieland:2013vf})}
      \end{enumerate}
    \end{itemize}
  \end{quote}

  Now, concluding, versus would conclude.
  However, focus is before concluding.
  So, would conclude and would conclude.

  Difficulty is, it's not at all clear this is the case.
  Need to be sure that there is a \requ{} for any \requ{}.
  Yet, from agent's perspective.
\end{note}

\begin{note}
  Interesting thing here is that it's not `just' the rule.

  \emph{And}, important difference is that the agent isn't moving from premises to conclusion.
  Following the standard interpretation, \citeauthor{Carroll:1895uj} gets us that there's a rule in play when agent concludes.
  (Or, more strictly, modus ponens\dots)
  But, this is very different from something similar being active when drawing some other conclusion.

  So, there is a link to \citeauthor{Carroll:1895uj}, but it is somewhat indirect.
  Still, this should soften the conclusion.

  In short, with \citeauthor{Carroll:1895uj} it's the rule.
  Here, it's the ability to employ the rule.
\end{note}

\subsubsection{Ryle}

\begin{note}
  Ideas regarding \citeauthor{Ryle:1946tu}'s distinction between knowing \emph{how} and knowing \emph{that} (Cf.~\citeyear{Ryle:1946tu}).

  Now, I confess my understanding of \citeauthor{Ryle:1946tu}'s distinction is limited --- I have not taken whatever opportunities I have had to read through \citeauthor{Ryle:1946tu}'s work.%
  \footnote{
    Though, I understand enough from passing commentary to note that the idea \emph{I} am perusing here does not, strictly, require that knowledge how and knowledge that are distinct kinds of knowledge.
    (See~\textcite{Pavese:2022up} for more!)
  }

  Following analogy from~\textcite{Ryle:2009us}:

  \begin{quote}
    Knowing `\emph{if p, then q}' is, \dots rather like being in possession of a railway ticket.
    It is having a licence or warrant to make a journey from London to Oxford.
    (Knowing a variable hypothetical or `law' is like having a season ticket.)
    As a person can have a ticket without actually travelling with it and without ever being in London or getting to Oxford, so a person can have an inference warrant without actually making any inferences and even without ever acquiring the premisses from which to make them.%
    \mbox{ }\hfill\mbox{(\citeyear[250]{Ryle:2009us})}
  \end{quote}

  Continuing~\citeauthor{Ryle:2009us}'s analogy, in the case of positive answers to \qzS{}:
  What matters is that the agent is currently in possession of the (season) ticket.

  Even if current possession of the (season) ticket is knowledge that, it is present knowledge.
  And, present without being applied.
\end{note}

%%% Local Variables:
%%% mode: latex
%%% TeX-master: "master"
%%% End:
