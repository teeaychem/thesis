\chapter{\fc{3}}
\label{cha:fcs}

\nocite{Ryle:1946tu}

\begin{note}
  This chapter introduces the idea of some proposition-value pairing \(\pv{\psi}{v'}\) being a \emph{\fc{}} from some pool of premises \(\Psi\) (for an agent).

  Intuitively, \(\pv{\psi}{v'}\) being a \fc{} from \(\Psi\) means an agent has the option to conclude \(\pv{\psi}{v'}\) from \(\Psi\).
\end{note}

\begin{note}
  Our main goal is a recipe to construct counterexamples to \issueConstraint{}.
  \fc{3} are a key ingredient.
  In particular, we argue the following conditional is true:

  \begin{itemize}
  \item
    If \(\pv{\psi}{v'}\) is a \fc{} from \(\Psi\), then a \ros{} holds between \(\pv{\psi}{v'}\) and \(\Psi\).
  \end{itemize}

  And, as an agent need not have concluded \(\pv{\psi}{v'}\) from \(\Psi\), it may be the case that a \ros{} holds between \(\pv{\psi}{v'}\) and \(\Psi\) while the agent does not a \wit{} for the \ros{}.

  Hence, \emph{if} the \ros{} between \(\pv{\psi}{v'}\) and \(\Psi\) answers \qWhyV{}, \issueConstraint{} fails to hold in general.
  However, this `if' is by no means straightforward, and does not follow from the idea of a \fc{} alone.
  An additional key idea is introduced in \autoref{cha:requs}, and a key motivating idea is introduced in \autoref{cha:typical}.
\end{note}

\begin{note}
  The chapter is divided as follows:
  \begin{TOCEnum}
  \item
    \TOCLine{cha:fcs:def}

    Definition and \illu{1} of \fc{0}.
  \item
    \TOCLine{cha:fcs:support}

    Link between \fc{1} and \ros{1}.
  \end{TOCEnum}
\end{note}


\section{\fc{3}}
\label{cha:fcs:def}

\begin{note}[\fc{2} definition]
  We define a \emph{\fc{0}} as follows:

  \begin{definition}[\fc{3}]
    \label{def:fc}
    \cenLine{
      \begin{VAREnum}
      \item
        Agent: \vAgent{}
      \item
        Proposition: \(\psi\)
      \item
        Value: \(v'\)
      \item
        \pool{0}: \(\Psi\)
      \item
        \mbox{ }
      \end{VAREnum}
    }

    \begin{itemize}
    \item
      \(\pv{\psi}{v'}\) from \(\Psi\) is a \emph{\fc{0}} for \vAgent{}.
    \end{itemize}

    \emph{If and only if}

    \begin{itemize}
    \item
      There is some action \(a\) such that both \ref{def:fc:act} and \ref{def:fc:result} are true:
      \begin{enumerate}[label=\alph*., ref=(\alph*), series=fcCounter]
      \item
        \label{def:fc:act}
        \vAgent{} may easily and immediately do \(a\).
      \item
        \label{def:fc:result}
        \vAgent{} is concluding \(\pv{\psi}{v'}\) from \(\Psi\), after \(a\) is done.
        \begin{itemize}
        \item
          Without use of any novel information obtained by doing \(a\).
        \end{itemize}
      \end{enumerate}
    \end{itemize}
    \vspace{-\baselineskip}
  \end{definition}

  The definition of a \fc{} parallels the definition of a \pevent{} (\autoref{def:potenital-event}, \autopageref{def:potenital-event}).

  \begin{proposition}[\pevent{2}]%
    \label{prop:fc:pevent}%
    There is a \pevent{} in which agent concludes.
  \end{proposition}

  \begin{argument}{prop:fc:pevent}
    By definition.
  \end{argument}

  So, the basic idea, it is possible for the agent to conclude.
  Where, the sense of possibility is given in terms of the progressive.
  Key is \assuPP{}.

  Qualification is that the role of the action is only the progressive.
\end{note}

\begin{note}
  A handful of \illu{1} in \autoref{cha:fcs:illu}.
  Consider simple things.
  For example, summation of relatively small terms, or figuring out what the date is.

  Not ideal agents.
  In the terminology of \citeauthor{Simon:1957aa} (\citeyear{Simon:1957aa}) `bounded'.
  (See, in particular, \citeyear[164]{Simon:1972aa})
  Some available conclusions have not been concluded.
\end{note}

\begin{note}
  Respect in which \fc{} is weak.
  Existential.
  However, concluding.
  This is somewhat strong.

  For example, action, start reading.
  However, not reading in full.
  Likewise, open book but not reading in full.
  For, though short I don't enjoy.
\end{note}


\begin{note}
  Though `\fc{0}' is a technical term, it is intended to correspond to a sense of the common term `foregone conclusion'.
  In this sense, a foregone conclusion is an inevitable result of reasoning.
  For example, consider the following passage from~\citeauthor{Machover:1996vu}'s~\citetitle{Machover:1996vu}:

  \begin{quote}
    I have omitted its proof, but added a detailed analysis of the meaning of the lemma and the reason why its proof works. When this is understood, the proof itself becomes a mere technicality, almost a foregone conclusion.\newline
    \mbox{ }\hfill\mbox{(\citeyear[viii]{Machover:1996vu})}
  \end{quote}

  \citeauthor{Machover:1996vu} is discussing a proof, and whether or not it is inevitable that one would complete the proof (conclude that the relevant theorem is true) after understanding the lemma and why it works.
  There is no relevant sense in which the truth of the theorem has been settled in advance of reasoning.
  Though, as the proof is somewhat difficult,~\citeauthor{Machover:1996vu} only states the proof is `almost' a foregone conclusion.%
  \footnote{
    The proof is question is of the G\"{o}del-Rosser First Incompleteness Theorem.
    (\citeyear[Cf.][226]{Machover:1996vu})
  }%
  \(^{,}\)%
  \footnote{
    For a similar example without qualification, consider:
    \begin{quote}
      [\dots] Russell's evaluation of such sentences as false is predetermined by his existence presuppositional semantics for the ‘existential' quantifier, and by the fact that his logic permits no alternative means of considering the semantic status of sentences ostensibly containing proper names for nonexistent objects.
      This makes it an altogether philosophically foregone conclusion that sentences like ‘Pegasus is winged,' which many logicians would otherwise consider to be true propositions of mythology, are false.%
      \mbox{ }\hfill\mbox{(\cite[6]{Jacquette:2002up})}
    \end{quote}
  }

  The above sense of the term `foregone conclusion' contrasts against a sense with which a conclusion which has been settled in advance of reasoning.%
  \footnote{
    For example:
    \begin{quote}
      When can a Bayesian select an hypothesis \emph{H} and design an experiment (or a sequence of experiments) to make certain that, given the experimental outcome(s), the posterior probability of \emph{H} will be greater than its prior probably?
      We discuss an elementary result that establishes sufficient conditions under which this reasoning to a foregone conclusion cannot occur.%
      \mbox{ }\hfill\mbox{(\cite[1228]{Kadane:1996vu})}
    \end{quote}
  }
  And, contrasts against a sense in which a forgone conclusion is some unavoidable state of affairs.%
  \footnote{
    For example:
  \begin{quote}
    [どうぜ][\dots] Expresses an attitude of resignation or carelessness on the part of the speaker, in the sense that regardless of what s/he does, the conclusion or outcome is foregone and cannot be changed by the will or effort of an individual.%
    \mbox{ }\hfill\mbox{(\cite[332--333]{kurufushamashii:2015un})}
  \end{quote}
  See also \citeauthor{Grice:1957vg}'s discussion of intention recognition (\citeyear[385]{Grice:1957vg}/\citeyear[219]{Grice:1989uf}).
  }
\end{note}

\begin{note}
  Non-technical `foregone conclusion' and technical `\fc{}' are similar.
  However, the modal.
  No suggestion this functions as an analysis.%
  \footnote{
    \label{fn:fc-ability}
    \fc{} is close to `able' while `foregone conclusion' is closer to not able to not.
  }
\end{note}

\begin{note}
  The progressive is key.

  Possibility is too general.
  There is a possibility in which I immediately obtain a comprehensive understanding of theoretical computer science and settle whether P is equal to NP (or show that the P versus NP problem is undecidable).
  Hence, \fc{1} are defined with respect to the progressive.

  Inherit various features of the  progressive.
  For example, \(\pv{\psi}{v'}\) from \(\Psi\) being a \fc{} is compatible with the agent failing to conclude \(\pv{\psi}{v'}\) from \(\Psi\) due to interruptions, etc.
  And, concluding may be a complex action.
\end{note}

\subsection{Illustrations}
\label{cha:fcs:illu}

\begin{note}
  We begin with a few \scen{1} in which \(\pv{\psi}{v'}\) from \(\Psi\) (plausibly) \emph{is} a \fc{}.
  Then, we consider some \scen{1} in which \(\pv{\psi}{v'}\) from \(\Psi\) is \emph{not} (clearly) a \fc{}.
\end{note}

\subsubsection{\illu{3} where \(\pv{\psi}{v'}\) is a \fc{1} from \(\Psi\)}
\label{cha:fcs:illu:yes}

\begin{note}[Chess I]
  \begin{scenario}[\citeauthor{Emms:2000aa}' Puzzle 113 (\citeyear[33]{Emms:2000aa})]%
    \label{illu:fc:chess:I}%
    \mbox{ }\hfill%
    \begin{adjustbox}{minipage=\linewidth,scale=.8}
      \centering
      \newchessgame[
      setwhite={pa2,pb2,pc2,pd3,pf2,pg3,ra1,re1,bd4,kg1,qe5},
      addblack={ra8,pa7,ba6,pb5,rc8,pd5,pf7,kg8,qg4,ph7,ph4},
      ]%
      \setchessboard{showmover=false}%
      \chessboard
    \end{adjustbox}%
    \label{fig:chess:easy}%
    \hfill\mbox{ }

    \begin{center}
      Is possible for White to checkmate in a single move?
    \end{center}
    \vspace{-\baselineskip}
  \end{scenario}
\end{note}

\begin{note}
  The conclusion of interest:%
  \footnote{
    Or: \pv{\prop{It is possible for White to checkmate in a single move}}{\val{True}}
  }

  \begin{enumerate}[label=C\arabic*., ref=(C\arabic*), series=fcCEx]
    \setcounter{enumi}{\thescenarioCounter - 1}
  \item
    \label{illu:fc:chess:I:c}
    \pv{\prop{White checkmates in a single move}}{\val{Possible}}
  \end{enumerate}

  Whether or not \ref{illu:fc:chess:I:c} is a \fc{} depends on the agent.
  For the present \illu{}, suppose the agent has a basic understanding of chess and will only settle whether it is possible for White to checkmate in a single move by applying that understanding.
  Further, strategy is exhaustive search.
\end{note}

\begin{note}
  To establish \ref{illu:fc:chess:I:c} is a \fc{}, we need some action for which both clauses of \autoref{def:fc} are satisfied.

  Consider the action described by:

  \begin{center}
    `Begin an attempt to solve \citeauthor{Emms:2000aa}' Puzzle 113'.
  \end{center}

  \noindent We walk through each clause in turn.
  For ease we omit the relevant \pool{}.

  \begin{itemize}[leftmargin=*]
  \item
    Clause~\ref{def:fc:act} is satisfied.

    The action is something the agent may easily and immediately do.

    In almost any state of affairs one may attempt to do almost any thing.
    For example, you may being an attempt to square the circle.
    Failure was assured, but you made an attempt.

    The action is restricted to begin in order to ease reasoning about Clause~\ref{def:fc:result}.
  \item
    Clause~\ref{def:fc:result} is satisfied.

    We break down the argument into three separate components.

    \begin{itemize}
    \item
      There is a possible event in which the agent concludes \ref{illu:fc:chess:I:c}.

      For, given the rules of chess it is possible for White to checkmate in a single move.
      The agent has a basic understanding of chess.
      And, the agent only needs to consider the appropriate move and to verity the move results in checkmate to conclude \autoref{illu:fc:chess:I:c}.

    \item
      The agent is concluding \ref{illu:fc:chess:I:c} after the action is done.

      Action is begin an attempt to solve Puzzle 113.
      After the agent has began an attempt, the remaining action is the attempt.
      Attempt is such that the agent is concluding \ref{illu:fc:chess:I:c}.

      Assumption, strategy is exhaustive search.
      So, agent starts by picking a move for White.

      Two conditionals.

      \begin{itemize}
      \item
        If agent picks \wmove{Qh8}, then agent is concluding \ref{illu:fc:chess:I:c}.%
        \smallskip

        Basic understanding of chess, hence applying this.
      \item
        If agent picks a move other than \wmove{Qh8} then after some reasoning picks a different move.%
        \smallskip

        For, understanding, identifies issue.
        Attempt, hence something else.
      \end{itemize}

      Now, whatever the agent picks, they eventually pick \wmove{Qh8}.
      And, \wmove{Qh8} then concluding.
      Hence, as any pick leads to concluding, agent is concluding.
    \item
      The agent does not use any novel information obtained by beginning an attempt.

      For, this action does not provide the agent  with any novel information.
      The agent has looked at the puzzle and has a basic understanding of the rules of chess.
      The agent does not appeal to any information that they do not already possess, and any information obtained follows.
    \end{itemize}
  \end{itemize}
\end{note}

\begin{note}
  Argument assumes a particular strategy.
  This need not be the case, but alternative strategies are difficult to describe.
  Still, I suggest, if you have a basic understanding of chess then I suggest you convince yourself the conclusion is a \fc{} by attempting to solve the puzzle.
  You need only conclude \ref{illu:fc:chess:I:c} and verify that you were concluding \ref{illu:fc:chess:I:c} after you began (and did not make use of any novel information).
\end{note}

\begin{note}
  In broad structure, the idea with \autoref{illu:fc:chess:I} is some effective method, sufficient information to apply method, and an opportunity to apply.

  For example, consider arithmetic.
  I expect the truth of \(13 \cdot 4 = 52\), \(96 \div 4 = 24\), and \(23 \cdot 15 = 345\) are \fc{1}.
  Likewise, if you have basic understanding of propositional logic, then the validity of various theorems are \fc{1}.
  And, if you enjoy Sudoku puzzles then, so long as you have the puzzle and sufficient time to spare, the solution to any puzzle is a \fc{}.
\end{note}

\begin{note}[Non-deductive \illu{1}]
  \autoref{illu:fc:chess:I}, some effective method.
  However, \(\pv{\psi}{v'}\) being a \fc{} from \(\Psi\) does not require an effective method.
  It need only be the case that the agent is concluding \(\pv{\psi}{v'}\) from \(\Psi\) after an action is done.
  Likewise, it need not be the conclusion is the case.

  \begin{scenario}[Sunny days]%
    \label{illu:fc:sunny}%
    It's mid summer day in the Bay Area.
  \end{scenario}

  \noindent For me, the following conclusion is a \fc{} from some \pool{}:

  \begin{enumerate}[label=C\arabic*., ref=(C\arabic*), resume*=fcCEx]
  \item
    \label{illu:fc:sunny:c}
    \pv{\prop{It will rain tomorrow}}{\val{False}}
  \end{enumerate}

  \noindent There is no effective method for me to determine whether it will rain tomorrow, and I recognise there may be rain tomorrow.
  Still, I am sufficiently committed to some uniformity principle.%
  \footnote{
    Cf. (\cite[70]{Hempel:1965aa}),~(\cite{Henderson:2022aa}).
  }
  And, that the principle together with past experience, ensure that if I consider whether it will rain tomorrow, I conclude it will not rain.%
  % \footnote{
  %   Same extends to various skeptical hypotheses.
  %   Entertain the possibility that there is no external world, but nothing that prevents me from concluding that there is an external world.
  %   Though, your perspective on such issues may differ.
  % }
\end{note}

\begin{note}[Poppies]
  To finish, we take something from literature:

  \begin{scenario}[Poppies]
    \label{illu:fc:poppies}
    \mbox{ }
    \vspace{-\baselineskip}
    \begin{quote}
      Was Tarquinius Superbus in seinem Garten mit den Mohnköpfen sprach, verstand der Sohn, aber nicht der Bote.

      [What Tarquinius Superbus said in the garden by means of the poppies, the son understood but the messenger did not].\newline
      \mbox{ }\hfill\mbox{(\cite[3]{Kierkegaard:1983ta}/\cite[190]{Hamann:1822vp})}
  \end{quote}
  \vspace{-\baselineskip}
  \end{scenario}

  \noindent The quote is from the epigraph to~\citeauthor{Kierkegaard:1983ta}'s \hyperlink{cite.Kierkegaard:1983ta}{Fear and Trembling}.
  \hyperlink{cite.Kierkegaard:1983ta}{H.\ Hong and E.\ Hong} detail the relevant background:

  \begin{quote}
    When the son of Tarquinius Superbus had craftily gotten Gabii in his power, he sent a messenger to his father asking what he should do with the city.
    Tarquinius, not trusting the messenger, gave no reply but took him into the garden, where with his cane he cut off the flowers of the tallest poppies.
    The son understood from this that he should eliminate the leading men of the city.%
    \mbox{ }\hfill\mbox{(\citeyear[339]{Kierkegaard:1983ta})}
  \end{quote}

  \noindent For Superbus' son, but not for the messenger the following was a \fc{} from some \pool{}:

  \begin{enumerate}[label=C\arabic*., ref=(C\arabic*), resume*=fcCEx]
  \item
    \label{illu:fc:poppies:c}
    \pv{\prop{Eliminate the leading men of the city}}{\val{Should}}
  \end{enumerate}

  \noindent Or, at least, Superbus \emph{expected}~\ref{illu:fc:poppies:c} be a \fc{} for his son.
\end{note}

\subsubsection{\(\pv{\psi}{v'}\) is not a \fc{1} from \(\Psi\)}
\label{cha:fcs:illu:no}

\begin{note}
  Two different ways.
  First, possible, though absence of resources.
  Second, the agent may conclude something which conflicts, and hence does not conclude.
\end{note}

\paragraph*{Resources}

\begin{note}[Chess II]
  Consider \autoref{illu:fc:chess:II} in the same context as \autoref{illu:fc:chess:I}:

  \begin{scenario}[\citeauthor{Emms:2000aa}' Puzzle 150 (\citeyear[33]{Emms:2000aa})]%
    \label{illu:fc:chess:II}%
    \mbox{ }\hfill%
    \begin{adjustbox}{minipage=\linewidth,scale=0.8}
      \centering
      \newchessgame[
      setwhite={ka5,pa3,pb4,pc4,pe5,pf6,bg5,bh5},
      addblack={pa6,pb7,pc6,pe6,pf7,kc7,nd7,nd4},
      ]%
      \setchessboard{showmover=false}%
      \chessboard
    \end{adjustbox}%
    \label{fig:chess:intro}%
    \hfill\mbox{ }

    \begin{center}
      It is possible for Black to checkmate in four moves?
    \end{center}
    \vspace{-\baselineskip}
  \end{scenario}

  \noindent Conclusion of interest is:

  \begin{enumerate}[label=C\arabic*., ref=(C\arabic*), resume*=fcCEx]
  \item
    \label{illu:fc:chess:II:c}
    \pv{\prop{Black checkmates in four moves}}{\val{Possible}}
  \end{enumerate}

  The difference between \autoref{illu:fc:chess:II} and \autoref{illu:fc:chess:I} is the difficulty of the puzzle.%
  \footnote{
    \citeauthor{Emms:2000aa} suggests:
    \textquote{%
      \variation{1... Nb6!}%
      (threatening \variation{2... Nb3\#})%
      \variation{2. b5}%
      (or \variation{2. Bd1 Nxc4+} \variation{3. Ka4 b5\#})%
      \variation{2... c5!}%
      \variation{3. bxa6 Nxc4+}%
      \variation{4. Ka4 b5\#}%
      \textbf{(0-1)}%
      }
      (\citeyear[46]{Emms:2000aa}).
    }
    An agent may do an action described by `begin an attempt to solve\dots'.
    And, there may be a possible event in which the agent concludes~\ref{illu:fc:chess:II:c}.
    However, a basic understanding of chess does not imply the capacity to work through a complex sequence of moves in the relevant situation.
    Indeed, the agent may simply lack the capacity.

    Further, even if an agent has the capacity, they have no interest in exercising the capacity.
    For some, chess puzzles aren't fun.

  %   Indeed, if the agent is inclined to randomly select sequences of moves and is short of patience, then an event in which the agent concludes~\ref{illu:fc:chess:II:c} is not in progress after the agent performs the action.
  % For, the agent may fail to consider an appropriate sequence before giving up and doing something else.

  Specifically,~\ref{illu:fc:chess:II:c} was not a \fc{} for me.%
  \footnote{
    I gave up after fifteen minutes or so, and I'm not sure I have the capacity.
  }
\end{note}

\begin{note}
  In general, it may be possible for the agent to conclude, but the possibility is weaker than the sense of possibility captured by the truth of the progressive.

  For example, uninteresting formal derivation or difficult arithmetic.

  Indeed, seems progressive rules out various instances of `lucky' conclusions.
\end{note}

\paragraph*{Conflict}

\begin{note}

  \begin{scenario}[Knowing whether and belief]%
    \label{ill:fcs:kw}%
    \citeauthor{Barker:1975un} suggests the following two principles hold with respect to knowing whether:%
    \footnote{
      \citeauthor{Barker:1975un} also, as far as I can tell, endorses the principles.
    }
    \begin{enumerate}[label=(\Alph*), ref=(\Alph*), noitemsep]
    \item
      \label{Barker:1975un:A}
      If \emph{S} knows whether \emph{p} and \emph{S} believes that \emph{p}, then \emph{p}.
    \item
      \label{Barker:1975un:B}
      If \emph{S} knows whether \emph{p} and \emph{S} believes that not-\emph{p}, then not-\emph{p}.%
      \mbox{ }\hfill\mbox{(\citeyear[281]{Barker:1975un})}
    \end{enumerate}
  \end{scenario}
  I suggest neither principle is \fc{}, as you may conclude counterexamples exist to both.%
  \footnote{
    For example, consider two agents, \emph{A} and \emph{B} playing chess where each move is timed.
    It's the end game, and \emph{A} believes that \emph{B} has a winning strategy.
    Further, \emph{A} (plausibly) knows whether \emph{B} has a winning strategy.
    For, an observer has determined whether or not \emph{B} has a winning strategy, and \emph{A} is capable of tracing the reasoning of the observer.
    So, if \ref{Barker:1975un:A} holds then \emph{B} has a winning strategy.
    But, the observer knows that \emph{B} \emph{does not} have a winning strategy, and \emph{A}'s belief is mistaken.
  }
  In contrast to \autoref{illu:fc:chess:II}, issue is conflicting conclusion.
\end{note}

\begin{note}
  Structurally, \autoref{ill:fcs:kw} is no different from an arithmetic equation which does not hold.
  For example, \(4 \times 3 = 14\).
\end{note}

\section{\fc{3} and \ros{1}}
\label{cha:fcs:support}

\begin{note}
  \autoref{cha:fcs:def}, defined, discussed, and provided some \illu{1} of \fc{1}.
  We now consider the relationship between \fc{1} and \ros{0}.

  As mentioned:

  \begin{itemize}
  \item
    If \(\pv{\psi}{v'}\) is a \fc{} from \(\Psi\), then a \ros{} holds between \(\pv{\psi}{v'}\) and \(\Psi\).
  \end{itemize}

  This, then, moves some way to \issueConstraint{}.
  Though, still need to show \ros{} answers \qWhyV{}.

  We argue for the conditional, and consider some worries.
\end{note}


\begin{note}
  We begin with the following proposition:

  \begin{proposition}[\potential{2} \ros{1}]
    \label{prop:fcs-only-if-pot-support}
    \cenLine{
      \begin{VAREnum}
      \item
        Agent: \vAgent{}
      \item
        Proposition: \(\psi\)
      \item
        Value: \(v'\)
      \item
        \pool{2}: \(\Psi\)
      \item
        \mbox{ }
      \end{VAREnum}
    }

    \begin{enumerate}
    \item[\emph{If}:]
      \(\pv{\psi}{v'}\) is a \fc{0} from \(\Psi\), for \vAgent{}.
    \item[\emph{Then}:]
      A \emph{\potential{0}} \ros{} between \(\pv{\psi}{v'}\) and \(\Psi\) holds, for \vAgent{}.
    \end{enumerate}
    \vspace{-\baselineskip}
  \end{proposition}

  \begin{argument}{prop:fcs-only-if-pot-support}
    Suppose \(\pv{\psi}{v'}\) is a \fc{0} from \(\Psi\), for \vAgent{}.
    Then, by \autoref{prop:fc:pevent} (\autopageref{prop:fc:pevent}) there is a \pevent{} in which \vAgent{} concludes \(\pv{\psi}{v'}\) from \(\Psi\).

    Now, consider the \pevent{} in which \vAgent{} concludes \(\pv{\psi}{v'}\) from \(\Psi\).
    As \vAgent{} concludes \(\pv{\psi}{v'}\) from \(\Psi\), then by~\supportI{} (\supportIpage{}), a \ros{} between \(\pv{\psi}{v'}\) and \(\Psi\) holds when \vAgent{} \eval{1} \(\psi\) as having value \(v'\).

    Hence, in whatever sense the event is \potential{0}, a \ros{} between \(\pv{\psi}{v'}\) and \(\Psi\) is likewise \potential{0}.
  \end{argument}
\end{note}

\begin{note}
  \emph{\potential{2}} \ros{}, but it does not follow that there is a \ros{}, for the agent.
  The following proposition argues that a \potential{2} \ros{} is a \ros{}:

  \begin{proposition}[Actual \ros{1}]
    \label{prop:pot-support-onlyIf-support}
    \cenLine{
      \begin{VAREnum}
      \item
        Agent: \vAgent{}
      \item
        Proposition: \(\psi\)
      \item
        Value: \(v'\)
      \item
        \pool{2}: \(\Psi\)
      \item
        \mbox{ }
      \end{VAREnum}
    }

    \begin{enumerate}
    \item[\emph{If}:]
      A potential \ros{} between \(\pv{\psi}{v'}\) and \(\Psi\) holds, for \vAgent{}.
    \item[\emph{Then}:]
      A \ros{0} between \(\pv{\psi}{v'}\) and \(\Psi\) holds, for \vAgent{}.
    \end{enumerate}
    \vspace{-\baselineskip}
  \end{proposition}

  \begin{argument}{prop:pot-support-onlyIf-support}
    It is possible for there to be.

    What is missing?

    Well, from \pevent{}, something.
    However, additional qualification.

    So, we have everything needed.
    Both necessary and sufficient.

    \begin{itemize}
    \item
      If \fc{} the every necessary property holds.

      Suppose some thing is necessary.
      By \supportII{} (\supportIIpage{}), this thing does not follow from \wit{}.
      By X \pevent{}.
      And, by qualification, do not need any novel information.

    \item
      If \fc{} then some sufficient property holds.

      Some collection of things which are sufficient.
      Observe, disjunction of all sufficient thing is a necessary condition.
      For, if \ros{} holds then some sufficient thing holds.
      However, we have argued every necessary things holds.
      Therefore, some sufficient thing must also hold.
    \end{itemize}
  \end{argument}

  \autoref{prop:pot-support-onlyIf-support} is odd.
  Only sufficient condition for \ros{} is conclusion.
  Hence, \pevent{} to get sufficient condition.
  However, \wit{} is not necessary.
  So, some weaker sufficient condition.
  Weaker sufficient condition must hold.

  Still, think in terms of propositional justification.%
  \footnote{
    Distinct, however, from something like reflection.
    For, dealing with ideals.
    And, no need to complete reasoning.
  }
\end{note}

\begin{note}
  \begin{proposition}[\fc{3} and \ros{1}]
    \label{prop:fcs-only-if-support}
    \cenLine{
      \begin{VAREnum}
      \item
        An agent: \vAgent{}
      \item
        A proposition: \(\phi\)
      \item
        A value: \(v\)
      \item
        A \pool{0}: \(\Phi\)
      \item
        \mbox{ }
      \end{VAREnum}
    }

    \begin{enumerate}
    \item[\emph{If}:]
      \(\pv{\psi}{v'}\) is a \fc{0} from \(\Psi\), for \vAgent{}.
    \item[\emph{Then}:]
      A \ros{0} between \(\pv{\phi}{v}\) and \(\Phi\) holds, for \vAgent{}.
    \end{enumerate}
    \vspace{-\baselineskip}
  \end{proposition}

  \begin{argument}{prop:fcs-only-if-support}
    Immediate by combining \autoref{prop:fcs-only-if-pot-support} and \autoref{prop:pot-support-onlyIf-support}.
  \end{argument}

  For, in order to argue against \issueConstraint{}, need some \(\pvp{\psi}{v'}{\Psi}\) such that answers \qWhyV{}.
  Consider \autoref{illu:gist:calc}.
  \fc{}.
  However, intuitively does not answer \qWhy{} nor \qWhyV{}.
\end{note}

\paragraph*{Worries}

\begin{note}
  Some doubt.
  Possible to conclude things which conflict.
  No restrictions placed on what an agent may conclude.
  But, this is no worse than \wit[es]{}.
  Possible to have concluded things which conflict.

  Though, nothing depends on this.
  If a simple way to rule out, then would.
  But, it's not so simple.
\end{note}

\begin{note}
  Could strengthen.

  \begin{enumerate}[label=\alph*., ref=(\alph*), resume*=fcCounter]
  \item
    For any proposition \(\phi'\), value \(v'\), and action \(a'\) such that \vAgent{} is concluding \(\pv{\phi'}{v'}\) from \(\Phi\) after \(a'\) is done:
    \begin{itemize}
    \item
      Either:
      \begin{enumerate}[label=\arabic*., ref=(\arabic*)]
      \item
        \(\phi'\) is \(\phi\) and \(v'\) is \(v\).%
        \footnote{
          In other words, \vAgent{} is concluding \(\pv{\phi}{v}\) from \(\Phi\) after doing \(a'\).
        }
      \item
        Throughout event in which \vAgent{} concludes \(\pv{\phi'}{v'}\) from \(\Phi\), there is some action \(b\) such that \vAgent{} is concluding \(\pv{\phi}{v}\) from \(\Phi\) after \(b\) is done.
      \end{enumerate}
    \end{itemize}
  \end{enumerate}

  This is a much stronger condition.
  Not only is it the case that concluding, but remains the case that concluding is available for any other conclusion.

  I like this condition.
  But, it's not of significant help.
  For, there is no guarantee that this extends further.

  Further, borders on normative considerations.

  Could say (channelling \cite{Carroll:1895uj}) that the agent \textquote{ca'n't help themselves}.
  However, what this modal amounts to is unclear.%
  \footnote{
    These concerns, in part, motivate \autoref{fn:fc-ability}.
  }

  Likewise, jumping to conclusions and so on.
\end{note}

\begin{note}
  Still, if you prefer something stronger, that's okay.
  There's nothing in what follows that strictly requires that definition of a \fc{} is sufficient.
  Though, some things will need to be restructured.
\end{note}

\begin{note}
  \color{red}
  Worry.
  \support{2} doesn't rely on witnessing.
  Now, if this goes through, then seems \support{} for any conclusion before making the conclusion.
  However, possible for the agent to reason to different conclusions.
  For, some faulty reasoning.
  Toggle the fault.
  Therefore, \support{} for contradictory conclusions.
\end{note}

\section{Summary}
\label{cha:fcs::summary}


% \subsubsection{\wit{3}}
% \label{sec:wit3}

% \begin{note}
%   Note, there is nothing that requires a \fc{} has no prior conclusion.

%   Hence, it may be the case that an agent has a \wit{} for a \fc{}.

%   \begin{proposition}[\wit{3} for \fc{1}]
%     \label{prop:wit-for-fc}
%     Possible for \wit{} from \fc{}.
%   \end{proposition}

%   \begin{argument}{prop:wit-for-fc}
%       Immediate by def of \wit{}.
%     \end{argument}
%   \end{note}


%%% Local Variables:
%%% mode: latex
%%% TeX-master: "master"
%%% End:


% \begin{note}
%   \begin{illustration}[Modal logic I]
%     \label{illu:fc:logic:CR}
%     The modal system obtained from adding \(\Diamond\Box p \rightarrow \Box\Diamond p\) as an axiom to \(\mathbf{K}\) is canonical for the Church-Rosser property.

%     I.e. the canonical model \(W,R,V\) for \(\mathbf{K} + \Diamond\Box p \rightarrow \Box\Diamond p\) is such that \(\forall s,t,u((Rst \land Rsu) \rightarrow \exists v(Rtv \land Ruv))\).
%   \end{illustration}

%   \autoref{illu:fc:logic:CR} is a \fc{} for me.
%   Though, in contrast to the previous \illu{1}, I think there is a reasonable change that \autoref{illu:fc:logic:CR} is not a \fc{} for you.

%   Fairly routine, but two important things.
%   First, grasp on the relevant concepts.
%   If you are unaware of how to construct canonical models for normal modal logics, then unlikely that you will complete the relevant proof.
%   Second, sufficient familiarity with the relevant concepts.
%   The proof is mostly straightforward, though some care needs to be taken in showing that the canonical model for \(\mathbf{K} + \Diamond\Box p \rightarrow \Box\Diamond p\) has the Church-Rosser property.
%   Proof by contradiction is my preferred way of obtaining the result, but this requires keeping certain facts about the canonical model in mind.%
%   \footnote{
%     A slightly more interesting variation is showing that \(\mathbf{K} + \Diamond\Box p \rightarrow \Box\Diamond p\) is (strongly) complete with respect to the class of frame which have the Church-Rosser property without detour via a canonical model.
%   }

%   Similar features as \illu{1} given above.

%   In particular, perhaps clearer than \autoref{illu:gist:Sudoku} in terms of mistakes.
%   For, go down some wrong path, still will not conclude until counterexample.
%   And, this is very hard to get.
% \end{note}

  %   \begin{itemize}
  %   \item
  %     The agent has a good understanding of some formal proof system.
  %     For example, some Fitch-style system.
  %   \item
  %     The agent has a good understanding of some method to construct semantic proofs.
  %     For example, by constructing truth tables, or reasoning about valuation functions.
  %   \item
  %     The agent understands the proof system is sound.
  %     That is to say, the agent understands there exists a proof of some sentence \(A\) \emph{only if} \(A\) is true given an arbitrary valuation.
  %   \end{itemize}
  %   The agent constructs the following proof:
  %   \begin{center}
  %     \begin{fitch}
  %       \phantlabel{illu:sPF:1}\fa \fh (P \rightarrow Q) \rightarrow P \\
  %       \phantlabel{illu:sPF:2}\fa \fa \fh  \lnot P \\
  %       \phantlabel{illu:sPF:3}\fa \fa \fa \fh  P \\
  %       \phantlabel{illu:sPF:4}\fa \fa \fa \fa  \bot & \(\bot\)\textbf{Intro:}\hyperref[illu:sPF:2]{2},\hyperref[illu:sPF:3]{3}\\
  %       \phantlabel{illu:sPF:5}\fa \fa \fa \fa  Q & \(\bot\)\textbf{Elim:}\hyperref[illu:sPF:4]{4}\\
  %       \phantlabel{illu:sPF:6}\fa \fa \fa P \rightarrow Q & \(\rightarrow\)\textbf{Intro:}\hyperref[illu:sPF:3]{3}--\hyperref[illu:sPF:5]{5} \\
  %       \phantlabel{illu:sPF:7}\fa \fa \fa P & \(\rightarrow\)\textbf{Elim:}\hyperref[illu:sPF:1]{1},\hyperref[illu:sPF:6]{6}\\
  %       \phantlabel{illu:sPF:8}\fa \fa \fa \bot & \(\bot\)\textbf{Intro:}\hyperref[illu:sPF:2]{2},\hyperref[illu:sPF:7]{7}\\
  %       \phantlabel{illu:sPF:9}\fa \fa \lnot\lnot P & \(\lnot\)\textbf{Intro:}\hyperref[illu:sPF:2]{2},\hyperref[illu:sPF:8]{8}\\
  %       \phantlabel{illu:sPF:10}\fa \fa P & \(\lnot\)\textbf{Elim:}\hyperref[illu:sPF:9]{9}\\
  %       \phantlabel{illu:sPF:11}\fa ((P \rightarrow Q) \rightarrow P) \rightarrow P & \(\rightarrow\)\textbf{Intro:}\hyperref[illu:sPF:1]{1}--\hyperref[illu:sPF:10]{10} \\
  %     \end{fitch}
  %   \end{center}
  %   \vspace{-\baselineskip}


%   For an additional example, consider the following from~\citeauthor{Grice:1957vg}'s~\citetitle{Grice:1957vg}:

%   \begin{quote}
%     He intends the audience's recognition of his intention to produce that response to be effective in producing that response.
%     He does not regard it as a foregone conclusion that his action will produce the intended response, whether or not his intention is recognised.%
%     \mbox{ }\hfill\mbox{(\citeyear[385]{Grice:1957vg}/\citeyear[Cf.][219]{Grice:1989uf})}
%   \end{quote}

%   In this case, the term `foregone conclusion' is embedded under negation, to highlight that the agent in question entertains the possibility that the agent's action will not produce the intended response.

% \begin{note}
%   Still, the conclusion need not be easy.

%   \begin{illustration}
%     \[\frac{(3 + \sqrt{3})^{2} + \sqrt{6}^{2} - (2\sqrt{3})^{2}}{2(3 + \sqrt{3})\sqrt{6}} = \frac{1}{\sqrt{2}}\]
%   \end{illustration}

%   I suspect this is a \fc{}.
%   Might need to do some work to recall principles, but it's okay.
% \end{note}

% \begin{note}[Propositional logic]
%   \begin{illustration}[Propositional theorems]
%     \label{illu:sketch:prop-logic}
%     Suppose an agent has a good grasp of propositional logic.
%     In particular:
%     The agent has a good understanding of some method to construct semantic proofs.
%     For example, by constructing truth tables, or reasoning about valuation functions.

%     The agent has some free time, and considers the formula:
%     \[
%       ((P \rightarrow Q) \rightarrow P) \rightarrow P
%     \]
%   \end{illustration}

%   Given the agent's understanding of propositional logic, the following is a \fc{}:
%   \begin{itemize}
%   \item
%     For any valuation \(v\), \(v \vDash ((P \rightarrow Q) \rightarrow P) \rightarrow P \)
%   \end{itemize}

%   There's nothing particularly special about the formula.
%   Given a good understanding, any formula of a reasonable length will do.

%   Note, the agent is concluding.
%   It's fine for the agent to work things through on a piece of paper.
% \end{note}

% \begin{note}[ML II]
%   \begin{illustration}[Modal logic II]
%     \label{illu:fc:ML2}
%     The modal system \(\mathbf{GL} = \mathbf{K} + \Box(\Box p \rightarrow p) \rightarrow \Box p\) is weakly complete with respect to the class of finite strict partial orders (that is, the class of finite irreflexive transitive frames).
%   \end{illustration}

%   \autoref{illu:fc:ML2} is similar in structure to \autoref{illu:fc:logic:CR}.
%   Indeed, both proofs involve constructing a canonical model.
%   The key distinguishing feature of \autoref{illu:fc:ML2}, however, is the difficulty of establishing the canonical model has the desired properties.
%   In particular, the general method I keep in mind for proving the relevant result requires a syntactic proof that \(\vdash_{\mathbf{GL}} \Box p \rightarrow \Box \Box p\).
%   And, as I have failed to recall the relevant syntactic on sufficient occasion, I do not consider the result a \fc{0} from my understanding of modal logic.

%   Hence, the result (plausibly) fails to be a \fc{} from my understanding of modal logic because there is no guarantee that I would provide a proof if I set out to do so.%
%   \footnote{
%     On the other hand, I have completed the relevant proof a sufficient number of times.
%     So, the result is a \fc{0} from whatever premises are associated with my memory.
%   }
% \end{note}

% \begin{note}
%   \illu{3} may be obtained by taking a proposition-value pairing which conflicts with \fc{}.
%   The following is \emph{not} a \fc{}.%
%   \footnote{
%     However, I suspect the following equation is a \fc{}:
%     \[\frac{(3 + \sqrt{3})^{2} + \sqrt{6}^{2} - (2\sqrt{3})^{2}}{2(3 + \sqrt{3})\sqrt{6}} = \frac{1}{\sqrt{2}}\]
%   }

%   \begin{illustration}
%     \[\frac{(3 + \sqrt{3})^{2} + \sqrt{6}^{2} - (2\sqrt{3})^{2}}{2(3 + \sqrt{3})\sqrt{6}} = \frac{1}{\sqrt{3}}\]
%   \end{illustration}
%   For, the equation does not hold.
% \end{note}

  %
  % \footnote{
  %   \nocite{Simon:1997aa}
  %   \textquote{[R]ationality can be bounded by assuming complexity in the cost function or other environmental constraints so great as to prevent the actor from calculating the best course of action} (\citeyear[164]{Simon:1972aa}).
  % }


% \begin{note}
%   This is no different from conclusion that it's 13:48 by looking at the clock.
%   I recognise there is a possibility that the clock is broken.
%   However, nothing to suggest it is.

%   This highlights qualification.
%   Without it, then so long as wearing watch, time is a \fc{}.
%   For, look at watch.
% \end{note}
