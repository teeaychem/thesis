\chapter{\fc{3}}
\label{cha:fcs}

\section{Introduction}
\label{cha:fcs:sec:introduction}

\begin{note}
  So, introduced question, \qzS{}.

  \requ{}, it is the case that the agent would conclude, from the agent's perspective.

  Potential event.

  Introduce idea of a \fc{}.

  Two main ideas.
  First, motivation for \fc{1}.
  Second, link to \support{}.
  And, some observations.
\end{note}

\begin{note}
  Support.

  Idea here is something distinguished about concluding, but is independent of whether or not the agent has witnessed concluding.

  \autoref{idea:support} and \autoref{idea:support:possible}.

  Noted, independence does not entail relation of support from agent's perspective without concluding.

  Parallel to propositional and doxastic justification.
\end{note}

\section{\fc{3}}
\label{sec:fc3-2}

\begin{note}
  The consequent here, potential event.
  Term this a \fc{}.

  \begin{restatable}[Foregone-conclusions]{definition}{definitionForegoneC}
    \label{def:fc}
    For an agent \vAgent{}, and some proposition-value-premises pairing \(\pvp{\psi}{v'}{\Psi}\):

    \begin{itemize}
    \item
      \(\pv{\phi}{v}\) is a \emph{\fc{0}} from \(\Phi\), with respect to \vAgent{}
    \end{itemize}
    \emph{If and only if}
    \begin{itemize}
    \item
      There is a potential event in which \vAgent{} concludes \(\pv{\phi}{v}\) from \(\Phi\).
    \end{itemize}
  \end{restatable}

  \fc{3} are stated from a neutral perspective --- at issue is whether there is a potential event in which the agent concludes --- however, our interest with \fc{1} will be from an agent's perspective.
  Specifically, whether \(\pv{\phi}{v}\) is a \fc{0} from \(\Phi\) with respect to \vAgent{}, from \vAgent{}' perspective.
  To simplify, we will shorten a positive instance to:
  `\(\pv{\phi}{v}\) is a \fc{0} from \(\Phi\), from \vAgent{}' perspective.'%
  \footnote{
    And, perhaps, in some cases to:
    `\(\pvp{\phi}{v}{\Phi}\) is a \fc{0}, from \vAgent{}' perspective.'
  }
\end{note}

\begin{note}
  Basic idea, is straightforward.
  Some motivation with instances of knowing how.
  Or, the agent having the ability to reason (in such a way that culmination of reasoning is a conclusion).

  However, the term `\fc{0}' is a little narrower than broad intuition may suggest.
  For, the event in which the agent conclude \(\pv{\phi}{v}\) from \(\Phi\) is qualified by the term `potential'.
  
\end{note}

\section{Potential}
\label{sec:potential}

\begin{note}
  Potential.

  As stated, for an agent.
  More explicit, indexed in various ways.
  Knowing whether, before learnt the rules of chess, valid move was not known whether.

  Relevant index is the agent, as they are.
  Potential, right now.
\end{note}

\begin{note}
  `Potential' is a term, other's may be substituted.
  Natural is `opportunity'.
  Though, it is not clear to me how natural this reading is.
  Consider the following passage from~\citeauthor{Austin:1961vz}:

  \begin{quote}
    Consider the case where what we wish to assert is that somebody had the opportunity to do something but lacked the ability---`He could have smashed that lob, if he had been any good at the smash': here the \emph{if}-clause, which may of course be suppressed and understood, relates not to opportunity but to ability.%
    \mbox{ }\hfill\mbox{(\citeyear[177]{Austin:1961vz})}
  \end{quote}
  Likewise for ability:
  \begin{quote}
    `He could have read \emph{Emma}, if he had had a copy', does seem to assert `categorically' that he had a certain ability, although he lacked the opportunity to exercise it.%
    \mbox{ }\hfill\mbox{(\citeyear[177]{Austin:1961vz})}
  \end{quote}

  Potential, inttuively, just in case both the ability and the opportunity.

  Example of what \textcite{Hackl:1998tt} terms `opportunity-can' (\citeyear[14]{Hackl:1998tt}):

  \begin{quote}
    \begin{enumerate}
    \item[(92)]
      \begin{enumerate}[label=\alph*., ref=(\alph*)]
      \item
        \label{Hackl:OC:a}
        A star gazer can see the solar eclipse of this year from the Cayman islands.\newline
        So if you were a star gazer and if you were on the Cayman islands at the right time you would see this year's solar eclipse.
      \item
        \label{Hackl:OC:b}
        John can see Mary from where he is standing.\newline
        So if you were standing in his place, you would see Mary.
      \end{enumerate}
    \end{enumerate}

    [\ref{Hackl:OC:b}] says that whoever is in this situation located at John's position and has normal eyesight and directs his/her gaze towards Mary will succeed in seeing Mary.%
    \mbox{ }\hfill\mbox{(\citeyear[39]{Hackl:1998tt})}
  \end{quote}
  Likewise for \ref{Hackl:OC:a}:
  A star gazer who is in the Cayman islands at the right time this year and looks for the solar eclipse will succeed in seeing the solar eclipse.%
  \footnote{
    Whether or not this is \emph{true} is a different matter\dots
  }
\end{note}


\begin{note}[Importance]
  Constraining \fc{1} to \emph{potential} events is of some importance.

  In following chapter, tie \fc{1} to instances of concluding.

  For example, event of concluding \(x + y = z\), and same.

  Potential, substitute conclusion.
  With failure cases, more difficult.
  Not the case potential, then would not have done the reasoning.
\end{note}

\section{Broad motivation}
\label{sec:broad-motivation}

\begin{note}
  \begin{illustration}[Sudoku]
    \label{illu:gist:sudoku}
    % https://tex.stackexchange.com/questions/91422/tikz-sudoku-circle-and-connect-with-lines-some-cells
    \begin{figure}[H]
      \mbox{ }\hfill
      \begin{subfigure}{0.45\linewidth}
        \centering
        \sudokuGrid{}
        \caption{The starting grid \dots}
        \label{fig:sudoku:grid}
      \end{subfigure}
      \hfill
      \begin{subfigure}{0.45\linewidth}
        \centering
        \sudokuGridHints{}
        \caption{\dots with hints}
        \label{fig:sudoku:hint}
      \end{subfigure}
      \hfill\mbox{ }
      \caption{Sudoku}
      \label{fig:sudoku}
    \end{figure}
    Know whether the hints are correct, and likewise know whether 3 is the correct number to place in the bottom left corner.
  \end{illustration}
\end{note}

\begin{note}
  Know whether \(x + y = z\).
  Indeed, for any \(z\), know whether \(x + y = z\).

  \[\frac{(3 + \sqrt{3})^{2} + \sqrt{6}^{2} - (2\sqrt{3})^{2}}{2(3 + \sqrt{3})\sqrt{6}} = \frac{1}{\sqrt{2}}\]

  Know whether equation is true.
\end{note}

\begin{note}
  Know whether the modal system \(\mathbf{K} + \Diamond\Box p \rightarrow \Box\Diamond p\) is weakly complete with respect to the class of frames which have the Church-Rosser property.%
  \footnote{
    \(\forall s,t,u((Rst \land Rsu) \rightarrow \exists v(Rtv \land Ruv))\).
  }

  The theorem is not quite routine.
  Some care needs to be taken when choosing valuation.
  However, fairly easy.

  Here, stress point that potential event.
  Even if fail on attempting, this doesn't raise a problem.
  Indeed, if get stuck, get hint, and I expect you will consider it the case that you could have done the thing.
\end{note}


\paragraph{Not clearly \fc{1}}

\begin{note}
  Provided a handful of (plausible) instances of knowing whether which (plausibly) involve \fc{1}.
  A pair of (plausible) instances whether which (plausibly) do not involve \fc{1}.
\end{note}

\begin{note}[ML II]
  The modal system \(\mathbf{K} + \Box(\Box p \rightarrow p) \rightarrow \Box p\) is weakly complete with respect to the class of finite strict partial orders (that is, the class of finite irreflexive transitive frames).
\end{note}

\begin{note}[Chess]
  \begin{figure}[H]
    \centering
    \begin{adjustbox}{minipage=\linewidth,scale=0.7}
      \centering
      \newchessgame[
      setwhite={ka5,pa3,pb4,pc4,pe5,pf6,bg5,bh5},
      addblack={pa6,pb7,pc6,pe6,pf7,kc7,nd7,nd4},
      ]%
      \setchessboard{showmover=false}%
      \chessboard
    \end{adjustbox}
    \caption{Chess board.}%\protect\footnotemark}
    \label{fig:chess:intro}
  \end{figure}

  It is possible for Black to checkmate in four moves.
  Not clear that this is a \fc{0} as potential.
  I, at least, fail.
  Perhaps understanding is good, and wouldn't be so hard.%
  \footnote{
    \citeauthor{Emms:2000aa}' Puzzle 150 (\citeyear[33]{Emms:2000aa}).
    \citeauthor{Emms:2000aa} provides the following solution:
    \begin{quote}
      \variation{1... Nb6!}
      (threatening \variation{2... Nb3\#})
      \variation{2. b5}
      (or \variation{2. Bd1 Nxc4+} \variation{3. Ka4 b5\#})
      \variation{2... c5!}
      \variation{3. bxa6 Nxc4+}
      \variation{4. Ka4 b5\#}
      \textbf{(0-1)}\nolinebreak
      \mbox{}
      \hfill
      (\citeyear[46]{Emms:2000aa})
    \end{quote}
    My statement above remains true---I don't have sufficient background to parse this solution!
  }
  Still, of interest is not only whether \emph{is} a \fc{0} but also from agent's perspective.
\end{note}

\paragraph{Knowing whether and knowing how}

\begin{note}
  Read the above examples in terms of knowing whether.

  More-or-less direct link between knowing whether and knowing how:

  Know whether \(?\{a,b,c,\dots\}\) know how to answer whether it is the case that \(?\{a,b,c,\dots\}\)
\end{note}

\begin{note}
  Ideas regarding \citeauthor{Ryle:1946tu}'s distinction between knowing \emph{how} and knowing \emph{that}~(Cf.~\citeyear{Ryle:1946tu}).

  Now, I confess my understanding of \citeauthor{Ryle:1946tu}'s distinction is limited --- I have not taken whatever opportunities I have had to read through \citeauthor{Ryle:1946tu}'s work.%
  \footnote{
    Though, I understand enough from passing commentary to note that the idea \emph{I} am perusing here does not, strictly, require that knowledge how and knowledge that are distinct kinds of knowledge.

    Intellectualist and anti-intellectualist views.

    For, granting that knowledge how reduces to knowledge that, it will remain the case that there is an event\dots
    (See~\textcite{Pavese:2022up} for more!)
  }

  Following analogy from~\textcite{Ryle:2009us}:

  \begin{quote}
    Knowing `\emph{if p, then q}' is, \dots rather like being in possession of a railway ticket.
    It is having a licence or warrant to make a journey from London to Oxford.
    (Knowing a variable hypothetical or `law' is like having a season ticket.)
    As a person can have a ticket without actually travelling with it and without ever being in London or getting to Oxford, so a person can have an inference warrant without actually making any inferences and even without ever acquiring the premisses from which to make them.%
    \mbox{ }\hfill\mbox{(\citeyear[250]{Ryle:2009us})}
  \end{quote}

  Continuing~\citeauthor{Ryle:2009us}'s analogy, in the case of \fc{1}:
  What matters is that the agent is currently in possession of the (season) ticket.
\end{note}

\begin{note}
  The relationship between \fc{1} and knowing whether is interesting.
  For, as stated, \fc{1}, just about potential event.
  Plausible paraphrase.

  Whether this goes in both directions.

  So, from knowing whether, generally understood, there's got to be an event.
  However, it is not clear to me that there is a `potential' constraint on the event.

  \cite{Bengson:2011th}.
  \begin{quote}
    \emph{Pi}.
    Louis, a competent mathematician, knows how to find the n\(^{\text{th}}\) numeral, for any numeral \(n\), in the decimal expansion of \(\pi\).
    He knows the algorithm and knows how to apply it in a given case.
    However, because of principled computational limitations, Louis (like all ordinary human beings) is unable to find the \(10^{46}\) numeral in the decimal expansion of \(\pi\).%
    \mbox{ }\hfill\mbox{(\citeyear[170]{Bengson:2011th})}
  \end{quote}

  Conversely, not clear \fc{0} leads to knowing whether.

  Issues here.

  First, factive constraint on knowledge.
  And, concluding need not be factive.

  Though, whether this matters in practice is not clear.
  For, from agent's perspective.

  Second, knowledge is not the right attitude from concluding.
  For example, focus is on deductive cases, but abductive conclusion.
  Or, some doubts about premises.
\end{note}


\begin{note}
  \citeauthor{Barker:1975un} suggests:%
  \footnote{
    \citeauthor{Barker:1975un} also, as far as I can tell, endorses the principles.
  }

  \begin{quote}
    Most theorists of knowledge would, it seems, accept the principles:
    \begin{enumerate}[label=(\Alph*), ref=(\Alph*)]
    \item
      \label{Barker:1975un:A}
      If \emph{S} knows whether \emph{p} and \emph{S} believes that \emph{p}, then \emph{p}.
    \item
      \label{Barker:1975un:B}
      If \emph{S} knows whether \emph{p} and \emph{S} believes that not-\emph{p}, then not-\emph{p}.\newline
      \mbox{ }\hfill\mbox{(\citeyear[281]{Barker:1975un})}
    \end{enumerate}
  \end{quote}
  Though, it seems to me both principles should be rejected.
  For example, consider two agents, \emph{A} and \emph{B} playing chess where each move is timed.
  It's the end game, and \emph{A} believes that \emph{B} has a winning strategy.
  Further, \emph{A} (plausibly) knows whether \emph{B} has a winning strategy.
  For, an observer has determined whether or not \emph{B} has a winning strategy, and \emph{A} is capable of tracing the reasoning of the observer.
  So, if \ref{Barker:1975un:A} holds then \emph{B} has a winning strategy.
  But, the observer knows that \emph{B} \emph{does not} have a winning strategy, and \emph{A}'s belief is mistaken.
\end{note}

\section{\support{2}}
\label{sec:proposition}

\begin{note}
  \begin{proposition}
    \label{prop:fcs-only-if-support}
    For an agent \vAgent{} and proposition-value-premises pairing \(\pvp{\phi}{v}{\Phi}\):
    \begin{enumerate}
    \item[\emph{If}:]
      \begin{enumerate}[label=\alph*., ref=(\alph*.)]
      \item
        \(\pv{\phi}{v}\) from \(\Phi\) is a \fc{0}, from \vAgent{}' perspective.
      \end{enumerate}
    \item[\emph{then}:]
      \begin{enumerate}[label=\alph*., ref=(\alph*.), resume]
      \item \support{2} holds between \(\pv{\phi}{v}\) and \(\Phi\), from \vAgent{}' perspective.
      \end{enumerate}
    \end{enumerate}
  \end{proposition}

  {
    \color{red}
    Why this is somewhat interesting.
  }
  However, before turning to the argument for \autoref{prop:fcs-only-if-support}, it is important to note the limitations of \autoref{prop:fcs-only-if-support} with respect to \issueConstraint{}.
  For, in order to argue against \issueConstraint{}, need some \(\pvp{\psi}{v'}{\Psi}\) such that answers \qWhyV{}.
  Answer \qWhyV{} only with dependence.
  Does not follow from \fc{0} that we get dependence.

  Note, also, qualifications.
  From the agent's perspective.
  Mirrored in both cases.
  Plausible that variant of \autoref{prop:fcs-only-if-support} holds unqualified.
  However, we have said nothing of \support{} independent of agent's perspective.
\end{note}

\begin{note}[Argument]
  Argument is straightforward.
  Possible support, by assumption.
  Contraposition.
  If not support, then no \fc{}.
\end{note}

\paragraph{Potential relations of support}

\begin{note}
  Start with the following proposition.
  \begin{proposition}
    If \fc{}, then potential relation of support.
  \end{proposition}

  Argument is fairly straightforward:
  \begin{argument}
    Suppose \fc{}.
    Then, from agent's perspective, potential event in which concludes.
    Now, consider the potential event.
    The culmination of the event, agent concludes.

    So, from~\autoref{idea:support}, a relation of support holds, from the agent's perspective.

    Therefore, in whatever sense event is potential, relation of support is likewise potential.
  \end{argument}
  From the agent's perspective, there is no difference between witnessed relation of support and potential relation of support.
\end{note}

\begin{note}
  \emph{Potential} relation of support, but it does not follow that there is a relation of support, from the agent's perspective.
\end{note}

\begin{note}
  \begin{proposition}
    Potential relation of support only if relation of support.
  \end{proposition}

  \begin{argument}
    \autoref{idea:support:possible}.
    It is possible for there to be.
    So, we have everything needed.
    Hence, form agent's perspective, relation of support.
  \end{argument}
\end{note}

\newpage

\section{Conclusions, forgone}
\label{sec:fc3-1}

\paragraph{Premises and past conclusions}

\begin{note}[Premises]
  So, as we have seen with testimony, status of a premises blocks a \requ{}.

  Whether the same may hold for this problem.

  It's the case that, part of agent's present epistemic state that they would conclude.

  Problem is, if attempt and fail, then this premise does nothing.
  Their present epistemic state develops into a dead-end.
\end{note}

\begin{note}[Note!]
  This doesn't hold in general, for all premises.

  In particular, premise is past conclusion.

  Consider cases of being somewhat impaired, e.g., via exhaustion.
  Indeed, exhaustion is interesting.
  Basic consistency checks.
  Should be the case that conclude A, but just concluded \emph{not}-A, or something like this\dots

  Denying that past continues to secure in all instances.
  So, just need the potential to revise perspective on previous conclusion.
\end{note}

\section{\fc{3} and support}
\label{cha:fcs:sec:fc3-support}

\begin{note}
  \begin{proposition}
    For any path, present epistemic state determines availability of path.
  \end{proposition}

  Start.
  Then, continue.
  Started from \(\Phi\), so will conclude.
  Hence, no matter choice made, must have taken the possibility of this choice into account.
  So, it must be the case that determined.

  Hence, if witness, then via some path.

  So, witnessing predetermined path.
  Any instances of concluding by witnessing reduces to witnessing predetermined path.

  Witnessing may provide information about path, but witnessing doesn't contribute given a \requ{}.

  For any X from W,
  present determines whether or not X from agent's point of view, then forgone conclusion.

  In other words, agent's present epistemic state determines.
  Agent may need to witness to figure out how determined, but witnessing does not influence.
\end{note}

\begin{note}[Two worries]
  Two worries.

  First, that even though \fc{0}, the agent would not conclude.
  Either because \(\Phi\) is unavailable, or because no potential witnessing event.
  So, can't remove \fc{0} from account of why.

  However, then \fc{0} does not support.

  If grant that \fc{0} supports, then this seems to work out.
  Further, if require existence, then things that support get very messy.
  Dopeganger cases.
  Reason is I saw A, but it wasn't A, appealing to something that doesn't exist.
  Various other cases like this.

  Difference.
  In these cases, have premise, thing is that the truth value is distinct.
  Here, possibly no premise.

  Well, this is different.
  However, I don't think this is sufficient to reject the idea.
  Just because this distinction doesn't arise in the case of witnessing doesn't really do much.

  Look, a `bad' premise offers no more support for the agent than no premise.

  Second, need \emph{that} \fc{0}.
  However, the point is that this is about the agent's present epistemic state.
  \emph{Without} \fc{0}, the agent would reason.
  This is just the key point reiterated.
  Know whether, \fc{0} just adds information about which.
\end{note}


\section{\fc{3} and ability}
\label{sec:fc3-ability}

\begin{note}
  If general ability, then \fc{1}.

  The basic idea is that, general ability to reason, specific instances of general ability, so these end up as \fc{1}.

  However, the link between these two is somewhat delicate.
  \fc{3} concern potential witnessing events.

  So, it's not the case that ability in general.
  For, may have ability with no opportunity.

  For, if not \fc{}, then no ability.

  Might think this fails to hold.
  For ability, possibility to fail, though preserve the ability.
  However, agent's own reasoning.
  While it's true the agent \emph{may} fail, the link is more general.
  It's not forward looking.
  Rather, there is some potential event.

  Now, the agent may be wrong.
  Fail to have the ability, or something about how things are doesn't line up, and so do not really have the option.
  Still, from agent's perspective.
\end{note}

\section{Scrap}

\begin{note}
  {
    \color{red}
    Move to \requ{} now things have been switched around.
  }
  \fc{} is broader than \requ{}.

  Consider calculator type cases.
\end{note}


%%% Local Variables:
%%% mode: latex
%%% TeX-master: "master"
%%% End:
