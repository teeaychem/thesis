\chapter{\fc{3}}
\label{cha:fcs}

\nocite{Ryle:1946tu}

\begin{note}
  This chapter introduces the core idea of this document:
  Some \prop{0}-\val{0} pair \(\pv{\phi}{v}\) being a \emph{\fc{}} from some \pool{0} \(\Psi\).

  Paraphrased, \(\pv{\phi}{v}\) is a \fc{} from \(\Phi\) for an agent just in case the agent has an option to conclude \(\pv{\psi}{v'}\) from \(\Psi\).

  The idea of a \fc{0} has two important roles:
  \begin{enumerate}
  \item
    Clarify the way in which counterexamples to \issueInclusion{} are possible.
  \item
    Identifying counterexamples to \issueInclusion{}.
  \end{enumerate}

  Specifically, \fc{1} for characterising \ros{}.
  \ros{} answer \qWhy{}.

  In this respect, \fc{1} are compatible with \issueInclusion{}.
\end{note}

\section{Definition}
\label{cha:fcs:def}

\begin{note}[\fc{2} definition]
  We define a \emph{\fc{0}} as follows:

  \begin{definition}[\fc{3}]
    \label{def:fc}
    \begin{itemize}
    \item
      \(\pv{\psi}{v'}\) is a \emph{\fc{0}} from \(\Psi\), for \vAgent{}.
    \end{itemize}

    \emph{If and only if}

    \begin{itemize}
    \item
      There is some action \(a\) \vAgent{} may immediately perform such that both \ref{def:fc:act} and \ref{def:fc:result} are true:
      \begin{enumerate}[label=\alph*., ref=(\alph*), series=fcCounter]
      \item
        \label{def:fc:act}
        The event \(e\) in which \vAgent{} does \(a\) is an event in which \vAgent{} is concluding \(\pv{\phi}{v}\) from \(\Phi\).
      \item
        \label{def:fc:result}
        There is some development \(e'\) of \(e\) in which \vAgent{} concludes \(\pv{\phi}{v}\) from \(\Phi\) without novel information.
      \end{enumerate}
    \end{itemize}
    \vspace{-\baselineskip}
  \end{definition}

  \noindent%
  The idea of a \fc{} is to capture a possible conclusion via the sense of possibility required for the progressive, and what this entails.

  % \begin{proposition}[\fc{3} and \pevent{1}]%
  %   \label{prop:fc:pevent}%
  %   \vspace{-\baselineskip}
  %   \begin{itenum}
  %   \item[\emph{If}:]
  %     \(\pv{\psi}{v'}\) from \(\Psi\) is a \emph{\fc{0}} for \vAgent{}.
  %   \item[\emph{Then}:]
  %     There is a \pevent{} in which \vAgent{} concludes \(\pv{\psi}{v'}\) from \(\Psi\).
  %   \end{itenum}
  %   \vspace{-\baselineskip}
  % \end{proposition}

  % \begin{argument}{prop:fc:pevent}
  %   By definition of a \pevent{} in which \vAgent{} \(\alpha\)'s (\peventpage{}) it must be the case there is some action \(a\) such that:
  %   \begin{enumerate}[noitemsep]
  %   \item[\ref{def:PE:action}]
  %     \vAgent{} may easily and immediately do \(a\).
  %   \item[\ref{def:PE:prog}]
  %     \(\text{Prog}(e, \alpha)\) is true of the event \(e\) in which \vAgent{} does \(a\).
  %   \end{enumerate}
  %   The definition of a \fc{} simply specifies \(\alpha\) as `concludes \(\pv{\psi}{v'}\) from \(\Psi\)', and adds a qualification.
  % \end{argument}

  The added qualification is important.
  Intuitively, \fc{1} follow from information the agent has.
  And, by performing an action an agent may obtain novel information which allows they to conclude some proposition has some value.
\end{note}

\begin{note}
  A handful of \illu{1} follow below.
  Still, though `\fc{0}' is a technical term, it is intended to associate with a sense of the common term `foregone conclusion'.
  In this sense, a foregone conclusion is an expected result of reasoning.%
  \footnote{
    For example, consider the following passage:

    \begin{quote}
      [\dots] Russell's evaluation of such sentences as false is predetermined by his existence presuppositional semantics for the ‘existential' quantifier, and by the fact that his logic permits no alternative means of considering the semantic status of sentences ostensibly containing proper names for nonexistent objects.
      This makes it an altogether philosophically foregone conclusion that sentences like ‘Pegasus is winged,' which many logicians would otherwise consider to be true propositions of mythology, are false.\newline
      \mbox{ }\hfill\mbox{(\cite[6]{Jacquette:2002up})}
    \end{quote}

    \noindent \citeauthor{Jacquette:2002up} is discussing what follows from~\citeauthor{Russell:1905aa}'s analysis of definite descriptions.
    Specifically, from \citeauthor{Russell:1905aa}'s analysis it follows \propI{Pegasus is winged} is \valI{False}.
    (There are no four-legged winged mammals, etc.)
  }
  This sense of the term `foregone conclusion' contrasts against a sense with which a conclusion which has been settled in advance of reasoning.%
  \footnote{
    For example:
    \begin{quote}
      When can a Bayesian select an hypothesis \emph{H} and design an experiment (or a sequence of experiments) to make certain that, given the experimental outcome(s), the posterior probability of \emph{H} will be greater than its prior probably?
      We discuss an elementary result that establishes sufficient conditions under which this reasoning to a foregone conclusion cannot occur.%
      \mbox{ }\hfill\mbox{(\cite[1228]{Kadane:1996vu})}
    \end{quote}
  }
  And, a sense in which a forgone conclusion is some unavoidable state of affairs.%
  \footnote{
    For example:
    \begin{quote}
      [どうぜ][\dots] Expresses an attitude of resignation or carelessness on the part of the speaker, in the sense that regardless of what s/he does, the conclusion or outcome is foregone and cannot be changed by the will or effort of an individual.%
      \mbox{ }\hfill\mbox{(\cite[332--333]{kurufushamashii:2015un})}
    \end{quote}
    See also \citeauthor{Grice:1957vg}'s discussion of intention recognition (\citeyear[385]{Grice:1957vg}/\citeyear[219]{Grice:1989uf}), and \citeauthor{Machover:1996vu}'s preface of their approach to the G\"{o}del-Rosser First Incompleteness Theorem (\citeyear[viii]{Machover:1996vu}).
  }

  Still, the similarities between the non-technical `foregone conclusion' and technical `\fc{}' are only for intuition.
  We do not present our definition of a \fc{} as an analysis of the non-technical term.%
  \footnote{
    \label{fn:fc-ability}
    In particular, I suspect `\fc{3}' is close to an ability modal while `foregone conclusion' is close to a compulsion modal.
    (See (\cite{Mandelkern:2017aa}) for discussion).
    Indeed, our definition of a \fc{} is close to an act conditional analysis of ability.
    However, there is no immediate parallel.
    We return to this observation in \autoref{cha:sec:fcs-def:ability}.
  }
\end{note}

\section{Illustrations}
\label{cha:fcs:illu}

\begin{note}
  We begin with a few \scen{1} in which \(\pv{\psi}{v'}\) from \(\Psi\) (plausibly) \emph{is} a \fc{}.
  Then, we consider some \scen{1} in which \(\pv{\psi}{v'}\) from \(\Psi\) is \emph{not} (clearly) a \fc{}.
\end{note}

\subsection{\scen{3} where some \prop{0}-\val{0} pair is a \fc{}}
\label{cha:fcs:illu:yes}

\begin{note}[Chess I]
  \begin{scenario}[\citeauthor{Emms:2000aa}' Puzzle 113 (\citeyear[33]{Emms:2000aa})]%
    \label{illu:fc:chess:I}%
    \mbox{ }\hfill%
    \begin{adjustbox}{minipage=\linewidth,scale=.8}
      \centering
      \newchessgame[
      setwhite={pa2,pb2,pc2,pd3,pf2,pg3,ra1,re1,bd4,kg1,qe5},
      addblack={ra8,pa7,ba6,pb5,rc8,pd5,pf7,kg8,qg4,ph7,ph4},
      ]%
      \setchessboard{showmover=false}%
      \chessboard
    \end{adjustbox}%
    \label{fig:chess:easy}%
    \hfill\mbox{ }

    \begin{center}
      Is possible for White to checkmate in a single move?
    \end{center}
    \vspace{-\baselineskip}
  \end{scenario}
\end{note}

\begin{note}
  The conclusion of interest:%
  \footnote{
    Or: \pv{\propI{It is possible for White to checkmate in a single move}}{\valI{True}}
  }

  \begin{enumerate}[label=C\thescenarioCounter., ref=(C\thescenarioCounter)]
  \item
    \label{illu:fc:chess:I:c}
    \pv{\propI{White checkmates in a single move}}{\valI{Possible}}
  \end{enumerate}

  \noindent Whether or not \ref{illu:fc:chess:I:c} is a \fc{} depends on the agent.
  For the present \illu{}, suppose the agent has a basic understanding of chess and will only settle whether it is possible for White to checkmate in a single move by applying that understanding.
  Further, strategy is exhaustive search.
\end{note}

\begin{note}
  To establish \ref{illu:fc:chess:I:c} is a \fc{}, we need some action for which both clauses of \autoref{def:fc} are satisfied.

  Consider the action described by:

  \begin{center}
    `Begin an attempt to solve \citeauthor{Emms:2000aa}' Puzzle 113'.
  \end{center}

  \noindent We walk through each clause in turn.
  For ease we omit the relevant \pool{}.

  \begin{itemize}[leftmargin=*]
  \item
    Clause~\ref{def:fc:act} is satisfied.

    The action is something the agent may easily and immediately do.

    In almost any state of affairs one may attempt to do almost any thing.
    For example, you may being an attempt to square the circle.
    Failure was assured, but you made an attempt.

    The action is restricted to begin in order to ease reasoning about Clause~\ref{def:fc:result}.
  \item
    Clause~\ref{def:fc:result} is satisfied.

    We break down the argument into three separate components.

    \begin{itemize}
    \item
      There is a possible event in which the agent concludes \ref{illu:fc:chess:I:c}.

      For, given the rules of chess it is possible for White to checkmate in a single move.
      The agent has a basic understanding of chess.
      And, the agent only needs to consider the appropriate move and to verity the move results in checkmate to conclude \ref{illu:fc:chess:I:c}.

    \item
      The agent is concluding \ref{illu:fc:chess:I:c} when the action is done.

      As the action is to begin an attempt to solve Puzzle 113, what follows is the attempt.
      To see the agent is concluding \ref{illu:fc:chess:I:c} when they attempt to do so, consider two (motivated) conditionals:

      \begin{enumerate}[label=\arabic*., ref=(\arabic*)]
      \item
        \label{illu:fc:chess:I:cond:1}
        If agent picks \wmove{Qh8}, then agent is concluding \ref{illu:fc:chess:I:c}.%
        \smallskip

        By assumption, the agent has a basic understanding of chess and is motivated.
        And, as the agent needs to verify \wmove{Qh8} in checkmate, the agent will.
      \item
        \label{illu:fc:chess:I:cond:2}
        If agent picks a move other than \wmove{Qh8} then after some reasoning the agent picks a novel move.%
        \smallskip

        By parallel reasoning.

        The agent needs only verify the move other than \wmove{Qh8} fails to result in checkmate, and then pick some other move.
        The agent will verify, given their understanding of chess.
        And, the agent will pick some novel move as their strategy is exhaustive search.
      \end{enumerate}

      Now, whichever move the agent picks, either \ref{illu:fc:chess:I:cond:1} or \ref{illu:fc:chess:I:cond:2} is true.
      And, as there are finitely many moves for the agent to pick, \ref{illu:fc:chess:I:cond:1} will (eventually be true).
      Hence, as an event in which the agent concludes \ref{illu:fc:chess:I:c} is in progress when \ref{illu:fc:chess:I:cond:1} is true, an event in which the agent concludes \ref{illu:fc:chess:I:c} is (also) in progress when \ref{illu:fc:chess:I:cond:2} is true.

    \item
      The agent does not use any novel information obtained by beginning an attempt.

      For, this action does not provide the agent  with any novel information.
      The agent has looked at the puzzle and has a basic understanding of the rules of chess.
      The agent does not appeal to any information that they do not already possess, and any information obtained follows.
    \end{itemize}
  \end{itemize}

  This argument assumes a particular strategy.
  This need not be the case, but alternative strategies are difficult to describe.
  If you have a basic understanding of chess then I suggest you convince yourself the conclusion is a \fc{} by attempting the puzzle.
  You need only conclude \ref{illu:fc:chess:I:c} and verify that you were concluding \ref{illu:fc:chess:I:c} after you began (and did not make use of any novel information).
\end{note}

\begin{note}
  In broad structure, the idea with \autoref{illu:fc:chess:I} is some effective method, sufficient information to apply method, and an opportunity to apply.

  For example, consider arithmetic.
  I expect the truth of \(13 \cdot 4 = 52\), \(96 \div 4 = 24\), and \(23 \cdot 15 = 345\) are \fc{1}.
  Likewise, if you have basic understanding of propositional logic, then the validity of various theorems are \fc{1}.
  And, if you enjoy Sudoku puzzles then, so long as you have the puzzle and sufficient time to spare, the solution to any puzzle is a \fc{}.
\end{note}



\begin{note}
  \begin{scenario}[\cite[9]{Dudeney:1995aa}]
    \label{scen:fc:chick}%
    Three chickens and one duck sold for as much as two geese; one chicken, two ducks, and three geese were sold together for \$25.00.
    What was the price of each bird in an exact number of dollars?
  \end{scenario}

  \begin{enumerate}[label=C\thescenarioCounter., ref=(C\thescenarioCounter)]
  \item
    \label{scen:fc:chick:c}
    \pv{\propI{The price of a chicken was \$2.00, for a duck \$4.00, and for a goose \$5.00}}{\valI{True}}
  \end{enumerate}
  %
  Solution to the puzzle follows from a grasp of basic algebra, capacity to recast natural language problems as algebraic problems, and some \emph{je ne sais quoi}.
\end{note}

\begin{note}
  \begin{scenario}[Fibonacci numbers]%
    \label{scen:fc:fib}%
    The \(i\)th number in the Fibonacci sequence is given by \(f(i)\), where:

    \[
      \begin{array}{rcl}
        f(0) &=& 0 \\
        f(1) &=& 1 \\
        f(n + 2) &=& f(n) + f(n + 1)
      \end{array}
    \]
  \end{scenario}

  \begin{enumerate}[label=C\thescenarioCounter., ref=(C\thescenarioCounter)]
  \item
    \label{scen:fc:fib:c}
    \pv{\propI{The sixth number in the Fibonacci sequence is 5}}{\valI{True}}
  \end{enumerate}
  %
  I leave you to verify you may immediately perform an action in which you are concluding \pv{\propI{The six number in the Fibonacci sequence is 5}}{\valI{True}} without novel information.
\end{note}


% \begin{note}[Non-deductive \illu{1}]
%   \autoref{illu:fc:chess:I} and parallel examples are motivated by an agent's grasp on some effective method to solve a type of problem.
%   However, \(\pv{\psi}{v'}\) being a \fc{} from \(\Psi\) does not require an effective method.
%   It need only be the case that the agent is concluding \(\pv{\psi}{v'}\) from \(\Psi\) after an action is done.
%   Consider, the following \scen{0}:

%   \begin{scenario}[Sunny days]%
%     \label{illu:fc:sunny}%
%     It's mid summer day in the Bay Area.
%   \end{scenario}

%   \noindent For me, the following conclusion is a \fc{} from some \pool{}:

%   \begin{enumerate}[label=C\thescenarioCounter., ref=(C\thescenarioCounter)]
%   \item
%     \label{illu:fc:sunny:c}
%     \pv{\propI{It will rain tomorrow}}{\valI{False}}
%   \end{enumerate}

%   \noindent%
%   There is no effective method for me to determine whether it will rain tomorrow, and I recognise there may be rain tomorrow.
%   Still, I am sufficiently committed to some uniformity principle.%
%   \footnote{
%     Cf. (\cite[70]{Hempel:1965aa}),~(\cite{Henderson:2022aa}).
%   }
%   And, that the principle together with past experience, ensure that if I consider whether it will rain tomorrow, I conclude it will not rain.%
% \end{note}



\subsection{\scen{3} where a \prop{0}-\val{0} pair is not a \fc{}}
\label{cha:fcs:illu:no}

\begin{note}
  \(\pv{\psi}{v'}\) may fail to be a \fc{} from \(\Phi\) in two basic ways:

  \begin{enumerate}[label=\alph*., ref=(\alph*), noitemsep]
  \item
    There is no action the agent may easily and immediately do.
  \item
    The agent may fail to be concluding \(\pv{\psi}{v'}\) from \(\Psi\), for any easy and immediate action.
  \end{enumerate}

  We focus on the latter type of case, which may be further sub-divided:

  \begin{itemize}[noitemsep]
  \item
    The agent does not have the capacity to conclude.
  \item
    The agent has the capacity, but would conclude something which conflicts with \(\pv{\psi}{v'}\).
  \end{itemize}

  The following \illu{} take each sub-case, respectively.
\end{note}

\paragraph*{Lack of capacity}



\begin{note}
  \begin{scenario}[Clocks]
    These examples contrast with conclusion the clock displaying incorrect time.
  Need an independent source.
  \end{scenario}
\end{note}

\begin{note}
  In general, it may be possible for the agent to conclude, but the possibility is weaker than the sense of possibility captured by the truth of the progressive.

  For further examples, consider uninteresting formal derivations or difficult arithmetic.
  One may calculate \(4^{4!}\), or \dots\space use a calculator.
\end{note}

\paragraph*{Novel information}

\begin{note}
  \begin{scenario}
    A miller was accustomed to take as toll one-tenth of the flour that he ground for his customers.
    How much did he grind for a man who had just one bushel after the toll had been taken?%
    \mbox{ }\hfill\mbox{(\cite[50]{Dudeney:1995aa})}
  \end{scenario}

  Personally, I have no idea bushel is.
  And, if you do know, then difference between British (equivalent to 36.4 litres) and US (equivalent to 35.2 litres) bushel.
\end{note}

\paragraph*{Conflict}

\begin{note}[A copper kettle]
  A further \illu{0} builds on a story as told by~\citeauthor{Freud:1960wx}.

  \begin{scenario}[A copper kettle]
    \label{illu:kettle}
    \mbox{ }
    \vspace{-\baselineskip}
    \begin{quote}
      `A.\ borrowed a copper kettle from B.\ and after he had returned it was sued by B.\ because the kettle now had a big hole in it which made it unusable.
      His defence was:
      ``First, I never borrowed a kettle from B.\ at all;
      secondly, the kettle had a hole in it already when I got it from him;
      and thirdly, I gave him back the kettle undamaged.%
      '''%
      \mbox{ }\hfill\mbox{(\citeyear[62]{Freud:1960wx})}
    \end{quote}
    An agent listens to A.'s defence.
  \end{scenario}

  For, A.\ has provided testimony only if what A.\ has said is true.
  And, what A.\ has said is true only if the three points of A.'s defence are jointly consistent.
  Putting these observations together, we have the following conditional:

  \begin{itemize}
  \item
    A.\ has provided testimony \emph{only if} if the three points of A.'s defence are jointly consistent.
  \end{itemize}
\end{note}

\begin{note}
  Structurally, \autoref{ill:fcs:kw} is no different from an formula or equation which does not hold.
  For example, \((\phi \rightarrow (\psi \rightarrow \phi)) \rightarrow \phi\), or \(4 \times 3 = 14\).
\end{note}

\section{Notes}

\begin{note}
  \color{red}
  Here, ability versus compulsion.
  Nothing really hangs on this.
\end{note}

\begin{note}
  % The definition of a \fc{} may be strengthened in various ways.
  % For example:

  % \begin{enumerate}[label=\alph*., ref=(\alph*), resume*=fcCounter]
  % \item
  %   \label{def:fc:alt:c}
  %   For any proposition \(\chi\), value \(v''\), and action \(b\) such that \vAgent{} is concluding \(\pv{\chi'}{v''}\) from \(X\) after \(b\) is done:
  %   \begin{itemize}
  %   \item
  %     Either~\ref{def:fc:extra:1} or~\ref{def:fc:extra:2} is the case:
  %     \begin{enumerate}[label=\arabic*., ref=\arabic*]
  %     \item
  %       \label{def:fc:extra:1}
  %       \(\chi\) is \(\psi\) and \(v''\) is \(v'\).
  %     \item
  %       \label{def:fc:extra:2}
  %       Throughout event in which \vAgent{} concludes \(\pv{\chi}{v''}\) from \(X\), there is some action \(c\) such that:
  %       \begin{itemize}
  %       \item
  %         \vAgent{} may easily and immediately do \(c\).
  %       \item
  %         \vAgent{} is concluding \(\pv{\psi}{v'}\) from \(\Psi\) when \vAgent{} does \(c\).
  %         \begin{itemize}
  %         \item
  %           Without use of any novel inf.\ obtained by doing \(a\) and \(b\).
  %         \end{itemize}
  %       \end{itemize}
  %     \end{enumerate}
  %   \end{itemize}
  % \end{enumerate}

  % \noindent Paraphrased:

  % \begin{itemize}[noitemsep]
  % \item
  %   Sub-clause~\ref{def:fc:extra:1} states \vAgent{} is concluding \(\pv{\psi}{v'}\) from \(\Psi\) after \vAgent{} does \(b\).
  % \item
  %   Sub-clause~\ref{def:fc:extra:2} states \vAgent{} \(\pv{\psi}{v'}\) from \(\Psi\) remains a \fc{} after \vAgent{} does \(b\).
  % \end{itemize}

  % \noindent This is a much stronger condition.
  % Not only is it the case that concluding, but remains the case that concluding is available for any other conclusion.

  % This rules out pairwise conflicts.%
  % \footnote{
  %   Though, does not rule out more.
  % }
  We do not add any additional clauses as this is all we need.
  Still, additional clauses may be added.

  No restrictions placed on what an agent may conclude.
  So, no restrictions placed on \fc{1}.

  If \fc{} then enough to conclude.
  And, if the may conclude conflicting things, so be it.
  This is no different from conclusions.

  \fc{3}, like conclusions, are descriptive.
  Do not include considerations about `good' or `bad' conclusions.
\end{note}

\begin{note}
  Still, if you prefer something stronger, that's okay.

  Indeed, probably something a little stronger than Clause~\autoref{def:fc:alt:c}.
  For, an agent may conclude two things and then fail.
  Perhaps, in the words of Achilles, a \fc{} should be such that \textquote{you ca'n't help yourself} (\cite[280]{Carroll:1895uj}).
  Though, what this amounts to\dots

  There's nothing in what follows that strictly requires that definition of a \fc{} is as general as it is.
  Additional restrictions limit instances of \fc{1}.
  And, use is existence of \fc{1}.
  So long as \fc{1} with no \wit{} exist, good.

  If do restrict, though, some things will need to be restructured.
\end{note}

\section*{Summary}

%%% Local Variables:
%%% mode: latex
%%% TeX-master: "master"
%%% End:


  %   \begin{itemize}
  %   \item
  %     The agent has a good understanding of some formal proof system.
  %     For example, some Fitch-style system.
  %   \item
  %     The agent has a good understanding of some method to construct semantic proofs.
  %     For example, by constructing truth tables, or reasoning about valuation functions.
  %   \item
  %     The agent understands the proof system is sound.
  %     That is to say, the agent understands there exists a proof of some sentence \(A\) \emph{only if} \(A\) is true given an arbitrary valuation.
  %   \end{itemize}
  %   The agent constructs the following proof:
  %   \begin{center}
  %     \begin{fitch}
  %       \phantlabel{illu:sPF:1}\fa \fh (P \rightarrow Q) \rightarrow P \\
  %       \phantlabel{illu:sPF:2}\fa \fa \fh  \lnot P \\
  %       \phantlabel{illu:sPF:3}\fa \fa \fa \fh  P \\
  %       \phantlabel{illu:sPF:4}\fa \fa \fa \fa  \bot & \(\bot\)\textbf{Intro:}\hyperref[illu:sPF:2]{2},\hyperref[illu:sPF:3]{3}\\
  %       \phantlabel{illu:sPF:5}\fa \fa \fa \fa  Q & \(\bot\)\textbf{Elim:}\hyperref[illu:sPF:4]{4}\\
  %       \phantlabel{illu:sPF:6}\fa \fa \fa P \rightarrow Q & \(\rightarrow\)\textbf{Intro:}\hyperref[illu:sPF:3]{3}--\hyperref[illu:sPF:5]{5} \\
  %       \phantlabel{illu:sPF:7}\fa \fa \fa P & \(\rightarrow\)\textbf{Elim:}\hyperref[illu:sPF:1]{1},\hyperref[illu:sPF:6]{6}\\
  %       \phantlabel{illu:sPF:8}\fa \fa \fa \bot & \(\bot\)\textbf{Intro:}\hyperref[illu:sPF:2]{2},\hyperref[illu:sPF:7]{7}\\
  %       \phantlabel{illu:sPF:9}\fa \fa \lnot\lnot P & \(\lnot\)\textbf{Intro:}\hyperref[illu:sPF:2]{2},\hyperref[illu:sPF:8]{8}\\
  %       \phantlabel{illu:sPF:10}\fa \fa P & \(\lnot\)\textbf{Elim:}\hyperref[illu:sPF:9]{9}\\
  %       \phantlabel{illu:sPF:11}\fa ((P \rightarrow Q) \rightarrow P) \rightarrow P & \(\rightarrow\)\textbf{Intro:}\hyperref[illu:sPF:1]{1}--\hyperref[illu:sPF:10]{10} \\
  %     \end{fitch}
  %   \end{center}
  %   \vspace{-\baselineskip}

% \begin{note}
%   Still, the conclusion need not be easy.

%   \begin{illustration}
%     \[\frac{(3 + \sqrt{3})^{2} + \sqrt{6}^{2} - (2\sqrt{3})^{2}}{2(3 + \sqrt{3})\sqrt{6}} = \frac{1}{\sqrt{2}}\]
%   \end{illustration}

%   I suspect this is a \fc{}.
%   Might need to do some work to recall principles, but it's okay.
% \end{note}


% \begin{note}[ML II]
%   \begin{illustration}[Modal logic II]
%     \label{illu:fc:ML2}
%     The modal system \(\mathbf{GL} = \mathbf{K} + \Box(\Box p \rightarrow p) \rightarrow \Box p\) is weakly complete with respect to the class of finite strict partial orders (that is, the class of finite irreflexive transitive frames).
%   \end{illustration}

%   \autoref{illu:fc:ML2} is similar in structure to \autoref{illu:fc:logic:CR}.
%   Indeed, both proofs involve constructing a canonical model.
%   The key distinguishing feature of \autoref{illu:fc:ML2}, however, is the difficulty of establishing the canonical model has the desired properties.
%   In particular, the general method I keep in mind for proving the relevant result requires a syntactic proof that \(\vdash_{\mathbf{GL}} \Box p \rightarrow \Box \Box p\).
%   And, as I have failed to recall the relevant syntactic on sufficient occasion, I do not consider the result a \fc{0} from my understanding of modal logic.

%   Hence, the result (plausibly) fails to be a \fc{} from my understanding of modal logic because there is no guarantee that I would provide a proof if I set out to do so.%
%   \footnote{
%     On the other hand, I have completed the relevant proof a sufficient number of times.
%     So, the result is a \fc{0} from whatever premises are associated with my memory.
%   }
% \end{note}

% \begin{note}
%   \illu{3} may be obtained by taking a proposition-value pairing which conflicts with \fc{}.
%   The following is \emph{not} a \fc{}.%
%   \footnote{
%     However, I suspect the following equation is a \fc{}:
%     \[\frac{(3 + \sqrt{3})^{2} + \sqrt{6}^{2} - (2\sqrt{3})^{2}}{2(3 + \sqrt{3})\sqrt{6}} = \frac{1}{\sqrt{2}}\]
%   }

%   \begin{illustration}
%     \[\frac{(3 + \sqrt{3})^{2} + \sqrt{6}^{2} - (2\sqrt{3})^{2}}{2(3 + \sqrt{3})\sqrt{6}} = \frac{1}{\sqrt{3}}\]
%   \end{illustration}
%   For, the equation does not hold.
% \end{note}

% \begin{note}
%   This is no different from conclusion that it's 13:48 by looking at the clock.
%   I recognise there is a possibility that the clock is broken.
%   However, nothing to suggest it is.

%   This highlights qualification.
%   Without it, then so long as wearing watch, time is a \fc{}.
%   For, look at watch.
% \end{note}

%%% Local Variables:
%%% mode: latex
%%% TeX-master: "master"
%%% TeX-engine: luatex
%%% End:
