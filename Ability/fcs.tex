\chapter{\fc{3}}
\label{cha:fcs}

\nocite{Ryle:1946tu}

\begin{note}
  This chapter introduces the idea of some proposition-value pairing \(\pv{\psi}{v'}\) being a \emph{\fc{}} from some pool of premises \(\Psi\) (for an agent).

  Intuitively, \(\pv{\psi}{v'}\) being a \fc{} from \(\Psi\) means an agent has the option to conclude \(\pv{\psi}{v'}\) from \(\Psi\).
\end{note}

\begin{note}
  Our main goal is a recipe to construct counterexamples to \issueConstraint{}.
  \fc{3} are a key ingredient.
  In particular, we argue the following conditional is true:

  \begin{itemize}
  \item
    If \(\pv{\psi}{v'}\) is a \fc{} from \(\Psi\), then a \ros{} holds between \(\pv{\psi}{v'}\) and \(\Psi\).
  \end{itemize}

  And, as an agent need not have concluded \(\pv{\psi}{v'}\) from \(\Psi\), it may be the case that a \ros{} holds between \(\pv{\psi}{v'}\) and \(\Psi\) while the agent does not a \wit{} for the \ros{}.

  Hence, \emph{if} the \ros{} between \(\pv{\psi}{v'}\) and \(\Psi\) answers \qWhyV{}, \issueConstraint{} fails to hold in general.
  However, this `if' is by no means straightforward, and does not follow from the idea of a \fc{} alone.
  An additional key idea is introduced in \autoref{cha:requs}, and a key motivating idea is introduced in \autoref{cha:typical}.
\end{note}

\begin{note}
  The chapter is divided as follows:
  \begin{TOCEnum}
  \item
    \TOCLine{cha:fcs:def}

    Definition and \illu{1} of \fc{0}.
  \item
    \TOCLine{cha:fcs:support}

    Link between \fc{1} and \ros{1}.
  \end{TOCEnum}
\end{note}


\section{\fc{3}}
\label{cha:fcs:def}

\begin{note}[\fc{2} definition]
  We define a \emph{\fc{0}} as follows:

  \begin{definition}[\fc{3}]
    \label{def:fc}
    \cenLine{
      \begin{itemize*}[noitemsep, label=\(\circ\)]
      \item
        Agent: \vAgent{}
      \item
        Proposition: \(\phi\)
      \item
        Value: \(v\)
      \item
        \pool{0}: \(\Phi\)
      \item
        \mbox{ }
      \end{itemize*}
    }

    \begin{itemize}
    \item
      \(\pv{\phi}{v}\) from \(\Phi\) is a \emph{\fc{0}} for \vAgent{}.
    \end{itemize}

    \emph{If and only if}

    \begin{itemize}
    \item
      There is some action \(a\) such that both \ref{def:fc:act} and \ref{def:fc:result} are true:
      \begin{enumerate}[label=\alph*., ref=(\alph*), series=fcCounter]
      \item
        \label{def:fc:act}
        \vAgent{} may easily and immediately do \(a\).
      \item
        \label{def:fc:result}
        \vAgent{} is concluding \(\pv{\phi}{v}\) from \(\Phi\), after \(a\) is done, without appeal to any novel information obtained by doing \(a\).
      \end{enumerate}
    \end{itemize}
    \vspace{-\baselineskip}
  \end{definition}

  The definition of a \fc{} parallels the definition of a \pevent{} (\autoref{def:potenital-event}, \autopageref{def:potenital-event}).

  \begin{proposition}[\pevent{2}]%
    \label{prop:fc:pevent}%
    There is a \pevent{} in which agent concludes.
  \end{proposition}

  \begin{argument}{prop:fc:pevent}
    By definition.
  \end{argument}

  So, the basic idea, it is possible for the agent to conclude.
  Where, the sense of possibility is given in terms of the progressive.
  Key is \assuPP{}.

  Qualification is that the role of the action is only the progressive.
\end{note}

\begin{note}
  A handful of \illu{1} in \autoref{cha:fcs:illu}.
  Consider simple things.
  For example, summation of relatively small terms, or figuring out what the date is.

  Not ideal agents.
  In the terminology of \citeauthor{Simon:1957aa} (\citeyear{Simon:1957aa}) `bounded'.
  (See in particular \citeyear[164]{Simon:1972aa})
  %
  % \footnote{
  %   \nocite{Simon:1997aa}
  %   \textquote{[R]ationality can be bounded by assuming complexity in the cost function or other environmental constraints so great as to prevent the actor from calculating the best course of action} (\citeyear[164]{Simon:1972aa}).
  % }
  Some available conclusions have not been concluded.
\end{note}

\begin{note}
  Respect in which \fc{} is weak.
  Existential.
  However, concluding.
  This is somewhat strong.

  For example, action, start reading.
  However, not reading in full.
  Likewise, open book but not reading in full.
  For, though short I don't enjoy.
\end{note}


\begin{note}
  Though `\fc{0}' is a technical term, it is intended to correspond to a sense of the common term `foregone conclusion'.
  In this sense, a foregone conclusion is an inevitable result of reasoning.
  For example, consider the following passage from~\citeauthor{Machover:1996vu}'s~\citetitle{Machover:1996vu}:

  \begin{quote}
    I have omitted its proof, but added a detailed analysis of the meaning of the lemma and the reason why its proof works. When this is understood, the proof itself becomes a mere technicality, almost a foregone conclusion.\newline
    \mbox{ }\hfill\mbox{(\citeyear[viii]{Machover:1996vu})}
  \end{quote}

  \citeauthor{Machover:1996vu} is discussing a proof, and whether or not it is inevitable that one would complete the proof (conclude that the relevant theorem is true) after understanding the lemma and why it works.
  There is no relevant sense in which the truth of the theorem has been settled in advance of reasoning.
  Though, as the proof is somewhat difficult,~\citeauthor{Machover:1996vu} only states the proof is `almost' a foregone conclusion.%
  \footnote{
    The proof is question is of the G\"{o}del-Rosser First Incompleteness Theorem.
    (\citeyear[Cf.][226]{Machover:1996vu})
  }%
  \(^{,}\)%
  \footnote{
    For a similar example without qualification, consider:
    \begin{quote}
      [\dots] Russell's evaluation of such sentences as false is predetermined by his existence presuppositional semantics for the ‘existential' quantifier, and by the fact that his logic permits no alternative means of considering the semantic status of sentences ostensibly containing proper names for nonexistent objects.
      This makes it an altogether philosophically foregone conclusion that sentences like ‘Pegasus is winged,' which many logicians would otherwise consider to be true propositions of mythology, are false.%
      \mbox{ }\hfill\mbox{(\cite[6]{Jacquette:2002up})}
    \end{quote}
  }

  The above sense of the term `foregone conclusion' contrasts against a sense with which a conclusion which has been settled in advance of reasoning.%
  \footnote{
    For example:
    \begin{quote}
      When can a Bayesian select an hypothesis \emph{H} and design an experiment (or a sequence of experiments) to make certain that, given the experimental outcome(s), the posterior probability of \emph{H} will be greater than its prior probably?
      We discuss an elementary result that establishes sufficient conditions under which this reasoning to a foregone conclusion cannot occur.%
      \mbox{ }\hfill\mbox{(\cite[1228]{Kadane:1996vu})}
    \end{quote}
  }
  And, contrasts against a sense in which a forgone conclusion is some unavoidable state of affairs.%
  \footnote{
    For example:
  \begin{quote}
    [どうぜ][\dots] Expresses an attitude of resignation or carelessness on the part of the speaker, in the sense that regardless of what s/he does, the conclusion or outcome is foregone and cannot be changed by the will or effort of an individual.%
    \mbox{ }\hfill\mbox{(\cite[332--333]{kurufushamashii:2015un})}
  \end{quote}
  See also \citeauthor{Grice:1957vg}'s discussion of intention recognition (\citeyear[385]{Grice:1957vg}/\citeyear[219]{Grice:1989uf}).
  }
\end{note}

\begin{note}
  Non-technical `foregone conclusion' and technical `\fc{}' are similar.
  However, the modal.
  No suggestion this functions as an analysis.%
  \footnote{
    \label{fn:fc-ability}
    \fc{} is close to `able' while `foregone conclusion' is closer to not able to not.
  }
\end{note}

\begin{note}
  The progressive is key.

  Possibility is too general.
  There is a possibility in which I immediately obtain a comprehensive understanding of theoretical computer science and settle whether P is equal to NP (or show that the P versus NP problem is undecidable).
  Hence, \fc{1} are defined with respect to the progressive.

  Inherit various features of the  progressive.
  For example, \(\pv{\psi}{v'}\) from \(\Psi\) being a \fc{} is compatible with the agent failing to conclude \(\pv{\psi}{v'}\) from \(\Psi\) due to interruptions, etc.
  And, concluding may be a complex action.
\end{note}

\subsection{Illustrations}
\label{cha:fcs:illu}

\begin{note}
  We first consider cases in which \(\pv{\phi}{v}\) from \(\Phi\) (plausibly) \emph{is} a \fc{}.
  Then, we consider cases in which \(\pv{\phi}{v}\) from \(\Phi\) is \emph{not} (clearly) a \fc{}.
\end{note}

\subsubsection*{\(\pv{\psi}{v'}\) is a \fc{1} from \(\Psi\)}
\label{cha:fcs:illu:yes}

\begin{note}[Chess I]
  \begin{illustration}[Chess I]%
    \label{illu:fc:chess:I}%
    Consider the following game state:

    \mbox{ }\hfill%
    \begin{adjustbox}{minipage=\linewidth,scale=.9}
      \centering
      \newchessgame[
      setwhite={pa2,pb2,pc2,pd3,pf2,pg3,ra1,re1,bd4,kg1,qe5},
      addblack={ra8,pa7,ba6,pb5,rc8,pd5,pf7,kg8,qg4,ph7,ph4},
      ]%
      \setchessboard{showmover=false}%
      \chessboard
    \end{adjustbox}%
    \label{fig:chess:easy}%
    \hfill\mbox{ }

    Is possible for White to checkmate in a single move?%
    \footnote{
      \citeauthor{Emms:2000aa}' Puzzle 113 (\citeyear[33]{Emms:2000aa}).
    }
  \end{illustration}
\end{note}

\begin{note}
  The conclusion of interest:%
  \footnote{
    May also consider whether is possible for White to checkmate in a single move to be a \fc{}, in the sense that it is possible to decide.
    In this case, reduces to solution.
    However, in general decidable may be \fc{} without either answer being a \fc{}.
    In particular, consider an arbitrary first-order formula.
    First-order logic is decidable, however determining which is a different matter.
  }
  \[
    \pv{\prop{It is possible for White to checkmate in a single move}}{\val{True}}
  \]

  Follows from some \pool{} which captures your understanding of chess.
\end{note}

\begin{note}
  Clearest way is to do the reasoning, and then observe would not have made a different conclusion.

  Consider different pieces.
  If like me, first move (\wmove{Qe8}) does not result in checkmate.
  However, do not conclude that it is not possible.
  Other moves to check (e.g\ \wmove{Qf6}, \wmove{Re4}, \wmove{h4}, etc.).

  At some point, consider \wmove{Qh8}, which results in checkmate.
\end{note}

\begin{note}
  Here, some thing with a clear method.
  In the worst case scenario, brute force.

  Other diversions with this feature.
  Understand method, then it's a matter of execution.
  For example, Sudoku propositional theorems, or arithmetic.

  Arithmetic.
  Goes back to \autoref{illu:gist:calc}.
\end{note}

\begin{note}
  Still, the conclusion need not be easy.

  \begin{illustration}
    \[\frac{(3 + \sqrt{3})^{2} + \sqrt{6}^{2} - (2\sqrt{3})^{2}}{2(3 + \sqrt{3})\sqrt{6}} = \frac{1}{\sqrt{2}}\]
  \end{illustration}

  I suspect this is a \fc{}.
  Might need to do some work to recall principles, but it's okay.
\end{note}

\begin{todo}
  I'm not sure if additional \illu{1} of this kind would be helpful.

  I could expand on those noted, but they aren't significantly different from \autoref{illu:fc:chess:I}.

  If you have a suggestion for something interesting, though, that would be great!
\end{todo}

% \begin{note}[Propositional logic]
%   \begin{illustration}[Propositional theorems]
%     \label{illu:sketch:prop-logic}
%     Suppose an agent has a good grasp of propositional logic.
%     In particular:
%     The agent has a good understanding of some method to construct semantic proofs.
%     For example, by constructing truth tables, or reasoning about valuation functions.

%     The agent has some free time, and considers the formula:
%     \[
%       ((P \rightarrow Q) \rightarrow P) \rightarrow P
%     \]
%   \end{illustration}

%   Given the agent's understanding of propositional logic, the following is a \fc{}:
%   \begin{itemize}
%   \item
%     For any valuation \(v\), \(v \vDash ((P \rightarrow Q) \rightarrow P) \rightarrow P \)
%   \end{itemize}

%   There's nothing particularly special about the formula.
%   Given a good understanding, any formula of a reasonable length will do.

%   Note, the agent is concluding.
%   It's fine for the agent to work things through on a piece of paper.
% \end{note}

\begin{note}[Non-deductive \illu{1}]
  The following is a simple \illu{} involving non-deductive conclusion:

  \begin{illustration}[Sunny days]
    It's mid summer in the Bay Area.
  \end{illustration}

  For me, the following conclusion is a \fc{} from some \pool{}:%
  \footnote{
    Alternatively: \(\pv{\prop{It will likely not rain tomorrow}}{\val{True}}\).
  }

  \[
    \pv{\prop{It will not rain tomorrow}}{\val{Likely}}
  \]

  Of course, I recognise there is a possibility that it \emph{will} rain tomorrow.
  Still, no matter the gravitas with which I consider the possibility of rain, I am sufficiently committed to some uniformity principle that the principle, combined with past experience, lead me to conclude that it will be sunny tomorrow.
  Hence, prior to reasoning, the truth of the proposition is a \fc{}.%
  \footnote{
    Same extends to various skeptical hypotheses.
    Entertain the possibility that there is no external world, but nothing that prevents me from concluding that there is an external world.
    Though, your perspective on such issues may differ.
  }

  Note, whether or not it rains tomorrow has no bearing on whether or not it is a \fc{} (for me) that it will not rain tomorrow.
  What happens in the future has no direct bearing on what I may (or may not) conclude in the present.%
  \footnote{
    Consider~\citeauthor{Russell:1912th}'s chicken\dots (Cf.~\citeyear[63]{Russell:1912th})
  }
\end{note}

\begin{note}
  This is no different from conclusion that it's 13:48 by looking at the clock.
  I recognise there is a possibility that the clock is broken.
  However, nothing to suggest it is.

  This highlights qualification.
  Without it, then so long as wearing watch, time is a \fc{}.
  For, look at watch.
\end{note}

\begin{note}[Poppies]
  We finish the \illu{1} with something a little speculative:

  \begin{illustration}[Poppies]
    \mbox{ }
    \vspace{-\baselineskip}
    \begin{quote}
      Was Tarquinius Superbus in seinem Garten mit den Mohnköpfen sprach, verstand der Sohn, aber nicht der Bote.

      [What Tarquinius Superbus said in the garden by means of the poppies, the son understood but the messenger did not].\newline
      \mbox{ }\hfill\mbox{(\cite[3]{Kierkegaard:1983ta}/\cite[190]{Hamann:1822vp})}
  \end{quote}
  \vspace{-\baselineskip}
  \end{illustration}

  The above quite is from the epigraph to~\citeauthor{Kierkegaard:1983ta}'s \hyperlink{cite.Kierkegaard:1983ta}{Fear and Trembling}.
  \hyperlink{cite.Kierkegaard:1983ta}{H.\ Hong and E.\ Hong} detail the relevant background:

  \begin{quote}
    When the son of Tarquinius Superbus had craftily gotten Gabii in his power, he sent a messenger to his father asking what he should do with the city.
    Tarquinius, not trusting the messenger, gave no reply but took him into the garden, where with his cane he cut off the flowers of the tallest poppies.
    The son understood from this that he should eliminate the leading men of the city.%
    \mbox{ }\hfill\mbox{(\citeyear[339]{Kierkegaard:1983ta})}
  \end{quote}

  For Superbus' son, but not for the messenger the following was a \fc{} from some \pool{}:
  \[
    \pv{\prop{Eliminate the leading men of the city}}{\val{Should}}
  \]

  Or, at the very least Superbus \emph{expected} the command to eliminate the leading men of the city to be a \fc{} for his son.
\end{note}

\subsubsection*{\(\pv{\psi}{v'}\) is \emph{not} a \fc{1} from \(\Psi\)}
\label{cha:fcs:illu:no}

\begin{note}
  Two different ways.
  First, the agent may not have sufficient resources to conclude.
  Second, the agent may conclude something which conflicts, and hence does not conclude.
\end{note}

\begin{note}[Chess II]
  Observation that absence of \fc{} due to failure to conclude extends to other cases.
  What follows is a more difficult chess problem.

  \begin{illustration}[Chess II]%
    \label{illu:fc:chess:II}%
    Consider the following game state:

    \mbox{ }\hfill%
    \begin{adjustbox}{minipage=\linewidth,scale=0.9}
      \centering
      \newchessgame[
      setwhite={ka5,pa3,pb4,pc4,pe5,pf6,bg5,bh5},
      addblack={pa6,pb7,pc6,pe6,pf7,kc7,nd7,nd4},
      ]%
      \setchessboard{showmover=false}%
      \chessboard
    \end{adjustbox}%
    \label{fig:chess:intro}%
    \hfill\mbox{ }

    It is possible for Black to checkmate in four moves?%
    \footnote{
      \citeauthor{Emms:2000aa}' Puzzle 150 (\citeyear[33]{Emms:2000aa}).
    }
  \end{illustration}

  It is plausible that I would not conclude that it is possible for Black to checkmate in four moves or conversely.

  Though perhaps the bound is too low.
  If I gave it my all and attempted to work my way though all the possibilities present it may be the case that I conclude either way.
  Still, there are a lot of moves to consider, and I lack any intuition about which is correct.%
  \footnote{
    \citeauthor{Emms:2000aa} provides the following solution:
    \begin{quote}
      \variation{1... Nb6!}
      (threatening \variation{2... Nb3\#})
      \variation{2. b5}
      (or \variation{2. Bd1 Nxc4+} \variation{3. Ka4 b5\#})
      \variation{2... c5!}
      \variation{3. bxa6 Nxc4+}
      \variation{4. Ka4 b5\#}
      \textbf{(0-1)}%
      \mbox{}
      \hfill
      (\citeyear[46]{Emms:2000aa})
    \end{quote}
    My statement above remains true---I don't have sufficient background to parse this solution.
  }
  And, if you think I am doing myself a disservice, then a variant of \autoref{illu:fc:chess:II} may be restated with and increase number of moves.
\end{note}

% \begin{note}[ML II]
%   \begin{illustration}[Modal logic II]
%     \label{illu:fc:ML2}
%     The modal system \(\mathbf{GL} = \mathbf{K} + \Box(\Box p \rightarrow p) \rightarrow \Box p\) is weakly complete with respect to the class of finite strict partial orders (that is, the class of finite irreflexive transitive frames).
%   \end{illustration}

%   \autoref{illu:fc:ML2} is similar in structure to \autoref{illu:fc:logic:CR}.
%   Indeed, both proofs involve constructing a canonical model.
%   The key distinguishing feature of \autoref{illu:fc:ML2}, however, is the difficulty of establishing the canonical model has the desired properties.
%   In particular, the general method I keep in mind for proving the relevant result requires a syntactic proof that \(\vdash_{\mathbf{GL}} \Box p \rightarrow \Box \Box p\).
%   And, as I have failed to recall the relevant syntactic on sufficient occasion, I do not consider the result a \fc{0} from my understanding of modal logic.

%   Hence, the result (plausibly) fails to be a \fc{} from my understanding of modal logic because there is no guarantee that I would provide a proof if I set out to do so.%
%   \footnote{
%     On the other hand, I have completed the relevant proof a sufficient number of times.
%     So, the result is a \fc{0} from whatever premises are associated with my memory.
%   }
% \end{note}

% \begin{note}
%   \illu{3} may be obtained by taking a proposition-value pairing which conflicts with \fc{}.
%   The following is \emph{not} a \fc{}.%
%   \footnote{
%     However, I suspect the following equation is a \fc{}:
%     \[\frac{(3 + \sqrt{3})^{2} + \sqrt{6}^{2} - (2\sqrt{3})^{2}}{2(3 + \sqrt{3})\sqrt{6}} = \frac{1}{\sqrt{2}}\]
%   }

%   \begin{illustration}
%     \[\frac{(3 + \sqrt{3})^{2} + \sqrt{6}^{2} - (2\sqrt{3})^{2}}{2(3 + \sqrt{3})\sqrt{6}} = \frac{1}{\sqrt{3}}\]
%   \end{illustration}
%   For, the equation does not hold.
% \end{note}

\begin{note}
  \begin{illustration}[Knowing whether and belief]
    \label{ill:fcs:kw}
    \citeauthor{Barker:1975un} suggests the following two principles hold with respect to knowing whether:%
    \footnote{
      \citeauthor{Barker:1975un} also, as far as I can tell, endorses the principles.
    }
    \begin{enumerate}[label=(\Alph*), ref=(\Alph*), noitemsep]
    \item
      \label{Barker:1975un:A}
      If \emph{S} knows whether \emph{p} and \emph{S} believes that \emph{p}, then \emph{p}.
    \item
      \label{Barker:1975un:B}
      If \emph{S} knows whether \emph{p} and \emph{S} believes that not-\emph{p}, then not-\emph{p}.%
      \mbox{ }\hfill\mbox{(\citeyear[281]{Barker:1975un})}
    \end{enumerate}
  \end{illustration}
  I suggest neither principle is \fc{}, as you may conclude counterexamples exist to both.%
  \footnote{
    For example, consider two agents, \emph{A} and \emph{B} playing chess where each move is timed.
    It's the end game, and \emph{A} believes that \emph{B} has a winning strategy.
    Further, \emph{A} (plausibly) knows whether \emph{B} has a winning strategy.
    For, an observer has determined whether or not \emph{B} has a winning strategy, and \emph{A} is capable of tracing the reasoning of the observer.
    So, if \ref{Barker:1975un:A} holds then \emph{B} has a winning strategy.
    But, the observer knows that \emph{B} \emph{does not} have a winning strategy, and \emph{A}'s belief is mistaken.
  }\(,\)%
  \footnote{
    This \illu{} results from an investigation into the literature on knowing how.
    For, it seems plausible that various \fc{1} may be analysed in terms of the know how of the agent.
    However, I did not find anything particularly helpful.
    And, \fc{1} are sufficiently distinct.
    For example, an agent may have know how but no opportunity.
  }

  Structurally, \autoref{ill:fcs:kw} is no different from an arithmetic equation which does not hold.
  For example, \(4 \times 3 = 14\).
\end{note}

\subsubsection{Observations}
\label{cha:fcs:observations}

\begin{note}
  This is somewhat weak.
  Some action.

  It may be the case that there are other actions.

  \begin{observation}%
    \label{prop:fc:no-arb}%
    Anticipation.
  \end{observation}

  \begin{motivation}{prop:fc:no-arb}
    For, need to anticipate the event.
  \end{motivation}

  Variety of consequences.
  For example, no arbitrary.
  Further, limited jumping to conclusions.

  Though, do not overstate.
  It may be the case that the agent jumps.
  But, if so, \fc{} all the same.
  \fc{3} are not normatively constrained.
  Rather, descriptive of an agent.

  \begin{observation}%
    \label{prop:fc:lim-luck}%
    Limits luck.
  \end{observation}

  \begin{motivation}{prop:fc:lim-luck}
    Limited.
    Depends on the event.

    Note, further, that this offers no guarantee that the conclusion is good.
    May be that the agent tends to jump to conclusions.
    However, this is fine.
    For, jumps to conclusions.
  \end{motivation}
\end{note}


\section{\fc{3} and \ros{1}}
\label{cha:fcs:support}

\begin{note}
  \autoref{cha:fcs:def}, defined, discussed, and provided some \illu{1} of \fc{1}.
  We now consider the relationship between \fc{1} and \ros{0}.

  As mentioned:

  \begin{itemize}
  \item
    If \(\pv{\psi}{v'}\) is a \fc{} from \(\Psi\), then a \ros{} holds between \(\pv{\psi}{v'}\) and \(\Psi\).
  \end{itemize}

  This, then, moves some way to \issueConstraint{}.
  Though, still need to show \ros{} answers \qWhyV{}.

  We argue for the conditional, and consider some worries.
\end{note}

\subsection{\potential{2} \ros{1}}

\begin{note}
  Start with the following proposition.

  \begin{proposition}
    \label{prop:fcs-only-if-pot-support}
    \begin{itemize*}[noitemsep, label=\(\circ\)]
    \item
      An agent: \vAgent{}
    \item
      A proposition: \(\phi\)
    \item
      A value: \(v\)
    \item
      A \pool{0}: \(\Phi\)
    \item
      \mbox{ }
    \end{itemize*}

    \begin{enumerate}
    \item[\emph{If}:]
      \begin{enumerate}[label=\alph*., ref=(\alph*.)]
      \item
        \(\pvp{\phi}{v}{\Phi}\) is a \fc{0}, for \vAgent{}.
      \end{enumerate}
    \item[\emph{Then}:]
      \begin{enumerate}[label=\alph*., ref=(\alph*.), resume]
      \item
        A \potential{0} \ros{} holds between \(\pv{\phi}{v}\) and \(\Phi\), for \vAgent{}.
      \end{enumerate}
    \end{enumerate}
    \vspace{-\baselineskip}
  \end{proposition}

  \begin{argument}{prop:fcs-only-if-pot-support}
    Suppose \(\pvp{\phi}{v}{\Phi}\) is a \fc{0}.
    Then, \pevent{} in which concludes.
    Now, consider the \pevent{}.
    The culmination of the event, agent concludes.

    So, from~\supportI{}, a \ros{} holds, for the agent.

    Therefore, in whatever sense event is \potential{0}, \ros{} between \(\pv{\phi}{v}\) and \(\Phi\) is likewise \potential{0}.
  \end{argument}
\end{note}

\begin{note}
  \emph{\potential{2}} \ros{}, but it does not follow that there is a \ros{}, for the agent.
\end{note}

\subsection{\ros{3}}
\label{sec:ros3}

\begin{note}
  \begin{proposition}
    \label{prop:pot-support-onlyIf-support}
    \begin{itemize*}[noitemsep, label=\(\circ\)]
    \item
      An agent: \vAgent{}
    \item
      A proposition: \(\phi\)
    \item
      A value: \(v\)
    \item
      A \pool{0}: \(\Phi\)
    \item
      \mbox{ }
    \end{itemize*}

    \begin{enumerate}
    \item[\emph{If}:]
      \begin{enumerate}[label=\alph*., ref=(\alph*.)]
      \item
        A potential \ros{} holds between \(\pv{\phi}{v}\) and \(\Phi\), for \vAgent{}.
      \end{enumerate}
    \item[\emph{then}:]
      \begin{enumerate}[label=\alph*., ref=(\alph*.), resume]
      \item
        A \ros{0} holds between \(\pv{\phi}{v}\) and \(\Phi\), for \vAgent{}.
      \end{enumerate}
    \end{enumerate}
    \vspace{-\baselineskip}
  \end{proposition}

  \autoref{prop:pot-support-onlyIf-support} is a strong proposition.
  Only sufficient condition for \ros{} is reasoning.
  However, \wit{} is not necessary.
  Hence, argument concerns lack of anything else.

  \begin{argument}{prop:pot-support-onlyIf-support}
    \supportII{}.
    It is possible for there to be.
    So, we have everything needed.
    Both necessary and sufficient.
    Hence, for agent, \ros{}.

    So, every necessary property that does not involve witnessing.
    But, then, every necessary property.
    Therefore, sufficient.
    For, if not sufficient, then missing a necessary property.
    Contradiction.

    Slight issue, disjunction of properties.
    But, this doesn't change the argument.
    Disjunction.
  \end{argument}

  Strong proposition.
  Still, think in terms of propositional justification.%
  \footnote{
    Distinct, however, from something like reflection.
    For, dealing with ideals.
    And, no need to complete reasoning.
  }

  Further, observe that there are no particularly interesting consequences of this.
  No answer to \qWhyV{}.
\end{note}

\subsection{Combined}
\label{sec:combined}

\begin{note}
  \begin{proposition}
    \label{prop:fcs-only-if-support}
    \begin{itemize*}[noitemsep, label=\(\circ\)]
    \item
      An agent: \vAgent{}
    \item
      A proposition: \(\phi\)
    \item
      A value: \(v\)
    \item
      A \pool{0}: \(\Phi\)
    \item
      \mbox{ }
    \end{itemize*}

    \begin{enumerate}
    \item[\emph{If}:]
      \begin{enumerate}[label=\alph*., ref=(\alph*.)]
      \item
        \(\pvp{\phi}{v}{\Phi}\) is a \fc{0}.
      \end{enumerate}
    \item[\emph{then}:]
      \begin{enumerate}[label=\alph*., ref=(\alph*.), resume]
      \item
        A \ros{0} holds between \(\pv{\phi}{v}\) and \(\Phi\), for \vAgent{}.
      \end{enumerate}
    \end{enumerate}
    \vspace{-\baselineskip}
  \end{proposition}

  {
    \color{red}
    Why this is somewhat interesting.
  }
  However, before turning to the argument for \autoref{prop:fcs-only-if-support}, it is important to note the limitations of \autoref{prop:fcs-only-if-support} with respect to \issueConstraint{}.
  For, in order to argue against \issueConstraint{}, need some \(\pvp{\psi}{v'}{\Psi}\) such that answers \qWhyV{}.
  Answer \qWhyV{} only with dependence.
  Does not follow from \fc{0} that we get dependence.
\end{note}

\paragraph*{Worries}

\begin{note}
  Some doubt.
  Possible to conclude things which conflict.
  No restrictions placed on what an agent may conclude.
  But, this is no worse than \wit[es]{}.
  Possible to have concluded things which conflict.

  Though, nothing depends on this.
  If a simple way to rule out, then would.
  But, it's not so simple.
\end{note}

\begin{note}
  Could strengthen.

  \begin{enumerate}[label=\alph*., ref=(\alph*), resume*=fcCounter]
  \item
    For any proposition \(\phi'\), value \(v'\), and action \(a'\) such that \vAgent{} is concluding \(\pv{\phi'}{v'}\) from \(\Phi\) after \(a'\) is done:
    \begin{itemize}
    \item
      Either:
      \begin{enumerate}[label=\arabic*., ref=(\arabic*)]
      \item
        \(\phi'\) is \(\phi\) and \(v'\) is \(v\).%
        \footnote{
          In other words, \vAgent{} is concluding \(\pv{\phi}{v}\) from \(\Phi\) after doing \(a'\).
        }
      \item
        Throughout event in which \vAgent{} concludes \(\pv{\phi'}{v'}\) from \(\Phi\), there is some action \(b\) such that \vAgent{} is concluding \(\pv{\phi}{v}\) from \(\Phi\) after \(b\) is done.
      \end{enumerate}
    \end{itemize}
  \end{enumerate}

  This is a much stronger condition.
  Not only is it the case that concluding, but remains the case that concluding is available for any other conclusion.

  I like this condition.
  But, it's not of significant help.
  For, there is no guarantee that this extends further.

  Further, borders on normative considerations.
  \autoref{prop:fc:no-arb}, no arbitrary.

  Could say (channelling \cite{Carroll:1895uj}) that the agent \textquote{ca'n't help themselves}.
  However, what this modal amounts to is unclear.%
  \footnote{
    These concerns, in part, motivate \autoref{fn:fc-ability}.
  }
\end{note}

\begin{note}
  Still, if you prefer something stronger, that's okay.
  There's nothing in what follows that strictly requires that definition of a \fc{} is sufficient.
  Though, some things will need to be restructured.
\end{note}

\begin{note}
  \color{red}
  Worry.
  \support{2} doesn't rely on witnessing.
  Now, if this goes through, then seems \support{} for any conclusion before making the conclusion.
  However, possible for the agent to reason to different conclusions.
  For, some faulty reasoning.
  Toggle the fault.
  Therefore, \support{} for contradictory conclusions.
\end{note}

\section{Summary}
\label{cha:fcs::summary}


% \subsubsection{\wit{3}}
% \label{sec:wit3}

% \begin{note}
%   Note, there is nothing that requires a \fc{} has no prior conclusion.

%   Hence, it may be the case that an agent has a \wit{} for a \fc{}.

%   \begin{proposition}[\wit{3} for \fc{1}]
%     \label{prop:wit-for-fc}
%     Possible for \wit{} from \fc{}.
%   \end{proposition}

%   \begin{argument}{prop:wit-for-fc}
%       Immediate by def of \wit{}.
%     \end{argument}
%   \end{note}


%%% Local Variables:
%%% mode: latex
%%% TeX-master: "master"
%%% End:


% \begin{note}
%   \begin{illustration}[Modal logic I]
%     \label{illu:fc:logic:CR}
%     The modal system obtained from adding \(\Diamond\Box p \rightarrow \Box\Diamond p\) as an axiom to \(\mathbf{K}\) is canonical for the Church-Rosser property.

%     I.e. the canonical model \(W,R,V\) for \(\mathbf{K} + \Diamond\Box p \rightarrow \Box\Diamond p\) is such that \(\forall s,t,u((Rst \land Rsu) \rightarrow \exists v(Rtv \land Ruv))\).
%   \end{illustration}

%   \autoref{illu:fc:logic:CR} is a \fc{} for me.
%   Though, in contrast to the previous \illu{1}, I think there is a reasonable change that \autoref{illu:fc:logic:CR} is not a \fc{} for you.

%   Fairly routine, but two important things.
%   First, grasp on the relevant concepts.
%   If you are unaware of how to construct canonical models for normal modal logics, then unlikely that you will complete the relevant proof.
%   Second, sufficient familiarity with the relevant concepts.
%   The proof is mostly straightforward, though some care needs to be taken in showing that the canonical model for \(\mathbf{K} + \Diamond\Box p \rightarrow \Box\Diamond p\) has the Church-Rosser property.
%   Proof by contradiction is my preferred way of obtaining the result, but this requires keeping certain facts about the canonical model in mind.%
%   \footnote{
%     A slightly more interesting variation is showing that \(\mathbf{K} + \Diamond\Box p \rightarrow \Box\Diamond p\) is (strongly) complete with respect to the class of frame which have the Church-Rosser property without detour via a canonical model.
%   }

%   Similar features as \illu{1} given above.

%   In particular, perhaps clearer than \autoref{illu:gist:sudoku} in terms of mistakes.
%   For, go down some wrong path, still will not conclude until counterexample.
%   And, this is very hard to get.
% \end{note}

  %   \begin{itemize}
  %   \item
  %     The agent has a good understanding of some formal proof system.
  %     For example, some Fitch-style system.
  %   \item
  %     The agent has a good understanding of some method to construct semantic proofs.
  %     For example, by constructing truth tables, or reasoning about valuation functions.
  %   \item
  %     The agent understands the proof system is sound.
  %     That is to say, the agent understands there exists a proof of some sentence \(A\) \emph{only if} \(A\) is true given an arbitrary valuation.
  %   \end{itemize}
  %   The agent constructs the following proof:
  %   \begin{center}
  %     \begin{fitch}
  %       \phantlabel{illu:sPF:1}\fa \fh (P \rightarrow Q) \rightarrow P \\
  %       \phantlabel{illu:sPF:2}\fa \fa \fh  \lnot P \\
  %       \phantlabel{illu:sPF:3}\fa \fa \fa \fh  P \\
  %       \phantlabel{illu:sPF:4}\fa \fa \fa \fa  \bot & \(\bot\)\textbf{Intro:}\hyperref[illu:sPF:2]{2},\hyperref[illu:sPF:3]{3}\\
  %       \phantlabel{illu:sPF:5}\fa \fa \fa \fa  Q & \(\bot\)\textbf{Elim:}\hyperref[illu:sPF:4]{4}\\
  %       \phantlabel{illu:sPF:6}\fa \fa \fa P \rightarrow Q & \(\rightarrow\)\textbf{Intro:}\hyperref[illu:sPF:3]{3}--\hyperref[illu:sPF:5]{5} \\
  %       \phantlabel{illu:sPF:7}\fa \fa \fa P & \(\rightarrow\)\textbf{Elim:}\hyperref[illu:sPF:1]{1},\hyperref[illu:sPF:6]{6}\\
  %       \phantlabel{illu:sPF:8}\fa \fa \fa \bot & \(\bot\)\textbf{Intro:}\hyperref[illu:sPF:2]{2},\hyperref[illu:sPF:7]{7}\\
  %       \phantlabel{illu:sPF:9}\fa \fa \lnot\lnot P & \(\lnot\)\textbf{Intro:}\hyperref[illu:sPF:2]{2},\hyperref[illu:sPF:8]{8}\\
  %       \phantlabel{illu:sPF:10}\fa \fa P & \(\lnot\)\textbf{Elim:}\hyperref[illu:sPF:9]{9}\\
  %       \phantlabel{illu:sPF:11}\fa ((P \rightarrow Q) \rightarrow P) \rightarrow P & \(\rightarrow\)\textbf{Intro:}\hyperref[illu:sPF:1]{1}--\hyperref[illu:sPF:10]{10} \\
  %     \end{fitch}
  %   \end{center}
  %   \vspace{-\baselineskip}


%   For an additional example, consider the following from~\citeauthor{Grice:1957vg}'s~\citetitle{Grice:1957vg}:

%   \begin{quote}
%     He intends the audience's recognition of his intention to produce that response to be effective in producing that response.
%     He does not regard it as a foregone conclusion that his action will produce the intended response, whether or not his intention is recognised.%
%     \mbox{ }\hfill\mbox{(\citeyear[385]{Grice:1957vg}/\citeyear[Cf.][219]{Grice:1989uf})}
%   \end{quote}

%   In this case, the term `foregone conclusion' is embedded under negation, to highlight that the agent in question entertains the possibility that the agent's action will not produce the intended response.