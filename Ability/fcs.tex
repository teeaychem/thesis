\chapter{\fc{3}}
\label{cha:fcs}

\nocite{Ryle:1946tu}

\section{Introduction}
\label{cha:fcs:sec:introduction}

\begin{note}
  \autoref{sec:clar:type-of-scen} introduced the general type of \scen{} we are interested in.

  Here, \fc{1}.
  Two key things.
  \autoref{cha:sec:fcs-def}, account of \fc{1}.
  \autoref{cha:fcs:sec:fcs-support}, \fc{1} and support.

  If \fc{}, then relation of support.

  Important idea.
  However, limited.
  Support without witnessing doesn't raise a problem for \issueConstraint{}.
  Rather, support needs to be \emph{why}.

  \autoref{cha:zS} will develop, \check{1} on concluding.

  Main focus is \autoref{cha:sec:fcs-def}.
  What it is for some proposition-value pairing \(\pv{\phi}{v}\) to be a \fc{} from some pool of premises \(\Phi\).
  With the right set-up, \support{} --- focus on \autoref{cha:fcs:sec:fcs-support} --- will be simple.

  Progressive, action which is concluding \(\pv{\phi}{v}\) from \(\Phi\).
  Contrast with ability, `ability to conclude \(\pv{\phi}{v}\) from \(\Phi\)'.
  Argue that there is no clear path.
  Though, implicit upshot is reducing ability to progressive (given assumption).
\end{note}

\begin{note}
  Breakdown
  \begin{itemize}
  \item
    \autoref{cha:sec:fcs-def} definition of \fc{0}.
  \item
    \autoref{cha:fcs:sec:fcs-support}, support.
  \end{itemize}
\end{note}

\begin{note}
  Support.

  Idea here is something distinguished about concluding, but is independent of whether or not the agent has witnessed concluding.

  \autoref{idea:support} and \autoref{idea:support:possible}.

  Noted, independence does not entail relation of support from agent's perspective without concluding.

  Parallel to propositional and doxastic justification.
\end{note}

\section{Intuition}
\label{sec:intuition}

\begin{note}
  General term `foregone conclusion' is ambiguous.
  \begin{itemize}
    \item
    Inevitable results of reasoning.
  \item
    Conclusion which has been settled in advance of reasoning.
  \end{itemize}

  Interest is with the second meaning.
  Two examples of general use.

  % \begin{quote}
  %   [どうぜ]\dots Expresses an attitude of resignation or carelessness on the part of the speaker, in the sense that regardless of what s/he does, the conclusion or outcome is foregone and cannot be changed by the will or effort of an individual.%
  %   \mbox{ }\hfill\mbox{(\citeyear[332--333]{kurufushamashii:2015un})}
  % \end{quote}
  A clear example of the first meaning is found in~\citeauthor{Machover:1996vu}'s~\citetitle{Machover:1996vu}:
w
  \begin{quote}
    I have omitted its proof, but added a detailed analysis of the meaning of the lemma and the reason why its proof works. When this is understood, the proof itself becomes a mere technicality, almost a foregone conclusion.%
    \mbox{ }\hfill\mbox{(\citeyear[viii]{Machover:1996vu})}
  \end{quote}

  \citeauthor{Machover:1996vu} is discussing a proof, and whether or not it is inevitable that one would complete the proof (conclude that the relevant theorem is true) after understanding the lemma and why it works.
  There is no relevant sense in which the truth of the theorem has been settled in advance of reasoning.
  Though, as the proof is somewhat difficult,~\citeauthor{Machover:1996vu} only states the proof is `almost' a foregone conclusion.%
  \footnote{
    The proof is question is of the G\"{o}del-Rosser First Incompleteness Theorem.
    (\citeyear[Cf.][226]{Machover:1996vu})
  }%
  \(^{,}\)
  \footnote{
    For a similar example without qualification, consider the following from~\textcite{Jacquette:2002up}:
    \begin{quote}
    It is nevertheless important to recognize that Russell's evaluation of such sentences as false is predetermined by his existence presuppositional semantics for the ‘existential' quantifier, and by the fact that his logic permits no alternative means of considering the semantic status of sentences ostensibly containing proper names for nonexistent objects.
    This makes it an altogether philosophically foregone conclusion that sentences like ‘Pegasus is winged,' which many logicians would otherwise consider to be true propositions of mythology, are false.%
    \mbox{ }\hfill\mbox{(\citeyear[6]{Jacquette:2002up})}
  \end{quote}
  }

  For an additional example, consider the following from~\citeauthor{Grice:1957vg}'s~\citetitle{Grice:1957vg}:%
  \footnote{
    Same is found in \textcite[219]{Grice:1989uf}.
  }
  \begin{quote}
    He intends the audience's recognition of his intention to produce that response to be effective in producing that response.
    He does not regard it as a foregone conclusion that his action will produce the intended response, whether or not his intention is recognized.\newline
    \mbox{ }\hfill\mbox{(\citeyear[385]{Grice:1957vg})}
  \end{quote}

  In this case, the term `foregone conclusion' is embedded under negation, to highlight that the agent in question entertains the possibility that the agent's action will not produce the intended response.

  By contrast, the following passage from \textcite{Kadane:1996vu} is an example of the second meaning:

  \begin{quote}
    When can a Bayesian select an hypothesis \emph{H} and design an experiment (or a sequence of experiments) to make certain that, given the experimental outcome(s), the posterior probability of \emph{H} will be greater than its prior probably?
    We discuss an elementary result that establishes sufficient conditions under which this reasoning to a foregone conclusion cannot occur.%
    \mbox{ }\hfill\mbox{(\citeyear[1228]{Kadane:1996vu})}
  \end{quote}

  At issue is whether a Bayesian may chose some hypothesis \emph{H} and then guarantee some increase in probability for \emph{H} by running some experiments.
  So, are there cases in which the Bayesian first chooses a hypothesis \emph{H} and then ensures they reason to an increase in the probability of \emph{H}?
\end{note}

\begin{note}
  Our interest is with the first meaning, though narrow.
  To avoid ambiguity with first meaning, write `\fc{0}' rather than `foregone conclusion'.
  That is, the hyphen signifies when we are speaking about the technical term.
\end{note}

\section{\fc{3}}
\label{cha:sec:fcs-def}

\begin{note}[\fc{2} definition]
  We define \(\pv{\phi}{v}\) being a \emph{\fc{0}} as follows:

  \begin{restatable}[\fc{3}]{definition}{definitionForegoneC}
    \label{def:fc}
    For an agent \vAgent{}, and some proposition-value-premises pairing \(\pvp{\phi}{v}{\Phi}\):

    \begin{itemize}
    \item
      \(\pv{\phi}{v}\) is a \emph{\fc{0}} from \(\Phi\), for \vAgent{}.\newline
      \mbox{ }\hfill(equiv.\ \(\pvp{\phi}{v}{\Phi}\) is a \fc{0} for \vAgent{})
    \end{itemize}
    \emph{If and only if}
    \begin{enumerate}[label=]
    \item
      Both~\ref{def:fc:is-pe-good} and~\ref{def:fc:no-pe-bad} are true:
      \begin{enumerate}[label=\alph*., ref=(\alph*)]
      \item
        \label{def:fc:is-pe-good}
        There is \emph{some} \pevent{} \(p\) in which \vAgent{} concludes \(\pv{\phi}{v}\) from \(\Phi\).
      \item
        \label{def:fc:no-pe-bad}
        There is \emph{no} \pevent{} \(p\) in which \vAgent{} concludes some proposition-value-premises pairing which is incompatible with concluding \(\pv{\phi}{v}\) from \(\Phi\).%
        \footnote{
          Incompatible:
          \(\pv{\chi}{v''}\) from \(X\) where:
        If conclude \(\pv{\chi}{v''}\) from \(X\), then does not conclude \(\pv{\phi}{v}\) from \(\Phi\).
        }
      \end{enumerate}
    \end{enumerate}
    \vspace{-\baselineskip}
  \end{restatable}
\end{note}

\begin{note}[Intuition]
  Significant attention will be given to what is means for there to be a potential event in which an agent performs some action in \autoref{cha:sec:fcs-def:potential-events}, below.
  However, the basic features of \autoref{def:fc} follow from substituting `possible' for `\pevent{}'.
  Given this substitution:

  \begin{itemize}[noitemsep]
  \item
    Clause~\ref{def:fc:is-pe-good} ensures that there is some possibility in which the agent to conclude \(\pv{\phi}{v}\) from \(\Phi\).
  \item
    Clause~\ref{def:fc:no-pe-bad} ensures that there is no possibility in which the agent concludes something incompatible with concluding \(\pv{\phi}{v}\) from \(\Phi\).
    Hence, Clause~\ref{def:fc:no-pe-bad} rules out the agent failing to conclude \(\pv{\phi}{v}\) from \(\Phi\) because some incompatible proposition-value-premises pairing is (also) a \fc{0}.
  \end{itemize}

  Intuitively, and inevitable result of possible reasoning.
  For, there is some possibility in which the agent concludes \(\pv{\phi}{v}\) from \(\Phi\) via Clause~\ref{def:fc:is-pe-good}.
  And, at no point prior to concluding could the agent have concluded some other proposition-value-premises pairing which would prevent the agent from concluding \(\pv{\phi}{v}\) from \(\Phi\) via~\ref{def:fc:no-pe-bad}.

  Further, not merely that agent would not prevent on success, but that there is no way to block success.
\end{note}

\begin{note}
  Still, mere possibility is too general.
  There is a possibility in which I immediately obtain a comprehensive understanding of theoretical computer science and settle whether P is equal to NP (or show that the P versus NP problem is undecidable).
  Hence, \fc{1} are defined with respect to \pevent{1}.

  For the moment we will set \pevent{1} aside, and consider a handful of \illu{1} in \autoref{cha:fcs:sec:illu}.
  In \autoref{cha:sec:fcs-def:potential-events} we will provide a fairly detailed account of \pevent{1}.
  (If you would prefer to skip the \illu{1}, please turn to~\autopageref{cha:sec:fcs-def:potential-events}.)
\end{note}

\begin{note}[Neutral perspective]
  \phantlabel{fcs-neutral-perspective}
  Still, before turning to the \illu{1} and \pevent{1}, a final observation:

  Observe that whether \(\pvp{\phi}{v}{\Phi}\) is a \fc{0} is stated from a neutral perspective --- at issue is whether there is a \pevent{} in which the agent concludes.
  Our interest with \fc{1} will be from an agent's perspective, rather than whether \(\pvp{\phi}{v}{\Phi}\) \emph{is} a \fc{0}.%
  \footnote{
    Specifically, whether \(\pv{\phi}{v}\) is a \fc{0} from \(\Phi\) with respect to \vAgent{}, from \vAgent{}' perspective.
  }
  But, by defining \fc{1} from a neutral perspective, we straightforwardly understand whether \(\pvp{\phi}{v}{\Phi}\) is a \fc{0} from the agent's perspective by shifting our perspective to match the agent's perspective.
\end{note}

\section{Illustrations}
\label{cha:fcs:sec:illu}

\begin{note}
  Intuition.
  In particular, proofs.

  Before turning to detailed account of \fc{1} in \autoref{cha:sec:fcs-def}, handful of \illu{1}.

  This section consists of two parts:
  \begin{itemize}
  \item
    \autoref{cha:fcs:sec:illu:yes}, \illu{} where \(\pv{\phi}{v}\) from \(\Phi\) (plausibly) \emph{is} a \fc{}.
  \item
    \autoref{cha:fcs:sec:illu:no}, \illu{} where \(\pv{\phi}{v}\) from \(\Phi\) (plausibly) is \emph{not} a \fc{}.
  \end{itemize}
\end{note}

\subsection{\fc{3}}
\label{cha:fcs:sec:illu:yes}

\begin{note}[Chess I]
  \begin{illustration}[Chess I]
    \label{illu:fc:chess:I}
    Consider the following game state:

    \mbox{ }\hfill%
    \begin{adjustbox}{minipage=\linewidth,scale=.9}
      \centering
      \newchessgame[
      setwhite={pa2,pb2,pc2,pd3,pf2,pg3,ra1,re1,bd4,kg1,qe5},
      addblack={ra8,pa7,ba6,pb5,rc8,pd5,pf2,kg8,qg4,ph7,ph4},
      ]%
      \setchessboard{showmover=false}%
      \chessboard
    \end{adjustbox}%
    \label{fig:chess:easy}%
    \hfill\mbox{ }

    Is possible for White to checkmate in a single move?%
    \footnote{
      \citeauthor{Emms:2000aa}' Puzzle 113 (\citeyear[33]{Emms:2000aa}).
    }
  \end{illustration}
\end{note}

\begin{note}
  \fc{2} of interest:%
  \footnote{
    May also consider whether is possible for White to checkmate in a single move to be a \fc{}, in the sense that it is possible to decide.
    In this case, reduces to solution.
    However, in general decidable may be \fc{} without either answer being a \fc{}.
    In particular, consider an arbitrary first-order formula.
    First-order logic is decidable, however determining which is a different matter.
  }
  \begin{itemize}
  \item
    It is possible for White to checkmate in a single move.
  \end{itemize}
  Clearest way is to do the reasoning, and then observe would not have made a different conclusion.
\end{note}

\begin{note}
  Consider different pieces.
  If like me, first move (\wmove{Qe8}) does not result in checkmate.
  However, do not conclude that it is not possible.
  Other moves to check (e.g\ \wmove{Qf6}, \wmove{Re4}, \wmove{h4}, etc.).

  At some point, consider moving the queen from e5 to h8, which results in checkmate.

  Simple.
  Key observation is that although not immediate, conclude it is possible for White to checkmate in a single move.
  And, would not have concluded otherwise, or indeed concluded something incompatible which would have prevented (for example, that it is possible for Black's king to move to b8 between the two rooks.)

  Perhaps get bored or distracted, and didn't conclude.
  Remains the case that \fc{}.
\end{note}

\begin{note}
  Following from chess, similar structure.
  \begin{illustration}[Sudoku]
    \label{illu:gist:sudoku}
    % https://tex.stackexchange.com/questions/91422/tikz-sudoku-circle-and-connect-with-lines-some-cells
    Consider the following Sudoku puzzle:%
    \footnote{
      From~\textcite[84]{Coussement:2007up}.
    }
    \vspace{\baselineskip}

    \mbox{ }\hfill%
    \begin{adjustbox}{minipage=0.45\linewidth,scale=1}
      \centering
        \begin{tikzpicture}[scale=.5]
          \begin{scope}
            \draw (0, 0) grid (9, 9);
            \draw[very thick, scale=3] (0, 0) grid (3, 3);
            \setcounter{row}{1}
            % Single entries
            \setrow { }{ }{ }  { }{ }{ }  {1}{ }{ }
            \setrow { }{ }{ }  { }{ }{ }  { }{5}{ }
            \setrow {9}{ }{ }  { }{ }{ }  { }{ }{2}
            \setrow { }{ }{3}  { }{2}{ }  { }{ }{ }
            \setrow { }{ }{ }  {8}{ }{ }  {4}{6}{5}
            \setrow { }{4}{ }  { }{5}{9}  { }{ }{8}
            \setrow { }{8}{7}  {2}{3}{1}  { }{4}{6}
            \setrow {2}{1}{ }  {5}{ }{ }  { }{ }{3}
            \setrow {3}{ }{6}  {4}{ }{8}  { }{ }{ }
        \end{scope}
      \end{tikzpicture}
    \end{adjustbox}%
    \hfill\mbox{ }
    \vspace{\baselineskip}
  \end{illustration}

  In contrast to~\autoref{illu:fc:chess:I}, \autoref{illu:gist:sudoku} involves a number of salient \fc{}.
  Specifically, for every empty cell in the Sudoku grid consider the disjunction:
  \[
    \bigvee\{ \phi \text{ is a \fc{}} \mid \phi \in \text{Choices} \}
  \]
  where \(\text{Choices} = \{ i\text{ is the correct number to place in the cell} \mid 1 \leq i \leq 9 \}\).

  For a particular instance:
  \begin{itemize}
  \item
    1 is the correct number to place in the centre cell of the centre sub-grid is a \fc{}.
  \end{itemize}

  In other words, the solution to the Sudoku puzzle is a \fc{}, and each part of the solution to the Sudoku puzzle is a \fc{}.

  Indeed, the relevant \fc{1} follow from a basic understanding the rules of Sudoku, in same way that the possibility for White to checkmate follows from understanding of the rules of chess in \autoref{illu:fc:chess:I}.

  Still, in contrast to \autoref{illu:fc:chess:I}, there is a interesting chance of error.
  For example, accidentally placing 9 above the 8 in the bottom centre sub-grid before observing the 9 in the same column.
  Or, 4 in the top-right of the top-left sub-grid before realising already placed 4 in top-left of the top-left sub-grid.

  However, errors do not (necessarily) amount to conclusions.
  One may make an error while completing the Sudoku puzzle, but refrain from concluding that the mistaken number is the correct number to place in the cell.
  Indeed, given the possibility of error one may only conclude \(i\) is the correct number to place in cell \(c\) only when all cells have been filled and they have ensured there are no errors.

  Though, if an agent is less cautious and is inclined to immediately conclude that \(i\) is the correct number to place in cell \(c\) then each instance of the disjunction may be false.
  For, prior to attempting the puzzle there is a possibility that the agent conclude \(i\) is the correct number to place in cell \(c\) and there is a possibility that the agent may conclude \(j\) is the correct number to place in cell \(c\), where \(i \ne j\).
  Hence, there is the possibility that the agent may conclude either of two (or more) conflicting proposition-value pairs.
\end{note}

\begin{note}
  Games.
  Clear set of rules, and understanding of rules leads to answers to certain questions being determined.
  Mathematics and logic.
\end{note}

\begin{note}
  \begin{illustration}[Fraction]
    \label{illu:fc:surds}
    \[\frac{(3 + \sqrt{3})^{2} + \sqrt{6}^{2} - (2\sqrt{3})^{2}}{2(3 + \sqrt{3})\sqrt{6}} = \frac{1}{\sqrt{2}}\]
  \end{illustration}

  Granting knowledge of a handful of equalities, beyond basic addition and subtraction,%
  \footnote{
    \(\sfrac{ab}{ac} = \sfrac{b}{c}\),
    \(\sqrt{a b} = \sqrt{a}\sqrt{b}\), and
    \((a + b)^{2} = (a^{2} + 2ab + b^{2})\).
  }
  whether or not the equation is true a \fc{}, and further the truth of the equation is a \fc{}.%
  \footnote{
    \label{illu:fc:surds:fn}
    First, consider the numerator.
    Each element of the numerator may be rewritten as follows:
    \((3 + \sqrt{3})^{2} = 12 + 6\sqrt{3}\), \(\sqrt{6}^{2} = 6\), \((2\sqrt{3})^{2} = 12\).
    By summing the elements we obtain \(6\sqrt{3} + 6\).
    Hence, by rewriting, the numerator may be replaced with, \(2(3\sqrt{3} + 3)\).

    Now consider the denominator.
    Observe we may cancel multiplication by \(2\) from both the numerator and denominator.
    Further, observe \(\sqrt{6} = \sqrt{2}\sqrt{3}\).
    Hence, by distributing we  obtain, \((3\sqrt{3} + \sqrt{3}\sqrt{3})\sqrt{2}\).
    Likewise, observe \(\sqrt{3}\sqrt{3} = \sqrt{9} = 3\).
    Hence, by rewriting the denominator reads \((3\sqrt{3} + 3)\sqrt{2}\).
    As both the numerator and denominator contain \((3\sqrt{3} + 3)\), we may cancel to obtain the desired equality.
  }
  Though, if like me a few dead-ends before stumbling across the path to the solution.
  However, don't conclude any intermediary miscalculations.
  And, keep going until solution is clear.
\end{note}

\begin{note}
  \begin{illustration}[Modal logic I]
    \label{illu:fc:logic:CR}
    The modal system obtained from adding \(\Diamond\Box p \rightarrow \Box\Diamond p\) as an axiom to \(\mathbf{K}\) is canonical for the Church-Rosser property.

    I.e. the canonical model \(W,R,V\) for \(\mathbf{K} + \Diamond\Box p \rightarrow \Box\Diamond p\) is such that \(\forall s,t,u((Rst \land Rsu) \rightarrow \exists v(Rtv \land Ruv))\).
  \end{illustration}

  \autoref{illu:fc:logic:CR} is a \fc{} for me.
  Though, in contrast to the previous \illu{1}, I think there is a reasonable change that \autoref{illu:fc:logic:CR} is not a \fc{} for you.

  Fairly routine, but two important things.
  First, grasp on the relevant concepts.
  If you are unaware of how to construct canonical models for normal modal logics, then unlikely that you will complete the relevant proof.
  Second, sufficient familiarity with the relevant concepts.
  The proof is mostly straightforward, though some care needs to be taken in showing that the canonical model for \(\mathbf{K} + \Diamond\Box p \rightarrow \Box\Diamond p\) has the Church-Rosser property.
  Proof by contradiction is my preferred way of obtaining the result, but this requires keeping certain facts about the canonical model in mind.%
  \footnote{
    A slightly more interesting variation is showing that \(\mathbf{K} + \Diamond\Box p \rightarrow \Box\Diamond p\) is (strongly) complete with respect to the class of frame which have the Church-Rosser property without detour via a canonical model.
  }

  Similar features as \illu{1} given above.

  In particular, perhaps clearer than \autoref{illu:gist:sudoku} and \autoref{illu:fc:surds} in terms of mistakes.
  For, go down some wrong path, still will not conclude until counterexample.
  And, this is very hard to get.
\end{note}

\begin{note}
  Four \illu{}.
  Share common characteristic.
  Result of deductive reasoning with more-or-less explicit collection of rules.

  These characteristics are not (nor any other shared characteristic) required.
  Clear account of why \(\pv{\phi}{v}\) from \(\Phi\) is a \fc{}.

  What is required is available information, conclusion, no divergence.
\end{note}

\begin{note}[Non-deductive \illu{1}]
  The following is a simple \illu{} involving non-deductive conclusion:
  \begin{illustration}[Sunny days]
    It's mid summer in the Bay Area.
  \end{illustration}
  For me, it is a \fc{} that it will not rain tomorrow.

  Of course, I recognise there is a possibility that it \emph{may} rain tomorrow.
  However, I haven't checked the weather forecast, and with no information to the contrary I see no way of \emph{failing} to conclude that tomorrow will be sunny.
  You may object, and perhaps I am too quick to conclude that it will not rain tomorrow.

  Still, no matter the gravitas with which I consider the possibility of rain, I am sufficiently committed to some uniformity principle that the principle, combined with past experience, lead me to conclude that it will be sunny tomorrow.
  Hence, prior to reasoning, the truth of the proposition is a \fc{}.%
  \footnote{
    Same extends to various skeptical hypotheses.
    Entertain the possibility that there is no external world, but nothing that prevents me from concluding that there is an external world.
    Though, your perspective on such issues may differ.
  }

  Note, whether or not it rains tomorrow has no bearing on whether or not it is a \fc{} (for me) that it will not rain tomorrow.
  What happens in the future has no direct bearing on what I may (or may not) conclude in the present.%
  \footnote{
    Consider~\citeauthor{Russell:1912th}'s chicken\dots (Cf.~\citeyear[63]{Russell:1912th})
  }
\end{note}

\begin{note}[Poppies]
  We conclude the \illu{1} with a slightly more speculative \illu{0}:
  \begin{illustration}[Poppies]
    \mbox{ }
    \vspace{-\baselineskip}
    \begin{quote}
      Was Tarquinius Superbus in seinem Garten mit den Mohnköpfen sprach, verstand der Sohn, aber nicht der Bote.

      [What Tarquinius Superbus said in the garden by means of the poppies, the son understood but the messenger did not].\newline
    \mbox{ }\hfill\mbox{(Cf.~\cite[3]{Kierkegaard:1983ta}, and~\cite[190]{Hamann:1822vp})}
  \end{quote}
  \vspace{-\baselineskip}
  \end{illustration}
  The above quite is from the epigraph to~\citeauthor{Kierkegaard:1983ta}'s \hyperlink{cite.Kierkegaard:1983ta}{Fear and Trembling}.
  \hyperlink{cite.Kierkegaard:1983ta}{H.\ Hong and E.\ Hong} detail the relevant background:

  \begin{quote}
    When the son of Tarquinius Superbus had craftily gotten Gabii in his power, he sent a messenger to his father asking what he should do with the city.
    Tarquinius, not trusting the messenger, gave no reply but took him into the garden, where with his cane he cut off the flowers of the tallest poppies.
    The son understood from this that he should eliminate the leading men of the city.%
    \mbox{ }\hfill\mbox{(\citeyear[339]{Kierkegaard:1983ta})}
  \end{quote}
  That he should eliminate the leading men of the city was a \fc{0} for Superbus' son, but not for the messenger.
  Or, at the very least Superbus \emph{expected} the command to eliminate the leading men of the city to be a \fc{} for his son.
\end{note}

\subsection{Not clearly \fc{1}}
\label{cha:fcs:sec:illu:no}

\begin{note}
  Provided a handful of (plausible) instances of knowing whether which (plausibly) involve \fc{1}.
  A pair of (plausible) instances whether which (plausibly) do not involve \fc{1}.
\end{note}

\begin{note}[ML II]
  \begin{illustration}[Modal logic II]
    \label{illu:fc:ML2}
    The modal system \(\mathbf{GL} = \mathbf{K} + \Box(\Box p \rightarrow p) \rightarrow \Box p\) is weakly complete with respect to the class of finite strict partial orders (that is, the class of finite irreflexive transitive frames).
  \end{illustration}

  \autoref{illu:fc:ML2} is similar in structure to \autoref{illu:fc:logic:CR}.
  Indeed, both proofs involve constructing a canonical model.
  The key distinguishing feature of \autoref{illu:fc:ML2}, however, is the difficulty of establishing the canonical model has the desired properties.
  In particular, the general method I keep in mind for proving the relevant result requires a syntactic proof that \(\vdash_{\mathbf{GL}} \Box p \rightarrow \Box \Box p\).
  And, as I have failed to recall the relevant syntactic on sufficient occasion, I do not consider the result a \fc{0} from my understanding of modal logic.

  Hence, the result (plausibly) fails to be a \fc{} from my understanding of modal logic because there is no guarantee that I would provide a proof if I set out to do so.

  On the other hand, I have completed the relevant proof a sufficient number of times.
  So, the result is a \fc{0} from whatever premises are associated with my memory.
\end{note}

\begin{note}[Chess II]
  Observation that absence of \fc{} due to failure to conclude extends to other cases.
  What follows is a more difficult chess problem.
  \begin{illustration}[Chess II]
    \label{illu:fc:chess:II}
    Consider the following game state:

    \mbox{ }\hfill%
    \begin{adjustbox}{minipage=\linewidth,scale=0.9}
      \centering
      \newchessgame[
      setwhite={ka5,pa3,pb4,pc4,pe5,pf6,bg5,bh5},
      addblack={pa6,pb7,pc6,pe6,pf7,kc7,nd7,nd4},
      ]%
      \setchessboard{showmover=false}%
      \chessboard
    \end{adjustbox}%
    \label{fig:chess:intro}%
    \hfill\mbox{ }

    It is possible for Black to checkmate in four moves?%
    \footnote{
      \citeauthor{Emms:2000aa}' Puzzle 150 (\citeyear[33]{Emms:2000aa}).
    }
  \end{illustration}
  As with \autoref{illu:fc:ML2} it is plausible that I would not conclude that it is possible for Black to checkmate in four moves or conversely.

  Though perhaps the bound is too low.
  If I gave it my all and attempted to work my way though all the possibilities present it may be the case that I conclude either way.
  Still, there are a lot of moves to consider, and I lack any intuition about which is correct.%
  \footnote{
    \citeauthor{Emms:2000aa} provides the following solution:
    \begin{quote}
      \variation{1... Nb6!}
      (threatening \variation{2... Nb3\#})
      \variation{2. b5}
      (or \variation{2. Bd1 Nxc4+} \variation{3. Ka4 b5\#})
      \variation{2... c5!}
      \variation{3. bxa6 Nxc4+}
      \variation{4. Ka4 b5\#}
      \textbf{(0-1)}%
      \mbox{}
      \hfill
      (\citeyear[46]{Emms:2000aa})
    \end{quote}
    My statement above remains true---I don't have sufficient background to parse this solution.
  }
  And, if you think I am doing myself a disservice, then a variant of \autoref{illu:fc:chess:II} may be restated with and increase number of moves.
\end{note}

\begin{note}
  Conflicting conclusions.

  \illu{3} may be obtained by taking a proposition-value pairing which conflicts with \fc{}.
  For example, consider again \autoref{illu:fc:surds}.
  The following is \emph{not} a \fc{}
  \[\frac{(3 + \sqrt{3})^{2} + \sqrt{6}^{2} - (2\sqrt{3})^{2}}{2(3 + \sqrt{3})\sqrt{6}} = \frac{1}{\sqrt{3}}\]
  For, by applying the reasoning outlined in \autoref{illu:fc:surds:fn}, I would conclude the left hand side of the equation is equal to \(\sfrac{1}{\sqrt{2}}\).
\end{note}

\begin{note}
  \begin{illustration}[Knowing whether and belief]
    \citeauthor{Barker:1975un} suggests the following two principles hold with respect to knowing whether:%
  \footnote{
    \citeauthor{Barker:1975un} also, as far as I can tell, endorses the principles.
  }
    \begin{enumerate}[label=(\Alph*), ref=(\Alph*), noitemsep]
    \item
      \label{Barker:1975un:A}
      If \emph{S} knows whether \emph{p} and \emph{S} believes that \emph{p}, then \emph{p}.
    \item
      \label{Barker:1975un:B}
      If \emph{S} knows whether \emph{p} and \emph{S} believes that not-\emph{p}, then not-\emph{p}.\newline
      \mbox{ }\hfill\mbox{(\citeyear[281]{Barker:1975un})}
    \end{enumerate}
  \end{illustration}
  I suggest neither principle is \fc{}, as you may conclude counterexamples exist to both.%
  \footnote{
    For example, consider two agents, \emph{A} and \emph{B} playing chess where each move is timed.
  It's the end game, and \emph{A} believes that \emph{B} has a winning strategy.
  Further, \emph{A} (plausibly) knows whether \emph{B} has a winning strategy.
  For, an observer has determined whether or not \emph{B} has a winning strategy, and \emph{A} is capable of tracing the reasoning of the observer.
  So, if \ref{Barker:1975un:A} holds then \emph{B} has a winning strategy.
  But, the observer knows that \emph{B} \emph{does not} have a winning strategy, and \emph{A}'s belief is mistaken.
  }
\end{note}


\section{\pevent{3}}
\label{cha:sec:fcs-def:potential-events}

\begin{note}
  \autoref{def:fc} appeals to~\ref{def:fc:is-pe-good} the existence of some \pevent{} and~\ref{def:fc:no-pe-bad} the non-existence of some \pevent{}.

  However, \autoref{def:fc} does not rely on anything more than existential quantification.
  The choice is deliberate.
  We given necessary and sufficient conditions for the \emph{existence} of some \pevent{} in terms of
  \begin{enumerate*}[label=(\roman*)]
  \item
    actions available to the agent, and
  \item
    truth conditions for the progressive.
  \end{enumerate*}
\end{note}

\begin{note}[\pevent{2} definition]
  We define a \pevent{} as follows:
  \begin{restatable}[\pevent{3}]{definition}{definitionPEvent}
    \label{def:potenital-event}
    For an agent \vAgent{} and action description \(\alpha\):
    \begin{itemize}
    \item
      There is a \pevent{} \(p\) in which \vAgent{} \(\alpha\)s
    \end{itemize}
    \emph{if and only if}
    \begin{enumerate}[label=]
    \item
      Both~\ref{def:PE:action} and~\ref{def:PE:prog} are true:
      \begin{enumerate}[label=\alph*., ref=(\alph*)]
      \item
        \label{def:PE:action}
        There is some action \(a\) that \vAgent{} may immediately perform.
      \item
        \label{def:PE:prog}
        \(\text{Prog}(e, \alpha)\) would be true in the event \(e\) of \vAgent{} doing \(a\).
      \end{enumerate}
    \end{enumerate}
    Where \(\text{Prog}(e, \alpha)\) stands for the progressive from of \(\alpha\) when evaluated with respect to \(e\).%
    \footnote{
      I.e.\ \(\text{Prog}(e, \alpha)\) is true \emph{iff} event \(e\) is an event of \(\alpha\)ing.
      See,~\textcite{Richards:1981wo},~\textcite{Portner:2011vi}, etc.
    }
  \end{restatable}

  In short,~\autoref{def:potenital-event} states that there is a \pevent{} in which an agent performs some action \(\alpha\) just in case there is some action the agent may (immediately) perform which would result in the agent \(\alpha\)ing.
\end{note}

\begin{note}[Division of labour between the clauses]
  The division of labour between clauses~\ref{def:PE:action} and~\ref{def:PE:prog} is, in reverse order:
  \begin{itemize}[noitemsep]
  \item
    Clause~\ref{def:PE:prog} captures a sense of possibility, via the progressive, such that that the agent \(\alpha\)s (but in such a way that how the agent \(\alpha\)s is not necessarily settled by the action).
\item
  Clause~\ref{def:PE:action} distinguishes the existence of a \pevent{} from the agent \(\alpha\)ing by existential quantification over actions, but binds the existence of a \pevent{} to circumstances be restriction to immediate actions.
  \end{itemize}
\end{note}

\begin{note}
  Following section \autoref{cha:sec:fcs-def:ability} will get problems with ability.
  Then, \autoref{cha:sec:fcs-def:progressive}, develop understanding of the progressive by examining and modifying \citeauthor{Landman:1992wh}'s (\citeyear{Landman:1992wh}) account of the progressive.
\end{note}

\subsection{The progressive}

\begin{note}[Interest with the progressive]
  Our interest with the progressive is due to the delicate sense of possibility required for a sentence stating an event in the progressive to be true.

  \phantlabel{imperfective-paradox:intro}
  Perhaps the clearest example is the `imperfective paradox' (\citeyear[cf.][Ch.3.1]{Dowty:1979vq}).

  \citeauthor{Bach:1986tb} summarises:
  \begin{quote}
    [H]ow can we characterize the meaning of a progressive sentences like \ref{Bach:impP:17} on the basis of the meaning of a simple sentence like \ref{Bach:impP:18} when \ref{Bach:impP:17} can be true of a history without \ref{Bach:impP:18} ever being true?
    \begin{enumerate}[label=(\arabic*), ref=(\arabic*)]
      \setcounter{enumi}{16}
    \item
      \label{Bach:impP:17}
      John was crossing the street.
    \item
      \label{Bach:impP:18}
      John crossed the street.%
      \mbox{ }\hfill\mbox{(\citeyear[12]{Bach:1986tb})}
    \end{enumerate}
  \end{quote}

  No completion is required, and often some surprise.
  Something unexpected happened while John was crossing the street.
  Sense of inertia associated with the agent \(\alpha\)ing.

  Expectation that that John reaching the other side of the street does not reduce to \(\{\text{logical}, \text{metaphysical}, \text{nomic}, \dots\}\) possibility.

  For, suppose John is sitting a multiple choice exam.
  To pass the exam John only needs to chose some number of correct choices.
  It is certainly logically, metaphysically, and nomically possible that John chooses a sufficient number of correct choices.
  However, it does not follow that John is passing the exam.%
  \footnote{
    See also Igal Kvart's example of Mary wiping out the Roman army (\cite[18]{Landman:1992wh}).
  }

  Likewise, there is no simple relation to counterfactuals.
  Consider a scenario in which John is passing the exam without external help.
  Then, a classmate slips John some answers, which John then uses.
  It is no longer true that John is passing the exam without external help.
  And, in the closest possible world where the classmate does not slip John answers, it need not be true that John passes the exam without external help.
  For, if John is surrounded by students of a similar mindset then it is plausible that the in closest possible world a different classmate slips John the same answers.
\end{note}

\begin{note}
  Way the modality functions is tied to the event.

  \citeauthor{Dowty:1979vq} adds:
  \begin{quote}
    Notice, furthermore, that to Say that John was drawing a circle is not the same as saying that John was drawing a triangle, the difference between the two activities obviously having to do with the difference between a circle and a triangle.
    Yet if neither activity necessarily involves the existence of such a figure, just how are the two to be distinguished?%
    \mbox{ }\hfill\mbox{(\citeyear[133]{Dowty:1979vq})}
  \end{quote}

  As \citeauthor{Dowty:1979vq} highlights, event is sufficiently specific to determine some outcome over some other.%
  \footnote{
    Though, the force of \citeauthor{Dowty:1979vq}'s observation is perhaps clearer by substituting `square' for `circle'.
    For, straight line\dots
  }
  So, the truth of the progressive doesn't require completion and doesn't require significant progress toward completion.
\end{note}

\begin{note}
  \autoref{def:potenital-event} relies on important (but common)%
  \footnote{
    See, for example:
    \textcite{Bennett:1972uw},
    \textcite{Dowty:1979vq},
    \textcite{Parsons:1990aa},
    \textcite{Landman:1992wh}, and
    \textcite{Portner:1998um}.

    However,~\autoref{assu:prog-modal-shift} is denied by~\textcite{Szabo:2004ul}.
    \citeauthor{Szabo:2004ul} writes:
    \begin{quote}
      Sometimes we are \emph{doing} things even though there is no real chance that we could get them \emph{done}, and this is true even if we abstract away from the possibility of miraculous intervention.%
      \mbox{ }\hfill\mbox{(\citeyear[40]{Szabo:2004ul})}
    \end{quote}
    To illustrate, \citeauthor{Szabo:2004ul} denies~\ref{Szabo:Arch} is necessarily false:
    \begin{quote}
      \begin{enumerate}[label=(\arabic*), ref=(\arabic*)]
        \setcounter{enumi}{12}
      \item
        \label{Szabo:Arch}
        As the architect was building the cathedral he knew that, although he would be building it for another year or so, he couldn't possibly complete it.%
        \mbox{ }\hfill\mbox{(\citeyear[38]{Szabo:2004ul})}
      \end{enumerate}
    \end{quote}
    Though,~\ref{Szabo:Arch} seems false to me, without some priming.
    And, the only priming on which~\ref{Szabo:Arch} reads true involves interpreting the architect's knowledge from the architect's perspective, allowing a failure of factivity, thus allowing the cathedral to be built.

    Still, \autoref{assu:prog-modal-shift} is an assumption.
    The goal is not to tie potential to progressive, but to evaluation associated with the progressive granting assumption.
  }
  assumption regarding the progressive.

  \begin{assumption}[Progressive perfection]
    \label{assu:prog-modal-shift}
    For any event \(e\) and action description \(\alpha\):
    \begin{enumerate}
    \item[\emph{If}:]
      \begin{enumerate}[label=\alph*., ref=(\alph*)]
      \item
        \(\text{Prog}(e, \alpha)\) is true.
      \end{enumerate}
    \item[\emph{Then}:]
      \begin{enumerate}[label=\alph*., ref=(\alph*), resume]
      \item
        There is some possible event \(e'\) such that \(e'\) is a continuation of \(e\) and \(\alpha\) is true of \(e'\).
      \end{enumerate}
    \end{enumerate}
    \vspace{-\baselineskip}
  \end{assumption}

  \autoref{assu:prog-modal-shift}, shift evaluation to some possible event in which something related is true.

  Possible here is arbitrary.
  Important is continuation.

  Though, in same way it is common to restrict attention to some sense of possibility via an adjective, we may speak instead of( event-)continuative-possibility.

  So, task of an account of the progressive is to narrow the relevant sense of continuative-possibility.
  \autoref{assu:prog-modal-shift} holds that success is a necessary condition on continuative-possibility.
  Applied, in particular, to concluding, \autoref{assu:prog-modal-shift} holds that a agent is concluding \(\pv{\phi}{v}\) from \(\Phi\) only if there is some continuative-possibility in which the agent concludes \(\pv{\phi}{v}\) from \(\Phi\).

  Here, \fc{}.
  concluding \(\pv{\phi}{v}\) from \(\Phi\).
  Concludes \(\pv{\phi}{v}\) from \(\Phi\).
  \fc{2}.
\end{note}

\begin{note}
  Paired with choice, allows complex `incomplete' actions.
  Again, progressive develops.

  \begin{illustration}[Darts]
    There is a \pevent{} in which agent scores 180 at darts just in case there is some action available to the agent, such that if the agent were to perform the action they would be scoring 180 at darts.
  \end{illustration}

  Slightly more interesting.
  Determine the available actions.
  Though, similar, no guarantee.
  Hand is knocked at point of release, still scoring.

  Scoring 180 is a complex action.
  Though, interesting.
  First throws don't matter.

  Again, key idea is that sufficient understanding of progressive.

  And, case of interest:

  \begin{illustration}[Concluding]
    There is a \pevent{} in which agent concludes \(\pv{\phi}{v}\) from \(\Phi\) just in case there is some action available to the agent, such that if the agent were to perform the action they would be concluding \(\pv{\phi}{v}\) from \(\Phi\).
  \end{illustration}

  What is it to be concluding something.
  Like crossing the road, fail to complete.
  Like darts, recover from a bad opening.
\end{note}

\begin{note}
  Intuitive distinction between which actions may and may not perform.

  However,~\ref{def:PE:action} without.
  Allow arbitrary division of actions, what matters is immediate.

  Then, agent doing \(a\) is in progressive, so make sure that doing \(a\) is also instance of \(\alpha\).
\end{note}


\begin{note}
  Still, no full account of the progressive.
  Quite difficult.
  Progressive is familiar, intuitive understanding.
  Work through in sufficient detail to be useful.

  A little on choice.
  Then, highlight issue with ability.
  Then, present and modify \citeauthor{Landman:1992wh}'s (\citeyear{Landman:1992wh}) account of the progressive.
\end{note}

\paragraph*{Summary}

\begin{note}[Summarising]
  To summarise the preceding:
  We began with the definition of a \fc{} (\autoref{def:fc}).
  Definition of a \fc{} relies of the idea of a \pevent{}.
  And, defined \pevent{} in terms of the truth of the progressive aspect applied to a minimal event.

  The exact details of \pevent{} depends on progressive.
  \autoref{cha:sec:fcs-def:progressive-landman}, \citeauthor{Landman:1992wh}'s (\citeyear{Landman:1992wh}) account of the progressive, reconstructed with (selected) observations from \textcite{Szabo:2004ul}.

  However, gnarly.
  Before turning to the progressive, consider ability.
  The following section --- \autoref{cha:sec:fcs-def:ability} --- will raise difficulties with this suggestion.%
  \footnote{
    And implicitly suggest that any sense of ability sufficient for purpose may be analysed in terms of progressive.
  }
\end{note}

\subsection{\fc{3} and ability}
\label{cha:sec:fcs-def:ability}

\begin{note}
  Alternative suggestion is to say \pevent{} just in case agent `can \(\alpha\)' or `has the ability to \(\alpha\)'.
  Preferable, ability.

  \begin{quote}
    \(\pv{\phi}{v}\) from \(\Phi\) is a \fc{} just in case agent has the ability to conclude \(\pv{\phi}{v}\) from \(\Phi\) (and has the ability to avoid concluding something incompatible).
  \end{quote}
\end{note}


\begin{note}
  \begin{itemize}[noitemsep]
  \item
    \autoref{cha:sec:fcs-def:ability:abil-gener-spec}, distinguish general and specific abilities.
  \item
    \autoref{cha:sec:fcs-def:ability:past}, specific ability and the past.
  \item
    \autoref{cha:sec:fcs-def:ability:control-intuition} `Control'.
  \end{itemize}
\end{note}

\subsubsection{Ability, general and specific}
\label{cha:sec:fcs-def:ability:abil-gener-spec}

\begin{note}
  Particular sense of ability.

  Recall \autoref{illu:fc:ML2}.

  In general, not a \fc{} that \(\mathbf{GL}\) is weakly complete with respect to the class of finite strict partial orders.
  For, method relies on a syntactic proof \(\vdash_{\mathbf{GL}} \Box p \rightarrow \Box \Box p\).

  In this respect, it seems I do not have the ability to prove \(\mathbf{GL}\) is weakly complete with respect to the class of finite strict partial orders.

  However, if I have just (by some luck) completed or (by some studying) rehearsed a syntactic proof \(\vdash_{\mathbf{GL}} \Box p \rightarrow \Box \Box p\), then the relevant theorem is a \fc{}.

  In short, it may be true that \(\pvp{\psi}{v'}{\Psi}\) is a \fc{} for an agent while it is false that the agent has the ability to conclude \(\pv{\psi}{v'}\) from \(\Psi\).
\end{note}

\begin{note}
  Still, while there may not be an \emph{immediate} link, whether or not \(\pvp{\psi}{v'}{\Psi}\) is a \fc{} may still reduce to ability, when ability is appropriately understood.

  \phantlabel{ability-g-s-dist}%
  \nocite{Maier:2018uo}
  For, we may distinguish between `general', `categorical' or `global' abilities and `specific' or `local' abilities.

  Following \textcite[2]{Whittle:2010wr} the distinction is roughly as follows:%
  \footnote{
    Though, see~\textcite[esp.\ \S4]{Kittle:2015tb} and~\textcite[1--2]{Kikkert:2022wp} for additional discussion.%
  }
  \begin{itemize}[noitemsep]
  \item
    General (or global) abilities concern `what an agent is able to do in a large range of circumstances', while
  \item
    Specific (or local) ability concern `what the agent is able to do now, in some particular circumstances'.
  \end{itemize}

  General is just given in terms of specific.
  Not conversely, where specific is general and circumstances permit.%
  \footnote{
    For an example of this approach, see \citeauthor{Austin:1961vz}'s (\citeyear{Austin:1961vz}) discussion of `categorical' abilities and opportunities:

    \begin{quote}
      Consider the case where what we wish to assert is that somebody had the opportunity to do something but lacked the ability---`He could have smashed that lob, if he had been any good at the smash':
      here the \emph{if}-clause, which may of course be suppressed and understood, relates not to opportunity but to ability.
      \dots
      `He could have read \emph{Emma}, if he had had a copy', does seem to assert `categorically' that he had a certain ability, although he lacked the opportunity to exercise it.%
      \mbox{ }\hfill\mbox{(\citeyear[177]{Austin:1961vz})}
    \end{quote}
  }

  Example of what \textcite{Hackl:1998tt} terms `opportunity-can' (\citeyear[14]{Hackl:1998tt}):

  \begin{quote}
    \begin{enumerate}
    \item[(92)]
      \begin{enumerate}[label=\alph*., ref=(\alph*)]
      \item
        \label{Hackl:OC:a}
        A star gazer can see the solar eclipse of this year from the Cayman islands.\newline
        So if you were a star gazer and if you were on the Cayman islands at the right time you would see this year's solar eclipse.
      \item
        \label{Hackl:OC:b}
        John can see Mary from where he is standing.\newline
        So if you were standing in his place, you would see Mary.
      \end{enumerate}
    \end{enumerate}

    [\ref{Hackl:OC:b}] says that whoever is in this situation located at John's position and has normal eyesight and directs his/her gaze towards Mary will succeed in seeing Mary.%
    \mbox{ }\hfill\mbox{(\citeyear[39]{Hackl:1998tt})}
  \end{quote}
  \citeauthor{Hackl:1998tt}'s analysis straightforwardly extends to \ref{Hackl:OC:a}:
  A star gazer who is in the Cayman islands at the right time this year and looks for the solar eclipse will succeed in seeing the solar eclipse.

  So, a tentative proposal is to understand whether or not \(\pvp{\psi}{v'}{\Psi}\) is a \fc{} for an agent in terms of whether or not the agent has the \emph{specific} ability to conclude \(\pv{\psi}{v'}\) from \(\Psi\).

  Hence, we set aside \citeauthor{Austin:1961vz}'s `categorical' ability.
  Likewise we set aside `general' accounts of ability such as~\citeauthor{Carter:2021wd}'s~(\citeyear{Carter:2021wd}) `fallibilist',~\citeauthor{Kikkert:2022wp}'s~(\citeyear{Kikkert:2022wp}) `robust', and \citeauthor{Maier:2013vk}'s (\citeyear{Maier:2013vk}) `general' account, among others.

  Two issues.
  Specific ability and the past.
  `Control'

  Discussion will centre around \textcite{Boylan:2020aa}.
  Clear that specific ability (\citeyear[23, fn.3]{Boylan:2020aa})
\end{note}

\subsubsection{(Specific) ability and the past}
\label{cha:sec:fcs-def:ability:past}

\begin{note}
  First is specific ability and what actually happens.
  Two entailments.
  First, \BoyPS{} following \textcite{Boylan:2020aa}.
  Second, \BoyPSC{} the converse of \BoyPS{}.

  Combined, had the ability to if and only if did.

  Embedded in the past.
  So, doesn't say too much.
  However, difficulty with this is \fc{} depends on something not happening.
\end{note}

\begin{note}
  The first entailment is termed `\BoyPS{}'.

  \begin{enumerate}[label=]
  \item
    \label{Boylan:Past-Success}
    \BoyPS{}: \(\text{Past}(S\text{ does }\phi) \Rightarrow \text{Past}(S\text{ is able to }\phi)\)%
    \mbox{ }\hfill\mbox{(\citeyear[\S1.1]{Boylan:2020aa})}
  \end{enumerate}

  \citeauthor{Boylan:2020aa} motivates \BoyPS{} in the following way:
  \begin{quote}
    \begin{quote}
      \textbf{Fluky Dartboard}.
      I am a terrible dartplayer.
      I struggle to even hit the board whenever I take a shot.
      However, I take my shot and I flukily hit the bullseye.
    \end{quote}

    Once I have taken the shot and hit the bullseye, I can compellingly argue:

    \begin{enumerate}
      \setcounter{enumi}{2}
    \item
      I hit the bullseye on that throw.\newline
      So, I was able to hit the bullseye on that throw.
    \end{enumerate}

    If you know that I have been successful, you must concede I was able to.%
    \mbox{ }\hfill\mbox{(\citeyear[2]{Boylan:2020aa})}
  \end{quote}

  Intuitions regarding \citeauthor{Boylan:2020aa}'s case may be unclear.
  However, recall we are interested in \emph{specific} ability.
  Therefore, the argument provided is consistent with \citeauthor{Boylan:2020aa} failing to have the \emph{general} ability to hit the bullseye.%
  \footnote{
    \textcite{Bhatt:2008aa} observes:
    \begin{quote}
      \begin{enumerate}[label=(\arabic*)]
        \setcounter{enumi}{314}
      \item
        (from~\cite{Thalberg:1969ta})
        \begin{enumerate}[label=\alph*., ref=(315\alph*)]
        \item
          \label{Bhatt:Thal-a}
          Yesterday, Brown hit three bulls-eyes in a row.
          Before he hit three bulls-eyes, he fired 600 rounds, without coming close to the bullseye; and his subsequent tries were equally wild.
        \item
          \label{Bhatt:Thal-b}
          Brown was able to hit three bulls-eyes in a row.
        \item
          \label{Bhatt:Thal-c}
          Brown had the ability to hit three bulls-eyes in a row.
        \end{enumerate}
      \end{enumerate}
      From~\ref{Bhatt:Thal-a}, we can conclude~\ref{Bhatt:Thal-b} but not~\ref{Bhatt:Thal-c}.
      Brown could have hit the target three times in a row by pure chance and he does not need to have had any ability for~\ref{Bhatt:Thal-b} to be true.%
      \mbox{ }\hfill\mbox{(\citeyear[167]{Bhatt:2008aa})}
    \end{quote}
    Distinction between `was able' and `had the ability'.
    \citeauthor{Boylan:2020aa} only `was able', and so agrees with \citeauthor{Bhatt:2008aa}.

    Still, the distinction between~\ref{Bhatt:Thal-b} but not~\ref{Bhatt:Thal-c} is due to the `specific'/`general' divide.
    And, indeed, \citeauthor{Bhatt:2008aa}'s proposal, \emph{to my understanding}, identifies `was able' with the specific reading of ability and `had the ability' with the general reading of ability.

    Indeed, \citeauthor{Boylan:2020aa} makes a similar observation with respect to \citeauthor{Maier:2018uo}'s (\citeyear{Maier:2018uo}) hybrid (modal/generic) account of ability. (\citeyear[23, fn.3]{Boylan:2020aa})
  }%
  \(^{,}\)%
  \footnote{
    Finding additional instances of \BoyPS{} has difficult.

    \textcite[1]{Boylan:2020aa} mentions \citeauthor{Austin:1961vz}'s  remark that `it follows merely from the premiss that he does it, that he has the ability to do it, according to ordinary English' (\citeyear[175]{Austin:1961vz}).
    However, reasoning patterns are often not made explicit.
    Most instances of `was able to' seem to correspond to the converse entailment, \BoyPSC{}, discussed below.

    Still, a clear instance comes from \textcite{Taylor:2011uh}:
    \begin{quote}
      Consider, then, the R-statement (S):

      \begin{quote}
        Stilpo walks through the Diomean Gate at t\textsubscript{2}
      \end{quote}

      and assume that statement, tenselessly expressed so as to avoid ambiguity in what follows, to be true.

      \hbox to \hsize{\hfil{\vdots}\hfil}

      if S is true, then it follows that Stilpo was able to be walking through the gate at t\textsubscript{2}, that being, in fact, precisely what he was doing.%
      \mbox{ }\hfill\mbox{(\cite[139--143]{Taylor:2011uh})}
    \end{quote}

    More generally, one may consider the following to be instance of \BoyPS{}, in which the cited proof explains(?) why the ability attribution is true:

    \begin{quote}
      Cantor was able to show (by a proof we will not reproduce here) that \([0, 1]\) is equivalent to the power set of the integers, and thus its cardinal number is \(2^{\aleph_{0}}\).\newline
      \mbox{ }\hfill\mbox{(\cite[65]{Partee:1990tu})}
    \end{quote}

    \begin{quote}
      Blok [\hyperlink{cite.Blok:1980th}{16}] was able to give a detailed analysis of frame incompleteness by drawing on algebraic methods.
      In particular, he did so by investigating splittings (a concept from lattice theory) of the lattice of normal modal logics\dots\newline
      \mbox{ }\hfill\mbox{(\cite[74]{Blackburn:2007wa})}
    \end{quote}
  }

  Second is the converse of \BoyPS{}
  \phantlabel{BoyPSC:Start}

  \begin{enumerate}[label=]
  \item
    \label{Boylan:Past-Success:C}
    \BoyPSC{}: \(\text{Past}(S\text{ is able to }\phi) \Rightarrow \text{Past}(S\text{ does }\phi)\)
  \end{enumerate}

  \citeauthor{Bhatt:2008aa}

  \begin{quote}
    The two readings associated with \emph{be able to} allow different interpretive possibilities for indefinite/bare plural subjects.

    \begin{enumerate}[label=(\arabic*), ref=(\arabic*)]
      \setcounter{enumi}{300}
    \item
      A fireman was/Firemen were able to eat five apples.
      \begin{enumerate}[label=\alph*., ref=(301\alph*)]
      \item
        \label{Bhatt:apples:ae}
        Yesterday at the apple eating contest, a fireman was/firemen were able to eat five apples.
        (Past episodic, actuality implication, existentially interpreted subject)
      \item
        In those days, a fireman were/firemen were able to eat five apples in an hour (Generic, no actuality implication, generically interpreted subject)%
        \mbox{}\hfill\mbox{(\citeyear[160]{Bhatt:2008aa})}
      \end{enumerate}
    \end{enumerate}
  \end{quote}

  \ref{Bhatt:apples:ae} `actuality implication'.%
  \footnote{
    Following \textcite{Alxatib:2019wf}:
\begin{quote}
      Actuality Entailments (AEs) are inferences from premises that appear to be modal, like~\ref{Alxatib:a}, but their content is that the modality is effectuated in the evaluation world~---~\ref{Alxatib:b}.

      \begin{enumerate}[label=(\arabic*)]
      \item
        \begin{enumerate}[label=\alph*., ref=(1\alph*)]
        \item
          \label{Alxatib:a}
          Pierre a dû \phantom{to.pfv} prendre le \phantom{e} train \newline
          Pierre had.to.\textsc{pfv} take \phantom{dre} the train\newline
          `Pierre had to take the train'
        \item
          \label{Alxatib:b}
          \emph{Inference}: Pierre took the train.%
          \mbox{}\hfill\mbox{(\citeyear[701]{Alxatib:2019wf})}
        \end{enumerate}
      \end{enumerate}
    \end{quote}

    \citeauthor{Alxatib:2019wf} stresses the reading of `had' in (1a) is `unambiguously deontic' (\citeyear[703]{Alxatib:2019wf}).

    See \textcite{Asher:2012vr}, \textcite{Bhatt:2008aa}, \textcite{Hacquard:2006to,Hacquard:2009ta}, \textcite{Palmer:1977wb}, \textcite{Pinon:2003te}, and~\textcite{Werner:2011tp} for examples and additional discussion of actuality entailments.
  }
  Follows that a fireman/firemen ate five apples.%
  \footnote{
    In contrast, to \BoyPS{}, examples of \BoyPSC{} are plentiful.
    Two examples involving reasoning follow:

    \begin{quote}
      One can then, because of the special ``linear'' nature of the electrical process, calculate the distortion of a very complicated signal, such as Uncle Fred's voice, simply by treating it as a series of gradual ``turnings on'' and ``turnings off'' of the unit step response and adding up their combined causal influence.
      Using his operational calculus, Heaviside was able to calculate the unit step response in very quick order and then solve more complicated cases in the manner suggested.%
      \mbox{ }\hfill\mbox{(\cite[316]{Wilson:1988wx})}
    \end{quote}

     \begin{quote}
       One senses from a reading of Russell how he was able to overlook this point:
       the trouble was his failure to focus upon the distinction between ``propositional functions'' as attributes, or relations-in intension, and ``propositional functions'' as expressions\dots%
      \mbox{ }\hfill\mbox{(\cite[152]{Quine:1967tv})}
    \end{quote}

  }

  % Put these together, and specific ability, embedded under past tense reduces to what happened.

  % \begin{enumerate}[label=]
  % \item
  %   \label{Boylan:Past-Success:IFF}
  %   \BoyPSIFF{}: \(\text{Past}(S\text{ is able to }\phi) \Longleftrightarrow \text{Past}(S\text{ does }\phi)\)
  % \end{enumerate}

  The difficulty with respect to \fc{1} is that concluding \(\pv{\phi}{v}\) from \(\Phi\) doesn't entail that \(\pvp{\phi}{v}{\Phi}\) was a \fc{}.
  For, Clause~\ref{def:fc:no-pe-bad} of \autoref{def:fc}.

  In short, \(\pvp{\phi}{v}{\Phi}\) is a \fc{} only if there is no potential event in which the agent concludes something incompatible with concluding \(\pv{\phi}{v}\) from \(\Phi\).

  For, given \BoyPS{} it is not possible to express Clause~\ref{def:fc:no-pe-bad} as:
  \begin{enumerate}[label=\emph{n}., ref=(\emph{n})]
  \item
    \label{Ability:past:narrow}
    The agent has the ability to \emph{not} [conclude something incompatible with concluding \(\pv{\phi}{v}\) from \(\Phi\)].
  \end{enumerate}
  As, so long as the agent concludes \(\pv{\phi}{v}\) from \(\Phi\), then the agent (plausibly) won't have concluded something incompatible, and hence the ability attribution will be true.

  Hence, negation must scope over the ability attribution, e.g.:

  \begin{enumerate}[label=\emph{w}., ref=(\emph{w})]
  \item
    \label{Ability:past:wide}
    The agent does \emph{not} [have the ability to conclude something incompatible with concluding \(\pv{\phi}{v}\) from \(\Phi\)].
  \end{enumerate}

  Still, \label{Ability:past:wide} will differ in truth value from \label{Ability:past:narrow} only if the negated variant of \BoyPS{} does not hold:
  \begin{enumerate}[label=]
  \item
    \label{Boylan:Past-Success:CQC}
    \BoyPSCQC{}: \(\text{Past}(S\text{ does \emph{not} }\phi) \Rightarrow \text{Past}(S\text{ is \emph{not} able to }\phi)\)
  \end{enumerate}
  \BoyPSCQC{} may seems false on first glace.
  However, the entailment may be motivated in parallel to \BoyPS{}.
  For, suppose \citeauthor{Boylan:2020aa} takes the shot and does not hit the bullseye.
  It seems we may then argue that:

  \begin{enumerate}[label=\arabic*\('\).]
    \setcounter{enumi}{2}
  \item
    You didn't hit the bullseye on that throw.\newline
    So, you were not able to to hit the bullseye on that throw.
  \end{enumerate}
  So, it is by no means clear that \BoyPSCQC{} fails for the relevant sense of ability.%
  \footnote{
    Further, observe the same substitution applied to \BoyPSC{} intuitively holds.
    In particular, consider \citeauthor{Bhatt:2008aa}'s \ref{Bhatt:apples:ae} where the firemen \emph{weren't} able to eat five apples.
  }
\end{note}

\begin{note}
  To summarise, immediate issue is with entailments.
  Seems these don't coincide with \fc{}.

  However, \BoyPS{}, \BoyPSC{}, nor \BoyPSCQC{} only concern past.

  Possible to give an indirect account of ability, tying specifically to present.

  Assuming unique sense of specific ability, explore.

  For, it is not immediate that \BoyPS{}, \BoyPSC{}, and \BoyPSCQC{} to hold where `Future' or `Present' is substituted for `Past'.

  Additional motivation required.

  If do, then significant problem.
  If do not, though, remains a general question about relationship.

  And, if distinct senses of specific ability, issue is identifying relevant sense for reduction.
  In particular, though \BoyPS{} and \BoyPSCQC{} are difficult, \BoyPSC{} is robust, and raises issue.

  Following section will focus on \BoyPS{} and idea of control.
\end{note}

\subsubsection{(Specific) ability and control}
\label{cha:sec:fcs-def:ability:control-intuition}

\begin{note}[Segue]
  \autoref{cha:sec:fcs-def:ability:past} raised concerns about specific ability and what does (or does not) happen.

  In this section we present and focus on idea of `control' common to a various analyses of specific ability.
  And, we will argue that idea of control as captured by the act conditional analysis of ability is incompatible with an account of \(\pvp{\phi}{v}{\Phi}\) being a \fc{1} for an agent in terms of the agent have the ability to conclude \(\pv{\phi}{v}\) from \(\Phi\).%
  \footnote{
    Part of the interest of \textcite{Boylan:2020aa} is combining the validity of \BoyPS{} with the failure of `Present Success'.
    However, the combination isn't of interest to us.
    Though, ensures specific ability.
  }
\end{note}

\begin{note}[\AbControl{}]
  \phantlabel{ability:control}
  \textcite{Mandelkern:2017aa} express the idea of control as follows:%
  \footnote{
    \label{fn:control-accounts}
    A similar account of the control intuition is found in \textcite{Jaster:2020wv}:

  \begin{quote}
    \dots think of the ability to sing a song, to build a shag, to play tennis --- all have an action as their manifestation: the agent controls what is going on and she also controls whether to exercise the ability at all.%
    \mbox{ }\hfill\mbox{(\cite[34]{Jaster:2020wv})}
  \end{quote}

  And, \citeauthor{Boylan:2020aa}'s (\citeyear{Boylan:2020aa}) statement of the control intuition is limited to a specific example:
      \begin{quote}
        Imagine a great wave is rising and I have dashed into the sea with my surfboard.
        You know nothing about me: perhaps I am one of the world’s great surfers; perhaps I am a fool. [\dots]

    When said before the fact, the claim that I can surf that wave is strong it says that surfing that wave is within my control.
    This intuition, call it the \emph{control intuition}\dots\newline
    \mbox{ }\hfill\mbox{(\citeyear[1]{Boylan:2020aa})}
  \end{quote}

  Similar accounts may be also be found in~\textcite{Brown:1988tl},~\textcite{Kikkert:2022wp}, and~\textcite{Horty:1995wu}.
  }
  {
    \newbox\qqBoxA
    \newdimen\qqCornerHgt
    \setbox\qqBoxA=\hbox{$\ulcorner$}
    \global\qqCornerHgt=\ht\qqBoxA
    \newdimen\qqArgHgt
    \def\Quinequote #1{%
      \setbox\qqBoxA=\hbox{$#1$}%
      \qqArgHgt=\ht\qqBoxA%
      \ifnum     \qqArgHgt<\qqCornerHgt \qqArgHgt=0pt%
      \else \advance \qqArgHgt by -\qqCornerHgt%
      \fi \raise\qqArgHgt\hbox{$\ulcorner$} \box\qqBoxA %
      \raise\qqArgHgt\hbox{$\urcorner$}}

    \begin{quote}
      When someone says \(\Quinequote{\text{I [am able to] }\varphi}\), she is assuring her interlocutors that \(\sem[c]{\varphi}\) is within her control in a certain way.\newline
      \mbox{ }\hfill\mbox{(\citeyear[326]{Mandelkern:2017aa})}
    \end{quote}
  }
  For ease of reference we will refer to the idea expressed via `\AbControl{}'.
\end{note}

\begin{note}[Control via \citeauthor{Schwarz:2020aa}]
  Similar to \citeauthor{Mandelkern:2017aa}, \textcite{Schwarz:2020aa} motivates \AbControl{} as follows:

  \begin{quote}
    Suppose Cyril does not know the first 10 digits of \(\pi\).
    Intuitively,~\ref{Schwarz:pi} is then false.

    \begin{enumerate}[label=(\arabic*), ref=(\arabic*)]
      \setcounter{enumi}{2}
    \item
      \label{Schwarz:pi}
      Cyril can recite the first 10 digits of \(\pi\).
    \end{enumerate}

    \dots when we say that someone can recite the first 10 digits of \(\pi\), we don't just mean that no relevant facts preclude them from uttering `three, one, four,' etc.
    Rather, the agent must have a certain kind of intentional control over performing the act under the description of `reciting digits of \(\pi\)'.%
    \mbox{ }\hfill\mbox{(\citeyear[2]{Schwarz:2020aa})}
  \end{quote}
\end{note}

\begin{note}[Control via \citeauthor{Boylan:2020aa}]
  Likewise, \citeauthor{Boylan:2020aa} (\citeyear{Boylan:2020aa}), inspired by~\textcite{Kenny:1976vh} motivates the idea of control with the following scenario:
  \begin{quote}
    \begin{quote}
      \textbf{Unreliable Dartboard}.
      I am a fairly bad dartplayer.
      I regularly hit the bottom half when I aim for the top; and vice versa.
      But I never miss the board entirely.
    \end{quote}

    I am about to take a shot.
    I am skilled enough to know I will hit the board; so I know the following:

    \begin{enumerate}[label=(\arabic*)]
      \setcounter{enumi}{6}
    \item
      I will hit the top half of the board on this throw or I will hit the bottom half of the board on this throw.
    \end{enumerate}

    But it does not seem that I should ascribe myself either of the following abilities here:

    \begin{enumerate}[label=(\arabic*), ref=(\arabic*), resume]
    \item
      I can hit the top on the throw.
    \item
      I can hit the bottom on this throw.
    \end{enumerate}

    Even the disjunction does not seem true:

    \begin{enumerate}[label=(\arabic*), ref=(\arabic*), resume]
    \item
      \label{Boylan:10}
      I can hit the top of the board on this throw or I can hit the bottom of the board on this throw.%
      \mbox{ }\hfill\mbox{(\citeyear[3]{Boylan:2020aa})}
    \end{enumerate}
  \end{quote}

  Intuitively, \citeauthor{Boylan:2020aa} lacks control over where the dart lands on the board, the exercises control over whether the dart lands on the dartboard.
  (\citeyear[\S2,19--20]{Boylan:2020aa})
\end{note}

\begin{note}[\BoyVS{}]
    As \citeauthor{Boylan:2020aa} observes,~\ref{Boylan:10} may be further expanded into a more complex disjunction of regions on the dartboard.
  (\citeyear[4]{Boylan:2020aa})
  For example, intuitively it is not the case that:
  \begin{enumerate}[label=(\arabic*'), resume]
    \setcounter{enumi}{10}
  \item
    I can hit outside the bullseye this throw or I can hit the upper-left-quadrant of the bullseye on this throw or I can hit the lower-right-quadrant of the bullseye on this throw or \dots
  \end{enumerate}

  Indeed, \AbControl{} leads to the \emph{invalidity} of \BoyVS{}:

  \begin{enumerate}[label=]
  \item
    \label{Boylan:Or-Success}
    \BoyVS{}: \(S\text{ will }\phi \lor S\text{ will }\psi \Rightarrow S\text{ is able to }\phi \lor S\text{ is able to }\psi\)\newline
    \mbox{ }\hfill\mbox{(\citeyear[\S1.2]{Boylan:2020aa})}
  \end{enumerate}
\end{note}

\begin{note}[Need to get precise]
  Now, \AbControl{} is an idea, but is under-specified by the motivation provided.
  \citeauthor{Mandelkern:2017aa} hedge with `in a certain way', \citeauthor{Schwarz:2020aa} hedges with `certain kind', and \citeauthor{Boylan:2020aa} does not provide an explicit statement of the idea (see Footnote~\ref{fn:control-accounts}).
  Hence, \(\pvp{\phi}{v}{\Phi}\) being a \fc{1} for an agent may be equivalent to the agent having the (controlled) ability to conclude \(\pv{\phi}{v}\) from \(\Phi\).
  Or, the proposed equivalence may fail.

  So, interest turns to details of the accounts of `is able to' advanced by \textcite{Mandelkern:2017aa} and \textcite{Boylan:2020aa} in order to obtain sufficient clarity on what \AbControl{} amounts to on their understanding.
\end{note}

\begin{note}[ACA]
  We present a generalised account of the `act conditional' analysis of ability, common to \textcite{Boylan:2020aa}, \textcite{Mandelkern:2017aa}, and \textcite{Schwarz:2020aa}.%
  \footnote{
    Though \citeauthor{Schwarz:2020aa} is non-committal with respect to a formal account of ability (\citeyear[cf.][13]{Schwarz:2020aa}), the spirit of \citeauthor{Schwarz:2020aa}'s analysis is sufficiently close to \citeauthor{Boylan:2020aa}'s for the issue to arise:
    `[A]n agent has the ability to \(\phi\) iff there are accessible worlds at which she \(\phi\)s simply by deciding to \(\phi\).' (\citeyear[19]{Schwarz:2020aa})
    Decision to action, but then the decision itself must sufficiently determine the action.
  }

  \[%
    \sem[c,w]{\text{S is able to }\varphi} = 1\text{ iff }\exists A \in \mathcal{A}_{S,c,w,t}\colon \forall v \in f_{c}(\text{S does }A,w),  \sem[c,v]{\varphi(S)} = 1%
  \]

  Where:
  \begin{itemize}
  \item
    \(f_{c}\) is a selection function from proposition-world pairings to set of worlds.
  \item
    \(\mathcal{A}_{S,c,w}\) is the set of actions that are available to \(S\) in context \(c\) and world \(w\).
  \end{itemize}

  So, \(S\text{is able to }\varphi\) is true at some world \(w\) in context \(c\), just in case there is some action available to the agent, such that for every world in which it is true that \(S\text{ tries to}A\) determined by the selection function \(f_{c}\), it is the case that \(S \varphi\text{s}\).%
  \footnote{
    Strictly, both \citeauthor{Mandelkern:2017aa} and \citeauthor{Boylan:2020aa} omit universal quantification over worlds returned by the selection function.
    For \citeauthor{Mandelkern:2017aa}, as discussed below, the selection function returns a unique world, though, as discussed below, this assumption is problematic.
    For \citeauthor{Boylan:2020aa}, universal quantification is implicit by embedding `\(\varphi(S)\)' under a modal `\(\mathcal{W}\)' corresponding to `will'.
    % Issue is `if performs act, then \dots'
    % Restrictor semantics for conditional.
    % \(\sem[c,w,f]{\text{if }\phi,\psi} = 1 \text{ iff } \sem[c,w,f^{\sem[c,w,f]{\phi}}]{\psi} = 1\).
    % With universal, effectively inserting a modal.
    % Complexity of \citeauthor{Boylan:2020aa}'s account is getting the right modal.
    % Simplicity of \citeauthor{Mandelkern:2017aa}'s account is avoiding modal by assuming unique world.
  }

  Paraphrased, the act conditional analysis of ability holds:
  `\(S\) is able to \(\varphi\)' is true just in case there is some action \(A\) available to \(S\) such that if \(S\) tried to \(A\) then S would \(\varphi\).
\end{note}

\begin{note}[Selection functions]
  The primary difference between the analyses of \citeauthor{Mandelkern:2017aa} and \citeauthor{Boylan:2020aa} is the specification of \(f_{c}\), though in practice the difference seems minor:
  \begin{itemize}
  \item
    For \citeauthor{Mandelkern:2017aa},
    \(f_{c}\) is~\citeauthor{Stalnaker:1968vt}'s selection function.
    I.e.\ \(f_{c}(\psi, w) = \{v\}\) where \(v\) is the `closest' world to \(w\) where \(\psi\) is true.
    (\citeyear[Cf.][314]{Mandelkern:2017aa})

    However, the assumption of a unique `closest' world is clearly problematic given \BoyVS{}.
    For, an agent has the ability to throw a dart at the dartboard.
    Hence, in the closest possible world where the agent attempts to throw a dart, the agent succeeds.
    Further, the dart lands at some exact region of the dartboard.
    Hence, as there is only one closest world to consider, `\(S\text{ it able to throw a dart at the dartboard}\)' is strengthened to `\(S\text{ it able to throw a dart at the \emph{exact region of the} dartboard}\)'.%
    \footnote{
      In particular, \citeauthor{Mandelkern:2017aa} do not require that what the agent tries to do and what the agent does satisfy the same description (\citeyear[310,314]{Mandelkern:2017aa}).

      The same problem applies to the `orthodox approach' of~\textcite{Hilpinen:1969vw}, \textcite{Kratzer:1977aa,Kratzer:1981vn}~and~\textcite{Lewis:1976us}.
      See \textcite[\S1.3]{Boylan:2020aa} and \textcite[\S2]{Mandelkern:2017aa} for more on the orthodox approach.
    }

    Still, a \citeauthor{Lewis:1973th}ian approach where the selection function return a set of `closest' worlds resolves this issue.
    For, we may assume that the closest possible worlds determine some inexact region of the dartboard.
  \item
    For \citeauthor{Boylan:2020aa}, rather than selecting `close' worlds, \(f_{c}\) selects all worlds which are identical to \(w\) up until time \(t\) (in which \(S\) does \(A\)).%
    \footnote{
      Strictly, there is more.
      For, Non-classical Strong Kleene account of disjunction.
      (\citeyear[\S5]{Boylan:2020aa})
      Though, I really don't get it.
      Just think of \(\mathcal{W}\) as \(G\).
      `Indeterminate' is just \(F \phi \land F \lnot \phi\).
    }
  \end{itemize}

  For present purposes, the key part of the act conditional analysis for capturing \AbControl{} is that \(S \varphi\)s \emph{follows from} \(S\text{ does }A\) in all worlds captured by the selection function.

  In this respect, that the agent \(\varphi\)s is a \emph{consequence} of performing \(A\).
  In particular, it is not possible for \(A\) to set `\(\varphi\)ing in motion'.
  For, we have seen with progressive and the imperfective paradox, \(\varphi\)ing does not entail an agent \(\varphi\)s.
\end{note}

\begin{note}[Availability]
  What counts as an available action is a more complex issue.
  Thankfully, the details of \citeauthor{Mandelkern:2017aa} and \citeauthor{Boylan:2020aa} may be avoided.%
  \footnote{
    Specifically, \citeauthor{Boylan:2020aa} says little on what makes it the case that an action is available to an agent:
    \begin{quote}
      I think of an agent's available actions as their options.
      And, for simplicity at least, we can typically think of options as a set of tryings.%
      \mbox{ }\hfill\mbox{(\citeyear[14]{Boylan:2020aa})}
    \end{quote}
    In contrast, \citeauthor{Mandelkern:2017aa} consider the issue in detail.
    In short:
    \begin{quote}
      [A]n action counts as practically available only if the agent knows that it is a way of bringing about the prejacent \emph{relative to a given description of her practical situation}.\newline
      \mbox{ }\hfill\mbox{(\citeyear[321]{Mandelkern:2017aa})}
    \end{quote}
    On my understanding, there is still some gap between knowing an action is a way of bringing something about and performing the action.

    Hence, the account allows for the possibility that an agent has the ability and fails.
    For, may fail to perform the relevant action.
      See \citeauthor{Maier:2013vk} the importance of allowing for failure.
  }

  For present purposes, a sufficient understanding of when an action is available to an agent by considering \citeauthor{Boylan:2020aa}'s scenario illustrating the invalidity of \BoyVS{}.
  For, if the agent throws a dart and it hits a certain region of the dartboard, then the agent performed the act of throwing the dart at that region.
  However, throwing the dart at that region could not have been an action available to the agent on pain of \BoyVS{} having a true premise and true conclusion.
  Hence, if an agent \emph{lacks} the ability to \(\varphi\), then it cannot be the case that there is an action \(A\) available in which the agent \(\varphi\)s by doing \(A\).
\end{note}

\begin{note}[Summary of \AbControl{}]
  To summarise, it seems that on an act conditional analysis of ability, \AbControl{} amounts to the availability of some action \(A\) such that the agent \(\varphi\)ing is a consequence of performing \(A\).
\end{note}

\begin{note}[Difficulty with \fc{1}]
  With understanding of \AbControl{} in hand, we now turn to \fc{1}.

  We make the simple observation that \(\pvp{\phi}{v}{\Phi}\) may be a \fc{} without the agent having appropriate control (in the sense of \AbControl{}) over concluding \(\pv{\phi}{v}\) from \(\Phi\).

  Specifically, cases where \(\pvp{\phi}{v}{\Phi}\) is a \fc{} but for any action \(A\), it is either the case that \(A\) is inconsequential or unavailable.

  The particular \fc{} is of little importance, so we take the abstract \(\pvp{\phi}{v}{\Phi}\)-pairing.
  The key to failure of \AbControl{} is the assumption that the agent does \emph{not} have the ability to avoid distraction.%
  \footnote{
    E.g.\ the agent may be interrupted at any time, become bored, or think of something else they would prefer to do.
  }
  In particular, consider relatively simple tasks such as long but simple calculations, simple sudoku puzzles, basic chess problems, or routine proofs.

  Now, suppose \(\pvp{\phi}{v}{\Phi}\) being a \fc{0} for an agent is equivalent to the agent having the ability to conclude \(\pv{\phi}{v}\) from \(\Phi\) (where \AbControl{} holds for the relevant sense of ability).

  Given \AbControl{} it must be the case that concluding \(\pv{\phi}{v}\) from \(\Phi\) is a result of performing some action \(A\).
  However, by assumption, the agent does not have the ability of avoid distraction, and so \(A\) is not an action available to the agent.
  For, if \(A\) were available to the agent, the agent would \emph{have} the ability to avoid distraction.

  Conversely, suppose any available action allows for the possibility of distraction.
  Then, it straightforwardly follows that concluding \(\pv{\phi}{v}\) from \(\Phi\) is \emph{not} a result of performing that action.
  For, if the agent gets distracted, then the do not conclude \(\pv{\phi}{v}\) from \(\Phi\).

  In short, \AbControl{} requires an available action such concluding \(\pv{\phi}{v}\) from \(\Phi\) is a result of performing some action.
  However, \(\pvp{\phi}{v}{\Phi}\) being a \fc{} tolerates the absence of any such action.

  We may express the difference in either of two ways.
  \begin{enumerate}[label=\arabic*.]
  \item
    In order for \(\pvp{\phi}{v}{\Phi}\) to be a \fc{} it need only be the case that there is some action which results in the agent concluding \(\pv{\phi}{v}\) from \(\Phi\) (and no action where the agent does not conclude anything incompatible), though this action does not need to be \emph{available} to the agent.
  \item
    In order for \(\pvp{\phi}{v}{\Phi}\) to be a \fc{} there must be some action available to the agent, but it need not be the case that concluding \(\pv{\phi}{v}\) from \(\Phi\) (and no action where the agent does not conclude anything incompatible), is a \emph{consequence} of performing the action.
  \end{enumerate}
  As indicated by interest in the progressive, I think the second expression is correct, but either is sufficient observe that \(\pvp{\phi}{v}{\Phi}\) being a \fc{} does not require \AbControl{} over concluding \(\pv{\phi}{v}\) from \(\Phi\).
\end{note}

\subsubsection{Summary}

\begin{note}
  Entertained reducing \fc{1} to ability.
  Specific, rather than general ability.
  However, questions about specific ability.
  Specifically, with what happens, and the past.
  And, \AbControl{}.
\end{note}

\begin{note}
  Naturally, I have not argued that there is no sense of `ability' such that \(\pvp{\phi}{v}{\Phi}\) being a \fc{0} for an agent is equivalent to the agent having the ability to conclude \(\pv{\phi}{v}\) from \(\Phi\).
  However, identifying the (or a) sense of ability suitable for the equivalence is difficult.

  Hence, we pursue an account of \(\pvp{\phi}{v}{\Phi}\) being a \fc{0} in terms of \pevent{1} and the progressive.
\end{note}

\paragraph[Independent difficulty]{Independent difficulty \hfill (Optional)}

\begin{note}
  This is the `plan' account of ability.
  It's kind of insane.
  Whether or not ability reduces to action such that choice and secure outcome.
\end{note}

\begin{note}
  This is kind of wild.
  For, actions are kind of huge.
  Similar to that paper with minimalism about intentions.
\end{note}

\begin{note}
  Our direct interest with account finishes with universal.
  However, clear additional problem.
  Co-operation.
\end{note}

\begin{note}
  Uh, think.
  Has the ability to X with my help.
  There's no action in advance.
  For, whatever is chosen, I intervene prior, changing the course.
  Well, the point is, I only help if the agent gives up on whatever they had been planning to do.

  This isn't odd, cooperative activity.
  So, actually, refine example a little.
  For, point is that there's the cooperation condition.
\end{note}

\subsection{The progressive}
\label{cha:sec:fcs-def:progressive}

\begin{note}
  \autoref{cha:sec:fcs-def:ability} raised difficulty with linking \fc{1} to ability.
  Specific.
  Ability and the past.
  Understanding of \AbControl{}.
\end{note}

\begin{note}
  Understand \fc{} in terms of \pevent{3}:
  \begin{quote}
    \definitionForegoneC*
  \end{quote}
  And, \pevent{3} in terms of the progressive.
  \begin{quote}
    \definitionPEvent*
  \end{quote}
  Choice of action, similar to act conditional analysis.

  Together with progressive captures specific.
  Just need to be concluding after performing action.
  This doesn't rely on general performance.

  Key difference is with respect to \AbControl{}.
  Consequence of the action is that \(S\) is \(\varphi\)\emph{ing}.
  That \(S\) \(\varphi\)s is not a consequence of the action, only that there is some possible continuation (\autoref{assu:prog-modal-shift}).

  Task is clarifying sense of possibility.

  To develop, focus on \citeauthor{Landman:1992wh}'s (\citeyear{Landman:1992wh}) account of the progressive.

  Focus on \citeauthor{Landman:1992wh} as goes through the reasoning.
  Interest with \citeauthor{Landman:1992wh}'s account is somewhat arbitrary.
  Minimal background, events and possible worlds, and counterfactuals.
  A number of important areas where \citeauthor{Landman:1992wh}'s analysis needs to be adjusted.
  Careful distinction between event and world event takes place in.
\end{note}


\subsubsection[\citeauthor{Landman:1992wh}'s account of the progressive (modified)]{\citeauthor{Landman:1992wh}'s (\citeyear{Landman:1992wh}) account of the progressive (modified)}
\label{cha:sec:fcs-def:progressive-landman}
\nocite{Portner:1998um}
\nocite{Engelberg:1999vi}


\begin{note}
  Borrow summary from \textcite{Szabo:2004ul}:
  \begin{quote}
    [A] progressive sentence is true at some time just in case some event occurs at that time, and if we follow the development of the event (within our world as long as it goes, then jumping into a nearby world, and iterating the process within the limits of reasonability) we will reach a possible world where the perfective correlate is true of the continuation.%
    \mbox{ }\hfill\mbox{(\citeyear[34]{Szabo:2004ul})}
  \end{quote}
  The perfective correlate, links to \autoref{assu:prog-modal-shift}.
\end{note}

\begin{note}
  \begin{enumerate}
  \item
    \label{prog:max:bad}
    Max is crossing the street.
  \end{enumerate}
  True just in case there is some continuation of the actual world such that in that world, Max crossed the street.

  In the actual world, Max doesn't cross the street because Max is hit by a bus cruising at thirty miles per hour.
  (\citeyear[764]{Portner:1998um})
  Intuitively, however, \ref{prog:max:bad} remains true.
  Max was hit by the bus, but Max was not destined to be hit by the bus.
  And, if Max had not been hit by the bus, Max would have continued to cross the street.
  In other words, there is some possible world \(v\) that branched from the actual world before Max was hit by the bus.
  And, in \(v\) Max was not by the bus, and continued a little further across the road.

  Still, behind bus \#1 a second bus, bus \#2, was ready to hit max.
  And, in \(v\) Max was hit by bus \#2.
  (\citeyear[766]{Portner:1998um})
  However, like bus \#1 in \(w\), Max was not destined to be hit by bus \#2 in \(v\).
  Hence, there is some world \(u\) which branched from \(v\) in which Max continued a little further across the road.

  So long as there are a finite number of busses and no bus is destined to hit Max, then prior to being hit by a given bus, Max makes it a little further across the road.
  And, so long as the road is finite, it follows that eventually Max will have crossed the road.
\end{note}

\begin{note}
  \autoref{fig:max-bus} is a recreation of \citeauthor{Portner:1998um}'s figure 1. (\citeyear[767]{Portner:1998um})
  \begin{figure}[!h]
    \centering
    \begin{tikzpicture}
      \tikzmath{
        % x positions
        \x1 = 11;
        \xb1 = 2/9*\x1; \xb2 = 4/9*\x1; \xb3 = 6/9*\x1;
        % y positions
        \y1 = 2/5*\x1; \ymid = 1/2*\y1;
        \yw1 = \y1; \yw2 = 1/2*\y1; \yw3 = 0*\y1; \yb2 = 1/5*\y1;
        % event e
        \xe = 1/2*\xb1; \yediff = \yw2 - \yb2;
        \ye = \yw2 - 1/2*\yediff;
        \enudge = .1;
        \xel = 0; \xer = \xb1; \yen = \yw2 - \enudge;
        % bus 1 description location
        \xbx = 1.5/9*\x1; \xby = 4/5*\y1;
        % bus 2 description location
        \xb9 = 2.5/9*\x1;
      }
      % Paths
      \draw[line width=0.25mm, line cap=round] (\xb1,\ymid) -- (\xb3,\yw1); % world 1
      \draw[line width=1mm, line cap=round] (0,\ymid) -- (\xb1,\ymid) -- (\xb2,\yb2) -- (\xb3,\yw2); % world 2
      \draw[line width=0.25mm, line cap=round] (\xb2,\yb2) -- (\xb3,\yw3); % world 3
      % World descriptions
      \filldraw[black] (\xb3,\yw1) circle (0pt) node[anchor=west, align=left]{world 1: Max hit by \\ bus \# 1};
      \filldraw[black] (\xb3,\yw2) circle (0pt) node[anchor=west, align=left]{world 2: Max \\ crosses street};
      \filldraw[black] (\xb3,\yw3) circle (0pt) node[anchor=west, align=left]{world 3: Max hit by \\ bus \# 2};
      % Event
      \draw[] (\xe,\ye) -- (\xel,\yen); % event l
      \draw[] (\xe,\ye) -- (\xer,\yen); % event r
      % Event description
      \filldraw[black] (\xe,\ye) circle (0pt) node[anchor=north, align=left]{event e};
      % Splits
      \filldraw[black, dashed] (\xbx,\xby) circle (0pt) node[anchor=south, align=left]{bus \#1 hits Max};
      \filldraw[black, dashed] (\xb9,\yw3) circle (0pt) node[anchor=north, align=left]{bus \#2 hits Max};
      % Split descriptions
      \draw[-Stealth, dashed] (\xbx,\xby) -- (\xb1,\ymid + \enudge); % bus 2 arrow
      \draw[-Stealth, dashed] (\xb9,\yw3) -- (\xb2 - \enudge,\yb2 - \enudge); % bus 2 arrow
    \end{tikzpicture}
    \caption{
      Continuation path of `Max was crossing the street'. \\
    }
    \label{fig:max-bus}
  \end{figure}
\end{note}

% \begin{note}
% {
%   \color{red}
%   It's not obvious that this is the case on \citeauthor{Landman:1992wh}'s account.
% }
%   Max was rather unfortunate to be hit by a bus, but fortunes may be reversed.
%   The same reasoning applies to

%   \begin{enumerate}
%   \item
%     \label{prog:max:good}
%     Max was failing at the exam.
%   \end{enumerate}

%   Multiple choice.
%   Max goes for broke.
%   For each question, Max puts down the right answer.
%   However, Max was not destined to write down the correct answer there is a branch from how things actually happened in which Max chose the incorrect answer.
%   And, after a sequence of incorrect choices, Max failed the exam.
% \end{note}

\paragraph[\citeauthor{Landman:1992wh}~(\citeyear{Landman:1992wh})]{\citeauthor{Landman:1992wh}'s (\citeyear{Landman:1992wh}) account of the progressive}

\begin{note}
  \citeauthor{Landman:1992wh}'s account of the progressive:

  \begin{quote}
    \(\sem{\text{PROG}(e, P)}_{w,g} = 1\) iff \(\exists f \exists v\colon \langle f,v \rangle \in \text{CON}(g(e), w)\)\newline
    \phantom{an} and \(\sem{P}_{v,g}(f) = 1\)\par

    where \(\text{CON}(g(e), w)\) is the continuation branch of \(g(e)\) in \(w\).\newline
    \mbox{ }\hfill\mbox{(\citeyear[27]{Landman:1992wh})}
  \end{quote}
  Account of continuation branch.
\end{note}

\begin{note}
  However, following proposal by \citeauthor{Szabo:2004ul}, variant account of the progressive which allows for continuation trees (as opposed to branches).

  \begin{quote}
    \begin{enumerate}[label=(\Roman*), ref=(\Roman*)]
      \setcounter{enumi}{5}
    \item
      \emph{Prog}[\(\varphi\)] is true at \(t\) in \(w\) iff there is an \(e\) at \(t\) in \(w\) and for every \(\langle e^{\ast}, w^{\ast} \rangle\) on the continuation tree for \(e\) in \(w\) if \(\varphi\) is not true of \(e^{\ast}\) at \(w^{\ast}\) then there is an \(\langle e', w' \rangle\) on the continuation tree for \(e\) in \(w\) such that \(e'\) is a continuation of \(e^{\ast}\) in \(w'\) and \(\varphi\) is true of \(e'\) at \(w'\).%
      \mbox{ }\hfill\mbox{(\citeyear[37]{Szabo:2004ul})}
    \end{enumerate}
  \end{quote}

  Continuation tree, as there may be no unique continuation branch for \(e\).
  \citeauthor{Szabo:2004ul}'s (proposed) definition covers possibility, and requires that for every way event develops, always part of some branch or has been true.
  I.e.\ that while false, there is some (possible) future where event is true.

  Note, continuation tree is generalisation of continuation branch.
  Hence, \autoref{fig:max-bus} does not represent a continuation tree.

  Indeed, \citeauthor{Szabo:2004ul}'s (proposed) definition is equivalent to \citeauthor{Landman:1992wh}'s definition if `continuation branch' is substituted for `continuation tree', for the relevant continuations will be limited to a single branch.

  Still, not simply adopting \citeauthor{Szabo:2004ul}'s (proposed) definition.
  Significant changes to continuation tree.
\end{note}

\begin{note}
  Immediate goal is to present \citeauthor{Landman:1992wh}'s account of a continuation branch.
  In turn, this will require expansion on three further points.
  Stages of an event.
  Continuations and stops with respect to events.
  Reasonable options.

  With understanding in hand algorithmic reconstruction of continuation branch.
  Expand in two ways.
  First, tree, to allow for forks.
  Second, a different account of how to identify branches.
\end{note}

\subparagraph{Continuation branch}

\begin{note}
  \citeauthor{Landman:1992wh}'s account of a continuation branch is as follows:
  \begin{quote}
    The \emph{continuation branch} for \(e\) in \(w\) is the smallest set of pairs of events and worlds such that
    \begin{enumerate}
    \item
      \label{Landman:CB:continues}
      for every event \(f\) in \(w\) such that \(e\) is a stage of \(\langle f,w \rangle \in C(e,w)\);
      the continuation stretch of \(e\) in \(w\);
    \item
      \label{Landman:CB:stops}
      if the continuation stretch of \(e\) in \(w\) stops in \(w\), it has a maximal element \(f\) and \(f\) stops in \(w\).
      Consider the closest world \(v\) where \(f\) does not stop:
      \begin{enumerate}[label=--]
      \item
        if \(v\) is not in \(\lRwe{w}{e}\), the continuation branch stops.
      \item
        if \(v\) is in \(\lRwe{w}{e}\), then \(\langle f,v \rangle \in C(e,w)\).
        In this case, we repeat the construction:
      \end{enumerate}
    \item
      \label{Landman:CB:continues:again}
      for every \(g\) in \(v\) such that \(f\) is a stage of \(g\), \(\langle g,v \rangle \in C(e,w)\), the continuation stretch of \(e\) in \(v\);
    \item
      \label{Landman:CB:stops:again}
      if the continuation stretch of \(e\) in \(v\) stops, we look at the closest world \(z\) where its maximal element \(g\) does not stop:
      \begin{enumerate}[label=--]
      \item
        if \(z\) is not in \(\lRwe{w}{e}\), the continuation branch stops.
      \item
        if \(z\) is in \(\lRwe{w}{e}\), then \(\langle g,z \rangle \in C(e,w)\) and we continue as above, etc.%
        \mbox{ }\hfill\mbox{(\citeyear[26--27]{Landman:1992wh})}
      \end{enumerate}
    \end{enumerate}
  \end{quote}

  Describes the process of following an event \(e\) and a world \(w\) and jumping to nearby reasonable worlds when \(e\) stops in \(w\) (or \(w'\), etc.).%
  \footnote{
    More general point.

    Whether it is possible to capture by event alone is unclear.
    As \textcite[1256]{Portner:2011vi} observes:
    \begin{enumerate}[label=\arabic*., ref=(\arabic*)]
    \item
      Max is crossing the street.
    \end{enumerate}
    May be re-described
    \begin{enumerate}[label=\arabic*\('\)., ref=(\arabic*\('\))]
    \item
      Max is walking into the path of an oncoming bus.
    \end{enumerate}

    However, re-description, at least on \citeauthor{Landman:1992wh}'s account.
    For, event stops.
    If walking into, then intuitively the event doesn't stop when Max is hit by the bus.
    Yet, only branch when stops, which must be before Max is hit.
    So, relevance of re-description is not isolated to branching.

    Perspective.
    Something which selects subset of information.%
    \footnotemark
    Hence, if perspective, then already have perspective.

    Deny same event.
    Rather, re-description of the interval via different event.

    Hence, isolate the how the event progresses independently of a world in which it progresses.
  }
  \footnotetext{
    \color{red}
    May captured by attention to fact.

  \citeauthor{Veltman:2005tj}'s (\citeyear{Veltman:2005tj}) revision to~\citeauthor{Tichy:1976tp}'s (\citeyear{Tichy:1976tp}) counterexample to \citeauthor{Lewis:1979vm}' (\citeyear{Lewis:1973th,Lewis:1979vm}) theory of counterfactuals.

  Some `facts' are fixed, and it is not possible to alter these facts under a counterfactual assumption.
  \(e\) has been set in motion.
  Now, keeping \(e\) fixed, if not X then e would continue.

  Fix what has been set in motion by \(e\), keep all laws the same.
  Then, buses at different times, but Mary doesn't get superpowers.
  }

  Observe, the construction of a continuation branch is iterative.
  Clauses~\ref{Landman:CB:continues:again} and~\ref{Landman:CB:stops:again} and duplicates of Clauses~\ref{Landman:CB:continues} and~\ref{Landman:CB:stops} shifted to \(g\) --- some development of \(e\) --- in some possible world \(v\).

  Reconstruction via a recursive algorithm.
  For the moment we leave \citeauthor{Landman:1992wh}'s account of a continuation branch without commentary.
\end{note}

\subparagraph{Stages}

\begin{note}
  An event being a stage of some other event.
  Clause~\ref{Landman:CB:continues} (and~\ref{Landman:CB:continues:again}).

  \citeauthor{Landman:1992wh}'s definition is light:
  \begin{quote}
    An event is a stage of another event if the second can be regarded as a more developed version of the first, that is, if we can point at it and say, ``It's the same event in a further stage of development.''\newline
    \mbox{ }\hfill\mbox{(\citeyear[23]{Landman:1992wh})}
  \end{quote}
  \citeauthor{Landman:1992wh}'s definition is from an agent's perspective.
  Even if the elaboration is ignored, the initial expansion is qualified by the term `can be regarded as'.
  However, we may provide a definition of a stage independent of an agent's perspective:
  \begin{definition}[Stage]
    For events \(e\) and \(f\):

    \begin{itemize*}
    \item
      \(e\) is a stage of \(f\)
    \item
      \emph{if and only if}
    \item
      \(f\) is a development of \(e\)
    \item
      \mbox{ }
    \end{itemize*}
  \end{definition}
  At issue is what it is for \(f\) to be a \emph{development} of \(e\).

  Stage is distinguished from part-of.
  \nocite{Davidson:1967aa}

  Buttering the toast.
  Bread was toasted.
  But, toasting the bread was not a stage of buttering the toast.
  Sufficiently distinct events.
  Marked by `toast'.
  Toasting and buttering the bread, then plausible stage.
  Likewise, scooping butter onto the knife, stage of buttering the toast.

  Stage does not entail anything of significance happens.
  Waiting for the post to arrive.
  Stage, nothing is happening, event has developed.

  So, definition by intuitive distinction.

  Importance is for shifting-to and tracing-events-through possible worlds.
\end{note}

\subparagraph{Continuations and Stops}

\begin{note}[Continuations and Stops]
  The definition of a stage is important for defining both the continuation and when an event stops.
  We present \citeauthor{Landman:1992wh}'s definition of both terms, and then provide restated definitions.
\end{note}

\begin{note}
  \citeauthor{Landman:1992wh} combines the definition of a continuation and when an event stops:
  \begin{quote}
    This is where stages come in: we cannot say that when an event stops in a world, there is no bigger event of which it is part in that world, but we can say that when it stops, there is no bigger event in the world of which it is a \emph{stage}:
    \begin{enumerate}[label=, noitemsep]
    \item
      Let \(e\) be an event that goes on at \(i\) in \(w\).
      Let \(f\) be an event that goes on at \(j\) in where \(i\) is a subinterval of \(j\).
    \item
      \(j\) is a continuation of \(e\) iff \(e\) is a stage of \(f\).
    \item
      Let \(j\) be a non-final interval.
    \item
      \(f\) stops at \(j\) in \(w\) iff no event of which \(f\) is a stage goes on beyond \(i\) in \(w\) (i.e., at a later ending interval).\newline
      \mbox{ }\hfill\mbox{(\citeyear[23--24]{Landman:1992wh})}
    \end{enumerate}
  \end{quote}

  To clarify the definitions, we borrow the relevant definitions regarding intervals from \textcite{Dowty:1979vq}:

  \begin{quote}
    \(I\) is a subinterval of \(J\) iff \(I \subseteq J\), where \(I\) and \(J\) are intervals.
    \(I\) is a proper subinterval of \(J\) iff \(I \subset J\).
    \(I\) is an \emph{initial subinterval} of \(J\) iff \(I\) is a subinterval of \(J\) and there is no \(t \in (J - I)\) for which there is \(t' \in I\) such that \(t \leq t'\).
    \emph{Final subinterval} is defined similarly\dots\newline
    \mbox{ }\hfill\mbox{(\citeyear[140]{Dowty:1979vq})}
  \end{quote}
  So, following \citeauthor{Dowty:1979vq}:
  \begin{quote}
    \(I\) is a \emph{final subinterval} of \(J\) iff \(I\) is a subinterval of \(J\) and there is no \(t \in (J - I)\) for which there is \(t' \in I\) such that \(t' \leq t\).
  \end{quote}
  And, from the definition of a non-final subinterval follows similarly:
  \begin{quote}
    \(I\) is a \emph{non-final subinterval} of \(J\) iff \(I\) is a subinterval of \(J\) and there is \emph{some} \(t \in (J - I)\) for which there is \(t' \in I\) such that \(t' \leq t\).
  \end{quote}
  Intuitively, then, \(I\) is a \emph{non}-final interval of \(J\), just in case \(I \subseteq J\) and \(J\) progresses further in time than \(I\).

  Let us now turn to restating the definitions of a continuation and a stop.
  We take each in turn.
\end{note}

\begin{note}
  The definition of \(f\) being a continuation of \(e\) requires two things:
  First, it must be the case the event at which \(e\) takes place is a subinterval of the event at which \(f\) takes place, and it must be the case that \(e\) is a stage of \(f\).%
  \footnote{
    Though, it seems to me the latter/\ref{def:Landman:conts:stage} implies the former/\ref{def:Landman:conts:interval}.
    I.e.\ if \(e\) is a stage of \(f\) then the interval at which \(e\) takes place must be a subinterval of the interval at which \(f\) takes place.
  }
  In full:

  \begin{definition}[Continuations]
    \label{def:Landman:conts}
    For events \(e\) and \(f\):
    \begin{itemize}
    \item \(f\) is a \emph{continuation} of \(e\)
    \end{itemize}
    \emph{if and only if}
    \begin{itemize}
    \item
      The following jointly hold:
      \begin{enumerate}[label=\alph*., ref=(\alph*)]
      \item
        \label{def:Landman:conts:interval}
        \begin{enumerate}
        \item[\emph{If}:]
          \begin{enumerate}[label=\roman*.]
          \item
            \(i\) is the interval at which \(e\) takes place in \(w\).
          \end{enumerate}
        \item[\emph{And}:]
          \begin{enumerate}[label=\roman*., resume]
          \item
            \(j\) is the interval at which \(f\) takes place.
          \end{enumerate}
        \item[\emph{Then}:]
          \begin{enumerate}[label=\roman*., resume]
          \item
            \(i\) is a subinterval of \(j\).
          \end{enumerate}
        \end{enumerate}
      \item
        \label{def:Landman:conts:stage}
        \(e\) is a stage of \(f\).
      \end{enumerate}
    \end{itemize}
    \vspace{-\baselineskip}
  \end{definition}
\end{note}

\begin{note}
  The definition of \(f\) stopping at \(j\) in \(w\) takes a little more work.

  There are two immediate issues with \citeauthor{Landman:1992wh}'s definition.

  First, the definition requires that \(j\) is a non-final interval.
  But, of what?
  By assumption \(i\) is a subinterval of \(j\), so \(j\) cannot be a non-final interval of \(i\).%
  \footnote{
    The term `non-final interval' only appears in the above quote from \citeauthor{Landman:1992wh}.
  }

  Second, there must be no event of which \(f\) is a stage that goes beyond \(i\) in \(w\).
  However, by assumption \(f\) is an event that goes on at \(j\), and \(i\) is a subinterval of \(j\).
  But why is \(f\) bound by some arbitrary interval \(i\)?

  I propose to resolve both issues with the following definition:
  \begin{definition}[Stops]
    \label{def:Landman:Stops}
    For an event \(e\), interval \(i\), and world \(w\):
    \begin{itemize}
    \item
      \(e\) \emph{stops} at \(i\) in \(w\)
    \end{itemize}
    \emph{If and only if:}
    \begin{itemize}
    \item
      There no is interval \(j\) nor event \(f\) such that the following jointly hold:
      \begin{enumerate}[label=\alph*., noitemsep]
      \item
        \(i\) is non-final subinterval of \(j\).
      \item
        \(f\) goes on at \(j\).
      \item
        \(f\) is a stage of \(e\).
      \end{enumerate}
    \end{itemize}
    \vspace{-\baselineskip}
  \end{definition}
  In short, there is no expansion from \(i\) \emph{forward} in time to obtain an interval \(j\) such that an event which is a stage of \(e\) got on at \(j\).

  Hence, \autoref{def:Landman:Stops} captures \citeauthor{Landman:1992wh}'s initial gloss: `[W]hen [\(e\)] stops, there is no bigger event in the world of which [\(e\)] is a \emph{stage}' (\citeyear[23]{Landman:1992wh})
\end{note}

\begin{note}[Example of \autoref{def:Landman:Stops}]
  Interesting case to consider.
  Time limits.
  Running the race in record time.
  But, falls short of the record.

  Now, it may be false.
  Or, it may be true.
  Stops, no longer possible to complete race in record time.
  Shift to possible world.
  However, stops when event is still in progress.
  So, shift two possible world at time when stop will be completing race in record time.
\end{note}

\subparagraph{Reasonable options}

\begin{note}
  Reasonable options
  \begin{quote}
    \label{def:LandRwe}
    \(v \in \lRwe{w}{e}\) iff there is a reasonable chance on the basis of what is internal to \(e\) in \(e\) that \(e\) continues in \(w\) as far as it does in \(v\).%
    \mbox{ }\hfill\mbox{(\citeyear[26]{Landman:1992wh})}
  \end{quote}

  Look at possible worlds.
  Consider event \(e\) in possible world \(v\).
  Consider \(e\) in actual world.
  If there is a reasonable chance that \(e\) continues in \(w\) as in \(v\), then \(v\) is reasonable, on the basis of what is `internal' to \(e\).

  Not at all clear.
  However, significant insight from role in \citeauthor{Landman:1992wh}'s account of the progressive.
  \autoref{cha:sec:fcs-def:progressive-landman:alg:branches} considers in detail.

  Briefly, however, `reasonable' is intuitively epistemic, but need not be.
\end{note}

\paragraph{An algorithmic (re)construction}
\label{cha:sec:fcs-def:progressive-landman:alg}
\nocite{Cormen:2009uw}

\begin{note}
  Breaking this down into a recursive algorithm.
  Goal is to create a tree, which will be a set of ordered event-world-pairing pairs indexed according to depth.
  For example:
  \[\text{Tree} = \{\langle \langle e,w \rangle_{1}, \langle f,w \rangle_{1} \rangle, \langle \langle f,w \rangle_{1}, \langle g,v \rangle_{2} \rangle, \langle \langle f,w \rangle_{1}, \langle g,v' \rangle_{2} \rangle, \dots \}\]

  Root, initial event world pair, and then branch from event world pair to some distinct event world pair when index changes.
  %
  \footnote{
    Indexing is required construct the tree as allow events to develop in different ways.
    However, also important for keeping track of the stage of construction.
  }

  We start by (re)constructing three basic algorithms.
  Following, we introduce a variation to one of these algorithms.
  And, finally, we (re)construct a recursive algorithm to build a continuation branch of some event-world pairing.

  \begin{itemize}[noitemsep]
  \item
    \AlgAC{(\(\langle g,u \rangle_{i}\))}%
    \hfill%
    \autoref{cha:sec:fcs-def:progressive-landman:alg:conts}
    \begin{itemize}
    \item
      An algorithm to obtain the continuation stretch of some event \(g\) in world \(u\).

      Builds on Clause~\ref{Landman:CB:continues} (and~\ref{Landman:CB:continues:again}).

      Expanded to help following algorithms.
    \end{itemize}
  \item
    \AlgGetStops{(\(\text{Continuations}\))}%
    \hfill%
    \autoref{cha:sec:fcs-def:progressive-landman:alg:stops}
    \begin{itemize}
    \item
      An algorithm to find the stopping points in some set of continuation stretches.

      Builds on the antecedent of Clause~\ref{Landman:CB:stops} (and~\ref{Landman:CB:stops:again})

      Follows \citeauthor{Landman:1992wh}, though will be revised.
    \end{itemize}
  \item
    \AlgFindBranches{(\(\langle \langle f,v \rangle_{i-1}, \langle g,u \rangle_{i}\rangle, e, w\))}%
    \hfill%
    \autoref{cha:sec:fcs-def:progressive-landman:alg:branches}
    \begin{itemize}
    \item
      An algorithm to identify alternative worlds in which \(g\) happens.

      Builds on the \emph{consequent} of Clause~\ref{Landman:CB:stops} (and~\ref{Landman:CB:stops:again})

      Adjusted from \citeauthor{Landman:1992wh}'s suggestions in two significant ways:
      \begin{itemize}[noitemsep]
      \item
        First, to avoid some issues with \citeauthor{Landman:1992wh}'s account of what is reasonable.
      \item
        Second, to allow multiple branches.
      \end{itemize}
    \end{itemize}
  \end{itemize}
  Combined:
  \begin{itemize}
  \item
    \AlgDevelopTree{(\(\text{Tree},e,w,n\))}%
    \hfill%
    \autoref{cha:sec:fcs-def:progressive-landman:alg:tree}
    \begin{itemize}
    \item
      Recursive algorithm to build tree.

      Unifies \AlgAC{}, \AlgGetStops{}, and \AlgFindBranches{}.

      Explicitly recasts the iteration implicit in \citeauthor{Landman:1992wh}'s definition of a continuation branch in terms of recursion.
    \end{itemize}
  \end{itemize}
  Alternative:
  \begin{itemize}
  \item
    \AlgGetPStops{(\(\text{Continuations}, e, w\))}%
    \hfill%
    \autoref{cha:sec:fcs-def:progressive-landman:alg:R-stops}
    \begin{itemize}
    \item
      Variation of \AlgGetStops{}.

      Considers `reasonable' ways in which \(g\) may have stopped.

      Ensures the continuation branch of a progressive does entail the progressive is true because the agent will \(\alpha\) in the actual world.
    \end{itemize}
  \end{itemize}
\end{note}

\subparagraph{Continuations}
\label{cha:sec:fcs-def:progressive-landman:alg:conts}

\begin{note}[\AlgAC{}]
  Clause~\ref{Landman:CB:continues} (and~\ref{Landman:CB:continues:again}) of \citeauthor{Landman:1992wh}'s definition of a continuation branch, which characterises the idea of a `continuation stretch' of some event \(g\) in world \(u\).
  We term the algorithm `\AlgAC{}':

  \begin{algorithm}[H]
    \label{PrAl:g-a-c}
    \caption{\AlgAC{}}
    \SetAlgoLined
    \DontPrintSemicolon
    \Input{\(\langle g,u \rangle_{i}\) \hfill An (indexed) event-world pairing}
    \KwResult{Continuation \hfill The continuation of \(g\) in \(u\)}
    \Begin{
      \(\text{Continuation} \longleftarrow \emptyset\)\;
      \label{PrAl:g-a-c:CSetInt}
      \(t_{s} \longleftarrow \text{start time of }g\text{ in }u\)\;
      \label{PrAl:g-a-c:ts}
      \(t_{e} \longleftarrow \text{end time of }g\text{ in }u\)\;
      \label{PrAl:g-a-c:te}
      \(g_{x} \longleftarrow g\)\;
      \label{PrAl:g-a-c:gx}
      \While{\(g_{x}\) is an event in \(u\)}
      {
        \label{PrAl:g-a-c:while:s}
        \(g_{x} \longleftarrow \emptyset\)\;
        \label{PrAl:g-a-c:gx:discard}
        \(t_{e} \longleftarrow t_{e} + 1\)\;
        \label{PrAl:g-a-c:te:plus}
        \(I \longleftarrow [t_{s},t_{e}]\)\;
        \label{PrAl:g-a-c:te:I}
        \For{\(g_{y} \in \{g_{y} \mid g_{y} \text{ is an event in } u\}\)}{
          \label{PrAl:g-a-c:for:s}
          \If{\(g_{y}\) occurs over \(J \subseteq I\) s.t. \(t_{e} \in J\) \emph{and} \(g_{y}\) is a stage of \(g\)}
          {
            \label{PrAl:g-a-c:for:test}
            \(\text{Continuation} \longleftarrow \text{Continuation} \cup \langle \langle g_{x},u \rangle_{i}, \langle g_{y},u \rangle_{i} \rangle\)\;
            \label{PrAl:g-a-c:C:new}
            \(g_{x} \longleftarrow g_{y}\)\;
            \label{PrAl:g-a-c:gx:new}
          }
        }
      }
      \label{PrAl:g-a-c:while:e}
      \Return{\(\text{Continuation}\)}
      \label{PrAl:g-a-c:return}
    }
    {
      \color{red}
      Need to fix this a little, for I want it to be the case that I get multiple events, maybe, and also that events get a choice of where to stop.
    }
  \end{algorithm}

  \AlgAC{} takes an event-world pairing \(\langle g,u \rangle_{i}\) and returns a set containing a (non-branching tree) which captures the continuation stretch of \(g\) in \(u\).

  Intuitively, \AlgAC{} starts with \(\langle g,u \rangle_{i}\).
  Then, \AlgAC{} finds the smallest event \(g^{+}\) such that \(g^{+}\) is a stage of \(g\) in \(u\), and adds \(\langle g,u \rangle_{i}, \langle g^{+},u \rangle_{i} \rangle\) as a continuation.
  Now, \(g\) may develop further in \(u\).
  So, \AlgAC{} continues to the smallest \(g^{++}\) such that \(g^{++}\) is a stage of \(g\) in \(u\) at a later time than \(g^{+}\), and adds \(\langle g^{+},u \rangle_{i}, \langle g^{++},u \rangle_{i} \rangle\) as a continuation.
  The hypothetical set \(\text{Continuation}\) is now \(\{\langle g,u \rangle_{i}, \langle g^{+},u \rangle_{i} \rangle, \langle g^{+},u \rangle_{i}, \langle g^{++},u \rangle_{i} \rangle\}\).

  This process repeats until there are no further stages of \(g\) in \(u\).
\end{note}

\begin{note}[Motivation for \AlgAC{}]
  % There are two pieces of motivation for the construction of \AlgAC{}.

  The construction of \AlgAC{} is motivated \citeauthor{Landman:1992wh}'s account of the progressive.
  For, \(\sem{\text{PROG}(e, P)}_{w,g}\) checks to see if there exists \emph{some} event-world pairing \(\langle f,v \rangle\) in the continuation branch of \(e\) (in \(w\)) such that \(\sem{P}\) is true of \(f\) in \(v\).
  Hence, it is not possible, in general, to only consider the `maximal' continuation of \(g\) in \(u\).
  Rather, we must have the option of identifying the particular stage \(f'\) of \(f\) such that \(P\) is true of \(f'\).

  % The second piece of motivation is the critiques of \textcite{Bonomi:1997uq} and \textcite{Szabo:2004ul} which suggest modifying \citeauthor{Landman:1992wh}'s account of the progressive to include universal quantification over branches of a continuation tree.
  % Therefore, it is important to ensure that the continuation of \(g\) in \(u\) does not itself involve branching.%
  % \footnote{
  %   I.e., this constraint rules out including both \(\langle \langle g,u \rangle_{i}, \langle g^{+},u \rangle_{i} \rangle\) and \(\langle \langle g,u \rangle_{i}, \langle g^{++},u \rangle_{i} \rangle\), as this would indicate a branch.
  % }
\end{note}

\begin{note}[Construction of \AlgAC{}]
  Finally, then, we have the way in which \(\text{Continuation}\) is constructed.

  We start by initialising \(\text{Continuation}\) as an empty set (\autoref{PrAl:g-a-c:CSetInt}).
  Then, we identify the start and end times of the interval in which \(g\) takes place (Lines~\ref{PrAl:g-a-c:ts} and~\ref{PrAl:g-a-c:te}).

  The task is then to continue to expand \(\text{Continuation}\) so long as there are further stages of \(g\) in \(u\).
  We achieve this a while loop %
  (Lines~\ref{PrAl:g-a-c:while:s}--\ref{PrAl:g-a-c:while:e}) %
  that will fail when we are no longer considering an event in \(u\).
  The variable event \(g_{x}\) is initially set to \(g\), to guaranteed at least one pass through the loop (\autoref{PrAl:g-a-c:gx}).

  In a instance of the while loop we first discard \(g_{x}\) to ensure the loop will fail if there are no further stages of \(g\) (\autoref{PrAl:g-a-c:gx:discard}).
  Then, we construct an interval \(I\) by shifting forward one step in time (Lines~\ref{PrAl:g-a-c:te:plus}~and~\ref{PrAl:g-a-c:te:I}).

  At this point, we have a new interval \(I\) to consider, but no immediate event.
  So, we consider every event \(g_{y}\) which occurs in \(u\) and test to see if \(g_{y}\) happens in \(I\) \emph{and} is a stage of \(g\) (Lines~\ref{PrAl:g-a-c:for:s}~and~\ref{PrAl:g-a-c:for:test}).
  If successful, we update \(\text{Continuation}\) and \(g_{x}\) (Lines~\ref{PrAl:g-a-c:C:new}~and~\ref{PrAl:g-a-c:gx:new}).

  Here single assumption:
  \begin{itemize}[noitemsep]
  \item
    If there is some further stage \(g_{z}\) of \(g_{x}\), then there is stage \(g_{y}\) of \(g_{x}\) at an interval obtained by stepping one tick forward in time.%
  \footnote{
    This may be avoided by separating the test on~\autoref{PrAl:g-a-c:for:test} into two separate tests, and shifting the reassignment on~\autoref{PrAl:g-a-c:gx:new} outside the scope of the if-clause.
    Though the while-loop will then only terminate when there are no further events in \(u\).
  }
  \end{itemize}

  After the stages of \(g\) in \(u\) have been exhausted, \AlgAC{} returns \(\text{Continuation}\) (\autoref{PrAl:g-a-c:return}) and terminates.
\end{note}

\subparagraph{Stops}
\label{cha:sec:fcs-def:progressive-landman:alg:stops}

\begin{note}[\AlgGetStops{}]
  To the \emph{antecedent} of Clause~\ref{Landman:CB:stops} (and~\ref{Landman:CB:stops:again}) of \citeauthor{Landman:1992wh}'s definition of a continuation branch asks whether the continuation stretch of \(g\) stops in \(u\), and considers the `maximal' event \(g'\) in \(u\) (if \(g\) stops).

  Intuitively, the algorithm `\AlgGetStops{}' searches through some continuation stretch of \(g\) in \(i\) (i.e.\ the result of \AlgAC{\((\langle g,u \rangle_{i})\)}) to identify the stopping point of \(g\) in \(u\).
  Hence, if \(g\) stops in \(u\) \AlgGetStops{} returns the relevant `maximal' event.
  Otherwise, \AlgGetStops{} does not return an event-world pairing.
  So, by inspecting the result of \AlgGetStops{} we may determine whether the antecedent of Clause~\ref{Landman:CB:stops} (and~\ref{Landman:CB:stops:again}) is fulfilled.

  Strictly, the construction of \AlgGetStops{} is generalised to cover finding the stopping points of multiple continuation stretches, but the same intuition extends to the more general case.

  After identifying the relevant stopping points of an event \(g\) in world \(u\), the following task will be to find continuations of \(g\) in other worlds.

  \AlgGetStops{} is as follows:

  \begin{algorithm}[H]
    \label{PrAl:g-s}
    \caption{\AlgGetStops{}}
    \SetAlgoLined
    \DontPrintSemicolon
    \Input{\(\text{Continuations}\) \hfill I.e.\ results of \AlgAC{} --- in general, a tree}
    \KwResult{Stops \hfill Stopping points for each event in \(\text{Continuations}\)}
    \Begin{
      \(\text{Stops} \longleftarrow \emptyset\)\;
      \label{PrAl:g-s:mk-st-Stops}
      \For{\(\langle \langle f,v \rangle_{i-1}, \langle g,u \rangle_{i} \rangle \in \text{Continuation}\)}
      {
        \label{PrAl:g-s:for:start}
        \If{\(g\text{ stops in }u\)}
        {
          \label{PrAl:g-s:if:start}
          % \(g' \longleftarrow\text{ the stopping point of }g\text{ in }u\)\;
          % \label{PrAl:g-s:max}
          \(\text{Stops} \longleftarrow \text{Stops} \cup \{\langle \langle f,v \rangle_{i-1}, \langle g,u \rangle_{i} \rangle\}\)\;
          \label{PrAl:g-s:make}
        }
      }
      \Return{\(\text{Stops}\)}
      \label{PrAl:g-s:return}
    }
  \end{algorithm}
\end{note}

\begin{note}[Construction of \AlgGetStops{}]
  The construction of \AlgGetStops{} is simple:%
  \footnote{
    \AlgGetStops{} differs in presentation from \citeauthor{Landman:1992wh}.
    Recall:
    \begin{quote}
      \begin{enumerate}
        \setcounter{enumi}{1}
      \item
        if the continuation stretch of \(e\) in \(w\) stops in \(w\), it has a maximal element \(f\) and \(f\) stops in \(w\).
        Consider the closest world \(v\) where \(f\) does not stop:\dots
      \end{enumerate}
    \end{quote}
    Clause~\ref{Landman:CB:continues} reads as an imperative to find the relevant maximal event.
    Still, so long as \AlgGetStops{} is applied to results \AlgAC{}, the result is equivalent, for the set of continuation given by \AlgAC{} will include the relevant maximal event.
  }

  \AlgGetStops{} initialises \(\text{Stops}\) as an empty set (\autoref{PrAl:g-s:mk-st-Stops}), and for every element of \(\text{Continuations}\), \AlgGetStops{} takes the right-event-world pairing and queries whether the event stops in the world (Lines~\ref{PrAl:g-s:for:start}--\ref{PrAl:g-s:if:start}).
  If the event stops, \AlgGetStops{} takes the point at which the event stops and adds the element of \(\text{Continuations}\)%
  % with the maximal continuation substituted for the event
  to \(\text{Stops}\) (\autoref{PrAl:g-s:make}).
  \AlgGetStops{} returns \(\text{Stops}\) (\autoref{PrAl:g-s:return}) and terminates.
\end{note}

\begin{note}
  Applied to a single instance of \AlgAC{}, \AlgGetStops{} will either return an empty set or a singleton set containing a pairing in which the event of the right element stops.
  And, applied to a set containing multiple results of \AlgAC{}, \AlgGetStops{} will either return an empty set or a set of size bounded by the instances of \AlgAC{}.
\end{note}

\begin{note}
  Role of \AlgGetStops{} is somewhat artificial.
  It would be far more effective to identify a stopping event in the construction of \AlgAC{}.

  However, after introducing \AlgFindBranches{} to find the continuations of a stopping event in some possible world (\autoref{cha:sec:fcs-def:progressive-landman:alg:branches}) we will introduce variant of \AlgGetStops{} (\autoref{cha:sec:fcs-def:progressive-landman:alg:R-stops}).
\end{note}

\subparagraph{Branches}
\label{cha:sec:fcs-def:progressive-landman:alg:branches}

\begin{note}
  The \emph{consequent} of Clause~\ref{Landman:CB:stops} (and~\ref{Landman:CB:stops:again}) of \citeauthor{Landman:1992wh}'s definition of a continuation branch requires identifying the continuation of an event \(g\) which stops in \(u\) in some possible world \(u'\), if the continuation exists.

  The algorithm `\AlgFindBranches{}' is designed to return the possible worlds in which \(g\) exists, and the continuation of \(g\) in the possible worlds may then be obtained via \AlgAC{}.

  The primary use-case is applying \AlgFindBranches{} to the result of \AlgGetStops{}.
\end{note}

\begin{note}[\AlgFindBranches{} alg]
  \AlgFindBranches{} is as follows:

  \begin{algorithm}[H]
    \label{PrAl:find-branches}
    \caption{\AlgFindBranches{}}
    \SetAlgoLined
    \DontPrintSemicolon
    \Input{\(\langle \langle f,v \rangle_{i-1}, \langle g,u \rangle_{i}\rangle\) \hfill An element of a tree\\
      \(e\) \hfill An event \\
      \(w\) \hfill A world
    }
    \KwResult{Branches \hfill A set containing event-world pairs \\
      \hfill such that \(g\) is a stage of \(e\) in \(u'\)}
    \Begin{
      \textcolor{comment}{\texttt{//} \(\text{CloseWorlds} \longleftarrow \{u' \mid u' \text{ is the closest world to } u\}\)}\;
      \label{PrAl:find-branches:close:Landman}
      \(\text{CloseWorlds} \longleftarrow \{u' \mid u' \text{ is among the closest worlds to } u\}\)\;
      \label{PrAl:find-branches:close}
      \textcolor{comment}{\texttt{//} \(\text{CloseWorlds} \longleftarrow \text{CloseWorlds} \cap \lRwe{w}{e}\)}\;
      \label{PrAl:find-branches:R}
      \(\text{Branches} \longleftarrow \emptyset\)\;
      \For{\(u' \in \text{CloseWorlds}\)}{
        \label{PrAl:find-branches:loop:start}
        \If{\(g\) is an event in \(u'\)}{
          \If{\(g \in \mRwe{w}{e}\)}
          {
            \label{PrAl:find-branches:Rprime:check}
            \(\text{Branches} \longleftarrow \text{Branches} \cup \{ \langle \langle f,v \rangle_{i-1}, \langle g,u' \rangle_{i}\rangle \}\)\;
          }
          \label{PrAl:find-branches:Rprime:check:end}
        }
      }
      \label{PrAl:find-branches:loop:end}
      \Return{\(\text{Branches}\)}
    }
  \end{algorithm}
\end{note}

\begin{note}
  From a broad perspective, \AlgFindBranches{} considers some event-world pairing \(\langle g,u \rangle\) and returns a collection of event-world pairings \(\langle g,u' \rangle\) where \(u'\) is some possible world.

  Still, \AlgFindBranches{} differs from \citeauthor{Landman:1992wh}'s method of identifying branches.
  For, \citeauthor{Landman:1992wh} identifies branches by considering possible worlds which are both close to \(u\) and elements of \(\lRwe{w}{e}\).
  \AlgFindBranches{}, by contrast, identifies branches by considering possible worlds that are close to \(u\) and events which are elements of \(\mRwe{w}{e}\).%
  \footnote{
    \label{fn:Alg:branches:getLandman}
    The following changes to \AlgFindBranches{} may be made obtain \citeauthor{Landman:1992wh}'s account:
    \begin{enumerate}[label=\arabic*., ref=(\arabic*), noitemsep]
    \item
      Uncomment \autoref{PrAl:find-branches:close:Landman} and comment \autoref{PrAl:find-branches:close} to retain \citeauthor{Landman:1992wh}'s assumption of a (unique) closest world to \(u\).
    \item
      Uncomment Line~\ref{PrAl:find-branches:R}, to restrict the possible worlds via \citeauthor{Landman:1992wh}'s \(\lRwe{w}{e}\).
    \item
      Comment Lines~\ref{PrAl:find-branches:Rprime:check} and~\ref{PrAl:find-branches:Rprime:check:end} to remove the restriction of events via our \(\mRwe{w}{e}\).
    \end{enumerate}
  }
  However, though \AlgFindBranches{} differs in method, \AlgFindBranches{} does not differ is spirit.

  Begin by identifying role.
  Highlight difficulty in achieving role.
  Consider \citeauthor{Landman:1992wh}'s \(\lRwe{w}{e}\).
  Provide variation \(\mRwe{w}{e}\).
\end{note}


\begin{note}
  There are two parts to role of \AlgFindBranches{}:
  \begin{enumerate}[label=\alph*., ref=(\alph*), noitemsep]
  \item
    Identify ways \(g\) may have come about, so \(g\) may continue.
  \item
    Ensure that the way in which \(g\) comes about is reasonable with respect to the initial event \(e\) in the world of origin.
  \end{enumerate}

  First part, then second part.
\end{note}

\begin{note}
  First part, shift to possible world.
  Hence, result of \AlgFindBranches{} is a collection of pairings such that \(g\) has happened in some possible world \(u'\).
  Note, \AlgFindBranches{} does not consider any further continuation of \(g\).%
  \footnote{
    Continuations of \(g\) in \(u'\) may (and will) be obtained via calling \(\AlgAC{(\langle g,u' \rangle_{i})}\).
  }

  Motivation for the first part follows from \autoref{assu:prog-modal-shift} and the observation that an agent may be \(\alpha\)ing though the agent does not proceed to \(\alpha\).

  Consider the following scenario:%
  \footnote{
    Variation on both:
    \begin{enumerate*}[label=(\alph*), ref=(\alph*)]
    \item
      \citeauthor{Landman:1992wh}'s scenario given on~\autopageref{cha:sec:fcs-def:progressive-landman} which involves Max walking across the road.
    \item
      \citeauthor{Schwarz:2020aa}'s (\citeyear{Schwarz:2020aa}) on to motivate \AbControl{} given above on~\autopageref{Schwarz:pi}.
    \end{enumerate*}
  }

  \begin{scenario}[State \(\pi\)]
    \label{scen:prog:Cyril:know}
    Cyril has been asked to state the first ten digits (of the decimal expansion) of \(\pi\).
    Cyril knows the first ten digits of \(\pi\).
    Cyril says `\(3\)' followed by `\(1\)'.
    However, interrupted a bird flying into the classroom through the open window.
  \end{scenario}
  Intuitively, as Cyril says `\(1\)' it is true that:
  \begin{enumerate}
  \item
    \label{Cyril:pi:progressive}
    Cyril is stating the first ten digits of \(\pi\).
  \end{enumerate}
  For, as with Max crossing the road, it seems if Cyril has not been interrupted Cyril would have gone on to say `\(4\)' followed by `\(2\)', etc.

  Stopping point is so that Cyril is not interrupted.
  Then, shift to a close world.
  For progressive to be true, differs enough so that as the event continues Cyril is not interrupted.

  However, incrementally.
  Possible world where Max is not hit by the bus because birds do not exist.
  Partial solution by looking at close worlds.
  Though an incomplete solution.
  For example, closest world where Cyril is not interrupted by the bird, Cyril says `\(4\)' and \emph{then} the bird flies through the window.
  Complete solution by looking at closest worlds relative world \(u\) in which \(g\) takes place, rather than the initial world \(w\).
  Term this process `\emph{drift}'.
  After drifted from initial world \(w\), so avoided first bird, may drift further from \(w\) to avoid second bird, and so on for any other birds flying toward the open window of the classroom.

  Closeness is difficult to specify, but sufficiently common tool that this does not impact sufficient understanding of the progressive.
  Intuitively, not possible to drift to non-existence of birds, as no close world in which birds exist a little less than they do in the actual world to get the process of drifting stated.
\end{note}

\begin{note}
  Still, though drift may fail to start with respect to certain possibilities, one drift starts, the result of drifting may lead to bad results.
  Worry is cases in which examining the continuation tree suggests instance of the progressive that is (intuitively) false is marked as true.

  To illustrate, consider the following variation on \autoref{scen:prog:Cyril:know}:

  \begin{scenario}[Not stating \(\pi\)]
    \label{scen:prog:Cyril:know:not}
    Cyril has been asked to state the first ten digits (of the decimal expansion) of \(\pi\).
    Cyril does \emph{not} know the first ten digits of \(\pi\).
    However, Cyril picks digits at random and says `\(3\)' followed by `\(1\)'.
    After saying `\(1\)' a bird flies into the classroom through the open window and interrupts Cyril.
  \end{scenario}

  \autoref{scen:prog:Cyril:know:not} mirrors \autoref{scen:prog:Cyril:know} with the exception that Cyril does \emph{not} know the first ten digits of \(\pi\) and instead guesses digits at random.

  As Cyril says `\(1\)' it is intuitively \emph{not} true that:
  \begin{enumerate}
  \item
    \label{Cyril:pi:progressive:again}
    Cyril is stating the first ten digits of \(\pi\).
  \end{enumerate}

  However, Cyril's choice of digits is random.
  So, it seems the worlds in which Cyril says `\(n\)' for \(0 \leq n \leq 9\) are all equally close.
  Therefore, there is a close world in which Cyril says `\(4\)'.
  Following, Cyril will continue to make a random choice and so there will be a close world to the world in which Cyril says `\(4\)' in which Cyril says `\(2\)', and so on.
  Hence, we may drift through a sequence of possible worlds until Cyril says the first ten digits of \(\pi\).%
  \footnote{
    And, indeed any finite sequence of digits from (the decimal representation of) \(\pi\).
    See \citeauthor{Landman:1992wh} (\citeyear[\S2.3]{Landman:1992wh}) for additional examples and discussion.
  }
  Yet, it seems \label{Cyril:pi:progressive:again} would be true, given \citeauthor{Landman:1992wh}'s account of the progressive.%
  \footnote{
    Strictly, \citeauthor{Landman:1992wh} assumes the existence of a closest world and hence it is not clear how to evaluate \autoref{scen:prog:Cyril:know:not} on \citeauthor{Landman:1992wh}'s account.

    \citeauthor{Landman:1992wh} states the assumption is made `for ease' (\citeyear[26]{Landman:1992wh}).
    However, \citeauthor{Landman:1992wh} does not clarify what the assumption eases.
    Therefore we assume for help in clarifying the current concern regarding branches that \citeauthor{Landman:1992wh}'s existential approach to the progressive would extend to continuations trees.
  }
\end{note}

\begin{note}
  At this point let us take a brief pause to observe the role of \autoref{PrAl:find-branches:close}.
  Here, \AlgFindBranches{} considers the closest worlds to \(u\).
  And, this line contrasts to \citeauthor{Landman:1992wh}'s assumption of a unique closest world which would be obtained by switching \autoref{PrAl:find-branches:close} with \autoref{PrAl:find-branches:close:Landman}.
  The change is motivated by the equally close worlds in which Cyril says `\(n\)' for \(0 \leq n \leq 9\) motivate entertaining, though is further motivated by considerations pressed by~\textcite{Bonomi:1997uq}.%
  \footnote{
    See in particular \citeauthor{Bonomi:1997uq}'s discussion of the `multiple-choice paradox' (\citeyear[\S4]{Bonomi:1997uq}) and (\cite[37]{Szabo:2004ul}).
  }
  This change, gives rise to the possibility of a continuation tree rather than a continuation branch.
\end{note}

\begin{note}
  Now, with a continuation \emph{tree} it is not clear that \autoref{scen:prog:Cyril:know:not} presents a difficulty.
  For, given the worlds in which Cyril says `\(n\)' for \(0 \leq n \leq 9\) are all equally close, there will be a branch of the continuation tree in which Cyril says `\(9\)', and as `\(9\)' is not the third digit of \(\pi\) it will not be the case that Cyril states the first ten digits of \(\pi\) on every branch of the continuation tree.

  However, a parallel difficulty arises.
  For, consider the following sentence:
  \begin{enumerate}
  \item
    \label{Cyril:pi:progressive:random}
    Cyril is randomly guessing the first ten digits of \(\pi\).
  \end{enumerate}
  \ref{Cyril:pi:progressive:random} seems true with respect to\autoref{scen:prog:Cyril:know:not}.
  However, consider the same sequence of drifting, such that Cyril utters the first ten digits of \(\pi\).
  Consider the last instance of the drift.
  Cyril has uttered the first nine digits of \(\pi\) without mistake.
  Consider \(w_{9}\), an arbitrary world from this collection of worlds.

  For \ref{Cyril:pi:progressive:random} to be true, it needs to be the case that in all the closest worlds to \(w_{9}\) in which Cyril (correctly) says the tenth digit (\(3\)), Cyril does so by randomly guessing.
  Now, Cyril started by guessing randomly.
  However, we have drifted a significant way from the actual world and Cyril has correctly said the first ten digits of \(\pi\).
  Hence, it is not clear that Cyril guesses randomly in all the closest worlds to \(w_{9}\).
  It seems plausible that Cyril did not guess randomly in at least one of the closest worlds to \(w_{9}\).

  Hence, on \citeauthor{Landman:1992wh}'s existential approach get something intuitively false being true.
  But, on \citeauthor{Szabo:2004ul}'s universal approach get something intuitively true being false.

  Broader perspective, just as things separate from event may stop event in progress, things may sustain an event.
\end{note}

\begin{note}
  Hence, we turn to the second part of the role of~\AlgFindBranches{}:
  \begin{enumerate}[label=\arabic*., ref=(\arabic*), noitemsep]
  \setcounter{enumi}{1}
  \item
    Ensure that the way in which \(g\) comes about is reasonable with respect to the initial event \(e\) in the world of origin \(w\).
  \end{enumerate}
  Given that Cyril knows the first ten digits of \(\pi\) in \autoref{scen:prog:Cyril:know} the drift through possible worlds is `reasonable'.
  By contrast, as Cyril does not know the first ten digits of \(\pi\) in \autoref{scen:prog:Cyril:know:not} the drift through possible worlds is `unreasonable'.

  Term this `\emph{regulated drift}'.
  Regulated by initial event and world.

  One way to think, property, at some point gets lost.
  Different way, property of the sequence as a whole.
  Comparison or relation between \(e\) in \(w\) and \(g\) in \(u\).
\end{note}

\begin{note}
  \citeauthor{Landman:1992wh}'s approach to regulating drift is to:
  \begin{enumerate*}[label=\roman*., ref=(\roman*)]
  \item
    intersect possible worlds with `reasonable' worlds with respect to the initial event \(e\) and word \(w\), and
  \item
    restrict drift to the intersection.%
  \footnote{
    Compare the consequent of Clause~\ref{Landman:CB:stops} (and~\ref{Landman:CB:stops:again}) of \citeauthor{Landman:1992wh}'s definition to \autoref{PrAl:find-branches:R} of \AlgFindBranches{}.
    See also \autoref{fn:Alg:branches:getLandman} on \autopageref{fn:Alg:branches:getLandman} for the full edit to \AlgFindBranches{}.
  }
  \end{enumerate*}

  The difficulty with \citeauthor{Landman:1992wh}'s proposal is that it is unclear what it is for a world to be `reasonable' with respect to \(e\) and \(w\).

  Recall, \citeauthor{Landman:1992wh} provided the following account:
  \begin{quote}
    \(v \in \lRwe{w}{e}\) iff there is a reasonable chance on the basis of what is internal to \(e\) in \(w\) that \(e\) continues in \(w\) as far as it does in \(v\).%
    \mbox{ }\hfill\mbox{(\citeyear[26]{Landman:1992wh})}
  \end{quote}
  An immediate difficulty is figuring out what makes for a `reasonable' chance.
  Intuitively, Cyril stating the correct sequence of digits in \autoref{scen:prog:Cyril:know:not} (where Cyril does not know the digits of \(\pi\)) would require considerable luck.%
  \footnote{
    If ten digits seems borderline, extend the sequence.
  }\(^{,}\)%
  \footnote{
    Following \citeauthor{Landman:1992wh}, also highlights no way to reduce to similarity between worlds.
    For, \citeauthor{Landman:1992wh} holds that unreasonable even though what happened.
    Suppose Cyril hadn't been interrupted and chooses correctly.
    Still not reasonable change when starting.
  }
  However, it is by no means clear that the chance of the bird \emph{not} flying into the classroom in~\autoref{scen:prog:Cyril:know} should be significantly different in chance to Cyril stating the digits of \(\pi\).%
  \footnote{
    Given a sufficiently determined bird and large enough window, the chance of the bird not flying into the classroom is likewise low.
  }
  Hence, it seems that a what captures `reasonable' chance is not some threshold.%
  \footnote{
    This objection parallels the issues~\citeauthor{Foley:2009vl} (\citeyear{Foley:2009vl}) raises with respect to the Lockean thesis (i.e.~the thesis that qualitative belief may be identified with quantitative/probabilistic belief above some threshold).
  }
  Further, even if `reasonable' chance is understood, \citeauthor{Landman:1992wh} appeals --- in part to address issues of the kind presented --- to `reasonable' chance \emph{on the basis of} what is internal to \(e\) in \(w\).
  Hence, of interest is not the chance of \(g\) happening in \(w\) simpliciter.
  And, unfortunately, \citeauthor{Landman:1992wh} does not elaborate on the distinction between what is internal or external to an event.%
  \footnote{
    And is criticised for this\dots \textcite[35]{Szabo:2004ul} \textcite[203,fn.2]{Bonomi:1997uq} \textcite[49--50]{Engelberg:1999vi}
  }
  Perhaps it is possible to suitably expand \citeauthor{Landman:1992wh}'s definition of \(\lRwe{w}{e}\).
  However, we will not pursue \citeauthor{Landman:1992wh}'s \(\lRwe{w}{e}\) further.
  Instead, we turn to a substitute which appeals to the idea of closeness that allowed for drift.
\end{note}

\begin{note}
  Our proposal in short, is that \(g\) is a \emph{reasonable continuation} of an event \(e\) in world \(w\) just if case, in some close world to \(w\) where \(g\) happens, \(g\) is a stage of \(e\).
  Paraphrased:
  It is the subjunctive conditional `if \(g\) were to happen \(g\) would be a development of \(e\)' is true when \(e\) happens in \(w\).

  Broadly, our proposal is to indirectly regulate the development of \(e\) as the event develops and drifts through possible worlds by ensuring that \(e\) remains (among) the most plausible ways for the development of \(e\) to happen.%
  \footnote{
    Visually, consider drifting through a Lewisian system of spheres.
    Consider the event.
    No shorter path.
  }

  Go back to Cyril.
  Separate at choice of number(s).
  For the progressive to be true, need it to be the case that Cyril says the first ten digits of \(\pi\).
  Consider the event of saying the first ten digits.
  Continues to guess correctly is a more distant possibility than Cyril is provided information and result is from distinct event.
  So, existential.

  Now, problem for our account.
  Cyril ends up knowing the first ten digits.
  However, seems that guessing is at least as good as knowing form the perspective of the actual world.

  The suggestion developed into a definition is as follows:
  \begin{definition}[Reasonable continuation of event]
    \label{def:myRwe}
    For an events \(e,g\) and a world \(w\):
    \begin{itemize}
    \item
      \(g \in \mRwe{w}{e}\). \hfill (\(g\) is a \emph{reasonable continuation} of event \(e\) in world \(w\).)
    \end{itemize}

    \emph{If and only if}

    \begin{itemize}
    \item
      The following conditional is true when evaluated from \(w\):
      \begin{itemize}
      \item
        If \(g\) happens, then it is not the case that \(g\) is not a stage of \(e\).
      \end{itemize}
    \item
      In other words:
      In some world close to \(w\) in which \(g\) happens, \(g\) is a stage of \(e\).
    \end{itemize}
    \vspace{-\baselineskip}
  \end{definition}

  Difficulty is given \(g\).
  Whether rule out \(g\) regardless of drifting to \(g\) and there being no more plausible way for \(g\) to have occurred.
  At issue is whether progressive may be true.

  \citeauthor{Engelberg:1999vi} (\citeyear{Engelberg:1999vi}), modified from~\textcite[475]{Asher:1992ug}:
  \begin{quote}
    Rebecca stood in front of a huge minefield, started walking, and walked about 50 yards into the minefield.%
    \mbox{ }\hfill\mbox{(\citeyear[49]{Engelberg:1999vi})}
  \end{quote}
  Reports most speakers find it acceptable to say `Rebecca is walking to the other side of the minefield'.
  (\citeyear[49]{Engelberg:1999vi}).
  Drift across possible worlds.
  Then, suppose Rebecca has crossed to the other side.
  Still the case that walking is most plausible.

  45.72 meters is quite far.
  Indeed, Cyril says the sixth digit, things get more difficult.
  As event develops, what is close to \(w\) changes.

  I am hesitant.
  However, need to find drift, so that still \(e\) in all branches, but doesn't intuitively seem like a case of the progressive.
  \(e\) has happened in \(w\).
  Possible to develop into \(g\), via drifting, and build up of drift 

  Important here is that we build the continuation tree and the events may be separate from \(e\).
  If include \(e\), then subjunctive is trivial.
  Of course, \(g\) may trivially be a stage of \(e\) by including \(e\).
  However, if separable, then possible to break.
  Consider how the progressive works in general.
  \citeauthor{Landman:1992wh} is existential.
  Account we have is universal-existential.
  For every branch.
  Hence, if separate, then failure on that branch will be sufficient to mark the instance of the progressive as false.

  .%
  \footnote{
    And, swimming, reaching the shore.
    But, it might also be the case that the seas are very kind of the particular day.
  }\(^{,}\)

  Also, universal.
  \(\mRwe{w}{e}\) isn't suitable for \citeauthor{Landman:1992wh}'s account of the progressive.

  Really part of \AlgAC{}.
  But better placed here due to function of respective algorithms.
  In particular, as, we won't include in tree if not returned by this algorithm.
\end{note}

\begin{note}
  Of course, if we're looking at \AlgFindBranches{} applied to \(\langle e,w \rangle\), then these two things (plausibly) coincide.
  Close world is such that continues only if same event.
  However, constructing the tree is an recursive process.
  \AlgFindBranches{} may be called multiple times.
\end{note}


\begin{note}
  Motivated restriction on world.
  What about converse?
  Is it possible to consider only the `reasonable' worlds?

  With respect to \(\mRwe{w}{e}\) no.
  For, requires \(g\), which is only obtained by drifting through worlds.

  With respect to \citeauthor{Landman:1992wh}'s \(\lRwe{w}{e}\), less clear.
  Though, no further speculation.
\end{note}

\begin{note}
  See if the event at some world continues in some other world.
  
  Note, importance of \(e\) and \(w\)!
  No shift in index, as the function identifies the same event in some other possible world.
\end{note}

\begin{note}
  Two general notes:

  \citeauthor{Landman:1992wh}, existential, so risk of finding some continuation in `unreasonable' worlds.
  Now suggested universal, then failure in an `unreasonable' world does not entail failure of the progressive.
\end{note}

\begin{note}
  Here, shift the index.
  When run recursive algorithm, each call of the algorithm will explore branches.
  Increase index for next call.

  Also, note, \AlgGetPStops{}.
  Pass the result of \AlgGetPStops{}, and this will get 
\end{note}

\subparagraph{Reasonable stops}
\label{cha:sec:fcs-def:progressive-landman:alg:R-stops}

\begin{note}
  In \autoref{cha:sec:fcs-def:progressive-landman:alg:R-stops} we gave a reconstruction of the antecedent of Clause~\ref{Landman:CB:stops} (and~\ref{Landman:CB:stops:again}) of \citeauthor{Landman:1992wh}'s definition of a continuation branch.

  In this section, we motivate and detail an alternative algorithm.

  The basic observations are
  \begin{enumerate}[label=\arabic*., ref=(\arabic*), noitemsep]
  \item
    \label{landman:alg:R-stops:ob:1}
    If an agent performs some action \(\alpha\), then so long as \(\alpha\) is not instantaneous, there is some prior event such that the agent is \(\alpha\)ing.
  \item
    \label{landman:alg:R-stops:ob:2}
    It is not the case that Observation~\ref{landman:alg:R-stops:ob:1} holds, in general.
  \end{enumerate}
  To resolve this tension, we consider a modification of \AlgGetStops{} which returns `reasonable' rather than actual stops of an event.
  We term the modification `\AlgGetPStops{}'.

  First, we substantiate Observations~\ref{landman:alg:R-stops:ob:1} and~\ref{landman:alg:R-stops:ob:2}.
\end{note}

\begin{note}
  Observation~\ref{landman:alg:R-stops:ob:1} follows from  Clause~\ref{Landman:CB:stops} (and~\ref{Landman:CB:stops:again}) of \citeauthor{Landman:1992wh}'s definition of a continuation branch.
  For, Clause~\ref{Landman:CB:stops} (and~\ref{Landman:CB:stops:again}) is a conditional:
  \begin{quote}
    \begin{enumerate}
      \setcounter{enumi}{1}
    \item
      if the continuation stretch of \(e\) in \(w\) stops in \(w\), it has a maximal element \(f\) and \(f\) stops in \(w\).
      Consider the closest world \(v\) where \(f\) does not stop: \dots
    \end{enumerate}
  \end{quote}
  Hence, for an event \(e\) and world \(w\), Clause~\ref{Landman:CB:stops} only includes the closest world where \(e\) does not stop \emph{if} \(e\) stops in \(w\).
  So, if \(e\) does not stop in the actual world, and \(e\) develops into an event \(e'\) such that the agent \(\alpha\)s in \(e'\), then it will be true that the agent is \(\alpha\)ing with respect to \(e\).
\end{note}

\begin{note}
  We now turn to Observation~\ref{landman:alg:R-stops:ob:2}.

  Observation~\ref{landman:alg:R-stops:ob:2} is the converse of the above%
  \footnote{
    See the discussion of the imperfective paradox on~\autopageref{imperfective-paradox:intro}.
  }
  observation that an event in which an agent \(\alpha\)s does not need to happen for it to be true that the agent is \(\alpha\)ing.
  For, restated, Observation~\ref{landman:alg:R-stops:ob:2} reads:%
  \footnote{
    \citeauthor{Landman:1992wh} considers a similar observation under the term `the problem of non-interruptions' (\citeyear[14--17,30--31]{Landman:1992wh}).
    \citeauthor{Landman:1992wh}'s suggestion is to distinguish events by the perspective of the agent. (\citeyear[31]{Landman:1992wh})
    The basic idea, as I understand \citeauthor{Landman:1992wh} is to deny that the event \(e'\) in which the agent \(\alpha\)s is sufficiently distinct from any prior event \(e\).
    Hence, it seems that strictly speaking \citeauthor{Landman:1992wh} does not endorse Observation~\ref{landman:alg:R-stops:ob:2}.
  }
  \begin{enumerate}[label=\arabic*\('\)., ref=(\arabic*\('\))]
    \setcounter{enumi}{1}
  \item
    There are cases in which \(e\) is not an instance of an agent \(\alpha\)ing, though \(e\) develops into \(e'\) and \(e'\) is an event in which the agent \(\alpha\)s.
  \end{enumerate}

    We present a pair of \illu{1} with both provide instances of Observation~\ref{landman:alg:R-stops:ob:2}.
    Following, we turn to theoretical motivation.
\end{note}

\begin{note}
  \begin{illustration}[Flipping coins]
    \label{illu:fc:coins}
    Agent is flipping a coin in the air and recording how it lands.%
    \footnote{
      Cf.~\textcite{Gelman:2002ww} and~\textcite{Keller:1986tz}.
    }
    The coin lands heads ten times in a row.

    The following seem true:
    \begin{itemize}
    \item
      The coin landed heads ten in a row.
    \item
      The agent landed the coin heads ten times in a row.
    \end{itemize}
    However, the following seem false:
    \begin{itemize}
    \item
      The coin was landing heads ten times in a row.
    \item
      The agent was landing a coin heads ten times in a row.
    \end{itemize}
    \vspace{-\baselineskip}
  \end{illustration}
  Intuition is clear.
  No time prior to the coin landing heads on the tenth flip was the coin landing heads ten times in a row.
  Consider prior.
  \begin{itemize}
  \item
    I am landing this coin heads ten times in a row.
  \end{itemize}
  Only reasonable interpretation is if the agent will continue to keep flipping the coin until the coin lands heads ten times in a row.
  However, if the agent is limited to ten flips, absurd.

  Though, the tenth flip is a clear continuation of the event containing the previous nine coin flips.

  Possible objection.
  Event was not the result of the agent.
  And, if agent then works out because prior to agent completing action, progressive.
  However, the action is flipping the coin so that it lands heads.
  No different from encrypting a message so that it will not be understood by the messenger.
  Once sent, the agent has no involvement in the messenger's understanding.
  Though, given a sufficiently strong algorithm, it seems true that the agent is encrypting the message when working through the algorithm.
\end{note}

\begin{note}[Second]
  \begin{illustration}[Chess III]
    \label{illu:fc:chess:III}
    Consider as the game state on the left transitions to the game state on the right.

    \noindent\mbox{ }\hspace{2em}%
    \begin{adjustbox}{minipage=.4\linewidth,scale=.75}%
      \centering%
      \newchessgame[%
      setwhite={ka6,pa5,pb6,pc7,rg2,bh1},%
      addblack={ka8,rg8,na3},%
      ]%
      \setchessboard{showmover=false}%
      \chessboard%
    \end{adjustbox}%
    \hfil%
    \begin{adjustbox}{minipage=.4\linewidth,scale=.75}%
      \centering%
      \newchessgame[%
      setwhite={ka6,pa5,pb6,pc7,rg2,bh1},%
      addblack={ka8,rc8,na3},%
      ]%
      \setchessboard{showmover=false}%
      \chessboard%
    \end{adjustbox}%
    \hspace{2em}\mbox{ }

    On the left, white has just moved their pawn from c6 to c7.
    Though, white is quite bad at chess.
    And, the strength of their pieces on the board is the result of black intentionally playing even worse.
    So, it is not true that:
    \begin{itemize}
    \item
      White is winning the game of chess.
    \end{itemize}
    For example, if black moves their rook to h8 then on the following turn white may move their pawn to c8 and exchange it for a rook which black may then capture, etc.\dots

    Still, on the following turn, black moves their rook from g8 to c8, as shown on the right.
    As a result:
    \begin{itemize}
    \item
      White wins the game of chess.
    \end{itemize}
  \end{illustration}

  May argue about the interpretations of `wins'.
  For, white still needs to make a move.
  However, any move made by white results in checkmate for black.
  Hence, true.

  In contrast to \autoref{illu:fc:coins}, \autoref{illu:fc:chess:III} does not involve luck.
  Instead, \autoref{illu:fc:chess:III} involves the action by some other agent which leads to desired outcome.
  %
  \footnote{
    More generally, one consider the structure of \autoref{illu:fc:chess:III} applied to any competitive activity in which a mistake by one participant hands victory to the other participant.
    It seems clear to me that a win does not entail that the participant was winning at any point throughout the game.
    E.g.\ a racing car driver may be handed a win due to their competitor having a bad pit stop, or a football team may be handed a win due to an unexpected injury on the other team, etc.
  }
\end{note}

\begin{note}[Theoretical motivation]
  On the one hand, intuition.
  On the other hand, motivated by \citeauthor{Landman:1992wh}'s analysis.
  Distinction between event and world.
  Interested in the event getting to completion regardless of what happens in the world.
  Hence, allows us to shift to possible worlds.
  Progressive is not false because of contingencies of the actual world.
  By parallel, progressive is not true because of contingencies of actual world.
\end{note}

\begin{note}
  Given this, understand both \illu{1}.
  Deviance.

  Once noted, plausible to distinguish where in event progressive comes true.
  Final \illu{}.
  \begin{illustration}[Choosing cards]
    An agent is presented with a shuffled deck of playing cards and is asked to choose a number \(1 < n \leq 52\).
    After the agent announces the number, the presenter of the deck of cards will remove \(n - 1\) cards from the top of the deck and reveal the \(n\)th card.

    The agent considers their options with gravitas.
    \begin{itemize}
    \item
      First, the agent decides they will choose and odd number
    \item
      Second, the agent decides they will choose a card from the top half of the deck.
    \item
      Third, the agent decides they will choose a prime number.
    \item
      Finally, the agent chooses \(17\).
    \end{itemize}
    The agent is unaware that the pack is shuffled so that every card corresponding to a prime number is a red card, though the \(9\)th card isn't a red card.
    Hence, after the third choice, but not before, it is true that:
    \begin{itemize}
    \item
      The agent is choosing a red card.
    \end{itemize}
    \vspace{-\baselineskip}
  \end{illustration}
  So, before third decision, agent may have chosen a non-prime number.
  In particular, \(9\) is not a prime number and the \(9\)th card is not a red card.
  However, after the decision, no turning back.
\end{note}

\begin{note}
  So, Observations~\ref{landman:alg:R-stops:ob:1} and~\ref{landman:alg:R-stops:ob:2}.

  The suggestion is simple.
  Rather than search the continuation of an event \(f\) in world \(v\) 

  Slight difficulty is closest world.
  Instead, closest worlds.
  Intuitively, obtained from counterfactuals against what happened in the actual world.
  However, reasonable, as these are worlds in which the event continues.
\end{note}

\begin{note}
  Modification builds on principles for \AlgFindBranches{}.
  For, \AlgFindBranches{} finds continuation of event in nearby world.


  \AlgGetPStops{}:

  \begin{algorithm}[H]
    \label{PrAl:g-s}
    \caption{\AlgGetPStops{}}
    \SetAlgoLined
    \DontPrintSemicolon
    \Input{%
      \(\text{Continuation}\) \hfill I.e.\ the result of \AlgAC{} --- in general, a tree \\
      \(e\) \hfill An event \\
      \(w\) \hfill A world
    }
    \KwResult{ReasonableStops \hfill \emph{Reasonable} stopping points \\
      \mbox{ } \hfill for each event in \(\text{Continuation}\)}
    \Begin{
      \(\text{R-Stops} \longleftarrow \emptyset\)\;
      \For{\(\langle \langle f,v \rangle_{i-1}, \langle g,u \rangle_{i} \rangle \in \text{Continuation}\)}
      {
        \(\text{Branches} \longleftarrow \AlgFindBranches{(\langle \langle f,v \rangle_{i-1}, \langle g,u \rangle_{i} \rangle, e,w)}\)\;
        \For{\(\langle \langle f,v \rangle_{i-1}, \langle g,u' \rangle_{i} \rangle \in \text{Branches}\)}
        {
          \If{\(g\text{ stops in }u'\)}
          {
            \(\text{R-Stops} \longleftarrow \text{R-Stops} \cup \{\langle \langle f,v \rangle_{i-1}, \langle g,u' \rangle_{i} \rangle\}\)\;
          }
        }
      }
      \Return{\(\text{R-Stops}\)}
    }
  \end{algorithm}

  \AlgGetPStops{} is a variation on \AlgGetStops{}.

  The for-loop from \AlgGetStops{}, repeated, and check from \AlgGetStops{} is repeated.
  However, check within a for-loop for whether there is a `reasonable world' in which the event of some event-world pairing stops.
\end{note}

\begin{note}
  Note, branch.
  However, only if result of \AlgGetPStops{} is part of tree.
  And, antecedent of conditional.
  Branches will be included, but only when calling \AlgFindBranches{}.
  This we now turn to.
\end{note}


\subparagraph{Build tree}
\label{cha:sec:fcs-def:progressive-landman:alg:tree}

\begin{note}
  With \AlgAC{}, \AlgGetStops{}/\AlgGetPStops{}, and \AlgFindBranches{} in hand, we now turn to \AlgDevelopTree{}, a recursive algorithm which completes any tree.

  Focus here is on linking \AlgAC{}, \AlgGetStops{}/\AlgGetPStops{}, and \AlgFindBranches{} in the same way as \citeauthor{Landman:1992wh}, and clarifying the `\dots and we continue as above, etc.' from \ref{Landman:CB:stops:again}.

  \begin{algorithm}[H]
    \label{PrAl:dev-tree}
    \caption{\AlgDevelopTree{}}
    \SetAlgoLined
    \DontPrintSemicolon
    \Input{%
      \(\text{Tree}\) \hfill A partially completed tree\\
      \(e\) \hfill The initial event of the tree\\
      \(w\) \hfill The initial world of the tree\\
      \(n\) \hfill An index to keep track recently added branches\\
    }
    \KwResult{Tree expanded so that there are no further stopping events}
    \Begin{
      \label{PrAl:dev-tree:start}
      \(\text{Stems} \longleftarrow \{ \langle g,u \rangle_{n} \mid \exists f,v \colon \langle \langle f,v \rangle_{n - 1}, \langle g,u \rangle_{n}\rangle \in \text{Tree}\}\)\;
      \label{PrAl:dev-tree:Extend:start}
      \label{PrAl:dev-tree:Extend:Stems}
      \(\text{GrownStems} \longleftarrow \emptyset\)\;
      \label{PrAl:dev-tree:Extend:FreshContsVar}
      \For{\(\langle g,u \rangle_{n} \in \text{Stems}\)}
      {
        \label{PrAl:dev-tree:Extend:Loop:start}
        \(\text{Growth} \longleftarrow \AlgAC{}(\langle g,u \rangle_{n})\)\;
        \(\text{GrownStems} \longleftarrow \text{GrownStems} \cup \text{Growth}\)\;
      }
      \label{PrAl:dev-tree:Extend:end}
      \(\text{Tree} \longleftarrow \text{Tree} \cup \text{GrownStems}\)\;
      \label{PrAl:dev-tree:Extend:merge}
      \textcolor{comment}{\texttt{//} \(\text{Stops} \longleftarrow \AlgGetStops{}(\text{GrownStems})\)}\;
      \label{PrAl:dev-tree:Stops:Land}%
      \(\text{Stops} \longleftarrow \AlgGetPStops{}(\text{GrownStems})\)\;
      \label{PrAl:dev-tree:Stops:Me}
      \(\text{Branches} \longleftarrow \emptyset\)\;
      \label{PrAl:dev-tree:pro-bra:start}
      \For{\(\langle \langle g,u \rangle_{n}, \langle h,u \rangle_{n+1}\rangle \in \text{Stops}\)}
      {
        \label{PrAl:dev-tree:Stops:cond:else:futureB:loop:start}
        \(\text{Temp} \longleftarrow \AlgFindBranches{}(\langle \langle g,u \rangle_{n}, \langle h,u \rangle_{n+1}\rangle, e, w)\)\;
        \label{PrAl:dev-tree:Stops:cond:else:futureB:loop:getBranches}
        \(\text{Branches} \longleftarrow \text{Branches} \cup \text{Temp}\)\;
        \label{PrAl:dev-tree:Stops:cond:else:futureB:loop:gather}
      }
      \label{PrAl:dev-tree:pro-bra:end}
      \eIf{\(\text{Branches} = \emptyset\)}{
        \label{PrAl:dev-tree:Stops:cond:more}
        \textbf{return} Tree\;
        \label{PrAl:dev-tree:Stops:cond:else:futureB:process:cancel}
      }
      {
        \(\text{Tree} \longleftarrow \text{Tree} \cup \text{Branches}\)\;
        \label{PrAl:dev-tree:Stops:cond:else:futureB:process:expand}
        \AlgDevelopTree{}(\(\text{Tree}, e,w, n+1\))\;
        \label{PrAl:dev-tree:Stops:cond:else:futureB:process:end}
      }
      \label{PrAl:dev-tree:Stops:cond:end}
    }
  \end{algorithm}

  \AlgDevelopTree{} processes single instance of branching on each run.
  So, given a tree, where terminal nodes are event world pairing \(\langle f,v \rangle_{i}\) such that \(f\) \emph{may} continue in \(v\).
  Term these terminal nodes `stems'.

  Tasks.
  \begin{enumerate}
  \item
    Extend stems.%
    \hfill%
    Lines~\ref{PrAl:dev-tree:Extend:start}--\ref{PrAl:dev-tree:Extend:merge}.
  \item
    Identify any (immediate) branches of the stems.%
    \hfill%
    \autoref{PrAl:dev-tree:Stops:Land} or \autoref{PrAl:dev-tree:Stops:Me}.
  \item
    Process continuations of branches.%
    \hfill%
    Lines~\ref{PrAl:dev-tree:pro-bra:start}--\ref{PrAl:dev-tree:pro-bra:end}.
  \item
    Either return tree or continue.%
    \hfill%
    Lines~\ref{PrAl:dev-tree:Stops:cond:more}--\ref{PrAl:dev-tree:Stops:cond:end}.
  \end{enumerate}

  \begin{itemize}
  \item
    Extend
    \begin{itemize}
    \item
      Lines~\ref{PrAl:dev-tree:Extend:start}--\ref{PrAl:dev-tree:Extend:end} extend tree given as input with continuations of all terminal branches.

      \autoref{PrAl:dev-tree:Extend:Stems}, collect all the fresh branches.
      \autoref{PrAl:dev-tree:Extend:FreshContsVar} create a set.
      Lines \ref{PrAl:dev-tree:Extend:Loop:start}--\ref{PrAl:dev-tree:Extend:end} loop over fresh branches, adding extensions to set.
    \item
      \autoref{PrAl:dev-tree:Extend:merge}, merge continuations.
      Now have extension of tree.
      However, only for fresh branches.
      Interest is whether there is further branching.
      Remainder of algorithm checks for branching, and for whether to continue.
    \end{itemize}
  \item Choices
    \begin{itemize}
    \item
      \autoref{PrAl:dev-tree:Stops:Land} gets \citeauthor{Landman:1992wh}.
    \item
      \autoref{PrAl:dev-tree:Stops:Me} gets revised.
    \end{itemize}
  \item Branches
    \begin{itemize}
    \item
      \autoref{PrAl:dev-tree:Stops:cond:start} to \autoref{PrAl:dev-tree:Stops:cond:end}, determining whether to pursue completion.
      Else 
      \begin{itemize}
      \item
        Have branches from \autoref{PrAl:dev-tree:Stops:Me}.
        \autoref{PrAl:dev-tree:Stops:cond:start} is simple check.
        If no stops, then \autoref{PrAl:dev-tree:Stops:cond:no-stops-finish} return Tree.
        Done.
      \item
        Else, some stops.
        Lines~\ref{PrAl:dev-tree:Stops:cond:else:start}--\ref{PrAl:dev-tree:Stops:cond:else:end}.
      \item
        Given stops, does the event continue in some other world?
        \autoref{PrAl:dev-tree:processbranch:start}, variable to collect branches.
        \autoref{PrAl:dev-tree:Stops:cond:else:futureB:loop:start} starts loop over stops.
        \autoref{PrAl:dev-tree:Stops:cond:else:futureB:loop:getBranches}, get branches via function.
        \autoref{PrAl:dev-tree:Stops:cond:else:futureB:loop:gather}, add these to future branches.
      \item
        \autoref{PrAl:dev-tree:Stops:cond:more} to \autoref{PrAl:dev-tree:Stops:cond:else:futureB:process:end} process future branches.
        \begin{itemize}
        \item
          \autoref{PrAl:dev-tree:Stops:cond:more} check to see if there are.
        \item
          \autoref{PrAl:dev-tree:Stops:cond:else:futureB:process:cancel} if none, then nothing more to do.
        \item
          Else, branched, so \autoref{PrAl:dev-tree:Stops:cond:else:futureB:process:expand}, expand tree and \autoref{PrAl:dev-tree:Stops:cond:else:futureB:process:end} recursive call, with index shifted.
        \end{itemize}
      \end{itemize}
    \end{itemize}
  \end{itemize}
  Now, start with \(\langle e,w \rangle\).
  Slight thing, ordered pairs.
  So, pass a vacant event.
  Run, \AlgDevelopTree{}(\(\langle \langle -,- \rangle_{0}, \langle e,w \rangle_{1} \rangle, e, w, 1\))
\end{note}

\begin{note}
  \autoref{PrAl:dev-tree:Stops:cond:tree-fix} is important given \AlgGetPStops{}.
  For, reasonable stop and may be stopping point which shows the progressive is false.%
  \footnote{
    Careful consider shows that we don't include all branches.
    This is an interesting issue.
    For, truth of the progressive looks to see whether description is true of event.
    May think that, given stops in reasonable worlds, should also consider continuations.

    Relevant possibility is in all cases where event stops, continuation where thing happens.
    But, some continuation in possible world where didn't stop and thing didn't happen.

    This doesn't strike me as plausible.

    Note, only relevant on first pass through the tree, and \AlgAC{} applies to all stops.

    Still, if inclined, cover the case by modifying \AlgGetPStops{} where conditional is switched to false, and adding the results to \(\text{Tree}\) prior to \autoref{PrAl:dev-tree:Stops:cond:start}.
    }
\end{note}

\begin{note}
  This gives us continuation tree.
  
  As an aside, this plausibly also gets failure of \BoyVS{}.
  Surprising, this seems to capture all the entailments that \citeauthor{Boylan:2020aa} is interested in.
  Though, I think we've just reduced ability to choice of action.
\end{note}

\begin{note}
  \pevent{} just in case there is some action avaiable to the agent, and were the agent to perform the action, the agent would be \(\alpha\)-ing.
\end{note}

\begin{note}
  With respect to concluding, reasonable constraint is neat.
  How the agent is in the actual world.
  And, shifts to closest worlds avoid blunders.
\end{note}

\begin{note}
  False negatives.
  \(\varphi\) to \(prog \varphi\).
  But, this entailment isn't really of interest.
  At least get this in the case of concluding.
\end{note}

\paragraph{Summary}

\begin{note}
  \citeauthor{Landman:1992wh}'s account with two key changes.

  First, allow for multiple branching.
  Motivated by concerns from \citeauthor{Bonomi:1997uq} and \citeauthor{Szabo:2004ul}.
  Important for \fc{1}, as explore all paths in reasoning.

  Second, possible stops.
  Other ways things could have gone.
  Motivated by intuitive instances of the progressive.
  Important for \fc{1}, again, all parts in reasoning.

  So, basically, progressive is true just in case, some event in progress, and no matter how the event develops, there is always some possibility in which complete.
  Possibility is existential as shift to possible world.
\end{note}

\paragraph{The progressive and ability}

\begin{note}
  Now, \BoyVS{}.
  Holds on \citeauthor{Landman:1992wh}'s account of the progressive.
  Holds on revised account, \emph{only} if disjoin all possible results.

  Two aspects.
  First, \AlgGetPStops{}, different from how things actually happen.
  Second, universal quantification over branches.

  So, if some point not considered, then disjunction without point fails to hold.

  This seems good.

  Then can't works, because \AlgGetPStops{}.

  Finally, \BoyPS{}.
  Complex.

  Holds, given the same method of evaluation.
  Which worlds are nearby worlds.

  Doesn't hold for the progressive in general.
  Though, I don't think it holds for ability.

  However, will hold when fully determined.
  Had the ability to win the race, after got into first.
  Had the ability to hit the bullseye, just before releasing the dart.

  Deals with wind.

  And, then this also gets joint abilities.
\end{note}

\subsection{`Foregone-concluding'}
\label{sec:fc-progressive}

\begin{note}
  So, action such that would be concluding.
  On understanding of the progressive, no matter what happens, still conclude.
  Might need to get `lucky' with action.
  However, start then path to the conclusion.

  This `luck' is limited.
  Nothing conflicting.
\end{note}

\begin{note}
  Worry, jumping to conclusions.
  But, in this case, \pevent{} in which agent concludes that step of reasoning is bad.
  Hence, incompatible.
\end{note}

\section{\fc{3} and \support{0}}
\label{cha:fcs:sec:fcs-support}

\begin{note}
  \autoref{cha:sec:fcs-def}, account of \fc{1}.
  Now tie to \support{0}.
\end{note}

\begin{note}
  \begin{proposition}
    \label{prop:fcs-only-if-support}
    For an agent \vAgent{} and proposition-value-premises pairing \(\pvp{\phi}{v}{\Phi}\):
    \begin{enumerate}
    \item[\emph{If}:]
      \begin{enumerate}[label=\alph*., ref=(\alph*.)]
      \item
        \(\pvp{\phi}{v}{\Phi}\) is a \fc{0}, from \vAgent{}' perspective.
      \end{enumerate}
    \item[\emph{then}:]
      \begin{enumerate}[label=\alph*., ref=(\alph*.), resume]
      \item
        \support{2} holds between \(\pv{\phi}{v}\) and \(\Phi\), from \vAgent{}' perspective.
      \end{enumerate}
    \end{enumerate}
    \vspace{-\baselineskip}
  \end{proposition}

  {
    \color{red}
    Why this is somewhat interesting.
  }
  However, before turning to the argument for \autoref{prop:fcs-only-if-support}, it is important to note the limitations of \autoref{prop:fcs-only-if-support} with respect to \issueConstraint{}.
  For, in order to argue against \issueConstraint{}, need some \(\pvp{\psi}{v'}{\Psi}\) such that answers \qWhyV{}.
  Answer \qWhyV{} only with dependence.
  Does not follow from \fc{0} that we get dependence.

  Note, also, qualifications.
  From the agent's perspective.
  Mirrored in both cases.
  Plausible that variant of \autoref{prop:fcs-only-if-support} holds unqualified.
  However, we have said nothing of \support{} independent of agent's perspective.
\end{note}

\begin{note}
  The argument for \autoref{prop:fcs-only-if-support} is {\color{red} mostly immediate for ideas regarding support}.

  \begin{goal}
    If conclude only if \fc{}, then support, in part, answers \qWhyV{}.
  \end{goal}

  So, to get answer to \qWhyV{}, need dependency.
  Here, if not \support{} then not \fc{}.
  If not \fc{} then not conclude.

  This is fine, just need to be careful with the counterfactual.
  Relation between \support{} and \fc{} is plain conditional.
  So, it survives any counterfactual changes.
\end{note}

\begin{note}[Argument]
  Argument is straightforward.
  Possible support, by assumption.
  Contraposition.
  If not support, then no \fc{}.
\end{note}

\paragraph{Potential relations of support}

\begin{note}
  Start with the following proposition.
  \begin{proposition}
    \label{prop:fcs-only-if-pot-support}
    For an agent \vAgent{} and proposition-value-premises pairing \(\pvp{\phi}{v}{\Phi}\):
    \begin{enumerate}
    \item[\emph{If}:]
      \begin{enumerate}[label=\alph*., ref=(\alph*.)]
      \item
        \(\pvp{\phi}{v}{\Phi}\) is a \fc{0}, from \vAgent{}' perspective.
      \end{enumerate}
    \item[\emph{then}:]
      \begin{enumerate}[label=\alph*., ref=(\alph*.), resume]
      \item
        (A) potential (relation of) \support{} holds between \(\pv{\phi}{v}\) and \(\Phi\), from \vAgent{}' perspective.
      \end{enumerate}
    \end{enumerate}
    \vspace{-\baselineskip}
  \end{proposition}

  Argument is fairly straightforward:
  \begin{argument}
    Suppose \(\pvp{\phi}{v}{\Phi}\) is a \fc{0}.
    Then, from agent's perspective, \pevent{} in which concludes.
    Now, consider the \pevent{}.
    The culmination of the event, agent concludes.

    So, from~\autoref{idea:support}, a relation of support holds, from the agent's perspective.

    Therefore, in whatever sense event is potential, \support{} between \(\pv{\phi}{v}\) and \(\Phi\) is likewise potential.
  \end{argument}
  From the agent's perspective, there is no difference between witnessed relation of support and potential relation of support.
\end{note}

\begin{note}
  \emph{Potential} relation of support, but it does not follow that there is a relation of support, from the agent's perspective.
\end{note}

\begin{note}
  \begin{proposition}
    \label{prop:pot-support-onlyIf-support}
    For an agent \vAgent{} and proposition-value-premises pairing \(\pvp{\phi}{v}{\Phi}\):
    \begin{enumerate}
    \item[\emph{If}:]
      \begin{enumerate}[label=\alph*., ref=(\alph*.)]
      \item
        (A) potential (relation of) \support{} holds between \(\pv{\phi}{v}\) and \(\Phi\), from \vAgent{}' perspective.
      \end{enumerate}
    \item[\emph{then}:]
      \begin{enumerate}[label=\alph*., ref=(\alph*.), resume]
      \item
        \support{2} holds between \(\pv{\phi}{v}\) and \(\Phi\), from \vAgent{}' perspective.
      \end{enumerate}
    \end{enumerate}
    \vspace{-\baselineskip}
  \end{proposition}

  \begin{argument}
    \autoref{idea:support:possible}.
    It is possible for there to be.
    So, we have everything needed.
    Both necessary and sufficient.
    Hence, form agent's perspective, relation of support.

    So, every necessary property that does not involve witnessing.
    But, then, every necessary property.
    Therefore, sufficient.
    For, if not sufficient, then missing a necessary property.
    Contradiction.

    Slight issue, disjunction of properties.
    But, this doesn't change the argument.
    Disjunction.
  \end{argument}
\end{note}

\begin{note}
  \color{red}
  Worry.
  \support{2} doesn't rely on witnessing.
  Now, if this goes through, then seems \support{} for any conclusion before making the conclusion.
  However, possible for the agent to reason to different conclusions.
  For, some faulty reasoning.
  Toggle the fault.
  Therefore, \support{} for contradictory conclusions.
\end{note}

\section{Conclusions, foregone}
\label{sec:fc3-1}

\paragraph{Premises and past conclusions}

\begin{note}[Premises]
  So, as we have seen with testimony, status of a premises blocks a \requ{}.

  Whether the same may hold for this problem.

  It's the case that, part of agent's present epistemic state that they would conclude.

  Problem is, if attempt and fail, then this premise does nothing.
  Their present epistemic state develops into a dead-end.
\end{note}

\begin{note}[Note!]
  This doesn't hold in general, for all premises.

  In particular, premise is past conclusion.

  Consider cases of being somewhat impaired, e.g., via exhaustion.
  Indeed, exhaustion is interesting.
  Basic consistency checks.
  Should be the case that conclude A, but just concluded \emph{not}-A, or something like this\dots

  Denying that past continues to secure in all instances.
  So, just need the potential to revise perspective on previous conclusion.
\end{note}

\section{\fc{3} and support}
\label{cha:fcs:sec:fc3-support}

\begin{note}
  \begin{proposition}
    For any path, present epistemic state determines availability of path.
  \end{proposition}

  Start.
  Then, continue.
  Started from \(\Phi\), so will conclude.
  Hence, no matter choice made, must have taken the possibility of this choice into account.
  So, it must be the case that determined.

  Hence, if witness, then via some path.

  So, witnessing predetermined path.
  Any instances of concluding by witnessing reduces to witnessing predetermined path.

  Witnessing may provide information about path, but witnessing doesn't contribute given a \requ{}.

  For any X from W,
  present determines whether or not X from agent's point of view, then \fc{}.

  In other words, agent's present epistemic state determines.
  Agent may need to witness to figure out how determined, but witnessing does not influence.
\end{note}

\begin{note}[Two worries]
  Two worries.

  First, that even though \fc{0}, the agent would not conclude.
  Either because \(\Phi\) is unavailable, or because no \pevent{}.
  So, can't remove \fc{0} from account of why.

  However, then \fc{0} does not support.

  If grant that \fc{0} supports, then this seems to work out.
  Further, if require existence, then things that support get very messy.
  Dopeganger cases.
  Reason is I saw A, but it wasn't A, appealing to something that doesn't exist.
  Various other cases like this.

  Difference.
  In these cases, have premise, thing is that the truth value is distinct.
  Here, possibly no premise.

  Well, this is different.
  However, I don't think this is sufficient to reject the idea.
  Just because this distinction doesn't arise in the case of witnessing doesn't really do much.

  Look, a `bad' premise offers no more support for the agent than no premise.

  Second, need \emph{that} \fc{0}.
  However, the point is that this is about the agent's present epistemic state.
  \emph{Without} \fc{0}, the agent would reason.
  This is just the key point reiterated.
  Know whether, \fc{0} just adds information about which.
\end{note}

\subsection{Interpretation}
\label{cha:zSpA:sec:interpretation}

\begin{note}
  Doxastic justification.
\end{note}

% \begin{note}
%   Important to observe here that with dispositions and ability, the subjunctive analysis is an analysis.
%   So, in principle possible to provide a distinct analysis.
%   This is surely the case, and I can probably find some example.

%   By contrast, in the case of positive answers to \qzS{}, the subjunctive is `built in' to the question.
% \end{note}

% \subsubsection{Dispositions and ability}
% \label{sec:dispositions}

% \begin{note}[Parallel between dispositions and ability]
%   Consider \citeauthor{Choi:2021wg}'s characterisation of the Simple Conditional Analysis of dispositions:
%   \begin{quote}
%     An object is disposed to \emph{M} when \emph{C} iff it would \emph{M} if it were the case that \emph{C}.\nolinebreak
%     \mbox{}\hfill\mbox{(\citeyear{Choi:2021wg})}
%   \end{quote}
%   For example, an object is disposed to dissolve when it is placed in water iff the object would dissolve if it were the case that it is placed in water.

%   The Simple Conditional Analysis may be challenged, but for our purposes it is adequate.
%   We are interested in the broad form of the truth condition, and various more refined analyses share the same broad form.
%   Note, in particular, that it being the case that \emph{C} and \emph{M} happening describes an event.
%   Given appropriate conditions; salt dissolves, glass breaks, and I mumble when I am tired.
%   The key idea is that the property of being disposed to \emph{M} when \emph{C} is analysed in terms of the (possible) event of \emph{M} happening when \emph{C}.

%   The parallel to ability is established by noting that ability may also be analysed in terms of a (possible) event, as we have seen.
%   In particular, by incorporating volition in the analysans of the Simple Conditional Analysis.
%   To illustrate, \citeauthor{Mandelkern:2017aa} trace the Conditional Analysis of ability  to \textcite{Hume:1748tp} and \textcite{Moore:1912te}, among others:
%   \begin{quote}
%     S can \(\phi\) iff S would \(\phi\) if S tried to \(\phi\)\nolinebreak
%     \mbox{}\hfill\mbox{(\citeyear[Cf.][308]{Mandelkern:2017aa})}
%   \end{quote}
%   Compare to the Simple Conditional Analysis of dispositions:
%   The object is some agent \emph{S}, \emph{C} is `S tried to \(\phi\)' and \emph{M} is `S \(\phi\)s' --- it is volition alone which distinguishes the analyses.
% \end{note}


\subsubsection{Doxastic justification}
\label{cha:fcs:sec:dox-just}

\begin{note}
  \citeauthor{Turri:2010aa}

  \begin{quote}
    Necessarily, for all S, \emph{p}, and \emph{t}, if \emph{p} is propositionally justified for S at \emph{t}, then \emph{p} is propositionally justified for S at \emph{t} because S currently possesses at least one means of coming to believe \emph{p} such that, were S to believe \emph{p} in one of those ways, S's belief would thereby be doxastically justified.%
    \mbox{ }\hfill\mbox{(\citeyear[316]{Turri:2010aa})}
  \end{quote}

  Key is that doxastic justification depends on what the agent does.

  \citeauthor{Turri:2010aa}'s focus is on how reasons are used.
  What the agent does.

  Seen with example.

  \begin{quote}
    \begin{enumerate}[label=(P\arabic*)]
      \setcounter{enumi}{4}
    \item
      The Spurs will win if they play the Pistons.
    \item
      The Spurs will play the Pistons.
    \end{enumerate}

    \hbox to \hsize{\hfil{\vdots}\hfil}

    \begin{enumerate}[label=(P\arabic*), resume]
    \item
      Therefore, the Spurs will win.%
    \mbox{ }\hfill\mbox{(\citeyear[317]{Turri:2010aa})}
    \end{enumerate}
  \end{quote}

  Rather than \emph{modus ponens}, `\emph{modus profusus}'.
  Conclude \(r\) from \(p\) and \(q\).
  (\citeyear[317]{Turri:2010aa})

  \begin{quote}
    The way in which the subject performs, the manner in which she makes use of her reasons, fundamentally determines whether her belief is doxastically justified.
    Poor utilization of even the best reasons for believing \emph{p} will prevent you from justifiedly believing or knowing that \emph{p}.%
    \mbox{ }\hfill\mbox{(\citeyear[316]{Turri:2010aa})}
  \end{quote}

  Variant of ~\cite{Prior:1960wh}'s `tonk' connective.
  Though, difference is between connective and rule.
  \(p\) tonk \(q\) would not be propositionally justified.
\end{note}

\begin{note}
  \citeauthor{Turri:2010aa} is similar to \citeauthor{Goldman:1979ui}

  Begin with justification.

  \begin{quote}
    \begin{enumerate}[label=(\arabic*)]
      \setcounter{enumi}{10}
    \item
      Person \emph{S} is \emph{ex ante} justified in believing \emph{p} at \emph{t} if and only if there is a reliable belief-forming operation available to \emph{S} which is such that if \emph{S} applied that operation to this total cognitive state at \emph{t}, \emph{S} would believe \emph{p} at \emph{t}-plus-delta (for a suitably small delta) and that belief would be \emph{ex post} justified.
    \end{enumerate}
  \end{quote}

  Where, sufficient condition for belief would be \emph{ex post} justified:
  \begin{quote}
    \begin{enumerate}[label=(\arabic*)]
      \setcounter{enumi}{4}
    \item
      If S's believing \emph{p} at \emph{t} results from a reliable cognitive belief-forming process (or set of processes), then S's belief in \emph{p} at \emph{t} is justified.%
      \mbox{ }\hfill\mbox{(\citeyear[13]{Goldman:1979ui})}
    \end{enumerate}
  \end{quote}
  Roughly, at least.
  \citeauthor{Goldman:1979ui} refines this a fair bit, but this isn't important.

  Availability of a reliable belief-forming operation!

  Relation here is brittle.
  Account of justification, apply to concluding.
  Well, then all we get is that before concluding, would make sense to conclude only if available.
  Running something like the \citeauthor{Carroll:1895uj} regress, not some state.
  But, this only tells us about suitability to conclude.

  Still, key point is process.

  Another useful thing to highlight is the suitably small delta.
  With \requ{}, this is captured in terms of the option.
\end{note}

\begin{note}
  Significant difference is in the case of justification, we're not interested in the agent's perspective.
  Hence, these accounts are understood in terms of the agent having the ability, roughly.
\end{note}

%%% Local Variables:
%%% mode: latex
%%% TeX-master: "master"
%%% End:
