\chapter{\fc{3}}
\label{cha:fcs}

\nocite{Ryle:1946tu}

\begin{note}
  Here, \fc{1}.
  Two key things.
  An account of \fc{1}, and connexion between \fc{1} and \ros{1}.

  \begin{itemize}
  \item
    If \fc{}, then \ros{}.
  \end{itemize}

  Important idea.
  However, limited.
  Support without witnessing doesn't raise a problem for \issueConstraint{}.
  Rather, support needs to be \emph{why}.

  Main focus is \autoref{cha:sec:fcs-def}.
  What it is for some proposition-value pairing \(\pv{\phi}{v}\) to be a \fc{} from some pool of premises \(\Phi\).
  With the right set-up, \ros{1} --- focus on \autoref{cha:fcs:sec:fcs-support} --- will be simple.

  Progressive, action which is concluding \(\pv{\phi}{v}\) from \(\Phi\).
  Contrast with ability, `ability to conclude \(\pv{\phi}{v}\) from \(\Phi\)'.
  Argue that there is no clear path.
  Though, implicit upshot is reducing ability to progressive (given assumption).
\end{note}

\begin{note}
  Breakdown
  \begin{itemize}
  \item
    \TOCLine{sec:intuition}
  \item
    \TOCLine{cha:sec:fcs-def}

    Definition of \fc{0}.
  \item
    \TOCLine{cha:fcs:sec:illu}
  \item
    \TOCLine{cha:fcs:sec:fcs-support}

    \ros{}.
  \end{itemize}
\end{note}

\section{Intuition}
\label{sec:intuition}

\begin{note}
  General term `foregone conclusion' is ambiguous.
  \begin{itemize}
    \item
    Inevitable results of reasoning.
  \item
    Conclusion which has been settled in advance of reasoning.
  \end{itemize}

  Interest is with the second meaning.
  Two examples of general use.

  % \begin{quote}
  %   [どうぜ]\dots Expresses an attitude of resignation or carelessness on the part of the speaker, in the sense that regardless of what s/he does, the conclusion or outcome is foregone and cannot be changed by the will or effort of an individual.%
  %   \mbox{ }\hfill\mbox{(\citeyear[332--333]{kurufushamashii:2015un})}
  % \end{quote}
  A clear example of the first meaning is found in~\citeauthor{Machover:1996vu}'s~\citetitle{Machover:1996vu}:
w
  \begin{quote}
    I have omitted its proof, but added a detailed analysis of the meaning of the lemma and the reason why its proof works. When this is understood, the proof itself becomes a mere technicality, almost a foregone conclusion.%
    \mbox{ }\hfill\mbox{(\citeyear[viii]{Machover:1996vu})}
  \end{quote}

  \citeauthor{Machover:1996vu} is discussing a proof, and whether or not it is inevitable that one would complete the proof (conclude that the relevant theorem is true) after understanding the lemma and why it works.
  There is no relevant sense in which the truth of the theorem has been settled in advance of reasoning.
  Though, as the proof is somewhat difficult,~\citeauthor{Machover:1996vu} only states the proof is `almost' a foregone conclusion.%
  \footnote{
    The proof is question is of the G\"{o}del-Rosser First Incompleteness Theorem.
    (\citeyear[Cf.][226]{Machover:1996vu})
  }%
  \(^{,}\)%
  \footnote{
    For a similar example without qualification, consider the following from~\textcite{Jacquette:2002up}:
    \begin{quote}
    It is nevertheless important to recognize that Russell's evaluation of such sentences as false is predetermined by his existence presuppositional semantics for the ‘existential' quantifier, and by the fact that his logic permits no alternative means of considering the semantic status of sentences ostensibly containing proper names for nonexistent objects.
    This makes it an altogether philosophically foregone conclusion that sentences like ‘Pegasus is winged,' which many logicians would otherwise consider to be true propositions of mythology, are false.%
    \mbox{ }\hfill\mbox{(\citeyear[6]{Jacquette:2002up})}
  \end{quote}
  }

  For an additional example, consider the following from~\citeauthor{Grice:1957vg}'s~\citetitle{Grice:1957vg}:%
  \footnote{
    Same is found in \textcite[219]{Grice:1989uf}.
  }
  \begin{quote}
    He intends the audience's recognition of his intention to produce that response to be effective in producing that response.
    He does not regard it as a foregone conclusion that his action will produce the intended response, whether or not his intention is recognized.\newline
    \mbox{ }\hfill\mbox{(\citeyear[385]{Grice:1957vg})}
  \end{quote}

  In this case, the term `foregone conclusion' is embedded under negation, to highlight that the agent in question entertains the possibility that the agent's action will not produce the intended response.

  By contrast, the following passage from \textcite{Kadane:1996vu} is an example of the second meaning:

  \begin{quote}
    When can a Bayesian select an hypothesis \emph{H} and design an experiment (or a sequence of experiments) to make certain that, given the experimental outcome(s), the posterior probability of \emph{H} will be greater than its prior probably?
    We discuss an elementary result that establishes sufficient conditions under which this reasoning to a foregone conclusion cannot occur.%
    \mbox{ }\hfill\mbox{(\citeyear[1228]{Kadane:1996vu})}
  \end{quote}

  At issue is whether a Bayesian may chose some hypothesis \emph{H} and then guarantee some increase in probability for \emph{H} by running some experiments.
  So, are there cases in which the Bayesian first chooses a hypothesis \emph{H} and then ensures they reason to an increase in the probability of \emph{H}?
\end{note}

\begin{note}
  Our interest is with the first meaning, though narrow.
  To avoid ambiguity with first meaning, write `\fc{0}' rather than `foregone conclusion'.
  That is, the hyphen signifies when we are speaking about the technical term.
\end{note}

\section{\fc{3}}
\label{cha:sec:fcs-def}

\begin{note}[\fc{2} definition]
  We define \(\pv{\phi}{v}\) being a \emph{\fc{0}} as follows:

  \begin{restatable}[\fc{3}]{definition}{definitionForegoneC}
    \label{def:fc}
    \begin{itemize*}[noitemsep, label=\(\circ\)]
    \item
      An agent: \vAgent{}
    \item
      A proposition: \(\phi\)
    \item
      A value: \(v\)
    \item
      A \poP{0}: \(\Phi\)
    \item
      \mbox{ }
    \end{itemize*}

    \begin{itemize}
    \item
      \(\pv{\phi}{v}\) from \(\Phi\) is a \emph{\fc{0}} for \vAgent{}.
      % \newline
      % \mbox{ }\hfill(equiv.\ \(\pvp{\phi}{v}{\Phi}\) is a \fc{0} for \vAgent{})
    \end{itemize}
    \emph{If and only if}
    \begin{enumerate}[label=]
    \item
      Both~\ref{def:fc:is-pe-good} and~\ref{def:fc:no-pe-bad} are true:
      \begin{enumerate}[label=\alph*., ref=(\alph*)]
      \item
        \label{def:fc:is-pe-good}
        There is \emph{some} \pevent{} \(p\) in which \vAgent{} concludes \(\pv{\phi}{v}\) from \(\Phi\).
      \item
        \label{def:fc:no-pe-bad}
        There is \emph{no} \pevent{} \(p\) in which \vAgent{} concludes some proposition-value-premises pairing which is incompatible with concluding \(\pv{\phi}{v}\) from \(\Phi\).%
        \footnote{
          Incompatible:
          \(\pv{\chi}{v''}\) from \(X\) where:
        If conclude \(\pv{\chi}{v''}\) from \(X\), then does not conclude \(\pv{\phi}{v}\) from \(\Phi\).
        }
      \end{enumerate}
    \end{enumerate}
    \vspace{-\baselineskip}
  \end{restatable}
\end{note}

\begin{note}[Intuition]
  Significant attention will be given to what is means for there to be a potential event in which an agent performs some action in \autoref{cha:sec:fcs-def:potential-events}, below.
  However, the basic features of \autoref{def:fc} follow from substituting `possible' for `\pevent{}'.
  Given this substitution:

  \begin{itemize}[noitemsep]
  \item
    Clause~\ref{def:fc:is-pe-good} ensures that there is some possibility in which the agent to conclude \(\pv{\phi}{v}\) from \(\Phi\).
  \item
    Clause~\ref{def:fc:no-pe-bad} ensures that there is no possibility in which the agent concludes something incompatible with concluding \(\pv{\phi}{v}\) from \(\Phi\).
    Hence, Clause~\ref{def:fc:no-pe-bad} rules out the agent failing to conclude \(\pv{\phi}{v}\) from \(\Phi\) because some incompatible proposition-value-premises pairing is (also) a \fc{0}.
  \end{itemize}

  Intuitively, and inevitable result of possible reasoning.
  For, there is some possibility in which the agent concludes \(\pv{\phi}{v}\) from \(\Phi\) via Clause~\ref{def:fc:is-pe-good}.
  And, at no point prior to concluding could the agent have concluded some other proposition-value-premises pairing which would prevent the agent from concluding \(\pv{\phi}{v}\) from \(\Phi\) via~\ref{def:fc:no-pe-bad}.

  Further, not merely that agent would not prevent on success, but that there is no way to block success.
\end{note}

\begin{note}
  Still, mere possibility is too general.
  There is a possibility in which I immediately obtain a comprehensive understanding of theoretical computer science and settle whether P is equal to NP (or show that the P versus NP problem is undecidable).
  Hence, \fc{1} are defined with respect to \pevent{1}.

  For the moment we will set \pevent{1} aside, and consider a handful of \illu{1} in \autoref{cha:fcs:sec:illu}.
  In \autoref{cha:sec:fcs-def:potential-events} we will provide a fairly detailed account of \pevent{1}.
  (If you would prefer to skip the \illu{1}, please turn to~\autopageref{cha:sec:fcs-def:potential-events}.)
\end{note}

\begin{note}[Neutral perspective]
  \phantlabel{fcs-neutral-perspective}
  Still, before turning to the \illu{1} and \pevent{1}, a final observation:

  Observe that whether \(\pvp{\phi}{v}{\Phi}\) is a \fc{0} is stated from a \agpe{neutral} --- at issue is whether there is a \pevent{} in which the agent concludes.
\end{note}

\subsection{\pevent{3}}
\label{cha:sec:fcs-def:potential-events}

\begin{note}
  \autoref{def:fc} appeals to~\ref{def:fc:is-pe-good} the existence of some \pevent{} and~\ref{def:fc:no-pe-bad} the non-existence of some \pevent{}.

  However, \autoref{def:fc} does not rely on anything more than existential quantification.
  The choice is deliberate.
  We given necessary and sufficient conditions for the \emph{existence} of some \pevent{} in terms of
  \begin{enumerate*}[label=(\roman*)]
  \item
    actions available to the agent, and
  \item
    truth conditions for the progressive.
  \end{enumerate*}
\end{note}

\begin{note}[\pevent{2} definition]
  We define a \pevent{} as follows:
  \begin{restatable}[\pevent{3}]{definition}{definitionPEvent}
    \label{def:potenital-event}
    For an agent \vAgent{} and action description \(\alpha\):
    \begin{itemize}
    \item
      There is a \pevent{} \(p\) in which \vAgent{} \(\alpha\)s
    \end{itemize}
    \emph{if and only if}
    \begin{enumerate}[label=]
    \item
      Both~\ref{def:PE:action} and~\ref{def:PE:prog} are true:
      \begin{enumerate}[label=\alph*., ref=(\alph*)]
      \item
        \label{def:PE:action}
        There is some action \(a\) that \vAgent{} may immediately perform.
      \item
        \label{def:PE:prog}
        \(\text{Prog}(e, \alpha)\) would be true in the event \(e\) of \vAgent{} doing \(a\).
      \end{enumerate}
    \end{enumerate}
    Where \(\text{Prog}(e, \alpha)\) stands for the progressive from of \(\alpha\) when evaluated with respect to \(e\) and \assuPP{} holds for the progressive.%
    \footnote{
      I.e.\ \(\text{Prog}(e, \alpha)\) is true \emph{iff} event \(e\) is an event of \(\alpha\)ing.
      See,~\textcite{Richards:1981wo},~\textcite{Portner:2011vi}, etc.
    }
  \end{restatable}

  In short,~\autoref{def:potenital-event} states that there is a \pevent{} in which an agent performs some action \(\alpha\) just in case there is some action the agent may (immediately) perform which would result in the agent \(\alpha\)ing.
\end{note}

\begin{note}[Division of labour between the clauses]
  The division of labour between clauses~\ref{def:PE:action} and~\ref{def:PE:prog} is, in reverse order:
  \begin{itemize}[noitemsep]
  \item
    Clause~\ref{def:PE:prog} captures a sense of possibility, via the progressive, such that that the agent \(\alpha\)s (but in such a way that how the agent \(\alpha\)s is not necessarily settled by the action).
\item
  Clause~\ref{def:PE:action} distinguishes the existence of a \pevent{} from the agent \(\alpha\)ing by existential quantification over actions, but binds the existence of a \pevent{} to circumstances be restriction to immediate actions.
  \end{itemize}
\end{note}

\begin{note}
  Following section \autoref{cha:sec:fcs-def:ability} will get problems with ability.
\end{note}

\subsection{The progressive}
\label{cha:fcs:sec:progressive}

\begin{note}[Interest with the progressive]
  Our interest with the progressive is due to the delicate sense of possibility required for a sentence stating an event in the progressive to be true.

  \phantlabel{imperfective-paradox:intro}
  Perhaps the clearest example is the `imperfective paradox' (\citeyear[cf.][Ch.3.1]{Dowty:1979vq}).

  \citeauthor{Bach:1986tb} summarises:
  \begin{quote}
    [H]ow can we characterize the meaning of a progressive sentences like \ref{Bach:impP:17} on the basis of the meaning of a simple sentence like \ref{Bach:impP:18} when \ref{Bach:impP:17} can be true of a history without \ref{Bach:impP:18} ever being true?
    \begin{enumerate}[label=(\arabic*), ref=(\arabic*)]
      \setcounter{enumi}{16}
    \item
      \label{Bach:impP:17}
      John was crossing the street.
    \item
      \label{Bach:impP:18}
      John crossed the street.%
      \mbox{ }\hfill\mbox{(\citeyear[12]{Bach:1986tb})}
    \end{enumerate}
  \end{quote}

  No completion is required, and often some surprise.
  Something unexpected happened while John was crossing the street.
  Sense of inertia associated with the agent \(\alpha\)ing.

  Expectation that that John reaching the other side of the street does not reduce to \(\{\text{logical}, \text{metaphysical}, \text{nomic}, \dots\}\) possibility.

  For, suppose John is sitting a multiple choice exam.
  To pass the exam John only needs to chose some number of correct choices.
  It is certainly logically, metaphysically, and nomically possible that John chooses a sufficient number of correct choices.
  However, it does not follow that John is passing the exam.%
  \footnote{
    See also Igal Kvart's example of Mary wiping out the Roman army (\cite[18]{Landman:1992wh}).
  }

  Likewise, there is no simple relation to counterfactuals.
  Consider a scenario in which John is passing the exam without external help.
  Then, a classmate slips John some answers, which John then uses.
  It is no longer true that John is passing the exam without external help.
  And, in the closest possible world where the classmate does not slip John answers, it need not be true that John passes the exam without external help.
  For, if John is surrounded by students of a similar mindset then it is plausible that the in closest possible world a different classmate slips John the same answers.
\end{note}

\begin{note}
  Way the modality functions is tied to the event.

  \citeauthor{Dowty:1979vq} adds:
  \begin{quote}
    Notice, furthermore, that to Say that John was drawing a circle is not the same as saying that John was drawing a triangle, the difference between the two activities obviously having to do with the difference between a circle and a triangle.
    Yet if neither activity necessarily involves the existence of such a figure, just how are the two to be distinguished?%
    \mbox{ }\hfill\mbox{(\citeyear[133]{Dowty:1979vq})}
  \end{quote}

  As \citeauthor{Dowty:1979vq} highlights, event is sufficiently specific to determine some outcome over some other.%
  \footnote{
    Though, the force of \citeauthor{Dowty:1979vq}'s observation is perhaps clearer by substituting `square' for `circle'.
    For, straight line\dots
  }
  So, the truth of the progressive doesn't require completion and doesn't require significant progress toward completion.
\end{note}

\begin{note}
  \autoref{def:potenital-event} relies on important (but common)%
  \footnote{
    See, for example:
    \textcite{Bennett:1972uw},
    \textcite{Dowty:1979vq},
    \textcite{Parsons:1990aa},
    \textcite{Landman:1992wh}, and
    \textcite{Portner:1998um}.

    However,~\assuPP{0} is denied by~\textcite{Szabo:2004ul}.
    \citeauthor{Szabo:2004ul} writes:
    \begin{quote}
      Sometimes we are \emph{doing} things even though there is no real chance that we could get them \emph{done}, and this is true even if we abstract away from the possibility of miraculous intervention.%
      \mbox{ }\hfill\mbox{(\citeyear[40]{Szabo:2004ul})}
    \end{quote}
    To illustrate, \citeauthor{Szabo:2004ul} denies~\ref{Szabo:Arch} is necessarily false:
    \begin{quote}
      \begin{enumerate}[label=(\arabic*), ref=(\arabic*)]
        \setcounter{enumi}{12}
      \item
        \label{Szabo:Arch}
        As the architect was building the cathedral he knew that, although he would be building it for another year or so, he couldn't possibly complete it.%
        \mbox{ }\hfill\mbox{(\citeyear[38]{Szabo:2004ul})}
      \end{enumerate}
    \end{quote}
    Though,~\ref{Szabo:Arch} seems false to me, without some priming.
    And, the only priming on which~\ref{Szabo:Arch} reads true involves interpreting the architect's knowledge from the \agpe{architect's}, allowing a failure of factivity, thus allowing the cathedral to be built.

    Still,~\assuPP{0} is an assumption.
    The goal is not to tie potential to progressive, but to evaluation associated with the progressive granting assumption.
  }
  assumption regarding the progressive.

  \begin{assumption}[\assuPP{2}]
    \label{assu:PP}
    For any event \(e\) and action description \(\alpha\):
    \begin{enumerate}
    \item[\emph{If}:]
      \begin{enumerate}[label=\alph*., ref=(\alph*)]
      \item
        \(e\) is an event of \(\alpha\)ing.%
        \hfill(I.e.\ \(\text{Prog}(\alpha)\) is true of \(e\).)
      \end{enumerate}
    \item[\emph{Then}:]
      \begin{enumerate}[label=\alph*., ref=(\alph*), resume]
      \item
        There is some possible event \(e'\) such that \(e'\) is a development of \(e\) and \(\alpha\) is true of \(e'\).
      \end{enumerate}
    \end{enumerate}
    \vspace{-\baselineskip}
  \end{assumption}

  \assuPP{2}, shift evaluation to some possible event in which something related is true.

  Possible here is arbitrary.
  Important is development.

  Though, in same way it is common to restrict attention to some sense of possibility via an adjective, we may speak instead of( event-)continuative-possibility.

  So, task of an account of the progressive is to narrow the relevant sense of continuative-possibility.
  \assuPP{2} holds that success is a necessary condition on continuative-possibility.
  Applied, in particular, to concluding, \assuPP{0} holds that a agent is concluding \(\pv{\phi}{v}\) from \(\Phi\) only if there is some continuative-possibility in which the agent concludes \(\pv{\phi}{v}\) from \(\Phi\).

  Here, \fc{}.
  concluding \(\pv{\phi}{v}\) from \(\Phi\).
  Concludes \(\pv{\phi}{v}\) from \(\Phi\).
  \fc{2}.
\end{note}

\begin{note}
  Paired with choice, allows complex `incomplete' actions.
  Again, progressive develops.

  \begin{illustration}[Darts]
    There is a \pevent{} in which agent scores 180 at darts just in case there is some action available to the agent, such that if the agent were to perform the action they would be scoring 180 at darts.
  \end{illustration}

  Slightly more interesting.
  Determine the available actions.
  Though, similar, no guarantee.
  Hand is knocked at point of release, still scoring.

  Scoring 180 is a complex action.
  Though, interesting.
  First throws don't matter.

  Again, key idea is that sufficient understanding of progressive.

  And, case of interest:

  \begin{illustration}[Concluding]
    There is a \pevent{} in which agent concludes \(\pv{\phi}{v}\) from \(\Phi\) just in case there is some action available to the agent, such that if the agent were to perform the action they would be concluding \(\pv{\phi}{v}\) from \(\Phi\).
  \end{illustration}

  What is it to be concluding something.
  Like crossing the road, fail to complete.
  Like darts, recover from a bad opening.
\end{note}

\begin{note}
  Intuitive distinction between which actions may and may not perform.

  However,~\ref{def:PE:action} without.
  Allow arbitrary division of actions, what matters is immediate.

  Then, agent doing \(a\) is in progressive, so make sure that doing \(a\) is also instance of \(\alpha\).
\end{note}


\begin{note}
  Still, no full account of the progressive.
  Quite difficult.
  Progressive is familiar, intuitive understanding.
  Work through in sufficient detail to be useful.

  A little on choice.
  Then, highlight issue with ability.
  Then, present and modify \citeauthor{Landman:1992wh}'s (\citeyear{Landman:1992wh}) account of the progressive.
\end{note}


\subsubsection{\citeauthor{Landman:1992wh}'s (\citeyear{Landman:1992wh}) account of the progressive}
\label{cha:sec:fcs-def:progressive-landman}
\nocite{Portner:1998um}
\nocite{Engelberg:1999vi}

\begin{note}
  In broad summary:%
  \footnote{
    \textcite{Szabo:2004ul}:
  \begin{quote}
    [A] progressive sentence is true at some time just in case some event occurs at that time, and if we follow the development of the event (within our world as long as it goes, then jumping into a nearby world, and iterating the process within the limits of reasonability) we will reach a possible world where the perfective correlate is true of the continuation.%
    \mbox{ }\hfill\mbox{(\citeyear[34]{Szabo:2004ul})}
  \end{quote}
  }
  \citeauthor{Landman:1992wh} holds that an action in the progressive holds of some event just in case the event, if allowed to develop, would develop into an event in which the action is performed.

  As we have seen with the perfective paradox, some action in the progressive need not continue in the actual world, and hence the core of \citeauthor{Landman:1992wh}'s account of the progressive is an account of allowing an event to continue.

  Roughly, on \citeauthor{Landman:1992wh}'s account the way in which an event is allowed to develop is indirectly captured by considering different ways in which the actual world may have been.

  In short, we follow an event through it's development in the actual world until it does not continue any further in the actual world.
  Then, just before the event stops, we jump to the closest `reasonable' world in which the same event happened and continues a little further (if such a world exists) and follow the development of the event in the close world until it does not continue any further.
  This process continues until it is not possible to jump to a `reasonable' world.
\end{note}

\begin{note}
  To see how the above sketch functions in practice, we follow~\citeauthor{Portner:1998um}'s (\citeyear[764--766]{Portner:1998um}) illustration of \citeauthor{Landman:1992wh}'s account.
\end{note}

\begin{note}
  Our interest is with the following sentence:
  \begin{enumerate}
  \item
    \label{prog:max:bad}
    Max is crossing the street.
  \end{enumerate}

  Hence, the relevant action of interest is the action of Max crossing the street.
  Let us fix the event \ref{prog:max:bad} is (assumed to be) true of as \(e\) and fix \(w\) for the world \(e\) happens in.

  Following \citeauthor{Landman:1992wh} (and in line with \assuPP{}), Max is crossing the street is true of \(e\) just in case, if allowed to develop, \(e\) would develop into an event in which Max crosses the street.

  Now, suppose that in \(w\) the event \(e\) does not develop any further.
  Instead (and quite unfortunately), Max is hit by a bus cruising at thirty miles per hour.
  Somewhat ominously, let us identify this bus as `bus \#1'.

  Still, \(e\) does not include Max being hit bus \#1, and had things been a little different, Max may have continued a little further across the street.
  Hence, there is some world \(v\) which is close to \(w\) in which \(e\) develop a little further.

  So far so good, but in \(w\) Max was hit by bus \#1 in \(w\).
  Hence, it seems that \(v\) also involves Max being hit by a bus.
  For, \(v\) is a possible world close to \(w\), and if Max avoids being hit by a bus altogether in \(v\) then there is surely some possible world \(v'\) closer to \(w\) than \(v\).
  So, although Max makes it a little further across the road in \(v\), Max is still hit by a bus.
  We identify the bus in \(v\) as `bus \#2'.

  However, as with bus \#1 in \(w\), it seems Max may have walked a little further across the street in possible worlds close to \(v\).
  Hence, by the same reasoning we may consider some possible world \(u\) close to \(v\) and so on\dots

  \autoref{fig:max-bus} is a modification of \citeauthor{Portner:1998um}'s figure 1. (\citeyear[767]{Portner:1998um})
  \begin{figure}[!h]
    \centering
    \begin{tikzpicture}
      \tikzmath{
        % x positions
        \x1 = 11;
        \xb1 = 2/9*\x1; \xb2 = 4/9*\x1; \xb3 = 6/9*\x1;
        % y positions
        \y1 = 2/5*\x1; \ymid = 1/2*\y1;
        \yw1 = \y1; \yw2 = 1/2*\y1; \yw3 = 0*\y1; \yb2 = 1/5*\y1;
        % event e
        \xe = 1/2*\xb1; \yediff = \yw2 - \yb2;
        \ye = \yw2 - 1/2*\yediff;
        \enudge = .1;
        \xel = 0; \xer = \xb1; \yen = \yw2 - \enudge;
        % bus 1 description location
        \xbx = 1.5/9*\x1; \xby = 4/5*\y1;
        % bus 2 description location
        \xb9 = 2.5/9*\x1;
      }
      % Paths
      \draw[line width=0.25mm, line cap=round] (\xb1,\ymid) -- (\xb3,\yw1); % world 1
      \draw[line width=1mm, line cap=round, dash pattern=on 175pt off 5pt on 5pt off 5pt on 5pt off 5pt on 5pt off 5pt on 5pt off 5pt on 5pt off 5pt on 5pt off 5pt on 5pt off 5pt on 5pt off 5pt on 5pt off 5pt on 5pt off 5pt on 5pt off 5pt on 5pt off 5pt on 5pt off 5pt on 5pt off 5pt on 5pt off 5pt on 5pt off 5pt on 5pt off 5pt on 5pt off 5pt on 5pt off 5pt] (0,\ymid) -- (\xb1,\ymid) -- (\xb2,\yb2) -- (\xb3,\yw2); % world 2
      \draw[line width=0.25mm, line cap=round] (\xb2,\yb2) -- (\xb3,\yw3); % world 3
      % World descriptions
      \filldraw[black] (\xb3,\yw1) circle (0pt) node[anchor=west, align=left]{world 1: Max hit by \\ bus \# 1};
      \filldraw[black] (\xb3,\yw2) circle (0pt) node[anchor=west, align=left]{world \(i\): Max \\ crosses street};
      \filldraw[black] (\xb3,\yw3) circle (0pt) node[anchor=west, align=left]{world 2: Max hit by \\ bus \# 2};
      % Event
      \draw[] (\xe,\ye) -- (\xel,\yen); % event l
      \draw[] (\xe,\ye) -- (\xer,\yen); % event r
      % Event description
      \filldraw[black] (\xe,\ye) circle (0pt) node[anchor=north, align=left]{event e};
      % Splits
      \filldraw[black, dashed] (\xbx,\xby) circle (0pt) node[anchor=south, align=left]{bus \#1 hits Max};
      \filldraw[black, dashed] (\xb9,\yw3) circle (0pt) node[anchor=north, align=left]{bus \#2 hits Max};
      % Split descriptions
      \draw[-Stealth, dashed] (\xbx,\xby) -- (\xb1,\ymid + \enudge); % bus 2 arrow
      \draw[-Stealth, dashed] (\xb9,\yw3) -- (\xb2 - \enudge,\yb2 - \enudge); % bus 2 arrow
    \end{tikzpicture}
    \caption{
      Continuation path of `Max was crossing the street'. \\
    }
    \label{fig:max-bus}
  \end{figure}

  The thick black line captures \(e\) as \(e\) is allowed to develop, and the changes in angle reflect shifts to alternative possible worlds.

  The dashed line indicates that Max may need to avoids being hit by additional busses in order for \(e\) to develop into an event in which Max crosses the road.

  Still, does \(e\) develop into an event in which Max crosses the road?
  If Max is hit by a bus in \(w\), then surely Max is hit by a bus in all the possible worlds close to \(w\).

  Here we turn to the key property of \citeauthor{Landman:1992wh}'s account which we have so far made without explicit comment:
  Prior to bus \#2 hitting Max in \(v\), we shift to \(u\) where \(u\) is a world which is close to \(v\) rather than \(w\).
  Hence, as the development of \(e\) develops, closeness is understood relative to the development of \(e\), rather than \(e\) itself as \(e\) happened in \(w\).
  So, as Max progresses a little further each time a possible world in which Max crosses the street gets a little closer until, eventually, Max crosses the street.

  This is the core idea of \citeauthor{Landman:1992wh}'s account.
  To borrow a piece of terminology from \textcite{Dowty:1979vq}, \(e\) has sufficient \emph{inertia} to develop in some possible world \(v\), and as \(e\) develops in \(v\), inertia continues to build until Max crosses the road.

  However, an important restriction is placed on shifts to possible worlds.
  Intuitively, Max does not avoid being hit by bus \#\(j\) because Max has the strength to stop as moving bus.
  Yet, a possible world in which Max has the strength to stop a moving bus may be close to the world in which Max is not hit by bus \#\(j - 1\).
  In \citeauthor{Landman:1992wh}'s terminology, the relevant possible worlds in which \(e\) develops must be `reasonable'.
\end{note}

\subparagraph*{Summary}

\begin{note}[Summarising]
  To summarise the preceding:
  We began with the definition of a \fc{} (\autoref{def:fc}).
  Definition of a \fc{} relies of the idea of a \pevent{}.
  And, defined \pevent{} in terms of the truth of the progressive aspect applied to a minimal event.

  The exact details of \pevent{} depends on progressive.

  However, gnarly.
  Before turning to the progressive, consider ability.
  The following section --- \autoref{cha:sec:fcs-def:ability} --- will raise difficulties with this suggestion.%
  \footnote{
    And implicitly suggest that any sense of ability sufficient for purpose may be analysed in terms of progressive.
  }
\end{note}


\section{Illustrations}
\label{cha:fcs:sec:illu}

\begin{note}
  Intuition.
  In particular, proofs.

  Before turning to detailed account of \fc{1} in \autoref{cha:sec:fcs-def}, handful of \illu{1}.

  This section consists of two parts:

  \begin{enumerate}[label=]
  \item
    \TOCLine{cha:fcs:sec:illu:yes}

    Cases in which \(\pv{\phi}{v}\) from \(\Phi\) (plausibly) \emph{is} a \fc{}.
  \item
    \TOCLine{cha:fcs:sec:illu:no}

    Cases in which where \(\pv{\phi}{v}\) from \(\Phi\) is \emph{not} (clearly) a \fc{}.
  \end{enumerate}
\end{note}

\subsection{\fc{3}}
\label{cha:fcs:sec:illu:yes}

\begin{note}[Chess I]
  \begin{illustration}[Chess I]
    \label{illu:fc:chess:I}
    Consider the following game state:

    \mbox{ }\hfill%
    \begin{adjustbox}{minipage=\linewidth,scale=.9}
      \centering
      \newchessgame[
      setwhite={pa2,pb2,pc2,pd3,pf2,pg3,ra1,re1,bd4,kg1,qe5},
      addblack={ra8,pa7,ba6,pb5,rc8,pd5,pf7,kg8,qg4,ph7,ph4},
      ]%
      \setchessboard{showmover=false}%
      \chessboard
    \end{adjustbox}%
    \label{fig:chess:easy}%
    \hfill\mbox{ }

    Is possible for White to checkmate in a single move?%
    \footnote{
      \citeauthor{Emms:2000aa}' Puzzle 113 (\citeyear[33]{Emms:2000aa}).
    }
  \end{illustration}
\end{note}

\begin{note}
  \fc{2} of interest:%
  \footnote{
    May also consider whether is possible for White to checkmate in a single move to be a \fc{}, in the sense that it is possible to decide.
    In this case, reduces to solution.
    However, in general decidable may be \fc{} without either answer being a \fc{}.
    In particular, consider an arbitrary first-order formula.
    First-order logic is decidable, however determining which is a different matter.
  }
  \begin{itemize}
  \item
    It is possible for White to checkmate in a single move.
  \end{itemize}
  Clearest way is to do the reasoning, and then observe would not have made a different conclusion.
\end{note}

\begin{note}
  Consider different pieces.
  If like me, first move (\wmove{Qe8}) does not result in checkmate.
  However, do not conclude that it is not possible.
  Other moves to check (e.g\ \wmove{Qf6}, \wmove{Re4}, \wmove{h4}, etc.).

  At some point, consider\wmove{Qh8}, which results in checkmate.

  Simple.
  Key observation is that although not immediate, conclude it is possible for White to checkmate in a single move.
  And, would not have concluded otherwise, or indeed concluded something incompatible which would have prevented (for example, that it is possible for Black's king to move to b8 between the two rooks.)

  Perhaps get bored or distracted, and didn't conclude.
  Remains the case that \fc{}.
\end{note}

\begin{note}[Propositional logic generalised]
  Second option, general type of \scen{0} in two ways:

  Soundness:

  \begin{quote}
    Make sure all the rules are sound, whether these are basic or derived rules.
  \end{quote}

  Semantic entailment:

  \begin{illustration}[Propositional logic generalised]
    \label{illu:sketch:prop-logic}
    Suppose an agent has a good grasp of propositional logic.
    In particular:
    \begin{itemize}
    \item
      The agent has a good understanding of some formal proof system.
      For example, some Fitch-style system.
    \item
      The agent has a good understanding of some method to construct semantic proofs.
      For example, by constructing truth tables, or reasoning about valuation functions.
    \item
      The agent understands the proof system is sound.
      That is to say, the agent understands there exists a proof of some sentence \(A\) \emph{only if} \(A\) is true given an arbitrary valuation.
    \end{itemize}
    The agent constructs a proof of \(A\).

    Given the agent's understanding of propositional logic, the agent observes:
    \begin{quote}
      The construction is a proof of \(A\) \emph{only if} \(A\) is true given an arbitrary valuation.
    \end{quote}
  \end{illustration}

  Intuitively, if the agent were to reason about whether \(A\) is true given an arbitrary valuation and failed to conclude \(A\) is true given an arbitrary valuation, then the agent would not conclude the construction is a proof of \(A\).

  If were to go for semantic, then by soundness, works out.
\end{note}

% \begin{note}
%   \begin{illustration}[Modal logic I]
%     \label{illu:fc:logic:CR}
%     The modal system obtained from adding \(\Diamond\Box p \rightarrow \Box\Diamond p\) as an axiom to \(\mathbf{K}\) is canonical for the Church-Rosser property.

%     I.e. the canonical model \(W,R,V\) for \(\mathbf{K} + \Diamond\Box p \rightarrow \Box\Diamond p\) is such that \(\forall s,t,u((Rst \land Rsu) \rightarrow \exists v(Rtv \land Ruv))\).
%   \end{illustration}

%   \autoref{illu:fc:logic:CR} is a \fc{} for me.
%   Though, in contrast to the previous \illu{1}, I think there is a reasonable change that \autoref{illu:fc:logic:CR} is not a \fc{} for you.

%   Fairly routine, but two important things.
%   First, grasp on the relevant concepts.
%   If you are unaware of how to construct canonical models for normal modal logics, then unlikely that you will complete the relevant proof.
%   Second, sufficient familiarity with the relevant concepts.
%   The proof is mostly straightforward, though some care needs to be taken in showing that the canonical model for \(\mathbf{K} + \Diamond\Box p \rightarrow \Box\Diamond p\) has the Church-Rosser property.
%   Proof by contradiction is my preferred way of obtaining the result, but this requires keeping certain facts about the canonical model in mind.%
%   \footnote{
%     A slightly more interesting variation is showing that \(\mathbf{K} + \Diamond\Box p \rightarrow \Box\Diamond p\) is (strongly) complete with respect to the class of frame which have the Church-Rosser property without detour via a canonical model.
%   }

%   Similar features as \illu{1} given above.

%   In particular, perhaps clearer than \autoref{illu:gist:sudoku} in terms of mistakes.
%   For, go down some wrong path, still will not conclude until counterexample.
%   And, this is very hard to get.
% \end{note}

\begin{note}
  Four \illu{}.
  Share common characteristic.
  Result of deductive reasoning with more-or-less explicit collection of rules.

  These characteristics are not (nor any other shared characteristic) required.
  Clear account of why \(\pv{\phi}{v}\) from \(\Phi\) is a \fc{}.

  What is required is available information, conclusion, no divergence.
\end{note}

% \begin{note}[Non-deductive \illu{1}]
%   The following is a simple \illu{} involving non-deductive conclusion:
%   \begin{illustration}[Sunny days]
%     It's mid summer in the Bay Area.
%   \end{illustration}
%   For me, it is a \fc{} that it will not rain tomorrow.

%   Of course, I recognise there is a possibility that it \emph{may} rain tomorrow.
%   However, I haven't checked the weather forecast, and with no information to the contrary I see no way of \emph{failing} to conclude that tomorrow will be sunny.
%   You may object, and perhaps I am too quick to conclude that it will not rain tomorrow.

%   Still, no matter the gravitas with which I consider the possibility of rain, I am sufficiently committed to some uniformity principle that the principle, combined with past experience, lead me to conclude that it will be sunny tomorrow.
%   Hence, prior to reasoning, the truth of the proposition is a \fc{}.%
%   \footnote{
%     Same extends to various skeptical hypotheses.
%     Entertain the possibility that there is no external world, but nothing that prevents me from concluding that there is an external world.
%     Though, your perspective on such issues may differ.
%   }

%   Note, whether or not it rains tomorrow has no bearing on whether or not it is a \fc{} (for me) that it will not rain tomorrow.
%   What happens in the future has no direct bearing on what I may (or may not) conclude in the present.%
%   \footnote{
%     Consider~\citeauthor{Russell:1912th}'s chicken\dots (Cf.~\citeyear[63]{Russell:1912th})
%   }
% \end{note}

\begin{note}[Poppies]
  We conclude the \illu{1} with a slightly more speculative \illu{0}:
  \begin{illustration}[Poppies]
    \mbox{ }
    \vspace{-\baselineskip}
    \begin{quote}
      Was Tarquinius Superbus in seinem Garten mit den Mohnköpfen sprach, verstand der Sohn, aber nicht der Bote.

      [What Tarquinius Superbus said in the garden by means of the poppies, the son understood but the messenger did not].\newline
    \mbox{ }\hfill\mbox{(Cf.~\cite[3]{Kierkegaard:1983ta}, and~\cite[190]{Hamann:1822vp})}
  \end{quote}
  \vspace{-\baselineskip}
  \end{illustration}
  The above quite is from the epigraph to~\citeauthor{Kierkegaard:1983ta}'s \hyperlink{cite.Kierkegaard:1983ta}{Fear and Trembling}.
  \hyperlink{cite.Kierkegaard:1983ta}{H.\ Hong and E.\ Hong} detail the relevant background:

  \begin{quote}
    When the son of Tarquinius Superbus had craftily gotten Gabii in his power, he sent a messenger to his father asking what he should do with the city.
    Tarquinius, not trusting the messenger, gave no reply but took him into the garden, where with his cane he cut off the flowers of the tallest poppies.
    The son understood from this that he should eliminate the leading men of the city.%
    \mbox{ }\hfill\mbox{(\citeyear[339]{Kierkegaard:1983ta})}
  \end{quote}
  That he should eliminate the leading men of the city was a \fc{0} for Superbus' son, but not for the messenger.
  Or, at the very least Superbus \emph{expected} the command to eliminate the leading men of the city to be a \fc{} for his son.
\end{note}

\subsection{Not clearly \fc{1}}
\label{cha:fcs:sec:illu:no}

\begin{note}
  Provided a handful of (plausible) instances of knowing whether which (plausibly) involve \fc{1}.
  A pair of (plausible) instances whether which (plausibly) do not involve \fc{1}.
\end{note}

\begin{note}[Chess II]
  Observation that absence of \fc{} due to failure to conclude extends to other cases.
  What follows is a more difficult chess problem.
  \begin{illustration}[Chess II]
    \label{illu:fc:chess:II}
    Consider the following game state:

    \mbox{ }\hfill%
    \begin{adjustbox}{minipage=\linewidth,scale=0.9}
      \centering
      \newchessgame[
      setwhite={ka5,pa3,pb4,pc4,pe5,pf6,bg5,bh5},
      addblack={pa6,pb7,pc6,pe6,pf7,kc7,nd7,nd4},
      ]%
      \setchessboard{showmover=false}%
      \chessboard
    \end{adjustbox}%
    \label{fig:chess:intro}%
    \hfill\mbox{ }

    It is possible for Black to checkmate in four moves?%
    \footnote{
      \citeauthor{Emms:2000aa}' Puzzle 150 (\citeyear[33]{Emms:2000aa}).
    }
  \end{illustration}
  It is plausible that I would not conclude that it is possible for Black to checkmate in four moves or conversely.

  Though perhaps the bound is too low.
  If I gave it my all and attempted to work my way though all the possibilities present it may be the case that I conclude either way.
  Still, there are a lot of moves to consider, and I lack any intuition about which is correct.%
  \footnote{
    \citeauthor{Emms:2000aa} provides the following solution:
    \begin{quote}
      \variation{1... Nb6!}
      (threatening \variation{2... Nb3\#})
      \variation{2. b5}
      (or \variation{2. Bd1 Nxc4+} \variation{3. Ka4 b5\#})
      \variation{2... c5!}
      \variation{3. bxa6 Nxc4+}
      \variation{4. Ka4 b5\#}
      \textbf{(0-1)}%
      \mbox{}
      \hfill
      (\citeyear[46]{Emms:2000aa})
    \end{quote}
    My statement above remains true---I don't have sufficient background to parse this solution.
  }
  And, if you think I am doing myself a disservice, then a variant of \autoref{illu:fc:chess:II} may be restated with and increase number of moves.
\end{note}

% \begin{note}[ML II]
%   \begin{illustration}[Modal logic II]
%     \label{illu:fc:ML2}
%     The modal system \(\mathbf{GL} = \mathbf{K} + \Box(\Box p \rightarrow p) \rightarrow \Box p\) is weakly complete with respect to the class of finite strict partial orders (that is, the class of finite irreflexive transitive frames).
%   \end{illustration}

%   \autoref{illu:fc:ML2} is similar in structure to \autoref{illu:fc:logic:CR}.
%   Indeed, both proofs involve constructing a canonical model.
%   The key distinguishing feature of \autoref{illu:fc:ML2}, however, is the difficulty of establishing the canonical model has the desired properties.
%   In particular, the general method I keep in mind for proving the relevant result requires a syntactic proof that \(\vdash_{\mathbf{GL}} \Box p \rightarrow \Box \Box p\).
%   And, as I have failed to recall the relevant syntactic on sufficient occasion, I do not consider the result a \fc{0} from my understanding of modal logic.

%   Hence, the result (plausibly) fails to be a \fc{} from my understanding of modal logic because there is no guarantee that I would provide a proof if I set out to do so.%
%   \footnote{
%     On the other hand, I have completed the relevant proof a sufficient number of times.
%     So, the result is a \fc{0} from whatever premises are associated with my memory.
%   }
% \end{note}

\begin{note}
  Conflicting conclusions.

  \illu{3} may be obtained by taking a proposition-value pairing which conflicts with \fc{}.
  The following is \emph{not} a \fc{}.%
  \footnote{
    However, I suspect
    \[\frac{(3 + \sqrt{3})^{2} + \sqrt{6}^{2} - (2\sqrt{3})^{2}}{2(3 + \sqrt{3})\sqrt{6}} = \frac{1}{\sqrt{2}}\]
    is a \fc{}.
  }

  \begin{illustration}
    \[\frac{(3 + \sqrt{3})^{2} + \sqrt{6}^{2} - (2\sqrt{3})^{2}}{2(3 + \sqrt{3})\sqrt{6}} = \frac{1}{\sqrt{3}}\]
  \end{illustration}
  For, the equation does not hold.
\end{note}

\begin{note}
  \begin{illustration}[Knowing whether and belief]
    \citeauthor{Barker:1975un} suggests the following two principles hold with respect to knowing whether:%
  \footnote{
    \citeauthor{Barker:1975un} also, as far as I can tell, endorses the principles.
  }
    \begin{enumerate}[label=(\Alph*), ref=(\Alph*), noitemsep]
    \item
      \label{Barker:1975un:A}
      If \emph{S} knows whether \emph{p} and \emph{S} believes that \emph{p}, then \emph{p}.
    \item
      \label{Barker:1975un:B}
      If \emph{S} knows whether \emph{p} and \emph{S} believes that not-\emph{p}, then not-\emph{p}.%
      \mbox{ }\hfill\mbox{(\citeyear[281]{Barker:1975un})}
    \end{enumerate}
  \end{illustration}
  I suggest neither principle is \fc{}, as you may conclude counterexamples exist to both.%
  \footnote{
    For example, consider two agents, \emph{A} and \emph{B} playing chess where each move is timed.
  It's the end game, and \emph{A} believes that \emph{B} has a winning strategy.
  Further, \emph{A} (plausibly) knows whether \emph{B} has a winning strategy.
  For, an observer has determined whether or not \emph{B} has a winning strategy, and \emph{A} is capable of tracing the reasoning of the observer.
  So, if \ref{Barker:1975un:A} holds then \emph{B} has a winning strategy.
  But, the observer knows that \emph{B} \emph{does not} have a winning strategy, and \emph{A}'s belief is mistaken.
  }
\end{note}

\section{\fc{3} and \ros{1}}
\label{cha:fcs:sec:fcs-support}

\begin{note}
  \autoref{cha:sec:fcs-def}, account of \fc{1}.
  Now tie to \ros{1}.
\end{note}

\begin{note}
  \begin{proposition}
    \label{prop:fcs-only-if-support}
    \begin{itemize*}[noitemsep, label=\(\circ\)]
    \item
      An agent: \vAgent{}
    \item
      A proposition: \(\phi\)
    \item
      A value: \(v\)
    \item
      A \poP{0}: \(\Phi\)
    \item
      \mbox{ }
    \end{itemize*}

    \begin{enumerate}
    \item[\emph{If}:]
      \begin{enumerate}[label=\alph*., ref=(\alph*.)]
      \item
        \(\pvp{\phi}{v}{\Phi}\) is a \fc{0}.
      \end{enumerate}
    \item[\emph{then}:]
      \begin{enumerate}[label=\alph*., ref=(\alph*.), resume]
      \item
        A \ros{0} holds between \(\pv{\phi}{v}\) and \(\Phi\), for \vAgent{}.
      \end{enumerate}
    \end{enumerate}
    \vspace{-\baselineskip}
  \end{proposition}

  {
    \color{red}
    Why this is somewhat interesting.
  }
  However, before turning to the argument for \autoref{prop:fcs-only-if-support}, it is important to note the limitations of \autoref{prop:fcs-only-if-support} with respect to \issueConstraint{}.
  For, in order to argue against \issueConstraint{}, need some \(\pvp{\psi}{v'}{\Psi}\) such that answers \qWhyV{}.
  Answer \qWhyV{} only with dependence.
  Does not follow from \fc{0} that we get dependence.
\end{note}

\begin{note}
  The argument for \autoref{prop:fcs-only-if-support} is {\color{red} mostly immediate for ideas regarding support}.

  \begin{goal}
    If conclude only if \fc{}, then support, in part, answers \qWhyV{}.
  \end{goal}

  So, to get answer to \qWhyV{}, need dependency.
  Here, if not \support{} then not \fc{}.
  If not \fc{} then not conclude.

  This is fine, just need to be careful with the counterfactual.
  Relation between \support{} and \fc{} is plain conditional.
  So, it survives any counterfactual changes.
\end{note}

\begin{note}[Argument]
  Argument is straightforward.
  Possible support, by assumption.
  Contraposition.
  If not support, then no \fc{}.
\end{note}

\subsection{Potential \ros{1}}

\begin{note}
  Start with the following proposition.
  \begin{proposition}
    \label{prop:fcs-only-if-pot-support}
    \begin{itemize*}[noitemsep, label=\(\circ\)]
    \item
      An agent: \vAgent{}
    \item
      A proposition: \(\phi\)
    \item
      A value: \(v\)
    \item
      A \poP{0}: \(\Phi\)
    \item
      \mbox{ }
    \end{itemize*}

    \begin{enumerate}
    \item[\emph{If}:]
      \begin{enumerate}[label=\alph*., ref=(\alph*.)]
      \item
        \(\pvp{\phi}{v}{\Phi}\) is a \fc{0}, for \vAgent{}.
      \end{enumerate}
    \item[\emph{then}:]
      \begin{enumerate}[label=\alph*., ref=(\alph*.), resume]
      \item
        A potential \ros{} holds between \(\pv{\phi}{v}\) and \(\Phi\), for \vAgent{}.
      \end{enumerate}
    \end{enumerate}
    \vspace{-\baselineskip}
  \end{proposition}

  \begin{argument}{prop:fcs-only-if-pot-support}
    Suppose \(\pvp{\phi}{v}{\Phi}\) is a \fc{0}.
    Then, \pevent{} in which concludes.
    Now, consider the \pevent{}.
    The culmination of the event, agent concludes.

    So, from~\supportI{}, a \ros{} holds, for the agent.

    Therefore, in whatever sense event is potential, \ros{} between \(\pv{\phi}{v}\) and \(\Phi\) is likewise potential.
  \end{argument}

  For agent, there is no difference between witnessed \ros{} and potential \ros{}.
\end{note}

\begin{note}
  There is room for small worry.
  For, require that agent does an action.
  As the agent does the action, other things could happen.

  Set up so that a soon as agent does action, remote control.

  I don't find this particularly plausible.
  Prefer to think of events to clarify relation to \supportI{}.
  However, reduce to what is currently true of the agent.
  In this sense, know how.
\end{note}

\begin{note}
  \emph{Potential} \ros{}, but it does not follow that there is a \ros{}, for the agent.
\end{note}

\begin{note}
  \begin{proposition}
    \label{prop:pot-support-onlyIf-support}
    \begin{itemize*}[noitemsep, label=\(\circ\)]
    \item
      An agent: \vAgent{}
    \item
      A proposition: \(\phi\)
    \item
      A value: \(v\)
    \item
      A \poP{0}: \(\Phi\)
    \item
      \mbox{ }
    \end{itemize*}

    \begin{enumerate}
    \item[\emph{If}:]
      \begin{enumerate}[label=\alph*., ref=(\alph*.)]
      \item
        A potential \ros{} holds between \(\pv{\phi}{v}\) and \(\Phi\), for \vAgent{}.
      \end{enumerate}
    \item[\emph{then}:]
      \begin{enumerate}[label=\alph*., ref=(\alph*.), resume]
      \item
        A \ros{0} holds between \(\pv{\phi}{v}\) and \(\Phi\), for \vAgent{}.
      \end{enumerate}
    \end{enumerate}
    \vspace{-\baselineskip}
  \end{proposition}

  \begin{argument}{prop:pot-support-onlyIf-support}
    \supportII{}.
    It is possible for there to be.
    So, we have everything needed.
    Both necessary and sufficient.
    Hence, for agent, \ros{}.

    So, every necessary property that does not involve witnessing.
    But, then, every necessary property.
    Therefore, sufficient.
    For, if not sufficient, then missing a necessary property.
    Contradiction.

    Slight issue, disjunction of properties.
    But, this doesn't change the argument.
    Disjunction.
  \end{argument}
\end{note}

\begin{note}
  \color{red}
  Worry.
  \support{2} doesn't rely on witnessing.
  Now, if this goes through, then seems \support{} for any conclusion before making the conclusion.
  However, possible for the agent to reason to different conclusions.
  For, some faulty reasoning.
  Toggle the fault.
  Therefore, \support{} for contradictory conclusions.
\end{note}

% \subsubsection{\wit{3}}
% \label{sec:wit3}

% \begin{note}
%   Note, there is nothing that requires a \fc{} has no prior conclusion.

%   Hence, it may be the case that an agent has a \wit{} for a \fc{}.

%   \begin{proposition}[\wit{3} for \fc{1}]
%     \label{prop:wit-for-fc}
%     Possible for \wit{} from \fc{}.
%   \end{proposition}

%   \begin{argument}{prop:wit-for-fc}
%       Immediate by def of \wit{}.
%     \end{argument}
% \end{note}

\section{`Foregone-concluding'}
\label{sec:fc-progressive}

\begin{note}
  So, action such that would be concluding.
  On understanding of the progressive, no matter what happens, still conclude.
  Might need to get `lucky' with action.
  However, start then path to the conclusion.

  This `luck' is limited.
  Nothing conflicting.
\end{note}

\begin{note}
  Worry, jumping to conclusions.
  But, in this case, \pevent{} in which agent concludes that step of reasoning is bad.
  Hence, incompatible.
\end{note}

%%% Local Variables:
%%% mode: latex
%%% TeX-master: "master"
%%% End:
