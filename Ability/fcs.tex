\chapter{\fc{3}}
\label{cha:fcs}

\nocite{Ryle:1946tu}



\begin{note}
  This chapter introduces the core idea of this document:
  Some \prop{0}-\val{0} pair \(\pv{\phi}{v}\) being a \emph{\fc{}} from some \pool{0} \(\Phi\).

  Paraphrased, \(\pv{\phi}{v}\) is a \fc{} from \(\Phi\) for an agent just in case there is a possible \eiw{0} the agent to conclude \(\pv{\phi}{v}\) from \(\Phi\) and \evals{} every \prop{} in \(\Phi\) with its corresponding \val{} prior to the event.
\end{note}

\begin{note}
  The idea of a \fc{0} is central to the way we identify \fingfr{1}, and the way we generate counterexamples to \issueInclusion{}.
  However, \fc{1} as defined are compatible with \issueInclusion{}.
\end{note}

\begin{note}
  The first section of this chapter defines \fc{1}.
  The second section of this chapter provide a handful of \illu{1} of \prop{0}-\val{0}-\pool{0} pairs which are \fc{1} and \prop{0}-\val{0}-\pool{0} pairs which are not.
\end{note}

\section{Definition}
\label{cha:fcs:def}

\begin{note}[\fc{2} definition]
  \begin{definition}[\fc{3}]%
    \label{def:fc}%
    \vspace{-\baselineskip}
    \begin{itemize}
    \item
      \(\pv{\phi}{v}\) is a \emph{\fc{0}} from \(\Phi\) for \vAgent{} through an event \(\ed{}\).
    \end{itemize}

    \emph{If and only if}

    \begin{itemize}
    \item
      For some partition of \(\edn{}\) into sub-events \(\edn{1}, \dots, \edn{k}\) there are descriptions \(\edo{1}, \dots, \edo{k}\) such that clauses \ref{def:fc:ai}, \ref{def:fc:act} and \ref{def:fc:result} hold:
      %
      \begin{enumerate}[label=\Alph*., ref=\Alph*, series=fcCounter]
      \item
        \label{def:fc:ai}
        \(\ed{i}\) is an \eiw{0} \vAgent{} may do some action \(a_{i}\).
        %
      \item
        \label{def:fc:act}
        The \eiw[\(\edn{a_{i}}\)]{0} \vAgent{} does \(a_{i}\) is an \eiw{0} \vAgent{} is concluding \(\pv{\phi}{v}\) from \(\Phi\).
        %
      \item
        \label{def:fc:result}
        For each \prop{0}-\val{0} pair \(\pv{\phi'}{v'}\) in \(\Phi\), \vAgent{} \evals{} \(\phi'\) as having value \(v'\) prior to doing \(a_{i}\).
        %
      \end{enumerate}
    \end{itemize}
    \vspace{-\baselineskip}
  \end{definition}

  \noindent%
  The core idea of \(\pv{\phi}{v}\) being a \fc{} from \(\Phi\) is the availability of an action such that an \eiw{0} the agent concludes \(\pv{\phi}{v}\) from \(\Phi\) is in progress when the agent does the action.

  In order to easily connect \fc{1} to other ideas in this document, \fc{1} are defined with respect to events.
  This slightly complicates the definition.
  For, of interest is that the agent do something and as a result be concluding \(\pv{\phi}{v}\) from \(\Phi\) through the event.
  And, depending on your view of actions, there may be no single action available to the agent through the relevant event.
  So, rather than presuppose there is some unique action a potentially restrict the events \autoref{def:fc} applies to, \autoref{def:fc} allows breaking down the relevant event into a series of sub-events such that each sub-event has an associated action.
  Still, \(\edn{}\) itself is trivially a partition and sub-event of \(\edn{}\), so this complication may be easily ignored.
\end{note}

\begin{note}
  \fc{3} are about the possibility of an agent concluding.

  The sense with which something is \emph{foregone} is captured by Clause~\ref{def:fc:ai} and \autoref{cons:no-f-ref} (\autopageref{cons:no-f-ref}).
  For, by \autoref{cons:no-f-ref} the description \(\edo{i}\) of \(\edn{i}\) is limited to what is true of \(\edn{i}\) when \(\edn{i}\) happens.
  So, when \(\edn{i}\) happens it is true that the agent may do some action \(a_{i}\), before the agent does \(a_{i}\).

  And, the sense with which the something foregone is a \emph{conclusion} is captured by Clause~\ref{def:fc:act} and \assuPP{} (\autoref{assu:PP}).
  For, by Clause~\ref{def:fc:ai} doing the action \(a_{i}\) results in the agent concluding \(\pv{\phi}{v}\) from \(\Phi\).
  And, by \assuPP{}, if an agent is concluding \(\pv{\phi}{v}\) from \(\Phi\) there is a possible \eiw{0} the agent concludes \(\pv{\phi}{v}\) from \(\Phi\).

  So, prior to the agent doing \(a_{i}\), something about \(\edn{i}\) ensures it is the case the agent may be concluding \(\pv{\phi}{v}\) from \(\Phi\), and hence by \assuPP{} this something ensures the agent may conclude \(\pv{\phi}{v}\) from \(\Phi\).
\end{note}

\begin{note}
  Still, our interest is with conclusions which already follow from an \agpe{}, and clauses~\ref{def:fc:ai} and \ref{def:fc:act} alone do not quite capture this way a conclusion may be foregone.

  For example, consider an agent working in a library.
  At any moment the agent may look up and read the time from a clock on the wall.
  And, if the agent starts to read the clock the agent is clearly concluding \pv{\propI{It's 17:30}}{\valI{True}} from some \pool{} which includes the \agents{} perception of the clock.

  The role of Clause~\ref{def:fc:result} is to strengthen the sense with which the conclusion of \(\pv{\phi}{v}\) from \(\Phi\) is foregone by ensuring everything the agent requires to conclude \(\pv{\phi}{v}\) from \(\Phi\) is available to the agent prior to doing action \(a_{i}\).
  So, prior to the agent doing \(a_{i}\), something about \(\edn{i}\) ensures it is the case the agent may conclude \(\pv{\phi}{v}\) from \(\Phi\) from the way things are from the \agpe{} prior to doing \(a_{i}\).

  Clause~\ref{def:fc:result} rules out \pv{\propI{It's 17:30}}{\valI{True}} being a \fc{} from some \pool{} which includes the \agents{} perception of the clock, so long as the agent has not looked at the clock.
  However, Clause~\ref{def:fc:result} allows \pv{\propM{\rootsCon{}}}{\valI{True}} to be a \fc{} from some \pool{} which includes the \agents{} understanding of factorisation in \autoref{illu:gist:roots:F}, given the \agents{} understanding of factorisation is fixed prior to the event of \autoref{illu:gist:roots:F}.%
  \footnote{
    There may be further considerations which favour strengthening \autoref{def:fc} further.
    In particular, one may construct cases where something important changes after the agent performs the action.
    Still, I have not found a clearly compelling case.
    Perhaps some action leads to a transformative experience, and the relevant conclusion is not sufficiently connected to the \agpe{\agents{} present}.
    Still, I doubt it is worth complicating \autoref{def:fc} any further.
    The role of \autoref{def:fc} is to identify certain a phenomenon.
    Without additional restrictions this phenomenon may be a little broader than intended, but still includes the phenomenon of direct interest.
  }
\end{note}

\begin{note}
  Given the above, our technical term `\fc{}' associates with (by is not an analysis of) a sense of the common term `foregone conclusion' as an expected result of reasoning.%
  \footnote{
    For example, \textcite[6]{Jacquette:2002up} writes:
    \textquote{
      [Russell's] presuppositional semantics for the ‘existential' quantifier, [\dots] makes it an altogether philosophically foregone conclusion that sentences like ‘Pegasus is winged,' [\dots] are false.%
      }

      \citeauthor{Jacquette:2002up} is discussing what follows from~\citeauthor{Russell:1905aa}'s analysis of definite descriptions.
    Specifically, from \citeauthor{Russell:1905aa}'s analysis it follows \propI{Pegasus is winged} is \valI{False}.
    (There are no four-legged winged mammals, etc.)
  }
  This sense of the term `foregone conclusion' contrasts against a sense with which a \prop{0}-\val{0} pair has been decided in advance of reasoning.%
  \footnote{
    For example, \textcite[1228]{Kadane:1996vu}:
    \textquote{
      When can a Bayesian select an hypothesis \emph{H} and design an experiment (or a sequence of experiments) to make certain that, given the experimental outcome(s), the posterior probability of \emph{H} will be greater than its prior probably?
      We discuss an elementary result that establishes sufficient conditions under which this reasoning to a foregone conclusion cannot occur.%
    }
  }
  And, a sense with which a forgone conclusion is some unavoidable state of affairs.%
  \footnote{
    For example, \textcite[332--333]{kurufushamashii:2015un} write:
    \textquote{
      [どうぜ][\dots] Expresses an attitude of resignation or carelessness on the part of the speaker, in the sense that regardless of what s/he does, the conclusion or outcome is foregone and cannot be changed by the will or effort of an individual.
    }

    See also \citeauthor{Grice:1957vg}'s discussion of intention recognition (\citeyear[385]{Grice:1957vg}/\citeyear[219]{Grice:1989uf}), and \citeauthor{Machover:1996vu}'s preface of their approach to the G\"{o}del-Rosser First Incompleteness Theorem (\citeyear[viii]{Machover:1996vu}).
  }
\end{note}



\section{An application and some \illu{1}}

\begin{note}
  This section contains an application of \autoref{def:fc} to~\autoref{illu:gist:roots:F}, and a handful of more relaxed \illu{1} of \fc{1}.

  We begin with the application as it highlights the way we put the definition of a \fc{} to use in the remainder of this document.
  And, as with \autoref{obs:se-inst} (\autopageref{obs:se-inst}) we will appeal to the application again later in the document.

  The \illu{1} which follow the application are a little more free-flowing and intended to help build intuition about the phenomenon the definition of a \fc{} is designed to capture.
  Still, as the \illu{1} are only for intuition, these may safely be skipped over.
  (Though, you may prefer to take a look at the \illu{1} prior to working through the application.)
\end{note}


\subsection{An application}

\begin{note}
  \begin{application}[A \fc{} of \autoref{illu:gist:roots:F}]%
    \label{obs:var:factor:fc}%
    Given:
    %
    \begin{itemize}
    \item
      \(\edn{}\) is the event described by \autoref{illu:gist:roots:F}.
    \item
      \(\Phi\) includes the \agents{} understanding of factorisation prior to \(\edn{}\).
    \item
      \(\edo{}\) is the description `the agent concludes \propM{\rootsCon{}} has \val{0} \valI{True} from \(\Phi\)'.
    \item
      \(\edn{\flat}\) covers Step~\ref{illu:gist:roots:F:factor} of the \agents{} reasoning in \autoref{illu:gist:roots:F}
    \end{itemize}
    %
    It is the case that:
    %
    \begin{itemize}
    \item
      For any description \(\edo{\flat}\) true of \(\edn{\flat}\) which captures \(\ed{}\) is in progress, \(\edo{\flat}\) entails:
      \begin{itemize}
      \item
        \pv{\propM{\rootsCon{}}}{\valI{True}} is a \fc{} from \(\Phi\) through \(\ed{\flat}\).
      \end{itemize}
    \end{itemize}
    \vspace{-\baselineskip}
  \end{application}

  \begin{dets}{obs:var:factor:fc}
    Consider any event \(\edn{i}\) in a partition \(\edn{1}, \dots, \edn{k}\) of \(\ed{\flat}\) into sub-events.
    And, let \(\ed{i}\) capture \(\edn{i}\) is an \eiw{0} the agent is concluding \pv{\propM{\rootsCon{}}}{\valI{True}} from \(\Phi\).%
    \footnote{
      Note, \(\edo{i}\) is clearly true of \(\edn{i}\) as \(\edn{i}\) is a sub-event of \(\edn{\flat}\) and \(\edo{\flat}\) is true of \(\edn{\flat}\).
    }
    \medskip

    \noindent%
    Our goal is to show clauses \ref{def:fc:ai}, \ref{def:fc:act} and  \ref{def:fc:result} of \autoref{def:fc} are satisfied.
    \medskip

    \noindent%
    To begin, observe that by assumption \(\ed{i}\) is such that the agent is concluding \pv{\propM{\rootsCon{}}}{\valI{True}} from \(\Phi\).
    Therefore, by \assuPP{} (\autopageref{assu:PP}) there is a possible \eiw{0} the agent concludes \pv{\propM{\rootsCon{}}}{\valI{True}} from \(\Phi\).
    And, as \(\edn{\flat}\) is the \eiw{0} the agent figures out \rootsConEqExV{6}{3}{2}, it is clear \(\ed{i}\) is not an \eiw{0} the agent concludes \pv{\propM{\rootsCon{}}}{\valI{True}} from \(\Phi\).
    Hence, in order for it to there to be a possible \eiw{0} the concludes \pv{\propM{\rootsCon{}}}{\valI{True}} from \(\Phi\), there must be some action the agent may do such that the agent is concluding \pv{\propM{\rootsCon{}}}{\valI{True}} from \(\Phi\) when they do the action.
    So, clauses~\ref{def:fc:ai} and~\ref{def:fc:act} of \autoref{def:fc} are satisfied.
    \medskip

    \noindent%
    Further, we assumed the agent understands factorisation prior to \(\ed{}\), and as \(\ed{\flat}\) is included in \(\ed{}\), the agent understands factorisation prior to \(\ed{\flat}\) by assumption.
    Hence, for each \prop{0}-\val{0} pair \(\pv{\phi'}{v'}\) in \(\Phi\), the agent \evals{} \(\phi'\) as having value \(v'\) prior to \(\ed{i}\).
    Therefore, Clause~\ref{def:fc:result} of \autoref{def:fc} is satisfied.
    \medskip

    \noindent%
    Finally, \(\edn{\flat 1}, \dots, \edn{\flat k}\) was an arbitrary partition of \(\ed{\flat}\) into sub-events and we have shown clauses \ref{def:fc:ai},~\ref{def:fc:act} and~\ref{def:fc:result} are satisfied with respect to any sub-event of the partition, it follows \pv{\propM{\rootsCon{}}}{\valI{True}} is a \fc{} from \(\Phi\) through \(\ed{\flat}\).
  \end{dets}
\end{note}

\begin{note}
  In short, \autoref{obs:var:factor:fc} reduces to observing that if \(\ed{\flat}\) is an \eiw{0} an agent is concluding \(\pv{\phi}{v}\) from \(\Phi\) then there must be some possible \eiw{0} the agent concludes \(\pv{\phi}{v}\) from \(\Phi\) by \assuPP{}.
  Hence, so long as the agent does not conclude \(\pv{\phi}{v}\) from \(\Phi\) during \(\edn{\flat}\) then through \(\edn{\flat}\) there must be some action such that the agent continues to be concluding \(\pv{\phi}{v}\) from \(\Phi\).
  So, clauses \ref{def:fc:ai} and \ref{def:fc:act} are almost trivially satisfied.
  And, Clause~\ref{def:fc:result} is satisfied by assumption.%
  \footnote{
    Note, \autoref{obs:se-inst} and \autoref{obs:var:factor:fc} combined show \pv{\propM{\rootsCon{}}}{\valI{True}} being a \fc{} from \(\Phi\) answers `why' (in the sense of \qWhy{}) the agent concluded \pv{\propM{\rootsCon{}}}{\valI{True}} from \(\Phi\).
  }
\end{note}



\subsection{Illustrations}
\label{cha:fcs:illu}


\begin{note}
  The following \illu{1} follow a set pattern.
  First, a \scen{0} is given as background.
  Following, a \prop{0}-\val{0}-\pool{0} pairing of interest is highlighted.
  And, discussion follows about whether the \prop{0}-\val{0} pair is a \fc{} from the \pool{0}.
\end{note}



\paragraph{Illustration 1}


\begin{note}
  \begin{scenario}[Paltry]
    \label{scen:fc:chick}%
    An agent reads Puzzle~25 of \citeauthor{Dudeney:1995aa}'s \citetitle{Dudeney:1995aa}:
    %
    \begin{quote}
      Three chickens and one duck sold for as much as two geese; one chicken, two ducks, and three geese were sold together for \$25.00. What was the price of each bird in an exact number of dollars?%
      \mbox{ }\hfill\mbox{(\citeyear[9]{Dudeney:1995aa})}
    \end{quote}
    \vspace{-\baselineskip}
  \end{scenario}

  \noindent%
  The \prop{0}-\val{0} pair of interest is the solution to the puzzle:
  %
  \begin{enumerate}[label=C\thescenarioCounter., ref=C\thescenarioCounter]
  \item
    \label{scen:fc:chick:c}
    \pv{\propI{The price of a chicken was \$2.00, for a duck \$4.00, and for a goose \$5.00}}{\valI{True}}
  \end{enumerate}
  %
  Intuitively, \ref{scen:fc:chick:c} is a \fc{} from some \pool{} \(\Phi_{\ref{scen:fc:chick:c}}\) which characterises the \agents{} understanding of natural langauge algebraic problems.
  For, so long as the agent has time to attempt a solution after reading the puzzle:
  %
  \begin{enumerate}[label=\alph*., ref=\alph*]
  \item
    \label{scen:fc:chick:ex:ai}
    The agent may attempt to solve Puzzle~25.
  \item
    \label{scen:fc:chick:ex:a}
    When the agent attempts to solve the puzzle, the agent is surely solving the puzzle given their understanding of natural langauge algebraic problems.
  \item
    \label{scen:fc:chick:ex:b}
    The \agents{} understanding of natural langauge algebraic problems is present prior to the agent attempting a solution.
  \end{enumerate}
  %
  Notes \ref{scen:fc:chick:ex:ai}, \ref{scen:fc:chick:ex:a}, and \ref{scen:fc:chick:ex:b} correspond to clauses \ref{def:fc:ai} \ref{def:fc:act} and \ref{def:fc:result} of \autoref{def:fc}, and we ignore the possibility of refining `trying' into a collection of finer-grained actions.
\end{note}


\begin{note}
  In general, puzzles like that of \autoref{scen:fc:chick} are a simple way of identifying \fc{1}.
  For, the puzzle is designed such that, given appropriate background, the agent has all the information required to solve the puzzle, once read.
  And, though we limit ourselves to a single example from \citeauthor{Dudeney:1995aa}'s book the book contains roughly 500 or so puzzles for which there is plausibly an associated \fc{0} from some \pool{} given a reader with the sufficient background.%
  \footnote{
    Some require \evals{} you may not have.
    For example, 527 on page 217 assumes knowledge of the size of some American coins.
    And others do not have well-defined answers, such 523 on page 215.
  }
  And, various other books full of similar puzzles are available.

  The downside of such puzzles is absence of a clear way to argue the \prop{0}-\val{0} pair is a \fc{}.
  Puzzles like Puzzle~25 involve some \emph{je ne sais quoi}.
  It is straightforward to write \(3c + 1d = 2g\) and \(1c + 2d + 3g = 25\), multiply the first equation by \(3\), the right equation by \(2\), substitute, and then simplify.
  However, this is a solution given in hindsight.
  And, the definition of a \fc{} requires such a solution to be given in foresight.

  Most cases are like \autoref{scen:fc:chick}.
  Some \prop{0}-\val{0} pair is intuitively a \fc{} from some \pool{}, though it is hard to specify what ensures \prop{0}-\val{0} pair is a \fc{} from the \pool{}.
  Still, in some cases we may do a little better.
  The following \illu{0} sketches an argument for some \prop{0}-\val{0} pair being a \fc{} from some \pool{}.
\end{note}



\paragraph{Illustration 2}


\begin{note}[Chess I]
  \begin{scenario}[Chess I]%
    \label{illu:fc:chess:I}%
    An agent reads \citeauthor{Emms:2000aa}' Puzzle 113 (\citeyear[33]{Emms:2000aa}):
    \begin{quote}
      \mbox{ }\hfill%
      \begin{adjustbox}{minipage=\linewidth,scale=0.75}
        \centering
        \newchessgame[
        setwhite={pa2,pb2,pc2,pd3,pf2,pg3,ra1,re1,bd4,kg1,qe5},
        addblack={ra8,pa7,ba6,pb5,rc8,pd5,pf7,kg8,qg4,ph7,ph4},
        ]%
        \setchessboard{showmover=false}%
        \chessboard
      \end{adjustbox}%
      \label{fig:chess:easy}%
      \hfill\mbox{ }
      \begin{center}
        Is possible for White to checkmate in a single move?
      \end{center}
    \end{quote}
    \vspace{-\baselineskip}
  \end{scenario}
\end{note}

\begin{note}
  The \prop{0}-\val{0} pair of interest is:

  \begin{enumerate}[label=C\thescenarioCounter., ref=C\thescenarioCounter]
  \item
    \label{illu:fc:chess:I:c}
    \pv{\propI{It is possible for White to checkmate in a single move}}{\valI{True}}
  \end{enumerate}
  %
  Consider a \pool{} which contains \prop{0}-\val{0} pairs which capture rules of chess and the game state.
  \pool{} prior to reasoning.
  For, the rules of chess are independent, and do not need to start reasoning in order to inspect setup of puzzle.

  The broad motivation for \ref{illu:fc:chess:I:c} being a \fc{} from \(\Phi_{\ref{illu:fc:chess:I:c}}\) parallels the discussion of  \ref{scen:fc:chick:c} being a \fc{} from \(\Phi_{\ref{scen:fc:chick:c}}\), above.

  Assume the relevant action is `begin an attempt to solve \citeauthor{Emms:2000aa}' Puzzle 113 by exhaustive search'.
  And, consider following (motivated) conditionals:

  \begin{enumerate}[label=\arabic*., ref=(\arabic*)]
  \item
    \label{illu:fc:chess:I:cond:1}
    If the agent picks \wmove{Qh8}, then agent is concluding \ref{illu:fc:chess:I:c} from \(\Phi_{\ref{illu:fc:chess:I:c}}\).
    \smallskip

    By assumption, the agent has a basic understanding of chess and is motivated.
    And, as the agent needs to verify \wmove{Qh8} in checkmate, the agent will.
  \item
    \label{illu:fc:chess:I:cond:2}
    If the agent picks a move other than \wmove{Qh8} then after some reasoning the agent picks a novel move.%
    \smallskip

    By parallel reasoning.
    For, the agent needs only verify the move other than \wmove{Qh8} fails to result in checkmate, and then pick some other move.
    The agent will verify, given their understanding of chess.
    And, the agent will pick some novel move as their strategy is exhaustive search.
  \end{enumerate}
  %
  Now, whichever move the agent picks, either \ref{illu:fc:chess:I:cond:1} or \ref{illu:fc:chess:I:cond:2} is true.
  And, as there are finitely many moves for the agent to pick, \ref{illu:fc:chess:I:cond:1} will (eventually be true).
  Hence, as an \eiw{0} the agent concludes \ref{illu:fc:chess:I:c} is in progress when \ref{illu:fc:chess:I:cond:1} is true, an \eiw{0} the agent concludes \ref{illu:fc:chess:I:c} is (also) in progress when \ref{illu:fc:chess:I:cond:2} is true.
\end{note}

\begin{note}
  Note, if \autoref{illu:fc:chess:I} is recast as a game of chess, and the agent of interest in White, it is plausible \ref{illu:fc:chess:I:c} has practical counterpart:
  %
  \begin{enumerate}[label=C\thescenarioCounter\('\)., ref=C\thescenarioCounter\('\)]
  \item
    \label{illu:fc:chess:I:c:p}
    \pv{\propI{\wmove{Qh8}}}{\valI{Do}}
  \end{enumerate}
  %
  For, a careful agent with a desire to win the game may do the reasoning sketched above, and then checkmate.
\end{note}



\paragraph{Illustration 3}

\begin{note}[Chess II]
  Still, a basic understanding of chess does not entail any chess puzzle is a \fc{}.
  For example, consider \citeauthor{Emms:2000aa}' Puzzle 150:

  \begin{scenario}[Chess II]%
    \label{illu:fc:chess:II}%
    An agent reads \citeauthor{Emms:2000aa}' Puzzle 150 (\citeyear[33]{Emms:2000aa}):
    \begin{quote}
      \mbox{ }\hfill%
      \begin{adjustbox}{minipage=\linewidth,scale=0.75}
        \centering
        \newchessgame[
        setwhite={ka5,pa3,pb4,pc4,pe5,pf6,bg5,bh5},
        addblack={pa6,pb7,pc6,pe6,pf7,kc7,nd7,nd4},
        ]%
        \setchessboard{showmover=false}%
        \chessboard
      \end{adjustbox}%
      \label{fig:chess:intro}%
      \hfill\mbox{ }
      \begin{center}
        It is possible for Black to checkmate in four moves?
      \end{center}
    \end{quote}
    \vspace{-\baselineskip}
  \end{scenario}

  \noindent%
  The conclusion of interest is:
  % 
  \begin{enumerate}[label=C\thescenarioCounter., ref=C\thescenarioCounter]
  \item
    \label{illu:fc:chess:II:c}
    \pv{\propI{It is possible for Black to checkmate in four moves}}{\valI{True}}
  \end{enumerate}
  % 
  The difference between \autoref{illu:fc:chess:II} and \autoref{illu:fc:chess:I} is the difficulty of the puzzle.%
  \footnote{
    \citeauthor{Emms:2000aa} suggests:
    \textquote{%
      \variation{1... Nb6!}%
      (threatening \variation{2... Nb3\#})%
      \variation{2. b5}%
      (or \variation{2. Bd1 Nxc4+} \variation{3. Ka4 b5\#})%
      \variation{2... c5!}%
      \variation{3. bxa6 Nxc4+}%
      \variation{4. Ka4 b5\#}%
      \textbf{(0-1)}%
    }
    (\citeyear[46]{Emms:2000aa}).
  }
  And, a basic understanding of chess does not imply the capacity to work through a complex sequence of moves in the relevant situation.%
  \footnote{
    I gave up after fifteen minutes or so.
    Hence it seems~\ref{illu:fc:chess:II:c} was not a \fc{} for me.
  }\(^{,}\)%
  \footnote{
    An agent may get guess a suitable sequence of four moves and verify they work, but if an agent is guessing, there is nothing to guarantee there next guess is correct.
    Hence, in this case \ref{illu:fc:chess:II:c} likewise fails to be a \fc{} from some \pool{}.
    And, more generally we define \fc{0} in terms of events in progress in part to rule out possible conclusions of this kind.
  }
  And, if the agent lacks the capacity to work through the relevant complex sequence, then an \eiw{0} the agent concludes \ref{illu:fc:chess:II:c} from some \pool{} is not in progress.
  Likewise, the agent may have the relevant capacity but have no interest in chess puzzles, be too resource constrained.
  % Still, though \ref{illu:fc:chess:II:c} may fail to be a \fc{} from \(\Phi_{\ref{illu:fc:chess:II:c}}\), \ref{illu:fc:chess:II:c} may be a \fc{} from some distinct \pool{} \(\Phi_{\ref{illu:fc:chess:II:c}}'\) but be a \fc{} from some distinct \pool{} \(\Phi'\) as the agent prefers reasoning from \(\Phi'\) as opposed to \(\Phi\).
  % Using a chess engine to solve chess problems seems to defeat the purpose of thinking about the problem, but I almost always prefer to use a calculator for mildly complex arithmetic such as \(4^{4!}\) over my own understanding of arithmetic.
\end{note}

% \paragraph{Illustration 4}

% \begin{note}
%   In some cases it is possible to easily point to something which prevents an agent from drawing a conclusion.
%   For example, consider the following story:

%   \begin{scenario}[A copper kettle]%
%     \label{illu:kettle}%
%     An agent is committed to the following principle:%
%   \footnote{
%     Motivated by the observations that
%     \begin{enumerate*}[label=(\alph*), ref=(\alph*)]
%     \item A.\ has provided testimony only if what A.\ has said is true.
%       And,
%     \item what A.\ has said is true only if the three points of A.'s defence are jointly consistent.
%     \end{enumerate*}
%   }
%   %
%   \begin{itemize}
%   \item
%     An agent has provided testimony \emph{only if} if what they say is internally consistent.
%   \end{itemize}
%   %
%   And listens to A.'s defence:
%     %
%     \begin{quote}
%       A.\ borrowed a copper kettle from B.\ and after he had returned it was sued by B.\ because the kettle now had a big hole in it which made it unusable. His defence was: ``%
%       First, I never borrowed a kettle from B.\ at all; secondly, the kettle had a hole in it already when I got it from him; and thirdly, I gave him back the kettle undamaged.%
%       ''%
%       \mbox{ }\hfill\mbox{(\cite[62]{Freud:1960wx})}
%     \end{quote}
%     \vspace{-\baselineskip}
%   \end{scenario}

%   \noindent%
%   The conclusion of interest is:
%   % 
%   \begin{enumerate}[label=C\thescenarioCounter., ref=C\thescenarioCounter]
%   \item
%     \label{illu:kettle:c}
%     \pv{\propI{A.'s defence is testimony}}{\valI{True}}
%   \end{enumerate}
%   % 
%   Further, suppose 
%   % 
%   It seems \ref{illu:kettle:c} is not a \fc{} from any \pool{} for the agent.
%   For, it is not possible for the agent to conclude \pv{\propI{The three points of A.'s defence are jointly consistent}}{\valI{True}} given the principle the agent is committed to.
%   Indeed, it seems clear \pv{\propI{The three points of A.'s defence are jointly \emph{in}consistent}}{\valI{True}} is a \fc{} from any relevant \pool{}.
%   For, it takes only a moments reflection to observe it is not possible to return a kettle one has not borrowed.
% \end{note}

% \begin{note}
%   Likewise, \propM{0 = 1}, value \valI{True}, and a \pool{} which includes a common understanding of arithmetic during their normal workday.
%   For, conflicts with a standard understanding of the natural numbers.

%   Of course, mistakes happen, and one may conclude some contradiction is true.
%   However, there is a distinction between a mistake which happens when an agent concludes, and a mistake present in the agent's understanding of some subject matter.
% \end{note}

% \begin{note}
%   Still, nothing prevents two conflicting \prop{0}-\val{0} pairs from being \fc{1}.
%   We argued \autoref{illu:kettle:c} is not a \fc{} due to requiring testimony is consistent.
%   However, an agent may be a little more flexible, and willing to ignore whether testimony is consistent if they are swayed by rhetoric.
%   Hence, 
%   For example, consistency of testimony, but may also be swayed by A.'s rhetoric.
%   So, focus on content, or focus on presentation.
%   Different actions, different conclusions.

%   A \fc{} only captures possible conclusion.
%   Nothing in particular hangs on this.
%   We refrain from placing additional constraints on \fc{1} in order to keep the definition of a \fc{} simple.
%   If you prefer to strengthen definition with justification, etc. that's okay.
%   And, we implicitly assume \fc{1} are sensible.
% \end{note}

% \paragraph{Illustration 5}

% \begin{note}
%   Finally, in some cases a \prop{0}-\val{0} pair may fail to be a \fc{} from a specific \pool{} due to the absence of some \prop{0}-\val{0} pairs.

%   \begin{scenario}[Stag hunt]%
%     \label{fc:sh}%
%     Two hunters may hunt either stag or hare.
%     It is possible for each hunter to capture hare alone, but to capture a stag requires cooperation.
%     Hence, the hunters expect a payoff for hunting hare regardless of what the other does, but they one expect a payoff for hunting hare if they do so together.
%     And, as a stag is much larger than a hare, the expected payoff for a hunting stag (together) is higher than hunting hare.
%   \end{scenario}

%   \noindent%
%   The conclusion of interest is:

%   \begin{enumerate}[label=C\thescenarioCounter., ref=C\thescenarioCounter]
%   \item
%     \label{fc:sh:c}
%     \pv{\propI{Hunt stag}}{\valI{Do}}
%   \end{enumerate}
%   % 
%   \ref{fc:sh:c} is plausibly fails to be a \fc{} for either hunter without information about what the other hunter is inclined to do, as there is no expected payoff for hunting stag alone.%
%   \footnote{
%     \citeauthor{Skyrms:2004aa} expands:
%     \begin{quote}
%       It is clear that a pessimist, who always expects the worst, would hunt hare [and] a cautious player, who was so uncertain that he thought the other player was as likely to do one thing as another, would also hunt hare.
%       [\dots]
%       That is not to say that rational players could not coordinate on the stag hunt equilibrium that gives them both a better payoff, but it is to say that they need a measure of trust to do so.%
%       \mbox{ }\hfill\mbox{(\citeyear[3]{Skyrms:2004aa})}
%     \end{quote}
%     And, we may say \pv{\propI{Hunt hare}}{\valI{Do}} is a \fc{} for the both a pessimist and cautious player.
%   }
% \end{note}

\section[Notes]{Notes \hfill (Optional)}



\begin{note}
  \fc{3} are defined via an \eiw{0} an agent is concluding.
  And, given \assuPP{}, an \eiw{0} an agent is concluding entails the existence of a (possible) \eiw{0} the agent concludes.
  So, definition similar to \autoref{def:fc} may be given which uses an \eiw{0} an agent concludes, rather than an \eiw{0} an agent is concluding.

  Still, we do not define \fc{3} via conclusions.
\end{note}

\begin{note}
  The primary motivations is pragmatic.

  Our interest with \fc{1} is something about the agent as they are which ensures it is possible for the agent to make the relevant conclusion.
  \autoref{def:fc} focuses on the agent as they by focusing on an action the agent may perform such that they are concluding.

  In contrast, stating \(\pv{\phi}{v}\) is a \fc{} from \(\Phi\) for an agent just in case the agent may perform such that they are \emph{conclude} \(\pv{\phi}{v}\) from \(\Phi\) draws attention to the event.
  This focuses on what may happen.
\end{note}

\begin{note}
  A secondary motivation is intuition.

  An agent is concluding just in case an \eiw{0} the agent concludes is in progress.
  Hence, sense of possibility with which an \eiw{0} the agent concludes is possible is constrained by an understanding of events in progress.
  (See~\autoref{sec:progressive}).

  This sense of possibility helps rule out conclusions which are possible, but are not, intuitively, forgone.

  For example, \autoref{illu:fc:chess:I}, the agent may conclude \pv{\propI{It is \emph{not} possible for White to checkmate in a single move}}{\valI{True}} from a \pool{} \(\Phi\) which includes their understanding of chess.
  Mistakes happen.

  Still, a mistake happening is distinct from a mistake being in progress.

  For example, it is unlikely there is an \eiw{0} the agent is \emph{concluding} \pv{\propI{It is not possible for White to checkmate in a single move}}{\valI{True}} from \(\Phi\).
\end{note}


\begin{note}
  While the definition of a \fc{1} is designed with certain features in mind, the way \fc{1} are defined plausibly fails to capture some natural cases.

  For example, consider the following \scen{0}:

  \begin{scenario}[ジョジョリオン]%
    \label{scen:jojo}%
    \nocite{huangmufeiluyan:2011aa}%
    An agent is skimming through the chapter titles of a book and translating the titles.
    A handful of translations and titles are:

    \begin{center}
      \bgroup
      \def\arraystretch{1.125}
      \begin{tabular}{R{.45\textwidth} L{.45\textwidth}}
        Translation & Title \\
        \hline
        Soft and wet & ソフト&ウェット \\
        Every day is summer vacation & 毎日が夏休み \\
        A hair clip from ??? period & 清の時代の髪留め \\
      \end{tabular}
      \egroup
    \end{center}

    \noindent%
    The agent then reads 「無事が何より」.
  \end{scenario}

  \noindent%
  The conclusion of interest is:
  %
  \begin{enumerate}[label=C\thescenarioCounter., ref=C\thescenarioCounter]
  \item
    \label{scen:jojo:c}
    \pv{「無事が何より」 \propI{translates to `Safety first'}}{\valI{True}}
  \end{enumerate}

  \noindent%
  It is plausible the agent may conclude \ref{scen:jojo:c}, so long as the agent is fairly good at translation.
  Still, \ref{scen:jojo:c} is plausibly not a \fc{} (as defined) from any relevant \pool{}.
  For, multiple translations are possible, and there may be nothing about the agent which favours one translation over any other.

  For example, 「無事が何より」 may (also) be translated to `Safety above everything else'.
  So, in order for the agent to be concluding \pv{「無事が何より」 \propI{translates to `Safety first'}}{\valI{True}}, something about the agent must favour `Safety first' as a translation over `Safety above everything else'.%
  \footnote{
    In some cases there is a unique translation, and hence a plausible \fc{}.
    For example, 「ソフト&ウェット」 is translated as 'Soft and Wet'.
    And, given some background context --- a number of other The Artist Formerly Known As Prince references --- and general knowledge, other translations (e.g.\ 'software and wetware') are ruled out.
  }
\end{note}

\begin{note}
  This noted, what matters for the overall argument of this document is what is captured by the definition of a \fc{} rather than what fails to be captured.
  So, there is no particular need for you to stick to \fc{1} as defined over an intuitive understanding of \fc{1}, so long as the extension of your intuitive understanding includes the extension of the definition.
\end{note}



\section*{Summary}


\begin{note}
  This chapter defined \fc{1}, applied the definition of a \fc{} to \autoref{illu:gist:roots:F} to highlight the way we put the definition of a \fc{} to use in the remainder of this document, and provided a handful of \illu{1} of \prop{0}-\val{0}-\pool{0} pairs such that the \prop{0}-\val{0} pair is (not) a \fc{} from the \pool{} to help firm intuition about the phenomenon captured.

  Additional applications and \illu{1} of \fc{1} follow in later chapters.
\end{note}


%%% Local Variables:
%%% mode: latex
%%% TeX-master: "master"
%%% TeX-engine: luatex
%%% End:
