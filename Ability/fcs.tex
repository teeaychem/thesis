\chapter{\fc{3}}
\label{cha:fcs}

\nocite{Ryle:1946tu}

\section{Introduction}
\label{cha:fcs:sec:introduction}

\begin{note}
  \autoref{sec:clar:type-of-scen} introduced the general type of \scen{} we are interested in.

  Here, \fc{1}.
  Two key things.
  \autoref{cha:sec:fcs-def}, account of \fc{1}.
  \autoref{cha:fcs:sec:fcs-support}, \fc{1} and support.

  If \fc{}, then relation of support.

  Important idea.
  However, limited.
  Support without witnessing doesn't raise a problem for \issueConstraint{}.
  Rather, support needs to be \emph{why}.

  \autoref{cha:zS} will develop, \check{1} on concluding.

  Main focus is \autoref{cha:sec:fcs-def}.
  What it is for some proposition-value pairing \(\pv{\phi}{v}\) to be a \fc{} from some pool of premises \(\Phi\).
  With the right set-up, \support{} --- focus on \autoref{cha:fcs:sec:fcs-support} --- will be simple.

  Progressive, action which is concluding \(\pv{\phi}{v}\) from \(\Phi\).
  Contrast with ability, `ability to conclude \(\pv{\phi}{v}\) from \(\Phi\)'.
  Argue that there is no clear path.
  Though, implicit upshot is reducing ability to progressive (given assumption).
\end{note}

\begin{note}
  Breakdown
  \begin{itemize}
  \item
    \autoref{cha:sec:fcs-def} definition of \fc{0}.
  \item
    \autoref{cha:fcs:sec:fcs-support}, support.
  \end{itemize}
\end{note}

\begin{note}
  Support.

  Idea here is something distinguished about concluding, but is independent of whether or not the agent has witnessed concluding.

  \autoref{idea:support} and \autoref{idea:support:possible}.

  Noted, independence does not entail relation of support from agent's perspective without concluding.

  Parallel to propositional and doxastic justification.
\end{note}

\section{Intuition}
\label{sec:intuition}

\begin{note}
  General term `foregone conclusion' is ambiguous.
  \begin{itemize}
    \item
    Inevitable results of reasoning.
  \item
    Conclusion which has been settled in advance of reasoning.
  \end{itemize}

  Interest is with the second meaning.
  Two examples of general use.

  % \begin{quote}
  %   [どうぜ]\dots Expresses an attitude of resignation or carelessness on the part of the speaker, in the sense that regardless of what s/he does, the conclusion or outcome is foregone and cannot be changed by the will or effort of an individual.%
  %   \mbox{ }\hfill\mbox{(\citeyear[332--333]{kurufushamashii:2015un})}
  % \end{quote}
  A clear example of the first meaning is found in~\citeauthor{Machover:1996vu}'s~\citetitle{Machover:1996vu}:

  \begin{quote}
    I have omitted its proof, but added a detailed analysis of the meaning of the lemma and the reason why its proof works. When this is understood, the proof itself becomes a mere technicality, almost a foregone conclusion.%
    \mbox{ }\hfill\mbox{(\citeyear[viii]{Machover:1996vu})}
  \end{quote}

  \citeauthor{Machover:1996vu} is discussing a proof, and whether or not it is inevitable that one would complete the proof (conclude that the relevant theorem is true) after understanding the lemma and why it works.
  There is no relevant sense in which the truth of the theorem has been settled in advance of reasoning.
  Though, as the proof is somewhat difficult,~\citeauthor{Machover:1996vu} only states the proof is `almost' a foregone conclusion.%
  \footnote{
    The proof is question is of the G\"{o}del-Rosser First Incompleteness Theorem.
    (\citeyear[Cf.][226]{Machover:1996vu})
  }%
  \(^{,}\)
  \footnote{
    For a similar example without qualification, consider the following from~\textcite{Jacquette:2002up}:
    \begin{quote}
    It is nevertheless important to recognize that Russell's evaluation of such sentences as false is predetermined by his existence presuppositional semantics for the ‘existential' quantifier, and by the fact that his logic permits no alternative means of considering the semantic status of sentences ostensibly containing proper names for nonexistent objects.
    This makes it an altogether philosophically foregone conclusion that sentences like ‘Pegasus is winged,' which many logicians would otherwise consider to be true propositions of mythology, are false.%
    \mbox{ }\hfill\mbox{(\citeyear[6]{Jacquette:2002up})}
  \end{quote}
  }

  For an additional example, consider the following from~\citeauthor{Grice:1957vg}'s~\citetitle{Grice:1957vg}:%
  \footnote{
    Same is found in \textcite[219]{Grice:1989uf}.
  }
  \begin{quote}
    He intends the audience's recognition of his intention to produce that response to be effective in producing that response.
    He does not regard it as a foregone conclusion that his action will produce the intended response, whether or not his intention is recognized.\newline
    \mbox{ }\hfill\mbox{(\citeyear[385]{Grice:1957vg})}
  \end{quote}

  In this case, the term `foregone conclusion' is embedded under negation, to highlight that the agent in question entertains the possibility that the agent's action will not produce the intended response.

  By contrast, the following passage from \textcite{Kadane:1996vu} is an example of the second meaning:

  \begin{quote}
    When can a Bayesian select an hypothesis \emph{H} and design an experiment (or a sequence of experiments) to make certain that, given the experimental outcome(s), the posterior probability of \emph{H} will be greater than its prior probably?
    We discuss an elementary result that establishes sufficient conditions under which this reasoning to a foregone conclusion cannot occur.%
    \mbox{ }\hfill\mbox{(\citeyear[1228]{Kadane:1996vu})}
  \end{quote}

  At issue is whether a Bayesian may chose some hypothesis \emph{H} and then guarantee some increase in probability for \emph{H} by running some experiments.
  So, are there cases in which the Bayesian first chooses a hypothesis \emph{H} and then ensures they reason to an increase in the probability of \emph{H}?
\end{note}

\begin{note}
  Our interest is with the first meaning, though narrow.
  To avoid ambiguity with first meaning, write `\fc{0}' rather than `foregone conclusion'.
  That is, the hyphen signifies when we are speaking about the technical term.
\end{note}

\section{Illustrations}

\begin{note}
  \begin{illustration}[Sudoku]
    \label{illu:gist:sudoku}
    % https://tex.stackexchange.com/questions/91422/tikz-sudoku-circle-and-connect-with-lines-some-cells
    \begin{figure}[H]
      \mbox{ }\hfill
      \begin{subfigure}{0.45\linewidth}
        \centering
        \sudokuGrid{}
        \caption{The starting grid \dots}
        \label{fig:sudoku:grid}
      \end{subfigure}
      \hfill
      \begin{subfigure}{0.45\linewidth}
        \centering
        \sudokuGridHints{}
        \caption{\dots with hints}
        \label{fig:sudoku:hint}
      \end{subfigure}
      \hfill\mbox{ }
      \caption{Sudoku}
      \label{fig:sudoku}
    \end{figure}
    Know whether the hints are correct, and likewise know whether 3 is the correct number to place in the bottom left corner.
  \end{illustration}
\end{note}

\begin{note}
  \[\frac{(3 + \sqrt{3})^{2} + \sqrt{6}^{2} - (2\sqrt{3})^{2}}{2(3 + \sqrt{3})\sqrt{6}} = \frac{1}{\sqrt{2}}\]

  Whether or not equation is true if a \fc{}.
\end{note}

\begin{note}
  \begin{illustration}
    Know whether the modal system \(\mathbf{K} + \Diamond\Box p \rightarrow \Box\Diamond p\) is weakly complete with respect to the class of frames which have the Church-Rosser property.%
    \footnote{ \(\forall s,t,u((Rst \land Rsu) \rightarrow \exists v(Rtv \land Ruv))\). }
  \end{illustration}

  The theorem is not quite routine.
  Some care needs to be taken when choosing valuation.
  However, fairly easy.

  Here, stress point that potential event.
  Even if fail on attempting, this doesn't raise a problem.
  Indeed, if get stuck, get hint, and I expect you will consider it the case that you could have done the thing.
\end{note}

\paragraph{Not clearly \fc{1}}

\begin{note}
  Provided a handful of (plausible) instances of knowing whether which (plausibly) involve \fc{1}.
  A pair of (plausible) instances whether which (plausibly) do not involve \fc{1}.
\end{note}

\begin{note}[ML II]
  \begin{illustration}
    The modal system \(\mathbf{GL} = \mathbf{K} + \Box(\Box p \rightarrow p) \rightarrow \Box p\) is weakly complete with respect to the class of finite strict partial orders (that is, the class of finite irreflexive transitive frames).
  \end{illustration}

  Almost routine.
  However, method involves a syntactic proof that \(\vdash_{\mathbf{GL}} \Box p \rightarrow \Box \Box p\).
  This, doubt.
\end{note}

\begin{note}[Chess]
  \begin{illustration}
    \begin{figure}[H]
      \centering
      \begin{adjustbox}{minipage=\linewidth,scale=0.7}
        \centering
        \newchessgame[
        setwhite={ka5,pa3,pb4,pc4,pe5,pf6,bg5,bh5},
        addblack={pa6,pb7,pc6,pe6,pf7,kc7,nd7,nd4},
        ]%
        \setchessboard{showmover=false}%
        \chessboard
      \end{adjustbox}
      \caption{Chess board.}%\protect\footnotemark}
      \label{fig:chess:intro}
    \end{figure}

    It is possible for Black to checkmate in four moves.
    \end{illustration}
  Not clear that this is a \fc{0} as potential.
  I, at least, fail.
  Perhaps understanding is good, and wouldn't be so hard.%
  \footnote{
    \citeauthor{Emms:2000aa}' Puzzle 150 (\citeyear[33]{Emms:2000aa}).
    \citeauthor{Emms:2000aa} provides the following solution:
    \begin{quote}
      \variation{1... Nb6!}
      (threatening \variation{2... Nb3\#})
      \variation{2. b5}
      (or \variation{2. Bd1 Nxc4+} \variation{3. Ka4 b5\#})
      \variation{2... c5!}
      \variation{3. bxa6 Nxc4+}
      \variation{4. Ka4 b5\#}
      \textbf{(0-1)}%
      \mbox{}
      \hfill
      (\citeyear[46]{Emms:2000aa})
    \end{quote}
    My statement above remains true---I don't have sufficient background to parse this solution!
  }
  Still, of interest is not only whether \emph{is} a \fc{0} but also from agent's perspective.
\end{note}


\section{\fc{3}}
\label{cha:sec:fcs-def}

\begin{note}
  We define \(\pv{\phi}{v}\) is a \emph{\fc{0}} from \(\Phi\) with respect to an agent via the conjunction of two clauses holding.
  Roughly, some action, which leads to concluding \(\pv{\phi}{v}\) from \(\Phi\).
  And, it is not the case that conclude something incompatible.

  Rough, the definition makes use of idea of a potential event.
  Detail what these amount to.

  \begin{restatable}[A \fc{0}]{definition}{definitionForegoneC}
    \label{def:fc}
    For an agent \vAgent{}, and some proposition-value-premises pairing \(\pvp{\psi}{v'}{\Psi}\):

    \begin{itemize}
    \item
      \(\pv{\phi}{v}\) is a \emph{\fc{0}} from \(\Phi\), for \vAgent{}.
    \end{itemize}
    \emph{If and only if}
    \begin{enumerate}[label=]
    \item
      Both~\ref{def:fc:is-pe-good} and~\ref{def:fc:no-pe-bad} are true:
      \begin{enumerate}[label=\alph*., ref=(\alph*)]
      \item
        \label{def:fc:is-pe-good}
        There is some potential event \(p\) such that \(p\) involves \vAgent{} concluding \(\pv{\phi}{v}\) from \(\Phi\).
      \item
        \label{def:fc:no-pe-bad}
        There is no potential event \(p\) such that \(p\) involves \vAgent{} concluding
        \(\pv{\chi}{v''}\) from \(X\) where:
        If conclude \(\pv{\chi}{v''}\) from \(X\), then does not conclude \(\pv{\phi}{v}\) from \(\Phi\).
      \end{enumerate}
    \end{enumerate}
    \vspace{-\baselineskip}
  \end{restatable}

  \phantlabel{fcs-neutral-perspective}
  \fc{3} are stated from a neutral perspective --- at issue is whether there is a potential event in which the agent concludes --- however, our interest with \fc{1} will be from an agent's perspective.
  Specifically, whether \(\pv{\phi}{v}\) is a \fc{0} from \(\Phi\) with respect to \vAgent{}, from \vAgent{}' perspective.
  To simplify, we will shorten a positive instance to:
  `\(\pvp{\phi}{v}{\Phi}\) is a \fc{0}, from \vAgent{}' perspective.'
\end{note}

\begin{note}
  Basic idea, is straightforward.
  Some motivation with instances of knowing how.
  Or, the agent having the ability to reason (in such a way that culmination of reasoning is a conclusion).

  However, the term `\fc{0}' is a little narrower than broad intuition may suggest.
  For, the event in which the agent conclude \(\pv{\phi}{v}\) from \(\Phi\) is qualified by the term `potential'.
\end{note}

\section{Potential events}
\label{cha:sec:fcs-def:potential-events}

\begin{note}
  \autoref{def:fc} appeals to~\ref{def:fc:is-pe-good} the existence of some potential event and~\ref{def:fc:no-pe-bad} the non-existence of some potential event.

  However, \autoref{def:fc} does not rely on anything more than existential quantification.
  The choice is deliberate.
  We given necessary and sufficient conditions for the \emph{existence} of some potential event in terms of
  \begin{enumerate*}[label=(\roman*)]
  \item
    actions available to the agent, and
  \item
    truth conditions for the progressive.
  \end{enumerate*}

  \begin{restatable}[Potential event]{definition}{definitionPEvent}
    \label{def:potenital-event}
    For an agent \vAgent{} and action description \(\alpha\):
    \begin{itemize}
    \item
      There is a potential event \(p\) in which \vAgent{} \(\alpha\)s
    \end{itemize}
    \emph{if and only if}
    \begin{enumerate}[label=]
    \item
      Both~\ref{def:PE:action} and~\ref{def:PE:prog} are true:
      \begin{enumerate}[label=\alph*., ref=(\alph*)]
      \item
        \label{def:PE:action}
        There is some minimal action \(a\) that \vAgent{} may immediately perform.
      \item
        \label{def:PE:prog}
        \(\text{Prog}(e, \alpha)\) would be true of the  event \(e\) of \vAgent{} doing \(a\).
      \end{enumerate}
    \end{enumerate}
    Where \(\text{Prog}(e, \alpha)\) stands for the progressive from of \(\alpha\) when evaluated with respect to \(e\).%
    \footnote{
      I.e.\ \(\text{Prog}(e, \alpha)\) is true \emph{iff} event \(e\) is an event of \(\alpha\)ing.
      See,~\textcite{Richards:1981wo},~\textcite{Portner:2011vi}, etc.
    }
  \end{restatable}

  In short,~\autoref{def:potenital-event} states that there is a potential event in which an agent performs some action \(\alpha\) just in case there is some action available to the agent which would result in the agent \(\alpha\)ing.
\end{note}

\begin{note}[Interest with the progressive]
  Our interest with the progressive is due to the delicate sense of possibility required for a sentence stating an event in the progressive tense to be true.
  Perhaps the clearest example is the `imperfective paradox' (\citeyear[cf.][Ch.3.1]{Dowty:1979vq}).

  \citeauthor{Bach:1986tb} summarises:
  \begin{quote}
    [H]ow can we characterize the meaning of a progressive sentences like \ref{Bach:impP:17} on the basis of the meaning of a simple sentence like \ref{Bach:impP:18} when \ref{Bach:impP:17} can be true of a history without \ref{Bach:impP:18} ever being true?
    \begin{enumerate}[label=(\arabic*), ref=(\arabic*)]
      \setcounter{enumi}{16}
    \item
      \label{Bach:impP:17}
      John was crossing the street.
    \item
      \label{Bach:impP:18}
      John crossed the street.%
      \mbox{ }\hfill\mbox{(\citeyear[12]{Bach:1986tb})}
    \end{enumerate}
  \end{quote}
  \citeauthor{Dowty:1979vq} adds:
  \begin{quote}
    Notice, furthermore, that to Say that John was drawing a circle is not the same as saying that John was drawing a triangle, the difference between the two activities obviously having to do with the difference between a circle and a triangle.
    Yet if neither activity necessarily involves the existence of such a figure, just how are the two to be distinguished?%
    \mbox{ }\hfill\mbox{(\citeyear[133]{Dowty:1979vq})}
  \end{quote}
  On the one hand, no completion.
  On the other hand, event is sufficiently specific to determine some outcome over some other.%
  \footnote{
    Though, the force of \citeauthor{Dowty:1979vq}'s observation is perhaps clearer by substituting `square' for `circle'.
    For, straight line\dots
  }
  So, the truth of the progressive doesn't require completion and doesn't require significant progress toward completion.

  However, completion is tied to the action in progress, rather than the possibility of there being an event in which progressive completed.

  Additional example of event.

  Clear example of this from Igal Kvart, via~\textcite{Landman:1992wh}:

  \begin{quote}
    Look at example~\ref{Landman:FRomans}:
    \begin{enumerate}[label=(\arabic*), ref=(\arabic*)]
      \setcounter{enumi}{19}
    \item
      \label{Landman:FRomans}
      Mary was wiping out the Roman army.
    \end{enumerate}

    The situation is that Mary, a person of moderate physical capacities, is battling the Roman army.
    She manages to kill a couple of soldiers before she gets killed.
    \ref{Landman:FRomans} is clearly false in this situation.\newline
    \mbox{ }\hfill\mbox{(\citeyear[18]{Landman:1992wh})}
  \end{quote}

  Mary may have survived, and more Roman soldiers may have perished by Mary's hand.
  However, relation of Mary to the Roman army means no sense in which event is Mary wiping out the Roman army.
\end{note}

\begin{note}
  \autoref{def:potenital-event} relies on important assumption regarding the progressive.

  \begin{assumption}[Progressive perfection]
    \label{assu:prog-modal-shift}
    For any event \(e\) and action description \(\alpha\):
    \begin{enumerate}
    \item[\emph{If}:]
      \begin{enumerate}[label=\alph*., ref=(\alph*)]
      \item
        \(\text{Prog}(e, \alpha)\) is true.
      \end{enumerate}
    \item[\emph{Then}:]
      \begin{enumerate}[label=\alph*., ref=(\alph*), resume]
      \item
        There is some \emph{possible} event \(e'\) such that \(e'\) is a continuation of \(e\) and \(\text{Past}(e',\alpha)\) is true.
      \end{enumerate}
    \end{enumerate}
    \vspace{-\baselineskip}
  \end{assumption}

  \autoref{assu:prog-modal-shift}, shift evaluation to some possible event in which something related is true.

  Possible here is unspecified/a placeholder.
  The task of an account of the progressive is to narrow the relevant sense of possibility.
  Key point is what holds of the possibility.
  Completed action.

  Past, as easier to parse.
  Agent is concluding, concludes, concluded.
  Easier, concluded.

  \autoref{assu:prog-modal-shift} is common for the progressive.%
  \footnote{
    See, for example:
    \textcite{Bennett:1972uw},
    \textcite{Dowty:1979vq},
    \textcite{Parsons:1990aa},
    \textcite{Landman:1992wh}, and
    \textcite{Portner:1998um}.

    However,~\autoref{assu:prog-modal-shift} is denied by~\textcite{Szabo:2004ul}.
    \citeauthor{Szabo:2004ul} writes:
    \begin{quote}
      Sometimes we are \emph{doing} things even though there is no real chance that we could get them \emph{done}, and this is true even if we abstract away from the possibility of miraculous intervention.%
      \mbox{ }\hfill\mbox{(\citeyear[40]{Szabo:2004ul})}
    \end{quote}
    To illustrate, \citeauthor{Szabo:2004ul} denies~\ref{Szabo:Arch} is necessarily false:
    \begin{quote}
      \begin{enumerate}[label=(\arabic*), ref=(\arabic*)]
        \setcounter{enumi}{12}
      \item
        \label{Szabo:Arch}
        As the architect was building the cathedral he knew that, although he would be building it for another year or so, he couldn't possibly complete it.%
        \mbox{ }\hfill\mbox{(\citeyear[38]{Szabo:2004ul})}
      \end{enumerate}
    \end{quote}
    Though,~\ref{Szabo:Arch} seems false to me, without some priming.
    And, the only priming on which~\ref{Szabo:Arch} reads true involves interpreting the architect's knowledge from the architect's perspective, allowing a failure of factivity, thus allowing the cathedral to be built.

    Still, \autoref{assu:prog-modal-shift} is an assumption.
    The goal is not to tie potential to progressive, but to evaluation associated with the progressive granting assumption.
  }


  Likewise for modals more generally.

  Possible only if possible world in which true.
  Will, future in which happens.

  Necessary (and sufficient).
\end{note}

\begin{note}[Action]
  In~\autoref{cha:sec:fcs-def:progressive-landman} we will provide a detailed account of the progressive, which builds on \textcite{Landman:1992wh}.

  However, the definition of a potential event, does not require that the agent is performing some action.
  Rather, \ref{def:PE:prog} requires both~\ref{def:PE:action} and~\ref{def:PE:prog} to be true:
  \begin{quote}
    \begin{enumerate}[label=\alph*., ref=(\alph*)]
      \item
        There is some immediate action \(a\) that \vAgent{} may perform.
      \item
        \(\text{Prog}(e, \alpha)\) would be true of the event \(e\) of \vAgent{} doing \(a\).\newline
    \mbox{ }\hfill\mbox{(Cf.~\autopageref{def:potenital-event})}
  \end{enumerate}
  \vspace{-\baselineskip}
  \end{quote}

  So, though we have sketched a general understanding of the progressive, action agent may perform.

  Intuitive distinction between which actions may and may not perform.

  However,~\ref{def:PE:action} without.
  Allow arbitrary division of actions, what matters is immediate.

  Then, agent doing \(a\) is in progressive, so make sure that the minimal \(a\) doing is also instance of \(\alpha\).

  Appealing to here is event for which progressive holds is sufficient without performing activity.
  So, identify the event quickly.

  Following \citeauthor{Dowty:1979vq}, progressive without drawing.
  Likewise, crossing road without yet having set foot on the road.

  What we avoid by minimal is large events.
  Arbitrary, so event in which the agent hits the centre of the dartboard.

  Way the agent throws the dart is an action, after throwing, no problem.

  However, not clear progressive holds when starting the action.
  Think racing.
  After some stuff happens, winning the race.
  However, with minimal, then there's going to be some shorter event.
  Not at the start.

  Likewise, the development of the action, sufficient to be throwing, but not throwing at the centre of the dartboard.

  Without minimal restriction, distinguish events in this way.
  With minimal restriction, converse problem.
  Minimal keeps reducing.
  However, bounded by the progressive being true of performing \(a\).
  Therefore, no immediate reduction to Zeno's paradox.
\end{note}


\begin{note}
  Still allows complex events.
  Again, progressive develops.

  \begin{illustration}[Darts]
    There is a potential event in which agent scores 180 at darts just in case there is some action available to the agent, such that if the agent were to perform the action they would be scoring 180 at darts.
  \end{illustration}

  Slightly more interesting.
  Determine the available actions.
  Though, similar, no guarantee.
  Hand is knocked at point of release, still scoring.

  Scoring 180 is a complex action.
  Though, interesting.
  First throws don't matter.

  Again, key idea is that sufficient understanding of progressive.

  And, case of interest:

  \begin{illustration}[Concluding]
    There is a potential event in which agent concludes \(\pv{\phi}{v}\) from \(\Phi\) just in case there is some action available to the agent, such that if the agent were to perform the action they would be concluding \(\pv{\phi}{v}\) from \(\Phi\).
  \end{illustration}

  What is it to be concluding something.
  Like crossing the road, fail to complete.
  Like darts, recover from a bad opening.
\end{note}

\begin{note}[Summarising]
  To summarise the preceding:
  We began with the definition of a \fc{} (\autoref{def:fc}).
  Definition of a \fc{} relies of the idea of a potential event.
  And, defined potential events in terms of the truth of the progressive aspect applied to a minimal event.

  The exact details of potential event depends on progressive.
  \autoref{cha:sec:fcs-def:progressive-landman}, \citeauthor{Landman:1992wh}'s (\citeyear{Landman:1992wh}) account of the progressive, reconstructed with (selected) observations from \textcite{Szabo:2004ul}.

  However, gnarly.
  Preferable, ability.
  \begin{quote}
    \(\pv{\phi}{v}\) from \(\Phi\) is a \fc{} just in case agent has the ability to conclude \(\pv{\phi}{v}\) from \(\Phi\) (and has the ability to avoid concluding something incompatible).
  \end{quote}
  Before turning to the progressive, consider ability.
  The following section --- \autoref{cha:sec:fcs-def:ability} --- will raise difficulties with this suggestion.%
  \footnote{
    And implicitly suggest that any sense of ability sufficient for purpose may be analysed in terms of progressive.
  }
\end{note}

\subsubsection{Ability(?)}
\label{cha:sec:fcs-def:ability}

\begin{note}
  Alternative suggestion is to say potential event just in case agent `can \(\alpha\)' or `has the ability to \(\alpha\)'.

  Particular sense of ability.

  \begin{enumerate}
  \item
    I have the ability to collect all 120 Power Stars in Super Mario 64.
  \item[]
    \begin{enumerate}
    \item[]
      \dots However, my Nintendo 64 is at home and I am at work.
    \end{enumerate}
  \end{enumerate}

  Distinction is familiar from~\citeauthor{Austin:1961vz}:

  \begin{quote}
    Consider the case where what we wish to assert is that somebody had the opportunity to do something but lacked the ability---`He could have smashed that lob, if he had been any good at the smash':
    here the \emph{if}-clause, which may of course be suppressed and understood, relates not to opportunity but to ability.%
    \mbox{ }\hfill\mbox{(\citeyear[177]{Austin:1961vz})}
  \end{quote}
  Likewise for ability:
  \begin{quote}
    `He could have read \emph{Emma}, if he had had a copy', does seem to assert `categorically' that he had a certain ability, although he lacked the opportunity to exercise it.%
    \mbox{ }\hfill\mbox{(\citeyear[177]{Austin:1961vz})}
  \end{quote}

  This combination is difficult.
  Broad sense of ability, from Austin.
  But, these two things don't need to come together.

  Made some mistaken in a proof.
  Ability to conclude.
  However, not `categorically' that I have ability.


  However, not this sense.
  In particular, \fc{} from current state.

  {
    \color{red} where is this example!
    I should also have it earlier.
  }

    \phantlabel{ability-g-s-dist}
  \nocite{Maier:2018uo}
  Distinction between `general' or `global' and `specific' or `local' abilities.
  \citeauthor{Whittle:2010wr}: `what an agent is able to do in a large range of circumstances' and `what the agent is able to do now, in some particular circumstances' (\citeyear[2]{Whittle:2010wr}).%
  \footnote{
    Though, see~\textcite[esp.\ \S4]{Kittle:2015tb} and~\textcite[1--2]{Kikkert:2022wp} for additional discussion on this distinction.
  }

  `general' ability conclude X.
  In large range of cases, conclude X.

  `specific' ability conclude \~X.
  Bad information.
  And, failure to conclude/lack of ability highlights that something has gone wrong.
  And, at issue is the specific instance.

  One way to view this, concluding is specific, reasoning is general.
  General speaking a strong relation between general reasoning and specific concluding.
  However, not always.

  So, interest is with `specific' or `local', and indirectly with `general' to the extent that general leads to an instance of specific.

  Example of what \textcite{Hackl:1998tt} terms `opportunity-can' (\citeyear[14]{Hackl:1998tt}):

  \begin{quote}
    \begin{enumerate}
    \item[(92)]
      \begin{enumerate}[label=\alph*., ref=(\alph*)]
      \item
        \label{Hackl:OC:a}
        A star gazer can see the solar eclipse of this year from the Cayman islands.\newline
        So if you were a star gazer and if you were on the Cayman islands at the right time you would see this year's solar eclipse.
      \item
        \label{Hackl:OC:b}
        John can see Mary from where he is standing.\newline
        So if you were standing in his place, you would see Mary.
      \end{enumerate}
    \end{enumerate}

    [\ref{Hackl:OC:b}] says that whoever is in this situation located at John's position and has normal eyesight and directs his/her gaze towards Mary will succeed in seeing Mary.%
    \mbox{ }\hfill\mbox{(\citeyear[39]{Hackl:1998tt})}
  \end{quote}
  \citeauthor{Hackl:1998tt}'s analysis straightforwardly extends to \ref{Hackl:OC:a}:
  A star gazer who is in the Cayman islands at the right time this year and looks for the solar eclipse will succeed in seeing the solar eclipse.

  Hence, we set aside \citeauthor{Austin:1961vz}'s `categorical' ability.
  Likewise we set aside `general' accounts of ability such as~\citeauthor{Carter:2021wd}'s~(\citeyear{Carter:2021wd}) `fallibilist',~\citeauthor{Kikkert:2022wp}'s~(\citeyear{Kikkert:2022wp}) `robust', and \citeauthor{Maier:2013vk}'s (\citeyear{Maier:2013vk}) `general' account, among others.

  To keep things narrow, focus on \citeauthor{Boylan:2020aa}'s (\citeyear{Boylan:2020aa}) account of `is able to'.
  Interest due to discussion of regarding two patterns of entailment.
\end{note}

\begin{note}
  \BoyPS{}.

  \begin{quote}
    \begin{enumerate}
    \item
      \label{Boylan:Past-Success}
      \BoyPS{}. \(\text{Past}(S\text{ }\phi) \leadsto \text{Past}(S\text{ [is able to] }\phi)\)%
      \mbox{ }\hfill\mbox{(\citeyear[\S1.1]{Boylan:2020aa})}
    \end{enumerate}
  \end{quote}
  \BoyPS{}.%
  \footnote{
    More generally, I understand \BoyPS{} to be an `actuality entailment'.

    Following \textcite{Alxatib:2019wf}:
    \begin{quote}
      Actuality Entailments (AEs) are inferences from premises that appear to be modal, like (1a), but their content is that the modality is effectuated in the evaluation world --- (1b).

      \begin{enumerate}[label=(\arabic*)]
      \item
        \begin{enumerate}[label=\alph*.]
        \item Pierre a dû \hspace{26pt} prendre le \hspace{3.5pt} train \newline
          Pierre had.to.\textsc{pfv} take \hspace{14pt} the train\newline
          \hspace{-4pt} ‘Pierre had to take the train'
        \item \emph{Inference}: Pierre took the train.%
          \mbox{}\hfill\mbox{(\citeyear[701]{Alxatib:2019wf})}
        \end{enumerate}
      \end{enumerate}
    \end{quote}

    Note, the reading of `had' in (1a) is `unambiguously deontic' (\citeyear[703]{Alxatib:2019wf}).
    Same holds for certain readings of ability.
    See~\textcite{Bhatt:1999ud,Bhatt:2008wt},~\textcite{Hacquard:2006to,Hacquard:2009ta},~\textcite{Pinon:2003te}, and~\textcite{Werner:2011tp} for examples and additional discussion of actuality entailments.
  }

  This doesn't hold for \fc{1}.
  So, problem with reducing to ability.

  Though, not so straightforward.

  \begin{quote}
    \begin{enumerate}[label=(\arabic*)]
      \setcounter{enumi}{314}
    \item
      (from~\cite{Thalberg:1969ta})
      \begin{enumerate}[label=\alph*., ref=(315\alph*)]
      \item
        \label{Bhatt:Thal-a}
        Yesterday, Brown hit three bulls-eyes in a row. Before he hit three bulls-eyes, he fired 600 rounds, without coming close to the bullseye; and his subsequent tries were equally wild.
      \item
        \label{Bhatt:Thal-b}
        Brown was able to hit three bulls-eyes in a row.
      \item
        \label{Bhatt:Thal-c}
        Brown had the ability to hit three bulls-eyes in a row.
      \end{enumerate}
    \end{enumerate}
    From~\ref{Bhatt:Thal-a}, we can conclude~\ref{Bhatt:Thal-b} but not~\ref{Bhatt:Thal-c}.
    Brown could have hit the target three times in a row by pure chance and he does not need to have had any ability for~\ref{Bhatt:Thal-b} to be true.%
    \mbox{ }\hfill\mbox{(\citeyear[167]{Bhatt:2008wt})}
  \end{quote}
  Distinction between `was able' and `had the ability'.
  So, it's not immediate that \BoyPS{} holds for ability in the relevant sense.

  Still, we are interested in `specific' abilities, rather than `general' abilities, and it may be that the distinction between~\ref{Bhatt:Thal-b} but not~\ref{Bhatt:Thal-c} is due to the `specific'/`general' divide.
  Indeed, \citeauthor{Bhatt:2008wt}'s proposal, in short, identifies `was able' with the specific reading of ability and `had the ability' with the general reading of ability.
  So, it is not clear that there is a stable account of ability that both:
  \begin{enumerate*}
  \item
    Captures specific ability, and
  \item
    Does not give rise to \BoyPS{}.
  \end{enumerate*}
\end{note}

\begin{note}[Segue]
  Hence, doubt there is a direct reduction to ability.
  However, indirect reduction.%
  \footnote{
    Indeed, potential events to progressive was indirect, focusing on (immediate) minimal actions.
    }

  In particular, \BoyPS{} relies on past.
  Indirect by shifting evaluation to the present.
  `At the time'.
\end{note}

\begin{note}
  \BoyPS{}, but denies \BoyVS{} (\citeyear[\S1.2]{Boylan:2020aa}):
  \begin{quote}
    \begin{enumerate}
    \item
      \label{Boylan:Or-Success}
      \BoyVS{}. \(S\text{ [will] }\phi \lor S\text{ [will] }\psi \leadsto S\text{ can }\phi \lor S\text{ can }\psi\)
    \end{enumerate}
  \end{quote}

  Future tense.

  However, invalidates \BoyPS{} from present perspective.
  So, some promise.
\end{note}

\begin{note}[Control intuition]
  \begin{quote}
    When said before the fact, the claim that I can surf that wave is strong  it says that surfing that wave is within my control.%
    \mbox{ }\hfill\mbox{(\citeyear[1]{Boylan:2020aa})}
  \end{quote}

  {
    \newbox\qqBoxA
    \newdimen\qqCornerHgt
    \setbox\qqBoxA=\hbox{$\ulcorner$}
    \global\qqCornerHgt=\ht\qqBoxA
    \newdimen\qqArgHgt
    \def\Quinequote #1{%
      \setbox\qqBoxA=\hbox{$#1$}%
      \qqArgHgt=\ht\qqBoxA%
      \ifnum     \qqArgHgt<\qqCornerHgt \qqArgHgt=0pt%
      \else \advance \qqArgHgt by -\qqCornerHgt%
      \fi \raise\qqArgHgt\hbox{$\ulcorner$} \box\qqBoxA %
      \raise\qqArgHgt\hbox{$\urcorner$}}

    \begin{quote}
      When someone says \(\Quinequote{\text{I can }\varphi}\), she is assuring her interlocutors that \(\sem[c]{\varphi}\) is within her control in a certain way.
      This is why we do not judge that Susie can hit the bull's-eye\dots%
      \mbox{ }\hfill\mbox{(\citeyear[326]{Mandelkern:2017aa})}
    \end{quote}
  }

  \begin{quote}
    think of the ability to sing a song, to build a shag, to play tennis – all have an action as their manifestation: the agent controls what is going on and she also controls whether to exercise the ability at all.%
    \mbox{ }\hfill\mbox{(\citeyear[34]{Jaster:2020wv})}
  \end{quote}

  \begin{quote}
    Suppose Cyril does not know the first 10 digits of \(\pi\).
    Intuitively,~\ref{Schwarz:pi} is then false.

    \begin{enumerate}[label=(\arabic*), ref=(\arabic*)]
      \setcounter{enumi}{2}
    \item
      \label{Schwarz:pi}
      Cyril can recite the first 10 digits of \(\pi\).
    \end{enumerate}

    \dots when we say that someone can recite the first 10 digits of \(\pi\), we don't just mean that no relevant facts preclude them from uttering `three, one, four,' etc.
    Rather, the agent must have a certain kind of intentional control over performing the act under the description of `reciting digits of \(\pi\)'.%
    \mbox{ }\hfill\mbox{(\citeyear[2]{Schwarz:2020aa})}
  \end{quote}

  This is the difficult intuition.
  In some respects control, and in other respects, no control.
  Interrupted, or get stuck in a loop.
  However, still a \fc{}.

  Might put this down to senses of control.
  However, unclear whether this is relevant for sense of control captured by the intuition.

  Focus on~\cite{Boylan:2020aa}'s analysis, same will apply to \textcite{Mandelkern:2017aa} and \textcite{Schwarz:2020aa} (I have not studied \textcite{Jaster:2020wv} in sufficient detail to extend the critique).
\end{note}

\begin{note}
  To get clear on how \citeauthor{Boylan:2020aa} understands control intuition, develop the formal core of \citeauthor{Boylan:2020aa}'s theory.
  Difficulty will not depend on particular details, but general understanding of control given by details.

  Need a handful of things.
  \begin{itemize}
  \item
    Unsettled world
  \item
    Selection function
  \item
    Semantics for will.
  \end{itemize}

  First, unsettled world.
  `while the past is settled, the future is open' (\citeyear[1]{Boylan:2020aa})

  \begin{quote}
    \emph{Unsettled World}. \(\mathcal{I}_{c} = \{w \mid w\text{ is identical to }w_{c}\text{ up until }t_{c}\}\)%
    \mbox{ }\hfill\mbox{(\citeyear[11]{Boylan:2020aa})}
  \end{quote}
  Relative to some context \(c\), which includes world \(w_{c}\) and a time \(t_{c}\), set of worlds which are identical to to \(w_{c}\) up until \(t_{c}\).

  Note, unique with respect to context.

  So, with when considering the actual world at some point in time, corresponding unsettled world is just set containing all worlds identical to the past.

  Caution, an unsettled world is a set of worlds.

  Selection function:

  \begin{quote}
    \(s(\mathcal{I}, \mathbf{A})\) picks out the closest (possibly unsettled) world to \(\mathcal{I}\) which settles that \(\mathbf{A}\) is true.%
    \mbox{ }\hfill\mbox{(\citeyear[11]{Boylan:2020aa})}
  \end{quote}
  So, selection function takes unsettled world and action as input and returns set containing a world or collection of worlds such that action is settled to be true.

  Key thing here is \(f(w,t)\).
  Intuitively, restrict which worlds are part of the unsettled world.
  Strictly, closest (unsettled) world.
  However, closeness is not of interest.

  \citeauthor{Boylan:2020aa} does not define settled.
  However, \(\mathbf{A}\) is true at every world in set.%
  \footnote{
    `If \(\phi\) is true at \(s(\mathcal{I}, f(w,t))\) (i.e. true throughout \(\mathcal{I}\)),\dots'. \mbox{(\citeyear[12]{Boylan:2020aa})}
  }

  \(\mathcal{W}\) `will'.

  \begin{quote}
    \begin{enumerate}
      \setcounter{enumi}{33}
    \item
      \begin{enumerate}
      \item
        \(\sem[w,t,f,\mathcal{I}]{\mathcal{W}\phi}\) is determinate only if either
        \begin{enumerate}
        \item
          \(s(\mathcal{I}, f(w,t)) \subseteq \sem[t,f,\mathcal{I}]{\phi}\) or
        \item
          \(s(\mathcal{I}, f(w,t)) \subseteq \sem[t,f,\mathcal{I}]{\lnot\phi}\)
        \end{enumerate}
      \item
        If determinate \(\sem[w,t,f,\mathcal{I}]{\mathcal{W}\phi} = 1\) iff \(s(\mathcal{I}, f(w,t)) \subseteq \sem[t,f,\mathcal{I}]{\phi}\)
      \end{enumerate}
    \end{enumerate}
  \end{quote}

  Here, \(w\) world, \(t\) time, \(f\) a function from a world and a time to a set of worlds, and \(\mathcal{I}\) an unsettled world.

  \(f(w,t) = \top\), then, abstracting from closest, all future possibilities.
  \(f(w,t) = \mathbf{A}\), then, all future possibilities in which \(\mathbf{A}\) happens.

  Mechanism is unclear.
  Intuitively, think of \(f\) as returning a collection of propositions where to evaluate, context is the same, but shifted time, and world is excluded if does not match with some proposition.
  So, truth of proposition at some (restricted) time in world.

  Key idea, restrict future worlds of interest so that the agent has performed some action.

  Now turn to `can'.

  \begin{quote}
    \begin{enumerate}
      \setcounter{enumi}{41}
    \item
      \(\sem[w,t,f,\mathcal{I}]{\text{S can }\phi} = 1\) iff for some \(\alpha \in \mathcal{A}\colon \sem[w,t,f^{\alpha},\mathcal{I}]{\mathcal{W}(\text{S }\phi\text{'s})} = 1\)\newline
      i.e.\ iff for some \(\alpha \in \mathcal{A}(w,t)\colon s(\mathcal{I}, f^{\alpha}(w,t)) \subseteq \sem[t,f^{\alpha},\mathcal{I}]{\text{S }\phi\text{'s}} = 1\)\newline
      \mbox{ }\hfill\mbox{(\citeyear[16]{Boylan:2020aa})}
    \end{enumerate}
  \end{quote}
  Here, key thing is \(f^{\alpha}\).
  We're making sure that the agent performs the action in the unsettled world.

  In short, we chose some action, and figure out whether it's a phing action.

  This is the interest.
  There is some action available to the agent, and this guarantees outcome.
  Restricts the future some that it happens.

  This captures the control intuition.

  So, this turns on what actions are `available' to the agent.
  \begin{quote}
    I think of an agent's available actions as their options.
    And, for simplicity at least, we can typically think of options as a set of tryings.\newline
    \mbox{ }\hfill\mbox{(\citeyear[14]{Boylan:2020aa})}
  \end{quote}

  So, key assumption is that the agent has \emph{choice} over actions.
  View ability in terms of having the option to determine some outcome.

  Now, \citeauthor{Boylan:2020aa} doesn't say much about this, though it's crucial.
  Still, get clear restrictions on what it means for an action to be available by invalidity of \BoyVS{} with respect to darts.
\end{note}


\begin{note}
  Now, \BoyPS{}, as context supplies \(\mathcal{I}\).
  This means, everything up until now is settled.
  So, when evaluate from the past, get a world in which \(\phi\), for sure.

  However, \BoyVS{} fails, because open.

  So, here's the thing.
  Failure of \BoyVS{} motivated our interest.
  But, to ensure failure, guaranteed by some action available to the agent.

  However, need to be careful about actions.
  Narrow, then works out, but then plausibly run into cases of \BoyVS{}.
  Broad, and then don't get instance of ability.

  But, for our purposes, no guarantee.
  Any action, may fail.

  In this respect, extends to \textcite{Mandelkern:2017aa} and \textcite{Schwarz:2020aa}.%
  \footnote{
    First, the issue straightforwardly extends to \citeauthor{Mandelkern:2017aa} as \citeauthor{Boylan:2020aa}'s analysis just is an instance of the act conditional analysis proposed by \citeauthor{Mandelkern:2017aa} (\citeyear[cf.][\S5]{Mandelkern:2017aa}).

    Second, though \citeauthor{Schwarz:2020aa} is non-committal with respect to a formal account of ability (\citeyear[cf.][13]{Schwarz:2020aa}), the spirit of \citeauthor{Schwarz:2020aa}'s analysis is sufficiently close to \citeauthor{Boylan:2020aa}'s for the issue to arise:
    `[A]n agent has the ability to \(\phi\) iff there are accessible worlds at which she \(\phi\)s simply by deciding to \(\phi\).' (\citeyear[19]{Schwarz:2020aa})
    Decision to action, but then the decision itself must sufficiently determine the action.
  }
\end{note}

\begin{note}
  Option here to weaken.
  \(\langle \mathcal{W} \rangle\) so non-empty intersection.
  `Orthodox' approach of~\textcite{Hilpinen:1969vw}%
  \footnote{
    Though,~\citeauthor{Hilpinen:1969vw} is a little more subtle.
    For,~\citeauthor{Hilpinen:1969vw} draws a distinction between:
    \begin{enumerate}[label=(A\(^{/}{'}\)), ref=(A\(^{/}{'}\))]
    \item
      \label{Hilpinen:A}
      John can catch the train / It is possible for John to catch the train.
    \end{enumerate}
    And:
    \begin{enumerate}[label=(C), ref=(C)]
    \item
      \label{Hilpinen:C}
      It is possible that John should catch the train.%
      \mbox{ }\hfill\mbox{(\citeyear[181--182]{Hilpinen:1969vw})}
    \end{enumerate}
    Where, the notion of possibility stated in~\ref{Hilpinen:A} is \emph{relativised} to John.
  }
  \textcite{Kratzer:1977aa,Kratzer:1981vn}~and ~\textcite{Lewis:1976us}.
  However, too weak.%
  \footnote{
    See also \textcite[\S1.3]{Boylan:2020aa} and \textcite[\S2]{Mandelkern:2017aa}
  }
  Violates control.


  For, now, just need some case.
  But, agent reasons to both.
\end{note}

\begin{note}
  Control is really strong.
  Lots of things could go wrong when reasoning.
  Interrupted at any time, or stick in a loop.
  Or, boredom.
  Chess, particular answer, but wow I'm likely to get bored.

  The right circumstances, but the point is, circumstances may not be right.

  Still, half the picture, sufficient control such that wouldn't prevent.
  But, avoiding failure is not the same as ensuring success.
\end{note}

\paragraph[Independent difficulty]{Independent difficulty \hfill (Optional)}

\begin{note}
  This is the `plan' account of ability.
  It's kind of insane.
  Whether or not ability reduces to action such that choice and secure outcome.
\end{note}

\begin{note}
  This is kind of wild.
  For, actions are kind of huge.
  Similar to that paper with minimalism about intentions.
\end{note}

\begin{note}
  Our direct interest with account finishes with universal.
  However, clear additional problem.
  Co-operation.
\end{note}

\begin{note}
  Uh, think.
  Has the ability to X with my help.
  There's no action in advance.
  For, whatever is chosen, I intervene prior, changing the course.
  Well, the point is, I only help if the agent gives up on whatever they had been planning to do.

  This isn't odd, cooperative activity.
  So, actually, refine example a little.
  For, point is that there's the cooperation condition.
\end{note}

\subsubsection[\citeauthor{Landman:1992wh}'s account of the progressive (modified)]{\citeauthor{Landman:1992wh}'s (\citeyear{Landman:1992wh}) account of the progressive (modified)}
\label{cha:sec:fcs-def:progressive-landman}
\nocite{Portner:1998um}
\nocite{Engelberg:1999vi}

\begin{note}
  \autoref{cha:sec:fcs-def:ability}, raised difficulty with linking \fc{1} to (specific) ability.
  Semantic issues, but key difficulty is `control intuition'.
  Or, more carefully, understanding of the control intuition.
\end{note}

\begin{note}
  Borrow summary from \textcite{Szabo:2004ul}:
  \begin{quote}
    [A] progressive sentence is true at some time just in case some event occurs at that time, and if we follow the development of the event (within our world as long as it goes, then jumping into a nearby world, and iterating the process within the limits of reasonability) we will reach a possible world where the perfective correlate is true of the continuation.%
    \mbox{ }\hfill\mbox{(\citeyear[34]{Szabo:2004ul})}
  \end{quote}
  The perfective correlate, links to \autoref{assu:prog-modal-shift}.
\end{note}

\begin{note}
  \begin{enumerate}
  \item
    \label{prog:max:bad}
    Max was crossing the street.
  \end{enumerate}
  True just in case there is some continuation of the actual world such that in that world, Max crossed the street.

  In the actual world, Max doesn't cross the street because Max is hit by a bus cruising at thirty miles per hour.
  (\citeyear[764]{Portner:1998um})
  Intuitively, however, \ref{prog:max:bad} remains true.
  Max was hit by the bus, but Max was not destined to be hit by the bus.
  And, if Max had not been hit by the bus, Max would have continued to cross the street.
  In other words, there is some possible world \(v\) that branched from the actual world before Max was hit by the bus.
  And, in \(v\) Max was not by the bus, and continued a little further across the road.

  Still, behind bus \#1 a second bus, bus \#2, was ready to hit max.
  And, in \(v\) Max was hit by bus \#2.
  (\citeyear[766]{Portner:1998um})
  However, like bus \#1 in \(w\), Max was not destined to be hit by bus \#2 in \(v\).
  Hence, there is some world \(u\) which branched from \(v\) in which Max continued a little further across the road.

  So long as there are a finite number of busses and no bus is destined to hit Max, then prior to being hit by a given bus, Max makes it a little further across the road.
  And, so long as the road is finite, it follows that eventually Max will have crossed the road.
\end{note}

\begin{note}
  \autoref{fig:max-bus} is a recreation of \citeauthor{Portner:1998um}'s figure 1. (\citeyear[767]{Portner:1998um})
  \begin{figure}[!h]
    \centering
    \begin{tikzpicture}
      \tikzmath{
        % x positions
        \x1 = 11;
        \xb1 = 2/9*\x1; \xb2 = 4/9*\x1; \xb3 = 6/9*\x1;
        % y positions
        \y1 = 2/5*\x1; \ymid = 1/2*\y1;
        \yw1 = \y1; \yw2 = 1/2*\y1; \yw3 = 0*\y1; \yb2 = 1/5*\y1;
        % event e
        \xe = 1/2*\xb1; \yediff = \yw2 - \yb2;
        \ye = \yw2 - 1/2*\yediff;
        \enudge = .1;
        \xel = 0; \xer = \xb1; \yen = \yw2 - \enudge;
        % bus 1 description location
        \xbx = 1.5/9*\x1; \xby = 4/5*\y1;
        % bus 2 description location
        \xb9 = 2.5/9*\x1;
      }
      % Paths
      \draw[line width=0.25mm, line cap=round] (\xb1,\ymid) -- (\xb3,\yw1); % world 1
      \draw[line width=1mm, line cap=round] (0,\ymid) -- (\xb1,\ymid) -- (\xb2,\yb2) -- (\xb3,\yw2); % world 2
      \draw[line width=0.25mm, line cap=round] (\xb2,\yb2) -- (\xb3,\yw3); % world 3
      % World descriptions
      \filldraw[black] (\xb3,\yw1) circle (0pt) node[anchor=west, align=left]{world 1: Max hit by \\ bus \# 1};
      \filldraw[black] (\xb3,\yw2) circle (0pt) node[anchor=west, align=left]{world 2: Max \\ crosses street};
      \filldraw[black] (\xb3,\yw3) circle (0pt) node[anchor=west, align=left]{world 3: Max hit by \\ bus \# 2};
      % Event
      \draw[] (\xe,\ye) -- (\xel,\yen); % event l
      \draw[] (\xe,\ye) -- (\xer,\yen); % event r
      % Event description
      \filldraw[black] (\xe,\ye) circle (0pt) node[anchor=north, align=left]{event e};
      % Splits
      \filldraw[black, dashed] (\xbx,\xby) circle (0pt) node[anchor=south, align=left]{bus \#1 hits Max};
      \filldraw[black, dashed] (\xb9,\yw3) circle (0pt) node[anchor=north, align=left]{bus \#2 hits Max};
      % Split descriptions
      \draw[-Stealth, dashed] (\xbx,\xby) -- (\xb1,\ymid + \enudge); % bus 2 arrow
      \draw[-Stealth, dashed] (\xb9,\yw3) -- (\xb2 - \enudge,\yb2 - \enudge); % bus 2 arrow
    \end{tikzpicture}
    \caption{
      Continuation path of `Max was crossing the street'. \\
    }
    \label{fig:max-bus}
  \end{figure}
\end{note}

\begin{note}
  Max was rather unfortunate to be hit by a bus, but fortunes may be reversed.
  The same reasoning applies to

  \begin{enumerate}
  \item
    \label{prog:max:good}
    Max was failing at the exam.
  \end{enumerate}

  Multiple choice.
  Max goes for broke.
  For each question, Max puts down the right answer.
  However, Max was not destined to write down the correct answer there is a branch from how things actually happened in which Max chose the incorrect answer.
  And, after a sequence of incorrect choices, Max failed the exam.
\end{note}

\paragraph[\citeauthor{Landman:1992wh}~(\citeyear{Landman:1992wh})]{\citeauthor{Landman:1992wh}'s (\citeyear{Landman:1992wh}) account of the progressive}

\begin{note}
  \citeauthor{Landman:1992wh}'s account of the progressive:

  \begin{quote}
    \(\sem{\text{PROG}(e, P)}_{w,g} = 1\) iff \(\exists f \exists v\colon \langle f,v \rangle \in \text{CON}(g(e), w)\)\newline
    \phantom{an} and \(\sem{P}_{v,g}(f) = 1\)\par

    where \(\text{CON}(g(e), w)\) is the continuation branch of \(g(e)\) in \(w\).\newline
    \mbox{ }\hfill\mbox{(\citeyear[27]{Landman:1992wh})}
  \end{quote}

  Immediate goal is to present \citeauthor{Landman:1992wh}'s account of a continuation branch.
  In turn, this will require expansion on three further points.
  Stages of an event.
  Continuations and stops with respect to events.
  Reasonable options.

  With understanding in hand algorithmic reconstruction of continuation branch.
  Expand in two ways.
  First, tree, to allow for forks.
  Second, a different account of how to identify branches.
\end{note}

\subparagraph{Continuation branch}

\begin{note}
  \citeauthor{Landman:1992wh}'s account of a continuation branch is as follows:
  \begin{quote}
    The \emph{continuation branch} for \(e\) in \(w\) is the smallest set of pairs of events and worlds such that
    \begin{enumerate}
    \item
      \label{Landman:CB:continues}
      for every event \(f\) in \(w\) such that \(e\) is a stage of \(\langle f,w \rangle \in C(e,w)\);
      the continuation stretch of \(e\) in \(w\);
    \item
      \label{Landman:CB:stops}
      if the continuation stretch of \(e\) in \(w\) stops in \(w\), it has a maximal element \(f\) and \(f\) stops in \(w\).
      Consider the closest world \(v\) where \(f\) does not stop:
      \begin{enumerate}[label=--]
      \item
        if \(v\) is not in \(R(e, w)\), the continuation branch stops.
      \item
        if \(v\) is in \(R(e, w)\), then \(\langle f,v \rangle \in C(e,w)\).
        In this case, we repeat the construction:
      \end{enumerate}
    \item
      \label{Landman:CB:continues:again}
      for every \(g\) in \(v\) such that \(f\) is a stage of \(g\), \(\langle g,v \rangle \in C(e,w)\), the continuation stretch of \(e\) in \(v\);
    \item
      \label{Landman:CB:stops:again}
      if the continuation stretch of \(e\) in \(v\) stops, we look at the closest world \(z\) where its maximal element \(g\) does not stop:
      \begin{enumerate}[label=--]
      \item
        if \(z\) is not in \(R(e, w)\), the continuation branch stops.
      \item
        if \(z\) is in \(R(e, w)\), then \(\langle g,z \rangle \in C(e,w)\) and we continue as above, etc.%
        \mbox{ }\hfill\mbox{(\citeyear[26--27]{Landman:1992wh})}
      \end{enumerate}
    \end{enumerate}
  \end{quote}

  Describes the process of following an event \(e\) and a world \(w\) and jumping to nearby reasonable worlds when \(e\) stops in \(w\) (or \(w'\), etc.).

  Observe, the construction of a continuation branch is iterative.
  Clauses~\ref{Landman:CB:continues:again} and~\ref{Landman:CB:stops:again} and duplicates of clauses~\ref{Landman:CB:continues} and~\ref{Landman:CB:stops} shifted to \(g\) --- some development of \(e\) --- in some possible world \(v\).

  Reconstruction via a recursive algorithm.
  For the moment we leave \citeauthor{Landman:1992wh}'s account of a continuation branch without commentary.
\end{note}

\subparagraph{Stages}

\begin{note}
  An event being a stage of some other event.
  Clause~\ref{Landman:CB:continues} (and~\ref{Landman:CB:continues:again}).

  \citeauthor{Landman:1992wh}'s definition is light:
  \begin{quote}
    An event is a stage of another event if the second can be regarded as a more developed version of the first, that is, if we can point at it and say, ``It's the same event in a further stage of development.''\newline
    \mbox{ }\hfill\mbox{(\citeyear[23]{Landman:1992wh})}
  \end{quote}
  \citeauthor{Landman:1992wh}'s definition is from an agent's perspective.
  Even if the elaboration is ignored, the initial expansion is qualified by the term `can be regarded as'.
  However, we may provide a definition of a stage independent of an agent's perspective:
  \begin{definition}[Stage]
    For events \(e\) and \(f\):

    \begin{itemize*}
    \item
      \(e\) is a stage of \(f\)
    \item
      \emph{If and only if}
    \item
      \(f\) is a development of \(e\).
    \end{itemize*}
  \end{definition}
  At issue is what it is for \(f\) to be a \emph{development} of \(e\).

  Stage is distinguished from part-of.
  \nocite{Davidson:1967aa}

  Buttering the toast.
  Bread was toasted.
  But, toasting the bread was not a stage of buttering the toast.
  Sufficiently distinct events.
  Marked by `toast'.
  Toasting and buttering the bread, then plausible stage.
  Likewise, scooping butter onto the knife, stage of buttering the toast.

  Stage does not entail anything of significance happens.
  Waiting for the post to arrive.
  Stage, nothing is happening, event has developed.

  So, definition by intuitive distinction.

  Importance is for shifting-to and tracing-events-through possible worlds.
\end{note}

\subparagraph{Continuations and Stops}

\begin{note}[Continuations and Stops]
  The definition of a stage is important for defining both the continuation and when an event stops.
  We present \citeauthor{Landman:1992wh}'s definition of both terms, and then provide restated definitions.
\end{note}

\begin{note}
  \citeauthor{Landman:1992wh} combines the definition of a continuation and when an event stops:
  \begin{quote}
    This is where stages come in: we cannot say that when an event stops in a world, there is no bigger event of which it is part in that world, but we can say that when it stops, there is no bigger event in the world of which it is a \emph{stage}:
    \begin{enumerate}[label=, noitemsep]
    \item
      Let \(e\) be an event that goes on at \(i\) in \(w\).
      Let \(f\) be an event that goes on at \(j\) in where \(i\) is a subinterval of \(j\).
    \item
      \(j\) is a continuation of \(e\) iff \(e\) is a stage of \(f\).
    \item
      Let \(j\) be a non-final interval.
    \item
      \(f\) stops at \(j\) in \(w\) iff no event of which \(f\) is a stage goes on beyond \(i\) in \(w\) (i.e., at a later ending interval).\newline
      \mbox{ }\hfill\mbox{(\citeyear[23--24]{Landman:1992wh})}
    \end{enumerate}
  \end{quote}

  To clarify the definitions, we borrow the relevant definitions regarding intervals from \textcite{Dowty:1979vq}:

  \begin{quote}
    \(I\) is a subinterval of \(J\) iff \(I \subseteq J\), where \(I\) and \(J\) are intervals.
    \(I\) is a proper subinterval of \(J\) iff \(I \subset J\).
    \(I\) is an \emph{initial subinterval} of \(J\) iff \(I\) is a subinterval of \(J\) and there is no \(t \in (J - I)\) for which there is \(t' \in I\) such that \(t \leq t'\).
    \emph{Final subinterval} is defined similarly\dots\newline
    \mbox{ }\hfill\mbox{(\citeyear[140]{Dowty:1979vq})}
  \end{quote}
  So, following \citeauthor{Dowty:1979vq}:
  \begin{quote}
    \(I\) is a \emph{final subinterval} of \(J\) iff \(I\) is a subinterval of \(J\) and there is no \(t \in (J - I)\) for which there is \(t' \in I\) such that \(t' \leq t\).
  \end{quote}
  And, from the definition of a non-final subinterval follows similarly:
  \begin{quote}
    \(I\) is a \emph{non-final subinterval} of \(J\) iff \(I\) is a subinterval of \(J\) and there is \emph{some} \(t \in (J - I)\) for which there is \(t' \in I\) such that \(t' \leq t\).
  \end{quote}
  Intuitively, then, \(I\) is a \emph{non}-final interval of \(J\), just in case \(I \subseteq J\) and \(J\) progresses further in time than \(I\).

  Let us now turn to restating the definitions of a continuation and a stop.
  We take each in turn.
\end{note}

\begin{note}
  The definition of \(f\) being a continuation of \(e\) requires two things:
  First, it must be the case the event at which \(e\) takes place is a subinterval of the event at which \(f\) takes place, and it must be the case that \(e\) is a stage of \(f\).%
  \footnote{
    Though, it seems to me the latter/\ref{def:Landman:conts:stage} implies the former/\ref{def:Landman:conts:interval}.
    I.e.\ if \(e\) is a stage of \(f\) then the interval at which \(e\) takes place must be a subinterval of the interval at which \(f\) takes place.
  }
  In full:

  \begin{definition}[Continuations]
    \label{def:Landman:conts}
    For events \(e\) and \(f\):
    \begin{itemize}
    \item \(f\) is a \emph{continuation} of \(e\)
    \end{itemize}
    \emph{if and only if}
    \begin{itemize}
    \item
      The following jointly hold:
      \begin{enumerate}[label=\alph*., ref=(\alph*)]
      \item
        \label{def:Landman:conts:interval}
        \begin{enumerate}
        \item[\emph{If}:]
          \begin{enumerate}[label=\roman*.]
          \item
            \(i\) is the interval at which \(e\) takes place in \(w\).
          \end{enumerate}
        \item[\emph{And}:]
          \begin{enumerate}[label=\roman*., resume]
          \item
            \(j\) is the interval at which \(f\) takes place.
          \end{enumerate}
        \item[\emph{Then}:]
          \begin{enumerate}[label=\roman*., resume]
          \item
            \(i\) is a subinterval of \(j\).
          \end{enumerate}
        \end{enumerate}
      \item
        \label{def:Landman:conts:stage}
        \(e\) is a stage of \(f\).
      \end{enumerate}
    \end{itemize}
    \vspace{-\baselineskip}
  \end{definition}
\end{note}

\begin{note}
  The definition of \(f\) stopping at \(j\) in \(w\) takes a little more work.

  There are two immediate issues with \citeauthor{Landman:1992wh}'s definition.

  First, the definition requires that \(j\) is a non-final interval.
  But, of what?
  By assumption \(i\) is a subinterval of \(j\), so \(j\) cannot be a non-final interval of \(i\).%
  \footnote{
    The term `non-final interval' only appears in the above quote from \citeauthor{Landman:1992wh}.
  }

  Second, there must be no event of which \(f\) is a stage that goes beyond \(i\) in \(w\).
  However, by assumption \(f\) is an event that goes on at \(j\), and \(i\) is a subinterval of \(j\).
  But why is \(f\) bound by some arbitrary interval \(i\)?

  I propose to resolve both issues with the following definition:
  \begin{definition}[Stops]
    \label{def:Landman:Stops}
    For an event \(e\), interval \(i\), and world \(w\):
    \begin{itemize}
    \item
      \(e\) \emph{stops} at \(i\) in \(w\)
    \end{itemize}
    \emph{If and only if:}
    \begin{itemize}
    \item
      There no is interval \(j\) nor event \(f\) such that the following jointly hold:
      \begin{enumerate}[label=\alph*., noitemsep]
      \item
        \(i\) is non-final subinterval of \(j\).
      \item
        \(f\) goes on at \(j\).
      \item
        \(f\) is a stage of \(e\).
      \end{enumerate}
    \end{itemize}
    \vspace{-\baselineskip}
  \end{definition}
  In short, there is no expansion from \(i\) \emph{forward} in time to obtain an interval \(j\) such that an event which is a stage of \(e\) got on at \(j\).

  Hence, \autoref{def:Landman:Stops} captures \citeauthor{Landman:1992wh}'s initial gloss: `[W]hen [\(e\)] stops, there is no bigger event in the world of which [\(e\)] is a \emph{stage}' (\citeyear[23]{Landman:1992wh})
\end{note}

\subparagraph{Reasonable options}

\begin{note}
  Reasonable options
  \begin{quote}
    \(v \in R(e, w)\) iff there is a reasonable chance on the basis of what is internal to \(e\) in \(e\) that \(e\) continues in \(w\) as far as it does in \(v\).%
    \mbox{ }\hfill\mbox{(\citeyear[26]{Landman:1992wh})}
  \end{quote}

  Look at possible worlds.
  Consider event \(e\) in possible world \(v\).
  Consider \(e\) in actual world.
  If there is a reasonable chance that \(e\) continues in \(w\) as in \(v\), then \(v\) is reasonable, on the basis of what is `internal' to \(e\).

  The function of \(R(e, w)\) is to constrain worlds.
  Possible world where May wipes out the Roman army, but not reasonable.
  However, constraint worlds relative to event, rather than anything else.

  Primary role of reasonable is:

  (11) Mary was crossing the street.

  (20) Mary was wiping out the Roman army.

  \citeauthor{Landman:1992wh}, existential, so risk of finding some continuation in `unreasonable' worlds.%
  \footnote{
    \color{red}
    If universal, then failure in an `unreasonable' world does not entail failure of the progressive.
  }

  No nearby world, but this isn't so clear.
  For, keep shifting.
  And, \citeauthor{Landman:1992wh} assumes closest.
  So, if branch multiple times, then can't easily capture with nearby worlds.

  Conversely, if just focus on the event, then there's no way to figure out how things external to the event influence.
\end{note}

\begin{note}
  Different perspective on~\citeauthor{Fine:1975tj}'s counterexample to~\citeauthor{Lewis:1973th}'s (\citeyear{Lewis:1973th}) theory of counterfactuals.

  In short, \citeauthor{Fine:1975tj} argues that the counterfactual `if Nixon had pressed the button there would have been a nuclear holocaust' is likely false on~\citeauthor{Lewis:1973th}'s theory.
  For, on~\citeauthor{Lewis:1973th}'s theory a counterfactual of the form `\(A \leadsto B\)' is true at a world just in case \(B\) is true at all the closest worlds in which \(A\) is true.

  And, as \citeauthor{Fine:1975tj} observes, it is plausible that in the closest worlds in which Nixon pressed the button, some malfunction occurs and a nuclear holocaust does not occur.
  For, generally speaking, the dissimilarity from our world due to a malfunction is small in contrast to the dissimilarity of a nuclear holocaust'.
  (\citeyear[452]{Fine:1975tj})

  Reasonable is constrained by the event, rather than similarity between worlds.
  So, general characteristics of world do not have a role in what is reasonable.

  Now, consider button pressing with no electrical fault.
  Reasonable that \(e\) continues the same way as it does in nuclear holocaust worlds.

  However, now suppose second check down the line.
  If just focus on event, then this has nothing to say about this.
\end{note}

\begin{note}
  Still, need to intersect with close worlds because progression of event may be blocked by things external to event.
\end{note}

\begin{note}
  Capture gist, the role of these things is clear.
  This doesn't amount to an account of how these things satisfy their role.
  However, reasonable is no less clear than nearby.%
  \footnote{
    \color{red}
    Plausibly captured by attention to fact.

  \citeauthor{Veltman:2005tj}'s (\citeyear{Veltman:2005tj}) revision to~\citeauthor{Tichy:1976tp}'s (\citeyear{Tichy:1976tp}) counterexample to \citeauthor{Lewis:1973th}'s theory (\citeyear{Lewis:1973th}) of counterfactuals.

  Some `facts' are fixed, and it is not possible to alter these facts under a counterfactual assumption.
  \(e\) has been set in motion.
  Now, keeping \(e\) fixed, if not X then e would continue.

  Fix what has been set in motion by \(e\), keep all laws the same.
  Then, buses at different times, but Mary doesn't get superpowers.
  }
\end{note}

\paragraph{An algorithmic (re)construction}

\begin{note}
  Breaking this down into a recursive algorithm.
  Goal is to create a tree, which will be a set of ordered event-world-pairing pairs indexed according to depth.
  For example:
  \[\text{Tree} = \{\langle \langle e,w \rangle_{1}, \langle f,w \rangle_{1} \rangle, \langle \langle f,w \rangle_{1}, \langle h,u \rangle_{2} \rangle, \langle \langle f,w \rangle_{1}, \langle h,u' \rangle_{2} \rangle, \dots \}\]

  Root, inital event world pair, and then branch from event world pair to some distinct event world pair just in case ordered pair.
  %
  \footnote{
    The indexing is not required to construct the tree.
    However, important for keeping track of the stage of construction.
  }

  {\color{red} Pseudocode}
  \nocite{Cormen:2009uw}

  We start by identifying three basic functions, and following a (re)construction of these functions with explanation added, we (re)construct a recursive function to build a continuation branch of some event-world pairing.

  The four basic algorithms will be termed:
  `\AlgAC{}', `\AlgGetStops{}', `\AlgGetPStops{}', and `\AlgFindBranches{}'.
  The recursive algorithm will be termed:
  `\AlgDevelopTree{}'.

  Interest will be in difference between \AlgGetStops{} and \AlgGetPStops{}.
\end{note}

\subparagraph{Continuations}

\begin{note}
  First function `\AlgAC{}'.
  Clause~\ref{Landman:CB:continues} (and~\ref{Landman:CB:continues:again}) of \citeauthor{Landman:1992wh}'s definition of a continuation branch.

  \begin{algorithm}[H]
    \label{PrAl:g-a-c}
    \caption{\AlgAC{}}
    \SetAlgoLined
    \DontPrintSemicolon
    \Input{\(\langle g,u \rangle_{i}\) \hfill An (indexed) event-world pairing}
    \KwResult{Continuations \hfill The continuations of event \(g\) in world \(u\)}
    \Begin{
      \(\text{Continuations} \longleftarrow \emptyset\)\;
      \label{PrAl:g-a-c:CSetInt}
      \(t_{s} \longleftarrow \text{start time of }g\text{ in }u\)\;
      \label{PrAl:g-a-c:ts}
      \(t_{e} \longleftarrow \text{end time of }g\text{ in }u\)\;
      \label{PrAl:g-a-c:te}
      \(g_{x} \longleftarrow g\)\;
      \label{PrAl:g-a-c:gx}
      \While{\(g_{x}\) is an event in \(u\)}
      {
        \label{PrAl:g-a-c:while:s}
        \(g_{x} \longleftarrow \emptyset\)\;
        \label{PrAl:g-a-c:gx:discard}
        \(t_{e} \longleftarrow t_{e} + 1\)\;
        \label{PrAl:g-a-c:te:plus}
        \(I \longleftarrow [t_{s},t_{e}]\)\;
        \label{PrAl:g-a-c:te:I}
        \For{\(g_{y} \in \{g_{y} \mid g_{y} \text{ is an event in } u\}\)}{
          \label{PrAl:g-a-c:for:s}
          \If{\(g_{y}\) occurs in \(I\) \emph{and} \(g_{y}\) is a stage of \(g\)}
          {
            \label{PrAl:g-a-c:for:test}
            \(\text{Continuations} \longleftarrow \text{Continuations } \cup \langle \langle g_{x},u \rangle_{i}, \langle g_{y},u \rangle_{i} \rangle\)\;
            \label{PrAl:g-a-c:C:new}
            \(g_{x} \longleftarrow g_{y}\)\;
            \label{PrAl:g-a-c:gx:new}
          }
        }
      }
      \label{PrAl:g-a-c:while:e}
      \Return{\(\text{Continuations}\)}
      \label{PrAl:g-a-c:return}
    }
  \end{algorithm}

  \AlgAC{} takes an event-world pairing \(\langle g,u \rangle_{i}\) and returns a set containing a (non-branching tree) which captures the continuation of \(g\) in \(u\).

  Intuitively, \AlgAC{} starts with \(\langle g,u \rangle_{i}\).
  Then, \AlgAC{} finds the smallest event \(g^{+}\) such that \(g^{+}\) is a stage of \(g\) in \(u\), and adds \(\langle g,u \rangle_{i}, \langle g^{+},u \rangle_{i} \rangle\) to the set of continuations.
  Now, \(g\) may develop further in \(u\).
  So, \AlgAC{} continues to the smallest \(g^{++}\) such that \(g^{++}\) is a stage of \(g\) in \(u\) at a later time than \(g^{+}\), and adds \(\langle g^{+},u \rangle_{i}, \langle g^{++},u \rangle_{i} \rangle\) to the set of continuations.
  The hypothetical set of continuations is now \(\{\langle g,u \rangle_{i}, \langle g^{+},u \rangle_{i} \rangle, \langle g^{+},u \rangle_{i}, \langle g^{++},u \rangle_{i} \rangle\}\).

  This process repeats until there are no further stages of \(g\) in \(u\).
\end{note}

\begin{note}
  There are two pieces of motivation for the construction of \(\text{Continuations}\).

  The first piece of motivation is \citeauthor{Landman:1992wh}'s account of the progressive.
  For, \(\sem{\text{PROG}(e, P)}_{w,g}\) checks to see if there exists \emph{some} event-world pairing \(\langle f,v \rangle\) in the continuation branch of \(e\) (in \(w\)) such that \(\sem{P}\) is true of \(f\) in \(v\).

  Hence, it is not possible, to consider the `maximal' continuation of \(g\) in \(u\).
  Rather, we must have the option of identifying event events as they develop.
  Need events as they develop.

  The second piece of motivation is the critiques of \textcite{Bonomi:1997uq} and \textcite{Szabo:2004ul} which press for a continuation tree as opposed to a continuation branch.
  And, given a continuation tree, an account of the progressive will involve universal quantification over branches.
  Therefore, it is important to ensure that the continuations of \(g\) in \(u\) do not involve branching.%
  \footnote{
    I.e., this constraint rules out including both \(\langle \langle g,u \rangle_{i}, \langle g^{+},u \rangle_{i} \rangle\) and \(\langle \langle g,u \rangle_{i}, \langle g^{++},u \rangle_{i} \rangle\), as this would indicate a branch.
  }
\end{note}

\begin{note}
  Finally, then, we have the way in which \(\text{Continuations}\) is constructed.

  We start by initialising \(\text{Continuations}\) as an empty set (\autoref{PrAl:g-a-c:CSetInt}).
  Then, we identify the start and end times of the interval in which \(g\) takes place (Lines~\ref{PrAl:g-a-c:ts} and~\ref{PrAl:g-a-c:te}).

  The task is then to continue to expand \(\text{Continuations}\) so long as there are further stages of \(g\) in \(u\).
  We achieve this a while loop %
  (Lines~\ref{PrAl:g-a-c:while:s}--\ref{PrAl:g-a-c:while:e}) %
  that will fail when we are no longer considering an event in \(u\).
  The variable event \(g_{x}\) is initially set to \(g\), to guaranteed at least one pass through the loop (\autoref{PrAl:g-a-c:gx}).

  In a instance of the while loop we first discard \(g_{x}\) to ensure the loop will fail if there are no further stages of \(g\) (\autoref{PrAl:g-a-c:gx:discard}).
  Then, we construct an interval \(I\) by shifting forward one step in time (Lines~\ref{PrAl:g-a-c:te:plus}~and~\ref{PrAl:g-a-c:te:I}).

  At this point, we have a new interval \(I\) to consider, but no immediate event.
  So, we consider every event \(g_{y}\) which occurs in \(u\) and test to see if \(g_{y}\) happens in \(I\) \emph{and} is a stage of \(g\) (Lines~\ref{PrAl:g-a-c:for:s}~and~\ref{PrAl:g-a-c:for:test}).
  If successful, we update \(\text{Continuations}\) and \(g_{x}\) (Lines~\ref{PrAl:g-a-c:C:new}~and~\ref{PrAl:g-a-c:gx:new}).

  Here we make two assumptions.

  First, if there is some stage of \(g\), then there is stage of \(g\) at an interval obtained by stepping forward in time one tick.%
  \footnote{
    This may be avoided by separating the test on~\autoref{PrAl:g-a-c:for:test} into two separate tests, and shifting the reassignment on~\autoref{PrAl:g-a-c:gx:new} outside the scope of the if-clause.
    Though the while-loop will then only terminate when there are no further events in \(u\).
  }

  Second, for any interval, if there is an event which is a stage of \(g\), then the event is unique.
  This assumption is key to ensure \(\text{Continuations}\) does not involve branching.

  Finally, after we have exhausted the stages of \(g\) in \(u\) we return \(\text{Continuations}\) (\autoref{PrAl:g-a-c:return}).
\end{note}

\subparagraph{Stops}

\begin{note}
  \AlgAC{} captures Clause~\ref{Landman:CB:continues} (and~\ref{Landman:CB:continues:again}) of \citeauthor{Landman:1992wh}'s definition of a continuation branch.

  If an event stops, then want to see if it's possible to shift to a new world to continue event.

  To capture the antecedent of Clause~\ref{Landman:CB:stops} (and~\ref{Landman:CB:stops:again}) of \citeauthor{Landman:1992wh}'s definition we introduce an algorithm termed `\AlgGetStops{}'.

  Following the account of \AlgGetStops{} we will introduce an algorithm termed `\AlgGetPStops{}'
  Consider how the event may have stopped, no just how the event stopped.
\end{note}

\begin{note}[\AlgGetStops{}]
  \AlgGetStops{} is as follows:

  \begin{algorithm}[H]
    \label{PrAl:g-s}
    \caption{\AlgGetStops{}}
    \SetAlgoLined
    \DontPrintSemicolon
    \Input{\(\text{Continuations}\) \hfill I.e.\ the result of \AlgAC{} --- in general, a tree}
    \KwResult{Stops \hfill Stopping points for each event in \(\text{Continuations}\)}
    \Begin{
      \(\text{Stops} \longleftarrow \emptyset\)\;
      \label{PrAl:g-s:mk-st-Stops}
      \For{\(\langle \langle f,v \rangle_{i-1}, \langle g,u \rangle_{i} \rangle \in \text{Continuations}\)}
      {
        \label{PrAl:g-s:for:start}
        \If{\(g\text{ stops in }u\)}
        {
          \label{PrAl:g-s:if:start}
          % \(g' \longleftarrow\text{ the stopping point of }g\text{ in }u\)\;
          % \label{PrAl:g-s:max}
          \(\text{Stops} \longleftarrow \text{Stops} \cup \{\langle \langle f,v \rangle_{i-1}, \langle g,u \rangle_{i} \rangle\}\)\;
          \label{PrAl:g-s:make}
        }
      }
      \Return{\(\text{Stops}\)}
      \label{PrAl:g-s:return}
    }
  \end{algorithm}

  \AlgGetStops{} initialises \(\text{Stops}\) as an empty set (\autoref{PrAl:g-s:mk-st-Stops}), and for every element of \(\text{Continuations}\), \AlgGetStops{} takes the right-event-world pairing and queries whether the event stops in the world (Lines~\ref{PrAl:g-s:for:start}--\ref{PrAl:g-s:if:start}).
  If the event stops, \AlgGetStops{} takes the point at which the event stops and adds the element of \(\text{Continuations}\) with the maximal continuation substituted for the event to \(\text{Stops}\) (\autoref{PrAl:g-s:make}).
  Finally, \AlgGetStops{} returns \(\text{Stops}\) (\autoref{PrAl:g-s:return}).
\end{note}

\begin{note}
  \AlgGetStops{} differs in presentation from \citeauthor{Landman:1992wh}.
  Recall:
  \begin{quote}
    \begin{enumerate}
      \setcounter{enumi}{2}
    \item
      if the continuation stretch of \(e\) in \(v\) stops, we look at the closest world \(z\) where its maximal element \(g\) does not stop: \dots
    \end{enumerate}
  \end{quote}
  Clause~\ref{Landman:CB:continues:again} reads as an imperative to find the relevant maximal event.
  \AlgGetStops{} does not explicitly highlight a maximal event\dots

  Still, the construction of \(\text{Continuations}\) parallels the continuation stretch of \(e\) in \(w\).
  Therefore, we obtain the relevant maximal event \(g\) by checking each member of \(\text{Continuations}\) to see whether the element is a stopping point.

  So, applied to the result of \AlgAC{}, \AlgGetStops{} will either return an empty set or a singleton set containing a pairing in which the event of the right element corresponds to the desired maximal event.
  And, applied to a set containing multiple results of \AlgAC{}, \AlgGetStops{} will either return an empty set or a set of size bounded by the instances of \AlgAC{} which implicitly contain the relevant maximal events.
\end{note}

\subparagraph{Possible stops}

\begin{note}
  Core insight.
  \citeauthor{Landman:1992wh} only considers stops that happen in some world.
  Hence, \AlgGetStops{} only finds the stopping point of an event in its continuation stretch.

  





  Here, we consider possible stops.
  Nearby world, and this stops the event from happening.
  Then, the key idea is that we get the progressive robust to possible stops.

  \AlgGetPStops{}:

  \begin{algorithm}[H]
    \label{PrAl:g-s}
    \caption{\AlgGetPStops{}}
    \SetAlgoLined
    \DontPrintSemicolon
    \Input{%
      \(\text{Continuations}\) \hfill I.e.\ the result of \AlgAC{} --- in general, a tree \\
      \(e\) \hfill An event \\
      \(w\) \hfill A world
    }
    \KwResult{ReasonableStops \hfill \emph{Reasonable} stopping points \\
      \mbox{ } \hfill for each event in \(\text{Continuations}\)}
    \Begin{
      \(\text{R-Stops} \longleftarrow \emptyset\)\;
      \For{\(\langle \langle f,v \rangle_{i-1}, \langle g,u \rangle_{i} \rangle \in \text{Continuations}\)}
      {
        \(\text{CloseWorlds} \longleftarrow \{u' \mid u' \text{ is among the closest world to } u\}\)\;
        \(\text{ReasonableWorlds} \longleftarrow \text{CloseWorlds} \cap R(e,w)\)\;
        \For{\(u' \in \text{ReasonableWorlds}\)}
        {
          {
            \If{\(g\text{ stops in }u'\)}
            {
              % \(g' \longleftarrow\text{ the stopping point of }g\text{ in }u'\)\;
              \(\text{R-Stops} \longleftarrow \text{R-Stops} \cup \{\langle \langle f,v \rangle_{i-1}, \langle g,u' \rangle_{i} \rangle\}\)\;
            }
          }
        }
      }
      \Return{\(\text{R-Stops}\)}
    }
  \end{algorithm}

  \AlgGetPStops{} is a variation on \AlgGetStops{}.
  The sole difference is that instead of checking every element of \(\text{Continuations}\) to identify the event-world pairings in which an event stops, \AlgGetPStops{} checks whether there is a `reasonable world' in which the event of some event-world pairing stops.
\end{note}

\begin{note}
  The for-loop from \AlgGetStops{}, repeated, and check from \AlgGetStops{} is repeated.
  However, check is within a for-loop over reasonable worlds.
  Reasonable worlds given by combination of closest possible worlds and what is reasonable.
\end{note}

\begin{note}
  Assume that \(u\) is always an element of closests worlds.
  Therefore, subset relation.
\end{note}

\begin{note}
  Think about the bus, could have been travelling a little faster.
  If so, interrupted sooner.
  Well, really, different outcome?
  Right, \citeauthor{Landman:1992wh}, p. 23, no bigger event of which it is a stage.
  So, if different outcome, then stop.
  This is sufficient for my purposes.
  I mean, the point is, no longer a stage of the maximal event in the actual world.
  This doesn't tell us anything about the description of the event, only that the resulting event differs from what actually happens.
\end{note}

\begin{note}
  Note, branch.
  However, only if result of \AlgGetPStops{} is part of tree.
  And, antecedent of conditional.
  Branches will be included, but only when calling \AlgFindBranches{}.
\end{note}

\subparagraph{Branches}

\begin{note}
  \AlgFindBranches{}.
  See if the event at some world continues in some other world.
  \citeauthor{Landman:1992wh} assumes closest world, but following \textcite[37]{Szabo:2004ul}, in part, allow for multiple closest worlds.%
  \footnote{
    Building on `multiple-choice paradox' from~\textcite{Bonomi:1997uq}
  }
  Note, importance of \(e\) and \(w\)!
  No shift in index, as the function identifies the same event in some other possible world.

    \begin{algorithm}[H]
    \SetAlgoLined
    \DontPrintSemicolon
    \Input{\(\langle \langle f,v \rangle_{i-1}, \langle g,u \rangle_{i}\rangle\) \hfill An element of a tree\\
      \(e\) \hfill An event \\
      \(w\) \hfill A world
    }
    \KwResult{Branches \hfill A set containing event-world pairs such that \(g\) continues in \(u'\)}
    \Begin{
      \(\text{CloseWorlds} \longleftarrow \{u' \mid u' \text{ is among the closest world to } u\}\)\;
      \(\text{ReasonableWorlds} \longleftarrow \text{CloseWorlds} \cap R(e,w)\)\;
      \(\text{Branches} \longleftarrow \emptyset\)\;
      \For{\(u' \in \text{ReasonableWorlds}\)}{
        \If{\(g\) is an event in \(u'\)}{
          \(\text{Branches} \longleftarrow \text{Branches} \cup \{ \langle \langle f,v \rangle_{i-1}, \langle g,u' \rangle_{i}\rangle \}\)\;
        }
      }
      \Return{\(\text{Branches}\)}
    }
    \caption{\AlgFindBranches{}\label{PrAl:find-branches}}
  \end{algorithm}
\end{note}

\begin{note}[Reasonable]
  Nearby worlds and what is reasonable.

  Restrictions on the branching.
  Nearby is no problem.
  Consider instances of the progressive, figure it out.
  Reasonable is more difficult.
  From the perspective of an agent.
  Semantics.

  Function here is to keep closeness tied to original event.
  As a descriptor, reasonable is difficult.
  However, the function here is not inherently epistemic.
\end{note}

\begin{note}
  Here, shift the index.
  When run recursive algorithm, each call of the algorithm will explore branches.
  Increase index for next call.

  Also, note, \AlgGetPStops{}.
  Pass the result of \AlgGetPStops{}, and this will get 
\end{note}

\subparagraph{Build tree}

\begin{note}
  With \AlgAC{}, \AlgGetStops{}/\AlgGetPStops{}, and \AlgFindBranches{} in hand, we now turn to \AlgDevelopTree{}, a recursive algorithm which completes any tree.

  Focus here is on linking \AlgAC{}, \AlgGetStops{}/\AlgGetPStops{}, and \AlgFindBranches{} in the same way as \citeauthor{Landman:1992wh}, and clarifying the `\dots and we continue as above, etc.' from \ref{Landman:CB:stops:again}.

  \begin{algorithm}[H]
    \label{PrAl:dev-tree}
    \caption{\AlgDevelopTree{}}
    \SetAlgoLined
    \DontPrintSemicolon
    \Input{%
      \(\text{Tree}\) \hfill A partially completed tree\\
      \(e\) \hfill The initial event of the tree\\
      \(w\) \hfill The initial world of the tree\\
      \(n\) \hfill An index to keep track recently added branches\\
    }
    \KwResult{Tree expanded so that there are no further stopping events}
    \Begin{
      \label{PrAl:dev-tree:start}
      \(\text{Stems} \longleftarrow \{ \langle g,u \rangle_{n} \mid \exists f,v \colon \langle \langle f,v \rangle_{n - 1}, \langle g,u \rangle_{n}\rangle \in \text{Tree}\}\)\;
      \label{PrAl:dev-tree:Extend:start}
      \label{PrAl:dev-tree:Extend:Stems}
      \(\text{GrownStems} \longleftarrow \emptyset\)\;
      \label{PrAl:dev-tree:Extend:FreshContsVar}
      \For{\(\langle g,u \rangle_{n} \in \text{Stems}\)}
      {
        \label{PrAl:dev-tree:Extend:Loop:start}
        \(\text{Growth} \longleftarrow \AlgAC{}(\langle g,u \rangle_{n})\)\;
        \(\text{GrownStems} \longleftarrow \text{GrownStems} \cup \text{Growth}\)\;
      }
      \label{PrAl:dev-tree:Extend:end}
      \(\text{Tree} \longleftarrow \text{Tree} \cup \text{GrownStems}\)\;
      \label{PrAl:dev-tree:Extend:merge}
      \textcolor{comment}{\texttt{//} \(\text{Stops} \longleftarrow \AlgGetStops{}(\text{GrownStems})\)}\;
      \label{PrAl:dev-tree:Stops:Land}%
      \(\text{Stops} \longleftarrow \AlgGetPStops{}(\text{GrownStems})\)\;
      \label{PrAl:dev-tree:Stops:Me}
      \eIf{\(\text{Stops} == \emptyset\)}
      {
        \label{PrAl:dev-tree:Stops:cond:start}
        \textbf{return} \(\text{Tree}\)\;
        \label{PrAl:dev-tree:Stops:cond:no-stops-finish}
      }
      {
        \label{PrAl:dev-tree:Stops:cond:else:start}
        \(\text{FutureBranches} \longleftarrow \emptyset\)\;
        \label{PrAl:dev-tree:Stops:cond:else:futureB:start}
        \For{\(\langle \langle g,u \rangle_{n}, \langle h,u \rangle_{n+1}\rangle \in \text{Stops}\)}
        {
          \label{PrAl:dev-tree:Stops:cond:else:futureB:loop:start}
          \(\text{Branches} \longleftarrow \AlgFindBranches{}(\langle \langle g,u \rangle_{n}, \langle h,u \rangle_{n+1}\rangle, e, w)\)\;
          \label{PrAl:dev-tree:Stops:cond:else:futureB:loop:getBranches}
          \(\text{FutureBranches} \longleftarrow \text{FutureBranches} \cup \text{Branches}\)\;
          \label{PrAl:dev-tree:Stops:cond:else:futureB:loop:gather}
        }
        \label{PrAl:dev-tree:Stops:cond:else:futureB:end}
        \eIf{\(\text{FutureBranches} == \emptyset\)}{
          \label{PrAl:dev-tree:Stops:cond:else:futureB:process:start}
          \textbf{return} Tree\;
          \label{PrAl:dev-tree:Stops:cond:else:futureB:process:cancel}
        }
        {
          \(\text{Tree} \longleftarrow \text{Tree} \cup \text{FutureBranches}\)\;
          \label{PrAl:dev-tree:Stops:cond:else:futureB:process:expand}
          \AlgDevelopTree{}(\(\text{Tree}, e,w, n+1\))\;
          \label{PrAl:dev-tree:Stops:cond:else:futureB:process:end}
        }
        \label{PrAl:dev-tree:Stops:cond:else:end}
      }
      \label{PrAl:dev-tree:Stops:cond:end}
    }
  \end{algorithm}

  \AlgDevelopTree{} processes single instance of branching on each run.
  So, given a tree, where terminal nodes are event world pairing \(\langle f,v \rangle_{i}\) such that \(f\) \emph{may} continue in \(v\).
  Term these terminal nodes `stems'.

  Three tasks.
  \begin{enumerate}
  \item
    Extend stems.%
    \hfill%
    Lines~\ref{PrAl:dev-tree:Extend:start}--\ref{PrAl:dev-tree:Extend:merge}.
  \item
    Identify any (immediate) branches of the stems.%
    \hfill%
    \autoref{PrAl:dev-tree:Stops:Land} or \autoref{PrAl:dev-tree:Stops:Me}.
  \item
    Process the (immediate) branches of the stems.%
    \hfill%
    Lines~\ref{PrAl:dev-tree:Stops:cond:start}--\ref{PrAl:dev-tree:Stops:cond:end}.
  \end{enumerate}

  \begin{itemize}
  \item
    Extend
    \begin{itemize}
    \item
      Lines~\ref{PrAl:dev-tree:Extend:start}--\ref{PrAl:dev-tree:Extend:end} extend tree given as input with continuations of all terminal branches.
      
      \autoref{PrAl:dev-tree:Extend:Stems}, collect all the fresh branches.
      \autoref{PrAl:dev-tree:Extend:FreshContsVar} create a set.
      Lines \ref{PrAl:dev-tree:Extend:Loop:start}--\ref{PrAl:dev-tree:Extend:end} loop over fresh branches, adding extensions to set.
    \item
      \autoref{PrAl:dev-tree:Extend:merge}, merge continuations.
      Now have extension of tree.
      However, only for fresh branches.
      Interest is whether there is further branching.
      Remainder of algorithm checks for branching, and for whether to continue.
    \end{itemize}
  \item Choices
    \begin{itemize}
    \item
      \autoref{PrAl:dev-tree:Stops:Land} gets \citeauthor{Landman:1992wh}.
    \item
      \autoref{PrAl:dev-tree:Stops:Me} gets revised.
    \end{itemize}
  \item Branches
    \begin{itemize}
    \item
      \autoref{PrAl:dev-tree:Stops:cond:start} to \autoref{PrAl:dev-tree:Stops:cond:end}, determining whether to pursue completion.
      Else 
      \begin{itemize}
      \item
        Have branches from \autoref{PrAl:dev-tree:Stops:Me}.
        \autoref{PrAl:dev-tree:Stops:cond:start} is simple check.
        If no stops, then \autoref{PrAl:dev-tree:Stops:cond:no-stops-finish} return Tree.
        Done.
      \item
        Else, some stops.
        Lines~\ref{PrAl:dev-tree:Stops:cond:else:start}--\ref{PrAl:dev-tree:Stops:cond:else:end}.
      \item
        Given stops, does the event continue in some other world?
        \autoref{PrAl:dev-tree:Stops:cond:else:futureB:start}, variable to collect branches.
        \autoref{PrAl:dev-tree:Stops:cond:else:futureB:loop:start} starts loop over stops.
        \autoref{PrAl:dev-tree:Stops:cond:else:futureB:loop:getBranches}, get branches via function.
        \autoref{PrAl:dev-tree:Stops:cond:else:futureB:loop:gather}, add these to future branches.
      \item
        \autoref{PrAl:dev-tree:Stops:cond:else:futureB:process:start} to \autoref{PrAl:dev-tree:Stops:cond:else:futureB:process:end} process future branches.
        \begin{itemize}
        \item
          \label{PrAl:dev-tree:Stops:cond:else:futureB:process:start} check to see if there are.
        \item 
          \autoref{PrAl:dev-tree:Stops:cond:else:futureB:process:cancel} if none, then nothing more to do.
        \item
          Else, branched, so \autoref{PrAl:dev-tree:Stops:cond:else:futureB:process:expand}, expand tree and \autoref{PrAl:dev-tree:Stops:cond:else:futureB:process:end} recursive call, with index shifted.
        \end{itemize}
      \end{itemize}
    \end{itemize}
  \end{itemize}
  Now, start with \(\langle e,w \rangle\).
  Slight thing, ordered pairs.
  So, pass a vacant event.
  Run, \AlgDevelopTree{}(\(\langle \langle -,- \rangle_{0}, \langle e,w \rangle_{1} \rangle, e, w, 1\))
\end{note}

\begin{note}
  This gives us continuation tree.
  Following \citeauthor{Szabo:2004ul}, revise account of the progressive.

  \begin{quote}
    \begin{enumerate}[label=(\Roman*), ref=(\Roman*)]
      \setcounter{enumi}{5}
    \item
      \emph{Prog}[\(\varphi\)] is true at \(t\) in \(w\) iff there is an \(e\) at \(t\) in \(w\) and for every \(\langle e^{\ast}, w^{\ast} \rangle\) on the continuation tree for \(e\) in \(w\) if \(\varphi\) is not true of \(e^{\ast}\) at \(w^{\ast}\) then there is an \(\langle e', w' \rangle\) on the continuation tree for \(e\) in \(w\) such that \(e'\) is a continuation of \(e^{\ast}\) in \(w'\) and \(\varphi\) is true of \(e'\) at \(w'\).%
      \mbox{ }\hfill\mbox{(\citeyear[37]{Szabo:2004ul})}
    \end{enumerate}
  \end{quote}

  Note, here, for every there exists.
  Existential is important, because there are so many events.

  As an aside, this plausibly also gets failure of \BoyVS{}.
  Surprising, this seems to capture all the entailments that \citeauthor{Boylan:2020aa} is interested in.
  Though, I think we've just reduced ability to choice of action.
\end{note}

\begin{note}
  potential event just in case there is some action avaiable to the agent, and were the agent to perform the action, the agent would be \(\alpha\)-ing.
\end{note}

\begin{note}
  With respect to concluding, reasonable constraint is neat.
  How the agent is in the actual world.
  And, shifts to closest worlds avoid blunders.
\end{note}

\begin{note}
  False negatives.
  \(\varphi\) to \(prog \varphi\).
  But, this entailment isn't really of interest.
  At least get this in the case of concluding.
\end{note}

\paragraph{The progressive and ability}

\begin{note}
  Now, \BoyVS{}.
  Holds on \citeauthor{Landman:1992wh}'s account of the progressive.
  Holds on revised account, \emph{only} if disjoin all possible results.

  Two aspects.
  First, \AlgGetPStops{}, different from how things actually happen.
  Second, universal quantification over branches.

  So, if some point not considered, then disjunction without point fails to hold.

  This seems good.

  Then can't works, because \AlgGetPStops{}.

  Finally, \BoyPS{}.
  Complex.

  Holds, given the same method of evaluation.
  Which worlds are nearby worlds.

  Doesn't hold for the progressive in general.
  Though, I don't think it holds for ability.

  However, will hold when fully determined.
  Had the ability to win the race, after got into first.
  Had the ability to hit the bullseye, just before releasing the dart.

  Deals with wind.

  And, then this also gets joint abilities.
\end{note}

\subsection{`Foregone-concluding'}
\label{sec:fc-progressive}

\begin{note}
  So, action such that would be concluding.
  On understanding of the progressive, no matter what happens, still conclude.
  Might need to get `lucky' with action.
  However, start then path to the conclusion.

  This `luck' is limited.
  Nothing conflicting.
\end{note}

\begin{note}
  Worry, jumping to conclusions.
  But, in this case, potential event in which agent concludes that step of reasoning is bad.
  Hence, incompatible.
\end{note}

\section{\fc{3} and \support{0}}
\label{cha:fcs:sec:fcs-support}

\begin{note}
  \autoref{cha:sec:fcs-def}, account of \fc{1}.
  Now tie to \support{0}.
\end{note}

\begin{note}
  \begin{proposition}
    \label{prop:fcs-only-if-support}
    For an agent \vAgent{} and proposition-value-premises pairing \(\pvp{\phi}{v}{\Phi}\):
    \begin{enumerate}
    \item[\emph{If}:]
      \begin{enumerate}[label=\alph*., ref=(\alph*.)]
      \item
        \(\pvp{\phi}{v}{\Phi}\) is a \fc{0}, from \vAgent{}' perspective.
      \end{enumerate}
    \item[\emph{then}:]
      \begin{enumerate}[label=\alph*., ref=(\alph*.), resume]
      \item
        \support{2} holds between \(\pv{\phi}{v}\) and \(\Phi\), from \vAgent{}' perspective.
      \end{enumerate}
    \end{enumerate}
    \vspace{-\baselineskip}
  \end{proposition}

  {
    \color{red}
    Why this is somewhat interesting.
  }
  However, before turning to the argument for \autoref{prop:fcs-only-if-support}, it is important to note the limitations of \autoref{prop:fcs-only-if-support} with respect to \issueConstraint{}.
  For, in order to argue against \issueConstraint{}, need some \(\pvp{\psi}{v'}{\Psi}\) such that answers \qWhyV{}.
  Answer \qWhyV{} only with dependence.
  Does not follow from \fc{0} that we get dependence.

  Note, also, qualifications.
  From the agent's perspective.
  Mirrored in both cases.
  Plausible that variant of \autoref{prop:fcs-only-if-support} holds unqualified.
  However, we have said nothing of \support{} independent of agent's perspective.
\end{note}

\begin{note}
  The argument for \autoref{prop:fcs-only-if-support} is {\color{red} mostly immediate for ideas regarding support}.

  \begin{goal}
    If conclude only if \fc{}, then support, in part, answers \qWhyV{}.
  \end{goal}

  So, to get answer to \qWhyV{}, need dependency.
  Here, if not \support{} then not \fc{}.
  If not \fc{} then not conclude.

  This is fine, just need to be careful with the counterfactual.
  Relation between \support{} and \fc{} is plain conditional.
  So, it survives any counterfactual changes.
\end{note}

\begin{note}[Argument]
  Argument is straightforward.
  Possible support, by assumption.
  Contraposition.
  If not support, then no \fc{}.
\end{note}

\paragraph{Potential relations of support}

\begin{note}
  Start with the following proposition.
  \begin{proposition}
    \label{prop:fcs-only-if-pot-support}
    For an agent \vAgent{} and proposition-value-premises pairing \(\pvp{\phi}{v}{\Phi}\):
    \begin{enumerate}
    \item[\emph{If}:]
      \begin{enumerate}[label=\alph*., ref=(\alph*.)]
      \item
        \(\pvp{\phi}{v}{\Phi}\) is a \fc{0}, from \vAgent{}' perspective.
      \end{enumerate}
    \item[\emph{then}:]
      \begin{enumerate}[label=\alph*., ref=(\alph*.), resume]
      \item
        (A) potential (relation of) \support{} holds between \(\pv{\phi}{v}\) and \(\Phi\), from \vAgent{}' perspective.
      \end{enumerate}
    \end{enumerate}
    \vspace{-\baselineskip}
  \end{proposition}

  Argument is fairly straightforward:
  \begin{argument}
    Suppose \(\pvp{\phi}{v}{\Phi}\) is a \fc{0}.
    Then, from agent's perspective, potential event in which concludes.
    Now, consider the potential event.
    The culmination of the event, agent concludes.

    So, from~\autoref{idea:support}, a relation of support holds, from the agent's perspective.

    Therefore, in whatever sense event is potential, \support{} between \(\pv{\phi}{v}\) and \(\Phi\) is likewise potential.
  \end{argument}
  From the agent's perspective, there is no difference between witnessed relation of support and potential relation of support.
\end{note}

\begin{note}
  \emph{Potential} relation of support, but it does not follow that there is a relation of support, from the agent's perspective.
\end{note}

\begin{note}
  \begin{proposition}
    \label{prop:pot-support-onlyIf-support}
    For an agent \vAgent{} and proposition-value-premises pairing \(\pvp{\phi}{v}{\Phi}\):
    \begin{enumerate}
    \item[\emph{If}:]
      \begin{enumerate}[label=\alph*., ref=(\alph*.)]
      \item
        (A) potential (relation of) \support{} holds between \(\pv{\phi}{v}\) and \(\Phi\), from \vAgent{}' perspective.
      \end{enumerate}
    \item[\emph{then}:]
      \begin{enumerate}[label=\alph*., ref=(\alph*.), resume]
      \item
        \support{2} holds between \(\pv{\phi}{v}\) and \(\Phi\), from \vAgent{}' perspective.
      \end{enumerate}
    \end{enumerate}
    \vspace{-\baselineskip}
  \end{proposition}

  \begin{argument}
    \autoref{idea:support:possible}.
    It is possible for there to be.
    So, we have everything needed.
    Both necessary and sufficient.
    Hence, form agent's perspective, relation of support.

    So, every necessary property that does not involve witnessing.
    But, then, every necessary property.
    Therefore, sufficient.
    For, if not sufficient, then missing a necessary property.
    Contradiction.

    Slight issue, disjunction of properties.
    But, this doesn't change the argument.
    Disjunction.
  \end{argument}
\end{note}

\begin{note}
  \color{red}
  Worry.
  \support{2} doesn't rely on witnessing.
  Now, if this goes through, then seems \support{} for any conclusion before making the conclusion.
  However, possible for the agent to reason to different conclusions.
  For, some faulty reasoning.
  Toggle the fault.
  Therefore, \support{} for contradictory conclusions.
\end{note}

\section{Conclusions, foregone}
\label{sec:fc3-1}

\paragraph{Premises and past conclusions}

\begin{note}[Premises]
  So, as we have seen with testimony, status of a premises blocks a \requ{}.

  Whether the same may hold for this problem.

  It's the case that, part of agent's present epistemic state that they would conclude.

  Problem is, if attempt and fail, then this premise does nothing.
  Their present epistemic state develops into a dead-end.
\end{note}

\begin{note}[Note!]
  This doesn't hold in general, for all premises.

  In particular, premise is past conclusion.

  Consider cases of being somewhat impaired, e.g., via exhaustion.
  Indeed, exhaustion is interesting.
  Basic consistency checks.
  Should be the case that conclude A, but just concluded \emph{not}-A, or something like this\dots

  Denying that past continues to secure in all instances.
  So, just need the potential to revise perspective on previous conclusion.
\end{note}

\section{\fc{3} and support}
\label{cha:fcs:sec:fc3-support}

\begin{note}
  \begin{proposition}
    For any path, present epistemic state determines availability of path.
  \end{proposition}

  Start.
  Then, continue.
  Started from \(\Phi\), so will conclude.
  Hence, no matter choice made, must have taken the possibility of this choice into account.
  So, it must be the case that determined.

  Hence, if witness, then via some path.

  So, witnessing predetermined path.
  Any instances of concluding by witnessing reduces to witnessing predetermined path.

  Witnessing may provide information about path, but witnessing doesn't contribute given a \requ{}.

  For any X from W,
  present determines whether or not X from agent's point of view, then \fc{}.

  In other words, agent's present epistemic state determines.
  Agent may need to witness to figure out how determined, but witnessing does not influence.
\end{note}

\begin{note}[Two worries]
  Two worries.

  First, that even though \fc{0}, the agent would not conclude.
  Either because \(\Phi\) is unavailable, or because no potential witnessing event.
  So, can't remove \fc{0} from account of why.

  However, then \fc{0} does not support.

  If grant that \fc{0} supports, then this seems to work out.
  Further, if require existence, then things that support get very messy.
  Dopeganger cases.
  Reason is I saw A, but it wasn't A, appealing to something that doesn't exist.
  Various other cases like this.

  Difference.
  In these cases, have premise, thing is that the truth value is distinct.
  Here, possibly no premise.

  Well, this is different.
  However, I don't think this is sufficient to reject the idea.
  Just because this distinction doesn't arise in the case of witnessing doesn't really do much.

  Look, a `bad' premise offers no more support for the agent than no premise.

  Second, need \emph{that} \fc{0}.
  However, the point is that this is about the agent's present epistemic state.
  \emph{Without} \fc{0}, the agent would reason.
  This is just the key point reiterated.
  Know whether, \fc{0} just adds information about which.
\end{note}

\subsection{Interpretation}
\label{cha:zSpA:sec:interpretation}

\begin{note}
  % Two instances where subjunctives of this form have a role.

  % Dispositions/ability.

  Doxastic justification.
\end{note}

% \begin{note}
%   Important to observe here that with dispositions and ability, the subjunctive analysis is an analysis.
%   So, in principle possible to provide a distinct analysis.
%   This is surely the case, and I can probably find some example.

%   By contrast, in the case of positive answers to \qzS{}, the subjunctive is `built in' to the question.
% \end{note}

% \subsubsection{Dispositions and ability}
% \label{sec:dispositions}

% \begin{note}[Parallel between dispositions and ability]
%   Consider \citeauthor{Choi:2021wg}'s characterisation of the Simple Conditional Analysis of dispositions:
%   \begin{quote}
%     An object is disposed to \emph{M} when \emph{C} iff it would \emph{M} if it were the case that \emph{C}.\nolinebreak
%     \mbox{}\hfill\mbox{(\citeyear{Choi:2021wg})}
%   \end{quote}
%   For example, an object is disposed to dissolve when it is placed in water iff the object would dissolve if it were the case that it is placed in water.

%   The Simple Conditional Analysis may be challenged, but for our purposes it is adequate.
%   We are interested in the broad form of the truth condition, and various more refined analyses share the same broad form.
%   Note, in particular, that it being the case that \emph{C} and \emph{M} happening describes an event.
%   Given appropriate conditions; salt dissolves, glass breaks, and I mumble when I am tired.
%   The key idea is that the property of being disposed to \emph{M} when \emph{C} is analysed in terms of the (possible) event of \emph{M} happening when \emph{C}.

%   The parallel to ability is established by noting that ability may also be analysed in terms of a (possible) event, as we have seen.
%   In particular, by incorporating volition in the analysans of the Simple Conditional Analysis.
%   To illustrate, \citeauthor{Mandelkern:2017aa} trace the Conditional Analysis of ability  to \textcite{Hume:1748tp} and \textcite{Moore:1912te}, among others:
%   \begin{quote}
%     S can \(\phi\) iff S would \(\phi\) if S tried to \(\phi\)\nolinebreak
%     \mbox{}\hfill\mbox{(\citeyear[Cf.][308]{Mandelkern:2017aa})}
%   \end{quote}
%   Compare to the Simple Conditional Analysis of dispositions:
%   The object is some agent \emph{S}, \emph{C} is `S tried to \(\phi\)' and \emph{M} is `S \(\phi\)s' --- it is volition alone which distinguishes the analyses.
%   For example, I have the ability to demonstrate that a rectangle with dimensions \(19\text{cm}\) by \(7\text{cm}\) has area \(133\text{cm}^{2}\) only if I would demonstrate that a rectangle with dimensions \(19\text{cm}\) by \(7\text{cm}\) has area \(133\text{cm}^{2}\) if it were the case that I tried that a rectangle with dimensions \(19\text{cm}\) by \(7\text{cm}\) has area \(133\text{cm}^{2}\).
% \end{note}


\subsubsection{Doxastic justification}
\label{cha:fcs:sec:dox-just}

\begin{note}
  \citeauthor{Turri:2010aa}

  \begin{quote}
    Necessarily, for all S, \emph{p}, and \emph{t}, if \emph{p} is propositionally justified for S at \emph{t}, then \emph{p} is propositionally justified for S at \emph{t} because S currently possesses at least one means of coming to believe \emph{p} such that, were S to believe \emph{p} in one of those ways, S's belief would thereby be doxastically justified.%
    \mbox{ }\hfill\mbox{(\citeyear[316]{Turri:2010aa})}
  \end{quote}

  Key is that doxastic justification depends on what the agent does.

  \citeauthor{Turri:2010aa}'s focus is on how reasons are used.
  What the agent does.

  Seen with example.

  \begin{quote}
    \begin{enumerate}[label=(P\arabic*)]
      \setcounter{enumi}{4}
    \item
      The Spurs will win if they play the Pistons.
    \item
      The Spurs will play the Pistons.
    \end{enumerate}

    \mbox{}\hfill\(\vdots\)\hfill\mbox{}

    \begin{enumerate}[label=(P\arabic*), resume]
    \item
      Therefore, the Spurs will win.%
    \mbox{ }\hfill\mbox{(\citeyear[317]{Turri:2010aa})}
    \end{enumerate}
  \end{quote}

  Rather than \emph{modus ponens}, `\emph{modus profusus}'.
  Conclude \(r\) from \(p\) and \(q\).
  (\citeyear[317]{Turri:2010aa})

  \begin{quote}
    The way in which the subject performs, the manner in which she makes use of her reasons, fundamentally determines whether her belief is doxastically justified.
    Poor utilization of even the best reasons for believing \emph{p} will prevent you from justifiedly believing or knowing that \emph{p}.%
    \mbox{ }\hfill\mbox{(\citeyear[316]{Turri:2010aa})}
  \end{quote}

  Variant of ~\cite{Prior:1960wh}'s `tonk' connective.
  Though, difference is between connective and rule.
  \(p\) tonk \(q\) would not be propositionally justified.
\end{note}

\begin{note}
  \citeauthor{Turri:2010aa} is similar to \citeauthor{Goldman:1979ui}

  Begin with justification.

  \begin{quote}
    \begin{enumerate}[label=(\arabic*)]
      \setcounter{enumi}{10}
    \item
      Person \emph{S} is \emph{ex ante} justified in believing \emph{p} at \emph{t} if and only if there is a reliable belief-forming operation available to \emph{S} which is such that if \emph{S} applied that operation to this total cognitive state at \emph{t}, \emph{S} would believe \emph{p} at \emph{t}-plus-delta (for a suitably small delta) and that belief would be \emph{ex post} justified.
    \end{enumerate}
  \end{quote}

  Where, sufficient condition for belief would be \emph{ex post} justified:
  \begin{quote}
    \begin{enumerate}[label=(\arabic*)]
      \setcounter{enumi}{4}
    \item
      If S's believing \emph{p} at \emph{t} results from a reliable cognitive belief-forming process (or set of processes), then S's belief in \emph{p} at \emph{t} is justified.%
      \mbox{ }\hfill\mbox{(\citeyear[13]{Goldman:1979ui})}
    \end{enumerate}
  \end{quote}
  Roughly, at least.
  \citeauthor{Goldman:1979ui} refines this a fair bit, but this isn't important.

  Availability of a reliable belief-forming operation!

  Relation here is brittle.
  Account of justification, apply to concluding.
  Well, then all we get is that before concluding, would make sense to conclude only if available.
  Running something like the \citeauthor{Carroll:1895uj} regress, not some state.
  But, this only tells us about suitability to conclude.

  Still, key point is process.

  Another useful thing to highlight is the suitably small delta.
  With \requ{}, this is captured in terms of the option.
\end{note}

\begin{note}
  Significant difference is in the case of justification, we're not interested in the agent's perspective.
  Hence, these accounts are understood in terms of the agent having the ability, roughly.
\end{note}

\section{Scrap}

\begin{note}
  {
    \color{red}
    Move to \requ{} now things have been switched around.
  }
  \fc{} is broader than \requ{}.

  Consider calculator type cases.
\end{note}


%%% Local Variables:
%%% mode: latex
%%% TeX-master: "master"
%%% End:
