\chapter{\fc{3}}
\label{cha:fcs}

\nocite{Ryle:1946tu}

\begin{note}
  This chapter introduces the idea of some proposition-value pairing \(\pv{\psi}{v'}\) being a \emph{\fc{}} from some pool of premises \(\Psi\) (for an agent).

  Intuitively, \(\pv{\psi}{v'}\) being a \fc{} from \(\Psi\) means an agent has the option to conclude \(\pv{\psi}{v'}\) from \(\Psi\).
\end{note}

\begin{note}
  Our main goal is a recipe to construct counterexamples to \issueConstraint{}.
  \fc{3} are a key ingredient.
  In particular, we argue the following conditional is true:

  \begin{itemize}
  \item
    If \(\pv{\psi}{v'}\) is a \fc{} from \(\Psi\), then a \ros{} holds between \(\pv{\psi}{v'}\) and \(\Psi\).
  \end{itemize}

  And, as an agent need not have concluded \(\pv{\psi}{v'}\) from \(\Psi\), it may be the case that a \ros{} holds between \(\pv{\psi}{v'}\) and \(\Psi\) while the agent does not a \wit{} for the \ros{}.

  Hence, \emph{if} the \ros{} between \(\pv{\psi}{v'}\) and \(\Psi\) answers \qWhyV{}, \issueConstraint{} fails to hold in general.
  However, this `if' is by no means straightforward, and does not follow from the idea of a \fc{} alone.
  An additional key idea is introduced in \autoref{cha:requs}, and a key motivating idea is introduced in \autoref{cha:typical}.
\end{note}

\begin{note}
  The chapter is divided as follows:
  \begin{TOCEnum}
  \item
    \TOCLine{cha:fcs:def}

    Definition and \illu{1} of \fc{0}.
  \item
    \TOCLine{cha:fcs:support}

    Link between \fc{1} and \ros{1}.
  \end{TOCEnum}
\end{note}


\section{\fc{3}}
\label{cha:fcs:def}

\begin{note}[\fc{2} definition]
  We define a \emph{\fc{0}} as follows:

  \begin{definition}[\fc{3}]
    \label{def:fc}
    \cenLine{
      \begin{VAREnum}
      \item
        Agent: \vAgent{}
      \item
        Proposition: \(\psi\)
      \item
        Value: \(v'\)
      \item
        \pool{0}: \(\Psi\)
      \item
        \mbox{ }
      \end{VAREnum}
    }

    \begin{itemize}
    \item
      \(\pv{\psi}{v'}\) is a \emph{\fc{0}} from \(\Psi\), for \vAgent{}.
    \end{itemize}

    \emph{If and only if}

    \begin{itemize}
    \item
      There is some action \(a\) such that both \ref{def:fc:act} and \ref{def:fc:result} are true:
      \begin{enumerate}[label=\alph*., ref=(\alph*), series=fcCounter]
      \item
        \label{def:fc:act}
        \(a\) is available to \vAgent{}.
        \mbox{ }\hfill (\vAgent{} may easily and immediately do \(a\).)
      \item
        \label{def:fc:result}
        \vAgent{} is concluding \(\pv{\psi}{v'}\) from \(\Psi\), when \vAgent{} does \(a\).
        \begin{itemize}
        \item
          Without use of any novel information obtained by doing \(a\).
        \end{itemize}
      \end{enumerate}
    \end{itemize}
    \vspace{-\baselineskip}
  \end{definition}

  This definition parallels the definition of a \pevent{} (\autopageref{def:potenital-event}).
  The idea of a \fc{} is to capture a possible conclusion via the sense of possibility required for the progressive, and what this entails.%
  \footnote{
    See \autoref{cha:clar:sec:Cing}, \autopageref{cha:clar:sec:Cing} for more details.
  }
  In particular, that there is an event in which the agent concludes via \assuPP{}.

  \begin{proposition}[\fc{3} and \pevent{1}]%
    \label{prop:fc:pevent}%
        \cenLine{
      \begin{VAREnum}
      \item
        Agent: \vAgent{}
      \item
        Proposition: \(\psi\)
      \item
        Value: \(v'\)
      \item
        \pool{0}: \(\Psi\)
      \item
        \mbox{ }
      \end{VAREnum}
    }
    \begin{itemize}
    \item
      \begin{itenum}
      \item[\emph{If}:]
        \(\pv{\psi}{v'}\) from \(\Psi\) is a \emph{\fc{0}} for \vAgent{}.
      \item[\emph{Then}:]
        There is a \pevent{} in which \vAgent{} concludes \(\pv{\psi}{v'}\) from \(\Psi\).
      \end{itenum}
    \end{itemize}
    \vspace{-\baselineskip}
  \end{proposition}

  \begin{argument}{prop:fc:pevent}
    By definition of a \pevent{} in which \vAgent{} \(\alpha\)'s (\peventpage{}) it must be the case there is some action \(a\) such that:
    \begin{enumerate}[noitemsep]
    \item[\ref{def:PE:action}]
      \vAgent{} may easily and immediately do \(a\).
    \item[\ref{def:PE:prog}]
      \(\text{Prog}(e, \alpha)\) is true of the event \(e\) in which \vAgent{} does \(a\).
    \end{enumerate}
    The definition of a \fc{} simply specifies \(\alpha\) as `concludes \(\pv{\psi}{v'}\) from \(\Psi\)', and adds a qualification.
  \end{argument}

  The added qualification is important.
  Intuitively, \fc{1} follow from information the agent has.
  And, by performing an action an agent may obtain novel information which allows they to conclude some proposition has some value.

  For example, consider an agent which a weak sense of time who is wearing a watch.
  There is a \pevent{} in which the agent concludes it's 5 o'clock.
  For, the agent may look at their watch.
  However, \pv{\propI{It's 5 o'clock}}{\valI{True}} is not a \fc{} for the agent, as without use of the clock's display, the agent would only conclude it's some time in the afternoon.
\end{note}

\begin{note}
  A handful of \illu{1} follow below.
  Still, though `\fc{0}' is a technical term, it is intended to associate with a sense of the common term `foregone conclusion'.
  In this sense, a foregone conclusion is an expected result of reasoning.
  For example, consider the following passage:

  \begin{quote}
    [\dots] Russell's evaluation of such sentences as false is predetermined by his existence presuppositional semantics for the ‘existential' quantifier, and by the fact that his logic permits no alternative means of considering the semantic status of sentences ostensibly containing proper names for nonexistent objects.
    This makes it an altogether philosophically foregone conclusion that sentences like ‘Pegasus is winged,' which many logicians would otherwise consider to be true propositions of mythology, are false.\newline
    \mbox{ }\hfill\mbox{(\cite[6]{Jacquette:2002up})}
  \end{quote}

  \noindent \citeauthor{Jacquette:2002up} is discussing what follows from~\citeauthor{Russell:1905aa}'s analysis of definite descriptions.
  Specifically, from \citeauthor{Russell:1905aa}'s analysis it follows \propI{Pegasus is winged} is false.
  (There are no four-legged winged mammals, etc.)

  The above sense of the term `foregone conclusion' contrasts against a sense with which a conclusion which has been settled in advance of reasoning.%
  \footnote{
    For example:
    \begin{quote}
      When can a Bayesian select an hypothesis \emph{H} and design an experiment (or a sequence of experiments) to make certain that, given the experimental outcome(s), the posterior probability of \emph{H} will be greater than its prior probably?
      We discuss an elementary result that establishes sufficient conditions under which this reasoning to a foregone conclusion cannot occur.%
      \mbox{ }\hfill\mbox{(\cite[1228]{Kadane:1996vu})}
    \end{quote}
  }
  And, a sense in which a forgone conclusion is some unavoidable state of affairs.%
  \footnote{
    For example:
    \begin{quote}
      [どうぜ][\dots] Expresses an attitude of resignation or carelessness on the part of the speaker, in the sense that regardless of what s/he does, the conclusion or outcome is foregone and cannot be changed by the will or effort of an individual.%
      \mbox{ }\hfill\mbox{(\cite[332--333]{kurufushamashii:2015un})}
    \end{quote}
    See also \citeauthor{Grice:1957vg}'s discussion of intention recognition (\citeyear[385]{Grice:1957vg}/\citeyear[219]{Grice:1989uf}), and \citeauthor{Machover:1996vu}'s preface of their approach to the G\"{o}del-Rosser First Incompleteness Theorem (\citeyear[viii]{Machover:1996vu}).
  }

  Still, the similarities between the non-technical `foregone conclusion' and technical `\fc{}' are only for intuition.
  We do not present our definition of a \fc{} as an analysis of the non-technical term.%
  \footnote{
    \label{fn:fc-ability}
    In particular, I suspect `\fc{3}' is close to an ability modal while `foregone conclusion' is close to a compulsion modal.
    (See (\cite{Mandelkern:2017aa}) for discussion).
    Indeed, our definition of a \fc{} is close to an act conditional analysis of ability.
    However, there is no immediate parallel.
    We return to this observation in \autoref{cha:sec:fcs-def:ability}.
  }
\end{note}

\subsection{Illustrations}
\label{cha:fcs:illu}

\begin{note}
  We begin with a few \scen{1} in which \(\pv{\psi}{v'}\) from \(\Psi\) (plausibly) \emph{is} a \fc{}.
  Then, we consider some \scen{1} in which \(\pv{\psi}{v'}\) from \(\Psi\) is \emph{not} (clearly) a \fc{}.
\end{note}

\subsubsection*{\illu{3} where \(\pv{\psi}{v'}\) is a \fc{1} from \(\Psi\)}
\label{cha:fcs:illu:yes}

\begin{note}[Chess I]
  \begin{scenario}[\citeauthor{Emms:2000aa}' Puzzle 113 (\citeyear[33]{Emms:2000aa})]%
    \label{illu:fc:chess:I}%
    \mbox{ }\hfill%
    \begin{adjustbox}{minipage=\linewidth,scale=.8}
      \centering
      \newchessgame[
      setwhite={pa2,pb2,pc2,pd3,pf2,pg3,ra1,re1,bd4,kg1,qe5},
      addblack={ra8,pa7,ba6,pb5,rc8,pd5,pf7,kg8,qg4,ph7,ph4},
      ]%
      \setchessboard{showmover=false}%
      \chessboard
    \end{adjustbox}%
    \label{fig:chess:easy}%
    \hfill\mbox{ }

    \begin{center}
      Is possible for White to checkmate in a single move?
    \end{center}
    \vspace{-\baselineskip}
  \end{scenario}
\end{note}

\begin{note}
  The conclusion of interest:%
  \footnote{
    Or: \pv{\propI{It is possible for White to checkmate in a single move}}{\valI{True}}
  }

  \begin{enumerate}[label=C\thescenarioCounter., ref=(C\thescenarioCounter)]
  \item
    \label{illu:fc:chess:I:c}
    \pv{\propI{White checkmates in a single move}}{\valI{Possible}}
  \end{enumerate}

  \noindent Whether or not \ref{illu:fc:chess:I:c} is a \fc{} depends on the agent.
  For the present \illu{}, suppose the agent has a basic understanding of chess and will only settle whether it is possible for White to checkmate in a single move by applying that understanding.
  Further, strategy is exhaustive search.
\end{note}

\begin{note}
  To establish \ref{illu:fc:chess:I:c} is a \fc{}, we need some action for which both clauses of \autoref{def:fc} are satisfied.

  Consider the action described by:

  \begin{center}
    `Begin an attempt to solve \citeauthor{Emms:2000aa}' Puzzle 113'.
  \end{center}

  \noindent We walk through each clause in turn.
  For ease we omit the relevant \pool{}.

  \begin{itemize}[leftmargin=*]
  \item
    Clause~\ref{def:fc:act} is satisfied.

    The action is something the agent may easily and immediately do.

    In almost any state of affairs one may attempt to do almost any thing.
    For example, you may being an attempt to square the circle.
    Failure was assured, but you made an attempt.

    The action is restricted to begin in order to ease reasoning about Clause~\ref{def:fc:result}.
  \item
    Clause~\ref{def:fc:result} is satisfied.

    We break down the argument into three separate components.

    \begin{itemize}
    \item
      There is a possible event in which the agent concludes \ref{illu:fc:chess:I:c}.

      For, given the rules of chess it is possible for White to checkmate in a single move.
      The agent has a basic understanding of chess.
      And, the agent only needs to consider the appropriate move and to verity the move results in checkmate to conclude \autoref{illu:fc:chess:I:c}.

    \item
      The agent is concluding \ref{illu:fc:chess:I:c} when the action is done.

      As the action is to begin an attempt to solve Puzzle 113, what follows is the attempt.
      To see the agent is concluding \ref{illu:fc:chess:I:c} when they attempt to do so, consider two (motivated) conditionals:

      \begin{enumerate}[label=\arabic*., ref=(\arabic*)]
      \item
        \label{illu:fc:chess:I:cond:1}
        If agent picks \wmove{Qh8}, then agent is concluding \ref{illu:fc:chess:I:c}.%
        \smallskip

        By assumption, the agent has a basic understanding of chess and is motivated.
        And, as the agent needs to verify \wmove{Qh8} in checkmate, the agent will.
      \item
        \label{illu:fc:chess:I:cond:2}
        If agent picks a move other than \wmove{Qh8} then after some reasoning the agent picks a novel move.%
        \smallskip

        By parallel reasoning.

        The agent needs only verify the move other than \wmove{Qh8} fails to result in checkmate, and then pick some other move.
        The agent will verify, given their understanding of chess.
        And, the agent will pick some novel move as their strategy is exhaustive search.
      \end{enumerate}

      Now, whichever move the agent picks, either \ref{illu:fc:chess:I:cond:1} or \ref{illu:fc:chess:I:cond:2} is true.
      And, as there are finitely many moves for the agent to pick, \ref{illu:fc:chess:I:cond:1} will (eventually be true).
      Hence, as an event in which the agent concludes \ref{illu:fc:chess:I:c} is in progress when \ref{illu:fc:chess:I:cond:1} is true, an event in which the agent concludes \ref{illu:fc:chess:I:c} is (also) in progress when \ref{illu:fc:chess:I:cond:2} is true.

    \item
      The agent does not use any novel information obtained by beginning an attempt.

      For, this action does not provide the agent  with any novel information.
      The agent has looked at the puzzle and has a basic understanding of the rules of chess.
      The agent does not appeal to any information that they do not already possess, and any information obtained follows.
    \end{itemize}
  \end{itemize}

  This argument assumes a particular strategy.
  This need not be the case, but alternative strategies are difficult to describe.
  If you have a basic understanding of chess then I suggest you convince yourself the conclusion is a \fc{} by attempting the puzzle.
  You need only conclude \ref{illu:fc:chess:I:c} and verify that you were concluding \ref{illu:fc:chess:I:c} after you began (and did not make use of any novel information).
\end{note}

\begin{note}
  In broad structure, the idea with \autoref{illu:fc:chess:I} is some effective method, sufficient information to apply method, and an opportunity to apply.

  For example, consider arithmetic.
  I expect the truth of \(13 \cdot 4 = 52\), \(96 \div 4 = 24\), and \(23 \cdot 15 = 345\) are \fc{1}.
  Likewise, if you have basic understanding of propositional logic, then the validity of various theorems are \fc{1}.
  And, if you enjoy Sudoku puzzles then, so long as you have the puzzle and sufficient time to spare, the solution to any puzzle is a \fc{}.
\end{note}

\begin{note}[Non-deductive \illu{1}]
  \autoref{illu:fc:chess:I} and parallel examples are motivated by an agent's grasp on some effective method to solve a type of problem.
  However, \(\pv{\psi}{v'}\) being a \fc{} from \(\Psi\) does not require an effective method.
  It need only be the case that the agent is concluding \(\pv{\psi}{v'}\) from \(\Psi\) after an action is done.
  Consider, the following \scen{0}:

  \begin{scenario}[Sunny days]%
    \label{illu:fc:sunny}%
    It's mid summer day in the Bay Area.
  \end{scenario}

  \noindent For me, the following conclusion is a \fc{} from some \pool{}:

  \begin{enumerate}[label=C\thescenarioCounter., ref=(C\thescenarioCounter)]
  \item
    \label{illu:fc:sunny:c}
    \pv{\propI{It will rain tomorrow}}{\valI{False}}
  \end{enumerate}

  \noindent%
  There is no effective method for me to determine whether it will rain tomorrow, and I recognise there may be rain tomorrow.
  Still, I am sufficiently committed to some uniformity principle.%
  \footnote{
    Cf. (\cite[70]{Hempel:1965aa}),~(\cite{Henderson:2022aa}).
  }
  And, that the principle together with past experience, ensure that if I consider whether it will rain tomorrow, I conclude it will not rain.%
\end{note}

\begin{note}[Poppies]
  To finish, we take something from literature:

  \begin{scenario}[Poppies]
    \label{illu:fc:poppies}
    \mbox{ }
    \vspace{-\baselineskip}
    \begin{quote}
      Was Tarquinius Superbus in seinem Garten mit den Mohnköpfen sprach, verstand der Sohn, aber nicht der Bote.

      [What Tarquinius Superbus said in the garden by means of the poppies, the son understood but the messenger did not].\newline
      \mbox{ }\hfill\mbox{(\cite[3]{Kierkegaard:1983ta}/\cite[190]{Hamann:1822vp})}
  \end{quote}
  \vspace{-\baselineskip}
  \end{scenario}

  \noindent The quote is from the epigraph to~\citeauthor{Kierkegaard:1983ta}'s \hyperlink{cite.Kierkegaard:1983ta}{Fear and Trembling}.
  \hyperlink{cite.Kierkegaard:1983ta}{H.\ Hong and E.\ Hong} detail the relevant background:

  \begin{quote}
    When the son of Tarquinius Superbus had craftily gotten Gabii in his power, he sent a messenger to his father asking what he should do with the city.
    Tarquinius, not trusting the messenger, gave no reply but took him into the garden, where with his cane he cut off the flowers of the tallest poppies.
    The son understood from this that he should eliminate the leading men of the city.%
    \mbox{ }\hfill\mbox{(\citeyear[339]{Kierkegaard:1983ta})}
  \end{quote}

  \noindent For Superbus' son, but not for the messenger the following was a \fc{} from some \pool{}:

  \begin{enumerate}[label=C\thescenarioCounter., ref=(C\thescenarioCounter)]
  \item
    \label{illu:fc:poppies:c}
    \pv{\propI{Eliminate the leading men of the city}}{\valI{Should}}
  \end{enumerate}

  \noindent Or, at least, Superbus \emph{expected}~\ref{illu:fc:poppies:c} be a \fc{} for his son.
\end{note}

\subsubsection*{\(\pv{\psi}{v'}\) is not a \fc{1} from \(\Psi\)}
\label{cha:fcs:illu:no}

\begin{note}
  \(\pv{\psi}{v'}\) may fail to be a \fc{} from \(\Phi\) in two basic ways:

  \begin{enumerate}[label=\alph*., ref=(\alph*), noitemsep]
  \item
    There is no action the agent may easily and immediately do.
  \item
    The agent may fail to be concluding \(\pv{\psi}{v'}\) from \(\Psi\), for any easy and immediate action.
  \end{enumerate}

  We focus on the latter type of case, which may be further sub-divided:

  \begin{itemize}[noitemsep]
  \item
    The agent does not have the capacity to conclude.
  \item
    The agent has the capacity, but would conclude something which conflicts with \(\pv{\psi}{v'}\).
  \end{itemize}

  The following \illu{} take each sub-case, respectively.
\end{note}

\paragraph*{Lack of capacity}

\begin{note}[Chess II]
  Consider \autoref{illu:fc:chess:II} in the same context as \autoref{illu:fc:chess:I}:

  \begin{scenario}[\citeauthor{Emms:2000aa}' Puzzle 150 (\citeyear[33]{Emms:2000aa})]%
    \label{illu:fc:chess:II}%
    \mbox{ }\hfill%
    \begin{adjustbox}{minipage=\linewidth,scale=0.8}
      \centering
      \newchessgame[
      setwhite={ka5,pa3,pb4,pc4,pe5,pf6,bg5,bh5},
      addblack={pa6,pb7,pc6,pe6,pf7,kc7,nd7,nd4},
      ]%
      \setchessboard{showmover=false}%
      \chessboard
    \end{adjustbox}%
    \label{fig:chess:intro}%
    \hfill\mbox{ }

    \begin{center}
      It is possible for Black to checkmate in four moves?
    \end{center}
    \vspace{-\baselineskip}
  \end{scenario}

  \noindent The conclusion of interest is:

  \begin{enumerate}[label=C\thescenarioCounter., ref=(C\thescenarioCounter)]
  \item
    \label{illu:fc:chess:II:c}
    \pv{\propI{Black checkmates in four moves}}{\valI{Possible}}
  \end{enumerate}

  The difference between \autoref{illu:fc:chess:II} and \autoref{illu:fc:chess:I} is the difficulty of the puzzle.%
  \footnote{
    \citeauthor{Emms:2000aa} suggests:
    \textquote{%
      \variation{1... Nb6!}%
      (threatening \variation{2... Nb3\#})%
      \variation{2. b5}%
      (or \variation{2. Bd1 Nxc4+} \variation{3. Ka4 b5\#})%
      \variation{2... c5!}%
      \variation{3. bxa6 Nxc4+}%
      \variation{4. Ka4 b5\#}%
      \textbf{(0-1)}%
      }
      (\citeyear[46]{Emms:2000aa}).
    }
    An agent may do an action described by `begin an attempt to solve\dots'.
    However, a basic understanding of chess does not imply the capacity to work through a complex sequence of moves in the relevant situation.

    Note, this sense of capacity is not limited to the possibility of concluding.
    There may be a possible event in which the agent concludes~\ref{illu:fc:chess:II:c}.
    For, the agent may get lucky and consider a suitable sequence of four moves.
    However, the possibility of a lucky sequence of events does not secure the truth of the progressive.
    For example, there is a possible event in which one wins the lottery after purchasing a ticket, but purchasing a ticket does not entail an event in which one wins the lottery is in progress.
    For some, chess puzzles aren't sufficiently fun to put the effort in.%
    \footnote{
      Specifically,~\ref{illu:fc:chess:II:c} was not a \fc{} for me.
      I gave up after fifteen minutes or so.
    }
\end{note}

\begin{note}
  In general, it may be possible for the agent to conclude, but the possibility is weaker than the sense of possibility captured by the truth of the progressive.

  For further examples, consider uninteresting formal derivations or difficult arithmetic.
  One may calculate \(4^{4!}\), or \dots\space use a calculator.
\end{note}

\paragraph*{Conflict}

\begin{note}

  \begin{scenario}[Knowing whether and belief]%
    \label{ill:fcs:kw}%
    \citeauthor{Barker:1975un} suggests the following two principles hold with respect to knowing whether:

    \begin{enumerate}[label=(\Alph*), ref=(\Alph*), noitemsep]
    \item
      \label{Barker:1975un:A}
      If \emph{S} knows whether \emph{p} and \emph{S} believes that \emph{p}, then \emph{p}.
    \item
      \label{Barker:1975un:B}
      If \emph{S} knows whether \emph{p} and \emph{S} believes that not-\emph{p}, then not-\emph{p}.%
      \mbox{ }\hfill\mbox{(\citeyear[281]{Barker:1975un})}
    \end{enumerate}
  \end{scenario}

  \noindent The conclusions of interest are:

  \begin{quote}
  \begin{enumerate}[label=C\Alph*., ref=(C\Alph*), noitemsep]
  \item
    \label{ill:fcs:kw:cA}
    \pv{\propI{If \emph{S} knows whether \emph{p} and \emph{S} believes that \emph{p}, then \emph{p}}}{\valI{True}}
  \item
    \label{ill:fcs:kw:cB}
    \pv{\propI{If \emph{S} knows whether \emph{p} and \emph{S} believes that not-\emph{p}, then not-\emph{p}}}{\valI{True}}
  \end{enumerate}
\end{quote}

  \noindent I expect neither~\ref{ill:fcs:kw:cA} nor~\ref{ill:fcs:kw:cB} are \fc{1}.
  For, rather than conclude either~\ref{ill:fcs:kw:cA} or~\ref{ill:fcs:kw:cB} one will find a counterexample.%
  \footnote{
    For example, consider an agent \emph{A} playing speed chess.
    It's the end game, and \emph{A} believes they have a winning strategy.
    \emph{A} knows whether they have a winning strategy, but doesn't have time to work through the details.
    Given \ref{Barker:1975un:A}, \emph{A} has a winning strategy.
    \emph{A} does not have a winning strategy.
  }
  And, given a counterexample one will not conclude the principle is true.
\end{note}

\begin{note}
  Structurally, \autoref{ill:fcs:kw} is no different from an formula or equation which does not hold.
  For example, \((\phi \rightarrow (\psi \rightarrow \phi)) \rightarrow \phi\), or \(4 \times 3 = 14\).
\end{note}

\subsection{Strengthening the definition}

\begin{note}
  The definition of a \fc{} may be strengthened in various ways.
  For example:

  \begin{enumerate}[label=\alph*., ref=(\alph*), resume*=fcCounter]
  \item
    \label{def:fc:alt:c}
    For any proposition \(\chi\), value \(v''\), and action \(b\) such that \vAgent{} is concluding \(\pv{\chi'}{v''}\) from \(X\) after \(b\) is done:
    \begin{itemize}
    \item
      Either~\ref{def:fc:extra:1} or~\ref{def:fc:extra:2} is the case:
      \begin{enumerate}[label=\arabic*., ref=\arabic*]
      \item
        \label{def:fc:extra:1}
        \(\chi\) is \(\psi\) and \(v''\) is \(v'\).
      \item
        \label{def:fc:extra:2}
        Throughout event in which \vAgent{} concludes \(\pv{\chi}{v''}\) from \(X\), there is some action \(c\) such that:
        \begin{itemize}
        \item
          \vAgent{} may easily and immediately do \(c\).
        \item
          \vAgent{} is concluding \(\pv{\psi}{v'}\) from \(\Psi\) when \vAgent{} does \(c\).
          \begin{itemize}
          \item
            Without use of any novel inf.\ obtained by doing \(a\) and \(b\).
          \end{itemize}
        \end{itemize}
      \end{enumerate}
    \end{itemize}
  \end{enumerate}

  \noindent Paraphrased:

  \begin{itemize}[noitemsep]
  \item
    Sub-clause~\ref{def:fc:extra:1} states \vAgent{} is concluding \(\pv{\psi}{v'}\) from \(\Psi\) after \vAgent{} does \(b\).
  \item
    Sub-clause~\ref{def:fc:extra:2} states \vAgent{} \(\pv{\psi}{v'}\) from \(\Psi\) remains a \fc{} after \vAgent{} does \(b\).
  \end{itemize}

  \noindent This is a much stronger condition.
  Not only is it the case that concluding, but remains the case that concluding is available for any other conclusion.

  This rules out pairwise conflicts.%
  \footnote{
    Though, does not rule out more.
  }

  We do not add any additional clauses as this is all we need.
  Still, additional clauses may be added.

  No restrictions placed on what an agent may conclude.
  So, no restrictions placed on \fc{1}.

  If \fc{} then enough to conclude.
  And, if the may conclude conflicting things, so be it.
  This is no different from conclusions.

  \fc{3}, like conclusions, are descriptive.
  Do not include considerations about `good' or `bad' conclusions.
\end{note}

\begin{note}
  Still, if you prefer something stronger, that's okay.

  Indeed, probably something a little stronger than Clause~\autoref{def:fc:alt:c}.
  For, an agent may conclude two things and then fail.
  Perhaps, in the words of Achilles, a \fc{} should be such that \textquote{you ca'n't help yourself} (\cite[280]{Carroll:1895uj}).
  Though, what this amounts to\dots

  There's nothing in what follows that strictly requires that definition of a \fc{} is as general as it is.
  Additional restrictions limit instances of \fc{1}.
  And, use is existence of \fc{1}.
  So long as \fc{1} with no \wit{} exist, good.

  If do restrict, though, some things will need to be restructured.
\end{note}

\section{\fc{3} and \ros{1}}
\label{cha:fcs:support}

\begin{note}
  We now consider the relationship between \fc{1} and \ros{0}.

  Our primary goal is to establish the following conditional:

  \begin{itemize}
  \item
    If \(\pv{\psi}{v'}\) is a \fc{} from \(\Psi\), then a \ros{} holds between \(\pv{\psi}{v'}\) and \(\Psi\).
  \end{itemize}

  \noindent%
  We break down the argument into two propositions, and then highlight the relation to \issueConstraint{}.
\end{note}

\begin{note}
  First, a helper proposition.
  Relates necessary and sufficient conditions.
  We begin by defining our understanding of necessary and sufficient conditions.

  \begin{definition}[Sufficient and necessary conditions]
    \label{def:NScon}
    \cenLine{
      \begin{VAREnum}
      \item
        Agent: \vAgent{}
      \item
        Condition: \(C\)
      \item
        Attribute: \(A\)
      \item
        \mbox{ }
      \end{VAREnum}
    }

    \begin{itemize}
    \item
      \(C\) is a \emph{sufficient} condition for attribute \(A\), if and only if:
      \begin{itemize}
      \item
        \emph{If} condition \(C\) obtains, \emph{then} \vAgent{} has attribute \(A\)
      \end{itemize}
    \item
      \(C\) is a \emph{necessary} condition for attribute \(A\), if and only if:
      \begin{itemize}
      \item
        \emph{If} \vAgent{} has attribute \(A\), \emph{then} condition \(C\) obtains.
      \end{itemize}
    \end{itemize}
    \vspace{-\baselineskip}
  \end{definition}

  \noindent%
  As usual, the conditionals by which sufficient and necessary conditions are defined are material conditions.
  Hence, \autoref{def:NScon} captures the \textquote{standard theory} of sufficient and necessary conditions
  (\cite[cf.][\S2]{Brennan:2022aa}).
  I doubt the follow proposition depends on \autoref{def:NScon}.
  However, the argument given does.

  \begin{proposition}[Attribution]
    \label{prop:attribution}
    \cenLine{
      \begin{VAREnum}
      \item
        Agent: \vAgent{}
      \item
        \mbox{ }
      \end{VAREnum}
    }

    \begin{itemize}
    \item
      For any attribute \(X\):
      \begin{itenum}
      \item[\emph{If}:]
        \vAgent{} satisfies all necessary conditions for having attribute \(X\).
      \item[\emph{Then}:]
        \vAgent{} satisfies some sufficient condition for having attribute \(X\).
      \end{itenum}
    \end{itemize}
    \vspace{-\baselineskip}
  \end{proposition}

  \begin{argument}{prop:attribution}
    We argue by contraposition.

    Consider sufficient conditions \(S_{1}, S_{2}, \dots, S_{j}\)

    If \vAgent{} has \(X\), then \vAgent{} satisfies some \(S_{i}\).
    For, \vAgent{} has \(x\).
    Hence, some condition obtains, and this is sufficient.

    Therefore, disjunction of \(S_{1}, S_{2}, \dots, S_{j}\).

    Suppose \vAgent{} does not satisfy disjunction.
    Then, \vAgent{} does not have attribute \(X\).
    For, if did, then some sufficient condition.

    Therefore, disjunction is necessary.

    If does not satisfy any sufficient, does not satisfy disjunction, does not satisfy all necessary.
  \end{argument}
\end{note}

\begin{note}
  Importance of \autoref{prop:attribution} is limitation on \ros{}.
  Only sufficient condition (\supportI{}).
  However, \supportII{} limits necessary conditions.
\end{note}


\begin{note}
  \begin{proposition}[\fc{3} and \ros{1}]
    \label{prop:fcs-only-if-pot-support}
    \cenLine{
      \begin{VAREnum}
      \item
        Agent: \vAgent{}
      \item
        Proposition: \(\psi\)
      \item
        Value: \(v'\)
      \item
        \pool{2}: \(\Psi\)
      \item
        \mbox{ }
      \end{VAREnum}
    }

    \begin{itenum}
    \item[\emph{If}:]
      \(\pv{\psi}{v'}\) is a \fc{0} from \(\Psi\) for \vAgent{} throughout \(e\).
    \item[\emph{Then}:]
      A \ros{0} between \(\pv{\psi}{v'}\) and \(\Psi\) holds for \vAgent{} throughout \(e\).
    \end{itenum}
    \vspace{-\baselineskip}
  \end{proposition}

  \begin{argument}{prop:fcs-only-if-pot-support}
    Suppose \(\pv{\psi}{v'}\) is a \fc{0} from \(\Psi\), for \vAgent{}.

    \smallskip
    Observe there is a \pevent{} in which \vAgent{} concludes \(\pv{\psi}{v'}\) from \(\Psi\).
    This is immediate by \autoref{prop:fc:pevent} (\autopageref{prop:fc:pevent}).

    \smallskip
    Consider the \pevent{} in which \vAgent{} concludes \(\pv{\psi}{v'}\) from \(\Psi\).
    As \vAgent{} concludes \(\pv{\psi}{v'}\) from \(\Psi\), then by~\supportI{} (\supportIpage{}), a \ros{} between \(\pv{\psi}{v'}\) and \(\Psi\) holds when \vAgent{} \eval{1} \(\psi\) as having value \(v'\).

    We have a \potential{} \ros{} between \(\pv{\psi}{v'}\) and \(\Psi\).

    Our goal is to establish this \potential{} \ros{} is a \ros{} \emph{simpliciter}.

    \smallskip
    Given the \pevent{}, \ros{}.
    Hence, the agent satisfies all necessary conditions.
    Therefore, by \autoref{prop:attribution}, some sufficient condition.

    Now \supportII{}, denies for all \(\phi\), \(v\), \(\Phi\).
    So, there may be cases.

    We have, \(\Psi\).
    sufficient to ensure progressive.

    \smallskip
    Something from \pevent{}.
    However, this cannot be a necessary condition in general.
    For, this contradicts \supportII{}.

    So, between now and \pevent{}, nothing necessary in general.

    \smallskip
    Deductive argument ends.
    At this point, pressure.

    There is nothing uniform between cases.
    Therefore, it seems no non-ad-hoc way of denying \ros{}.
  \end{argument}

  Think in terms of propositional justification.%
  \footnote{
    Distinct, however, from something like reflection.
    For, dealing with ideals.
    And, no need to complete reasoning.
  }
  Or, if you prefer, substitute \autoref{prop:fcs-only-if-pot-support} for \supportII{}.
  I prefer deductive arguments, and have the argument as motivation for idea/assumption.
  However, does not make sense with structure.
\end{note}

\begin{note}
  Definition of a \fc{} is action, concluding, qualification.

  Key feature is sufficient to get conclusion.

  Qualification is key.
  For, else something may fail.
\end{note}

\begin{note}
  To clarify the importance of \autoref{prop:fcs-only-if-pot-support}:

  \begin{proposition}
    \label{prop:fc-wit}
    \cenLine{
      \begin{VAREnum}
      \item
        An agent: \vAgent{}
      \item
        A proposition: \(\phi\)
      \item
        A value: \(v\)
      \item
        A \pool{0}: \(\Phi\)
      \item
        \mbox{ }
      \end{VAREnum}
    }
    \begin{itenum}
    \item[\emph{If}:]
      Both~\ref{prop:fc-wit:fc} and~\ref{prop:fc-wit:noC} hold:
      \begin{enumerate}[label=\alph*., ref=(\alph*)]
      \item
        \label{prop:fc-wit:fc}
        \(\pv{\psi}{v'}\) is a \fc{0} from \(\Psi\), for \vAgent{}
      \item
        \label{prop:fc-wit:noC}
        \vAgent{} has not concluded \(\pv{\psi}{v'}\) from \(\Psi\).
      \end{enumerate}
    \item[\emph{Then}:]
      Both~\ref{prop:fc-wit:ros} and~\ref{prop:fc-wit:noW} hold:
      \begin{enumerate}[label=\alph*\('\)., ref=(\alph*\('\))]
      \item
        \label{prop:fc-wit:ros}
        A \ros{} between \(\pv{\psi}{v'}\) and \(\Psi\), for \vAgent{}.
      \item
        \label{prop:fc-wit:noW}
        \vAgent{}~does~not~have~a~\wit{}~for~the~\ros{}~btw.\~\(\pv{\psi}{v'}\)~and~\(\Psi\).
      \end{enumerate}
    \end{itenum}
    \vspace{-\baselineskip}
  \end{proposition}

  \begin{argument}{prop:fc-wit}
    \ref{prop:fc-wit:fc} entails \ref{prop:fc-wit:ros} by \autoref{prop:fcs-only-if-pot-support}.

    \noindent \ref{prop:fc-wit:noC} entails \ref{prop:fc-wit:noW} by the definition of a \wit{} (\witpage{}).
  \end{argument}

  For, in order to argue against \issueConstraint{}, need some \(\pvp{\psi}{v'}{\Psi}\) such that answers \qWhyV{}.
  Consider \autoref{scen:calc}.
  \fc{}.
  However, intuitively does not answer \qWhy{} nor \qWhyV{}.
\end{note}


\section*{Summary}
% \label{cha:fcs::summary}


% \subsubsection{\wit{3}}
% \label{sec:wit3}

% \begin{note}
%   Note, there is nothing that requires a \fc{} has no prior conclusion.

%   Hence, it may be the case that an agent has a \wit{} for a \fc{}.

%   \begin{proposition}[\wit{3} for \fc{1}]
%     \label{prop:wit-for-fc}
%     Possible for \wit{} from \fc{}.
%   \end{proposition}

%   \begin{argument}{prop:wit-for-fc}
%       Immediate by def of \wit{}.
%     \end{argument}
%   \end{note}


%%% Local Variables:
%%% mode: latex
%%% TeX-master: "master"
%%% End:


% \begin{note}
%   \begin{illustration}[Modal logic I]
%     \label{illu:fc:logic:CR}
%     The modal system obtained from adding \(\Diamond\Box p \rightarrow \Box\Diamond p\) as an axiom to \(\mathbf{K}\) is canonical for the Church-Rosser property.

%     I.e. the canonical model \(W,R,V\) for \(\mathbf{K} + \Diamond\Box p \rightarrow \Box\Diamond p\) is such that \(\forall s,t,u((Rst \land Rsu) \rightarrow \exists v(Rtv \land Ruv))\).
%   \end{illustration}

%   \autoref{illu:fc:logic:CR} is a \fc{} for me.
%   Though, in contrast to the previous \illu{1}, I think there is a reasonable change that \autoref{illu:fc:logic:CR} is not a \fc{} for you.

%   Fairly routine, but two important things.
%   First, grasp on the relevant concepts.
%   If you are unaware of how to construct canonical models for normal modal logics, then unlikely that you will complete the relevant proof.
%   Second, sufficient familiarity with the relevant concepts.
%   The proof is mostly straightforward, though some care needs to be taken in showing that the canonical model for \(\mathbf{K} + \Diamond\Box p \rightarrow \Box\Diamond p\) has the Church-Rosser property.
%   Proof by contradiction is my preferred way of obtaining the result, but this requires keeping certain facts about the canonical model in mind.%
%   \footnote{
%     A slightly more interesting variation is showing that \(\mathbf{K} + \Diamond\Box p \rightarrow \Box\Diamond p\) is (strongly) complete with respect to the class of frame which have the Church-Rosser property without detour via a canonical model.
%   }

%   Similar features as \illu{1} given above.

%   In particular, perhaps clearer than \autoref{illu:gist:Sudoku} in terms of mistakes.
%   For, go down some wrong path, still will not conclude until counterexample.
%   And, this is very hard to get.
% \end{note}

  %   \begin{itemize}
  %   \item
  %     The agent has a good understanding of some formal proof system.
  %     For example, some Fitch-style system.
  %   \item
  %     The agent has a good understanding of some method to construct semantic proofs.
  %     For example, by constructing truth tables, or reasoning about valuation functions.
  %   \item
  %     The agent understands the proof system is sound.
  %     That is to say, the agent understands there exists a proof of some sentence \(A\) \emph{only if} \(A\) is true given an arbitrary valuation.
  %   \end{itemize}
  %   The agent constructs the following proof:
  %   \begin{center}
  %     \begin{fitch}
  %       \phantlabel{illu:sPF:1}\fa \fh (P \rightarrow Q) \rightarrow P \\
  %       \phantlabel{illu:sPF:2}\fa \fa \fh  \lnot P \\
  %       \phantlabel{illu:sPF:3}\fa \fa \fa \fh  P \\
  %       \phantlabel{illu:sPF:4}\fa \fa \fa \fa  \bot & \(\bot\)\textbf{Intro:}\hyperref[illu:sPF:2]{2},\hyperref[illu:sPF:3]{3}\\
  %       \phantlabel{illu:sPF:5}\fa \fa \fa \fa  Q & \(\bot\)\textbf{Elim:}\hyperref[illu:sPF:4]{4}\\
  %       \phantlabel{illu:sPF:6}\fa \fa \fa P \rightarrow Q & \(\rightarrow\)\textbf{Intro:}\hyperref[illu:sPF:3]{3}--\hyperref[illu:sPF:5]{5} \\
  %       \phantlabel{illu:sPF:7}\fa \fa \fa P & \(\rightarrow\)\textbf{Elim:}\hyperref[illu:sPF:1]{1},\hyperref[illu:sPF:6]{6}\\
  %       \phantlabel{illu:sPF:8}\fa \fa \fa \bot & \(\bot\)\textbf{Intro:}\hyperref[illu:sPF:2]{2},\hyperref[illu:sPF:7]{7}\\
  %       \phantlabel{illu:sPF:9}\fa \fa \lnot\lnot P & \(\lnot\)\textbf{Intro:}\hyperref[illu:sPF:2]{2},\hyperref[illu:sPF:8]{8}\\
  %       \phantlabel{illu:sPF:10}\fa \fa P & \(\lnot\)\textbf{Elim:}\hyperref[illu:sPF:9]{9}\\
  %       \phantlabel{illu:sPF:11}\fa ((P \rightarrow Q) \rightarrow P) \rightarrow P & \(\rightarrow\)\textbf{Intro:}\hyperref[illu:sPF:1]{1}--\hyperref[illu:sPF:10]{10} \\
  %     \end{fitch}
  %   \end{center}
  %   \vspace{-\baselineskip}


%   For an additional example, consider the following from~\citeauthor{Grice:1957vg}'s~\citetitle{Grice:1957vg}:

%   \begin{quote}
%     He intends the audience's recognition of his intention to produce that response to be effective in producing that response.
%     He does not regard it as a foregone conclusion that his action will produce the intended response, whether or not his intention is recognised.%
%     \mbox{ }\hfill\mbox{(\citeyear[385]{Grice:1957vg}/\citeyear[Cf.][219]{Grice:1989uf})}
%   \end{quote}

%   In this case, the term `foregone conclusion' is embedded under negation, to highlight that the agent in question entertains the possibility that the agent's action will not produce the intended response.

% \begin{note}
%   Still, the conclusion need not be easy.

%   \begin{illustration}
%     \[\frac{(3 + \sqrt{3})^{2} + \sqrt{6}^{2} - (2\sqrt{3})^{2}}{2(3 + \sqrt{3})\sqrt{6}} = \frac{1}{\sqrt{2}}\]
%   \end{illustration}

%   I suspect this is a \fc{}.
%   Might need to do some work to recall principles, but it's okay.
% \end{note}

% \begin{note}[Propositional logic]
%   \begin{illustration}[Propositional theorems]
%     \label{illu:sketch:prop-logic}
%     Suppose an agent has a good grasp of propositional logic.
%     In particular:
%     The agent has a good understanding of some method to construct semantic proofs.
%     For example, by constructing truth tables, or reasoning about valuation functions.

%     The agent has some free time, and considers the formula:
%     \[
%       ((P \rightarrow Q) \rightarrow P) \rightarrow P
%     \]
%   \end{illustration}

%   Given the agent's understanding of propositional logic, the following is a \fc{}:
%   \begin{itemize}
%   \item
%     For any valuation \(v\), \(v \vDash ((P \rightarrow Q) \rightarrow P) \rightarrow P \)
%   \end{itemize}

%   There's nothing particularly special about the formula.
%   Given a good understanding, any formula of a reasonable length will do.

%   Note, the agent is concluding.
%   It's fine for the agent to work things through on a piece of paper.
% \end{note}

% \begin{note}[ML II]
%   \begin{illustration}[Modal logic II]
%     \label{illu:fc:ML2}
%     The modal system \(\mathbf{GL} = \mathbf{K} + \Box(\Box p \rightarrow p) \rightarrow \Box p\) is weakly complete with respect to the class of finite strict partial orders (that is, the class of finite irreflexive transitive frames).
%   \end{illustration}

%   \autoref{illu:fc:ML2} is similar in structure to \autoref{illu:fc:logic:CR}.
%   Indeed, both proofs involve constructing a canonical model.
%   The key distinguishing feature of \autoref{illu:fc:ML2}, however, is the difficulty of establishing the canonical model has the desired properties.
%   In particular, the general method I keep in mind for proving the relevant result requires a syntactic proof that \(\vdash_{\mathbf{GL}} \Box p \rightarrow \Box \Box p\).
%   And, as I have failed to recall the relevant syntactic on sufficient occasion, I do not consider the result a \fc{0} from my understanding of modal logic.

%   Hence, the result (plausibly) fails to be a \fc{} from my understanding of modal logic because there is no guarantee that I would provide a proof if I set out to do so.%
%   \footnote{
%     On the other hand, I have completed the relevant proof a sufficient number of times.
%     So, the result is a \fc{0} from whatever premises are associated with my memory.
%   }
% \end{note}

% \begin{note}
%   \illu{3} may be obtained by taking a proposition-value pairing which conflicts with \fc{}.
%   The following is \emph{not} a \fc{}.%
%   \footnote{
%     However, I suspect the following equation is a \fc{}:
%     \[\frac{(3 + \sqrt{3})^{2} + \sqrt{6}^{2} - (2\sqrt{3})^{2}}{2(3 + \sqrt{3})\sqrt{6}} = \frac{1}{\sqrt{2}}\]
%   }

%   \begin{illustration}
%     \[\frac{(3 + \sqrt{3})^{2} + \sqrt{6}^{2} - (2\sqrt{3})^{2}}{2(3 + \sqrt{3})\sqrt{6}} = \frac{1}{\sqrt{3}}\]
%   \end{illustration}
%   For, the equation does not hold.
% \end{note}

  %
  % \footnote{
  %   \nocite{Simon:1997aa}
  %   \textquote{[R]ationality can be bounded by assuming complexity in the cost function or other environmental constraints so great as to prevent the actor from calculating the best course of action} (\citeyear[164]{Simon:1972aa}).
  % }


% \begin{note}
%   This is no different from conclusion that it's 13:48 by looking at the clock.
%   I recognise there is a possibility that the clock is broken.
%   However, nothing to suggest it is.

%   This highlights qualification.
%   Without it, then so long as wearing watch, time is a \fc{}.
%   For, look at watch.
% \end{note}

%%% Local Variables:
%%% mode: latex
%%% TeX-master: "master"
%%% TeX-engine: luatex
%%% End:
