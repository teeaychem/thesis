\chapter{\fc{3}}
\label{cha:foregone-conclusions}

\begin{note}[Foregone-conclusions]
  Basic idea of a foregone-conclusion.

  \begin{restatable}[Foregone-conclusions]{definition}{definitionForegoneC}
    \(\pv{\phi}{v}\) is a foregone-conclusion from some pool of premises \(\Phi\) just in case, given the agent's present epistemic, the agent would not fail to conclude \(\pv{\phi}{v}\) from \(\Phi\) were the agent to reason.
  \end{restatable}

  Whether foregone-conclusion takes agent's present epistemic state as a function.
  However, does not need to be the case that the agent recognises foregone-conclusion.

  At most, witnessing provides information about method.

  For any property \(P\) which would follow from any instance of witnessing reasoning \(\pv{\phi}{v}\) from \(\Phi\), the agent's present epistemic state is sufficient to determine \(P\) without witnessing reasoning from \(\Phi\) to \(\pv{\phi}{v}\).

  Suppose \(P\) follows from concluding.
  Forgegone-conclusion.
  So, agent's present epistemic state, agent would not fail.
  However, it then follows that \(P\).

  Here, restricted \(P\) to follow from any.
  Hence, if there are multiple methods, \(P\) may be restricted.

  However, broaden.
\end{note}

\begin{note}[Intuitive cases]
  Knowing whether and knowing how to.
  More or less interchangeable.

  Know whether \(x + y = z\).
  Know how to calculate \(x + y\).
  Indeed, for any \(z\), know whether \(x + y = z\).

  Sudoku puzzles.
  Know how to figure out.
  So, know whether any solution is valid.

  Of course, in certain cases, there are shortcuts.
  Two even numbers, then know whether by checking whether the last digit is even or odd.
  And, other cases, contingent shortcut, such as two of the same number in a square for Sudoku.

  So, really, knowing how to.
\end{note}

\begin{note}[Weaken]
  \fc{2} is weaker.
  Knowing, factive.
  Though, plausible that these amount to the same thing in various cases.
  Either because \fc{} is determined by knowing how to.
  Or, because knowing is weakened to the agent's perspective.
\end{note}

\begin{note}
  \begin{proposition}
    For any path, present epistemic state determines availability of path.
  \end{proposition}

  Start.
  Then, continue.
  Started from \(\Phi\), so will conclude.
  Hence, no matter choice made, must have taken the possibility of this choice into account.
  So, it must be the case that determined.

  Hence, if witness, then via some path.

  So, witnessing predetermined path.
  Any instances of concluding by witnessing reduces to witnessing predetermined path.

  Witnessing may provide information about path, but witnessing doesn't 


  For any X from W,
  present determines whether or not X from agent's point of view, then forgone conclusion.

  In other words, agent's present epistemic state determines.
  Agent may need to witness to figure out how determined, but witnessing does not influence.
\end{note}

\begin{note}[Non-cases]
  Now, \(p, p \rightarrow q \vdash q\) case.
  Well, determines \(q\), if we ignore possibility of revision.
  However, this doesn't tell us about all X.

  This is much stronger.
  There is nothing witnessing adds which is not already determined.

  2 + 2, peano arithmetic.

  More intuitive.
  Though, still question about deriving 2 + 2 = 4 from peano arithmetic.

  Understanding of arithmetic.
  Then, add two numbers.
  Forgone-conclusion.

  % Sudoku puzzle.
  % Solution is a foregone-conclusion.
\end{note}

\begin{note}
  To \illu{0}, questions and answers.

  Do you know whether \(83\) is prime?

  Not off the top of my head.

  Do you know whether \(28 + 55 = 83\)?

  Sure, but give me a moment.

  Do you know whether \dots

  No.

  Of course, might hold that the agent needs to have figured things out.
  But, then we have a plausible reduction.
  Knowing whether, and witnessing whether.
  Common component.

  Now, idea is a little different, as knowledge implies factivity.
  Interest with concluding is that not necessarily factive.
  From the agent's perspective.

  `Determining whether'.
  Or, rather `\fc{0}'.
\end{note}

\begin{note}
  Similar to Goldman, etc.?
  The idea is justification\dots
\end{note}

\begin{note}[Trimming]
  \begin{proposition}
    Basically, there's no role for anything beyond \(\Phi\) in the case of a foregone-conclusion.

    \(\Phi\), in the context of the agent's present epistemic state is sufficient to secure the conclusion, and the possibility of witnessing reasoning.
  \end{proposition}

  \begin{proposition}
    Foregone-conclusion just in case \(\Phi\) supports \(\pv{\phi}{v}\).
    \begin{argument}
      In short, given agent's present epistemic state, there's a guaranteed path from \(\Phi\) to \(\pv{\phi}{v}\).
    \end{argument}
    In other words, if \(\Phi\) does not support \(\pv{\phi}{v}\), then \(\pvp{\phi}{v}{\Phi}\) is not a foregone-conclusion.
  \end{proposition}

  Now, as the agent has not witnessed reasoning, need information that \(\pvp{\phi}{v}{\Phi}\) is a foregone-conclusion in order to recognise this.
  However, with information that \(\pvp{\phi}{v}{\Phi}\) is a foregone-conclusion, the information has no role in supporting \(\pv{\phi}{v}\).
\end{note}

\begin{note}
  So, that \(\pvp{\phi}{v}{\Phi}\) is a \fc{0} provides information, and explains, in part or whole, \emph{how} the agent concludes \(\pv{\phi}{v}\).
  However, \emph{why} is accounted for by \(\Phi\).
\end{note}

\begin{note}
  Why does it matter which?

  \begin{idea}[Reduction]
    If foregone-conclusion, then witness relation established by being a foregone-conclusion.
    Hence, reduction.
  \end{idea}
  Here, reduction.
  Restriction to `some'.
  It is not the case that every conclusion is a foregone-conclusion.

  However, if foregone-conclusions are of interest, then some motivation.

  Still, why of interest?
  What role do foregone-conclusion have?

  In particular.
  Need to get that some conclusion is a foregone-conclusion.
  From some reasoning.
  So, some pool of premises.
  And, that pool of premises is sufficient for conclusion.

  So, source for 2 + 2 = 4 and general ability.
  Already concluded that 2 + 2 = 4.
  Not quite, still going from general ability to 2 + 2 = 4.

  So, it's not clear that reduction is irrelevant.

  However, it is also not clear that this reduction is general.
  Want to show that foundation for reduction is not limited.
\end{note}

\begin{note}[Two worries]
  Two worries.

  First, that even though \fc{0}, the agent would not conclude.
  Either because \(\Phi\) is unavailable, or because no potential witnessing event.
  So, can't remove \fc{0} from account of why.

  However, then \fc{0} does not support.

  If grant that \fc{0} supports, then this seems to work out.
  Further, if require existence, then things that support get very messy.
  Dopeganger cases.
  Reason is I saw A, but it wasn't A, appealing to something that doesn't exist.
  Various other cases like this.

  Difference.
  In these cases, have premise, thing is that the truth value is distinct.
  Here, possibly no premise.

  Well, this is different.
  However, I don't think this is sufficient to reject the idea.
  Just because this distinction doesn't arise in the case of witnessing doesn't really do much.

  Look, a `bad' premise offers no more support for the agent than no premise.

  Second, need \emph{that} \fc{0}.
  However, the point is that this is about the agent's present epistemic state.
  \emph{Without} \fc{0}, the agent would reason.
  This is just the key point reiterated.
  Know whether, \fc{0} just adds information about which.
\end{note}


\paragraph{Foregone-concluding}

\begin{note}[Foregone-concluding]
  Pair this with a key idea.

  \begin{restatable}[Foregone-concluding]{idea}{ideaForegoneCing}
    \label{idea:reassignment}
    If foregone-conclusion, then may conclude.
    %\vspace{-\baselineskip}
  \end{restatable}

  Cases where concluding by witnessing reduces to witnessing forgone conclusion.
  \emph{Concluding \(\pv{\psi}{v'}\) from \(\Psi\) is just witnessing foregone-conclusion.}
  So, reduction, in certain cases.
  Further, if forgone conclusion, then conclude.
  At least, in certain cases.
\end{note}

\begin{note}[???]
  Only argue for a positive resolution to~{\color{red} issue:Main} given~\autoref{idea:reassignment}.

  And, leave~\autoref{idea:reassignment} as an idea.
  Insight into adopting this idea, or something like this.
\end{note}

\begin{note}
  Positive resolution may read easier if something like `in committing'.
  Commit to location from map, sum from arithmetic.
  Indeed, perhaps intuitive sense is just commitment via witnessing reasoning.

  But, we then have a reduction.
  Question is, what work does commitment do, and what work does witnessing do?

  Still, why think this?
  Why not think that concluding leads to commitments.
  Independent consequence.
  In same way that knowledge as basic entails justification, same for commitments.
  Indeed, in same way that relevant justification may be distinctive in the case of knowledge, same for commitment with concluding.
\end{note}

\begin{note}
  Assume motivate.
  What exactly is concluding?
  This will be beyond the scope of this document.
  I hope motivating a difference in extension motivates further questions about what the relation is, and the importance of witnessing in certain cases.
  As we will see, making this argument is by no means straightforward.

  I think there needs to be some instance of witnessing --- concluding does not arise from nowhere.
  Still, if witnessing leads to additional conclusions, then what do we get from concluding?
  What we will get is a general closure condition.

  To do so we narrow things down a little.
  Focus on particular types of concluding, and when concluding is accompanied by an additional property.
  Still, if in these instances no witnessing, then not more generally.
  Additional property, concluding some proposition from some premises.

  Argument, won't directly rely on intuitions about whether agent has concluded.
\end{note}

%%% Local Variables:
%%% mode: latex
%%% TeX-master: "master"
%%% End:
