\chapter{\ESU{}, \gsi{}, and \aben{the}}
\label{sec:first-conditional}

\begin{note}[Summary]
  In this section we argue that \ESU{} constrains what an agent may claim support for when reasoning from general to specific ability.

  In short, \ESU{} rules out claiming support by \adB{}.
\end{note}

\begin{note}
  Two ways corresponding to two sides of (specific) ability
  First, with respect to \aben{the}: appealing to (specific) ability.
  Second, with respect to \gsi{}: establishing (specific) ability from (general) ability.
\end{note}

\begin{note}[Expand: \gsi{}]
  Start with \gsi{}.

  Agent is claiming support for specific ability.
  Hence, claiming support that there is a potential event in which they \emph{V} that \(\phi\).
  Expanding potential event, claiming support that sufficient resources are available.
  Note, the agent may not (merely) \emph{expect} that sufficient resources are available, as availability of resources is part of claim for potential event.
  Rather, the agent may expect that there are no defeaters to claim that resources are available.

  To illustrate.
  Suppose I claim support that I know the train will be late.
  It's not (merely) that I expect that the train will be late.
  In order to claim support, some considerations sufficient to establish that the train will be late, and that there are no defeaters for these considerations.
  Expect would be absence of materia that train is on time.
  But absence alone doesn't push either way.\nolinebreak
  \footnote{
    Absence may be materia, though.
    For example, at least five minutes before train will arrive there is a message broadcast at the station.
    We are at the station, and it is a three minutes before the train is scheduled to arrive.
  }

  Task is to account for why an agent may claim support for availability of sufficient resources.
  In rough outline, answer is simple.
  Claimed support for general ability.
  Specific ability to \emph{V} that \(\phi\) is an `instance' of the general ability.
  So, given context, general ability supplements sufficient additional premises and steps of reasoning.

  However, without witnessing specific ability, agent is not aware of which additional premises and steps of reasoning are used.
\end{note}

\begin{note}[Moving to incompatibility]
  Incompatibly with \ESU{} will be from common point of appeal to sufficient resources.
  To this we now turn.
\end{note}

\section{Constraints on reasoning with \gsi{} given \ESU{}}
\label{sec:incomp-wr-ura}

\begin{note}[Argument outline]
  \color{red}
  There are two issues.
  \begin{itemize}
  \item \ESU{} means that the agent may not `directly' establish the existence of a witnessing event.
    For, in order to do so, the agent would need to claim support that the conclusion follows from some collection of premises and steps of reasoning.
    However, as the agent does not witness, then this isn't possible given \ESU{}.
    \begin{itemize}
    \item The objection here is that the agent doesn't necessarily need to go directly.
      It's possible that the agent claim support for some property, hence gets specific ability, and then reasons that this means that there's a witnessing event.
      So, to the extent that \WR{} needs first the existence of a witnessing event, \ESU{} might be okay.
    \end{itemize}
  \item Second, the agent can't reason with the details of the witnessing event.
    This then blocks \WR{}, and does so conclusively.
    For, the agent is not permitted to appeal to a relation of support between premises and conclusion.
    The agent is only permitted to appeal to the existence of an event that would establish such a relation.
    So this is the main objection.
  \end{itemize}
\end{note}

Key proposition of this section.

\begin{note}[Proposition]
    \begin{restatable}[\ESU{} and \adB{}]{proposition}{propNoESUandADB}\label{mcA:WR-and-denied-claim}
    \emph{If} \ESU{} is true \emph{then} no claiming support by \adB{} with respect to derived \AR{} or \WR{}.
  \end{restatable}
\end{note}

\begin{note}
  Argument is straightforward.
  \adB{} then claiming support by property of witnessing event.
  But, agent has not used those things in reasoning.
\end{note}

\begin{note}
  So, the key thing with this proposition is that in cases where an agent reasons with \gsi{-}, the agent claims support for a property.
  Hence, if an agent reasons from specific ability via \aben{the}, then must be an instance of \AR{}.

  Now, \autoref{mcA:WR-and-denied-claim} doesn't say that \ESU{} and \WR{} are incompatible in general.
  We'll see this in the argument for~\autoref{mcA:WR-and-denied-claim}.

  The argument that \WR{} is an incorrect interpretation of (specific) abilities of the form \emph{S} has the ability to \emph{V} that \(\phi\) (for which \aben{the} entailment holds) has two components.
  First, difficulty establishing by \gsi{}.
  Second, rules out \aben{the}.

  Difficulties with \gsi{}.
  Result of claiming support by \gsi{} is that agent claims support for specific.
  And, given \WR{}, this involves claiming support that some sufficient collection of premises and steps of reasoning are available to the agent.
  Given \ESU{}, the agent is required to use these steps and premises in order to appeal.
  Therefore, \ESU{} requires a partial witnessing event.
  Partial only, as the depending on how premises and steps are understood, certain premises or steps may be reused, and a single use may be sufficient for \ESU{}.

  The issue is strengthened when turning to \aben{the}.
  For, the conclusion is that \(\phi\) is the case.
  And, a partial witnessing event does not establish that \(\phi\) is the case.
\end{note}

\section{Summary}
\label{sec:uRa-and-wr-summary}

\begin{note}[Table]
  \begin{figure}[H]
    \centering
    \saMtxRuledOutESU{}
    \repeatCaptionPrime{fig:saMtxRuledOut}{Distinction matrix}
  \end{figure}
\end{note}

\begin{note}[Summarising]
  ???
\end{note}

%%% Local Variables:
%%% mode: latex
%%% TeX-master: "master"
%%% End: