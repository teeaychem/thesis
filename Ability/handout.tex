\makeatletter
\renewcommand{\PackageInfo}[2]{}% Remove package information
\renewcommand{\@font@info}[1]{}% Remove font information
% \renewcommand{\@latex@info}[1]{}% Remove LaTeX information
\makeatother

\PassOptionsToPackage{unicode}{hyperref}

\documentclass[10pt]{article}
\usepackage[british]{babel}

\usepackage[margin=.75in]{geometry}

\usepackage{amsthm}         % (in part) For the defined environments
\usepackage{mathtools}      % Improves  on amsmaths/mtpro2
\usepackage{amssymb}

% % % My packages % % %
\usepackage{myNotation}
\usepackage{ThesisCustom}
\usepackage{CustomEnvs}
\usepackage{ThesisFig}
% % % % % % % % % % % %

% \usepackage{selnolig}% For suppressing certain typographic ligatures automatically
% % % % % % %
\usepackage{xfrac}
\usepackage{array}
\usepackage{arydshln}
\usepackage{multirow}

% % % The bibliography % % %
\usepackage[%
backend=biber,
style=authoryear-comp,
bibstyle=authoryear,
citestyle=authoryear-comp,
uniquename=false,
backref=false,
hyperref=true,
url=false,
isbn=false,
doi=false,
useprefix=true,
maxbibnames=99,
]{biblatex}
\DeclareFieldFormat{postnote}{#1}
\DeclareFieldFormat{multipostnote}{#1}
% \setlength\bibitemsep{1.5\itemsep}
\newcommand{\noopsort}[1]{}
\addbibresource{Ability.bib}
\DefineBibliographyExtras{british}{\def\finalandcomma{\addcomma}} % Enable Oxford Comma
% % % % % % % % % % % % % % %

\usepackage[inline]{enumitem}
\SetEnumitemValue{labelindent}{standard}{.25\parindent}
\setlist[itemize]{labelindent=standard} %, leftmargin=1.5em}
\setlist[enumerate]{labelindent=standard} %, leftmargin=1.5em}

\newlist{itenum}{enumerate}{1}
\setlist[itenum]{
  style=standard,
  font=\normalfont,
  labelwidth = \widthof{\emph{Then}:},
}

\usepackage[export]{adjustbox}
\usepackage{subcaption}
\usepackage{float}

% % % % % % % % % % % % TIKZ
\usepackage{tikz}
\usetikzlibrary{bending,arrows,calc,arrows.meta,patterns,fadings}
\usetikzlibrary{trees}
\usetikzlibrary{backgrounds, positioning, fit, backgrounds}
\usetikzlibrary{math}
\usetikzlibrary{tikzmark}
% % % % % % % % % % % %


\usepackage{graphicx} % for images (png/jpeg etc.)
\usepackage{caption} % for \caption* command

% % % % % % % % % % % % MY COMMANDS

\makeatletter
\renewcommand\paragraph{\@startsection{paragraph}{4}{\z@}%
  {-3.25ex\@plus -1ex \@minus -.2ex}%
  {1.5ex \@plus .2ex}%
  {\normalfont\normalsize\bfseries}}
\makeatother

\makeatletter
\renewcommand\subparagraph{\@startsection{subparagraph}{4}{\z@}%
  {-3.25ex\@plus -1ex \@minus -.2ex}%
  {1.5ex \@plus .2ex}%
  {\normalfont\normalsize\bfseries}}
\makeatother

\makeatletter
\newcommand\lLine{{\color{lightgray} \noindent\rule{\textwidth}{0.4pt}}}
\newcommand\sepLine{
  \vfill
  \par\noindent\rule{\textwidth}{0.4pt}%
  \vspace{-10pt}%
  \par\noindent\rule{\textwidth}{0.4pt}
  \vfill}
\makeatother



% % % Fonts% %
\usepackage[no-math]{fontspec}
\defaultfontfeatures{Ligatures={NoDiscretionary, NoHistoric, NoRequired, NoContextual}}
% \protrudechars=2 %
\adjustspacing=2 %
\newfontfeature{Microtype}{protrusion=default;expansion=default;}
\setmainfont[Microtype]{Libertinus Serif}
\setsansfont[Microtype,Scale=MatchLowercase]{Libertinus Sans}
\setmonofont{Libertinus Mono}

\usepackage[math-style=ISO]{unicode-math}
\setmathfont{Libertinus Math}

\usepackage{pifont}
\newcommand{\hand}{\ding{43}}

\usepackage[
breaklinks,
bookmarks=false,
hidelinks,
linkcolor=highlight,
citecolor=highlight,
% colorlinks=true,
colorlinks=false,
]{hyperref}

\usepackage{csquotes}

\addto\extrasbritish{
  \def\chapterautorefname{Chapter}
  \def\sectionautorefname{Section}
  \def\subsectionautorefname{Section}
  \def\subsubsectionautorefname{Section}
  \def\paragraphautorefname{Section}
  \def\subparagraphautorefname{Section}
  \def\AlgoLineautorefname{Line}
  \def\pageautorefname{Page}
  \def\footnoteautorefname{Footnote}
  \def\scenarioCounterautorefname{Scenario}
  \def\observationCounterautorefname{Observation}
  \def\specificationCounterautorefname{Specification}
  \def\illustrationCounterautorefname{Illustration}
  \def\sketchCounterautorefname{Sketch}
  \def\linkCounterautorefname{Link}
  \def\constraintCounterautorefname{Constraint}
  \def\questionCounterautorefname{Question}
  \def\assumptionCounterautorefname{Assumption}
  \def\defiCounterautorefname{Definition}
  \def\propCounterautorefname{Proposition}
  \def\ideaCounterautorefname{Idea}
  \def\condCounterautorefname{Condition}
  \def\intuitionCounterautorefname{Intuition}
  \def\notationCounterautorefname{Notation}
  \def\applicationCounterautorefname{Application}
}

\title{
  Foregone-conclusions \quad / \quad Handout
}
% \author{Ben Sparkes}
\date{ }

\begin{document}

\maketitle

\begin{thesis}[First pass]
  A constraint on answers to \emph{why} an \eiw{0} an agent concludes happens via \emph{how} the \eiw{0} an agent concludes happens does not hold.
\end{thesis}

\section{\eiw{3} an agent concludes}
\label{sec:overview}

\begin{note}
  \par\noindent\rule{\textwidth}{0.4pt}
  \begin{minipage}{.5\linewidth}
    \begin{rscenariox}{illu:gist:roots:F}{Quadratic roots --- Factors}%
      % \autoref{illu:gist:roots:F}
      An agent reasons as follows:
      %
      \begin{enumerate}[label=\arabic*., ref=\arabic*]
      \item
        \label{illu:gist:roots:F:eq}
        For some \(x \in \mathbb{R}\), \rootsConEq{}.
      \item
        \label{illu:gist:roots:F:factor}
        \rootsConEqFac{}.
      \item
        \label{illu:gist:roots:F:zero}
        Either \(\rootsConEqFacL{} = 0\) or \(\rootsConEqFacR{} = 0\).
      \item
        \label{illu:gist:roots:F:case:a}
        If \(\rootsConEqFacL{} = 0\), then \(x = \rootsConEqFacLx{}\).
      \item
        \label{illu:gist:roots:F:case:b}
        If \(\rootsConEqFacR{} = 0\), then \(x = \rootsConEqFacRx{}\).
      \item
        \label{illu:gist:roots:F:factor:done}
        Either \(x = \rootsConEqFacLx{}\) or \(x = \rootsConEqFacRx{}\).
      \end{enumerate}
      %
      The agent concludes:
      \rootsCon{}.
    \end{rscenariox}
  \end{minipage}
  \begin{minipage}{.5\linewidth}
    \begin{rscenariox}{illu:gist:roots:QF}{Quadratic roots --- Formula}%
      % \autoref{illu:gist:roots:QF}
      An agent reasons as follows:
      %
      \begin{enumerate}[label=\arabic*., ref=\arabic*]
      \item
        \label{illu:gist:roots:QF:eq}
        For some \(x \in \mathbb{R}\), \rootsConEq{}.
      \item
        \label{illu:gist:roots:QF:qf}
        The quadratic formula states \(x = \rootsQEq{}\).
      \item
        \label{illu:gist:roots:QF:subs}
        Set \(a \coloneq \rootsQa{}\), \(b \coloneq \rootsQb{}\), and \(c \coloneq \rootsQc{}\).%
        %\footnotemark
      \item
        \label{illu:gist:roots:QF:qf-subs}
        \(x = \rootsQEqFill{\rootsQa{}}{\rootsQb{}}{\rootsQc{}}\).
      \item
        \label{illu:gist:roots:QF:qf:1}
        \(x = \sfrac{(- 1 \pm 5)}{12}\).
      \item
        \label{illu:gist:roots:QF:qf:done}
        Either \(x = \rootsConEqFacLx{}\) or \(x = \rootsConEqFacRx{}\).
      \end{enumerate}
      %
      The agent concludes:
      \rootsCon{}.
    \end{rscenariox}
  \end{minipage}
  \par\noindent\rule{\textwidth}{0.4pt}
  % \footnotetext{
  %   \(a\) is the coefficient of the \(x^{2}\) term, \(b\) is the coefficient of the \(x\) term, and \(c\) is the constant.
  % }
\end{note}

% \begin{note}
  % Likewise:

  % \begin{itemize}
  % \item
  %   In \autoref{illu:gist:roots:F} the agent does not conclude \propI{\rootsCon{}} from their understanding of the quadratic formula.
  % \item
  %   In \autoref{illu:gist:roots:QF} the agent does not conclude \propI{\rootsCon{}} from their understanding of factorisation.
  % \end{itemize}
% \end{note}


\subsection{Abstractions}


\begin{note}
  % Things:

  \begin{itemize}
  \item
    \underline{\textbf{\prop{3}}}: Some way things may be.
    \begin{itemize}
    \item
      \propI{The soup is sweet}, \propI{Foxes like pineapples}, \propI{The time is 17:30}, \dots \hfill \(\phi, \psi, \dots\)
    \end{itemize}
  \item
    \underline{\textbf{\val{3}}}: Agent's perspectives on the way some things are.
    \begin{itemize}
    \item
      \valI{True}, \valI{False}, \valI{Desirable}, \valI{Probable}, \dots \hfill \(v, v', \dots\)
    \end{itemize}
  \item
    \underline{\textbf{\pool{3}}}: Collections of \prop{1} paired with \val{1}.
    \begin{itemize}
    \item
      A \pool{} associated with an agent's understanding of factorisation, chess, sudoku puzzles, \dots \hfill \(\Phi, \Psi, \dots\)
    \end{itemize}
  \end{itemize}

  % Think of this as a more flexible approach to propositional attitudes.
  % Instead of agent evaluates \prop{} \(\phi\) to have \val{} \(v\), say agent from attitude \(A\) toward proposition \(\phi\).
  % However, speaking in terms of \prop{1} and \val{1} helps highlight that the argument does not rest on any particular understanding of propositions or attitudes.

  % \noindent
  % Relations:
  \lLine

  \begin{itemize}
  \item
    An agent concludes a \prop{0} \(\phi\) has some \val{0} \(v\) \underline{\textbf{from}} some \pool{} \(\Phi\).
    \begin{itemize}
    \item
      In \autoref{illu:gist:roots:F} the agent concludes \propI{\rootsCon{}} has \val{} \valI{True} from a \pool{} which includes \propM{\rootsConEq{}} having \val{0} \valI{True} and \prop{0}-\val{0} pairs associated with the \agents{} understanding of factorisation.
    \end{itemize}
  \item
    From an \agpe{}, a \prop{0} \(\psi\) having some \val{0} \(v'\) \underline{\textbf{\fof{}}} some \pool{} \(\Psi\).
    \begin{itemize}
    \item
      In \autoref{illu:gist:roots:QF}, at Step 3, \propI{\rootsCon{}} having \val{} \valI{True} \fof{} a \pool{} which includes \propM{\rootsConEq{}} having \val{0} \valI{True} and \prop{0}-\val{0} pairs associated with the \agents{} understanding of the quadratic formula.
    \end{itemize}
    \begin{itemize}
    \item[\hand]
      A \fof{} relation may hold between arbitrary \prop{0}-\val{0}-\pool{0} pairings and does not require a from relation.
      \begin{itemize}
      \item[\(\leadsto\)]
        Maybe think propositional justification, but only descriptive.
      \end{itemize}
    \end{itemize}
  \end{itemize}

  % If you like, think of this as a relation of support.
  % However, \fofr{1} are understood descriptively, i.e.\ there's no normative constraints on \fofr{1} here.
  % If you'd like additional motivation for this sort of thing, consider propositional justification.
\end{note}

\section{Two questions and a constraint \hfill (\qWhy{}, \qHow{}, and \issueInclusion{})}
\label{sec:target}

\vfill

\begin{note}
  \begin{question}{questionWhy}{\qWhy{}}
    Given \(e\) is an \eiw{0} \vAgent{} concludes \prop{} \(\phi\) has \val{} \(v\) from \pool{} \(\Phi\):
    \begin{itemize}
    \item
      Which \fofr{1} are included in (partial) explanations of \emph{why} \(e\) is an \eiw{0} \vAgent{} concludes \prop{} \(\phi\) has \val{} \(v\) from \pool{} \(\Phi\) (rather than an \eiw{0} some other thing happens)?
    \end{itemize}
    \vspace{-1\baselineskip}
  \end{question}
\end{note}


\lLine


\begin{note}
  \begin{itemize}
  \item
    The sense of `why' in \qWhy{} asks what (if anything) favoured \(e\) being an \eiw{} \(A\) happened, rather than \(e\) being an event in which \(B\) happened.
    \begin{itemize}
    \item
      Cards have been shuffled and \mainCard{} is \mainCardPos{}.
      \begin{itemize}
      \item
        If random shuffle, no answer to why \mainCard{} is \mainCardPos{}. \hfill (Any card may have been there.)
      \item
        If sleight of hand, answer to why \mainCard{} is \mainCardPos{}. \hfill (The result was forced.)
      \end{itemize}
    \end{itemize}
  \end{itemize}

  \begin{itemize}
  \item[\hand]
    \qWhy{} asks about explanatory \fingfr{1}.
  \item
    \propM{\rootsCon{}} having \val{} \valI{True} \fof{} a \pool{} which contains \propM{\rootsConEq{}} having \val{} \valI{True} and various other \prop{0}-\val{0} pairs associated with the agent's understanding of factorisation.
  \end{itemize}

\end{note}

\sepLine

\begin{note}
  \begin{question}{questionHow}{\qHow{}}
    Given \(e\) is an \eiw{0} \vAgent{} concludes \prop{} \(\phi\) has \val{} \(v\) from \pool{} \(\Phi\):
    \begin{itemize}
    \item
      Which past or present events (partially) explain \emph{how} \(e\) is an \eiw{0} \vAgent{} concludes \prop{} \(\phi\) has \val{} \(v\) from \pool{} \(\Phi\) (rather than an \eiw{0} some other thing happens)?
    \end{itemize}
    \vspace{-1.5\baselineskip}
  \end{question}
\end{note}

\lLine

\begin{note}
  \begin{itemize}
  \item[\hand]
    \qHow{} asks about what happened.
  \end{itemize}

  \begin{itemize}
  \item
    The sense of `how' in \qHow{} asks what (if anything) resulted in \(e\) being an \eiw{} \(A\) happened, rather than \(e\) being an event in which \(B\) happened.
  \end{itemize}
\end{note}

\sepLine

\begin{note}
  \begin{constraint}{consInclusion}{\issueInclusion{}}
    Given \(e\) is an \eiw{0} \vAgent{} concludes \prop{} \(\phi\) has \val{} \(v\) from \pool{} \(\Phi\):
    \begin{itenum}
    \item[\emph{If}:]
      \prop{2} \(\psi\) having \val{} \(v\) \fingf{} \pool{} \(\Psi\) answers \qWhy{}.
    \item[\emph{Then}:]
      A prior or present \eiw{0} the agent concludes \(\psi\) has \val{} \(v\) from \pool{} \(\Psi\) answers \qHow{}.
    \end{itenum}
    \vspace{-\baselineskip}
  \end{constraint}
\end{note}

\lLine

\begin{note}
  \begin{itemize}
  \item
    I think this is (maybe) intuitive.
    \begin{itemize}
    \item
      Sleight of hand happened, the agent concluded via factorisation/the quadratic formula.
    \end{itemize}
  \item
    \citeauthor{Davidson:1963aa} opens \citetitle{Davidson:1963aa} with the following question:

    \begin{quote}
      What is the relation between a reason and an action when the reason explains the action by giving the agent's reason for doing what he did?
      We may call such explanations \emph{rationalizations}, and say that the reason rationalizes the action.%
      \mbox{ }\hfill\mbox{(\citeyear[685]{Davidson:1963aa})}
    \end{quote}

    Paraphrased:
    %
    \begin{quote}
      A rationalisation is an explanation of why an agent did action \(a\) by giving the agent's reason for doing \(a\).
    \end{quote}
    %
    \citeauthor{Davidson:1963aa} argues rationalisations are causal explanations.
    So, this is plausibly an instance of an answer to a `why' question being constrained by answers to a `how' question.
    \begin{itemize}
    \item
      Though, rationalisations are different from \fofr{1} and there's no mapping between the two.
    \end{itemize}
  \end{itemize}
\end{note}

\sepLine

\begin{note}
  \begin{thesis}[Second pass]
    \issueInclusion{} does not hold.
  \end{thesis}
\end{note}

\vfill


\newpage

\section{Answers to \qWhy{} via \progEx{0}}
\label{sec:answers-qwhy}

\begin{note}
  Basic idea: Obtain answers to \qWhy{} by considering conclusions in progress.
\end{note}

\subsection{Events in which an event in progress}
\label{sec:events-progress}

\begin{note}
  \begin{quote}
    Events \(\ed{\flat}\) such that some event \(\edn{}\) which satisfies description \(\edo{}\) is in progress.
  \end{quote}
\end{note}

\begin{note}
  Roughly, think of events in progress in terms of the (English) progressive aspect:\footnote{
    This is a common analysis of the progressive.
    See, e.g., \cite{Bennett:1972uw,Dowty:1979vq,Parsons:1990aa,Landman:1992wh,Portner:1998um}.
  }
  %
  \begin{enumerate}
  \item
    The agent is making soup.%\newline
    \mbox{ } \hfill \(\leadsto\) An \eiw{0} the agent makes soup is in progress.\newline
    \mbox{ }\hfill \emph{{\color{gray} Some event \(\edn{}\) which satisfies description} {\color{darkgray} `The agent makes soup'} {\color{gray} is in progress}}.
  \item
    The agent is concluding \rootsCon{}.%\newline
    \mbox{ } \hfill \(\leadsto\) An \eiw{0} the agent concludes \rootsCon{} is in progress.\newline
    \mbox{ }\hfill \emph{{\color{gray} Some event \(\edn{}\) which satisfies description} {\color{darkgray} `The agent concludes \rootsCon{}' } {\color{gray} is in progress}.}
  \item
    The agent is passing the exam.%\newline
    \mbox{ } \hfill \(\leadsto\) An \eiw{0} the agent passes the exam is in progress.\newline
    \mbox{ }\hfill \emph{{\color{gray} Some event \(\edn{}\) which satisfies description} {\color{darkgray} `The agent passes the exam'} {\color{gray} is in progress}.}
  \end{enumerate}
  %
  This doesn't quite work, as the progressive has some quirks.
  Still, it's close enough to fix a grasp on events in progress.
\end{note}

\sepLine

\begin{note}
  \begin{observation}[Favours I]
    \label{obs:favours}
    \vspace{-\baselineskip}
    \begin{itenum}
    \item[\emph{If}:]
      \(\ed{\flat}\) is such that some event \(\edn{}\) under description \(\edo{}\) is in progress.
    \item[\emph{Then}:]
      \(\ed{}\) is favoured over any event which in incompatible with \(\ed{}\) happening.
    \end{itenum}
    \vspace{-\baselineskip}
  \end{observation}

  \begin{motivation}{obs:favours}
    \vspace{-\baselineskip}
    \begin{enumerate}[label=\Alph*.]
    \item
      If concluding by factorisation, then not concluding by the quadratic equation.
    \item
      If rolling a biased die, then not leaving things up to chance.
    \item
    If team A is winning, then not team B is winning.
  \end{enumerate}
  \vspace{-\baselineskip}
  \end{motivation}
\end{note}

\lLine

\begin{note}
  Shifting perspective on \autoref{obs:favours}:

  \begin{observation}[Favours II]
    \label{obs:favoursII}
    \vspace{-\baselineskip}
    \begin{itenum}
    \item[\emph{If}:]
      \(\ed{\flat}\) is such that \(\ed{}\) is in progress.
    \item[\emph{And}:]
      \(\ed{}\) happened.
    \item[\emph{Then}:]
      The details of \(\edn{\flat}\) captured by \(\edo{\flat}\) favoured \(\ed{}\) happening over any other event.
    \end{itenum}
    \vspace{-\baselineskip}
  \end{observation}

  \begin{itemize}
  \item[\hand]
    Details of events in progress are answers to why the event happened, given the sense of `why' present in \qWhy{}.
  \end{itemize}
\end{note}

\sepLine

\begin{note}
  Small issue:

  \begin{observation}[Squandered progress]%
    \label{obs:se-need-hCon}%
    It may be the case that:
    \begin{center}
      \begin{itemize*}
      \item
        \(\ed{\flat}\) is such that some event \(\ed[ ]{+}\) is in progress.
      \item
        An event \(\ed[ ]{\times}\) happens.
      \item
        \(\ed[ ]{\times}\) does not (partly) happen as a result of \(\ed{\flat}\).
      \end{itemize*}
    \end{center}
    \vspace{-\baselineskip}
  \end{observation}

  \begin{motivation}{obs:se-need-hCon}
    \vspace{-\baselineskip}
    \begin{enumerate}[label=\Alph*.]
    \item
      Agent Kujan is learning Roger Kint is Keyser S\"{o}ze, but is spoiled.
    \item
      Agent is passing an exam, and without external help, but answers are passed to them.
    \item
      Team A is winning the game due to playing well, but Team A wins the game by default as Team B forfeits due to injury.
    \end{enumerate}
    \vspace{-.5\baselineskip}
  \end{motivation}
\end{note}

\vfill

\newpage

\subsection{\se{3}, \progEx{0}, and \fingfr{1}}

\begin{note}
  \begin{rdefinition}{def:se}{\se{3}}
    \vspace{-\baselineskip}
    \begin{itemize}
    \item
      \(\ed{\flat}\) is a \emph{\se{0}} of \(\ed[ ]{\times}\).
    \end{itemize}
    \emph{If and only if}:
    \begin{itemize}
    \item
      Clauses~\ref{assu:p:se:prog} and \ref{assu:p:se:hCon} hold:
      \begin{enumerate}[label=\Alph*., ref=\Alph*]
      \item
        \label{assu:p:se:prog}
        \(\ed{\flat}\) is such that some event \(\edn{+}\) under description \(\edo{}\) is in progress.
      \item
        \label{assu:p:se:hCon}
        \(\ed[ ]{\times}\) partly happens as a result of \(\ed{\flat}\).
      \end{enumerate}
    \end{itemize}
    \vspace{-\baselineskip}
  \end{rdefinition}

  \lLine

  \begin{itemize}
  \item
    Clause~\ref{assu:p:se:prog} `looks forward':

    The description \(\edo{\flat}\) of \(\ed{\flat}\) captures that an event described by \(\edo{}\) is in progress.\newline
    \mbox{}\hfill\leadsto As Clause~\ref{assu:p:se:prog} concerns \(\ed{\flat}\), no specific event which satisfies \(\edo{}\) is fixed.
  \item
    Clause~\ref{assu:p:se:hCon} `looks backward':

    \(\ed{\times}\) as described by \(\edo{}\) happened in part as a result of \(\ed{\flat}\) as described by \(\edo{\flat}\).
  \end{itemize}
\end{note}

\sepLine

\begin{note}
  \begin{idea}[\progEx{2}]%
    \label{prop:PEbasic}
    Given \(\ed{}\) is an \eiw{0} \vAgent{} does \(a\), and the sense of `why' present in \qWhy{}:

    \begin{itenum}
    \item[\emph{If}:]
      \(\ed{\flat}\) is a \se{} of \(\ed{}\).
    \item[\emph{Then:}]
      \(\edo{\flat}\) being true of \(\ed{\flat}\) explains `why' \(\ed{}\) happened.
    \end{itenum}
    \vspace{-\baselineskip}
  \end{idea}
\end{note}

\lLine

\begin{note}
  There's some more to this, but in short:

  \begin{itemize}
  \item
    Clause~\ref{assu:p:se:prog} allows us to apply observations \ref{obs:favours} and \ref{obs:favoursII}.
  \item
    Clause~\ref{assu:p:se:hCon} allows us to avoid Observation \ref{obs:se-need-hCon}.
  \end{itemize}
\end{note}


\sepLine

\begin{note}
  \begin{idea}[Explanatory \fofr{1}]
    \vspace{-\baselineskip}
    \begin{itenum}
    \item[\emph{If}:]
      \(\ed{\flat}\) is an \eiw{} an agent concluding \prop{} \(\phi\) has \val{} \(v\) from \pool{} \(\Phi\).
    \item[\emph{Then}:]
      For the agent, \(\phi\) having \val{} \(v\) \fof{} \pool{} \(\Phi\) through \(\ed{\flat}\).
    \end{itenum}
    \vspace{-\baselineskip}
  \end{idea}
\end{note}

\lLine

\begin{note}
  \begin{itemize}
  \item
    Consider the contrapositive.

    If, for the agent, \(\phi\) having \val{} \(v\) does not \fof{} \pool{} \(\Phi\), then there is nothing which ensures the agent concluding \(\phi\) has \val{} \(v\) from \pool{} \(\Phi\).
  \end{itemize}

  \begin{itemize}
  \item[\hand]
    So, this is the way we obtain \fofr{1} which answer \qWhy{}.
  \item
    In the thesis, this idea is tightened up significantly and is where `\fc{1}' have a key role.
  \end{itemize}
\end{note}


\vfill

\newpage

\section{Failure}

% \sepLine

\begin{note}
  \begin{idea}[Sometimes an agent's reasoning is of a particular type, and so it must be that other \fofr{1} hold]
    \emph{Sometimes}:
    \begin{itenum}
    \item[\emph{If}:]
      \(\ed{\flat}\) is an \eiw{} an agent concluding \prop{} \(\phi\) has \val{} \(v\) from \pool{} \(\Phi\).
    \item[\emph{Then}:]
      For the agent, \(\psi\) having \val{} \(v'\) \fof{} \pool{} \(\Psi\) through \(\ed{\flat}\).
    \end{itenum}
    \vspace{-\baselineskip}
  \end{idea}
\end{note}

\lLine

\begin{note}
    \begin{scenario}[Lucas numbers]%
    \label{scen:LucasNums}%
    The Lucas numbers are recursively defined by: \qquad
    \(
      L_{n} = \left\{
        \begin{array}{ll}
          2 & \text{if } n = 0 \\
          1 & \text{if } n = 1 \\
          L_{n-1} + L_{n-2} & \text{if } n > 1 \\
        \end{array}
      \right.
    \)

    \smallskip
    \noindent An agent sets out to calculate the some of the Lucas numbers by the recursive definition.
    %
    \begin{center}
      \(
      \begin{array}{cccccc}
        L_{0} & L_{1} & L_{2} & L_{3} & L_{4} & \cdots \\
        \hline
        2 & 1 & 3 & 4 & 7 & \cdots \\
      \end{array}
      \)
    \end{center}
    \vspace{-2\baselineskip}
  \end{scenario}

  \begin{itemize}
  \item[\hand]
    For negative cases which appeal to this idea, consider:
    Selection tasks (\cite{Wason:1966aa}) and various observations regarding Decision theory (\cite{Allais:1979aa}, \cite{Ellsberg:1961aa}, \cite{Quinn:1990aa}) and logical principles (\cite{Makinson:1965aa}, \cite{Kyburg:1997aa}).\newline
    \mbox{}\hfill \leadsto As subjects fail at selection tasks, they do not reason about natural language conditionals as material conditionals.
  \end{itemize}
\end{note}

\sepLine

\begin{note}
  \begin{idea}[Oops]
    \emph{Sometimes}:
    \begin{itemize}
    \item
      \(\ed{\flat}\) is an \eiw{} an agent concluding \prop{} \(\phi\) has \val{} \(v\) from \pool{} \(\Phi\).
    \item
      \begin{itenum}
      \item[\emph{If}:]
        \(\ed{\flat}\) is an \eiw{} an agent concluding \prop{} \(\phi\) has \val{} \(v\) from \pool{} \(\Phi\).
      \item[\emph{Then}:]
        For the agent, \(\psi\) having \val{} \(v'\) \fof{} \pool{} \(\Psi\) through \(\ed{\flat}\).
      \end{itenum}
    \item
      The agent has not concluded \(\psi\) has \val{} \(v'\) from \pool{} \(\Psi\).
    \end{itemize}
    \vspace{-\baselineskip}
  \end{idea}
\end{note}

\lLine

\begin{note}
    \begin{scenario}[Sudoku puzzles]%
    \label{illu:gist:sudoku}%
    % https://tex.stackexchange.com/questions/91422/tikz-sudoku-circle-and-connect-with-lines-some-cells
    An agent (you) who has an understanding of Sudoku solves \sudokuPuzL{} (though does not work on \sudokuPuzR{}).%
    \footnote{
      The puzzles are from (\cite[54,56]{Coussement:2007up}).
    }
    \bigskip

    \begin{figure}[h!]
      \centering
      \begin{subfigure}{.45\linewidth}
        \centering
        \begin{tikzpicture}[scale=.5]
          \begin{scope}
            \draw (0, 0) grid (9, 9);
            \draw[very thick, scale=3] (0, 0) grid (3, 3);
            \setcounter{row}{1}
            \setrow { }{ }{ }  { }{ }{ }  { }{ }{ }
            \setrow {5}{ }{ }  { }{ }{ }  {3}{2}{ }
            \setrow { }{ }{ }  {4}{7}{9}  {6}{8}{ }
            \setrow {7}{ }{3}  { }{ }{4}  { }{ }{ }
            \setrow {4}{ }{5}  { }{ }{2}  { }{6}{9}
            \setrow {9}{2}{8}  {7}{5}{6}  {4}{ }{1}
            \setrow {3}{5}{ }  { }{2}{ }  { }{9}{ }
            \setrow { }{ }{1}  { }{9}{3}  {5}{ }{ }
            \setrow { }{ }{ }  { }{ }{8}  {1}{7}{ }
          \end{scope}
        \end{tikzpicture}
        \caption*{\sudokuPuzL{}}
      \end{subfigure}
      \begin{subfigure}{.45\linewidth}
        \centering
        \begin{tikzpicture}[scale=.5]
          \begin{scope}
            \draw (0, 0) grid (9, 9);
            \draw[very thick, scale=3] (0, 0) grid (3, 3);
            \setcounter{row}{1}
            % Single entries
            \setrow {4}{6}{3}  {7}{ }{8}  { }{ }{ }
            \setrow {2}{1}{ }  { }{ }{ }  { }{ }{7}
            \setrow { }{7}{5}  {4}{ }{ }  { }{3}{ }
            \setrow { }{2}{ }  {8}{7}{ }  { }{ }{3}
            \setrow {3}{9}{1}  { }{ }{2}  {5}{ }{8}
            \setrow { }{ }{6}  {1}{3}{ }  { }{ }{2}
            \setrow { }{3}{8}  { }{1}{6}  {7}{5}{4}
            \setrow { }{ }{ }  {9}{ }{ }  { }{ }{1}
            \setrow { }{ }{ }  { }{ }{ }  {2}{ }{ }
          \end{scope}
        \end{tikzpicture}
        \caption*{\sudokuPuzR{}}
      \end{subfigure}
    \end{figure}
    \vspace{-\baselineskip}
  \end{scenario}
\end{note}

\vfill

\end{document}

\begin{note}
  Well, a more general perspective on Wason, etc.\ is in terms of mistakes.
  Intuitively, it's only the case that an agent is making a mistake, relative to their perspective, if in typical circumstances the agent does things differently.
  In parallel, the agent is only doing things correctly, if in typical circumstances the agent does things the same.

  If there are no typical circumstances, then there's no generality to the agent's reasoning, and therefore nothing which establishes whether the agent is mistakenly or correctly doing things, from their perspective.
  The only other option is some independent measure, but an independent measure is only good for evaluating what the agent has done, not accounting for why the agent did that thing.

  In short, when we describe an agent reasoning, we have a type of reasoning in mind.
  And, this implies the agent may reason in other ways.
  In turn, this implies various following from relations hold.
\end{note}

\section{Extras}

\subsection*{\scen{3}}
\label{sec:scen3}

\begin{note}
  \begin{scenario}[Paltry]
    \label{scen:fc:chick}%
    An agent reads Puzzle~25 of \citeauthor{Dudeney:1995aa}'s \citetitle{Dudeney:1995aa}:
    %
    \begin{quote}
      Three chickens and one duck sold for as much as two geese; one chicken, two ducks, and three geese were sold together for \$25.00. What was the price of each bird in an exact number of dollars?%
      \mbox{ }\hfill\mbox{(\citeyear[9]{Dudeney:1995aa})}
    \end{quote}
    \vspace{-\baselineskip}
  \end{scenario}
\end{note}

\begin{note}
  \begin{rscenario}{scen:countS}{Countersign}%
    \indent The captain mumbled, ``I come from Miran.''

    The man returned the gambit, grimly.
    ``Miran is early this year.''

    The captain said, ``No earlier than last year.''

    But the man did not step aside.
    He said, ``Who are you?''

    ``Aren't you Fox?''

    ``Do you always answer by asking?''

    The captain took an imperceptibly longer breath, and then said calmly,
    ``I am Han Pritcher, Captain of the Fleet, and member of the Democratic Underground Party.
    Will you let me in?''%
    \mbox{ }\hfill\mbox{(\cite[70]{Asimov:1945aa})}%
    \newline
  \end{rscenario}
\end{note}




%%% Local Variables:
%%% mode: latex
%%% TeX-master: t
%%% TeX-engine: luatex
%%% TeX-master: "handout"
%%% End:

