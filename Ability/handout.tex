\makeatletter
\renewcommand{\PackageInfo}[2]{}% Remove package information
\renewcommand{\@font@info}[1]{}% Remove font information
% \renewcommand{\@latex@info}[1]{}% Remove LaTeX information
\makeatother

\PassOptionsToPackage{unicode}{hyperref}

\documentclass[10pt]{article}
\usepackage[british]{babel}

\usepackage[margin=.75in]{geometry}

\usepackage{amsthm}         % (in part) For the defined environments
\usepackage{mathtools}      % Improves  on amsmaths/mtpro2
\usepackage{amssymb}

% % % My packages % % %
\usepackage{myNotation}
\usepackage{ThesisCustom}
\usepackage{CustomEnvs}
\usepackage{ThesisFig}
% % % % % % % % % % % %

% \usepackage{selnolig}% For suppressing certain typographic ligatures automatically
% % % % % % %
\usepackage{xfrac}
\usepackage{array}
\usepackage{arydshln}
\usepackage{multirow}

% % % The bibliography % % %
\usepackage[%
backend=biber,
style=authoryear-comp,
bibstyle=authoryear,
citestyle=authoryear-comp,
uniquename=false,
backref=false,
hyperref=true,
url=false,
isbn=false,
doi=false,
useprefix=true,
maxbibnames=99,
]{biblatex}
\DeclareFieldFormat{postnote}{#1}
\DeclareFieldFormat{multipostnote}{#1}
% \setlength\bibitemsep{1.5\itemsep}
\newcommand{\noopsort}[1]{}
\addbibresource{Ability.bib}
\DefineBibliographyExtras{british}{\def\finalandcomma{\addcomma}} % Enable Oxford Comma
% % % % % % % % % % % % % % %

\usepackage[inline]{enumitem}
\SetEnumitemValue{labelindent}{standard}{.25\parindent}
\setlist[itemize]{labelindent=standard} %, leftmargin=1.5em}
\setlist[enumerate]{labelindent=standard} %, leftmargin=1.5em}

\newlist{itenum}{enumerate}{1}
\setlist[itenum]{
  style=standard,
  font=\normalfont,
  labelwidth = \widthof{\emph{Then}:},
}

\usepackage[export]{adjustbox}
\usepackage{subcaption}
\usepackage{float}

% % % % % % % % % % % % TIKZ
\usepackage{tikz}
\usetikzlibrary{bending,arrows,calc,arrows.meta,patterns,fadings}
\usetikzlibrary{trees}
\usetikzlibrary{backgrounds, positioning, fit, backgrounds}
\usetikzlibrary{math}
\usetikzlibrary{tikzmark}
% % % % % % % % % % % %


\usepackage{graphicx} % for images (png/jpeg etc.)
\usepackage{caption} % for \caption* command

% % % % % % % % % % % % MY COMMANDS

\makeatletter
\renewcommand\paragraph{\@startsection{paragraph}{4}{\z@}%
  {-3.25ex\@plus -1ex \@minus -.2ex}%
  {1.5ex \@plus .2ex}%
  {\normalfont\normalsize\bfseries}}
\makeatother

\makeatletter
\renewcommand\subparagraph{\@startsection{subparagraph}{4}{\z@}%
  {-3.25ex\@plus -1ex \@minus -.2ex}%
  {1.5ex \@plus .2ex}%
  {\normalfont\normalsize\bfseries}}
\makeatother

\makeatletter
\newcommand\lLine{{\color{lightgray} \noindent\rule{\textwidth}{0.4pt}}}
\newcommand\sepLine{
  \vfill
  \par\noindent\rule{\textwidth}{0.4pt}
  \vfill}
\makeatother



% % % Fonts% %
\usepackage[no-math]{fontspec}
\defaultfontfeatures{Ligatures={NoDiscretionary, NoHistoric, NoRequired, NoContextual}}
% \protrudechars=2 %
\adjustspacing=2 %
\newfontfeature{Microtype}{protrusion=default;expansion=default;}
\setmainfont[Microtype]{Libertinus Serif}
\setsansfont[Microtype,Scale=MatchLowercase]{Libertinus Sans}
\setmonofont{Libertinus Mono}

\usepackage[math-style=ISO]{unicode-math}
\setmathfont{Libertinus Math}

\usepackage{pifont}
\newcommand{\hand}{\ding{43}}

\usepackage[
breaklinks,
bookmarks=false,
hidelinks,
linkcolor=highlight,
citecolor=highlight,
% colorlinks=true,
colorlinks=false,
]{hyperref}

\usepackage{csquotes}

\addto\extrasbritish{
  \def\chapterautorefname{Chapter}
  \def\sectionautorefname{Section}
  \def\subsectionautorefname{Section}
  \def\subsubsectionautorefname{Section}
  \def\paragraphautorefname{Section}
  \def\subparagraphautorefname{Section}
  \def\AlgoLineautorefname{Line}
  \def\pageautorefname{Page}
  \def\footnoteautorefname{Footnote}
  \def\scenarioCounterautorefname{Scenario}
  \def\observationCounterautorefname{Observation}
  \def\specificationCounterautorefname{Specification}
  \def\illustrationCounterautorefname{Illustration}
  \def\sketchCounterautorefname{Sketch}
  \def\linkCounterautorefname{Link}
  \def\constraintCounterautorefname{Constraint}
  \def\questionCounterautorefname{Question}
  \def\assumptionCounterautorefname{Assumption}
  \def\defiCounterautorefname{Definition}
  \def\propCounterautorefname{Proposition}
  \def\ideaCounterautorefname{Idea}
  \def\condCounterautorefname{Condition}
  \def\intuitionCounterautorefname{Intuition}
  \def\notationCounterautorefname{Notation}
  \def\applicationCounterautorefname{Application}
}

\title{
  Foregone-conclusions \quad / \quad Handout
}
% \author{Ben Sparkes}
\date{ }

\begin{document}

\maketitle

\begin{thesis}[First pass]
  A constraint on answers to \emph{why} an \eiw{0} an agent concludes happens via \emph{how} the \eiw{0} an agent concludes happens does not hold.
\end{thesis}

\section{\eiw{3} an agent concludes}
\label{sec:overview}

\begin{note}
  \par\noindent\rule{\textwidth}{0.4pt}
  \begin{minipage}{.5\linewidth}
    \begin{rscenariox}{illu:gist:roots:F}{Quadratic roots --- Factors}%
      % \autoref{illu:gist:roots:F}
      An agent reasons as follows:
      % 
      \begin{enumerate}[label=\arabic*., ref=\arabic*]
      \item
        \label{illu:gist:roots:F:eq}
        For some \(x \in \mathbb{R}\), \rootsConEq{}.
      \item
        \label{illu:gist:roots:F:factor}
        \rootsConEqFac{}.
      \item
        \label{illu:gist:roots:F:zero}
        Either \(\rootsConEqFacL{} = 0\) or \(\rootsConEqFacR{} = 0\).
      \item
        \label{illu:gist:roots:F:case:a}
        If \(\rootsConEqFacL{} = 0\), then \(x = \rootsConEqFacLx{}\).
      \item
        \label{illu:gist:roots:F:case:b}
        If \(\rootsConEqFacR{} = 0\), then \(x = \rootsConEqFacRx{}\).
      \item
        \label{illu:gist:roots:F:factor:done}
        Either \(x = \rootsConEqFacLx{}\) or \(x = \rootsConEqFacRx{}\).
      \end{enumerate}
      % 
      The agent concludes:
      \rootsCon{}.
    \end{rscenariox}
  \end{minipage}
  \begin{minipage}{.5\linewidth}
    \begin{rscenariox}{illu:gist:roots:QF}{Quadratic roots --- Formula}%
      % \autoref{illu:gist:roots:QF}
      An agent reasons as follows:
      % 
      \begin{enumerate}[label=\arabic*., ref=\arabic*]
      \item
        \label{illu:gist:roots:QF:eq}
        For some \(x \in \mathbb{R}\), \rootsConEq{}.
      \item
        \label{illu:gist:roots:QF:qf}
        The quadratic formula states \(x = \rootsQEq{}\).
      \item
        \label{illu:gist:roots:QF:subs}
        Set \(a \coloneq \rootsQa{}\), \(b \coloneq \rootsQb{}\), and \(c \coloneq \rootsQc{}\).%
        %\footnotemark
      \item
        \label{illu:gist:roots:QF:qf-subs}
        \(x = \rootsQEqFill{\rootsQa{}}{\rootsQb{}}{\rootsQc{}}\).
      \item
        \label{illu:gist:roots:QF:qf:1}
        \(x = \sfrac{(- 1 \pm 5)}{12}\).
      \item
        \label{illu:gist:roots:QF:qf:done}
        Either \(x = \rootsConEqFacLx{}\) or \(x = \rootsConEqFacRx{}\).
      \end{enumerate}
      % 
      The agent concludes:
      \rootsCon{}.
    \end{rscenariox}
  \end{minipage}
  \par\noindent\rule{\textwidth}{0.4pt}

  % \footnotetext{
  %   \(a\) is the coefficient of the \(x^{2}\) term, \(b\) is the coefficient of the \(x\) term, and \(c\) is the constant.
  % }
\end{note}

\begin{note}
  % Likewise:

  % \begin{itemize}
  % \item
  %   In \autoref{illu:gist:roots:F} the agent does not conclude \propI{\rootsCon{}} from their understanding of the quadratic formula.
  % \item
  %   In \autoref{illu:gist:roots:QF} the agent does not conclude \propI{\rootsCon{}} from their understanding of factorisation.
  % \end{itemize}
\end{note}


\subsection{Abstractions}


\begin{note}
  Things:

  \begin{itemize}
  \item
    \prop{3} \(\leadsto\) ways things may be.
    \begin{itemize}
    \item
      \propI{The soup is sweet}, \propI{Foxes like pineapples}, \propI{The time is 17:30}, and so on\dots
    \end{itemize}
  \item
    \val{3} \(\leadsto\) an \agpe{} on some way things may be.
    \begin{itemize}
    \item
      \valI{True}, \valI{False}, \valI{Desirable}, \valI{Probable}, and so on\dots
    \end{itemize}
  \item
    \pool{3} \(\leadsto\) collections of \prop{1} paired with \val{1}.
    \begin{itemize}
    \item
      A \pool{} associated with an agent's understanding of factorisation, chess, sudoku puzzles, and so on\dots
    \end{itemize}
  \end{itemize}

  % Think of this as a more flexible approach to propositional attitudes.

  \noindent
  Relations:

  \begin{itemize}
  \item
    An agent concludes a \prop{0} has some \val{0} \underline{\emph{from}} some \pool{}.
    \begin{itemize}
    \item
      In \autoref{illu:gist:roots:F} the agent concludes \propI{\rootsCon{}} has \val{} \valI{True} from a \pool{} which includes \propM{\rootsConEq{}} having \val{0} \valI{True} and \prop{0}-\val{0} pairs associated with the \agents{} understanding of factorisation.
    \end{itemize}
  \item
    A \prop{0} having some \val{0} \underline{\emph{\fof{}}} some \pool{} (from the \agpe{}, relative to an event).
    \begin{itemize}
    \item
      In \autoref{illu:gist:roots:QF}, at Step 3, \propI{\rootsCon{}} having \val{} \valI{True} followed from a \pool{} which includes \propM{\rootsConEq{}} having \val{0} \valI{True} and \prop{0}-\val{0} pairs associated with the \agents{} understanding of the quadratic formula.
    \end{itemize}
    \begin{itemize}
    \item[\hand]
      A \fof{} relation may hold between arbitrary \prop{0}-\val{0}-\pool{0} pairings (different from rationalisations).
    \end{itemize}
  \end{itemize}
  % If you like, think of this as a relation of support.
  % However, there's nothing normative here.
  % This is what distinguishes \fofr{} from rationalisations.
  % For the moment, intuitive.
  % Shortly give a pair of sufficient conditions for a \fofr{}.
\end{note}

% \newpage

\section{Two questions and a constraint \hfill (\qWhy{}, \qHow{}, and \issueInclusion{})}
\label{sec:target}

\vfill

\begin{note}
  \begin{question}{questionWhy}{\qWhy{}}
    Given \(e\) is an \eiw{0} \vAgent{} concludes \prop{} \(\phi\) has \val{} \(v\) from \pool{} \(\Phi\):
    \begin{itemize}
    \item
      Which \fofr{1} are included in (partial) explanations of \emph{why} \(e\) is an \eiw{0} \vAgent{} concludes \prop{} \(\phi\) has \val{} \(v\) from \pool{} \(\Phi\) (rather than an \eiw{0} some other thing happens)?
    \end{itemize}
    \vspace{-1\baselineskip}
  \end{question}
\end{note}


\lLine


\begin{note}
  \begin{itemize}
  \item[\hand]
    \qWhy{} asks about explanatory \fingfr{1}.
  \end{itemize}

  \begin{itemize}
  \item
    The sense of `why' in \qWhy{} asks what (if anything) favoured \(e\) being an \eiw{} \(A\) happened, rather than \(e\) being an event in which \(B\) happened.
    \begin{itemize}
    \item
      Shuffling cards.
      \begin{itemize}
      \item
        If random shuffle, no answer to why.
      \item
        If performing sleight of hand, answer to why for chosen card.
      \end{itemize}
    \item
      Biased die.
    \end{itemize}
  \end{itemize}
\end{note}

\sepLine

\begin{note}
  \begin{question}{questionHow}{\qHow{}}
    Given \(e\) is an \eiw{0} \vAgent{} concludes \prop{} \(\phi\) has \val{} \(v\) from \pool{} \(\Phi\):
    \begin{itemize}
    \item
      Which past or present events (partially) explain \emph{how} \(e\) is an \eiw{0} \vAgent{} concludes \prop{} \(\phi\) has \val{} \(v\) from \pool{} \(\Phi\) (rather than an \eiw{0} some other thing happens)?
    \end{itemize}
    \vspace{-1.5\baselineskip}
  \end{question}
\end{note}

\lLine

\begin{note}
  \begin{itemize}
  \item[\hand]
    \qHow{} asks about what happened.
  \end{itemize}

  \begin{itemize}
  \item
    The sense of `how' in \qHow{} asks what (if anything) resulted in \(e\) being an \eiw{} \(A\) happened, rather than \(e\) being an event in which \(B\) happened.
  \end{itemize}
\end{note}

\sepLine

\begin{note}
  \begin{constraint}{consInclusion}{\issueInclusion{}}
    \mbox{ }
    \vspace{-\baselineskip}
    \begin{itenum}
    \item[\emph{If}:]
      \fingfb{\(\pv{\psi}{v'}\)}{\(\Psi\)} answers \qWhy{}.
    \item[\emph{Then}:]
      An \eiw{0} the agent concludes \(\pv{\psi}{v'}\) from \(\Psi\) answers \qHow{}.
    \end{itenum}
    \vspace{-\baselineskip}
  \end{constraint}
\end{note}

\lLine

\begin{note}
  \begin{itemize}
  \item
    I think this is (maybe) intuitive.
    \begin{itemize}
    \item
      Sleight of hand happened, bias of a die took effect.
    \end{itemize}
  \item
    \citeauthor{Davidson:1963aa} opens \citetitle{Davidson:1963aa} with the following question:

    \begin{quote}
      What is the relation between a reason and an action when the reason explains the action by giving the agent's reason for doing what he did?
      We may call such explanations \emph{rationalizations}, and say that the reason rationalizes the action.%
      \mbox{ }\hfill\mbox{(\citeyear[685]{Davidson:1963aa})}
    \end{quote}

    Paraphrased:
    % 
    \begin{quote}
      A rationalisation is an explanation of why an agent did action \(a\) by giving the agent's reason for doing \(a\).
    \end{quote}
    % 
    \citeauthor{Davidson:1963aa} argues rationalisations are causal explanations.
    So, this is plausibly an instance of answer to a `why' question being constrained by answers to a `how' question.
  \item[\hand]
    Rationalisations are different from \fofr{1}!
  \end{itemize}
\end{note}

\sepLine

\begin{note}
  \begin{thesis}[Second pass]
    \issueInclusion{} does not hold.
  \end{thesis}
\end{note}

\vfill


\newpage

\section{Answers to \qWhy{} via \progEx{0}}
\label{sec:answers-qwhy}

\begin{note}
  Basic idea: Obtain answers to \qWhy{} by considering conclusions in progress.
\end{note}

\subsection{Events, in progress}
\label{sec:events-progress}

\begin{note}
  Some event \(\ed{\flat}\) such that some event \(\edn{}\) under description \(\edo{}\) is in progress.

  \begin{itemize}
  \item[\hand]
    Some event \(\edn{}\) which satisfies description \(\edo{}\) is in progress.
  \end{itemize}
\end{note}

\begin{note}
  Roughly, think of events in progress in terms of the (English) progressive aspect.
  %
  \begin{itemize}
  \item
    The agent is making soup.%\newline
    \mbox{ } \hfill \(\leadsto\) An \eiw{0} the agent makes soup is in progress.

    \mbox{ }\hfill \(\edo{}:\) `The agent makes soup.'
  \item
    The agent is reading Henley's `Invictus'.%\newline
    \mbox{ } \hfill \(\leadsto\) An \eiw{0} the agent reads Henley's `Invictus' is in progress.

    \mbox{ }\hfill \(\edo{}:\) `The agent reads Henley's `Invictus'.'
  \item
    The agent is passing the exam.%\newline
    \mbox{ } \hfill \(\leadsto\) An \eiw{0} the agent passes the exam is in progress.

    \mbox{ }\hfill \(\edo{}:\) `The agent passes the exam.'
  \end{itemize}
  %
  This isn't quite right, as the progressive has some quirks.
  Though, it's good enough.
\end{note}

\lLine

\begin{note}
  \begin{observation}[Favours]
    If \(\ed{\flat}\) is such that \(\ed{}\) is in progress, then \(\ed{}\) is favoured over any event which in incompatible with \(\ed{}\) happening.
  \end{observation}
\end{note}

\begin{note}
  \begin{itemize}
  \item
    If concluding by factorisation, then not concluding by the quadratic equation.
  \item
    If rolling a biased die, then not leaving things up to chance.
  \item
    If team A is winning, then not team B is winning.
  \end{itemize}
\end{note}

\lLine

\begin{note}
  \begin{observation}[Squandered progress]%
    \label{obs:se-need-hCon}%
    It may be the case that:
    \begin{center}
      \begin{itemize*}
      \item
        \(\ed{\flat}\) is such that some event \(\ed[ ]{+}\) is in progress.
      \item
        An event \(\ed[ ]{\times}\) happens.
      \item
        \(\ed[ ]{\times}\) does not partly happen as a result of \(\ed{\flat}\).
      \end{itemize*}
    \end{center}
    \vspace{-\baselineskip}
  \end{observation}

  \begin{itemize}
  \item
    Agent changes to quadratic formula.
  \item
    Forgets sleight of hand.
  \item
    Team B forfeits due to injury.
  \end{itemize}

  \begin{itemize}
  \item
    \(\ed{\flat}\): An agent is passing an exam without external help.
  \item
    \(\ed{}\): The agent passes the exam.
  \end{itemize}

  \(\ed{\flat}\) is such that an event \(\ed{}\) is in progress, but the agent may pass the exam as the result of answers passed to them.
\end{note}

\newpage

\subsection{\se{3}}

\begin{note}
  \begin{rdefinition}{def:se}{\se{3}}
    \vspace{-\baselineskip}
    \begin{itemize}
    \item
      \(\ed{\flat}\) is a \emph{\se{0}} of \(\ed{}\).
    \end{itemize}
    \emph{If and only if}:
    \begin{itemize}
    \item
      Clauses~\ref{assu:p:se:prog} and \ref{assu:p:se:hCon} hold:
      \begin{enumerate}[label=\Alph*., ref=\Alph*]
      \item
        \label{assu:p:se:prog}
        \(\ed{\flat}\) is such that \(\ed{}\) is in progress.
      \item
        \label{assu:p:se:hCon}
        \(\ed{}\) partly happens as a result of \(\ed{\flat}\).
      \end{enumerate}
    \end{itemize}
    \vspace{-\baselineskip}
  \end{rdefinition}

  \noindent%
  \(\ed{\flat}\) being a \emph{\se{0}} of \(\ed{}\) concerns specific events \(\ed{\flat}\) and \(\ed{}\).
  And, of interest is whether specific descriptions \(\edo{}\) and \(\edo{\flat}\) capture a particular connexion between \(\ed{\flat}\) and \(\ed{}\).

  Still, our interest is only with parts of \(\ed{\flat}\) and \(\ed{}\), respectively.
  Intuitively:
  %
  \begin{itemize}
  \item
    Clause~\ref{assu:p:se:prog} `looks forward':

    The description \(\edo{\flat}\) of \(\ed{\flat}\) captures that an event described by \(\edo{}\) is in progress.
  \item
    Clause~\ref{assu:p:se:hCon} `looks backward':

    \(\ed{}\) as described by \(\edo{}\) happened in part as a result of \(\ed{\flat}\) as described by \(\edo{\flat}\).
  \end{itemize}

  The role of Clause~\ref{assu:p:se:prog} is to ensure the event \(\ed{}\) is favoured over some other event.

  And, the role of Clause~\ref{assu:p:se:hCon} is to ensure \(\ed{}\) happens as a result of being favoured.
\end{note}


\begin{note}
  Consider \autoref{illu:gist:roots:F}.
  Agent's conclusion was in progress, but plausible order of steps 4 and 5 was not in progress.
  Same with Step 5 of \autoref{illu:gist:roots:QF}, and here unspecified --- add or subtract first.

  So distinct between these two things helps fix an event fairly easily, while narrowing down certain parts of the event.
\end{note}


\subsection{\progExI{}}
\label{sec:progex}


\begin{note}
  Idea is, over any other event.
  Event over any other event.
  So, events in progress.
  Some examples.
  A little more careful.
  Event in progress, an event of interest happens as a result.
  Given this, event in progress is a partial answer to why the event happened, given the sense of why outlined.
\end{note}


\begin{note}
    \begin{rproposition}{prop:PEbasic}{\progExI{}}%
    Given \(\ed{}\) is an \eiw{0} \vAgent{} does \(a\), and the sense of `why' present in \qWhy{}:

    \begin{itenum}
    \item[\emph{If}:]
      \(\ed{\flat}\) is a \se{} of \(\ed{}\).
    \item[\emph{Then:}]
      \(\edo{\flat}\) being true of \(\ed{\flat}\) explains `why' \(\ed{}\) happened.
    \end{itenum}
    \vspace{-\baselineskip}
  \end{rproposition}

  \noindent%
  Given \(\ed{\flat}\) is a \se{} of \(\ed{}\), \(\edo{\flat}\) being true of \(\ed{}\) explains `why' \(\ed{}\) happened in a basic sense, as by Clause~\ref{assu:p:se:hCon} \(\ed{}\) partly happened as a result of \(\ed{\flat}\).
  And, Clause~\ref{assu:p:se:prog} allows us to expand this observation to see \(\ed{}\) partly happened as a result of something which favoured \(\ed{}\) happening over any (incompatible) event.
\end{note}

\paragraph{Example}

\begin{note}
  Step 2 of the scenarios.
\end{note}

\subsubsection{\progEx{2}, \fingfr{1}, and \fc{1}}
\label{sec:progex-fingfr1}

\begin{note}
  So, question is whether \eiw{} conclusion is in progress entails a \fingfr{}.
  I think this is the case.

  In short, often, to describe event in progress, generality.

  Entails various other conclusions are \fc{1}.
\end{note}

\begin{note}
  Everything is worked through in a fairly general way, and relevant ideas, definitions, and so on are applied to \scen{1} to demonstrate the way \issueInclusion{} fails.

  And, the thesis closes with the reader themselves creating a \scen{0} for which \issueInclusion{} fails.
\end{note}

\newpage

\section{Extras}

\subsection*{Rationalisations}
\label{sec:rationalisations}

\begin{note}
  

  \noindent%
\end{note}

\begin{note}
  Given a \eiw{} an agent concludes some \prop{0} \(\phi\) has some \val{0} \(v\) from some \pool{} \(\Phi\), a rationalisation may explain why the agent \emph{concluded} \(\phi\) has \val{0} from \(\Phi\).
  And, the various \prop{0}-\val{0} pairs in the \pool{} may be (part of) the \agents{} reason.

  Still, there is no (clear) way to capture some (maybe distinct) \prop{0}-\val{0} pair \fof{} some (maybe distinct) \pool{} via a rationalisation.
\end{note}

\subsection*{\scen{3}}
\label{sec:scen3}

\begin{note}
  \begin{rscenario}{scen:countS}{Countersign}%
    \indent The captain mumbled, ``I come from Miran.''

    The man returned the gambit, grimly.
    ``Miran is early this year.''

    The captain said, ``No earlier than last year.''

    But the man did not step aside.
    He said, ``Who are you?''

    ``Aren't you Fox?''

    ``Do you always answer by asking?''

    The captain took an imperceptibly longer breath, and then said calmly,
    ``I am Han Pritcher, Captain of the Fleet, and member of the Democratic Underground Party.
    Will you let me in?''%
    \mbox{ }\hfill\mbox{(\cite[70]{Asimov:1945aa})}%
    \newline
  \end{rscenario}
\end{note}

\begin{note}
    \begin{scenario}[Lucas numbers]%
    \label{scen:LucasNums}%
    The Lucas numbers are recursively defined as follows:%
    \[
      L_{n} = \left\{
        \begin{array}{ll}
          2 & \text{if } n = 0 \\
          1 & \text{if } n = 1 \\
          L_{n-1} + L_{n-2} & \text{if } n > 1 \\
        \end{array}
      \right.
    \]
    %
    An agent is alone and quite bored.
    The agent sets out to calculate the first ten Lucas numbers by applying the recursive definition.
    %
    The agent begins:
    %
    \[
      \begin{array}{cccccc}
        L_{0} & L_{1} & L_{2} & L_{3} & L_{4} & \cdots \\
        \hline
        2 & 1 & 3 & 4 & 7 & \cdots \\
      \end{array}
    \]
    %
    And, eventually concludes:
    \begin{center}
      \pv{\propI{The first ten Lucas numbers are 2, 1, 3, 4, 7, 11, 18, 29, 47, and 76}}{\valI{True}}
    \end{center}
    %
    From some \pool{} \(\Phi\) which includes the \agents{} understanding of the Lucas numbers.
  \end{scenario}
\end{note}

\end{document}

%%% Local Variables:
%%% mode: latex
%%% TeX-master: t
%%% TeX-engine: luatex
%%% TeX-master: "handout"
%%% End:

