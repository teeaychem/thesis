\makeatletter
\renewcommand{\PackageInfo}[2]{}% Remove package information
\renewcommand{\@font@info}[1]{}% Remove font information
% \renewcommand{\@latex@info}[1]{}% Remove LaTeX information
\makeatother

\PassOptionsToPackage{unicode}{hyperref}

\documentclass[10pt]{article}
\usepackage[british]{babel}

\usepackage[margin=.75in]{geometry}

\usepackage{amsthm}         % (in part) For the defined environments
\usepackage{mathtools}      % Improves  on amsmaths/mtpro2
\usepackage{amssymb}

% % % My packages % % %
\usepackage{myNotation}
\usepackage{ThesisCustom}
\usepackage{CustomEnvs}
\usepackage{ThesisFig}
% \usepackage[show]{Handout}
\usepackage[hide]{Handout}
% % % % % % % % % % % %

\usepackage{multicol}

% \usepackage{selnolig}% For suppressing certain typographic ligatures automatically
% % % % % % %
\usepackage{xfrac}
\usepackage{array}
\usepackage{arydshln}
\usepackage{multirow}

\newcolumntype{L}[1]{>{\raggedright\let\newline\\\arraybackslash\hspace{0pt}}m{#1}}
\newcolumntype{C}[1]{>{\centering\let\newline\\\arraybackslash\hspace{0pt}}m{#1}}
\newcolumntype{R}[1]{>{\raggedleft\let\newline\\\arraybackslash\hspace{0pt}}m{#1}}


\usepackage{xskak} % For chess diagram

% https://tex.stackexchange.com/questions/212710/fill-space-created-by-phantom-with-other-text
% \usepackage{calc}
% \newcommand{\textover}[3][l]{%
%  % #1 is the alignment, default l
%  % #2 is the text to be printed
%  % #3 is the text for setting the width
%  \makebox[\widthof{#3}][#1]{#2}%
%  }

% % % The bibliography % % %
\usepackage[%
backend=biber,
style=authoryear-comp,
bibstyle=authoryear,
citestyle=authoryear-comp,
uniquename=false,
backref=false,
hyperref=true,
url=false,
isbn=false,
doi=false,
useprefix=true,
maxbibnames=99,
]{biblatex}
\DeclareFieldFormat{postnote}{#1}
\DeclareFieldFormat{multipostnote}{#1}
% \setlength\bibitemsep{1.5\itemsep}
\newcommand{\noopsort}[1]{}
\addbibresource{Ability.bib}
\DefineBibliographyExtras{british}{\def\finalandcomma{\addcomma}} % Enable Oxford Comma
% % % % % % % % % % % % % % %

\usepackage[inline]{enumitem}
\SetEnumitemValue{labelindent}{standard}{.25\parindent}
\setlist[itemize]{labelindent=standard} %, leftmargin=1.5em}
\setlist[enumerate]{labelindent=standard} %, leftmargin=1.5em}

\newcommand{\squareBullet}{\textcolor{black}{\raisebox{.45ex}{\rule{.6ex}{.6ex}}}}

\newlist{itenum}{enumerate}{1}
\setlist[itenum]{
  style=standard,
  font=\normalfont,
  labelwidth = \widthof{\emph{Then}:},
}

\usepackage[export]{adjustbox}
\usepackage{subcaption}
\usepackage{float}

% % % % % % % % % % % % TIKZ
\usepackage{tikz}
\usetikzlibrary{bending,arrows,calc,arrows.meta,patterns,fadings}
\usetikzlibrary{trees}
\usetikzlibrary{backgrounds, positioning, fit, backgrounds}
\usetikzlibrary{math}
\usetikzlibrary{tikzmark}
% % % % % % % % % % % %


\usepackage{graphicx} % for images (png/jpeg etc.)
\usepackage{caption} % for \caption* command

% % % % % % % % % % % % MY COMMANDS

\makeatletter
\renewcommand\paragraph{\@startsection{paragraph}{4}{\z@}%
  {-3.25ex\@plus -1ex \@minus -.2ex}%
  {1.5ex \@plus .2ex}%
  {\normalfont\normalsize\bfseries}}
\makeatother

\makeatletter
\renewcommand\subparagraph{\@startsection{subparagraph}{4}{\z@}%
  {-3.25ex\@plus -1ex \@minus -.2ex}%
  {1.5ex \@plus .2ex}%
  {\normalfont\normalsize\bfseries}}
\makeatother

\makeatletter
\newcommand\lLine{{\color{gray} \noindent\rule{\textwidth}{0.4pt}}}
\newcommand\sepLine{
  \vfill
  \par\noindent\rule{\textwidth}{0.4pt}%
  \vspace{-10pt}%
  \par\noindent\rule{\textwidth}{0.4pt}
  \vfill}
\makeatother


% % % Fonts% %
\usepackage[no-math]{fontspec}
\defaultfontfeatures{Ligatures={NoDiscretionary, NoHistoric, NoRequired, NoContextual}}
% \protrudechars=2 %
\adjustspacing=2 %
\newfontfeature{Microtype}{protrusion=default;expansion=default;}
\setmainfont[Microtype]{Libertinus Serif}
\setsansfont[Microtype,Scale=MatchLowercase]{Libertinus Sans}
\setmonofont{Libertinus Mono}

\usepackage[match]{luatexja-fontspec}

\usepackage[math-style=ISO]{unicode-math}
\setmathfont{Libertinus Math}

\usepackage{pifont}
\newcommand{\hand}{\ding{43}}

\usepackage[
breaklinks,
bookmarks=false,
hidelinks,
linkcolor=highlight,
citecolor=highlight,
% colorlinks=true,
colorlinks=false,
]{hyperref}

\usepackage{csquotes}

\addto\extrasbritish{
  \def\chapterautorefname{Chapter}
  \def\sectionautorefname{Section}
  \def\subsectionautorefname{Section}
  \def\subsubsectionautorefname{Section}
  \def\paragraphautorefname{Section}
  \def\subparagraphautorefname{Section}
  \def\AlgoLineautorefname{Line}
  \def\pageautorefname{Page}
  \def\footnoteautorefname{Footnote}
  \def\scenarioCounterautorefname{Scenario}
  \def\observationCounterautorefname{Observation}
  \def\specificationCounterautorefname{Specification}
  \def\illustrationCounterautorefname{Illustration}
  \def\sketchCounterautorefname{Sketch}
  \def\linkCounterautorefname{Link}
  \def\constraintCounterautorefname{Constraint}
  \def\questionCounterautorefname{Question}
  \def\assumptionCounterautorefname{Assumption}
  \def\defiCounterautorefname{Definition}
  \def\propCounterautorefname{Proposition}
  \def\ideaCounterautorefname{Idea}
  \def\condCounterautorefname{Condition}
  \def\intuitionCounterautorefname{Intuition}
  \def\notationCounterautorefname{Notation}
  \def\applicationCounterautorefname{Application}
}

\title{\fc{3}}
\author{Sparkes}
\date{2023/06/13}

\begin{document}

\maketitle

\begin{comment}
  \begin{itemize}
  \item
    Goal for this talk is to cover the foundations of the thesis.
    If things go well, at the end of the presentation you will have a fairly good idea of:
    \begin{enumerate}
    \item
      Broadly, what the thesis is about.
    \item
      What I argued for.
    \item
      The `core' of the argument.

      The thesis is very detail oriented.
      However, most of the thesis amounts to refining a particular idea to obtain a specific result.
      The way the idea is refined is beyond the scope of a presentation like this, but I think the idea itself almost fits.
    \end{enumerate}
  \end{itemize}
\end{comment}


\begin{thesis}[First pass]
  A constraint:

  \quad On answers to: \emph{Why} \textover[c]{an}{the} \eiw{0} \textover[c]{an}{the} agent concludes happened.

  \quad \textover[r]{By answers to}{On answers to}: \textover[c]{\emph{How}}{\emph{Why}} the \eiw{0} the agent concludes happened.

  Does not hold.
\end{thesis}

\begin{comment}
  The phrase `an \eiw{0} an agent concludes happens' is carefully chosen.
  It might be more natural to ask why `an agent concludes'.
  However, this suggests something about agency.
  At issue is why the event happened, rather than why the agent concluded.
\end{comment}

\section{\eiw{3} an agent concludes and some abstractions}

\begin{note}
  \par\noindent\rule{\textwidth}{0.4pt}
  \begin{minipage}{.5\linewidth}
    \begin{rscenariox}{illu:gist:roots:F}{Quadratic roots --- Factorisation}%
      An agent reasons as follows:
      %
      \begin{enumerate}[label=\arabic*., ref=\arabic*]
      \item
        \label{illu:gist:roots:F:eq}
        For some \(x \in \mathbb{R}\), \rootsConEq{}.
      \item
        \label{illu:gist:roots:F:factor}
        \rootsConEqFac{}.
      \item
        \label{illu:gist:roots:F:zero}
        Either \(\rootsConEqFacL{} = 0\) or \(\rootsConEqFacR{} = 0\).
      \item
        \label{illu:gist:roots:F:case:a}
        If \(\rootsConEqFacL{} = 0\), then \(x = \rootsConEqFacLx{}\).
      \item
        \label{illu:gist:roots:F:case:b}
        If \(\rootsConEqFacR{} = 0\), then \(x = \rootsConEqFacRx{}\).
      \item
        \label{illu:gist:roots:F:factor:done}
        Either \(x = \rootsConEqFacLx{}\) or \(x = \rootsConEqFacRx{}\).
      \end{enumerate}
      %
      The agent concludes:
      \rootsCon{}.
    \end{rscenariox}
  \end{minipage}
  \begin{minipage}{.5\linewidth}
    \begin{rscenariox}{illu:gist:roots:QF}{Quadratic roots --- Formula}%
      An agent reasons as follows:
      %
      \begin{enumerate}[label=\arabic*., ref=\arabic*]
      \item
        \label{illu:gist:roots:QF:eq}
        For some \(x \in \mathbb{R}\), \rootsConEq{}.
      \item
        \label{illu:gist:roots:QF:qf}
        The quadratic formula states \(x = \rootsQEq{}\).
      \item
        \label{illu:gist:roots:QF:subs}
        Set \(a \coloneq \rootsQa{}\), \(b \coloneq \rootsQb{}\), and \(c \coloneq \rootsQc{}\).%
      \item
        \label{illu:gist:roots:QF:qf-subs}
        \(x = \rootsQEqFill{\rootsQa{}}{\rootsQb{}}{\rootsQc{}}\).
      \item
        \label{illu:gist:roots:QF:qf:1}
        \(x = \sfrac{(- 1 \pm 5)}{12}\).
      \item
        \label{illu:gist:roots:QF:qf:done}
        Either \(x = \rootsConEqFacLx{}\) or \(x = \rootsConEqFacRx{}\).
      \end{enumerate}
      %
      The agent concludes:
      \rootsCon{}.
    \end{rscenariox}
  \end{minipage}
  \par\noindent\rule{\textwidth}{0.4pt}
\end{note}

\begin{comment}
  \(a\) is the coefficient of the \(x^{2}\) term, \(b\) is the coefficient of the \(x\) term, and \(c\) is the constant.
\end{comment}

% \begin{comment}
%   \begin{itemize}
%   \item
%     In \autoref{illu:gist:roots:F} the agent does not conclude \propI{\rootsCon{}} from their understanding of the quadratic formula.
%   \item
%     In \autoref{illu:gist:roots:QF} the agent does not conclude \propI{\rootsCon{}} from their understanding of factorisation.
%   \end{itemize}
% \end{comment}

% \subsection{Abstractions}

\begin{note}
  \begin{itemize}[label = \squareBullet]
  \item
    \textbf{\prop{3}}: Some way things may be. \hfill {\color{gray} (Propositions)}
    \begin{itemize}
    \item
      \propI{The soup is sweet}, \propI{Foxes like pineapples}, \propI{The time is 17:30}, \dots \hfill \(\phi, \psi, \dots\)
    \end{itemize}
  \item
    \textbf{\val{3}}: An agent's perspectives on the way some things may be. \hfill {\color{gray} (Propositional attitudes)}
    \begin{itemize}
    \item
      \valI{True}, \valI{False}, \valI{Desirable}, \valI{Probable}, \dots \hfill \(v, v', \dots\)
    \end{itemize}
  \item
    \textbf{\pool{3}}: Collections of \prop{1} paired with \val{1}. \hfill {\color{gray} (Collections of propositional attitudes)}
    \begin{itemize}
    \item
      A \pool{} associated with an agent's understanding of factorisation, chess, sudoku puzzles, \dots \hfill \(\Phi, \Psi, \dots\)
    \end{itemize}
  \end{itemize}

\begin{comment}
  Think of this as a more flexible approach to propositional attitudes.
  Instead of agent evaluates \prop{} \(\phi\) to have \val{} \(v\), say agent formed attitude \(A\) toward proposition \(\phi\).

  \prop{3} correspond to propositions and \val{1} correspond to attitudes.
\end{comment}

  \lLine

  \begin{itemize}[label = \squareBullet]
  \item
    An agent concludes a \prop{0}-\val{0} pair \(\pv{\phi}{v}\) \textbf{from} some \pool{} \(\Phi\). \hfill (During some event.)
    \begin{itemize}
    \item
      In \autoref{illu:gist:roots:F} the agent concludes \pv{\propI{\rootsCon{}}}{\valI{True}} from a \pool{} which includes \pv{\propM{\rootsConEq{}}}{\valI{True}} and \prop{0}-\val{0} pairs associated with the \agents{} understanding of factorisation.
    \end{itemize}
  \item
    A \prop{2}-\val{0} pair \(\pv{\phi}{v}\) \textbf{\fof{}} some \pool{} \(\Phi\), from an \agpe{}. \hfill (Relative to some event.)
    \begin{itemize}
    \item
      In \autoref{illu:gist:roots:QF}, at Step 3, \pv{\propI{\rootsCon{}}}{\valI{True}} \fof{} a \pool{} which includes \pv{\propM{\rootsConEq{}}}{\valI{True}} and \prop{0}-\val{0} pairs associated with the \agents{} understanding of the quadratic formula.
    \end{itemize}
    \begin{itemize}
    \item[\hand]
      A \fof{} relation may hold between arbitrary \prop{0}-\val{0}-\pool{0} pairings and does not require a from relation.
      \begin{itemize}
      \item
        No developed theory of \fofr{1}.
        Instead, sufficient conditions.
        E.g.\ from \(\Rightarrow\) \fof{}.
      % \item[\(\leadsto\)]
      %   Maybe think propositional justification, but only descriptive.
      \end{itemize}
    \end{itemize}
  \end{itemize}
\end{note}

\begin{comment}
  Relative to an \eiw{} agent concludes.

  E.g.\ agent concluded by factorisation, but had the option to use the quadratic formula.

  For additional examples, consider practice problems.
  If you've been learning the basics of factorisation, then \fofr{} should (hopefully) hold for practice problems.
\end{comment}

\begin{comment}
  The constraint on reasoning that is stated with these abstraction is intended to be theory-neutral level.
\end{comment}


\begin{comment}
  If you like, think of this as a relation of support.
  However, \fofr{1} are understood descriptively, i.e.\ there's no normative constraints on \fofr{1} here.
  If you'd like additional motivation for this sort of thing, consider propositional justification.
\end{comment}

\newpage

\section{Two questions and a constraint \hfill (\qWhy{}, \qHow{}, and \issueInclusion{})}
\label{sec:target}

\vfill

\begin{comment}
  With from and \fofr{1} in hand, two questions.
\end{comment}

\begin{note}
  \begin{question}{questionWhy}{\qWhy{}}
    Given \(e\) is an \eiw{0} an agent concludes \(\pv{\phi}{v}\) from \(\Phi\):
    \begin{itemize}[label = \squareBullet]
    \item
      Which \fofr{1} are included in (partial) explanations of \emph{why} \(e\) is an \eiw{0} the agent concludes \(\pv{\phi}{v}\) from \(\Phi\) (rather than an \eiw{0} some other thing happens)?
    \end{itemize}
    \vspace{-1.25\baselineskip}
  \end{question}
\end{note}


\lLine


\begin{note}
  \begin{itemize}
  \item
    The sense of `why' in \qWhy{} asks what (if anything) favoured \(e\) being an \eiw{} \(A\) happened, rather than \(e\) being an \eiw{} \(B\) happened.
    %
    \begin{itemize}
    \item
      Cards have been shuffled and \mainCard{} is \mainCardPos{}.
      \begin{itemize}
      \item
        If random shuffle, no answer to why \mainCard{} is \mainCardPos{}. \hfill (Any card may have been there.)
      \item
        If sleight of hand, answer to why \mainCard{} is \mainCardPos{}. \hfill (The result was forced.)
      \end{itemize}
    \end{itemize}
  \end{itemize}

  \begin{itemize}
  \item[\hand]
    \qWhy{} asks about (partially) `explanatory' \fingfr{1}.
  \item
    Intuitively, \pv{\propM{\rootsCon{}}}{\valI{True}} \fof{} a \pool{} which contains \pv{\propM{\rootsConEq{}}}{\valI{True}} and various other \prop{0}-\val{0} pairs associated with the agent's understanding of factorisation. (Similar to forcing a card.)\par

    I do not presuppose \qWhy{} has answers.
    Rather, I argue \qWhy{} has answers (sketched below).
  \end{itemize}
\end{note}

\begin{comment}
  Note, \qWhy{} asks about \emph{partial} explanations.
  So, these explanations need not be sufficient.
  And, \qWhy{} does not require the (partial) explanations are necessary.

  In this respect, you're allowed to lose some interest here.
  And, I think the permissiveness of explanation here is potentially an issue.
  That being said, I think some work could be done to add some interesting constraints here.
  The issue is incorporating these constraints into arguments.
  On the one hand, there's an issue of whether the relevant arguments can be made.
  And, on the other hand, there's an issue of complexity.
\end{comment}

\sepLine

\begin{note}
  \begin{question}{questionHow}{\qHow{}}
    Given \(e\) is an \eiw{0} an agent concludes \(\pv{\phi}{v}\) from \(\Phi\):
    \begin{itemize}[label = \squareBullet]
    \item
      Which past or present events (partially) explain \emph{how} \(e\) is an \eiw{0} the agent concludes \(\pv{\phi}{v}\) from \(\Phi\) (rather than an \eiw{0} some other thing happens)?
    \end{itemize}
    \vspace{-1.5\baselineskip}
  \end{question}
\end{note}

\lLine

\begin{note}
  \begin{itemize}[noitemsep]
  \item
    The sense of `how' in \qHow{} asks what (if anything) resulted in \(e\) being an \eiw{} \(A\) happened, rather than \(e\) being an event in which \(B\) happened  --- e.g.\ from relations.
  \item
    E.g.\ what happened when the agent shuffled cards or made a conclusion.
  \end{itemize}
\end{note}

\sepLine

\begin{note}
  \begin{constraintD}{consInclusion}{\issueInclusion{}}
    Given \(e\) is an \eiw{0} an agent concludes \(\pv{\phi}{v}\) from \(\Phi\):
    \begin{itenum}
    \item[\emph{If}:]
      Some \prop{0}-\val{0} pair \(\pv{\psi}{v'}\) \fingf{} \(\Psi\) answers \qWhy{}.
    \item[\emph{Then}:]
      A prior or present \eiw{0} the agent concludes \(\pv{\psi}{v'}\) from \(\Psi\) answers \qHow{}.
    \end{itenum}
    \vspace{-1.25\baselineskip}
  \end{constraintD}
\end{note}

\begin{comment}
  This is where some important work is done.
  The thing is, this is set up in a careful way.
  \fofr{3} answer both \qWhy{} and \qHow{}.
  So, it's possible for a \fofr{} between a \prop{0}-\val{0} pair and \pool{} distinct from the \prop{0}-\val{0}-\pool{0} pair directly related to the conclusion to answer both \qWhy{} and \qHow{}.

  Key observation is that if the agent has not concluded \(\pv{\psi}{v'}\) from \(\Psi\) then \(\pv{\psi}{v'}\) \fof{} \(\Psi\) does not answer \qHow{}.
\end{comment}

\lLine

\begin{note}
  \begin{itemize}[noitemsep]
  \item
    I.e.\ The only answers to \qWhy{} are when \(\pv{\phi}{v}\) \fof{} \(\Phi\). \hfill (Unless the agent has concluded \(\pv{\psi}{v'}\) from \(\Psi\).)
    \begin{itemize}
    \item[\hand]
      A \fofr{} only answers \qWhy{} if it has been `witnessed' by a from relation.
    \end{itemize}
  \item
    I think \issueInclusion{} is (maybe) intuitive.
    \begin{itemize}
    \item
      Sleight of hand happened, the agent concluded via factorisation/the quadratic formula.
    \end{itemize}
  \item
    \citeauthor{Davidson:1963aa} opens \citetitle{Davidson:1963aa} with the following question:

    \begin{quote}
      What is the relation between a reason and an action when the reason explains the action by giving the agent's reason for doing what he did?
      We may call such explanations \emph{rationalizations}, and say that the reason rationalizes the action.%
      \mbox{ }\hfill\mbox{(\citeyear[685]{Davidson:1963aa})}
    \end{quote}

    % Paraphrased:
    % %
    % \begin{quote}
    %   A rationalisation is an explanation of why an agent did action \(a\) by giving the agent's reason for doing \(a\).
    % \end{quote}
    % %
    \citeauthor{Davidson:1963aa} argues rationalisations are causal explanations.
    So, this is plausibly an instance of an answer to a `why' question being constrained by answers to a `how' question.
    (See also \cite{Hieronymi:2011aa}.)
    \begin{itemize}
    \item
      Though, rationalisations are different from \fofr{1} and there's no mapping between the two.
    \end{itemize}
  \end{itemize}
\end{note}

\sepLine

\begin{note}
  \begin{thesis}[Second pass]
    The Constraint (\issueInclusion{}) does not hold.
  \end{thesis}
\end{note}

% \begin{comment}
%   Might like to view \qWhy{} and \qHow{} as, respectively, asking about constitutive and causal explanations.
%   Still, I see no need to relate these questions to broader types of explanation.
%   If they do correspond, then great.
%   But, I'm not interested in whether a connexion between these types of explanation holds in general, or in the specific case of an \eiw{} an agent concludes.

%   As focus on getting answers to \qWhy{}, maybe you'll tell me more in the Q\&A.
% \end{comment}

\begin{comment}
  For \issueInclusion{} to fail, need a counterexample.
  Though this is more involved.

  \begin{enumerate}
  \item
    Sense of `why' and `how'.
  \item
    Answers to \qWhy{}.
  \item
    Way to identify \fingfr{}.
  \end{enumerate}
\end{comment}

\begin{comment}
  Not really about the result.
  Argument.
  Fairly technical.
  Framework so it's possible to create a direct argument.
  Only issue is whether the definitions, ideas, etc.\ work out.
  I.e.\ whether you agree.

  Stress this point.
  Well, what do I want to stress?
  `Makes sense'.
  So, think of epistemology.
  Various cases, intuition about whether knows.
  Intuition is taken as input for creating theory.
  This is the sort of Rawlsian reflective equilibrium.

  Important constraint.
  Should not be the case framework presupposes answer.
  See how these definitions come together.
\end{comment}



\vfill


\newpage

\section{Answers to \qWhy{} via \progEx{0}}
\label{sec:answers-qwhy}

\begin{note}
  Basic idea: Obtain answers to \qWhy{} by considering conclusions in progress.
\end{note}

\begin{comment}
  Part of why this is interesting.
  Maybe.
  No subjunctives, with the exception of thinking about events in progress.
  If you can do events in progress without subjunctives, then, things go through.
  Even projections.
  These talk about possibility, and entail various subjunctives, but that's about it.
\end{comment}

\begin{note}
  \hfill\hspace{\dimexpr-\fboxrule-\fboxsep\relax}\fbox{%
    \begin{minipage}[t]{.6\linewidth}
      Events \(\ed{\flat}\) such that some event \(\eXdn{}\) which satisfies description \(\edo{}\) is in progress.
    \end{minipage}
  }\hfill
\end{note}

\begin{note}
  Roughly, think of events in progress in terms of the (English) progressive aspect:\footnote{
    This is a common analysis of the progressive.
    See, e.g., \cite{Bennett:1972uw,Dowty:1979vq,Parsons:1990aa,Landman:1992wh,Portner:1998um}.
  }
  %
  \begin{enumerate}
  \item
    The agent is making soup.%\newline
    \mbox{ } \hfill \(\leadsto\) An \eiw{0} the agent makes soup is in progress.\newline
    \mbox{ }\hfill \emph{{\color{gray} Some event \(\eXdn{}\) which satisfies description} {\color{darkgray} `The agent makes soup'} {\color{gray} is in progress}}.
  \item
    The agent is concluding \rootsCon{}.%\newline
    \mbox{ } \hfill \(\leadsto\) An \eiw{0} the agent concludes \rootsCon{} is in progress.\newline
    \mbox{ }\hfill \emph{{\color{gray} Some event \(\eXdn{}\) which satisfies description} {\color{darkgray} `The agent concludes \rootsCon{}' } {\color{gray} is in progress}.}
  \item
    The agent is passing the exam.%\newline
    \mbox{ } \hfill \(\leadsto\) An \eiw{0} the agent passes the exam is in progress.\newline
    \mbox{ }\hfill \emph{{\color{gray} Some event \(\eXdn{}\) which satisfies description} {\color{darkgray} `The agent passes the exam'} {\color{gray} is in progress}.}
  \end{enumerate}
  %
\begin{comment}
  This doesn't quite work, as the progressive has some quirks.
  Still, it's close enough to fix a grasp on events in progress.
\end{comment}
\end{note}

\sepLine

\begin{note}
  \begin{observation}[Favours]
    \label{obs:favours}
    \vspace{-\baselineskip}
    \begin{itenum}
    \item[\emph{If}:]
      \(\ed{\flat}\) is such that some event \(\eXdn{}\) under description \(\edo{}\) is in progress.
    \item[\emph{Then}:]
      \(\ed{}\) is favoured over any event which in incompatible with \(\ed{}\) happening.
    \end{itenum}
    \vspace{-\baselineskip}
  \end{observation}

  \begin{motivation}{obs:favours}
    \vspace{-\baselineskip}
    \begin{enumerate}[label=\Alph*.]
    \item
      If concluding by factorisation, then not concluding by the quadratic equation.
    \item
      If rolling a biased die, then not leaving things up to chance.
    \item
    If Team A is winning, then Team B isn't winning.
  \end{enumerate}
  \vspace{-\baselineskip}
  \end{motivation}
\end{note}

\begin{comment}
  Well, at least as far as intuition goes.
  Things are a little more delicate in the main document.
\end{comment}

\lLine

\begin{note}
  Shifting perspective:
  \setcounter{observationCounter}{0}

  \begin{observation}[Favors]
    \label{obs:favoursII}
    \vspace{-\baselineskip}
    \begin{itenum}
    \item[\emph{If}:]
      \(\ed{\flat}\) is such that some event \(\eXdn{}\) which satisfies description \(\edo{}\) is in progress \emph{and} \(\ed{}\) happens.
    \item[\emph{Then}:]
      The details of \(\edn{\flat}\) captured by \(\edo{\flat}\) favoured \(\ed{}\) happening over any other event.
    \end{itenum}
    \vspace{-\baselineskip}
  \end{observation}

  \begin{itemize}
  \item[\hand]
    Details of events in progress are answers to why the event happened, given the sense of `why' present in \qWhy{}.
  \end{itemize}
\end{note}

\sepLine

\begin{note}
  \begin{idea}[\fc{3}]
    \vspace{-\baselineskip}
    \begin{itenum}
    \item[\emph{If}:]
      The following two conditions hold:
      \begin{itemize}
      \item
        \(\ed{\flat}\) is an \eiw{} an agent either is, or has the option to be, concluding \(\pv{\phi}{v}\) from \(\Phi\).
      \item
        The agent already evaluates each \prop{0} in \(\Phi\) with it's paired \val{}.
      \end{itemize}
    \item[\emph{Then}:]
      For the agent, \(\pv{\phi}{v}\) \fof{} \(\Phi\) through \(\ed{\flat}\).
    \end{itenum}
    \vspace{-\baselineskip}
  \end{idea}
\end{note}

\begin{comment}
  This avoids over-generating \fofr{1}.
\end{comment}

\lLine

\begin{comment}
  Something about \(\edo{\flat}\) ensures an event which satisfies \(\edo{}\) is in progress.
  In this respect, a \fofr{}.
\end{comment}

\begin{note}
  \begin{itemize}
  \item
    Consider the contrapositive:

    \begin{itenum}[noitemsep]
    \item[\emph{If}:]
      For the agent, it is not the case \(\pv{\phi}{v}\) \fof{} \(\Phi\).
    \item[\emph{Then}:]
      There is nothing which ensures the agent concluding \(\pv{\phi}{v}\) from \(\Phi\).
    \end{itenum}
  \end{itemize}
\end{note}

\sepLine

\begin{note}
  \begin{itemize}
  \item
    \progEx{2} paired with \fc{1} is the way we get \fofr{1} which answer \qWhy{}.
    \begin{itemize}
    \item[\hand]
      If an agent is concluding \(\pv{\phi}{v}\) from \(\Phi\) then:
      \begin{itemize}
      \item
        An event in which the agent concludes \(\pv{\phi}{v}\) from \(\Phi\) is favoured over any other event.
        \begin{itemize}
        \item
          And, included is that \(\pv{\phi}{v}\) \fof{} \(\Phi\), from the agent's perspective.
        \end{itemize}
      \end{itemize}
    \end{itemize}
  \end{itemize}
\end{note}

\vfill

\newpage


\section{Failure}

% \sepLine

\begin{note}
  \begin{idea}[Sometimes an agent's reasoning is of a type, and so it must be that other \fofr{1} hold]
    \emph{Sometimes}:
    \begin{itenum}
    \item[\emph{If}:]
      \(\ed{\flat}\) is an \eiw{} an agent concluding \(\pv{\phi}{v}\) from \(\Phi\).
    \item[\emph{Then}:]
      For the agent, \(\pv{\psi}{v'}\) \fof{} \(\Psi\) through \(\ed{\flat}\).
    \end{itenum}
    \vspace{-\baselineskip}
  \end{idea}
\end{note}

\lLine

\begin{note}
    \begin{scenario}[Lucas numbers]%
    \label{scen:LucasNums}%
    The Lucas numbers are recursively defined by: \qquad
    \(
      L_{n} = \left\{
        \begin{array}{ll}
          2 & \text{if } n = 0 \\
          1 & \text{if } n = 1 \\
          L_{n-1} + L_{n-2} & \text{if } n > 1 \\
        \end{array}
      \right.
    \)

    \smallskip
    \noindent An agent sets out to calculate the some of the Lucas numbers by the recursive definition.
    %
    \begin{center}
      \(
      \begin{array}{cccccc}
        L_{0} & L_{1} & L_{2} & L_{3} & L_{4} & \cdots \\
        \hline
        2 & 1 & 3 & 4 & 7 & \cdots \\
      \end{array}
      \)
    \end{center}
    \vspace{-2\baselineskip}
  \end{scenario}

\begin{comment}
  Note, the agent doesn't (necessarily) conclude \(L_{4}\) from \prop{0}-\val{0} pairs which detail \(L_{2}\) and \(L_{3}\).
  For, \(L_{2}\) and \(L_{3}\) both \fof{} the recursive definition.

  The agent's doesn't need anything other than the recursive definition to get \(L_{4}\), though they'll get \(L_{2}\) and \(L_{3}\) on the way to \(L_{4}\).
\end{comment}

  \begin{itemize}[noitemsep]
  \item[\hand]
    For negative cases which appeal to this idea, consider selection tasks (\cite{Wason:1966aa}):
    \begin{itemize}[noitemsep]
    \item[\leadsto]
      As agents fail at selection tasks, they do not reason about natural language conditionals as material conditionals.
    \item
      Contraposed:
      \begin{itemize}
      \item Agents reason about nat.\ lang.\ conditionals as material conditionals \emph{only if} they succeed at selection tasks.
      \end{itemize}
    \end{itemize}
  \item
    See also various observations regarding Decision theory (\cite{Allais:1979aa}, \cite{Ellsberg:1961aa}, \cite{Quinn:1990aa}), structural principles (\cite{Makinson:1965aa}, \cite{Kyburg:1997aa}), and folk psychology (\cite{Bratman:1981aa,Bratman:1987aa})
  \end{itemize}
\end{note}

\sepLine

\begin{note}
  \begin{idea}[Typical failure]
    \emph{Sometimes}:
    \begin{enumerate}[label=\Alph*.]
    \item
      \begin{itenum}
      \item[\emph{If}:]
        \(\ed{\flat}\) is an \eiw{} an agent concluding \(\pv{\phi}{v}\) from \(\Phi\).
      \item[\emph{Then}:]
        For the agent, \(\pv{\psi}{v'}\) \fof{} \(\Psi\) through \(\ed{\flat}\).
      \end{itenum}
    \item
      \(\ed{\flat}\) is an \eiw{} an agent concluding \(\pv{\phi}{v}\) from \(\Phi\).
    \item
      The agent has not concluded \(\pv{\psi}{v'}\) from \(\Psi\).
    \end{enumerate}
    \vspace{-\baselineskip}
  \end{idea}
\end{note}

\begin{comment}
  Intuitively:
      \begin{itemize}
    \item
      \begin{itenum}
      \item[\emph{If}:]
        \(\pv{\phi}{v}\) is a \fc{} from \(\Phi\).
      \item[\emph{Then}:]
        \(\pv{\psi}{v'}\) is a \fc{} from \(\Psi\).
      \end{itenum}
    \item
      \(\pv{\phi}{v}\) is a \fc{} from \(\Phi\).
    \item
      The agent has not concluded \(\pv{\psi}{v'}\) from \(\Psi\).
    \end{itemize}

    What secures the relevant \fc{1} is that \(\ed{\flat}\) is an \eiw{} an agent concluding \(\pv{\phi}{v}\) from \(\Phi\).
\end{comment}

\lLine

\begin{note}
    \begin{scenario}[Sudoku puzzles]%
    \label{illu:gist:sudoku}%
    % https://tex.stackexchange.com/questions/91422/tikz-sudoku-circle-and-connect-with-lines-some-cells
    An agent (you) who has an understanding of Sudoku solves \sudokuPuzL{} (though does not work on \sudokuPuzR{}).%
    \footnote{
      The puzzles are from (\cite[54,56]{Coussement:2007up}).
    }
    \bigskip

    \begin{figure}[h!]
      \centering
      \begin{subfigure}{.45\linewidth}
        \centering
        \begin{tikzpicture}[scale=.5]
          \begin{scope}
            \draw (0, 0) grid (9, 9);
            \draw[very thick, scale=3] (0, 0) grid (3, 3);
            \setcounter{row}{1}
            \setrow { }{ }{ }  { }{ }{ }  { }{ }{ }
            \setrow {5}{ }{ }  { }{ }{ }  {3}{2}{ }
            \setrow { }{ }{ }  {4}{7}{9}  {6}{8}{ }
            \setrow {7}{ }{3}  { }{ }{4}  { }{ }{ }
            \setrow {4}{ }{5}  { }{ }{2}  { }{6}{9}
            \setrow {9}{2}{8}  {7}{5}{6}  {4}{ }{1}
            \setrow {3}{5}{ }  { }{2}{ }  { }{9}{ }
            \setrow { }{ }{1}  { }{9}{3}  {5}{ }{ }
            \setrow { }{ }{ }  { }{ }{8}  {1}{7}{ }
          \end{scope}
        \end{tikzpicture}
        \caption*{\sudokuPuzL{}}
      \end{subfigure}
      \begin{subfigure}{.45\linewidth}
        \centering
        \begin{tikzpicture}[scale=.5]
          \begin{scope}
            \draw (0, 0) grid (9, 9);
            \draw[very thick, scale=3] (0, 0) grid (3, 3);
            \setcounter{row}{1}
            % Single entries
            \setrow {4}{6}{3}  {7}{ }{8}  { }{ }{ }
            \setrow {2}{1}{ }  { }{ }{ }  { }{ }{7}
            \setrow { }{7}{5}  {4}{ }{ }  { }{3}{ }
            \setrow { }{2}{ }  {8}{7}{ }  { }{ }{3}
            \setrow {3}{9}{1}  { }{ }{2}  {5}{ }{8}
            \setrow { }{ }{6}  {1}{3}{ }  { }{ }{2}
            \setrow { }{3}{8}  { }{1}{6}  {7}{5}{4}
            \setrow { }{ }{ }  {9}{ }{ }  { }{ }{1}
            \setrow { }{ }{ }  { }{ }{ }  {2}{ }{ }
          \end{scope}
        \end{tikzpicture}
        \caption*{\sudokuPuzR{}}
      \end{subfigure}
    \end{figure}
    \vspace{-\baselineskip}
  \end{scenario}
\end{note}

\vfill

\begin{comment}
  In summary, constraint on answers to \qWhy{} in terms of answers to \qHow{} does not hold.

  In short, answering \emph{how} details a particular event.

  Yet, when asking about \emph{why} some thing we're often interested in classifying a particular event as an instance of a general phenomenon.

  \emph{Why} was the agent's reasoning an instance of applying their understanding of Sudoku, rather than something else?
  E.g., guessing randomly.

  To do this, one ends up citing the way things happen in other cases.

  In very short, a single instance of reasoning is an instance of a general type of reasoning \emph{only if} the particular instance may be adapted to a different instance of the type, resources permitting.

  So, that a particular \fofr{} answers \qWhy{} (sometimes) also entails various other \fofr{1} answer \qWhy{}.

  \lLine{}
\end{comment}

\begin{comment}
  Perhaps this result is due to the level of generality at which \qWhy{} and \qHow{} are asked?

  I don't think this is right.

  For example, suppose we have very detailed picture of what happens when someone is doing a Sudoku puzzle.
  And, there's some neural activity which is only activated when the agent is doing Sudoku puzzles.
  Sure, we have plausibly have an answer to both \qWhy{} which at first glance only looks at what happened (in line with \issueInclusion{}).
  However, what makes it the case that the neural activity is that of solving Sudoku puzzles?
  Well, plausibly due to the same neural activity in other events.
\end{comment}

\vfill

\newpage

\section{\progEx{2}, a little more}

\begin{note}
  \begin{observation}[Squandered progress]%
    \label{obs:se-need-hCon}%
    It may be the case that:
    \begin{center}
      \begin{itemize}
      \item
        \(\ed{\flat}\) is such that some event \(\eXdn{}\) which satisfies description \(\edo{}\) is in progress.
      \item
        \(\ed{}\) happens.
      \item
        \(\ed{}\) does not (partly) happen as a result of \(\ed{\flat}\).
      \end{itemize}
    \end{center}
    \vspace{-\baselineskip}
  \end{observation}

  \begin{motivation}{obs:se-need-hCon}
    \vspace{-\baselineskip}
    \begin{enumerate}[label=\Alph*.]
    \item
      Agent Kujan is learning Roger Kint is Keyser S\"{o}ze, but is spoiled.
    \item
      Agent is passing an exam, and without external help, but answers are passed to them.
    \item
      Team A is winning the game due to playing well, but Team A wins the game by default as Team B forfeits due to injury.
    \end{enumerate}
    \vspace{-.5\baselineskip}
  \end{motivation}
\end{note}

\sepLine

\begin{note}
  \begin{rdefinition}{def:se}{\se{3}}
    \vspace{-\baselineskip}
    \begin{itemize}
    \item
      \(\ed{\flat}\) is a \emph{\se{0}} of \(\ed{}\).
    \end{itemize}
    \emph{If and only if}:
    \begin{itemize}
    \item
      Clauses~\ref{assu:p:se:prog} and \ref{assu:p:se:hCon} hold:
      \begin{enumerate}[label=\Alph*., ref=\Alph*]
      \item
        \label{assu:p:se:prog}
        \(\ed{\flat}\) is such that some event \(\eXdn{}\) under description \(\edo{}\) is in progress.
      \item
        \label{assu:p:se:hCon}
        \(\ed{}\) partly happens as a result of \(\ed{\flat}\).
      \end{enumerate}
    \end{itemize}
    \vspace{-\baselineskip}
  \end{rdefinition}

  \lLine

  \begin{itemize}
  \item
    Clause~\ref{assu:p:se:prog} `looks forward':

    The description \(\edo{\flat}\) of \(\ed{\flat}\) captures that an event described by \(\edo{}\) is in progress.\newline
    \mbox{}\hfill\leadsto As Clause~\ref{assu:p:se:prog} concerns \(\ed{\flat}\), no specific event which satisfies \(\edo{}\) is fixed.
  \item
    Clause~\ref{assu:p:se:hCon} `looks backward':

    \(\ed{}\) as described by \(\edo{}\) happened in part as a result of \(\ed{\flat}\) as described by \(\edo{\flat}\).
  \end{itemize}
\end{note}

\sepLine

\begin{note}
  \begin{idea}[\progEx{2}]%
    \label{prop:PEbasic}
    Given \(\ed{}\) is an \eiw{0} an agent does \(a\), and the sense of `why' present in \qWhy{}:

    \begin{itenum}
    \item[\emph{If}:]
      \(\ed{\flat}\) is a \se{} of \(\ed{}\).
    \item[\emph{Then:}]
      \(\edo{\flat}\) being true of \(\ed{\flat}\) explains `why' \(\ed{}\) happened.
    \end{itenum}
    \vspace{-\baselineskip}
  \end{idea}
\end{note}

\begin{comment}
  Split perspective.
  Important.
  Weak assumption about events in progress.
  Only that there is some possible event.
  With hindsight, fix the particular event which happened.
\end{comment}

\lLine

\begin{note}
  There's some more to this, but in short:

  \begin{itemize}
  \item
    Clause~\ref{assu:p:se:prog} allows us to apply Observation \ref{obs:favours}.
  \item
    Clause~\ref{assu:p:se:hCon} allows us to avoid Observation \ref{obs:se-need-hCon}.
  \end{itemize}

  \begin{itemize}
  \item[\hand]
    It's fine to look backwards, as our interest is in why (and how) an event happened.
    \begin{itemize}
    \item
      Though, when looking forward we do not do so with the benefit of hindsight (hence the existential).
    \end{itemize}
  \end{itemize}
\end{note}

\vfill


\newpage

\section{More on \fc{1}}

\begin{note}
    \begin{rdefinition}{def:fc}{\fc{3}}%
    \vspace{-\baselineskip}
    \begin{itemize}
    \item
      \prop{2}-\val{0} pair \(\pv{\phi}{v}\) is a \emph{\fc{0}} from \pool{} \(\Phi\) for an agent through an event \(\ed{}\).
    \end{itemize}

    \emph{If and only if}

    \begin{itemize}
    \item
      For some partition of \(\edn{}\) into sub-events \(\edn{1}, \dots, \edn{k}\) there are descriptions \(\edo{1}, \dots, \edo{k}\) such that clauses \ref{def:fc:ai}, \ref{def:fc:act} and \ref{def:fc:result} hold:
      %
      \begin{enumerate}[label=\Alph*., ref=\Alph*, series=fcCounter]
      \item
        \label{def:fc:ai}
        \(\ed{i}\) is an \eiw{0} the agent may do some action \(a_{i}\).
        %
      \item
        \label{def:fc:act}
        The \eiw[\(\edn{a_{i}}\)]{0} the agent does \(a_{i}\) is an \eiw{0} the agent is concluding \(\pv{\phi}{v}\) from \(\Phi\).
        %
      \item
        \label{def:fc:result}
        For each \prop{0}-\val{0} pair \(\pv{\phi'}{v'}\) in \(\Phi\), the agent \evals{} \(\phi'\) as having value \(v'\) prior to doing \(a_{i}\).
        %
      \end{enumerate}
    \end{itemize}
    \vspace{-1.5\baselineskip}
  \end{rdefinition}
\end{note}

\sepLine

\begin{note}
  \begin{scenario}[Paltry]
    \label{scen:fc:chick}%
    An agent reads Puzzle~25 of \citeauthor{Dudeney:1995aa}'s \citetitle{Dudeney:1995aa}:
    %
    \begin{quote}
      Three chickens and one duck sold for as much as two geese; one chicken, two ducks, and three geese were sold together for \$25.00. What was the price of each bird in an exact number of dollars?%
      \mbox{ }\hfill\mbox{(\citeyear[9]{Dudeney:1995aa})}
    \end{quote}
    \vspace{-1.5\baselineskip}
  \end{scenario}
\end{note}

\lLine

\begin{note}[Chess]
  \begin{scenario}[Chess]%
    \label{illu:fc:chess:I}%
    An agent reads \citeauthor{Emms:2000aa}' Puzzle 113 (\citeyear[33]{Emms:2000aa}):\vspace{-\baselineskip}
    \begin{quote}
      \mbox{ }\hfill%
      \begin{adjustbox}{minipage=\linewidth, scale=.75}
        \centering
        \newchessgame[
        setwhite={pa2,pb2,pc2,pd3,pf2,pg3,ra1,re1,bd4,kg1,qe5},
        addblack={ra8,pa7,ba6,pb5,rc8,pd5,pf7,kg8,qg4,ph7,ph4},
        ]%
        \setchessboard{showmover=false}%
        \chessboard
      \end{adjustbox}%
      \label{fig:chess:easy}%
      \hfill\mbox{ }
      \begin{center}
        Is possible for White to checkmate in a single move?
      \end{center}
    \end{quote}
    \vspace{-2\baselineskip}
  \end{scenario}
\end{note}

\lLine

\begin{note}
    \begin{scenario}[ジョジョリオン]%
    \label{scen:jojo}%
    \nocite{huangmufeiluyan:2011aa}%
    An agent is skimming through the chapter titles of a book and translating the titles.
    A handful of translations and titles are:

    \begin{center}
      \bgroup
      \def\arraystretch{1.125}
      \begin{tabular}{R{.45\textwidth} L{.45\textwidth}}
        Translation & Title \\
        \hline
        Soft and wet & ソフト&ウェット \\
        Every day is a summer holiday & 毎日が夏休み \\
        A hair clip from ??? period & 清の時代の髪留め \\
      \end{tabular}
      \egroup
    \end{center}

    \noindent%
    The agent then reads 「無事が何より」.
    \vspace{-\baselineskip}
  \end{scenario}
\end{note}
\mbox{}\newline

\sepLine

\begin{multicols}{2}
  {
    \renewcommand*{\bibfont}{\tiny}
    \printbibliography[heading=none]
  }
\end{multicols}

\end{document}

\begin{note}
  Well, a more general perspective on Wason, etc.\ is in terms of mistakes.
  Intuitively, it's only the case that an agent is making a mistake, relative to their perspective, if in typical circumstances the agent does things differently.
  In parallel, the agent is only doing things correctly, if in typical circumstances the agent does things the same.

  If there are no typical circumstances, then there's no generality to the agent's reasoning, and therefore nothing which establishes whether the agent is mistakenly or correctly doing things, from their perspective.
  The only other option is some independent measure, but an independent measure is only good for evaluating what the agent has done, not accounting for why the agent did that thing.

  In short, when we describe an agent reasoning, we have a type of reasoning in mind.
  And, this implies the agent may reason in other ways.
  In turn, this implies various following from relations hold.
\end{note}



%%% Local Variables:
%%% mode: latex
%%% TeX-master: t
%%% TeX-engine: luatex
%%% TeX-master: "handout"
%%% End:

