\chapter{Old \nI{} arguments}

\paragraph{Another}

\begin{note}[Propositions applied to \nI{}]
  Hence, obtain from \nIBackground{} that \(\psi\) having value \(v'\) is expectation by \autoref{prop:info-to-expectation}.

  In turn, \ref{nI:going-by-value} restricts attention to when \(\phi\) having value \(v\) is required part of reasoning.
  Hence, \autoref{assu:expect-nai} prevents claiming support.
\end{note}

\begin{note}
  So, we assume that \nIBackground{} obtain.

  Important step is establishing that \(\psi\) having value \(v'\) is an expectation of the claimed support for \(\phi\) having value \(v\).

  In outline.
  \begin{itemize}
  \item \ref{nI:claimed-support} claimed support for \(\phi\) having value \(v\).
  \item \ref{nI:psi-is-new} \(\psi\) was not considered.
  \item Interdependence between claimed support for \(\phi\) having value \(v\) and claiming support for \(\psi\) having value \(v'\).
  \item So, from previous two, \(\psi\) not having value \(v'\) was an unrecognised defeater.
  \item Hence, \(\psi\) having value \(v'\) is expectation.
  \end{itemize}
\end{note}

\begin{note}
  So, focus of argument is primarily on why~\ref{nI:inclusion} leads to expectation that \(\psi\) has value \(v'\) with respect to \(\psi\) having value \(v'\).
\end{note}

\begin{note}
  See that \nI{} is somewhat weak.
  Requires going by \(\phi\).
  \emph{And} that \(\psi\) is expectation.
  May be that resolution to instance of \nI{} is to remove \(\phi\) as expectation.

  So, upshot has two parts.
  Either remove expectation, or abandon claiming support for \(\psi\) having value \(v'\) via appeal to \(\phi\) having value \(v\).
\end{note}

\paragraph{Another}

\begin{note}
  Important that expectation.
  Because, even though move to \(\phi\) would bring \(\psi\), would have taken care of whether bringing in \(\psi\) was robust against possibility of not \(\psi\).

  That is, because claimed support for \(\psi\) is expectation, the agent ends up requiring that claimed support for \(\psi\).

  Right, okay, and then \autoref{assu:expect-nai} works to show that getting to \(\psi\) as result of reasoning wouldn't allow getting rid of claiming support for \(\psi\) as expectation.
\end{note}

\begin{note}
  The real key point, then, is that have to get \(\psi\) from \emph{expectation} of claimed support for \(\psi\).
  But, then, haven't got claimed support against possible defeaters.
  And, as a result, haven't got \(\psi\) against possible defeaters.

  That is, no account of \(\psi\) regardless because

  Wait, part of the point is that we've got claiming support for \(\psi\) as a possible defeater.
  Without this, there's not much of a worry about \(\psi\).
  So, this then bring in expectation.
  So, when move to \(\phi\), also get \(\psi\).
  \emph{And}, because CS\(\psi\) is expectation, this means that it's not `proper' claimed support.
  So, the expectation of claimed support does not provide grounds for \(\psi\) regardless of whether or not \(\psi\).
  Hence, not only does \(\phi\) bring \(\psi\), but the reasoning is not secured against \(\psi\) not holding up.
  Hence, not claiming support.
  Because, reasoning is not such that compatible with the possibility of \(\psi\) not being the case.

  For, if not \(\psi\) then move from claimed support for \(\phi\) to \(\phi\) is undercut.
  Because, only expectation of claimed support for \(\psi\).

  Move to \(\phi\) needs claiming support for \(\psi\) (by expectation), and in turn brings that \(\psi\).
  Possible that \(\psi\) is not the case as claiming support, but then as claimed support for \(\psi\) is only an expectation, no reasoning about whether this holds up when \(\psi\) is not the case.

  Possible that \(\lnot\psi\) is such that this breaks the claimed support for \(\phi\).
  In turn, breaks moving to \(\phi\).

  Hence, require \(\psi\) going by \(\phi\).
  And, if not \(\psi\) then no going by \(\phi\).

  I.e.\ if go to \(\phi\) then \(\psi\) and if not \(\psi\) then no going by \(\phi\), and hence no account of \(\psi\).
  So, only way of getting to \(\psi\) requires \(\psi\).

  So, issue is that not only does \(\phi\) bring \(\psi\).
  But also that the move to \(\phi\) is premised on \(\psi\) in such a way that doesn't work without \(\psi\), and therefore really is a requirement.
  With the additional observation being that because claimed support for \(\psi\) is an expectation, this step is not robust given the possibility of \(\lnot\psi\).
\end{note}

\begin{note}
  This ends up with something different to an expectation, though a variation really.
  For, the problem is with the particular application of the claimed support.
  Call this ??

  Point is only applies when going to \(\phi\).
\end{note}

\paragraph{Another}

\begin{note}
  \begin{itemize}
  \item Get that \(\psi\) having value \(v'\) is an expectation of the claimed support for \(\phi\) having value \(v\).
    \begin{itemize}
    \item This part is straightforward from definitions.
    \end{itemize}
  \item Not possible to consider possibility of \(\psi\) not having value \(v'\) while holding on to \(\phi\) having value \(v\).
  \end{itemize}

  \begin{itemize}
  \item It is not possible for the agent to appeal to \(\phi\) having value \(v\) from claimed support that \(\phi\) has value \(v\) while considering it possible that \(\psi\) does not have value \(v'\).
  \end{itemize}
  Given the above, it is not possible for the agent to \emph{claim support} for \(\psi\) having value \(v'\).
  For, such reasoning is not compatible with the possibility that \(\psi\) does not have value \(v'\).

  \begin{proof}
    Claimed support for \(\phi\) having value \(v\).
    \(\phi\) has value \(v\).
    In order to do this, position to claim support for \(\psi\) having value \(v'\).
    And, claimed support for \(\phi\) having value \(v\) `guarantees' claimed support for \(\psi\) having value \(v'\).
    For, granting \(\phi\) has value \(v\) is such that the claimed support is not \misled{}, and therefore, consider that claimed support for \(\psi\) having value \(v'\) would not be \misled{}, which is only the case if \(\psi\) has value \(v'\) (at present time).
    So, also \(\psi\) has value \(v'\).

    This comes from a and b.

    If \(\psi\) does not have value \(v'\), then moving to \(\phi\) is undercut, because claimed support for \(\phi\) having value \(v\) \emph{is} either \mom{}. (Stronger than mere possibility.)

    This comes from b.

    Considering the possibility leads agent to consider that claimed support for \(\phi\) having value \(v\) is not enough to move to \(\phi\) having value \(v\).
    Claimed support for \(\phi\) is such that it ensures that the agent has resources to deal with possible defeater of \(\psi\) not having value \(v'\).

    Hence, the result of reasoning is not compatible with possibility that the agent is \mom{} with respect to \(\psi\) having value \(v'\).
    Therefore, the result of reasoning is not an instance of claiming support.

    {
      \color{red}
      The really important thing to keep in mind here is that the problem isn't just from \(\psi\) not having value \(v'\).
      Rather, it's that if \(\psi\) does not have value \(v'\) then this reveals some problem in the claimed support for \(\phi\) having value \(v\).
      I.e.\ it's that appealing to the claimed support for \(\phi\) having value \(v\) `includes' \(\psi\) having value \(v'\) when moving to \(\phi\) having value \(v\).

      It is, of course, the case that for all claimed support:
      If \(\phi\) does not have value \(v\) then the claimed support for \(\phi\) having value \(v\) is either \mom{}.
      However, the problem highlighted here is distinct.
      At issue is not that the claimed support for \(\psi\) having value \(v'\) would be \mom{} if \(\psi\) does not have value \(v'\).
      Rather, the problem is that the way in which the agent claims support for \(\psi\) having value \(v'\) fails in a recognised way if \(\psi\) does not have value \(v'\).
    }
  \end{proof}

  Corollary here is that there is no way to remove \(\psi\) having value \(v'\) as an expectation of the claimed support by appeal to the claimed support for \(\psi\) having value \(v\) alone.

  This will be useful when we apply \nI{} to cases of interest.

  Still, the argument seems stronger, as it rests on interdependence.
  Given this, may wonder whether it is important that \(\psi\) having value \(v'\) is an expectation of the claimed support.

  First, some caution.
  Interdependence is key, but expectation allowed us to focus on this.
  The purpose of the initial assumptions is to ensure that \(\psi\) having value \(v'\) turns out to be an expectation, and hence requires some novel reasoning about the possible defeater.

  Still, it may be possible to strengthen this argument.
  It may be in some cases that the kind of interdependence is sufficient to require a retraction of previous considerations regarding \(\psi\) having value \(v'\).
  This would then allow the argument presented to apply not only instances of expectation, but also instances of retraction.

  We do not take a stand on this.
  This would require additional argumentation concerning the reasoning that goes in to claiming support.
  \nI{} is a sufficient condition, but it need not be the weakest sufficient condition.\nolinebreak
  \footnote{
    The weakest sufficient condition relies on making precise things we've left as open in assumptions.
  }
\end{note}

\begin{note}
  Put a different way:
  Interdependence is such that when move to \(\phi\), \(\psi\) also has to come with.
\end{note}

\begin{note}
  The converse does not hold.
  It need not be the case that the claimed support for \(\psi\) is such that requires claimed support for \(\phi\).

  This is important to note.
  For, general and specific.

  Should have an example here.
\end{note}



%%% Local Variables:
%%% mode: latex
%%% TeX-master: "master"
%%% End:
