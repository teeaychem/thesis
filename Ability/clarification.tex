\chapter{Clarification}
\label{cha:clar}

\begin{note}
  \autoref{cha:introduction} introduced the focal point of this document:
  The issue of whether a particular relation holds between answers to two distinct questions.
  Issue: \issueInclusion{}.
  Answers to why an agent concludes  --- answers to \qWhy{} --- constrained by how an agent concluded --- answers to \qHow{}.
  Our overall goal is to argue that there are counterexamples to \issueInclusion{}.

  The purpose of the present chapter is clarification.
  At the close of this chapter we will have variants of \qWhy{}, \qHow{}, and \issueInclusion{} and an understanding of how the variants relate to the initial questions and relation.

  Purpose of variants is to provide clarity.
  As a result, slight generalisation.
\end{note}

\begin{note}
  The chapter is split into two parts.
  In the first part we develop some foundations.
  Second part, foundations to develop variants.
  For each variant, link to initial question.

  \begin{itemize}
  \item
    Concluding \hfill \autoref{cha:clar:sec:CCC:pvp}
    \begin{itemize}
    \item
      What we are interested in.
    \end{itemize}
  \end{itemize}
\end{note}

\section{Concluding, concludes, conclusion}
\label{cha:clar:sec:CCC}

\begin{note}
  \qWhy{}, \qHow{}, and \issueInclusion{} are all stated with respect to concluding.
  Variations developed below will likewise stated with respect to concluding.
\end{note}

\begin{note}
  Our interest is in concluding.
  In this section, overview by considering the interaction between three terms:
  `concluding', `concludes', `conclusion'.

  Roughly, we hold:

  \begin{itemize}
  \item
    A conclusion of an agent is the state of a proposition being paired with a value by the agent.
  \item
    An event in which an agent concludes results in a conclusion of an agent.
  \item
    An event of concluding is such that an event in which an agent concludes is in progresses.
  \end{itemize}

  Hence, we understand `concluding' in terms of `concludes', and `concludes' in terms of `conclusion'.
  The characterisation of conclusions as proposition-value pairings is fundamental to the presentation of this document, and our initial focus will be on how such a pairing is understood.
  The events in which an agent concludes or is concluding are then understood in terms of what the event results in (`concludes') or is progressing towards (`concluding').

  In general, we understand a conclusion as the result of reasoning.
  Hence, an event in which an agent concludes is an event in which the agent reasons, and an event of concluding is an event of reasoning.

  Key distinction is that reasoning need result in the agent pairing a proposition with a value.
  \phantlabel{reasoning-vs-concluding-progressive}

  Indeed, we will often contrast concluding with reasoning.
  For, if the agent is concluding, then event in which the agent concludes is in progress.
  The event may be interrupted, and the agent may fail to conclude.
  However, if the event is not interrupted, will result in a conclusion.

  But contrast, the same need not be true of reasoning.
\end{note}

\subsection{Conclusions as proposition-value pairings}
\label{cha:clar:sec:CCC:pvp}

\begin{note}
  Basic assumption:

  \begin{assumption}
    \label{assu:concluding:pvp}
    For an agent \vAgent{}:

    \begin{itemize}
    \item
      A conclusion is a pairing of a proposition with a value, from \agpe{\vAgent{}'}.
    \end{itemize}

    Where:
    \begin{itemize}[noitemsep]
    \item
      A proposition is some state of affairs.
    \item
      A value captures an evaluation of a proposition.
    \end{itemize}
    \vspace{-\baselineskip}
  \end{assumption}

  Propositions are states of affairs, whether actual, possible, true, false, ideal, or regrettable.
  Values capture how the agent evaluates the state of affairs.
  Conclusion is pairing proposition with some value.

  The converse need not be the case; a proposition may be paired with some value from an \agpe{}, without the pairing being a conclusion.
  The role of \autoref{assu:concluding:pvp} is to capture what a conclusion amounts to.
\end{note}

\begin{note}
  Before discussing the details of \autoref{assu:concluding:pvp}, we present some examples of proposition-value pairings:

  \begin{itemize}
  \item
    The bag is heavy.\newline
    \mbox{ }\hfill%
    \(\pv{\text{The bag is heavy}}{\text{True}}\)
  \item
    The cyclist isn't avoiding potholes.\newline
    \mbox{ }\hfill%
    \(\pv{\text{The cyclist is avoiding potholes}}{\text{False}}\)
  \item I like bitter coffee.\newline
    \mbox{ }\hfill%
    \(\pv{\text{Bitter coffee}}{\text{Desirable}}\)
  \item
    It's too bad the balloon is deflating.\newline
    \mbox{ }\hfill%
    \(\pv{\text{The balloon is deflating}}{\text{Unfortunate}}\)
  \item
    The dog shouldn't be drinking from the pond.\newline
    \mbox{ }\hfill%
    \(\pv{\text{The dog isn't drinking from the pond}}{\text{Ought-to-be}}\)
  \item
    There is no chance \dots
    \mbox{ }\hfill%
    \(\pv{\text{\dots}}{\text{Improbable}}\)
  \end{itemize}

  Nothing in particular hangs on these examples.
  In particular, the relevant values may be restricted to `true' and `false' and the corresponding propositions expanded.
  For example:
  \begin{itemize}
  \item
    The dog shouldn't be drinking from the pond.\newline
    \mbox{ }\hfill%
    \(\pv{\text{The dog shouldn't be drinking from the pond}}{\text{True}}\)
  \end{itemize}

  However, odd, at least to me.
  Conclusion is about how bitter the coffee is, state of affairs is distinguished from the \agpe{} on the state of affairs.
\end{note}

\begin{note}
  Compound:
  \begin{itemize}
  \item I like how bitter the coffee is.\newline
    \mbox{ }\hfill%
    \(\pv{\text{The coffee is bitter}}{\text{True}}\)\newline
    \mbox{ }\hfill%
    \(\pv{\text{The coffee is bitter}}{\text{Desirable}}\)
  \end{itemize}

  Likewise, from the \agpe{}, it may be the case that the proposition that the dog is sleeping is evaluated as both true and desirable.
  For example, the agent may evaluate the proposition as true because the agent perceives the proposition to be the case.
  And, the agent may evaluate the proposition as desirable because the agent is planning to take the dog on a long walk later in the day.
\end{note}

\begin{note}
  We will make use of the following notation:
  \begin{notation}[Proposition-value pairs]
    \mbox{ }
    \vspace{-\baselineskip}
    \begin{itemize}
    \item
      Lower case letters of the Greek alphabet, \(\phi, \psi, \dots\) correspond to propositions, i.e.\ states of affairs.
    \item
      The Latin letter \(v\) corresponds to a some value.
      And in cases where more than one value may be relevant we will write \(v', v'' \dots\).
      However, \(v, v'\), and \(v''\), etc., may all correspond to the same value.
    \item
      For any proposition \(\phi\) and value \(v\), we abbreviate the pairing of \(\phi\) and \(v\) as \(\pv{\phi}{v}\).
    \end{itemize}
    \vspace{-\baselineskip}
  \end{notation}

  In general, any occurrence of \(v\) may be read as `true'.
  When speaking abstractly, nothing depends on particular value.


  % \begin{itemize}
  % \item
  %   On occasion we use \(\overline{v}\) to indicate some value other than \(v\) of the same type as \(v\) and \(v_{?}\) to indicate \(\phi\) is not paired with any proposition of the same type as \(v\).

  %   In summary, the notation on the left is in accordance with the sentence on the right with a default reading below:
  %   \begin{itemize}
  %   \item
  %     \(\pv{\phi}{v}\) \hfill \(\phi\) has value \(v\).%
  %     \newline
  %     \mbox{ }\hfill \(\phi\) is true.
  %   \item
  %     \(\pv{\phi}{\overline{v}}\) \hfill \(\phi\) has some value other than \(v\) of the same type.%
  %     \newline
  %     \mbox{ }\hfill \(\phi\) has some value other than true of the same type.%
  %     \newline
  %     \mbox{ }\hfill I.e.\ \(\phi\) is false.
  %   \item
  %     \(\pv{\phi}{v_{?}}\) \hfill \(\phi\) is not evaluated to have a value of type \(v\).%
  %     \newline
  %     \mbox{ }\hfill \(\phi\) is not evaluated as true or false.
  %   \end{itemize}
  % \end{itemize}
\end{note}

\begin{note}
  We have seen a few examples of what conclusions \emph{are} given \autoref{assu:concluding:pvp}.
  The motivating idea behind \autoref{assu:concluding:pvp} is that an agent's evaluation of a proposition is fundamental.

  In other words, the respective evaluations of a proposition as true and of a proposition as desirable may function analogously to beliefs and desires in folk-psychology.
  Indeed, an agent evaluating a proposition as true may be treated as equivalent to the agent believing the proposition is true, and the agent evaluating a proposition as desirable as equivalent to desiring the (truth of the) proposition.%
  \footnote{
    \label{fn:belief-is-difficult}
    Though I don't think belief is really so straightforward.

    Consider the Jeremy Goodman's example of three-horse race from~\textcite{Hawthorne:2016wv}:
    \begin{quote}
      Assume that horse A is more likely to win than horse B which in turn is more likely to win then horse C (so the probabilities of winning could be known to be 45, 28, 27\%).
      In this case it seems fine to say `I think horse A will win' or `I believe horse A will win'.%
      \mbox{ }\hfill\mbox{(\citeyear[1440]{Hawthorne:2016wv})}
    \end{quote}
    As \citeauthor{Hawthorne:2016wv} observe: ``[I]t is awful to say, in this case, `I think horse A will win but I don't believe it will'.''
    (\citeyear[1440, fn.17]{Hawthorne:2016wv})
  }

  If you are inclined to the folk-psychological picture, then the benefit of speaking in terms of proposition-value pairing from the \agpe{} is an easy way to talk about how things are from the \agpe{}, rather than recasting the \agpe{} in terms of the propositional attitudes the agent has.

  And, if you are not inclined to the folk-psychological picture then the sketched equivalence should suggest how to understand the way in which proposition-value pairing are fundamental.
  If an agent evaluates some proposition \(\phi\) as true then \(\phi\) is the case, from the \agpe{}, etc.
\end{note}

\begin{note}
  In particular, we distinguish proposition-value pairing from a proposition-value pairing \emph{given various assumptions}.
  When we speak of the proposition `Rover will fall asleep soon' pairing with the value `true', from an \agpe{}, then Rover \emph{will} fall asleep soon, from the \agpe{}.
  By contrast, if the agent has been entertaining the idea that Rover is tired then the agent is entertaining the idea that the proposition `Rover is tired' has the value `true', but we do not consider the proposition and valid paired.
\end{note}

\begin{note}
  \phantlabel{mention:concluding-non-factive}
  Further, we allow for the possibility that some proposition is paired with some value from the \agpe{}, though in reality things are otherwise.
  Hence, we do not require, for example, that an agent concludes \(\phi\) is true only if \(\phi\) is the case.

  Following an example of \citeauthor{Williams:1979wi}, an agent may conclude it is true that the stuff is gin, while in fact the stuff is petrol (\citeyear[18]{Williams:1979wi}).
  A slightly more involved case is given in the footnote to this sentence.%
  \footnote{
    An agent may conclude \(0.999\dots \ne 1\).
    And, the agent may, when concluding, hold themselves to have a conventional understanding of real numbers.

    The qualification is important, there are various interpretations under which \(0.999\dots \ne 1\), but (it is convention that) the Archimedean property holds for real numbers.

    Still, the agent may have failed to grasp the Archimedean property does not hold for real numbers, and so may reason that, though \(0.999\dots\) approaches \(1\), there must be \emph{some} difference between \(0.999\dots\) and \(1\) --- no matter how small --- and some difference between to things is sufficient to establish that they are not equal.
  }
\end{note}

\begin{note}
  We will not impose an explicit restrictions on which proposition may be paired with which values, nor on the circumstances under which propositions may be paired with values.

  However, we will try to keep things intuitive.
\end{note}

\subsection{Conclusions from \poP{1}}
\label{sec:pools-premises}

\begin{note}

  \begin{assumption}
    \label{assu:concluding:pools}
    For an agent \vAgent{}:

    \begin{enumerate}
    \item[\emph{If}:]
      \vAgent{} concludes \(\phi\) has value \(v\).
    \item[\emph{Then}:]
      \vAgent{} concludes \(\phi\) has value \(v\) from some \poP{} \(\Phi\).
    \end{enumerate}
    Where:
    \begin{itemize}
    \item
      A \poP{} is some collection of proposition-value pairings.
    \end{itemize}
    \vspace{-\baselineskip}
  \end{assumption}

  \autoref{assu:concluding:pools} is primarily a matter of convenience.
  In various cases it is intuitive that an agent concludes \(\phi\) has value \(v\) from some premises, and we will have interest in keeping track of which premises an agent concludes from.

  For a handful of brief examples, consider the following:

  \begin{itemize}[noitemsep]
  \item
    \emph{A} testified that \(\phi\) is true, so \(\phi\) is true.
  \item
    \(\phi\) would be satisfactory for every member of the group, so \(\phi\) ought to be the case.
  \item
    The song is produced by \emph{B}, so listening to it is desirable.
  \item
    The device reads `\(\phi\)' and is reliable, so \emph{not}-\(\phi\) is improbable.
  \end{itemize}

  Each sentence may be used to identify a conclusion, marked by `so' with some relevant premises.
  I doubt each sentence captures the full extent of the premises involved, but details regarding the premises will not be of interest.

  Further, it is not clear that concluding always involves specific premises.
  For example, consider the parallel between the~\citeauthor{Ramsey:1929tf} test for conditionals and a Fitch-style rule for conditional introduction in propositional logic.%
  \footnote{
    \textcite{Read:1995wf} describe the test as follows:

    \begin{quote}
      One should believe a conditional. `if \emph{A} then \emph{B}' if one would come to believe \emph{B} if one were to add A to one's stock of beliefs.%
      \mbox{ }\hfill\mbox{(\citeyear[47]{Read:1995wf})}
    \end{quote}

    A Fitch-style rule for conditional introduction in propositional logic is as follows --- Cf. (\cite[206]{Barwise:1999tu}), (\cite{Pelletier:2021vp}):
    \begin{center}
      \begin{fitch}
        \ftag{\scriptsize i}{\fa \fh P} & \\
        \ftag{\scriptsize }{\fa \fa \vdots} & \\
        \ftag{\scriptsize j}{\fa \fa Q} & \\
        \ftag{\scriptsize j+1}{\fa P \rightarrow Q} & \(\rightarrow\)\textbf{Intro:} \emph{i}--\emph{j} \\
      \end{fitch}
    \end{center}

    The rule states that at any point in a proof, one may assume \(P\), then, after deriving \(Q\) from the assumption of \(P\), one may discharge assumption of \(P\) and introduce the conditional \(P \rightarrow Q\).
    Note, the assumption of \(P\) on line \emph{i} corresponds to adding \(P\) to collection of propositions proved or assumed up to line \emph{i}, and hence to one's stock of beliefs.

    Now, both the \citeauthor{Ramsey:1929tf} test and the Fitch-style rule for conditional introduction describe a clear \emph{processes} for which a conditional is a conclusion, but there may not be any \emph{premises} associated with the conclusion.
    To illustrate, consider the following derivation which does not involve any premises:

    \begin{center}
      \begin{fitch}
        \ftag{\scriptsize 1}{\fa \fh P \land Q} & \\
        \ftag{\scriptsize 2}{\fa \fa Q} & \(\land\)\textbf{Elim:} \emph{1} \\
        \ftag{\scriptsize 3}{\fa (P \land Q) \rightarrow Q} & \(\rightarrow\)\textbf{Intro:} \emph{1}--\emph{2} \\
      \end{fitch}
    \end{center}
  }
  However, \autoref{assu:concluding:pools} splits the difference by allowing the relevant \poP{} to be the empty set.
\end{note}

\begin{note}
  \begin{notation}[\poP{3}]
    \mbox{ }
    \vspace{-\baselineskip}
    \begin{itemize}
    \item
      We use upper case letters of the Greek alphabet, \(\Phi, \Psi, \dots\), to refer to collections of proposition-value pairings.

      For example, \(\Phi\) may contain \(\pv{\phi}{v}, \pv{\psi}{v'}\) and so on, or \(\Phi\) may be empty.

      \(\Phi\) will often be a \poP{} relative to some conclusion.
    \item
      \(\pvp{\phi}{v}{\Phi}\) abbreviates the association of some proposition-value pair \(\pv{\phi}{v}\) with a collection of proposition-value pairings \(\Phi\).

      Usually \(\pvp{\phi}{v}{\Phi}\) will abbreviate \(\pv{\phi}{v}\) being a conclusion from the \poP{} \(\Phi\).
    \end{itemize}
  \end{notation}
\end{note}

\subsection{Concludes and concluding}
\label{cha:clar:sec:CCC:c-and-c}

\begin{note}
  \autoref{cha:clar:sec:CCC:pvp} made the assumption that conclusions are pairings of a proposition \(\phi\) with a value \(v\), from the \agpe{}.

  In this section we turn to two events concerning conclusions:
  \begin{itemize}
  \item
    The event in which an agent concludes.
  \item
    The event in which an agent is concluding.
  \end{itemize}
  This section will focus on the event in which an agent concludes.
\end{note}

\begin{note}
  We start with an assumption regarding an event in which an agent concludes:

  \begin{assumption}[Concludes is inclusive]
    \label{assu:concluding:span}
    For an agent \vAgent{}, proposition-value pair \(\pv{\phi}{v}\), and \poP{} \(\Phi\):

    \begin{itemize}
    \item
      The event in which an agent concludes \(\pv{\phi}{v}\) from \(\Phi\) spans the agent moving from the premises of \(\Phi\) to the conclusion that \(\phi\) has value \(v\).
    \end{itemize}
    \vspace{-\baselineskip}
  \end{assumption}

  The key feature of \autoref{assu:concluding:span}  is the \emph{scope} of concluding.
  The event of an agent is not limited to the agent pairing proposition \(\phi\) with value \(v\), but includes how the agent comes to pair \(\phi\) with \(v\).

  \autoref{assu:concluding:span} implicitly characterises an event in which an agent concludes via what a conclusion (and \poP{}) is.

  A slightly more natural paraphrase of \autoref{assu:concluding:span} states that the event in which an agent concludes \(\pv{\phi}{v}\) from \(\Phi\) spans the agents \emph{reasoning} to \(\pv{\phi}{v}\) from \(\Phi\).

  Indeed, we will continue to speak in terms of reasoning to ease things a little.
\end{note}

\begin{note}
  Rather than being an assumption proper, I'm inclined to think that \autoref{assu:concluding:span} identifies a particular sense of the term `concludes'.
  The sense of concludes which \emph{includes} the relevant reasoning leading up to some conclusion and does not focus only on the more-or-less instantaneous event in which the agent forms the relevant conclusion.

  In this respect, \autoref{assu:concluding:span} is less of an assumption and more of an indication about the sense in which I will use `concludes'.
\end{note}

\begin{note}
  To illustrate, we contrast to instances of `concludes'.
  The first is from \textcite{Gardner:1986wp} and the second from \textcite{Bratman:1979aa}.

  The following passage from \citeauthor{Gardner:1986wp}'s discussion of Newcomb's problem:

  \begin{quote}
    A large number of those who recommended taking only the second box performed the expected-value calculation and concluded that, provided the probability that the Being was correct was at least .5005, they would take only the second box.%
    \mbox{ }\hfill\mbox{(\citeyear[166]{Gardner:1986wp})}
  \end{quote}

  On my reading, \citeauthor{Gardner:1986wp} distinguishes the performing the expected-value calculation from the conclusion that they would take the box given a certain probability.

  In contrast to the passage from \citeauthor{Gardner:1986wp}, consider the following passage from \citeyear{Bratman:1979aa}:

  \begin{quote}
    Sam thinks both that in certain respects his drinking would be, \emph{prima facie}, best, and that in certain other respects his abstaining would be, \emph{prima facie}, best.
    Weighing these conflicting considerations he concludes that it would be best to abstain, rather than drink.%
    \mbox{ }\hfill\mbox{(\citeyear[156]{Bratman:1979aa})}
  \end{quote}

  It seems to me \citeauthor{Bratman:1979aa} describes includes the weighing of conflicting considerations in the event of concluding --- there is no `and' to distinguish the event of Sam weighting the conflicting considerations and the event in which Sam concludes that it would be best to abstain, rather than drink.
\end{note}

\begin{note}
  In addition, it seems to me that whether or not one defaults to thinking of a more-or-less instantaneous event or an inclusive event comes apart in multi-agent cases.

  For example, consider Sam and Max working on some problem \(p\) together.
  Sam and Max trade ideas back and forth, and settle on an answer \(a\).

  Consider the events described by the following sentence:
  \begin{enumerate}[label=\arabic*., ref=(\arabic*)]
  \item
    \label{Cing:SandM:j}
    Sam and Max (jointly) conclude \(a\) is an answer to problem \(p\).
  \item
    \label{Cing:SandM:s}
    Sam concludes \(a\) is an answer to problem \(p\).
  \item
    \label{Cing:SandM:m}
    Max concludes \(a\) is an answer to problem \(p\).
  \end{enumerate}

  My default is to consider the events captured by \ref{Cing:SandM:s} and \ref{Cing:SandM:m} in line with \citeauthor{Gardner:1986wp}.
  When considering Sam or Max in isolation, the event in which Sam (or Max) concludes more-or-less instantaneous event.
  Though, the relevant events may be expanded to include the trading of ideas by appending `by trading ideas with Max/Sam'.

  By contrast, I read \ref{Cing:SandM:j} in line with \citeauthor{Bratman:1979aa}.
  When considering Sam and Max as a pair, in which a non-distributive reading of `conclude' is forced by the parenthetical `jointly', the event includes the trading of ideas back and forth.
  Though, the relevant event includes sub-event in which Sam and Max separately evaluate that \(a\) is an answer to problem \(p\).
\end{note}

\begin{note}
  Our understanding of `concluding' is that an event in which an agent is concluding some proposition-value pair \(\pv{\phi}{v}\) from some \poP{} \(\Phi\) is just an event such that the event in which the agent concludes \(\pv{\phi}{v}\) from \(\Phi\) is in progress.

  In other worlds, an event of concluding involves some reasoning from some sub-set of the \poP{} \(\Phi\).

  In this respect, an event of concluding need not involve the agent pairing \(\phi\) with \(v\), it need only be the case that if the relevant event went on uninterrupted, then event would develop into an event in which the agent concludes \(\pv{\phi}{v}\) from \(\Phi\), and hence pairs \(\phi\) with \(v\).

  As noted on \autopageref{reasoning-vs-concluding-progressive}, an important distinction between reasoning and concluding is that we don't get this.
  {
    \color{red} Up to here.
  }
\end{note}

\subsubsection{Reasoning}
\label{sec:reasoning-1}

\begin{note}
  We have assumed that concluding is an instance of reasoning, and more specifically that concluding involves an agent reasoning from some \poP{} to some conclusion, where both premises and conclusions are understood in terms of proposition-value pairings.
  In this respect, the assumptions we have made place constraints on certain instances of reasoning.
  However, aside from understanding reasoning in terms of proposition-value pairings, we do not place further constraints on reasoning.

  Note, the absence of placed constraints does not imply that reasoning is unconstrained.
  Rather, we adopt a neutral stance on what may count as instance of reasoning.
\end{note}

\begin{note}[No constraints]
  To illustrate, consider the contrast between \citeauthor{Broome:2013aa}'s (\citeyear{Broome:2013aa}) rule following account of reasoning and \citeauthor{Wedgwood:2006ui}'s (\citeyear{Wedgwood:2006ui}) reason-based account of reasoning.

  \citeauthor{Broome:2013aa}'s rule following account of reasoning is unconstrained in terms of the rules an agent may follow.
  This aspect  of \citeauthor{Broome:2013aa}'s account is highlighted in the following passage:

  \begin{quote}
    [S]hould we exclude this bizarre rule: from the proposition that it is raining and the proposition that if it is raining the snow will melt, to derive the proposition that you hear trumpets.
    Following this rule would lead you to believe you hear trumpets when you believe it is raining and believe that if it is raining the snow will melt.
    If you did this, should we count you as reasoning?

    I think we should.
    If you derive this conclusion by operating on the premises, following the rule, we should count you as reasoning.
    \dots
    I think we should not impose a limit on rules.%
    \mbox{}\hfill\mbox{(\citeyear[233]{Broome:2013aa})}
  \end{quote}

  By constraint, \citeauthor{Wedgwood:2006ui}'s reason-based account of reasoning:

  \begin{quote}
    Reasoning, I shall assume, is the process of \emph{revising ones beliefs or intentions, for a reason}.%
    \mbox{ }\hfill\mbox{(\citeyear[600]{Wedgwood:2006ui})}
  \end{quote}

  \citeauthor{Wedgwood:2006ui} understands reasons in terms of intelligibility.%
  \footnote{
    Strictly, \citeauthor{Wedgwood:2006ui} understanding reasoning in terms of dispositions that respond to what `rationalizes' (\citeyear[672]{Wedgwood:2006ui}) and \citeauthor{Wedgwood:2006ui} considers `rationalizes' and  `makes intelligible' as equivalent.
    Therefore, the following conditional, states a sufficient rather than necessary condition which leads to \citeauthor{Wedgwood:2006ui}'s assumption, rather than being an expansion of what \citeauthor{Wedgwood:2006ui}'s assumption amounts to.
    I find this particularly confusion.
  }

  \begin{quote}
    If a set of antecedent mental states makes it rational for one to form a new belief or intention, then those antecedent mental states are surely of a suitable type and content so that it is \emph{intelligible} that they could represent one's reason for forming that belief or intention.\newline
    \mbox{ }\hfill\mbox{(\citeyear[662]{Wedgwood:2006ui})}
  \end{quote}

  Now, perhaps the presence of the rule in \citeauthor{Broome:2013aa}'s example makes it intelligible that the agent concludes that they hear trumpets from the relevant premises.
  However, there is also a clear sense in which the reasoning in \citeauthor{Broome:2013aa}'s example in \emph{unintelligible}.

  \citeauthor{Wedgwood:2006ui} provides the following contrasts to clarify their understanding of intelligibility:

  \begin{quote}
    [T]he belief that the \emph{Oxford Dictionary of National Biography} says that Hume died in 1776 seems a mental state of a suitable type and content so that it could intelligibly represent one's reason for believing that Hume died in 1776.
    On the other hand, it is not (except in the presence of some rather extraordinary background beliefs) a mental state of a suitable type and content so that it could intelligibly represent one's reason for believing that every even number is the sum of two primes.%
    \mbox{ }\hfill\mbox{(\citeyear[662]{Wedgwood:2006ui})}
  \end{quote}

  Does the rules that the agent follows in \citeauthor{Broome:2013aa}'s example count as a sufficiently extraordinary background belief?
  Indeed, \citeauthor{Broome:2013aa} do not require that, in general, an agent has beliefs concerning of the rules they follow in reasoning (\citeyear[Cf.][\S13.2]{Broome:2013aa}).
\end{note}

\begin{note}
  By refraining from placing any further constraints on reasoning we remain neutral on whether an unconstrained approach to reasoning is line with \citeauthor{Broome:2013aa} is correct, or whether a constrained approach to reasoning in line with \citeauthor{Wedgwood:2006ui} is correct.

  Still, in general, we aim to consider instances of reasoning (and concluding) which are intuitive.
  Hence, we hope that the instances of reasoning we consider are compatible with whatever way reasoning is understood.
\end{note}

\subsection{Attitudes}

\begin{note}
  Conclusion of reasoning is a proposition-value pairing.
  In general, this does not provide information about the attitude that the agent holds toward the proposition.

  For example, conclude that \(\phi\) has value `true'.
  Conclusion may amount to knowledge, belief, or some other veridical attitude.

  We will have little interest in propositional attitudes.
  Our interest is in concluding, and it is rarely the case that one concludes that they have some attitude toward a proposition.
  Conclude that the cat is on the mat, do not conclude that I believe the cat is on the mat.

  Set aside interest in attitudes when concluding.
  Interest will be in proposition-value-premises pairings from \agpe{}.
\end{note}

\begin{note}
  Some difficulty in understanding answers to \qWhy{} and agent concludes.
  Evaluations are not attitudes.
  However, attitude entail evaluations.
\end{note}

\begin{note}
  \citeauthor{Davidson:1963aa}, reasons in terms of attitudes.

  \begin{quote}
    \emph{R} is a primary reason why an agent performed the action \emph{A} under the description \emph{d} only if \emph{R} consists of a pro attitude of the agent toward actions with a certain property, and a belief of the agent that \emph{A}, under the description \emph{d}, has that property.%
    \mbox{ }\hfill\mbox{(\citeyear[687]{Davidson:1963aa})}
  \end{quote}

  However, foundation of (pro-)attitude is proposition-value pairing.
  So, pro-attitude, evaluation with something like desire, and belief with evaluation of true.

  In this respect, \citeauthor{Davidson:1963aa} is agent neutral.
  Primary reason is not seen from agent's point of view.
  Instead, from our point of view.
  However, entailed (or corresponding) account from agent's point of view.%
  \footnote{
    Clearer given principle \citeauthor{Smith:1987vk} derives from the Humean Theory of Motivation, as captured by~\ref{Smithh:HtM:2}:
    \begin{quote}
      \begin{enumerate}[label=\textsc{P1}., ref=(\textsc{P1})]
      \item
        \label{Smithh:HtM:2}
        Agent A at t has a motivating reason to \(\phi\) only if there is some \(\psi\) such that, at t, A desires to \(\psi\) and believes that were he to \(\phi\) he would \(\psi\).%
        \mbox{ }\hfill\mbox{(\citeyear[36]{Smith:1987vk})}
      \end{enumerate}
    \end{quote}

    Understand desire of A to \(\psi\) in terms of the evaluation of \(\psi\) as desirable, and the belief as evaluating if do \(\phi\) then \(\psi\).

    \citeauthor{Smith:1987vk} suggests comparison to \textcite{Davidson:1963aa}.
    Note, in particular \citeauthor{Smith:1987vk}'s use of `motivating reason' is equivalent to \citeauthor{Davidson:1963aa}'s use of `primary reason' rather than `reason'.
  }

  So, \citeauthor{Davidson:1963aa}'s constraint in terms of cause is a little more delicate.
\end{note}

\begin{note}
  Hope this is straightforward.%
  \footnote{
    Discussion by \citeauthor{Collins:1997wn} and \citeauthor{Dancy:2000aa}.
  }
  Contrast is possible, from agent's point of view it is that agent has relevant propositional attitudes.

  Consider the follow passage from \citeauthor{Hume:2011aa}'s \hyperlink{cite.Hume:2011aa}{Treatise}:

  \begin{quote}
    ’Tis also obvious, that this emotion rests not here, but making us cast our view on every side, comprehends whatever objects are connected with its original one by the relation of cause and effect.
    Here then reasoning takes place to discover this relation; and according as our reasoning varies, our actions receive a subsequent variation.
    But ’tis evident in this case, that the impulse arises not from reason, but is only directed by it.
    ’Tis from the prospect of pain or pleasure that the aversion or propensity arises towards any object: And these emotions extend themselves to the causes and effects of that object, as they are pointed out to us by reason and experience.%
    \mbox{ }\hfill\mbox{(\hyperlink{cite.Hume:2011aa}{T.2.3.3})}
  \end{quote}

  The passage captures means-end \emph{reasoning}.
  Evaluate an object according to the prospect of pain or pleasure.
\end{note}

%%% Local Variables:
%%% mode: latex
%%% TeX-master: "master"
%%% End:
