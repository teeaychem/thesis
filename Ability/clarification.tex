\chapter{Conclusions}
\label{cha:clar}

\begin{note}
  \autoref{cha:intro} introduced the focal point of this document:
  \issueInclusion{}, which holds answers to \qWhy{} are constrained by answers to \qHow{}.
  Our overall goal is to argue that there are counterexamples to \issueInclusion{}.

  The purpose of the present chapter is clarification.
  At the close of this chapter we will have variants of \qWhy{}, \qHow{}, and \issueInclusion{} and an understanding of how the variants relate to the initial questions and relation.

  Purpose of variants is to provide clarity.
  As a result, slight generalisation.
\end{note}

\begin{note}
  The chapter is split into two parts.
  In the first part we develop some foundations.
  Second part, foundations to develop variants.
  For each variant, link to initial question.

  \begin{enumerate}[label=, leftmargin=0pt]
  \item
    \TOCLine{cha:clar:sec:Cons}

    Conclusions.
    Proposition-value pair and \pool{1}.
  \item
    \TOCLine{cha:clar:sec:Concls}

    Event in which an agent forms a conclusion.
    Specific event.
    Link to reasoning.
  \item
    \TOCLine{cha:clar:sec:Cing}

    Progressive.
    Discussion of the progressive.
    Important assumption `\assuPP{}'.
  \end{enumerate}
\end{note}

\begin{note}
  Definitions.
  Specifications.

  Definition when introducing a new term.
  Specification when specifying a particular sense of a pre-existing term.

  Propositions, and observations.
  Proposition for things which are helpful and important.
  Observations for things which are not important, but may be helpful for understanding definitions or specifications.
  Argue for propositions, and motivate observations.
\end{note}

\section{Conclusions}
\label{cha:clar:sec:Cons}

\begin{note}
  In this section, overview by considering the relation between three terms:
  `conclusion', `concludes', and `concluding'.

  Roughly, we hold:

  \begin{itemize}
  \item
    A conclusion of an agent is the state of a proposition being paired with a value by the agent.
  \item
    An event in which an agent concludes results in a conclusion of an agent.
  \item
    An event of concluding is such that an event in which an agent concludes is in progresses.
  \end{itemize}

  Hence, we understand `concluding' in terms of `concludes', and `concludes' in terms of `conclusion'.
\end{note}

% \begin{note}
%   The characterisation of conclusions as proposition-value pairs is fundamental to the presentation of this document, and our initial focus will be on how such a pairing is understood.
%   The events in which an agent concludes or is concluding are then understood in terms of what the event results in (`concludes') or is progressing towards (`concluding').

%   In general, we understand a conclusion as the result of reasoning.
%   Hence, an event in which an agent concludes is an event in which the agent reasons, and an event of concluding is an event of reasoning.

%   Key distinction is that reasoning need result in the agent pairing a proposition with a value.
%   \phantlabel{reasoning-vs-concluding-progressive}

%   Indeed, we will often contrast concluding with reasoning.
%   For, if the agent is concluding, then event in which the agent concludes is in progress.
%   The event may be interrupted, and the agent may fail to conclude.
%   However, if the event is not interrupted, will result in a conclusion.

%   But contrast, the same need not be true of reasoning.
% \end{note}

\subsection{Conclusions as proposition-value pairs}
\label{cha:clar:sec:Cons:pvp}

\begin{note}
  Basic idea:

  \begin{idea}[Conclusions are Proposition-value pairs]
    \label{assu:concluding:pvp}
    For an agent \vAgent{}:

    \begin{itemize}
    \item
      A conclusion is a pairing of a proposition with a value.
    \end{itemize}

    Where:
    \begin{enumerate}[noitemsep, label=]
    \item
      \begin{itemize}[noitemsep]
      \item
        A proposition is some state of affairs.
      \item
        A value captures the \agpe{} on the state of affairs.
      \end{itemize}
    \end{enumerate}
    \vspace{-\baselineskip}
  \end{idea}

  The role of \autoref{assu:concluding:pvp} is to capture what a conclusion amounts to.

  Propositions are states of affairs, whether actual, possible, true, false, ideal, or regrettable.
  Values captures an \agpe{} on the state of affairs,%
  \footnote{
    Or the way in which the agent evaluates the state of affairs.
  }
  whether being actual, being possible, being true, being false, being idea, or being regrettable.

  \begin{notation}[Proposition-value pairs]
    \begin{itemize}
    \item
      Use san serif font for propositions and values.
      For example \prop{The hedge is shaped like a squirrel} is the proposition which captures the state of affairs in which there's some hedge which is shaped like a squirrel.
      And, \val{True} is the value `True'.
    \item
      We abbreviate some proposition-value pair with angled brackets as so:
      \(\pv{\prop{Proposition}}{\val{Value}}\)
    \end{itemize}
    \vspace{-\baselineskip}
  \end{notation}

  Here are some examples:

  \begin{enumerate}[label=\arabic*., ref=(\arabic*),noitemsep, series=propValExC]
  \item
    The bag is heavy.%
    \hfill%
    \(\pv{\prop{The bag is heavy}}{\val{True}}\)
  \item
    The room isn't full.%
    \hfill%
    \(\pv{\prop{The room is full}}{\val{False}}\)
  \item
    \label{pvEx:bC}
    I like bitter coffee.%
    \hfill%
    \(\pv{\prop{Bitter coffee}}{\val{Desirable}}\)
  \item
    I should go to work.%
    \hfill%
    \(\pv{\prop{I go to work}}{\val{Ought}}\)
  \item
    There's no chance they find the%
    \hfill%
    \(\pv{\prop{The needle is found}}{\val{Unlikely}}\)
  \item
    It's too bad the balloon deflated.%
    \hfill%
    \(\pv{\prop{The balloon deflated}}{\val{Unfortunate}}\)
  \end{enumerate}

  Nothing of importance rests on the value chosen.
  Indeed, the relevant values may be restricted to \val{True} and \val{False}, which we understand as synonyms for \val{Actual} and \val{Not actual}.
  For example, \ref{pvEx:bC} may be captured by \(\pv{\prop{I like bitter coffee}}{\val{True}}\).

  However, a range of values allows us to easily capture different perspectives on the same state of affairs.
  For example:
  \begin{enumerate}[label=\arabic*., ref=(\arabic*),noitemsep, resume*=propValExC]
  \item I like how bitter the coffee is.%
    \hfill%
    \(\pv{\prop{The coffee is bitter}}{\val{True}}\)\newline
    \hfill%
    \(\pv{\prop{The coffee is bitter}}{\val{Desirable}}\)
  \end{enumerate}
\end{note}

\begin{note}
  In particular, we distinguish proposition-value pairing from a proposition-value pairing \emph{given various assumptions}.
  When we speak of the proposition \prop{Rover will fall asleep soon} pairing with the value \val{True}, then Rover \emph{will} fall asleep soon, for the agent.
  By contrast, if the agent has been entertaining the idea that Rover is tired then the agent is entertaining the idea that the proposition \prop{Rover is tired} has the value \val{True}, but we do not consider the proposition and valid paired.
\end{note}

\begin{note}
  \phantlabel{mention:concluding-non-factive}
  Further, we allow for the possibility that some proposition is paired with some value for the agent, though in reality things are otherwise.
  Hence, we do not require, for example, that an agent concludes \(\phi\) is true only if \(\phi\) is the case.

  Following an example of \citeauthor{Williams:1979wi}, an agent may conclude it is true that the stuff is gin, while in fact the stuff is petrol (\citeyear[18]{Williams:1979wi}).
  A slightly more involved case is given in the footnote to this sentence.%
  \footnote{
    An agent may conclude \(0.999\dots \ne 1\).
    And, the agent may, when concluding, hold themselves to have a conventional understanding of real numbers.

    The qualification is important, there are various interpretations under which \(0.999\dots \ne 1\), but (it is convention that) the Archimedean property holds for real numbers.

    Still, the agent may have failed to grasp the Archimedean property does not hold for real numbers, and so may reason that, though \(0.999\dots\) approaches \(1\), there must be \emph{some} difference between \(0.999\dots\) and \(1\) --- no matter how small --- and some difference between to things is sufficient to establish that they are not equal.
  }
\end{note}


\begin{note}
  % We have seen a few examples of what conclusions \emph{are} given \autoref{assu:concluding:pvp}.
  % The motivating idea behind \autoref{assu:concluding:pvp} is that an agent's evaluation of a proposition is fundamental.

  % In other words, the respective evaluations of a proposition as true and of a proposition as desirable may function analogously to beliefs and desires in folk-psychology.
  % Indeed,
  In general, this does not provide information about the attitude that the agent holds toward the proposition.
  For example, conclude that \(\phi\) has value \val{True}.
  Conclusion may amount to knowledge, belief.
  

  An agent evaluating a proposition as true may be treated as capturing relevant aspects of the the agent believing or desiring the proposition.%
  \footnote{
    % \mbox{ }\hfill\mbox{(\hyperlink{cite.Hume:2011aa}{T.2.3.3})}
    \nocite{Hume:2011aa}
    Clearer given principle \citeauthor{Smith:1987vk} derives from the Humean Theory of Motivation, as captured by~\ref{Smithh:HtM:2}:
    \begin{quote}
      \begin{enumerate}[label=\textsc{P1}., ref=(\textsc{P1})]
      \item
        \label{Smithh:HtM:2}
        Agent A at t has a motivating reason to \(\phi\) only if there is some \(\psi\) such that, at t, A desires to \(\psi\) and believes that were he to \(\phi\) he would \(\psi\).%
        \mbox{ }\hfill\mbox{(\citeyear[36]{Smith:1987vk})}
      \end{enumerate}
    \end{quote}

    Understand desire of A to \(\psi\) in terms of the evaluation of \(\psi\) as desirable, and the belief as evaluating if do \(\phi\) then \(\psi\).

    \citeauthor{Smith:1987vk} suggests comparison to \textcite{Davidson:1963aa}.
    Note, in particular \citeauthor{Smith:1987vk}'s use of `motivating reason' is equivalent to \citeauthor{Davidson:1963aa}'s use of `primary reason' rather than `reason'.
  }
  \(^{,}\)%
  \footnote{
    \label{fn:belief-is-difficult}
    Though I don't think belief is straightforward.

    Consider the Jeremy Goodman's example of three-horse race:
    \begin{quote}
      Assume that horse A is more likely to win than horse B which in turn is more likely to win then horse C (so the probabilities of winning could be known to be 45, 28, 27\%).
      In this case it seems fine to say `I think horse A will win' or `I believe horse A will win'.%
      \mbox{ }\hfill\mbox{(\cite[1440]{Hawthorne:2016wv})}
    \end{quote}
    As \citeauthor{Hawthorne:2016wv} observe: \textquote{[I]t is awful to say, in this case, `I think horse A will win but I don't believe it will'.}
    (\citeyear[1440, fn.17]{Hawthorne:2016wv})
  }

  If you are inclined to the folk-psychological picture, then the benefit of speaking in terms of proposition-value pairs for the agent is an easy way to talk about how things are for the agent, rather than recasting in terms of propositional attitudes the agent has.

  And, if you are not inclined to the folk-psychological picture then the sketched equivalence should suggest how to understand the way in which proposition-value pairing are fundamental.
  If an agent evaluates some proposition \(\phi\) as true then \(\phi\) is the case, for the agent, etc.
\end{note}

\begin{note}
  \begin{notation}[Proposition-value pairs]
    \mbox{ }
    \vspace{-\baselineskip}
    \begin{itemize}
    \item
      Lower case letters of the Greek alphabet, \(\phi, \psi, \dots\) correspond to propositions, i.e.\ states of affairs.
    \item
      The Latin letter \(v\) corresponds to a some value.
      And in cases where more than one value may be relevant we will write \(v', v'' \dots\).

      (However, \(v, v'\), and \(v''\), etc., may all correspond to the same value.)
    \end{itemize}
    \vspace{-\baselineskip}
  \end{notation}
\end{note}



\subsection{Conclusions from \pool{1}}
\label{sec:pools-premises}

\begin{note}

  \begin{definition}[\pool{3}]
    \label{def:pools}
    \cenLine{
      \begin{itemize*}[noitemsep, label=\(\circ\)]
      \item
        Agent: \vAgent{}
      \item
        Proposition: \(\phi\)
      \item
        Value: \(v\)
      \item
        \mbox{ }
      \end{itemize*}
    }

    \begin{itemize}
    \item
      A \emph{\pool{}} is a collection of proposition-value pairs.
    \item
      For every \(\pv{\phi}{v}\) in \pool{}, the agent evaluates \(\phi\) as having value \(v\).
    \end{itemize}
    \vspace{-\baselineskip}
  \end{definition}

  \begin{assumption}[\pool{3}]
    \label{assu:concluding:pools}
    \cenLine{
      \begin{itemize*}[noitemsep, label=\(\circ\)]
      \item
        Agent: \vAgent{}
      \item
        Proposition: \(\phi\)
      \item
        Value: \(v\)
      \item
        \mbox{ }
      \end{itemize*}
    }

    \begin{itemize}[noitemsep]
    \item
      \begin{enumerate}
      \item[\emph{If}:]
        \vAgent{} concludes \(\phi\) has value \(v\).
      \item[\emph{Then}:]
        \vAgent{} concludes \(\phi\) has value \(v\) from some \pool{} \(\Phi\).
      \end{enumerate}
    \end{itemize}
    \vspace{-\baselineskip}
  \end{assumption}

  \autoref{assu:concluding:pools} functions to secure a simple way to describe where conclusions come from.
  An \agpe{}, the way in which things are for the agent.
  Proposition-value pairs.
  \pool{} is a collection of proposition-value pairs, and so captures the relevant stuff.

  However, this does not assume that conclusions always follow from something.
  \pool{} is a collection, and may be empty.%
  \footnote{
    To motivate, consider the parallel between the~\citeauthor{Ramsey:1929tf} test for conditionals and a Fitch-style rule for conditional introduction in propositional logic.
    \textcite{Read:1995wf} describe the test as follows:

    \begin{quote}
      One should believe a conditional. `if \emph{A} then \emph{B}' if one would come to believe \emph{B} if one were to add A to one's stock of beliefs.%
      \mbox{ }\hfill\mbox{(\citeyear[47]{Read:1995wf})}
    \end{quote}

    A Fitch-style rule for conditional introduction in propositional logic is as follows on the left, which an instance on the right --- cf. (\cite[206]{Barwise:1999tu}), (\cite{Pelletier:2021vp}):

    \begin{center}
      \begin{fitch}
        \ftag{\text{\scriptsize \emph{i}}}{\fa \fh P} & \\
        \ftag{\text{\scriptsize }}{\fa \fa \vdots} & \\
        \ftag{\text{\scriptsize \emph{j}}}{\fa \fa Q} & \\
        \ftag{\text{\scriptsize \emph{j+1}}}{\fa P \rightarrow Q} & \(\rightarrow\)\textbf{Intro:}\emph{i}--\emph{j} \\
      \end{fitch}%
      \hfil%
      \begin{fitch}
        \ftag{\text{\scriptsize 1}}{\fa \fh P \land Q} & \\
        \ftag{\text{\scriptsize 2}}{\fa \fa Q} & \(\land\)\textbf{Elim:}\emph{1} \\
        \ftag{\text{\scriptsize 3}}{\fa (P \land Q) \rightarrow Q} & \(\rightarrow\)\textbf{Intro:}\emph{1}--\emph{2} \\
      \end{fitch}
    \end{center}

    The rule states that at any point in a proof, one may assume \(P\), then, after deriving \(Q\) from the assumption of \(P\), one may discharge assumption of \(P\) and introduce the conditional \(P \rightarrow Q\).
    Note, the assumption of \(P\) on line \emph{i} corresponds to adding \(P\) to collection of propositions proved or assumed up to line \emph{i}, and hence to one's stock of beliefs.

    Now, both the \citeauthor{Ramsey:1929tf} test and the Fitch-style rule for conditional introduction describe a clear \emph{processes} for which a conditional is a conclusion, but there may not be any \emph{premises} associated with the conclusion.
    Observe, the instance derivation does not involve any premises.
  }


  For a handful of brief examples, consider the following:

  \begin{itemize}[noitemsep]
  \item
    \emph{A} testified that \(\phi\) is true, so \(\phi\) is true.
  \item
    \(\phi\) would be nice, so \(\phi\) ought to be the case.
  \item
    The song is produced by \emph{B}, so listening to it is desirable.
  \item
    The device reads `\(\phi\)' and is reliable, so \emph{not}-\(\phi\) is improbable.
  \end{itemize}

  Each sentence may be used to identify a conclusion, marked by `so' with some \pool{}.
  I doubt each sentence captures the full extent of the premises involved, but details regarding the premises will not be of interest.
\end{note}

\begin{note}
  \begin{notation}[\pool{3}]
    \mbox{ }
    \vspace{-\baselineskip}
    \begin{itemize}
    \item
      We use upper case letters of the Greek alphabet, \(\Phi, \Psi, \dots\), to refer to collections of proposition-value pairs.

      For example, \(\Phi\) may contain \(\pv{\phi}{v}, \pv{\psi}{v'}\) and so on, or \(\Phi\) may be empty.

    \item
      \(\pvp{\phi}{v}{\Phi}\) abbreviates the association of some proposition-value pair \(\pv{\phi}{v}\) with a collection of proposition-value pairs \(\Phi\).
    \end{itemize}
  \end{notation}
\end{note}



\section{Concludes}
\label{cha:clar:sec:Concls}

\begin{note}
  The way an event in which some agent concludes some proposition has some value happened

  Event in which an agent concludes involves pairing proposition with value.
  Of interest is scope of event.

  The event is not limited to the agent pairing proposition \(\phi\) with value \(v\), but includes how the agent comes to pair \(\phi\) with \(v\).
\end{note}

\begin{note}
  We start with an assumption regarding an event in which an agent concludes:

  \begin{specification}[Concludes is inclusive]
    \label{assu:concluding:span}
        \cenLine{
      \begin{itemize*}[noitemsep, label=\(\circ\)]
      \item
        Agent: \vAgent{}
      \item
        Proposition: \(\phi\)
      \item
        Value: \(v\)
      \item
        \pool{2}: \(\Phi\)
      \item
        \mbox{ }
      \end{itemize*}
    }

    \begin{itemize}
    \item
      The event in which an agent concludes \(\pv{\phi}{v}\) from \(\Phi\) spans the agent moving from the premises of \(\Phi\) to the conclusion that \(\phi\) has value \(v\).
    \end{itemize}
    \vspace{-\baselineskip}
  \end{specification}

  Paraphrased, \autoref{assu:concluding:span} states that the event in which an agent concludes \(\pv{\phi}{v}\) from \(\Phi\) spans the agents \emph{reasoning} to \(\pv{\phi}{v}\) from \(\Phi\).
\end{note}

\begin{note}
  \autoref{assu:concluding:span} specifies a particular sense of the term `concludes'.
  The sense of concludes which \emph{includes} the relevant reasoning leading up to some conclusion and does not focus only on the more-or-less instantaneous event in which the agent forms the relevant conclusion.
\end{note}

\begin{note}
  To illustrate, we contrast to instances of `concludes'.

  The first instance is from \citeauthor{Gardner:1986wp}'s discussion of Newcomb's problem:

  \begin{quote}
    A large number of those who recommended taking only the second box performed the expected-value calculation and concluded that, provided the probability that the Being was correct was at least .5005, they would take only the second box.%
    \mbox{ }\hfill\mbox{(\citeyear[166]{Gardner:1986wp})}
  \end{quote}

  The second instance is from \citeauthor{Bratman:1979aa}'s discussion of temptation:

  \begin{quote}
    Sam thinks both that in certain respects his drinking would be, \emph{prima facie}, best, and that in certain other respects his abstaining would be, \emph{prima facie}, best.
    Weighing these conflicting considerations he concludes that it would be best to abstain, rather than drink.%
    \mbox{ }\hfill\mbox{(\citeyear[156]{Bratman:1979aa})}
  \end{quote}

  Both passages capture a similar process.
  From some \agpe{} various states of affairs are evaluated in different ways, and after some consideration the agent chooses to do something.

  The contrast between the two passages in the event captured by `concludes'.
  \citeauthor{Gardner:1986wp} presents the scenario as a sequence;
  Two verbs (`performs', `concludes') are linked by `and'.
  \citeauthor{Bratman:1979aa}, modifies a verb (`concludes') by a different verb (`weighs').
  On a natural reading, the event in which the agent concludes as described by \citeauthor{Gardner:1986wp} is distinct from the event in which the agent performs the expected-value calculation.
  And, by contrast, \citeauthor{Bratman:1979aa} describes a single event in which the agent concludes by weighing conflicting considerations.

  \autoref{assu:concluding:span} is compatible with \citeauthor{Bratman:1979aa}'s description.
  However, if the agent concluded they would take only the second box from whatever premises are associated with the expected-value calculation given \citeauthor{Gardner:1986wp}'s description, then the event \citeauthor{Gardner:1986wp} identifies with `concludes' is too narrow.
  Following \citeauthor{Bratman:1979aa}, we may recast \citeauthor{Gardner:1986wp}'s description as \textquote{Performing these expected-value calculations, they concluded  they would take only the second box}.%
  \footnote{
    In addition, it seems to me that whether or not one defaults to thinking of a more-or-less instantaneous event or an inclusive event comes apart in multi-agent cases.

  For example, consider Sam and Max working on some problem \(p\) together.
  Sam and Max trade ideas back and forth, and settle on an answer \(a\).

  Consider the events described by the following sentence:
  \begin{enumerate}[label=\arabic*., ref=(\arabic*), noitemsep]
  \item
    \label{Cing:SandM:j}
    Sam and Max (jointly) conclude \(a\) is an answer to problem \(p\).
  \item
    \label{Cing:SandM:s}
    Sam concludes \(a\) is an answer to problem \(p\).
  \item
    \label{Cing:SandM:m}
    Max concludes \(a\) is an answer to problem \(p\).
  \end{enumerate}

  My default is to consider the events captured by \ref{Cing:SandM:s} and \ref{Cing:SandM:m} in line with \citeauthor{Gardner:1986wp}.
  When considering Sam or Max in isolation, the event in which Sam (or Max) concludes more-or-less instantaneous event.
  Though, the relevant events may be expanded to include the trading of ideas by appending `by trading ideas with Max/Sam'.

  By contrast, I read \ref{Cing:SandM:j} in line with \citeauthor{Bratman:1979aa}.
  When considering Sam and Max as a pair, in which a non-distributive reading of `conclude' is forced by the parenthetical `jointly', the event includes the trading of ideas back and forth.
  Though, the relevant event includes sub-event in which Sam and Max separately evaluate that \(a\) is an answer to problem \(p\).
  }
\end{note}

\begin{note}
  \begin{observation}
    \label{obs:newPVp-newE}
    Event, agent beings with \pool{} \(\Phi\).
    Adds \(\pv{\psi}{v'}\).
    Then, there is no event where the agent concludes \(\pv{\phi}{v}\) from \(\Phi\).
  \end{observation}

  \begin{motivation}{obs:newPVp-newE}
    \autoref{def:pools}.
    Assumed agent concludes from \pool{}, and that everything in \pool{} is evaluated.
    Therefore, agent does not conclude from \(\Phi\).
    For, needed \(\pv{\psi}{v'}\).
    Further, conclusion includes \pool{}.
    So, the only event is after \(\pv{\psi}{v'}\).
  \end{motivation}

  This is a little unintuitive.
  However, result of being somewhat specific.
  I think it is fine to say concludes.
  However, specific sense we're working with, this does not hold.
\end{note}

\subsection{Reasoning}
\label{sec:reasoniing}

\begin{note}
  Conclusions are the result of reasoning.

  \begin{assumption}[Conclusions by reasoning]
    \label{assu:ConRea}
        \cenLine{
      \begin{itemize*}[noitemsep, label=\(\circ\)]
      \item
        Agent: \vAgent{}
      \item
        Proposition: \(\phi\)
      \item
        Value: \(v\)
      \item
        \pool{2}: \(\Phi\)
      \item
        Event: \(e\)
      \item
        \mbox{ }
      \end{itemize*}
    }

    \begin{itemize}
    \item
      \begin{itemize}
      \item[\emph{If}:]
        \(e\) is an event in which \vAgent{} concludes \(\pv{\phi}{v}\) from \(\Phi\).
      \item[\emph{Then}:]
        \(e\) is an event in which \vAgent{} \emph{reasons} to \(\pv{\phi}{v}\) from \(\Phi\).
      \end{itemize}
    \end{itemize}
    \vspace{-\baselineskip}
  \end{assumption}

  \autoref{assu:ConRea} links concludes to reasons.

  For the most part we have little interest in reasoning.
  As highlighted in \autoref{cha:intro}, relation between proposition, value and \pool{}.

  \begin{itemize}
  \item
    Drop to reasoning, where natural
  \item
    Link to accounts which focus on reasoning.
  \end{itemize}
\end{note}

\begin{note}
  We do not place any constraints on reasoning.

  \citeauthor{Broome:2013aa}'s rule following account of reasoning is unconstrained in terms of the rules an agent may follow.
  This aspect  of \citeauthor{Broome:2013aa}'s account is highlighted in the following passage:%
  \footnote{
     \citeauthor{Broome:2013aa} do not require that, in general, an agent has beliefs concerning of the rules they follow in reasoning (\citeyear[cf.][\S13.2]{Broome:2013aa}).
  }

  \begin{quote}
    [S]hould we exclude this bizarre rule: from the proposition that it is raining and the proposition that if it is raining the snow will melt, to derive the proposition that you hear trumpets.
    Following this rule would lead you to believe you hear trumpets when you believe it is raining and believe that if it is raining the snow will melt.
    If you did this, should we count you as reasoning?

    I think we should.
    If you derive this conclusion by operating on the premises, following the rule, we should count you as reasoning.
    \dots
    I think we should not impose a limit on rules.%
    \mbox{}\hfill\mbox{(\citeyear[233]{Broome:2013aa})}
  \end{quote}

  May be tempting to say the agent does not.
  Some difficulty.%
  \footnote{
    For contrast, \citeauthor{Wedgwood:2006ui} assumes reasoning \textquote{is the process of \emph{revising ones beliefs or intentions, for a reason}}.%
    (\citeyear[600]{Wedgwood:2006ui})

   \begin{quote}
    If a set of antecedent mental states makes it rational for one to form a new belief or intention, then those antecedent mental states are surely of a suitable type and content so that it is \emph{intelligible} that they could represent one's reason for forming that belief or intention.\newline
    \mbox{ }\hfill\mbox{(\citeyear[662]{Wedgwood:2006ui})}
  \end{quote}

   \citeauthor{Wedgwood:2006ui} provides the following contrasts to clarify their understanding of intelligibility:

  \begin{quote}
    [T]he belief that the \emph{Oxford Dictionary of National Biography} says that Hume died in 1776 seems a mental state of a suitable type and content so that it could intelligibly represent one's reason for believing that Hume died in 1776.
    On the other hand, it is not (except in the presence of some rather extraordinary background beliefs) a mental state of a suitable type and content so that it could intelligibly represent one's reason for believing that every even number is the sum of two primes.%
    \mbox{ }\hfill\mbox{(\citeyear[662]{Wedgwood:2006ui})}
  \end{quote}

  Strictly, \citeauthor{Wedgwood:2006ui} understanding reasoning in terms of dispositions that respond to what `rationalizes' (\citeyear[672]{Wedgwood:2006ui}) and \citeauthor{Wedgwood:2006ui} considers `rationalizes' and  `makes intelligible' as equivalent.
    Therefore, the following conditional, states a sufficient rather than necessary condition which leads to \citeauthor{Wedgwood:2006ui}'s assumption, rather than being an expansion of what \citeauthor{Wedgwood:2006ui}'s assumption amounts to.
    I find this particularly confusion.
  }

  Impacts link to other accounts of reasoning.

  However, core cases.
\end{note}

\begin{note}
  One difficult observation.
  Conclusions may be instantaneous.
  Place no constraints.
\end{note}

\section{Concluding}
\label{cha:clar:sec:Cing}

\begin{note}
  Conclusions and event in which an agent concludes in hand.
  Final thing of interest.
  Event in which an agent is `concluding'.

  English does not have a specific way of expressing progressive.
  Still, progressive.
  Only progressive.%
  \footnote{
    This is difficult.
    I have considered ensuring every instance of an `-ing' verb expresses the progressive, but this is quite hard (see the fourth word in this sentence).
    And, I have considered writing `\(\text{Prog}(\text{concludes})\)' in place of `concluding', but this is too clumsy.
  }

  Progressive is used extensively throughout this document.
  Clarify why, basic assumption, and illustration.
\end{note}

\subsection{Progressive}

\subsubsection{The progressive}

\begin{note}[Interest with the progressive]
  Our interest with the progressive is due to the sense of possibility required for a sentence stating an event in the progressive to be true.

  \phantlabel{imperfective-paradox:intro}
  Perhaps the clearest example is the `imperfective paradox' (\citeyear[cf.][Ch.3.1]{Dowty:1979vq}).

  \citeauthor{Bach:1986tb} summarises:
  \begin{quote}
    [H]ow can we characterize the meaning of a progressive sentences like \ref{Bach:impP:17} on the basis of the meaning of a simple sentence like \ref{Bach:impP:18} when \ref{Bach:impP:17} can be true of a history without \ref{Bach:impP:18} ever being true?
    \begin{enumerate}[label=(\arabic*), ref=(\arabic*)]
      \setcounter{enumi}{16}
    \item
      \label{Bach:impP:17}
      John was crossing the street.
    \item
      \label{Bach:impP:18}
      John crossed the street.%
      \mbox{ }\hfill\mbox{(\citeyear[12]{Bach:1986tb})}
    \end{enumerate}
  \end{quote}

  No completion is required, and often some surprise.
  Something unexpected happened while John was crossing the street.
  Sense of inertia associated with the agent \(\alpha\)ing.

  Expectation that that John reaching the other side of the street does not reduce to \(\{\text{logical}, \text{metaphysical}, \text{nomic}, \dots\}\) possibility.

  For, suppose John is sitting a multiple choice exam.
  To pass the exam John only needs to chose some number of correct choices.
  It is certainly logically, metaphysically, and nomically possible that John chooses a sufficient number of correct choices.
  However, it does not follow that John is passing the exam.%
  \footnote{
    See also Igal Kvart's example of Mary wiping out the Roman army (\cite[18]{Landman:1992wh}).
  }

  Likewise, there is no simple relation to counterfactuals.
  Consider a scenario in which John is passing the exam without external help.
  Then, a classmate slips John some answers, which John then uses.
  It is no longer true that John is passing the exam without external help.
  And, in the closest possible world where the classmate does not slip John answers, it need not be true that John passes the exam without external help.
  For, if John is surrounded by students of a similar mindset then it is plausible that the in closest possible world a different classmate slips John the same answers.
\end{note}

\begin{note}
  Way the modality functions is tied to the event.

  \citeauthor{Dowty:1979vq} adds:
  \begin{quote}
    Notice, furthermore, that to Say that John was drawing a circle is not the same as saying that John was drawing a triangle, the difference between the two activities obviously having to do with the difference between a circle and a triangle.
    Yet if neither activity necessarily involves the existence of such a figure, just how are the two to be distinguished?%
    \mbox{ }\hfill\mbox{(\citeyear[133]{Dowty:1979vq})}
  \end{quote}

  As \citeauthor{Dowty:1979vq} highlights, event is sufficiently specific to determine some outcome over some other.
  So, the truth of the progressive doesn't require completion and doesn't require significant progress toward completion.
\end{note}

\subsubsection{\assuPP{2} and \pevent{1}}
\label{sec:assupp2}

\begin{note}
  Important (but common)%
  \footnote{
    See, e.g.:
    \cite{Bennett:1972uw},
    \cite{Dowty:1979vq},
    \cite{Parsons:1990aa},
    \cite{Landman:1992wh},
    \cite{Portner:1998um}.

    However,~\assuPP{0} is denied by~\textcite{Szabo:2004ul}.
    \citeauthor{Szabo:2004ul} argues:
    \begin{quote}
      Sometimes we are \emph{doing} things even though there is no real chance that we could get them \emph{done}, and this is true even if we abstract away from the possibility of miraculous intervention.%
      \mbox{ }\hfill\mbox{(\citeyear[40]{Szabo:2004ul})}
    \end{quote}
    To illustrate, \citeauthor{Szabo:2004ul} denies~\ref{Szabo:Arch} is necessarily false:
    \begin{quote}
      \begin{enumerate}[label=(\arabic*), ref=(\arabic*)]
        \setcounter{enumi}{12}
      \item
        \label{Szabo:Arch}
        As the architect was building the cathedral he knew that, although he would be building it for another year or so, he couldn't possibly complete it.%
        \mbox{ }\hfill\mbox{(\citeyear[38]{Szabo:2004ul})}
      \end{enumerate}
    \end{quote}
    Though,~\ref{Szabo:Arch} seems false to me.
    The only sense with which I read~\ref{Szabo:Arch} as true involves allowing factivity of knowledge to fail, thus allowing the cathedral to be built.

    (See also (\cite[1245]{Portner:2011vi}) for additional, distinct, discussion of (\cite{Szabo:2004ul}).)

    Still, we do not, strictly, require~\assuPP{0}.
    The role of~\assuPP{0} is to ensure the existence of a possible event, where the sense of `possible' is captured by the progressive.
    Hence, if~\assuPP{0} fails then while we lack motivation, it is not clear the events fail to exist under whatever the modality amounts to.
  }
  assumption regarding the progressive:

  \begin{assumption}[\assuPP{2}]
    \label{assu:PP}
    For any event \(e\) and action description \(\alpha\):
    \begin{enumerate}
    \item[\emph{If}:]
      \begin{enumerate}[label=\alph*., ref=(\alph*)]
      \item
        \(\text{Prog}(e,\alpha)\) is true.%
        \hfill(I.e.\ \(e\) is an event of \(\alpha\)ing.)
      \end{enumerate}
    \item[\emph{Then}:]
      \begin{enumerate}[label=\alph*., ref=(\alph*), resume]
      \item
        There is some \progAdj{0} event \(e'\) such that~\ref{assu:PP:pe:dev} and~\ref{assu:PP:pe:verb} are both true:
        \begin{enumerate}[label=\roman*., ref=(\roman*)]
        \item
          \label{assu:PP:pe:dev}
          \(e'\) is a development of \(e\).
        \item
          \label{assu:PP:pe:verb}
          \(\alpha\) is true of \(e'\).
        \end{enumerate}
      \end{enumerate}
    \end{enumerate}
    \vspace{-\baselineskip}
  \end{assumption}

  \assuPP{2}, shift evaluation to some \progAdj{0} event in which something related is true.

  So, task of an account of the progressive is to narrow the relevant sense of `\progAdj{0}'.

  Applied, in particular, to concluding, \assuPP{0} holds that a agent is concluding \(\pv{\phi}{v}\) from \(\Phi\) only if there is some \progAdj{0} event in which the agent concludes \(\pv{\phi}{v}\) from \(\Phi\).
\end{note}

\begin{note}
  With \assuPP{} in hand, define \pevent{} to refer to event.

  We define a \pevent{} as follows:
  \begin{restatable}[\pevent{3}]{definition}{definitionPEvent}
    \label{def:potenital-event}
        \cenLine{
      \begin{itemize*}[noitemsep, label=\(\circ\)]
      \item
        Agent: \vAgent{}
      \item
        Event: \(e\)
      \item
        Action description: \(\alpha\)
      \item
        \mbox{ }
      \end{itemize*}
    }

    \begin{itemize}
    \item
      There is a \pevent{} \(p\) in which \vAgent{} \(\alpha\)s
    \end{itemize}

    \emph{If and only if}:

    \begin{enumerate}[label=]
    \item
      There is some action \(a\) such that both~\ref{def:PE:action} and~\ref{def:PE:prog} are true:
      \begin{enumerate}[label=\alph*., ref=(\alph*)]
      \item
        \label{def:PE:action}
        \vAgent{} may immediately perform \(a\).
      \item
        \label{def:PE:prog}
        \(\text{Prog}(e, \alpha)\) would be true in the event \(e\) of \vAgent{} doing \(a\).
      \end{enumerate}
    \end{enumerate}

    Where \(\text{Prog}(e, \alpha)\) stands for the progressive from of \(\alpha\) when evaluated with respect to \(e\) and \assuPP{} holds for the progressive.%
    \footnote{
      I.e.\ \(\text{Prog}(e, \alpha)\) is true \emph{iff} event \(e\) is an event of \(\alpha\)ing.
      See,~\textcite{Richards:1981wo},~\textcite{Portner:2011vi}, etc.
    }
  \end{restatable}

  In short,~\autoref{def:potenital-event} states that there is a \pevent{} in which an agent performs some action \(\alpha\) just in case there is some action the agent may (immediately) perform which would result in the agent \(\alpha\)ing.
\end{note}

\begin{note}
  Paired with choice, allows complex `incomplete' actions.
  Again, progressive develops.

  \begin{illustration}[Darts]
    There is a \pevent{} in which agent scores 180 at darts just in case there is some action available to the agent, such that if the agent were to perform the action they would be scoring 180 at darts.
  \end{illustration}

  Slightly more interesting.
  Determine the available actions.
  Though, similar, no guarantee.
  Hand is knocked at point of release, still scoring.

  Scoring 180 is a complex action.
  Though, interesting.
  First throws don't matter.

  Again, key idea is that sufficient understanding of progressive.

  And, case of interest:

  \begin{illustration}[Concluding]
    There is a \pevent{} in which agent concludes \(\pv{\phi}{v}\) from \(\Phi\) just in case there is some action available to the agent, such that if the agent were to perform the action they would be concluding \(\pv{\phi}{v}\) from \(\Phi\).
  \end{illustration}

  What is it to be concluding something.
  Like crossing the road, fail to complete.
  Like darts, recover from a bad opening.
\end{note}

\subparagraph{\citeauthor{Landman:1992wh}'s (\citeyear{Landman:1992wh}) account of the progressive}
\label{cha:sec:fcs-def:progressive-landman}
\nocite{Portner:1998um}
\nocite{Engelberg:1999vi}

\begin{note}
  No full account of the progressive.
\end{note}

\begin{note}
  In broad summary:%
  \footnote{
    \textcite{Szabo:2004ul}:
  \begin{quote}
    [A] progressive sentence is true at some time just in case some event occurs at that time, and if we follow the development of the event (within our world as long as it goes, then jumping into a nearby world, and iterating the process within the limits of reasonability) we will reach a possible world where the perfective correlate is true of the continuation.%
    \mbox{ }\hfill\mbox{(\citeyear[34]{Szabo:2004ul})}
  \end{quote}
  }
  \citeauthor{Landman:1992wh} holds that an action in the progressive holds of some event just in case the event, if allowed to develop, would develop into an event in which the action is performed.

  As we have seen with the perfective paradox, some action in the progressive need not continue in the actual world, and hence the core of \citeauthor{Landman:1992wh}'s account of the progressive is an account of allowing an event to continue.

  Roughly, on \citeauthor{Landman:1992wh}'s account the way in which an event is allowed to develop is indirectly captured by considering different ways in which the actual world may have been.

  In short, we follow an event through it's development in the actual world until it does not continue any further in the actual world.
  Then, just before the event stops, we jump to the closest `reasonable' world in which the same event happened and continues a little further (if such a world exists) and follow the development of the event in the close world until it does not continue any further.
  This process continues until it is not possible to jump to a `reasonable' world.
\end{note}

\begin{note}
  To see how the above sketch functions in practice, we follow~\citeauthor{Portner:1998um}'s (\citeyear[764--766]{Portner:1998um}) illustration of \citeauthor{Landman:1992wh}'s account.
\end{note}

\begin{note}
  Our interest is with the following sentence:
  \begin{enumerate}
  \item
    \label{prog:max:bad}
    Max is crossing the street.
  \end{enumerate}

  Hence, the relevant action of interest is the action of Max crossing the street.
  Let us fix the event \ref{prog:max:bad} is (assumed to be) true of as \(e\) and fix \(w\) for the world \(e\) happens in.

  Following \citeauthor{Landman:1992wh} (and in line with \assuPP{}), Max is crossing the street is true of \(e\) just in case, if allowed to develop, \(e\) would develop into an event in which Max crosses the street.

  Now, suppose that in \(w\) the event \(e\) does not develop any further.
  Instead (and quite unfortunately), Max is hit by a bus cruising at thirty miles per hour.
  Somewhat ominously, let us identify this bus as `bus \#1'.

  Still, \(e\) does not include Max being hit bus \#1, and had things been a little different, Max may have continued a little further across the street.
  Hence, there is some world \(v\) which is close to \(w\) in which \(e\) develop a little further.

  So far so good, but in \(w\) Max was hit by bus \#1 in \(w\).
  Hence, it seems that \(v\) also involves Max being hit by a bus.
  For, \(v\) is a possible world close to \(w\), and if Max avoids being hit by a bus altogether in \(v\) then there is surely some possible world \(v'\) closer to \(w\) than \(v\).
  So, although Max makes it a little further across the road in \(v\), Max is still hit by a bus.
  We identify the bus in \(v\) as `bus \#2'.

  However, as with bus \#1 in \(w\), it seems Max may have walked a little further across the street in possible worlds close to \(v\).
  Hence, by the same reasoning we may consider some possible world \(u\) close to \(v\) and so on\dots

  \autoref{fig:max-bus} is a modification of \citeauthor{Portner:1998um}'s figure 1. (\citeyear[767]{Portner:1998um})
  \begin{figure}[!h]
    \centering
    \begin{tikzpicture}
      \tikzmath{
        % x positions
        \x1 = 11;
        \xb1 = 2/9*\x1; \xb2 = 4/9*\x1; \xb3 = 6/9*\x1;
        % y positions
        \y1 = 2/5*\x1; \ymid = 1/2*\y1;
        \yw1 = \y1; \yw2 = 1/2*\y1; \yw3 = 0*\y1; \yb2 = 1/5*\y1;
        % event e
        \xe = 1/2*\xb1; \yediff = \yw2 - \yb2;
        \ye = \yw2 - 1/2*\yediff;
        \enudge = .1;
        \xel = 0; \xer = \xb1; \yen = \yw2 - \enudge;
        % bus 1 description location
        \xbx = 1.5/9*\x1; \xby = 4/5*\y1;
        % bus 2 description location
        \xb9 = 2.5/9*\x1;
      }
      % Paths
      \draw[line width=0.25mm, line cap=round] (\xb1,\ymid) -- (\xb3,\yw1); % world 1
      \draw[line width=1mm, line cap=round, dash pattern=on 175pt off 5pt on 5pt off 5pt on 5pt off 5pt on 5pt off 5pt on 5pt off 5pt on 5pt off 5pt on 5pt off 5pt on 5pt off 5pt on 5pt off 5pt on 5pt off 5pt on 5pt off 5pt on 5pt off 5pt on 5pt off 5pt on 5pt off 5pt on 5pt off 5pt on 5pt off 5pt on 5pt off 5pt on 5pt off 5pt on 5pt off 5pt on 5pt off 5pt] (0,\ymid) -- (\xb1,\ymid) -- (\xb2,\yb2) -- (\xb3,\yw2); % world 2
      \draw[line width=0.25mm, line cap=round] (\xb2,\yb2) -- (\xb3,\yw3); % world 3
      % World descriptions
      \filldraw[black] (\xb3,\yw1) circle (0pt) node[anchor=west, align=left]{world 1: Max hit by \\ bus \# 1};
      \filldraw[black] (\xb3,\yw2) circle (0pt) node[anchor=west, align=left]{world \(i\): Max \\ crosses street};
      \filldraw[black] (\xb3,\yw3) circle (0pt) node[anchor=west, align=left]{world 2: Max hit by \\ bus \# 2};
      % Event
      \draw[] (\xe,\ye) -- (\xel,\yen); % event l
      \draw[] (\xe,\ye) -- (\xer,\yen); % event r
      % Event description
      \filldraw[black] (\xe,\ye) circle (0pt) node[anchor=north, align=left]{event e};
      % Splits
      \filldraw[black, dashed] (\xbx,\xby) circle (0pt) node[anchor=south, align=left]{bus \#1 hits Max};
      \filldraw[black, dashed] (\xb9,\yw3) circle (0pt) node[anchor=north, align=left]{bus \#2 hits Max};
      % Split descriptions
      \draw[-Stealth, dashed] (\xbx,\xby) -- (\xb1,\ymid + \enudge); % bus 2 arrow
      \draw[-Stealth, dashed] (\xb9,\yw3) -- (\xb2 - \enudge,\yb2 - \enudge); % bus 2 arrow
    \end{tikzpicture}
    \caption{
      Continuation path of `Max was crossing the street'. \\
    }
    \label{fig:max-bus}
  \end{figure}

  The thick black line captures \(e\) as \(e\) is allowed to develop, and the changes in angle reflect shifts to alternative possible worlds.

  The dashed line indicates that Max may need to avoids being hit by additional busses in order for \(e\) to develop into an event in which Max crosses the road.

  Still, does \(e\) develop into an event in which Max crosses the road?
  If Max is hit by a bus in \(w\), then surely Max is hit by a bus in all the possible worlds close to \(w\).

  The key property of \citeauthor{Landman:1992wh}'s account is that closeness is understood relative to the development of \(e\), rather than \(e\) itself as \(e\) happened in \(w\).
  Prior to bus \#2 hitting Max in \(v\), we shifted to \(u\), a world which is close to \(v\) rather than \(w\).
  So, as Max progresses a little further each time a possible world in which Max crosses the street gets a little closer until, eventually, Max crosses the street.
  To borrow a piece of terminology from \textcite{Dowty:1979vq}, \(e\) has sufficient \emph{inertia} to develop in some possible world \(v\), and as \(e\) develops in \(v\), inertia continues to build until Max crosses the road.

  However, an important restriction is placed on shifts to possible worlds.
  Intuitively, it is not the case that Max avoids being hit by bus \#\(j\) because Max has the strength to stop as moving bus.
  Yet, a possible world in which Max has the strength to stop a moving bus may be close to the world in which Max is not hit by bus \#\(j - 1\).
  In \citeauthor{Landman:1992wh}'s terminology, the relevant possible worlds in which \(e\) develops must be `reasonable'.
\end{note}

\subsection{Concluding}

\begin{note}
  With some understanding of the progressive.
  Return to concluding.
\end{note}

\begin{note}
  Our understanding of `concluding' is that an event in which an agent is concluding some proposition-value pair \(\pv{\phi}{v}\) from some \pool{} \(\Phi\) is just an event such that the event in which the agent concludes \(\pv{\phi}{v}\) from \(\Phi\) is in progress.

  In this respect, an event of concluding need not involve the agent pairing \(\phi\) with \(v\), it need only be the case that if the relevant event went on uninterrupted, then event would develop into an event in which the agent concludes \(\pv{\phi}{v}\) from \(\Phi\), and hence pairs \(\phi\) with \(v\).
\end{note}

\begin{note}
  \begin{proposition}[Possible event in which an agent concludes]
    \label{prp:peventC}
    \cenLine{
      \begin{itemize*}[noitemsep, label=\(\circ\)]
      \item
        Agent: \vAgent{}
      \item
        Proposition: \(\phi\)
      \item
        Value: \(v\)
      \item
        \pool{2}: \(\Phi\)
      \item
        Event: \(e\)
      \item
        \mbox{ }
      \end{itemize*}
    }

    \begin{itemize}
    \item
      \begin{itemize}
      \item[\emph{If}:]
        \(e\) is an event in which \vAgent{} is concluding \(\pv{\phi}{v}\) from \(\Phi\).
      \item[\emph{Then}:]
        There is some \pevent{} \(e'\) and \(e'\) is an event in which \vAgent{} concludes \(\pv{\phi}{v}\) from \(\Phi\).
      \end{itemize}
    \end{itemize}
    \vspace{-\baselineskip}
  \end{proposition}

  \begin{argument}{prp:peventC}
    Immediate by~\autoref{assu:PP}.
  \end{argument}

  \begin{observation}[Conclusions without concluding]
    \label{obs:cds-arb}
    \cenLine{
      \begin{itemize*}[noitemsep, label=\(\circ\)]
      \item
        Agent: \vAgent{}
      \item
        Proposition: \(\phi\)
      \item
        Value: \(v\)
      \item
        \pool{2}: \(\Phi\)
      \item
        Event: \(e\)
      \item
        \mbox{ }
      \end{itemize*}
    }

    \begin{itemize}
    \item
      The following conditional is not necessarily true:
      \begin{itemize}
      \item[\emph{If}:]
        \(e\) is an event in which \vAgent{} concludes \(\pv{\phi}{v}\) from \(\Phi\).
      \item[\emph{Then}:]
        There is some sub-event \(e'\) of \(e\) such that:
        \begin{enumerate}
        \item
          \(e'\) is an event in which \vAgent{} is concluding \(\pv{\phi}{v}\) from \(\Phi\).
        \end{enumerate}
      \end{itemize}
    \end{itemize}
  \end{observation}

  \begin{argument}{obs:cds-arb}
    There are three ways the conditional may fail.

    First, it may be the case that an agent instantly concludes \(\phi\) has value \(v\).
    For example, an agent looks outside, sees a bird in a tree and concludes the proposition \prop{There is a bird in that tree} has value \val{True}.

    Second, there may have been no `inertia' with respect to the events prior to the event in which the agent concludes.
    For example, there are exactly five intermediate logics that have the interpolation property (\cite[cf.][]{Maksimova:1977un}).
    However, \citeauthor{Maksimova:1977un} may have been ready to give up on the proof at any point.%
    \footnote{%
      Admittedly, it is unlikely \citeauthor{Maksimova:1977un} would have given up, but the point stands.
    }

    Third, it may fail to be the case that the \emph{agent} is concluding.
    For example, consider being guided through a complex argument.
    Follow along, and conclude.
    At each step, the guide highlights which sub-conclusions to draw.
    However, guide goes away.
    And, given complexity, no repetition without guide.
  \end{argument}
\end{note}

\begin{note}
  \begin{specification}
    
    Distinction between `concluding' and `going to conclude'
  \end{specification}

  \begin{motivation}
    Builds on \autoref{obs:newPVp-newE}
  \end{motivation}
  {
    \color{red}
    There is, then, a difference between event in which agent is concluding \(\pv{\phi}{v}\) from \(\Phi\).
    And, event in which agent is `going' to conclude \(\pv{\phi}{v}\) from \(\Phi\).
    Where, `going' is progressive, e.g.\ going on a journey.
    I.e.\ develops such that the agent concludes \(\pv{\phi}{v}\) from \(\Phi\).

    Indeed, both develop.
    The difference is that with the former, \pool{} is present.
    With the latter, the event develops such that the \pool{} is present.
  }
\end{note}


%%% Local Variables:
%%% mode: latex
%%% TeX-master: "master"
%%% End:


% \begin{itemize}
% \item
%   On occasion we use \(\overline{v}\) to indicate some value other than \(v\) of the same type as \(v\) and \(v_{?}\) to indicate \(\phi\) is not paired with any proposition of the same type as \(v\).

%   In summary, the notation on the left is in accordance with the sentence on the right with a default reading below:
%   \begin{itemize}
%   \item
%     \(\pv{\phi}{v}\) \hfill \(\phi\) has value \(v\).%
%     \newline
%     \mbox{ }\hfill \(\phi\) is true.
%   \item
%     \(\pv{\phi}{\overline{v}}\) \hfill \(\phi\) has some value other than \(v\) of the same type.%
%     \newline
%     \mbox{ }\hfill \(\phi\) has some value other than true of the same type.%
%     \newline
%     \mbox{ }\hfill I.e.\ \(\phi\) is false.
%   \item
%     \(\pv{\phi}{v_{?}}\) \hfill \(\phi\) is not evaluated to have a value of type \(v\).%
%     \newline
%     \mbox{ }\hfill \(\phi\) is not evaluated as true or false.
%   \end{itemize}
% \end{itemize}
