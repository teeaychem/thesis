\chapter{Clarification}
\label{cha:clarification}

\begin{note}
  In~\autoref{cha:introduction} we considered an instance of concluding, raised two questions, and stated an issue with respect to these questions.

  In particular, questions: \qWhy{} and \qHow{}, and relation between these two questions, \issueInclusion{}.

  The purpose of the present chapter is clarification.
  At the close of this chapter we will have variants of \qWhy{}, \qHow{}, and \issueInclusion{} and an understanding of how the variant of \issueInclusion{} may fail to hold.

  Restatements of \qWhy{} and \qHow{} which remove reference to `why' and `how'.
  In such a way that basic intuitions remain.

  Provide variations, and link these to original questions.

  In turn, variation of \issueInclusion{}.
  Follows from the variations.

  However, highlight how may fail, and other important aspects.
\end{note}

\section{Outline}
\label{cha:clarification:sec:outline}

\begin{note}[Support]
  Generic, two ideas.

  First, if concluded then (a relation of) \support{}.
  Second, possible for \support{} without reasoning from.

  Constraints are fairly weak.

  In order to get support without witnessing, need it to be the case that only thing the agent hasn't done is concluded.
\end{note}

\begin{note}[Support and \qWhy{}]
  Key idea, if support, then proposition-value pair answers \qWhy{}.

  The possibility this allows is \support{} answering why.
  And, without reasoning.

  This is strong, and this is not intended as a general constraint on \qWhy{}.
  For sure, may be various ways in which could understand \qWhy{} for which this doesn't hold.

  Still, minimal, and for our interests required.
\end{note}

\begin{note}[\qHow{}]
  Witnessed reasoning.

  Agent has reasoned from pool of premises to conclusion.
\end{note}

\begin{note}[Combined]
  Support between \(\Phi\) and \(\pv{\phi}{v}\) answers why only if witnessed reasoning from \(\Phi\) to \(\pv{\phi}{v}\).
\end{note}

\begin{note}
  If the argument is successful, then I think the weak link is support.
  But, options for rejecting a fairly limited.
\end{note}

\begin{note}
  With the issue clarified, the following chapter focuses on \scen{1} of interest.
  With these in hand, final chapter of this part sketches the argument.
\end{note}


\begin{note}
  The role of this chapter is to refine important things about concluding.

  What it is about concluding we're interested in.
  This is support.
  Leads to a refinement of why? and how?
\end{note}

\begin{note}
  The goal here is to refine understanding of key question.
  And, to introduce type of \scen{} of interest.
\end{note}

\section{\dots, from the agent's perspective}
\label{cha:introduction:sec:agents-perspective}

\begin{note}
  First.
  The agent's perspective.

  Mentioned before, but make this clear.
\end{note}

\begin{note}
  \qWhy{} and \qHow{}, questions asked from the agent's perspective.

  Noted this distinction when considering \citeauthor{Davidson:1963aa}.

  Continue in this way.
  Variations, and resulting issue, from agent's perspective.
\end{note}

\paragraph{Limitations}

\begin{note}
  Perhaps some diminishing interest.
  \citeauthor{Davidson:1963aa}, reasons not from the agent's perspective.
  Rather, derived from the agent's perspective.

  This is important to keep in mind.
  Parallel with knowledge.
  Factivity, we have from the agent's perspective, and from an independent perspective.

  As everything will be qualified, nothing stronger.
  Parallel, with knowledge, from the agent's perspective doesn't tell us whether agent knows, and so answers to \qWhy{} and \qHow{}, as from agent's perspective don't necessarily say anything with respect to \emph{why} and \emph{how} from an independent perspective.

  Still, agent's perspective is arguable important.
  This is what we get with \citeauthor{Davidson:1963aa}.
  And, even if you don't go in for \citeauthor{Davidson:1963aa}'s theory, we have things like the humean theory of motivation.

  Expressed by \citeauthor{Smith:1987vk}:%
  \footnote{
    \citeauthor{Smith:1987vk} here, suggests comparison to \textcite{Davidson:1963aa}.
    Note, in particular \citeauthor{Smith:1987vk}'s use of `motivating reason' is equivalent to \citeauthor{Davidson:1963aa}'s use of `primary reason' rather than `reason'.
  }
  \begin{quote}
    \begin{enumerate}[label=R1., ref=(R1)]
    \item
    R at t constitutes a motivating reason of agent A to \(\phi\) iff there is some \(\psi\) such that R at t consists of a desire of A to \(\psi\) and a belief that were he to \(\phi\) he would \(\psi\).%
    \mbox{ }\hfill\mbox{(\citeyear[36]{Smith:1987vk})}
  \end{enumerate}
\end{quote}

  Potential for some interesting issues to arise.
  For example\dots
  Humean theory, Putting the belief and desire together.

  \citeauthor{Smith:1987vk}, motivating reason.

  \begin{quote}
    \begin{enumerate}[label=R2., ref=(R2)]
    \item
      Agent A at t has a motivating reason to \(\phi\) only if there is some \(\psi\) such that, at t, A desires to \(\psi\) and believes that were he to \(\phi\) he would \(\psi\).%
      \mbox{ }\hfill\mbox{(\citeyear[36]{Smith:1987vk})}
    \end{enumerate}
  \end{quote}

  Plausibly, motivating reason obtained from belief and desire.%
  \footnote{
    Consider the follow passage from \citeauthor{Hume:2011aa}'s \hyperlink{cite.Hume:2011aa}{Treatise}:
    \begin{quote}
      ’Tis also obvious, that this emotion rests not here, but making us cast our view on every side, comprehends whatever objects are connected with its original one by the relation of cause and effect.
      Here then reasoning takes place to discover this relation; and according as our reasoning varies, our actions receive a subsequent variation.
      But ’tis evident in this case, that the impulse arises not from reason, but is only directed by it.
      ’Tis from the prospect of pain or pleasure that the aversion or propensity arises towards any object: And these emotions extend themselves to the causes and effects of that object, as they are pointed out to us by reason and experience.%
      \mbox{ }\hfill\mbox{(\hyperlink{cite.Hume:2011aa}{T.2.3.3})}
    \end{quote}
    Means end reasoning, roughly.
  }

  Way in which the belief and desire come together.

  So, relation between how agent performed some action and why.
  Back to \citeauthor{Davidson:1963aa}.

  Though, we will not pursue these.
  Task is to establish in general.
  Indeed, concluding, just to keep things simple.
\end{note}

\paragraph{Characterisation}

\begin{note}
  The qualifier `from the agent's perspective' limits the scope of argument.

  However, the qualifier also affords us some ease of expression.

  For, rather than speaking of an agent's beliefs, desires, and so on, we understand propositions as situations which are evaluated by the agent.
  And, `the agent's perspective' captures the fundamental way in which an agent evaluates a proposition.

  For example, the reference of `the dog is sleeping' is a situation in which the dog is sleeping.
  Hence, when we speak of the proposition that the dog is sleeping, we mean this situation, and nothing more.

  Though, from the agent's perspective, it may be the case that the proposition that the dog is sleeping is evaluated as both true and desirable.
  For example, the agent may evaluate the proposition as true because the agent perceives the proposition to be the case.
  And, the agent may evaluate the proposition as desirable because the agent is planning to take the dog on a long walk later in the day.%
  \footnote{
    A handful of abstract instances:
    \begin{itemize}[noitemsep]
    \item \(\phi\) is assigned the value `true'. \hfill (\(\phi\) is true.)
    \item \(\phi\) is assigned the value `ought to be'. \hfill (\(\phi\) ought to be the case.)
    \item \(\phi\) is assigned the value `desirable'. \hfill (\(\phi\) is desirable.)
    \item \(\phi\) is assigned the value `improbable'. \hfill (\(\phi\) is improbable.)
    \end{itemize}
  }\(^{,}\)
  \footnote{
    Nothing in particular hangs on the distinction between different values.
    If you prefer, you may expand the proposition (\world{}) to include additional factors, and consider only the values `true' and `false'.
    For example, the proposition that \emph{I desire the bath to be warm} is false, as opposed me assigned the proposition that \emph{the bath is warm} the value `undesirable'.
  }

  In this respect, we may capture how things are from the agent's perspective without taking a stance on how to characterise things independently of the agent's perspective.

  To illustrate, it may be the case that some proposition is evaluated as true, from the agent's perspective, just in case the agent \emph{believes} the proposition.%
  \footnote{
    \label{fn:belief-is-difficult}
    Though I don't think belief is really so straightforward.

    Consider the Jeremy Goodman's example of three-horse race from~\textcite{Hawthorne:2016wv}:
    \begin{quote}
      Assume that horse A is more likely to win than horse B which in turn is more likely to win then horse C (so the probabilities of winning could be known to be 45, 28, 27\%).
      In this case it seems fine to say `I think horse A will win' or `I believe horse A will win'.%
      \mbox{ }\hfill\mbox{(\citeyear[1440]{Hawthorne:2016wv})}
    \end{quote}
    As \citeauthor{Hawthorne:2016wv} observe: ``[I]t is awful to say, in this case, `I think horse A will win but I don't believe it will'.''
    (\citeyear[1440, fn.17]{Hawthorne:2016wv})
  }
  And, it may be the case that some proposition is evaluated as desirable, from the agent's perspective, just in case the agent \emph{desires} the proposition.

  However, the relevant characterisation of belief and desire will be set aside.%
  \footnote{
    No clear distinction between knowledge and belief in terms of valuations.
    But, difference won't be important.
  }

  Variables, range of attitudes.
  Concluding, proposition has value, no need to limit.
\end{note}

\begin{note}
  From the agent's perspective, a proposition-value pair.
  Some notation:

  \begin{notation}[Proposition-value pair]
    Proposition-value pair, abbreviate \(\pv{\phi}{v}\).

    Collection of proposition-value pairs, \(\Phi\).

    \(\pvp{\phi}{v}{\Phi}\), relation.
  \end{notation}

  \begin{notation}
    Fix \(\phi\), \(v\), and \(\Phi\).
    Three distinct options:
    \begin{enumerate}
    \item
      \(\pv{\phi}{v}\) \hfill \(\phi\) has value \(v\).
    \item
      \(\pv{\phi}{\overline{v}}\) \hfill \(\phi\) has some value other than \(v\) of the same type.
    \item
      \(\pv{\phi}{v_{?}}\) \hfill \(\phi\) is not evaluated to have a value of type \(v\).
    \end{enumerate}
  \end{notation}

  The second may be further distinguished.
  That \(\phi\) has some other value, and some specific value.
  We have no need for such a finer grained distinction.

  Safe to default to `true'.
\end{note}

\paragraph{Three distinct options}

\begin{note}
  Perspective, so agent may be wrong.

  Example.
  Sam shorter than Taylor.
  Then, from epistemic state, Taylor shorter than Sam.
  Doesn't matter whether Sam is shorter than Taylor.
  From perspective of agent's epistemic state.
  Likewise, doesn't matter whether agent recognises shorter, so long as principle.

  Similarly, classical and intuitionistic.
  From int.\ perspective not a proof.
  From classical, also a proof of \(\phi\).
\end{note}

\begin{note}
  \phantlabel{mention:concluding-non-factive}
  The role of the agent's perspective is important with concluding.

  In this respect, doesn't matter whether premises have values, or whether conclusion has value.%
  \footnote{
    To illustrate, an agent may conclude \(0.999\dots \ne 1\).
    And, the agent may, when concluding, hold themselves to have a conventional understanding of real numbers.

    The qualification is important, there are various interpretations under which \(0.999\dots \ne 1\), but (it is convention that) the Archimedean property holds for real numbers.

    Still, the agent may have failed to grasp the Archimedean property does not hold for real numbers, and so may reason that, though \(0.999\dots\) approaches \(1\), there must be \emph{some} difference between \(0.999\dots\) and \(1\) --- no matter how small --- and some difference between to things is sufficient to establish that they are not equal.
  }
\end{note}

\paragraph{Concluding}

\begin{note}
  Our interest is with the process of an agent concluding some proposition-value pair from some pool of premises.

  Concluding is an event --- concluding is something that happens.
  And, in particular, concluding is an act --- an agent concludes.%
  \footnote{
    \color{red}
    This passive construction is to avoid characterising an act as something brought about by an agent.
    Of course, agent being involved doesn't make an act.
    Involvement of agent, exist, but not an act.
    On the other hand, waking up.
    {
      \color{red} This line of inquiry doesn't really matter.
    }
  }

  Assumptions:

  \begin{assumption}
    \label{assu:concluding:pvp}
    What an agent concludes is always some proposition-value pair.
  \end{assumption}

  \begin{assumption}
    \label{assu:concluding:pools}
    An agent always concludes some proposition-value pair from some pool of premises, where a pool of premises is some collection of proposition-value pairs.
  \end{assumption}

  \begin{assumption}
    \label{assu:concluding:consistent}
    perspective is unsuitable for further conclusions only if exactly one:
    \begin{enumerate}
    \item
      \(\pv{\phi}{v}\) \hfill \(\phi\) has value \(v\).
    \item
      \(\pv{\phi}{\overline{v}}\) \hfill \(\phi\) has some value other than \(v\) of the same type.
    \item
      \(\pv{\phi}{v_{?}}\) \hfill \(\phi\) is not evaluated to have a value of type \(v\).
    \end{enumerate}
  \end{assumption}
\end{note}

\begin{note}
  I take \autoref{assu:concluding:pvp} to be fairly straightforward.
  In natural language, \emph{that} clauses.
  Default interpretation, true.

  In terms of reasoning, understanding of concluding includes both theoretical and practical, and if you think this division is non-exhaustive, then some long as the result of the type of reasoning is the agent's evaluation of some situation, \autoref{assu:concluding:pvp} is designed, at least, to not rule out the culmination of the type of reasoning as an instance of concluding.%
  \footnote{
    Though, concluding may be stronger.
  }
\end{note}

\begin{note}
  \autoref{assu:concluding:pools} may be a little less straightforward.%
  \footnote{
    \begin{itemize}[noitemsep]
    \item \emph{A} testified that \(\phi\) is true, so \(\phi\) is true.
    \item \(\phi\) would satisfy every member of the group, so \(\phi\) ought to be the case.
    \item The song is produced by \emph{B}, so it is desirable that I listen to it.
    \item The device reads \(\phi\) and is reliable, so \emph{not}-\(\phi\) is improbable.
    \end{itemize}
  }

  Not so clear that concluding always involves premises.

  Indeed, have seen in \autoref{cha:introduction} some difficulty in stating what premises would be involved in concluding via understanding of arithmetic.

    Robinson or Peano arithmetic together with the two numbers joined by an operator.
  Though, I doubt this.

  For a clear example, consider the rule for conditional introduction a Fitch-style rule for propositional logic.
    (\cite[cf.][206]{Barwise:1999tu})

  \begin{quote}
    \fitchctx{
     \subproof{\pline{P}}{
         \ellipsesline\\
         \pline{Q}
     }
     \pline{P \lif Q}
   }
  \end{quote}

  {
    \color{red}
    The rule states that at any point in a proof, assume \formula{P}, then, after deriving \formula{Q} from \formula{P} discharge assumption of \formula{P} and introduce \formula{P \lif Q}.
  }

  Start with an assumption , derive proposition-value pair.
  Conclude if X then Y.
  Indeed,~\citeauthor{Ramsey:1929tf} test for conditionals.%
  \footnote{
    See, for example,~\textcite{Read:1995wf}.
  }

  No premises.

  Parallel, pool of premises may be empty.

  Rather than being motivated by some pre-theoretical understanding of concluding, \autoref{assu:concluding:pools} is primarily a matter of convenience.
  Far easier to talk about concluding proposition-value pair from some pool of premises, than to cover distinct cases.

  What it means to say an agent concluded from empty pool of premises, set aside.

  Indeed, taking in terms of proposition-value-premises pairings will be key for our statement of \autoref{idea:support}, a key idea concerning concluding.
\end{note}

\begin{note}
  Final, consistency of agent's perspective.
\end{note}

\section{\support{2}}
\label{cha:clarification:support}

\begin{note}
  \color{red}
  Support is important for \qWhy{}.
  And, for issue.

  Two ideas, with some motivation.

  Finally, highlight an important thing support does.
\end{note}

\subsection{Support I}

\begin{note}
  \begin{idea}[Support I]
    \label{idea:support}
    For an agent \vAgent{}, a proposition-value pair \(\pv{\phi}{v}\) and pool of premises \(\Phi\):

    \begin{itemize}
    \item[\emph{If}]
      \begin{enumerate}[label=\alph*., ref=(\alph*)]
      \item
        \vAgent{} concluded \(\pv{\phi}{v}\) from \(\Phi\).
      \end{enumerate}
      \item[\emph{then}]
        \begin{enumerate}[label=\alph*., ref=(\alph*), resume]
        \item
          (A relation of) \emph{\support{}} held between \(\pv{\phi}{v}\) and \(\Phi\), when \vAgent{} concluded \(\pv{\phi}{v}\) from \(\Phi\), from the \vAgent{}' perspective.
        \end{enumerate}
      \end{itemize}
      \vspace{-\baselineskip}
  \end{idea}

  Primary role of \autoref{idea:support} is to distinguish concluding from other acts.

  `relation of support', but `\support{}' is clear.
\end{note}

\begin{note}
  As we will shortly see, links to reasons.
  However, support is distinct from reasons.

  \support{2} holds between conclusion and pool of premises.
  Between some proposition-value pair and a collection of proposition-value pairs.

  Explanatory reason holds between something which explains and an action.

  Support involved in providing explanatory reason.
\end{note}

\subsubsection{Motivation for support I}

\begin{note}
  For example,~\citeauthor{Boghossian:2014aa}'s Taking Condition:%
  \footnote{
    There are various objections to the taking condition.

    See, for example,~\textcite{Hlobil:2014tq}, \textcite{McHugh:2016vp}, and~\textcite{Wright:2014tt}.

    \citeauthor{Hlobil:2014tq} argues against the Taking Condition as it distracts from what accounts of reasoning out to explain, rather than arguing against the Taking Condition directly.

    \citeauthor{McHugh:2016vp} summarise various objects to the taking condition, and present district arguments against against (distinct) ideas in favour of the taking condition.
    In particular,~\autoref{idea:support} is closer to what \citeauthor{McHugh:2016vp} term the `Consequence Condition' (\citeyear[cf.][316]{McHugh:2016vp}), which \citeauthor{McHugh:2016vp} also (indirectly) argue against.
    However, \citeauthor{McHugh:2016vp} does not consider an alternative account of what distinguishes concluding from any other action, and as~\autoref{idea:support} is designed to capture this distinction, it is unclear to me whether \citeauthor{McHugh:2016vp}'s arguments apply to~\autoref{idea:support} (if, indeed, they are sound).

    \citeauthor{Wright:2014tt} denies that reasoning must involve a state which connects premises to conclusions. (\citeyear[Cf.][33-34]{Wright:2014tt})
    Note,~\autoref{idea:support} is compatible with \citeauthor{Wright:2014tt}'s objection.
  }

  \begin{quote}
    (Taking Condition):
    Inferring necessarily involves the thinker \emph{taking} his premises to support his conclusion and drawing his conclusion because of that fact.%
    \mbox{}\hfill\mbox{(\citeyear[5]{Boghossian:2014aa})}
  \end{quote}

  \begin{quote}
    The intuition behind the Taking Condition is that no causal process counts as inference, unless it consists in an attempt to arrive at a belief by figuring out what, in some suitably broad sense, is supported by other things one believes.%
    \mbox{}\hfill\mbox{(\citeyear[5]{Boghossian:2014aa})}
  \end{quote}

  Inference --- and hence reasoning with beliefs --- rather than reasoning more broadly (\citeyear[cf][2]{Boghossian:2014aa}).

  \citeauthor{Boghossian:2014aa}, `taking'.
  Taking is required for support to explain.
  No explanation without taking.
\end{note}

\begin{note}
  \citeauthor{Broome:2002aa}'s `jogging account' of reasoning:

  \begin{quote}
    [I]n reasoning you call to mind some of the premises, and doing so jogs into operation an automatic process that causes you to acquire a conclusion-attitude.%
    \mbox{}\hfill\mbox{(\citeyear[226]{Broome:2002aa})}
  \end{quote}

  {
    \color{red}
    We will talk about reasoning later.
    For the moment, substitute in concluding.
    }

  \citeauthor{Broome:2002aa} argues some things which satisfy jogging are clearly not reasoning.
  For example, \citeauthor{Broome:2002aa} considers
  {
    \color{red}
    Suppose you believe that it is raining and that if it is raining the snow will melt. Suppose these beliefs are conscious, and suppose they cause you to believe you hear trumpets.
  }
  (\citeyear[225,226--227]{Broome:2002aa})%
  \footnote{
    Does not rule out concluding (\citeyear[231,233]{Broome:2002aa}).
  }

  \citeauthor{Broome:2002aa} endorses rule following.

  \begin{quote}
    Active reasoning is a particular sort of process by which conscious premise-attitudes cause you to acquire a conclusion-attitude.
    The process is that you operate on the contents of your premise-attitudes following a rule, to construct the conclusion, which is the content of a new attitude of yours that you acquire in the process.\newline
    \mbox{ }\hfill\mbox{(\citeyear[234]{Broome:2002aa})}
  \end{quote}

  Understand support of \autoref{idea:support} in terms of having followed a rule.
\end{note}

\begin{note}
  Present purposes, assume support is involved in providing explanatory reasons for concluding.
\end{note}

\begin{note}
  Key assumption regarding concluding.
  Further assumptions detailed in~\autoref{chapter:concluding}.
\end{note}

\begin{note}
  This is somewhat subtle.
  As we're treating support very generally, don't require appreciation or anything like this.
  In this respect, quite different from \citeauthor{Boghossian:2014aa}.

  Of course, on a theory like \citeauthor{Boghossian:2014aa}'s, support does have a role in how, as the taking condition requires the agent takes conclusion.
  Here, relevant that \support{}, taking is then evaluation, and this does not need to be intentional.%
  \footnote{
    Problems of rule following.
  }

  In some respects, plausible.
  Given \autoref{idea:support}, support is key to characterising some act as concluding.
  From the agent's perspective, \(\Phi\) support \(\pv{\phi}{v}\).
  However, this is a state of affairs.
  Does not follow that \support{} from the agent's perspective is also part of that state of affairs.

  Consider, for example, \citeauthor{Wright:2014tt}'s `Simple Proposal':
  \begin{quote}
    \dots consider instead the proposal, not that the status of the transition as inferential depends on the thinker's judgments about his reasons, but that it depends on \emph{what his reasons are}.
    We want his acceptance of the premises to supply his \emph{actual} reasons for accepting the conclusion.

    \mbox{}\hfill\(\vdots\)\hfill\mbox{}

    Call this the Simple Proposal.
    It says that a thinker infers q from p\(_{1}\) \(\cdots\) p\(_{\text{n}}\) when he accepts each of p\(_{1}\) \(\cdots\) p\(_{\text{n}}\), moves to accept q, and does so for the reason that he accepts p\(_{1}\) \(\cdots\) p\(_{\text{n}}\).\newline
    \mbox{}\hfill\mbox{(\citeyear[33]{Wright:2014tt})}
  \end{quote}

  \support{2} from accepting pool of premises and then accepting conclusion.
  However, from agent's perspective, relation between pool of premises and conclusion is not part of why agent moves to accept conclusion, pool of premises alone is sufficient.

  Still, \support{}.
  What answers is relation between premises and conclusion.
\end{note}

\subsection{Support II}

\begin{note}
  Motivation for~\autoref{idea:support}, distinguish concluding.
  \support{2} may capture what is distinctive.

  For present purposes, abstract nature of support, second function.
  Distinguish support from concluding.
\end{note}

\begin{note}
  \begin{idea}[Support II]
    \label{idea:support:possible}
    For an agent \vAgent{}, and proposition-value-premises pairing \(\pvp{\phi}{v}{\Phi}\):

    \begin{itemize}
    \item
      It is possible for there to be \support{} between \(\pv{\phi}{v}\) and \(\Phi\) without \vAgent{} having concluded \(\pv{\phi}{v}\) from \(\Phi\), from \vAgent{}' perspective.
    \end{itemize}
    \vspace{-\baselineskip}
  \end{idea}

  \autoref{idea:support:possible} is important for the arguments to follow.
  For, in \autoref{cha:clarification:sec:support-qWhy} we will prevent a variation of \qWhy{} in which support between a proposition-value pair and a pool of premises answers, in part, why an agent concluded.
  And, in \dots \qHow{}, reasoning.

  \autoref{idea:support:possible} then, establishes the \emph{possibility} of support answering why, without the proposition-value-premises pairing being involved in how.

  The agent, is such that, from their perspective they have the option of concluding \(\pv{\phi}{v}\) from \(\Psi\).
  If so, then by \autoref{idea:support}, taking the option would involve a \support{}.
  However, \autoref{idea:support:possible}, possible without.
  Therefore, \support{}, from the agent's perspective.
\end{note}

\begin{note}
  Important point to stress here, \support{} from agent's perspective.
  However, there is no immediate link between such a relation of \support{} and \qWhy{}.

  Brief motivation for \autoref{idea:support:possible}, then turn to clarifying this.
\end{note}

\subsubparagraph{Motivation via prop and dox}

\begin{note}
  \autoref{idea:support:possible}, motivation from the distinction between propositional and doxastic justification.%
  \footnote{
    \citeauthor{Firth:1978vi}'s (\citeyear{Firth:1978vi}) distinction between doxastic and propositional justification (or warrant).
    See also \citeauthor{Silva:2020aa} (\citeyear{Silva:2020aa}) --- esp.\ fn.\ 1.
  }

  Here, this clearly holds.

  Whether agent gets that they have propositional justification without doxastic justification is somewhat difficult.

  Various cases, fails.

  For example, look at Sudoku.
  Now, in a sense propositional justification for whether or not number is in square.
  Good understanding of rules, and easy to do.

  Now, suppose filled in with some clues.
  Here, justification for which numbers are not possible.
  Doxastic justification, via testimony.
  However, seems also get that you have propositional justification from understanding of Sudoku.

  However, from that have propositional justification, don't get doxastic justification.
  For this, would need to go from understanding of Sudoku to rejection of number.
  Going from A to B is built in to understanding of doxastic justification.
\end{note}

\begin{note}
  This proposition, idea is that allow support to be arbitrarily strong, under constraint that no witnessing.

  So, witnessing isn't necessary for \support{}.
  However, it may still be the case that witnessing is necessary to determine whether \support{}, from agent's perspective.

  Strictly speaking, though, this isn't too important.

  Consider against propositional and doxastic justification.
  Similar issue, it seems plausible that in order to get that propositional justification for square via rules, first need to get doxastic justification via testimony.
  Only after doxastic justification do you get propositional justification.

  So, same holds for concluding.

  Indeed, could go through all of this with propositional and doxastic justification.
  Though, a little less clear.
  Propositional justification is in part answer to why without doxastic justification.
  Difficulty is arguing for this.
  For, depends on theory.
  Well, in particular, what grants propositional justification.
  In example, easy to get.

  Concluding, deny this.
  Indeed, seems to be intuitive thing about reasoning.
\end{note}

\subsection{Summary of support}
\label{sec:summary-support}

\begin{note}
  Understanding of support is generic.

  From agent's perspective.
  Perhaps the calculator is faulty, or the agent has an a unsteady grip on arithmetic.
  Still, conclusion from somewhere.
  Something account for why the agent holds.
\end{note}


\section{Witnessing}
\label{sec:reasoning}

\begin{note}
  \begin{definition}[Witnessing]
    An agent has \emph{witnessed} reasoning to \(\pv{\phi}{v}\) from \(\Phi\) \emph{only if} at some point in the past, the agent has reasoned to \(\pv{\phi}{v}\) from \(\Phi\).
  \end{definition}

  Term `witnessing' to make it clear that the agent has reasoned.
  The agent hasn't (merely), for example, entertained or forbidden reasoning from \(\Phi\) to \(\pv{\phi}{v}\).

  Performed some act.
  Moved, somehow, from pools of proposition-value pairs to other pools of proposition-value pairs.

  Do not assume that result of reasoning is concluding.
  However, as we will see, concluding only if reasoning.
\end{note}

\begin{note}
  Example, pairing of \(23 \times 15 = 345\) and the testimony of the calculator that \(23 \times 15 = 345\) answers, in part, how the agent concluded \(23 \times 15 = 345\) in \autoref{illu:gist:calc}.
  Slightly more natural to say `the testimony of the calculator that \(23 \times 15 = 345\)', but \emph{paring}.

  Observed that, intuitively, pairing of \(23 \times 15 = 345\) and whatever pool of premises would be associated with the agent applying their understanding of arithmetic does not answer, in part, how the agent concluded \(23 \times 15 = 345\).
\end{note}

\begin{note}
  \begin{idea}
    \label{assu:C-culmination-of-R}
    An agent has concluded \(\pv{\phi}{v}\) from \(\Phi\) only if agent has witnessed reasoning to \(\pv{\phi}{v}\) from \(\Phi\).
  \end{idea}

  \autoref{assu:C-culmination-of-R} distinguishes concluding.

  Together with earlier idea, if concluded, then both witnessed, and \support{}.

  Extension, \support{} \emph{by} witnessing.

  However, with Support II, it's not the case that \support{} only if witnessed reasoning.
\end{note}


\begin{note}
  \autoref{assu:C-culmination-of-R}.

  Concluding is the result of witnessing reasoning.
\end{note}

\subsection{No constraints}

\begin{note}[No constraints]
  No constraints placed on reasoning.

  In line with, for example, \citeauthor{Broome:2013aa}.

  \begin{quote}
    Following this rule would lead you to believe you hear trumpets when you believe it is raining and believe that if it is raining the snow will melt. If you did this, should we count you as reasoning?

    I think we should. If you derive this conclusion by operating on the premises, following the rule, we should count you as reasoning.

    \mbox{}\hfill\(\vdots\)\hfill\mbox{}

    I think we should not impose a limit on rules.%
    \mbox{}\hfill\mbox{(\citeyear[233]{Broome:2013aa})}
  \end{quote}
\end{note}

\begin{note}
  Plausibly contrasts with an account of reasoning such as \citeauthor{Wedgwood:2006ui}'s (\citeyear{Wedgwood:2006ui}) account of reasoning.

  \begin{quote}
    % So, the disposition that one must manifest in forming a belief in \emph{p} by means of reasoning must be one that can be specified by means of a function that, for \emph{any} proposition \emph{q} within the relevant range, maps the stimulus event-type \emph{coming to be in some mental states or other that rationalize forming a belief in q} onto the response event-type \emph{forming a belief in q}.
    [T]he disposition that one must manifest in reasoning is a disposition that responds to the fact that one is in some mental states or other that rationalize one's forming the belief or intention in question.\newline
    \mbox{ }\hfill\mbox{(\citeyear[672]{Wedgwood:2006ui})}
  \end{quote}

  Where:

  \begin{quote}
    If a set of antecedent mental states makes it rational for one to form a new belief or intention, then those antecedent mental states are surely of a suitable type and content so that it is \emph{intelligible} that they could represent one's reason for forming that belief or intention.\newline
    \mbox{ }\hfill\mbox{(\citeyear[662]{Wedgwood:2006ui})}
  \end{quote}

  Of course, this is only sufficient condition, but \citeauthor{Wedgwood:2006ui}'s account of reasoning is developed under the assumption the condition is also necessary.%
  \footnote{
    So, here, problem with fallacious reasoning that \citeauthor{Wedgwood:2006ui} notes.
    Following brief discussion:
    \begin{quote}
      I shall proceed on the idealizing assumption that the only way in which a set of antecedent mental states can rationalize the formation of a belief or intention is by making that new belief or intention a \emph{rational} belief or intention to form.\newline
      \mbox{ }\hfill\mbox{(\citeyear[662]{Wedgwood:2006ui})}
    \end{quote}
    However, it's not clear to me that fallacious reasoning is a pressing problem for an account such as \citeauthor{Wedgwood:2006ui}'s.
    For, fallacious reasoning may be reasoning in the same way that a fake water pistol is a water pistol.
    I.e.\ fallacious reasoning need not be reasoning proper.
  }
\end{note}

\begin{note}
  So, allow reasoning to be broad.
  Hence, concluding also.
  Still, compatible with account such as \citeauthor{Wedgwood:2006ui}'s.

  No constraints means that we do not place constraints, it does not mean that we assume there are no constraints on reasoning.
\end{note}

\begin{note}
  Here, with \citeauthor{Wedgwood:2006ui}, antecedent mental states rationalise.
  Understand this in terms of \qWhy{}.
  So, this motivates.
  The mental states rationalise, hence, it seems, full account of why agent concludes in terms of the reasoning the agent does.

  There is also the disposition.
  \citeauthor{Wedgwood:2006ui} doesn't explicit include this in terms of rationalising.
  Though, I think this should plausibly be the case.%
  \footnote{
    See discussion regarding \citeauthor{Turri:2010aa}'s (\citeyear{Turri:2010aa}) account of doxastic justification in~\autoref{cha:fcs:sec:dox-just} (on~\autopageref{cha:fcs:sec:dox-just}).

    In short, \(p\) and \emph{if} \(p\) then \(q\) rationalise concluding \(q\).
    However, disposition.

    Tighten \citeauthor{Wedgwood:2006ui}'s account some disposition \emph{only} responds to rationalising mental states.
    Though, have no constrained the disposition.
    And, that the disposition is so constrained is, in part, why.
  }
\end{note}

\subsection[Reasoning-to vs.\ concluding]{Reasoning and concluding \hfill (Optional)}

\begin{note}
  Reasoning that concludes \(\phi\) has value \(v\) is distinct from reasoning:
  \begin{enumerate}[label=\Alph*., ref=(\Alph*)]
  \item
    \label{CS:delicacy:O}
    Whose conclusion is (merely) \emph{about} \(\phi\) having value \(v\), and does not `require' \(\phi\) has value \(v\).
  \item
    \label{CS:delicacy:A}
    That concludes \(\phi\) has value \(v\) \emph{assuming} \dots\space --- where `\emph{assuming} \dots\space' is expanded to include some proposition-value pair \(\pv{\chi}{v}\) such that they agent has \emph{not} concluded \(\pv{\chi}{v}\).
  \end{enumerate}
\end{note}

\begin{note}[Epistemic operator]
  The statement of \ref{CS:delicacy:O} is loose, but the underlying idea is straightforward.

  we typically express the conclusion of the agent's reasoning which is (merely) about \(\phi\) having value \(v\) with some adjective.
  For example:

  \begin{quote}
    \vAgent{} \(\{ \text{hopes}, \text{imagines}, \text{desires}, \text{thinks}, \dots \}\) \(\phi\) is true.%
    \footnote{
      More generally, has value \(v\).
    }
  \end{quote}
  Each candidate adjective may be read in a way which is compatible with past, present, or future reasoning which concludes \(\phi\) does not have value `true'.
  In this sense, the reasoning is (merely) \emph{about} \(\phi\) being true, and does not require \(\phi\) is true.

  However, the relevant adjective may also be read as an evaluation of the relevant situation.
  It may be \(\phi\) \emph{is} hoped for, imagined, desired, thought, and so on.
  In this sense, the reasoning concludes \(\phi\) has some value \(v\), where the value is some value distinct from `true'.

  For the most part, `true' will be the relevant value \(v\) when discussing instances of reasoning.
  And, you may (unless there is clear tension) substitute mention or use of `\(\phi\) having value \(v\)' for `\(\phi\) being true'.
  Still, we do not require that `true' is the unique value of interest, and on occasion we will observe how various different valuations interact with observations made.
\end{note}

\begin{note}[Suppositions]
  \ref{CS:delicacy:A} is more straightforward.
  Contrast the instances of reasoning in~\autoref{fig:Rover}.
  \begin{figure}[h!]
    \mbox{}\hfill
    \begin{subfigure}{0.45\linewidth}
      \begin{enumerate}[label=\arabic*.,ref=(\arabic*)]
      \item
        \label{fig:Rover:CS:1}
        Rover is tired.
      \item
        \label{fig:Rover:CS:2}
        Rover will fall asleep soon.
      \end{enumerate}
      \caption{}
      \label{fig:Rover:CS}
    \end{subfigure}
    \hfill
    \begin{subfigure}{0.45\linewidth}
      \begin{enumerate}[label=\arabic*\('\).,ref=(\arabic*\('\))]
      \item
        \label{fig:Rover:nCS:1}
        \emph{Supposing} Rover is tired.
      \item
        \label{fig:Rover:nCS:2}
        Rover will fall asleep soon.
      \end{enumerate}
      \caption{}
      \label{fig:Rover:nCS}
    \end{subfigure}
    \hfill\mbox{}
    \caption{Two instance of reasoning}
    \label{fig:Rover}
  \end{figure}

  \ref{fig:Rover:CS} and \ref{fig:Rover:nCS} are distinguishing by whether or not the premise that Rover is tired is a supposition \ref{fig:Rover:nCS} or not \ref{fig:Rover:CS}.

  It may be the case that Rover would fall asleep soon if Rover were tired, but as \ref{fig:Rover:nCS:1} does not concern the actual state of Rover, \ref{fig:Rover:nCS:2} need not concern the actual state of Rover.

  By contrast,~\ref{fig:Rover:CS} is an instance of reasoning about how things are.
  The agent is holding that it is the case that Rover is tired, and therefore it is the case that Rover will fall asleep soon.

  I take this distinction to be straightforward.
  Though, it is not always immediately clear how the distinction applies to some conclusion of reasoning when stated independently of the preceding reasoning.
  Consider the conclusion:

  \begin{quote}
    Rover is likely asleep.
  \end{quote}

  Following the distinction, there are two broad ways of interpreting the conclusion:
  \begin{enumerate}
  \item
    The agent has made some plausible assumptions and granting those assumptions, Rover is asleep.
  \item
    For every proposition-value pair \(\chi_{i}\) having value \(v_{i}\) the agent has appealed to, the agent hold it to be the case \(\chi_{i}\) has value \(v_{i}\).
    However, the agent only concludes from \(\chi_{i}\) having values \(v_{i}\) that there is some (objective or subjective) chance that Rover is asleep.
  \end{enumerate}
  So, though claiming support is for some proposition \(\phi\) having some value \(v\), it is not clearly the case that claiming support is restricted to certain proposition-value combinations.
\end{note}

\subsection{Summary}
\label{sec:summary-1}

\begin{note}
  Introduced support and reasoning.
  Assumptions regarding support, and reasoning.
  In particular, support without reasoning.

  In the following sections, relate \support{} and reasoning to \qWhy{} and \qHow{}.
\end{note}

\section{Support and \qWhy{}}
\label{cha:clarification:sec:support-qWhy}
\label{cha:clar:expand:qWhy}

\begin{note}[Introduction]
  We now turn to \qWhy{}:
  \vspace{-\baselineskip}
  \begin{quote}
    \questionWhyBasic*
  \end{quote}
  Example, pairing of \(23 \times 15 = 345\) and the testimony of the calculator that \(23 \times 15 = 345\) answers, in part, why the agent concluded \(23 \times 15 = 345\) in \autoref{illu:gist:calc}.
  Slightly more natural to say `the testimony of the calculator that \(23 \times 15 = 345\)', but \emph{paring}.

  Observed that, intuitively, pairing of \(23 \times 15 = 345\) and whatever pool of premises would be associated with the agent applying their understanding of arithmetic does not answer, in part, why the agent concluded \(23 \times 15 = 345\).
\end{note}

\begin{note}
  Motivation by intuition, but also via positive resolution to \issueInclusion{}.
\end{note}

\paragraph{Variant of \qWhy{}}
\label{cha:clar:expand:qWhy:variant}

\begin{note}[The variant]
  Given role of support, and ideas, variant of \qWhy{}:

  \begin{question}[\qWhyV{}]
    \label{q:why:v}
    Which proposition-value-premises pairings are such that:
    \begin{enumerate}
    \item[\emph{If}:]
      \begin{enumerate}[label=\alph*., ref=(\alph*)]
      \item
        \support{} between the proposition-value pair and the pool of premises failed to hold, from \vAgent{}' perspective.
      \end{enumerate}
    \item[\emph{then}:]
      \begin{enumerate}[label=\alph*., ref=(\alph*), resume]
      \item \vAgent{} would not have concluded \(\pv{\phi}{v}\) from \(\Phi\).
      \end{enumerate}
    \end{enumerate}
  \end{question}

  \support{}, rather than simple pairing.

  Note, here, elimination of `why'.

  Dependence.
\end{note}

\begin{note}
  \qWhyV{} dependence.

  As a result, no guarantee that support between \(\pv{\phi}{v}\) and \(\Phi\) answers \qWhyV{}.
  However, given \autoref{idea:support}, if the agent has concluded, then \support{} between \(\pv{\phi}{v}\) and \(\Phi\) is available.

  This should also not be too much of a worry.
  We also offered no guarantee that \(\pvp{\phi}{v}{\Phi}\) is an answer to \qWhy{}.

  Further, plausible that this may be argued for.
  However, our interest is with other proposition-value-premises pairings.
\end{note}

\begin{note}
  Link between \qWhy{} and \qWhyV{} follows by:

  \begin{restatable}{link}{linkSupportWhy}
    \label{link:why:support:pvpp}
    For an agent \vAgent{}, and proposition-value-premises pairings \(\pvp{\psi}{v'}{\Psi}\), \(\pvp{\phi}{v}{\Phi}\):
    \begin{itemize}
    \item[\emph{If}]
      \begin{enumerate}[label=\alph*., ref=(\alph*)]
      \item
        \support{2} between \(\pv{\psi}{v'}\) and \(\Psi\) is, in part, an answer to \qWhyV{} \vAgent{} concluded \(\pv{\phi}{v}\) from \(\Phi\).
      \end{enumerate}
    \item[\emph{then}]
      \begin{enumerate}[label=\alph*., ref=(\alph*), resume]
      \item
        \(\pvp{\psi}{v'}{\Psi}\) is, in part, an answer to \qWhy{} \vAgent{} concluded \(\pv{\phi}{v}\) from \(\Phi\).
      \end{enumerate}
    \end{itemize}
    \vspace{-\baselineskip}
  \end{restatable}
  {
    \color{red}
    View this as a constraint on our understanding of \qWhy{}.
    Or, as a constraint on support.
    (The only difficulty here is if, somehow, understand \qWhy{} in such a way that rules this out\dots)
  }
  Difficulty for any negative view is getting support without how.

  If with intuition for positive resolution, then this should also look fine.
\end{note}

\subsubsection{Dispositions and Deviance}
\label{sec:deviance-1}

\begin{note}
  The statement of \autoref{q:why:v} is designed to allow the connexion between \support{} failing to hold and the agent not concluding to be obscure from the agent's perspective.

  Consider the following suggestions:

  \begin{suggestion}
    Which proposition-value-premises pairings are such that:
    \begin{itemize}
    \item
      From \vAgent{}' perspective:
      \begin{itemize}
      \item
        If \support{} between the proposition-value pair and the pool of premises failed to hold, then \vAgent{} would not have concluded \(\pv{\phi}{v}\) from \(\Phi\).
      \end{itemize}
    \end{itemize}
  \end{suggestion}

  Here, conditional is from the agent's perspective.
  This doesn't get a link to concluding.

  In this case, there is no guarantee that \support{} mattered for concluding.
  Gets close to dispositional view of reasons.

  However, it's not quite the same.
  \qWhyV{} is target, but there is no difficulty in suggestions entailing.

  Specifically, answers to first suggestion for which the following also holds.

    \begin{suggestion}
    Which proposition-value-premises pairings are such that:
    \begin{enumerate}
    \item[\emph{If}:]
      \begin{enumerate}[label=\alph*., ref=(\alph*)]
      \item
        \support{2} between the proposition-value pair and the pool of premises failed to hold, from \vAgent{}' perspective.
      \end{enumerate}
    \item[\emph{then}:]
      \begin{enumerate}[label=\alph*., ref=(\alph*), resume]
      \item \vAgent{} would not have concluded \(\pv{\phi}{v}\) from \(\Phi\), from \vAgent{}' perspective.
      \end{enumerate}
    \end{enumerate}
  \end{suggestion}

  And, here, where the b condition falls out.
  Cases in which deviance fails.
\end{note}

\subsubsection{Identifying something as a premise}
\label{cha:clarification:sec:embedding}

\begin{note}
  Above, we observed how \label{idea:support} and~\autoref{idea:support:possible} may combine in such a way that \support{} holds between \(\pv{\psi}{v'}\) and \(\Phi\) (from an agent's perspective), though the agent has not concluded \(\pv{\psi}{v'}\) from \(\Phi\).

  Here, then, way in which \support{} may answer \qWhyV{}, and hence pairing for \qWhy{} via~\autoref{link:why:support:pvpp}.

  Indeed, this is exactly what we will argue for.

  However, relation between \support{} holding from agent's perspective and \support{} answering \qWhyV{} is delicate.

  For, there is a significant difference between the following two answers to \qWhyV{}:
  \begin{enumerate}
  \item
    \support{2} between \(\pv{\psi}{v'}\) and \(\Psi\).
  \item
    \support{2} between (\emph{\support{} between \(\pv{\psi}{v'}\) and \(\Psi\)}) and \(\pv{\phi}{v}\).
  \end{enumerate}
  Or, more generally in the case of 2, \support{} between \(\Phi\) and \(\pv{\phi}{v}\) such that it being true that \support{} holds between \(\pv{\psi}{v'}\) and \(\Psi\) is an element of the pool of premises \(\Phi\).

  Speaking of \support{} may be strained, but it helps make this distinction.
  And, this distinction is important.%
  \footnote{
    This distinction will return in \autoref{cha:zSpAwhy}.
  }

  Difference between \support{}, in part, answering \qWhyV{} and \support{} embedded as a premise.
\end{note}

\paragraph{Conditional}

\begin{note}
  Simple illustration, conditional.

  On the assumption of \(\pv{\chi}{v''}\), get \(\pv{\psi}{v'}\).
  Turn this into a conditional.

  Now, \(\pv{\chi}{v''}\).

  So, given the reasoning, \support{} between \(\pv{\chi}{v''}\) and \(\pv{\chi}{v''}\).

  However, conclude from conditional via detachment.

  So, what answers \qWhyV{} is \support{} between \(\pv{\chi}{v''}\) and conditional and \(\pv{\psi}{v'}\).

  Though, it may also be the case that \support{} between \(\pv{\chi}{v''}\) and \(\pv{\chi}{v''}\) remains important.
  For, still have the option to conclude.

  However, this gets embedded.
  This \support{} does not answer \qWhyV{}.
  For, linked to conditional, and premise.
\end{note}

\begin{note}
  Phrase this in terms of memory.
  Concluded.
  \support{2} held.
  However, \support{} continues to hold.
  However, still conclude from memory.
\end{note}

\subparagraph{\citeauthor{Owens:2006tw}}

\begin{note}
  For example, \citeauthor{Owens:2006tw} argues for a belief expression model of assertion in which the rationality of a belief formed by an agent via testimony is connected to justification of the testifier:

  \begin{quote}
    Trusting an expression of belief by accepting what a speaker says involves entering a state of mind which gets its rationality from the rationality of the belief expressed.
    This state's rationality depends on the speaker's justification for the belief he expresses, not on his justification for the action of expressing it.
    And to hear a speaker as making a sincere assertion, as expressing a belief, is \emph{ceteris paribus} to feel able to tap into \emph{that} justification (whether or not his assertion was directed at you) by accepting what he says.%
    \mbox{}\hfill\mbox{(\citeyear[123]{Owens:2006tw})}
  \end{quote}

  On the view advanced by \citeauthor{Owens:2006tw}, justification.
  View in terms of \support{}.

  \support{} directly.
  Rationality of agent is rationality of speaker.

  However, `depends'.

  Distinction between rationality of state, and relation between rationality of state and rationality of state.

  Inclined to think \citeauthor{Owens:2006tw} is arguing for the former.%
  \footnote{
    \begin{quote}
      If we are to believe what the speaker indicates he believes, either the speaker must justify this belief to us, or we must supply some justification of our own
      \dots
      Neither act can be part of a rationality preserving mechanism for belief.%
      \mbox{ }\hfill\mbox{(\citeyear[123--124]{Owens:2006tw})}
    \end{quote}
  }
  Though, it is not clear to me that embedded isn't a viable option.

  Regardless, distinction that is important.
\end{note}

\begin{note}
  Same distinction holds for answers to \qWhyV{}.

  It may be the case that \support{} between \(\pv{\psi}{v'}\) and \(\Phi\) is, from the agent's perspective, involved in concluding \(\pv{\phi}{v}\) from \(\Phi\).

  However, no immediate move from this to \support{} being, in part, an answer to \qWhyV{}.
\end{note}


\subparagraph[Conversely]{Conversely \hfill (Optional)}

\begin{noteP}[Converse to~\autoref{link:why:support:pvpp}]
  \autoref{link:why:pvpp:support} is the converse of~\autoref{link:why:support:pvpp}:

  \begin{restatable}[]{link}{linkWhySupport}
    \label{link:why:pvpp:support}
    For an agent \vAgent{}, and proposition-value-premises pairings \(\pvp{\psi}{v'}{\Psi}\) and \(\pvp{\phi}{v}{\Phi}\):
    \begin{itemize}
    \item[\emph{If}]
      \begin{enumerate}[label=\alph*., ref=(\alph*)]
      \item
        \(\pvp{\psi}{v'}{\Psi}\) is, in part, an answer to why \vAgent{} concluded \(\pv{\phi}{v}\) from \(\Phi\).
      \end{enumerate}
    \item[\emph{then}]
      \begin{enumerate}[label=\alph*., ref=(\alph*), resume]
      \item
        \support{2} between \(\pv{\psi}{v'}\) and \(\Psi\) is, in part, an answer to why \vAgent{} concluded \(\pv{\phi}{v}\) from \(\Phi\).
      \end{enumerate}
    \end{itemize}
    \vspace{-\baselineskip}
  \end{restatable}

  Combined, \autoref{link:why:pvpp:support} and \autoref{link:why:support:pvpp} may be taken to provide a characterisation the way in which a proposition-value-premises pairing \(\pvp{\psi}{v'}{\Psi}\) answers, in part, why an agent concludes \(\pv{\phi}{v}\) from \(\Phi\):
  \support{2} holds between \(\pvp{\psi}{v'}{\Psi}\) and \(\Psi\).

  In other words, support captures the relevant relation between a conclusion proposition-value pair and pool of premises, from perspective of agent, when concluding.
  However, we will not press this idea further.
\end{noteP}

\section{Reasoning and \qHow{}}
\label{sec:overview:reasoning}
\label{cha:clar:expand:qHow}

\begin{note}[Introduction]
  We now turn to \qHow{}:
  \vspace{-\baselineskip}
  \begin{quote}
    \questionHowBasic*
  \end{quote}

  In parallel to \qWhyV{}, our goal is to provide a variation of \qHow{} which does captures conditions for answers \qHow{}.
  The type of link between \qWhyV{} and \qWhy{} was sufficiency.%
  \footnote{
    In short, we have stated a conclusion of \(\pv{\phi}{v}\) from \(\Phi\) {\color{red} depending} on \support{} between \(\pv{\psi}{v'}\) and \(\Psi\) (from the concluder's perspective)is sufficient for \(\pvp{\psi}{v'}{\Psi}\) to answer why the concluder concluded \(\pv{\phi}{v}\) from \(\Phi\).
  }
  The type of link between \qHow{} and it's variant will is that of necessity.

  In this respect, the variant of \qHow{} is weak:

  \begin{question}[\qHowV{}]
    \label{q:how:v}
    Which proposition-value-premises pairing are such that the agent has witnessed reasoning to \(\pv{\psi}{v'}\) from \(\Psi\)?
  \end{question}

  In short, \qWhyV{} asks whether the concluder has witnessed reasoning to \(\pv{\psi}{v'}\) from \(\Psi\).
\end{note}

\begin{note}[The link]
  The link is:

  \begin{restatable}{link}{linkHowWitnessing}
    \label{link:how-witnessing}
    For any proposition-value-premises pairing \(\pvp{\psi}{v'}{\Psi}\):
    \begin{itemize}
    \item[\emph{If}]
      \begin{enumerate}[label=\alph*., ref=(\alph*)]
      \item
        \(\pvp{\psi}{v'}{\Psi}\) is, in part, an answer \qHow{}.
      \end{enumerate}
    \item[\emph{then}]
      \begin{enumerate}[label=\alph*., ref=(\alph*), resume]
      \item
        \(\pvp{\psi}{v'}{\Psi}\) is, in part, an answer \qHowV{}.
      \end{enumerate}
    \end{itemize}
    \vspace{-\baselineskip}
  \end{restatable}

  Really broad in two ways.%
  \footnote{
    Hence, no converse, in contrast to \autoref{link:why:support:pvpp}.
  }

  \qHowV{} is weak in two key ways:
  \begin{enumerate}
  \item
    \qHowV{} makes not reference to the agent concluding \(\pv{\phi}{v}\) from \(\Phi\).
  \item
    \qHowV{} only requires the agent has witnessed reasoning to \(\pv{\psi}{v'}\) from \(\Psi\).
  \end{enumerate}

  Intuitively, fair too weak to provide an answer to \qWhy{}.
  However, this is of little interest.
  Interest is in arguing against \issueInclusion{}.

  Pair sufficient for \qWhy{} and necessary for \qHow{}, then depends on \support{} such that agent has not witnessed.
  Make this clear in \autoref{cha:clar:expand:issue}.

  Stronger idea, then in principle, easier counterexamples.
\end{note}

\begin{note}
  Now, the motivation for \autoref{link:how-witnessing} is straightforward.
  For, from \autoref{assu:C-culmination-of-R}, concluded then witnessed.

  Concluding, going to \(\pv{\phi}{v}\) from \(\Phi\).

  So, if the agent has concluded \(\pv{\phi}{v}\) from \(\Phi\), then the agent has witnessed reasoning.
\end{note}

\begin{note}
  Of some interest is weaker.
  Provide short motivation for option of past tense and for something weaker than concluding.
\end{note}

\paragraph{No present tense}

\begin{note}[Illustration]
  To illustrate, consider an agent working on some mathematical problem.

  As part of their work on the problem the agent concludes the hypotenuse of some right-angled triangle is \(\sqrt{74}\text{cm}\) by use of the Pythagorean theorem.

  Further, the agent has, at some point in the past proved the Pythagorean theorem from more basic principles.

  Now, generally speaking, it seems to me it may be the case that the agent concludes the hypotenuse of the triangle is \(\sqrt{74}\text{cm}\), in part, from those more basic principles.
  For example, the agent may have just completed their proof of the Pythagorean theorem and the reasoning from the more basic principles to the hypotenuse of the triangle may be considered a single unified instances of reasoning, with an intermediary conclusion.

  Still, suppose the agent proved the Pythagorean theorem some years ago.

  Perhaps the agent's reasoning from more basic principles continues to provide, in part, an answer to how the agent concluded the hypotenuse of the triangle is \(\sqrt{74}\text{cm}\).
  There may be a gap of some years, but it may be the case that the agent uses the Pythagorean theorem (in some sense of the word) \emph{because} they concluded the theorem from more basic principles.
  {
    \color{red}
    Hence, \support{}, and this in part answers \qWhyV{}.
  }

  On the other hand, one may be inclined to hold that the more basic principles the agent proved the Pythagorean theorem from have no role in explaining how the agent concluded the hypotenuse of the triangle is \(\sqrt{74}\text{cm}\) in the present.
  Rather, the Pythagorean theorem is a more-or-less fundamental premise of the agent's present reasoning.

  At best, the agent's memory of reasoning from more basic principles to the Pythagorean theorem may, in part, answer how the agent concluded hypotenuse of the triangle is \(\sqrt{74}\text{cm}\).
  The reasoning from the more basic principles, given that it happened so long ago, is irrelevant.

  \autoref{link:how-witnessing} is designed to allow either opinion on the agent's conclusion.
\end{note}

\paragraph{Reasoning}

\begin{note}
  Second point, about reasoning.

  This is broad, but the point here is to allow there's something distinct about concluding.
  So, reasoning may be good, but strictly the agent did not conclude.
\end{note}


\newpage


\section{\issueInclusion{}}
\label{cha:clar:expand:issue}

\begin{note}
  \autoref{cha:clar:expand:qWhy} refined \qWhy{} in terms of support.
  \autoref{cha:clar:expand:qHow} refined, or perhaps weakened, \qHow{} in terms of witnessing.

  Interest in \qWhy{} and \qHow{} in terms of whether a constraint holds.

  \begin{quote}
    \vspace{-\baselineskip}
    \issueInclusionFirst*
  \end{quote}

  Introduced in~\autoref{cha:introduction}.
  Provided some intuitive motivation via \autoref{illu:gist:calc}.
  And, theoretical motivation via \citeauthor{Davidson:1963aa}' causal theory of action.

  We now bring refinements to questions together to provide refinement to \issueInclusion{}.
\end{note}

\begin{note}
 Following:

  \begin{restatable}[]{proposition}{propVariationsAndInclusion}
    \label{prop:support-and-witnessing}
    Grating a positive resolution to \issueInclusion{}, and given \autoref{link:why:support:pvpp} and \autoref{link:how-witnessing}:
    \begin{enumerate}
    \item[\emph{If}]
      \begin{enumerate}[label=\alph*., ref=(\alph*)]
      \item \vAgent{} concluded \(\pv{\phi}{v}\) from \(\Phi\).
      \end{enumerate}
    \item[\emph{And}]
      \begin{enumerate}[label=\alph*., ref=(\alph*), resume]
      \item
        \vAgent{} would not have concluded \(\pv{\phi}{v}\) from \(\Phi\), If \support{} between \(\pv{\psi}{v'}\) and \(\Psi\) failed to hold, from \vAgent{}' perspective.
      \end{enumerate}
    \item[\emph{Then}]
      \begin{enumerate}[label=\alph*., ref=(\alph*), resume]
      \item
        \vAgent{} has witnessed reasoning to \(\pv{\psi}{v'}\) from \(\Psi\).
      \end{enumerate}
    \end{enumerate}
  \end{restatable}

  \autoref{prop:support-and-witnessing} is straightforward.
  A visual representation is given in~\autoref{fig:relations-between-whys-and-hows}.
\end{note}

\begin{figure}[H]
  \centering
  \begin{tikzpicture}
    \tikzset{ansStyle/.style={
        draw=gray,
        text width=.45\textwidth,
        rounded corners=2pt,
      }
    }
    %
    \node[ansStyle] (whyO) at (0,0) %
    {\qWhyV{} is answered by support between \(\pv{\psi}{v'}\) and \(\Psi\).};
    %
    \node[ansStyle] (whyA) at (2,-1.5) %
    {\qWhy{} is answered by \(\pvp{\psi}{v'}{\Psi}\).};
    %
    \node[ansStyle] (howA) at (4,-3) %
    {\qHow{} is answered by \(\pvp{\psi}{v'}{\Psi}\).};
    %
    \node[ansStyle] (witA) at (6,-4.5) %
    {\qHowV{} is answered by witnessed reasoning from \(\Psi\) to \(\pv{\psi}{v'}\).};
    %
    \path[->] ($(whyO.south)!0.9!(whyO.south west)$) edge [out=270, in=180] (whyA);
    \path[->] ($(whyA.south)!0.9!(whyA.south west)$) edge [out=270, in=180] (howA);
    \path[->] ($(howA.south)!0.9!(howA.south west)$) edge [out=270, in=180] (witA);
    %
    \node[text width=.5\textwidth] (1) at (1,-.8) %
    {Via~\autoref{link:why:support:pvpp}.};
    %
    \node[text width=.75\textwidth] (2) at (4.5,-2.25) %
    {Via a positive resolution to~\issueInclusion{}.};
    %
    \node[text width=.5\textwidth] (3) at (5,-3.625) %
    {Via~\autoref{link:how-witnessing}.};
    %
    \draw[->, gray] ($(whyA.north)!0.9!(whyA.north east)$) to [out=90, in=0] ($(whyO.east)$);
    %
    \node[text width=.5\textwidth, text=gray] (1p) at (8,-.8) %
    {Via~\autoref{link:why:pvpp:support}.};
  \end{tikzpicture}%
  \caption{Relation between positive answers to questions.}
  \label{fig:relations-between-whys-and-hows}
\end{figure}

\begin{note}
  Following~\autoref{prop:support-and-witnessing}, we refine \issueInclusion{} to \issueConstraint{} by asking whether~\autoref{prop:support-and-witnessing} is the case:

  \begin{restatable}[\issueConstraint{}]{issue}{rIssueConstraint}
    \label{issue:has-witnessed}
    For an agent \vAgent{}:

    Is it the case that:

    \begin{enumerate}
    \item[\emph{If}]
      \begin{enumerate}[label=\alph*., ref=(\alph*)]
      \item \vAgent{} concluded \(\pv{\phi}{v}\) from \(\Phi\).
      \end{enumerate}
    \item[\emph{And}]
      \begin{enumerate}[label=\alph*., ref=(\alph*), resume]
      \item
        \vAgent{} would not have concluded \(\pv{\phi}{v}\) from \(\Phi\), if \support{} between \(\pv{\psi}{v'}\) and \(\Psi\) failed to hold, from \vAgent{}' perspective.
      \end{enumerate}
    \item[\emph{Then}]
      \begin{enumerate}[label=\alph*., ref=(\alph*), resume]
      \item
        \vAgent{} has witnessed reasoning to \(\pv{\psi}{v'}\) from \(\Psi\).
      \end{enumerate}
    \end{enumerate}
  \end{restatable}

  If negative resolution, three possibilities:
  ~\autoref{link:why:support:pvpp} fails,~\autoref{link:how-witnessing} fails, or a negative resolution to \issueInclusion{}.

  ~\autoref{link:why:support:pvpp} and~\autoref{link:how-witnessing} are sufficiently intuitive, negative resolution to \issueInclusion{}.

  However, \issueConstraint{} is sufficiently intuitive, that failure of \issueConstraint{} works.
\end{note}

\begin{note}
  As with \issueInclusion{}, \issueConstraint{} distinguishes classes of theories.
  A positive resolution to \issueInclusion{} will not directly provide general answer to \qWhy{} or \qHow{}.
  Though, a positive answer will rule out certain answers.
\end{note}

\subsection{Summary}
\label{cha:clar:expand:issue:summary}

\begin{note}
  Three key things.

  Support.
  Witnessing.
  Issue.
\end{note}

\begin{note}
  Focus on \issueConstraint{}.
  \vspace{-\baselineskip}
  \begin{quote}
    \rIssueConstraint*
  \end{quote}
  Sufficient clarity on both `why?' and `how?'.
  Link we have argued for.
  And, further, independently of argument, it seems to me that a positive resolution to \issueConstraint{} is equally compelling as positive resolution to \issueInclusion{}.
\end{note}

\begin{note}
  This is the important thing, and in this respect it doesn't matter whether past or present.
  Whether a \support{} holds only if witnessed.
  Whether resolution to \qWhy{} only if the agent has witnessed.
\end{note}

\begin{note}
  Difficulty with all of this is that the accounts seem to be consistent, but do not explicitly motivate this constraint.
\end{note}

\begin{note}
  An additional example, \citeauthor{Hieronymi:2011aa}
  \begin{quote}
    The proposal starts with this simple thought: whenever an agent acts for reasons, the agent, in some sense, takes certain considerations to settle the question of whether so to act, therein intends so to act, and executes that intention in action.

    If this much is uncontroversial (and, under some interpretation, I believe it must be), we can use it as a form for filling out.
    I propose, then, that we explain an event that is an action done for reasons by appealing to the fact that the agent took certain considerations to settle the question of whether to act in some way, therein intended so to act, and successfully executed that intention in action.
    I suggest that \emph{this} complex fact, \dots is the reason that rationalizes the action---that explains the action by giving the agent’s reason for acting.%
    \mbox{ }\hfill\mbox{(\citeyear[431]{Hieronymi:2011aa})}
  \end{quote}

  From the deliberation.

  So, reason is the complex fact.
  Complex fact gives the reason the agent acted, and so content of constituent considerations from agent's perspective.
\end{note}

%%% Local Variables:
%%% mode: latex
%%% TeX-master: "master"
%%% End:
