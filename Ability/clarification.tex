\chapter{Clarification}
\label{cha:clarification}

\begin{note}
  \autoref{cha:introduction} introduced the focal point of this document:
  The issue of whether a particular relation holds between answers to two distinct questions.
  Issue: \issueInclusion{}.
  Answers to why an agent concludes  --- answers to \qWhy{} --- constrained by how an agent concluded --- answers to \qHow{}.
  Our overall goal is to argue that there are counterexamples to \issueInclusion{}.

  The purpose of the present chapter is clarification.
  At the close of this chapter we will have variants of \qWhy{}, \qHow{}, and \issueInclusion{} and an understanding of how the variants relate to the initial questions and relation.

  Purpose of variants is to provide clarity.
  As a result, slight generalisation.
\end{note}

\begin{note}
  The chapter is split into two parts.
  In the first part we develop some foundations.
  Second part, foundations to develop variants.
  For each variant, link to initial question.

  \begin{itemize}
  \item
    Concluding \hfill \autoref{cha:introduction:sec:CCC:pvp}    Concluding \hfill \autoref{cha:introduction:sec:CCC:pvp}
    \begin{itemize}
    \item
      What we are interested in.
    \end{itemize}
  \item
    \dots, from the agent's perspective \hfill \autoref{cha:introduction:sec:agents-perspective}
    \begin{itemize}
    \item
      A common qualifier.
      Have seen in \autoref{cha:introduction}.
      Section gives a detail account and overview of role.
    \end{itemize}
  \end{itemize}

  \begin{itemize}
  \item
    Support and \qWhy{} \hfill \autoref{cha:clar:expand:qWhy}
  \item
    Witnessing and \qHow{} \hfill \autoref{cha:clar:expand:qHow}
  \item
    \issueInclusion{} \hfill \autoref{cha:clar:expand:issue}
  \end{itemize}
\end{note}

\section{Concluding, concludes, conclusion}
\label{cha:introduction:sec:CCC}

\begin{note}
  \qWhy{}, \qHow{}, and \issueInclusion{} are all stated with respect to concluding.
  Variations developed below will likewise stated with respect to concluding.
\end{note}

\begin{note}
  Our interest is in concluding.
  In this section, overview by considering the interaction between three terms:
  `concluding', `concludes', `conclusion'.

  Roughly, we hold:

  \begin{itemize}
  \item
    A conclusion of an agent is the state of a proposition being paired with a value by the agent.
  \item
    An event in which an agent concludes results in a conclusion of an agent.
  \item
    An event of concluding is such that an event in which an agent concludes is in progresses.
  \end{itemize}

  Hence, we understand `concluding' in terms of `concludes', and `concludes' in terms of `conclusion'.
  The characterisation of conclusions as proposition-value pairings is fundamental to the presentation of this document, and our initial focus will be on how such a pairing is understood.
  The events in which an agent concludes or is concluding are then understood in terms of what the event results in (`concludes') or is progressing towards (`concluding').

  In general, we understand a conclusion as the result of reasoning.
  Hence, an event in which an agent concludes is an event in which the agent reasons, and an event of concluding is an event of reasoning.

  Key distinction is that reasoning need result in the agent pairing a proposition with a value.
  \phantlabel{reasoning-vs-concluding-progressive}

  Indeed, we will often contrast concluding with reasoning.
  For, if the agent is concluding, then event in which the agent concludes is in progress.
  The event may be interrupted, and the agent may fail to conclude.
  However, if the event is not interrupted, will result in a conclusion.

  But contrast, the same need not be true of reasoning.
\end{note}

\subsection{Conclusions as proposition-value pairings}
\label{cha:introduction:sec:CCC:pvp}

\begin{note}
  Basic assumption:

  \begin{assumption}
    \label{assu:concluding:pvp}
    For an agent \vAgent{}:

    \begin{itemize}
    \item
      A conclusion is a pairing of a proposition with a value, from some \vAgent{}'s perspective.
    \end{itemize}

    Where:
    \begin{itemize}[noitemsep]
    \item
      A proposition is some state of affairs.
    \item
      A value captures an evaluation of a proposition.
    \end{itemize}
  \end{assumption}

  Propositions are states of affairs, whether actual, possible, true, false, ideal, or regrettable.
  Values capture how the agent evaluates the state of affairs.
  Conclusion is pairing proposition with some value.

  The converse need not be the case; a proposition may be paired with some value from an agent's perspective, without the pairing being a conclusion.
  The role of \autoref{assu:concluding:pvp} is to capture what a conclusion amounts to.
\end{note}

\begin{note}
  Before discussing the details of \autoref{assu:concluding:pvp}, we present some examples of proposition-value pairings:

  \begin{itemize}
  \item
    The bag is heavy.\newline
    \mbox{ }\hfill%
    \(\pv{\text{The bag is heavy}}{\text{True}}\)
  \item
    The cyclist isn't avoiding potholes.\newline
    \mbox{ }\hfill%
    \(\pv{\text{The cyclist is avoiding potholes}}{\text{False}}\)
  \item I like bitter coffee.\newline
    \mbox{ }\hfill%
    \(\pv{\text{Bitter coffee}}{\text{Desirable}}\)
  \item
    It's too bad the balloon is deflating.\newline
    \mbox{ }\hfill%
    \(\pv{\text{The balloon is deflating}}{\text{Unfortunate}}\)
  \item
    The dog shouldn't be drinking from the pond.\newline
    \mbox{ }\hfill%
    \(\pv{\text{The dog isn't drinking from the pond}}{\text{Ought-to-be}}\)
  \item
    There is no chance \dots
    \mbox{ }\hfill%
    \(\pv{\text{\dots}}{\text{Improbable}}\)
  \end{itemize}

  Nothing in particular hangs on these examples.
  In particular, the relevant values may be restricted to `true' and `false' and the corresponding propositions expanded.
  For example:
  \begin{itemize}
  \item
    The dog shouldn't be drinking from the pond.\newline
    \mbox{ }\hfill%
    \(\pv{\text{The dog shouldn't be drinking from the pond}}{\text{True}}\)
  \end{itemize}

  However, odd, at least to me.
  Conclusion is about how bitter the coffee is, state of affairs is distinguished from agent's perspective on the state of affairs.
\end{note}

\begin{note}
  Compound:
  \begin{itemize}
  \item I like how bitter the coffee is.\newline
    \mbox{ }\hfill%
    \(\pv{\text{The coffee is bitter}}{\text{True}}\)\newline
    \mbox{ }\hfill%
    \(\pv{\text{The coffee is bitter}}{\text{Desirable}}\)
  \end{itemize}

  Likewise, from the agent's perspective, it may be the case that the proposition that the dog is sleeping is evaluated as both true and desirable.
  For example, the agent may evaluate the proposition as true because the agent perceives the proposition to be the case.
  And, the agent may evaluate the proposition as desirable because the agent is planning to take the dog on a long walk later in the day.
\end{note}

\begin{note}
  We will make use of the following notation:
  \begin{itemize}
  \item
    Lower case letters of the Greek alphabet, \(\phi, \psi, \dots\) correspond to propositions, i.e.\ states of affairs.
  \item
    The Latin letter \(v\) corresponds to a some value.
    And in cases where more than one value may be relevant we will write \(v', v'' \dots\).
    However, \(v, v'\), and \(v''\), etc., may all correspond to the same value.
  \item
    For any proposition \(\phi\) and value \(v\), we abbreviate the pairing of \(\phi\) and \(v\) as \(\pv{\phi}{v}\).
  \end{itemize}

  In general, any occurrence of \(v\) may be read as `true'.
  When speaking abstractly, nothing depends on particular value.

  \begin{itemize}
  \item
    On occasion we use \(\overline{v}\) to indicate some value other than \(v\) of the same type as \(v\) and \(v_{?}\) to indicate \(\phi\) is not paired with any proposition of the same type as \(v\).

    In summary, the notation on the left is in accordance with the sentence on the right with a default reading below:
    \begin{itemize}
    \item
      \(\pv{\phi}{v}\) \hfill \(\phi\) has value \(v\).%
      \newline
      \mbox{ }\hfill \(\phi\) is true.
    \item
      \(\pv{\phi}{\overline{v}}\) \hfill \(\phi\) has some value other than \(v\) of the same type.%
      \newline
      \mbox{ }\hfill \(\phi\) has some value other than true of the same type.%
      \newline
      \mbox{ }\hfill I.e.\ \(\phi\) is false.
    \item
      \(\pv{\phi}{v_{?}}\) \hfill \(\phi\) is not evaluated to have a value of type \(v\).%
      \newline
      \mbox{ }\hfill \(\phi\) is not evaluated as true or false.
    \end{itemize}
  \end{itemize}
\end{note}

\begin{note}
  We have seen a few examples of what conclusions \emph{are} given \autoref{assu:concluding:pvp}.
  The motivating idea behind \autoref{assu:concluding:pvp} is that an agent's evaluation of a proposition is fundamental.

  In other words, the respective evaluations of a proposition as true and of a proposition as desirable may function analogously to beliefs and desires in folk-psychology.
  Indeed, an agent evaluating a proposition as true may be treated as equivalent to the agent believing the proposition is true, and the agent evaluating a proposition as desirable as equivalent to desiring the (truth of the) proposition.%
  \footnote{
    \label{fn:belief-is-difficult}
    Though I don't think belief is really so straightforward.

    Consider the Jeremy Goodman's example of three-horse race from~\textcite{Hawthorne:2016wv}:
    \begin{quote}
      Assume that horse A is more likely to win than horse B which in turn is more likely to win then horse C (so the probabilities of winning could be known to be 45, 28, 27\%).
      In this case it seems fine to say `I think horse A will win' or `I believe horse A will win'.%
      \mbox{ }\hfill\mbox{(\citeyear[1440]{Hawthorne:2016wv})}
    \end{quote}
    As \citeauthor{Hawthorne:2016wv} observe: ``[I]t is awful to say, in this case, `I think horse A will win but I don't believe it will'.''
    (\citeyear[1440, fn.17]{Hawthorne:2016wv})
  }

  If you are inclined to the folk-psychological picture, then the benefit of speaking in terms of proposition-value pairing from the agent's perspective is an easy way to talk about how things are from the perspective of the agent, rather than recasting the agent's perspective in terms of the propositional attitudes the agent has.

  And, if you are not inclined to the folk-psychological picture then the sketched equivalence should suggest how to understand the way in which proposition-value pairing are fundamental.
  If an agent evaluates some proposition \(\phi\) as true then \(\phi\) is the case, from the agent's perspective, etc.
\end{note}

\begin{note}
  In particular, we distinguish proposition-value pairing from a proposition-value pairing \emph{given various assumptions}.
  When we speak of the proposition `Rover will fall asleep soon' pairing with the value `true', from an agent's perspective, then Rover \emph{will} fall asleep soon, from the agent's perspective.
  By contrast, if the agent has been entertaining the idea that Rover is tired then the agent is entertaining the idea that the proposition `Rover is tired' has the value `true', but we do not consider the proposition and valid paired.
\end{note}

\begin{note}
  \phantlabel{mention:concluding-non-factive}
  Further, we allow for the possibility that some proposition is paired with some value from the perspective of an agent, though in reality things are otherwise.
  Hence, we do not require, for example, that an agent concludes \(\phi\) is true only if \(\phi\) is the case.

  Following an example of \citeauthor{Williams:1979wi}, an agent may conclude it is true that the stuff is gin, while in fact the stuff is petrol (\citeyear[18]{Williams:1979wi}).
  A slightly more involved case is given in the footnote to this sentence.%
  \footnote{
    An agent may conclude \(0.999\dots \ne 1\).
    And, the agent may, when concluding, hold themselves to have a conventional understanding of real numbers.

    The qualification is important, there are various interpretations under which \(0.999\dots \ne 1\), but (it is convention that) the Archimedean property holds for real numbers.

    Still, the agent may have failed to grasp the Archimedean property does not hold for real numbers, and so may reason that, though \(0.999\dots\) approaches \(1\), there must be \emph{some} difference between \(0.999\dots\) and \(1\) --- no matter how small --- and some difference between to things is sufficient to establish that they are not equal.
  }
\end{note}

\begin{note}
  We will not impose an explicit restrictions on which proposition may be paired with which values, nor on the circumstances under which propositions may be paired with values.

  However, we will try to keep things intuitive.
\end{note}

\subsection{Conclusions from pools of premises}
\label{sec:pools-premises}

\begin{note}

  \begin{assumption}
    \label{assu:concluding:pools}
    For an agent \vAgent{}:

    \begin{enumerate}
    \item[\emph{If}:]
      \vAgent{} concludes \(\phi\) has value \(v\).
    \item[\emph{Then}:]
      \vAgent{} concludes \(\phi\) has value \(v\) from some pool of premises \(\Phi\).
    \end{enumerate}
    Where:
    \begin{itemize}
    \item
      A pool of premises is some collection of proposition-value pairings.
    \end{itemize}
    \vspace{-\baselineskip}
  \end{assumption}

  \autoref{assu:concluding:pools} is primarily a matter of convenience.
  In various cases it is intuitive that an agent concludes \(\phi\) has value \(v\) from some premises, and we will have interest in keeping track of which premises an agent concludes from.

  For a handful of brief examples, consider the following:

  \begin{itemize}[noitemsep]
  \item
    \emph{A} testified that \(\phi\) is true, so \(\phi\) is true.
  \item
    \(\phi\) would be satisfactory for every member of the group, so \(\phi\) ought to be the case.
  \item
    The song is produced by \emph{B}, so listening to it is desirable.
  \item
    The device reads `\(\phi\)' and is reliable, so \emph{not}-\(\phi\) is improbable.
  \end{itemize}

  Each sentence may be used to identify a conclusion, marked by `so' with some relevant premises.
  I doubt each sentence captures the full extent of the premises involved, but details regarding the premises will not be of interest.

  Further, it is not clear that concluding always involves specific premises.
  For example, consider the parallel between the~\citeauthor{Ramsey:1929tf} test for conditionals and a Fitch-style rule for conditional introduction in propositional logic.%
  \footnote{
    \textcite{Read:1995wf} describe the test as follows:

    \begin{quote}
      One should believe a conditional. `if \emph{A} then \emph{B}' if one would come to believe \emph{B} if one were to add A to one's stock of beliefs.%
      \mbox{ }\hfill\mbox{(\citeyear[47]{Read:1995wf})}
    \end{quote}

    A Fitch-style rule for conditional introduction in propositional logic is as follows --- Cf. (\cite[206]{Barwise:1999tu}), (\cite{Pelletier:2021vp}):
    \begin{center}
      \begin{fitch}
        \ftag{\scriptsize i}{\fa \fh P} & \\
        \ftag{\scriptsize }{\fa \fa \vdots} & \\
        \ftag{\scriptsize j}{\fa \fa Q} & \\
        \ftag{\scriptsize j+1}{\fa P \rightarrow Q} & \(\rightarrow\)\textbf{Intro:} \emph{i}--\emph{j} \\
      \end{fitch}
    \end{center}

    The rule states that at any point in a proof, one may assume \(P\), then, after deriving \(Q\) from the assumption of \(P\), one may discharge assumption of \(P\) and introduce the conditional \(P \rightarrow Q\).
    Note, the assumption of \(P\) on line \emph{i} corresponds to adding \(P\) to collection of propositions proved or assumed up to line \emph{i}, and hence to one's stock of beliefs.

    Now, both the \citeauthor{Ramsey:1929tf} test and the Fitch-style rule for conditional introduction describe a clear \emph{processes} for which a conditional is a conclusion, but there may not be any \emph{premises} associated with the conclusion.
    To illustrate, consider the following derivation which does not involve any premises:

    \begin{center}
      \begin{fitch}
        \ftag{\scriptsize 1}{\fa \fh P \land Q} & \\
        \ftag{\scriptsize 2}{\fa \fa Q} & \(\land\)\textbf{Elim:} \emph{1} \\
        \ftag{\scriptsize 3}{\fa (P \land Q) \rightarrow Q} & \(\rightarrow\)\textbf{Intro:} \emph{1}--\emph{2} \\
      \end{fitch}
    \end{center}
  }
  However, \autoref{assu:concluding:pools} splits the difference by allowing the relevant pool of premises to be the empty set.
\end{note}

\subsection{Concludes and concluding}
\label{cha:introduction:sec:CCC:c-and-c}

\begin{note}
  \autoref{cha:introduction:sec:CCC:pvp} made the assumption that conclusions are pairings of a proposition \(\phi\) with a value \(v\), from the perspective of an agent.

  In this section we turn to two events concerning conclusions:
  \begin{itemize}
  \item
    The event in which an agent concludes.
  \item
    The event in which an agent is concluding.
  \end{itemize}
  This section will focus on the event in which an agent concludes.
\end{note}

\begin{note}
  We start with an assumption regarding an event in which an agent concludes:

  \begin{assumption}[Concludes is inclusive]
    \label{assu:concluding:span}
    For an agent \vAgent{}, proposition-value pair \(\pv{\phi}{v}\), and pool of premises \(\Phi\):

    \begin{itemize}
    \item
      The event in which an agent concludes \(\pv{\phi}{v}\) from \(\Phi\) spans the agent moving from the premises of \(\Phi\) to the conclusion that \(\phi\) has value \(v\).
    \end{itemize}
    \vspace{-\baselineskip}
  \end{assumption}

  The key feature of \autoref{assu:concluding:span}  is the \emph{scope} of concluding.
  The event of an agent is not limited to the agent pairing proposition \(\phi\) with value \(v\), but includes how the agent comes to pair \(\phi\) with \(v\).

  \autoref{assu:concluding:span} implicitly characterises an event in which an agent concludes via what a conclusion (and pool of premises) is.

  A slightly more natural paraphrase of \autoref{assu:concluding:span} states that the event in which an agent concludes \(\pv{\phi}{v}\) from \(\Phi\) spans the agents \emph{reasoning} to \(\pv{\phi}{v}\) from \(\Phi\).

  Indeed, we will continue to speak in terms of reasoning to ease things a little.
\end{note}

\begin{note}
  Rather than being an assumption proper, I'm inclined to think that \autoref{assu:concluding:span} identifies a particular sense of the term `concludes'.
  The sense of concludes which \emph{includes} the relevant reasoning leading up to some conclusion and does not focus only on the more-or-less instantaneous event in which the agent forms the relevant conclusion.

  In this respect, \autoref{assu:concluding:span} is less of an assumption and more of an indication about the sense in which I will use `concludes'.
\end{note}

\begin{note}
  To illustrate, we contrast to instances of `concludes'.
  The first is from \textcite{Gardner:1986wp} and the second from \textcite{Bratman:1979aa}.

  The following passage from \citeauthor{Gardner:1986wp}'s discussion of Newcomb's problem:

  \begin{quote}
    A large number of those who recommended taking only the second box performed the expected-value calculation and concluded that, provided the probability that the Being was correct was at least .5005, they would take only the second box.%
    \mbox{ }\hfill\mbox{(\citeyear[166]{Gardner:1986wp})}
  \end{quote}

  On my reading, \citeauthor{Gardner:1986wp} distinguishes the performing the expected-value calculation from the conclusion that they would take the box given a certain probability.

  In contrast to the passage from \citeauthor{Gardner:1986wp}, consider the following passage from \citeyear{Bratman:1979aa}:

  \begin{quote}
    Sam thinks both that in certain respects his drinking would be, \emph{prima facie}, best, and that in certain other respects his abstaining would be, \emph{prima facie}, best.
    Weighing these conflicting considerations he concludes that it would be best to abstain, rather than drink.%
    \mbox{ }\hfill\mbox{(\citeyear[156]{Bratman:1979aa})}
  \end{quote}

  It seems to me \citeauthor{Bratman:1979aa} describes includes the weighing of conflicting considerations in the event of concluding --- there is no `and' to distinguish the event of Sam weighting the conflicting considerations and the event in which Sam concludes that it would be best to abstain, rather than drink.
\end{note}

\begin{note}
  In addition, it seems to me that whether or not one defaults to thinking of a more-or-less instantaneous event or an inclusive event comes apart in multi-agent cases.

  For example, consider Sam and Max working on some problem \(p\) together.
  Sam and Max trade ideas back and forth, and settle on an answer \(a\).

  Consider the events described by the following sentence:
  \begin{enumerate}[label=\arabic*., ref=(\arabic*)]
  \item
    \label{Cing:SandM:j}
    Sam and Max (jointly) conclude \(a\) is an answer to problem \(p\).
  \item
    \label{Cing:SandM:s}
    Sam concludes \(a\) is an answer to problem \(p\).
  \item
    \label{Cing:SandM:m}
    Max concludes \(a\) is an answer to problem \(p\).
  \end{enumerate}

  My default is to consider the events captured by \ref{Cing:SandM:s} and \ref{Cing:SandM:m} in line with \citeauthor{Gardner:1986wp}.
  When considering Sam or Max in isolation, the event in which Sam (or Max) concludes more-or-less instantaneous event.
  Though, the relevant events may be expanded to include the trading of ideas by appending `by trading ideas with Max/Sam'.

  By contrast, I read \ref{Cing:SandM:j} in line with \citeauthor{Bratman:1979aa}.
  When considering Sam and Max as a pair, in which a non-distributive reading of `conclude' is forced by the parenthetical `jointly', the event includes the trading of ideas back and forth.
  Though, the relevant event includes sub-event in which Sam and Max separately evaluate that \(a\) is an answer to problem \(p\).
\end{note}

\begin{note}
  Our understanding of `concluding' is that an event in which an agent is concluding some proposition-value pair \(\pv{\phi}{v}\) from some pool of premises \(\Phi\) is just an event such that the event in which the agent concludes \(\pv{\phi}{v}\) from \(\Phi\) is in progress.

  In other worlds, an event of concluding involves some reasoning from some sub-set of the pool of premises \(\Phi\).

  In this respect, an event of concluding need not involve the agent pairing \(\phi\) with \(v\), it need only be the case that if the relevant event went on uninterrupted, then event would develop into an event in which the agent concludes \(\pv{\phi}{v}\) from \(\Phi\), and hence pairs \(\phi\) with \(v\).

  As noted on \autopageref{reasoning-vs-concluding-progressive}, an important distinction between reasoning and concluding is that we don't get this.
  {
    \color{red} Up to here.
  }
\end{note}

\subsubsection{Reasoning}
\label{sec:reasoning-1}

\begin{note}
  We have assumed that concluding is an instance of reasoning, and more specifically that concluding involves an agent reasoning from some pool of premises to some conclusion, where both premises and conclusions are understood in terms of proposition-value pairings.
  In this respect, the assumptions we have made place constraints on certain instances of reasoning.
  However, aside from understanding reasoning in terms of proposition-value pairings, we do not place further constraints on reasoning.

  Note, the absence of placed constraints does not imply that reasoning is unconstrained.
  Rather, we adopt a neutral stance on what may count as instance of reasoning.
\end{note}

\begin{note}[No constraints]
  To illustrate, consider the contrast between \citeauthor{Broome:2013aa}'s (\citeyear{Broome:2013aa}) rule following account of reasoning and \citeauthor{Wedgwood:2006ui}'s (\citeyear{Wedgwood:2006ui}) reason-based account of reasoning.

  \citeauthor{Broome:2013aa}'s rule following account of reasoning is unconstrained in terms of the rules an agent may follow.
  This aspect  of \citeauthor{Broome:2013aa}'s account is highlighted in the following passage:

  \begin{quote}
    [S]hould we exclude this bizarre rule: from the proposition that it is raining and the proposition that if it is raining the snow will melt, to derive the proposition that you hear trumpets.
    Following this rule would lead you to believe you hear trumpets when you believe it is raining and believe that if it is raining the snow will melt.
    If you did this, should we count you as reasoning?

    I think we should.
    If you derive this conclusion by operating on the premises, following the rule, we should count you as reasoning.
    \dots
    I think we should not impose a limit on rules.%
    \mbox{}\hfill\mbox{(\citeyear[233]{Broome:2013aa})}
  \end{quote}

  By constraint, \citeauthor{Wedgwood:2006ui}'s reason-based account of reasoning:

  \begin{quote}
    Reasoning, I shall assume, is the process of \emph{revising ones beliefs or intentions, for a reason}.%
    \mbox{ }\hfill\mbox{(\citeyear[600]{Wedgwood:2006ui})}
  \end{quote}

  \citeauthor{Wedgwood:2006ui} understands reasons in terms of intelligibility.%
  \footnote{
    Strictly, \citeauthor{Wedgwood:2006ui} understanding reasoning in terms of dispositions that respond to what `rationalizes' (\citeyear[672]{Wedgwood:2006ui}) and \citeauthor{Wedgwood:2006ui} considers `rationalizes' and  `makes intelligible' as equivalent.
    Therefore, the following conditional, states a sufficient rather than necessary condition which leads to \citeauthor{Wedgwood:2006ui}'s assumption, rather than being an expansion of what \citeauthor{Wedgwood:2006ui}'s assumption amounts to.
    I find this particularly confusion.
  }

  \begin{quote}
    If a set of antecedent mental states makes it rational for one to form a new belief or intention, then those antecedent mental states are surely of a suitable type and content so that it is \emph{intelligible} that they could represent one's reason for forming that belief or intention.\newline
    \mbox{ }\hfill\mbox{(\citeyear[662]{Wedgwood:2006ui})}
  \end{quote}

  Now, perhaps the presence of the rule in \citeauthor{Broome:2013aa}'s example makes it intelligible that the agent concludes that they hear trumpets from the relevant premises.
  However, there is also a clear sense in which the reasoning in \citeauthor{Broome:2013aa}'s example in \emph{unintelligible}.

  \citeauthor{Wedgwood:2006ui} provides the following contrasts to clarify their understanding of intelligibility:

  \begin{quote}
    [T]he belief that the \emph{Oxford Dictionary of National Biography} says that Hume died in 1776 seems a mental state of a suitable type and content so that it could intelligibly represent one's reason for believing that Hume died in 1776.
    On the other hand, it is not (except in the presence of some rather extraordinary background beliefs) a mental state of a suitable type and content so that it could intelligibly represent one's reason for believing that every even number is the sum of two primes.%
    \mbox{ }\hfill\mbox{(\citeyear[662]{Wedgwood:2006ui})}
  \end{quote}

  Does the rules that the agent follows in \citeauthor{Broome:2013aa}'s example count as a sufficiently extraordinary background belief?
  Indeed, \citeauthor{Broome:2013aa} do not require that, in general, an agent has beliefs concerning of the rules they follow in reasoning (\citeyear[Cf.][\S13.2]{Broome:2013aa}).
\end{note}

\begin{note}
  By refraining from placing any further constraints on reasoning we remain neutral on whether an unconstrained approach to reasoning is line with \citeauthor{Broome:2013aa} is correct, or whether a constrained approach to reasoning in line with \citeauthor{Wedgwood:2006ui} is correct.

  Still, in general, we aim to consider instances of reasoning (and concluding) which are intuitive.
  Hence, we hope that the instances of reasoning we consider are compatible with whatever way reasoning is understood.
\end{note}

\subsection{Attitudes}

\begin{note}
  Conclusion of reasoning is a proposition-value pairing.
  In general, this does not provide information about the attitude that the agent holds toward the proposition.

  For example, conclude that \(\phi\) has value `true'.
  Conclusion may amount to knowledge, belief, or some other veridical attitude.

  We will have little interest in propositional attitudes.
  Our interest is in concluding, and it is rarely the case that one concludes that they have some attitude toward a proposition.
  Conclude that the cat is on the mat, do not conclude that I believe the cat is on the mat.

  Set aside interest in attitudes when concluding.
  Interest will be in proposition-value-premises pairings from agent's perspective, as will be discussed further in \autoref{cha:introduction:sec:agents-perspective}.
\end{note}

\begin{note}
  Some difficulty in understanding answers to \qWhy{} and agent concludes.
  Evaluations are not attitudes.
  However, attitude entail evaluations.
\end{note}

\begin{note}
  \citeauthor{Davidson:1963aa}, reasons in terms of attitudes.

  \begin{quote}
    \emph{R} is a primary reason why an agent performed the action \emph{A} under the description \emph{d} only if \emph{R} consists of a pro attitude of the agent toward actions with a certain property, and a belief of the agent that \emph{A}, under the description \emph{d}, has that property.%
    \mbox{ }\hfill\mbox{(\citeyear[687]{Davidson:1963aa})}
  \end{quote}

  However, foundation of (pro-)attitude is proposition-value pairing.
  So, pro-attitude, evaluation with something like desire, and belief with evaluation of true.

  In this respect, \citeauthor{Davidson:1963aa} is agent neutral.
  Primary reason is not seen from agent's point of view.
  Instead, from our point of view.
  However, entailed (or corresponding) account from agent's point of view.%
  \footnote{
    Clearer given principle \citeauthor{Smith:1987vk} derives from the Humean Theory of Motivation, as captured by~\ref{Smithh:HtM:2}:
    \begin{quote}
      \begin{enumerate}[label=\textsc{P1}., ref=(\textsc{P1})]
      \item
        \label{Smithh:HtM:2}
        Agent A at t has a motivating reason to \(\phi\) only if there is some \(\psi\) such that, at t, A desires to \(\psi\) and believes that were he to \(\phi\) he would \(\psi\).%
        \mbox{ }\hfill\mbox{(\citeyear[36]{Smith:1987vk})}
      \end{enumerate}
    \end{quote}

    Understand desire of A to \(\psi\) in terms of the evaluation of \(\psi\) as desirable, and the belief as evaluating if do \(\phi\) then \(\psi\).

    \citeauthor{Smith:1987vk} suggests comparison to \textcite{Davidson:1963aa}.
    Note, in particular \citeauthor{Smith:1987vk}'s use of `motivating reason' is equivalent to \citeauthor{Davidson:1963aa}'s use of `primary reason' rather than `reason'.
  }

  So, \citeauthor{Davidson:1963aa}'s constraint in terms of cause is a little more delicate.
\end{note}

\begin{note}
  Hope this is straightforward.%
  \footnote{
    Discussion by \citeauthor{Collins:1997wn} and \citeauthor{Dancy:2000aa}.
  }
  Contrast is possible, from agent's point of view it is that agent has relevant propositional attitudes.

  Consider the follow passage from \citeauthor{Hume:2011aa}'s \hyperlink{cite.Hume:2011aa}{Treatise}:

  \begin{quote}
    ’Tis also obvious, that this emotion rests not here, but making us cast our view on every side, comprehends whatever objects are connected with its original one by the relation of cause and effect.
    Here then reasoning takes place to discover this relation; and according as our reasoning varies, our actions receive a subsequent variation.
    But ’tis evident in this case, that the impulse arises not from reason, but is only directed by it.
    ’Tis from the prospect of pain or pleasure that the aversion or propensity arises towards any object: And these emotions extend themselves to the causes and effects of that object, as they are pointed out to us by reason and experience.%
    \mbox{ }\hfill\mbox{(\hyperlink{cite.Hume:2011aa}{T.2.3.3})}
  \end{quote}

  The passage captures means-end \emph{reasoning}.
  Evaluate an object according to the prospect of pain or pleasure.
\end{note}

\subsection{Notation}

\begin{note}
  In previous sections introduced key aspects of how we understand reasoning.
  In this section we introduce some core notation that we will use throughout the remainder of the document with respect to concluding.
\end{note}

\begin{note}
  \begin{itemize}
  \item
    We use upper case letters of the Greek alphabet, \(\Phi, \Psi, \dots\), to refer to collections of proposition-value pairings.

    For example, \(\Phi = \{ \pv{\phi}{v}, \pv{\psi}{v'}, \dots \}\).

    \(\Phi\) will often be a pool of premises relative to some conclusion.
  \item
    \(\pvp{\phi}{v}{\Phi}\) abbreviates the association of some proposition-value pair \(\pv{\phi}{v}\) with a collection of proposition-value pairings \(\Phi\).

    Usually \(\pvp{\phi}{v}{\Phi}\) will abbreviate \(\pv{\phi}{v}\) being a conclusion from the pool of premises \(\Phi\).
  \end{itemize}
\end{note}

\section{\dots, from the agent's perspective}
\label{cha:introduction:sec:agents-perspective}

\begin{note}
  Qualification.
  Adopt the agent's perspective.

  \begin{itemize}
  \item
    Role agent's perspective has.
  \end{itemize}
\end{note}


\subsection{Role}
\label{sec:role}

\begin{note}
  Explicitly develop variant to \qWhy{}, from the agent's perspective.

  Noted this distinction when considering \citeauthor{Davidson:1963aa}.

  Initial paraphrase.

  why from the agent's perspective.

  Motivate why something is important by consider what else holds from the agent's perspective.
\end{note}

\begin{note}
  Gets close to dispositional view of reasons.
  Reasons may be different, but issues plausibly extend to \support{}.%
  \footnote{
    Or, conversely, \support{} is plausibly an instance of a reason.
  }

  \begin{quote}
    reasons are those a person would offer if asked to justify his belief.
  \end{quote}

  For, no clear link between the reasons agent would offer and reasons for belief.
  \textcite{Harman:1973ww} provides a handful of examples.
  For example, reasons offered as response to what will convince audience, rather than reasons the agent thinks are any good.
  (\citeyear[26--27]{Harman:1973ww})

  I agree with \citeauthor{Harman:1973ww}.
  There is 

  Focus on saying, but the more general point holds.

  Issue here is sound.

  However, distinguish from a different point that \citeauthor{Harman:1973ww} raises:

  \begin{quote}
    The same point follows from the fact that in most cases we cannot say in any detail why we believe as we do.
    At best we can give a vague indication of reasons we find convincing.
    It is only in rare cases that we can tell a person’s detailed reasons from what he can say about them.
    Indeed it is doubtful that we can ever fully specify our reasons.%
    \mbox{ }\hfill\mbox{(\citeyear[28]{Harman:1973ww})}
  \end{quote}

  Distinct from \citeauthor{Harman:1973ww}'s concern.
  Perspective when concluding, and this is fundamental sense in which proposition has value.
  Care needed is to ensure that specification of what is the case from agent's perspective is natural.
\end{note}

\begin{note}
  However, more significant worry.
  The agent's perspective does not necessarily reflect what happens.

  Parallel with knowledge.
  Factivity, we have from the agent's perspective, and from an independent perspective.

  To say it is true that X knows \(\phi\) is true, from agent's perspective allows for the possibility that X does not know \(\phi\) is true.

  So answers to \qWhy{}, as from agent's perspective don't necessarily say anything with respect to \emph{why} and \emph{how} from an independent perspective.

  As everything will be qualified, nothing stronger.
\end{note}


\begin{note}
  Assume from agent's perspective.
  And, for some proposition-value-premises pairing \(\pvp{\psi}{v'}{\Psi}\):

  So, grant this is how things are from the agent's perspective.
  With caution, the qualifier `from the agent's perspective' distributes to the antecedent and consequent of the conditional.

  \begin{enumerate}
  \item[\emph{If}:]
    \begin{enumerate}[label=\alph*., ref=(\alph*)]
    \item
      \support{2} between \(\pv{\psi}{v'}\) and \(\Psi\) failed to hold, from the agent's perspective.
    \end{enumerate}
  \item[\emph{then}:]
    \begin{enumerate}[label=\alph*., ref=(\alph*), resume]
    \item
      The agent would not have concluded \(\pv{\phi}{v}\) from \(\Phi\), from \vAgent{}' perspective.
    \end{enumerate}
  \end{enumerate}

  Now, desired result follows so long the agent would not have concluded, from the agent's perspective, implies that the agent would not have concluded.

  So, at the time of concluding.
  The agent's perspective makes a difference.

  Note, here, are interested in an instance.
  For the relevant implication to hold, it does not need to be the case that implication holds for any instance, etc.

  This is plausible.
  For, concluding is something done by the agent.
  How things are from the agent's perspective matter.
  Goes back to \citeauthor{Davidson:1963aa}.

  In other words, in order for this implication to fail on \citeauthor{Davidson:1963aa}'s theory, some deviant causal chain.

  Significant problem in providing a theory such that deviance is ruled out in general.
  However, relying here only on an instance.
\end{note}

\begin{note}
  Plausible holds for basic cases.
  If testimony was no good, would not conclude.
  From the agent's perspective.
  Hence, testimony answers, in part, why.
  And, through link.
\end{note}

\begin{note}
  Might be suspect, given voluntarism about concluding.
  But, this, I think, is a mistake.
  There's no suggestion that agent may choose whether or not to conclude in these types of cases.
\end{note}

\section{Support and \qWhy{}}
\label{cha:clarification:sec:support-qWhy}
\label{cha:clar:expand:qWhy}

\begin{note}[Introduction]
  We now turn to \qWhy{}:
  \vspace{-\baselineskip}
  \begin{quote}
    \questionWhyBasic*
  \end{quote}
  Example, pairing of \(23 \times 15 = 345\) and the testimony of the calculator that \(23 \times 15 = 345\) answers, in part, why the agent concluded \(23 \times 15 = 345\) in \autoref{illu:gist:calc}.
  Slightly more natural to say `the testimony of the calculator that \(23 \times 15 = 345\)', but \emph{paring}.

  Observed that, intuitively, pairing of \(23 \times 15 = 345\) and whatever pool of premises would be associated with the agent applying their understanding of arithmetic does not answer, in part, why the agent concluded \(23 \times 15 = 345\).
\end{note}

\begin{note}
  Motivation by intuition, but also via positive resolution to \issueInclusion{}.
\end{note}

\subsection{Variant of \qWhy{}}
\label{cha:clar:expand:qWhy:variant}

\begin{note}[The variant]
  Given role of support, and ideas, variant of \qWhy{}:

  \begin{question}[\qWhyV{}]
    \label{q:why:v}
    Which proposition-value-premises pairings are such that:

    \begin{enumerate}[label=]
    \item
      \begin{enumerate}[label=\alph*., ref=(\alph*)]
      \item
        \support{2} between the proposition-value pair and the pool of premises holds, from \vAgent{}' perspective.
      \end{enumerate}
    \item[And:]
    \item
      \begin{enumerate}
      \item[\emph{If}:]
        \begin{enumerate}[label=\alph*., ref=(\alph*), resume]
          \setcounter{enumiii}{1}
        \item
          \support{2} between the proposition-value pair and the pool of premises failed to hold, from \vAgent{}' perspective.
        \end{enumerate}
      \item[\emph{then}:]
        \begin{enumerate}[label=\alph*., ref=(\alph*), resume]
        \item \vAgent{} would not have concluded \(\pv{\phi}{v}\) from \(\Phi\).
        \end{enumerate}
      \end{enumerate}
    \end{enumerate}
\end{question}

  \support{}, rather than simple pairing.

  Note, here, elimination of `why'.

  Dependence.
\end{note}

\begin{note}
  \qWhyV{} dependence.

  As a result, no guarantee that support between \(\pv{\phi}{v}\) and \(\Phi\) answers \qWhyV{}.
  However, given \autoref{idea:support}, if the agent has concluded, then \support{} between \(\pv{\phi}{v}\) and \(\Phi\) is available.

  This should also not be too much of a worry.
  We also offered no guarantee that \(\pvp{\phi}{v}{\Phi}\) is an answer to \qWhy{}.

  Further, plausible that this may be argued for.
  However, our interest is with other proposition-value-premises pairings.
\end{note}

\begin{note}
  Link between \qWhy{} and \qWhyV{} follows by:

  \begin{restatable}{link}{linkSupportWhy}
    \label{link:why:support:pvpp}
    For an agent \vAgent{}, and proposition-value-premises pairings \(\pvp{\psi}{v'}{\Psi}\), \(\pvp{\phi}{v}{\Phi}\):

    \begin{itemize}
    \item[\emph{If}]
      \begin{enumerate}[label=\alph*., ref=(\alph*)]
      \item
        \support{2} between \(\pv{\psi}{v'}\) and \(\Psi\) is, in part, an answer to \qWhyV{} \vAgent{} concluded \(\pv{\phi}{v}\) from \(\Phi\).
      \end{enumerate}
    \item[\emph{then}]
      \begin{enumerate}[label=\alph*., ref=(\alph*), resume]
      \item
        \(\pvp{\psi}{v'}{\Psi}\) is, in part, an answer to \qWhy{} \vAgent{} concluded \(\pv{\phi}{v}\) from \(\Phi\).
      \end{enumerate}
    \end{itemize}
    \vspace{-\baselineskip}
  \end{restatable}
  {
    \color{red}
    View this as a constraint on our understanding of \qWhy{}.
    Or, as a constraint on support.
    (The only difficulty here is if, somehow, understand \qWhy{} in such a way that rules this out\dots)
  }
  Difficulty for any negative view is getting support without how.

  If with intuition for positive resolution, then this should also look fine.
\end{note}

\subsection{\support{2}}
\label{sec:support2}

{
  \color{red} Include some examples.
}

\begin{note}
  Given variant, have an understanding of role.
  We now introduce to explicit ideas which govern our understanding of \support{}.
\end{note}


\begin{note}
  \begin{idea}[\supportI{}]
    \label{idea:support}
    For an agent \vAgent{}, a proposition-value pair \(\pv{\phi}{v}\), pool of premises \(\Phi\), and event \(e\):

    \begin{itemize}
    \item[\emph{If}]
      \begin{enumerate}[label=\alph*., ref=(\alph*)]
      \item
        \(e\) is an event in which \vAgent{} concludes \(\pv{\phi}{v}\) from \(\Phi\).
      \end{enumerate}
      \item[\emph{then}]
        \begin{enumerate}[label=\alph*., ref=(\alph*), resume]
        \item
          When \vAgent{} pairs \(\phi\) with \(v\) as a sub-event of \(e\), (a relation of) \emph{\support{}} holds between \(\pv{\phi}{v}\) and \(\Phi\), as \vAgent{} concludes \(\pv{\phi}{v}\) from \(\Phi\), from \vAgent{}' perspective.
        \end{enumerate}
      \end{itemize}
      \vspace{-\baselineskip}
  \end{idea}

  \supportI{} captures way in which pool of premises are related to conclusion by event in which agent concludes.

  In this respect, static account of how the agent has come to pair \(\phi\) with \(v\).
  However, allow for the possibility that \support{} regardless of whether agent has concluded \(\pv{\phi}{v}\) from \(\Phi\).

  Intuitively, \support{} between \(\pv{\phi}{v}\) and \(\Phi\) is always an answer to \qWhyV{}.
  For, if \support{} failed to hold, then the event would not be an event in which the agent concludes \(\pv{\phi}{v}\) from \(\Phi\).

  Important role of \supportI{} is to capture, in a static way, relationship between \(\pv{\phi}{v}\) and \(\Phi\) that has been established by reasoning from \(\pv{\phi}{v}\) to \(\Phi\) when agent pairs \(\phi\) with \(v\) to form a conclusion.

  Little interest in what \support{} amounts to.
\end{note}

\begin{note}
  Note, however, that \supportI{} is if-then conditional.
  The converse to \supportI{} does not necessarily hold.
\end{note}

\begin{note}
  \begin{idea}[\supportII{}]
    \label{idea:support:possible}
    For an agent \vAgent{}, a proposition-value pair \(\pv{\phi}{v}\), and pool of premises \(\Phi\):

    \begin{itemize}
    \item
      It is possible for there to be \support{} between \(\pv{\phi}{v}\) and \(\Phi\) without \vAgent{} having concluded \(\pv{\phi}{v}\) from \(\Phi\), from \vAgent{}' perspective.
    \end{itemize}
    \vspace{-\baselineskip}
  \end{idea}

  Note, \supportII{} does not require there exist instances of \support{}.
  \autoref{cha:fcs} will focus on argument that there are instances of \support{} without event of in which agent concludes.
  These we will term \fc{1}.%
  \footnote{
    \citeauthor{Firth:1978vi}'s (\citeyear{Firth:1978vi}) distinction between doxastic and propositional justification (or warrant).
    See also \citeauthor{Silva:2020aa} (\citeyear{Silva:2020aa}) --- esp.\ fn.\ 1.

    {\color{red}
      Compare \citeauthor{Firth:1978vi}'s example with Holmes and Watson (\citeyear[218]{Firth:1978vi}).
      Watson is presented with all the evidence Holmes used to that the coachman committed the murder, and that this provides Watson with sufficient epistemic reasons regardless of whether or not Watson forms any attitude, but it is not clear that Watson has the understanding to piece together the evidence laid before them.
    }
  }

  \supportII{} establishes the \emph{possibility} of \support{} answering why, without the proposition-value-premises pairing being involved in how.

  Still, looking ahead a little, \supportII{} should not be of any immediate concern.
  For, suppose consider an instance of \support{} without event in which agent concludes.
  There mere existence of such an relation does not show that the relation is relevant to why.
\end{note}

\subsubsection{Contrasts}

\begin{note}
  \autoref{idea:support} is distinct from~\citeauthor{Boghossian:2014aa}'s Taking Condition:%
  \footnote{
    There are various objections to the taking condition.

    See, for example,~\textcite{Hlobil:2014tq}, \textcite{McHugh:2016vp}, and~\textcite{Wright:2014tt}.

    \citeauthor{Hlobil:2014tq} argues against the Taking Condition as it distracts from what accounts of reasoning out to explain, rather than arguing against the Taking Condition directly.

    \citeauthor{McHugh:2016vp} summarise various objects to the taking condition, and present district arguments against against (distinct) ideas in favour of the taking condition.
    In particular,~\autoref{idea:support} is closer to what \citeauthor{McHugh:2016vp} term the `Consequence Condition' (\citeyear[cf.][316]{McHugh:2016vp}), which \citeauthor{McHugh:2016vp} also (indirectly) argue against.
    However, \citeauthor{McHugh:2016vp} does not consider an alternative account of what distinguishes concluding from any other action, and as~\autoref{idea:support} is designed to capture this distinction, it is unclear to me whether \citeauthor{McHugh:2016vp}'s arguments apply to~\autoref{idea:support} (if, indeed, they are sound).

    \citeauthor{Wright:2014tt} denies that reasoning must involve a state which connects premises to conclusions. (\citeyear[Cf.][33-34]{Wright:2014tt})
    Note,~\autoref{idea:support} is compatible with \citeauthor{Wright:2014tt}'s objection.
  }

  \begin{quote}
    (Taking Condition):
    Inferring necessarily involves the thinker \emph{taking} his premises to support his conclusion and drawing his conclusion because of that fact.%
    \mbox{}\hfill\mbox{(\citeyear[5]{Boghossian:2014aa})}
  \end{quote}

  \begin{quote}
    The intuition behind the Taking Condition is that no causal process counts as inference, unless it consists in an attempt to arrive at a belief by figuring out what, in some suitably broad sense, is supported by other things one believes.%
    \mbox{}\hfill\mbox{(\citeyear[5]{Boghossian:2014aa})}
  \end{quote}

  In short, \citeauthor{Boghossian:2014aa} assumes that inferring is a causal process, and the role of the Taking Condition is to distinguish the specific process of inferring from some other causal process.
  Of course, \citeauthor{Boghossian:2014aa} states the Taking Condition in terms of inferring, but extends to concluding.

  In contrast, we do not require that \support{} has any particular role an event in which an agent concludes \(\pv{\phi}{v}\) from \(\Phi\).
  As mentioned, static (and partial) perspective on dynamic process.
\end{note}

\begin{note}
  Indeed, our understanding of \support{} is close to \citeauthor{Wright:2014tt}'s `Simple Proposal':
  \begin{quote}
    \dots consider instead the proposal, not that the status of the transition as inferential depends on the thinker's judgments about his reasons, but that it depends on \emph{what his reasons are}.
    We want his acceptance of the premises to supply his \emph{actual} reasons for accepting the conclusion.

    \mbox{}\hfill\(\vdots\)\hfill\mbox{}

    Call this the Simple Proposal.
    It says that a thinker infers q from p\(_{1}\) \(\cdots\) p\(_{\text{n}}\) when he accepts each of p\(_{1}\) \(\cdots\) p\(_{\text{n}}\), moves to accept q, and does so for the reason that he accepts p\(_{1}\) \(\cdots\) p\(_{\text{n}}\).%
    \mbox{}\hfill\mbox{(\citeyear[33]{Wright:2014tt})}
  \end{quote}

  On my reading, \citeauthor{Wright:2014tt}'s simple proposal is that, from the agent's perspective, the relation between pool of premises and conclusion need not be part of why agent moves to accept conclusion.
  Considering how the agent concludes \(\pv{\phi}{v}\) from \(\Phi\) is sufficient.

  Still, there is an important between \autoref{idea:support} and \citeauthor{Wright:2014tt}'s Simple Proposal.
  For, \autoref{idea:support} is an entailment, while \citeauthor{Wright:2014tt}'s Simple Proposal is an identity statement.
  Inferring, on the Simple Proposal, is an agent accepting some conclusion for the reason that they accept some (pool of) premises.
  \autoref{idea:support} does not entail that concluding is nothing more than pairing \(\phi\) with value \(v\) as a result of \(\Phi\).

  Goal is to argue that there is more to concluding, no matter how premises are understood.
\end{note}

\begin{note}
  The same observation extends to \citeauthor{Broome:2002aa}'s (\citeyear{Broome:2013aa}) rule following account of (active) reasoning.

  \begin{quote}
    Active reasoning is a particular sort of process by which conscious premise-attitudes cause you to acquire a conclusion-attitude.
    The process is that you operate on the contents of your premise-attitudes following a rule, to construct the conclusion, which is the content of a new attitude of yours that you acquire in the process.\newline
    \mbox{ }\hfill\mbox{(\citeyear[234]{Broome:2002aa})}
  \end{quote}

  Understand relation of \support{} entailed by an event in which an agent concludes \(\pv{\phi}{v}\) from \(\Phi\) by \autoref{idea:support} in terms of having followed a rule.
  However, if this entailment holds then the converse entailment, that the agent concluded by following the rules that gave rise to relation of \support{} does not hold.
\end{note}

\subsection{Two observations}
\label{sec:two-observations}

\subsubsection{Dispositions and Deviance}
\label{cha:clarification:sec:dis-and-dev}

\begin{note}
  Clarify~\autoref{q:why:v}, but also highlight way in which~\autoref{q:why:v} may be indirectly answered.
\end{note}

\begin{note}
  The statement of \autoref{q:why:v} is designed to establish a direct connexion between some relation of \support{} and concluding.

  Consider the following suggestions:

  \begin{question}
    \label{q:why:v:perspective}
    Which proposition-value-premises pairings are such that:
    \begin{itemize}
    \item
      From \vAgent{}' perspective:
      \begin{enumerate}[label=]
      \item
        \begin{enumerate}[label=\alph*., ref=(\alph*)]
        \item
          \support{2} between the proposition-value pair and the pool of premises holds.
        \end{enumerate}
      \item[And:]
      \item
        \begin{enumerate}
        \item[\emph{If}:]
          \begin{enumerate}[label=\alph*., ref=(\alph*), resume]
            \setcounter{enumiii}{1}
          \item
            \support{2} between the proposition-value pair and the pool of premises failed to hold.
          \end{enumerate}
        \item[\emph{then}:]
          \begin{enumerate}[label=\alph*., ref=(\alph*), resume]
          \item \vAgent{} would not have concluded \(\pv{\phi}{v}\) from \(\Phi\).
          \end{enumerate}
        \end{enumerate}
      \end{enumerate}
    \end{itemize}
  \end{question}

  Here, conditional is from the agent's perspective.
  This doesn't get a link to concluding.
  In this case, there is no guarantee that \support{} mattered for concluding.
  {
    \color{red}
    Discussion from \autoref{sec:role}
  }

  \begin{proposition}
    Instances where \(\pvp{\psi}{v'}{\Psi}\) answers \autoref{q:why:v} in virtue of answering \autoref{q:why:v:perspective}.
  \end{proposition}

  Key is link between agent's perspective and concluding.
\end{note}

\subsubsection{Identifying something as a premise}
\label{cha:clarification:sec:embedding}

\begin{note}
  Above, we observed how \autoref{idea:support} and~\supportII{} may combine in such a way that \support{} holds between \(\pv{\psi}{v'}\) and \(\Psi\) (from an agent's perspective), though the agent has not concluded \(\pv{\psi}{v'}\) from \(\Psi\).

  Here, then, way in which \support{} may answer \qWhyV{}, and hence pairing for \qWhy{} via~\autoref{link:why:support:pvpp}.

  Indeed, this is exactly what we will argue for.

  However, relation between \support{} holding from agent's perspective and \support{} answering \qWhyV{} is delicate.

  For, there is a significant difference between the following two answers to \qWhyV{}:
  \begin{enumerate}
  \item
    \support{2} between \(\pv{\psi}{v'}\) and \(\Psi\).
  \item
    \support{2} between (\emph{\support{} between \(\pv{\psi}{v'}\) and \(\Psi\)}) and \(\pv{\phi}{v}\).
  \end{enumerate}
  Or, more generally in the case of 2, \support{} between \(\Phi\) and \(\pv{\phi}{v}\) such that it being true that \support{} holds between \(\pv{\psi}{v'}\) and \(\Psi\) is an element of the pool of premises \(\Phi\).

  Speaking of \support{} may be strained, but it helps make this distinction.
  And, this distinction is important.%
  \footnote{
    This distinction will return in \autoref{cha:zSpAwhy}.
  }

  Difference between \support{}, in part, answering \qWhyV{} and \support{} embedded as a premise.
\end{note}

\paragraph{Conditional}

\begin{note}
  Simple illustration, conditional.

  On the assumption of \(\pv{\chi}{v''}\), get \(\pv{\psi}{v'}\).
  Turn this into a conditional.

  Now, \(\pv{\chi}{v''}\).

  So, given the reasoning, \support{} between \(\pv{\chi}{v''}\) and \(\pv{\chi}{v''}\).

  However, conclude from conditional via detachment.

  So, what answers \qWhyV{} is \support{} between \(\pv{\chi}{v''}\) and conditional and \(\pv{\psi}{v'}\).

  Though, it may also be the case that \support{} between \(\pv{\chi}{v''}\) and \(\pv{\chi}{v''}\) remains important.
  For, still have the option to conclude.

  However, this gets embedded.
  This \support{} does not answer \qWhyV{}.
  For, linked to conditional, and premise.
\end{note}

\begin{note}
  Phrase this in terms of memory.
  Concluded.
  \support{2} held.
  However, \support{} continues to hold.
  However, still conclude from memory.
\end{note}

\subparagraph{\citeauthor{Owens:2006tw}}

\begin{note}
  For example, \citeauthor{Owens:2006tw} argues for a belief expression model of assertion in which the rationality of a belief formed by an agent via testimony is connected to justification of the testifier:

  \begin{quote}
    Trusting an expression of belief by accepting what a speaker says involves entering a state of mind which gets its rationality from the rationality of the belief expressed.
    This state's rationality depends on the speaker's justification for the belief he expresses, not on his justification for the action of expressing it.
    And to hear a speaker as making a sincere assertion, as expressing a belief, is \emph{ceteris paribus} to feel able to tap into \emph{that} justification (whether or not his assertion was directed at you) by accepting what he says.%
    \mbox{}\hfill\mbox{(\citeyear[123]{Owens:2006tw})}
  \end{quote}

  On the view advanced by \citeauthor{Owens:2006tw}, justification.
  View in terms of \support{}.

  \support{} directly.
  Rationality of agent is rationality of speaker.

  However, `depends'.

  Distinction between rationality of state, and relation between rationality of state and rationality of state.

  Inclined to think \citeauthor{Owens:2006tw} is arguing for the former.%
  \footnote{
    \begin{quote}
      If we are to believe what the speaker indicates he believes, either the speaker must justify this belief to us, or we must supply some justification of our own
      \dots
      Neither act can be part of a rationality preserving mechanism for belief.%
      \mbox{ }\hfill\mbox{(\citeyear[123--124]{Owens:2006tw})}
    \end{quote}
  }
  Though, it is not clear to me that embedded isn't a viable option.

  Regardless, distinction that is important.
\end{note}

\begin{note}
  Same distinction holds for answers to \qWhyV{}.

  It may be the case that \support{} between \(\pv{\psi}{v'}\) and \(\Phi\) is, from the agent's perspective, involved in concluding \(\pv{\phi}{v}\) from \(\Phi\).

  However, no immediate move from this to \support{} being, in part, an answer to \qWhyV{}.
\end{note}


\subsection[Conversely]{Conversely \hfill (Optional)}

\begin{noteP}[Converse to~\autoref{link:why:support:pvpp}]
  \autoref{link:why:pvpp:support} is the converse of~\autoref{link:why:support:pvpp}:

  \begin{restatable}[]{link}{linkWhySupport}
    \label{link:why:pvpp:support}
    For an agent \vAgent{}, and proposition-value-premises pairings \(\pvp{\psi}{v'}{\Psi}\) and \(\pvp{\phi}{v}{\Phi}\):
    \begin{itemize}
    \item[\emph{If}]
      \begin{enumerate}[label=\alph*., ref=(\alph*)]
      \item
        \(\pvp{\psi}{v'}{\Psi}\) is, in part, an answer to why \vAgent{} concluded \(\pv{\phi}{v}\) from \(\Phi\).
      \end{enumerate}
    \item[\emph{then}]
      \begin{enumerate}[label=\alph*., ref=(\alph*), resume]
      \item
        \support{2} between \(\pv{\psi}{v'}\) and \(\Psi\) is, in part, an answer to why \vAgent{} concluded \(\pv{\phi}{v}\) from \(\Phi\).
      \end{enumerate}
    \end{itemize}
    \vspace{-\baselineskip}
  \end{restatable}

  Combined, \autoref{link:why:pvpp:support} and \autoref{link:why:support:pvpp} may be taken to provide a characterisation the way in which a proposition-value-premises pairing \(\pvp{\psi}{v'}{\Psi}\) answers, in part, why an agent concludes \(\pv{\phi}{v}\) from \(\Phi\):
  \support{2} holds between \(\pvp{\psi}{v'}{\Psi}\) and \(\Psi\).

  In other words, support captures the relevant relation between a conclusion proposition-value pair and pool of premises, from perspective of agent, when concluding.
  However, we will not press this idea further.
\end{noteP}

\section{Witnessing and \qHow{}}
\label{sec:overview:reasoning}
\label{cha:clar:expand:qHow}

\begin{note}[Introduction]
  We now turn to \qHow{}:
  \vspace{-\baselineskip}
  \begin{quote}
    \questionHowBasic*
  \end{quote}

  In parallel to \qWhyV{}, our goal is to provide a variation of \qHow{} which does captures conditions for answers \qHow{}.
  The type of link between \qWhyV{} and \qWhy{} was sufficiency.%
  \footnote{
    In short, we have stated a conclusion of \(\pv{\phi}{v}\) from \(\Phi\) {\color{red} depending} on \support{} between \(\pv{\psi}{v'}\) and \(\Psi\) (from the concluder's perspective)is sufficient for \(\pvp{\psi}{v'}{\Psi}\) to answer why the concluder concluded \(\pv{\phi}{v}\) from \(\Phi\).
  }
  The type of link between \qHow{} and it's variant will is that of necessity.

  In this respect, the variant of \qHow{} is weak:

  \begin{question}[\qHowV{}]
    \label{q:how:v}
    Which proposition-value-premises pairing are such that the agent has witnessed \support{} between \(\pv{\psi}{v'}\) from \(\Psi\)?
  \end{question}

  We define witnessing as follows:

  \begin{definition}[Witnessing]
    \label{def:witnessing}
    An agent has \emph{witnessed} \support{} between \(\pv{\phi}{v}\) from \(\Phi\) \emph{only if} there is some event \(e\) such that \(e\) is an event in which \vAgent{} concludes \(\pv{\phi}{v}\) from \(\Phi\).
  \end{definition}

  Intuitively, \support{} answers \qWhyV{} \emph{only if} witnessed \support{}.
  Indeed, this is exactly the variant to \issueInclusion{} we will state in~\autoref{cha:clar:expand:issue}.
\end{note}


\begin{note}
  Example, pairing of \(23 \times 15 = 345\) and the testimony of the calculator that \(23 \times 15 = 345\) answers, in part, how the agent concluded \(23 \times 15 = 345\) in \autoref{illu:gist:calc}.
  Slightly more natural to say `the testimony of the calculator that \(23 \times 15 = 345\)', but \emph{paring}.

  Observed that, intuitively, pairing of \(23 \times 15 = 345\) and whatever pool of premises would be associated with the agent applying their understanding of arithmetic does not answer, in part, how the agent concluded \(23 \times 15 = 345\).
\end{note}

\begin{note}[The link]
  The link is:

  \begin{restatable}{link}{linkHowWitnessing}
    \label{link:how-witnessing}
    For any proposition-value-premises pairing \(\pvp{\psi}{v'}{\Psi}\):
    \begin{itemize}
    \item[\emph{If}]
      \begin{enumerate}[label=\alph*., ref=(\alph*)]
      \item
        \(\pvp{\psi}{v'}{\Psi}\) is, in part, an answer \qHow{}.
      \end{enumerate}
    \item[\emph{then}]
      \begin{enumerate}[label=\alph*., ref=(\alph*), resume]
      \item
        \(\pvp{\psi}{v'}{\Psi}\) is, in part, an answer \qHowV{}.
      \end{enumerate}
    \end{itemize}
    \vspace{-\baselineskip}
  \end{restatable}

  Weak.%
  \footnote{
    Hence, no converse, in contrast to \autoref{link:why:support:pvpp}.
  }
  \begin{enumerate}
  \item
    \qHowV{} only requires the agent has witnessed reasoning to \(\pv{\psi}{v'}\) from \(\Psi\).
  \end{enumerate}

  Intuitively, may be too weak to provide an answer to \qWhy{}.
  However, this is of no concern.
  Interest is in arguing against \issueInclusion{}.

  Pair sufficient for \qWhy{} and necessary for \qHow{}, then depends on \support{} such that agent has not witnessed.
  Make this clear in \autoref{cha:clar:expand:issue}.

  Stronger idea, then in principle, easier counterexamples.
\end{note}

\begin{note}
  Now, the motivation for \autoref{link:how-witnessing} is straightforward.
  For, from by assumption, \(e\) in which concludes is such that that witnessed.

  Concluding, going to \(\pv{\phi}{v}\) from \(\Phi\).

  So, if the agent has concluded \(\pv{\phi}{v}\) from \(\Phi\), then the agent has witnessed reasoning.
\end{note}

\begin{note}
  Of some interest is weaker.
  Provide short motivation for option of past tense and for something weaker than concluding.
\end{note}

\paragraph{No present tense}

\begin{note}[Illustration]
  To illustrate, consider an agent working on some mathematical problem.

  As part of their work on the problem the agent concludes the hypotenuse of some right-angled triangle is \(\sqrt{74}\text{cm}\) by use of the Pythagorean theorem.

  Further, the agent has, at some point in the past proved the Pythagorean theorem from more basic principles.

  Now, generally speaking, it seems to me it may be the case that the agent concludes the hypotenuse of the triangle is \(\sqrt{74}\text{cm}\), in part, from those more basic principles.
  For example, the agent may have just completed their proof of the Pythagorean theorem and the reasoning from the more basic principles to the hypotenuse of the triangle may be considered a single unified instances of reasoning, with an intermediary conclusion.

  Still, suppose the agent proved the Pythagorean theorem some years ago.

  Perhaps the agent's reasoning from more basic principles continues to provide, in part, an answer to how the agent concluded the hypotenuse of the triangle is \(\sqrt{74}\text{cm}\).
  There may be a gap of some years, but it may be the case that the agent uses the Pythagorean theorem (in some sense of the word) \emph{because} they concluded the theorem from more basic principles.
  {
    \color{red}
    Hence, \support{}, and this in part answers \qWhyV{}.
  }

  On the other hand, one may be inclined to hold that the more basic principles the agent proved the Pythagorean theorem from have no role in explaining how the agent concluded the hypotenuse of the triangle is \(\sqrt{74}\text{cm}\) in the present.
  Rather, the Pythagorean theorem is a more-or-less fundamental premise of the agent's present reasoning.

  At best, the agent's memory of reasoning from more basic principles to the Pythagorean theorem may, in part, answer how the agent concluded hypotenuse of the triangle is \(\sqrt{74}\text{cm}\).
  The reasoning from the more basic principles, given that it happened so long ago, is irrelevant.

  \autoref{link:how-witnessing} is designed to allow either opinion on the agent's conclusion.
\end{note}

\paragraph{Reasoning}

\begin{note}
  One potential objection to \qWhyV{} is the idea that there may be instances where an agent reasoned to \(\phi\) having value \(v\) but did not conclude \(\phi\) has value \(v\).
  Yet, at some point later strengthened the result of reasoning to a conclusion.
  May think that in line with \supportII{}, a relation of \support{} held, or came to hold between \(\phi\) and \(v\), and the relation of \support{}, in part, answers \qWhyV{}.

  For example, do a proof but worry about particular step.
  Investigate step, and strengthen.
  Strictly, have not concluded \(\pv{\phi}{v}\) from \(\Phi\).
  Rather, from previous reasoning and premise that reasoning was sound.

  Still, further proof, and would not conclude if prior proof was not sound.

  Plausible to weaken \autoref{def:witnessing} to reasoning.
  Hence, reasoning later seen to be sound would qualify.
\end{note}

\section{\issueInclusion{}}
\label{cha:clar:expand:issue}

\begin{note}
  \autoref{cha:clar:expand:qWhy} refined \qWhy{} in terms of support.
  \autoref{cha:clar:expand:qHow} refined, or perhaps weakened, \qHow{} in terms of witnessing.

  Interest in \qWhy{} and \qHow{} in terms of whether a constraint holds.

  \begin{quote}
    \vspace{-\baselineskip}
    \issueInclusionFirst*
  \end{quote}

  Introduced in~\autoref{cha:introduction}.
  Provided some intuitive motivation via \autoref{illu:gist:calc}.
  And, theoretical motivation via \citeauthor{Davidson:1963aa}' causal theory of action.

  We now bring refinements to questions together to provide refinement to \issueInclusion{}.
\end{note}

\begin{note}
 Following:

  \begin{restatable}[]{proposition}{propVariationsAndInclusion}
    \label{prop:support-and-witnessing}
    Grating a positive resolution to \issueInclusion{}, and given \autoref{link:why:support:pvpp} and \autoref{link:how-witnessing}:
    \begin{enumerate}
    \item[\emph{If}:]
      \begin{enumerate}[label=\alph*., ref=(\alph*)]
      \item \vAgent{} concluded \(\pv{\phi}{v}\) from \(\Phi\).
      \end{enumerate}
    \item[\emph{And}:]
      \begin{enumerate}[label=\alph*., ref=(\alph*), resume]
      \item
        \vAgent{} would not have concluded \(\pv{\phi}{v}\) from \(\Phi\), if \support{} between \(\pv{\psi}{v'}\) and \(\Psi\) failed to hold, from \vAgent{}' perspective.
      \end{enumerate}
    \item[\emph{Then}:]
      \begin{enumerate}[label=\alph*., ref=(\alph*), resume]
      \item
        \vAgent{} has witnessed \support{} between \(\pv{\psi}{v'}\) and \(\Psi\).
      \end{enumerate}
    \end{enumerate}
  \end{restatable}

  \autoref{prop:support-and-witnessing} is straightforward.
  A visual representation is given in~\autoref{fig:relations-between-whys-and-hows}.
\end{note}

\begin{figure}[H]
  \centering
  \begin{tikzpicture}
    \tikzset{ansStyle/.style={
        draw=gray,
        text width=.45\textwidth,
        rounded corners=2pt,
      }
    }
    %
    \node[ansStyle] (whyO) at (0,0) %
    {\qWhyV{} is answered by support between \(\pv{\psi}{v'}\) and \(\Psi\).};
    %
    \node[ansStyle] (whyA) at (2,-1.5) %
    {\qWhy{} is answered by \(\pvp{\psi}{v'}{\Psi}\).};
    %
    \node[ansStyle] (howA) at (4,-3) %
    {\qHow{} is answered by \(\pvp{\psi}{v'}{\Psi}\).};
    %
    \node[ansStyle] (witA) at (6,-4.5) %
    {\qHowV{} is answered by witnessed \support{} between \(\pv{\psi}{v'}\) and \(\Psi\).};
    %
    \path[->] ($(whyO.south)!0.9!(whyO.south west)$) edge [out=270, in=180] (whyA);
    \path[->] ($(whyA.south)!0.9!(whyA.south west)$) edge [out=270, in=180] (howA);
    \path[->] ($(howA.south)!0.9!(howA.south west)$) edge [out=270, in=180] (witA);
    %
    \node[text width=.5\textwidth] (1) at (1,-.8) %
    {Via~\autoref{link:why:support:pvpp}.};
    %
    \node[text width=.75\textwidth] (2) at (4.5,-2.25) %
    {Via a positive resolution to~\issueInclusion{}.};
    %
    \node[text width=.5\textwidth] (3) at (5,-3.625) %
    {Via~\autoref{link:how-witnessing}.};
    %
    \draw[->, gray] ($(whyA.north)!0.9!(whyA.north east)$) to [out=90, in=0] ($(whyO.east)$);
    %
    \node[text width=.5\textwidth, text=gray] (1p) at (8,-.8) %
    {Via~\autoref{link:why:pvpp:support}.};
  \end{tikzpicture}%
  \caption{Relation between positive answers to questions.}
  \label{fig:relations-between-whys-and-hows}
\end{figure}

\begin{note}
  Following~\autoref{prop:support-and-witnessing}, we refine \issueInclusion{} to \issueConstraint{} by asking whether~\autoref{prop:support-and-witnessing} is the case:

  \begin{restatable}[\issueConstraint{}]{issue}{rIssueConstraint}
    \label{issue:has-witnessed}
    For an agent \vAgent{}, proposition-value pairs \(\pv{\phi}{v}\), \(\pv{\psi}{v'}\), and pools of premises \(\Phi\), \(\Psi\):

    Is it the case that:

    \begin{enumerate}
    \item[\emph{If}:]
      \begin{enumerate}[label=\alph*., ref=(\alph*)]
      \item \vAgent{} concluded \(\pv{\phi}{v}\) from \(\Phi\).
      \end{enumerate}
    \item[\emph{And}:]
      \begin{enumerate}[label=\alph*., ref=(\alph*), resume]
      \item
        \vAgent{} would not have concluded \(\pv{\phi}{v}\) from \(\Phi\), if \support{} between \(\pv{\psi}{v'}\) and \(\Psi\) failed to hold, from \vAgent{}' perspective.
      \end{enumerate}
    \item[\emph{Then}:]
      \begin{enumerate}[label=\alph*., ref=(\alph*), resume]
      \item
        \vAgent{} has witnessed \support{} between \(\pv{\psi}{v'}\) and \(\Psi\).
      \end{enumerate}
    \end{enumerate}
  \end{restatable}

  If negative resolution, three possibilities:
  ~\autoref{link:why:support:pvpp} fails,~\autoref{link:how-witnessing} fails, or a negative resolution to \issueInclusion{}.

  ~\autoref{link:why:support:pvpp} and~\autoref{link:how-witnessing} are sufficiently intuitive, negative resolution to \issueInclusion{}.

  Of course, negative resolution may also be uninteresting.
  For, agent's perspective in antecedent.
  However, failures of deviance again.
  If agent's perspective is linked to what the agent would do, then problem.
\end{note}

\begin{note}
  As with \issueInclusion{}, \issueConstraint{} distinguishes classes of theories.
  A positive resolution to \issueInclusion{} will not directly provide general answer to \qWhy{} or \qHow{}.
  Though, a positive answer will rule out certain answers.
\end{note}

\subsection{Summary}
\label{cha:clar:expand:issue:summary}

\begin{note}
  Three key things.

  Support.
  Witnessing.
  Issue.
\end{note}

\begin{note}
  Focus on \issueConstraint{}.
  \vspace{-\baselineskip}
  \begin{quote}
    \rIssueConstraint*
  \end{quote}
  Sufficient clarity on both `why?' and `how?'.
  Link we have argued for.
  And, further, independently of argument, it seems to me that a positive resolution to \issueConstraint{} is equally compelling as positive resolution to \issueInclusion{}.
\end{note}

\begin{note}
  This is the important thing, and in this respect it doesn't matter whether past or present.
  Whether a \support{} holds only if witnessed.
  Whether resolution to \qWhy{} only if the agent has witnessed.
\end{note}

\begin{note}
  Difficulty with all of this is that the accounts seem to be consistent, but do not explicitly motivate this constraint.
\end{note}

\begin{note}
  An additional example, \citeauthor{Hieronymi:2011aa}
  \begin{quote}
    The proposal starts with this simple thought: whenever an agent acts for reasons, the agent, in some sense, takes certain considerations to settle the question of whether so to act, therein intends so to act, and executes that intention in action.

    If this much is uncontroversial (and, under some interpretation, I believe it must be), we can use it as a form for filling out.
    I propose, then, that we explain an event that is an action done for reasons by appealing to the fact that the agent took certain considerations to settle the question of whether to act in some way, therein intended so to act, and successfully executed that intention in action.
    I suggest that \emph{this} complex fact, \dots is the reason that rationalizes the action---that explains the action by giving the agent’s reason for acting.%
    \mbox{ }\hfill\mbox{(\citeyear[431]{Hieronymi:2011aa})}
  \end{quote}

  From the deliberation.

  So, reason is the complex fact.
  Complex fact gives the reason the agent acted, and so content of constituent considerations from agent's perspective.

  In particular, note here that everything is directed at the question.
  Premise-conclusion relationship.
\end{note}

%%% Local Variables:
%%% mode: latex
%%% TeX-master: "master"
%%% End:
