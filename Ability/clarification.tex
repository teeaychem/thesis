\chapter{Conclusions}
\label{cha:clar}

\begin{note}
  \autoref{cha:intro} introduced conclusions as proposition-value pairs from some \pool{} of premises.
  The present chapter details the way in which we understand conclusions in greater detail, together with events in which an agent concludes and events in which an agent is concluding.

  The chapter is split into three main sections:

  \begin{TOCEnum}
  \item
    \TOCLine{cha:clar:sec:Cons}

    Conclusions as \evalN{} of a proposition by a value. \pool{3}.
  \item
    \TOCLine{cha:clar:sec:Concls}

    The scope of an event in which an agent concludes.
    A link between events in an agent concludes and events in which an agent reasons.
  \item
    \TOCLine{cha:clar:sec:Cing}

    Our understanding of the progressive aspect (in particular the assumption of `\assuPP{}').
    Definition of a \pevent{}.
  \end{TOCEnum}

  The sections are cumulative.
  We use conclusions to understand an event in which an agent concludes, and events in which an agent concludes to understand events in which an agent is concluding.
\end{note}

\section{Conclusions}
\label{cha:clar:sec:Cons}

\begin{note}
  We split our discussion of conclusions into two sections.
  The first section covers conclusions as \evalN{1} and the second covers conclusions from \pool{1}.
\end{note}

\subsection{Conclusions as \evalN{1}}
\label{cha:clar:sec:Cons:pvp}

\begin{note}
  This section clarifies our understanding of conclusions in terms of \evalN{}.
  We define propositions and values, then turn to \evalN{1} and conclusions.
\end{note}
%
\begin{note}
  \begin{definition}[Propositions and values]
    \label{def:prop-val}
    \mbox{ }%
    \vspace{-\baselineskip}
    \begin{enumerate}[noitemsep, label=]
    \item
      \begin{itemize}
      \item
        A \emph{proposition} is some state of affairs.
      \item
        A \emph{value} is an \agpe{} on a state of affairs.
      \end{itemize}
    \end{enumerate}
    \vspace{-\baselineskip}
  \end{definition}

  Propositions are states of affairs, whether actual, possible, true ideal, or regrettable, etc.
  Values captures an \agpe{} on the state of affairs, whether \emph{being} actual, possible, true, ideal, or regrettable, etc.

  \begin{notation}
    \item
      Specific propositions and values are written in a \textsf{sans-serif} font.

      For example \prop{The hedge is shaped like a squirrel} is the proposition/state of affairs in which a hedge is shaped like a squirrel.
      \val{True} is the value `True'.
    \item
      Arbitrary propositions and values are written using:
      \begin{itemize}[noitemsep]
      \item
        Lower case Greek letters, \(\phi, \psi, \dots\) for propositions.
      \item
        The Latin letter \(v\) with some number of \('\)'s, \(v, v', v'' \dots\), for values.
      \end{itemize}
    \item
      \(\pv{\phi}{v}\) links a proposition and a value.

      For example:
      \(\pv{\prop{The hedge is shaped like a squirrel}}{\val{True}}\)
    \end{notation}
\end{note}

\begin{note}
  An \evalN{} (partially) captures the way things are for an agent.

  \begin{definition}[\evalN{3}]
    \label{def:evals}
    \mbox{ }
    \vspace{-\baselineskip}
    \begin{itemize}
    \item
      An \emph{\evalN{}} pairs a proposition and a value such that:
      \begin{itemize}
      \item
        The pairing captures the way things are from an \agpe{}.
      \end{itemize}
    \end{itemize}
    \vspace{-\baselineskip}
  \end{definition}

  \begin{notation}
  \item
    When context is clear, \(\pv{\phi}{v}\) abbreviates an \evalN{0} of \(\phi\) with \(v\).
  \end{notation}
\end{note}

\begin{note}
  Some examples:

  \begin{enumerate}[label=\arabic*., ref=(\arabic*),noitemsep, series=propValExC]
  \item
    The bag is heavy.%
    \hfill%
    \(\pv{\prop{The bag is heavy}}{\val{True}}\)
  \item
    The room isn't full.%
    \hfill%
    \(\pv{\prop{The room is full}}{\val{False}}\)
  \item
    \label{pvEx:bC}
    I like bitter coffee.%
    \hfill%
    \(\pv{\prop{Bitter coffee}}{\val{Desirable}}\)
  \item
    I should go to work.%
    \hfill%
    \(\pv{\prop{I go to work}}{\val{Ought}}\)
  \item
    There's no chance they find the%
    \hfill%
    \(\pv{\prop{The needle is found}}{\val{Unlikely}}\)
  \item
    It's too bad the balloon deflated.%
    \hfill%
    \(\pv{\prop{The balloon is deflated}}{\val{Unfortunate}}\)
  \end{enumerate}

  Nothing of importance rests on the value chosen.
  Indeed, the relevant values may be restricted to \val{True} and \val{False}, which we understand as synonyms for \val{Actual} and \val{Not actual}.
  For example, \ref{pvEx:bC} may be recast as \(\pv{\prop{I like bitter coffee}}{\val{True}}\).

  Still, a range of values allows us to easily capture different perspectives on the same state of affairs.
  For example:
  \begin{enumerate}[label=\arabic*., ref=(\arabic*),noitemsep, resume*=propValExC]
  \item I like how bitter the coffee is.%
    \hfill%
    \(\pv{\prop{The coffee is bitter}}{\val{True}}\)\newline
    \hfill%
    \(\pv{\prop{The coffee is bitter}}{\val{Desirable}}\)
  \end{enumerate}

  In this respect, \evalN{} are similar to propositional attitudes.
  I both \emph{believe} \prop{The coffee is bitter} and \emph{desire} \prop{The coffee is bitter}.
  However, \evalN{1} are coarser than propositional attitudes.
  For example, \(\pv{\prop{The coffee is bitter}}{\val{True}}\) does not distinguish between whether the agent \emph{believes} or \emph{knows} \prop{The coffee is bitter}.%
  \footnote{
    \label{fn:belief-is-difficult}
    Though I don't think belief is straightforward.
    Consider the Jeremy Goodman's example:
    \begin{quote}
      Assume that horse A is more likely to win than horse B which in turn is more likely to win then horse C (so the probabilities of winning could be known to be 45, 28, 27\%).
      In this case it seems fine to say `I think horse A will win' or `I believe horse A will win'.%
      \mbox{ }\hfill\mbox{(\cite[1440]{Hawthorne:2016wv})}
    \end{quote}
    % As \citeauthor{Hawthorne:2016wv} observe: \textquote{[I]t is awful to say, in this case, `I think horse A will win but I don't believe it will'.}
    % (\citeyear[1440, fn.17]{Hawthorne:2016wv})
  }

  Still, \evalN{1} capture the common characteristic of all propositional attitudes:
  Some proposition being some way from the \agpe{}.
  And, with respect to conclusions are interest is limited to this characteristic.
  We have no interest with any particular propositional attitude.
\end{note}

\begin{note}
  \nocite{Scriven:1962vq}
  \nocite{Woodward:2021ue}
  \nocite{Perry:1979vc}
  \nocite{Perry:1986aa}
  \nocite{Collins:1997wn}

  \evalN{3} capture an \agpe{} on the way things are.
  To clarify, consider \autoref{fig:zollner-illusion}, an instance of the \citeauthor{Zollner:1860vx} illusion.

  \begin{figure}[!h]
    \centering
    \def\svgwidth{\columnwidth}
    \input{ZIOh.pdf_tex}
    \caption{An instance of the \citeauthor{Zollner:1860vx} illusion --- crop of ~\textcite{Fibonacci:2007vj}}
    \label{fig:zollner-illusion}
  \end{figure}

  \autoref{fig:zollner-illusion} contains eight long black lines which are crossed with short black lines.
  If the illusion has been successfully recreated, then on first pass the long lines do not appear parallel to one another.
  So, from \agpe{your}, \prop{The long lines are parallel} is \evaled{} \val{False}.

  However, on closer inspect (perhaps with the aid of the ruler) you may determine the long lines are parallel.
  If you are convinced that the lines are parallel, then from \agpe{your}, \prop{The long lines are parallel} is re-\evaled{} \val{True}.
  That is, your \evalN{} captures the way things are from \agpe{your}, rather than the way things visually appear, etc.%
  \footnote{
    Consider also Cypher's thoughts while eating steak. (\cite[330--331]{Wachowski:2000uh})
  }
\end{note}

\begin{note}
  As \evalN{1} capture an \agpe{} on the way the things are we distinguish \evalN{1} from a proposition-value pairings \emph{given various assumptions}.
  When we speak of an agent's \evalN{} of \prop{Rover will fall asleep soon} as \val{True}, then Rover \emph{will} fall asleep soon, for the agent.
  By contrast, if the agent has been entertaining the idea that Rover is tired then the agent is entertaining the idea that the proposition \prop{Rover is tired} has the value \val{True}, but we do not consider the proposition-value pair \evaled{0}.

  \phantlabel{mention:concluding-non-factive}
  Further, we allow for the possibility that some proposition is \evaled{0} with some value, though things are otherwise.
  Following an example of \citeauthor{Williams:1979wi}, \prop{The stuff is gin} may be \eval{} \val{True} though the stuff really is petrol (\citeyear[18]{Williams:1979wi}).%
  \footnote{
    For a slightly more involved case, an agent may \eval{} \prop{\(0.999\dots \ne 1\)} as \val{True} while holding themselves to have a conventional understanding of real numbers.
    (There are various interpretations under which \(0.999\dots \ne 1\), but it's convention the property holds for the reals.)

    For, the agent may have failed to grasp the Archimedean property does not hold for real numbers, and may think that, though \(0.999\dots\) approaches \(1\), there must be \emph{some} difference between \(0.999\dots\) and \(1\) --- no matter how small --- and so they are not equal.
  }
\end{note}

\begin{note}
  Conclusions are \evalN{1}:

  \begin{definition}[Conclusions as \evalN{1}]
    \label{assu:concluding:pvp}
    \mbox{ }
    \vspace{-\baselineskip}
    \begin{itemize}
    \item
      A \emph{conclusion} is an \evalN{} made as the result of some process.
    \end{itemize}
    \vspace{-\baselineskip}
  \end{definition}

  Specifically, conclusions are \evalN{} which arise from some process.
  Emphasis may be placed on `\evalN{}' (Prospero's epilogue was the conclusion of the play) or `the result of some process' (some evaluations may be innate).
\end{note}

\subsection{Conclusions from \pool{1}}
\label{sec:pools-premises}

\begin{note}
  With a definition of \evalN{1} in hand, we define \pool{1} as follows:

  \begin{definition}[\pool{3}]
    \label{def:pools}
    \mbox{ }
    \vspace{-\baselineskip}
    \begin{itemize}
    \item
      A \emph{\pool{}} is a (maybe empty) collection of \evalN{1}.
    \end{itemize}
    \vspace{-\baselineskip}
  \end{definition}

  \begin{notation}
  \item
    Arbitrary \pool{1} are written using upper case Greek letters; \(\Phi, \Psi, \dots\).

    For example, \(\Phi\) may contain \(\pv{\phi}{v}, \pv{\psi}{v'}, \dots\) or be empty.
  \end{notation}

  The role of a \pool{} in this document is to keep track of which \evalN{1} an agent makes a conclusion from.

  \begin{assumption}[Conclusions from \pool{3}]
    \label{assu:concluding:pools}
    \cenLine{
      \begin{VAREnum}
      \item
        Agent: \vAgent{}
      \item
        Proposition: \(\phi\)
      \item
        Value: \(v\)
      \item
        \mbox{ }
      \end{VAREnum}
    }
    \begin{itemize}[noitemsep]
    \item
      \begin{enumerate}
      \item[\emph{If}:]
        \(\phi\) having value \(v\) is a conclusion by \vAgent{}.
      \item[\emph{Then}:]
        \(\phi\) having value \(v\) is a conclusion by \vAgent{} from some \pool{} \(\Phi\).
      \end{enumerate}
    \end{itemize}
    \vspace{-\baselineskip}
  \end{assumption}

  \begin{notation}
  \item
    \(\pvp{\phi}{v}{\Phi}\) associates some pair/\evalN{} \(\pv{\phi}{v}\) with a \pool{} \(\Phi\).
  \end{notation}

  In many cases, a \pool{}, or an collection of \evalN{} which form the basis of a \pool{} are clearly associated with an conclusion.
  For a handful of brief examples, consider the following:

  \begin{enumerate}[label=\arabic*., ref=(\arabic*), noitemsep]
  \item
    \emph{A} testified that \(\phi\) is true, so \(\phi\) is true.
  \item
    \(\phi\) would be nice, so \(\phi\) ought to be the case.
  \item
    The song is produced by \emph{B}, so I want to listening to it.
  \item
    The device reads `\(\phi\)' and is reliable, so \emph{not}-\(\phi\) is unlikely.
  \end{enumerate}

  Each sentence may be used to identify a conclusion, marked by `so' with some \pool{}.
  I doubt each sentence captures the full extent of the premises involved.
  We have no interest in specifying exactly which \evalN{} belong to any given \pool{}.

  Further, by allowing \pool{1} to be empty, \autoref{assu:concluding:pools} does not rule out cases where an agent makes a conclusion independent of any prior \evalN{1}.%
  \footnote{
    To motivate, consider the parallel between the~\citeauthor{Ramsey:1929tf} test for conditionals and a Fitch-style rule for conditional introduction in propositional logic.
    \textcite{Read:1995wf} describe the test as follows:

    \begin{quote}
      One should believe a conditional. `if \emph{A} then \emph{B}' if one would come to believe \emph{B} if one were to add A to one's stock of beliefs.%
      \mbox{ }\hfill\mbox{(\citeyear[47]{Read:1995wf})}
    \end{quote}

    A Fitch-style rule for conditional introduction in propositional logic is as follows on the left, which an instance on the right (cf. \cite[206]{Barwise:1999tu}, \cite{Pelletier:2021vp}):

    \begin{center}
      \begin{fitch}
        \ftag{\text{\scriptsize \emph{i}}}{\fa \fh P} & \\
        \ftag{\text{\scriptsize }}{\fa \fa \vdots} & \\
        \ftag{\text{\scriptsize \emph{j}}}{\fa \fa Q} & \\
        \ftag{\text{\scriptsize \emph{j+1}}}{\fa P \rightarrow Q} & \(\rightarrow\)\textbf{Intro:}\emph{i}--\emph{j} \\
      \end{fitch}%
      \hfil%
      \begin{fitch}
        \ftag{\text{\scriptsize 1}}{\fa \fh P \land Q} & \\
        \ftag{\text{\scriptsize 2}}{\fa \fa Q} & \(\land\)\textbf{Elim:}\emph{1} \\
        \ftag{\text{\scriptsize 3}}{\fa (P \land Q) \rightarrow Q} & \(\rightarrow\)\textbf{Intro:}\emph{1}--\emph{2} \\
      \end{fitch}
    \end{center}

    The rule states that at any point in a proof, one may assume \(P\), then, after deriving \(Q\) from the assumption of \(P\), one may discharge assumption of \(P\) and introduce the conditional \(P \rightarrow Q\).
    Note, the assumption of \(P\) on line \emph{i} corresponds to adding \(P\) to collection of propositions proved or assumed up to line \emph{i}, and hence to one's stock of beliefs.

    Now, both the \citeauthor{Ramsey:1929tf} test and the Fitch-style rule for conditional introduction describe a clear \emph{processes} for which a conditional is a conclusion, but there may not be any \emph{premises} associated with the conclusion.
    Observe, the instance derivation does not involve any premises.
  }
\end{note}

\section{Concludes}
\label{cha:clar:sec:Concls}

\begin{note}
  With conclusions and \pool{1} in hand, we now turn to events in which an agent concludes some proposition has some value.
  Recall, our primary interest is with understanding the way an event in which some agent concludes some proposition has some value happened.

  We do not place any constraints on what happens when an agent concluded.
  However, the \emph{scope} of the event is important.
  We link events in which an agent concludes to reasoning and use this link to clarify scope.
\end{note}

\subsection{Reasoning}
\label{cha:clar:sec:Concls:reasoning}

\begin{note}
  We assume conclusions are the result of reasoning:

  \begin{assumption}[Conclusions by reasoning]
    \label{assu:ConRea}
        \cenLine{
      \begin{VAREnum}
      \item
        Agent: \vAgent{}
      \item
        Proposition: \(\phi\)
      \item
        Value: \(v\)
      \item
        \pool{2}: \(\Phi\)
      \item
        Event: \(e\)
      \item
        \mbox{ }
      \end{VAREnum}
    }

    \begin{itemize}
    \item
      \begin{itemize}
      \item[\emph{If}:]
        \(e\) is an event in which \vAgent{} concludes \(\pv{\phi}{v}\) from \(\Phi\).
      \item[\emph{Then}:]
        \(e\) is an event in which \vAgent{} \emph{reasons} to \(\pv{\phi}{v}\) from \(\Phi\).
      \end{itemize}
    \end{itemize}
    \vspace{-\baselineskip}
  \end{assumption}

  An event in which an agent concludes is particular case of an event in which an agent reasons.

  For example, an agent may conclude \(\pv{\psi}{\val{True}}\) from a \pool{} containing \(\pv{\phi}{\val{True}}\) and \(\pv{\prop{If }\phi\prop{ then }\psi}{\val{True}}\) by reasoning via \emph{modus ponens}.
  Given~\autoref{assu:ConRea}, the event in which the agent concludes includes the the agent's use of \emph{modus ponens}.

  However, we have little interest in what reasoning is.%
  \footnote{
    As highlighted in \autoref{cha:intro}, our interest is with \emph{relations} between conclusions and \pool{}.
  }
  Hence, we do not place any constraints on reasoning.
  By this I mean, any instance of reasoning we talk about should be recognised as reasoning by any sensible account of what reasoning is and so you are free to import any constraint you wish.%
  \footnote{
    In this respect, our understanding of an event in which an agent reasons is compatible with, e.g., \citeauthor{Broome:2013aa}'s (\citeyear{Broome:2013aa}) rule following account of reasoning.

    Consider in particular \citeauthor{Broome:2013aa}'s discussion of reasoning from \(\pv{\prop{It is raining}}{\val{True}}\) and \(\pv{\prop{If it is raining the snow will melt}}{\val{True}}\) to \(\pv{\prop{I hear trumpets}}{\val{True}}\).
    The conclusion does not follow by something like \emph{modus pones}.
    However, \citeauthor{Broome:2013aa} grants the possibility of a distinct rule which allows the agent to move from the premises to the conclusion, and holds that \emph{if} the agent is following the rule, the agent is reasoning.
    (\citeyear[233]{Broome:2013aa})
  }%
  \(^{,}\)%
  \footnote{
    For contrast, \citeauthor{Wedgwood:2006ui} assumes reasoning \textquote{is the process of \emph{revising ones beliefs or intentions, for a reason}} (\citeyear[600]{Wedgwood:2006ui}).

   \begin{quote}
    If a set of antecedent mental states makes it rational for one to form a new belief or intention, then those antecedent mental states are surely of a suitable type and content so that it is \emph{intelligible} that they could represent one's reason for forming that belief or intention.%
    \mbox{ }\hfill\mbox{(\citeyear[662]{Wedgwood:2006ui})}
  \end{quote}

   \citeauthor{Wedgwood:2006ui} provides the following contrasts to clarify their understanding of intelligibility:

  \begin{quote}
    [T]he belief that the \emph{Oxford Dictionary of National Biography} says that Hume died in 1776 seems a mental state of a suitable type and content so that it could intelligibly represent one's reason for believing that Hume died in 1776.
    On the other hand, it is not (except in the presence of some rather extraordinary background beliefs) a mental state of a suitable type and content so that it could intelligibly represent one's reason for believing that every even number is the sum of two primes.%
    \mbox{ }\hfill\mbox{(\citeyear[662]{Wedgwood:2006ui})}
  \end{quote}

  Strictly, \citeauthor{Wedgwood:2006ui} understanding reasoning in terms of dispositions that respond to what `rationalizes' (\citeyear[672]{Wedgwood:2006ui}) and \citeauthor{Wedgwood:2006ui} considers `rationalizes' and  `makes intelligible' as equivalent.
    Therefore, the following conditional, states a sufficient rather than necessary condition which leads to \citeauthor{Wedgwood:2006ui}'s assumption, rather than being an expansion of what \citeauthor{Wedgwood:2006ui}'s assumption amounts to.
    I find this particularly confusing.
  }

  There are two basic upshots of \autoref{assu:ConRea}:

  \begin{itemize}[noitemsep]
  \item
    When discussing an event in which an agent concludes, we may talk in terms of the agent's reasoning, where natural.
  \item
    Accounts of reasoning constrain account of conclusions.
  \end{itemize}
\end{note}

\subsection{Scope}
\label{sec:scope}

\begin{note}
  A further upshot of \autoref{assu:ConRea} is the \emph{scope} of an event in which an agent concludes:
  The event in which an agent concludes \(\pv{\phi}{v}\) from \(\Phi\) \emph{spans} the event in which the agent reasons to the conclusion \(\phi\) has value \(v\) from the \pool{} \(\Phi\).

  Hence, \autoref{assu:ConRea} specifies a particular sense of the term `concludes'.
  The sense of concludes which \emph{includes} the relevant reasoning leading up to some conclusion and does not focus only on the more-or-less instantaneous event in which the agent forms the relevant conclusion.
\end{note}

\begin{note}
  To illustrate, we contrast to instances of the term `concludes'.

  The first instance is from \citeauthor{Gardner:1986wp}'s discussion of Newcomb's problem:

  \begin{quote}
    A large number of those who recommended taking only the second box performed the expected-value calculation and concluded that, provided the probability that the Being was correct was at least .5005, they would take only the second box.%
    \mbox{ }\hfill\mbox{(\citeyear[166]{Gardner:1986wp})}
  \end{quote}

  The second instance is from \citeauthor{Bratman:1979aa}'s discussion of temptation:

  \begin{quote}
    Sam thinks both that in certain respects his drinking would be, \emph{prima facie}, best, and that in certain other respects his abstaining would be, \emph{prima facie}, best.
    Weighing these conflicting considerations he concludes that it would be best to abstain, rather than drink.%
    \mbox{ }\hfill\mbox{(\citeyear[156]{Bratman:1979aa})}
  \end{quote}

  Both passages capture a similar process.
  From some \agpe{} various states of affairs are evaluated in different ways, and after some consideration the agent chooses to do something.

  The contrast between the two passages in the event captured by `concludes'.
  \citeauthor{Gardner:1986wp} presents the scenario as a sequence;
  Two verbs (`performs', `concludes') are linked by `and'.
  \citeauthor{Bratman:1979aa}, modifies a verb (`concludes') by a different verb (`weighs').
  On a natural reading, the event in which the agent concludes as described by \citeauthor{Gardner:1986wp} is distinct from the event in which the agent performs the expected-value calculation.
  And, by contrast, \citeauthor{Bratman:1979aa} describes a single event in which the agent concludes by weighing conflicting considerations.

  \autoref{assu:ConRea} is compatible with \citeauthor{Bratman:1979aa}'s description.
  However, if the agent concluded they would take only the second box from whatever premises are associated with the expected-value calculation given \citeauthor{Gardner:1986wp}'s description, then the event \citeauthor{Gardner:1986wp} identifies with `concludes' is too narrow.
  Following \citeauthor{Bratman:1979aa}, we may recast \citeauthor{Gardner:1986wp}'s description as \textquote{Performing these expected-value calculations, they concluded  they would take only the second box}.%
  \footnote{
    In addition, it seems to me that whether or not one defaults to thinking of a more-or-less instantaneous event or an inclusive event comes apart in multi-agent cases.

  For example, consider Sam and Max working on some problem \(p\) together.
  Sam and Max trade ideas back and forth, and settle on an answer \(a\).

  Consider the events described by the following sentence:
  \begin{enumerate}[label=\arabic*., ref=(\arabic*), noitemsep]
  \item
    \label{Cing:SandM:j}
    Sam and Max (jointly) conclude \(a\) is an answer to problem \(p\).
  \item
    \label{Cing:SandM:s}
    Sam concludes \(a\) is an answer to problem \(p\).
  \item
    \label{Cing:SandM:m}
    Max concludes \(a\) is an answer to problem \(p\).
  \end{enumerate}

  My default is to consider the events captured by \ref{Cing:SandM:s} and \ref{Cing:SandM:m} in line with \citeauthor{Gardner:1986wp}.
  When considering Sam or Max in isolation, the event in which Sam (or Max) concludes more-or-less instantaneous event.
  Though, the relevant events may be expanded to include the trading of ideas by appending `by trading ideas with Max/Sam'.

  By contrast, I read \ref{Cing:SandM:j} in line with \citeauthor{Bratman:1979aa}.
  When considering Sam and Max as a pair, in which a non-distributive reading of `conclude' is forced by the parenthetical `jointly', the event includes the trading of ideas back and forth.
  Though, the relevant event includes sub-event in which Sam and Max separately evaluate that \(a\) is an answer to problem \(p\).
  }
\end{note}

\newpage

\begin{note}
  \begin{observation}[Bounds]%
    \label{obs:newPVp-newE}%
    Suppose an agent concludes \(\pv{\phi}{v}\).
    The agent beings reasoning from a \pool{} of premises \(\Phi\).
    However, part way through the agent comes to \eval{} \(\psi\) to have value \(v'\) independently from \(\Phi\) and appeals to \(\pv{\psi}{v'}\) to conclude \(\pv{\phi}{v}\).

    The event in which the agent concludes \emph{includes} but is \emph{not limited to} starting with the agent \evaling{} \(\psi\) with \(v'\).
  \end{observation}

  \begin{motivation}{obs:newPVp-newE}%
    By stipulation, the agent concludes \(\pv{\phi}{v}\) from \(\Phi\) together with \(\pv{\psi}{v'}\).
    So, from \autoref{assu:ConRea} it follows the event in which the agent concludes includes the \evalion{} \(\psi\) has value \(v'\).
    However, \autoref{assu:ConRea} does not limit the event in which the agent concludes to the event in which the agent reasons.
  \end{motivation}

  We tacitly assume the event in which an agent concludes \(\pv{\phi}{v}\) from \(\Phi\) is the minimal event in which the agent reasons to \(\pv{\phi}{v}\) from \(\Phi\).
  However, nothing in particular hangs on this.
\end{note}

\section{Concluding}
\label{cha:clar:sec:Cing}

\begin{note}
  We have covered conclusions as \evalN{1} and events in which an agent concludes.
  The final things of interest is events in which an agent in concluding.

  Our interest is with understanding the way an event in which some agent \emph{concludes} some proposition has some value happened.
  However, the way in which we develop this understanding will involve events such that an event in which the agent concludes is \emph{in progress} --- the agent is `concluding'.
\end{note}

\begin{note}
  English does not have a quick, unambiguous, way of expressing events in progress.%
  \footnote{
    This is difficult.
    I have considered ensuring every instance of an `-ing' verb expresses the progressive, but this is quite hard (see the fourth word in this sentence).
    And, I have considered writing `\(\text{Prog}(\text{concludes})\)' in place of `concluding', but this is too clumsy.
  }
  For, consider the sentence:
  \begin{enumerate}[label=\arabic*., ref=(\arabic*)]
  \item
    \label{prog:abmig}
    \textquote{John is studying for an exam}.
  \end{enumerate}
  \ref{prog:abmig} may be understand to express either the continuous or progressive aspect.

  Under the continuous aspect, \ref{prog:abmig} captures something about John, rather than something about an event happening.
  Hence, it need to be the case that John is engaged in an event of studying when \ref{prog:abmig} is said.
  For example, we may expand:
  \textquote{Sam is studying for an exam, but is taking a short nap.}

  By contrast, \ref{prog:abmig} under the progressive captures an event in where John studying is in progress.%
  \footnote{
    See,~\textcite{Richards:1981wo},~\textcite{Portner:2011vi}, etc.
  }
  For example, we may expand:
  \textquote{Sam is studying for an exam, so they aren't taking a nap.}
\end{note}

\begin{note}
  Our use of the term \textquote{concluding} will always capture an event under the progressive aspect.
  \textquote{The agent is concluding \(\phi\) has value \(v\)} is read in parallel to a natural reading of  \textquote{Taylor walking to the shops} or \textquote{Riley is filling a bathtub}.
  An event is in progress, but the event has not been completed.
\end{note}

\begin{note}
  Our interest with events in which an agent in concluding in the sense of possibility the progressive captures.
  We use this sense of possibility to capture various modal aspects of an event in which an agent concludes.

  First we briefly discuss the progressive and the sense of possibility the progressive captures.
  Second, we define a `\pevent{}', esp.\ a `\pevent{} in which an agent concludes'.
\end{note}

\subsection{The progressive and \assuPP{}}

\begin{note}[Interest with the progressive]
  Our interest with the progressive is due to the sense of possibility required for a sentence stating an event in the progressive to be true.
  \phantlabel{imperfective-paradox:intro}
  The puzzle is perhaps clearest with the `imperfective paradox' (\cite[cf.][Ch.3.1]{Dowty:1979vq}).

  \citeauthor{Bach:1986tb} summarises:
  \begin{quote}
    [H]ow can we characterize the meaning of a progressive sentences like \ref{Bach:impP:17} on the basis of the meaning of a simple sentence like \ref{Bach:impP:18} when \ref{Bach:impP:17} can be true of a history without \ref{Bach:impP:18} ever being true?
    \begin{enumerate}[label=(\arabic*), ref=(\arabic*)]
      \setcounter{enumi}{16}
    \item
      \label{Bach:impP:17}
      John was crossing the street.
    \item
      \label{Bach:impP:18}
      John crossed the street.%
      \mbox{ }\hfill\mbox{(\citeyear[12]{Bach:1986tb})}
    \end{enumerate}
  \end{quote}
\end{note}

\begin{note}
  The `paradox' amounts to two seemingly incompatible observations:
  \begin{enumerate}[noitemsep]
  \item
    \ref{Bach:impP:17} means an event in which John is crosses the street is in progress.
  \item
    There need not be an (actual) event in which John is crosses the street.%
  \end{enumerate}

  For example, John may have been hit by a bus.
  In parallel, it may be true that you are falling asleep before a fire alarm is set off.
\end{note}

\begin{note}
  This `paradox' suggests the following:

    \begin{assumption}[\assuPP{2}]
    \label{assu:PP}
    For any event \(e\) and action description \(\alpha\):
    \begin{enumerate}
    \item[\emph{If}:]
      \begin{enumerate}[label=\alph*., ref=(\alph*)]
      \item
        \(\text{Prog}(e,\alpha)\) is true.%
        \hfill(I.e.\ \(e\) is an event of \(\alpha\)ing.)
      \end{enumerate}
    \item[\emph{Then}:]
      \begin{enumerate}[label=\alph*., ref=(\alph*), resume]
      \item
        There is some \progAdj{0} event \(e'\) such that~\ref{assu:PP:pe:dev} and~\ref{assu:PP:pe:verb} are both true:
        \begin{enumerate}[label=\roman*., ref=(\roman*)]
        \item
          \label{assu:PP:pe:dev}
          \(e'\) is a development of \(e\).
        \item
          \label{assu:PP:pe:verb}
          \(\alpha\) is true of \(e'\).
        \end{enumerate}
      \end{enumerate}
    \end{enumerate}
    \vspace{-\baselineskip}
  \end{assumption}

  \assuPP{2} is a common feature of analyses of the progressive.%
  \footnote{
    See, e.g.:
    \cite{Bennett:1972uw},
    \cite{Dowty:1979vq},
    \cite{Parsons:1990aa},
    \cite{Landman:1992wh},
    \cite{Portner:1998um}.

    However,~\assuPP{0} is denied by~\textcite{Szabo:2004ul}.
    \citeauthor{Szabo:2004ul} argues:
    \begin{quote}
      Sometimes we are \emph{doing} things even though there is no real chance that we could get them \emph{done}, and this is true even if we abstract away from the possibility of miraculous intervention.%
      \mbox{ }\hfill\mbox{(\citeyear[40]{Szabo:2004ul})}
    \end{quote}
    To illustrate, \citeauthor{Szabo:2004ul} denies~\ref{Szabo:Arch} is necessarily false:
    \begin{quote}
      \begin{enumerate}[label=(\arabic*), ref=(\arabic*)]
        \setcounter{enumi}{12}
      \item
        \label{Szabo:Arch}
        As the architect was building the cathedral he knew that, although he would be building it for another year or so, he couldn't possibly complete it.%
        \mbox{ }\hfill\mbox{(\citeyear[38]{Szabo:2004ul})}
      \end{enumerate}
    \end{quote}
    Though,~\ref{Szabo:Arch} seems always false to me.
    The only sense with which I read~\ref{Szabo:Arch} as true under the progressive requires factivity of knowledge to fail, thus allowing the cathedral to be built.

    (See also (\cite[1245]{Portner:2011vi}) for additional, distinct, discussion of (\cite{Szabo:2004ul}).)

    Still, we do not, strictly, require~\assuPP{0}.
    The role of~\assuPP{0} is to ensure the existence of a possible event, where the sense of `possible' is captured by the progressive.
    Hence, if~\assuPP{0} fails then while we lack motivation, it is not clear the events fail to exist under whatever the modality amounts to.
  }
  We assume \assuPP{} holds and offer no (further) account what makes it the case that an action in the progressive is true.%
  \footnote{
    \nocite{Portner:1998um}
    \nocite{Engelberg:1999vi}
    I suggest \textcite{Landman:1992wh} as an introduction, as \citeauthor{Landman:1992wh} considers a variety of important considerations in a straightforward way.
    Further, \citeauthor{Landman:1992wh}'s account of the progressive is developed in terms of closeness between worlds, and the way in which \citeauthor{Landman:1992wh} wrangles closeness to provide an account of the progressive is illustrative of the difficulties and does not rely too heavily on being well versed in formal semantics.

    \textcite{Szabo:2004ul} provides a concise summary:
    \begin{quote}
      [A] progressive sentence is true at some time just in case some event occurs at that time, and if we follow the development of the event (within our world as long as it goes, then jumping into a nearby world, and iterating the process within the limits of reasonability) we will reach a possible world where the perfective correlate is true of the continuation.%
      \mbox{ }\hfill\mbox{(\citeyear[34]{Szabo:2004ul})}
    \end{quote}
    See also (\cite[764--766]{Portner:1998um}) for a more in depth summary.

    For some of the issues with \citeauthor{Landman:1992wh}'s account, see (\cite{Bonomi:1997uq}), (\cite[49--50]{Engelberg:1999vi}), (\cite[35]{Szabo:2004ul}), (\cite[767]{Portner:1998um}), and (\cite[esp.][1256]{Portner:2011vi}).
  }
  Our interest with the progressive is limited to \assuPP{}.
  For, given \assuPP{}, an event in progress has a modal entailment:
  There is some possible completion of the event.%
  \footnote{
    Note, however, possible does not necessarily entail `non-actual'.

    Contrast \textquote{listening to a radio drama} and \textquote{listening to an entire radio drama}.
    Suppose the radio drama accidentally stops halfway through.
    The former does not require a non-actual possible event, as the event in which the agent was listening to a radio drama is an event in which the agent listens to a radio drama.
    To listen does not require to listen until completion.
    The latter, by contrast, requires a non-actual event, as there is no actual event in which the agent listens to the entire radio drama.
    In parallel, concluding \(\pv{\phi}{v}\) from \(\Phi\) requires completion.
  }
  Applied to concluding, \assuPP{} entails:
  A agent is concluding \(\pv{\phi}{v}\) from \(\Phi\) \emph{only if} there's a \progAdj{0} event where the agent concludes \(\pv{\phi}{v}\) from \(\Phi\).
\end{note}

\begin{note}
  As written, \assuPP{} just entails that there is some possible event.
  However, the role of the progressive in this document relies on the sense of possibility captured.
  As we do not give an account of the progressive we do not give an account of the sense of possibility captured.
  Instead, I rely on your tacit understanding of when and where to use the progressive.

  Still, we make a few observations to highlight some features.

  \begin{observation}[\assuPP{2} and existential modality]%
    \label{obs:prog-not-reg-poss}%
    The sense of possibility in \assuPP{} does not reduce to existential \(\{\text{logical}, \text{metaphysical}, \text{nomic}, \dots\}\) possibility.
  \end{observation}
  \begin{motivation}{obs:prog-not-reg-poss}
    Suppose John is sitting a multiple choice exam.
    To pass the exam John only needs to chose some number of correct choices.
    It is certainly logically, metaphysically, and nomically possible that John chooses a sufficient number of correct choices.
    However, it does not follow that John is passing the exam.%
    \footnote{
      See also Igal Kvart's example of Mary wiping out the Roman army (\cite[18]{Landman:1992wh}).
    }
  \end{motivation}

  \begin{observation}[\assuPP{2} and counterfactuals]%
    \label{obs:prog-not-cfs}%
    There is no simple relation between the sense of possibility in \assuPP{} and counterfactuals.
  \end{observation}
  \begin{motivation}{obs:prog-not-cfs}
    Suppose John is passing an exam without external help.
    Then, a classmate slips John some answers, which John then uses.
    It is no longer true that John is passing the exam without external help.
    And, in the closest possible world where the classmate does not slip John answers, it need not be true that John passes the exam without external help.
    For, if John is surrounded by students of a similar mindset then it is plausible that the in closest possible world a different classmate slips John the same answers.
  \end{motivation}

  \begin{observation}[\assuPP{} and uniqueness]%
    \label{obs:prog-no-unique}%
    The progressive may be true of an event without the event being sufficiently developed to `indicate' a unique outcome.
  \end{observation}
  \begin{motivation}{obs:prog-no-unique}
    Suppose John has drawn a straight line on a piece of paper.
    It may be true that John is drawing a triangle.
    However, the straight line is compatible with John drawing an \(n\)-sided polygon, for any \(n\) within a some reasonable bound.%
    \footnote{
      This observation is inspired by \citeauthor{Dowty:1979vq}'s example involving a circle and a triangle (\citeyear[133]{Dowty:1979vq}).
    }
  \end{motivation}
\end{note}

\begin{note}
  Loosely paraphrased, if an event is in progress, then there is something about the way things are which ensures the existence of a possible completion event (\autoref{obs:prog-no-unique}) which is robust against external influence (\autoref{obs:prog-not-cfs}) and does not require luck (\autoref{obs:prog-not-reg-poss}).
\end{note}

\begin{note}
  Our interest with the progressive is with respect to events in which an agent is concluding.
  The progressive entails (given \assuPP{}) there is some possible event in which the agent concludes.

  \begin{proposition}[Possible event in which an agent concludes]
    \label{prp:peventC}
    \cenLine{
      \begin{VAREnum}
      \item
        Agent: \vAgent{}
      \item
        Proposition: \(\phi\)
      \item
        Value: \(v\)
      \item
        \pool{2}: \(\Phi\)
      \item
        Event: \(e\)
      \item
        \mbox{ }
      \end{VAREnum}
    }

    \begin{itemize}
    \item
      \begin{itemize}
      \item[\emph{If}:]
        \(e\) is an event in which \vAgent{} is concluding \(\pv{\phi}{v}\) from \(\Phi\).
      \item[\emph{Then}:]
        There is a possible event \(e'\) in which \vAgent{} concludes \(\pv{\phi}{v}\) from \(\Phi\).
      \end{itemize}
    \end{itemize}
    \vspace{-\baselineskip}
  \end{proposition}

  \begin{argument}{prp:peventC}
    Immediate by \assuPP{}~(\autoref{assu:PP}).
  \end{argument}

  Hence, the progressive allows us to capture conclusions which are available to the agent, but not yet made.
\end{note}

\begin{note}
  Before moving on, observe the quasi-converse of \assuPP{} does not necessarily hold:

  \begin{observation}[Conclusions without concluding]
    \label{obs:cds-arb}
    \cenLine{
      \begin{VAREnum}
      \item
        Agent: \vAgent{}
      \item
        Proposition: \(\phi\)
      \item
        Value: \(v\)
      \item
        \pool{2}: \(\Phi\)
      \item
        Event: \(e\)
      \item
        \mbox{ }
      \end{VAREnum}
    }

    \begin{itemize}
    \item
      The following conditional is not necessarily true:
      \begin{itemize}
      \item[\emph{If}:]
        \(e\) is an event in which \vAgent{} concludes \(\pv{\phi}{v}\) from \(\Phi\).
      \item[\emph{Then}:]
        There is some sub-event \(e'\) of \(e\) such that:
        \begin{itemize}
        \item
          \(e'\) is an event in which \vAgent{} is concluding \(\pv{\phi}{v}\) from \(\Phi\).
        \end{itemize}
      \end{itemize}
    \end{itemize}
    \vspace{-\baselineskip}
  \end{observation}

  \begin{argument}{obs:cds-arb}
    There are (at least) two ways the conditional may fail:

    \begin{itemize}[leftmargin=*]
    % \item
    %   It may be the case that an agent instantly concludes \(\phi\) has value \(v\).

    %   For example, an agent looks outside, sees a bird in a tree and concludes \prop{There is a bird in that tree} has value \val{True}.%
    %   \footnote{
    %     This example highlights an unintuitive consequence of \autoref{assu:ConRea}.
    %     For, conclusions may be `instantaneous' or without any clear reasoning.
    %     I think it is plausible to understand this conclusion in this example to follow from reasoning
    %     %
    %     % This process may be \emph{fast} without being `instantaneous'.

    %     We set this issue aside.
    %     You may revise \autoref{assu:ConRea} to exclude `instantaneous' instances, if you like.
    %   }
    \item
      There may have been no `inertia' with respect to the events prior to the event in which the agent concludes.

    For example, there are exactly five intermediate logics that have the interpolation property (\cite[cf.][]{Maksimova:1977un}).
    However, \citeauthor{Maksimova:1977un} may have been ready to give up on the proof at any point.%
    \footnote{%
      Though, admittedly, it is unlikely \citeauthor{Maksimova:1977un} would have given up.
    }
  \item
    It may fail to be the case that the \emph{agent} is concluding.

    For example, consider being guided through a complex argument.
    (E.g. \citeauthor{Maksimova:1977un}'s proof.)
    At each step, the guide signposts the way to the conclusion.
    However, without the guide the agent would fail to reach the conclusion.
  \end{itemize}
  \vspace{-\baselineskip}
  \end{argument}
\end{note}

\subsection{\pevent{3}}
\label{sec:assupp2}

\begin{note}
  we noted the progressive allows us to capture conclusions which are available to the agent, but not yet made.
  Though, to identify such conclusions via the progressive it must be the conclusion is in progress.
  Still, with the progressive in hand we may define broader modalities in terms of the progressive.
  In particular, we define a \pevent{} as follows:

  \begin{definition}[\pevent{3}]
    \label{def:potenital-event}
    \cenLine{
      \begin{VAREnum}
      \item
        Agent: \vAgent{}
      \item
        Event: \(e\)
      \item
        Action description: \(\alpha\)
      \item
        \mbox{ }
      \end{VAREnum}
    }

    \begin{itemize}
    \item
      There is a \emph{\pevent{}} in which \vAgent{} \(\alpha\)s.
    \end{itemize}

    \emph{If and only if}:

    \begin{itemize}
    \item
      There is some action \(a\) such that both~\ref{def:PE:action} and~\ref{def:PE:prog} are true:

      \begin{enumerate}[label=\alph*., ref=(\alph*)]
      \item
        \label{def:PE:action}
        \vAgent{} may easily and immediately do \(a\).
      \item
        \label{def:PE:prog}
        \(\text{Prog}(e, \alpha)\) is true of the event \(e\) in which \vAgent{} does \(a\).
      \end{enumerate}
    \end{itemize}
    \vspace{-\baselineskip}
  \end{definition}

  The role of \pevent{1} is to capture an event which is `possible' in the sense of the progressive without the truth of the progressive.
  Of course, the added modality to move to an event in which the progressive is true, and this modality may be scrutinised.
  Still, I take an `easy and immediate' action to be sufficiently intuitive.%
  \footnote{
    For background reading, \autoref{def:potenital-event} parallels \citeauthor{Mandelkern:2017aa}'s act conditional analysis of ability, where `practically available' parallels `easy and immediate' (\citeyear[\S5]{Mandelkern:2017aa}).
    Likewise, consider `options' in \citeauthor{Boylan:2020aa}'s `determinacy' analysis (\citeyear[\S4]{Boylan:2020aa}).
  }
\end{note}

\begin{note}
  To close this section, let's put \pevent{1} to work.

  \begin{illustration}[Darts]
    There is a \pevent{} in which agent wins at darts just in case there is some action available to the agent, such that if the agent were to perform the action they would be winning at darts.

    Winning is a complex action.
  An agent has three dart throws to lower their score from 501 to 0 before play switches to the other player, and play continues until neither player may lower their score further on their next turn (without going past 0).
  Playing a game is a complex action, as the region of a dartboard an agent wishes to hit changes according to previous throws.
  For example, if the agent's score is 51 with three throws remaining, the agent will not wish to hit bullseye, as there is no way to reduce their score by a single point using two darts.
  If the agent goes on to hit 20, then the score of the remaining to darts should equal 31, and so on.
  \end{illustration}

  Note, the initial sequence of actions may be more or less arbitrary.
  It is not possible to score 501 in three or six dart throws, so an agent \emph{could} start by throwing a few darts blindly, so long as they have sufficient skill to recover on subsequent throws.

  Of course, throwing darts is quite different from concluding, but this note extends.
  An agent may be concluding a theorem is true even though their first line of enquiry turns out to be a dead end, etc.

  This is the appeal of the progressive.
  Speaking of \pevent{1} allows us to talk about the way in which an agent may bring about an event without the agent being engaged in the process of bringing the event about.
\end{note}

\newpage

\section*{Summary}
% \label{cha:clar:sec:sum}

\begin{note}
  This chapter detailed the way in which we understand conclusions, events in which an agent concludes, and events in which an agent is concluding.
\end{note}

\begin{note}
  A brief summary:
  \begin{itemize}
  \item
    Conclusions are \evalN{1} (\autoref{assu:concluding:pvp}).

    Specifically, an \eval{} of a proposition/state of affairs by some value (\autoref{def:prop-val}), where proposition-value pair captures the way things are from the relevant \agpe{} (\autoref{def:evals}).
  \item
    A conclusion is always from a \pool{1} of \evalN{1} (\autoref{assu:concluding:pools}).
  \item
    An event in which an agent concludes \(\pv{\phi}{v}\) from \(\Phi\) is an event in which the agent reasons to \(\pv{\phi}{v}\) from \(\Phi\) (\autoref{assu:ConRea}).
  \item
    An event in which an agent is concluding is an event in which an event in which the agent concludes is in progress.

    By \assuPP{} (\autoref{assu:PP}), if an agent is concluding there is a possible event in which the agent concludes (\autoref{prp:peventC}).
  \item
    \pevent{3} (\autoref{def:potenital-event}) are defined in terms of the availability of an action for which the progressive would be true, and hence the sense of modality captured by `\potential{0}' is primarily reduces to the sense of `possible' identified by \assuPP{}.
  \end{itemize}
\end{note}

%%% Local Variables:
%%% mode: latex
%%% TeX-master: "master"
%%% End:


% \begin{itemize}
% \item
%   On occasion we use \(\overline{v}\) to indicate some value other than \(v\) of the same type as \(v\) and \(v_{?}\) to indicate \(\phi\) is not paired with any proposition of the same type as \(v\).

%   In summary, the notation on the left is in accordance with the sentence on the right with a default reading below:
%   \begin{itemize}
%   \item
%     \(\pv{\phi}{v}\) \hfill \(\phi\) has value \(v\).%
%     \newline
%     \mbox{ }\hfill \(\phi\) is true.
%   \item
%     \(\pv{\phi}{\overline{v}}\) \hfill \(\phi\) has some value other than \(v\) of the same type.%
%     \newline
%     \mbox{ }\hfill \(\phi\) has some value other than true of the same type.%
%     \newline
%     \mbox{ }\hfill I.e.\ \(\phi\) is false.
%   \item
%     \(\pv{\phi}{v_{?}}\) \hfill \(\phi\) is not evaluated to have a value of type \(v\).%
%     \newline
%     \mbox{ }\hfill \(\phi\) is not evaluated as true or false.
%   \end{itemize}
% \end{itemize}

% \subsection[\citeauthor{Landman:1992wh}'s (\citeyear{Landman:1992wh}) account of the progressive]{\citeauthor{Landman:1992wh}'s (\citeyear{Landman:1992wh}) account of the progressive \hfill (Optional)}
% \label{cha:sec:fcs-def:progressive-landman}
% \nocite{Portner:1998um}
% \nocite{Engelberg:1999vi}

% \begin{note}
%   Progressive has an important role.
%   Natural language.
%   However, useful to refine intuitions.

%   In this section we more-or-less follow~\citeauthor{Portner:1998um}'s (\citeyear{Portner:1998um}) summary of \citeauthor{Landman:1992wh}'s (\citeyear{Landman:1992wh}) of the progressive.

%   \citeauthor{Landman:1992wh}'s account of the progressive has a number of drawbacks.
%   However, for illustrative purposes, the way in which \citeauthor{Landman:1992wh} develops an algorithmic understanding hopefully helps clarify the kind of thing the progressive is.
% \end{note}

% \begin{note}
%   In broad summary:
%   \citeauthor{Landman:1992wh} holds that an action in the progressive holds of some event just in case the event, if `allowed' to develop, would develop into an event in which the action is performed.

%   As we have seen with the perfective paradox, some action in the progressive need not continue in the actual world, and hence the core of \citeauthor{Landman:1992wh}'s account of the progressive is an account of allowing an event to continue.

%   In slightly more detail, the way in which an event is `allowed' to develop is captured by idea of a continuation tree, which we present via a recursive algorithm.
%   \footnote{
%     See \citeauthor{Landman:1992wh} (\citeyear[26--27]{Landman:1992wh}) for \citeauthor{Landman:1992wh}'s account of algorithm.
%     \citeauthor{Landman:1992wh}'s construction is iterative, though I find the recursive easier.

%     \textcite{Szabo:2004ul} provides a concise summary:
%     \begin{quote}
%       [A] progressive sentence is true at some time just in case some event occurs at that time, and if we follow the development of the event (within our world as long as it goes, then jumping into a nearby world, and iterating the process within the limits of reasonability) we will reach a possible world where the perfective correlate is true of the continuation.%
%       \mbox{ }\hfill\mbox{(\citeyear[34]{Szabo:2004ul})}
%     \end{quote}
%   }

%   \begin{algorithm}[H]
%     \caption{Build continuation branch}
%     \label{PrAl:basic}
%     \SetAlgoLined
%     \DontPrintSemicolon
%     \Input{
%       \(e,w,v\)
%     }
%     \KwResult{The continuation of \(e\)}
%     \Begin{
%       Continue the development of \(e\) in \(v\)\;
%       \If{\(e\) reaches a point where \(e\) does not develop further in \(v\)}{
%         Let \(u\) be the closest world to \(v\).\;
%         \If{\(u\) exists and is plausible with respect to \(w\)}{
%           Extend the continuation of \(e\) via \(C(e,w,u)\)\;
%         }
%       }
%       \Return{\(e\)}
%     }
%   \end{algorithm}

%   Progressive is true just in case some event on continuation path.
% \end{note}

% \begin{note}
%   \autoref{PrAl:basic} takes three arguments:
%   An event \(e\), an initial world \(w\), and a possible world \(v\).
%   \autoref{PrAl:basic} starts by developing \(e\) in \(v\).
%   \(e\) may develop in \(v\) without a problem.
%   If so, \autoref{PrAl:basic} simply returns the way in which \(e\) developed.

%   For example, \(e\) may be the event of an agent may be walking across the road.
%   If the agent crosses the road in \(v\), then \(e\) developed without a problem, and \autoref{PrAl:basic} returns \(e\).

%   However, \(e\) may be interrupted.
%   For example, the agent may be hit by a bus.
%   If so, there is a point just before the agent is hit by a bus in which the agent is crossing the road.
%   We are then instructed to consider the closest world \(u\) to \(v\).
%   If \(u\) exists and is `plausible' with respect to the initial world \(w\), then we switch to \(e\) in \(u\) and repeat the construction.

%   There are three clear issues.

%   First, assumption of a closest world.
%   Taken up elsewhere.
%   \autoref{PrAl:basic} follows \citeauthor{Landman:1992wh} in assuming that if a close world exists, then a (unique) closest world exists.

%   The other two problems are more significant.

%   Illustrated the idea of \(e\) developing in terms of the progressive.
%   It is not clear how to specify \(e\) without appeal to the progressive.

%   Plausible.%
%   \footnote{
%     \citeauthor{Landman:1992wh} uses the term `reasonable'.
%     We substitute `plausible' for `reasonable' in order to avoid epistemic connotations.
%     See \textcite[17--19,24--26]{Landman:1992wh} for \citeauthor{Landman:1992wh}'s discussion of what a world being `reasonable' amounts to.
%   }
%   However, no clear understanding of what plausibility amounts to.
%   It is not closeness.

%   Goal is to allow event to develop, but make sure the way in which the event develops makes sense.
%   By shifting to close worlds, consider how things may have otherwise been.
%   And, by doing this relative, allow the continuation to be somewhat distant from initial world.
%   However, as drift to distant worlds, make sure that the incremental changes do not mean the possible world is `too distant' from initial world.

%   Still, this idea of mixing two modalities captures interest of the progressive.

%   Setting aside difficulties.
%   To construct continuation branch for event \(e\) in initial world, run \((C(e,w,v)\).
%   So, possible world is initial world.
% \end{note}

% \begin{note}
%   To see how the above sketch functions in practice, we follow~\citeauthor{Portner:1998um}'s (\citeyear[764--766]{Portner:1998um}) illustration of \citeauthor{Landman:1992wh}'s account.
% \end{note}

% \begin{note}
%   Our interest is with the following sentence:
%   \begin{enumerate}
%   \item
%     \label{prog:max:bad}
%     Max is crossing the street.
%   \end{enumerate}
%   Evaluated with respect to an event \(e\) in world \(w\).
%   (I.e. \(\text{Prog}(e,\text{Max crossed the street})\) is true in \(w\).)

%   Following \citeauthor{Landman:1992wh} (and in line with \assuPP{}), Max is crossing the street is true of \(e\) just in case there is a continuation branch such that \(e\) develops into an event in which Max crosses the street.

%   Now, suppose that in \(w\) \(e\) develops into an event \(e'\) where Max is hit by a bus cruising at thirty miles per hour (before Max crosses the street).
%   Somewhat ominously, let us identify this bus as `bus \#1'.

%   Event \(e'\) includes Max being hit bus \#1, but \(e\) does not, and had things been a little different, it is reasonable Max continued a little further across the street.
%   For example, if the bus had been travelling at twenty five miles per hour, Max may have been just ahead of the bus.
%   Hence, we may consider some world \(v\) which is close to \(w\) in which \(e\) develop a little further.

%   So far so good, but in \(w\) Max was hit by bus \#1 in \(w\).
%   Hence, as \(v\) is \emph{close} to \(w\), the way in which \(v\) is close to \(w\) may require that Max is hit by a bus in \(w\).%
%   \footnote{
%     \citeauthor{Fine:1975tj}'s notice on \citeauthor{Lewis:1973th}'s (\citeyear{Lewis:1973th}) initial account of counterfactuals or \citeauthor{Veltman:2005tj}'s (\citeyear{Veltman:2005tj}) use of an example given by \textcite{Tichy:1976tp} to raise problems for \citeauthor{Lewis:1979vm}'s (\citeyear{Lewis:1979vm}) revision.
%   }
%   So, although Max makes it a little further across the road in \(v\), Max is still hit by a bus.
%   We identify the bus in \(v\) as `bus \#2'.
%   (Perhaps the bus swerves a little to the left.)

%   However, as with bus \#1 in \(w\), it seems Max may have walked a little further across the street in some possible world close to \(v\).
%   Hence, by the same reasoning we may consider some possible world \(u\) close to \(v\) and so on\dots

%   Still, does \(e\) develop into an event in which Max crosses the road?
%   By shifting through close worlds, avoid bus.
%   So long as result of shifting is plausible, then find an event in which Max crosses the road.

%   % The key property of \citeauthor{Landman:1992wh}'s account is that closeness is understood relative to the development of \(e\), rather than \(e\) itself as \(e\) happened in \(w\).
%   % Prior to bus \#2 hitting Max in \(v\), we shifted to \(u\), a world which is close to \(v\) rather than \(w\).
%   % So, as Max progresses a little further each time a possible world in which Max crosses the street gets a little closer until, eventually, Max crosses the street.
%   To borrow a piece of terminology from \textcite{Dowty:1979vq}, \(e\) has sufficient \emph{inertia} to develop in some possible world \(v\), and as \(e\) develops in \(v\), inertia continues to build until Max crosses the road.

%   % However, an important restriction is placed on shifts to possible worlds.
%   % Intuitively, it is not the case that Max avoids being hit by bus \#\(j\) because Max has the strength to stop as moving bus.
%   % Yet, a possible world in which Max has the strength to stop a moving bus may be close to the world in which Max is not hit by bus \#\(j - 1\).
%   % In \citeauthor{Landman:1992wh}'s terminology, the relevant possible worlds in which \(e\) develops must be `reasonable'.
% \end{note}

% \begin{note}
%   \autoref{fig:max-bus} is a modification of \citeauthor{Portner:1998um}'s figure 1. (\citeyear[767]{Portner:1998um})
%   \begin{figure}[!h]
%     \centering
%     \begin{tikzpicture}
%       \tikzmath{
%         % x positions
%         \x1 = 10;
%         \xb1 = 2/9*\x1; \xb2 = 4/9*\x1; \xb3 = 6/9*\x1;
%         % y positions
%         \y1 = 1.25/5*\x1; \ymid = 1/2*\y1;
%         \yw1 = \y1; \yw2 = 1/2*\y1; \yw3 = 0*\y1; \yb2 = 1/5*\y1;
%         % event e
%         \xe = 1/2*\xb1; \yediff = \yw2 - \yb2;
%         \ye = \yw2 - 1/2*\yediff;
%         \enudge = .1;
%         \xel = 0; \xer = \xb1; \yen = \yw2 - \enudge;
%         % bus 1 description location
%         \xbx = 1.5/9*\x1; \xby = 4/5*\y1;
%         % bus 2 description location
%         \xb9 = 2.5/9*\x1;
%       }
%       % Paths
%       \draw[line width=0.25mm, line cap=round] (\xb1,\ymid) -- (\xb3,\yw1); % world 1
%       \draw[line width=1mm, line cap=round, dash pattern=on 175pt off 5pt on 5pt off 5pt on 5pt off 5pt on 5pt off 5pt on 5pt off 5pt on 5pt off 5pt on 5pt off 5pt on 5pt off 5pt on 5pt off 5pt on 5pt off 5pt on 5pt off 5pt on 5pt off 5pt on 5pt off 5pt on 5pt off 5pt on 5pt off 5pt on 5pt off 5pt on 5pt off 5pt on 5pt off 5pt on 5pt off 5pt on 5pt off 5pt] (0,\ymid) -- (\xb1,\ymid) -- (\xb2,\yb2) -- (\xb3,\yw2); % world 2
%       \draw[line width=0.25mm, line cap=round] (\xb2,\yb2) -- (\xb3,\yw3); % world 3
%       % World descriptions
%       \filldraw[black] (\xb3,\yw1) circle (0pt) node[anchor=west, align=left]{world 1: Max hit by \\ bus \# 1};
%       \filldraw[black] (\xb3,\yw2) circle (0pt) node[anchor=west, align=left]{world \(i\): Max \\ crosses street};
%       \filldraw[black] (\xb3,\yw3) circle (0pt) node[anchor=west, align=left]{world 2: Max hit by \\ bus \# 2};
%       % Event
%       \draw[] (\xe,\ye) -- (\xel,\yen); % event l
%       \draw[] (\xe,\ye) -- (\xer,\yen); % event r
%       % Event description
%       \filldraw[black] (\xe,\ye) circle (0pt) node[anchor=north, align=left]{event e};
%       % Splits
%       \filldraw[black, dashed] (\xbx,\xby) circle (0pt) node[anchor=south, align=left]{bus \#1 hits Max};
%       \filldraw[black, dashed] (\xb9,\yw3) circle (0pt) node[anchor=north, align=left]{bus \#2 hits Max};
%       % Split descriptions
%       \draw[-Stealth, dashed] (\xbx,\xby) -- (\xb1,\ymid + \enudge); % bus 2 arrow
%       \draw[-Stealth, dashed] (\xb9,\yw3) -- (\xb2 - \enudge,\yb2 - \enudge); % bus 2 arrow
%     \end{tikzpicture}
%     \caption{
%       Continuation path of `Max was crossing the street'. \\
%     }
%     \label{fig:max-bus}
%   \end{figure}

%   The thick black line captures \(e\) as \(e\) is allowed to develop, and the changes in angle reflect shifts to alternative possible worlds.

%   The dashed line indicates that Max may need to avoids being hit by additional busses in order for \(e\) to develop into an event in which Max crosses the road.
% \end{note}

  % \citeauthor{Broome:2013aa}'s rule following account of reasoning is unconstrained in terms of the rules an agent may follow.
  % This aspect  of \citeauthor{Broome:2013aa}'s account is highlighted in the following passage:%
  % \footnote{
  %   \citeauthor{Broome:2013aa} doesn't require an agent has beliefs about rules followed (\citeyear[cf.][\S13.2]{Broome:2013aa}).
  % }

  % \begin{quote}
  %   [S]hould we exclude this bizarre rule: from the proposition that it is raining and the proposition that if it is raining the snow will melt, to derive the proposition that you hear trumpets.
  %   Following this rule would lead you to believe you hear trumpets when you believe it is raining and believe that if it is raining the snow will melt.
  %   If you did this, should we count you as reasoning?

  %   I think we should.
  %   If you derive this conclusion by operating on the premises, following the rule, we should count you as reasoning.
  %   \dots
  %   I think we should not impose a limit on rules.%
  %   \mbox{}\hfill\mbox{(\citeyear[233]{Broome:2013aa})}
  % \end{quote}


% \begin{note}
%   \begin{observation}
%     \label{obs:going-C}
%     Distinction between `concluding' and `going to conclude'
%   \end{observation}

%   \begin{motivation}{obs:going-C}
%     Builds on \autoref{obs:newPVp-newE}
%   \end{motivation}
%   {
%     \color{red}
%     There is, then, a difference between event in which agent is concluding \(\pv{\phi}{v}\) from \(\Phi\).
%     And, event in which agent is `going' to conclude \(\pv{\phi}{v}\) from \(\Phi\).
%     Where, `going' is progressive, e.g.\ going on a journey.
%     I.e.\ develops such that the agent concludes \(\pv{\phi}{v}\) from \(\Phi\).

%     Indeed, both develop.
%     The difference is that with the former, \pool{} is present.
%     With the latter, the event develops such that the \pool{} is present.
%   }
% \end{note}

  % \footnote{
  %   % \mbox{ }\hfill\mbox{(\hyperlink{cite.Hume:2011aa}{T.2.3.3})}
  %   \nocite{Hume:2011aa}
  %   Clearer given principle \citeauthor{Smith:1987vk} derives from the Humean Theory of Motivation, as captured by~\ref{Smithh:HtM:2}:
  %   \begin{quote}
  %     \begin{enumerate}[label=\textsc{P1}., ref=(\textsc{P1})]
  %     \item
  %       \label{Smithh:HtM:2}
  %       Agent A at t has a motivating reason to \(\phi\) only if there is some \(\psi\) such that, at t, A desires to \(\psi\) and believes that were he to \(\phi\) he would \(\psi\).%
  %       \mbox{ }\hfill\mbox{(\citeyear[36]{Smith:1987vk})}
  %     \end{enumerate}
  %   \end{quote}
  %
  %   Understand desire of A to \(\psi\) in terms of the evaluation of \(\psi\) as desirable, and the belief as evaluating if do \(\phi\) then \(\psi\).
  %
  %   \citeauthor{Smith:1987vk} suggests comparison to \textcite{Davidson:1963aa}.
  %   Note, in particular \citeauthor{Smith:1987vk}'s use of `motivating reason' is equivalent to \citeauthor{Davidson:1963aa}'s use of `primary reason' rather than `reason'.
  % }%
  %   \(^{,}\)%


% \begin{note}
%   Though we do not make any assumptions about reasoning, we do allow for there to be no significant event in which the agent reasons/concludes.

%   \begin{assumption}[Conclusions may be `instantaneous']
%     \label{assu:concluding:instant}
%         \cenLine{
%       \begin{VAREnum}
%       \item
%         Agent: \vAgent{}
%       \item
%         Proposition: \(\phi\)
%       \item
%         Value: \(v\)
%       \item
%         \mbox{ }
%       \end{VAREnum}
%     }
%     \begin{itemize}
%     \item
%       An event in which an agent concludes \(\pv{\phi}{v}\) but is there is no sub-event in which the agent is concluding.
%     \end{itemize}
%     \vspace{-\baselineskip}
%   \end{assumption}

%   The primary role of \autoref{assu:concluding:instant} is to explicitly allow the converse to .
%   Useful for understanding definitions, propositions, and so on.

%   You may have no interest, and we have no particular interest.
%   Useful, however.
% \end{note}
