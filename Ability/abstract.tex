\chapter*{Abstract}
\label{cha:abstract}

\begin{note}
  This document is about partial explanations for why an \eiw{} an agent concludes happened.
\end{note}

\begin{note}
  The kind of partial explanations for why an \eiw{} an agent concludes happened of interest are routine.

  For example, while playing a game of chess an agent concludes moving one of their rooks to e4 secures checkmate.
  At least in part, this conclusion is explained by observing the agent has a good understanding of chess and applied their understanding to identify the checkmate.

  In a more detail, the agent may have performed an exhaustive search over the remaining pieces for a checkmate.
  Or, perhaps the agent had cornered their opponent with a particular strategy.

  In contrast, some things are not partial explanations for why the \eiw{} the agent concludes to move one of their rooks to e4 happened.

  For example, consider the agent's conclusion to eat chocolate rather than vanilla ice cream while playing, or whatever the agent concludes to do in the opening of the next game they play.
  Neither seems irrelevant.
\end{note}

\begin{note}
  The argument of this document is oriented around a general constraint on partial explanations for why an \eiw{} an agent concludes happened.

  Very roughly, the constraint holds that partial explanations are limited by what happens when the agent concludes, or what happened prior to the agent concluding.

  I argue the constraint does not hold.
\end{note}

\begin{note}
  The contribution of this document may be split into four connected areas:

  First, a fairly theory-neutral framework to talk about the way \eiw{1} an agent concludes happen.

  Second, a statement of the relevant constraint within the framework.

  Third, a deductive argument which shows the constraint fails when certain conditions obtain.

  Fourth, supporting argument to establish the relevant conditions often obtain, and so failure of the constraint is fairly common.
\end{note}


\begin{note}
  This document is orientated around the constraint.
  Still, the focus of the document is the details of the argument against the constraint.

  The argument highlights the way a handful of observations combine.
  As such, I have worked to ensure the observations lead to general ideas which may be applied or adapted to other arguments.
\end{note}



%%% Local Variables:
%%% mode: latex
%%% TeX-master: "master"
%%% TeX-engine: luatex
%%% End:
