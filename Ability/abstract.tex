\chapter*{Abstract}
\label{cha:abstract}

\begin{note}
  The argument of this document is oriented around a general constraint on partial explanations of why an \eiw{} an agent concludes happened.
\end{note}

\begin{note}
  The kind of partial explanations for why an \eiw{} an agent concludes happened of interest are routine.

  For example, while playing a game of chess an agent concludes moving one of their rooks to e4 secures checkmate.
  And, at least in part, this conclusion is explained by observing the agent identified the checkmate by applying their understanding of chess.

  In more detail, the agent may have performed an exhaustive search of some fixed depth over the remaining pieces.
  Or, perhaps the agent had finished cornering their opponent with a particular strategy.
\end{note}

\begin{note}
  In contrast, some things are not partial explanations for why the \eiw{} the agent concludes to move one of their rooks to e4 happened.

  For example, consider the agent's conclusion to eat chocolate rather than vanilla ice cream while playing, or whatever the agent concludes to do in the opening of the next game they play.
  Neither seems relevant.
\end{note}

\begin{note}
  Very roughly, the constraint holds that partial explanations are limited by what happens when the agent concludes or prior to the agent concluding.

  I argue the constraint does not hold.
\end{note}

\begin{note}
  The contribution of this document may be split into four connected parts:

  First, a fairly theory-neutral framework to talk about the way \eiw{1} an agent concludes happen.

  Second, a statement of the relevant constraint within the framework.

  Third, a deductive argument which shows the constraint fails when certain conditions obtain.

  Fourth, supporting argument to establish the relevant conditions (often) obtain, and the constraint (often) fails.
\end{note}


\begin{note}
  Though this document is orientated around the constraint, the focus of the document is the details of the argument against the constraint.

  The argument highlights the way a handful of observations combine.
  And, as such, I have worked to ensure the observations lead to general ideas which may be applied or adapted to projects.
\end{note}



%%% Local Variables:
%%% mode: latex
%%% TeX-master: "master"
%%% TeX-engine: luatex
%%% End:
