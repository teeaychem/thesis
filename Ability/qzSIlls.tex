\chapter{\zSN{2}: Additional \illu{1}}
\label{cha:zS:sec:question:illu}

\begin{note}
  Some \illu{1}.
  Two broad sections.

  \begin{itemize}
  \item
    First,~\autoref{cha:zS:sec:question:illu:basic} provides a collection of \illu{1} in which an agent (plausibly) would or would not have \zS{} if they were to conclude some proposition-value pair.

    This section is further subdivided:
    \begin{itemize}
    \item
      \autoref{cha:zS:sec:question:illu:basic:does-not-have} provides a pair of \illu{1} in which an agent (plausibly) would not have \zS{} if they made the relevant conclusion.
    \item
      \autoref{cha:zS:sec:question:illu:basic:has} provides a handful of \illu{1} in which an agent (plausibly) would have \zS{} if they made the relevant conclusion.
      In particular, this section provides discussion of when an agent would (plausibly) have \zS{} due to the absence of any \requ{1}.
    \end{itemize}
  \item
    Second, \autoref{cha:zS:sec:question:illu:carroll} provides an involved \illu{1} which links \qzS{} to \citeauthor{Carroll:1895uj}'s tale about the Tortoise and Achilles.

    These \illu{1} as a whole are optional, but this section is particularly so.
    The discussion of \citeauthor{Carroll:1895uj}'s tale is involved, and focuses in roughly equal parts on providing further clarification of \zS{} and providing a novel (at least to my awareness) \emph{interpretation} of \citeauthor{Carroll:1895uj}'s tale.%
    \footnote{
      I doubt very much that the interpretation given is what \citeauthor{Carroll:1895uj} had in mind.
      On the other hand, it's not clear that \citeauthor{Carroll:1895uj} had any particular interpretation in mind (\cite[Cf.][]{Thomson:2010tt}).
    }
  \end{itemize}
\end{note}

\section{Basic \illu{1}}
\label{cha:zS:sec:question:illu:basic}

\subsection{The agent (plausibly) does not have \zS{}}
\label{cha:zS:sec:question:illu:basic:does-not-have}

\begin{note}[Simple negative answers]
  Variant on lost keys, where agent considers plausible they may reason from some other premise.
  {
    \color{red}
    \begin{illustration}
      \begin{enumerate}
      \item
        A search for `Measurement Theory' via the LCC `H61 .R593' returned no results.
      \item
        The library does not have a copy of `Measurement Theory'.
      \end{enumerate}
    \end{illustration}

    So, if conclude, then option of concluding the library does not use DDC indexing.
    May fail to conclude.
  }

  A more direct variant is spot the difference without a clear statement of how many differences are in the picture.
  Continue to search.
\end{note}

\begin{note}[A copper kettle]
  A further \illu{0} builds on a story as told by~\citeauthor{Freud:1960wx}.
  \begin{illustration}[A copper kettle]
    \label{illu:kettle}
    \mbox{ }
    \vspace{-\baselineskip}
    \begin{quote}
      `A.\ borrowed a copper kettle from B.\ and after he had returned it was sued by B.\ because the kettle now had a big hole in it which made it unusable.
      His defence was:
      ``%
      First, I never borrowed a kettle from B.\ at all;
      secondly, the kettle had a hole in it already when I got it from him;
      and thirdly, I gave him back the kettle undamaged.%
      '''\newline
      \mbox{ }\hfill\mbox{(\citeyear[62]{Freud:1960wx})}
    \end{quote}
    An agent listens to A.'s defence, but does not conclude A.\ has provided testimony.
  \end{illustration}

  The agent's failure to conclude A.\ has provided testimony may be understood in terms of a \requ{}.
  For, A.\ has provided testimony only if what A.\ has said is true.
  And, what A.\ has said is true only if the three points of A.'s defence are jointly consistent.
  Putting these observations together, we have the following conditional:
  \begin{itemize}
  \item
    A.\ has provided testimony \emph{only if} if the three points of A.'s defence are jointly consistent.
  \end{itemize}
  Hence, failure for the agent to conclude the consequent would prevent the agent from concluding the antecedent.

  Here, then, clause \ref{idea:requ:nPsi-nPhi} of \iRequ{} is satisfied.

  Likewise, there are only three points, and checking for consistency does not require the agent to establish whether the points are (actually) true.
  Hence, clause \ref{idea:requ:pool} of \iRequ{} is also satisfied.

  Sp, before concluding A.\ has provided testimony, the agent reasons about whether the three points of A.'s defence are jointly consistent.
  After, the agent does not conclude A.\ has provided testimony.%
  \footnote{
    Any pair of points are jointly inconsistent.
    For example, consider the first and third:
    If A.\ never borrowed the kettle from B, then it is not possible for A.\ to have returned the kettle to B.
  }

  The story as told by \citeauthor{Freud:1960wx} is comical, but the \requ{0} identified is fairly general.
  In many cases one may only accept a story if the details add up, and imbalance would lead to rejection.
\end{note}

\begin{note}[Wally]
  \begin{illustration}[Where's Wally]
    \label{illu:CS:wheres-wally}
    \nagent{15} has a book containing numerous drawings of bustling scenes in which various characters are doing a variety of things.
    And, somewhere in each scene is a character called `Wally', identifiable by a collection of individually necessary and jointly sufficient distinguishing features.
    These features include a red and white striped jumper, blue trousers, short brown wavy hair, and so on.

    \nagent{15} has searched through one particular scene, and has identified a character with a variety of the features.
    Before concluding that the character is Wally, \nagent{15} remembers that there is a picture of Wally On the cover of the book, with all the identifying features present.

    Wally is always wearing a pair of round glasses, but this was not a feature \nagent{15} kept in mind when searching for Wally.
    So, perhaps the character \nagent{15} identified is not wearing round glasses  --- \nagent{15} only recalls the features they identified.
  \end{illustration}

  Interest is with whether \nagent{15} may conclude from the variety of features identified that the character is Wally.

  The difficulty for \nagent{15} is that if \nagent{15} were to check whether the character is wearing a pair of round glasses, and the character is not wearing a pair of round glasses, then \nagent{15} would conclude that the character is not Wally.
  Hence, a \requ{}.
  And, not a \fc{}.
\end{note}

\subsection{The agent (plausibly) has \zS{}}
\label{cha:zS:sec:question:illu:basic:has}

\subsubsection{Already}

\begin{note}
  And, more commonplace examples involve the gradual accumulation of proposition-value pairs.

  \begin{illustration}
    \nagent{16} \emph{said} they're coming to the party, but you know from \nagent{17} that \nagent{16} is coming to the party only if \nagent{18} is coming to the party.
    Without further information, reasoning about whether \nagent{18} is coming to the party might prevent you from taking \nagent{16} at the word.
    However, you have already have conformation from \nagent{18} that they are coming to the party.
  \end{illustration}
\end{note}

\begin{note}
  In this case, \requ{}, but concluded.
\end{note}

\begin{note}
  The logic instance is interesting here.
  Only if option of concluding via syntactic derivation.

  Well, in this case, agent collapses both at once.
\end{note}

\subsubsection{No option}

\begin{note}
  Failures of clause~\ref{idea:requ:pool} of \iRequ{} seem common.

  Indeed, any agent will have some limitations on the proposition-value pairs they have the option of concluding from some pool of premises.

  For non-idealised agents, simple examples involve lack of ability or information.

  For example, I do not think I have the ability to provide syntactic derivations of arbitrary propositional tautologies, even when limited to small number of propositional atoms, and so there is no corresponding contrast to the key conditional of \autoref{illu:sketch:prop-logic}.%
  \footnote{
    \begin{quote}
      The construction is a proof of \(A\) \emph{only if} \(A\) is true given an arbitrary valuation.
    \end{quote}
  }
  Similarly, a skilled ecologist may have a good understanding of tracking, and so have the option of drawing various conclusions from a set of animal footprints, but without the set of footprints available as premises, lack the option to draw any conclusions.

  The second option holds equally for idealised agents that lack omnipotence, at least.
  However, even granting omnipotence clause~\ref{idea:requ:pool} fails for various instances of \(\pvp{\psi}{v'}{\Psi}\).
  For, recall that \requ{1} are from an agent's perspective.
  And, generally speaking, there is no requirement that \(\pv{\psi}{v'}\) follows from \(\Psi\).
  Hence, a for a non-idealised agent \(\pv{\psi}{v'}{\Psi}\) may be a \requ{} of concluding \(\pv{\phi}{v}\) from \(\Phi\), in part because the agent's present epistemic state is not sound.
  And, so long as an idealised agent's present epistemic state is always sound, the agent will not have the option for any unsound \(\pvp{\psi}{v'}{\Psi}\) instance.
\end{note}

\begin{note}
  The following is a simple \illu{0}.

  \begin{illustration}[Testimony as a layperson]
    \label{illu:testimony-layperson}
    An agent is informed that there are exactly five intermediate logics that have the interpolation property.\nolinebreak
  \footnote{Cf.\ \textcite{Maksimova:1977un}}

    The agent does not have the means to query the proof.

    The agent concludes there are exactly five intermediate logics that have the interpolation property.
  \end{illustration}

  The agent does not have the means to query the proof that there are exactly five intermediate logics that have the interpolation property, hence clause~\ref{idea:requ:pool} fails.
\end{note}

\begin{note}
  Note, in cases such as \autoref{illu:testimony-layperson}, clause~\ref{idea:requ:nPsi-nPhi} of \iRequ{} almost trivially fails.
  For, clause~\ref{idea:requ:nPsi-nPhi} is read with respect to the agent's present epistemic state.
  Therefore, if the agent does not have the option of concluding the theorem is true, the agent is bound to fail to conclude the theorem is true prior to reasoning about whether the statement is testimony.
  Hence, it seems straightforward that failure to conclude has no bearing on whether the agent would conclude the statement is testimony.
\end{note}

\subsubsection{No relation}

\begin{note}
  Finally, clause~\ref{idea:requ:nPsi-nPhi} of \iRequ{} may fail to hold.
  In such cases \qzS{} may have a positive answer, because there is no \(\pvp{\psi}{v'}{\Psi}\) such that the agent would not conclude \(\pv{\phi}{v}\) from \(\Phi\) if the agent failed to conclude \(\pv{\psi}{v'}\) from \(\Psi\).

  Generally speaking, \(\pvp{\psi}{v'}{\Psi}\) not being a \requ{} of concluding \(\pv{\phi}{v}\) from \(\Phi\) is this way is fairly uninteresting.
  For, clause~\ref{idea:requ:nPsi-nPhi} will fail to hold when \(\pvp{\psi}{v'}{\Psi}\) and \(\pvp{\phi}{v}{\Phi}\) are unrelated (from the agent's perspective).
  For example, an agent may have the option of concluding \(9^{6} \times 3^{2} = 3^{14}\), but failing to do so would not prevent the agent from concluding their dog would like a walk from the behaviour it is exhibiting.

  Still, we offer a slightly more subtle \illu{0}:
\end{note}

\begin{note}[No entanglement]
  \begin{illustration}[Multiplying Turing Machines]
    \label{illu:turing-add-mult}
    An agent regards following statement as testimony:

    \begin{itemize}
    \item
      If you manage to construct a Turing Machine that performs addition, then you will have no trouble constructing a Turing Machine that performs multiplication.
    \end{itemize}

    The agent then constructs a Turing Machine that performs addition.

    And, given the statement they received as testimony, the agent considers it to be the case that they will have no trouble constructing a Turing Machine that performs multiplication.
  \end{illustration}

  Now, given the testimony, the agent \emph{may} think that they have constructed a Turing Machine that performs addition \emph{only if} they are able to construct a Turing Machine that performs multiplication.
  However, such a conditional does not follow directly from the statement received as testimony.

  There is (without further context) no indication that failure to construct a Turing Machine that performs multiplication would reveal a problem with the Turing Machine that performs addition.
  Strictly, the statement received as testimony only states the agent will have no trouble constructing such a Turing Machine.
  Hence, even if the agent does have trouble constructing such a machine, it is not clear this trouble would indicate a problem with their machine for addition.
  Instead, this may only suggest the content of the statement they received as testimony is not, in fact, true.

  So, while the agent \emph{may} think clause~\ref{idea:requ:nPsi-nPhi} of \iRequ{} is satisfied, the agent may also think clause~\ref{idea:requ:nPsi-nPhi} of \iRequ{} is not satisfied.%
  \footnote{
    \autoref{illu:turing-add-mult} is also interesting with respect to the agent concluding the statement is testimony.

    For, is it the case that constructing a TM that adds and having trouble constructing a TM that multiplies is a \requ{} of concluding the information is testimony?

    I am not sure.

    On one hand, it seems plausible the agent has the option of concluding whether the conjunction holds given their present epistemic state.

    On the other hand, it also seems plausible that the agent may conclude they will not have trouble constructing a TM that multiplies \emph{even if} they were to have trouble to constructing a TM that multiplies on attempting.
    For, the agent may only be interested in concluding the future facing `will' if they have not yet attempted to construct a TM that multiplies.
    %
    % In general, I doubt that \qzS{} has a positive answer in cases of testimony as the constraint a positive answer imposes seems too stringent.
    % However, here I think it is plausible  \qzS{} does have a positive answer.
  }
\end{note}

\begin{note}[Maybe?]
  Would not conclude external world if failed to conclude have hands.

  {
    \color{red}
    Might contrast in an interesting way with something from Wright.
  }
\end{note}

\subsubsection{Same time}

\begin{note}[Importance of at same time]
  Propositional logic.
  These premises allow to conclude two things.
  Then, conclusion that \(\phi \land \psi\) is simultaneously a conclusion that \(\phi\) and that \(\psi\).

  Or, apples in a bag.
  Five.
  Well, could do at least four, three, etc.
  Conclude at the same time.
\end{note}

\begin{note}
  For example, counterexample for some formula of propositional logic.
  Constructed a truth table.
  Identified a line.
  If counterexample, then line makes any tautology of propositional logic true.
  And, do not need to appeal to the line being a counterexample to the relevant formula to do so.
  reason from line to any recognised tautology.
  Conclude, tautology would be true.
\end{note}

\subsubsection{Caution}

\begin{note}
  So, with \citeauthor{Dretske:1970to}, things work out.
  Wouldn't reason to a different conclusion.
  However, this isn't to say that the agent gets knowledge.
  I'm inclined to say the agent does know, but there's no immediate link between \qzS{} and knowledge.

  Maybe also include Harman here, noting the possibility of checking the car.
\end{note}

\begin{note}
  Here, close to \citeauthor{Dretske:1970to}'s zebra case?
\end{note}

\begin{note}
  \begin{illustration}
    Computer, not turning on.
    Broken.
    Check that it's plugged in.
  \end{illustration}

  Well, this is not about reasoning.
\end{note}

\begin{note}
  Consider a familiar situation from \citeauthor{Harman:1986ux}:

  \begin{quote}
    Mary believes, prior to looking in the cupboard, that if she looks in the cupboard, then she will see a box of Cheerios.
    Mary then looks in the cupboard and does not see a box of Cheerios.
    Hence, Mary abandons her belief about what she would see if she looks in the cupboard.\nolinebreak
    \mbox{}\hfill\mbox{(\citeyear[Cf.][Chs.1\&2]{Harman:1986ux})}
  \end{quote}

  What this a \requ{}?
  No.
  Needed to look in the cupboard!

  However, \requ{} may fail to hold.

  So, book.
  Two problems.
  One is difficult.
  Think about other.
  Clear counterexample.
  No longer book.
  Hence, no longer \requ{}.
\end{note}

\paragraph{A difficult case}

\begin{note}[Spot the difference]
  \begin{illustration}[Spot the difference]
    \label{illu:CS:spot-the-diff}
    The agent has been working through a spot-the-difference to pass some time.

    Though the time is not completely passed, the agent examined the two images with what seems sufficient care to claim support that they have found all the differences.
    However, the agent did not keep track of the number of differences.

    The agent announces `I have found all the differences' and their companion responds `All fifteen?'

    \begin{enumerate}[label=\arabic*., ref=(I\ref{illu:CS:spot-the-diff}.\arabic*)]
      \setcounter{enumi}{-1}
    \item
      \label{illu:CS:spot-the-diff:info}
      If I have found all the differences, I have found fifteen differences.
    \end{enumerate}

    The agent then reasons as follows:

    \begin{enumerate}[label=\arabic*., ref=(I\ref{illu:CS:spot-the-diff}.\arabic*), resume]
    \item Exhaustive search.
    \item
      \label{illu:CS:spot-the-diff:all}
      I found all the differences.
    \item
      \label{illu:CS:spot-the-diff:fif}
      So, I have found fifteen differences. \hfill (From \ref{illu:CS:spot-the-diff:info} and \ref{illu:CS:spot-the-diff:all})
    \end{enumerate}
  \end{illustration}

  Before going further, structure of this.

  The agent performed some reasoning, and concluded that they found all the differences.
  However, that reasoning is mentioned but not stated in the \illu{0}.
  Rather, present is distinct instance of reasoning after being provided with information.
  ``If not 15, then problem''.
  Present reasoning appeals to past reasoning, and draws out consequence of this given new information.
  Important: the present reasoning does not consider possibility that the agent did not find all 15 differences.
  Instead, consequence of conclusion of previous instance of reasoning.
  Still, epistemically possible that the agent did not find 15 differences.
\end{note}

\begin{note}
  Information leads to \requ{}.

  Possibility of not fifteen.
  And, not merely that the agent performed the reasoning, but that the reasoning identified all.
  If not fifteen, then not all, so would involve appeal to something that is not the case.

  And, present reasoning does not include reasoning about \requ{}.
\end{note}

\begin{note}
  \color{red}
  Though, this is interesting.
  For, the agent may have found fifteen.
  This, then, helps stress the point that it's not just reasoning to the conclusion.
\end{note}

\section[The Tortoise and Achilles]{A dialogue between the Tortoise and Achilles}
\label{cha:zS:sec:question:illu:carroll}

\begin{note}
  In this section, relate \qzS{} to an interpretation of a dialogue between a Tortoise and Achilles, as told by \textcite{Carroll:1895uj}.

  Two things here.
  First, how \qzS{} relates to the puzzle.
  Second, how this is compatible with some other lessons drawn.

  Start with the key points of \textcite{Carroll:1895uj} and our interpretation.
  Provide \illu{0}.
  Relate to \citeauthor{Boghossian:2008vf}'s suggestion regarding rule primativism.
\end{note}

\begin{note}
  \color{red}
  Adding to all the other interpretations of this paper\dots
\end{note}

\begin{note}
  \begin{quote}
    ``Plenty of blank leaves, I see!'' the Tortoise cheerily remarked.
    ``We shall need them \emph{all}!''
    (Achilles shuddered.)
    ``Now write as I dictate:---

    \begin{enumerate}[label=(\emph{\Alph*}), ref=\emph{\Alph*}]
    \item
      \label{AatT:a}
      Things that are equal to the same are equal to each other.
    \item
      \label{AatT:b}
      The two sides of this Triangle are things that are equal to the same.
    \item
      \label{AatT:c}
      If~\ref{AatT:a} and~\ref{AatT:b} are true,~\ref{AatT:z} must be true.
      \setcounter{enumi}{25}
    \item
      \label{AatT:z}
      The two sides of this Triangle are equal to each other.''
    \end{enumerate}

    ``You should call it~\ref{AatT:d}, not~\ref{AatT:z},'' said Achilles.
    ``It comes \emph{next} to the other three.
    If you accept~\ref{AatT:a} and~\ref{AatT:b} and~\ref{AatT:c}, you \emph{must} accept~\ref{AatT:z}.''

    ``And why \emph{must} I?''

    ``Because it follows \emph{logically} from them.
    If~\ref{AatT:a} and~\ref{AatT:b} and~\ref{AatT:c} are true,~\ref{AatT:z} \emph{must} be true.
    You don't dispute \emph{that}, I imagine?''

    ``If~\ref{AatT:a} and~\ref{AatT:b} and~\ref{AatT:c} are true,~\ref{AatT:z} \emph{must} be true,'' the Tortoise thoughtfully repeated.
    ``That's \emph{another} Hypothetical, isn't it?
    And, if I failed to see its truth, I might accept~\ref{AatT:a} and~\ref{AatT:b} and~\ref{AatT:c}, and \emph{still} not accept~\ref{AatT:z}, mightn't I ?''

    \mbox{}\hfill\(\vdots\)\hfill\mbox{}

    ``Then Logic would take you by the throat, and force you to do it!''
    Achilles triumphantly replied.
    ``Logic would tell you 'You ca'n't help yourself.''%
    \mbox{ }\hfill\mbox{(\citeyear[279--280]{Carroll:1895uj})}
  \end{quote}

  The Tortoise has written down three premises,~\ref{AatT:a},~\ref{AatT:b}, and~\ref{AatT:c}.
  Achilles holds that~\ref{AatT:z} follows from~\ref{AatT:a},~\ref{AatT:b}, and~\ref{AatT:c}.
  The Tortoise observes they have the possibility of refraining to accept~\ref{AatT:z} follows from~\ref{AatT:a},~\ref{AatT:b}, and~\ref{AatT:c}.
  And (initially), the Tortoise does not accept~\ref{AatT:z} follows from~\ref{AatT:a},~\ref{AatT:b}, and~\ref{AatT:c}.
  Achilles requests the Tortoise accepts that~\ref{AatT:z} follows from~\ref{AatT:a},~\ref{AatT:b}, and~\ref{AatT:c}, and the Tortoise complies.
  Specifically, the Tortoise grants:

  \begin{quote}
    \begin{enumerate}[label=(\emph{\Alph*}), ref=\emph{\Alph*}]
      \setcounter{enumi}{3}
    \item
      \label{AatT:d}
      If~\ref{AatT:a} and~\ref{AatT:b} and~\ref{AatT:c} are true,~\ref{AatT:z} must be true.%
      \mbox{ }\hfill\mbox{(\citeyear[279]{Carroll:1895uj})}
    \end{enumerate}
  \end{quote}

  But, does not accept~\ref{AatT:z} follows from~\ref{AatT:a},~\ref{AatT:b},~\ref{AatT:c}, and~\ref{AatT:d}.

  General pattern.
  Pool of premises and conclusion.
  The Tortoise does not accept the conclusion follows from the pool of premises.
  Achilles requests the proposition \emph{that} the conclusion follows from the pool of premises is added to the pool of premises.
  The Tortoise does not accept the conclusion follows from the expanded pool of premises.

  Indeed,~\ref{AatT:c} in the passage is granted because the Tortoise (initially) did not accept~\ref{AatT:z} follows from~\ref{AatT:a} and~\ref{AatT:b}.
  (\citeyear[279]{Carroll:1895uj})
\end{note}

\begin{note}
  Problem of priority.
  \begin{quote}
    My paradox \dots turns on the fact that, in a Hypothetical, the \emph{truth} of the Protasis, the \emph{truth} of the Apodosis, and the \emph{validity of the sequence}, are 3 distinct Propositions.

    \mbox{}\hfill\(\vdots\)\hfill\mbox{}

    Suppose I say ``I grant~\ref{AatT:a} and~\ref{AatT:b} and~\ref{AatT:c}, but I do \emph{not} grant that I am thereby \emph{obliged} to grant~\ref{AatT:z}.''
    Surely, my granting~\ref{AatT:z} must \emph{wait} until I have been made to see the validity of this sequence: i.e.\ in order to grant~\ref{AatT:z}, I must grant~\ref{AatT:a},~\ref{AatT:b},~\ref{AatT:c}, and~\ref{AatT:d}! And so on.%
    \mbox{ }\hfill\mbox{(\citeyear[472]{Carroll:1977wl})}
  \end{quote}

  So, granting that \emph{C} or \emph{D} is true does not amount to granting that it's okay to move form the premises to the conclusion.
  What makes it okay to go from premises to conclusion?

  How distinct?
  \citeauthor{Carroll:1977wl} has priority.
  Valid only if implication is true.

  So, why not say both ways?
  Both at the same time.%
  \footnote{
    Applies to any logic for which the deduction theorem holds.
  }

  Well, rule is general.
  Here, get lots of instances of the rule.
  So, at least one way of understanding the puzzle is in terms of the relation between a general rule and particular instances of the rule.%
  \footnote{
    Or, no distinction between premises and rules.
    See, for example,~\textcite{Smiley:1995wk}.
    \begin{quote}
      Any attempt by Carroll to tackle the question of inference was bound to begin in confusion and end in constipation---all those premises piling up, but no motion.\newline
      \mbox{ }\hfill\mbox{(\citeyear[727]{Smiley:1995wk})}
    \end{quote}
  }

  The Tortoise has not (yet) accepted something of a general form, then no persuasion by requesting the Tortoise to accept specific instances of the specific form.

  Failure of Achilles is to provide the Tortoise with motivation to adopt the general rule.

  In this respect, variation on suggestion --- e.g.\ \textcite[21--22,33]{Thomson:2010tt} and~\textcite[573]{Wisdom:1974uc} ---  that the infinite regress \citeauthor{Carroll:1895uj} noted is a `red herring' and the task of Achilles is to clarify to the Tortoise \emph{that} they are under logically necessity to move from~\ref{AatT:a} and~\ref{AatT:b} to~\ref{AatT:z}.
\end{note}

\begin{note}
  However, equally, all these instances of the rule.
  Follow \textcite{Wisdom:1974uc}, go from the particulars.
  But, on the interpretation pressed here, this does nothing for the general problem.

  May think implicit in \citeauthor{Carroll:1895uj}'s paper, what other resource does Achilles have?
\end{note}


\begin{note}
  So, wrong to think that need to cover each individual case in order to get full.
  However, granting this, doubts about particular cases may block adopting rule.

  This is what \qzS{} picks up on.
\end{note}

\begin{note}
  Simple example, testimony.%
  \footnote{
    In keeping with \citeauthor{Carroll:1895uj}'s interest in \emph{modus ponens}, the similar reasoning may be constructed with apparent counterexamples to \emph{modus ponens}.
    For example, consider \textcite{McGee:1985tz}.
  }

  \begin{illustration}[Dodgson's testimony I]
    An agent reasons as follows:
    \begin{enumerate}[label=\arabic*., ref=(\arabic*)]
    \item
      \label{testimony:state}
      Charles Dodgson has testified to me that Lewis Carroll wrote \emph{Alice in Wonderland}.
    \item
      \label{testimony:result}
      I know Lewis Carroll wrote \emph{Alice in Wonderland}.
    \end{enumerate}
    But does not conclude~\ref{testimony:result} from~\ref{testimony:state}.
  \end{illustration}

  Going from~\ref{testimony:state} to~\ref{testimony:result}, general form:

  \begin{enumerate}[label=\(\gamma\)., ref=(\(\gamma\))]
  \item
    \label{testimony:general}
    If someone has testified to me that that \(\phi\) has value \(v\), then I know \(\phi\) has value \(v\).
  \end{enumerate}

  In other words, \emph{if} the agent were to conclude~\ref{testimony:result} from~\ref{testimony:state}, the agent would be committed to proposition of general form, and in turn to concluding \(\phi\) has value \(v\) from received testimony that \(\phi\) has value \(v\).

  This means the agent applies to other instances of received testimony.
  Though, the agent may fail to conclude \(\phi\) has value \(v\) from received testimony that \(\phi\) has value \(v\).

  Now, does the agent consider it the case that there may be some proposition-value pair such that the agent may fail to conclude?

  At a cursory glance,~\ref{testimony:general} seems about as good as conditional detachment, though I think an agent may consider this to be the case.
  And, the agent may even have a particular \(\pvp{\psi}{v'}{\Psi}\) in mind.

  To give a unrealistic but clear example, consider expanding previous \illu{0}.
  The agent recalls Charles Dodgson said more\dots

  \begin{illustration}[Dodgson's testimony II]
    The agent reasons as follows:
    \begin{enumerate}[label=\arabic*\('\)., ref=(\arabic*\('\))]
    \item
      \label{testimony:v:state}
      Charles Dodgson has testified to me that Lewis Carroll wrote \emph{Alice in Wonderland} and I don't know Lewis Carroll wrote \emph{Alice in Wonderland}.
    \end{enumerate}

    Given~\ref{testimony:general},~\ref{testimony:v:result} may be obtained from~\ref{testimony:v:state}.

    \begin{enumerate}[label=\arabic*\('\)., ref=(\arabic*\('\)), resume]
    \item
      \label{testimony:v:result}
      I know that Lewis Carroll wrote \emph{Alice in Wonderland} and I don't know Lewis Carroll wrote \emph{Alice in Wonderland}.
    \end{enumerate}

    Distribution of knowledge over conjunction.

    \begin{enumerate}[label=\arabic*\('\)., ref=(\arabic*\('\)), resume]
    \item
      \label{testimony:v:result:dist}
      I know that Lewis Carroll wrote \emph{Alice in Wonderland} and I know that I don't know Lewis Carroll wrote \emph{Alice in Wonderland}.
    \end{enumerate}

    Take right conjunct, and factivity of knowledge, get~\ref{testimony:v:right:fact}:

    \begin{enumerate}[label=\arabic*\('\)., ref=(\arabic*\('\)), resume]
    \item
      \label{testimony:v:right:fact}
      I don't know Lewis Carroll wrote \emph{Alice in Wonderland}.
    \end{enumerate}

    Combine left conjunct of~\ref{testimony:v:result:dist} with~\ref{testimony:v:right:fact} to obtain~\ref{testimony:v:bad}:

    \begin{enumerate}[label=\arabic*\('\)., ref=(\arabic*\('\)), resume]
    \item
      \label{testimony:v:bad}
      I know that Lewis Carroll wrote \emph{Alice in Wonderland} and I don't know that Lewis Carroll wrote \emph{Alice in Wonderland}.
    \end{enumerate}
    However, as~\ref{testimony:v:bad} is clearly a contradiction, the agent does not conclude~\ref{testimony:v:bad} from~\ref{testimony:v:state}.
  \end{illustration}

  Hence, reject~\ref{testimony:general}, general inference.
  For, if conclude then also this variant reasoning.

  Of course, various ways around this problem.
  But, without solution in hand, still enough to block acceptance of general.%
  \footnote{
    More interesting case is the surprise exam paradox.
    Arguably Same property of coming to know something after testimony.
    (Cf.~\textcite{Chow:1998vw} and~\textcite{Gerbrandy:2007vm})
  }


  (
  So, this can be re-formed with testimony.
  Get a different conclusion.
  It's possible to conclude that this is a bad conditional.

  Now, this is interesting.
  Because, typically, this has been taking to present a problem of infinite regress.
  Note, the formulation of \qzS{} is silent on this.
  We're not interested in whether there's some reasoning for the conditional, which does seem difficult, but whether there's something that could lead to a problem.
  )

  Alternatively, consider \citeauthor{Harman:1986ux} here, where logic might do other things.
\end{note}

% \begin{note}
%   \color{red}
%   Distinction this is pulling on is syntax versus semantics, roughly.
%   Well, I think this is one way of viewing what's going on._
%   The semantic rule of \emph{modus ponens} might be fine, but it's not clear syntax lines up.
%   Still, I don't think it's worth the added time and complexity to make this point.
% \end{note}

\begin{note}
  So, fails to conclude because unclear about what else would follow.

  Key, agent does not withhold because need something positive prior, but because unsure about what else happens if they do conclude.
\end{note}

\begin{note}
  Important, the agent may also conclude.
  Here, we've got a very cautious agent.
  But, \qzS{} is all about the agent's perspective.

  So, may go for testimony.
  Likewise, may conclude keys are lost even if there is some other place.

  So, three things.

  First, alternative interpretation of \citeauthor{Carroll:1895uj}'s paradox.
  Second, seen how \qzS{} may apply to reasoning involving testimony.
  Third, and most important, no regress.
\end{note}

\subsection{\textcite{Boghossian:2008vf}}

\begin{note}
  Interpretation is different, but related, from \textcite{Boghossian:2008vf}.

  \begin{quote}
    The Carrollian argument, \dots is meant to raise a problem for the \emph{justification} of our rules of inference---How can we justify our belief that Modus Ponens, for example, is a good rule of inference?\newline
    \mbox{ }\hfill\mbox{(\citeyear[493]{Boghossian:2008vf})}
  \end{quote}

  Considered whether, descriptively, the agent may wonder about some problems happening.

  May think related, no worries only if justification.
  However, this is problematic.

  What~\textcite{Boghossian:2008vf} motivates (though does not explicitly endorse) is possibility that any motivation for general rule requires appeal to general rule.

  A kind of primitivism about rule following.
  \begin{quote}
    [W]e would have to take as primitive a \emph{general (often conditional) content serving as the reason for which one believes something}, without this being mediated by inference of any kind.%
    \mbox{ }\hfill\mbox{(\citeyear[500]{Boghossian:2008vf})}
  \end{quote}

  More basic than justification, but also, if primitive then no justification.

  Important is that the interpretation outlined, and the role of \qzS{} is compatible with this primitivism.
  Not saying that the agent needs \abspec{} ability.
  However, am holding that worries about bad cases are sufficient.

  In failure cases, not about what is required for the agent to conclude, but why an agent may fail to conclude.
  Two different cases.
  First, agent may have a general worry, in this case, there's no particular \(\pvp{\psi}{v'}{\Psi}\) at issue, and so no particular \(\pvp{\psi}{v'}{\Psi}\) is part of why the agent does not conclude.
  Second, agent may have a specific worry.
  In this case, there is some particular \(\pvp{\psi}{v'}{\Psi}\).
  Lack of support between \(\pv{\psi}{v'}\) and \(\Psi\) is part of why agent does not conclude.

  Parallel in converse cases.
\end{note}

%%% Local Variables:
%%% mode: latex
%%% TeX-master: "master"
%%% End:
