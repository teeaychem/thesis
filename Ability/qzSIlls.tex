\chapter{\requ{3}: Additional \illu{1}}
\label{cha:zS:sec:question:illu}

\begin{note}
  Some \illu{1}.
  Two broad sections.

  \begin{itemize}
  \item
    The discussion of \citeauthor{Carroll:1895uj}'s tale is involved, and focuses in roughly equal parts on providing further clarification of \requ{1} and providing a novel (at least to my awareness) \emph{interpretation} of \citeauthor{Carroll:1895uj}'s tale.%
    \footnote{
      I doubt very much that the interpretation given is what \citeauthor{Carroll:1895uj} had in mind.
      On the other hand, it's not clear that \citeauthor{Carroll:1895uj} had any particular interpretation in mind (\cite[Cf.][]{Thomson:2010tt}).
    }
  \end{itemize}
\end{note}

% \section{Basic \illu{1}}
% \label{cha:zS:sec:question:illu:basic}

% \subsection{\requ{3} which are not \fc{1}}
% \label{cha:zS:sec:question:illu:basic:does-not-have}

% \subsection{\requ{3} which are \fc{1}}
% \label{cha:zS:sec:question:illu:basic:has}

% \subsubsection{Already}

% \begin{note}
%   Recall \autoref{prop:wit-for-fc} (\autopageref{prop:wit-for-fc}).
%   It may be the case that agent has a \wit{} for a \fc{}.
%   In turn, converse.
%   \wit{} therefore \fc{}.
% \end{note}

% \begin{note}
%   And, more commonplace examples involve the gradual accumulation of proposition-value pairs.

%   \begin{illustration}
%     \nagent{16} \emph{said} they're coming to the party, but you know from \nagent{17} that \nagent{16} is coming to the party only if \nagent{18} is coming to the party.
%     Without further information, reasoning about whether \nagent{18} is coming to the party might prevent you from taking \nagent{16} at the word.
%     However, you have already have conformation from \nagent{18} that they are coming to the party.
%   \end{illustration}
% \end{note}

% \begin{note}
%   In this case, \requ{}, but concluded.
% \end{note}

% \begin{note}
%   The logic instance is interesting here.
%   Only if option of concluding via syntactic derivation.

%   Well, in this case, agent collapses both at once.
% \end{note}

% \subsubsection{No collateral conclusions}

% \begin{note}
%   For non-idealised agents, simple examples involve lack of ability or information.

%   For example, I do not think I have the ability to provide syntactic derivations of arbitrary propositional tautologies, even when limited to small number of propositional atoms, and so there is no corresponding contrast to the key conditional of \autoref{illu:sketch:prop-logic}.%
%   \footnote{
%     \begin{quote}
%       The construction is a proof of \(A\) \emph{only if} \(A\) is true given an arbitrary valuation.
%     \end{quote}
%   }
%   Similarly, a skilled ecologist may have a good understanding of tracking, and so have the option of drawing various conclusions from a set of animal footprints, but without the set of footprints available as premises, lack the option to draw any conclusions.
% \end{note}

% \subsubsection{No relation}

% \begin{note}
%   Finally, the conditional may fail to hold.
%   In such cases \qzS{} may have a positive answer, because there is no \(\pvp{\psi}{v'}{\Psi}\) such that the agent would not conclude \(\pv{\phi}{v}\) from \(\Phi\) if the agent failed to conclude \(\pv{\psi}{v'}\) from \(\Psi\).

%   Generally speaking, \(\pvp{\psi}{v'}{\Psi}\) not being a \requ{} of concluding \(\pv{\phi}{v}\) from \(\Phi\) is this way is fairly uninteresting.
%   For, the conditional will fail to be true when \(\pvp{\psi}{v'}{\Psi}\) and \(\pvp{\phi}{v}{\Phi}\) are unrelated (from the \agpe{}).
%   For example, an agent may have the option of concluding \(9^{6} \times 3^{2} = 3^{14}\), but failing to do so would not prevent the agent from concluding their dog would like a walk from the behaviour it is exhibiting.

%   Still, we offer a slightly more subtle \illu{0}:
% \end{note}

% \begin{note}[No entanglement]
%   \begin{illustration}[Multiplying Turing Machines]
%     \label{illu:turing-add-mult}
%     An agent regards following statement as testimony:

%     \begin{itemize}
%     \item
%       If you manage to construct a Turing Machine that performs addition, then you will have no trouble constructing a Turing Machine that performs multiplication.
%     \end{itemize}

%     The agent then constructs a Turing Machine that performs addition.

%     And, given the statement they received as testimony, the agent considers it to be the case that they will have no trouble constructing a Turing Machine that performs multiplication.
%   \end{illustration}

%   Now, given the testimony, the agent \emph{may} think that they have constructed a Turing Machine that performs addition \emph{only if} they are able to construct a Turing Machine that performs multiplication.
%   However, such a conditional does not follow directly from the statement received as testimony.

%   There is (without further context) no indication that failure to construct a Turing Machine that performs multiplication would reveal a problem with the Turing Machine that performs addition.
%   Strictly, the statement received as testimony only states the agent will have no trouble constructing such a Turing Machine.
%   Hence, even if the agent does have trouble constructing such a machine, it is not clear this trouble would indicate a problem with their machine for addition.
%   Instead, this may only suggest the content of the statement they received as testimony is not, in fact, true.

%   So, while the agent \emph{may} think the conditional is true, the agent may also think the conditional is false.%
%   \footnote{
%     \autoref{illu:turing-add-mult} is also interesting with respect to the agent concluding the statement is testimony.

%     For, is it the case that constructing a TM that adds and having trouble constructing a TM that multiplies is a \requ{} of concluding the information is testimony?

%     I am not sure.

%     On one hand, it seems plausible the agent has the option of concluding whether the conjunction holds given their present epistemic state.

%     On the other hand, it also seems plausible that the agent may conclude they will not have trouble constructing a TM that multiplies \emph{even if} they were to have trouble to constructing a TM that multiplies on attempting.
%     For, the agent may only be interested in concluding the future facing `will' if they have not yet attempted to construct a TM that multiplies.
%     %
%     % In general, I doubt that \qzS{} has a positive answer in cases of testimony as the constraint a positive answer imposes seems too stringent.
%     % However, here I think it is plausible  \qzS{} does have a positive answer.
%   }
% \end{note}

% \begin{note}[Maybe?]
%   Would not conclude external world if failed to conclude have hands.

%   {
%     \color{red}
%     Might contrast in an interesting way with something from Wright.
%   }
% \end{note}

% \subsubsection{Same time}

% \begin{note}
%   Generally think of \requ{} as arising prior to conclusion.
%   However, may arise when the agent concludes.
% \end{note}

% \begin{note}
%   Counterexample for some formula of propositional logic.
%   Constructed a truth table.
%   Identified a line.
%   If counterexample, then line makes any tautology of propositional logic true.
%   And, do not need to appeal to the line being a counterexample to the relevant formula to do so.
%   reason from line to any recognised tautology.
%   Conclude, tautology would be true.
% \end{note}

% \paragraph{Difficult cases}

% \subsubsection{Changes}
% \label{sec:changes}

% \begin{note}[Typos]
%   So, homework.
%   Some problems.
%   Unless have some answer for each of problems, fail.

%   Think about other.
%   Clear counterexample.
%   No confident in other questions.
%   Hence, no longer \requ{}.
%   For, the homework may be invalidated.

%   This one happened.
% \end{note}

\section[The Tortoise and Achilles]{A dialogue between the Tortoise and Achilles}
\label{cha:zS:sec:question:illu:carroll}

\begin{note}
  In this section, relate \requ{1} to an interpretation of a dialogue between a Tortoise and Achilles, as told by \textcite{Carroll:1895uj}.

  Two things here.
  First, how \requ{1} relates to the puzzle.
  Second, how this is compatible with some other lessons drawn.

  Start with the key points of \textcite{Carroll:1895uj} and our interpretation.
  Provide \illu{0}.
  Relate to \citeauthor{Boghossian:2008vf}'s suggestion regarding rule primativism.
\end{note}

\begin{note}
  \color{red}
  Adding to all the other interpretations of this paper\dots
\end{note}


\begin{note}
  Problem of priority.
  \begin{quote}
    My paradox \dots turns on the fact that, in a Hypothetical, the \emph{truth} of the Protasis, the \emph{truth} of the Apodosis, and the \emph{validity of the sequence}, are 3 distinct Propositions.

    \mbox{}\hfill\(\vdots\)\hfill\mbox{}

    Suppose I say ``I grant~\ref{AatT:a} and~\ref{AatT:b} and~\ref{AatT:c}, but I do \emph{not} grant that I am thereby \emph{obliged} to grant~\ref{AatT:z}.''
    Surely, my granting~\ref{AatT:z} must \emph{wait} until I have been made to see the validity of this sequence: i.e.\ in order to grant~\ref{AatT:z}, I must grant~\ref{AatT:a},~\ref{AatT:b},~\ref{AatT:c}, and~\ref{AatT:d}! And so on.%
    \mbox{ }\hfill\mbox{(\citeyear[472]{Carroll:1977wl})}
  \end{quote}

  So, granting that \emph{C} or \emph{D} is true does not amount to granting that it's okay to move form the premises to the conclusion.
  What makes it okay to go from premises to conclusion?

  How distinct?
  \citeauthor{Carroll:1977wl} has priority.
  Valid only if implication is true.

  So, why not say both ways?
  Both at the same time.%
  \footnote{
    Applies to any logic for which the deduction theorem holds.
  }

  Well, rule is general.
  Here, get lots of instances of the rule.
  So, at least one way of understanding the puzzle is in terms of the relation between a general rule and particular instances of the rule.%
  \footnote{
    Or, no distinction between premises and rules.
    See, for example,~\textcite{Smiley:1995wk}.
    \begin{quote}
      Any attempt by Carroll to tackle the question of inference was bound to begin in confusion and end in constipation---all those premises piling up, but no motion.\newline
      \mbox{ }\hfill\mbox{(\citeyear[727]{Smiley:1995wk})}
    \end{quote}
  }

  The Tortoise has not (yet) accepted something of a general form, then no persuasion by requesting the Tortoise to accept specific instances of the specific form.

  Failure of Achilles is to provide the Tortoise with motivation to adopt the general rule.

  In this respect, variation on suggestion --- e.g.\ \textcite[21--22,33]{Thomson:2010tt} and~\textcite[573]{Wisdom:1974uc} ---  that the infinite regress \citeauthor{Carroll:1895uj} noted is a `red herring' and the task of Achilles is to clarify to the Tortoise \emph{that} they are under logically necessity to move from~\ref{AatT:a} and~\ref{AatT:b} to~\ref{AatT:z}.
\end{note}

\begin{note}
  However, equally, all these instances of the rule.
  Follow \textcite{Wisdom:1974uc}, go from the particulars.
  But, on the interpretation pressed here, this does nothing for the general problem.

  May think implicit in \citeauthor{Carroll:1895uj}'s paper, what other resource does Achilles have?
\end{note}


\begin{note}
  So, wrong to think that need to cover each individual case in order to get full.
  However, granting this, doubts about particular cases may block adopting rule.

  This is what \requ{} picks up on.
\end{note}

\begin{note}
  Simple example, testimony.%
  \footnote{
    In keeping with \citeauthor{Carroll:1895uj}'s interest in \emph{modus ponens}, the similar reasoning may be constructed with apparent counterexamples to \emph{modus ponens}.
    For example, consider \textcite{McGee:1985tz}.
  }

  \begin{illustration}[Dodgson's testimony I]
    An agent reasons as follows:
    \begin{enumerate}[label=\arabic*., ref=(\arabic*)]
    \item
      \label{testimony:state}
      Charles Dodgson has testified to me that Lewis Carroll wrote \emph{Alice in Wonderland}.
    \item
      \label{testimony:result}
      I know Lewis Carroll wrote \emph{Alice in Wonderland}.
    \end{enumerate}
    But does not conclude~\ref{testimony:result} from~\ref{testimony:state}.
  \end{illustration}

  Going from~\ref{testimony:state} to~\ref{testimony:result}, general form:

  \begin{enumerate}[label=\(\gamma\)., ref=(\(\gamma\))]
  \item
    \label{testimony:general}
    If someone has testified to me that that \(\phi\) has value \(v\), then I know \(\phi\) has value \(v\).
  \end{enumerate}

  In other words, \emph{if} the agent were to conclude~\ref{testimony:result} from~\ref{testimony:state}, the agent would be committed to proposition of general form, and in turn to concluding \(\phi\) has value \(v\) from received testimony that \(\phi\) has value \(v\).

  This means the agent applies to other instances of received testimony.
  Though, the agent may fail to conclude \(\phi\) has value \(v\) from received testimony that \(\phi\) has value \(v\).

  Now, does the agent consider it the case that there may be some proposition-value pair such that the agent may fail to conclude?

  At a cursory glance,~\ref{testimony:general} seems about as good as conditional detachment, though I think an agent may consider this to be the case.
  And, the agent may even have a particular \(\pvp{\psi}{v'}{\Psi}\) in mind.

  To give a unrealistic but clear example, consider expanding previous \illu{0}.
  The agent recalls Charles Dodgson said more\dots

  \begin{illustration}[Dodgson's testimony II]
    The agent reasons as follows:
    \begin{enumerate}[label=\arabic*\('\)., ref=(\arabic*\('\))]
    \item
      \label{testimony:v:state}
      Charles Dodgson has testified to me that Lewis Carroll wrote \emph{Alice in Wonderland} and I don't know Lewis Carroll wrote \emph{Alice in Wonderland}.
    \end{enumerate}

    Given~\ref{testimony:general},~\ref{testimony:v:result} may be obtained from~\ref{testimony:v:state}.

    \begin{enumerate}[label=\arabic*\('\)., ref=(\arabic*\('\)), resume]
    \item
      \label{testimony:v:result}
      I know that Lewis Carroll wrote \emph{Alice in Wonderland} and I don't know Lewis Carroll wrote \emph{Alice in Wonderland}.
    \end{enumerate}

    Distribution of knowledge over conjunction.

    \begin{enumerate}[label=\arabic*\('\)., ref=(\arabic*\('\)), resume]
    \item
      \label{testimony:v:result:dist}
      I know that Lewis Carroll wrote \emph{Alice in Wonderland} and I know that I don't know Lewis Carroll wrote \emph{Alice in Wonderland}.
    \end{enumerate}

    Take right conjunct, and factivity of knowledge, get~\ref{testimony:v:right:fact}:

    \begin{enumerate}[label=\arabic*\('\)., ref=(\arabic*\('\)), resume]
    \item
      \label{testimony:v:right:fact}
      I don't know Lewis Carroll wrote \emph{Alice in Wonderland}.
    \end{enumerate}

    Combine left conjunct of~\ref{testimony:v:result:dist} with~\ref{testimony:v:right:fact} to obtain~\ref{testimony:v:bad}:

    \begin{enumerate}[label=\arabic*\('\)., ref=(\arabic*\('\)), resume]
    \item
      \label{testimony:v:bad}
      I know that Lewis Carroll wrote \emph{Alice in Wonderland} and I don't know that Lewis Carroll wrote \emph{Alice in Wonderland}.
    \end{enumerate}
    However, as~\ref{testimony:v:bad} is clearly a contradiction, the agent does not conclude~\ref{testimony:v:bad} from~\ref{testimony:v:state}.
  \end{illustration}

  Hence, reject~\ref{testimony:general}, general inference.
  For, if conclude then also this variant reasoning.

  Of course, various ways around this problem.
  But, without solution in hand, still enough to block acceptance of general.%
  \footnote{
    More interesting case is the surprise exam paradox.
    Arguably Same property of coming to know something after testimony.
    (Cf.~\textcite{Chow:1998vw} and~\textcite{Gerbrandy:2007vm})
  }


  (
  So, this can be re-formed with testimony.
  Get a different conclusion.
  It's possible to conclude that this is a bad conditional.

  Now, this is interesting.
  Because, typically, this has been taking to present a problem of infinite regress.
  Note, the formulation of \requ{} is silent on this.
  We're not interested in whether there's some reasoning for the conditional, which does seem difficult, but whether there's something that could lead to a problem.
  )

  Alternatively, consider \citeauthor{Harman:1986ux} here, where logic might do other things.
\end{note}

% \begin{note}
%   \color{red}
%   Distinction this is pulling on is syntax versus semantics, roughly.
%   Well, I think this is one way of viewing what's going on._
%   The semantic rule of \emph{modus ponens} might be fine, but it's not clear syntax lines up.
%   Still, I don't think it's worth the added time and complexity to make this point.
% \end{note}

\begin{note}
  So, fails to conclude because unclear about what else would follow.

  Key, agent does not withhold because need something positive prior, but because unsure about what else happens if they do conclude.
\end{note}

\begin{note}
  Important, the agent may also conclude.
  Here, we've got a very cautious agent.
  But, \requ{} is all about the \agpe{}.

  So, may go for testimony.
  Likewise, may conclude keys are lost even if there is some other place.

  So, three things.

  First, alternative interpretation of \citeauthor{Carroll:1895uj}'s paradox.
  Second, seen how \requ{} may apply to reasoning involving testimony.
  Third, and most important, no regress.
\end{note}

% \subsection{\textcite{Boghossian:2008vf}}

% \begin{note}
%   Interpretation is different, but related, from \textcite{Boghossian:2008vf}.

%   \begin{quote}
%     The Carrollian argument, \dots is meant to raise a problem for the \emph{justification} of our rules of inference---How can we justify our belief that Modus Ponens, for example, is a good rule of inference?\newline
%     \mbox{ }\hfill\mbox{(\citeyear[493]{Boghossian:2008vf})}
%   \end{quote}

%   Considered whether, descriptively, the agent may wonder about some problems happening.

%   May think related, no worries only if justification.
%   However, this is problematic.

%   What~\textcite{Boghossian:2008vf} motivates (though does not explicitly endorse) is possibility that any motivation for general rule requires appeal to general rule.

%   A kind of primitivism about rule following.
%   \begin{quote}
%     [W]e would have to take as primitive a \emph{general (often conditional) content serving as the reason for which one believes something}, without this being mediated by inference of any kind.%
%     \mbox{ }\hfill\mbox{(\citeyear[500]{Boghossian:2008vf})}
%   \end{quote}

%   More basic than justification, but also, if primitive then no justification.

%   Important is that the interpretation outlined, and the role of \requ{} is compatible with this primitivism.
%   Not saying that the agent needs \abspec{} ability.
%   However, am holding that worries about bad cases are sufficient.

%   In failure cases, not about what is required for the agent to conclude, but why an agent may fail to conclude.
%   Two different cases.
%   First, agent may have a general worry, in this case, there's no particular \(\pvp{\psi}{v'}{\Psi}\) at issue, and so no particular \(\pvp{\psi}{v'}{\Psi}\) is part of why the agent does not conclude.
%   Second, agent may have a specific worry.
%   In this case, there is some particular \(\pvp{\psi}{v'}{\Psi}\).
%   Lack of support between \(\pv{\psi}{v'}\) and \(\Psi\) is part of why agent does not conclude.

%   Parallel in converse cases.
% \end{note}

%%% Local Variables:
%%% mode: latex
%%% TeX-master: "master"
%%% TeX-engine: luatex
%%% End:
