\chapter{Counterexamples}
\label{cha:ces}

\begin{note}
  \autoref{cha:clar:sec:literature}, in addition to intuition, constraint seems to often be a theoretical assumption.

  Purpose of variants is to motivate counterexamples to constraint.
  Specifically in terms of answers to \qWhyV{} which are not answers to \qHowV{}.
  In other words, \ros{} such that \ros{} explains, in part, why agent concludes but is such that the agent does not have a \wit{} for the \ros{}.

  In this section we outline in rough form how we will (attempt) to provide counterexamples.

  In short, need:
  An agent, event in which agent concludes \(\pv{\phi}{v}\) from \(\Phi\), and \ros{} between \(\pv{\psi}{v'}\) and \(\Psi\) such that:

  \begin{itemize}
  \item
    The agent does not have a \wit{} for the \ros{} between \(\pv{\psi}{v'}\) and \(\Psi\).
  \item
    The \ros{} between \(\pv{\psi}{v'}\) and \(\Psi\), in part, answers \qWhyV{}.
  \end{itemize}

  Our goal is motivate a general method for generating examples in which some \ros{} for which an agent does not have a \wit{} such that the \ros{} answers \qWhyV{}.
\end{note}

\section{\cScen{1}}
\label{sec:cscen1}

\begin{note}
  \begin{illustration}
    Wason selection task.

    If X then Y.
  \end{illustration}

  This is really nice, if flip over the card, then would be concluding.

  Kind of trivial, huh.

  Of course, don't need to do this to conclude.
  However, expected.
  Indeed, something surprising if it is not a \fc{}.

  No \wit{}, clearly, as you have no seen the other side of the card.

  If any doubts, then clear problem.
\end{note}

\begin{note}
  \begin{illustration}
    Semantics for squish.
  \end{illustration}
\end{note}



\begin{note}[Prior to concluding\dots]
  Not particularly marked.
  Allow agent to have built up a bunch of stuff while reasoning.

  Example.

  \begin{illustration}[Velocity]
    \label{ill:velocity}
    Agent is provided with information about how far a car has travelled north as a function of time travelled.

    From this, take the derivative of the function to obtain the (instantaneous) velocity of the car at a handful of points in time.

    And, from the (instantaneous) velocity of the car, the agent calculates the (instantaneous) acceleration of the car at each of the points in time.

    The agent also has information about the speed of the car as a function of time travelled, and the agent may calculate speed by the taking magnitude of the (instantaneous) velocity of the car.
  \end{illustration}

  \autoref{ill:velocity}, two step calculation.
  Velocity, acceleration.
  After the first step, check by taking the magnitude.
  Calculation of velocity is correct only if taking the magnitude matches speed.

  So, two events for which the agent is concluding.
  Distinct \requ{1} associated with each event.
\end{note}

\section{\sR{2} and \wit{1}}
\label{sec:sr2}

\begin{note}
  Plausible that \wit{} for each component of the type of reasoning!

  But, what to make of this?
\end{note}

\section{Point}
\label{sec:point}

\begin{note}
  Now, the upshot here is perhaps quite minor.

  \issueConstraint{} still does work to see which \ros{} the agent has a \wit{} for.

  The point is more-or-less this.
  \issueConstraint{} does this work, but it does not do any additional work.
\end{note}

%%% Local Variables:
%%% mode: latex
%%% TeX-master: "master"
%%% End:
