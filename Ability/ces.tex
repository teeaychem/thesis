\chapter{Counter-samples}
\label{cha:ces}

\begin{note}
  \autoref{cha:lit}, in addition to intuition, constraint seems to often be a theoretical assumption.

  Purpose of variants is to motivate counterexamples to constraint.
  Specifically in terms of answers to \qWhyV{} which are not answers to \qHowV{}.
  In other words, \ros{} such that \ros{} explains, in part, why agent concludes but is such that the agent does not have a \wit{} for the \ros{}.

  In this section we outline in rough form how we will (attempt) to provide counterexamples.

  In short, need:
  An agent, event in which agent concludes \(\pv{\phi}{v}\) from \(\Phi\), and \ros{} between \(\pv{\psi}{v'}\) and \(\Psi\) such that:

  \begin{itemize}
  \item
    The agent does not have a \wit{} for the \ros{} between \(\pv{\psi}{v'}\) and \(\Psi\).
  \item
    The \ros{} between \(\pv{\psi}{v'}\) and \(\Psi\), in part, answers \qWhyV{}.
  \end{itemize}

  Our goal is motivate a general method for generating examples in which some \ros{} for which an agent does not have a \wit{} such that the \ros{} answers \qWhyV{}.
\end{note}

\section{Nonogram puzzles}
\label{sec:nonogram-puzzles}

\url{https://blog-imgs-144.fc2.com/o/t/o/otokojima/logic_228.htm}

\section{Sudoku puzzles}

\begin{note}
  \begin{illustration}[Sudoku]
    \label{illu:gist:sudoku}
    % https://tex.stackexchange.com/questions/91422/tikz-sudoku-circle-and-connect-with-lines-some-cells
    Consider the following Sudoku puzzle:%
    \footnote{
      From~\textcite[84]{Coussement:2007up}.
    }
    \vspace{\baselineskip}

    \mbox{ }\hfill%
    \begin{adjustbox}{minipage=0.45\linewidth,scale=1}
      \centering
      \begin{tikzpicture}[scale=.5]
        \begin{scope}
          \draw (0, 0) grid (9, 9);
          \draw[very thick, scale=3] (0, 0) grid (3, 3);
          \setcounter{row}{1}
          % Single entries
          \setrow { }{ }{ }  { }{ }{ }  {1}{ }{ }
          \setrow { }{ }{ }  { }{ }{ }  { }{5}{ }
          \setrow {9}{ }{ }  { }{ }{ }  { }{ }{2}
          \setrow { }{ }{3}  { }{2}{ }  { }{ }{ }
          \setrow { }{ }{ }  {8}{ }{ }  {4}{6}{5}
          \setrow { }{4}{ }  { }{5}{9}  { }{ }{8}
          \setrow { }{8}{7}  {2}{3}{1}  { }{4}{6}
          \setrow {2}{1}{ }  {5}{ }{ }  { }{ }{3}
          \setrow {3}{ }{6}  {4}{ }{8}  { }{ }{ }
        \end{scope}
      \end{tikzpicture}
    \end{adjustbox}%
    \hfill\mbox{ }\newline
    \mbox{ }
  \end{illustration}

  Interactive.
  Fill in the grid.
  Difference between filling in the grid and concluding that solution to the puzzle.
  So, before conclude for any particular square, or for the grid as a whole.
  Is it the case that you would fill in the grid the same way?

  Key intuition, stop before committing to any mistake.

  Two aspects.

  First, may give up completely.
  Second, catch any mistakes and fix before moving on.
\end{note}

\begin{note}
  \autoref{illu:gist:sudoku} parallels \autoref{scen:squish}.

  In both \illu{1}, \(\pv{\phi}{v}\) follows from \(\Phi\) via a rules.

  However, rules are not of direct interest.
  \autoref{scen:squish} is a syntactic proof, but variant \scen{0} in which semantic proof.
  Relevant reasoning may be rule governed, but semantic proofs are not constrained.

  Rather, familiarity.

  The type of reasoning is general.
  Syntactic and semantic proofs, Sudoku puzzles, simple instances of chess problems, all seem to involve general reasoning.
  Likewise, counting, adding, subtracting, and so on.
  Competence established through various proofs, puzzles, problems, and practice.

  In this respect, there is no reasonable doubt of \tR{}.
  At issue is whether specific performance.
\end{note}

\begin{note}
  Same as \autoref{prop:requ-not-n-ce} on \autopageref{prop:requ-not-n-ce}.
  Attention only wrt.\ to reasoning.
  Hence, \wit{}.
\end{note}

\begin{note}
  Intuition for each of the points.

  \begin{itemize}
  \item
    \fc{1}, understand how to solve Sudoku puzzles, repeat any instances.
  \item
    Catch mistakes.
  \end{itemize}
\end{note}

\section{\scen{3}}
\label{sec:cscen1}


\begin{note}
  \begin{illustration}
    Semantics for squish.
  \end{illustration}
\end{note}



\begin{note}[Prior to concluding\dots]
  Not particularly marked.
  Allow agent to have built up a bunch of stuff while reasoning.

  Example.

  \begin{illustration}[Velocity]
    \label{ill:velocity}
    Agent is provided with information about how far a car has travelled north as a function of time travelled.

    From this, take the derivative of the function to obtain the (instantaneous) velocity of the car at a handful of points in time.

    And, from the (instantaneous) velocity of the car, the agent calculates the (instantaneous) acceleration of the car at each of the points in time.

    The agent also has information about the speed of the car as a function of time travelled, and the agent may calculate speed by the taking magnitude of the (instantaneous) velocity of the car.
  \end{illustration}

  \autoref{ill:velocity}, two step calculation.
  Velocity, acceleration.
  After the first step, check by taking the magnitude.
  Calculation of velocity is correct only if taking the magnitude matches speed.

  So, two events for which the agent is concluding.
  Distinct \requ{1} associated with each event.
\end{note}

\section{\tR{2} and \wit{1}}
\label{sec:sr2}

\begin{note}
  Plausible that \wit{} for each component of the type of reasoning!

  But, what to make of this?
\end{note}

\section{Point}
\label{sec:point}

\begin{note}
  Now, the upshot here is perhaps quite minor.

  \issueConstraint{} still does work to see which \ros{} the agent has a \wit{} for.

  The point is more-or-less this.
  \issueConstraint{} does this work, but it does not do any additional work.
\end{note}

%%% Local Variables:
%%% mode: latex
%%% TeX-master: "master"
%%% End:
