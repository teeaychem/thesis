\chapter{Counterexamples to \issueInclusion{}}
\label{cha:ces}


\begin{note}
  Here, should have the things used to capture \tCV{}.
  So, the arithmetic example.
  With Wason, specific cards.
  Of course, when get complete, \wit{}.
  But, for each card, there's a conclusion.
  This is sufficient.

  And, translation.
  Various translations that agent does not perform.
  In particular, when there are multiple possible translations.
\end{note}


\section{Counterexamples}
\label{sec:counterexamples}


\begin{note}
  \begin{proposition}[\tCV{3} and failures of \issueInclusion{}]
    \label{prop:tCV-WhyV-ces}
    \vspace{-\baselineskip}
    \begin{itenum}
    \item[\emph{If}:]
      Conditions \ref{prop:hinge:X:tCed}, \ref{prop:hinge:X:e:act:se}, \ref{prop:hinge:X:tC}, \ref{prop:hinge:X:e:pro} and \ref{prop:hinge:X:e:pro:fc} hold:
      \begin{enumerate}[label=\arabic*., ref=\arabic*]
      \item
        \label{prop:hinge:X:tCed}
        \(\ed{}\) is an event in which \vAgent{} (\typeAdv{}) concludes \(\pv{\phi}{v}\) from \(\Phi\) (by \torNa{} \(T\)).
      \item
        \label{prop:hinge:X:e:act:se}
        \(\ed{\flat}\) is a \se{} of \(\ed{}\).
      \item
        \label{prop:hinge:X:tC}
        \(\ed{\flat}\) is an event in which \vAgent{} is \tCV{} \(\pv{\phi}{v}\) from \(\Phi\) by \torNa{} \(T\).
      \item
        \label{prop:hinge:X:e:pro}
        \(\tproS{}\) is a projection of type \(T\) with respect to \(\ed{}\) \vAgent{}.
      \item
        \label{prop:hinge:X:e:pro:fc}
        For some event \(\edn{}\) and \prop{0}-\val{0}-\pool{0} pair \(\pv{\psi}{v'}\) and \(\Psi\) in \(\tproS{}\):
        \begin{enumerate}[label=\alph*., ref=\theenumi\alph*]
        \item
          \label{prop:hinge:X:e:pro:fc:i}
          \(\edn{\sharp}\) is the result of an action \(a\) done by \vAgent{} in \(\ed{\flat}\).
        \item
          \label{prop:hinge:X:e:pro:fc:ii}
          \vAgent{} \evals{} \(\psi'\) as having value \(v''\) prior to doing \(a\), for each \prop{0}-\val{0} pair \(\pv{\psi'}{v''}\) in \(\Psi\).
        \item
          \label{prop:hinge:X:e:pro:fc:iii}
          \vAgent{} does not have a \wit{} for a \ros{} between \(\pv{\psi}{v'}\) and \(\Psi\) when \vAgent{} concludes \(\pv{\phi}{v}\) from \(\Phi\).
        \end{enumerate}
      \end{enumerate}
    \item[\emph{Then}:]
      \issueInclusion{} does not hold.
    \end{itenum}
    \vspace{-\baselineskip}
  \end{proposition}

  \noindent%
  \autoref{prop:tCV-WhyV-ces} is a variation on \autoref{prop:requ-WhyV-ces} (\autopageref{prop:requ-WhyV-ces}), where the condition of an agent \tCV{} helps identify \fc{1}, and in turn \requ{1} sufficient to show \issueInclusion{} does not hold.

    \begin{argument}{prop:tCV-WhyV-ces}
    Suppose \ref{prop:hinge:X:tCed}, \ref{prop:hinge:X:e:act:se}, \ref{prop:hinge:X:tC}, \ref{prop:hinge:X:e:pro} and \ref{prop:hinge:X:e:pro:fc} hold.
    \medskip

    \noindent%
    First, observe clauses~\ref{def:requ:se} and~\ref{def:requ:fc} of the definition of a \requ{} (\autoref{def:requ}, \autopageref{def:requ}) are satisfied with respect to \(\pv{\psi}{v'}\), \(\Psi\), and \(\ed{\flat}\):

    \begin{itemize}
    \item
      Clause~\ref{def:requ:se} is satisfied by assumption.
      For, by Clause~\ref{prop:hinge:X:e:act:se} \(\ed{\flat}\) is a \se{} of \(\ed{}\).
    \item
      Clause~\ref{def:requ:fc} is satisfied by observing conditions \ref{prop:hinge:X:tC}, \ref{prop:hinge:X:e:pro}, and \ref{prop:hinge:X:e:pro:fc} are the relevant conditions of \autoref{prop:tCV-fc} (\autopageref{prop:tCV-fc}).
      Hence, \(\pv{\psi}{v'}\) is a \fc{0} from \(\Psi\) for the agent through \(\ed{\flat}\).
    \end{itemize}
    %
    So, \(\pv{\psi}{v'}\) being a \fc{} from \(\Psi\) through \(\ed{\flat}\) is a \requ{} of \(\ed{}\) by \autoref{def:requ}.
    \medskip

    \noindent%
    Now, as \(\pv{\psi}{v'}\) being a \fc{} from \(\Psi\) through \(\ed{\flat}\) is a \requ{} of \(\ed{}\), a \ros{} between \(\pv{\psi}{v'}\) and \(\Psi\) holds through \(\ed{\flat}\) by \supportII{} (\autopageref{idea:support:possible}).
    And, in turn, by \autoref{prop:requ-WhyV} (\autopageref{prop:requ-WhyV}), the \ros{} between \(\pv{\psi}{v'}\) and \(\Psi\) answers \qWhy{}.

    So, if \issueInclusion{} holds, there is some past or present event with respect to \(\ed{}\) in which the agent concludes \(\pv{\psi}{v'}\) from \(\Psi\).
    However, by Condition~\ref{prop:hinge:X:e:pro:fc:iii} there is no such event.%
    \footnote{
      Note, the argument for \autoref{prop:tCV-WhyV-ces} only assumes \(\ed{}\) is an event in which \vAgent{} concludes \(\pv{\phi}{v}\) from \(\Phi\).
      The role of \tCV{} is limited to the event \(\edn{\flat}\) in which the conclusion of \(\edn{}\) is in progress.
      Still, as \(\ed{\flat}\) is an event in which the agent is \tCV{} \(\pv{\phi}{v}\) from \(\Phi\) by \torNa{} \(T\), and \(\ed{}\) happens as a result of \(\ed{\flat}\), it is natural for \(\ed{}\) to be an event in which \vAgent{} \typeAdv{} concludes \(\pv{\phi}{v}\) from \(\Phi\) by \torNa{} \(T\).
    }
  \end{argument}
\end{note}


\section{Counter-samples}
\label{sec:counter-samples}


\begin{note}
  With \autoref{prop:tCV-WhyV-ces} in hand, our final task is to show there are cases in which Conditions \ref{prop:hinge:X:tCed}, \ref{prop:hinge:X:e:act:se}, \ref{prop:hinge:X:tC}, \ref{prop:hinge:X:e:pro} and \ref{prop:hinge:X:e:pro:fc} hold.
\end{note}


\begin{note}
  A few counterexamples.

  The first counterexample is fully worked out and the following counterexamples highlight key features.
\end{note}



\subsection{\autoref{illu:gist:roots:F}}


\begin{note}
  The key features of a counterexample has been developed with respect to \autoref{illu:gist:roots:F} throughout this document.

  \begin{application}[\autoref{illu:gist:roots:F} and the failure of \issueInclusion{}]%
    \label{app:sc1-ce}%
    There is a plausible reading of \autoref{illu:gist:roots:F} such that \issueInclusion{} does not hold.
  \end{application}

  \begin{dets}{app:sc1-ce}
    Our goal is to show conditions \ref{prop:hinge:X:tCed}, \ref{prop:hinge:X:e:act:se}, \ref{prop:hinge:X:tC}, \ref{prop:hinge:X:e:pro} and \ref{prop:hinge:X:e:pro:fc} of \autoref{prop:tCV-WhyV-ces} are satisfied under a plausible reading of \autoref{illu:gist:roots:F}.

    The event and descriptions of interest are familiar from previous applications:
    %
    \begin{itemize}
    \item
      \(\edn{}\) is the event described by \autoref{illu:gist:roots:F}.
    \item
      \(\edo{}\) is the description:
      `The agent concludes \pv{\propM{\rootsCon{}}}{\valI{True}} from \(\Phi\)'.
    \item
      \(\edn{\flat}\) covers Step~\ref{illu:gist:roots:F:factor} of the \agents{} reasoning in \autoref{illu:gist:roots:F}.
    \item
      \(\edo{\flat}\) is the description:
      `The agent figures out \rootsConEqFac{} with the aim to identify the factors of \rootsConEq{}'.
    \end{itemize}
    %
    Where, as before, \(\Phi\) captures the \agents{} understanding of factorisation prior to \(\edn{}\).

    We now highlight the way conditions \ref{prop:hinge:X:tCed}, \ref{prop:hinge:X:e:act:se}, \ref{prop:hinge:X:tC} are satisfied in turn:

    \begin{enumerate}
    \item
      \(\edo{}\) is true of \(\ed{}\).
      Hence, \(\ed{}\) is an event in which the agent concludes \pv{\propM{\rootsCon{}}}{\valI{True}} from \(\Phi\).%
    \footnote{
      Further, \(\ed{}\) satisfies to optional typical component of Condition~\ref{prop:hinge:X:tCed}, where the relevant \torNa{} is factorisation by reasoning similar to that found in \autoref{prop:p-t} (\autopageref{prop:p-t}).

      In short, the only way for an agent to concludes \pv{\propM{\rootsCon{}}}{\valI{True}} from a \pool{} which captures the \agents{} understanding of factorisation is for the agent to factor.
    }
  \item
    By \autoref{obs:se-inst} (\autopageref{obs:se-inst}) --- which argues \(\ed{\flat}\) is a \se{} of \(\ed{}\).
  \item
    By a mix of \autoref{obs:se-inst} and \autoref{prop:p-t} (\autopageref{prop:p-t}).

    For, by \autoref{obs:se-inst}, \(\ed{\flat}\) is a \se{} of \(\ed{}\).
    Hence, by Clause~\ref{assu:p:se:prog} of the definition of a \se{} (\autoref{def:se}, \autopageref{def:se}), \(\ed{\flat}\) is such that \(\ed{}\) is in progress.
    In other words, \(\ed{\flat}\) is an event in which the agent is concluding \pv{\propM{\rootsCon{}}}{\valI{True}} from \(\Phi\).
    And, as \(\Phi\) captures the \agents{} understanding of factorisation, \autoref{prop:p-t} entails \(\edo{\flat}\) is an event in which the agent is \tCV{} \pv{\propM{\rootsCon{}}}{\valI{True}} from \(\Phi\) by factorisation.
  \end{enumerate}

  Conditions \ref{prop:hinge:X:e:pro} and \ref{prop:hinge:X:e:pro:fc} concern some \tpro{} \(\mathbb{P}\).
  The \tpro{} of interest is given by \autoref{app:sc1-typ} (\autopageref{app:sc1-typ}) which argued, there is a plausible reading of \autoref{illu:gist:roots:F} such that:
  %
  \begin{quote}
    There is a \tpro{} \(\mathbb{P}\) which consists of:
    \begin{itemize}
    \item
      \prop{2}-\val{0}-\pool{0} pairings, for each \(n\) in \(\rootsConAltSet{}\), such that:
      \begin{itemize}
      \item
        The \pool{} contains an equation of the form \rootsConEqGen{}.
      \item
        The \prop{0}-\val{0} pair is of the form \pv{\propI{\rootsConGen{}}}{\valI{True}}, where \(m \times (m - 1) = n\).
      \end{itemize}
    \item
      Events \(\edn{\ast}\) such that:
      \(\edn{\ast}\) is the outcome of an action done by \vAgent{} in \(\edn{\flat}\).
    \end{itemize}
  \end{quote}
  %
  The \tpro{} \(\mathbb{N}\) satisfies Condition \ref{prop:hinge:X:e:pro}.
  And, by inspection, each \prop{0}-\val{0}-\pool{0} and event pairing in the \tpro{} \(\mathbb{N}\) satisfy sub-conditions \ref{prop:hinge:X:e:pro:fc:i} and \ref{prop:hinge:X:e:pro:fc:ii} of Condition~\ref{prop:hinge:X:e:pro:fc}.

  So, the only task which remains is to argue it is plausible the agent does not have a \wit{} for a \ros{} between some \prop{0}-\val{0}-\pool{0} pairing in \(\mathbb{P}\).
  I take it to be plausible the agent is concluding \pv{\propM{\rootsCon{}}}{\valI{True}} from \(\Phi\) though the agent has never concluded, e.g.\ \rootsConGenV{5}{4} from a \pool{} which contains \rootsConEqV{20}.%
  \footnote{
    I'm fairly confident I did this in the course of developing this document.
    Though, as I do not have recollection of every instance of factorisation I've done, I'm not certain.
  }%
  \newline
  \end{dets}
\end{note}


\begin{note}
  With \autoref{app:sc1-ce} the argument of this document is complete.
  Plausibly, \issueInclusion{} does not hold.

  Still, a genuine counterexample to \issueInclusion{} requires an actual event\dots
\end{note}

\begin{note}
  \begin{scenario}[Sudoku puzzles]%
    \label{illu:gist:sudoku}%
    % https://tex.stackexchange.com/questions/91422/tikz-sudoku-circle-and-connect-with-lines-some-cells
    An agent (you) starts works through the Sudoku puzzle on the left (and does not work through the Sudoku puzzle on the right).
    \medskip

    \mbox{ }\hfill%
    \begin{adjustbox}{minipage=0.45\linewidth,scale=1}
      \centering
      \begin{tikzpicture}[scale=.5]
        \begin{scope}
          \draw (0, 0) grid (9, 9);
          \draw[very thick, scale=3] (0, 0) grid (3, 3);
          \setcounter{row}{1}
          % Single entries
          \setrow { }{ }{ }  { }{ }{ }  { }{ }{ }
          \setrow {5}{ }{ }  { }{ }{ }  {3}{2}{ }
          \setrow { }{ }{ }  {4}{7}{9}  {6}{8}{ }
          \setrow {7}{ }{3}  { }{ }{4}  { }{ }{ }
          \setrow {4}{ }{5}  { }{ }{2}  { }{6}{9}
          \setrow {9}{2}{8}  {7}{5}{6}  {4}{ }{1}
          \setrow {3}{5}{ }  { }{2}{ }  { }{9}{ }
          \setrow { }{ }{1}  { }{9}{3}  {5}{ }{ }
          \setrow { }{ }{ }  { }{ }{8}  {1}{7}{ }
        \end{scope}
      \end{tikzpicture}
    \end{adjustbox}%
    \begin{adjustbox}{minipage=0.45\linewidth,scale=1}
      \centering
      \begin{tikzpicture}[scale=.5]
        \begin{scope}
          \draw (0, 0) grid (9, 9);
          \draw[very thick, scale=3] (0, 0) grid (3, 3);
          \setcounter{row}{1}
          % Single entries
          \setrow {4}{6}{3}  {7}{ }{8}  { }{ }{ }
          \setrow {2}{1}{ }  { }{ }{ }  { }{ }{7}
          \setrow { }{7}{5}  {4}{ }{ }  { }{3}{ }
          \setrow { }{2}{ }  {8}{7}{ }  { }{ }{3}
          \setrow {3}{9}{1}  { }{ }{2}  {5}{ }{8}
          \setrow { }{ }{6}  {1}{3}{ }  { }{ }{2}
          \setrow { }{3}{8}  { }{1}{6}  {7}{5}{4}
          \setrow { }{ }{ }  {9}{ }{ }  { }{ }{1}
          \setrow { }{ }{ }  { }{ }{ }  {2}{ }{ }
        \end{scope}
      \end{tikzpicture}
    \end{adjustbox}%
    \hfill\mbox{ }\newline
    \mbox{ }\newline
    \mbox{ }\hfill(\cite[54,56]{Coussement:2007up})\newline
  \end{scenario}
\end{note}


\begin{note}
  So long as you understand the way Sudoku puzzles work, I think it is clear you are concluding a solution to the left Sudoku puzzle.
\end{note}


\begin{note}
  \begin{itemize}
  \item
    \(\ed{}\) event in which you complete the left Sudoku puzzle.
  \item
    \(\ed{\flat}\) figure out the very centre cell is 8.
  \item
    \se{}.
    Event in progress.
    Result of.

    In progress, for sure.
    Result of, well you need to figure out each cell.
  \end{itemize}

  Further, \tCV{}.

  Now, \tpro{}.

  Plausible \tpro{} is empty cells and corresponding values for the Sudoku puzzle on the right.
  And, indeed, other Sudoku puzzles in \citetitle{Coussement:2007up}, of the same difficulty.
  However, as our interest is obtaining a counterexample to \issueConstraint{}, single \prop{0}-\val{0} pair.

  Figure out the very centre cell is 4.

  Event, while you were concluding.
  Nothing prevented you ignoring my request.
\end{note}

\begin{note}
  Explains why, as favoured completing.
  If not \fc{}, then not clear you were concluding very centre cell is 8.
\end{note}


\begin{note}
  If you don't, then random numbers.
  Sudoku soup.
  But, the same thing which favours entails solution to the second puzzle.
  Granting that you're willing to attempt the second puzzle it follows that if you'd make Sudoku soup of the second puzzle, there is nothing which favours the event in which you solve the chosen Sudoku puzzle.
\end{note}


\begin{note}
  Key to the application is descriptions, and what these descriptions entail.

  Two key parts.
  First, \se{}.
  Recall details of \autoref{obs:se-inst} and in particular \autoref{prop:se-d-lim}.

  \begin{quote}
    \begin{itenum}
    \item[\emph{If}:]
      \(\ed{}\) partly happens as a result of \(\ed{\flat}\).
    \item[\emph{Then}:]
      \(\edo{\flat}\) does not include features of \(\edn{\flat}\) that \(\ed{}\) does not partly happen as a result of.
    \end{itenum}
  \end{quote}

  Second, typical.
  From description alone.
\end{note}


\subsection{Previous \scen{1}}
\label{sec:previous-scen1}


\begin{note}
  Various other \scen{1}.
  The quadratic formula, Fibonnaci and Lucas numbers.
  Vary the prices.
  Chess puzzles.
  Agent doing well at selection tasks. (reference here to domain specific reasoning.)
  Completing textbook exercises.
\end{note}



\section*{Summary}
\label{sec:summary-1}

\begin{note}
  Provided a counterexample to \issueInclusion{}.

  A little more specifically, highlighted the way results of previous chapters combine to form a detailed, plausible, counterexample to \issueConstraint{} and then sketched the way you provided a real counterexample to \issueConstraint{} (so long as you played along).
\end{note}




\section[Outside the scope]{Outside the scope \hfill (Optional)}
\label{sec:outside-scope}


\begin{note}
  Kind of interesting that Searle goes for a lookup table.
  Motivation here is semantic content, rather than syntactic association.
\end{note}


\begin{note}
  \begin{scenario}[ジョジョリオン]%
    \nocite{huangmufeiluyan:2011aa}%
    Book.
    List of chapters.
    Translating the chapter titles.

    \begin{center}
      \bgroup
      \def\arraystretch{1.125}
      \begin{tabular}{R{.45\textwidth} L{.45\textwidth}}
        \hline\hline
        Translation & Title \\
        \hline
        Soft and wet & ソフト&ウェット \\
        Safety first & 無事が何より \\
        Every day is summer vacation & 毎日が夏休み \\
        A hair clip from ??? period & 清の時代の髪留め \\
      \end{tabular}
      \egroup
    \end{center}
  \end{scenario}

  \noindent%
  Translation.
  In order to translate, sufficient understanding of two languages.

  Language, key feature is novel sentences.
  Much like previous scenario, chapter titles, so further chapter titles.
  
  Still, in contrast to scenario,
  
  Partial translations.
  Variety of translations.

  \begin{center}
    \bgroup
    \def\arraystretch{1.125}
    \begin{tabular}{R{.45\textwidth} L{.45\textwidth}}
      \hline\hline
      Translation & Title \\
      \hline
      Safety above everything else & \multirow{3}*{無事が何より} \\
      Safety first & \\
      Gotta be safe & \\
    \end{tabular}
    \egroup
  \end{center}

  With mathematical exercises, may expect agent to re-express solution.
  For example, \(200\), \(2^{3} \times 5^{2}\).
  However, distinct from finding solution.
  No clear problem if only solve with \(200\) and then re-express.
  
  With translation, re-expression after translation seems problematic.
  Suggests syntactic correspondence, but no clear grasp of semantic correspondence.
  
  Indeed, in the case of translation, absence of \fc{} (under standard circumstances) is plausibly important.
  For example, 「ソフト&ウェット」 is translated as 'Soft and Wet'.
  The background context is a number of other The Artist Formerly Known As Prince references (e.g.\ ペイズリー・パーク/Paisley Park).
  However, in a different context 「ソフト&ウェット」 may be translated as 'software and wetware'.
\end{note}


\begin{note}
  There are two issue here.

  First, \fc{1}.
  It's not clear that any particular translation is a \fc{}.
  Need something which ensures translation is in progress.
  However, multiple translations.
  It is not clear whether any particular is in progress.

  Second, \tpro{}.
  In may cases it is not clear any particular translation is required.
\end{note}


%%% Local Variables:
%%% mode: latex
%%% TeX-master: "master"
%%% TeX-engine: luatex
%%% End:
