\chapter{Counterexamples to \issueInclusion{}}
\label{cha:ces}


\begin{note}
  Here, should have the things used to capture \tCV{}.
  So, the arithmetic example.
  With Wason, specific cards.
  Of course, when get complete, \wit{}.
  But, for each card, there's a conclusion.
  This is sufficient.

  And, translation.
  Various translations that agent does not perform.
  In particular, when there are multiple possible translations.
\end{note}




\section{Counterexamples}
\label{sec:counterexamples}


\begin{note}
  \begin{proposition}[\tCV{3} and \issueInclusion{}]
    \label{prop:tCV-WhyV-ces}
    \vspace{-\baselineskip}
    \begin{itenum}
    \item[\emph{If}:]
      Conditions \ref{prop:hinge:X:tCed}, \ref{prop:hinge:X:e:act:se}, \ref{prop:hinge:X:tC}, \ref{prop:hinge:X:e:pro} and \ref{prop:hinge:X:e:pro:fc} hold:
      \begin{enumerate}[label=\arabic*., ref=\arabic*]
      \item
        \label{prop:hinge:X:tCed}
        \(\ed{}\) is an event in which \vAgent{} is \tCs{} \(\pv{\phi}{v}\) from \(\Phi\) by \torNa{} \(T\).
      \item
        \label{prop:hinge:X:e:act:se}
        \(\ed{\flat}\) is a \se{} of \(\ed{}\).
      \item
        \label{prop:hinge:X:tC}
        \(\ed{\flat}\) is an event in which \vAgent{} is \tCV{} \(\pv{\phi}{v}\) from \(\Phi\) by \torNa{} \(T\).
      \item
        \label{prop:hinge:X:e:pro}
        \(\tproS{}\) is a projection of a type \(T\) with respect to \(\ed{}\) \vAgent{}.
      \item
        \label{prop:hinge:X:e:pro:fc}
        For some event \(\edn{}\) and \prop{0}-\val{0}-\pool{0} pair \(\pv{\psi}{v'}\) and \(\Psi\) in \(\tproS{}\):
        \begin{enumerate}[label=\alph*., ref=\theenumi\alph*]
        \item
          \label{prop:hinge:X:e:pro:fc:i}
          \(\edn{\sharp}\) is the result of an action \(a\) done by \vAgent{} in \(\ed{\flat}\).
        \item
          \label{prop:hinge:X:e:pro:fc:ii}
          \vAgent{} \evals{} \(\psi'\) as having value \(v''\) prior to doing \(a\), for each \prop{0}-\val{0} pair \(\pv{\psi'}{v''}\) in \(\Psi\).
        \item
          \label{prop:hinge:X:e:pro:fc:iii}
          \vAgent{} does not have a \wit{} for a \ros{} between \(\pv{\psi}{v'}\) and \(\Psi\) when \vAgent{} concludes \(\pv{\phi}{v}\) from \(\Phi\).
        \end{enumerate}
      \end{enumerate}
    \item[\emph{Then}:]
      \issueInclusion{} does not hold.
    \end{itenum}
    \vspace{-\baselineskip}
  \end{proposition}

  \begin{argument}{prop:tCV-WhyV-ces}
    Suppose \ref{prop:hinge:X:tCed}, \ref{prop:hinge:X:e:act:se}, \ref{prop:hinge:X:tC}, \ref{prop:hinge:X:e:pro} and \ref{prop:hinge:X:e:pro:fc} hold.
    \medskip

    \noindent%
    First, observe clauses~\ref{def:requ:se} and~\ref{def:requ:fc} of the definition of a \requ{} (\autoref{def:requ}, \autopageref{def:requ}) are satisfied with respect to \(\pv{\psi}{v'}\), \(\Psi\), and \(\ed{\flat}\):

    \begin{itemize}
    \item
      Clause~\ref{def:requ:se} is satisfied by assumption.
      For, by Clause~\ref{prop:hinge:X:e:act:se} \(\ed{\flat}\) is a \se{} of \(\ed{}\).
    \item
      Clause~\ref{def:requ:fc} is satisfied by observing conditions \ref{prop:hinge:X:tC}, \ref{prop:hinge:X:e:pro}, and \ref{prop:hinge:X:e:pro:fc} are the relevant conditions of \autoref{prop:tCV-fc} (\autopageref{prop:tCV-fc}).
      Hence, \(\pv{\psi}{v'}\) is a \fc{0} from \(\Psi\) for the agent through \(\ed{\flat}\).
    \end{itemize}
    %
    So, \(\pv{\psi}{v'}\) being a \fc{} from \(\Psi\) through \(\ed{\flat}\) is a \requ{} of \(\ed{}\) by \autoref{def:requ}.
    \medskip

    \noindent%
    Now, as \(\pv{\psi}{v'}\) being a \fc{} from \(\Psi\) through \(\ed{\flat}\) is a \requ{} of \(\ed{}\), a \ros{} between \(\pv{\psi}{v'}\) and \(\Psi\) holds through \(\ed{\flat}\) by \supportII{} (\autopageref{idea:support:possible}).
    And, in turn, by \autoref{prop:requ-WhyV} (\autopageref{prop:requ-WhyV}), the \ros{} between \(\pv{\psi}{v'}\) and \(\Psi\) answers \qWhy{}.

    So, if \issueInclusion{} holds, there is some past or present event with respect to \(\ed{}\) in which the agent concludes \(\pv{\psi}{v'}\) from \(\Psi\).
    However, by Condition~\ref{prop:hinge:X:e:pro:fc:iii} there is no such event.
  \end{argument}
\end{note}



\begin{note}
  I like to read \autoref{prop:tCV-WhyV-ces} as a dilemma:

  \begin{quote}
    \begin{itemize}[labelwidth = \widthof{\emph{Either}:}]
    \item[\emph{Either}:]
      There are no cases in which conditions \ref{prop:hinge:X:tCed}, \ref{prop:hinge:X:e:act:se}, \ref{prop:hinge:X:tC}, \ref{prop:hinge:X:e:pro} and \ref{prop:hinge:X:e:pro:fc} jointly hold.
    \item[\emph{Or}:]
      \issueInclusion{} does not hold.
    \end{itemize}
  \end{quote}
  %
  And the thing is, there are cases in which conditions \ref{prop:hinge:X:tCed}, \ref{prop:hinge:X:e:act:se}, \ref{prop:hinge:X:tC}, \ref{prop:hinge:X:e:pro} and \ref{prop:hinge:X:e:pro:fc} jointly hold.
\end{note}




\section{Counter-samples}
\label{sec:counter-samples}

\begin{note}
  \begin{application}%
    \label{app:sc1-ce}%
    There is a reading of \autoref{illu:gist:roots:F}:
    \begin{itenum}
    \item[\emph{If}:]
      \begin{itemize}
      \item
        The agent is \tCV{} \pv{\propI{\rootsCon{}}}{\valI{True}} from \(\Phi\) by factorisation.
      \item
        {
          \color{blue}
          Instance.
        }
      \end{itemize}
    \item[\emph{Then}:]
      \issueInclusion{} does not hold.
    \end{itenum}
  \end{application}

  \begin{dets}{app:sc1-ce}
    For \(n\) in  \(\{6, 12, 20, 35, 42, 56\}\):
    \begin{itemize}
    \item
      The agent may switch to \rootsConEqGen{} and be concluding \pv{\propI{\rootsConGen{}}}{\valI{True}} from \(\Phi'\).
    \end{itemize}

    The agent has not concluded, e.g., \(42\).
  \end{dets}
\end{note}


\subsection{\autoref{illu:gist:roots:F}}





\section{A counter-sample}

\begin{note}
  \begin{scenario}[Sudoku puzzles]
    \label{illu:gist:sudoku}
    % https://tex.stackexchange.com/questions/91422/tikz-sudoku-circle-and-connect-with-lines-some-cells

    \mbox{ }\hfill%
    \begin{adjustbox}{minipage=0.45\linewidth,scale=1}
      \centering
      \begin{tikzpicture}[scale=.5]
        \begin{scope}
          \draw (0, 0) grid (9, 9);
          \draw[very thick, scale=3] (0, 0) grid (3, 3);
          \setcounter{row}{1}
          % Single entries
          \setrow { }{ }{ }  { }{ }{ }  {1}{ }{ }
          \setrow { }{ }{ }  { }{ }{ }  { }{5}{ }
          \setrow {9}{ }{ }  { }{ }{ }  { }{ }{2}
          \setrow { }{ }{3}  { }{2}{ }  { }{ }{ }
          \setrow { }{ }{ }  {8}{ }{ }  {4}{6}{5}
          \setrow { }{4}{ }  { }{5}{9}  { }{ }{8}
          \setrow { }{8}{7}  {2}{3}{1}  { }{4}{6}
          \setrow {2}{1}{ }  {5}{ }{ }  { }{ }{3}
          \setrow {3}{ }{6}  {4}{ }{8}  { }{ }{ }
        \end{scope}
      \end{tikzpicture}
    \end{adjustbox}%
    \begin{adjustbox}{minipage=0.45\linewidth,scale=1}
      \centering
      \begin{tikzpicture}[scale=.5]
        \begin{scope}
          \draw (0, 0) grid (9, 9);
          \draw[very thick, scale=3] (0, 0) grid (3, 3);
          \setcounter{row}{1}
          % Single entries
          \setrow {1}{ }{3}  { }{ }{ }  {8}{9}{5}
          \setrow { }{ }{ }  { }{1}{3}  { }{ }{7}
          \setrow {8}{6}{ }  { }{7}{ }  {3}{2}{ }
          \setrow { }{ }{ }  {2}{3}{6}  { }{ }{ }
          \setrow {4}{1}{9}  { }{8}{5}  { }{6}{3}
          \setrow { }{ }{2}  { }{ }{ }  { }{ }{ }
          \setrow {2}{ }{ }  { }{ }{ }  {4}{ }{6}
          \setrow { }{ }{ }  { }{ }{ }  { }{8}{ }
          \setrow { }{ }{8}  { }{9}{ }  { }{ }{ }
        \end{scope}
      \end{tikzpicture}
    \end{adjustbox}%
    \hfill\mbox{ }\newline
    \mbox{ }\newline
    \mbox{ }\hfill(\cite[84,85]{Coussement:2007up})\newline
  \end{scenario}

  \noindent%
  If you understand the rules of Sudoku, this counter-sample is interactive.
  For, start to fill in either grid and then return here.
\end{note}


\begin{note}
  Basic intuition.

  Why an event happens.
  Something which favours the event.
  Now, understanding of Sudoku.
  If you don't, then random numbers.
  Sudoku soup.
  But, the same thing which favours entails solution to the second puzzle.
  Granting that you're willing to attempt the second puzzle it follows that if you'd make Sudoku soup of the second puzzle, there is nothing which favours the event in which you solve the chosen Sudoku puzzle.
\end{note}


\begin{note}
  Detailed argument.
\end{note}



\paragraph*{An illustration}


\begin{note}
  \begin{illustration}[ジョジョリオン]%
    \nocite{huangmufeiluyan:2011aa}%
    Book.
    List of chapters.
    Is the agent reading the chapter titles?

    \begin{center}
      \bgroup
      \def\arraystretch{1.125}
      \begin{tabular}{R{.45\textwidth} L{.45\textwidth}}
        \multicolumn{2}{c}{Translations and chapter titles in `reading'} \\
        \hline\hline
        Translation & Title \\
        \hline
        Soft and wet & ソフト&ウェット \\
        \hdashline
        Safety above everything else & \multirow{3}*{無事が何より} \\
        Safety first & \\
        Gotta be safe & \\
        \hdashline
        Every day is summer vacation & 毎日が夏休み \\
        \hdashline
        A hair clip from ??? period & 清の時代の髪留め \\
      \end{tabular}
      \egroup
    \end{center}

    \noindent%
    First and third examples are straightforward.
    Difference is katakana and common kanji with a little grammar.
    Second, overly literal and slightly different translations due to lack of information about context.
    Fourth, do not need complete translation.

    \noindent%
    However, agent may fail to translate certain chapter titles such as:

    \begin{center}
      \bgroup
      \def\arraystretch{1.125}
      \begin{tabular}{R{.45\textwidth} L{.45\textwidth}}
        \multicolumn{2}{c}{Translations and chapter titles not in `reading'} \\
        \hline\hline
        Translation & Title \\
        \hline
        Software and wet & ソフト&ウェット \\
        \hdashline
        The Qing Dynasty Hair Clip & 清の時代の髪留め \\
        \hdashline
        ??? & 母と子
      \end{tabular}
      \egroup
    \end{center}

    \noindent%
    First, translation ruled out by context, though possible.
    Second and third, require knowledge that may be expected for fluency, but not reading.
    Second, did not require complete translation.
    Third, failure to translate \textquote{Mother and child}.
  \end{illustration}
\end{note}



\section{Prior \scen{1}}
\label{sec:cscen1}

\begin{note}
  Selection tasks and \autoref{illu:tR:powers}.
\end{note}

\section{Point}
\label{sec:point}

\begin{note}
  Now, the upshot here is perhaps quite minor.

  \issueInclusion{} still does work to see which \ros{} the agent has a \wit{} for.

  The point is more-or-less this.
  \issueInclusion{} does this work, but it does not do any additional work.
\end{note}


\subsection{`Why'}
\label{sec:why}


\begin{note}
  The argument provided against \issueInclusion{} does not show that the relevant counterexamples to \issueInclusion{} are \emph{necessary} for answering \qWhy{}.

  The point, really, is that this doesn't work as a pre-theoretical constraint.


  Further, whether it is possible to do this in terms of what is the case from the \agpe{}.
  Davidson, etc.
\end{note}

%%% Local Variables:
%%% mode: latex
%%% TeX-master: "master"
%%% TeX-engine: luatex
%%% End:
