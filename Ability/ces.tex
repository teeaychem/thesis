\chapter{Counterexamples}
\label{cha:ces}

\begin{note}
  \autoref{cha:clar:sec:literature}, in addition to intuition, constraint seems to often be a theoretical assumption.

  Purpose of variants is to motivate counterexamples to constraint.
  Specifically in terms of answers to \qWhyV{} which are not answers to \qHowV{}.
  In other words, \ros{} such that \ros{} explains, in part, why agent concludes but is such that the agent does not have a \wit{} for the \ros{}.

  In this section we outline in rough form how we will (attempt) to provide counterexamples.

  In short, need:
  An agent, event in which agent concludes \(\pv{\phi}{v}\) from \(\Phi\), and \ros{} between \(\pv{\psi}{v'}\) and \(\Psi\) such that:

  \begin{itemize}
  \item
    The agent does not have a \wit{} for the \ros{} between \(\pv{\psi}{v'}\) and \(\Psi\).
  \item
    The \ros{} between \(\pv{\psi}{v'}\) and \(\Psi\), in part, answers \qWhyV{}.
  \end{itemize}

  Our goal is motivate a general method for generating examples in which some \ros{} for which an agent does not have a \wit{} such that the \ros{} answers \qWhyV{}.
\end{note}


%%% Local Variables:
%%% mode: latex
%%% TeX-master: "master"
%%% End:
