\chapter{Counterexamples to \issueInclusion{}}
\label{cha:ces}


\begin{note}
  Here, should have the things used to capture \tCV{}.
  So, the arithmetic example.
  With Wason, specific cards.
  Of course, when get complete, \wit{}.
  But, for each card, there's a conclusion.
  This is sufficient.

  And, translation.
  Various translations that agent does not perform.
  In particular, when there are multiple possible translations.
\end{note}


\section{Counterexamples}
\label{sec:counterexamples}


\begin{note}
  \begin{proposition}[\tCV{3} and failures of \issueInclusion{}]
    \label{prop:tCV-WhyV-ces}
    \vspace{-\baselineskip}
    \begin{itenum}
    \item[\emph{If}:]
      Conditions \ref{prop:hinge:X:tCed}, \ref{prop:hinge:X:e:act:se}, \ref{prop:hinge:X:tC}, \ref{prop:hinge:X:e:pro} and \ref{prop:hinge:X:e:pro:fc} hold:
      \begin{enumerate}[label=\arabic*., ref=\arabic*]
      \item
        \label{prop:hinge:X:tCed}
        \(\ed{}\) is an event in which \vAgent{} (\typeAdv{}) concludes \(\pv{\phi}{v}\) from \(\Phi\) (by \torNa{} \(T\)).
      \item
        \label{prop:hinge:X:e:act:se}
        \(\ed{\flat}\) is a \se{} of \(\ed{}\).
      \item
        \label{prop:hinge:X:tC}
        \(\ed{\flat}\) is an event in which \vAgent{} is \tCV{} \(\pv{\phi}{v}\) from \(\Phi\) by \torNa{} \(T\).
      \item
        \label{prop:hinge:X:e:pro}
        \(\tproS{}\) is a projection of a type \(T\) with respect to \(\ed{}\) \vAgent{}.
      \item
        \label{prop:hinge:X:e:pro:fc}
        For some event \(\edn{}\) and \prop{0}-\val{0}-\pool{0} pair \(\pv{\psi}{v'}\) and \(\Psi\) in \(\tproS{}\):
        \begin{enumerate}[label=\alph*., ref=\theenumi\alph*]
        \item
          \label{prop:hinge:X:e:pro:fc:i}
          \(\edn{\sharp}\) is the result of an action \(a\) done by \vAgent{} in \(\ed{\flat}\).
        \item
          \label{prop:hinge:X:e:pro:fc:ii}
          \vAgent{} \evals{} \(\psi'\) as having value \(v''\) prior to doing \(a\), for each \prop{0}-\val{0} pair \(\pv{\psi'}{v''}\) in \(\Psi\).
        \item
          \label{prop:hinge:X:e:pro:fc:iii}
          \vAgent{} does not have a \wit{} for a \ros{} between \(\pv{\psi}{v'}\) and \(\Psi\) when \vAgent{} concludes \(\pv{\phi}{v}\) from \(\Phi\).
        \end{enumerate}
      \end{enumerate}
    \item[\emph{Then}:]
      \issueInclusion{} does not hold.
    \end{itenum}
    \vspace{-\baselineskip}
  \end{proposition}

  \noindent%
  \autoref{prop:tCV-WhyV-ces} is a variation on \autoref{prop:requ-WhyV-ces} (\autopageref{prop:requ-WhyV-ces}), where the condition of an agent \tCV{} helps identify \fc{1}, and in turn \requ{1} sufficient to show \issueInclusion{} does not hold.

    \begin{argument}{prop:tCV-WhyV-ces}
    Suppose \ref{prop:hinge:X:tCed}, \ref{prop:hinge:X:e:act:se}, \ref{prop:hinge:X:tC}, \ref{prop:hinge:X:e:pro} and \ref{prop:hinge:X:e:pro:fc} hold.
    \medskip

    \noindent%
    First, observe clauses~\ref{def:requ:se} and~\ref{def:requ:fc} of the definition of a \requ{} (\autoref{def:requ}, \autopageref{def:requ}) are satisfied with respect to \(\pv{\psi}{v'}\), \(\Psi\), and \(\ed{\flat}\):

    \begin{itemize}
    \item
      Clause~\ref{def:requ:se} is satisfied by assumption.
      For, by Clause~\ref{prop:hinge:X:e:act:se} \(\ed{\flat}\) is a \se{} of \(\ed{}\).
    \item
      Clause~\ref{def:requ:fc} is satisfied by observing conditions \ref{prop:hinge:X:tC}, \ref{prop:hinge:X:e:pro}, and \ref{prop:hinge:X:e:pro:fc} are the relevant conditions of \autoref{prop:tCV-fc} (\autopageref{prop:tCV-fc}).
      Hence, \(\pv{\psi}{v'}\) is a \fc{0} from \(\Psi\) for the agent through \(\ed{\flat}\).
    \end{itemize}
    %
    So, \(\pv{\psi}{v'}\) being a \fc{} from \(\Psi\) through \(\ed{\flat}\) is a \requ{} of \(\ed{}\) by \autoref{def:requ}.
    \medskip

    \noindent%
    Now, as \(\pv{\psi}{v'}\) being a \fc{} from \(\Psi\) through \(\ed{\flat}\) is a \requ{} of \(\ed{}\), a \ros{} between \(\pv{\psi}{v'}\) and \(\Psi\) holds through \(\ed{\flat}\) by \supportII{} (\autopageref{idea:support:possible}).
    And, in turn, by \autoref{prop:requ-WhyV} (\autopageref{prop:requ-WhyV}), the \ros{} between \(\pv{\psi}{v'}\) and \(\Psi\) answers \qWhy{}.

    So, if \issueInclusion{} holds, there is some past or present event with respect to \(\ed{}\) in which the agent concludes \(\pv{\psi}{v'}\) from \(\Psi\).
    However, by Condition~\ref{prop:hinge:X:e:pro:fc:iii} there is no such event.%
    \footnote{
      Note, the argument for \autoref{prop:tCV-WhyV-ces} only assumes \(\ed{}\) is an event in which \vAgent{} concludes \(\pv{\phi}{v}\) from \(\Phi\).
      The role of \tCV{} is limited to the event \(\edn{\flat}\) in which the conclusion of \(\edn{}\) is in progress.
      Still, as \(\ed{\flat}\) is an event in which the agent is \tCV{} \(\pv{\phi}{v}\) from \(\Phi\) by \torNa{} \(T\), and \(\ed{}\) happens as a result of \(\ed{\flat}\), it is natural for \(\ed{}\) to be an event in which \vAgent{} \typeAdv{} concludes \(\pv{\phi}{v}\) from \(\Phi\) by \torNa{} \(T\).
    }
  \end{argument}
\end{note}


\section{Counter-samples}
\label{sec:counter-samples}


\begin{note}
  With \autoref{prop:tCV-WhyV-ces} in hand, our final task is to show there are cases in which Conditions \ref{prop:hinge:X:tCed}, \ref{prop:hinge:X:e:act:se}, \ref{prop:hinge:X:tC}, \ref{prop:hinge:X:e:pro} and \ref{prop:hinge:X:e:pro:fc} hold.
\end{note}


\begin{note}
  A few counterexamples.

  The first counterexample is fully worked out and the following counterexamples highlight key features.
\end{note}



\subsection{\autoref{illu:gist:roots:F}}


\begin{note}
  The key features of a counterexample has been developed with respect to \autoref{illu:gist:roots:F} throughout this document.

  \begin{application}%
    \label{app:sc1-ce}%
    There is a reading of \autoref{illu:gist:roots:F}:
    \begin{itenum}
    \item[\emph{If}:]
      \begin{itemize}
      \item
        The agent is \tCV{} \pv{\propI{\rootsCon{}}}{\valI{True}} from \(\Phi\) by factorisation.
      \item
        {
          \color{blue}
          Instance.
        }
      \end{itemize}
    \item[\emph{Then}:]
      \issueInclusion{} does not hold.
    \end{itenum}
  \end{application}

  \begin{dets}{app:sc1-ce}
    \begin{itemize}
    \item
      \autoref{obs:se-inst}, \se{}.

      This needs to be strengthened a little.
    \item
      \autoref{app:sc1-typ}, \fc{1}.
    \end{itemize}
    For \(n\) in  \(\{6, 12, 20, 35, 42, 56\}\):
    \begin{itemize}
    \item
      The agent may switch to \rootsConEqGen{} and be concluding \pv{\propI{\rootsConGen{}}}{\valI{True}} from \(\Phi'\).
    \end{itemize}

    The agent has not concluded, e.g., \(42\).
  \end{dets}
\end{note}



\begin{note}
  Various other \scen{1}.
  The quadratic formula, Fibonnaci and Lucas numbers.
  Vary the prices.
  Chess puzzles.
  Agent doing well at selection tasks. (reference here to domain specific reasoning.)
  Completing textbook exercises.
\end{note}


\subsection{Sudoku puzzles}


\begin{note}
  The second counterexample is interactive.

  \begin{scenario}[Sudoku puzzles]%
    \label{illu:gist:sudoku}%
    % https://tex.stackexchange.com/questions/91422/tikz-sudoku-circle-and-connect-with-lines-some-cells
    An agent (you) starts working on the left (or right) Sudoku puzzle.
    \medskip

    \mbox{ }\hfill%
    \begin{adjustbox}{minipage=0.45\linewidth,scale=1}
      \centering
      \begin{tikzpicture}[scale=.5]
        \begin{scope}
          \draw (0, 0) grid (9, 9);
          \draw[very thick, scale=3] (0, 0) grid (3, 3);
          \setcounter{row}{1}
          % Single entries
          \setrow { }{ }{ }  { }{ }{ }  {1}{ }{ }
          \setrow { }{ }{ }  { }{ }{ }  { }{5}{ }
          \setrow {9}{ }{ }  { }{ }{ }  { }{ }{2}
          \setrow { }{ }{3}  { }{2}{ }  { }{ }{ }
          \setrow { }{ }{ }  {8}{ }{ }  {4}{6}{5}
          \setrow { }{4}{ }  { }{5}{9}  { }{ }{8}
          \setrow { }{8}{7}  {2}{3}{1}  { }{4}{6}
          \setrow {2}{1}{ }  {5}{ }{ }  { }{ }{3}
          \setrow {3}{ }{6}  {4}{ }{8}  { }{ }{ }
        \end{scope}
      \end{tikzpicture}
    \end{adjustbox}%
    \begin{adjustbox}{minipage=0.45\linewidth,scale=1}
      \centering
      \begin{tikzpicture}[scale=.5]
        \begin{scope}
          \draw (0, 0) grid (9, 9);
          \draw[very thick, scale=3] (0, 0) grid (3, 3);
          \setcounter{row}{1}
          % Single entries
          \setrow {1}{ }{3}  { }{ }{ }  {8}{9}{5}
          \setrow { }{ }{ }  { }{1}{3}  { }{ }{7}
          \setrow {8}{6}{ }  { }{7}{ }  {3}{2}{ }
          \setrow { }{ }{ }  {2}{3}{6}  { }{ }{ }
          \setrow {4}{1}{9}  { }{8}{5}  { }{6}{3}
          \setrow { }{ }{2}  { }{ }{ }  { }{ }{ }
          \setrow {2}{ }{ }  { }{ }{ }  {4}{ }{6}
          \setrow { }{ }{ }  { }{ }{ }  { }{8}{ }
          \setrow { }{ }{8}  { }{9}{ }  { }{ }{ }
        \end{scope}
      \end{tikzpicture}
    \end{adjustbox}%
    \hfill\mbox{ }\newline
    \mbox{ }\newline
    \mbox{ }\hfill(\cite[84,85]{Coussement:2007up})\newline
  \end{scenario}
\end{note}


\begin{note}
  Basic intuition.

  Why an event happens.
  Something which favours the event.
  Now, understanding of Sudoku.
  If you don't, then random numbers.
  Sudoku soup.
  But, the same thing which favours entails solution to the second puzzle.
  Granting that you're willing to attempt the second puzzle it follows that if you'd make Sudoku soup of the second puzzle, there is nothing which favours the event in which you solve the chosen Sudoku puzzle.
\end{note}


\begin{note}
  Multiple \fc{1}.
  For example, 8 in upper mid, or 6 in left mid.
\end{note}





\subsection{Translation}


\begin{note}
  Kind of interesting that Searle goes for a lookup table.
  Motivation here is semantic content, rather than syntactic association.
\end{note}


\begin{note}
  \begin{scenario}[ジョジョリオン]%
    \nocite{huangmufeiluyan:2011aa}%
    Book.
    List of chapters.
    Translating the chapter titles.

    \begin{center}
      \bgroup
      \def\arraystretch{1.125}
      \begin{tabular}{R{.45\textwidth} L{.45\textwidth}}
        \hline\hline
        Translation & Title \\
        \hline
        Soft and wet & ソフト&ウェット \\
        Safety first & 無事が何より \\
        Every day is summer vacation & 毎日が夏休み \\
        A hair clip from ??? period & 清の時代の髪留め \\
      \end{tabular}
      \egroup
    \end{center}
  \end{scenario}

  \noindent%
  Translation.
  In order to translate, sufficient understanding of two languages.

  Language, key feature is novel sentences.
  Much like previous scenario, chapter titles, so further chapter titles.
  
  Still, in contrast to scenario,
  
  partial translations.
  variety of translations.

  \begin{center}
    \bgroup
    \def\arraystretch{1.125}
    \begin{tabular}{R{.45\textwidth} L{.45\textwidth}}
      \hline\hline
      Translation & Title \\
      \hline
      Safety above everything else & \multirow{3}*{無事が何より} \\
      Safety first & \\
      Gotta be safe & \\
    \end{tabular}
    \egroup
  \end{center}

  
  With mathematical exercises, may expect agent to re-express solution.
  For example, \(200\), \(2^{3} \times 5^{2}\).
  However, distinct from finding solution.
  No clear problem if only solve with \(200\) and then re-express.
  
  With translation, re-expression after translation seems problematic.
  Suggests syntactic correspondence, but no clear grasp of semantic correspondence.
  
  Indeed, in the case of translation, absence of \fc{} (under standard circumstances) is plausibly important.
  For example, 「ソフト&ウェット」 is translated as 'Soft and Wet'.
  The background context is a number of other The Artist Formerly Known As Prince references (e.g.\ ペイズリー・パーク/Paisley Park).
  However, in a different context 「ソフト&ウェット」 may be translated as 'software and wetware'.
\end{note}



\section{Point}
\label{sec:point}

\begin{note}
  Now, the upshot here is perhaps quite minor.

  \issueInclusion{} still does work to see which \ros{} the agent has a \wit{} for.

  The point is more-or-less this.
  \issueInclusion{} does this work, but it does not do any additional work.
\end{note}



%%% Local Variables:
%%% mode: latex
%%% TeX-master: "master"
%%% TeX-engine: luatex
%%% End:
