\chapter{Motivation}
\label{cha:motivation}

\section{Theories}

\subsection{Reasoning}

\begin{note}[Some clarification]
  \begin{quote}
    What is the relation between a reason and an action when the reason explains the action by giving the agent's reason for doing what he did?%
    \mbox{}\hfill\mbox{({\citeyear[685]{Davidson:1963aa}})}
  \end{quote}

  Burgular case.
  There are reasons that an agent has to act which are not necessarily reasons for which the agent acts.

  Davidson distinction between \emph{reasons that one has to act} and \emph{reasons for which one acts}.

  \begin{quote}
    Davidson's article [asserts] a demand for a more ordinary form of explanation: an explanation which shows, not merely what, from another's point of view, \emph{could} count in favour of acting, but why that person did, in fact, act.%
    \mbox{}\hfill\mbox{(\citeyear[417]{Hieronymi:2011aa})}
  \end{quote}

  Following Neta, Anscombe.

  \begin{quote}
    a reason for which you act is always an answer to “a certain sense of the question ‘Why?' \dots that in which the answer, if positive, gives a reason for acting”
  \end{quote}

  Answer question from observer or the agent's perspective.

  With respect to reasoning, a pool of premises, or pools of premises.

  Distinction between \emph{how} and \emph{why} is with respect to \emph{reasons for which one acts}.

  Conclusion from some reasoning.
  Some trace.

  It seems not all components of the trace are relevant to why.
  Possible to go on a detour, or to simplify a line of reasoning.
\end{note}

\begin{note}[Inclusion]
  Note, a positive resolution to \autoref{issue:why-inc-in-how} does not require an answer to why to be equivalent to an answer to how.

  Cases of redundancy.
\end{note}

\paragraph*{Motivation}

\begin{note}
  We being with~\cite{Armstrong:1968vh}.
  \begin{quote}
    We are not concerned here with logicians' questions about inference, but solely with the psychological process of inferring.
    The primary sense of the word is that in which it involves acquiring a belief on the basis of a belief already held.

    \mbox{}\hfill\(\vdots\)\hfill\mbox{}

    \dots to say that A infers \emph{p} from \emph{q} is simply to say that A's believing \emph{q} \emph{causes} him to acquire the belief \emph{p}.
    And the sense of `cause' employed here is the common or billiardball sense of `cause', whatever that sense is.%
    \mbox{}\hfill\mbox{(\citeyear[194]{Armstrong:1968vh})}
  \end{quote}
  \cite{Armstrong:1968vh} goes on to consider some issues, but these reflect the `simplicity' of saying that inference is a matter of causation between beliefs, rather than causation.
  `With these qualifications, it seems that our causal account of inferring can stand.'
  (\citeyear[197]{Armstrong:1968vh})

  Of course, \citeauthor{Armstrong:1968vh} talks about `inference' rather than `reasoning', and in doing so restricts the scope of the proposal from arbitrary ways in which a proposition may be evaluated.
\end{note}

\begin{note}[Wedgwood]
  \begin{quote}
    Reasoning is a causal process, in which one mental event (say, one's accepting the conclusion of a certain argument) is caused by an antecedent mental event (say, one's considering the premises of the argument).%
    \mbox{}\hfill\mbox{(\cite[660]{Wedgwood:2006ui})}
  \end{quote}

  \citeauthor{Wedgwood:2006ui} illustrates how to go about witnessing.
  Of course, \citeauthor{Wedgwood:2006ui} here does not require positive resolution.
  Only talking about instances of reasoning.
  It is not immediate that any account of why is answered by reasoning.

  \begin{quote}
    Reasoning, I shall assume, is the process of \emph{revising ones beliefs or intentions, for a reason}.%
    \mbox{}\hfill\mbox{(\citeyear[660]{Wedgwood:2006ui})}
  \end{quote}

  Various connexions to other elements of epistemic state.
  At least include support.

  So, plausibly goes all the way.

  If why, then concluding is the result of reasoning.
  All reasoning is causal.
  So, positive resolution to \autoref{issue:why-inc-in-how}.
\end{note}

\begin{note}[Broome]
  {
    \color{red}
    In a footnote.
  }
  \citeauthor{Broome:2013aa} also.
  \begin{quote}
    So far as I can see, then, no further conditions need be added.
    I have arrived at necessary and sufficient conditions for a process to be active reasoning.
    Active reasoning is a particular sort of process by which conscious premise-attitudes cause you to acquire a conclusion-attitude.
    The process is that you operate on the contents of your premise-attitudes following a rule, to construct the conclusion, which is the content of a new attitude of yours that you acquire in the process.%
    \mbox{}\hfill\mbox{(\cite[234]{Broome:2013aa})}
  \end{quote}
\end{note}

\begin{note}[Illustration]
  Causal trace from conclusion to calculator.
  But, no relevance to understanding of arithmetic.
\end{note}

\begin{note}[Boghossian]
  Consider \citeauthor{Boghossian:2014aa}'s Taking Condition:
  \begin{quote}
    (Taking Condition): Inferring necessarily involves the thinker \emph{taking} his premises to support his conclusion and drawing his conclusion because of that fact.%
    \mbox{}\hfill\mbox{(\Citeyear[5]{Boghossian:2014aa})}
  \end{quote}
  As with \citeauthor{Armstrong:1968vh}, inference --- and hence reasoning with beliefs --- rather than reasoning more broadly (\Citeyear[cf][2]{Boghossian:2014aa}).

  \begin{quote}
    The intuition behind the Taking Condition is that no causal process counts as inference, unless it consists in an attempt to arrive at a belief by figuring out what, in some suitably broad sense, is supported by other things one believes.

    In the relevant sense, reasoning is something we \emph{do}, not just something that happens to us.
    And it is something \emph{we} do, not just something that is done by sub-personal bits of us.
    And it is something that we do with an \emph{aim}---that of figuring out what follows or is supported by other things one believes.
    It's hard to see how to respect these features of reasoning without something like the Taking Condition.%
    \mbox{}\hfill\mbox{(\Citeyear[5]{Boghossian:2014aa})}
  \end{quote}

  \begin{quote}
    \dots the property of a person's thinking something \emph{for a reason} is not response-dependent.
    To say that R was S's reason for A'ing implies that S took R to support his A'ing at the time that he A'ed, and that his so taking it led to his A'ing.

    I don't see how S's being disposed to \emph{say} that R was his reason for A'ing could \emph{make it the case} that he took R to support his A'ing and that this taking had a certain causal impact.
    His saying it might be very good evidence that R was his reason.
    But saying that R was his reason can't be \emph{constitutive} of R's being his reason.
    Causation is part of the idea of R's being his reason---and causation can't be a response-dependent property.%
    \mbox{}\hfill\mbox{(\citeyear[10--11]{Boghossian:2014aa})}
  \end{quote}

  {
    \color{red}
    See also \textcite[26--28]{Harman:1973ww} for an argument against dispositional accounts.
    In short, disposed to offer reasons targeted to audience.
    And, even if sincerity, possibility of identifying possible reasons rather than actual reasons.
    More basic, identification of reasons and conscious reasons.
  }

  {
    \color{red}
    Here, it's a kind of sufficiency.
    But, this final passage strengthens things.
  }

  {
    \color{green}
    Note, also, that though speak in terms of support, not assuming the taking condition.
  }
\end{note}

\begin{note}[Responding to reasons]
  As final motivation, consider the proposal at the core of \citeauthor{Lord:2018aa}'s (\Citeyear{Lord:2018aa}) thesis that being rational is to correctly respond to reasons.

  \begin{quote}
    \textbf{Correctly Responding:} What it is for A's \(\phi\)-ing to be ex post rational is for A to possess sufficient reason S to \(\phi\) and for A's \(\phi\)-ing to be a manifestation of knowledge about how to use S as sufficient reason to \(\phi\).%
    \mbox{}\hfill\mbox{(\Citeyear[143]{Lord:2018aa})}
  \end{quote}

  An agent's action is rational only if the action is a manifestation of some know-how.
  \citeauthor{Lord:2018aa} summaries:

  \begin{quote}
    \dots when one manifests one's know-how, dispositions that are directly sensitive to normative facts are manifesting. Thus, the competences involved in the relevant know-how make one directly sensitive to the normative facts%
    \mbox{}\hfill\mbox{(\Citeyear[16]{Lord:2018aa})}
  \end{quote}

  For our purposes, following example of manifesting know-how directly relates to reasoning:

  \begin{quote}
    The most salient disposition [when appealing to \emph{p} as a reason]%
    \footnote{Note, \citeauthor{Lord:2018aa} (explicitly) not talking about believing that \emph{p} is a reason, but argues that the cited disposition to present both when appealing to p as a reason and believing that \emph{p} is a reason.}
    is the disposition to (competently) use \emph{p} as a premise in reasoning.%
    \mbox{}\hfill\mbox{(\Citeyear[25]{Lord:2018aa})}
  \end{quote}

  Hence, suppose an agent concludes.
  Then, if the agent does not witness reasoning from pool of premises, it seems the agent does not manifest know-how, which is required for the appeal to meet \citeauthor{Lord:2018aa}'s account of rational action.

  Of course, that the noted disposition is the most salient does not rule out alternative, less noteworthy, dispositions.
  However, it is unclear to me how to \emph{manifest} know-how without use.
\end{note}

\begin{note}[Illustration]
  Clear that there is no manifestation of understanding of arithmetic.
\end{note}

\paragraph*{Less clear cases}

\begin{note}[Valaris]
  Maybe?

  Same observation as \citeauthor{Wedgwood:2006ui}, basically.
  Here, sufficient with additional hypothesis.
\end{note}

\begin{note}
  A note on \citeauthor{Valaris:2014un} here would be useful.
  For, \citeauthor{Valaris:2014un} offers something close, but suitably distinct.
  Reasoning is not to be identified with a causal process, but that's different.
  Constructs a coherent account of reasoning which does not involve causation.

  It's also limited.
  Because, there's a difference between believing p from availability of premises, and believing p from premises.

  It looks as though \citeauthor{Valaris:2014un} falls under witnessing.
\end{note}

\begin{note}[Other cases are less clear]
  For example, \citeauthor{Wright:2014tt}'s `Simple Proposal':
  \begin{quote}
    But consider instead the proposal, not that the status of the transition as inferential depends on the thinker's judgments about his reasons, but that it depends on \emph{what his reasons are}.
    We want his acceptance of the premises to supply his \emph{actual} reasons for accepting the conclusion.

    \mbox{}\hfill\(\vdots\)\hfill\mbox{}

    Call this the Simple Proposal.
    It says that a thinker infers q from p\(_{1}\) \(\cdots\) p\(_{\text{n}}\) when he accepts each of p\(_{1}\) \(\cdots\) p\(_{\text{n}}\), moves to accept q, and does so for the reason that he accepts p\(_{1}\) \(\cdots\) p\(_{\text{n}}\).\newline
      \mbox{}\hfill\mbox{(\Citeyear[33]{Wright:2014tt})}
    \end{quote}

    However, \citeauthor{Wright:2014tt} denies that reasoning must involve a state which connects premises to conclusions and so however \citeauthor{Wright:2014tt}'s Simple Proposal is developed, it will not involve a doxastic state:

    \begin{quote}
      What is needed, then, is an account of, or at least some insight into, what it is for certain intentional states of a thinker to be his actual reasons for his transition to another intentional state.

      [Which avoids] committing to the notion that doing something for certain reasons must involve a state that somehow registers those reasons as reasons for what one does.%
      \mbox{}\hfill\mbox{(\Citeyear[34]{Wright:2014tt})}
    \end{quote}
\end{note}

\subsection{The basing relation}

\begin{note}[Theories of basing]
  Connexion between reasoning and basing.
\end{note}

\begin{note}
  \citeauthor{Pollock:1999tm} introduce the basing relation with the following observation:
  \begin{quote}
    To be justified in believing something it is not sufficient merely to \emph{have} a good reason for believing it.
    One could have a good reason at one's disposal but never make the connection.
    \dots
    Surely, you are not justified in believing [something], despite the fact that you have impeccable reasons for it at your disposal.
    What is lacking is that you do not believe the conclusion on the basis of those reasons.\linebreak
    \mbox{}\hfill\mbox{(\Citeyear[35]{Pollock:1999tm})}
  \end{quote}
  The observation falls short of being an account of the basing relation, but the intuition \citeauthor{Pollock:1999tm} appeal to is instructive.
  It seems that an agent must connect reasons and the content of a belief in order for the belief to be formed on the basis of those reasons, and hence be justified by those reasons.
\end{note}

\begin{note}
  Apply same question to the basing relation.
\end{note}

\begin{note}
  Basing relation is delicate.
  No clear connexion with concluding.

  For example, consider \citeauthor{Evans:2013tw}' dispositional theory:

  \begin{quote}
    S's belief that \emph{p} is based on \emph{m} iff S is disposed to revise her belief that \emph{p} when she loses \emph{m}.%
    \mbox{}\hfill\mbox{(\citeyear[2952]{Evans:2013tw})}
  \end{quote}

  Intuitively, no witnessing.

  However, \citeauthor{Evans:2013tw}' dispositional theory is designed to capture why an agent sustains a belief, rather than why an agent forms a belief.%
  \footnote{
    See also \textcite{Audi:1986to} for a discussion of cases in which `[b]elieving for a reason does not entail having \textbf{come} to believe for that reason, or for any reason.' (\citeyear[32--33]{Audi:1986to})
  }

  \begin{quote}
    [T]he core of the basing relation is a particular sort of dependence:
    for one belief to be based on another is for the one to depend on the other in the right way.
    I think that the sort of dependence in question involves how one would respond were the basis of one's belief lost.
    Intuitively, one would revise one's belief, were its basis lost.
    If one's belief that \emph{p} really is based on one's belief that \emph{q}, one responds to a loss of the belief that \emph{q} by revising one's belief that \emph{p}.
    This is why we say that a belief stands or falls with its basis.%
    \mbox{}\hfill\mbox{(\citeyear[2951]{Evans:2013tw})}
  \end{quote}
\end{note}

\begin{note}
  More generally, what \citeauthor{Leite:2004uv} terms the `Spectatorial Conception' of the basing relation:%
  \footnote{
    See also~\textcite{Bondy:2018tk} for a detailed discussion of how the Spectatorial Conception of the basing relation relates to~\citeauthor{Schaffer:2010vq}'s (\citeyear{Schaffer:2010vq}) debasing demon.
  }
  \begin{quote}
    [T]he facts which determine basing relations are in place independently of the person’s explicit deliberation, reasoning, or declaration of reasons and are not directly determined by any of the person’s explicit deliberative or justificatory activity.%
    \mbox{}\hfill\mbox{(\citeyear[229]{Leite:2004uv})}
  \end{quote}

  Roughly, our approach to concluding is in line with the Spectatorial Conception.

  However, to the extent \citeauthor{Leite:2004uv} motivates rejecting the Spectatorial Conception, a problem.

  {
    \color{red}
    Also,~\cite{Sylvan:2016wq}
    Relation to inference.
  }
\end{note}

\begin{note}
  So, in short, there is no general connexion to the basing relation.

  For, in general, the relevant pre-theoretic characterisation of a basing relation between some basis and proposition-value pair need not bear any (pre-theoretic) relation to the agent concluding the proposition-value pair from the basis.

  Hence, there is no immediate connexion between issues~\ref{issue:why-inc-in-how} and~\ref{issue:has-witnessed} and accounts of the basing relation in general.

  Of course, if one is inclined to reject the Spectatorial Conception of the basing relation, then one may also be inclined to defer attention from issues~\ref{issue:why-inc-in-how} and~\ref{issue:has-witnessed}.
  For, it may be, following~\citeauthor{Evans:2013tw}, that why an agent comes to have some attitude toward some proposition-value pair would only be of interest in the context of why the agent sustains the attitude toward the proposition-value pair.
  And, at the same time, why an agent comes to have some attitude toward some proposition-value pair may have no clear relation to why the agent sustains the attitude toward the proposition-value pair.
\end{note}

\begin{note}
  So, I do not see any clear relation between issues~\ref{issue:why-inc-in-how} and~\ref{issue:has-witnessed} and accounts of the basing relation in general.

  However, specific account of the basing relation may hold some interest.
  In particular, we will briefly discuss two accounts of the basing relation which motivate interest in the broader idea of witnessing over causation.

  These accounts are:
  \begin{itemize}
  \item
    \citeauthor{Swain:1981wd}'s (\citeyear{Swain:1981wd}) causal-counterfactual theory of the basing relation.
  \item
    \citeauthor{Tolliver:1982us}'s (\citeyear{Tolliver:1982us}) doxastic theory of the basing relation.
  \end{itemize}
  We will explore both accounts in some detail, and particular attention will be given to the counterexamples \citeauthor{Tolliver:1982us} presses against \citeauthor{Swain:1981wd}'s theory.
\end{note}

\paragraph*{\citeauthor{Swain:1981wd}}

\begin{note}[\citeauthor{Swain:1981wd}]
  \begin{quote}
    \begin{enumerate}[label=(DB)]
    \item
      S's belief that \(h\) is based upon the set of causal reasons \(r\) at \(t\) \(=_{\text{\emph{df}}}\)
      \begin{enumerate}[label=(\arabic*)]
      \item
        S believes that \(h\) at \(t\); and
      \item
        For every member, \(r_{i}\) of \(R\), there is some time \(t_{n}\) (simultaneous or prior to \(t\)) such that
        \begin{enumerate}[label=(\alph*)]
        \item
          S has (or had) \(r_{j}\) at \(t_{n}\); and
        \item
          Either
          \begin{enumerate}
          \item[(\(i\))]
            S's having \(r_{j}\) at \(t_{n}\) is a cause or genuine overdeterminant of S's believing \(h\) at \(t\) or S's having \(r_{j}\) at \(t_{n}\) is a pseudo-overdeterminant of S's believing that \(h\) at \(t\);
          \item[(\(i\) + 1)]
            for some \(r_{i}\) and \(t_{i}\) that satisfy condition (i), S's having \(r_{j}\) at \(t_{n}\) is either a cause or a pseudo-overdeterminant of S's having \(r_{i}\) at \(t_{i}\).
          \end{enumerate}
        \end{enumerate}
      \end{enumerate}
    \end{enumerate}
  \end{quote}

  Pseudo-overdetermination:

  \begin{quote}
    \begin{enumerate}
    \item[(DPO)]
      Where \(c\) and \(e\) are occurrent events, \(c\) is a pseudo-overdeterminant of \(e\) if:
      \begin{enumerate}[label=(\arabic*)]
      \item
        \(c\) is not a cause of \(e\); and
      \item
        there is some set of occurrent events, \(D = \{d_{1}, d_{2},\dots, d_{n}\}\) (possibly having only one member), such that
        \begin{enumerate}
        \item
          each \(d_{i}\) in \(D\) is a cause of \(e\); and
        \item
          if no member of \(D\) had occurred, but \(c\) and \(e\) had occurred anyway, then there would have been a causal chain from \(c\) to \(e\), and \(c\) would have been causally prior to \(e\).
        \end{enumerate}
      \end{enumerate}
    \end{enumerate}
  \end{quote}

  Key here is occurrent events.

  In other words, only if the agent has witnessed reasoning.
\end{note}

\begin{note}[Swain]
  Here, with peduo overdetermination.

  With Swain, do not have witnessing.
  Well, it seems.
  But, this is not quite right.

  Here, very delicate.
  Present epistemic state.

  Key observation is that although Swain does not offer a causal account of the basing relation, it seems as though witnessed reasoning is required for pseudo-overdetermination.
\end{note}

\paragraph*{\citeauthor{Tolliver:1982us}}

\begin{note}[\citeauthor{Tolliver:1982us}]
  \citeauthor{Tolliver:1982us} is interesting.

  ``The pendulum case (and others like it) concerns problems arising from basing relations between beliefs which imply each other.''

  \begin{quote}
    Suppose A is a physics student who has learned that, from the period of a pendulum, one can calculate its length, and \emph{vice versa}.
    A observes that a particular pendulum \emph{b} has length \emph{l}.
    He calculates that \emph{b} must have period \emph{p}.
    As a result of his calculations, A has a couple of general beliefs:
    ``For all things, call them x, if x is a pendulum of length \emph{l} (call this ``Lx''), then x is a pendulum of period \emph{p} (``Px''), and (x) (if Px, then Lx).
    It seems clear that A's reason for believing Pb is his belief that Lb.
    But is A's belief that \emph{b} has period \emph{p} also one of A's reasons for his believing that \emph{b} has length \emph{l}?
    It would appear not.\newline
    \mbox{ }\hfill\mbox{(\citeyear[152]{Tolliver:1982us})}%
    \footnote{%
      The punctuation of this passage follows \citeauthor{Tolliver:1982us}'s paper, I am not sure why the initial left double quotation mark is not closed, nor am I clear on where it should be closed\dots

      As an aside, \citeauthor{Tolliver:1982us} presents the \scen{0} in this way to ease comparison with \citeauthor{Armstrong:1973vr}'s (\citeyear{Armstrong:1973vr}) account of the basing relation.
      We have, and will continue, to focus on \citeauthor{Swain:1981wd}'s account as \citeauthor{Armstrong:1973vr}'s account explicitly involves causation in a way \citeauthor{Swain:1981wd}'s does not.
    }
  \end{quote}
  \citeauthor{Tolliver:1982us} does not expand on why it appears A's belief that \emph{b} has period \emph{p} is \emph{not} also one of A's reasons for his believing that \emph{b} has length \emph{l}.
\end{note}

\begin{note}[Back to Swain]
  Have:
  \(P = 2\pi\sqrt{\sfrac{L}{g}}\) iff \(L = g \left(\sfrac{P}{2\pi}\right)^{2}\)

  Here, these are numbers rather than variables.

  So, if nothing changes, \(g(\sfrac{P}{2\pi})^{2}\) is \(L\).
  But, then \(P\) is involved.

  This is on 155 of \citeauthor{Tolliver:1982us}.

  So, the objection is delicate.
  The problem is not that the agent would measure the period, and then calculate the length.
  Rather, it's that we have a biconditional.
  And, it is assumed to be immediate that given the biconditional, substitution, roughly.

  Hence, the apparent flaw with Swain's account is the witnessed reasoning from length to period.
  This leads to an strange form of pseudo-overdetermination.
\end{note}

\begin{note}[\citeauthor{Tolliver:1982us}'s theory]
  \citeauthor{Tolliver:1982us} proposes the following doxastic account of the basing relation.
  \begin{quote}
    \begin{enumerate}[label=(B\('\))]
    \item
      A bases his belief that q on p at time t, iff
      \begin{enumerate}[label=(\arabic*)]
      \item
        A believes that q at t and A believes that p at t, and
      \item
        A believes that the truth of p is evidence for the truth of q at t,andd
      \item
        Where A's estimate of the likelihood of q equals h at t \((0 < \text{h} \leq 1)\), \emph{if it were the case that}:
        \begin{enumerate}[label=(\roman*)]
        \item
          A's second·order estimate of the L-proposition ``the likelihood of q is greater than or equal to h'' is less prior to t than it at t, and
        \item
          A did not believe p prior to t, and
        \item
          A came to belive p at t,
        \end{enumerate}
        then, at t, A's second-order estimate of the L-proposition ``the likelihood of q is greater than or equal to h'' would be greater than it was prior to t.%
        \mbox{}\hfill\mbox{(\citeyear[159]{Tolliver:1982us})}
      \end{enumerate}
    \end{enumerate}
  \end{quote}

  Intuitively, combining to believe p makes a difference.

  Now, return to the pendulum case.
  Here, \citeauthor{Tolliver:1982us} argues for no basing.

  \begin{quote}
    [A]ll we need consider is whether the introduction of the belief that Pb into A's doxastic framework is sufficient to increase A's propensity to hold an estimate of the likelihood of Lb higher than or equal to its present value.
    The answer is ``no.''
    The strength of A's propensity for his estimate of the likelihood of Lb not to drop below its present level is not at all increased by his belief that Pb.
    A's belief that Pb will not tend to counteract the influence of any factor tending to reduce A's confidence in the truth of Lb.
  \end{quote}

  So, does belief about the period make any difference with respect to length.

  \citeauthor{Tolliver:1982us}'s point is that this does nothing.

  Footnote, because of no information about period.

  But, this is puzzling.

  Unless witness reasoning from period to length, then it's not clear why getting the period would have any influence.
  This is the agent's estimate.
  Not, a second order estimate given the agent's beliefs.

  So, not causal, and no clear statement of witnessing, but hard to understand without.
\end{note}


%%% Local Variables:
%%% mode: latex
%%% TeX-master: "master"
%%% End: