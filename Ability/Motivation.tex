\chapter{Motivation}
\label{cha:motivation}

\begin{note}
  In this chapter, collect various account from the literature which motivate\dots

  {
    \color{red}
    Here, restate relevant issues/ideas.
  }

  Split into reasoning and the basing relation.
\end{note}

\section{Theories}

\subsection{Reasoning}

\begin{note}[Some clarification]
  \begin{quote}
    What is the relation between a reason and an action when the reason explains the action by giving the agent's reason for doing what he did?%
    \mbox{}\hfill\mbox{({\citeyear[685]{Davidson:1963aa}})}
  \end{quote}

  Burgular case.
  There are reasons that an agent has to act which are not necessarily reasons for which the agent acts.

  Davidson distinction between \emph{reasons that one has to act} and \emph{reasons for which one acts}.

  \begin{quote}
    Davidson[ asserts] a demand for a more ordinary form of explanation:
    an explanation which shows, not merely what, from another's point of view, \emph{could} count in favour of acting, but why that person did, in fact, act.%
    \mbox{}\hfill\mbox{(\citeyear[417]{Hieronymi:2011aa})}
  \end{quote}
\end{note}

\begin{note}

%   Note, a positive resolution to \autoref{issue:why-inc-in-how} does not require an answer to why to be equivalent to an answer to how.

%   Cases of redundancy.
% \end{note}

% \paragraph*{Causal}

% \begin{note}[Wedgwood]
%   \begin{quote}
%     Reasoning is a causal process, in which one mental event (say, one's accepting the conclusion of a certain argument) is caused by an antecedent mental event (say, one's considering the premises of the argument).%
%     \mbox{}\hfill\mbox{(\cite[660]{Wedgwood:2006ui})}
%   \end{quote}

%   \citeauthor{Wedgwood:2006ui} illustrates how to go about witnessing.
%   Of course, \citeauthor{Wedgwood:2006ui} here does not require positive resolution.
%   Only talking about instances of reasoning.
%   It is not immediate that any account of why is answered by reasoning.

%   \begin{quote}
%     Reasoning, I shall assume, is the process of \emph{revising ones beliefs or intentions, for a reason}.%
%     \mbox{}\hfill\mbox{(\citeyear[660]{Wedgwood:2006ui})}
%   \end{quote}

%   Various connexions to other elements of epistemic state.
%   At least include support.

%   So, plausibly goes all the way.

%   If why, then concluding is the result of reasoning.
%   All reasoning is causal.
%   So, positive resolution to \autoref{issue:why-inc-in-how}.
% \end{note}

% \begin{note}[Broome]
%   {
%     \color{red}
%     In a footnote.
%   }
%   \citeauthor{Broome:2013aa} also.
%   \begin{quote}
%     So far as I can see, then, no further conditions need be added.
%     I have arrived at necessary and sufficient conditions for a process to be active reasoning.
%     Active reasoning is a particular sort of process by which conscious premise-attitudes cause you to acquire a conclusion-attitude.
%     The process is that you operate on the contents of your premise-attitudes following a rule, to construct the conclusion, which is the content of a new attitude of yours that you acquire in the process.%
%     \mbox{}\hfill\mbox{(\cite[234]{Broome:2013aa})}
%   \end{quote}
% \end{note}

% \begin{note}[Illustration]
%   Causal trace from conclusion to calculator.
%   But, no relevance to understanding of arithmetic.
% \end{note}

% \begin{note}[Boghossian]
%   Consider \citeauthor{Boghossian:2014aa}'s Taking Condition:
%   \begin{quote}
%     (Taking Condition): Inferring necessarily involves the thinker \emph{taking} his premises to support his conclusion and drawing his conclusion because of that fact.%
%     \mbox{}\hfill\mbox{(\citeyear[5]{Boghossian:2014aa})}
%   \end{quote}
%   As with \citeauthor{Armstrong:1968vh}, inference --- and hence reasoning with beliefs --- rather than reasoning more broadly (\citeyear[cf][2]{Boghossian:2014aa}).

%   \begin{quote}
%     The intuition behind the Taking Condition is that no causal process counts as inference, unless it consists in an attempt to arrive at a belief by figuring out what, in some suitably broad sense, is supported by other things one believes.

%     In the relevant sense, reasoning is something we \emph{do}, not just something that happens to us.
%     And it is something \emph{we} do, not just something that is done by sub-personal bits of us.
%     And it is something that we do with an \emph{aim}---that of figuring out what follows or is supported by other things one believes.
%     It's hard to see how to respect these features of reasoning without something like the Taking Condition.%
%     \mbox{}\hfill\mbox{(\citeyear[5]{Boghossian:2014aa})}
%   \end{quote}

  % \begin{quote}
  %   \dots the property of a person's thinking something \emph{for a reason} is not response-dependent.
  %   To say that R was S's reason for A'ing implies that S took R to support his A'ing at the time that he A'ed, and that his so taking it led to his A'ing.

  %   I don't see how S's being disposed to \emph{say} that R was his reason for A'ing could \emph{make it the case} that he took R to support his A'ing and that this taking had a certain causal impact.
  %   His saying it might be very good evidence that R was his reason.
  %   But saying that R was his reason can't be \emph{constitutive} of R's being his reason.
  %   Causation is part of the idea of R's being his reason---and causation can't be a response-dependent property.%
  %   \mbox{}\hfill\mbox{(\citeyear[10--11]{Boghossian:2014aa})}
  % \end{quote}
\end{note}

\subsection{The basing relation}

\begin{note}[Theories of basing]
  Connexion between reasoning and basing.
\end{note}

\begin{note}
  \citeauthor{Pollock:1999tm} introduce the basing relation with the following observation:
  \begin{quote}
    To be justified in believing something it is not sufficient merely to \emph{have} a good reason for believing it.
    One could have a good reason at one's disposal but never make the connection.
    \dots
    Surely, you are not justified in believing [something], despite the fact that you have impeccable reasons for it at your disposal.
    What is lacking is that you do not believe the conclusion on the basis of those reasons.\linebreak
    \mbox{}\hfill\mbox{(\citeyear[35]{Pollock:1999tm})}
  \end{quote}
  The observation falls short of being an account of the basing relation, but the intuition \citeauthor{Pollock:1999tm} appeal to is instructive.
  It seems that an agent must connect reasons and the content of a belief in order for the belief to be formed on the basis of those reasons, and hence be justified by those reasons.
\end{note}

\begin{note}
  Apply same question to the basing relation.
\end{note}

\begin{note}
  Basing relation is delicate.
  No clear connexion with concluding.

  For example, consider \citeauthor{Evans:2013tw}' dispositional theory:

  \begin{quote}
    S's belief that \emph{p} is based on \emph{m} iff S is disposed to revise her belief that \emph{p} when she loses \emph{m}.%
    \mbox{}\hfill\mbox{(\citeyear[2952]{Evans:2013tw})}
  \end{quote}

  Intuitively, no witnessing.

  However, \citeauthor{Evans:2013tw}' dispositional theory is designed to capture why an agent sustains a belief, rather than why an agent forms a belief.%
  \footnote{
    See also \textcite{Audi:1986to} for a discussion of cases in which `[b]elieving for a reason does not entail having \textbf{come} to believe for that reason, or for any reason.' (\citeyear[32--33]{Audi:1986to})
  }

  \begin{quote}
    [T]he core of the basing relation is a particular sort of dependence:
    for one belief to be based on another is for the one to depend on the other in the right way.
    I think that the sort of dependence in question involves how one would respond were the basis of one's belief lost.
    Intuitively, one would revise one's belief, were its basis lost.
    If one's belief that \emph{p} really is based on one's belief that \emph{q}, one responds to a loss of the belief that \emph{q} by revising one's belief that \emph{p}.
    This is why we say that a belief stands or falls with its basis.%
    \mbox{}\hfill\mbox{(\citeyear[2951]{Evans:2013tw})}
  \end{quote}
\end{note}

\begin{note}
  More generally, what \citeauthor{Leite:2004uv} terms the `Spectatorial Conception' of the basing relation:%
  \footnote{
    See also~\textcite{Bondy:2018tk} for a detailed discussion of how the Spectatorial Conception of the basing relation relates to~\citeauthor{Schaffer:2010vq}'s (\citeyear{Schaffer:2010vq}) debasing demon.
  }
  \begin{quote}
    [T]he facts which determine basing relations are in place independently of the person’s explicit deliberation, reasoning, or declaration of reasons and are not directly determined by any of the person’s explicit deliberative or justificatory activity.%
    \mbox{}\hfill\mbox{(\citeyear[229]{Leite:2004uv})}
  \end{quote}

  Roughly, our approach to concluding is in line with the Spectatorial Conception.

  However, to the extent \citeauthor{Leite:2004uv} motivates rejecting the Spectatorial Conception, a problem.

  {
    \color{red}
    Also,~\cite{Sylvan:2016wq}
    Relation to inference.
  }
\end{note}

\begin{note}
  So, in short, there is no general connexion to the basing relation.

  For, in general, the relevant pre-theoretic characterisation of a basing relation between some basis and proposition-value pair need not bear any (pre-theoretic) relation to the agent concluding the proposition-value pair from the basis.

  Hence, there is no immediate connexion between issues~\ref{issue:why-inc-in-how} and~\ref{issue:has-witnessed} and accounts of the basing relation in general.

  Of course, if one is inclined to reject the Spectatorial Conception of the basing relation, then one may also be inclined to defer attention from issues~\ref{issue:why-inc-in-how} and~\ref{issue:has-witnessed}.
  For, it may be, following~\citeauthor{Evans:2013tw}, that why an agent comes to have some attitude toward some proposition-value pair would only be of interest in the context of why the agent sustains the attitude toward the proposition-value pair.
  And, at the same time, why an agent comes to have some attitude toward some proposition-value pair may have no clear relation to why the agent sustains the attitude toward the proposition-value pair.
\end{note}

\begin{note}
  So, I do not see any clear relation between issues~\ref{issue:why-inc-in-how} and~\ref{issue:has-witnessed} and accounts of the basing relation in general.

  However, specific account of the basing relation may hold some interest.
  In particular, we will briefly discuss two accounts of the basing relation which motivate interest in the broader idea of witnessing over causation.

  These accounts are:
  \begin{itemize}
  \item
    \citeauthor{Swain:1981wd}'s (\citeyear{Swain:1981wd}) causal-counterfactual theory of the basing relation.
  \item
    \citeauthor{Tolliver:1982us}'s (\citeyear{Tolliver:1982us}) doxastic theory of the basing relation.
  \end{itemize}
  We will explore both accounts in some detail, and particular attention will be given to the counterexamples \citeauthor{Tolliver:1982us} presses against \citeauthor{Swain:1981wd}'s theory.
\end{note}

\paragraph*{\citeauthor{Swain:1981wd}}

\begin{note}[\citeauthor{Swain:1981wd}]
  \begin{quote}
    \begin{enumerate}[label=(DB)]
    \item
      S's belief that \(h\) is based upon the set of causal reasons \(r\) at \(t\) \(=_{\text{\emph{df}}}\)
      \begin{enumerate}[label=(\arabic*)]
      \item
        S believes that \(h\) at \(t\); and
      \item
        For every member, \(r_{i}\) of \(R\), there is some time \(t_{n}\) (simultaneous or prior to \(t\)) such that
        \begin{enumerate}[label=(\alph*)]
        \item
          S has (or had) \(r_{j}\) at \(t_{n}\); and
        \item
          Either
          \begin{enumerate}
          \item[(\(i\))]
            S's having \(r_{j}\) at \(t_{n}\) is a cause or genuine overdeterminant of S's believing \(h\) at \(t\) or S's having \(r_{j}\) at \(t_{n}\) is a pseudo-overdeterminant of S's believing that \(h\) at \(t\);
          \item[(\(i\) + 1)]
            for some \(r_{i}\) and \(t_{i}\) that satisfy condition (i), S's having \(r_{j}\) at \(t_{n}\) is either a cause or a pseudo-overdeterminant of S's having \(r_{i}\) at \(t_{i}\).
          \end{enumerate}
        \end{enumerate}
      \end{enumerate}
    \end{enumerate}
  \end{quote}

  Pseudo-overdetermination:

  \begin{quote}
    \begin{enumerate}
    \item[(DPO)]
      Where \(c\) and \(e\) are occurrent events, \(c\) is a pseudo-overdeterminant of \(e\) if:
      \begin{enumerate}[label=(\arabic*)]
      \item
        \(c\) is not a cause of \(e\); and
      \item
        there is some set of occurrent events, \(D = \{d_{1}, d_{2},\dots, d_{n}\}\) (possibly having only one member), such that
        \begin{enumerate}
        \item
          each \(d_{i}\) in \(D\) is a cause of \(e\); and
        \item
          if no member of \(D\) had occurred, but \(c\) and \(e\) had occurred anyway, then there would have been a causal chain from \(c\) to \(e\), and \(c\) would have been causally prior to \(e\).
        \end{enumerate}
      \end{enumerate}
    \end{enumerate}
  \end{quote}

  Key here is occurrent events.

  In other words, only if the agent has witnessed reasoning.
\end{note}

\begin{note}[Swain]
  Here, with peduo overdetermination.

  With Swain, do not have witnessing.
  Well, it seems.
  But, this is not quite right.

  Here, very delicate.
  Present epistemic state.

  Key observation is that although Swain does not offer a causal account of the basing relation, it seems as though witnessed reasoning is required for pseudo-overdetermination.
\end{note}

\paragraph*{\citeauthor{Tolliver:1982us}}

\begin{note}[\citeauthor{Tolliver:1982us}]
  \citeauthor{Tolliver:1982us} is interesting.

  ``The pendulum case (and others like it) concerns problems arising from basing relations between beliefs which imply each other.''

  \begin{quote}
    Suppose A is a physics student who has learned that, from the period of a pendulum, one can calculate its length, and \emph{vice versa}.
    A observes that a particular pendulum \emph{b} has length \emph{l}.
    He calculates that \emph{b} must have period \emph{p}.
    As a result of his calculations, A has a couple of general beliefs:
    ``For all things, call them x, if x is a pendulum of length \emph{l} (call this ``Lx''), then x is a pendulum of period \emph{p} (``Px''), and (x) (if Px, then Lx).
    It seems clear that A's reason for believing Pb is his belief that Lb.
    But is A's belief that \emph{b} has period \emph{p} also one of A's reasons for his believing that \emph{b} has length \emph{l}?
    It would appear not.\newline
    \mbox{ }\hfill\mbox{(\citeyear[152]{Tolliver:1982us})}%
    \footnote{%
      The punctuation of this passage follows \citeauthor{Tolliver:1982us}'s paper, I am not sure why the initial left double quotation mark is not closed, nor am I clear on where it should be closed\dots

      As an aside, \citeauthor{Tolliver:1982us} presents the \scen{0} in this way to ease comparison with \citeauthor{Armstrong:1973vr}'s (\citeyear{Armstrong:1973vr}) account of the basing relation.
      We have, and will continue, to focus on \citeauthor{Swain:1981wd}'s account as \citeauthor{Armstrong:1973vr}'s account explicitly involves causation in a way \citeauthor{Swain:1981wd}'s does not.
    }
  \end{quote}
  \citeauthor{Tolliver:1982us} does not expand on why it appears A's belief that \emph{b} has period \emph{p} is \emph{not} also one of A's reasons for his believing that \emph{b} has length \emph{l}.
\end{note}

\begin{note}[Back to Swain]
  Have:
  \(P = 2\pi\sqrt{\sfrac{L}{g}}\) iff \(L = g \left(\sfrac{P}{2\pi}\right)^{2}\)

  Here, these are numbers rather than variables.

  So, if nothing changes, \(g(\sfrac{P}{2\pi})^{2}\) is \(L\).
  But, then \(P\) is involved.

  This is on 155 of \citeauthor{Tolliver:1982us}.

  So, the objection is delicate.
  The problem is not that the agent would measure the period, and then calculate the length.
  Rather, it's that we have a biconditional.
  And, it is assumed to be immediate that given the biconditional, substitution, roughly.

  Hence, the apparent flaw with Swain's account is the witnessed reasoning from length to period.
  This leads to an strange form of pseudo-overdetermination.
\end{note}

\begin{note}[\citeauthor{Tolliver:1982us}'s theory]
  \citeauthor{Tolliver:1982us} proposes the following doxastic account of the basing relation.
  \begin{quote}
    \begin{enumerate}[label=(B\('\))]
    \item
      A bases his belief that q on p at time t, iff
      \begin{enumerate}[label=(\arabic*)]
      \item
        A believes that q at t and A believes that p at t, and
      \item
        A believes that the truth of p is evidence for the truth of q at t,andd
      \item
        Where A's estimate of the likelihood of q equals h at t \((0 < \text{h} \leq 1)\), \emph{if it were the case that}:
        \begin{enumerate}[label=(\roman*)]
        \item
          A's second·order estimate of the L-proposition ``the likelihood of q is greater than or equal to h'' is less prior to t than it at t, and
        \item
          A did not believe p prior to t, and
        \item
          A came to belive p at t,
        \end{enumerate}
        then, at t, A's second-order estimate of the L-proposition ``the likelihood of q is greater than or equal to h'' would be greater than it was prior to t.%
        \mbox{}\hfill\mbox{(\citeyear[159]{Tolliver:1982us})}
      \end{enumerate}
    \end{enumerate}
  \end{quote}

  Intuitively, combining to believe p makes a difference.

  Now, return to the pendulum case.
  Here, \citeauthor{Tolliver:1982us} argues for no basing.

  \begin{quote}
    [A]ll we need consider is whether the introduction of the belief that Pb into A's doxastic framework is sufficient to increase A's propensity to hold an estimate of the likelihood of Lb higher than or equal to its present value.
    The answer is ``no.''
    The strength of A's propensity for his estimate of the likelihood of Lb not to drop below its present level is not at all increased by his belief that Pb.
    A's belief that Pb will not tend to counteract the influence of any factor tending to reduce A's confidence in the truth of Lb.
  \end{quote}

  So, does belief about the period make any difference with respect to length.

  \citeauthor{Tolliver:1982us}'s point is that this does nothing.

  Footnote, because of no information about period.

  But, this is puzzling.

  Unless witness reasoning from period to length, then it's not clear why getting the period would have any influence.
  This is the agent's estimate.
  Not, a second order estimate given the agent's beliefs.

  So, not causal, and no clear statement of witnessing, but hard to understand without.
\end{note}

\subsubsection{Enthymematic inferences}

\begin{note}[\citeauthor{Moretti:2019wx}]
  Above we considered how various account of the basing relation seem to imply \issueConstraint{}/\issueInclusion{}.
  Roughly, because such accounts of the basing relation required a premise or step of reasoning to be used in order to be a candidate member of the base of some conclusion of reasoning --- motivated by either causal and representational considerations.
  In contrast, \citeauthor{Moretti:2019wx} argue for an account of the basing relation which does not entail \issueConstraint{}/\issueInclusion{}.

  In our terminology, \citeauthor{Moretti:2019wx} argue that: A belief held by an agent may be \emph{based} on premises that the agent did not use when forming the belief.

  The following is a fragment of the general principle relating propositional justification to well-grounded belief (alternatively doxastistcally justified belief) containing the two clauses of interest:

  \begin{quote}
    IF

    \dots

    OR

    \begin{enumerate*}[label=(\arabic*.2\(^{\ast}\))]
    \item\label{LT:1.2} Q is propositionally justified for S in virtue of P1, P2, \(\dots\), Pn being justifiedly true from her perspective because S justifiedly believes P1, P2, \(\dots\), Pn, and in virtue of her being aware that Q is an inductive or deductive consequence of P1, P2, \(\dots\), Pn jointly, and
    \item\label{LT:2.2} S carries out a \emph{plain} inference from P1, P2, \(\dots\), Pn to Q.
    \end{enumerate*}

    OR

    \begin{enumerate*}[label=(\arabic*.3), ref=(\arabic*.3)]
    \item\label{LT:1.3} Q is propositionally justified for S in virtue of P1, P2, \(\dots\), Pn being justifiedly true from her perspective, though S doesn't believe at least some P1, P2, \(\dots\), Pn, and in virtue of S being aware that Q is an inductive or deductive consequence of P1, P2, \(\dots\), Pn jointly, and
    \item\label{LT:2.3} S carries out a (fully or partly) \emph{enthymematic inference} from P1, P2, \(\dots\), Pn to Q.
    \end{enumerate*}

    THEN
    \begin{enumerate}[label=(3)]
    \item S's belief that Q is well-grounded.\nolinebreak
      \mbox{}\hfill\mbox{(\citeyear[87]{Moretti:2019wx})}
    \end{enumerate}
  \end{quote}

  The `plain' inference of~\ref{LT:1.2} and~\ref{LT:2.2} corresponds to cases in which an agent uses P1, P2, \(\dots\), Pn to reason to Q.
  By contrast, the `enthymematic' inference of~\ref{LT:1.3} and~\ref{LT:2.3} involves reasoning in which an agent does not use some or all of P1, P2, \(\dots\), Pn to reason to Q as the agent does not believe some of P1, P2, \(\dots\), Pn (though the agent has propositional support for each of P1, P2, \(\dots\), Pn).

  To illustrate the distinction between `plain' and `enthymematic' inferences (\citeyear[Cf.][85]{Moretti:2019wx}) consider reasoning from the premise that \nagent{5} is shorter than \nagent{6} to the conclusion that someone is taller than \nagent{5}.
  An instance of plain (non-enthymematic) may take the intermediary step that \nagent{6} is taller than \nagent{5} before abstracting from \nagent{6}.
  In contrast, an instance of enthymematic reasoning consists of the (single) premise and conclusion noted without forming the belief that \nagent{6} is taller than \nagent{5}.\nolinebreak
  \footnote{Cf.\ (\citeyear[87--89]{Moretti:2019wx}) for examples given by \citeauthor{Moretti:2019wx}.}

  The key idea is that if an agent reasons enthymematically, then the agent's belief may be based on those premises that the agent would use in the corresponding plain inference.
  (\citeyear[Cf.][86--87]{Moretti:2019wx})
  Hence, we have a proposal on which an agent's belief may be supported by premises and steps of reasoning that an agent has not used.
  And, in addition, because S carries out a (fully or partly) enthymematic inference \ref{LT:2.3}, it seems S \emph{may} appeal to P1, P2, \(\dots\), Pn when reasoning to Q, in conflict with \issueConstraint{}/\issueInclusion{}.

  Whether or not \citeauthor{Moretti:2019wx}'s account is correct is not of interest.
  Rather, \emph{grating} that \citeauthor{Moretti:2019wx}'s account is correct allows us to make two (related) observations.
  First, \citeauthor{Moretti:2019wx}'s account does not conflict with \issueConstraint{}/\issueInclusion{}.
  Second, \dots ???
\end{note}

\begin{note}[First point]
  To establish the first point we require further details about how \citeauthor{Moretti:2019wx} define a (fully or partly) enthymematic inference.
  The following quote combines the relevant definitions:
  \begin{quote}
    \textbf{(}[\textbf{Partly}/\textbf{Fully}] \textbf{Enthymematic Inference)}

    S carries out a [\emph{partly}/\emph{fully}] \emph{enthymematic} inference from P1, P2, \(\dots\), Pn to Q if and only if
    \begin{enumerate}[label=(\alph*), ref=(\alph*)]
       \setcounter{enumi}{1}
    \item \emph{S doesn't actually believe} [\emph{at least some of the premises}/\emph{any of}] P1, P2, \(\dots\), Pn, though some constituents M1, M2, \(\dots\), Mm of S's perspective cause in S the \emph{disposition} to believe P1, P2, \(\dots\), Pn, and
    \item M1, M2, \(\dots\), Mm [together with the premises believed by S jointly/jointly] cause S's belief that Q through a process that is shaped by S's taking Q to be a consequence of P1, P2, \(\dots\), Pn at a personal level.\nolinebreak
      \mbox{}\hfill\mbox{(\citeyear[85]{Moretti:2019wx})}
    \end{enumerate}
  \end{quote}

  In short, an enthymematic inference involves reasoning with premises M1, M2, \(\dots\), Mm which are related to the premises P1, P2, \(\dots\), Pn of some corresponding plain inference.
  In order to complete the definition, we require an account of what it is for S to take Q to be a consequence of P1, P2, \(\dots\), Pn at a personal level:

  \begin{quote}
    \textbf{(Personal Level\(^{\ast}\))}

    S's mental states M1, M2, \(\dots\), Mm and any premises believed by S, among P1, P2, \(\dots\), Pn, jointly cause S's belief that Q through a process shaped by S's taking Q to be a consequence of P1, P2, \(\dots\), Pn at a personal level if and only if M1, M2, \(\dots\), Mm and any premise believed by S, among P1, P2, \(\dots\), Pn, jointly cause S to believe Q and S would adduce the reasons that P1, P2, \(\dots\), Pn and that Q is a consequence of P1, P2, \(\dots\), Pn in response to a request to explain why she believes Q.\nolinebreak
    \mbox{}\hfill\mbox{(\citeyear[85--86]{Moretti:2019wx})}
  \end{quote}

  So, loosely reconstructed an enthymematic inference involves constituents M1, M2, \(\dots\), Mm of S's perspective which ensure that S has the disposition to believe P1, P2, \(\dots\), Pn.
  And, the way in which M1, M2, \(\dots\), Mm lead to S forming the belief that Q allow S to explain that they believe Q on the basis of P1, P2, \(\dots\), Pn.
  In short, an enthymematic inference is an inference in which may be \emph{post hoc} expanded to some corresponding plain inference (in part) because performing the enthymematic inference requires the agent to be disposed to believe the required premises of the corresponding plain inference.
  And, as such the premises of the corresponding plain inference may be considered as (constitutive of) the basis of S's belief that Q.

  In contrast, \issueConstraint{}/\issueInclusion{} concerns the way in which M1, M2, \(\dots\), Mm lead to S forming the belief that Q do not necessarily require the agent to appeal to P1, P2, \(\dots\), Pn.
  It is consistent with \citeauthor{Moretti:2019wx} account that the reasoning from M1, M2, \(\dots\), Mm to Q may only appeal to premises and steps of reasoning used.
  That Q may be based on P1, P2, \(\dots\), Pn is due to the requirement that S is disposed to believe P1, P2, \(\dots\), Pn and the possibility of S retroactively appealing to Q being a consequence of P1, P2, \(\dots\), Pn.
  Hence, the account does not conflict with \issueConstraint{}/\issueInclusion{}.

  The insight offered is that there does not necessarily need to be a structure preserving mapping between premises and steps providing propositional support for a belief and the premises and steps appealed to when forming the belief.
  However, this does not constrain what the agent appeals to when forming a belief.
\end{note}

\begin{note}
  {
    \color{red}
    What to make of this?

    \textbf{
      It's not from the agent's perspective.
      Instead, it's a reconstruction.
    }

    I mean, from one perspective, \citeauthor{Moretti:2019wx} can understood as being motivated by justification.
    In this sense, there are parallels to \citeauthor{Lehrer:1971aa}.
    Intuition that ethymemetic reasoning doesn't appropriately justify, but at the same time the agent has a justified belief.

    Still, this seems to be a distinct question.
    Why is the agent justified is distinct from why did the agent conclude.

    So, although this is somewhat similar, it's not clear that there's sufficient relation to issues of interest.
  }
\end{note}



%%% Local Variables:
%%% mode: latex
%%% TeX-master: "master"
%%% End: