%%% Local Variables:
%%% TeX-master: "master"
%%% End:


\chapter{Inertia}
\label{cha:inertia}

\section{Inertia}
\label{sec:inertia}

Argument against~\ref{denied-claim} is that it conflicts with \nI{-} in scenarios highlighted by~\ref{prem:ab}.

\begin{enumerate}[label=\nI{}, ref=\nI{}]
\item\label{prem:ni} An agent is not able to obtain support for some proposition \(\psi\) on the basis of information that some the support the agent has for \(\phi\) is misleading or mistaken if \(\psi\) is not the case.
\end{enumerate}

There is lack of an explanatory connexion between support for \(\phi\) and support for \(\psi\).

\section{nI}
\label{sec:ni}

Case in which the support the agent has does not trace from access, or the agent obtains support on the basis that the support the agent does have is misleading.

Note, talk about `obtaining' support.
So, we have a setup where:

\begin{enumerate}
\item Agent has support.
\item Say, for some proposition \(\psi\).
\item The support for \(\psi\) is not, given the agent's doxastic state, also support for \(\psi\).
\item Agent receives information to the extent that the agent's support for \(\psi\) is misleading if \(\psi\) is not the case.
\item Agent holds that the support for \(\psi\) supports \(\phi\) being the case, because else their support is misleading.
\end{enumerate}

Here, the support doesn't do anything for \(\phi\).

This principle is less obvious, but I think fine.

Consider a few good cases and some bad cases.

Good case, deduction.
So, if support for \(\phi\) then support for \(\phi \lor \psi\), because, intuitively, support for \(\phi\) is also support for \(\phi \lor \psi\).
It is puzzling to motivate on the basis that the agent obtains support for \(\phi \lor \psi\) because if \(\phi \lor \psi\) is not the case then their support for \(\phi\) would be misleading.
Of course, this is the case.
However, that is not why the agent has support for \(\phi \lor \psi\).


Suppose quiz.
Interested in answer to a certain question, mix with questions I know the answer to.
Contestant gets the answers correct.
Support that the contestant is good with respect to the domain.
Have support for the question I'm interested in.

Problem if that I hold the answer to the question is such and such because otherwise support for contestant understanding the domain would be misleading.

These are two cases where things look straightforward, and the agent has support by some other route.

If X then support is misleading
Therefore, not-X.

???

If that's not a boat then the radar is faulty.

Have support that the radar is working well.
So, it is a boat.

??? Though in this case it looks as though there's a way to go from functioning radar to boat.

Slight variant.
Checked the radar, support that it's functioning correctly.
Not looking at the screen.

If there's a blip on the radar, then it's not functioning correctly.
Support that it's functioning correct.
So, there's no blip on the radar.



Parked car outside.
If car is stolen then support that this is a safe neighbourhood would be misleading.
Support for car is not stolen.


If the seeds haven't started sprouting yet, then you've been misled about the planting conditions.
Have not been misled about the planting conditions.
So, the seeds have started sprouting.

\begin{note}[Denying~\ref{prem:ni}]
  Worry about coherence.

  If the agent does not obtain support, then in a situation where:
  \begin{itemize}
  \item Support for \(\phi\)
  \item Information that if \(\phi\) is the case then \(\psi\) is the case.
  \item No support for \(\psi\).
  \end{itemize}

  So, agent is committed in some sense to \(\psi\) being the case, given the support they have for \(\phi\) and the information.

  I have support for \(\phi\) and information that \(\phi \rightarrow \psi\), but no support for \(\psi\).

  Looks like a failure of closure.

  However, closure is typically formulated with entailment.
  In each of these cases, there's no entailment.
  Distinction between the support for \(\phi\) and \(\phi\).

  Would need something to the effect of positive attitude only if support.

  However,~\ref{prem:ni} only denies support.
  It does not deny that the agent is require to have some positive attitude toward the proposition.
\end{note}


\section{Motivation for \nI{}}
\label{sec:motivation-ni}

\begin{itemize}
\item Because, the support the agent has is independent of \(A(\psi)\)/\(\psi\).
\end{itemize}

\begin{itemize}
\item An agent is not able to obtain support for some proposition \(\psi\) on the basis of information that the support the agent has for \(\phi\) is misleading if \(\psi\) is not the case.
\end{itemize}

\begin{itemize}
\item The relevant information must also provide some support for \(\phi\).
\item One way of getting to this is by ordering support.
\item If the constraint is established prior to obtaining support, then this may limit support.
\end{itemize}

\begin{itemize}
\item This is also related to Harman.
\item For, the principle there is that one is not in a position to hold that support is going to be misleading.
\item Support for \(\phi\) does not show that future support for \(\psi\) is misleading when \(\psi \vdash \lnot\phi\).
\item Support for \(\phi\) does not show that\dots
\end{itemize}

\begin{itemize}
\item Important to note is that this does not deny closure.
\item First, doxastic.
\item Second, no requirement that the required information is a (known, logical) entailment.
\end{itemize}

Why does witnessing work?

\begin{itemize}
\item Because the agent is not basing things on the support they have.
\item The things about misleading support is that it doesn't say there's anything problematic about the information received.
\end{itemize}

