\begin{note}
  \begin{restatable}[Entanglement]{definition}{defEntanglement}\label{def:CS-depends}
    \(\psi\) having value \(v'\) \emph{is entangled with} claimed support for \(\phi\) having value \(v\) just in case the agent would consider the claimed support for \(\phi\) having value \(v\) to be either \mom{} if \(\psi\) does not have value \(v'\).
  \end{restatable}

  There's nothing too special about a entanglement.
  In particular, there is no requirement that an agent has claimed support for any entangled proposition.
  Simple statement about relation between claimed support for \(\phi\) having value \(v\) and the evaluation of other propositions.

  \autoref{prop:RBV:inheritance} follows from~\autoref{def:CS-depends}.

  \begin{restatable}[]{proposition}{propInheritance}\label{prop:RBV:inheritance}
    Reasoning from \(\phi\) having value \(v\), then:
    \begin{enumerate}
    \item Any claimed support as result depends on \(\phi\) having value \(v\). And:
    \item If agent considers that \(\phi\) having value \(v\) requires \(\xi\) having value \(v''\) then any claimed support as a result (also) depends on \(\xi\) having value \(v''\).
    \end{enumerate}
  \end{restatable}

  Again, a entanglement isn't anything special.

  {
    \color{red}
    Also note that the second part of \ref{prop:RBV:inheritance} is about consideration.
    It doesn't matter whether things actually are this way.
  }
\end{note}

\begin{note}
  \begin{enumerate}
  \item By \ref{nI:going-by-value}, \(\phi\) having value \(v\) is entangled in the claimed support for \(\psi\) having value \(v'\).
  \item Given that \(\phi\) has value \(v\), the agent's claimed support for \(\phi\) having value \(v\) is not \misled{}.
  \item By \ref{nI:inclusion:position} it follows that the agent is in position to claim support for \(\psi\) having value \(v'\).
  \item By \ref{nI:inclusion:bound} the agent consider that they would not be \mistaken{} when claiming support for \(\psi\) having value \(v'\).
  \item Hence, \(\psi\) having value \(v'\) is entangled.
  \end{enumerate}
\end{note}

\begin{note}
  This seems kind of strong.
  Given relation set up by~\ref{nI:inclusion} there's a problem with claimed support for \(\phi\) having value \(v\) and \(\phi\) having value \(v\) as a dependency.

  However, this is not so much of an issue.
  First, the relation set up by~\ref{nI:inclusion} is specific to \(\psi\) having value \(v'\).
  Second, the relationship itself is quite a strong condition.
  This kind of interdependence requires quite a lot.
\end{note}

% % % % % % % % % % % % % % % % % % % % % % % % % % % % % % % % % % %

% % % % % % % % % % % % % % % % % % % % % % % % % % % % % % % % % % %

% % % % % % % % % % % % % % % % % % % % % % % % % % % % % % % % % % %

\subparagraph*{Variant}

\begin{note}
  This is all I need.
  For:
  \begin{itemize}
  \item If \(\phi\) does not have value \(v\) then claimed support for \(\psi\) having value \(v'\) fails.
  \item If claimed support for \(\phi\) having value \(v\) is misled, then \(\phi\) does not have value \(v\).
  \item If the agent would be mistaken when claimed support for \(\psi\) having value \(v'\) then the agent's claimed support for \(\phi\) having value \(v\) would be mistaken.
  \item If \(\psi\) does not have value \(v'\) then the agent would be mistaken when claiming support for \(\psi\) having value \(v'\).
  \end{itemize}
  The key point here is that because the claimed support for \(\psi\) having value \(v'\) depends on \(\phi\) having value \(v\), it also depends on everything that the agent thinks must be the case if \(\phi\) has value \(v\).

  Now, the problem here is that this just establishes a dependency.
  It does not follow from this that there's a violation of \eiS{}.

  This is easy to see, as dependency as defined is reflexive.

  The point is that if \(\phi\) doesn't have value \(v'\) then \(\phi\) doesn't have value \(v\).
  And, \(\phi\) was part of way in which agent claimed support.
  So, \(\phi\) already requires that \(\psi\) has value \(v'\).

  Okay, so if \(\psi\) doesn't have value \(v'\), then \(\phi\) doesn't have value \(v\).
  Hence, the agent doesn't get to appeal to \(\phi\) having value \(v\) to secure \(\psi\) having value \(v'\).

  In short, problematic consequences from the role of \(\phi\) having value \(v\) as a dependency given the perceived relationship between claimed support for \(\phi\) having value \(v\) and claiming support for \(\psi\) having value \(v'\).

  I.e.\ the reasoning does not go through without \(\psi\) having value \(v'\).

  {
    \color{red}
    Note, \nfcs{} is really important here for the possibility of these things obtaining.
    The issue won't arise if the agent is in a position to rule out some of this stuff.
    However, \nI{} itself doesn't apply to such cases.
  }
\end{note}

\begin{note}
  There is a variant of the argument given which adds additional constraints.
  I think this variant is plausible, and given that it adds additional constraints avoids some concerns about \nI{} over-generating failures of claiming support.
  However, requires a premise/conjecture that I feel is too imprecise.

  Briefly sketch.
\end{note}

\begin{note}
  \begin{itemize}
  \item Claimed support is not factive, \nfcs{}.
  \item Not possible to get that \(\phi\) has value \(v\) from claimed support that \(\phi\) has value \(v\).
  \item Hence, moving from claimed support for \(\phi\) having value \(v\) to \(\phi\) having value \(v\) requires assumption that claimed support for \(\phi\) is \nmom{}.
  \end{itemize}

  So, idea is that if possible \mom{} then don't get to go to \(\phi\) via claimed support.

  Of course, agent may assume \(\phi\), but then no account of why \(\phi\) unless this kind of thing is applied.

  \begin{itemize}
  \item In order to `get' \(\phi\) from CS\(\phi\), assume \nmom{}.
  \item Conversely, if do not assume \nmom{}, then don't `get' \(\phi\) from CS\(\phi\).
  \end{itemize}
  This is more or less immediate from \nfcs{}.
  As possible misled, then that one has claimed support does not guarantee \(\phi\).

  \begin{itemize}
  \item Any reasoning from \(\phi\) proceeds under CS\(\phi\) \nmom{}.
  \end{itemize}

  That is, if appealed to CS\(\phi\) in order to secure reasoning from \(\phi\) \dots
\end{note}

\begin{note}
  \begin{itemize}
  \item Move to \(\phi\) having value \(v\).
  \item So, clear that things won't work out if \(\phi\) does not have value \(v\).
  \end{itemize}
\end{note}

% % % % % % % % % % % % % % % % % % % % % % % % % % % % % % % % % % %

% % % % % % % % % % % % % % % % % % % % % % % % % % % % % % % % % % %

% % % % % % % % % % % % % % % % % % % % % % % % % % % % % % % % % % %

\begin{note}[Revised]
  \begin{enumerate}
  \item Going by value.
  \item Inherit constraints on claimed support for \(\phi\).
  \item So, in position to claim support for \(\psi\).
  \item Nothing particularly problematic here, generally speaking --- claiming support is hard.
  \item However, claiming support for \(\psi\).
  \item So, we now have the assumption of being in position as background for claiming support for \(\psi\).
  \item This is where the issue is, though it's not super obvious.
    \begin{enumerate}
    \item Arguing that there's no way to discharge assumption seems fine.
    \item For, nothing available to the agent.
    \item Difficulty is in understanding why this prevents claiming support for \(\psi\).
    \item Simple idea is that being in a position to claim support also requires the agent to hold that \(\psi\) is the case.
      \begin{enumerate}
      \item If this holds, then the agent hasn't done anything of interest.
      \item For, need to assume what one is trying to claim support for.
      \end{enumerate}
    \end{enumerate}
  \item So, from~\ref{nI:inclusion} if the agent goes by value to \(\phi\) then also go by value to \(\psi\).
    For,~\ref{nI:inclusion:bound} requires that the agent holds that they're not mistaken or misled with respect to \(\psi\) if not mistaken or misled with respect to \(\phi\).
    Hence, if agent is going to value \(\phi\), agent must also be going to value of \(\psi\).
  \item Note,~\ref{nI:inclusion:bound} is the main thing here, but it's benign without~\ref{nI:inclusion:position} (there's a note on this above).
  \item So, we've got \(\psi\) already, before even appealing to the conditional\dots
  \end{enumerate}
  So, now, the remaining issue is why having \(\psi\) is bad.
  \begin{enumerate}
  \item Well, goal is to appeal to the conditional.
  \item However, the conditional doesn't establish anything more than the agent has assumed.
  \item And, no way for \(\psi\) not to be an assumption.
  \end{enumerate}
\end{note}

% % % % % % % % % % % % % % % % % % % % % % % % % % % % % % % % % % %

% % % % % % % % % % % % % % % % % % % % % % % % % % % % % % % % % % %

% % % % % % % % % % % % % % % % % % % % % % % % % % % % % % % % % % %

\paragraph{\ref{nI:going-by-value}}


\begin{note}[Background to Limitation]
  The broad target of \nI{} is to identify sufficient conditions for a way of claiming support to be undercut.
  The content of~\ref{nI:going-by-value} is role of \(\phi\) having value \(v\) in claiming support.
  % an account of the particular way of claiming support, and the statement that \emph{that} way of claiming support is undercut given~\ref{nI:claimed-support} and~\ref{nI:inclusion}.

  Restated:

  \begin{quote}
    \begin{enumerate}
    \item[\ref{nI:going-by-value}] \nIClauseValue{}
      \begin{enumerate}[label=\alph*.]
      \item[\ref{nI:going-by-value:phi}] \nIClauseValuePhi{}
      \end{enumerate}
    \end{enumerate}
  \end{quote}

  In this section we clarify the way of claiming support in some detail.
\end{note}

\begin{note}
  Specifically, we will split this section into two parts.
  First, a discussion of the way in which the agent claims support and second, an statement and argument for a proposition that we will appeal to when arguing for \nI{}.
  The proposition is related to the specific way of claiming support described by~\ref{nI:going-by-value}.
  However, the proposition is stated with respect to way of claiming support, rather than specific instance.
\end{note}

\begin{note}[\ref{nI:going-by-value} Restated]
  \ref{nI:going-by-value} describes a particular way of claiming support (and denies that the agent may claim support in such a way if certain conditions are met).
\end{note}

\begin{note}[An instance of \adA{}]
  From~\ref{nI:claimed-support} the agent has claimed support that \(\phi\) has value \(v\).

  Hence, the type of reasoning here is an instance of \adA{}.
  Indeed, the way in which the agent claims support in illustrations conforms to the \hyperref[abstract-adA]{basic (abstract) instance} of \adA{} present in~\autoref{sec:ability-ads-adc}.
\end{note}

\begin{note}
  \color{red}
  The thing that's going to be important here is that the agent claims support for \(\psi\) (in part) from claimed support for \(\phi\).
  And, that the agent goes to the value of \(\phi\).
  The move to value is important because requires that the agent considers that the claimed support for \(\phi\) is \nmom{}.
\end{note}

\begin{note}[Breakdown]
  So, have:
  \begin{enumerate}
  \item Claimed support for \(\phi\) having value \(v\), and
  \item Claimed support that if \(\phi\) has value \(v\) then \(\psi\) has value \(v'\)
  \end{enumerate}
  And, in turn.
  \begin{enumerate}
  \item \(\phi\) has value \(v\), therefore
  \item \(\psi\) has value \(v'\).
  \end{enumerate}
  Two important steps.
  \begin{enumerate}
  \item Move from claimed support for \(\phi\) having value \(v\) to \(\phi\) having value \(v\).
  \item Move from \(\psi\) having value \(v'\) to claiming support that \(\psi\) has value \(v'\).
  \end{enumerate}
\end{note}

\begin{note}[The two key moves]
  Focus now on the two key moves.
\end{note}

\begin{note}[First move]
  Distinction between claiming support that \(\phi\) has value \(v\) and reasoning about \(\phi\) having value \(v\).

  Okay, the key thing here is that this move involves no possibility of misled.
  So, claimed support, fine with possible defeaters, expect that not misled.

  This is the key thing to keep in mind.
  Important point is that the agent requires a bunch of stuff to hold that they have not necessarily claimed support for.
\end{note}

% \begin{note}[Familiar from literature]
%   Zebra.
%   Light bulb.

%   Exhibit this basic problem.
%   There's no problem with the agent claiming support, but if the agent then goes by value they get more.
%   In these cases, knowledge does a lot of the work (and we'll see why different later).
%   Strengthens the problem, as knowledge is taken to be factive.
%   However, we don't necessarily have this issue with respect to \nI{}.
%   Here, more general, same move works with respect belief, etc.
% \end{note}

\begin{note}[Second move]
  Claiming support for \(\psi\) having value \(v'\).

  Issue is a continuation of the previous.
  By reasoning from \(\phi\) having value \(v\) the agent has gone beyond claimed support.
  And, given that the agent has got a whole bunch of stuff from \(\phi\) having value \(v\), the agent needs to resolve these issues as the agent isn't thinking that these are the case when they claim support for \(\psi\) having value \(v'\).

  Look, this works a lot of the time, so there's some way of understanding this.
  With respect to \nI{}, interest is in whether this trivialises the reasoning.
\end{note}

% \begin{note}[Example]
%   Simple way to trace this is by thinking about Tarskian treatment of quantifiers.\nolinebreak
%   \footnote{
%     Conditionals are similar, but distinct.
%     Antecedent is true, and then discharge assumption.
%     Here, typically no claimed support that the antecedent is true.
%   }
%   Explored above.
%   Here, \adB{}.
%   Update the assignment function to point to a specific individual.
%   Use this to then reason as normal.
%   However, when we're done, discharge this assumption.
% \end{note}

% \begin{note}
%   Slightly different example, here, where things seem fine.
%   \begin{illustration}
%     Searching for YYY in the building.
%     If I search, there are various defeaters I expect not to obtain.
%     Ask secretary.
%     If not in LLL, then not in building.
%     Now I only need to check LLL.
%     From this, \RBV{} and then claim support for potential defeaters of obscure places.
%     Why?
%     Because regardless of my reasoning, XXX not being in LLL entails that XXX in not in the building.
%     Given this type of value information, establish an also-is relation of support.
%     XXX not in LLL also-is support for XXX not in building.
%     Without, don't get this.
%   \end{illustration}
%   Here, the searching the room is fine.
%   However, now extend by info from uketuke.
%   This then no defeaters from uketuke.
%   I mean, the agent has only checked one room.

%   Claim support, and here it seems that claimed support now rests on expectation that these defeaters don't hold.
% \end{note}

\begin{note}[Other types of reasoning]
  \RBV{} isn't the only way to reason.
  Key is that first step, often don't need it to be the case that one requires that possible defeaters don't hold.

  Simple example:
  If you've claimed support for \(\phi\), then \(\psi\).

  The agent doesn't need to reason with \(\phi\) having value \(v\).

  Handful of examples
\end{note}

\begin{note}
  Consider again the rectangle.
  The rectangle has certain dimensions, and calculations are only relevant given dimensions.
  So, the reasoning proceeds independently.
  There's nothing in the core part of the reasoning that requires the rectangle itself to have the noted dimensions.
  At no point do I need to appeal to it being true.
  Rather, I background that it is true, and derive additional results.
  {
    \color{red}
    I offer a different perspective on this.
  }

  Contrast, if the alarm is beeping then there is a fire.
  It matters whether or not the alarm really is beeping.

  % Here, a useful illustration is something like reliable then reliable on this instance.
  % In this type of reasoning, it seems there's something missing between \(\phi\) and \(\psi\).
  % So, this wouldn't be a case of \RBV{}.
  % That is, there's something additional going on with \(\phi\) = reliable in general to \(\psi\) = true on this occasion.
  % Indeed, it seems that depending on how \(\psi\) is understood, then either no \ref{nI:received-info} and hence no \RBV{}, or no \ref{nI:inclusion}, because a single failure is not sufficient to raise a problem with \(\phi\).
\end{note}

\begin{note}
  Finally, a variant on the zebra and red wall cases.

  It doesn't look as the wall \dots
  And, it doesn't look as though it's a cleverly disguised mule.

  This is admittedly an odd proposition.
  So, it seems plausible to obtain this by reasoning.
  A cleverly disguised mule doesn't look like a cleverly disguised mule --- it's cleverly disguised.

  Almost trivial.

  Important here is that no need to \RBV{}.
  Indeed, \RBV{} is strange.
  Isn't at all obvious why not being a cleverly disguised mule would allow one to claim support that it doesn't look like a cleverly disguised mule.

  Of course, the issue is that it's hard to do anything interesting with these propositions.
  Doesn't seem that the claimed support even permits \RBV{}.
  However, further exploration of this topic would go beyond present interest, so we'll stop here.
\end{note}




\paragraph*{Summary}

\begin{note}[Summary]
  In summary, we've expanded on the type of reasoning that~\ref{nI:going-by-value} captures.
  We termed this type of reasoning `\RBV{-}' as the distinguishing feature is reasoning from a proposition having a certain value.
  Two things which follow from this:
  \begin{itemize}
  \item Requires expectations to hold.
  \item And, this requires some care when claiming support for \(\psi\) having value \(v'\).
  \end{itemize}
  So, the task of \ref{nI:inclusion} is to narrow down the conditions to a specific problem --- given~\ref{nI:claimed-support}.

  Broadly speaking, {\color{red} undercutting defeater}.
p  So, \RBV{} won't do for claiming support.
  After discussing~\ref{nI:inclusion} in some detail, we'll turn to the argument.
\end{note}