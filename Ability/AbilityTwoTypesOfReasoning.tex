\chapter{Two ways of claiming support}
\label{sec:two-ways-of-claiming-support}

\section{\adA{} and \adB{}}
\label{sec:ability-ads-adc}

\begin{note}[Recall\dots]
  Let us briefly summarise our progress so far.

  In~\autoref{sec:cases-interest} we introduced particular instances of an agent claiming support for some conclusion which involved two key steps.
  The reasoning, in outline:
  \begin{enumerate}[label=\arabic*., ref=(\arabic*)]
  \item\label{NUR:ro:i} I have some general ability \(\gamma\).
  \item\label{NUR:ro:ii} If I have general ability \(\gamma\) then I have some specific ability \(\varsigma\) to \emph{V} that \(\phi\).
  \item\label{NUR:ro:iii} I have the specific ability \(\varsigma\) to \emph{V} that \(\phi\). \hfill (From~\ref{NUR:ro:i} and~\ref{NUR:ro:ii})
  \item\label{NUR:ro:iv} It is only possible to \emph{V} that \(\phi\) if \(\phi\) is already the case.
  \item\label{NUR:ro:v} \(\phi\) is the case. \hfill (From~\ref{NUR:ro:iii} and~\ref{NUR:ro:iv})
  \end{enumerate}

  The reasoning involves claiming support by two important steps.

  \begin{itemize}
  \item The first step, from~\ref{NUR:ro:i} and~\ref{NUR:ro:ii} to~\ref{NUR:ro:iii}, involves the conditional of~\ref{NUR:ro:ii}, termed `\gsi{0}', clarified in~\autoref{sec:type-scenario}. And,
  \item The second step, from~\ref{NUR:ro:iii} and~\ref{NUR:ro:iv} to~\ref{NUR:ro:v}, is an instance of `\aben{an}', clarified in~\autoref{sec:ability-entailment}.
  \end{itemize}

  Both steps involve reasoning, and in particular claiming support, by appeal to ability.
  The first step, claiming support from a specific ability from a general ability.
  The second step, claiming support for some proposition from specific ability.

  Issue is how the agent claims support.
  In turn, how an agent reasons with ability.
  The sketch captures key premises and steps, but does not provide an interpretation of those steps.

  In~\autoref{sec:wr-ar} we introduced two (schematic) interpretations of specific ability --- \AR{} and \WR{}.
  A few purposes for these (schematic) interpretations.
  First, some insight into how an agent may claim support.
  \AR{} some property, \WR{} some event.
  No stance on these.
  Distinction on one hand allows us to state in greater detail, and on the other hand ensures that the arguments to follow do not presuppose a particular (schematic) interpretation of ability.

  Now, two steps under either \AR{} or \WR{} may look straightforward.
  Both involve conditionals, so matter of something like \emph{modus ponens}.
  In this respect, distinction between \AR{} and \WR{} matters only for finer details of how the agent claims support, rather than the reasoning involved.
  `Something like \emph{modus ponens}' sufficiently similar so that it's the conditional that's doing the work.
  In the sense that the conditional form does most of the work.
  \AR{} and \WR{} why it's claiming support as opposed to some other kind, say subjunctive or suppositional, reasoning.

  However, this is not immediate.
  The sketch does not provide an interpretation of those steps.
  The purpose of this section is to outline an alternative way of claiming support that may be applied to the sketch.
  Key part in argument against \ESU{} (and for \EAS{}).

  Start with \illu{0}.
  Then, definition.
  Applied to further \illu{1}.
  Applied to \AR{} and \WR{}.


  Distinction between reasoning \adA{} and \adB{} is of interest to use with respect to these two instances of claiming support in particular.

  So, pair \AR{} and \WR{} with \adA{} and \adB{} and we have various ways of understanding how agent claim support.

  Keep in mind that here we're elaborating on what this sketch amounts to.
\end{note}



\subsection{Applied to ability}
\label{sec:applied-ability}

\subsubsection{\adA{}}
\label{sec:ads}

\subsubsection{\adB{}}
\label{sec:adc}

\begin{note}
  Core idea is that claim support by what follows from ability.

  Helpful to highlight parallel distinction with respect to other instances of claiming support.
\end{note}

\begin{note}
  Similar to verifying an algorithm may be implemented.
  Break down all of the steps in the algorithm, and then ensure that it is possible to express each of the steps in the programming language of choice.

  \begin{quote}
    \textsc{factorial}(\(n\)):\newline
    \textbf{if} \(n = 1\)\newline
    \mbox{}\indent \textbf{return} \(1\)\newline
    \textbf{else}\newline
    \mbox{}\indent \textbf{return} \(n \times\) \textsc{factorial}(\(n-1\))
  \end{quote}

  Fortran 77 does not support recursion, a function may not call an instance of itself.\nolinebreak
  \footnote{
    This is not to say that one may not compute factorials using Fortran 77.
    It's a Turing complete language.
    However, would require a different (non-recursive) algorithm.
  }
  By contrast, the recursive factorial algorithm may implemented in languages that support recursion, such as Lisp or Python.

  \adA{} and \adB{}, recursion, or function calling an instance of itself.

  \AR{}, properties of the language, \WR{} adds in particular event.
\end{note}

\begin{note}
  Turning to reasoning, very similar idea.
  Features of programming languages are resources for doing something, in the same way that premises and steps of reasoning are resources for reaching some conclusion.
\end{note}

\section{Recap}
\label{sec:recap-reasoning}

\begin{note}
  Ability.
  Instance of reasoning with ability.
  Two distinctions which apply here.
  \AR{} and \WR{} for ability.
  \adB{} and \adA{} for how ability is used to claim support.
\end{note}

\begin{note}[Two ways to get to \(\phi\)]
  Two key steps.
  \begin{itemize}
  \item \gsi{}.
  \item \aben{the}.
  \end{itemize}

  First, general to specific, then specific to \(\phi\).

  Second, general to specific, then general to \(\phi\).
\end{note}

\begin{note}
  Second is something like evidence of evidence is evidence.

  Here, the important difference is that the agent only needs to appeal to general ability.
  And, they've claimed support for this.

  The point is that it's not clear the agent is required to do anything too much with the specific ability.
\end{note}

\begin{note}[Focus]
  Common here is \gsi{}.
  Needed in both cases.

  However, it's also the case that the `final' bit of reasoning is more or less \aben{the}.

  So, in a sense both ways of reasoning depend on these two things.

  The only difference is the particular for \aben{the} would take.

  There is a difference.
  For, general and specific are different.
  I'm not clear on whether this amounts to a significant difference.
  Still, even if it does, argument will proceed even if there is something significant to be made of this.
\end{note}

\begin{note}[Summarising]
  Above we introduced \gsi{}.
  Limited information of the form `If \emph{S} has a (general) ability to \(\gamma\), then \emph{S} has a (specific) ability to \emph{V} that \(\phi\) (as an instance of the general ability).'
  We then noted that certain instances of the (specific) ability to \emph{V} that \(\phi\) entail that \(\phi\) is the case.
  Two interpretations of \aben{the}, \AR{} and \WR{}.

  Our focus now turns back to \gsi{}.
  For those instances of \gsi{} when \aben{the} holds, the interpretations \AR{} and \WR{} detail what the agent obtains by reasoning from general to specific ability.
  In other words, \emph{what} the agent is claiming support for.

  As noted, using a conditional such as \gsi{} is not automatic.
  The informer has not provided the agent with any additional way to claim support that the agent has the general ability.
  Rather, outlined something that follows \emph{if} the agent has the general ability.

  So, it is up to the agent to resolve in either way.
  If the agent wants to use the information, then the agent needs to reason from general to specific.
  The issue is that without any additional reasoning, it seems there's no clear way to determine which way the agent should go.
  Here is where the distinction between \AR{} and \WR{} is important.
  Interpretation of specific ability informs how the agent move from general to specific.

  Following two propositions outline combination.
  {
    \color{red}
    The key thing here is about claiming that one has a specific ability.
  }
\end{note}

\begin{note}[Unified idea]
  Claim support for premises and steps of reasoning.

  Easiest with \WR{}.
  What's missing here is the use of the premises in reasoning.
  Hence, contrast to \ESU{} which we'll talk in some detail about below.

  Same applies to \AR{} by general property reduction.

  So, \AR{} and \WR{} allow the agent to do the same thing, but in slightly different ways.
\end{note}

\begin{note}[\gsi{}++]
  First, \gsi{} applied to \AR{}
  \begin{restatable}[\textsf{|gs-I\space·\space H|}]{idea}{ideaCSbyAR}\label{idea:CS-by-AR}
    % In order for \emph{S} to have the (specific) ability to \emph{V} that \(\phi\) for which \aben{the} holds, claimed support for general and claimed support for \gsi{} are sufficient to claim support that \emph{S} has the property of being able to \emph{V} that \(\phi\).
    Suppose an agent has:
    \begin{enumerate}
    \item Claimed support for some general ability \(\gamma\).
    \item Claimed support that if they have the general ability \(\gamma\) then they have some specific ability to \emph{V} that \(\phi\) (for which \aben{the} holds).
    \end{enumerate}
    Then:
    \begin{enumerate}[resume]
    \item \emph{S} may claim support for having the specific ability \(\sigma\) by reasoning that they have the property of being able to \emph{V} that \(\phi\).
    \end{enumerate}
    \vspace{-\baselineskip}
  \end{restatable}

  Second, \gsi{} applied to \WR{}

  \begin{restatable}[\textsf{|gs-I\space·\space W|}]{idea}{ideaCSbyWR}\label{idea:CS-by-WR}\label{W:s}
    % In order for \emph{S} to have the (specific) ability to \emph{V} that \(\phi\) for which \aben{the} holds, claimed support for general and claimed support for \gsi{} are sufficient to claim support that there is a potential witnessing event in which \emph{S} \emph{V}s that \(\phi\).
    Suppose an agent has claimed support for some general ability \(\gamma\) and has claimed support that if they have the general ability \(\gamma\) then they have some specific ability to \emph{V} that \(\phi\) for which \aben{the} holds.
    Then, an agent may claim support for having the specific ability \(\sigma\) by reasoning that there is a potential witnessing event in which \emph{S} \emph{V}s that \(\phi\).
  \end{restatable}
\end{note}

\begin{note}[Alternatives]
  Appeal to premises and steps is not required by either \AR{} or \WR{}.
  However, most plausible account of what is going on.

  Explored some alternatives for \AR{}, but unclear what is of importance other than reasoning, and hence premises and steps.
  And, in this respect, basic \AR{} seems like a dead end.
  Premises and steps allow the agent to claim support in the same way as they would allow the agent to claim support when used in reasoning.
  It's not at all clear to me that basic \AR{} makes sense from this perspective.
\end{note}

\begin{note}[Limitation of intuition]
  Focused on idea that claiming support in same way as reasoning.

  This is not to imply equivalence of claimed support.

  Said too little about claimed support to make any strong remarks about equivalence.
  Still, intuitive that additional way of being \mom{}.
  For, haven't done the reasoning, so \mom{} about this.
  Not the case if the agent has done the reasoning.
\end{note}

\subsubsection{Old notes}

\begin{note}[\gsi{}++ applied : \AR{}]
  \AR{} doesn't need to much expansion.
  Silent on what the property is.
  One way to view is that general ability reduces to sufficient collection of specific.
  \gsi{} conditional informs the agent that specific instance, so required for general ability.
  \gsi{} is novel, but support claimed is for quantifier over all core instances.

  Similar to a standard induction principle.

  With respect to chess, this is one such principle.
\end{note}

\begin{note}[\gsi{}++ applied : \WR{}]
  \WR{} is different.
  Witnessing event.
  So, \emph{V}ing that \(\phi\).
  Break down \emph{V}ing that \(\phi\) into a series of actions performed by the agent.
  General ability secures performing each of those actions.

  Turning to the chess example.
  Here, appeal to sufficient understanding of the rules of chess, and the combination of these.
  More broadly, premises and steps of reasoning.

  It's this kind of stuff that \WR{} uses.
\end{note}

\begin{note}[Impact of distinction]
  Return to the impact of the distinction.

  \AR{}, focus on generator.
  Hence, task is to establish that the agent has resources to generate.
  So, in a sense, with \gsi{} we go from existence of general to existence of specific generator.
  Note, this isn't to say that there's something like a general generator.
  It may be the case by general ability we have quantification over specific abilities.
  If so, then claim support that there's a particular specific generator.
  Nor that specific generators are unique.
  The available resources may overlap.
  Still, some thing that is true of the agent, and claimed support for general ability is sufficient to claim support that the `some thing' holds.

  \WR{}, focus on generated event.
  So, agent doesn't necessarily need to establish a generator, but rather ensure that event may be generated.
  Hence, \gsi{}, general ability allows the agent to generate witnessing event for specific ability.
  Not looking to claim support that `some thing' is true of the agent.
  Rather, claimed support for general ability, and appeal to the actions that this allows the agent to perform.
\end{note}

\begin{note}[Intuition for \AR{} and \WR{}]
  Both \AR{} and \WR{} are ways to understand \aben{the}, which is in turn about what is entailed by an agent having a (certain kind of) specific ability.

  \AR{} focuses on the idea that the agent may claim support from having the attribute (or the truth) of the specific ability.
  \AR{} requires support for attribute, which in turn suggests in a position to claim support for premises and steps.
  \AR{} doesn't require agent to claim support for premises and steps.

  \WR{} focuses on the idea that the agent may claim support from witnessing (or using) the specific ability.
  \WR{} requires support for premises and steps, which in turn suggests in a position to claim support for attribute.
  \WR{} doesn't require agent to claim support for attribute.
\end{note}

\begin{note}[Quite brief]
  Sketches of \AR{} and \WR{} are brief.
  Expand on these in the following sections (\ref{sec:first-conditional} and~\ref{sec:second-conditional}) to some extent.
\end{note}

\begin{note}
  Role of ability to secure witnessing event.

  The distinction may be highlighted by a distinct set of implications\nolinebreak
  \footnote{
    Though not necessarily entailments.
  }
  \nagent{4} is dehydrated, so \nagent{4} is tired.
  \nagent{4} took a long walk in the sun, so \nagent{4} is tired.

  First, implication follows from some property.
  Second, implication follows from the result of some action.

  As with ability, both implications may be true.
  Still, difference in terms of whether one appeals to some property of \nagent{4}, or some action that \nagent{4} performed.
  As with ability, there is some ambiguity.
  There's the fact that \nagent{4} took a walk in the sun, and there's the action of \nagent{4} taking a walk in the sun.
\end{note}

\begin{note}[Why]
  So, \AR{}, some property.
  With \WR{}, it's the event that matters.
  In turn, moving from premises to some conclusion.
  Appeal to the event involves appeal to constituents of event.

  Return to \ESU{}.
  No inherent conflict with either \AR{} or \WR{}.
  Difference between property and witness.
  Requirement is that claimed support for premises is sufficient to claim support for conclusion.
  With \AR{}, claimed support for property --- need enough to be sure property is adequate.
  With \WR{}, claimed support for witness --- need enough to make sure that witness is adequate.

  \nagent{4} is thirsty, no implication.
  \nagent{4} walked, no implication.

  Does not matter that thirst is part of the relevant instance of being dehydrated.
  Nor that the witnessing event of walking was a long walk in the sun.
  Deny claimed support as did not reason from such premises.
\end{note}

\subsection{Summary of distinctions}
\label{sec:summary-distinctions}

\begin{note}
  \begin{figure}[H]
    \centering
    \saMtxInterpreted{}
    \repeatCaption{fig:saMtxInterpreted}{Distinction matrix with \aben{the}}
  \end{figure}
\end{note}

\begin{note}
  Basic \AR{} with \adB{} has `?'.
  For, as noted above it's not clear what this amounts to.
\end{note}

\subsection{The distinctions are (sufficiently) exhaustive}
\label{sec:ar-wr-are}

\begin{note}
  Two pairs.
  \AR{} and \WR{}, \adA{} and \adB{}.

  Goal is to argue that:
  \begin{itemize}
  \item \AR{} and \WR{} are exhaustive.
  \item \adA{} and \adB{}, sufficient, as relative to \(\phi\).
  \end{itemize}
\end{note}

%%% Local Variables:
%%% mode: latex
%%% TeX-master: "master"
%%% End: