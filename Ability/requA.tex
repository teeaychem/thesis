\chapter{\requ{3}}
\label{cha:requs}

\begin{note}
  This chapter introduces the idea of a \(\pv{\psi}{v'}\) being a \fc{} from \(\Psi\) as a \requ{1} of an event in which an agent concludes \(\pv{\phi}{v}\) from \(\Phi\).

  The role of \requ{1} is to capture a core feature of failures of \issueConstraint{}.
  Following the introduction of \requ{1} we provide a detailed example of a \requ{0} in isolation from \issueConstraint{} and then clarify the connexion between \requ{1}, \qWhyV{}, and \issueConstraint{}.

  Still, the existence of \requ{1} does not presuppose \issueConstraint{} fails to hold.%
  \footnote{
    I.e.\ one may jointly hold:
    \begin{enumerate*}[label=(\alph*)]
    \item
      various \requ{1} exist, and
    \item
      answers to \qWhyV{} are constrained by answers to \qHowV{} via \issueConstraint{}
    \end{enumerate*}%
    .
    For more, see \autoref{prop:requ-not-n-ce} on \autopageref{prop:requ-not-n-ce}, below.
  }
  Key to failure the failure of \issueConstraint{} is the existence of \requ{1} \emph{such that} \issueConstraint{} fails to hold.
  We suggest provide a plausible example at the end of the chapter.
  However, the argument for \requ{1} such that \issueConstraint{} fails to hold takes place in \autoref{cha:typical}.
\end{note}



\section{\requ{3}}
\label{cha:requs:requs}

\begin{note}
  \begin{definition}[A \requ{0}]%
    \label{def:requ}%
    For any event \(\ed{}\) such that \(\ed{}\) is an event in which \vAgent{} concludes \(\pv{\phi}{v}\) from \(\Phi\):
    %
    \begin{itemize}
    \item
      \(\pv{\psi}{v'}\) being a \fc{} from \(\Psi\) for \vAgent{} throughout \(\ed{\flat}\) is a \emph{\requ{}} of \(\ed{}\).
    \end{itemize}

    \emph{If and only if}

    \begin{itemize}
    \item
      Clauses \ref{def:requ:se} and \ref{def:requ:fc} hold:
      \begin{enumerate}[label=\Alph*., ref=\Alph*]
      \item
        \label{def:requ:se}
        \(\ed{\flat}\) is a \se{} of \(\ed{}\).
      \item
        \label{def:requ:fc}
        \(\pv{\psi}{v'}\) is a \fc{} from \(\Psi\) for \vAgent{} throughout \(\ed{\flat}\).
      \end{enumerate}
    \end{itemize}
    \vspace{-\baselineskip}
  \end{definition}

  \noindent%
  The definition of a \requ{} captures a particular instance of \progEx{} (\autopageref{obs:PE}).
  Specifically, when there is some \se{} \(\ed{\flat}\) of \(\ed{}\) such that \(\ed{\flat}\) is a \se{} of \(\ed{}\) \emph{only if} \(\pv{\psi}{v'}\) is a \fc{} from \(\Psi\) for the agent throughout \(\ed{\flat}\).
\end{note}


\begin{note}
  Expanded, the definition of a \requ{} concerns an event \(\ed{\flat}\) in which:
  %
  \begin{enumerate}[label=(\alph*), ref=(\alph*)]
  \item
    A conclusion of \(\pv{\phi}{v}\) from \(\Phi\) is in progress, and
  \item
    If the agents conclusion of \(\pv{\phi}{v}\) from \(\Phi\) in progress during \(\ed{\flat}\) happens, then the conclusion happens as a result of \(\ed{\flat}\).
  \item
    \(\pv{\psi}{v'}\) being a \fc{} from \(\Psi\) is required for \(\ed{\flat}\) to be an event such that a conclusion of \(\pv{\phi}{v}\) from \(\Phi\) is in progress.
  \end{enumerate}
  %
  The conditions capture the key way in which a \ros{} answers \qWhyV{}.
  For, with the definition of a \requ{} and \autoref{prop:requ-WhyV} in hand, we only need to identify a \requ{} in order to find an answer to \qWhyV{}:

  \begin{proposition}[\requ{3} and \qWhyV{}]
    \label{prop:requ-WhyV}
    \vspace{-\baselineskip}
    \begin{itenum}
    \item[\emph{If}:]
      Conditions~%
      \ref{def:rCs:C}~and~\ref{def:rCs:Cing}~jointly hold:
      \begin{enumerate}[label=\arabic*., ref=\arabic*]
      \item
        \label{def:rCs:C}
        \(\ed{}\) is an event in which \vAgent{} concludes \(\pv{\phi}{v}\) from \(\Phi\).
      \item
        \label{def:rCs:Cing}
        There is some event \(\ed{\flat}\) such that:
        \begin{itemize}
        \item
          \(\pv{\psi}{v'}\) being a \fc{} from \(\Psi\) for \vAgent{} throughout \(\ed{\flat}\) is a \requ{} of \(\ed{}\).
        \end{itemize}
      \end{enumerate}
    \item[\emph{Then}:]
      A \ros{0} between \(\pv{\psi}{v'}\) and \(\Psi\) answers \qWhyV{}.
    \end{itenum}
    \vspace{-\baselineskip}
  \end{proposition}

  \noindent%
  The argument for \autoref{prop:requ-WhyV} revisits part of the argument made when we connected a \ros{} to a \se{} with respect to \qWhyV{} and \autoref{illu:gist:roots:F} in \autoref{cha:var}.

  \begin{argument}{prop:requ-WhyV}
    Assume conditions~\ref{def:rCs:C}~and~\ref{def:rCs:Cing}~jointly hold.

    From Condition~\ref{def:rCs:C}, \(\ed{}\) is an event in which \vAgent{} concludes \(\pv{\phi}{v}\) from \(\Phi\).
    So, \qWhyV{} applies to \(\ed{}\).

    Our task is to show the conditional of \qWhyV{} is true.

    Let \(\ed{\flat}\) by an event which satisfies the existential of Condition~\ref{def:rCs:Cing}.
    So, \(\ed{\flat}\) is such that:
    \begin{itemize}
    \item
      \(\pv{\psi}{v'}\) being a \fc{} from \(\Psi\) for \vAgent{} throughout \(\ed{\flat}\) is a \requ{} of \(\ed{}\).
    \end{itemize}

    Further, suppose a \ros{} between \(\pv{\psi}{v'}\) and \(\Psi\) fails to hold for \vAgent{} through \(\ed{\flat}\).

    Now, as an instance of \supportII{}:
    %
    \begin{itenum}
    \item[\emph{If}:]
      \(\pv{\psi}{v'}\) is a \fc{0} from \(\Psi\) for \vAgent{} throughout \(\ed{\flat}\).
    \item[\emph{Then}:]
      A \ros{0} between \(\pv{\psi}{v'}\) and \(\Psi\) holds for \vAgent{} throughout \(\ed{\flat}\).
    \end{itenum}
    %
    \noindent%
    Hence, as a \ros{0} between \(\pv{\psi}{v'}\) and \(\Psi\) \emph{fails} to hold for \vAgent{} throughout \(\ed{\flat}\), it follows that \(\pv{\psi}{v'}\) is \emph{not} a \fc{0} from \(\Psi\) for \vAgent{} throughout \(\ed{\flat}\).

    And, as \(\ed{\flat}\) satisfies the existential of \ref{def:rCs:Cing} and is \(\pv{\psi}{v'}\) is \emph{not} a \fc{0} from \(\Psi\) for \vAgent{} throughout \(\ed{\flat}\), it follows that \(\ed{\flat}\) is not, or does not develop into, \(\ed{}\).
    This contradicts \(\ed{\flat}\) being a \se{0} of \(\ed{}\), and hence contradicts Condition~\ref{def:rCs:Cing}.
  \end{argument}
\end{note}



\section{An \illu{0}}
\label{sec:some-requ1-beginnote}


\begin{note}
  We work through a \scen{} to illustrate the definition of a \requ{}.
  We focus on identifying a single \requ{}, though the argument presented generalises to identify additional \requ{1} in the same \scen{0}.

  \begin{scenario}[Lucas numbers]%
    \label{scen:LucasNums}%
    The Lucas numbers are recursively defined as follows:%
    \footnote{
      Starting at \(2\), see \hyperlink{cite.OEIS.:aa}{OEIS sequence A000032} for more details.
    }
    %
    \[
      L_{n} = \left\{
        \begin{array}{ll}
          2 & \text{if } n = 0 \\
          1 & \text{if } n = 1 \\
          L_{n-1} + L_{n-2} & \text{if } n > 1 \\
        \end{array}
      \right.
    \]
    %
    An agent is alone and quite bored.
    The agent sets out to calculate the first ten Lucas numbers by applying the recursive definition.
    %
    The agent begins:
    %
    \[
      \begin{array}{cccccc}
        L_{0} & L_{1} & L_{2} & L_{3} & L_{4} & \cdots \\
        \hline
        2 & 1 & 3 & 4 & 7 & \cdots \\
      \end{array}
    \]
    %
    And, eventually concludes:
    %
    \[
      \pv{\propI{The first ten Lucas numbers are 2, 1, 3, 4, 7, 11, 18, 29, 47, and 76}}{\valI{True}}
    \]
    %
    From some \pool{} \(\Phi\) which captures the \agents{} understanding of the Lucas numbers.
  \end{scenario}

  \noindent%
  The event of interest is \(\ed{}\) is when the agent concludes \pv{\propM{L_{[:9]} = [2, \dots, 76]}}{\valI{True}} from \(\Phi\), after studying the definition of Lucas numbers.%
  \footnote{
    \propM{L_{[:9]} = [2, \dots, 76]} shortens \propI{The first ten Lucas numbers are 2, 1, 3, 4, 7, 11, 18, 29, 47, and 76}.
  }

  \begin{observation}[\requ{3} of \autoref{scen:LucasNums}]%
    \label{obs:LucasRequ}%
    There is some \se{} \(\ed{\flat}\) of \(\ed{}\) such that:
    %
    \begin{itemize}
    \item
      \(\pv{\propM{L_{8} = 47}}{\valI{True}}\) being a \fc{} from \(\Phi\) throughout \(\ed{\flat}\) is a \requ{0} of \(\ed{\flat}\).
    \end{itemize}
    \vspace{-\baselineskip}
  \end{observation}

  \noindent%
  The motivation for \autoref{obs:LucasRequ} parallels the way in which we connected a \ros{} to a \se{} when we applied \qWhyV{} to \autoref{illu:gist:roots:F} in \autoref{cha:var} (\autopageref{qWhyV:ex:con}).

  \begin{motivation}{obs:LucasRequ}
    \(e_{d}\) is such that the agent concludes \pv{\propM{L_{[:9]} = [2, \dots, 76]}}{\valI{True}} from \(\Phi\) by the recursive definition.

    As \propM{L_{[:9]} = [2, \dots, 76]} contains the first nine Lucas numbers, there must be some sub-event in which the agent concludes \(\pv{\propM{L_{9} = 76}}{\valI{True}}\) by adding \(L_{8}\) and \(L_{7}\).
    For, the agent does not have a method to obtain \(L_{9}\) without \(L_{8}\) and \(L_{7}\), given the agent is reasoning by the recursive definition.

    And, in turn, there must be a sub-event in which the agent concludes \(\pv{\propM{L_{8} = 47}}{\valI{True}}\) from \(\Phi\).
    Likewise for \(L_{7}\), \(L_{6}\), and \(L_{5}\).

    I particular, consider the event \(\ed{\flat}\) just after the agent has concluded \(\pv{\propM{L_{6} = 18}}{\valI{True}}\) and \(\pv{\propM{L_{5} = 11}}{\valI{True}}\).
    \medskip

    Suppose \(\ed{\flat}\) is a \se{} of \(\ed{}\).
    Our task is to show \(\pv{\propM{L_{8} = 47}}{\valI{True}}\) is a \fc{} from \(\Phi\) throughout \(\ed{\flat}\).

    By assumption, \(\ed{\flat}\) is a \se{} of \(\ed{}\).
    Hence, \(\ed{\flat}\) is such that the agent is concluding \pv{\propM{L_{[:9]} = [2, \dots, 76]}}{\valI{True}}.
    As observed the agent must conclude \(\pv{\propM{L_{8} = 47}}{\valI{True}}\) in order to conclude \pv{\propM{L_{[:9]} = [2, \dots, 76]}}{\valI{True}}.
    So, as agent is concluding \pv{\propM{L_{[:9]} = [2, \dots, 76]}}{\valI{True}} and has not yet concluded \(\pv{\propM{L_{8} = 47}}{\valI{True}}\), the agent must also --- in part --- be concluding \(\pv{\propM{L_{8} = 47}}{\valI{True}}\).
    \medskip

    Now, by \assuPP{} an agent is concluding \(\pv{\propM{L_{8} = 47}}{\valI{True}}\) by the recursive definition there is some possible event in which the agent concludes \(\pv{\propM{L_{8} = 47}}{\valI{True}}\) by the recursive definition.
    Hence, there must be some action available to the agent such that the agent adds \(L_{6}\) and \(L_{5}\) to get \(L_{7}\) and then adds \(L_{7}\) and \(L6\) to get \(L_{8}\).
    And, as \(\pv{\propM{L_{8} = 47}}{\valI{True}}\) follows from the \agents{} understanding of the Lucas numbers, the agent concludes \(\pv{\propM{L_{8} = 47}}{\valI{True}}\) from \(\Phi\) when the agent does the relevant action.
    Therefore, \(\pv{\propM{L_{8} = 47}}{\valI{True}}\) is a \fc{} from \(\Phi\).
  \end{motivation}

  \noindent%
  With minor adjustments the motivation for \autoref{obs:LucasRequ} also shows \(\pv{\propM{L_{7} = 29}}{\valI{True}}\) being a \fc{} from \(\Phi\) throughout \(\ed{\flat}\) is a \requ{0} of \(\ed{\flat}\).
  And, likewise for \(L_{6}\), \(L_{5}\), and so on until the base cases of \(L_{1}\) and \(L_{0}\), as these are given by the \agents{} understanding of the Lucas numbers.

  In parallel, \pv{\rootsCon{}}{\valI{True}} being a \fc{} from \(\Phi\) through \(\ed{\flat}\) is a \requ{} for \(\ed{}\) (being an event in which the agent concludes \pv{\rootsCon{}}{\valI{True}} from \(\Phi\)) with respect to \autoref{illu:gist:roots:F} by the argument given in \autoref{cha:var}.
\end{note}

\begin{note}
  A \requ{} is nothing special.
  It's an observation about what is.
\end{note}

% \paragraph*{Lost keys}

% \begin{note}
%   \requ{} does not require an event has happened.
%   In particular, consider development of an event.
%   \requ{} helps observe an event is not such that concluding.

%   \begin{scenario}[Lost keys]%
%     \label{illu:lost-key}%
%     An agent thinks they may have lost their keys.
%     They usually leave place my keys on the right side of their desk, near a copy of~\citeauthor{Vickers:1989tr}'s~\citetitle{Vickers:1989tr} they've been saving for a rainy day.
%     Their keys aren't there.

%     They've searched over, under, and beside the desk.
%     They haven't found their keys.

%     Still, the agent holds the following principle:
%     \begin{quote}
%       If they think of a place to look for an object, they do not conclude the object is lost without searching the place.
%     \end{quote}
%   \end{scenario}

%   \begin{observation}%
%     \label{obv:lK:requ}%
%     \pv{\propI{I have no further idea of where to look}}{\valI{True}} being a \fc{} from \(\Psi\) is a \requ{} of the agent concluding \pv{\propI{I have lost my keys}}{\valI{True}} from \(\Phi\).
%   \end{observation}

%   \begin{motivation}{obv:lK:requ}
%     Suppose agent is concluding.
%     Then, by \assuPP{} event in which agent concludes.
%     Therefore, \requ{} of any such event.
%     For, else the agent goes and searches.

%     Antecedent assumes there is some action.
%     Therefore, problem.
%   \end{motivation}

%   Note, this doesn't rule out \(l\) and no as a \fc{}.
%   The problem is, not in a position to determine which conclusion.
% \end{note}



\section{\requ{3} and \issueConstraint{}}
\label{sec:comining-ingredients}

\begin{note}
  Given \autoref{prop:requ-WhyV}, we only need to identify a \requ{} in order to find an answer to \qWhyV{}.
  Hence, the definition of a \requ{} helps state when \issueConstraint{} fails to hold:

  \begin{proposition}[\requ{3} and \issueConstraint{}]
    \label{prop:requ-WhyV-ces}
    \vspace{-\baselineskip}
    \begin{itenum}
    \item[\emph{If}:]
      Conditions~\ref{prop:requ-WhyV-ces:C},~\ref{prop:requ-WhyV-ces:Cing} and \ref{prop:requ-WhyVCes:noW} hold:
      \begin{enumerate}[label=\arabic*., ref=\arabic*]
      \item
        \label{prop:requ-WhyV-ces:C}
        \(\ed{}\) is an event in which \vAgent{} concludes \(\pv{\phi}{v}\) from \(\Phi\).
      \item
        \label{prop:requ-WhyV-ces:Cing}
        There is some event \(\ed{\flat}\) such that:
        \begin{itemize}
        \item
          \(\pv{\psi}{v'}\) being a \fc{} from \(\Psi\) for \vAgent{} throughout \(\ed{\flat}\) is a \requ{} of \(\ed{}\).
        \end{itemize}
      \item
        \label{prop:requ-WhyVCes:noW}
      \vAgent{} does not have a \wit{} for a \ros{} between \(\pv{\psi}{v'}\) and \(\Psi\) when \vAgent{} concludes \(\pv{\phi}{v}\) from \(\Phi\).
      \end{enumerate}
    \item[\emph{Then}:]
      \issueConstraint{} does not hold.
    \end{itenum}
    \vspace{-\baselineskip}
  \end{proposition}

  \noindent%
  \autoref{prop:requ-WhyV-ces} is more-or-less immediate from \autoref{prop:requ-WhyV}:

  \begin{argument}{prop:requ-WhyV-ces}
    Assume conditions~\ref{prop:requ-WhyV-ces:C},~\ref{prop:requ-WhyV-ces:Cing} and \ref{prop:requ-WhyVCes:noW} hold.

    Conditions~\ref{prop:requ-WhyV-ces:C}~and~\ref{prop:requ-WhyV-ces:Cing} entail a \ros{0} between \(\pv{\psi}{v'}\) and \(\Psi\) answers \qWhyV{} via \autoref{prop:requ-WhyV}.
    So, by \issueConstraint{} it must be the case that \vAgent{} has a \wit{} for the \ros{} between \(\pv{\psi}{v'}\) and \(\Psi\).
    However, by Condition~\ref{prop:requ-WhyVCes:noW}, the agent does not have a \wit{} for a \ros{} between \(\pv{\psi}{v'}\) and \(\Psi\) when \vAgent{} concludes \(\pv{\phi}{v}\) from \(\Phi\).
  \end{argument}
\end{note}


\begin{note}
  \autoref{prop:requ-WhyV-ces} provides a collection of conditions which, if satisfied.
  Specifically, in terms of \requ{1}.
  Still, the existence of \requ{1} is compatible with \issueConstraint{}.

  \begin{observation}%
    \label{prop:requ-not-n-ce}%
    \autoref{prop:requ-WhyV} is compatible with \issueConstraint{}.
  \end{observation}

  \begin{motivation}{prop:requ-not-n-ce}
    \requ{3} are about an agent concluding.
    \fc{} when the agent concludes.
    An agent may fail to have a \wit{} for \fc{}.

    However, \issueConstraint{}, \wit{} when the agent concludes.
    Therefore, so long as \wit{} when the agent concludes, fine.

    Two possibilities.
    \begin{enumerate}
    \item
      The agent may already have a \wit{} for a \ros{} which follows from a \fc{}.

      In other words, the agent may have already concluded \(\pv{\psi}{v'}\) from \(\Psi\).

      This is the case with respect to \autoref{obs:LucasRequ}.
      For, the agent has concluded \(\pv{\propM{L_{8} = 47}}{\valI{True}}\) from \(\Phi\) prior to the \agents{} conclusion of \pv{\propM{L_{[:9]} = [2, \dots, 76]}}{\valI{True}} from \(\Phi\).
    \item
      Agent may obtain a \wit{} when the agent concludes.

      In other words, the event in which the agent concludes \(\pv{\phi}{v}\) from \(\Phi\) is a \wit{} for the \ros{} between \(\pv{\psi}{v'}\) from \(\Psi\).

      This is the case with respect to \autoref{illu:gist:roots:F}.
      For, \pv{\rootsCon{}}{\valI{True}} being a \fc{} from \(\Phi\) is a \requ{} of some \se{}, though the event of interest is such that the agent concludes \pv{\rootsCon{}}{\valI{True}} from \(\Phi\).
    \end{enumerate}
    \vspace{-\baselineskip}
  \end{motivation}
\end{note}


\begin{note}
  Given \autoref{prop:requ-not-n-ce}, the idea of a \requ{} is compatible with \issueInclusion{}.
  Still, the idea of a \requ{1} hints the failure of \issueConstraint{}.
\end{note}


\begin{note}
  For example, consider \autoref{scen:LucasNums}.
  In particular, \(\ed{\flat}\) again.
  \(l_{8}\) was a \fc{}.
  It seems \(\pv{\propM{L_{10} = 123}}{\valI{True}}\) is likewise a \fc{} from \(\Phi\) throughout \(\ed{\flat}\).

  Is it possible that the agent is concluding and \(L_{10}\) is not a \fc{}?

  There is a clear sense in which the answer is yes.
  For, agent may be limited, or something may happen.

  Still, exclude such cases with a description.

  So, granting there is nothing which prevents the agent from continuing to calculate Lucas numbers, is it possible that the agent is concluding and \(L_{10}\) is not a \fc{}?

  I doubt this is the case.
  The same thing that gets \(L_{8}\) and \(L_{9}\) gets \(L_{10}\).

  Then, \requ{}.
  So, just need \se{}.

  Still, this is quite and focuses on a single \scen{}.
\end{note}


\section*{Summary}


\begin{note}
  This chapter defined \requ{1}.

  Specifically, we defined when \(\pv{\psi}{v'}\) being a \fc{} from \(\Psi\) for an agent throughout \(\ed{\flat}\) is a \requ{} of some event \(\ed{}\) in which an agent concludes \(\pv{\phi}{v}\) from \(\Phi\).
  This holds just in case \(\ed{\flat}\) is a \se{} of \(\ed{}\) \emph{only if} \(\pv{\psi}{v'}\) is a \fc{} from \(\Psi\) for the agent throughout \(\ed{\flat}\).

  The role of this definition is to easily capture when a \ros{} between \(\pv{\psi}{v'}\) and \(\Psi\) answers to \qWhyV{}.
  For, by \autoref{prop:requ-WhyV}, a \ros{} between \(\pv{\psi}{v'}\) and \(\Psi\) answers to \qWhyV{} when \(\pv{\psi}{v'}\) being a \fc{} from \(\Psi\) is a \requ{} of some \se{} of the event in which the agent concludes \(\pv{\phi}{v}\) from \(\Phi\).

  And, \autoref{prop:requ-WhyV-ces} highlighted when \issueConstraint{} fails to hold due to some \requ{1}.
\end{note}


\begin{note}
  The definition of a \requ{} is not designed to do anything more than help identify answers to \qWhyV{} and counterexamples to \issueConstraint{}.
\end{note}


% \footnote{
%     \nocite{Tichy:1976tp}%
%     The \itc{} of \qWhyV{} captures the presence of a lawlike constraint, as it applies regardless of whether or not the agent goes on to conclude \(\pv{\phi}{v}\) from \(\Phi\).

%     In the literature on subjunctive conditionals, the conditionals which constrain the development of events are sometimes termed `laws' (\cite{Chisholm:1955aa,Lewis:1979vm,Veltman:2005tj}) or thing that `lump together' certain facts (\cite{Kratzer:1981aa,Kratzer:1989aa}).
%   }


%%% Local Variables:
%%% mode: latex
%%% TeX-master: "master"
%%% TeX-engine: luatex
%%% End:

