\chapter{\requ{3}}
\label{cha:requs}

\begin{note}
  \autoref{cha:fcs} introduced \fc{1}.

  \fc{1} are such that:
  \begin{itemize}
  \item
    \ros{} holds between \(\pv{\psi}{v'}\) and \(\Psi\) (for the agent).
  \item
    The agent doesn't have a \wit{} for the \ros{} between \(\pv{\psi}{v'}\) and \(\Psi\).
  \end{itemize}

  A \requ{} is a relation between \(\pv{\psi}{v'}\) from \(\Psi\) being a \fc{} and the agent \emph{concluding} \(\pv{\phi}{v}\) from \(\Phi\).
\end{note}

\begin{note}
  Important to generate counterexamples to \issueConstraint{}.
  Still, only certain \requ{1} generate counterexamples to \issueConstraint{}.%
  \footnote{
    I.e.\ one may jointly hold:
    \begin{enumerate*}[label=(\alph*)]
    \item
      various \requ{1} exist, and
    \item
      answers to \qWhyV{} are constrained by answers to \qHowV{} via \issueConstraint{}
    \end{enumerate*}%
    .
    For more, see \autoref{prop:requ-not-n-ce} on \autopageref{prop:requ-not-n-ce}, below.
  }
  Hence, we introduce and motivate the idea of a \requ{0} with \requ{1} which are compatible with \issueConstraint{}.
\end{note}

\section{\requ{3}}
\label{cha:requs:requs}

\begin{note}
  Idea of a \requ{} is something that must be the case in order for the agent to be concluding.

  Various things need to be the case.
  In general, the agent must be alive and conscious.
  Specific instances, further constraints.
  For example, the coffee is too hot.
  Cup of coffee and a preference for the temperature of coffee.

  A \requ{0} is a \prop{0}-\val{0} pair which must be a \fc{0} for the agent in order for the agent to be concluding some \prop{0}-\val{0} pair.

  In particular, this thing is a \fc{}.
  We define \requ{1} as follows:

  \begin{definition}[A \requ{0}]%
    \label{def:requ}%
    For any event \(e\) such that \(e\) is an event in which \vAgent{} concludes \(\pv{\phi}{v}\) from \(\Phi\):
    %
    \begin{itemize}
    \item
      \(\pv{\psi}{v'}\) being a \fc{} from \(\Psi\) for \vAgent{} throughout \(e^{\flat}\) is a \emph{\requ{}} of \(e\).
    \end{itemize}

    \emph{If and only if}

    \begin{itemize}
    \item
      \begin{itenum}
      \item[\emph{If}:]
        \(e^{\flat}\) develops into \(e\).
      \item[\emph{Then}:]
        \(\pv{\psi}{v'}\) is a \fc{} from \(\Psi\) for \vAgent{} throughout \(e^{\flat}\).
      \end{itenum}
    \end{itemize}
    \vspace{-\baselineskip}
  \end{definition}

  {\color{red}

    \(e^{\flat}\) develops into an event in which \vAgent{} concludes \(\pv{\phi}{v}\) from \(\Phi\).
  }

  \noindent%
  Shorten the defined term to:
  \begin{notationList}
  \item
    \(\pvp{\psi}{v'}{\Psi}\) is a \requ{} of an agent concluding \(\pv{\phi}{v}\) from \(\Phi\).
  \end{notationList}
\end{note}

\begin{note}
  Similar to \qWhyV{} on \autopageref{questionWhyV}.
  Something about event which gets this constraint.

  \itc{2}, so either not concluding or \fc{}.

  Something about the event secures requirement.

  Agent is or is not concluding \(\pv{\phi}{v}\) from \(\Phi\).
  However, requirement if agent is concluding.
  And, possible block if agent is not.

  However, it does not follow from \(\pvp{\phi}{v}{\Phi}\) being a \requ{} of an agent concluding \(\pv{\phi}{v}\) from \(\Phi\) that either:
  \begin{itemize}
  \item
    \(\pv{\phi}{v}\) being a \fc{} from \(\Psi\) is relevant to the agent concluding \(\pv{\phi}{v}\) from \(\Phi\), if the agent is concluding \(\pv{\phi}{v}\) from \(\Phi\).
  \item
    \(\pv{\phi}{v}\) not being a \fc{} from \(\Psi\) is relevant to the agent not concluding \(\pv{\phi}{v}\) from \(\Phi\), if the agent is not concluding \(\pv{\phi}{v}\) from \(\Phi\).
  \end{itemize}

  For \requ{} to be of interest, additional considerations.
  Following section.

  First, we two \illu{1} of \requ{1}.
\end{note}

\paragraph*{\illu{3}}

\begin{note}
  \begin{scenario}[Lucas numbers]%
    \label{scen:LucasNums}%
    The Lucas numbers are recursively defined as follows:%
    \footnote{
      Starting at \(2\), see \hyperlink{cite.OEIS.:aa}{OEIS sequence A000032} for more details.
    }
    %
    \[
      L_{n} = \left\{
        \begin{array}{ll}
          2 & \text{if } n = 0 \\
          1 & \text{if } n = 1 \\
          L_{n-1} + L_{n-2} & \text{if } n > 1 \\
        \end{array}
      \right.
    \]

    An agent is alone and quite bored.
    The agent sets out to calculate the first ten Lucas numbers by applying the recursive definition.

    The agent begins:

    \[
    \begin{array}{cccccc}
      L_{0} & L_{1} & L_{2} & L_{3} & L_{4} & \cdots \\
      \hline
      2 & 1 & 3 & 4 & 7 & \cdots \\
    \end{array}
    \]
  \end{scenario}

  \begin{observation}[\requ{3} of \autoref{scen:LucasNums}]%
    \label{obs:LucasRequ}%
    For the event \(e\) as described by \autoref{scen:LucasNums}:
    %
    \begin{itemize}
    \item
      \(\pv{\propI{L\textsubscript{8} = 47}}{\valI{True}}\) from \(\Phi_{8}\) and \(\pv{\propI{L\textsubscript{7} = 29}}{\valI{True}}\) from \(\Phi_{7}\) are both \requ{1} of the agent concluding \pv{\propI{L\textsubscript{[:9]} = [2, 1, 3, 4, 7, 11, 18, 29, 47 76]}}{\valI{True}} from \(\Phi\).
    \end{itemize}
    %
    Where \propI{L\textsubscript{[:9]} = [2, 1, 3, 4, 7, 11, 18, 29, 47 76]} abbreviates \propI{The first ten Lucas numbers are 2, 1, 3, 4, 7, 11, 18, 29, 47, and 76}, and the \pool{1} \(\Phi_{7}\), \(\Phi_{8}\), and \(\Phi\) are left unspecified.
  \end{observation}

  \begin{motivation}{obs:LucasRequ}
    Applying \autoref{def:requ} it is sufficient to establish:

    \begin{itenum}
    \item[\emph{If}:]
      \(e\) is an event in which the agent is concluding\newline \pv{\propI{L\textsubscript{[:9]} = [2, 1, 3, 4, 7, 11, 18, 29, 47 76]}}{\valI{True}} from \(\Phi\).
    \item[\emph{Then}:]
      \pv{\propI{L\textsubscript{8} = 47}}{\valI{True}} from \(\Phi_{8}\) is a \fc{} for the agent.
    \end{itenum}
    %
    And, likewise for \pv{\propI{L\textsubscript{7} = 29}}{\valI{True}} from \(\Phi_{8}\).

    Suppose the agent is concluding \pv{\propI{L\textsubscript{[:9]} = [2, 1, 3, 4, 7, 11, 18, 29, 47 76]}}{\valI{True}} from \(\Phi\).

    By \assuPP{}, there is an event in which the agent concludes\newline \pv{\propI{L\textsubscript{[:9]} = [2, 1, 3, 4, 7, 11, 18, 29, 47 76]}}{\valI{True}} from \(\Phi\).
    For, the proposition contains the tenth Lucas number, and so it seems the agent must have concluded \pv{\propI{L\textsubscript{9} = 76}}{\valI{True}} from some \pool{} \(\Phi_{9}\).

    And, as the agent is reasoning by the recursive definition, the agent obtains \(L_{9}\) by adding \(L_{8}\) and \(L_{7}\).
    Hence, prior to the conclusion \pv{\propI{L\textsubscript{9} = 76}}{\valI{True}}, it seems the agent must conclude \pv{\propI{L\textsubscript{8} = 47}}{\valI{True}} and \pv{\propI{L\textsubscript{7} = 29}}{\valI{True}}.

    And, as the agent is alone, it seems it must be the case the agent concludes \pv{\propI{L\textsubscript{8} = 47}}{\valI{True}} and \pv{\propI{L\textsubscript{7} = 29}}{\valI{True}} without any novel information.

    So, events.
    Hence, \fc{1}.
  \end{motivation}

  Important parts of the motivation for \autoref{obs:LucasRequ} are need in order to conclude, and no other option.

  By the same reasoning, \(L_{6}\) and \(L_{5}\).

  Indeed, \requ{1} of the event which secures \fc{}.

  And, \(L_{9}\) itself.%
  \footnote{
    However, it is not always the case that \(\pvp{\phi}{v}{\Phi}\) is a \requ{} of an agent concluding \(\pv{\phi}{v}\) from \(\Phi\).
    For example, consider the partnered case from \autoref{obs:cds-arb}
  }
\end{note}

\begin{note}
  Is the agent concluding?
  This depends on the event.
  Are the relevant \prop{0}-\val{0} pairs \fc{1}?

  For example, if the agent makes a mistake, then not concluding.
\end{note}

\begin{note}
  As defined, there is little to a \requ{1}.
  The role of a \requ{} is to link an agent concluding to \fc{1}.
\end{note}


\section{\requ{3}, \qWhyV{}, and \issueConstraint{}}
\label{sec:comining-ingredients}

\begin{note}
  Section links \requ{0} to \qWhyV{} and \issueConstraint{}.

  Start by defining a collection of conditions.
  Then, \qWhy{}.
  From this, additional condition to get counterexample to \issueConstraint{}.
\end{note}

\subsection{\requ{3} and \qWhyV{}}

\begin{note}
  \begin{proposition}[\requ{3} and \qWhyV{}]
    \label{prop:requ-WhyV}
    \vspace{-\baselineskip}
    \begin{itenum}
    \item[\emph{If}:]
      Conditions~%
      \ref{def:rCs:C}~and~%
      \ref{def:rCs:Cing}~%
      jointly hold:
      \begin{enumerate}[label=\arabic*., ref=(\arabic*)]
      \item
        \label{def:rCs:C}
        \(e\) is an event in which \vAgent{} concludes \(\pv{\phi}{v}\) from \(\Phi\).
      \item
        \label{def:rCs:Cing}
        There is some sub-event \(e^{\flat}\) of \(e\) such that:
        \begin{enumerate}[label=\alph*., ref=(\arabic{enumi}\alph*)]
        \item
          \label{def:rCs:Cing:requ}
          \(\pv{\psi}{v'}\) being a \fc{} from \(\Psi\) for \vAgent{} throughout \(e^{\flat}\) is a \requ{} of \(e\).
        \end{enumerate}
      \end{enumerate}
    \item[\emph{Then}:]
      A \ros{0} between \(\pv{\psi}{v'}\) and \(\Psi\) answers \qWhyV{}.
    \end{itenum}
    \vspace{-\baselineskip}
  \end{proposition}

  \begin{argument}{prop:requ-WhyV}
    Assume conditions~\ref{def:rCs:C}~and~\ref{def:rCs:Cing}~jointly hold.

    From Condition~\ref{def:rCs:C}, \(e\) is an event in which \vAgent{} concludes \(\pv{\phi}{v}\) from \(\Phi\).
    So, \qWhyV{} applies to \(e\).

    Our task is to show the conditional of \qWhyV{} is true.

    Consider the event \(e^{\flat}\).
    Suppose a \ros{} between \(\pv{\psi}{v'}\) and \(\Psi\) fails to hold for \vAgent{} through \(e^{\flat}\).

    \supportII{}:

    \begin{itenum}
    \item[\emph{If}:]
      \(\pv{\psi}{v'}\) is a \fc{0} from \(\Psi\) for \vAgent{} throughout \(e^{\sharp}\).
    \item[\emph{Then}:]
      A \ros{0} between \(\pv{\psi}{v'}\) and \(\Psi\) holds for \vAgent{} throughout \(e^{\sharp}\).
    \end{itenum}

    \noindent%
    So, by our supposition \(\pv{\psi}{v'}\) not a \fc{0} from \(\Psi\) for \vAgent{} throughout \(e^{\flat}\).
    Then, \(e^{\flat}\) is not, or does not develop into, \(e\).
    This contradicts  \ref{def:rCs:Cing}.
  \end{argument}
\end{note}

\begin{note}
  Key.

  Intuition is that \tC{}.
  Therefore, matters whether or not \fc{}.
  Hence, matters whether or not \ros{}.
\end{note}

\subsection{\requ{3} and \issueConstraint{}}
\label{cha:binding:sec:requ-iC}

\begin{note}
  With \autoref{prop:requ-WhyV} we observe the way in which the \rCon{0} may provide counterexamples to \issueConstraint{}.

  In short, we need some instance in which an agent concludes \(\pv{\phi}{v}\) from \(\Phi\) such that \(\pvp{\psi}{v'}{\Psi}\) is a \requ{} of the agent concluding \(\pv{\phi}{v}\) from \(\Phi\), yet the agent does not have a \wit{} for the \ros{} between \(\pv{\phi}{v}\) and \(\Psi\).

  In full:

  \begin{proposition}[\requ{3} and \issueConstraint{}]
    \label{prop:requ-WhyV-ces}
    \vspace{-\baselineskip}
    \begin{itenum}
    \item[\emph{If}:]
      Conditions~\ref{prop:requ-WhyV-ces:C}~and~\ref{prop:requ-WhyV-ces:Cing}~jointly hold:
      \begin{enumerate}[label=\arabic*., ref=(\arabic*)]
      \item
        \label{prop:requ-WhyV-ces:C}
        \(e\) is an event in which \vAgent{} concludes \(\pv{\phi}{v}\) from \(\Phi\).
      \item
        \label{prop:requ-WhyV-ces:Cing}
        There is some sub-event \(e^{\flat}\) of \(e\) such that:
        \begin{enumerate}[label=\alph*., ref=(\arabic{enumi}\alph*)]
        \item
          \label{prop:requ-WhyV-ces:Cing:requ}
          \(\pv{\psi}{v'}\) being a \fc{} from \(\Psi\) for \vAgent{} throughout \(e^{\flat}\) is a \requ{} of \(e\).
        \end{enumerate}
      \end{enumerate}
    \item[\emph{And}:]
      \label{prop:requ-WhyVCes:noW}
      \vAgent{} does not have a \wit{} for a \ros{} between \(\pv{\phi}{v}\) and \(\Psi\) when \vAgent{} concludes \(\pv{\phi}{v}\) from \(\Phi\).
    \item[\emph{Then}:]
      \(\pvp{\psi}{v'}{\Psi}\) is a counterexample to \issueConstraint{}.
    \end{itenum}
    \vspace{-\baselineskip}
  \end{proposition}

  \begin{argument}{prop:requ-WhyV-ces}
    More-or-less immediate.

    % If the \rCon{0} hold of \(e\) with respect to \(\langle \vAgent{}, \pvp{\phi}{v}{\Phi}, \pvp{\psi}{v'}{\Psi}, e^{\flat} \rangle\), then \(\pvp{\psi}{v'}{\Psi}\), in part, answers \qWhyV{}, given~\autoref{prop:requ-WhyV}.
    % And, granting \vAgent{} does not have a \wit{} for a \ros{} between \(\pv{\phi}{v}\) and \(\Psi\) when \vAgent{} concludes \(\pv{\phi}{v}\) from \(\Phi\), this immediately contradicts \issueConstraint{}.
  \end{argument}
\end{note}

\begin{note}
  \autoref{prop:requ-WhyV-ces} does not guarantee the existence of counterexamples to \issueConstraint{}.
  And, as we were careful not to presuppose counterexamples to \issueConstraint{} when developing \tC{0}, \fc{1}, and \requ{1}, we have no yet seen any explicit counterexamples to \issueConstraint{}.

  \autoref{cha:ces} provides examples which satisfy the conditions of \autoref{prop:requ-WhyV-ces}.
\end{note}

\begin{note}
  The existence of \requ{1} are compatible with \issueConstraint{}.

  \begin{observation}%
    \label{prop:requ-not-n-ce}%
    \autoref{prop:requ-WhyV} is compatible with \issueConstraint{}.
  \end{observation}

  \begin{motivation}{prop:requ-not-n-ce}
    \requ{3} are about an agent concluding.
    \fc{} when the agent concludes.
    An agent may fail to have a \wit{} for \fc{}.

    However, \issueConstraint{}, \wit{} when the agent concludes.
    Therefore, so long as \wit{} when the agent concludes, fine.

    Two possibilities.
    \begin{enumerate}
    \item
      Agent already has a \wit{}.

      For example, first \scen{0}.
      If concluding sequence, \fc{}.
      Prior to getting \(L_{8}\), \fc{}.

      However, expectation of \wit{} when sequence.
      Indeed, \(L_{9}\) from \(L_{8} + L_{7}\).
    \item
      Agent obtains \wit{} prior to conclusion.

      Consider \autoref{scen:squish}.
      The agent has previously concluded \sqE{} is sound.
      Therefore, the agent has a \wit{} for \ros{}.
    \end{enumerate}
    \vspace{-\baselineskip}
  \end{motivation}

  Note, however, that \autoref{prop:requ-not-n-ce} is weak.
  Compatible in terms of existential.
  Hence we only needed to show a single case in which \wit{} for \requ{}.
  This does not suggest that \requ{1} are not in tension with \issueConstraint{}.
\end{note}

\section{Additional motivation for \requ{1}}
\label{sec:addit-motiv-requ1}

\begin{note}
  \begin{scenario}[Lost keys]%
    \label{illu:lost-key}%
    An agent thinks they may have lost their keys.
    They usually leave place my keys on the right side of their desk, near a copy of~\citeauthor{Vickers:1989tr}'s~\citetitle{Vickers:1989tr} they've been saving for a rainy day.
    Their keys aren't there.

    They've searched over, under, and beside the desk.
    They haven't found their keys.

    Still, the agent holds the following principle:
    \begin{quote}
      If the agent thinks of a place to look for an object, the agent does not conclude the object is lost without searching the place.
    \end{quote}
  \end{scenario}

  \noindent%
  % At issue is whether the agent concludes they have lost their keys without any further search.
  % Fix the event to begin when the \agents{} completes their initial search of the desk and extend the event until the agent either
  % \begin{enumerate*}[label=(\alph*), ref=(\alph*)]
  % \item
  %   concludes they have lost their keys without any further search, or
  % \item
  %   sets of on a further search.
  % \end{enumerate*}
  Given the presentation of \autoref{illu:lost-key} it seems:
  \begin{itemize}
  \item
    \pvp{\propI{The agent has no further idea of where to look}}{\valI{True}}{\Psi} is a \requ{} of the agent concluding \pv{\propI{The agent has lost their keys}}{\valI{True}} from \(\Phi\).
  \end{itemize}
  The principle the agent holds requires the agent to conclude the have no further idea of where to look before concluding they have lost their keys.%
  \footnote{
    Assumes the agent has negative influence.

    I think this is clear.
    Various things that influence.

    And, though possible that there are various demons that change things, do not consider these relevant.

    Note, this does not entail that the agent has positive influence, nor choice.
  }
  Then, by principle, it must be the case that no further place.
  For, else go look at that place.
  And, if go look, then the relevant event is over.

  \requ{2}.
  However, only given the principle.
  It may be the case that the agent abandons principle.
  However, if so then description is not true of relevant event.
  Subtle point.
  Need something about the way things are which constrains the way things may develop.

  If this is not the case, then no \requ{1}.
  However, I assume this is the case.%
  \footnote{
    Various demons, rule out by description.
  }
\end{note}

\begin{note}
  Interesting features:

  \begin{observation}
    It need may be the case that both:
    \begin{itemize}
    \item
      \pvp{\propI{The agent has no further idea of where to look}}{\valI{True}}{\Psi}
    \item
      \pvp{\propI{The agent's may be by the coffee machine}}{\valI{True}}{\Psi}
    \end{itemize}
    are \fc{1}.
  \end{observation}

  This is fine.
  If the latter, then the agent does not conclude they have lost their keys.

  Now, with conditional, is it the case that concluding?
  Is the agent going to think of something?

  If the agent does, then they go and search.
  If the agent does not, then they conclude lost.

  There is not immediate answer.
  However, either or.

  To illustrate further.
  Suppose coffee.
  May think to retrace steps.
  If so, no conclusion.
  However, the agent may fail to think of anything.
  If so, the agent concludes.

  Still, it is not the case that conclusion is in progress, for the agent may think of a place to search.

  It may be the case that \fc{} and not concluding.
  But, this is no problem.
  For, \fc{1} are not exclusive.
  But, this is consistent with \requ{}.
\end{note}


%%% Local Variables:
%%% mode: latex
%%% TeX-master: "master"
%%% TeX-engine: luatex
%%% End:

