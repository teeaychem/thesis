\chapter{\requ{3}}
\label{cha:requs}

\begin{note}
  This chapter introduces the idea of a \(\pv{\psi}{v'}\) being a \fc{} from \(\Psi\) as a \requ{1} of an \eiw{0} an agent concludes \(\pv{\phi}{v}\) from \(\Phi\).

  The role of \requ{1} is to capture the way we identify answers to \qWhy{}, and in turn help characterise failures of \issueInclusion{}.

  Following the definition of \requ{1} we state connexion between \requ{1}, \qWhy{}, and \issueInclusion{} and provide a few examples.
\end{note}



\section{\requ{3}}
\label{cha:requs:requs}

\begin{note}
  \begin{rdefinition}{def:requ}{A \requ{0}}%
    % \autoref{def:requ}
    For an \eiw[\(\ed{}\)]{0} \vAgent{} concludes \(\pv{\phi}{v}\) from \(\Phi\):
    %
    \begin{itemize}
    \item
      \(\pv{\psi}{v'}\) being a \fc{} from \(\Psi\) for \vAgent{} through \(\ed{\flat}\) is a \emph{\requ{}} of \(\ed{}\).
    \end{itemize}

    \emph{If and only if}

    \begin{itemize}
    \item
      Clauses \ref{def:requ:se} and \ref{def:requ:fc} hold:
      %
      \begin{enumerate}[label=\Alph*., ref=\Alph*]
      \item
        \label{def:requ:se}
        \(\ed{\flat}\) is a \se{} of \(\ed{}\).
      \item
        \label{def:requ:fc}
        \(\edo{\flat}\) entails:
        \(\pv{\psi}{v'}\) is a \fc{} from \(\Psi\) for \vAgent{} through \(\edn{\flat}\).
      \end{enumerate}
    \end{itemize}
    \vspace{-\baselineskip}
  \end{rdefinition}

  \noindent%
  The definition of a \requ{} captures a particular instance of \progEx{} (\autopageref{sec:ProgEx}).
  Specifically, when there is some \se{} \(\ed{\flat}\) of \(\ed{}\) such that \(\ed{\flat}\) is a \se{} of \(\ed{}\) \emph{only if} \(\pv{\psi}{v'}\) is a \fc{} from \(\Psi\) for the agent through \(\ed{\flat}\).
\end{note}


\begin{note}
  Expanded, the definition of a \requ{} concerns an event \(\ed{\flat}\) such that:
  %
  \begin{itemize}
  \item
    A conclusion of \(\pv{\phi}{v}\) from \(\Phi\) is in progress through \(\ed{\flat}\), and if the agents conclusion in progress happens, it happens as a result of \(\ed{\flat}\).
  \item
    \(\pv{\psi}{v'}\) being a \fc{} from \(\Psi\) is required for \(\ed{\flat}\) to be an event such that a conclusion of \(\pv{\phi}{v}\) from \(\Phi\) is in progress.
  \end{itemize}
  %
  The conditions capture the key way a \fingfr{} answers \qWhy{}:

  \begin{proposition}[\requ{3} and \qWhy{}]
    \label{prop:requ-WhyV}
    \vspace{-\baselineskip}
    \begin{itenum}
    \item[\emph{If}:]
      Conditions~%
      \ref{def:rCs:C}~and~\ref{def:rCs:Cing}~jointly hold:
      \begin{enumerate}[label=\arabic*., ref=\arabic*]
      \item
        \label{def:rCs:C}
        \(\ed{}\) is an \eiw{0} \vAgent{} concludes \(\pv{\phi}{v}\) from \(\Phi\).
      \item
        \label{def:rCs:Cing}
        There is some event \(\ed{\flat}\) such that:
        \begin{itemize}
        \item
          \(\pv{\psi}{v'}\) being a \fc{} from \(\Psi\) for \vAgent{} through \(\ed{\flat}\) is a \requ{} of \(\ed{}\).
        \end{itemize}
      \end{enumerate}
    \item[\emph{Then}:]
      \fingfb{\(\pv{\psi}{v'}\)}{\(\Psi\)} answers \qWhy{}.
    \end{itenum}
    \vspace{-\baselineskip}
  \end{proposition}

  \begin{argument}{prop:requ-WhyV}
    Assume conditions~\ref{def:rCs:C}~and~\ref{def:rCs:Cing} hold.

    From Condition~\ref{def:rCs:C}, \(\ed{}\) is an \eiw{0} an agent concludes \(\pv{\phi}{v}\) from \(\Phi\).
    So, \qWhy{} applies to \(\ed{}\).

    Let \(\ed{\flat}\) by an event which satisfies the existential of Condition~\ref{def:rCs:Cing}.
    So, \(\pv{\psi}{v'}\) being a \fc{} from \(\Psi\) for the agent through \(\ed{\flat}\) is a \requ{} of \(\ed{}\).
    Hence, by \supportII{}, \(\edo{\flat}\) entails it is true of \(\edn{\flat}\) that:
    \fofb{\(\pv{\psi}{v'}\)}{\(\Psi\)} for the agent through \(\ed{\flat}\).
    This means we have an \se{} \(\ed{\flat}\) of \(\ed{}\) such that:
    \(\edo{\flat}\) entails is true of \(\edn{\flat}\) that:
    \fofb{\(\pv{\psi}{v'}\)}{\(\Psi\)} through \(\ed{\flat}\).
    So, by \autoref{sketch:PE:cROS} (\autopageref{sketch:PE:cROS}) \fingfb{\(\pv{\psi}{v'}\)}{\(\Psi\)} answers \qWhy{}.
  \end{argument}
\end{note}



\subsection{Two applications}
\label{sec:some-requ1-beginnote}


\begin{note}
  We provide two applications of a \requ{}.
  The first highlights a \requ{} of \autoref{illu:gist:roots:F}.
  The second works through a new \scen{0}.
\end{note}



\paragraph*{Factorisation}


\begin{note}
  \begin{application}[A \requ{} of \autoref{illu:gist:roots:F}]
    \label{app:requ-scen1}
    Given:
        \begin{itemize}
    \item
      \(\edn{}\) is the event described by \autoref{illu:gist:roots:F}.
    \item
      \(\Phi\) includes the \agents{} understanding of factorisation prior to \(\edn{}\).
    \item
      \(\edo{}\) is the description `the agent concludes \propM{\rootsCon{}} has \val{0} \valI{True} from \(\Phi\)'.
    \item
      \(\edn{\flat}\) covers Step~\ref{illu:gist:roots:F:factor} of the \agents{} reasoning in \autoref{illu:gist:roots:F}
    \item
      \(\edo{\flat}\) is the description:
      `The agent figures out \rootsConEqExV{6}{3}{2} with the aim to identify the factors of \rootsConEq{}'.
    \end{itemize}
    %
    It is the case that:
    %
    \begin{itemize}
    \item
      \pv{\propM{\rootsCon{}}}{\valI{True}} being a \fc{} from \(\Phi\) through \(\ed{\flat}\) is a \requ{} of \(\ed{}\).
    \end{itemize}
    \vspace{-\baselineskip}
  \end{application}


  \begin{dets}{app:requ-scen1}
    \autoref{obs:se-inst} (\autopageref{obs:se-inst}) argued \(\ed{\flat}\) is a \se{} of \(\ed{}\).
    And, by \autoref{obs:var:factor:fc} (\autopageref{obs:var:factor:fc}) \(\edo{\flat}\) entails:
    \pv{\propM{\rootsCon{}}}{\valI{True}} is a \fc{} from \(\Phi\) through \(\ed{\flat}\).
  \end{dets}

  \noindent%
  Note, \pv{\propM{\rootsCon{}}}{\valI{True}} being a \fc{} from \(\Phi\) through \(\ed{\flat}\) being a \requ{} of \(\ed{}\) captures the way \fingfb{\pv{\propM{\rootsCon{}}}{\valI{True}}}{\(\Phi\)} answers \qWhy{}.
  For, the details of \autoref{appl:qWhyV-s1} simply expand on the details of \autoref{app:requ-scen1} with the observations made in the argument for \autoref{prop:requ-WhyV}.
\end{note}



\paragraph*{Lucas numbers}


\begin{note}
  We work through a \scen{} to illustrate the definition of a \requ{}.
  We focus on identifying a single \requ{}, though the argument presented generalises to identify additional \requ{1} in the same \scen{0}.

  \begin{scenario}[Lucas numbers]%
    \label{scen:LucasNums}%
    The Lucas numbers are recursively defined as follows:%
    \footnote{
      Starting at \(2\), see \hyperlink{cite.OEIS.:aa}{OEIS sequence A000032} for more details.
    }
    %
    \[
      L_{n} = \left\{
        \begin{array}{ll}
          2 & \text{if } n = 0 \\
          1 & \text{if } n = 1 \\
          L_{n-1} + L_{n-2} & \text{if } n > 1 \\
        \end{array}
      \right.
    \]
    %
    An agent is alone and quite bored.
    The agent sets out to calculate the first ten Lucas numbers by applying the recursive definition.
    %
    The agent begins:
    %
    \[
      \begin{array}{cccccc}
        L_{0} & L_{1} & L_{2} & L_{3} & L_{4} & \cdots \\
        \hline
        2 & 1 & 3 & 4 & 7 & \cdots \\
      \end{array}
    \]
    %
    And, eventually concludes:
    \begin{center}
      \pv{\propI{The first ten Lucas numbers are 2, 1, 3, 4, 7, 11, 18, 29, 47, and 76}}{\valI{True}}
    \end{center}
    %
    From some \pool{} \(\Phi\) which includes the \agents{} understanding of the Lucas numbers.
  \end{scenario}

  \noindent%
  The event of interest is \(\ed{}\) is when the agent concludes \pv{\propM{L_{[:9]} = [2, \dots, 76]}}{\valI{True}} from \(\Phi\), after studying the definition of Lucas numbers.%
  \footnote{
    \propM{L_{[:9]} = [2, \dots, 76]} shortens \propI{The first ten Lucas numbers are 2, 1, 3, 4, 7, 11, 18, 29, 47, and 76}.
  }

  \begin{application}[\requ{3} of \autoref{scen:LucasNums}]%
    \label{obs:LucasRequ}%
    There is some event \(\ed{\flat}\) such that:
    %
    \begin{itemize}
    \item
      \pv{\propM{L_{8} = 47}}{\valI{True}} being a \fc{} from \(\Phi\) through \(\ed{\flat}\) is a \requ{0} of \(\ed{}\).
    \end{itemize}
    \vspace{-\baselineskip}
  \end{application}

  \noindent%
  The motivation for \autoref{obs:LucasRequ} largely follows the motivation given for a \se{} and \fc{} in connexion with the second step of the \agents{} reasoning in \autoref{illu:gist:roots:F} --- see \autoref{obs:se-inst} (\autopageref{obs:se-inst}) and \autoref{obs:var:factor:fc} (\autopageref{obs:var:factor:fc}), respectively.

  So, rather than work through the full definitions of a \se{} and a \fc{} we first identify an event of interest and then sketch the key observations.

  \begin{dets}{obs:LucasRequ}
    By assumption \(e_{d}\) is such that the agent concludes \pv{\propM{L_{[:9]} = [2, \dots, 76]}}{\valI{True}} from \(\Phi\), where \(\Phi\) includes the \agents{} understanding of the Lucas numbers.

    Now, \propM{L_{[:9]} = [2, \dots, 76]} contains the first nine Lucas numbers.
    Implicit in the description of \autoref{scen:LucasNums} is that the agent has no calculated the first nine Lucas numbers prior to \(\ed{}\).
    And, explicit in the description of \autoref{scen:LucasNums} is that the agent concludes by applying the given definition.
    So, as \propM{L_{[:9]} = [2, \dots, 76]} contains the first nine Lucas numbers there must be some sub-\eiw{0} the agent concludes \pv{\propM{L_{9} = 76}}{\valI{True}} by adding \(L_{8}\) and \(L_{7}\).
    And, in turn, there must be a sub-\eiw{0} the agent concludes \pv{\propM{L_{8} = 47}}{\valI{True}} by adding \(L_{7}\) and \(L_{6}\), and so on\dots

    Fix \(\edn{\flat}\) as the event when the agent concludes \pv{\propM{L_{7} = 29}}{\valI{True}}, and take \(\edo{\flat}\) as a natural description of the event.
    \smallskip

    \noindent%
    The first task is to show \(\ed{\flat}\) is a \se{} of \ed{}.

    Similar to the argument for \autoref{obs:se-inst} (\autopageref{obs:se-inst}), observe it is not possible for the agent to conclude \pv{\propM{L_{[:9]} = [2, \dots, 76]}}{\valI{True}} from \(\Phi\) if the \agents{} conclusion is not a result (in part) of figuring out \pv{\propM{L_{7} = 29}}{\valI{True}}.
    For, \(\Phi\) includes the \agents{} understanding of the Lucas numbers and there is no way for the agent to figure out \propM{L_{[:9]} = [2, \dots, 76]} without figuring out \propM{L_{7} = 29} unless the agent appeals to something other than their understanding of the Lucas numbers.
    So, the agent concludes \pv{\propM{L_{[:9]} = [2, \dots, 76]}}{\valI{True}} as a result of concluding \pv{\propM{L_{7} = 29}}{\valI{True}}.

    Given the above Clause~\ref{assu:p:se:hCon} of \autoref{def:se} is satisfied, and all that remains is to finesse the relevant description to ensure it follows an \eiw{0} the agent concludes \pv{\propM{L_{[:9]} = [2, \dots, 76]}}{\valI{True}} from \(\Phi\) is in progress.
    \smallskip

    \noindent%
    The second task is to show \pv{\propM{L_{8} = 47}}{\valI{True}} is a \fc{} from \(\Phi\) through~\(\ed{\flat}\).

    The broad idea parallels \autoref{obs:var:factor:fc} (\autopageref{obs:var:factor:fc}).
    In short, \(\ed{\flat}\) is an \eiw{0} the agent concludes \pv{\propM{L_{7} = 29}}{\valI{True}}.
    And, \(\ed{\flat}\) is such that the agent is concluding \pv{\propM{L_{[:9]} = [2, \dots, 76]}}{\valI{True}}.
    As observed the agent must conclude \pv{\propM{L_{8} = 47}}{\valI{True}} in order to conclude \pv{\propM{L_{[:9]} = [2, \dots, 76]}}{\valI{True}} from \(\Phi\).
    And, the agent has not yet concluded \pv{\propM{L_{8} = 47}}{\valI{True}}.
    Therefore, as there is some action available to the agent such that the agent continues to be concluding \pv{\propM{L_{[:9]} = [2, \dots, 76]}}{\valI{True}} from \(\Phi\), the same action also ensures involves the agent concluding \pv{\propM{L_{8} = 47}}{\valI{True}} from \(\Phi\).
  \end{dets}

  \noindent%
  With minor adjustments the motivation for \autoref{obs:LucasRequ} also shows \pv{\propM{L_{9} = 76}}{\valI{True}} being a \fc{} from \(\Phi\) through \(\ed{\flat}\) is a \requ{0} of \(\ed{\flat}\).
  The motivation may also be adapted to earlier events and numbers in the sequence until the base cases of \(L_{1}\) and \(L_{0}\), as these are given by the \agents{} understanding of the Lucas numbers.
\end{note}



\section{\requ{3} and failures of \issueInclusion{}}
\label{sec:comining-ingredients}

\begin{note}
  Given \autoref{prop:requ-WhyV}, identifying a \requ{} sufficient to find an answer to \qWhy{}.

  Still, answers to \qWhy{} may be compatible with \issueInclusion{} as an agent may already have a \wit{} for the relevant \fingfr{}.
  For example, in \autoref{obs:LucasRequ} \fingfb{\pv{\propM{L_{8} = 47}}{\valI{True}}}{\(\Phi\)} answers \qWhy{} the agent concluded \pv{\propM{L_{[:9]} = [2, \dots, 76]}}{\valI{True}} from \(\Phi\).
  However, it is also the case that the agent concluded \pv{\propM{L_{8} = 47}}{\valI{True}} from \(\Phi\) prior to the conclusion of \pv{\propM{L_{[:9]} = [2, \dots, 76]}}{\valI{True}}.

  Likewise, in \autoref{illu:gist:roots:F} \fingfb{\pv{\propM{\rootsCon{}}}{\valI{True}}}{\(\Phi\)} answers \qWhy{} the agent concluded \pv{\propM{\rootsCon{}}}{\valI{True}} from \(\Phi\).
  However, when the agent concludes \pv{\propM{\rootsCon{}}}{\valI{True}} from \(\Phi\) the agent has a \wit{} for \fingfb{\pv{\propM{\rootsCon{}}}{\valI{True}}}{\(\Phi\)}.
  %
  So, to find a counterexample to \issueInclusion{} we need to ensure the agent does not have a \wit{0} for the relevant \fingfr{1}:

  \begin{proposition}[\requ{3} and \issueInclusion{}]
    \label{prop:requ-WhyV-ces}
    \vspace{-\baselineskip}
    \begin{itenum}
    \item[\emph{If}:]
      Conditions~\ref{prop:requ-WhyV-ces:C},~\ref{prop:requ-WhyV-ces:Cing} and \ref{prop:requ-WhyVCes:noW} hold:
      \begin{enumerate}[label=\arabic*., ref=\arabic*]
      \item
        \label{prop:requ-WhyV-ces:C}
        \(\ed{}\) is an \eiw{0} \vAgent{} concludes \(\pv{\phi}{v}\) from \(\Phi\).
      \item
        \label{prop:requ-WhyV-ces:Cing}
        There is some event \(\ed{\flat}\) such that:
        \begin{itemize}
        \item
          \(\pv{\psi}{v'}\) being a \fc{} from \(\Psi\) for \vAgent{} through \(\ed{\flat}\) is a \requ{} of \(\ed{}\).
        \end{itemize}
      \item
        \label{prop:requ-WhyVCes:noW}
      \vAgent{} does not have a \wit{} for \fingfb{\(\pv{\psi}{v'}\)}{\(\Psi\)} when \vAgent{} concludes \(\pv{\phi}{v}\) from \(\Phi\).
      \end{enumerate}
    \item[\emph{Then}:]
      \issueInclusion{} does not hold.
    \end{itenum}
    \vspace{-\baselineskip}
  \end{proposition}

  \noindent%
  \autoref{prop:requ-WhyV-ces} primarily follows from \autoref{prop:requ-WhyV}:

  \begin{argument}{prop:requ-WhyV-ces}
    Assume conditions~\ref{prop:requ-WhyV-ces:C},~\ref{prop:requ-WhyV-ces:Cing} and \ref{prop:requ-WhyVCes:noW} hold.

    Conditions~\ref{prop:requ-WhyV-ces:C}~and~\ref{prop:requ-WhyV-ces:Cing} entail \fingfb{\(\pv{\psi}{v'}\)}{\(\Psi\)} answers \qWhy{} via \autoref{prop:requ-WhyV}.

    So, by \issueInclusion{}, an \eiw{0} the agent concludes \(\pv{\psi}{v'}\) from \(\Psi\) answers \qHow{}.
    Hence, by \autoref{def:witnessing} the agent must have a \wit{} for \fingfb{\(\pv{\psi}{v'}\)}{\(\Psi\)} when \vAgent{} concludes \(\pv{\phi}{v}\) from \(\Phi\).
    However, by Condition~\ref{prop:requ-WhyVCes:noW}, the agent does not have such a \wit{}.
    % for \fingfb{\(\pv{\psi}{v'}\)}{\(\Psi\)} when \vAgent{} concludes \(\pv{\phi}{v}\) from \(\Phi\).
  \end{argument}
\end{note}

\begin{note}
  The following chapter introduces a final idea to help identify \fc{1}.
  And, in turn \requ{1} for which the agent does not have a \wit{} for the corresponding \fingfr{}.
\end{note}


\section*{Summary}


\begin{note}
  This chapter defined \(\pv{\psi}{v'}\) being a \fc{} from \(\Psi\) for an agent through \(\ed{\flat}\) is a \requ{} of some \eiw[\(\ed{}\)]{0} an agent concludes \(\pv{\phi}{v}\) from \(\Phi\) --- when \(\ed{\flat}\) is a \se{} of \(\ed{}\) \emph{only if} \(\pv{\psi}{v'}\) is a \fc{} from \(\Psi\) for the agent through \(\ed{\flat}\).

  The role of this definition is to easily capture when \fingfb{\(\pv{\psi}{v'}\)}{\(\Psi\)} answers to \qWhy{}.
  \autoref{prop:requ-WhyV} stated conditions for \fingfb{\(\pv{\psi}{v'}\)}{\(\Psi\)} to answer \qWhy{} in terms of \requ{1} and \autoref{prop:requ-WhyV-ces} highlighted when \issueInclusion{} fails due to some \requ{1}.
\end{note}


\begin{note}
  % The definition of a \requ{} is not designed to do anything more than help identify answers to \qWhy{} and counterexamples to \issueInclusion{}.

  Note, the existence of \requ{1} does not presuppose \issueInclusion{} fails to hold.%
  \footnote{
    I.e.\ one may jointly hold:
    \begin{enumerate*}[label=(\alph*)]
    \item
      various \requ{1} exist, and
    \item
      answers to \qWhy{} are constrained by answers to \qHow{} via \issueInclusion{}
    \end{enumerate*}%
    .
  }
  Key to failure the failure of \issueInclusion{} is the existence of \requ{1} \emph{such that} \issueInclusion{} fails to hold.
  The argument for \requ{1} such that \issueInclusion{} fails to hold takes place in \autoref{cha:typical}.
\end{note}


% \footnote{
%     \nocite{Tichy:1976tp}%
%     The \itc{} of \qWhyV{} captures the presence of a lawlike constraint, as it applies regardless of whether or not the agent goes on to conclude \(\pv{\phi}{v}\) from \(\Phi\).

%     In the literature on subjunctive conditionals, the conditionals which constrain the development of events are sometimes termed `laws' (\cite{Chisholm:1955aa,Lewis:1979vm,Veltman:2005tj}) or thing that `lump together' certain facts (\cite{Kratzer:1981aa,Kratzer:1989aa}).
%   }


%%% Local Variables:
%%% mode: latex
%%% TeX-master: "master"
%%% TeX-engine: luatex
%%% End:

