\chapter{\requ{3}}
\label{cha:requs}

\begin{note}
  This chapter introduces the idea of a \(\pv{\psi}{v'}\) being a \fc{} from \(\Psi\) as a \requ{1} of an event in which an agent concludes \(\pv{\phi}{v}\) from \(\Phi\).

  The role of \requ{1} is to capture a core feature of failures of \issueConstraint{}.
  Following the introduction of \requ{1} we provide a detailed example of a \requ{0} in isolation from \issueConstraint{} and then clarify the connexion between \requ{1}, \qWhyV{}, and \issueConstraint{}.

  Still, the existence of \requ{1} does not presuppose \issueConstraint{} fails to hold.%
  \footnote{
    I.e.\ one may jointly hold:
    \begin{enumerate*}[label=(\alph*)]
    \item
      various \requ{1} exist, and
    \item
      answers to \qWhyV{} are constrained by answers to \qHowV{} via \issueConstraint{}
    \end{enumerate*}%
    .
    For more, see \autoref{prop:requ-not-n-ce} on \autopageref{prop:requ-not-n-ce}, below.
  }
  Key to failure the failure of \issueConstraint{} is the existence of \requ{1} \emph{such that} \issueConstraint{} fails to hold.
  We suggest provide a plausible example at the end of the chapter.
  However, the argument for \requ{1} such that \issueConstraint{} fails to hold takes place in \autoref{cha:typical}.
\end{note}



\section{\requ{3}}
\label{cha:requs:requs}

\begin{note}
  \begin{definition}[A \requ{0}]%
    \label{def:requ}%
    For any event \(\ed{}\) such that \(\ed{}\) is an event in which \vAgent{} concludes \(\pv{\phi}{v}\) from \(\Phi\):
    %
    \begin{itemize}
    \item
      \(\pv{\psi}{v'}\) being a \fc{} from \(\Psi\) for \vAgent{} throughout \(\ed{\flat}\) is a \emph{\requ{}} of \(\ed{}\).
    \end{itemize}

    \emph{If and only if}

    \begin{itemize}
    \item
      {
        \color{blue}
        For some minimal description \(\edo{m}\) of \(\edn{\flat}\) which captures \(\ed{}\) is in progress throughout \(\edn{\flat}\),%
      }
      clauses \ref{def:requ:se} and \ref{def:requ:fc} hold:
      \begin{enumerate}[label=\Alph*., ref=\Alph*]
      \item
        \label{def:requ:se}
        \(\ed[m]{\flat}\) is a \se{} of \(\ed{}\).
      \item
        \label{def:requ:fc}
        \(\pv{\psi}{v'}\) is a \fc{} from \(\Psi\) for \vAgent{} throughout \(\ed[m]{\flat}\).
      \end{enumerate}
    \end{itemize}
    \vspace{-\baselineskip}
  \end{definition}

  \noindent%
  The definition of a \requ{} captures a particular instance of \progEx{} (\autopageref{obs:PE}).
  Specifically, when there is some \se{} \(\ed{\flat}\) of \(\ed{}\) such that \(\ed{\flat}\) is a \se{} of \(\ed{}\) \emph{only if} \(\pv{\psi}{v'}\) is a \fc{} from \(\Psi\) for the agent throughout \(\ed{\flat}\).
\end{note}


\begin{note}
  Expanded, the definition of a \requ{} concerns an event \(\ed{\flat}\) such that:
  %
  \begin{itemize}%[label=\roman*., ref=(\roman*)]
  \item
    A conclusion of \(\pv{\phi}{v}\) from \(\Phi\) is in progress, and
  \item
    If the agents conclusion of \(\pv{\phi}{v}\) from \(\Phi\) in progress during \(\ed{\flat}\) happens, then the conclusion happens as a result of \(\ed{\flat}\).
  \item
    \(\pv{\psi}{v'}\) being a \fc{} from \(\Psi\) is required for \(\ed{\flat}\) to be an event such that a conclusion of \(\pv{\phi}{v}\) from \(\Phi\) is in progress.
  \end{itemize}
  %
  The conditions capture the key way in which a \ros{} answers \qWhyV{}.
  For, with the definition of a \requ{} and \autoref{prop:requ-WhyV} in hand, we only need to identify a \requ{} in order to find an answer to \qWhyV{}:

  \begin{proposition}[\requ{3} and \qWhyV{}]
    \label{prop:requ-WhyV}
    \vspace{-\baselineskip}
    \begin{itenum}
    \item[\emph{If}:]
      Conditions~%
      \ref{def:rCs:C}~and~\ref{def:rCs:Cing}~jointly hold:
      \begin{enumerate}[label=\arabic*., ref=\arabic*]
      \item
        \label{def:rCs:C}
        \(\ed{}\) is an event in which \vAgent{} concludes \(\pv{\phi}{v}\) from \(\Phi\).
      \item
        \label{def:rCs:Cing}
        There is some event \(\ed{\flat}\) such that:
        \begin{itemize}
        \item
          \(\pv{\psi}{v'}\) being a \fc{} from \(\Psi\) for \vAgent{} throughout \(\ed{\flat}\) is a \requ{} of \(\ed{}\).
        \end{itemize}
      \end{enumerate}
    \item[\emph{Then}:]
      A \ros{0} between \(\pv{\psi}{v'}\) and \(\Psi\) answers \qWhyV{}.
    \end{itenum}
    \vspace{-\baselineskip}
  \end{proposition}

  \begin{argument}{prop:requ-WhyV}
    \color{blue}
    Assume conditions~\ref{def:rCs:C}~and~\ref{def:rCs:Cing} hold.

    From Condition~\ref{def:rCs:C}, \(\ed{}\) is an event in which \vAgent{} concludes \(\pv{\phi}{v}\) from \(\Phi\).
    So, \qWhyV{} applies to \(\ed{}\).
    Our main task is to show the conditional of \qWhyV{} (\autopageref{questionWhyV}) is true.
    \medskip

    \noindent%
    Let \(\ed{\flat}\) by an event which satisfies the existential of Condition~\ref{def:rCs:Cing}.
    So, \(\pv{\psi}{v'}\) being a \fc{} from \(\Psi\) for \vAgent{} throughout \(\ed{\flat}\) is a \requ{} of \(\ed{}\).
    Hence, by \supportII{} A \ros{0} between \(\pv{\psi}{v'}\) and \(\Psi\) holds for \vAgent{} throughout \(\ed{\flat}\).

    So, it is immediately the case that:
    \begin{itenum}
    \item[\emph{If}:]
      \(\ed{\flat}\) is an event such that \(\ed{}\) is in progress.
    \item[\emph{Then}:]
      A \ros{0} between \(\pv{\psi}{v'}\) and \(\Psi\) holds for \vAgent{} throughout \(\ed{\flat}\)
    \end{itenum}
    %
    % %
    % Further, suppose a \ros{} between \(\pv{\psi}{v'}\) and \(\Psi\) fails to hold for \vAgent{} through \(\ed{\flat}\).

    % Now, as an instance of :
    % %
    % \begin{itenum}
    % \item[\emph{If}:]
    %   \(\pv{\psi}{v'}\) is a \fc{0} from \(\Psi\) for \vAgent{} throughout \(\ed{\flat}\).
    % \item[\emph{Then}:]
      
    % \end{itenum}
    % %
    % \noindent%
    % Hence, as a \ros{0} between \(\pv{\psi}{v'}\) and \(\Psi\) \emph{fails} to hold for \vAgent{} throughout \(\ed{\flat}\), it follows that \(\pv{\psi}{v'}\) is \emph{not} a \fc{0} from \(\Psi\) for \vAgent{} throughout \(\ed{\flat}\).

    % And, as \(\ed{\flat}\) satisfies the existential of \ref{def:rCs:Cing} and is \(\pv{\psi}{v'}\) is \emph{not} a \fc{0} from \(\Psi\) for \vAgent{} throughout \(\ed{\flat}\), it follows that \(\ed{\flat}\) is not, or does not develop into, \(\ed{}\).
    % This contradicts \(\ed{\flat}\) being a \se{0} of \(\ed{}\), and hence contradicts Condition~\ref{def:rCs:Cing}.
  \end{argument}

  The argument requires a little commentary.

  By assumption, \(\ed{\flat}\) is a \se{} of \(\ed{}\).
  In this respect, \(\ed{}\) is in progress.
  The way in which an answer to \qWhyV{} intuitively follows is that 

\end{note}


\begin{note}
  Given \autoref{obs:se-inst} (\autopageref{obs:se-inst}) and \autoref{obs:var:factor:fc} (\autopageref{obs:var:factor:fc}) it immediately follows \pv{\rootsCon{}}{\valI{True}} being a \fc{} from \(\Phi\) through \(\ed{\flat}\) is a \requ{} of \(\ed{}\), where \(\ed{}\) captures \autoref{illu:gist:roots:F}  \(\ed{\flat}\) and captures Step~\ref{illu:gist:roots:F:eq} of the \agents{} reasoning.

  For, by \autoref{obs:se-inst} \(\ed{\flat}\) is a \se{} of \(\ed{}\) and by \autoref{obs:var:factor:fc} \pv{\rootsCon{}}{\valI{True}} is a \fc{} from \(\Phi\) throughout \(\ed{\flat}\).
\end{note}




\section{An \illu{0}}
\label{sec:some-requ1-beginnote}


\begin{note}
  We work through a \scen{} to illustrate the definition of a \requ{}.
  We focus on identifying a single \requ{}, though the argument presented generalises to identify additional \requ{1} in the same \scen{0}.

  \begin{scenario}[Lucas numbers]%
    \label{scen:LucasNums}%
    The Lucas numbers are recursively defined as follows:%
    \footnote{
      Starting at \(2\), see \hyperlink{cite.OEIS.:aa}{OEIS sequence A000032} for more details.
    }
    %
    \[
      L_{n} = \left\{
        \begin{array}{ll}
          2 & \text{if } n = 0 \\
          1 & \text{if } n = 1 \\
          L_{n-1} + L_{n-2} & \text{if } n > 1 \\
        \end{array}
      \right.
    \]
    %
    An agent is alone and quite bored.
    The agent sets out to calculate the first ten Lucas numbers by applying the recursive definition.
    %
    The agent begins:
    %
    \[
      \begin{array}{cccccc}
        L_{0} & L_{1} & L_{2} & L_{3} & L_{4} & \cdots \\
        \hline
        2 & 1 & 3 & 4 & 7 & \cdots \\
      \end{array}
    \]
    %
    And, eventually concludes:
    \begin{center}
      \pv{\propI{The first ten Lucas numbers are 2, 1, 3, 4, 7, 11, 18, 29, 47, and 76}}{\valI{True}}
    \end{center}
    %
    From some \pool{} \(\Phi\) which captures the \agents{} understanding of the Lucas numbers.
  \end{scenario}

  \noindent%
  The event of interest is \(\ed{}\) is when the agent concludes \pv{\propM{L_{[:9]} = [2, \dots, 76]}}{\valI{True}} from \(\Phi\), after studying the definition of Lucas numbers.%
  \footnote{
    \propM{L_{[:9]} = [2, \dots, 76]} shortens \propI{The first ten Lucas numbers are 2, 1, 3, 4, 7, 11, 18, 29, 47, and 76}.
  }

  \begin{observation}[\requ{3} of \autoref{scen:LucasNums}]%
    \label{obs:LucasRequ}%
    There is some event \(\ed{\flat}\) such that:
    %
    \begin{itemize}
    \item
      \pv{\propM{L_{8} = 47}}{\valI{True}} being a \fc{} from \(\Phi\) throughout \(\ed{\flat}\) is a \requ{0} of \(\ed{}\).
    \end{itemize}
    \vspace{-\baselineskip}
  \end{observation}

  \noindent%
  The motivation for \autoref{obs:LucasRequ} largely follows the motivation given for a \se{} and \fc{} in connexion with the second step of the \agents{} reasoning in \autoref{illu:gist:roots:F} --- see \autoref{obs:se-inst} (\autopageref{obs:se-inst}) and \autoref{obs:var:factor:fc} (\autopageref{obs:var:factor:fc}), respectively.

  So, rather than work through the full definitions of a \se{} and a \fc{} we first identify an event of interest and then sketch the key observations.


  \begin{motivation}{obs:LucasRequ}
    By assumption \(e_{d}\) is such that the agent concludes \pv{\propM{L_{[:9]} = [2, \dots, 76]}}{\valI{True}} from \(\Phi\), where \(\Phi\) captures the \agents{} understanding of the Lucas numbers.

    Now, \propM{L_{[:9]} = [2, \dots, 76]} contains the first nine Lucas numbers.
    Implicit in the description of \autoref{scen:LucasNums} is that the agent has no calculated the first nine Lucas numbers prior to \(\ed{}\).
    And, explicit in the description of \autoref{scen:LucasNums} is that the agent concludes by applying the given definition.
    So, as \propM{L_{[:9]} = [2, \dots, 76]} contains the first nine Lucas numbers there must be some sub-event in which the agent concludes \pv{\propM{L_{9} = 76}}{\valI{True}} by adding \(L_{8}\) and \(L_{7}\).
    And, in turn, there must be a sub-event in which the agent concludes \pv{\propM{L_{8} = 47}}{\valI{True}} by adding \(L_{7}\) and \(L_{6}\), and so on\dots

    %\pv{\propM{L_{7} = 29}}{\valI{True}} and \pv{\propM{L_{6} = 18}}{\valI{True}}

    Fix \(\edn{\flat}\) as the event when the agent concludes \pv{\propM{L_{7} = 29}}{\valI{True}}, and take \(\edo{\flat}\) as a natural description of the event.
    \medskip

    \noindent%
    The first task is to show \(\ed{\flat}\) is a \se{} of \ed{}.

    The key to \(\ed{\flat}\) being a \se{} of \ed{} is Clause~\ref{assu:p:se:hCon} of \autoref{def:se} .
    Roughly:
    \emph{If} \(\ed{}\) happens \emph{then} \(\ed{}\) happens as a result of \(\ed{\flat}\).

    In other words, our task is to show the agent concludes \pv{\propM{L_{[:9]} = [2, \dots, 76]}}{\valI{True}} as a result of concluding \pv{\propM{L_{7} = 29}}{\valI{True}}.

    As with \autoref{obs:se-inst} (\autopageref{obs:se-inst}) this reduces to the observation that it is not possible for the agent to conclude \pv{\propM{L_{[:9]} = [2, \dots, 76]}}{\valI{True}} from \(\Phi\) if the \agents{} conclusion is not a result (in part) of figuring out \pv{\propM{L_{7} = 29}}{\valI{True}}.
    For, \(\Phi\) captures the \agents{} understanding of the Lucas numbers and there is no way for the agent to figure out \propM{L_{[:9]} = [2, \dots, 76]} without figuring out \propM{L_{7} = 29} unless the agent appeals to something other than their understanding of the Lucas numbers.
    \medskip

    \noindent%
    The second task is to show \pv{\propM{L_{8} = 47}}{\valI{True}} is a \fc{} from \(\Phi\) through~\(\ed{\flat}\).

    The broad idea parallels \autoref{obs:var:factor:fc} (\autopageref{obs:var:factor:fc}).
    In short, \(\ed{\flat}\) is an event in which the agent concludes \pv{\propM{L_{7} = 29}}{\valI{True}}.
    And, \(\ed{\flat}\) is such that the agent is concluding \pv{\propM{L_{[:9]} = [2, \dots, 76]}}{\valI{True}}.
    As observed the agent must conclude \pv{\propM{L_{8} = 47}}{\valI{True}} in order to conclude \pv{\propM{L_{[:9]} = [2, \dots, 76]}}{\valI{True}} from \(\Phi\).
    And, the agent has not yet concluded \pv{\propM{L_{8} = 47}}{\valI{True}}.
    Therefore, as there is some action available to the agent such that the agent continues to be concluding \pv{\propM{L_{[:9]} = [2, \dots, 76]}}{\valI{True}} from \(\Phi\), the same action also ensures involves the agent concluding \pv{\propM{L_{8} = 47}}{\valI{True}} from \(\Phi\).
  \end{motivation}

  \noindent%
  With minor adjustments the motivation for \autoref{obs:LucasRequ} also shows \pv{\propM{L_{9} = 76}}{\valI{True}} being a \fc{} from \(\Phi\) throughout \(\ed{\flat}\) is a \requ{0} of \(\ed{\flat}\).
  And, likewise, the motivation may be adapted to earlier events and numbers in the sequence until the base cases of \(L_{1}\) and \(L_{0}\), as these are given by the \agents{} understanding of the Lucas numbers.
\end{note}


% \begin{note}
%   A \requ{} is nothing special.
%   It's an observation about what is.
% \end{note}

% \paragraph*{Lost keys}

% \begin{note}
%   \requ{} does not require an event has happened.
%   In particular, consider development of an event.
%   \requ{} helps observe an event is not such that concluding.

%   \begin{scenario}[Lost keys]%
%     \label{illu:lost-key}%
%     An agent thinks they may have lost their keys.
%     They usually leave place my keys on the right side of their desk, near a copy of~\citeauthor{Vickers:1989tr}'s~\citetitle{Vickers:1989tr} they've been saving for a rainy day.
%     Their keys aren't there.

%     They've searched over, under, and beside the desk.
%     They haven't found their keys.

%     Still, the agent holds the following principle:
%     \begin{quote}
%       If they think of a place to look for an object, they do not conclude the object is lost without searching the place.
%     \end{quote}
%   \end{scenario}

%   \begin{observation}%
%     \label{obv:lK:requ}%
%     \pv{\propI{I have no further idea of where to look}}{\valI{True}} being a \fc{} from \(\Psi\) is a \requ{} of the agent concluding \pv{\propI{I have lost my keys}}{\valI{True}} from \(\Phi\).
%   \end{observation}

%   \begin{motivation}{obv:lK:requ}
%     Suppose agent is concluding.
%     Then, by \assuPP{} event in which agent concludes.
%     Therefore, \requ{} of any such event.
%     For, else the agent goes and searches.

%     Antecedent assumes there is some action.
%     Therefore, problem.
%   \end{motivation}

%   Note, this doesn't rule out \(l\) and no as a \fc{}.
%   The problem is, not in a position to determine which conclusion.
% \end{note}



\section{\requ{3} and \issueConstraint{}}
\label{sec:comining-ingredients}

\begin{note}
  Given \autoref{prop:requ-WhyV}, we only need to identify a \requ{} in order to find an answer to \qWhyV{}.
  Hence, the definition of a \requ{} helps state when \issueConstraint{} fails to hold:

  \begin{proposition}[\requ{3} and \issueConstraint{}]
    \label{prop:requ-WhyV-ces}
    \vspace{-\baselineskip}
    \begin{itenum}
    \item[\emph{If}:]
      Conditions~\ref{prop:requ-WhyV-ces:C},~\ref{prop:requ-WhyV-ces:Cing} and \ref{prop:requ-WhyVCes:noW} hold:
      \begin{enumerate}[label=\arabic*., ref=\arabic*]
      \item
        \label{prop:requ-WhyV-ces:C}
        \(\ed{}\) is an event in which \vAgent{} concludes \(\pv{\phi}{v}\) from \(\Phi\).
      \item
        \label{prop:requ-WhyV-ces:Cing}
        There is some event \(\ed{\flat}\) such that:
        \begin{itemize}
        \item
          \(\pv{\psi}{v'}\) being a \fc{} from \(\Psi\) for \vAgent{} throughout \(\ed{\flat}\) is a \requ{} of \(\ed{}\).
        \end{itemize}
      \item
        \label{prop:requ-WhyVCes:noW}
      \vAgent{} does not have a \wit{} for a \ros{} between \(\pv{\psi}{v'}\) and \(\Psi\) when \vAgent{} concludes \(\pv{\phi}{v}\) from \(\Phi\).
      \end{enumerate}
    \item[\emph{Then}:]
      \issueConstraint{} does not hold.
    \end{itenum}
    \vspace{-\baselineskip}
  \end{proposition}

  \noindent%
  \autoref{prop:requ-WhyV-ces} is more-or-less immediate from \autoref{prop:requ-WhyV}:

  \begin{argument}{prop:requ-WhyV-ces}
    Assume conditions~\ref{prop:requ-WhyV-ces:C},~\ref{prop:requ-WhyV-ces:Cing} and \ref{prop:requ-WhyVCes:noW} hold.

    Conditions~\ref{prop:requ-WhyV-ces:C}~and~\ref{prop:requ-WhyV-ces:Cing} entail a \ros{0} between \(\pv{\psi}{v'}\) and \(\Psi\) answers \qWhyV{} via \autoref{prop:requ-WhyV}.
    So, by \issueConstraint{} it must be the case that \vAgent{} has a \wit{} for the \ros{} between \(\pv{\psi}{v'}\) and \(\Psi\).
    However, by Condition~\ref{prop:requ-WhyVCes:noW}, the agent does not have a \wit{} for a \ros{} between \(\pv{\psi}{v'}\) and \(\Psi\) when \vAgent{} concludes \(\pv{\phi}{v}\) from \(\Phi\).
  \end{argument}
\end{note}


\begin{note}
  \autoref{prop:requ-WhyV-ces} provides a collection of conditions which, if satisfied.
  Specifically, in terms of \requ{1}.
  Still, the existence of \requ{1} is compatible with \issueConstraint{}.

  \begin{observation}%
    \label{prop:requ-not-n-ce}%
    \autoref{prop:requ-WhyV} is compatible with \issueConstraint{}.
  \end{observation}

  \begin{motivation}{prop:requ-not-n-ce}
    \requ{3} are about an agent concluding.
    \fc{} when the agent concludes.
    An agent may fail to have a \wit{} for \fc{}.

    However, \issueConstraint{}, \wit{} when the agent concludes.
    Therefore, so long as \wit{} when the agent concludes, fine.

    Two possibilities.
    \begin{enumerate}
    \item
      The agent may already have a \wit{} for a \ros{} which follows from a \fc{}.

      In other words, the agent may have already concluded \(\pv{\psi}{v'}\) from \(\Psi\).

      This is the case with respect to \autoref{obs:LucasRequ}.
      For, the agent has concluded \pv{\propM{L_{8} = 47}}{\valI{True}} from \(\Phi\) prior to the \agents{} conclusion of \pv{\propM{L_{[:9]} = [2, \dots, 76]}}{\valI{True}} from \(\Phi\).
    \item
      Agent may obtain a \wit{} when the agent concludes.

      In other words, the event in which the agent concludes \(\pv{\phi}{v}\) from \(\Phi\) is a \wit{} for the \ros{} between \(\pv{\psi}{v'}\) from \(\Psi\).

      This is the case with respect to \autoref{illu:gist:roots:F}.
      For, \pv{\rootsCon{}}{\valI{True}} being a \fc{} from \(\Phi\) is a \requ{} of some \se{}, though the event of interest is such that the agent concludes \pv{\rootsCon{}}{\valI{True}} from \(\Phi\).
    \end{enumerate}
    \vspace{-\baselineskip}
  \end{motivation}
\end{note}


\begin{note}
  Given \autoref{prop:requ-not-n-ce}, the idea of a \requ{} is compatible with \issueInclusion{}.
  Still, the idea of a \requ{1} hints the failure of \issueConstraint{}.
\end{note}


\begin{note}
  For example, consider \autoref{scen:LucasNums}.
  In particular, \(\ed{\flat}\) again.
  \(l_{8}\) was a \fc{}.
  It seems \pv{\propM{L_{10} = 123}}{\valI{True}} is likewise a \fc{} from \(\Phi\) throughout \(\ed{\flat}\).

  Is it possible that the agent is concluding and \(L_{10}\) is not a \fc{}?

  There is a clear sense in which the answer is yes.
  For, agent may be limited, or something may happen.

  Still, exclude such cases with a description.

  So, granting there is nothing which prevents the agent from continuing to calculate Lucas numbers, is it possible that the agent is concluding and \(L_{10}\) is not a \fc{}?

  I doubt this is the case.
  The same thing that gets \(L_{8}\) and \(L_{9}\) gets \(L_{10}\).

  Then, \requ{}.
  So, just need \se{}.

  Still, this is quite and focuses on a single \scen{}.
\end{note}


\section*{Summary}


\begin{note}
  This chapter defined \requ{1}.

  Specifically, we defined when \(\pv{\psi}{v'}\) being a \fc{} from \(\Psi\) for an agent throughout \(\ed{\flat}\) is a \requ{} of some event \(\ed{}\) in which an agent concludes \(\pv{\phi}{v}\) from \(\Phi\).
  This holds just in case \(\ed{\flat}\) is a \se{} of \(\ed{}\) \emph{only if} \(\pv{\psi}{v'}\) is a \fc{} from \(\Psi\) for the agent throughout \(\ed{\flat}\).

  The role of this definition is to easily capture when a \ros{} between \(\pv{\psi}{v'}\) and \(\Psi\) answers to \qWhyV{}.
  For, by \autoref{prop:requ-WhyV}, a \ros{} between \(\pv{\psi}{v'}\) and \(\Psi\) answers to \qWhyV{} when \(\pv{\psi}{v'}\) being a \fc{} from \(\Psi\) is a \requ{} of some \se{} of the event in which the agent concludes \(\pv{\phi}{v}\) from \(\Phi\).

  And, \autoref{prop:requ-WhyV-ces} highlighted when \issueConstraint{} fails to hold due to some \requ{1}.
\end{note}


\begin{note}
  The definition of a \requ{} is not designed to do anything more than help identify answers to \qWhyV{} and counterexamples to \issueConstraint{}.
\end{note}


% \footnote{
%     \nocite{Tichy:1976tp}%
%     The \itc{} of \qWhyV{} captures the presence of a lawlike constraint, as it applies regardless of whether or not the agent goes on to conclude \(\pv{\phi}{v}\) from \(\Phi\).

%     In the literature on subjunctive conditionals, the conditionals which constrain the development of events are sometimes termed `laws' (\cite{Chisholm:1955aa,Lewis:1979vm,Veltman:2005tj}) or thing that `lump together' certain facts (\cite{Kratzer:1981aa,Kratzer:1989aa}).
%   }


%%% Local Variables:
%%% mode: latex
%%% TeX-master: "master"
%%% TeX-engine: luatex
%%% End:

