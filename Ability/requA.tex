\chapter{\requ{3}}
\label{cha:requs}

\begin{note}
  A \requ{} is a relation between \(\pv{\psi}{v'}\) from \(\Psi\) being a \fc{} and an event in which an agent concludes \(\pv{\phi}{v}\) from \(\Phi\).
\end{note}

\begin{note}
  Important to generate counterexamples to \issueConstraint{}.
  Still, only certain \requ{1} generate counterexamples to \issueConstraint{}.%
  \footnote{
    I.e.\ one may jointly hold:
    \begin{enumerate*}[label=(\alph*)]
    \item
      various \requ{1} exist, and
    \item
      answers to \qWhyV{} are constrained by answers to \qHowV{} via \issueConstraint{}
    \end{enumerate*}%
    .
    For more, see \autoref{prop:requ-not-n-ce} on \autopageref{prop:requ-not-n-ce}, below.
  }
  Hence, we introduce and motivate the idea of a \requ{0} with \requ{1} which are compatible with \issueConstraint{}.
\end{note}

\section{\requ{3}}
\label{cha:requs:requs}

\begin{note}
  \begin{definition}[A \requ{0}]%
    \label{def:requ}%
    For any event \(\ed{}\) such that \(\ed{}\) is an event in which \vAgent{} concludes \(\pv{\phi}{v}\) from \(\Phi\):
    %
    \begin{itemize}
    \item
      \(\pv{\psi}{v'}\) being a \fc{} from \(\Psi\) for \vAgent{} throughout \(\ed{\flat}\) is a \emph{\requ{}} of \(\ed{}\).
    \end{itemize}

    \emph{If and only if}

    \begin{itemize}
    \item
      \begin{itenum}
      \item[\emph{If}:]
        \(\ed{\flat}\) is a \se{} of \(\ed{}\).
      \item[\emph{Then}:]
        \(\pv{\psi}{v'}\) is a \fc{} from \(\Psi\) for \vAgent{} throughout \(\ed{\flat}\).
      \end{itenum}
    \end{itemize}
    \vspace{-\baselineskip}
  \end{definition}

  \noindent%
  Broad idea of a \requ{} is that we fix an event, and then consider what must be the case with respect to some sub-event.
  Specifically, applied to an event in which an agent concludes, and holds just in case what must be the case is a \fc{}.%
  \footnote{
    \itc{2}, so either does not develop or \fc{}.

    Something about the event secures requirement.
  }%
  \footnote{
    Note, it is not always the case that \(\pvp{\phi}{v}{\Phi}\) is a \requ{} of an agent concluding \(\pv{\phi}{v}\) from \(\Phi\).
    For example, consider the partnered case from \autoref{obs:cds-arb}
  }
  % We define \requ{} in terms of \fc{1} as a \ros{} is an abstraction for which we only have sufficient conditions.
  % A \fc{} on the other hand, is something we have defined.
\end{note}

\subsection{Illustrations}
\label{sec:illustrations}

\begin{note}
  Work through a \scen{} to illustrate the definition of a \requ{}.
\end{note}

\paragraph*{Lucas numbers}

\begin{note}
  \begin{scenario}[Lucas numbers]%
    \label{scen:LucasNums}%
    The Lucas numbers are recursively defined as follows:%
    \footnote{
      Starting at \(2\), see \hyperlink{cite.OEIS.:aa}{OEIS sequence A000032} for more details.
    }
    %
    \[
      L_{n} = \left\{
        \begin{array}{ll}
          2 & \text{if } n = 0 \\
          1 & \text{if } n = 1 \\
          L_{n-1} + L_{n-2} & \text{if } n > 1 \\
        \end{array}
      \right.
    \]
    %
    An agent is alone and quite bored.
    The agent sets out to calculate the first ten Lucas numbers by applying the recursive definition.
    %
    The agent begins:
    %
    \[
      \begin{array}{cccccc}
        L_{0} & L_{1} & L_{2} & L_{3} & L_{4} & \cdots \\
        \hline
        2 & 1 & 3 & 4 & 7 & \cdots \\
      \end{array}
    \]
    %
    And, eventually concludes:
    %
    \[
      \pv{\propI{The first ten Lucas numbers are 2, 1, 3, 4, 7, 11, 18, 29, 47, and 76}}{\valI{True}}
    \]
    %
    From some \pool{} \(\Phi\).
  \end{scenario}

  Event of interest is \(\ed{}\) such that \propI{L\textsubscript{[:9]} = [2, 1, 3, 4, 7, 11, 18, 29, 47, 76]}{\valI{True}} from \(\Phi\).%
  \footnote{
    \propI{L\textsubscript{[:9]} = [2, 1, 3, 4, 7, 11, 18, 29, 47, 76]} shortens \propI{The first ten Lucas numbers are 2, 1, 3, 4, 7, 11, 18, 29, 47, and 76}.
  }

  \begin{observation}[\requ{3} of \autoref{scen:LucasNums}]%
    \label{obs:LucasRequ}%
    There is some sub-event \(\ed{\flat}\) of \(\ed{}\) such that:
    %
    \begin{itemize}
    \item
      \(\pv{\propI{L\textsubscript{8} = 47}}{\valI{True}}\) from some \pool{} \(\Phi_{8}\) is \requ{1} of the agent concluding .
    \end{itemize}
  \end{observation}

  \begin{motivation}{obs:LucasRequ}
    The agent concludes \pv{\propI{L\textsubscript{[:9]} = [2, 1, 3, 4, 7, 11, 18, 29, 47 76]}}{\valI{True}} from \(\Phi\).

    The agent reasons by the recursive definition.
    Hence, the agent obtains \(L_{9}\) by adding \(L_{8}\) and \(L_{7}\).
    So, agent concludes \(\pv{\propI{L\textsubscript{8} = 47}}{\valI{True}}\) from \(\Phi_{8}\).
    And, \pv{\propI{L\textsubscript{7} = 29}}{\valI{True}} from \(\Phi_{7}\)
    For, without \(L\textsubscript{8}\) and \(L\textsubscript{7}\), the agent does not get \(L\textsubscript{9}\).

    Sub-event \(e^{8}\)

    Now, \(e^{8}\) agent concludes.
    So, by parallel reasoning, \(L\textsubscript{7}\) and \(L\textsubscript{6}\).
    But, then \(L\textsubscript{7}\), \(L\textsubscript{6}\) and definition.
    Hence, action, and when the agent does this, \(L\textsubscript{8}\).

    So, must be \(L\textsubscript{8}\) and clear just prior to this, \(L\textsubscript{8}\) is a \fc{}.
  \end{motivation}

  The same applies to \(L\textsubscript{7}\), \(L\textsubscript{6}\), and so on until we reach the base cases of \(L\textsubscript{1}\) and \(L\textsubscript{0}\), as these are given.
\end{note}

\begin{note}
  A \requ{} is nothing special.
  It's an observation about what is.

  In this case, given an event, and works backwards given additional details to observe which \fc{1} held.
  Note, however, that when the agent concludes \pv{\propI{L\textsubscript{[:9]} = [2, 1, 3, 4, 7, 11, 18, 29, 47 76]}}{\valI{True}} from \(\Phi\) the agent has a \wit{} for all \ros{} mentioned.
\end{note}

\paragraph*{Lost keys}

\begin{note}
  \requ{} does not require an event has happened.
  In particular, consider development of an event.
  \requ{} helps observe an event is not such that concluding.

  \begin{scenario}[Lost keys]%
    \label{illu:lost-key}%
    An agent thinks they may have lost their keys.
    They usually leave place my keys on the right side of their desk, near a copy of~\citeauthor{Vickers:1989tr}'s~\citetitle{Vickers:1989tr} they've been saving for a rainy day.
    Their keys aren't there.

    They've searched over, under, and beside the desk.
    They haven't found their keys.

    Still, the agent holds the following principle:
    \begin{quote}
      If they thinks of a place to look for an object, they do not conclude the object is lost without searching the place.
    \end{quote}
  \end{scenario}

  \begin{observation}
    \label{obv:lK:requ}
    If some action such that agent consider may be in  \(l\), then agent is not concluding.
  \end{observation}

  {
    \color{red}
  \begin{itemize}
  \item
    \pv{\propI{The agent has no further idea of where to look}}{\valI{True}} being a \fc{} from \(\Psi\) is a \requ{} of the agent concluding \pv{\propI{The agent has lost their keys}}{\valI{True}} from \(\Phi\).
  \end{itemize}
  }

  \begin{motivation}{obv:lK:requ}
    Suppose agent is concluding.
    Then, by \assuPP{} event in which agent concludes.
    Therefore, \requ{} of any such event.
    For, else the agent goes and searches.

    Antecedent assumes there is some action.
    Therefore, problem.
  \end{motivation}

  Note, this doesn't rule out \(l\) and no as a \fc{}.
  The problem is, not in a position to determine which conclusion.
\end{note}

\section{\requ{3}, \qWhyV{}, and \issueConstraint{}}
\label{sec:comining-ingredients}

\begin{note}
  Section links \requ{0} to \qWhyV{} and \issueConstraint{}.

  Start by defining a collection of conditions.
  Then, \qWhy{}.
  From this, additional condition to get counterexample to \issueConstraint{}.
\end{note}

\subsection{\requ{3} and \qWhyV{}}

\begin{note}
  \begin{proposition}[\requ{3} and \qWhyV{}]
    \label{prop:requ-WhyV}
    \vspace{-\baselineskip}
    \begin{itenum}
    \item[\emph{If}:]
      Conditions~%
      \ref{def:rCs:C}~and~\ref{def:rCs:Cing}~jointly hold:
      \begin{enumerate}[label=\arabic*., ref=\arabic*]
      \item
        \label{def:rCs:C}
        \(\ed{}\) is an event in which \vAgent{} concludes \(\pv{\phi}{v}\) from \(\Phi\).
      \item
        \label{def:rCs:Cing}
        There is some sub-event \(\ed{\flat}\) of \(\ed{}\) such that:
        \begin{itemize}
        \item
          \(\pv{\psi}{v'}\) being a \fc{} from \(\Psi\) for \vAgent{} throughout \(\ed{\flat}\) is a \requ{} of \(\ed{}\).
        \end{itemize}
      \end{enumerate}
    \item[\emph{Then}:]
      A \ros{0} between \(\pv{\psi}{v'}\) and \(\Psi\) answers \qWhyV{}.
    \end{itenum}
    \vspace{-\baselineskip}
  \end{proposition}

  \begin{argument}{prop:requ-WhyV}
    Assume conditions~\ref{def:rCs:C}~and~\ref{def:rCs:Cing}~jointly hold.

    From Condition~\ref{def:rCs:C}, \(\ed{}\) is an event in which \vAgent{} concludes \(\pv{\phi}{v}\) from \(\Phi\).
    So, \qWhyV{} applies to \(\ed{}\).

    Our task is to show the conditional of \qWhyV{} is true.

    Let \(\ed{\flat}\) by an event which satisfies the existential of Condition~\ref{def:rCs:Cing}.

    Further, suppose a \ros{} between \(\pv{\psi}{v'}\) and \(\Psi\) fails to hold for \vAgent{} through \(\ed{\flat}\).

    Now, as an instance of \supportII{}:
    %
    \begin{itenum}
    \item[\emph{If}:]
      \(\pv{\psi}{v'}\) is a \fc{0} from \(\Psi\) for \vAgent{} throughout \(\ed{\flat}\).
    \item[\emph{Then}:]
      A \ros{0} between \(\pv{\psi}{v'}\) and \(\Psi\) holds for \vAgent{} throughout \(\ed{\flat}\).
    \end{itenum}
    %
    \noindent%
    Hence, as a \ros{0} between \(\pv{\psi}{v'}\) and \(\Psi\) \emph{fails} to hold for \vAgent{} throughout \(\ed{\flat}\), it follows that \(\pv{\psi}{v'}\) is \emph{not} a \fc{0} from \(\Psi\) for \vAgent{} throughout \(\ed{\flat}\).

    And, as \(\ed{\flat}\) satisfies the existential of \ref{def:rCs:Cing} and is \(\pv{\psi}{v'}\) is \emph{not} a \fc{0} from \(\Psi\) for \vAgent{} throughout \(\ed{\flat}\), it follows that \(\ed{\flat}\) is not, or does not develop into, \(\ed{}\).
    This contradicts \(\ed{\flat}\) being a \se{0} of \(\ed{}\), and hence contradicts Condition~\ref{def:rCs:Cing}.
  \end{argument}
\end{note}



\subsection{\requ{3} and \issueConstraint{}}
\label{cha:binding:sec:requ-iC}



\paragraph*{Compatibility}


\begin{note}
  The existence of \requ{1} are compatible with \issueConstraint{}.

  \begin{observation}%
    \label{prop:requ-not-n-ce}%
    \autoref{prop:requ-WhyV} is compatible with \issueConstraint{}.
  \end{observation}

  \begin{motivation}{prop:requ-not-n-ce}
    \requ{3} are about an agent concluding.
    \fc{} when the agent concludes.
    An agent may fail to have a \wit{} for \fc{}.

    However, \issueConstraint{}, \wit{} when the agent concludes.
    Therefore, so long as \wit{} when the agent concludes, fine.

    Two possibilities.
    \begin{enumerate}
    \item
      The agent may already have a \wit{} for a \ros{} which follows from a \fc{}.

      In other words, the agent may have already concludes \(\pv{\psi}{v'}\) from \(\Psi\).
    \item
      Agent may obtain a \wit{} prior to conclusion.

      For example, first \scen{0}.
      If concluding sequence, \fc{}.
      Prior to getting \(L_{8}\), \fc{}.

      However, expectation of \wit{} when sequence.
      Indeed, \(L_{9}\) from \(L_{8} + L_{7}\).
    \end{enumerate}
    \vspace{-\baselineskip}
  \end{motivation}

  \noindent%
  Hence, \autoref{prop:requ-WhyV-ces} does not guarantee the existence of counterexamples to \issueConstraint{}.
\end{note}



\paragraph*{Tension}


\begin{note}
  The existence of \requ{1} is suggests tension with \issueConstraint{}.
\end{note}

\begin{note}
  \begin{observation}[Tension]
    \label{obs:reqTension}%
    \autoref{prop:requ-WhyV} is in tension with \issueConstraint{}.
  \end{observation}

  Before motivating \autoref{obs:reqTension} we establish a consequence of \autoref{prop:requ-WhyV}:

  \begin{proposition}[\requ{3} and \issueConstraint{}]
    \label{prop:requ-WhyV-ces}
    \vspace{-\baselineskip}
    \begin{itenum}
    \item[\emph{If}:]
      Conditions~\ref{prop:requ-WhyV-ces:C}~and~\ref{prop:requ-WhyV-ces:Cing}~jointly hold:
      \begin{enumerate}[label=\arabic*., ref=(\arabic*)]
      \item
        \label{prop:requ-WhyV-ces:C}
        \(\ed{}\) is an event in which \vAgent{} concludes \(\pv{\phi}{v}\) from \(\Phi\).
      \item
        \label{prop:requ-WhyV-ces:Cing}
        There is some sub-event \(\ed{\flat}\) of \(\ed{}\) such that:
        \begin{enumerate}[label=\alph*., ref=(\arabic{enumi}\alph*)]
        \item
          \label{prop:requ-WhyV-ces:Cing:requ}
          \(\pv{\psi}{v'}\) being a \fc{} from \(\Psi\) for \vAgent{} throughout \(\ed{\flat}\) is a \requ{} of \(\ed{}\).
        \end{enumerate}
      \end{enumerate}
    \item[\emph{And}:]
      \label{prop:requ-WhyVCes:noW}
      \vAgent{} does not have a \wit{} for a \ros{} between \(\pv{\psi}{v'}\) and \(\Psi\) when \vAgent{} concludes \(\pv{\phi}{v}\) from \(\Phi\).
    \item[\emph{Then}:]
      \issueConstraint{} does not hold.
    \end{itenum}
    \vspace{-\baselineskip}
  \end{proposition}

  \begin{argument}{prop:requ-WhyV-ces}
    Suppose both antecedents holds.

    Note conditions~\ref{prop:requ-WhyV-ces:C}~and~\ref{prop:requ-WhyV-ces:Cing} from the first antecedent entail a \ros{0} between \(\pv{\psi}{v'}\) and \(\Psi\) answers \qWhyV{} via \autoref{prop:requ-WhyV}.
    So, by \issueConstraint{} it must be the case that \vAgent{} has a \wit{} for the \ros{} between \(\pv{\psi}{v'}\) and \(\Psi\).
    However, by the second antecedent, \vAgent{} does not have a \wit{} for a \ros{} between \(\pv{\psi}{v'}\) and \(\Psi\) when \vAgent{} concludes \(\pv{\phi}{v}\) from \(\Phi\).
  \end{argument}


  \begin{motivation}{obs:reqTension}
    By \autoref{prop:requ-WhyV-ces}, \fc{} which is a \requ{} such that the agent does not have a \ros{}.

    \supportII{}.
    This is possible.
  \end{motivation}

  Example, \(L_{10}\).
  When conclude \(L_{9}\), one more?
  If so, \fc{}.
  Now, suppose \(L_{10}\) is \emph{not} a \fc{}.

  Is this compatible with conclusion of \(L_{9}\) from \(\Phi_{L_{9}}\).

  At issue is not whether \eval{} \(L_{9}\).
  That clearly happened.
  Rather, understanding of the event in which conclude \(L_{9}\).
\end{note}


%%% Local Variables:
%%% mode: latex
%%% TeX-master: "master"
%%% TeX-engine: luatex
%%% End:

