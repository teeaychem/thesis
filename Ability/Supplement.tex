\chapter{\LCS{}}
\label{cha:lcs-extra}

\section{Contrast to other conditions}
\label{sec:contr-other-cond}

\begin{note}
  Two conditions.

  First, \citeauthor{Wright:2011wn} on warrant transmission.

  Second, \citeauthor{Weisberg:2010to} on bootstrapping.
\end{note}

\begin{note}
  Use to argue that \nI{} is unique.

  Also, observe some interesting things about \nI{}.

  \citeauthor{Wright:2011wn} by difference in extension.
  \citeauthor{Weisberg:2010to} by difference in intension.

  Respective approaches are motivated by ease of demonstrating the relevant difference in extension and intension.
  \citeauthor{Wright:2011wn}'s template(s) match scenarios fairly well, and so extension.
  \citeauthor{Weisberg:2010to}'s make certain things explicit with work for difference in intension.

  However, will suggest that observations made with respect to \citeauthor{Weisberg:2010to} also extend to \citeauthor{Wright:2011wn}.
\end{note}

\subsection{Wright on warrant transmission (failure)}

\begin{note}[How transmission failure relates]
  Inclined to think these are really the same.

  Note, in particular, \citeauthor{Wright:2000tq} is interested in transmission of \emph{second-order} warrant.
  So, not about whether the agent has warrant, but whether the agent may \emph{claim} to have warrant.
  (\Citeyear[89]{Wright:2011wn})

  In parallel, \nI{} is about claiming support, and not about whether the agent has support.
\end{note}

\begin{note}
  Basic ideas go back (at least) to the Proper Execution Principle of~\textcite{Wright:1991vn}, and in particular the \widt{} of~\textcite{Wright:2000tq} and (\Citeyear{Wright:2003aa}).\nolinebreak
  \footnote{See also~\textcite{Wright:1986ug,Wright:2002uk} and \textcite{Wright:2004uo}.}

  The \widt{} is as follows:
  \phantlabel{widt}
  \begin{quote}
    A body of evidence, \emph{e}, is an information-dependent warrant for a particular proposition P if regarding \emph{e} as warranting P rationally requires certain kinds of collateral information, \emph{I}.
    Some examples of such \emph{e}, P and \emph{I} [\dots] have the feature that elements of the relevant \emph{I} are themselves entailed by P (together perhaps with other warranted premises).
    In that case, any warrant supplied by \emph{e} for P will not be transmissible to those elements of \emph{I}.\nolinebreak
    \mbox{}\hfill\mbox{(\Citeyear[143]{Wright:2000tq})}
  \end{quote}

  The ellipses skip a quick illustration given by~\citeauthor{Wright:2000tq} in favour of the following illustration.
  \begin{quote}
    \vspace{-\baselineskip}
    \begin{illustration}
      You go to the zoo, see several zebras in an enclosure, and opine that these animals are zebras.
      Well, you know what zebras look like, and these animals look just like that.
      Surely you are fully warranted in your belief.
      But if the animals are zebras, then it follows that they are not mules painstakingly and skilfully disguised as zebras.\linebreak
      \mbox{}\hfill\mbox{(\Citeyear[154]{Wright:2000tq})}
      \newline\mbox{ }
    \end{illustration}
  \end{quote}

  Here, the body of evidence, \emph{e}, is what you've seen, the proposition P is that those animals are zebras.
  At issue is whether the warrant for P transmits to the proposition that those animals are not mules <adjectives> disguised to look just like zebras.
  In other words, at issue is whether the proposition that those animals are not mules <adjectives> disguised to look just like zebras is collateral information required for what you've seen to warrant those animals being zebras.

  From a broader perspective, the relevant collateral information is, well, there need be no specific collateral information across all possible ways of filling out the remaining details of the scenario, so let's say the collateral information is that things are as they appear.
  If so, the noted proposition is certainly required.

  More specifically, you're at a zoo, so something looking like a zebra seems sufficient to claim warrant that the thing is a zebra, and hence not a disguised mule.
  As such, \citeauthor{Wright:2000tq} holds there is a problem because, generally speaking, \dots

  \begin{quote}
    \dots\space there are external preconditions for the effectiveness of your\linebreak method---casual observation---whose satisfaction you will very likely, without compromise of the warrant you acquire for those beliefs, have done nothing special to ensure.

    [\dots]

    Can the warrants you acquire licitly be transmitted to the claim that those preconditions \emph{are} met---or at least that they are not frustrated in those specific respects?
    It should seem obvious that they cannot.\linebreak
    \mbox{}\hfill\mbox{(\Citeyear[154]{Wright:2000tq})}
  \end{quote}
  We won't go further into why \citeauthor{Wright:2000tq} thinks the result should seem obvious.
  Rather, the above should give you an idea of the phenomenon \citeauthor{Wright:2000tq} is interested in.
  And, with this, we can begin a comparison with \nI{}.
\end{note}

\begin{note}
  In relation to \nI{}, similar idea of undercutting.\nolinebreak
  \footnote{
    Indeed, \citeauthor{Wright:1991vn} notes that Stephen Yablo suggested the kind of defeat in question might be called undercutting in reference to \citeauthor{Pollock:1987un} (\Citeyear[95,fn.9]{Wright:1991vn}).

    I should perhaps note here that I developed~\nI{} after struggling to apply the ideas of \citeauthor{Wright:2011wn} to the scenarios of interest involving ability.
    And, after developing an initial draft of~\nI{} I took to the literature to see if there were any developed ideas that are either equivalent or imply \nI{}.
    This, quite naturally, led me to \citeauthor{Pollock:1987un}'s distinction between overriding and undercutting defeaters, and some of the references above which use `undercutting' in a broader sense than \citeauthor{Pollock:1987un}'s original formulation.
    Hence, it seemed to me that framing \nI{} in terms of identifying something of an undercutting defeater might be a helpful guide.
    If I had found this footnote earlier, I may have had an easier time developing the initial draft of~\nI{}.
  }
  {
    \color{expand}
    Information-dependence blocks transmission of warrant, but does not suggest anything about the relevant elements of \emph{I}.
  }

  \citeauthor{Wright:2000tq} views this as a requirement, I haven't made this move.
  Significant part of \citeauthor{Wright:2000tq}\nolinebreak
  \footnote{
    See Pryor.
    This is what dogmatism denies.
  }
  , but I don't think this is the thing to focus on.\nolinebreak
  \footnote{
    You might be inclined to think understanding of claimed support should be strengthened.
    I don't want to take a stance on this, and hence problematic for me to distinguish on this basis.
  }

  {% to delete?
    Still, looking at the \emph{form} of \nI{} and \citeauthor{Wright:2000tq}'s \widt{}, there is a difference.
  }

  Instead
  \nI{} is concerned with the way in which in agent uses claimed support for a pair of propositions to claim support (or warrant) for some other proposition.
  By contrast, the \widt{} is concerned with preconditions for claiming warrant (or support) for a proposition that `might' be used to claim support.

  How agent claims support for \(\psi\) given claimed support for \(\phi\) and an implication from \(\phi\) to \(\psi\).

  \(\phi\) and \(\phi \rightarrow \psi\).
  Whether evidence, or claimed support, for this pair requires some collateral information entailed by \(\psi\).

  So, \nI{} doesn't care too much about \(\phi\) and \(\phi \rightarrow \psi\) whereas the template does.

  This is the thing of interest.

  Note, not possible to apply the template in a different way.

  So, in terms of \ideaCS{}.
  If going by value, require \(\psi\) to be the case.

  For template, need some warranted proposition P.
  Can't be \(\psi\), as template needs warrant, which we're denying.
  So, need something between \(\phi\) and \(\phi \rightarrow \psi\) and \(\psi\).
  Seems there's no proposition here.

  At issue is whether this difference in form corresponds to a difference in substance.
\end{note}

\begin{note}[The revised template]
  The \widt{} is intuitive, but has some downsides.

  Foremost, we would like additional clarity with respect to collateral information.
  As things stand, it's a little vague as to what constitutes and external preconditions for the effectiveness of a method.
  Further exposition might resolve this problem, but \dots

  More significant the \widt{} has been surpassed.\nolinebreak
  \footnote{
    This is a somewhat subtle issue.

    The revised template is, strictly speaking, a revision of the disjunctive template.
    And, \citeauthor{Wright:2002uk} initially distinguished the two templates:

    The \widt{} was designed to identify failures of transmission following from accumulation of defeasible evidence.

    And, by contrast, the disjunctive template was designed to identify failures of transmission following from by some faculty, such as perception or memory.

    See, for example, \textcite{Wright:2002uk} in which both templates are discussed separately, and \textcite[91]{Wright:2011wn} where the difference in motivation is restated.

    Still, \citeauthor{Wright:2011wn} observes that both templates the `base' for failure of transmission is the same in both cases.
    And, in turn, that the initial formulation of the disjunctive template yields unintuitive results when applied to cases covered by the \widt{} is a significant problem.
    (\Citeyear[91]{Wright:2011wn})

    So, \wrt{} is, strictly speaking, not a revision of the \widt{}, but rather the disjunctive template.
    However, \wrt{} is also designed to apply to the cases covered by the \widt{}, given that \citeauthor{Wright:2011wn}
    holds that both the \widt{} and the disjunctive template capture the same core phenomenon.
    Therefore, we have omitted these turns from the body of the paper.
  }
  (\Citeyear[90]{Wright:2011wn})
  This doesn't prevent a comparison, as such, but it may lead one to consider the comparison disingenuous.
  I don't think this is the case, and I began with the information-dependence as it is the spirit, rather than the letter, of the template which is at issue.

  So, to contrast \citeauthor{Wright:2000tq}'s template and \nI{} in detail, let's switch to \phantlabel{wrt}\citeauthor{Wright:2011wn}'s revised template:

  \begin{quote}
    Where A entails B, a rational claim to warrant for A is not transmissible to B if there is some proposition C such that:
    \begin{enumerate}[label=\roman*., ref=(\roman*)]
    \item\label{WT:i} The process/state of accomplishing the relevant putative warrant for A is subjectively compatible with C's holding: things could be with one in all respects exactly as they subjectively are yet C be true
    \item\label{WT:ii} C is incompatible (not necessarily with A but) with some presupposition of the cognitive project of obtaining a warrant for A in the relevant fashion, and
    \item\label{WT:iii} Not-B entails C\nolinebreak
      \mbox{}\hfill\mbox{(\Citeyear[93]{Wright:2011wn})}
    \end{enumerate}
  \end{quote}
  Where
  \begin{quote}
    A presupposition of a cognitive project is any condition P such that to doubt P (in advance of executing the project) would rationally commit one to doubting the significance, or competence of the\linebreak project, irrespective of its outcome.\nolinebreak
    \mbox{}\hfill\mbox{(\Citeyear[92]{Wright:2011wn})}
  \end{quote}

  In relation to the \widt{} discussed above:
  \citeauthor{Wright:2011wn}'s account of presuppositions of cognitive projects clarifies what collateral information amounts to.
  Cases of transmission failure are going to arise when one attempts to claim warrant for a condition for which doubt toward would undercut any outcome of the project.\nolinebreak
  \footnote{
    What matters is whether the relevant cognitive project has a presupposition of this kind, not whether the agent has done `anything special to ensure such presuppositions are satisfied.
    }

  In turn,~\ref{WT:i} to~\ref{WT:iii} detail how warrant for the relevant proposition depends on such collateral information.
\end{note}

\begin{note}
  The core intuition remains the same:
  Failure of transmission from a fixed proposition to conditions that need to be met in order for the agent to claim warrant for the fixed proposition.
\end{note}

\begin{note}[Applied to a case]
  Let's apply \wrt{} to the illustration used above to check:

  The relevant instances of A and B are, respectively:
  \begin{itemize}
  \item[A.] Those animals are zebras
  \item[B.] Those animals are not mules disguised to look like zebras
  \end{itemize}
  And, we may take C to be not-A. (\Citeyear[90]{Wright:2011wn})

  Now,~\ref{WT:i} to~\ref{WT:iii} are satisfied:

  \begin{itemize}
  \item[{\hyperref[WT:i]{i:}}] The process of accomplishing putative warrant for A is that the animals appear to be zebras, and things could be exactly as they \emph{appear} to be and yet the animals are not zebras.
  \item[{\hyperref[WT:ii]{ii:}}] The animals not being zebras is incompatible with A, of course --- given that C is not-A.\nolinebreak
    \footnote{
      For more details: (\Citeyear[90--96]{Wright:2011wn})
    }
    And, more broadly, the animals not being zebras is incompatible with moving from appearance to fact.
  \item[{\hyperref[WT:iii]{iii:}}] If those animals \emph{are} mules disguised to look like zebras, then those animals are not zebras.
  \end{itemize}
  Hence, \wrt{} identifies a failure of transmission in much the same way as we saw above with respect to the \widt{}.
\end{note}

\begin{note}[Entailment]
  Now, turning to the comparison proper.

  First, we'll map A and B from \wrt{} to \(\phi\) and \(\psi\), respectively, from \nI{}.

  An immediate difference is that \wrt{} requires an entailment between the relevant A and B while \nI{} does not include such a requirement.
  % requires that the agent has claimed support that if \(\phi\) has value \(v\) then \(\psi\) has value \(v'\).
  Still, to keep things simple, we'll assume that the agent has claimed support for an entailment from \(\phi\) having value \(v\) to \(\psi\) having value \(v'\).
  \wrt{} doesn't require that the agent has warrant, or has claimed support, for the entailment between A and B, and hence will apply in the case that the agent has.
    % So, \ref{nI:received-info} may be seen as an instance of \wrt{}'s initial condition with some superfluous detail.

  Likewise, \wrt{} is concerned with whether a claim to warrant for A is transmissible to B in general, and so not assume the agent has claimed warrant for A.
  So, as~\ref{nI:background} requires that the agent has reasoned to \(\phi\), we may consider~\ref{nI:background} as superfluous from the perspective of \wrt{}.
\end{note}

\begin{note}[Gist of why these are different]
  \color{red}
  Hence, it seems to me at issue is whether~\ref{WT:i} -- \ref{WT:iii} from \wrt{} and~\ref{nI:inclusion} do different things.
\end{note}

\begin{note}[Two questions]
  Let's break this down into two questions.

  \begin{enumerate}
  \item Whether an instance of~\ref{nI:inclusion} obtaining means that the agent makes a presupposition of the kind identified by \wrt{}.
  \item And, conversely, whether an instance~\ref{nI:inclusion} obtains if the agent has made a presupposition of the kind identifies by \wrt{}.
  \end{enumerate}
\end{note}

\begin{note}[Second question]
  The second question is straightforward to answer in the negative.

  Testimony, sight, whatever.
  Here, doesn't need to be any other way for the agent to claim support for the relevant proposition.
  {
    \color{red}
    To illustrate, zebra.
    Not in a position to claim support by some other way that moving from appearance is bad.
  }
\end{note}

\begin{note}[First question]
  The first question is more involved, and requires some care.

  Consider again the general presupposition that things are as they appear.
  Or, even more generally, that one is not in some sceptical scenario, such as a dream or a vat (cf.~\Citeyear{Wright:2002uk}, \Citeyear[97--98]{Wright:2011wn}).
  The difficult here is that it's easy to trivialise the question if the relevant presupposition is any presupposition.
  For, it seems any cognitive project will require some presupposition.

  Instead, the question is whether~\ref{nI:inclusion} obtaining means there is some \emph{related} presupposition.

  And, it seems this need not the case.

  For,~\ref{nI:inclusion} is, intuitively, about whether an agent is confident they have some way to claim support for \(\psi\), other than appealing to \(\phi\) and an implication from \(\phi\) to \(\psi\).
  But, it doesn't seem to follow that doubt about whether the agent has some other way of claiming support for \(\psi\) prior to claiming support for \(\phi\) and the implication from \(\phi\) to \(\psi\) would undercut claiming support for \(\phi\) or the implication from \(\phi\) to \(\psi\).

  I suspect this point is best argued for by illustrations, and we will consider a handful below.
  Still, it may be helpful to first outline the target of such illustrations in some detail.

  \ref{nI:inclusion} is about whether an agent is confident they have some other way to claim support for \(\psi\), but consists of two parts.
  \ref{nI:inclusion:position} requires that the agent is confident that the support claimed for \(\phi\) and would be \mom{} if the agent is not in a position to claim support for \(\psi\) some other way.
  And,~\ref{nI:inclusion:bound} requires that the agent is confident that the claimed support for \(\phi\) is a guarantee of sorts for claiming support for \(\psi\).

  Now, we're interested in deriving a related presupposition from \ref{nI:inclusion}.
  Still, the presupposition needs to be with respect to claimed support.
  So, as~\ref{nI:inclusion:bound} is a condition which concerns (as yet) unclaimed support, the related should follow from~\ref{nI:inclusion:position} --- in particular, from the possibility of the claimed support for \(\phi\) being \mom{}.

  However, claimed support for \(\phi\) being \mom{} reduces to either the claimed support indicating \(\phi\) has some value it does not have (misled) or the claimed support relies on factors that do not indicate the value of the \(\phi\) (mistaken).

  The point here is that claimed support being \mom{} is a relatively broad phenomenon.
  For example, an instance of inductive support may indicate the value of \(\phi\), and hence not be mistaken, but misled due to constrained sampling.
  Consider, by way of quick illustration, testing a random number generator by sampling its output.
  It may take a significant sample size to identify a bias, and hence bug in the source code.

  Yet, \citeauthor{Wright:2011wn}'s notion of a presupposition requires that doubt about the presupposition is such that doubt about the presupposition, \emph{in advance of following through on the project}, would \emph{require} doubt about the significance or competence of the project --- regardless of its outcome.

  So, suppose an instance of~\ref{nI:inclusion} may obtain because an agent has claimed inductive support for \(\phi\), has claimed support that \(\phi\) entails \(\psi\), and has some independent check on whether \(\psi\) is the case.

  If such an instance of~\ref{nI:inclusion} obtaining means that the agent makes a presupposition of the kind identified by \wrt{}, then the relevant presupposition should concern the nature of the claimed inductive support.

  However, it seems fundamental to claimed inductive support that one may doubt the inductive support is not misled without a requirement that one doubts the significance, or competence, of claiming such inductive support.

  I may doubt that I have obtained a sufficiently large sample to conclude that there are no bugs in the source code of the random number generator without being required to doubt the significance, or competence, of the sample acquired.
\end{note}

\begin{note}
  \color{red}
  The big difference is doubt versus an \requ{}.
  \requ{2} is weaker, in sense that it's just something that follows.
  However, as \requ{} is weaker it might also be easier to deal with --- `reasoning'.
  (As kind of seen in the zebra case.)
\end{note}

\begin{note}
  To summarise:
  \ref{nI:inclusion} concerns (an agent's confidence in) a particular kind of relationship holding between claimed support for \(\phi\) and claiming support for \(\psi\) from the perspective of whether the respective instances of support are (or would be) \mom{}.
  This relationship may arise from claimed inductive support for \(\phi\).
  If so, a positive answer to the first question would require a corresponding presupposition with respect to the claimed indicative support for \(\phi\).
  Yet, such a presupposition seems incompatible with the nature of inductive support.
\end{note}

\begin{note}
  Stress, briefly, that this does not indicate anything problematic about \citeauthor{Wright:2011wn}'s template.
  I'm inclined to think the template is sound.
  The issue is whether (at least some) of the instances captured by \nI{} fall outside the scope of \citeauthor{Wright:2011wn}'s template.
  Given that both \nI{} and \citeauthor{Wright:2011wn}'s template are sufficient, there's no tension between the two if it is the case.
\end{note}

\begin{note}
  Let's now return to~\autoref{ill:CE:main} from the start of this section in which we examined a researcher may claim support that there are no counterexamples to a theory they have developed.

  Given that we have already seen how \nI{} applies to both illustrations, and outlined the theoretical difference between \nI{} and \wrt{}, we will focus only on why \wrt{} does not seem to apply to the illustration.
  In particular, why it seems there is no plausible candidate for the required `C proposition' of \wrt{}.
\end{note}

\begin{note}[\autoref{ill:CE:main}]
  \autoref{ill:CE:main} considered a researcher who has claimed inductive support for some theory.
  The instance of reasoning we took interest with was as follows:

  \begin{itemize}
  \item I have claimed support that the theory is adequate.
  \item So, given the claimed support, theory is adequate.
  \item Therefore, as the theory is adequate, given the claimed support, it follows that there are no counterexamples.
  \item Hence, I claim support that there are no counterexamples to the theory.
  \end{itemize}

  As we assumed an entailment from an adequate theory to an absence of counterexamples to the theory, we have the following two instances of A and B with respect to \wrt{}:

  \begin{enumerate}[label=\Alph*., ref=(\Alph*)]
  \item\label{wrt:difference:theory:A} The theory is adequate
  \item\label{wrt:difference:theory:B} There are no counterexamples to the theory.\nolinebreak
    \footnote{
      With respect to~\autoref{ill:CE:colleague}, we would have:
      \begin{enumerate}[label=\Alph*., ref=(\Alph*)]
      \item The colleague has failed to identify a counterexample to the theory.
      \end{enumerate}
    }
  \end{enumerate}

  If we are to identify failure of warrant transmission from~\ref{wrt:difference:theory:A} to~\ref{wrt:difference:theory:B} via \wrt{}, then there must be some proposition C such that (paraphrased):

  \begin{enumerate}[label=\roman*., ref=(\roman*)]
  \item\label{wrt:CE:maini} The process of claiming warrant for the theory being adequate is subjectively compatible with C holding.
  \item\label{wrt:CE:mainii} C is incompatible with either the adequacy, or some presupposition of the cognitive project of claiming warrant for the adequacy, of the theory.
  \item\label{wrt:CE:mainiii} The existence of a counterexample to the theory entails C.
  \end{enumerate}

  Well,~\ref{wrt:CE:maini} seems okay.
  Interested in claimed inductive warrant/support.
  And, claiming inductive support seems subjectively compatible with an entailment from some counterexample holding.
  It seems possible that things could be exactly as they subjectively are, yet the theory is inadequate because there is an unobserved instance of the phenomenon which constitutes a counterexample to the theory.

  So,~\ref{wrt:CE:mainii} and~\ref{wrt:CE:mainiii}.

  Working backwards.

  From~\ref{wrt:CE:mainiii}:
  C needs to be entailed by the existence of a counterexample.

  Paired with~\ref{wrt:CE:mainii}, the existence of a counterexample needs to entail something that is incompatible with either the adequacy, or some presupposition of the cognitive project of claiming warrant for the adequacy, of the theory.

  The problem:
  Claiming inductive warrant.
  Seems compatible with some counterexample holding.
  Applied to various instances of the phenomenon, and the theory holds up.
  Possible that it doesn't hold up under some instance of the phenomenon.

  So,
  Suppose the existence of a counterexample entails something that is incompatible with either the adequacy, or some presupposition of the cognitive project of claiming warrant for the adequacy, of the theory.
  Then, it seems the theory denies the possibility of such a counterexample.

  The difficulty is that we're talking generally about some theory for which the researcher has claimed inductive warrant for.
  I see no reason to think that any theory which fits this broad description will deny the possibility of certain counterexamples.

  There, may be that there are assumptions.
  Theories are built on other theories.
  However, interest is in a counterexample to the theory --- not a counterexample to theoretical foundations.

  Claiming warrant that there are no counterexamples in general seems to be the issue, rather than the specific kind a counterexample that would be required for \wrt{} to apply.

  Indeed, we can revise the relevant B instance given \citeauthor{Wright:2011wn}'s notion of a presupposition:
  \begin{enumerate}[label=\Alph*\('\)., ref=(\Alph*\('\))]
    \setcounter{enumi}{1}
  \item No counterexample consistent with the presuppositions.
  \end{enumerate}

  Evaluation of the reasoning seems the same.
  Indeed, natural assumption that there are no such presuppositions, so the concerns raised in~\autoref{ill:CE:main} remain.
\end{note}

\begin{note}[Looking ahead]
  \color{later}
  Difference is one thing, but also difference with respect to cases of interest.
  So, looking ahead, ability.

  Simple variation on second example.
  Ability to demonstrate that instance of phenomenon is covered by theory/not a counterexample.

  Follows from understanding of the theory.
  Seems just as bad.
  And, \citeauthor{Wright:2011wn} doesn't apply to either.
  Just need a little more work.
  Claiming support for general ability, so we add claimed (inductive) support for theory together with understanding of theory.
  Follows to specific ability as instance of general.

  So, while not focusing on cases involving ability from perspective of motivating \nI{}, differences here still relevant.
\end{note}

\subsection{Weisberg}

\begin{note}
  \color{red}
  Difference to \wnf{} is that there's no clear account of why \(\psi\) is needed, the problem, instead, is that it is not possible for the agent to get rid of \(\psi\).
\end{note}

\begin{note}[Intro to \wnf{}]
  Case.
  Condition.
  Contrast.

  Applies to inductive reasoning.
  \nI{} isn't strictly concerned with inductive reasoning.
  However, application is focused on this, and we have appealed to inductive reasoning extensively when contrasting \nI{} to \citeauthor{Wright:2011wn}'s templates.
\end{note}

\begin{note}[Bootstrapping]
  To illustrate \wnf{}, let's consider a case of bootstrapping introduced by~\textcite{Vogel:2000tl}'s --- here following \citeauthor{Weisberg:2010to}'s presentation:
  \begin{quote}
    \begin{illustration}\label{ill:gas-gauge}
      \emph{The Gas Gauge}. The gas gauge in \nagent{9}'s car is reliable, though she has no evidence about its reliability.
      On one occasion the gauge reads F, leading her to believe that the tank is full, which it is.
      She notes that on this occasion the tank reads F and is full.
      She then repeats this procedure many times on other occasions, eventually coming to believe that the gauge reliably indicates when the tank is full.\nolinebreak
      \mbox{}\hfill\mbox{(\Citeyear[526--527]{Weisberg:2010to})}\linebreak
      \mbox{}
    \end{illustration}
  \end{quote}
  \citeauthor{Vogel:2000tl} argued that kind of reasoning present in~\autoref{ill:gas-gauge} is a problem for reliabilist theories of knowledge, and others have argued the problem may be extended further (see \textcite[\S1]{Weisberg:2010to} for more details).

  However, our interest in the reasoning present in~\autoref{ill:gas-gauge} and \wnf{} is merely that the reasoning is intuitively problematic, \wnf{} is an account of why, and \wnf{} may capture the same phenomenon as \nI{}.
\end{note}

\begin{note}[No feedback]
  \begin{quote}\phantlabel{wnf}
    \textbf{No Feedback} If
    \begin{enumerate*}[label=(\roman*)]
    \item\label{W:NF:i} \(L_{1}-L_{n}\) are inferred from \(P_{1}-P_{m}\), and
    \item\label{W:NF:ii} \(C\) is inferred from \(L_{1}-L_{n}\) (and possibly some of \(P_{1}-P_{m}\)) by an argument whose justificatory power depends on making \(C\) at least \(x\) probable,\nolinebreak
      \footnote{
        There may be some ambiguity here.
        As we will see when examining an illustration below, the arguments justificatory power should be read in terms of depending on \emph{having made} \(C\) at least x probable rather than \emph{establishing that} \(C\) at least \(x\) probable.
        (It is in this sense that \(C\) is being `amplified'.)
        By contrast, the following clause requires that \(P_{1}-P_{m}\) \emph{are making} \(C\) at least \(x\) probable without the help of \(L_{1}-L_{n}\).
      }
      and
    \item\label{W:NF:iii} \(P_{1}-P_{m}\) do not make \(C\) at least \(x\) probable without the help of \(L_{1}-L_{n}\), then the argument for \(C\) is defeated.\linebreak
      \mbox{}\hfill\mbox{(\Citeyear[533--534]{Weisberg:2010to})}
    \end{enumerate*}
  \end{quote}
  Where `\(P\)' stands for a premise(s), and `\(L\)' for a lemma(s). (Cf.~\Citeyear[533]{Weisberg:2010to})

  Again, we have a condition in which an argument would be undercut.
  \wnf{} suggests only that the argument for \(C\) would be defeated, but leaves open the status of \(C\).
\end{note}

\begin{note}[\wnf{} intuition]
  \citeauthor{Weisberg:2010to} motivates with the following intuition.
  \begin{quote}
    The idea is that the amplification of an already amplified signal distorts the original signal, resulting in feedback, and bootstrapping is just ``epistemic feedback''.
    Bootstrapping is an undesirable result of amplifying the output of ampliative inference without restriction.\linebreak
    \mbox{}\hfill\mbox{(\Citeyear[534]{Weisberg:2010to})}
  \end{quote}

  To summarise.
  {
    \color{red}
    So, the point is that there's some amplification applied to \(C\) in order to get \(L\) from \(P\).
    And, this in turn blocks an argument to \(C\).
    For, already included amplification to \(C\).
    
  }
  {
    \color{red}
    \phantlabel{wnf:expectation}
    Note, doesn't rule out \(L_{1}-L_{n}\).
    Here, similar to expecting that defeaters don't hold.
    Or, following \citeauthor{Weisberg:2010to}, drawing conclusions from evidence.
  }
\end{note}

\begin{note}
  After walking through how \wnf{} applies to~\autoref{ill:gas-gauge} we will motivate a connexion between \wnf{} and \nI{}, before arguing that the two are sufficiently distinct.
\end{note}

\begin{note}
  The overall conclusion \nagent{9} draws in~\autoref{ill:gas-gauge} is that the gauge reliably indicates when the gas tank is full.
  Still, this overall conclusion is drawn from repeated instances of reasoning on particular occasions that concludes that the gauge is reliable on that occasion.
  And, the fault identified by \wnf{} concerns the reasoning on particular occasions.
  Intuitively, if \nagent{9} fails to establish the reliability of the gauge on any particular occasion by the particular instances of reasoning, then the conclusions of those particular instances of reasoning are unavailable for \nagent{9} to draw the general conclusion.

  So, to begin let us summarise the pattern to which each particular instances of reasoning conforms:

  \begin{enumerate}
  \item\label{W:GG:i} The gauge is reliable. \hfill (Background assumption)\nolinebreak
    \footnote{
      Have as background assumption because in the original, \nagent{9} skips over this as an explicit step.
      However, following \citeauthor{Weisberg:2010to} possible to reformulate to some level of probability sufficient to go to 3, such that the overall result of argument is to raise probability. (\Citeyear[528]{Weisberg:2010to})
    }
  \item\label{W:GG:v} It is sufficiently likely that the gauge is functioning correctly on this occasion. \hfill \mbox{(From~\ref{W:GG:i}, `Amplification')}
  \item\label{W:GG:ii} The gauge reads full. \hfill (Observation)
  \item\label{W:GG:iii} So, the tank is full. \hfill (From~\ref{W:GG:v} \&~\ref{W:GG:ii})
  \item\label{W:GG:iv} Hence, the gauge is functioning correctly on this occasion. \hfill (From~\ref{W:GG:ii} \&~\ref{W:GG:iii})
  \end{enumerate}

  The `feedback' in this reasoning pattern involves establishing (an instance of) the reliability of the gauge from an assumption that the gauge is reliable.

  From the perspective of \wnf{} we have:
  \begin{itemize}
  \item[P:] The gauge reads full.
  \item[L:] The tank is full.
  \item[C:] The gauge is functioning correctly on this occasion.
  \end{itemize}

  And, each of the clauses of \wrt{} are satisfied, for:
  \begin{itemize}[labelwidth=\widthof{(iii)}]
  \item[{\hyperref[W:NF:i]{i:}}] That the tank is full is inferred from the gauge reading full (together with the background assumption applied to the particular occasion).
  \item[{\hyperref[W:NF:ii]{ii:}}] That the gauge is functioning correctly on this occasion is inferred from the tank being full (and the gauge readings full) by an argument whose justificatory power depends on it being probable the gauge functioning correctly on this occasion.
  \item[{\hyperref[W:NF:iii]{iii:}}] That the gauge reads full does not make it probable the gauge functioning correctly on this occasion without the help of it being the case that the tank is full.
  \end{itemize}

  In short, the reasoning from~\ref{W:GG:i} to~\ref{W:GG:iv} captures the (intuitive) idea that \nagent{9}'s reasoning is flawed because and agent doesn't get to use reasoning that proceeds from an assumption to infer that the assumption holds.

  {
    In terms of \citeauthor{Weisberg:2010to}'s presentation, the agent makes an ampliative inference from \(P_{1}-P_{m}\) to \(L_{1}-L_{n}\), requires certain things to be the case, and, results of amplification inference don't provide one with an argument for source of distortion.
    }

  \nagent{9} requires the gauge functioning correctly on this occasion to infer that the gas tank is full, but observing that the gauge functioning correctly follows given the assumption that the gauge functioning correctly doesn't make it any more likely that the gauge really is functioning correctly.
\end{note}

\begin{note}[Different from \citeauthor{Wright:2011wn}]
    {
    Here, very similar to \citeauthor{Wright:2011wn}'s \wrt{}.
    Difference is with respect to \ref{WT:iii}.
    Not-\(C\) does not necessarily entail something incompatible.

    For, \nagent{9} needs sufficiently reliable.
    And, it doesn't follow from the gauge is not functioning on this occasion that it is not sufficiently likely, nor that it is not possible to move from sufficiently likely to working.

    Still, \wnf{} does seem to fall within the general scope of \widt{}.
  }
\end{note}

\begin{note}[In relation to \nI{}]
  We turn now to the relationship between \citeauthor{Weisberg:2010to}'s \wnf{} and \nI{}.

  Recall, \ideaCS{}:
  Antecedent check on reasoning.

  The argument for \nI{} rests on \ideaCS{}, and \citeauthor{Weisberg:2010to}'s \wnf{} may, likewise, be seen to rest on \ideaCS{}.

  For, it seems that any claimed support for the conclusion of an argument that satisfies the clauses of \wnf{} would violate \ideaCS{}.

  Consider \wnf{} once again.
  An argument for a relevant instance of \(C\) is defeated because \(C\) being probable to some degree is required in order to obtain additional lemmas used to construct an argument for \(C\).
  Recast, then, the agent may not construct an argument for \(C\) if the agent requires \(C\) to be probable to some degree.
  Or, equivalently, the agent may not construct an argument for \(C\) if it is a requirement for the success of the argument that the agent is not misled about degree to which \(C\) is probable.
  I.e.\ the argument would indicate the value, or probability, of \(C\) regardless of whether the claimed support is misled because the argument is only successful if \(C\) is probable to the relevant degree.

  So, is it the case that \citeauthor{Weisberg:2010to}'s \wnf{} and \nI{} are equivalent accounts of how \ideaCS{} constrains claiming support, if some (perhaps) superficial details about `probability' or `being in a position to claim support' are either removed or revised?
\end{note}

\begin{note}[Technicality]
  There is an initial difference with respect to scope of application.
  \wnf{} only applies to inductive reasoning (Cf.~\Citeyear[533]{Weisberg:2010to}), while \nI{} makes no such restriction.

  Still, I don't think too much should hang on this difference.
  We have motivated \nI{} primarily with respect to inductive reasoning, and reasoning with \gsi{0} is also, plausibly, an instance of inductive reasoning.
  So, even if there is room for a technicality, it doesn't matter for the cases of interest.
\end{note}

\begin{note}
  \color{red}
  Intuition is that in case of \nI{} the agent needs to make \(\psi\) probable to some degree.
  For, intuitively, needs to be that \(\lnot\psi\) is not an actual defeater.
  I.e.\ the probability of \(\lnot\psi\) needs to be low, and so the probability of \(\psi\) needs to be high.
  If this is right, then it looks as though \nI{} collapses into an instance of \wnf{}.
\end{note}

\begin{note}
  \color{red}
  The problem here is that in order to get \nI{} from \wnf{} we need to assume that any argument relies on making possible defeaters sufficiently improbable and hence that the negation of the defeater is sufficiently probable.

  Ugh.
  This is more complex.
  First, it's not clear that \wnf{} applies to the kind of cases of interest.
  For, need three parts.

  So, would need to be the case that general ability does not make \(\psi\) sufficiently probable.

  Well, it's hard to evaluate this.
  Given the information, it seems it does.
  But, suppose it does not.
  Then, would need it to be the case that general to specific requires \(\psi\).
  This goes back to assumption before.

  However, this then seems problematic.
  For, then run through the novice.
  It seems that here the novice would require that the phenomenon is not a counterexample in order to apply.
  And that seems really odd.

  I mean, there is an issue with the novice as the phenomena does come up as an \requ{}, my solution is to allow the agent to consider this, as it's testimony, or something, but then the parallel for \wnf{} would be to grant that theory alone is sufficient.
  However, if this is the case, then it seems general alone is going to be sufficient.

  And, \nI{} is also going to apply even when \(P\) alone makes \(C\) sufficiently probable.
  Because, it's all about whether something has been considered, not what difference it makes.
  I mean, that's kind of tough, because without considering \(C\) it's hard to know what the probability distribution is.

  \nI{} is more about the impact of unconsidered information.
\end{note}

\begin{note}[Key difference]
  Barring technicalities, still, a fundamental difference is present.

  As we have seen, \wnf{} holds when an agent makes an inductive inference from some premises to some lemmas which require the relevant conclusion of the argument to be probable to a certain degree.

  We \hyperref[wnf:expectation]{noted} that \wnf{} does not require that the agent has claimed support that the conclusion of the argument is probable to the required degree.
  However, clause~\ref{W:NF:ii} of \wnf{} explicitly states the argument of interest depends on relevant conclusion being probable to the required degree.
  So, the reasoning outlined by \wnf{} involves the agent being committed to the relevant conclusion being probable to the required degree as a part of the instance of reasoning to which \wnf{} applies.
  And, hence, tension with \ideaCS{}.

  \nI{} is fundamentally different.

  In the case of \nI{},~\ref{nI:inclusion} ensures that when an agent moves from reasoning from claimed support for \(\phi\) having value \(v\) to reasoning from \(\phi\) having value \(v\) the agent assumes \(\psi\) has value \(v'\).
  In turn, this assumption leads to tension with \ideaCS{} as any reasoning that proceeds from \(\phi\) having value \(v\) depends on \(\psi\) having value \(v'\).
  Hence, even if the agent reasons from the implication that \(\psi\) has value \(v'\) when \(\phi\) has value \(v\), such reasoning is constrained to a context in which \(\psi\) having value \(v'\) must already be assumed.
  And, so, it is not possible for the agent to claim support that \(\psi\) has value \(v'\) that indicates that \(\psi\) has value \(v'\) regardless of whether or not the claimed support is \mom{} because the reasoning from \(\phi\) having value \(v\) to \(\psi\) having value \(v'\) fails if \(\psi\) does not have value \(v'\).

  This is, admittedly, complex.
  To simplify, \wnf{} holds when an agent assumes the relevant conclusion (is probable to a certain degree) as part of the reasoning to the relevant conclusion.
  And, in contrast, \nI{} holds when an agent is required to assume that the relevant conclusion (has some value) in order to reason to the relevant conclusion.

  Simplified further, for~\wnf{} the issue is with respect to the reasoning that would be performed by the agent.
  And, by contrast, for~\nI{} the issue is with respect to what the agent must hold in order to perform the relevant reasoning.

  Admittedly this is a somewhat delicate.
  However, this distinction is important as it highlights how \nI{} is concerned with the specific details of the way in which the agent reasons, rather than any pattern of reasoning itself.

  As we saw when contrasting Illustrations~\ref{ill:CE:main},~\ref{ill:CE:colleague}, and~\ref{ill:CE:testimony}, the reasoning allows an agent to claim support in certain cases (\ref{ill:CE:testimony}), but not in others (\ref{ill:CE:main} and~\ref{ill:CE:colleague}).

  So, there is a significant difference between the cores of \wnf{} and \nI{}.
  Issues with reasoning, and issues with assumptions prior to reasoning, respectively.
\end{note}

\begin{note}[Generalising to \widt{}]
  Noted above that \wnf{} seems to fall in the scope of \widt{}.
  If so, same kind of problem for \widt{}.

  However, relies on additional background about claiming support for presuppositions.
    And, while it is true that \citeauthor{Wright:2011wn} holds such views, it was instructive to observe that difference regardless.

    No such alternative with relation to \citeauthor{Weisberg:2010to}, as it's built into \wnf{}.
\end{note}

\begin{note}[Hm]
  \color{return}
  This is brief, but it's not clear more needs to be said.
  These are distinct phenomenon, even if they turn out to be extensionally equivalent.
  Conjecture that these wouldn't be, as similar issue with respect to difference between \citeauthor{Wright:2011wn}'s \wrt{} and \nI{} arising from \ref{nI:inclusion}.
  Working through illustrations is significant due to difference in formulation.
  There, difference in extension for difference in intension.
  Here, directly observed difference in intension.

  Broader consequences are of some interest (at least to me).
  However, aren't of interest to core line of argument.
  So, pursuing this will be left for some other time.
\end{note}

% \paragraph{Pryor}

% \begin{note}[Objection wrt.\ \citeauthor{Pryor:2000tl}]
%   I've probably already mentioned this somewhere else, but there's a way of reading \citeauthor{Pryor:2000tl} that conflicts.
%   For, Dogmatism could be read as allowing for \RBV{} in cases where it applies.

%   However, this requires some care.

%   Initial point is that it's not clear whether there's a dogmatist position with respect to claiming support.

%   Fruther, there are two possible ways to approach the issue.
%   First, as above.

%   Second, there's some operator which is dogmatic, and consequences stay within the scope of this operator.
%   So, for example with the zebra, the agent sees that the animal is a zebra, and hence sees that it's not a cleverly disguised mule.
%   On this reading, the agent does not \RBV{}.
% \end{note}


\chapter{\EAS{}}
\label{cha:supp:EAS}

\subsubsection{Application}
\label{sec:application}

\begin{note}[Desire]
  Finally, while the examples of reasoning given have concluded with the truth of some proposition --- that a rectangle has some specific area, or that a given fountain pen floats in water, etc.\ --- our interest with \EAS{} is broader.

  In many cases the assigned value truth, falsity, or something in between.
  However, claiming support, and in turn; \USE{}, \ESU{}, and \EAS{} are all neutral with respect to the value assigned to the proposition.
  Therefore, we may consider other values while investigating, and as an application of \ESU{} and \EAS{}.
  In particular, consider reasoning which concludes with the desirability of some proposition.\nolinebreak
  \footnote{
    \color{red}
    Mistaken or misled.
    Yes, I think this holds up.

    Strong view on which an agent may be mistaken about desires in the same way as an agent may be mistaken about evidence.
    View on which desires are independent of representation.
    Hence, misleading or mistaken support when an agent fails to represent desire.
  }

  To illustrate this point, consider temptation.\nolinebreak
  \footnote{
    \color{red}
    Whether or not this is `genuine' temptation isn't of \dots
  }
  Specifically we will consider a slight variation on \citeauthor{Bratman:1999ac}'s `two glasses of wine' (\Citeyear[38]{Bratman:1999ac}) case of temptation.\nolinebreak
  \footnote{
    \color{red}
    See also \textcite{Bratman:2007ab}
  }
\end{note}

\begin{note}[The Pianist]
  Consider a pianist who frequently performs at a club.
  Before each performance the pianist gets nervous and has the option of drinking a glass of wine.
  A glass of wine would also lead to a worse performance.
  However, the glass of wine would help with the pianist's nerves.
  Both are learnt with some experience.
  Hence, if the pianist reasons about what to do:
  \begin{itemize}
  \item When the pianist does not feel the nerves of an upcoming performance they reason to a preference abstaining from drinking a glass of wine.
  \item Yet, when nerves are felt the pianist reasons to a preference for drinking a glass of wine.
  \end{itemize}
  The pattern is stable, and has held over many performances.

  Still, while nerves sometimes get to the pianist, they abstain from drinking a glass of wine most of the time.

  That the our pianist abstains is not necessarily surprising --- it is not uncommon to resist temptation.
  Though it is puzzling.
  The pianist's reasoning is unwavering throughout the span of time in which the pianist has the option of drinking the wine; they reason to preference for drinking a glass of wine.
  So, if the pianist abstains, the pianist acts in opposition to their preference when given the option of drinking a glass of wine, and does so purposefully.
\end{note}

\begin{note}[Reasoning and desire]
  To clarify the puzzle, let us state a basic conjecture regarding preferences and acts.

  \begin{conjecture}\label{conj:resolve-issue-act}
    Any instance of purposeful rational action performed by an agent is the result of the agent resolving the issue of how to act.
    Where:
    \begin{enumerate}
    \item An act is an candidate resolution for how to act only if the agent has claimed support for preference for some proposition and has an expectation that act would bring about the proposition.
    \item An act is an admissible resolution only if there no other candidate resolutions for which the agent has a stronger (combined) preference with respect to the proposition(s) that the agent expects to be brought about by performing the act.
    \end{enumerate}
  \end{conjecture}

  \autoref{conj:resolve-issue-act} understands rational action as the result of an agent resolving the issue of how to act --- choosing which act from a collection of options to perform.
  By understanding rational action as the result of an agent resolving the issue of how to act we may break down the reasoning involved in purposeful rational action into two steps.

  First, what makes an act a candidate resolution, and second what makes an act an admissible resolution.

  An act is a possible resolution just in case the agent links the result of acting to some proposition the agent has a preference for.

  And, an  act is an admissible resolution just in case the agent has no stronger preference for some other proposition that the agent expect could be brought about by some other candidate action.
  (Or, more generally, when an agent is uncertain about which proposition may be brought about by some act, a combined preference regarding each potential proposition.)

  In short, \autoref{conj:resolve-issue-act} is more-or-less the core of an standard decision theoretic account of maximising expected utility without commitment to particular details.\nolinebreak
  \footnote{
    \color{red}
    Cf.\ \textcite{Steele:2020tr}.
    \citeauthor{Davidson:1963aa} `Primary reason' (\Citeyear{Davidson:1963aa})
  }
\end{note}

\begin{note}[Use of conjecture]
  \autoref{conj:resolve-issue-act} fixes an understanding of purposeful rational action, and in turn establishes two ways in which the pianist may resist drinking a glass of wine:
  \begin{enumerate}
  \item Drinking the glass of wine is not a candidate resolution.
    \autoref{conj:resolve-issue-act} states a necessary condition for a candidate resolution, but further conditions may rule out possible resolutions which satisfy the necessary condition stated.
  \item Drinking the glass of wine is not an admissible resolution.
    In particular, because the pianist has a claims support for a stronger preference toward the result of abstaining.
  \end{enumerate}

  We will provide a brief argument that \ESU{} requires the former to be the case and provide an example of how further conditions may rule out possible resolutions.
  In short, \ESU{} requires an agent to witness reasoning to a conclusion in order to claim support for such a conclusion, and as the pianist reasons to a preference for having drunk a glass of wine when performing, abstaining is not an admissible resolution.
  Then, we will turn to \EAS{}, and suggest that it allows the latter to be the case while granting that the only reasoning that the pianist witnesses establishes a preference for having drunk a glass of wine when performing.
  In short, so long as the pianist may claim support for the ability to reason to a stronger preference for the result of abstaining, then by \EAS{} the agent may claim support for a stronger preference for the result of abstaining.
\end{note}

\begin{note}
  Suppose \ESU{} is true.
  By \autoref{conj:resolve-issue-act} an agent resolves an issue of how to act by determining candidate resolutions.
  In turn, a candidate resolution results from the agent claiming support for a preference toward some proposition.
  And, \ESU{} requires that an agent must witness some instance of reasoning in order to claim support for the conclusion of the instance of reasoning.

  Turning to the pianist, we have assumed that before taking to the stage the pianist reasons to preference that favours drinking a glass of wine.
  So, given \autoref{conj:resolve-issue-act} abstaining is not an admissible resolution because the agent has a stronger preference from drinking a glass of wine before taking the stage.
  And, by \ESU{} it is not possible for the pianist to claim support for a preference that would lead to abstention because the pianist must witness the relevant reasoning in order to claim support.
  Therefore, the pianist must rule out drinking a glass of wine as a candidate resolution for how to act.
\end{note}

\begin{note}[Intention and \ESU{}]
  \color{red}

  \citeauthor{Bratman:2007ab} argues that such cases may be understood through an theory on which intentions constrain reasoning.
  If the pianist intends no to drink the wine, and this intention persists, then drinking the wine is no an available conclusion of reasoning.
  Key here is that intention does not interact with the pianist's preferences.
  It remains the case that the pianist would prefer.

  The role of intention in the role of \citeauthor{Bratman:2007ab} account is to constrain possible resolutions for how to act.
  An intention to not drink prevents drinking a glass of wine from being a candidate resolution for how to act (see in particular \textcite[\S3.3]{Bratman:1987aa}).

  However, because ruled out, the pianist does not have the option of acting on that preference.
  Rather, act in a way that is compatible with intention.
  For the pianist, we may assume abstention is the only act compatible with the intention.\nolinebreak
  \footnote{
    \color{red}
    Variation.
    Block contribution of nerves.
    So, intention to not allow nerves to contribute to reasoning.
    Compatible with drinking, but given the way the scenario has been constructed, will result in not drinking.
  }
  So, \citeauthor{Bratman:2007ab} is an example of how to resolve weakness of will given relation between reasoning and action expressed by \autoref{conj:resolve-issue-act} and \ESU{}.
\end{note}

\begin{note}[Broader]
  I think, in broad strokes, phenomena fit this kind of theory.
  Abstracting from the details of any particular theory, it seems plausible that candidate resolutions to the issue of how to act are subject to conditions that extend beyond whether the agent has claimed support for preference for some proposition and has an expectation that act would bring about the proposition --- \autoref{conj:resolve-issue-act} only stipulates that the given constraint is a necessary condition for candidate resolutions.

  However, not clear to me that all phenomena fit such a theory.
  The pianist's reasoning seems distorted.
  The nerves felt before taking to the stage plausibly interfere with the pianists reasoning about candidate resolutions to the issue of how to act, and so the pianist's reasoning plausibly does not resolve the issue of how to act in line with the premises and steps of reasoning that are available to the agent.
  So much, I suspect, is intuitive.
  However, any interpretation of the pianist compatible with \ESU{} is committed to the agent resolving the issue of how to act given unreliable reasoning.
  A \citeauthor{Bratman:1987aa}-like intention would rule out drinking a glass of wine as a resolution to the issue of how to act, but whatever reasoning the agent performs given the intention is still influenced by the nerves felt.

  In the following paragraphs we will suggest that the pianist may resist the conclusion of reasoning performed given nerves because it is distorted.
  We start with two additional conjectures.
\end{note}

\begin{note}[Preferences change with reasoning]
  \begin{conjecture}\label{conj:pref-vs-reasoning}
    Whether, or to what degree, an agent claims support for a preference toward a proposition may differ from whether, or to what degree, the agent would claim support for a preference toward the proposition given varying information.
  \end{conjecture}

  Loosely paraphrased,~\autoref{conj:pref-vs-reasoning} states that preferences an agent reasons to are subject to change given change in the information that the agent reasons from.
  Hence, it seems~\autoref{conj:pref-vs-reasoning} may be considered a truism rather than a conjecture.
  Indeed, varying information is a common component in the construction of cyclical preferences (cf.\ \cite{Sobel:1997wt},\cite{Schumm:1987wx},\cite{Davidson:1955wo}, etc.)

  To illustrate, we consider a case in which difference arises from information that does not contribute to the agent's preferential evaluation of the proposition.

  Suppose an agent has a established preference for meeting person who is Black Panther over meeting the person who is Storm by reasoning.
  However, the agent is not aware that Black Panther is T'Challa nor that Storm is Ororo Monroe.
  Indeed, the agent does not have any information about the referent of `T'Challa' or `Ororo Monroe' and so does establish a preference for meeting T'Challa or Ororo Monroe by reasoning (nor vice-versa).

  Still, it seems that if the agent were provided with the information that Black Panther is T'Challa and that Storm is Ororo Monroe then the agent would reason to a preference for meeting T'Challa or Ororo Monroe.
  And, such information would not contribute to the agent's preferential evaluation of the relevant propositions.
  The agent's preferences for the relevant propositions are determined by considerations that are independent of the terms used to refer to the relevant individuals --- e.g.\ the person who helped defeat Thanos and the person who helped defeat Magneto.

  With relation to \autoref{conj:resolve-issue-act}, whether or not an act is a possible resolution for issue of how to act may depend on what information agent reasons from.
  We now introduce a further conjecture:

  \begin{conjecture}\label{conj:more-info-is-good}
    Generally speaking: When resolving how to act, claimed support for some preference toward a proposition given more information is given greater weight than claimed support for preference toward the (same) proposition given less information.\nolinebreak
  \footnote{
    Note, `more' and `less' information are relative, and it may not be possible to compare distinct bodies of information.
    If so, \autoref{conj:more-info-is-good} does not state anything about the importance of either body of information.
    For example, an agent may have information about the subjective taste of a meal and about the nutritional value of the meal.
    Still, without a way to compare information about subjective taste to information nutritional value in an information there would be no sense in which the former could be considered to hold `more' information that the latter (or vice-versa).
    Though this is not to rule out such a comparison --- we do not place constraints on what comparisons an agent may make.
  }
\end{conjecture}

  \autoref{conj:more-info-is-good} speaks more generally than~\autoref{conj:pref-vs-reasoning} and as a result may be closer to or further from a truism depending on your point of view.
  Still, the core idea is simple:
  Claimed support for some preference toward a proposition given more information is worth more than claimed support for some preference toward a proposition given less information because more information typically increases the reduces the likelihood that the claimed support is either mistaken or misled.\nolinebreak
  \footnote{
    Cf.\ \cite{Good:1966wx} for a related idea --- though see also \cite{Bradley:2016wo}.
  }
  In other words, strength of preference in the sense of \autoref{conj:resolve-issue-act} is proportional to information used to establish preference given a fixed proposition.

  To illustrate, consider an agent resolving whether to banded or off-brand multi-vitamins.
  The agent's method of establishing a preference is to look on each container, and work through the lists of vitamins comparing whether a vitamin is included, and if so to what quantity, weighing some vitamins more heavily than others.
  As the agent works through the lists of vitamins the agent moves from less information to more.
  Still, after each vitamin on the list the agent marks a preference.
  For example, the branded multi-vitamins have 2,500 IU of vitamin A while the off-brand have 2,000 IU, so the agent's initial preference leads to purchasing the branded multi-vitamins.
  However, the branded multi-vitamins have 50mg of vitamin C while the off-brand have 60mg, and given information about vitamins A and C the agent's preference leads to purchasing the off-brand multi-vitamins.
  And so on until the agent has compared the contents of the branded and off-brand multi-vitamins.

  \autoref{conj:more-info-is-good} holds because it seems implausible that the agent could be understood as acting rationally by purchasing either multi-vitamin from a preference determined by a partial comparison between the multi-vitamins given that the agent has established a preference given a full comparison between the multi-vitamins.
\end{note}

\begin{note}[Back to the pianist]
  To summarise the two conjectures:
  \autoref{conj:pref-vs-reasoning} holds that an agent's preferences may vary with the information that the agent uses to claim support for those preference.
  And, \autoref{conj:more-info-is-good} holds, generally speaking, that claimed support for a preference from more information is given greater weight than claimed support for a preference from less information.

  Now, we return to the pianist.
  We make two observations.
  First, given \autoref{conj:pref-vs-reasoning} it may be possible to trace the change in the preference that the pianist reasons to (from abstention to a glass of wine, and vice-versa) to variation in the information that the pianist reasons with.
  In particular, it may be the case that the pianist's nerves prevent them from reasoning with information that they would otherwise reason with, hence the preference to abstain arrived at by reasoning before and after the span of time in which the pianist has the opportunity to drink a glass of wine is a preference arrived at given more information than the preference to drink a glass of wine.
  Second, given \autoref{conj:more-info-is-good} and the present interpretation, the agent may give greater weight to the preference to reasoning that results in a preference for abstention.

  In short, it may be true that pianist's nerves interfere with the reasoning they perform, and without such interference the agent would reason to abstaining from drinking a glass of wine before any given performance.\nolinebreak
  \footnote{
    \color{red}
    Plausible, though not immediate.
    To clear things up a little, consider a hangover.
    Reasoning sucks, I did not desire to miss lunch with a friend, I forgot, because of the hangover.
    However, I did desire to go to bed early, because of the hangover.
    Question is whether the nerves and the glass are like missing lunch or going to bed early.
    If the former, then reasoning is at issue.
    If the latter, then reasoning is at the issue.

    Conjecture, the former.
    No additional information from nerves.
    Still, when the nerves are present the pianist has a hard time reasoning with them.
  }

  The difficulty for the pianist is that such interference is always present;
  It is not straightforward for the pianist to give greater weight to abstaining, as the pianist does not reason to abstaining given their nerves.
  However, if the pianist may claim support for  ability to reason to abstaining, then by \EAS{} the agent may claim support for a preference for abstaining.

  Indeed, it seems the pianist may claim support for having the ability to establish a preference for abstaining when presented with the option of drinking a glass of wine.
  Two observations:
  \begin{itemize}
  \item First, the pianist has performed such reasoning many times, both before and after performances, and the result of such reasoning is stable: abstention.
  \item Second, given the assumptions made about the agent's nerves, the agent retains the relevant premises and steps of reasoning when presented with the choice to drink a glass of wine.
    The nerves felt when presented with the choice to drink a glass of wine only ensure that witnessing this ability is difficult to a degree sufficient for the agent to fail each attempt at witnessing the ability.
  \end{itemize}

  So, when presented with the choice to drink a glass of wine:
  If the  pianist may claim support for having the ability to establish a preference for abstaining, then by \EAS{} the pianist may claim support for a preference for abstaining.
  And, in turn, by \autoref{conj:more-info-is-good} the pianist may give greater weight to their preference for abstaining because the claim of support for a preference to abstain would be arrived at by taking into consideration additional information.

  Given this interpretation of the pianist, `giving into temptation' would be for the pianist to disregard the reasoning that the pianist is able to perform.
  Conversely, `resisting temptation' is for the pianist act in accordance with the preference that they would reason to given the information available to them.
  So, in contrast to accounts of temptation constrained by \ESU{} the pianist need not prevent themselves from reasoning to drinking a glass of wine.
  Rather, the pianist need only reflect on what they are able to reason to.
\end{note}

\begin{note}[Quick objection]
  There is a quick objection to consider before moving on:
  Does the agent have the ability to reason to stronger preference for the result of abstaining?
  For, it seems natural for the pianist to express that they do not have the ability to reason to a preference other than for having drunk a glass of wine when performing \emph{because} of how nerves interfere with their reasoning.

  Ability fluctuates.
  At present I claim support that I have the ability to prove that S4 is sound and complete with respect to transitive frames.
  Part of my claim is that I understand the details of Lindenbaum's Lemma.
  And, if I were to forget the details ofLindenbaum's Lemma then I would lack the ability to construct the relevant proof.
  So, I may lose the ability, but even if I do lose the ability due to forgetting the details of Lindenbaum's Lemma, I may regain the ability by revising the relevant details.

  However, there may be a difference between my loss of ability due to forgetting the details of Lindenbaum's Lemma and the pianist's nerves.
  Forgetting the details of Lindenbaum's Lemma ensures that I lack premises (or steps of reasoning) required to witness the proof.
  By contrast, it is not clear that the pianist's nerves entail that the pianist lack premises (or steps of reasoning) required to reason to stronger preference for the result of abstaining.

  Let us distinguish to ways in which an agent may be said to lack an ability to reason to some conclusion.
  First, the agent lacks sufficient resources; premises and steps of reasoning.
  Second, impediments to the agent using sufficient resources.

  From the first, I have the ability to enumerate all of the positive integers in decimal representation as I have sufficient resources to produce a decimal representation of the first positive integer and I have sufficient resources to produce a decimal representation of any successor integer.
  From the second, I am clearly bounded to enumerate only a finite collection of the positive integers given my mortality and so lack such an ability.

  The success of the quick objection relies on an ability to reason to some conclusion entailing that there are impediments to the agent using sufficient resources to reason to the conclusion.
  I suspect this entailment does not hold for the sense of ability at issue.
  Rather, I suggest that what matters is that the conclusion follows from the premises and steps of reasoning.

  Whether or not this a compelling suggestion is up to you.
  Whether or not \EAS{} holds does not depend on impediments to the agent using sufficient resources entail lack of ability.
  Though, interest in the details of ability and whether they relate to cases of temptation such as the pianist do depend on whether or not \EAS{} holds.
  Therefore, our primary focus will be on showing that \EAS{} holds.
\end{note}


%%% Local Variables:
%%% mode: latex
%%% TeX-master: "master"
%%% End:
