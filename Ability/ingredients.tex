\begin{note}

  \autoref{part:prep} introduced basics.
  \begin{TOCEnum}
  \item
    \TOCLine{cha:clar}

    Conclusion, concludes, concluding, reasoning.
  \item
    \TOCLine{cha:ros}
  \item
    \TOCLine{cha:var}

    \qWhyV{}, \qHowV{}, and \issueConstraint{}.
  \end{TOCEnum}

  The primary goal of this document is develop a recipe for counterexamples to \issueConstraint{}.

  Interest is with whether there are cases in which answers to \qWhyV{} and not constrained by answers to \qHowV{} as specified by \issueConstraint{}.
  In detail, our goal is to develop \scen{1} in which:

  \begin{enumerate}
  \item
    An agent has concluded \(\pv{\phi}{v}\) from \(\Phi\).
  \item
    If an \ros{} failed to hold between \(\pv{\psi}{v'}\) and \(\Psi\) (for the agent) then the agent would not have concluded \(\pv{\phi}{v}\) from \(\Phi\).
  \item
    The agent does not have a \wit{} for the \ros{} between \(\pv{\psi}{v'}\) and \(\Psi\).%
    \mbox{ }\hfill%
    (I.e.\ the agent has not concluded \(\pv{\psi}{v'}\) from \(\Psi\).)
  \end{enumerate}

  In such cases, the \ros{} between \(\pv{\psi}{v'}\) answers \qWhyV{}, but as the agent does not have a \wit{} for the \ros{}.
\end{note}

\begin{note}
  Three basic ingredients.

  \begin{itemize}
  \item
    \tR{2}.
  \item
    \fc{3}.
  \item
    \requ{3}.
  \end{itemize}

  Mix these together, and with some care, counterexamples.
\end{note}

%%% Local Variables:
%%% mode: latex
%%% TeX-master: "master"
%%% End:
