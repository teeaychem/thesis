\chapter{\tpro{3} and dispositions}
\label{cha:tc2-dispositions}


\begin{note}
  This chapter briefly draws parallels between the way we understand \tpro{1} with respect to an agent \tC{} and (what we term) the \dBCA{0} of dispositions.
  In addition, we highlight a type of objection to the \dSCA{0} which does not apply to the \dBCA{0}.
\end{note}



\section{\tCV{2} and the \dBCA{0}}
\label{sec:dispositions}


\begin{note}
  Consider the `\dBCA{0}' of dispositions:%

  \begin{sketch}[\dBCA{2} --- \dBCAa{0}]
    \label{sketch:dBCA}
    \vspace{-\baselineskip}
    \begin{itemize}
    \item
      Object \(o\) has disposition \(d\)
    \end{itemize}
    \emph{If and only if}:
    \begin{itemize}
    \item
      There are descriptors \(C\) and \(M\) such that:%
      \footnote{
        Read `condition' and `manifestation'.
      }
        \emph{If} it were the case that \(C\), \emph{then} \(o\) would \(M\).
    \end{itemize}
    \vspace{-\baselineskip}
  \end{sketch}

  \noindent%
  The \dBCA{0} is common.
  For example:
  %
  \begin{quote}
    To say that an object \(a\) is (water-) \emph{soluble} at time \(t\) is to say that if \(a\) were in water at \(t\), \(a\) would dissolve at \(t\).%
    \mbox{ }\hfill\mbox{(\cite[203]{Quine:2013aa})}
  \end{quote}
  %
  \begin{quote}
    Dispositional words like `know', `believe', `aspire', `clever' and `humorous' are determinable dispositional words.
    They signify abilities, tendencies or pronenesses to do, not things of one unique kind, but things of lots of different kinds.%
    \mbox{ }\hfill\mbox{(\cite[118]{Ryle:1949aa})}
  \end{quote}
  %
  \begin{quote}
    [A] statement like

    \(w\) is inflammable

    amounts to [\dots] some such fainthearted counterfactual as

    If all conditions had been propitious and \(w\) had been heated enough, it would have burned.\newline
    \mbox{ }\hfill\mbox{(\cite[39]{Goodman:1983aa})}
  \end{quote}
\end{note}

\begin{note}
  There are clear parallels between the \dBCA{0} (\autoref{sketch:dBCA}) and our definition of a \tpro{} (\autoref{def:tpro}, \autopageref{def:tpro}).

  For, contrast the conditional by which \tpro{1} are defined with the \emph{if \dots then \dots} direction of \autoref{sketch:dBCA}.
  The conditional by which \tpro{1} are defined concerns a collection of \prop{0}-\val{0}-\pool{0} and event pairings.
  However, it is easy to rephrase the consequent of the conditional in terms of a condition \(C_{\mathbb{P}}\) the events of \(\mathbb{P}\) satisfy, and in turn, the relevant manifestation condition \(M_{\mathbb{P}}\) is the agent concluding \(\pv{\psi}{v'}\) from \(\Psi\).
\end{note}


\begin{note}
  Indeed, a straightforward consequence of the \dBCA{0} mirrors \autoref{prop:tpro-switch} (\autopageref{prop:tpro-switch}):

  \begin{proposition}[Basic proposition]%
    \label{obs:disp:basic}%
    The \dBCA{0} entails:

    \begin{itenum}
    \item[\emph{If}:]
      Object \(o\) has disposition \(d\)
    \item[\emph{Then}:]
      There are descriptors \(C'\) and \(M'\) such that:
      \begin{itemize}
      \item
        For every \scen{0}:
        \emph{If} \(C'\) is the case, \emph{then} \(o\) manifests \(M'\).
      \end{itemize}
    \end{itenum}
    \vspace{-\baselineskip}
  \end{proposition}

  \begin{argument}{obs:disp:basic}
    Suppose the \dBCA{0} holds and consider some object \(o\) with disposition \(d\).
    Given the \dBCAa{0}, there are descriptors \(C\) and \(M\) such that:
    \emph{If} it were the case that \(C\), \emph{then} \(o\) would \(M\).

    Consider the \scen{1} where \(C\) are the case, and what \(M\) captures.
    Now, re-express \(C\) as \(C'\) and \(M\) as \(M'\) such that \(C'\) and \(M'\) do not depend on context (e.g.\ world) of evaluation.%
    \footnote{
      We understand the \dBCAa{} as specifying \(C\) and \(M\) relative to a context, and hence \(C'\) and \(M'\) are given with respect to some context.
      However, it does not follow that \(C\) and \(M\) need to take the relevant context as an argument.
      (\autoref{idea:tC} is also understood this way.)
    }%
    \(^{,}\)%
    \footnote{
      Though, in general, \(C'\) need not capture all conditions.
      For, \autoref{obs:disp:basic} is a \emph{only if} statement.
    }
  \end{argument}
\end{note}


\section{The \dSCA{}}
\label{sec:dsca2}


\begin{note}
  With the parallel between the \dBCAa{} and \autoref{def:tpro} in hand, we briefly observe some issues for the \emph{simple} conditional analysis of dispositions do not apply to the \dBCAa{} (and, hence, similar issues do not extend to \autoref{def:tpro}).
  

  The \dSCA{} is as follows:%
  \footnote{
    Compare to, e.g. \citeauthor{Lewis:1997wg}'s account:
    \textquote{%
      Something \(x\) is disposed at time \(t\) to give response \(r\) to stimulus \(s\) iff, if \(x\) were to undergo stimulus \(s\) at time \(t\), \(x\) would give response \(r\).%
    }
    (\citeyear[143]{Lewis:1997wg})
  }

  \begin{sketch}[The \dSCA{} --- \dSCAa{}, cf.\ \cite[\S1.2]{Choi:2021wg}]%
    \label{sketch:dSCA}
    \vspace{-\baselineskip}
    \begin{itemize}
    \item
      An object \(o\) disposed to \(M\) when \(C\)
    \end{itemize}
    \emph{If and only if}:
    \begin{itemize}
    \item
      \(o\) would \(M\) if it were the case that \(C\).
    \end{itemize}
    \vspace{-\baselineskip}
  \end{sketch}

  \noindent%
  Observe, \(C\) and \(M\) are used to characterise the disposition, and therefore the choice of \(C\) and \(M\) in the counterfactual is fixed.
  By contrast, the \dBCAa{} analysed a disposition \(d\) which did not contain \(C\) and \(M\), and hence allowed free choice of \(C\) and \(M\) in the corresponding counterfactual.%
    \footnote{
    Examples of \dBCA{1} are those cited by \citeauthor{Choi:2021wg} as endorsements of \dSCA{1}.
    This is not the case.
    The same is true of \citeauthor{Manley:2008aa} (\citeyear[60]{Manley:2008aa}).
    Though, \citeauthor{Manley:2008aa}'s discussion is almost identical to that of \citeauthor{Fara:2006aa}~(\citeyear[\S2.1]{Fara:2006aa})\dots

    Anyway, \citeauthor{Choi:2021wg} distinguish the \dSCAa{} from `Entailment':
    %
    \begin{quote}
      \emph{F} is a disposition iff there are an associated stimulus condition and manifestation such that, necessarily, \emph{x} has \emph{F} only if \emph{x} would produce the manifestation if it were in the stimulus condition.%
      \mbox{ }\hfill\mbox{(\citeyear[\S2.1]{Choi:2021wg})}
    \end{quote}
    %
    And, `Entailment' is equivalent to the \dBCAa{0}.
    However, \citeauthor{Choi:2021wg} add:
    %
    \begin{quote}
      If disposition ascriptions do not entail corresponding counterfactual conditionals, then Entailment is hopeless.
      Note that the apparent counterexamples to [the \dSCAa{}] may seem to show just that.
      But let's leave this claim aside [\dots].
    \end{quote}
    %
    There is nothing to leave aside.
    The \dSCAa{} and the \dBCAa{} are substantially different.
  }

  Further, well-known counterexamples to the \dSCAa{} require the choice of \(C\) and \(M\) being fixed.
  For example, consider~\citeauthor{Clarke:2010aa} idea of `masks' as \textquote{something that prevents a disposition from manifesting despite the occurrence of that disposition's characteristic stimulus} (\citeyear[153]{Clarke:2010aa}).%
  \footnote{
    Parallel observations apply to (\cite{Johnston:1992aa}) and (\cite{Martin:1994aa}).
  }
  A sweet is digested when ingested, but a sweet wrapped in plastic is not.
  Hence, being wrapped in plastic is a mask of `an object is disposed to be digested when ingested' given the \dSCAa{}.
  For, there is little preventing someone from ingesting a sweet wrapped in plastic, and it is not the case that a sweet wrapped in plastic is digested when ingested.

  Still, masks are not a problem for the \dBCAa{}.
  For, \dBCAa{} does not place constraints on the relevant \(C\) and \(M\) descriptors.
  Indeed, the relevant \(C\) and \(M\) descriptors are not assumed to be part of the disposition being analysed under a \dBCAa{}, in contrast to the \dSCAa{}.

  For sure, masks may be counterexamples to \emph{instances} of the \dBCAa{}, but in contrast to the \dSCAa{}, masks are not counterexamples to the (basic conditional) \emph{analysis}.%
  \footnote{
    As \citeauthor{Bonevac:2011tz} stress:
    \begin{quote}
      Counterexamples must be deployed as counterexamples to specific proposals.
      The example of a glass packed in styrofoam can perhaps show that fragile cannot be analysed as would break if struck, but it shows nothing about a proposed analysis of fragile as would break if struck when unwrapped, and certainly shows nothing about any proposed analysis of a different dispositional term, such as irascible.%
      \mbox{ }\hfill\mbox{(\citeyear[1144]{Bonevac:2011tz})}
    \end{quote}
    %
    \nocite{Manley:2007aa}
    In response,~\citeauthor{Manley:2011aa} (\citeyear{Manley:2011aa}) argue that what is of interest is whether the \dBCAa{0} functions as an analysis of disposition \emph{ascriptions} --- not whether the \dBCAa{0} is true (\citeyear[cf.][\S1.3]{Manley:2011aa}).
  }%
  \(^{,}\)%
  \footnote{
    This observation predates masks.
    Consider, e.g., the following passage from \citeauthor{Goodman:1983aa}:

    \begin{quote}
      [W]e can define ``flexible'' if we find an auxiliary manifest predicate that is suitably related to ``flexes'' through `causal' principles or laws.
      The problem of dispositions is to define the nature of the connection involved here:
      the problem of characterizing a relation such that if the initial manifest predicate ``Q'' stands in this relation to another manifest predicate or conjunction of manifest predicates ``A'', then ``A'' may be equated with the dispositional counterpart---``Q-able'' or ``Q\textsc{d}''---of the predicate ``Q''.\nolinebreak
      \mbox{ }\hfill\mbox{(\citeyear[45]{Goodman:1983aa} --- first published in 1955)}
    \end{quote}
  }
\end{note}


%%% Local Variables:
%%% mode: latex
%%% TeX-master: "master"
%%% TeX-engine: luatex
%%% End:
