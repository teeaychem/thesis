%%% Local Variables:
%%% TeX-master: "master"
%%% End:

\chapter{Access}
\label{cha:access}

\begin{note}[Basic motivation]
  \ref{denied-claim} provides an account of why an agent has support for some propositions and not others.
  Intuition that in two contrasting cases, one in which the agent does the reasoning, and one in which the agent does not, that the one in which the agent does not is not in a position to claim support.

  As this is only (it seems, relevant) difference, then this motivates~\ref{denied-claim} to some degree.
  However, this kind of motivating is lacking.

  Sam is not tall, Taylor is tall.
  Difference is some fixed height.
  So, tallness requires this particular height.
\end{note}

\begin{note}[Necessary condition]
  \ref{denied-claim} may be strengthened.

  For example, a synchronic/simultaneous requirement.
  \citeauthor{Goldman:2011vn} writes:
  \begin{quote}
    How can evidentialism cope with this problem? It could improve its handling of these cases by abandoning the simultaneity requirement, the requirement that justifying evidence must be possessed at the same time as the belief. But this requirement is a core part of internalism, to which mentalist evidentialism adheres.\nolinebreak
    \mbox{}\hfill\mbox{(\citeyear[261]{Goldman:2011vn})}
  \end{quote}
\end{note}

\begin{note}[Requires foundationalism?]
  Might want to deny~\ref{denied-claim} as it seems to weakly conflict with coherentism, for example.

  For,~\ref{denied-claim} suggests a basing relation, and this doesn't seem to work too well with coherentism.
  Either coherentism is false, or coherence doesn't make a difference in cases of reasoning.
  Neither seem plausible.

  Suggest that there is a plausible revision.

  In the literature, there's attempts to make this work.
  Especially with regard to credences.

  \textcite[\S2]{Silva:2020aa} has some pointers.
\end{note}

\section{Generalising uRA to all instances of reasoning}
\label{sec:generalising-ura-all}

\begin{note}[Broader conditional]
  Still, the observations made when establishing~\ref{P:ab-and-dc:W} broader relation between \ref{denied-claim} and ability.
  For, we noted that \ref{denied-claim} is incompatible with \WR{}.
  To recap:
  \begin{enumerate}[label=(B\arabic*), ref=(B\arabic*)]
  \item Assume an agent appeals to entailment from the ability to \emph{V} that \(\phi\) in order to obtain \(\phi\).
  \item The agent makes use of (the reasoning sketched in) either \AR{} or \WR{}.
  \item \WR{} is incompatible with~\ref{denied-claim}.
  \item The agent makes use of (the reasoning sketched in) \AR{}.
  \end{enumerate}
\end{note}



\section{Literature}
\label{sec:literature}

\subsection{Simple instances}
\label{sec:simple-instances}

\begin{quote}
  A person may know some propositions that logically entail some proposition that the person scarcely understands and surely does not know to follow from the things she does know.
  The logical route from what she knows to this proposition may be complex and go beyond her understanding, or even the understanding of any person.
  In our view, the person is not then justified in believing the consequence, even though it is entailed by her evidence.
  It is noteworthy that, to become justified in believing the proposition, she has to learn something new---namely, its logical connection to her evidence.\nolinebreak
  \mbox{}\hfill\mbox{(\citeyear[94]{Conee:wk})}
\end{quote}
Falls a little short.
Talking about a specific kind of proposition.
Interest is in the suggestion for why this fails.
Still, doesn't require the agent to reason.
However, this is a natural variant of~\ref{denied-claim}, {\color{red} and I should make this clear when introducing the claim that this variant is available}.


\subsection{\citeauthor{Littlejohn:2018uq} and forking paths}
\label{sec:littlejohn}

\textcite{Littlejohn:2018uq} denies the path principle:

\begin{quote}
  The Path Principle: The types of support relations that hold between a thinker's evidence and the propositions she grasps wholly determine whether there is propositional justification for believing these propositions.\nolinebreak
  \mbox{}\hfill\mbox{(\citeyear[224]{Littlejohn:2018uq})}
\end{quote}

Failure of the path principle is similar to failure of~\ref{denied-claim}.
In that, there are cases in which agent has something without accessing support.
proposition support, but so long as it is possible for the agent to obtain doxastic support, then this is something similar.
Different, however, in the details.
\ref{denied-claim} concerns cases of reasoning from premises to conclusion.
So, compatible with failure of the path principle in cases which do not involve reasoning.

\citeauthor{Littlejohn:2018uq} observes the following two corollaries of the path principle.

\begin{quote}
  The Dependence Thesis: There is no situation in which it is appropriate for a thinker to believe p where the thinker does not possess evidence that provides the right support for believing p.\nolinebreak
  \mbox{}\hfill\mbox{(\citeyear[227]{Littlejohn:2018uq})}
\end{quote}

\begin{quote}
  The Sufficiency Thesis: There is no situation in which a thinker's evidence provides the right support for believing p if there is a situation where this type of evidential support relation fails to provide sufficient support for believing a suitable counterpart of p.\nolinebreak
  \mbox{}\hfill\mbox{(\citeyear[227]{Littlejohn:2018uq})}
\end{quote}

\citeauthor{Littlejohn:2018uq} argues against these:

\begin{quote}
  The first problem has to do with the Dependence Thesis.
  However we understand evidence and its possession, there are some cases of justification without evidential support.
  The second problem has to do with the Sufficiency Thesis.
  Even when there is evidence that supports a proposition that a thinker justifiably believes, that type of support might hold in other cases and fail to justify a thinker's beliefs.\nolinebreak
  \mbox{}\hfill\mbox{(\citeyear[227]{Littlejohn:2018uq})}
\end{quote}

It is only \citeauthor{Littlejohn:2018uq}'s argument against the sufficiency thesis that relates to~\ref{denied-claim}.
However, this trades on whether what the agent has is always sufficient.
Hence, requires cases in which the agent has support.




\section{Basing and \ref{denied-claim}}

\begin{note}[Basing]
  Following \citeauthor{Silva:2020aa}:
  \begin{quote}
    \textbf{(Basing)} S's belief that p is appropriately connected to S's sufficient epistemic reasons, R, to believe that p iff S's belief that p is based on R.\linebreak
    \mbox{}\hfill\mbox{(\citeyear{Silva:2020aa})}
  \end{quote}
\end{note}


\begin{note}[No doxastic support]
  A little more work to argue that the agent does not have doxastic support.
  Idea is to assume that agent forms attitude, and question whether it satisfies basing requirement.
  In part,~\ref{denied-claim}.
\end{note}

\begin{note}[Examining doxastic via basing]
  Following paragraphs serve two purposes.
  First, variety of accounts of basing are incompatible with doxastic support.
  Second, that the support in doxastic support is no different from support in propositional support --- relation to the agent.

  If already convinced, then may skip ahead.
\end{note}

\begin{note}[Taxonomy of basing]
  Follow taxonomy presented by \textcite{Korcz:2021ue}.
  \begin{itemize}
  \item Causal.
  \item Counterfactual.
  \item Doxastic.
  \item Causal-doxastic.
  \end{itemize}
\end{note}

\begin{note}[Causal]
  Causal is ruled out, as we're interested in support from general ability, roughly, and there's no plausible causal relation.
  The information that specific ability follows does the work.

  \cite{Moser:1989tv}
  \begin{quote}
    \emph{S}'s believing or assenting to \emph{P} is based on his justifying propositional reason \emph{Q} \(=_{\text{df}}\) \emph{S}'s believing or assenting to \emph{P} is causally sustained in a nondeviant manner by his believing or assenting to \emph{Q}, and by his associating \emph{P} and \emph{Q}.\nolinebreak
    \mbox{}\hfill\mbox{(\citeyear[157]{Moser:1989tv})}
  \end{quote}

  \begin{quote}
    \emph{S} occurrently satisfies an association relation between \emph{E} and \emph{P} \(=_{\text{df}}\)
    \begin{enumerate*}[label=(\roman*)]
    \item \emph{S} has a \emph{de re} awareness of \emph{E}'s supporting \emph{P}, and
    \item as a nondeviant result of this awareness, \emph{S} is in a dispositional state whereby if he were to focus his attention only on his evidence for \emph{P} (while all else remained the same), he would focus his attention on \emph{E}.
      \newline
      \mbox{}\hfill\mbox{(\citeyear[141--142]{Moser:1989tv})}
    \end{enumerate*}
  \end{quote}

  \cite{Ye:2019ux}
  \begin{quote}
    \textbf{Causation Caused by Believing (CCB)}

    One's belief that p is based on reason R just in case R causes the belief and the causation is caused by one's believing that R supports p.
    \newline
    \mbox{}\hfill\mbox{(\citeyear[27]{Ye:2019ux})}
  \end{quote}
  In our cases, no causation from premises.
  Further, if agent considers \nI{}, then won't get the second part of supporting relation.
\end{note}

\begin{note}[Counterfactual]
  Counterfactual.
  Based on reason is caused or would have caused in appropriate circumstances.
  So, belief ends up being based on all pseudo-overdeterminants, for example.
  Built in to \citeauthor{Swain:1981wd}'s account is that the agent did some reasoning, and over-determinant is a substitute for reasoning performed.

  Here, may suggest that agent has done some reasoning, as they've gone from ability to conclusion.
  This reduces to an instance of \AR{} in order for agent to obtain support on reasoning \emph{performed}.
  No clear modification, as then agent would end up with a basic relation in cases where no relation obtains.
\end{note}

\begin{note}[Doxastic]
  Doxastic.
  \cite{Tolliver:1982us}

  \begin{quote}
    \begin{enumerate}[label=(B)]
    \item A bases his belief that q on p at time t, iff
      \begin{enumerate}[label=(\arabic*)]
      \item A believes that q at t and A believes that p at t, and
      \item A believes that the truth of p is evidence for the truth of q at t, and
      \item \space[\dots]\footnotemark
        \mbox{}\hfill\mbox{(\citeyear[159]{Tolliver:1982us})}
      \end{enumerate}
    \end{enumerate}
    \footnotetext{
        The final condition is designed to rule out certain problematic cases which expands on \citeauthor{Tolliver:1982us}'s understanding of what it is for the truth of p to be evidence for the truth of q.
        \begin{quote}
        \begin{enumerate}[label=(\arabic*)]
          \setcounter{enumi}{2}
        \item Where A's estimate of the likelihood of q equals h at t \(0 < h \leq 1\), \emph{if it were the case that}:
          \begin{enumerate}[label=(\roman*)]
          \item A's second·order estimate ofthe L-proposition ``the likelihood of q is greater than or equal to h'' is less prior to t than it at t, and
          \item A did not believe p prior to t, and
          \item A came to belive pa at t,
          \end{enumerate}
          then, at t, A's second-order estimate of the L-proposition ``the likelihood of q is greater than or equal to h'' would be greater than it was prior to t.
          \end{enumerate}
        \end{quote}
      }
    \end{quote}
    The second condition qualifies \citeauthor{Tolliver:1982us}'s account as doxastic.
    The agent has a `meta-belief' that that the truth of the reason is evidence for the truth of the content of the agent's belief.
    For example, agent believes that the premises available for existence of (particular) strategy are evidence for existence of strategy.
    This does not require a causal relation between the reason and the belief, or between the relevant premises and the existence of the (particular) strategy.

    Feature of the meta-belief is that it is not part of the established basing relation.

    Two issues with \citeauthor{Tolliver:1982us}.
    A component of the first condition --- believing p at t --- and the second condition.

    On certain accounts of belief, believing p at t, will hold true.
    Whatever the relevant premises are, the agent believes them.
    However, this is not particularly clear.

    The second condition is the primary difficulty.
    Motivated by \citeauthor{Lehrer:1971aa}.
    \cite{Korcz:2000uo} outlines the argument (\citeyear[534]{Korcz:2000uo}).
    Hence, suggests such meta-beliefs are sometimes sufficient to establish basing relation.

    \citeauthor{Tolliver:1982us} talks of belief, \citeauthor{Korcz:2000uo} talks interchangeably of awareness or meta-belief.
    Continue to talk of meta-belief.

    \citeauthor{Tolliver:1982us} provides an analysis of what it is to believe that the truth of p is evidence for the truth of q. (\citeyear[156--157]{Tolliver:1982us})
    Key is that \citeauthor{Tolliver:1982us}'s condition requires the agent to have information about the relation.
    Incompatible with cases of interest, where agent doesn't have information about how (specific) ability leads to conclusion.

    \citeauthor{Korcz:2000uo} proposes that p and q contribute to causing the meta-belief.
    So, this suggests that the only cases are those in which there's some rewriting of an established basing relation.

    If follow \citeauthor{Korcz:2000uo}, then it seems as though we exclude instances of `first time' support.
    If we continue to follow \citeauthor{Tolliver:1982us}, then support for meta-belief.

    If don't require support, then things get difficult.
    On the one hand, problematic examples \citeauthor{Korcz:2000uo} mentions.
    As noted, \citeauthor{Korcz:2000uo} uses causation.
    Could substitute with some kind of support, but this kind of support is absent from scenarios of interest.

    On the other hand, if restricted independently of support, then it seems many basing instances are going to come for free.


  Here we have a kind of meta-belief about relation of support.
  As with counterfactual, the issue is with the meta-belief.
  No clear way to establish this without a variant of \AR{}.
  Here, this doesn't require~\ref{denied-claim}, which may be of some interest.

  Causal-doxastic inherit the problems of more basic.

  Upshot is that following the taxonomy, there's no clear basing relation in scenarios of interest.
  Influence of~\ref{denied-claim}.

  Of course, doesn't establish, as not requirement that doxastic support requires one such account of basing relation.
  However, more than suggestive that there's no doxastic support.

  
\end{note}

\begin{note}[Example of \citeauthor{Neta:2019aa} on basing]
    \begin{quote}
    For an agent A to C for reason R involves A’s \emph{de se}, object-involving representation of a particular explanatory relation between R, on the one hand, and her C’ing, on the other, and that object-involving representation represents that same explanatory relation under the category ex post justifying.\nolinebreak
    \mbox{}\hfill\mbox{(\citeyear[204]{Neta:2019aa})}
  \end{quote}
  Distinct from other accounts.

  \emph{De se} so that there no possible mistake about whether it is the agent that is C'ing.
  The explanatory connexion is represented.
  \citeauthor{Neta:2019aa} is non-committal to what is involved with representation.

  Key is `\emph{that same explanatory relation}'.
\end{note}

\begin{note}[Summarising]
  Seems plausible to grant that the agent has propositional support.
  Similarly plausible that the agent does not have doxastic support.

  Doxastic support captures something stronger than what is available to the agent.
  By doing reasoning, agent would establish doxastic support.

  Still, do not need to claim doxastic support.
  Goal is to allow agent to establish support from premises.
  This may fall (qualitatively) short of being doxastic support.

  Interest with above, as this is intuitively a case of, or closely related to, basing.
\end{note}


\section{Notes}
\label{sec:notes}

\begin{note}[Memory]
  \ref{denied-claim} doesn't require that the agent continues to have access to reasons.

  \textcite[208]{Goldman:1999tr} has a nice statement regarding the \emph{problem of forgotten evidence} for internalism.
  \ref{denied-claim} only concerns the agent obtaining something like `first-time' support.
  If agent reasons to conclusion from premises, and forgets premises, remains true that the agent obtained support by reasoning from those premises.
\end{note}