%%% Local Variables:
%%% TeX-master: "master"
%%% End:

\chapter{Access}
\label{cha:access}

\begin{note}[Basic motivation]
  \ref{denied-claim} provides an account of why an agent has support for some propositions and not others.
  Intuition that in two contrasting cases, one in which the agent does the reasoning, and one in which the agent does not, that the one in which the agent does not is not in a position to claim support.

  As this is only (it seems, relevant) difference, then this motivates~\ref{denied-claim} to some degree.
  However, this kind of motivating is lacking.

  Sam is not tall, Taylor is tall.
  Difference is some fixed height.
  So, tallness requires this particular height.
\end{note}

\begin{note}[Necessary condition]
  \ref{denied-claim} may be strengthened.

  For example, a synchronic/simultaneous requirement.
  \citeauthor{Goldman:2011vn} writes:
  \begin{quote}
    How can evidentialism cope with this problem? It could improve its handling of these cases by abandoning the simultaneity requirement, the requirement that justifying evidence must be possessed at the same time as the belief. But this requirement is a core part of internalism, to which mentalist evidentialism adheres.\nolinebreak
    \mbox{}\hfill\mbox{(\citeyear[261]{Goldman:2011vn})}
  \end{quote}
\end{note}

\begin{note}[Requires foundationalism?]
  Might want to deny~\ref{denied-claim} as it seems to weakly conflict with coherentism, for example.

  For,~\ref{denied-claim} suggests a basing relation, and this doesn't seem to work too well with coherentism.
  Either coherentism is false, or coherence doesn't make a difference in cases of reasoning.
  Neither seem plausible.

  Suggest that there is a plausible revision.

  In the literature, there's attempts to make this work.
  Especially with regard to credences.

  \textcite[\S2]{Silva:2020aa} has some pointers.
\end{note}

\section{Generalising uRA to all instances of reasoning}
\label{sec:generalising-ura-all}

\begin{note}[Broader conditional]
  Still, the observations made when establishing~\ref{P:ab-and-dc:W} broader relation between \ref{denied-claim} and ability.
  For, we noted that \ref{denied-claim} is incompatible with \WR{}.
  To recap:
  \begin{enumerate}[label=(B\arabic*), ref=(B\arabic*)]
  \item Assume an agent appeals to entailment from the ability to \emph{V} that \(\phi\) in order to obtain \(\phi\).
  \item The agent makes use of (the reasoning sketched in) either \AR{} or \WR{}.
  \item \WR{} is incompatible with~\ref{denied-claim}.
  \item The agent makes use of (the reasoning sketched in) \AR{}.
  \end{enumerate}
\end{note}



\section{Literature}
\label{sec:literature}

\subsection{Simple instances}
\label{sec:simple-instances}

\begin{quote}
  A person may know some propositions that logically entail some proposition that the person scarcely understands and surely does not know to follow from the things she does know.
  The logical route from what she knows to this proposition may be complex and go beyond her understanding, or even the understanding of any person.
  In our view, the person is not then justified in believing the consequence, even though it is entailed by her evidence.
  It is noteworthy that, to become justified in believing the proposition, she has to learn something new---namely, its logical connection to her evidence.\nolinebreak
  \mbox{}\hfill\mbox{(\citeyear[94]{Conee:wk})}
\end{quote}
Falls a little short.
Talking about a specific kind of proposition.
Interest is in the suggestion for why this fails.
Still, doesn't require the agent to reason.
However, this is a natural variant of~\ref{denied-claim}, {\color{red} and I should make this clear when introducing the claim that this variant is available}.


\subsection{\citeauthor{Littlejohn:2018uq} and forking paths}
\label{sec:littlejohn}

\textcite{Littlejohn:2018uq} denies the path principle:

\begin{quote}
  The Path Principle: The types of support relations that hold between a thinker's evidence and the propositions she grasps wholly determine whether there is propositional justification for believing these propositions.\nolinebreak
  \mbox{}\hfill\mbox{(\citeyear[224]{Littlejohn:2018uq})}
\end{quote}

Failure of the path principle is similar to failure of~\ref{denied-claim}.
In that, there are cases in which agent has something without accessing support.
proposition support, but so long as it is possible for the agent to obtain doxastic support, then this is something similar.
Different, however, in the details.
\ref{denied-claim} concerns cases of reasoning from premises to conclusion.
So, compatible with failure of the path principle in cases which do not involve reasoning.

\citeauthor{Littlejohn:2018uq} observes the following two corollaries of the path principle.

\begin{quote}
  The Dependence Thesis: There is no situation in which it is appropriate for a thinker to believe p where the thinker does not possess evidence that provides the right support for believing p.\nolinebreak
  \mbox{}\hfill\mbox{(\citeyear[227]{Littlejohn:2018uq})}
\end{quote}

\begin{quote}
  The Sufficiency Thesis: There is no situation in which a thinker's evidence provides the right support for believing p if there is a situation where this type of evidential support relation fails to provide sufficient support for believing a suitable counterpart of p.\nolinebreak
  \mbox{}\hfill\mbox{(\citeyear[227]{Littlejohn:2018uq})}
\end{quote}

\citeauthor{Littlejohn:2018uq} argues against these:

\begin{quote}
  The first problem has to do with the Dependence Thesis.
  However we understand evidence and its possession, there are some cases of justification without evidential support.
  The second problem has to do with the Sufficiency Thesis.
  Even when there is evidence that supports a proposition that a thinker justifiably believes, that type of support might hold in other cases and fail to justify a thinker's beliefs.\nolinebreak
  \mbox{}\hfill\mbox{(\citeyear[227]{Littlejohn:2018uq})}
\end{quote}

It is only \citeauthor{Littlejohn:2018uq}'s argument against the sufficiency thesis that relates to~\ref{denied-claim}.
However, this trades on whether what the agent has is always sufficient.
Hence, requires cases in which the agent has support.

\section{Notes}
\label{sec:notes}

\begin{note}[Memory]
  \ref{denied-claim} doesn't require that the agent continues to have access to reasons.

  \textcite[208]{Goldman:1999tr} has a nice statement regarding the \emph{problem of forgotten evidence} for internalism.
  \ref{denied-claim} only concerns the agent obtaining something like `first-time' support.
  If agent reasons to conclusion from premises, and forgets premises, remains true that the agent obtained support by reasoning from those premises.
\end{note}