\makeatletter
\renewcommand{\PackageInfo}[2]{}% Remove package information
\renewcommand{\@font@info}[1]{}% Remove font information
% \renewcommand{\@latex@info}[1]{}% Remove LaTeX information
\makeatother

\PassOptionsToPackage{unicode}{hyperref}

\documentclass[10pt]{report}
% \usepackage[margin=1in]{geometry}
% \newcommand\hmmax{0}
% \newcommand\bmmax{0}
% % % Fonts% %
% \usepackage{luatexja}

\usepackage[T1]{fontenc}
\usepackage[complete, subscriptcorrection, slantedGreek, mtpfrak, mtpbb, mtpcal]{mtpro2}
\usepackage{bm}% Access to bold math symbols
\usepackage[no-math]{fontspec}
% \usepackage{fontspec}
\defaultfontfeatures{Ligatures=TeX}%,Numbers={Proportional}}
   % \protrudechars=2 % or \pdfprotrudechars=2 and
\adjustspacing=2 %    \pdfadjustspacing=2 with luatex < v0.85
\newfontfeature{Microtype}{protrusion=default;expansion=default;}
\directlua{fonts.protrusions.setups.default.factor=.5}
\setmainfont[Microtype,Ligatures=TeX,BoldFont={*-Semibold}]{Source Serif 4}
\setsansfont[Microtype,Scale=MatchLowercase,Ligatures=TeX,BoldFont={*-Semibold}]{Source Sans 3}
\setmonofont[Scale=0.8]{Source Code Pro}

% \usepackage{selnolig}% For suppressing certain typographic ligatures automaically
% % % % % % %

\usepackage{amsthm}         % (in part) For the defined environments
\usepackage{mathtools}      % Improves  on amsmaths/mtpro2
\usepackage{xfrac}

% \usepackage{silence}
% \WarningsOff[marginnote]% Suppress warnings related to package

\counterwithout{footnote}{chapter} % For continuous footnote numbering
\counterwithout{figure}{chapter}

% % % The bibliography % % %
\usepackage[%
  backend=biber,
  style=authoryear-comp,
  bibstyle=authoryear,
  citestyle=authoryear-comp,
  uniquename=false,
  backref=false,
  hyperref=true,
  url=false,
  isbn=false,
  doi=false,
  useprefix=true,
  ]{biblatex}
\DeclareFieldFormat{postnote}{#1}
\DeclareFieldFormat{multipostnote}{#1}
% \setlength\bibitemsep{1.5\itemsep}
\newcommand{\noopsort}[1]{}
\addbibresource{Ability.bib}
% % % % % % % % % % % % % % %

\usepackage[inline]{enumitem}
% \setlist[enumerate]{noitemsep}
% \setlist[itemize]{noitemsep}
\setlist[description]{style=unboxed,leftmargin=\parindent,labelindent=\parindent,font=\normalfont\space}

% % % Misc packages % % %
% \usepackage{setspace}
% \usepackage{refcheck} % Can be used for checking references
% \usepackage{lineno}   % For line numbers
% \usepackage{hyphenat} % For \hyp{} hyphenation command, and general hyphenation stuff
\usepackage{nonumonpart} % Disable page numbers on 'Part' title pages
% % % % % % % % % % % % %

% % % Red Math % % %
\usepackage[usenames, dvipsnames]{xcolor}
% \usepackage{everysel}
% \EverySelectfont{\color{black}}
% \everymath{\color{red}}
% \everydisplay{\color{black}}
\definecolor{fuchsia}{HTML}{FE4164}%Neon Fuchsia %{F535AA}%Neon Pink
\definecolor{details}{HTML}{FE4164}%Neon Fuchsia %{F535AA}%Neon Pink
\definecolor{expand}{HTML}{C61AFF}
\definecolor{later}{HTML}{A65978}
\definecolor{return}{HTML}{660066}
\definecolor{reword}{HTML}{F57F17}
\definecolor{link}{HTML}{66FFFF}
% % % % % % % % % %

\usepackage[export]{adjustbox}
\usepackage{subcaption}
\usepackage{float}

% \usepackage{pifont}
% \newcommand{\hand}{\ding{43}}

\usepackage{tabularray} % For fancy table features

\usepackage{multirow}
% \usepackage{adjustbox}

\usepackage{multicol}

\setcounter{secnumdepth}{4}
\setcounter{tocdepth}{4}

% % % % % % % % % % % % TIKZ
\usepackage{tikz}
\usetikzlibrary{bending,arrows,calc,arrows.meta,patterns,fadings}
\usetikzlibrary{trees}
\usetikzlibrary{backgrounds, positioning, fit, backgrounds}
\usepackage{tikz-qtree} %for simple tree syntax
% \usetikzlibrary{positioning,shapes.multipart} %for structured nodes
\usetikzlibrary{tikzmark}
% % % % % % % % % % % %

\usepackage{graphicx} % for images (png/jpeg etc.)
\usepackage{caption} % for \caption* command

% \usepackage{svg}
% \usepackage[off]{svg-extract}
% \svgsetup{clean=true}

\usepackage{extarrows}

% % % % % % % % % % % % MY COMMANDS
\newcommand{\hozlinedash}[0]{%
  \noindent\hdashrule[0.5ex][c]{\textwidth}{.1pt}{2.5pt}
}

\def\nleadsto{\mathrel{%
    \mathchoice{\NLEADSTO}{\NLEADSTO}{\scriptsize\NLEADSTO}{\tiny\NLEADSTO}%
}}
\def\NLEADSTO{{%
    \setbox0\hbox{\leadsto}%
    \rlap{\hbox to \wd0{\hss\hspace{-10pt}\not\hss}}\box0
  }}

\usepackage{dashrule}
\newcommand{\hozline}[0]{%
  \noindent\hdashrule[0.5ex][c]{\textwidth}{.1pt}{}
}
% % % % % % % % % % % %
\usepackage{xskak} % For chess diagram
\usepackage{lplfitch} % For fitch proofs

% % % % % % % % % % % %

% % % My packages % % %
\usepackage{CustomTheoremsThesis}
% \usepackage{FuturePromisedEvents}
\usepackage{ThesisCustom}
\usepackage{Notes}
% % % % % % % % % % % %

\makeatletter
\renewcommand\paragraph{\@startsection{paragraph}{4}{\z@}%
  {-3.25ex\@plus -1ex \@minus -.2ex}%
  {1.5ex \@plus .2ex}%
  {\normalfont\normalsize\bfseries}}
\makeatother

\makeatletter
\renewcommand\subparagraph{\@startsection{subparagraph}{4}{\z@}%
  {-3.25ex\@plus -1ex \@minus -.2ex}%
  {1.5ex \@plus .2ex}%
  {\normalfont\normalsize\bfseries}}
\makeatother

% https://tex.stackexchange.com/questions/94402/creating-a-subsubparagraph
\makeatletter
\newcounter{subsubparagraph}[subparagraph]
\renewcommand\thesubsubparagraph{%
  \thesubparagraph.\@arabic\c@subsubparagraph}
\newcommand\subsubparagraph{%
  \@startsection{subsubparagraph}    % counter
    {6}                              % level
    {\z@}                     % indent
    {3.25ex \@plus 1ex \@minus .2ex} % beforeskip
    {1.5ex \@plus .2ex}%             % afterskip
    {\normalfont\normalsize\bfseries}}
\newcommand\l@subsubparagraph{\@dottedtocline{6}{10em}{5em}}
\newcommand{\subsubparagraphmark}[1]{}
\makeatother


\usepackage[
breaklinks,
bookmarks=false,
hidelinks,
colorlinks=true,
linkcolor=fuchsia,
citecolor=fuchsia,
]{hyperref}

% https://tex.stackexchange.com/questions/83872/getting-hyperref-to-work-with-citeyear-and-citeyearpar-in-biblatex

\DeclareCiteCommand{\citeyear}
{\usebibmacro{prenote}}
{\bibhyperref{\printfield{year}}\bibhyperref{\printfield{extrayear}}}
{\multicitedelim}
{\usebibmacro{postnote}}

\DeclareCiteCommand{\citeyearpar}[\mkbibparens]
{\usebibmacro{prenote}}
{\bibhyperref{\printfield{year}}\bibhyperref{\printfield{extrayear}}}
{\multicitedelim}
{\usebibmacro{postnote}}

% https://tex.stackexchange.com/questions/75902/hyperlinking-author-names-in-biblatex-when-using-citeauthor

\DeclareCiteCommand{\citeauthor}
{\boolfalse{citetracker}%
  \boolfalse{pagetracker}%
  \usebibmacro{prenote}%
}%
{\ifciteindex%
{\indexnames{labelname}}%
{}%
\printtext[bibhyperref]{\printnames{labelname}}}%
{\multicitedelim}%
{\usebibmacro{postnote}}

\usepackage[british]{babel}
\addto\extrasbritish{
  \def\chapterautorefname{Chapter}
  \def\sectionautorefname{Section}
  \def\subsectionautorefname{Section}
  \def\subsubsectionautorefname{Section}
}
% Following commands mean autoref writes `section' rather that `subsection', etc.
% https://tex.stackexchange.com/questions/177007/autoref-showing-subsection-and-subsubsection
\let\subsectionautorefname\sectionautorefname
\let\subsubsectionautorefname\sectionautorefname

\title{
  Witnessing foregone-conclusions
}
\author{Ben Sparkes}
% \date{ }

\begin{document}

\pagenumbering{roman}

\maketitle

\begin{quote}
  \textsc{Othello} O monstrous, monstrous!

  \textsc{Iago}\phantom{O monstrous, monstrous! Nay,} Nay, this was but his dream.

  \textsc{Othello} But this denoted a foregone conclusion.

  \textsc{Iago} 'Tis a shrewd doubt, though it be but a dream;\newline
  \phantom{\textsc{Iago} 'Tis}And this may help to thicken other proofs\newline
  \phantom{\textsc{Iago} 'Tis}That do demonstrate thinly.\newline
  \mbox{ }\hfill\mbox{(\citetitle{Shakespeare:2003vc}, 3.3.428--432)}
\end{quote}

\tableofcontents

\newpage

\pagenumbering{arabic}

% \part*{Introduction}
% \label{part:introduction}

% \chapter{\qWhy{} and \qHow{}}
\label{cha:intro}

\begin{note}
  Our interest is understanding the way an event in which some agent concludes some proposition has some value happened.

  \begin{scenario}[Multiplication]%
    \label{illu:gist:calc}%
    An agent enters `\(23 \cdot 15\)' into a calculator and presses a button marked `\(=\)'.
    The calculator displays `\(345\)'.

    The agent pauses for a moment.
    They have a good understanding of arithmetic.
    And, given the display on the calculator, it follows that if the calculator is trustworthy, then they would not fail to conclude \(23 \cdot 15 = 345\) via their understanding of arithmetic.

    The agent concludes \(23 \cdot 15 = 345\).
  \end{scenario}

  Intuitively, the agent concluded \(23 \cdot 15 = 345\) from the calculator.
  % Personifying a little, we may say the agent concluded \(23 \cdot 15 = 345\) from the testimony of the calculator.
  More could be said about the way the agent concluded \(23 \cdot 15 = 345\) from the calculator, but these details won't be of too much interest.%
  \footnote{
    For example, classify the agent as regarding the calculator as a source of testimony, adding that testimony are factive, and so concluded \(23 \cdot 15 = 345\) from the testimony of the calculator.
  }

  The agent's understanding of arithmetic, it seems, had no significant role.
  So long as the calculator was correct, the agent had the opportunity to conclude \(23 \cdot 15 = 345\).
  Still, the agent did not perform any arithmetic.
\end{note}

\begin{note}
  Abstracting a little, \(23 \cdot 15 = 345\) corresponds to a state of affairs.
  If the truths of mathematics are necessary, then this state of affairs never fails to be, but \(23 \cdot 15 = 345\) captures the way things are in the same way as `kangaroos have tails' captures the way things are.
  For ease we refer to states of affairs as propositions.

  Continuing the abstraction, when the agent concludes \(23 \cdot 15 = 345\), we may say the agent concludes \(23 \cdot 15 = 345\) is true.
  Even if \(23 \cdot 15 = 345\) never fails to be, the calculator may have been faulty and the agent may have concluded \(23 \cdot 15 = 543\).

  So, say the agent concluded the proposition `\(23 \cdot 15 = 543\)' has value `\(\text{True}\)'.

  The agent also (intuitively) concluded `\(23 \cdot 15 = 543\)' has value `\(\text{True}\)' from the calculator.
  Term this a premise.%
  \footnote{
    Strictly, of interest is that the relevant conclusions follow from the agent evaluating the relevant state of affairs as \(\text{True}\).
    That is, with respect to~\autoref{illu:gist:calc}, of interest is that the agent concludes `\(23 \cdot 15 = 345\)' has value `\(\text{True}\)' \emph{from} the agent evaluating `the calculator testified \(23 \cdot 15 = 345\)' as `\(\text{True}\)'.
    And, with respect to~\autoref{illu:gist:calc} it is likewise the birds `perceiving' an explanation from Snowball exists.
    However, we adopt the shorthand for ease of expression.
  }
  Though, perhaps there are additional premises that may be mentioned.
  So, say the agent concluded `\(23 \cdot 15 = 543\)' has value `\(\text{True}\)' from a \pool{} of premises which includes the calculator.
\end{note}

\begin{note}
  Abstracting from almost all the details of \autoref{illu:gist:calc}, then we may say that an agent concluded some proposition has some value from some \pool{} of premises.
  % We assume an event in which an agent concludes always amounts to a conclusion that some proposition has some value from some \pool{} of premises.

  For ease, we abbreviate `\pool{} of premises' to `\pool{}'.
\end{note}

\begin{note}
  Our interest is understanding the way an event in which some agent concludes some proposition \(\phi\) has some value \(v\) happened.

  In this respect, our interest is with the intuition that the agent concluded `\(23 \cdot 15 = 543\)' has value `\(\text{True}\)' \emph{from} the calculator.
  This `\emph{from}' does work.
  Our understanding of \autoref{illu:gist:calc} would be different if the agent concluded `\(23 \cdot 15 = 543\)' has value `\(\text{True}\)' from the their understanding of arithmetic, or if the agent concluded `\(23 \cdot 15 = 543\)' has value `\(\text{True}\)' from their conviction that the calculator guessed correctly.

  What this relation amounts to is something we will not discuss.
  The core intuition is that the relation of interest captures something beyond the observation that there was a sequence of events in which the agent used the calculator and then concluded `\(23 \cdot 15 = 543\)' has value `\(\text{True}\)'.
  The conclusion was `from', `due to', `because of' \dots etc., the relevant \pool{}.

  % It seems clear the same type of relation does not hold between `\(23 \cdot 15 = 543\)', `\(\text{True}\)', and the agent's understanding of arithmetic.
\end{note}

\begin{note}
  \begin{scenario}[Animalism]
    \label{scen:animalism}
    `Four legs good, two legs bad.'
    This, he said, contained the essential principle of Animalism.
    Whoever had thoroughly grasped it would be safe from human influences.
    The birds at first objected, since it seemed to them that they also had two legs, but Snowball proved to them that this was not so.

    `A bird's wing, comrades,' he said, `\textsc{is} an organ of propulsion and not of manipulation.
    It should therefore be regarded as a leg.
    The distinguishing mark of Man is the \emph{hand}, the instrument with which he does all his mischief.'

    The birds did not understand Snowball's long words, but they accepted his explanation, and all the humbler animals set to work to learn the new maxim by heart.
    \textsc{four legs good, two legs bad}, was inscribed on the end wall of the barn\dots%
    \mbox{ }\hfill\mbox{(\cite[25]{Orwell:1976aa})}%
    \newline
  \end{scenario}

  The agents of interest are the birds, and the conclusion is the essential principle of Animalism:
  `Four legs good, two legs bad'.

  Snowball provided an argument against an objection from the birds, and the birds concluded `Four legs good, two legs bad' from Snowball's explanation.

  Though, as \citeauthor{Orwell:1976aa} highlights, the birds did not conclude `Four legs good, two legs bad' from \emph{the content of} Snowball's explanation.
  Some words were too long.
  Instead, the birds concluded `Four legs good, two legs bad', at least in part, from \emph{Snowball's explanation}.

  In parallel to the agent's understanding of arithmetic from~\autoref{illu:gist:calc} the birds of~\autoref{scen:animalism} did not, in part, conclude `Four legs good, two legs bad' from the content of Snowball's explanation.
\end{note}

\begin{note}
  In line with the abstraction given, we may say:
  The birds conclude the proposition `Four legs good, two legs bad' has value `\(\text{True}\)' from the existence of Snowball's explanation.
  And, a relation holds between `Four legs good, two legs bad', `\(\text{True}\)' and the existence of Snowball's explanation, for each bird.
\end{note}

\section*{\qWhy{}, \qHow{} and \issueInclusion{}}
\label{cha:intro:why-how}

\begin{note}
  Our interest is understanding the way an event in which an agent concludes some proposition has some value happened.

  \phantlabel{how-and-why-first-mention}
  We distinguish two questions; `\qWhy{}' and `\qHow{}':

  \begin{question}{questionWhy}{\qWhy{}}
    \cenLine{
      \begin{VAREnum}
      \item
        Agent: \vAgent{}
      \item
        Proposition: \(\phi\)
      \item
        Value: \(v\)
      \item
        Event: \(e\)
      \item
        \mbox{ }
      \end{VAREnum}
    }
    \medskip

    Given \(e\) is an event in which \vAgent{} concludes \(\phi\) has value \(v\):
    \begin{itemize}
    \item
      Which relations between some proposition, value, and \pool{} explain \emph{why} \(e\) is such that \vAgent{} concluded \(\phi\) has value \(v\)?
    \end{itemize}
    \vspace{-\baselineskip}
  \end{question}

  \begin{question}{questionHow}{\qHow{}}
    \label{q:how}
    \cenLine{
      \begin{VAREnum}
      \item
        Agent: \vAgent{}
      \item
        Proposition: \(\phi\)
      \item
        Value: \(v\)
      \item
        Event: \(e\)
      \item
        \mbox{ }
      \end{VAREnum}
    }
    \medskip

    Given \(e\) is an event in which \vAgent{} concludes \(\phi\) has value \(v\):
    \begin{itemize}
    \item
      Which events explain \emph{how} \(e\) is such that \vAgent{} concluded \(\phi\) has value \(v\)?
    \end{itemize}
    \vspace{-\baselineskip}
  \end{question}
\end{note}

\begin{note}
  \qWhy{} seeks an understanding of the way an agent concluded \(\phi\) has value \(v\) in terms of relations between propositions, values, and \pool{1}.

  Following the analysis of \scen{3}~\ref{illu:gist:calc}~and~\ref{scen:animalism}, the relevant relations, respectively, seemed to be:

  \begin{itemize}[noitemsep]
  \item
    The relation between `\(23 \cdot 15 = 345\)', `\(\text{True}\)', and the calculator.
  \item
    The relation between `Four legs good, two legs bad', `\(\text{True}\)' and the existence of Snowball's explanation.
  \end{itemize}

  As highlighted with respect to \autoref{illu:gist:calc} this relation seems to do work in guiding our understanding of the an event.
  To observe that a relation held indicates something in addition to the observation of what happened.%
  \footnote{
    The \illu{1} of \qWhy{} by \scen{3}~\ref{illu:gist:calc}~and~\ref{scen:animalism} is somewhat limited, as no \scen{0} involves an account of what motivates the respective agents.
    \qWhy{} is not designed to rule out what motivates an agent.
    For example, one may add to \autoref{illu:gist:calc} that the agent wants to calculate \(23 \cdot 15\).
    Functions as a premise, given the understanding of \pool{1}.
    More detail follows in~\autoref{cha:clar}.
  }

  What happened does not entail a corresponding relation.
  For example, in \autoref{illu:gist:calc} the agent presses a button marked `\(=\)', but no relation seems to hold between `\(23 \cdot 15 = 345\)', `\(\text{True}\)', and the pressing the button marked `\(=\)'.

  Further, understanding not only of what happened, but an understanding of what happened as opposed to anything else happening.
  The relation between `Four legs good, two legs bad', `\(\text{True}\)', and existence of Snowball's explanation grants an understanding of why no parallel relation holds between `Four legs good, two legs bad', `\(\text{False}\)', and existence of Snowball's explanation.
  Nothing, in principle, prevents the conclusion `Four legs good, two legs bad' has value `\(\text{False}\)' happening after Snowball's explanation.
  % It seems, respectively, the calculator and Snowball's explanation are sufficient to understand the way the agent concluded the relevant proposition has the specified value.
  %
  % We do not, it seems, need to consider the agent calculating \(23 \cdot 15\) by their understanding of arithmetic, nor the birds understanding Snowball's explanation.
  %
  % In particular why this conclusion as opposed to any other conclusion.
\end{note}

\begin{note}
  \qHow{} seeks an understanding of the way an agent concluded \(\phi\) has value \(v\) in terms of what happened.
  So, answers to \qHow{} when applied to \scen{3}~\ref{illu:gist:calc}~and~\ref{scen:animalism}, at least, are the events captured by the descriptions given.

  No relation seems to hold between `\(23 \cdot 15 = 345\)', `\(\text{True}\)', and the pressing the button marked `\(=\)', but pressing the button marked `\(=\)' answers, in part \qHow{}.
  If the agent's hadn't pressed the button, the calculator wouldn't have displayed `\(345\)'.

  Whether or not a relation that answers \qWhy{} is something which answers \qHow{} is not something we have an opinion on.
  Still, in the case of \scen{3}~\ref{illu:gist:calc}~and~\ref{scen:animalism} it seems clear the events provide `witnesses' for the respective relations.
  Something about \scen{3}~\ref{illu:gist:calc}~and~\ref{scen:animalism} secures the relevant relations.
  % Reasoned from \pool{} to conclusion.
  % Events of \scen{3}~\ref{illu:gist:calc}~and~\ref{scen:animalism} seem to provide an account of the way conclusion was obtained from \pool{}.
\end{note}

\begin{note}
  It does not follow that answers to \qWhy{} are given in terms of answers to \qHow{}.
  To illustrate this point, consider the following \scen{}:

  \begin{scenario}[England AD 932]
    \label{scen:king}
    \vspace{-\baselineskip}
    \begin{screenplay}
    \item[OLD WOMAN:]
      Well, how did you become King, then?
    \item[ARTHUR:]
      The Lady of the Lake, her arm clad in purest shimmering samite, held Excalibur aloft from the bosom of the waters to signify that by Divine Providence\space\dots\space I, Arthur, was to carry Excalibur\dots\space that is why I am your King.
    \item[DENNIS:]
      Look, strange women lying on their backs in ponds handing over swords\space\dots\space that's no basis for a system of government.
      Supreme executive power derives from a mandate from the masses not from some farcical aquatic ceremony.%
      \mbox{ }\hfill\mbox{(\cite[8--9]{Cleese:1974aa})}
    \end{screenplay}
    \vspace{-\baselineskip}
  \end{scenario}

  The old woman asks Arthur \emph{how} the become king.
  Arthur provides an answer in terms of some events which happened, but emphasises that those events are \emph{why} Arthur is king.
  In turn, Dennis accepts the answer provided by Arthur as an account of how Arthur became king but rejects the answer an account of why Arthur is king.
  Instead, an answer is expected in involve the absence of an appropriate system of governance in England.
  So, the answer provided by Arthur is accepted by Dennis as an answer to how, but not as an answer to why.
\end{note}

\begin{note}
  The distinction between `why?' and `how?' present in \autoref{scen:king} parallels \qWhy{} and \qHow{} with respect to conclusions.
  In \autoref{illu:gist:calc}, the agent considered performing mental arithmetic, but the consideration of calculating \(23 \cdot 15 = 345\) seems no basis for the conclusion.
  Instead, it is the use of the calculator which is a basis.

  And, when we consider the relation between `\(23 \cdot 15 = 345\)', `\(\text{True}\)', and the calculator, we understand why (and not merely how) the agent concluded \(23 \cdot 15 = 345\).
\end{note}

\begin{note}
  Our interest is understanding the way an event in which an agent concludes some proposition has some value happened.
  \qWhy{} and \qHow{} are distinct questions, and the following constraint on answers to \qWhy{} in terms of answers to \qHow{} seems intuitive:

  \begin{constraint}{consInclusion}{\issueInclusion{}}
    \mbox{ }
    \vspace{-\baselineskip}
    \begin{itemize}
    \item
      \begin{itenum}
      \item[\emph{If}:]
        Some relation between proposition, value, and \pool{} is, in part, an answer to \qWhy{}.
      \item[\emph{Then}:]
      An event in which the agent concludes the proposition has the value from the \pool{} is, in part, \qHow{}.
    \end{itenum}
  \end{itemize}
  \vspace{-\baselineskip}
  \end{constraint}

  In short, for any relation between a proposition, value, and \pool{} which grants some understanding of the way an agent concluded \(\phi\) has value \(v\), there is an event such that the agent has concluded the proposition has some value from the \pool{}.
\end{note}

\begin{note}
  I consider \issueInclusion{} intuitive.
  In principle, relation between any proposition, value, and \pool{} may answer \qWhy{}.
  However, no other relation.

  The analysis of \scen{3}~\ref{illu:gist:calc}~and~\ref{scen:animalism} may be taken to motivate \issueInclusion{}, but I expect the analysis given is sensible given the adherence to \issueInclusion{}.

  \begin{itemize}[noitemsep]
  \item
    Consider \autoref{illu:gist:calc}.

    The agent concluded `\(23 \cdot 15 = 345\)' and `\(\text{True}\)' from the calculator, and hence if a relation between these items answers \qWhy{} (which it intuitively does), then there is a corresponding answer to \qHow{}.

    And, while a relation may hold between `\(23 \cdot 15 = 345\)', `\(\text{True}\)' and whatever \pool{} would be associated with the agent calculating \(23 \cdot 15 = 345\) from their understanding of Arithmetic.
    There is no event in which the agent concluded `\(23 \cdot 15 = 345\)', has value `\(\text{True}\)' by performing arithmetic.
    So, there is no answer to \qHow{} with respect to this \pool{}, and given \issueInclusion{}, it is not possible for the relation to answer \qWhy{}.
  \item
    Parallel reasoning applies to~\autoref{scen:animalism}.

    The bird's conclusion of `Four legs good, two legs bad' has value `\(\text{True}\)' from the existence of Snowball's explanation answers \qHow{}, and the relation between the proposition, value, and \pool{} answers \qWhy{}, in line with \issueInclusion{}.

    And, while a relation may hold between `Four legs good, two legs bad' `\(\text{True}\)' and the \emph{content} of Snowball's explanation, the birds failed to understand the content.
    Hence, with no answer to \qHow{}, it is not possible for this relation to answer \qWhy{}.
  \end{itemize}
  Hence, \issueInclusion{} seems plausible, and seems to do work.%
  \footnote{
    Note, \qHow{} does not explicitly require the relevant event to be the event in which the agent concludes \(\phi\) has value \(v\) from the \pool{}.
    Hence, a previous event in which the agent concluded `\(23 \cdot 15 = 345\)' has value `\(\text{True}\)' may answer \qHow{}.
    Still, given \(23 \cdot 15 = 345\) in~\autoref{illu:gist:calc} functions as some arbitrary multiplication that the agent may calculation, we may assume the agent has never calculated \(23 \cdot 15 = 345\).
  }\(^{,}\)%
  \footnote{
    Alternatively, \issueInclusion{} may be thought to narrow the range on answers to \qWhy{}.
    That is to say, \issueInclusion{} functions to disambiguate the sense of `why' used in the statement of \qWhy{}.
    If so, then \issueInclusion{} is not a constraint, it is simply part of the way \qWhy{} is understood.
    However, I do not think this is the case.
    I think it is plausible that the sense of `why' present in \qWhy{} may be understood without reference to \issueInclusion{}, and hence that \issueInclusion{} amounts to a substantive constraint.
    The document will assume this is the case, but only the framing depends on this.
  }
\end{note}

\begin{note}
  Still, we have only considered two \scen{1} and \issueInclusion{} is a general constraint.
  Hence, if there is some doubt regarding \issueInclusion{} then further argument is required.
  It is not clear that there are \scen{1} in which \issueInclusion{} fails to hold.
  However, lack of apparent counterexamples is not argument.

  Though I would very much like to provide an argument for \issueInclusion{}, I am not aware of any.%
  \footnote{
    The particular construct of \qWhy{}, \qHow{}, and \issueInclusion{} is somewhat idiosyncratic, and so it is no surprise there are no direct argument for \issueInclusion{}.
    However, I take the idea captured by \issueInclusion{} to be intuitive, and I am not aware of any argument for this idea.
  }
  Perhaps \issueInclusion{} is sufficiently intuitive that it is taken for granted.
  Or, perhaps the failure of \issueInclusion{} is recognised and I have yet to stumble upon the acknowledgement of its failure.

  In any case, no argument will be provided.
  The goal of this document is to provide a recipe for generating counterexamples to \issueInclusion{}.
  By recipe, features of \scen{1} which lead to violations.

  Before closing this introduction, say a little more regarding the recipe for counterexamples.
  First, though no clear arguments for \issueInclusion{}, motivation by relation to reasons.
\end{note}

\section*{\issueInclusion{} and reasons}
\label{sec:reasons}

\begin{note}
  \qWhy{} and \qHow{} are questions about the way an event in which an agent concludes some proposition has some value happened.

  \qWhy{} seeks `explanatory reasons', summarised by \citeauthor{Hieronymi:2011aa} (\citeyear{Hieronymi:2011aa}) as:

  \begin{quote}
    [T]he reasons why things happen, or why things are the way they are.\newline
    \mbox{ }\hfill\mbox{(\citeyear[410]{Hieronymi:2011aa})}
  \end{quote}

  To borrow an an example given by \citeauthor{Hieronymi:2011aa}, the extreme heat or the faulty construction is a reason why the engine failed (\citeyear[409]{Hieronymi:2011aa}).
  Likewise, the engine failing may be a reason why the steamboat is moored, and the steamboat being moored is a reason why the hotel is fully booked, etc.

  With \qWhy{}, the relevant explanatory reason is a relation between a proposition, value, and a \pool{}.
  The relation between `\(23 \cdot 15 = 345\)', `\(\text{True}\)' and the calculator captures a reason why the agent has concluded `\(23 \cdot 15 = 345\)' has value `\(\text{True}\)'.
\end{note}

\begin{note}
  In general, explanatory reasons are distinct from an \emph{agent's} reasons, whatever these may be.
  Though, in some cases, explanatory reasons may be involve an agent's reasons.
  This contrast and involvement forms the opening question of \citeauthor{Davidson:1963aa}'s \citetitle{Davidson:1963aa}:

  \begin{quote}
    What is the relation between a reason and an action when the reason explains the action by giving the agent's reason for doing what he did?
    We may call such explanations \emph{rationalizations}, and say that the reason \emph{rationalizes} the action.%
    \mbox{}\hfill\mbox{(\citeyear[685]{Davidson:1963aa})}
  \end{quote}

  \citeauthor{Davidson:1963aa}'s primary interest is with an explanatory reason which explains an action by giving the agent's reason for performing the action.
  At issue is why the agent did what they did, and the explanatory reason for this involves the agent's reason.

  % A conclusion is also the result of an act and isolate the relevant belief component as the premises.
\end{note}

\begin{note}
  Rationalisation for \citeauthor{Davidson:1963aa} cover all actions.
  A conclusion is a particular specific type of act.%
  % \footnote{
  %   `The agent concluded \(\phi\) has value \(v\)' may be re-expressed as `the agent performed an act in which they concluded \(\phi\) has value \(v\)', in the same way as `the agent buttered the toast' may be re-expressed as `the agent performed an act in which the toast was buttered'.
  % }
  And, with respect to conclusions, a sensible candidate for an agent's reasons are \pool{0}.
  For example, we may say with relative ease that:%
  \footnote{
    These may also be understood, in line with \citeauthor{Smith:1994wo} (\citeyear{Smith:1994wo}), as `motivating reasons':
    \begin{quote}
      The distinctive feature of a motivating reason to \(\phi\) is that, in virtue of having such a reason, an agent is in a state that is \emph{explanatory} of her \(\phi\)-ing, at least other things being equal --- other things must be equal because an agent may have a motivating reason to \(\phi\) without that reason's being overriding.%
      \mbox{}\hfill\mbox{(\citeyear{Smith:1994wo})}
    \end{quote}
    However, our interest is with the \emph{relation} between a proposition, value, and \pool{}.
  }

  \begin{itemize}[noitemsep]
  \item
    The reason for which the agent concluded \(23 \cdot 15 = 345\) was the calculator.
  \item
    The reason for which the birds concluded `Four legs good, two legs bad' was (the existence of) Snowball's explanation.
  \end{itemize}

  So, it seems relations between propositions, values, and \pool{1} may be understood as \citeauthor{Davidson:1963aa}ian rationalisations.
  Or, conversely, one may grant a relation between a proposition, value, and a \pool{1} provides an explanatory reason, and understand the explanatory reason in terms of an agent's reasons by identifying the \pool{1} as the agent's reasons.%
  \footnote{
    For \citeauthor{Davidson:1963aa} an agent's reason is, roughly; \textquote{some feature, consequence, or aspect of the action the agent wanted, desired, prized, held dear, thought dutiful, beneficial, obligatory, or agreeable} (\citeyear[685]{Davidson:1963aa}).
  In this respect, belief-desire pair, and action is something like buttering the toast.
  However, interest with conclusions, and hence ignore the pro-attitude component.
}
\end{note}

\begin{note}
  Now, the key argument of made in \citetitle{Davidson:1963aa} is:
  \begin{quote}
    [R]ationalization is a species of ordinary causal explanation.%
    \mbox{ }\hfill\mbox{(\citeyear[685]{Davidson:1963aa})}
  \end{quote}

  If we grant relations between a proposition, value, and \pool{1} are rationalisations, and rationalisations are causal explanations, and \pool{1} capture an agent's reasons then \issueInclusion{} follows.

  For, suppose some relation between a proposition, value, and \pool{} which answers \qWhy{}.
  By assumption, this relation amounts to a rationalisation.
  And, by assumption rationalisations as causal explanations.
  Hence, there must be some causal relation holding between a \pool{1} and the agent's conclusion that the proposition has the value.
  And, grating there is a causal relation between the agent's conclusion that the proposition has the value and a \pool{1}, then there is some event in which the agent concludes the proposition has the value from the \pool{1}.
  This event is an answer to \qHow{}.
\end{note}

\begin{note}
  \citeauthor{Davidson:1963aa}'s account considers actions in general.
  Still, granting that a conclusion is always the result of reasoning, \citeauthor{Broome:2013aa}'s account of active reasoning explicitly links the same idea to conclusions:
  \begin{quote}
    I have arrived at necessary and sufficient conditions for a process to be active reasoning.

    Active reasoning is a particular sort of process by which conscious premise-attitudes cause you to acquire a conclusion-attitude.
    The process is that you operate on the contents of your premise-attitudes following a rule, to construct the conclusion, which is the content of a new attitude of yours that you acquire in the process.%
    \mbox{ }\hfill\mbox{(\citeyear[234]{Broome:2013aa})}
  \end{quote}

  \citeauthor{Broome:2019aa}'s account of active reasoning parallels \citeauthor{Davidson:1963aa}'d understanding of rationalisations.
  However, \citeauthor{Broome:2019aa}'s account is strictly stronger.
  For, \citeauthor{Davidson:1963aa} assumes rationalisations are such that an action is explained by giving an agent's reason for doing what they did.
  Hence, in certain cases there may be cases in which rationalisations do not exhaust explanatory reasons.
  \citeauthor{Broome:2019aa}, however, holds that causation of a conclusion-attitude by premise-attitudes is sufficient for a process to be active reasoning, and so we do not need to consider anything other than the causal relation to understand the way the event happened.%
  \footnote{
    I claimed above that I am not aware of any argument for \issueInclusion{}.
    However, if the link between \citeauthor{Broome:2013aa}'s account of active reasoning and \issueInclusion{} holds, it seems whatever argument \citeauthor{Broome:2013aa} has given from the account of active reasoning should extend to \issueInclusion{}.

    Still, while \citeauthor{Broome:2013aa} motivate the account of active reasoning, at issue is whether the condition \citeauthor{Broome:2013aa} gives is sufficient.
    \citeauthor{Broome:2013aa} considers and dismisses a number of other conditions to secure sufficiency
    (\citeyear[cf.][\S13.2]{Broome:2013aa}), and while I think this is good motivation, it falls short of an argument.
    For, I see no clear way to extend the specific dismissals to ensure no other condition is required for sufficiency in general.

    Likewise, we noted how \citeauthor{Davidson:1963aa}'s account is compatible with other explanatory reasons.
    %focuses on causal explanation, and does not rule out that there may be cases in a reason explains an action by given the agent's reason for doing what they did in addition to other considerations (which may be causally involved).
  }
\end{note}

\begin{note}
  So, capturing explanatory reasons in terms of causal explanation motivations \issueInclusion{}.
  However, abstracting a little, the role of causation in \citeauthor{Davidson:1963aa} and \citeauthor{Broome:2013aa}'s account is of interest only to the extent that the relevant causal relations ensure the relation between a proposition, value, and \pool{} which holds between the conclusion and the \pool{} is sufficient.
  And, such accounts do not need to appeal to causation to ensure this sufficiency.

  For a final illustration, consider \citeauthor{Hieronymi:2011aa}'s (\citeyear{Hieronymi:2011aa}) account of action explanation:

  \begin{quote}
    [W]henever an agent acts for reasons, the agent, in some sense, takes certain considerations to settle the question of whether so to act, therein intends so to act, and executes that intention in action.

    If this much is uncontroversial (and, under some interpretation, I believe it must be), we can use it as a form for filling out.
    I propose, then, that we explain an event that is an action done for reasons by appealing to the fact that the agent took certain considerations to settle the question of whether to act in some way, therein intended so to act, and successfully executed that intention in action.
    I suggest that \emph{this} complex fact [\dots] explains the action by giving the agent's reason for acting.%
    \mbox{ }\hfill\mbox{(\citeyear[421]{Hieronymi:2011aa})}
  \end{quote}

  The key feature of \citeauthor{Hieronymi:2011aa}'s account is the agent taking certain considerations to settle the question of whether to act in some way.
  \citeauthor{Hieronymi:2011aa} only requires the agent took certain considerations to settle a question, whether this taking amounts to causation or otherwise.
  \citeauthor{Hieronymi:2011aa} understands this taking to be an activity, and hence there is a corresponding event.
  Therefore, if a relation between a proposition, value, and \pool{} answers \qWhy{} and this relation is understood in terms of the agent taking the \pool{} to settle the question of whether to conclude the proposition has the value, we must have, by \citeauthor{Hieronymi:2011aa}'s account, an answer to \qHow{} in line with \issueInclusion{}.
\end{note}

\begin{note}
  Key idea.
  \issueInclusion{} is motivated by how an agent concludes being sufficient to understand why.
  Specifying the \pool{} gets us a reason, and this reason seems to be sufficient.
  Only need to observe cause and effect, or something like this.
\end{note}


\section*{Questioning \issueInclusion{}}

\begin{note}
  Our interest is understanding the way an event in which some agent \vAgent{} concludes some proposition \(\phi\) has some value \(v\) happened.

  We introduced two questions (\qWhy{}, \qHow{}) and a constraint (\issueInclusion{}) on answers to \qWhy{} by answers to \qHow{}.
  And, we motivated the constraint in part by a pair of \scen{1} (\scen{3}~\ref{illu:gist:calc}~and~\ref{scen:animalism}) and then more generally in terms of reasons.
  %(\cite{Davidson:1963aa},\cite{Broome:2019aa},\cite{Hieronymi:2011aa})

  I think \issueInclusion{}, while intuitive, fails to hold in general.
\end{note}

\begin{note}
  I think that answers to \qWhy{} reduce to psychological facts of an agent, specifically psychological facts which hold of the agent when they conclude.%
  \footnote{
    This is incompatible with views of reasons explanation advanced by~(\cite{Dancy:2000aa}) and~(\cite{Alvarez:2013aa}), in which the state of affairs may be an agent's reason (and not just the evaluation of some state of affairs).
    Though, the argument will not depend on assuming that the relations reduce to psychological facts.

    See~(\cite[413--418]{Hieronymi:2011aa}),~(\cite[3--5]{DOro:2013vh}), and~(\cite[\S2]{Alvarez:2017aa}) for more.
  }
  However, I do not think that a relation between the proposition, value, and \pool{} exhaust the relevant psychological facts.
  In particular, I think that in various cases additional relations between distinct propositions, values and \pool{1} answer \qWhy{} without there being an event in which the agent concluded the propositions have values from \pool{1}.
\end{note}

\begin{note}
  As seen above with respect to \autoref{scen:animalism}, it follows from \issueInclusion{} that the content of Snowball's explanation is irrelevant.
  For, the birds do not understand Snowball's explanation, hence there is no event in which the birds reason from the content to `Four legs good, two legs bad' has value `\(\text{True}\)'.
  And, I think this is correct.

  However, parallel reasoning entails the agent's understanding of arithmetic is irrelevant with respect to \autoref{illu:gist:calc}.
  And, I do not think this entailment holds.
  I think it may be the case that a relation between `\(23 \cdot 15 = 543\)', `\(\text{True}\)' and the agent's understanding of arithmetic \emph{may} answer \qWhy{}.
  Whether this is the case will depend on whether some additional details hold of~\autoref{illu:gist:calc}.
  Still, as the details matter, the entailment, and hence \issueInclusion{}, is not right.
\end{note}

\begin{note}
  In other words, I think there are counterexamples to \issueInclusion{}.
  The primary goal of this document is to provide a recipe for generating counterexample to \issueInclusion{}.

  The document is split into four parts:

  \begin{TOCEnum}
  \item
    \autoref{part:prep}: \nameref{part:prep}.

    We begin by clarifying our understanding of conclusions, \qWhy{}, \qHow{}, and \issueInclusion{}.
    In particular, we provide variations to \qWhy{}, \qHow{}, and \issueInclusion{} in order to precisely capture what the counterexamples we generate our counterexamples to.
  \item
    \autoref{part:ing}: \nameref{part:ing}

    We detail three ideas which will be used to generate counterexamples.
    Each idea captures some phenomenon.
    The counterexamples occur when each idea applies to a \scen{0} in which an agent concludes some proposition has some value from some \pool{}.

    The key idea is that of a \fc{}.
    However, the ideas of \tC{}, and \requ{} will tie \fc{1} to instances in which an agent concludes.
  \item
    \autoref{part:dir}: \nameref{part:dir}

    With the preparation and ingredients in hand, we show how to combine the ingredients to generate counterexamples to \issueInclusion{}.

    We also provide a few sample \scen{1} where \issueInclusion{} fails, and consider any leftover issues from the recipe.
    Here we return to \autoref{illu:gist:calc}.
  \end{TOCEnum}

  This covers how things will happen.
  I do not have a brief account of what will happen.
  The details are too important.
  The recipe is not based on an intuitive understanding of any particular \scen{}.
  Instead, the recipe is based on the way a number of ideas come together when we understand the way some agent concludes some proposition \(\phi\) has some value \(v\) happened.
\end{note}



% \begin{note}
%   Now, all this has been said without giving attention to the conditional observed by the agent in \autoref{illu:gist:calc}:

%   \begin{itemize}
%   \item
%     If the calculator is trustworthy, then the agent would not fail to conclude \(23 \cdot 15 = 345\) via their understanding of arithmetic.
%   \end{itemize}

%   Both natural, and somewhat surprising.

%   Consider the contraposition.

%   \begin{itemize}
%   \item
%     If the agent were to fail to conclude \(23 \cdot 15 = 345\) via their understanding of arithmetic, then the calculator is not trustworthy.
%   \end{itemize}

%   \begin{itemize}
%   \item
%     Is it possible for the agent to fail to conclude \(23 \cdot 15 = 345\) via their understanding of arithmetic?
%   \end{itemize}

%   If possible, then difficulty.
%   For, testimony, so must be, but at the same time, possible that it is not.
%   If not possible, then it doesn't seem that observing \(23 \cdot 15 = 345\) via the testimony of the calculator is sufficient.
%   For, by the previous observation, difficulty with the testimony of the calculator.

%   \begin{itemize}
%   \item
%     Testimony of the calculator \emph{only if} \(23 \cdot 15 = 345\) is a \fc{0} given the agent's understanding of arithmetic.
%   \end{itemize}
% \end{note}

% \begin{note}
%   Important:

%   Multiple ways to conclude.
%   So, have a check.

%   Differs to, for example, concluding {\color{red} ???} from a scientific calculator.
%   {\color{red} ???} goes beyond typical understanding of arithmetic.
%   Parallel pair of conditionals does not hold.

%   Or, alternatively, testimony that {\color{red} ???}.
%   Beyond understanding.
% \end{note}

% \begin{note}
%   I am not sure what to make of \ref{illu:gist:calc}.
%   Understanding of arithmetic is a partial check.
%   However, testimony.

%   Unsure because status of a premise.

%   Basic contrary idea only requires some instances.

%   Argument against this intuition.
%   Type of cases, premises are fixed.
%   Check on own reasoning.

%   First, expand on intuition.
%   Then, introduce type of \scen{0} of interest.
% \end{note}

% \begin{note}
%   Follow part, foundations.
%   Following, turn to details.

% {\color{red} Place somewhere?}.%
%   \footnote{
%     Roughly, if it were the agent failed to conclude \(23 \cdot 15 = 345\) in \autoref{illu:gist:calc}, then there would be conflict between the agent's understanding of arithmetic and the testimony of the calculator.
%     Expressed differently, there would be conflict between the agent's failure to conclude \(23 \cdot 15 = 345\) by their understanding of arithmetic, and a premises involved in concluding \(23 \cdot 15 = 345\) via the calculator.
%     I.e. supposing the agent concludes \(23 \cdot 15 \ne 345\), then for the agent the calculator is not a source of testimony.
%     In the \scen{1} of interest, this hypothetical --- or in some cases possible --- conflict will strictly be between the agent's reasoning from \pool{1} to conclusions.
%   }
% \end{note}

% \begin{note}
%   To illustrate, consider \citeauthor{Broome:2013aa}'s (\citeyear{Broome:2013aa}) account of a `motivating reason'%
%   \footnote{
%     \citeauthor{Broome:2013aa} contrasts `motivating reasons' to `normative reasons'.
%     \begin{quote}
%       Whereas motivating reasons explain or help to explain why a person does something, normative reasons explain or help to explain why a person ought to do something, or to believe something, or to hope for something, or to like something, or in general to F, where ‘F' stands for a verb phrase.%
%     \mbox{}\hfill\mbox{(\citeyear[47]{Broome:2013aa})}
%     \end{quote}
%   }
%   \begin{quote}
%     Sometimes the explanation of why a person does something has a particular character:
%     roughly, it involves the person's rationality in a distinctive way that I shall not try to describe.
%     Then we say the person does what she does for a reason.
%     We might say ‘The reason for which Hannibal used elephants was to terrorize the Romans'.
%     The reason for which a person does something is called a ‘motivating reason'.
%     In general, a motivating reason is whatever explains or helps to explain what a person does in the distinctive way that involves her rationality.
%     \mbox{}\hfill\mbox{(\citeyear[46--47]{Broome:2013aa})}
%   \end{quote}
% \end{note}

% \begin{note}
%   \color{red}

%   Further,

%   For \citeauthor{Davidson:1963aa}, primary reason.

%   \begin{quote}
%     \emph{R} is a primary reason why an agent performed the action \emph{A} under the description \emph{d} only if \emph{R} consists of a pro attitude of the agent toward actions with a certain property, and a belief of the agent that \emph{A}, under the description \emph{d}, has that property.\newline
%     \mbox{ }\hfill\mbox{(\citeyear[687]{Davidson:1963aa})}
%   \end{quote}

%   We have distinguished \qWhy{} from pro-attitudes.
%   However, fill in whatever motivation.
%   What matters is the belief.
%   This is the relevant proposition-value pair.

%   If \citeauthor{Davidson:1963aa}, then granting restriction, seems we don't need to look beyond the proposition-value pair.
% \end{note}

% \begin{note}
%   So, answer to \qWhy{} is constrained by answer to \qHow{} by getting to a reason.

%   For, causal relation.
%   Indeed, from `explanatory', these things are identical.
%   However, from `motivating' still distinction.
%   Agent's reason is causal, but the content is not necessarily causal.%
%   \footnote{
%     On my understanding of \citeauthor{Davidson:1963aa}, there's a tight link between then content of some state and the causal relations that arise from the state.

%     So, go from content to state, and then proceed from here.

%     This, I think, is correct.
%     And, the problem of deviant causal chains highlights this.
%     For, \citeauthor{Davidson:1963aa} recognises there's a problem with the link between content and the causal relations which hold between the states.

%     \begin{quote}
%       Beliefs and desires that would rationalize an action if they caused it in the right way—through a course of practical reasoning, as we might try saying---may cause it in other ways.%
%       \mbox{ }\hfill\mbox{(\citeyear[79]{Davidson:1973vd})}
%     \end{quote}
%   }

%   Here, we get a causal trace.
%   No need to look for any relation of support other than premises of reasoning.
% \end{note}

% \begin{note}
%   \begin{scenario}[Sudoku]
%     \label{scen:sudoku:intro}
%     An agent has some free time.
%     They take out copy of \citetitle{Coussement:2007up} and open the book to some Sudoku puzzle.
%     The agent methodically fills in each cell of the puzzle.

%     The agent pauses for a moment.
%     They have a good understanding of the rules of Sudoku and some free time.
%     Hence, if solution is correct, then would not fail to complete any other Sudoku puzzle in the book.

%     If encounter difficulty, then re-examine proposed solution.

%     The agent concludes the proposed solution is the correct solution to the Sudoku puzzle.
%   \end{scenario}

%   Proposition, `The solution is correct', value `\(\text{True}\)'.
%   The \pool{} includes the initial Sudoku grid and the rules of Sudoku.

%   Relation.

%   Possible relation between some other puzzle and the initial Sudoku grid for that other puzzle and the rules of Sudoku.

%   However, as with \scen{3}~\ref{illu:gist:calc}~and~\ref{scen:animalism}, no role.
% \end{note}

% \begin{note}
%   More broadly, I take the basic idea to capture a pre-theoretical constraint on classes of theories.
%   There are theories that agree with the basic idea, such as \citeauthor{Davidson:1963aa}' causal theory of action (when the action is concluding) and there \emph{may be} theories which do not agree with the basic idea --- though I do not know of any specific theories that are explicitly of this kind.
% \end{note}



%%% Local Variables:
%%% mode: latex
%%% TeX-master: "master"
%%% TeX-engine: luatex
%%% End:


\part{Foundations}

\chapter{Introduction}
\label{cha:introduction}

\begin{note}
  \begin{itemize}
  \item
    Initial \scen{0}, intuitions, questions, relation between answers, motivation for positive answer.
  \end{itemize}
\end{note}

\section{Concluding: Why? and How?}
\label{sec:overview:issue}

\subsection{Initial \scen{0}}

\begin{note}
  \begin{scenario}[Multiplication]
    \label{illu:gist:calc}
    An agent enters `\(23 \times 15\)' into a calculator and presses the button marked `\(=\)'.
    The calculator displays `\(345\)'.

    \mbox{ }

    The agent observes they have the option to calculate \(23 \times 15\) via their understanding of arithmetic.
    And, \emph{if} the calculator is trustworthy, then they would not fail to conclude \(23 \times 15 = 345\) via their understanding of arithmetic.

    \mbox{ }

    The agent concludes \(23 \times 15 = 345\).
  \end{scenario}

  Intuitively, the agent concluded \(23 \times 15 = 345\) from the calculator.
  Personifying a little, we may say the agent has concluded \(23 \times 15 = 345\) from the testimony of the calculator.%
  \footnote{
    Indeed, for our purposes \autoref{illu:gist:calc} may be recast in terms of an agent asking another agent to solve `\(23 \times 15\)', but using a calculator is more natural.
  }

  Still, as noted by the agent, they had the option of concluding \(23 \times 15 = 345\) through their own understanding of arithmetic, and hence without the use of the calculator.

  Still, it is intuitive that an agent concludes some proposition has some value%
  \footnote{In \scen{0} the proposition is `\(23 \times 15 = 345\)' and the value is `true'.
    In isolation, `\(23 \times 15 = 345\)' describes some possible state of affairs, and assigning the value `true' indicates the possible state of affairs is the actual state of affairs.
    Still, the agent may have concluded `\(23 \times 15 = 345\)' is `desired', `impossible', `probable', and so on.
    When speaking generally, we will keep explicit which value a proposition is paired with, though when describing specific \scen{0} or examples, we will leave the associated value implicit.
  }
  from some pool of premises only if the agent reasoned from those premises to the proposition-value pair.

  In other words, the agent may have concluded \(23 \times 15 = 345\) via their understanding of arithmetic, but as the agent did not calculate \(23 \times 15\) themselves, they did not conclude \(23 \times 15 = 345\) from whatever premises would be involved when reasoning via their understanding of arithmetic.

  Indeed, given the agent's understanding of arithmetic, it seems clear that prior to using the calculator the agent knew \emph{whether} \(23 \times 15 = 345\).
  Though, knowing whether \(23 \times 15 = 345\) is not knowing \(23 \times 15 = 345\).
  For example, I expect --- though one of the following equalities does not hold --- you know whether \(345 \times 11 = 3,795\), whether \(3,795 \div 5 = 760\), and whether \(760 \div 8 = 95\).

  Rephrasing things a little, and keeping track of truth, we may say \(23 \times 15 = 345\) was a \fc{} for the agent in \autoref{illu:gist:calc}.
  And, for you, \(345 \times 11 = 3,795\), \(3,795 \div 5 \ne 760\), and \(760 \div 8 = 95\) are \fc{1}.

  Instead, it seems the agent concluded \(23 \times 15 = 345\) by use of the calculator, regardless of whether \(23 \times 15 = 345\) was a \fc{1}.
  And, likewise, if you have concluded \(3,795 \div 5 \ne 760\) from the paragraph above, it was due to my testimony, and not your understanding of arithmetic.
\end{note}

\begin{note}[Summary of basic intuitions]
  So, intuitively, in \autoref{illu:gist:calc} the agent concluded \(23 \times 15 = 345\) from the testimony of the calculator.
  And, intuitively, \(23 \times 15 = 345\) being a \fc{0} had no significant role in the agent concluding \(23 \times 15 = 345\) from the testimony of the calculator.
\end{note}

\subsection{Two questions: Why? and How?}

\begin{note}[Not just concluding]
  So far we have spoken about intuitions of the basic form:

  \begin{itemize}
  \item
    Agent \(A\) concluded some proposition \(\phi\) has some value \(v\) from some pool of premises \(\Phi\).
  \end{itemize}

  I take intuitions of this basic form to be readily available in a variety of \scen{1}, and I also take those intuitions expressed in regards to \autoref{illu:gist:calc} to be fairly clear.

  \phantlabel{how-and-why-first-mention}
  Still, concluding is something an agent does, and in this respect there are (at least) two distinct questions intuitions of the above form may answer:%

  \begin{restatable}[\qWhy{}]{question}{questionWhyBasic}
    \label{q:why}
    Which proposition-value-premises pairings an involved in explaining \emph{why} the agent concluded \(\phi\) has value \(v\)?
  \end{restatable}

  \begin{restatable}[\qHow{}]{question}{questionHowBasic}
    \label{q:how}
    Which proposition-value-premises pairings an involved in explaining \emph{how} the agent concluded \(\phi\) has value \(v\)?
  \end{restatable}

  In basic form, focus is on what the agent did, or alternatively whether some action way performed.
  Did the agent conclude \(\phi\) has value \(v\)?
  Or, did the agent conclude \(\phi\) has value \(v\) from the pool of premises \(\Phi\)?
  Or, perhaps, did the agent fail to conclude \(\phi\) has value \(v\)?

  `How?' and `why?' by contrast, take for granted the agent concluded \(\phi\) has value \(v\) and consider, respectively, how and why this action was performed.

  How do intuitions regarding whether or not an agent performed an action relation to how and why the agent performed the action?
\end{note}

\begin{note}
  \color{red}
  Designed to allow intermediate conclusions.
\end{note}

\begin{note}
  \qWhy{} and \qHow{} are distinct questions.
  Particular instances of broader `why?' and `how?'.
  And, generally speaking, `why?' and `how?' may have distinct answers.

  For example, consider asking why and how some agent arrived at some location.
  `By bicycle' answers how the agent arrived at the post office, but does not answer why the agent arrived at the post office.
  Likewise, `to post a letter' intuitively answers why the agent arrived at the post office, but does not answer how the agent arrived at the post office.%
  \footnote{
    Of course, things get complex.
    Action, motivating reason, belief-desire pair.
    Hence, desire to post a letter is part of how.

    Preface with `intuitively'.
    The point is that may refine intuition regarding the agent concluding \(23 \times 15 = 345\).
  }

  So, the intuitions expressed with respect to \autoref{illu:gist:calc} may answer `how?' but not `why?', or `why?' but not `how?'.

  Still, I suspect this is not the case.
  In the case of \autoref{illu:gist:calc} both have the same rough answer:

  \begin{itemize}
  \item
    The pairing of testimony of the calculator with \(23 \times 15 = 345\), is, in part, \emph{both} how \emph{and} why the agent concluded \(23 \times 15 = 345\).
  \end{itemize}
  That premises associated with the agent's understanding of arithmetic do not answer either `how?' or `why?' is implicit by omission.

  Again, the agent used the testimony of the calculator to conclude \(23 \times 15 = 345\), and the agent appealed to the testimony of the calculator to conclude \(23 \times 15 = 345\).
  The agent did not use their understanding of arithmetic to conclude \(23 \times 15 = 345\), and the agent did not appeal to their understanding of arithmetic to conclude \(23 \times 15 = 345\).
\end{note}

\subsection{Relationship between \qWhy{} and \qHow{}}

\begin{note}
  Our observation that the testimony of the calculator seems to answer both `why?' and `how?' the agent concluded \(23 \times 15 = 345\) suggests, even if --- as a single \scen{0} --- only slightly, the following basic idea:%
  \footnote{
    The observation also suggests the converse holds --- an answer, in part, to `how?' is also, in part, an answer to `why?' --- though I think the converse faces some immediate difficulties.
    For, it seems answers to `how?' may include details that are irrelevant to `why?'.
    For example, typing digits and operators into the calculator answers, in part, how the agent concluded \(23 \times 15 = 345\) but these actions seems irrelevant to why the agent concluded \(23 \times 15 = 345\).
    Rather, an answer to `why?' seems to be limited to the calculator providing testimony that \(23 \times 15 = 345\), regardless of whether it was the agent who used the calculator, or whether the agent observed someone else using the calculator.
  }

  \phantlabel{how-and-why-relation-first-mention}
  \begin{restatable}[\issueInclusion{}]{issue}{issueInclusionFirst}
    \label{issue:why-inc-in-how}
    Some proposition-value-premises pairing is, in part, an answer to \qWhy{} only if that proposition-value-premises pairing is (also), in part, an answer to \qHow{}.
  \end{restatable}

  In other words, in order for premise to, in part, answer \emph{why} an agent concluded \(\phi\) has value \(v\), that premise must also, in part, answer \emph{how} the agent concluded \(\phi\) has value \(v\).

  With respect to \autoref{illu:gist:calc}, the testimony of the calculator satisfies the constraint imposed by the idea, while the agent's understanding of arithmetic does not.
  Specifically, the testimony of the calculator was part of how the agent concluded \(23 \times 15 = 345\), and so the testimony of the calculator may be, in part, an answer to why the agent concluded \(23 \times 15 = 345\).
  However, the agent's understanding of arithmetic was \emph{not} part of how the agent concluded \(23 \times 15 = 345\), and so the agent's understanding of arithmetic \emph{may not}, in part, an answer to why the agent concluded \(23 \times 15 = 345\).

  In addition, the basic idea may be taken to capture some explanatory significance and we may even say:
  The agent's understanding of arithmetic is not, in part, an answer to why the agent concluded \(23 \times 15 = 345\) \emph{because} the agent's understanding of arithmetic was \emph{not} part of how the agent concluded \(23 \times 15 = 345\).
\end{note}

\subsection{Motivation from \citeauthor{Davidson:1963aa}}

\begin{note}
  The basic idea, rather than intuitions regarding specific \scen{1} is our interest.
  Roughly, at least.%
  \footnote{
    We will shortly refine our understanding of `why?' and `how?' to focus on support between premises and conclusions, and will motivate a slightly weaker idea with respect to support.
  }

  Additional \scen{1} may provide additional motivation for the basic idea.
  Though, I think the basic idea is sufficiently intuitive independently of individual \scen{1}.
  Instead, observe the basic idea may be motivated not only by \scen{1}, but also by theories.
  Perhaps the most prominent is \citeauthor{Davidson:1963aa}' causal theory of action.

  \citeauthor{Davidson:1963aa} opens \textcite{Davidson:1963aa} with the following question:

  \begin{quote}
    What is the relation between a reason and an action when the reason explains the action by giving the agent's reason for doing what he did?
    We may call such explanations \emph{rationalizations}, and say that the reason \emph{rationalizes} the action.%
    \mbox{}\hfill\mbox{(\citeyear[685]{Davidson:1963aa})}
  \end{quote}

  As noted, concluding is an action, and hence \qWhy{} is a particular instance of \citeauthor{Davidson:1963aa}' question.
  And, answers to \qWhy{} will be reasons that rationalise the agent concluding \(\phi\) has value \(v\).

  \citeauthor{Davidson:1963aa} argues, in short, for the following answer to the relation between a reason and the rationalisation of an action:

  \begin{quote}
    \begin{enumerate}[label=\arabic*]
      [R]ationalization is a species of ordinary causal explanation.\newline
      \mbox{ }\hfill\mbox{(\citeyear[685]{Davidson:1963aa})}
    \end{enumerate}
  \end{quote}

  Following \citeauthor{Davidson:1963aa}, an answer to \qWhy{} is a rationalisation, rationalisation is an instance of ordinary causal explanation.
  So, the answer to why an agent concluded \(\phi\) has value \(v\) will, in part, by a cause of the agent concluding \(\phi\) has value \(v\).
  Therefore, an answer to why an agent concluded \(\phi\) has value \(v\) is, in part, an answer to how the agent concluded \(\phi\) has value \(v\).

  Implicit in this quick argument is the idea that a causal explanation answers `how?'.
  Note, however, that we did not appeal to the converse.
  Causal theories of action seem to motivate the basic idea, though I do not think the basic idea (directly, at least) motivates causal theories of action (or, specifically, concluding).
  In other words, for our purposes, answers to `how?' need not be causal explanations, though they may be.

  More broadly, I take the basic idea to capture a pre-theoretical constraint on classes of theories.
  There are theories that agree with the basic idea, such as \citeauthor{Davidson:1963aa}' causal theory of action (when the action is concluding) and there \emph{may be} theories which do not agree with the basic idea --- though I do not know of any specific theories that are explicitly of this kind.
\end{note}

\subsection{Questioning the intuitive relationship}

\begin{note}
  The basic idea is more-or-less the basic issue of this document.

  Both intuitions, such as those regarding \autoref{illu:gist:calc}, and theories, such as \citeauthor{Davidson:1963aa} causal theory of action, provide motivation for the basic issue.

  Our goal is to motivate the following, basic, contrary idea:

  \begin{itemize}
  \item
    There are cases in which something is, in part, an answer to `why?' and that something is \emph{not} (also), in part, an answer to `how?'.
  \end{itemize}

  The basic contrary idea is the negation of the basic idea.
  For, the basic idea states, roughly, answers to `why?' are always included in answers to `how?' while the basic contrary idea states that there are cases in which something that answers `why?' does not also answer `how?'.

  The basic contrary idea, then, has the form of an existential.
  We will not motivate the idea that there is always something which answers `why?' but does not also answer `how?'.
  And, in particular, it may be the case that the intuitions observed with respect to \autoref{illu:gist:calc} are correct.
\end{note}

\begin{note}
  Now, all this has been said without giving attention to the conditional observed by the agent in \autoref{illu:gist:calc}:

  \begin{itemize}
  \item
    If the calculator is trustworthy, then the agent would not fail to conclude \(23 \times 15 = 345\) via their understanding of arithmetic.
  \end{itemize}

  Both natural, and somewhat surprising.

  Consider the contraposition.

  \begin{itemize}
  \item
    If the agent were to fail to conclude \(23 \times 15 = 345\) via their understanding of arithmetic, then the calculator is not trustworthy.
  \end{itemize}

  \begin{itemize}
  \item
    Is it possible, from the agent's perspective, fail to conclude \(23 \times 15 = 345\) via their understanding of arithmetic?
  \end{itemize}

  If possible, then difficulty.
  For, testimony, so must be, but at the same time, possible that it is not.
  If not possible, then it doesn't seem that observing \(23 \times 15 = 345\) via the testimony of the calculator is sufficient.
  For, by the previous observation, difficulty with the testimony of the calculator.

  \begin{itemize}
  \item
    Testimony of the calculator \emph{only if} \(23 \times 15 = 345\) is a \fc{0} given the agent's understanding of arithmetic.
  \end{itemize}
\end{note}

\begin{note}
  The only if, interesting.

  Calculator provides information about what is a \fc{}.
  Agent's understanding of arithmetic is why it is a \fc{}.

  So, if being a \fc{0} is involved in concluding, then it may be that understanding of arithmetic is, in part, an answer to `why?'.
  And, as the agent has not concluded, not, in part, an answer to `how?'.
\end{note}

\begin{note}
  Important:

  Multiple ways to conclude.
  So, have a check.

  Differs to, for example, concluding {\color{red} ???} from a scientific calculator.
  {\color{red} ???} goes beyond typical understanding of arithmetic.
  Parallel pair of conditionals does not hold.

  Or, alternatively, testimony that {\color{red} ???}.
  Beyond understanding.
\end{note}

\begin{note}
  I am not sure what to make of \ref{illu:gist:calc}.
  Understanding of arithmetic is a partial check.
  However, testimony.

  Unsure because status of a premise.

  Basic contrary idea only requires some instances.

  Argument against this intuition.
  Type of cases, premises are fixed.
  Check on own reasoning.

  First, expand on intuition.
  Then, introduce type of \scen{0} of interest.
\end{note}

\subsection{Key things going forward}

\begin{note}
  Three things of interest.

  \begin{enumerate}
  \item
    \scen{1} like \autoref{illu:gist:calc} in which an agent concludes some proposition has some value.
  \item
    Relation between Why and how and agent concludes.
    Basic idea, and basic contrary idea.
  \item
    Idea of a \fc{} and how \fc{1} may be involved in concluding.
  \end{enumerate}


  Chapter will be split into three parts
  In \autoref{cha:clarification}, we will clarify what is of interest.
  In particular, \autoref{cha:clarification} will be split into two subsections.

  In~\autoref{sec:clarification:support} we will clarify why we are interested in intuitions regarding \scen{1} such as \autoref{illu:gist:calc}.


  First, clarify what is of interest.
  Two things in particular.
  1. What it is about concluding.
  2. Scenarios of interest.
  \autoref{illu:gist:calc} is similar to the type of \scen{0} that will be the focus on this document.
  However, the argument we provide will not directly apply to \scen{1} like \autoref{illu:gist:calc}.

  In \autoref{sec:clar:type-of-scen} we present the type of \scen{0} we will present an instance of the type of \scen{0} interested in, provide a general description of the \scen{0} type.
  Further, we will provide a detailed contrast between the type of \scen{0} we are interested in and \autoref{illu:gist:calc}.%
  \footnote{
    Roughly, if it were the agent failed to conclude \(23 \times 15 = 345\) in \autoref{illu:gist:calc}, then there would be conflict between the agent's understanding of arithmetic and the testimony of the calculator.
    Expressed differently, there would be conflict between the agent's failure to conclude \(23 \times 15 = 345\) by their understanding of arithmetic, and a premises involved in concluding \(23 \times 15 = 345\) via the calculator.
    I.e. supposing the agent concludes \(23 \times 15 \ne 345\), then from the agent's perspective the calculator is not a source of testimony.
    In the \scen{1} of interest, this hypothetical --- or in some cases possible --- conflict will strictly be between the agent's reasoning from pools of premises to conclusions.
  }

  Generally speaking, I am unsure about the intuitions expressed with respect to~\autoref{illu:gist:calc}.
  \Autoref{illu:gist:calc} shares an important feature with the type of \scen{0} that we will explore in detail.
  Yet, \dots

  I am inclined to think intuitions are fine.
  variation of scenario, what the difference is, and why focus.
  Second, sketch general argument of the paper, and role of \fc{1}.
\end{note}



%%% Local Variables:
%%% mode: latex
%%% TeX-master: "master"
%%% End:

\part{A question and it's answers}

\chapter{\zSN{2}}
\label{cha:zS}

\begin{note}[Intro, locating]
  Now turn to \zSN{}.

  Our goal motivate a negative resolution to~\issueConstraint{}.
  And, as sketched in~\autoref{cha:sketch}, \zSN{} has key role in developing tension.

  Question.
  Positive answer answers why.
  Positive answer only if either concluded or \fc{}.
  Concluded or \fc{} answers why.
  \fc{} answers why only if relation of support answers why.
  Relation of support answers why only if proposition-value-premises pairing answers why.

  But, \fc{}, so proposition-value-premises pairing is not an answer to how, as the agent has not witnessed reasoning.

  Now, it is consistent that positive answer only if agent has concluded, hence has witnessed reasoning, and hence is, in part, an answer to how.

  Figuring out instances where this does not hold will wait until later.


  In contrast to sketch given, being with focus on motivation.

  Motivate \zSN{} via plain language question, similar to initial \qWhy{} and \qHow{} from~\autoref{cha:introduction}.
  Follow similar pattern.
  Just as \qWhy{} and \qHow{} were developed in~\autoref{cha:clarification}, \zSN{}, is developed in a similar way.
  Here, like with \qWhy{}, specifically, two things reduce to the same.
\end{note}

\begin{note}[Map]
  Focus is a question.

  Certain kind of support \zSN{}, or \zS{}.

  Understanding of \zS{}.
\end{note}

\begin{note}
  In order to establish tension we narrow our attention to when concluding \(\pv{\phi}{v}\) concluding \(\pv{\phi}{v}\) involves the agent establishing a particular property with respect to \(\pv{\phi}{v}\).
  We term the property `\zSN{0}', or `\zS{}' for short.

  Positive resolution only requires existence of cases.
  Hence, existence of cases with this property.
  This will be sufficient.
  Any case of concluding which involves \csVImp{} will also be an instance of concluding.
\end{note}

\paragraph{Naming}

\begin{note}[Naming]
  Our choice of the term `\zgb{0}' is metaphorical.
  \zgb{2} is a family of flower plants which, typically, have the appearance of a single stem with no branches.
  If one starts just before the flower and works back down the stem, one will not find a branch which, if taken, would lead to a different flower.
  In comparison, if one starts with an agent's epistemic state prior to the agent concluding \(\pv{\phi}{v}\) from \(\Phi\) and~\autoref{question:zs} has a negative answer with respect to \(\pvp{\phi}{v}{\Phi}\), then one will not find a branch which leads to a different conclusion.

  I have some doubts as to whether or not this metaphor really works, but some term is required.
  `Palm-tree-support', or `Arecaceae-support' would also work.
\end{note}

\begin{note}[The token `\qzS{}']
  As {\color{red} noted}, we {\color{red} will} associate \zS{} with positive answers to~\autoref{question:zs}.
  So, in anticipation of this connexion we will use the token `\qzS{}' to name and refer to \autoref{question:zs}.
  Though, to be clear, \zS{} will concern an agent \emph{after} concluding \(\pv{\phi}{v}\) from \(\Phi\) while \qzS{} concerns the agent \emph{when} concluding \(\pv{\phi}{v}\) from \(\Phi\).
  So, strictly speaking an instance of \zS{} is not equivalent with a positive answer to some instance of \qzS{}.%
  \footnote{
    Whether, when concluding \(\pv{\phi}{v}\) from \(\Phi\), it is the case that the following conditional which quantifies over proposition-value-premises pairings:

    \begin{itemize}
    \item
      For all \(\pvp{\psi}{v'}{\Psi}\) [\emph{if}~\ref{question:zs:subjunctive} and~\ref{question:zs:option} hold, \emph{then} \ref{question:zs:may-fail} holds].
    \end{itemize}
    Where `hold(s)' expands to `hold(s) from the agent's perspective'.
  }
\end{note}

\section{\zS{}}
\label{cha:zS:sec:question}

\begin{note}
  A question about an agent's epistemic state when concluding \(\pv{\phi}{v}\) from \(\Phi\).
  Goal here is to get why involved.
\end{note}

\subsection{A question}
\label{cha:zS:sec:the-question}

\begin{note}
  We begin with the question.

  \begin{restatable}[\qzS{}]{question}{questionZS}
    \label{question:zs}
    For an agent \vAgent{}, when concluding \(\pv{\phi}{v}\) from \(\Phi\), is it the case that:

    \begin{itemize}
    \item
      From \vAgent{}' perspective:
      \begin{itemize}
      \item
        For any proposition-value-premises pairing \(\pvp{\psi}{v'}{\Psi}\):
        \begin{itemize}
        \item[\emph{If}:]
          \begin{enumerate}[label=\alph*., ref=(\alph*)]
          \item
            \label{question:zs:option}
            \vAgent{} has the option of concluding \(\pv{\psi}{v'}\) from \(\Psi\), given the agent's reasoning from \(\Phi\) to \(\pv{\phi}{v}\).
          \end{enumerate}
        \item[\emph{and}:]
          \begin{enumerate}[label=\alph*., ref=(\alph*), resume]
          \item
            \label{question:zs:subjunctive}
            \vAgent{} would not conclude \(\pv{\phi}{v}\) from \(\Phi\), if \vAgent{} were to attempt and fail to conclude \(\pv{\psi}{v'}\) from \(\Psi\).%
          \end{enumerate}
        \item[\emph{then}:]
          \begin{enumerate}[label=\alph*., ref=(\alph*), resume]
          \item
            \label{question:zs:may-fail}
            \vAgent{} would conclude \(\pv{\psi}{v'}\) from \(\Psi\), if \vAgent{} were to attempt to conclude \(\pv{\psi}{v'}\) from \(\Psi\).
          \end{enumerate}
        \end{itemize}
      \end{itemize}
    \end{itemize}
    \vspace{-\baselineskip}
  \end{restatable}
\end{note}

\begin{note}
  \qzS{} is delicate.

  Implicit is link between present conclusion and potential conclusions, from the agent's perspective.

  Uncertainty about whether or not one has the option to conclude something blocks present conclusion.

  So, positive answer, for any potential conclusion, would conclude in the present.

  In negative answers, potential conclusion, but either no or negative opinion on concluding.

  This doesn't directly tell us anything about whether or not the agent concludes \(\pv{\phi}{v}\) from \(\Phi\).
  Agent may continue, or may not.

  Cover basics of \qzS{} and then link to \qWhy{}.
  Specifically, in \autoref{cha:zS:section:qzs-and-why}.

  Most of the focus here is on connecting \qzS{} with \qWhy{}.
\end{note}

\paragraph{Cases}

\begin{note}
  \qzS{} concerns an agent when concluding \(\pv{\phi}{v}\) from \(\Phi\), but just prior to having concluding \(\pv{\phi}{v}\) from \(\Phi\).
  And, in paraphrase, asks whether there is some proposition-value-premises pairing \(\pvp{\psi}{v'}{\Psi}\) such that for the agent and from their perspective:
  \begin{itemize}
  \item
    The agent \emph{may} fail to conclude \(\pv{\psi}{v'}\) from \(\Psi\), and hence \emph{may} refrain from concluding \(\pv{\phi}{v}\) from \(\Phi\).
  \end{itemize}

  This paraphrase combines and makes implicit various aspects of clauses~\ref{question:zs:option},~\ref{question:zs:subjunctive}, and~\ref{question:zs:may-fail} to form a single simple statement which is assessed from the agent's perspective.

  We step through~\ref{question:zs:may-fail},~\ref{question:zs:option}, and then~\ref{question:zs:subjunctive}:

  \begin{itemize}
  \item
    \ref{question:zs:may-fail}, the question, would the agent conclude \(\pv{\psi}{v'}\) from \(\Psi\) for any such \(\pvp{\psi}{v'}{\Psi}\).

    the question is whether the agent may \emph{refraining} to conclude \(\pv{\phi}{v}\) from \(v\), while \autoref{question:zs} (in particular~\ref{question:zs:may-fail}) asks whether the \emph{would} conclude \(\pv{\phi}{v}\) from \(v\).

  Still, given the broader context,%
  \footnote{
    In general the agent may also get sidetracked and so on, but the broader context \autoref{question:zs} is that the agent is (in the process of) concluding \(\pv{\phi}{v}\) from \(\Phi\), and likewise for the paraphrase.
  }
  these are equivalent questions.
  The difference is how negative and positive answers are characterised.
  \autoref{question:zs} has a positive answer just in case the agent would conclude, and the paraphrase has a positive answer just in case the agent would \emph{not} conclude.

  \emph{At issue} is whether there exists some \(\pvp{\psi}{v'}{\Psi}\).
\item
    \ref{question:zs:option} is implicit in the reading of `may' --- the agent may fail to conclude \(\pv{\psi}{v'}\) from \(\Psi\) only if the agent has the option of concluding \(\pv{\psi}{v'}\) from \(\Psi\).

    In~\autoref{question:zs}, ensuring that the antecedent of the `would' conditional is always interpreted relative to the agent's present epistemic state.%
    \footnote{
      Basic understanding of subjunctive conditionals: Lewis, Stalnaker, Veltman, etc.
    }
    In the paraphrase the three clauses are combined into a single statement with an implicit temporal ordering, so the relevant subjunctive antecedents are already in place.%
    \footnote{
      Possible to omit \ref{question:zs:option} if rewrite \ref{question:zs:subjunctive}.
    }

  \item
    The relationship between failing to conclude \(\pv{\psi}{v'}\) from \(\Psi\) and refraining to conclude \(\pv{\phi}{v}\) from \(\Phi\) captured by~\ref{question:zs:subjunctive} is replaced in the paraphrase by `hence'.
    Implicit is that failing to conclude \(\pv{\psi}{v'}\) from \(\Psi\) explains why the agent would refrain from concluding \(\pv{\phi}{v}\) from \(\Phi\).

  \end{itemize}
\end{note}

\begin{note}[Quantified conditional]
  Whether there is such a \(\pvp{\psi}{v'}{\Psi}\).

  Clauses~\ref{question:zs:option},~\ref{question:zs:subjunctive}, and~\ref{question:zs:may-fail} of~\autoref{question:zs} from a quantified conditional, so there are three distinct ways in which~\ref{question:zs} may receive a positive answer.
  \begin{itemize}
  \item
    For any proposition-value-premises pairing \(\pvp{\psi}{v'}{\Psi}\), either:
    \begin{enumerate}[label=\alph*\('\).]
    \item
      It is not the case that the agent has the option of concluding \(\pv{\psi}{v'}\) from \(\Psi\), given the agent's reasoning from \(\Phi\) to \(\pv{\phi}{v}\).
    \item
      The agent may conclude \(\pv{\phi}{v}\) from \(\Phi\), (even) if they were to attempt and fail to conclude \(\pv{\psi}{v'}\) from \(\Psi\).
    \item
      The agent would conclude \(\pv{\psi}{v'}\) from \(\Psi\), if they attempted to do so.
    \end{enumerate}
  \end{itemize}

  In other words, negative answer only if from the agent's perspective there is some \(\pvp{\psi}{v'}{\Psi}\) such that the agent may fail to conclude \(\pv{\psi}{v'}\) from \(\Psi\), and if the agent were to fail they would not conclude \(\pv{\phi}{v}\) from \(\Phi\).%
  \footnote{
    Same for the paraphrase.
    Existential.
  }
\end{note}

\begin{note}
  Our interest is primarily in cases where both~\ref{question:zs:option} and~\ref{question:zs:subjunctive} hold for some \(\pvp{\psi}{v'}{\Psi}\)-pairing.

  \autoref{cha:zS:sec:question:illu} will contain additional \illu{1} of positive and negative answers to~\autoref{question:zs}, and in particular \illu{1} where~\ref{question:zs:option} and~\ref{question:zs:subjunctive} fail.

  However, our motivation for considering~\autoref{question:zs} is the relation between concluding (or failing to conclude)  \(\pv{\phi}{v}\) from \(\Phi\) and concluding (or failing to conclude) \(\pv{\psi}{v'}\) from \(\Psi\).

  First link the kind of case we are interested in asking~\autoref{question:zs} to \cScen{1}, and make this link explicit by revisiting a pair of \scen{1}.

  Second, highlight why.

  Third, features in additional detail.

  Following, in~\autoref{cha:zS:sec:zS}, kind of support: \zS{}.
  Examples of \zS{} will also provide additional examples of \scen{0}.
\end{note}

\begin{note}[\cScen{1}, examples]
  {
    \color{red}
    In such cases,~\autoref{question:zs} focuses on whether the agent, from their perspective, would conclude \(\pv{\psi}{v'}\) from \(\Psi\).

    We are already familiar with cases of this kind.
    Indeed, as case of this kind is a \cScen{0}.
    For, \cScen{1}, \dots.

    For the relevant conditionals in \cScen{1}, is it the case, from the agent's perspective, that they would conclude.
  }

  To illustrate, consider again \autoref{illu:gist:roots} and \autoref{illu:sketch:prop-logic}:
\end{note}

\begin{note}[\autoref{illu:gist:roots}]
  \autoref{illu:gist:roots} involves an agent concluding either \(x = 1\) or \(x = -\sfrac{1}{2}\) from premise that for some \(x \in \mathbb{R}\), \(2x^{2} - x - 1 = 0\).%
  \footnote{
    Abstractly, \autoref{illu:gist:roots} is a case where the agent would not conclude \(\pv{\phi}{v}\) from \(\Phi\) if the agent failed to conclude \(\pv{\phi}{v}\) from \(\Psi\).
    I.e.\ the relevant conclusion is the same in both proposition-value-premises pairings, the only difference is the relevant pools of premises (and method of reasoning).
  }
  And, when concluding either \(x = 1\) or \(x = -\sfrac{1}{2}\) the agent observes that \emph{if} \(x = 1\) or \(x = -\sfrac{1}{2}\), then they would also be able to observe this via factorisation.

  In other words, if the agent attempted to conclude either \(x = 1\) or \(x = -\sfrac{1}{2}\) via factorisation and failed, the agent would not conclude either \(x = 1\) or \(x = -\sfrac{1}{2}\) via (their application of) the quadratic formula.
\end{note}

\begin{note}[\autoref{illu:sketch:prop-logic}]
  Likewise, \autoref{illu:sketch:prop-logic} involves an agent concluding some sentence \(A\) is a syntactic theorem of propositional logic via a formula derivation.%
  \footnote{
    Abstractly, \autoref{illu:gist:roots} is a case where the agent would not conclude \(\pv{\phi}{v}\) from \(\Phi\) if the agent failed to conclude \(\pv{\psi}{v}\) from \(\Psi\), where \(\phi\) is distinct from \(\psi\) and \(\Phi\) is distinct from \(\Psi\).
    Soundness (and completeness) relates syntactic and semantic theorems of propositional logic, but these are distinct, as may be observed by considering, for example, a logic which is incomplete, or an unsound proof system.
  }

  And, when concluding \(A\) is a syntactic theorem, the agent observes that \(A\) is a syntactic theorem only if \(A\) is also a semantic theorem (from soundness).

  In other words, if the agent attempt to show \(A\) is true under an arbitrary valuation and failed, the agent would not conclude \(A\) is a syntactic theorem.
\end{note}

\begin{note}[In general]
  Generally speaking, the proposition-value-premises pairing present in a \cScen{0} just is what is required for both~\ref{question:zs:option} and~\ref{question:zs:subjunctive} to hold.
  Hence, when~\autoref{question:zs} is paired with an \cScen{0},~\autoref{question:zs} asks whether the agent, from their perspective, would conclude the relevant proposition-value pair from the relevant pool of premises.
\end{note}


\begin{note}
  With example, failure to conclude because 


  Broader, failure to conclude because no conclusion.

  `\emph{Unless}'
  {
    \color{red}
    Strong understanding.
    In short, always question regarding anything weaker.
    However, we will argue for this.
  }
\end{note}

\subsection[Visualisation]{Visualisation of what is at issue when asking \qzS{}}

\begin{figure}[h]
  \centering
  \begin{tikzpicture}
    \node (origin) at (0,0) {};
    \node (psiSplit) at (1,0) {};
    \node (phiSplit) at (4,0) {};
    %
    \node[anchor=west] (psiV) at  (6,-1)  {\(\pvp{\psi}{v'}{\Psi}\)};
    \node[anchor=west] (psiNv) at (6,-2) {\(\pvp{\psi}{\overline{v'}}{\Psi}\)};
    \node[anchor=west] (psiQ) at (6,-3) {\(\pvp{\psi}{?}{\Psi}\)};
    %
    % \node[anchor=west] (psiVPhiV) at (9,-1) {\(\pv{\phi}{v}\)};
    \node[anchor=west] (psiNvPhiU) at (9,-2) {\(\pv{\phi}{\{\overline{v},?\}}\)};
    \node[anchor=west] (psiQPhiU) at (9,-3) {\(\pv{\phi}{\{\overline{v},?\}}\)};
    %
    \node[anchor=west] (phiQ) at (10,1) {\(\pv{\phi}{?}\)};
    \node[anchor=west] (phiNv) at (10,2) {\(\pv{\phi}{\overline{v}}\)};;
    \node[anchor=west] (phiV) at (10,0) {\(\pv{\phi}{v}\)};
    %
    \draw[-]  (origin) -- (phiV);
    %
    \path[-,dashed] (phiSplit) edge [out=0, in=180] (phiNv);
    \path[-,dashed] (phiSplit) edge [out=0, in=180] (phiQ);
    %
    \path[-.] (psiSplit) edge [out=0, in=180] (psiV);
    \path[-, dashed] (psiSplit) edge [out=0, in=180] (psiNv);
    \path[-, dashed] (psiSplit) edge [out=0, in=180] (psiQ);
    %
    \draw[<-,dotted] (psiV) edge [out=0, in=180] (phiV);
    \draw[->, dotted] (psiNv) edge (psiNvPhiU);
    \draw[->, dotted] (psiQ) edge (psiQPhiU);
    \end{tikzpicture}
    \caption{Sketch of when \qzS{} has a negative answer.}
    \label{fig:csN:illu:overview}
  \end{figure}

\begin{note}[Figure]
  \autoref{fig:csN:illu:overview} provides a rough visualisation of~\qzS{}.

  The flat line captures the agent's reasoning, which concludes with \(\pv{\phi}{v}\).
  In concluding \(\pv{\phi}{v}\) the agent rules out two possibilities with respect to \(\phi\).
  First, that \(\phi\) does not have value \(v\), indicated by \(\pv{\phi}{\overline{v}}\).
  Second, that the agent does not assign any value to \(v\), indicated by \(\pv{\phi}{?}\).
  Prior to concluding \(\pv{\phi}{v}\), the agent's reasoning may have branched to either alternative path, but as the agent has concluded \(\pv{\phi}{v}\), neither path is viable, and hence both paths are represented with a dashed line.

  So far, we have seen only that the agent has concluded \(\pv{\phi}{v}\).

  We now consider some proposition-value-premises pairing \(\pv{\psi}{v'}{\Psi}\) such that if the agent were to fail to conclude \(\pv{\psi}{v'}\) from \(\Phi\), the agent would not conclude \(\pv{\phi}{v}\) from \(\Phi\).

  Intuitively, the dotted arrows from the various combinations of \(\psi\) and \(\{v',\overline{v'},?\}\) read, from top to bottom:
  \begin{itemize}
  \item If \(\phi\) has value \(v\) then the agent may conclude \(\pv{\psi}{v'}\) from \(\Psi\), and:
  \item If the agent concludes \(\psi\) has some value \(\overline{v'}\) from \(\Psi\), then the agent either concludes \(\phi\) has some value other than \(v\), or the agent fails to reach a conclusion regarding \(\phi\) from \(\Phi\).
    Both options are combined via the shorthand \(\pv{\phi}{\{\overline{v},?\}}\).
  \item
    And, likewise if the agent fails to conclude \(\pv{\psi}{v'}\) from \(\Psi\).
  \end{itemize}

  With respect to concluding, observe that prior to ruling out alternative branches with respect to \(\pv{\phi}{\{\overline{v},?\}}\), the agent may have reasoned about whether \(\psi\) has value \(v\).
  And, from the agent's perspective, \(\phi\) has value \(v\) only if \(\psi\) has value \(v'\).
  If \(\psi\) does not have value \(v'\), then either \(\phi\) does not have value \(v\), or the agent's reasoning would not conclude with a value for \(\phi\), indicated by \(\pv{\phi}{\{\overline{v},?\}}\).

  Hence, prior to concluding \(\pv{\phi}{v}\), the agent has concluded \(\pv{\psi}{v'}\).
\end{note}

\begin{note}
  Broadly, then, we may say that an agent has {\color{red} particular kind of conclusion} for \(\pv{\phi}{v}\) just in case when concluding \(\pv{\phi}{v}\) it is not the case that the agent's reasoning would have branched to a different conclusion with respect to \(\phi\).

  However, the visualisation of~\autoref{fig:csN:illu:overview} and this broad statement of {\color{red} positive answer to \qzS{}} are a little too broad.
  For, we are only interested in proposition-value pairs guaranteed by \(\phi\) having value \(v\).
  {\color{red} positive answer to \qzS{}} is not global with respect to all proposition-value pairs that the agent may have reasoned about, but local to those guaranteed by the proposition.
\end{note}

\section{\emph{Why}}
\label{cha:zS:section:qzs-and-why}

\begin{note}
  Intuitively, positive answer in \scen{1}.

  Some care has been taken.

  \autoref{illu:sketch:prop-logic}.
  Only if semantic proof.
  Syntactic proofs, at least in my experience, may be out of reach.
  However, semantic proofs, often straightforward.

  Converse may hold, but more challenging than \autoref{illu:sketch:prop-logic}.

  Similarly, \autoref{illu:gist:roots}.
  Factorisation isn't too difficult.

  \autoref{illu:sketch:math}.
  \(345 \div 15 = 23\) \emph{only if} \(23 \times 15 = 345\).

  Agent has the opportunity, and current result looks good.
\end{note}

\begin{note}
  Now, answers, in part, why.
\end{note}

\begin{note}
  Rough understanding.
  In terms of broader argument, \emph{why}.

  Idea is that agent's perspective regarding \(\pvp{\psi}{v'}{\Psi}\) in part explains why.

  Preferred \illu{0} concerns whether one has or has not lost their keys.
\end{note}

\begin{note}[Motivating \illu{0}]
  {
    \color{red}
    Interest in terms of explaining why and agent did or didn't conclude.
  }

  \begin{illustration}[Lost keys]
    \label{illu:lost-key}
    Tempting as it may be to conclude that a pair of keys are lost after some searching, if the keys really are lost then there aren't in a handful of places you haven't yet thought to look. And, until you have concluded that the keys really aren't in those places, and that there is no-where else to look, the keys aren't really lost.
  \end{illustration}

  For the agent's perspective, there is some \(\pvp{\psi}{v'}{\Psi}\), and this explains why the agent does not conclude they have lost their keys.
\end{note}

\begin{note}
  Similar points for examples given.

  Check via factorising, check via a semantic proof.
\end{note}

\begin{note}
  Here, important, \issueConstraint{} asks about whether a relation of support is part of why an agent \emph{concludes} (only if agent has witnessed).

  With~\autoref{illu:lost-key},~\autoref{illu:gist:roots}, and~\autoref{illu:sketch:prop-logic}, failure to conclude!
\end{note}

\begin{note}
  Still, \emph{negative} answers to~\autoref{question:zs}.

  Interest, what about \emph{positive} answers?

  Positive answer only if from agent's perspective, would conclude.
  No witnessing.
  Relation of support is part of why.

  Lack of support explains, in part, why agent does \emph{not} conclude.
  Conversely, presence of support explains, in part, why agent \emph{does} conclude.
\end{note}

\paragraph{Basic principle}

\begin{note}
  \begin{idea}[\autoref{question:zs} and \emph{Why}]
    \label{prop:qzS-answers-why}
    There are cases in which:

    \begin{itemize}
    \item[\(\pm\)]
      Answers to \autoref{question:zs} answer, in part, why an agent concludes or does not conclude \(\pv{\phi}{v}\) from \(\Phi\).
    \end{itemize}

    Specifically:
    \begin{itemize}
    \item[\(-\)]
      A negative answer to~\autoref{question:zs} answers, in part, why an agent does not conclude \(\pv{\phi}{v}\) from \(\Phi\).
    \item[\(+\)]
      A positive answer to~\autoref{question:zs} answers, in part, why an agent does conclude \(\pv{\phi}{v}\) from \(\Phi\).
    \end{itemize}
  \end{idea}

  Motivation for \autoref{prop:qzS-answers-why} is by cases.
  See additional cases in \autoref{cha:zS:sec:question:illu}.

  Weak point.
  \autoref{prop:qzS-answers-why} is central to overall argument.
  Hence, something else which captures cases.
  Something else does not involve whether or not the agent would conclude.

  I do not think so.
  But, generalising from exhaustion.
  Have not exhausted every possibility, but I've exhausted myself.

  Preferable, I think, to hold that \qzS{} is not relevant.
  \autoref{prop:qzS-answers-why} is an existential.
  Generally speaking, would be good.
  Only trouble when \(\pvp{\psi}{v'}{\Psi}\) is such that the agent has not witnessed reasoning to \(\pv{\psi}{v'}\) from \(\Psi\).
  So, if \qzS{} only applies in such cases, then no problem.

  However, already seen, \autoref{illu:lost-key}.
  Absence of reasoning.

  So, narrow to no cases where a positive answer, but consider \cScen{1}.
\end{note}

\begin{note}
  Run this through \scen{1} listed.
\end{note}

\begin{note}
  \fc{}!
  Or, forgone conclusion, but this is different from a \fc{}.
\end{note}

\begin{note}
  What's needed is positive answer only if support.

  Here, maybe illustrate with general ability.
  Got X.
  General ability.
  So, specific ability to Y.
\end{note}

\begin{note}
  {
    \color{red}
    There's a difference between answering `no' and failing to answer.
    But, the point I'm arguing for works given this distinction.
    There's no real different.
    I mean, the conditional is either true or false.
    But, it's possible that the falsity of the conditional has a role where the truth of the conditional does not.
  }
\end{note}

\subsection{Deviance}
\label{sec:deviance}

\begin{note}
  Here, causal deviance.
\end{note}

\begin{note}
  Problem is, there's no way to guarantee a link between positive answer to \qzS{} and the agent concluding or not refraining from concluding.
\end{note}

\begin{note}
  Argument relies on tying content to explanation.

  In this respect, there is room for an objection.
  Deviant causal chains.
  Point here is that there are cases where these come apart.

  This isn't only a problem for causal theories of reasoning.
  The point is, some instantiation, and so long as act may be caused by something else, then possibly caused by the instantiation.

  So, possible here.

  Well, hold on.
  What is need is the relevance of the content.
  For this objection to work, need to take a theoretical perspective.
  See, in Davidson's case, the idea is fusing these two things together.
  We answer two different questions with a common thing viewed in two ways.

  Still, I think the objection can be pressed!
  Only \emph{really} an explanation is no deviance.
  To the same extent that potential event matters, it matters to the agent that there is no deviance.

  {
    \color{red}
    Resolution is, if deviance, then no agency.
  }

  I think this makes sense, or at least makes enough sense.
  Answers to `why', on this understanding, are tentative.

  Or, rest on presupposition that agent performed the action.

  So, contingent on showing there is no causal deviance.

  This is different to error.
  With error, thing appealed to isn't the case, but appeal still did work.
  Here, it doesn't matter whether or not the case, no work is done.

  In contrast to more typical instances of the problem, don't need to rule out deviant causal chains.
  Instead, just need one instance to fail to hold.
  One instance of non-deviousness.

  Still a problem for a compatible account which avoids.
  For, here, there can't be any direct link from perspective to reason.

  For example, \citeauthor{Hieronymi:2011aa}

    \begin{quote}
      [W]e explain an event that is an action done for reasons by appealing to the fact that the agent took certain considerations to settle the question of whether to act in some way, therein intended so to act, and successfully executed that intention in action.
    [\emph{T}]\emph{his} complex fact, [\dots] is the reason that rationalizes the action---that explains the action by giving the agent's reason for acting.%
    \mbox{ }\hfill\mbox{(\citeyear[431]{Hieronymi:2011aa})}
  \end{quote}

  So, here, considerations which settle question, and in so settling question.
  Link between settling the question and acting.

  Following \citeauthor{Hieronymi:2011aa}, no room for deviance.
  Too tight.

  In other words, so long as this fact holds, there is no distinction between settling the question and acting.
  Therefore, no deviance.

  Compatible, I think.
  Question is whether in resolving \qzS{} is sufficiently tied to resolving the question \citeauthor{Hieronymi:2011aa} identifies.
  And, plausibly is.
  This is what the motivation for \qzS{} did.

  Trouble is, for our purposes, need at least sufficient conditions for when this complex fact obtains.
  And, no account of this.

  \citeauthor{Hieronymi:2011aa} notes the gaps.

  Some tension.
  These considerations aren't premises.
\end{note}

\begin{note}
  So, the other option is to embrace deviant causal chains.
  Have the content, but this doesn't work in the way the agent thinks it does.

  Example from Davidson.

  The trouble here is that the content and resulting action match.
  So, things make sense from the agent's perspective.

  Deviant, but maybe not so deviant here.

  Systematic deviance, where content is separated from role of mental state.

  But, I see no motivation for this.

  Solution to causal chains doesn't get round this, because the result is a restricted account.
  So, there's no guaranteed trade-off here.
  Trouble is, it seems hard to see a case where this wouldn't be the case.
\end{note}


\subsection{Details}
\label{cha:zS:sec:details}

\subsubsection{Clauses~\ref{question:zs:option},~\ref{question:zs:subjunctive}, and the idea of a \requ{0}}
\label{cha:zS:sec:clauses-idea-requ1}

\paragraph{The components of \qzS{}}

\begin{note}
  The primary clauses of interest are clauses~\ref{question:zs:option} and~\ref{question:zs:subjunctive}.

  Intuitively, clause~\ref{question:zs:option} means that, so long as \(\phi\) has value \(v\), the agent has the option of checking whether it makes sense for the agent to conclude \(\pv{\phi}{v}\) from \(\Phi\).
\end{note}
  And, clause~\ref{question:zs:subjunctive} expresses that concluding \(\pv{\psi}{v'}\) from \(\Psi\) is a check on whether it makes sense for the agent, from their perspective, to conclude \(\pv{\phi}{v}\) from \(\Phi\).

\paragraph{General}

\begin{note}
  Constraints placed on \(\pvp{\psi}{v'}{\Psi}\).
  From reasoning involved in process of concluding \(\pv{\phi}{v}\) from \(\Phi\).
  Would lead to failure.

  Conditionals.

  Involved in concluding \(\pv{\phi}{v}\) from \(\Phi\).
  First, enough to break.
  Second, reasoning makes this proposition-value-premises pairing available.

  Pair of additional features

  Second highlights why \(\pvp{\psi}{v'}{\Psi}\) is of interest.
  However, in this respect, not strictly required.
  Given universal, will also include these.

  First,
  Don't need \(\phi\) to have value \(v\).
  Also, implicit, no revision.
  Built up various things in reasoning, and given all of this\dots


  And, maybe reasoning offers something new.
  Though, not the case that \(\pvp{\psi}{v'}{\Psi}\) only from something new.
  Might be the case that negative answer because go off on wrong reasoning.
\end{note}

\paragraph{Option}

\begin{note}
  Hence, it need not be the case that the agent has the option of concluding \(\pv{\psi}{v'}\) from \(\Psi\) from their epistemic state prior to starting line of reasoning (as the agent has not yet concluded that \(\phi\) has value \(v\)).
\end{note}


\paragraph{Would not conclude}

\begin{note}
  Noted failure.
\end{note}

{
  \color{red}
  Note, \ref{question:zs:may-fail} is delicate.
  For, the combination of \ref{question:zs:subjunctive} and \ref{question:zs:option} suggest there is a way of concluding \(\pv{\psi}{v'}\) from \(\Psi\).
  Hence, \ref{question:zs:may-fail} may be read in reference to this.
  However, \ref{question:zs:may-fail} is intended to allow other ways of concluding \(\pv{\psi}{v'}\) from \(\Psi\).
  What matters is that the agent has not concluded \(\pv{\psi}{v'}\) from \(\Psi\), the agent has the option, and the agent may fail.%
  \footnote{
    This is important for witnessing, but also motivated by different methods.
    A different way to putting this is that concluding is two place relation.
    Between premises and conclusion.
    Concluding is not a three place relation between premises, conclusion, and method.
    I should really have this stated as an assumption.

    Still, there is a variant where method comes into play, as I have this via ability.
  }
}

%%%% TEMP from question
\footnote{
  Clause~\ref{question:zs:subjunctive} is expressed by a subjunctive conditional as there is no requirement that the agent will attempt to conclude \(\pv{\psi}{v'}\) from \(\Psi\).

  \color{red}
  As this alternative expression makes clear,~\autoref{question:zs} focuses on the agent (and their epistemic state).
  At no point do we consider any variation of the agent's epistemic state.
  Likewise,~\autoref{question:zs} concerns only the agent's perspective on concluding \(\pv{\psi}{v'}\) from \(\Psi\).
  Whether or not the agent would conclude \(\pv{\psi}{v'}\) from \(\Psi\) is irrelevant.
  What matters is whether, from the agent's perspective, there is potential for reasoning about whether \(\pv{\psi}{v'}\) follows from \(\Psi\) to block concluding \(\pv{\phi}{v}\) from \(\Phi\).
}

\paragraph*{Minor clarifications}

\begin{note}[Importance of \csN{}]
  First, agent's reasoning.
  At issue is whether the agent may reason to a different conclusion.
  There's nothing that would lead me elsewhere.

  Second, agent's reasoning.
  Independent of whether \(\phi\) has value \(v\), \(\psi\) has value \(v'\), or any of the premises.
  Need not be the case that satisfaction amounts to anything substantial.
  No clause for justification, etc.

  Third, competence, rather than performance.
\end{note}

\subsection{\requ{3}}

\begin{note}
  We begin by refining the relevant \(\pvp{\psi}{v'}{\Psi}\) proposition-value-premises pairings of interest from~\qzS{}.
  We term such proposition-value-premises pairings `\requ{1}' of concluding \(\pv{\phi}{v}\) from \(\Phi\).
\end{note}

\begin{note}[Notion of a \requ{}]
  \begin{idea}[\iRequ{}]
    \label{idea:requ}
    \(\pvp{\phi}{v'}{\Psi}\) is a \emph{\requ{}} of concluding \(\pv{\phi}{v}\) from \(\Phi\), with respect to an agent \vAgent{}'s epistemic state if:
    \begin{enumerate}
    \item
      \label{idea:requ:main}
      From the perspective of \vAgent{}' epistemic state, \(\phi\) has value \(v\) only if:
      \begin{enumerate}[label=\alph*., ref=\named{R:\alph*}]
      \item
        \label{idea:requ:pool}
        \vAgent{} has the option of concluding \(\pv{\psi}{v'}\) from \(\Psi\) where:
        \begin{enumerate}[label=\roman*., ref=\named{R:a.\roman*}, series=csIdeaCounter]
        \item
          \label{idea:requ:pool:int}
          \vAgent{} may conclude \(\pv{\psi}{v'}\) from \(\Psi\) without concluding \(\pv{\phi}{v}\) from \(\Phi\) as an intermediary step.
        \item
          \label{idea:requ:pool:ind}
          For any proposition-value pair \(\pv{\psi_{i}}{v_{i}}\) in \(\Psi\), \vAgent{} either has concluded or may conclude \(\pv{\psi_{i}}{v_{i}}\) without concluding \(\pv{\phi}{v}\) from \(\Phi\).
        \end{enumerate}
      \item
        \label{idea:requ:nPsi-nPhi}
        If \vAgent{} were to fail to conclude \(\pv{\psi}{v'}\) from \(\Psi\) prior to reasoning about whether \(\phi\) has value \(v\) given \(\Phi\), \vAgent{} would not conclude \(\pv{\phi}{v}\) from \(\Phi\).
      \end{enumerate}
    \end{enumerate}
    \vspace{-\baselineskip}
  \end{idea}

  With the key clause linking~\autoref{idea:requ} to~\qzS{} is clause~\ref{idea:requ:nPsi-nPhi}.
  For, clause~\ref{idea:requ:nPsi-nPhi} captures the core idea of failure to conclude \(\pv{\psi}{v'}\) from \(\Psi\) leading to failure to conclude \(\pv{\phi}{v}\) from \(\Phi\).

  The role of clause~\ref{idea:requ:pool} is explicitly state various properties \(\pv{\psi}{v'}{\Psi}\) must have in order for any failure to conclude \(\pv{\psi}{v'}\) from \(\Psi\) is relevant to concluding \(\pv{\phi}{v}\) from \(\Phi\).%
  \footnote{
    Indeed, we take \ref{idea:requ:pool:int} and~\ref{idea:requ:pool:ind} to be more-or-less implicit constraints on \(\pvp{\psi}{v'}{\Psi}\) in the statement of \qzS{}.
  }
  In particular, \ref{idea:requ:pool:int} and~\ref{idea:requ:pool:ind} are required to ensure the agent may conclude \(\pv{\psi}{v'}\) from \(\Psi\) independently of concluding \(\pv{\phi}{v}\) from \(\Phi\).

    For, if \ref{idea:requ:pool:int} and \ref{idea:requ:pool:ind} were to fail to hold then:
  \begin{itemize}
  \item
    By~\ref{idea:requ:pool:int}, the agent would need to conclude \(\pv{\phi}{v}\) from \(\Phi\) as a sub-conclusion when reasoning from the relevant pool of premises \(\Psi\).
    Hence, it would not be possible to conclude \(\pv{\psi}{v'}\) from \(\Psi\) without first concluding \(\pv{\phi}{v}\) from \(\Phi\).
  \item
    And, likewise, by~\ref{idea:requ:pool:ind}, the agent need to have already concluded \(\pv{\phi}{v}\) from \(\Phi\) in order to appeal to some of the proposition-value pairs in the relevant pool of premises \(\Psi\).
  \end{itemize}

  Conversely, if both~\ref{idea:requ:pool:int} and~\ref{idea:requ:pool:ind} hold, the agent may conclude \(\pv{\psi}{v'}\) from \(\Psi\) independently of concluding \(\pv{\phi}{v}\) from \(\Phi\).

  Note, however, neither~\ref{idea:requ:pool:int} nor~\ref{idea:requ:pool:ind} rule out the possibility of the agent concluding \(\pv{\phi}{v}\) from \(\Phi\) when concluding \(\pv{\psi}{v'}\) from \(\Psi\) or, conversely, concluding \(\pv{\psi}{v'}\) from \(\Psi\) when concluding \(\pv{\phi}{v'}\) from \(\Phi\).
  There may be an interesting variant of the notion of a \requ{} with such a constraint in place, but such a constraint is not of interest with respect to \qzS{}.
  For, at issue is only whether the agent may at interest is only failure to conclude \(\pv{\psi}{v'}\) from \(\Psi\), and both~\ref{idea:requ:pool:int} and~\ref{idea:requ:pool:ind} ensure lack of concluding \(\pv{\phi}{v}\) from \(\Phi\) will not prevent the agent from reaching a conclusion regarding whether \(\psi\) has value \(v\) given \(\Psi\).
\end{note}

\begin{note}
  \color{red}
  Has the option.
\end{note}

\begin{note}[\requ{2}: Partial check]
  Intuitively, concluding \(\pv{\psi}{v'}\) from \(\Psi\) would serve as a partial check on whether the agent may reason to a conclusion other than \(\pv{\phi}{v}\), captured by~\ref{idea:requ:nPsi-nPhi}.

  Concluding \(\pv{\psi}{v'}\) from \(\Psi\) is a check.
  For, if the agent were to fail to conclude \(\pv{\psi}{v'}\) from \(\Psi\) then, the agent would not conclude \(\pv{\phi}{v}\) from \(\Phi\), from the agent's perspective.
  Hence, contraposing, the agent would conclude \(\pv{\phi}{v}\) from \(\Phi\) only if the agent would conclude \(\pv{\psi}{v'}\) from \(\Psi\), from the agent's perspective.

  However, the check is partial, as it need not be the case that the agent would conclude \(\pv{\psi}{v'}\) from \(\Psi\) only if the agent \(\pv{\phi}{v}\) from \(\Phi\).
  Therefore, failing to conclude \(\pv{\psi}{v'}\) from \(\Psi\) may block concluding \(\pv{\phi}{v}\) (from the perspective of the agent) though concluding \(\pv{\psi}{v'}\) from \(\Psi\) need not ensure that the agent would conclude \(\pv{\phi}{v}\).

  Combining these two ideas, intuitively, \(\pv{\psi}{v'}\) is a \requ{} of concluding \(\pv{\phi}{v}\) just in case there is some pool of premises \(\Psi\) such that determining whether the agent would conclude \(\pv{\psi}{v'}\) is an independent partial check on whether the agent may reason to a conclusion other than \(\pv{\phi}{v}\).
\end{note}


\subsubsection{Abstract}

\begin{note}[Abstract motivation]
  First, the combination of~{\color{red} ???} and~{\color{red} ???} keeps the complexity of resolving~\qzS{} relatively low.
  We only need to consider what \(\phi\) having value \(v\) would commit the agent to, so to speak.
  For example, we do not need to consider the consequences of the agent reasoning about some arbitrary proposition-value pair \(\pv{\chi}{v''}\) prior to concluding \(\pv{\phi}{v}\) from \(\Phi\).

  Of course, if the agent would conclude both \(\pv{\chi}{v''}\) and \(\pv{\chi}{\overline{v''}}\) for some \(\chi\), then it seems the agent's epistemic state is in bad shape.
  Still, given that an agent will typically revise their epistemic state upon concluding both \(\pv{\chi}{v''}\) and \(\pv{\chi}{\overline{v''}}\) for some \(\chi\), such concerns may be isolated to a distinct question.

  Second, we avoid --- to some extent --- concerns about over-generating.
  Our overall argument will put~\autoref{question:zs} to work in motivating a negative resolution to~{\color{red} issue:Main}.
  Primarily by drawing consequences from what we will argue is an equivalent characterisation of negative resolutions to~\autoref{question:zs}.
  A concern is that this overall argument may over-generate.
  Given that a negative resolution to~{\color{red} issue:Main} seems by no means clear, unintended consequences of our interest in~\autoref{question:zs} may diminish interest in the tension we hope to motivate, and hence tip favour to a positive answer to~{\color{red} issue:Main}.
  Whether or not there are unintended consequences of broadening the scope of~\autoref{question:zs} is unclear to me.
  Still, without need to investigate, we may ignore any such consequences that may arise.
\end{note}

\subsection*{Narrowing \requ{1}}

\begin{note}[Expanding pool constraints]
  To~\ref{idea:requ:pool} of~\autoref{idea:requ} the following clause may also be added:
  \begin{enumerate}[label=]
  \item
    \begin{enumerate}[label=]
    \item
      \begin{enumerate}[label=\roman*., ref=(\roman*), resume*=csIdeaCounter]
        \setcounter{enumiii}{3}
      \item
        \label{idea:requ:pool:method}
        Concluding \(\pv{\psi}{v'}\) from \(\Psi\) involves the same general method the agent would use to conclude \(\pv{\phi}{v}\) from \(\Phi\).
      \end{enumerate}
    \end{enumerate}
  \end{enumerate}
  We omit~\autoref{idea:requ:pool:method} from the idea of \csN{} for two (related) reasons.
  First, it is not clear what `the same general method' amounts to in detail.
  Second, avoiding questions about method affords flexibility when providing \illu{1} of \zS{}.
  However,~\autoref{idea:requ:pool:method} may be imposed with no loss to the role of \zS{} in the overall argument.
  However, always a check on whether one has the general ability.
\end{note}

\begin{note}
  Reasoning, \support{}, would not reason to a different conclusion.

  Specifically, \requ{} of some conclusion.
  So long as conclusion, then it is possible to reason about whether \(\psi\) has value \(v'\), and unless conclude \(\psi\) has value \(v'\), would not conclude \(\phi\) has value \(v\).

  Intuitively, \requ{} as an independent check on the reasoning.
  If don't hold \(\psi\) from premises, then question about whether \(\phi\).

  Claiming support, necessary condition is satisfying all \requ{1}.
  Claiming support, then, is weaker than having support.
  Restricted to whether conclusion of reasoning would introduce a \requ{}.
  And, may be further restricted without impact to the tension we will develop to whether the conclusion would `clearly' introduce a \requ{}.
\end{note}

\subsubsection{Prior to concluding\dots}

\begin{note}[Prior to concluding\dots]
  An important feature of \qzS{} \dots

  Not particularly marked.
  Allow agent to have built up a bunch of stuff in the reasoning.

  Example.

  \begin{illustration}[Velocity]
    \label{ill:velocity}
    Agent is provided with information about how far a car has travelled north as a function of time travelled.

    From this, take the derivative of the function to obtain the (instantaneous) velocity of the car at a handful of points in time.

    And, from the (instantaneous) velocity of the car, the agent calculates the (instantaneous) acceleration of the car at each of the points in time.

    The agent also has information about the speed of the car as a function of time travelled, and the agent may calculate speed by the taking magnitude of the (instantaneous) velocity of the car.
  \end{illustration}

  

  \autoref{ill:velocity}, two step calculation.
  Velocity, acceleration.
  After the first step, check by taking the magnitude.
  Calculation of velocity is correct only if taking the magnitude matches speed.

  Just before concluding to include cases such as this.

  Note, \cScen{}.
\end{note}

\begin{note}
  Example highlights how `intermediate conclusions' relate.
  Further point of interest:
  Failure to conclude.

  Two ways to view agent's calculation of the velocity of the car.

  First, as a conclusion.
  Same status as the function.

  Or, as temporary.

  Difference in how we understand agent's present epistemic state.

  On first, the agent's present epistemic state is inconsistent.
  Two proposition-value pairs which conflict.
  Not possible for the car to have velocity the agent calculated and acceleration the agent has been informed of.

  May also be that the function and information about acceleration are inconsistent, but may also be that the agent made a mistake in calculating the velocity of the car.

  On second, the agent's present epistemic state may%
  \footnote{
    Don't have complete perspective on agent's present epistemic state.
  }
  consistent.

  For, made a mistake.
  But, proposition-value pair is not part of present epistemic state, so distinguished from function and information about acceleration, which are consistent.

  This is a distinction we have little interest in.
  What matters is failure to conclude speed.
  Result is either revising inconsistent epistemic state, or abandoning intermediary steps of reasoning.
\end{note}


\section{\zS{}}
\label{cha:zS:sec:zS}

\begin{note}[Answers to \qzS{}]
  For ease of reference, define.
  \begin{definition}[\izS{}]
    \label{idea:zS}
    For an agent \vAgent{}, after concluding \(\pv{\phi}{v}\) from \(\Phi\):
    \begin{itemize}
    \item
      \vAgent{} has \zS{} with respect to \(\pvp{\phi}{v}{\Phi}\).
    \end{itemize}

    \emph{if and only if}

    \begin{itemize}
    \item
      \qzS{} had a \emph{positive} answer when \vAgent{} was concluding \(\pv{\phi}{v}\) from \(\Phi\).
    \end{itemize}
  \end{definition}

  Slight problem here with going for after concluding.
  Possible, it seems, for relation of support to no longer hold.
  Likewise for \zS{}.
  For ease, assume agent has not retracted conclusion.
\end{note}

\begin{note}
  Now, two basic propositions follow.

  \begin{proposition}[When a agent has \zS{}]
    Agent and proposition-value-premises pairing.

    \begin{itemize}
    \item
      Agent \emph{has} \zS{} for \(\pv{\phi}{v}\) after concluding \(\pv{\phi}{v}\) from \(\Phi\).
    \end{itemize}

    \emph{if and only if}

    \begin{itemize}
    \item
      Either:
      \begin{enumerate}[label=(\alph*), ref=\alph*]
      \item
        There is no proposition-value-premises pairing \(\pvp{\psi}{v'}{\Psi}\) for which the relevant conditions are met.
      \item
        There is some proposition-value-premises pairing \(\pvp{\psi}{v'}{\Psi}\) but, from the agent's perspective, the agent would not fail to conclude \(\pv{\psi}{v'}\) from \(\Psi\).
      \end{enumerate}
    \end{itemize}
  \end{proposition}

  Second, when an agent does not have \zS{}.

  \begin{proposition}[When a agent does not have \zS{}]
    Agent and proposition-value-premises pairing.
    \begin{itemize}
    \item
      Agent \emph{does not} have \zS{} for \(\pv{\phi}{v}\) after concluding \(\pv{\phi}{v}\) from \(\Phi\).
    \end{itemize}

    \emph{if and only if}

    \begin{itemize}
    \item
      There is some proposition-value-premises pairing \(\pvp{\psi}{v'}{\Psi}\) such that, from the agent's perspective, the agent may fail to conclude \(\pv{\psi}{v'}\) from \(\Psi\).
    \end{itemize}
  \end{proposition}
\end{note}

\begin{note}
  Note, concluding is neither safe nor sensitive (assumption), so neither is \zS{}.
\end{note}

\section{Notes}
\label{cha:zS:sec:notes}

\paragraph*{When}

\begin{note}
  \emph{When} concluding \(\pv{\phi}{v}\) from \(\Phi\) in order to keep things simple.
  A variant of the question may be asked if the agent has (already) concluded \(\pv{\phi}{v}\) from \(\Phi\).
  Here, rather than asking whether the agent would not conclude \(\pv{\phi}{v}\) from \(\Phi\), we may ask whether the agent would revise their conclusion of \(\pv{\phi}{v}\) from \(\Phi\).
\end{note}

\paragraph*{Whether the agent may conclude \(\phi\) has value \(v\), regardless of \(\Phi\)}

\begin{note}
  Not about the proposition-value pair.
  Rather, it is about the concluding.
  At interest is not whether \(\phi\) has value \(v\), but whether it makes sense to conclude \(\pv{\phi}{v}\) from \(\Phi\).
  Of course, if the agent has no information about whether \(\phi\) has value \(v\), then this is also part of the picture, but that is a consequence of the base concern.
\end{note}

\paragraph*{Introduced by \(\pv{\phi}{v}\) from \(\Phi\)}

\begin{note}[Proposition-value-premise pairing introduced by \(\pv{\phi}{v}\) from \(\Phi\)]
  This restriction may seem arbitrary, and to some degree I think it is.
  Ideally, an agent concluding \(\pv{\phi}{v}\) is an instance of \csN{} just in case the agent would not have reasoned to a different conclusion if they were to reason first about any other proposition-value pair.
  However, the advantage of focusing on some proposition-value pair `required' by \(\phi\) having value \(v\) is a significant constraint on the range of proposition-value pairs an agent needs to consider in order to \csN{}.

  In general, it may not be clear which proposition-value pairs may lead an agent to fail to conclude \(\phi\) has value \(v\), but so long the proposition-value pair of interest is given by \(\phi\) having value \(v\), an exhaustive search over all other proposition-value pairs may be avoided.

  Indeed, we will say that an agent has \emph{\support{}} for \(\phi\) having value \(v\) just in case they would not have reasoned otherwise, and reserve \emph{\claiming{}} \support{} for the weaker notion.
\end{note}


\paragraph*{Inductive, abductive, etc.\ reasoning}

\begin{note}
  Narrow, but not too narrow.
\end{note}

\begin{note}
  This doesn't rule out inductive or abductive reasoning.
  Consider standard induction.
  Here, there may be novel information, but this is not available from the agent's present epistemic state, and \qzS{} only concerns the agent's present epistemic state.
  Perhaps the possibility alone would prevent conclusion.
  However, it seems most conclude in recognition of such possibility.
  Instead, what one would need is considerations against uniformity principle.

  Same for any bridge between probabilistic and full.
  Toss a coin \(n\) times, conclude it is fair.
  Possible to toss \(m\) more times, not fair.
  However, \(n\) is sufficient, then no problem.
  It is true that there is something more you could do, but this would require acquiring new information.
\end{note}



\paragraph*{Fragility}

\begin{note}
  Kind of fragility.
  If concludes \(\pv{\phi}{v}\) from \(\Phi\), and does not have \zS{}, then from agent's perspective, possibility of revision.

  Indeed, may break down into two components.

  First, possibility of different conclusion.
  Agent's epistemic state is potentially unstable.

  Second, isolation of potential instability to \(\pv{\phi}{v}\) from \(\Phi\).
\end{note}


\paragraph*{Normative?}

\begin{note}[Just a property]
  There's no kind of normative evaluation here.
  We do not hold any conclusion for which this fails is bad.
  Nor do we hold that any conclusion for which this holds is good.
  Indeed, \zS{} is narrow, far too narrow for a general evaluation.

  Indeed, whether or not \zS{} doesn't tell us anything about the relationship between \(\pv{\phi}{v}\) and \(\Phi\) in general, as relative to an agent's epistemic state.
  May be that there is some \(\pvp{\psi}{v'}{\Psi}\), but only due to some quirk of the agent.
\end{note}

\paragraph*{The agent concluding \(\pv{\psi}{v'}\) from \(\Psi\)}

\begin{note}
  From the perspective of the agent.
  It doesn't matter whether the agent really has the option.
  Indeed, this perspective is important for fragility.
\end{note}

\section{Scraps}
\label{sec:scraps}

\begin{note}
  \emph{However}, caution.
  For, as we have seen with testimony, it may be the case that status of a premises blocks a \requ{}.
  And, the argument given relies on the existence of a \requ{}.
  So, it may be the case that past reasoning blocks a \requ{}.
  Still, here, only need to deny this.
  Not saying that in every case agent's present reasoning is given priority.
  (Indeed, consider cases of being somewhat impaired, e.g., via exhaustion.
  Indeed, exhaustion is interesting.
  Basic consistency checks.
  Should be the case that conclude A, but just concluded \emph{not}-A, or something like this\dots)
  Rather, denying that past continues to secure in all instances.
  So, just need the potential to revise perspective on any previous conclusion.
\end{note}

\begin{note}
  An interesting observation here is that in certain this all arises, to a certain extent, because of general abilities.
  General ability spans multiple different proposition-value-premises pairings.
  Hence, all of these function as \requ{1}, so long as the agent has the option.

  General ability spans multiple different proposition-value-premises pairings.
  Hence, all of these function as \requ{1}, so long as the agent has the option.

  \begin{itemize}
  \item
    General and specific abilities.
  \item
    Answers to why, then.
    Note, here, that opportunity is interesting.
    The whole conjunction of all instance of the general ability is plausibly not a \requ{}.
    However, all that's needed is the \emph{individual} instances, and for these to raise a problem.
  \item
    The point is, \requ{1} for any general ability, and these are also \requ{1} for main pairing.
    (%
    Note --- or perhaps emphasise --- here, that the problem is \emph{not} recursive.
    Instead, the problem is about the spread.%
    )
  \item
    Here, then, ability is both the problem and the answer.
    What's interesting is the way in which ability functions.
    It's not merely \emph{that} the agent has the ability.
    Instead, it \emph{is} the ability.
  \end{itemize}
\end{note}

\begin{note}
  So, the way in which past reasoning relates is by ensuring that the agent would reach the same conclusion.
  About the agent's reasoning.
  \emph{How} rather than \emph{that}.

  Look, what we are getting is that the agent would conclude.
  If something were to happen, then some action would be performed.
  There's no distinction between the answer and performing the act, roughly.
  Or, better put, the answer \emph{is about present reasoning}.
  Answer states that in present reasoning, would not fail.


  It is about the agent's present epistemic state, and in particular what the agent's present epistemic state is capable of.

  In other words, ability.
  What answers is ability, in the sense that ability iff would.

  This is very important to the understanding of \fc{}.

  And, I kind of want to have ability as a gloss, while focusing on \fc{} to avoid going into ability in too much detail.

  So, positive answer, then it's the pairing \emph{being} a \fc{}.
  (I should always use this instance of the copula.)
\end{note}

%%% Local Variables:
%%% mode: latex
%%% TeX-master: "master"
%%% End:


\chapter{Positive answers to \qzS{}}
\label{cha:positive-answers}

\begin{note}
  Focus of this chapter is two key propositions which expand on positive answers to \qzS{}.

  First, potential event.
  Second, how the potential event functions.

  Combined, potential event answers, but is not a premise of the agent's reasoning.

  In following chapter, expand on this with \fc{1}, and relations of support.
\end{note}

\begin{note}
  Details matters.

  Key point of interest is propositions.
  Arguments matter.

  Present first proposition.
  Then second.
  Argument for second.
  Expand on second.

  Argument for the first in detail.
  Here, make use of ideas from the argument for the second.
\end{note}

\section{Outline}
\label{sec:outline}

\begin{note}
  So, the way in which past reasoning relates is by ensuring that the agent would reach the same conclusion.
  About the agent's reasoning.
  \emph{How} rather than \emph{that}.

  Look, what we are getting is that the agent would conclude.
  If something were to happen, then some action would be performed.
  There's no distinction between the answer and performing the act, roughly.
  Or, better put, the answer \emph{is about present reasoning}.
  Answer states that in present reasoning, would not fail.

  In this respect, \fc{}.

  Perhaps obvious, this is what the question asks.
  But, very important.
  Characterisation of the answer in terms of something forward looking.
  \fc{}.
  It is about the agent's present epistemic state, and in particular what the agent's present epistemic state is capable of.

  In other words, ability.
  What answers is ability, in the sense that ability iff would.

  This is very important to the understanding of \fc{}.

  And, I kind of want to have ability as a gloss, while focusing on \fc{} to avoid going into ability in too much detail.

  So, positive answer, then it's the pairing \emph{being} a \fc{}.
  (I should always use this instance of the copula.)
\end{note}

\begin{note}
  An interesting observation here is that in certain this all arises, to a certain extent, because of general abilities.
  General ability spans multiple different proposition-value-premises pairings.
  Hence, all of these function as \requ{1}, so long as the agent has the option.
\end{note}

\section{Outline}

\begin{note}
  \begin{itemize}
  \item
    The point is, \requ{1} for any general ability, and these are also \requ{1} for main pairing.
    (%
    Note --- or perhaps emphasise --- here, that the problem is \emph{not} recursive.
    Instead, the problem is about the spread.%
    )
  \item
    Here, then, ability is both the problem and the answer.
    What's interesting is the way in which ability functions.
    It's not merely \emph{that} the agent has the ability.
    Instead, it \emph{is} the ability.
  \end{itemize}
\end{note}

\begin{note}
  Somewhere at the end, or perhaps on a speculative chapter:
  Deduction theorem for reasoning.
  And, support, so why not conclude from without witnessing the reasoning.
  This would just be witnessing a foregone-conclusion.
\end{note}

\section{Fragments}

\begin{note}
  A \fc{} is not something which functions as a premise.
  This is a category mistake.

  So, answer to \qzS{} can't be a premise, because, in principle, for any premise, if we have a \requ{}, then any premise which states that the \requ{} is satisfied is subject to the \requ{}.

  I mean, do I get an infinite regress here?
  Even if I do, I don't think it's important.
  The point is, even granting that the agent is correct, hum.
  The \requ{} remains, but that is true in both cases.
  The task isn't getting rid of the \requ{}.
  Rather, the task is to show that the agent would conclude.
  But, now, fails to answer, because think!

  \fc{} answers by pointing to the reasoning.
  Alternative answers by not doing so.
  As you have not pointed to the reasoning, it remains the case that whatever this is, the \requ{} applies.

  This is probably a better way of doing things.
  I have a clearer understanding of pointing to the reasoning.
  And, with respect to the reasoning, it's clear.
  There's no question about whether the agent would conclude, that is what the agent is pointing to.
  The only question is whether there really is such an event.

  By contrast, if we're not pointing to the reasoning, then\dots
  Whatever it is the agent is pointing to, the agent has the option of appealing to this regardless of whether there is a witnessing event.
  Hence, if there is no witnessing event, then useless???

  \emph{The same \requ{} applies}
  Because, failure to conclude, then this thing is useless.

  This is quite subtle.
  The point is, potential witnessing event.
  Without this, without this doing the work, whatever one thinks of, the same question still applies.
  (No recursion!)

  If one does appeal to the potential witnessing event, then one is pointing to the very thing that matters.

  Now, failure to conclude, then something has gone wrong.
  Yes.
  The key observation is that in the failure case, the alternative thing, whatever this happens to be, persists, or at least may persist.
  It's independent of there being a potential witnessing event.

  This is what shows it doesn't work.

  So, looking.
  If mistake, then bad things all around.

  If not a mistake, then still a problem.
  For, independence of potential witnessing event.
  Therefore, failure would prevent from doing work.

  Thing is, right about potential, done.
  Right about alternative, then still a question regarding the potential.
\end{note}

\section{Potential events}
\label{sec:positive-answers-qzs}

\begin{note}

  When concluding \(\pv{\phi}{v}\) from \(\Phi\), an instance of \qzS{} has a positive answer only if, for any \requ{} \(\pvp{\psi}{v'}{\Psi}\) of concluding \(\pv{\phi}{v}\) from \(\Phi\), from the agent's perspective, the agent would conclude \(\pv{\psi}{v'}\) from \(\Psi\).

  Expanding, the following \hyperref[prop:PWEs]{proposition}:

  \begin{proposition}[Potential events]
    \label{prop:PWEs}
    For an agent \vAgent{}, when concluding \(\pv{\phi}{v}\) from \(\Phi\):

    \begin{itemize}
    \item[]
      For any thing \(\qzSaV{}\):
      \begin{enumerate}[label=\alph*., ref=(\alph*)]
      \item
        \label{prop:PWEs:a}
        \(\qzSaV{}\) is a positive answer to \qzS{}.
      \end{enumerate}
      \begin{itemize}
      \item[\emph{Only if}]
        For any proposition-value-premises pairing \(\pvp{\psi}{v'}{\Psi}\):
        \begin{itemize}
        \item[\emph{If}]
          \begin{enumerate}[label=\alph*., ref=(\alph*), resume]
          \item
            \label{prop:PWEs:b}
            \(\pvp{\psi}{v'}{\Psi}\) is a \requ{} of \vAgent{} concluding \(\pv{\phi}{v}\) from \(\Phi\).
          \end{enumerate}
        \item[\emph{then}]
          \begin{enumerate}[label=\alph*., ref=(\alph*), resume]
          \item
            \label{prop:PWEs:c}
            \(\qzSaV{}\) involves, in part, there being a potential event in which \vAgent{} concludes \(\pv{\psi}{v'}\) from \(\Psi\), from \vAgent{}' perspective.
          \end{enumerate}
        \end{itemize}
      \end{itemize}
    \end{itemize}
  \end{proposition}

  {
    \color{red}
    \(\qzSaV{}\) involves, in part, \vAgent{}' judgement that there is a potential event in which \vAgent{} concludes \(\pv{\psi}{v'}\) from \(\Psi\).
  }

  \autoref{prop:PWEs}, takes an arbitrary positive answers to \qzS{}, \(\qzSaV{}\), and expresses a \emph{necessary} condition on \(\qzSaV{}\).%
  \footnote{
    Otherwise expressed, \autoref{prop:PWEs} has the following form:

    \begin{quote}
      For an agent \vAgent{}, when concluding \(\pv{\phi}{v}\) from \(\Phi\):
      \begin{quote}
        For any thing \(\qzSaV{}\), (\hyperref[prop:PWEs:a]{a} \emph{only if} (for any \(\pvp{\psi}{v'}{\Psi}\), \emph{if} \hyperref[prop:PWEs:b]{b} \emph{then} \hyperref[prop:PWEs:c]{c}))
      \end{quote}
    \end{quote}
    Alternatively, \autoref{prop:PWEs} may be reformulated in terms of a single conditional by shifting the internal quantifier to scope over the both conditionals, and appealing to \emph{import-export}:
    \begin{quote}
      \begin{quote}
        For any thing \(\qzSaV{}\), and for any \(\pvp{\psi}{v'}{\Psi}\), ((\hyperref[prop:PWEs:a]{a} and \hyperref[prop:PWEs:b]{b}) \emph{only if} \hyperref[prop:PWEs:c]{c})
      \end{quote}
    \end{quote}
  }

  Hence, \autoref{prop:PWEs} states that in order for some thing \(\qzSaV{}\) to be a positive answer to an instance of \qzS{}, \(\qzSaV{}\) involves, in part, a potential event in which \vAgent{} concludes \(\pv{\psi}{v'}\) from \(\Psi\).
\end{note}

\begin{note}
  Two minor points of clarification, and then why \autoref{prop:PWEs} is difficult.
  Following, argument for \autoref{prop:PWEs}.

  Two minor points of clarification are:
  Necessary condition.
  Potential event.
\end{note}

\begin{note}
  First, the form of \autoref{prop:PWEs} as a necessary condition has syntactic and dialectical motivation.

  The syntactic motivation is straightforward:
  \begin{shiftpar}
    \ref{prop:PWEs:b} and \ref{prop:PWEs:c} are within the scope of quantification over all proposition-value-premises pairings, with \ref{prop:PWEs:b} serving to restrict instances of \ref{prop:PWEs:c} to pairings which are \requ{}.
    And, we have observed that, for an agent, there may be more than one \requ{} of concluding \(\pv{\phi}{v}\) from \(\Phi\).
    Therefore, it is not possible for the potential event of \ref{prop:PWEs:c} to, in general, be identified with \(\qzSaV{}\).%
    \footnote{
      For example, suppose two \requ{1}.
      \(\pvp{\psi}{v'}{\Psi}\) and \(\pvp{\theta}{v''}{\Theta}\).
      A potential event in which \vAgent{} concludes \(\pv{\psi}{v'}\) from \(\Psi\) (trivially) involves a potential event in which \vAgent{} concludes \(\pv{\psi}{v'}\) from \(\Psi\).
      However, such a potential event need not involve the agent concluding \(\pv{\theta}{v''}\) from \(\Theta\).
    }
  \end{shiftpar}

  The dialectical motivation is likewise straightforward:

  \begin{shiftpar}
    We will have no interest in whether or not the collection of potential events for every \requ{} is sufficient for \(\qzSaV{}\) is a positive answer to \qzS{}.
    I suspect such a collection would be sufficient, but nothing will depend on this.
  \end{shiftpar}
\end{note}

\begin{note}
  Second, what is meant by a `potential event'.
  Potential is subjunctive, an instance of a possible event.
  However, interested in answers to \qzS{}.
  Constraints on the event.
  `Potential' is a restriction of possible which satisfies constraints.
\end{note}

\subsection{A simple argument}
\label{cha:zSpA:sec:simple-argument}

\begin{note}
  The simple argument for~\autoref{prop:PWEs} highlights that \qzS{} asks whether a conditional holds, and observes it is not possible for the conditional to hold without~\autoref{prop:PWEs} holding.

  Recall \qzS{}:
  \begin{quote}
    \questionZS*
  \end{quote}
\end{note}

\begin{note}[Simple argument for~\autoref{prop:PWEs}]
  The simple argument for~\autoref{prop:PWEs} is direct.
  Take some arbitrary thing \(\qzSaV{}\), some \(\pvp{\psi}{v'}{\Psi}\), assume \requ{}, potential event.

  In full:


  Consider an agent when concluding \(\pv{\phi}{v}\) from \(\Phi\).
  Positive answer, and let \(\qzSaV{}\) be positive answer.

  Further, assume \(\pvp{\psi}{v'}{\Psi}\) is a \requ{}.
  { \color{red} Seen, a and b just express \requ{}\dots}

  From \qzS{} \ref{question:zs:may-fail}, it follows that from the agent's perspective, the agent would conclude \(\pv{\psi}{v'}\) from \(\Psi\), if \vAgent{} were to attempt to conclude \(\pv{\psi}{v'}\) from \(\Psi\).

  Agent has the option, from \ref{question:zs:option}.
  So, there is a potential event.
  And, given \ref{question:zs:may-fail}, concludes.
\end{note}

\begin{note}[Alternatively, by contradiction]
  No potential event.
  Then, focusing only on \(X\), we have the possibility of \(X\) and there not being a potential witnessing event.
  So, we then have \(X\) and failure to conclude.
  But then \ref{question:zs:may-fail} still.
\end{note}

\begin{note}
  Argument captures the core.
  However, details matter, and expand on this later.
  Get to the important proposition.
\end{note}

\paragraph{The difficulty}

\begin{note}
  For the moment, simple argument.
  However, we will return to this argument in~\autoref{cha:zSpA:sec:return-simple-arg}.

  Looking ahead, subtle flaw.
  From the agent's perspective, and aware of this.
  So, don't get potential event.
  Rather, perspective that there is a potential event.

  Clearest formulation:
  Agent perceives \dots

  And, hold from perspective potential event, while no opinion on whether there really is a potential event.
\end{note}

\begin{note}[Why this matters]
  Why does this matter?

  Broad scope.
  Relation of support.
  Distinction between relation of support, and agent's perspective that there is a relation of support.
\end{note}

\section{The role of the potential event}
\label{sec:no-premise}

\begin{note}
  Stated \ref{prop:PWEs:c}.

  Focus on event.

  This is, delicate.

  From agent's perspective, it is true that there is a potential event in which \dots

  So, functions as a premise.
  In this respect, no problem.
  Agent will witness reasoning.
\end{note}

\begin{note}
  In general?
  No, because here there's the assumption of \(\pvp{\psi}{v'}{\Psi}\) being a \requ{}.

  \begin{proposition}[The role of potential events]
    \label{prop:qzS-ans-event}
    For an agent \vAgent{}, when concluding \(\pv{\phi}{v}\) from \(\Phi\) such that \qzS{} has positive answer:
    \begin{enumerate}
    \item[\emph{If}]
      \begin{enumerate}[label=\alph*., ref=(\alph*)]
      \item \(\pvp{\psi}{v'}{\Psi}\) is a \requ{} of concluding \(\pv{\phi}{v}\) from \(\Phi\).
      \end{enumerate}
    \item[\emph{then}]
      \begin{enumerate}[label=\alph*., ref=(\alph*), resume]
      \item
        Potential event conclude \(\pv{\psi}{v'}\) from \(\Psi\) is not a premise or intermediary step of \vAgent{}' reasoning.
      \end{enumerate}
    \end{enumerate}
  \end{proposition}

  \autoref{prop:qzS-ans-event} builds on~\autoref{prop:PWEs}.%
  \footnote{
    Consider some agent, and assume \qzS{} has a positive answer when the agent is concluding \(\pv{\phi}{v}\) from \(\Phi\).
    Further, assume \(\pvp{\psi}{v'}{\Psi}\) is a \requ{} of concluding \(\pv{\phi}{v}\) from \(\Phi\).
    Then, from~\autoref{prop:PWEs} it follows that, from the agent's point of view there is a potential event in which the agent concludes \(\pv{\psi}{v'}\) from \(\Psi\).
    And,~\autoref{prop:qzS-ans-event} adds that the potential event, from the agent's perspective, is neither a premise nor intermediary step of the agent's reasoning.
  }
\end{note}

\begin{note}
  Proposition is negative.
  Doesn't assign a clear status to the potential event.

  No clear account of the role.
  Reasoning, premises, rules, conclusion.
  But, rule goes from proposition-value pairs to proposition-value pair.

  Still, like a rule.
\end{note}

\begin{note}
  Disposition.

  From agent's perspective, answer, disposition.

  Same problem.
  That disposed also as premise.
  But, clear reading where this isn't a premise.

  Well, maybe in terms of the car breaking down.
  Well, here, non-premise conclusion structure.
  Rather, a statement of how things are.
  Explains in the same way.

  So, with premise an conclusion, it being true explains why it is true.
  With other form of explanation, the event of X explains the event of X.
  Causal explanation.
  \citeauthor{Scriven:1962vq} vs Hempel.
\end{note}


\subsection{Understanding}
\label{sec:understanding}

\paragraph{Making this clear with ability (maybe)}

\begin{note}
  Ability to conclude.
  On the one hand, static perspective on ability.
  On the other hand, dynamic perspective.
\end{note}

\subsection{Argument}
\label{sec:argument-1}

\begin{note}
  Argument for~\autoref{prop:qzS-ans-event}.
\end{note}

\begin{note}[A \deadEnd{}]
  Piece of terminology for the argument.

  \begin{definition}[A \deadEnd{0}]
    \label{def:dead-end}
    For an agent \vAgent{}:

    \begin{itemize}
      \item
        \vAgent{}' epistemic state is a \emph{\deadEnd{0}} if:
        \begin{itemize}
        \item
          No conclusion without revision, from \vAgent{}' perspective.
        \end{itemize}
      \end{itemize}
      \vspace{-\baselineskip}
    \end{definition}

    Typical instance of dead end is conflicting proposition-value pairings.

    Qualifier, from \vAgent{}' perspective.
    Some difficulty.
    \deadEnd{} prevents agent from concluding.

    In this respect, \deadEnd{} does not need to be genuine.
    For example, further reasoning, so not conflicting.
    Understand as distinct epistemic state, as no longer block.

    Conversely, don't need proposition-value pairs to be in genuine conflict.
\end{note}

\begin{note}
  \begin{argument}
    Well, it's an event!
    Wow, this doesn't work, because I don't have the option of making a distinction.
    Yikes.
  \end{argument}
  Still, direct argument.
  \begin{argument}
    {
      \color{green}
      So, another way to do this:
      Construct so that the answer has not yet been given.
      Now, we have \(\Phi\).
      Problem is, adding anything to \(\Phi\), there question still remains.
      For, failure would dead-end anything added.

      In \emph{this} presentation, parallel to \citeauthor{Carroll:1895uj}.
    }
  \end{argument}

  Important, this is still from the agent's perspective.
  So, there's no guarantee that there really is a potential witnessing event.%
  \footnote{
    This works even with reduction to knowledge that, as there's no guarantee that the agent really has the knowledge (that).
  }
  What we get is the way in which \qzS{} receives a positive answer from the agent's perspective, and that is, in part, in terms of potential witnessing event, or something that depends on such an event.
\end{note}

\paragraph{Premises and past conclusions}

\begin{note}[Premises]
  So, as we have seen with testimony, status of a premises blocks a \requ{}.

  Whether the same may hold for this problem.

  It's the case that, part of agent's present epistemic state that they would conclude.

  Problem is, if attempt and fail, then this premise does nothing.
  Their present epistemic state develops into a dead-end.
\end{note}

\begin{note}[Note!]
  This doesn't hold in general, for all premises.

  In particular, premise is past conclusion.

  Consider cases of being somewhat impaired, e.g., via exhaustion.
  Indeed, exhaustion is interesting.
  Basic consistency checks.
  Should be the case that conclude A, but just concluded \emph{not}-A, or something like this\dots

  Denying that past continues to secure in all instances.
  So, just need the potential to revise perspective on previous conclusion.
\end{note}

\begin{note}[Past conclusions and positive answers]
  \begin{itemize}
  \item
    \emph{If} positive answer due to some past conclusion \emph{then} possible for the agent to conclude.
  \end{itemize}
  This conditional is immediate, because \qzS{} is about whether the agent would conclude, given that they have the option.
  \begin{itemize}
  \item
    \emph{If} possible to conclude, \emph{then} fact is insufficient.
  \end{itemize}
  This conditional is also immediate, because if the agent failed to conclude, then the fact that they had concluded wouldn't go anywhere.
\end{note}


\subsection{Literature}
\label{sec:literature}

\subsubsection{\citetitle{Carroll:1895uj}}
\label{sec:carroll}

\begin{note}
  \color{red}

  ~\cite{Besson:2018wz} in here somewhere.
\end{note}

\begin{note}
  \color{red}
  Point here is role of rule of inference is key.
  And,~\autoref{prop:PWEs} is observing this.
\end{note}

\begin{note}
  Similar to \citeauthor{Carroll:1895uj}.
  \begin{quote}
    Logic would take you by the throat, and \emph{force} you to do it!%
    \mbox{ }\hfill\mbox{(\citeyear[280]{Carroll:1895uj})}
  \end{quote}
  Looking at something static.
  Achilles fails to convey this to the Tortoise, arguably through some fault of Achilles' own.

  In parallel, we could stack up additional passives in the same way, but there's little interest in doing so.
  The point is the base \requ{} is not satisfied.
\end{note}

\begin{note}
  So, with \citeauthor{Carroll:1895uj}, we get a rule of inference, great.

  \citeauthor{Wieland:2013vf} characterises the general understanding of \textcite{Carroll:1895uj} in terms of two lessons:
  \begin{quote}
    [T]he negative lesson is that if you add ever more premises to an argument \dots, then you will never demonstrate that its conclusion follows logically.%
    \mbox{ }\hfill\mbox{(\citeyear[984]{Wieland:2013vf})}
  \end{quote}

  Parallel, static answers, still option for concluding otherwise.

  \begin{quote}
    [T]he positive lesson is that rules of inference, rather than premises of the form `if premises such and such are true, then the conclusion is true', will do the job.%
    \mbox{ }\hfill\mbox{(\citeyear[984]{Wieland:2013vf})}
  \end{quote}

  Parallel, the dynamic status of a rule.
\end{note}

\begin{note}
  Similar, but a little different.
\end{note}

\begin{note}
  No regress.

  Following \citeauthor{Wieland:2013vf}:

  \begin{quote}
    \begin{itemize}[noitemsep]
    \item[IR]
      For any item x of a certain type, S \(\varphi\)-s x only if
      \begin{enumerate}[label=(\roman*),noitemsep]
      \item
        there is a new item y of that same type, and
      \item
        S \(\varphi\)-s y.%
        \mbox{ }\hfill\mbox{(\citeyear[996]{Wieland:2013vf})}
      \end{enumerate}
    \end{itemize}
  \end{quote}

  Now, concluding, versus would conclude.
  However, focus is before concluding.
  So, would conclude and would conclude.

  Difficulty is, it's not at all clear this is the case.
  Need to be sure that there is a \requ{} for any \requ{}.
  Yet, from agent's perspective.
\end{note}

\begin{note}
  Interesting thing here is that it's not `just' the rule.

  \emph{And}, important difference is that the agent isn't moving from premises to conclusion.
  Following the standard interpretation, \citeauthor{Carroll:1895uj} gets us that there's a rule in play when agent concludes.
  (Or, more strictly, modus ponens\dots)
  But, this is very different from something similar being active when drawing some other conclusion.

  So, there is a link to \citeauthor{Carroll:1895uj}, but it is somewhat indirect.
  Still, this should soften the conclusion.

  In short, with \citeauthor{Carroll:1895uj} it's the rule.
  Here, it's the ability to employ the rule.
\end{note}


\subsubsection{Dispositions}
\label{sec:dispositions}

\begin{note}[Parallel between dispositions and ability]
  Consider \citeauthor{Choi:2021wg}'s characterisation of the Simple Conditional Analysis of dispositions:
  \begin{quote}
    An object is disposed to \emph{M} when \emph{C} iff it would \emph{M} if it were the case that \emph{C}.\nolinebreak
    \mbox{}\hfill\mbox{(\citeyear{Choi:2021wg})}
  \end{quote}
  For example, an object is disposed to dissolve when it is placed in water iff the object would dissolve if it were the case that it is placed in water.

  The Simple Conditional Analysis may be challenged, but for our purposes it is adequate.
  We are interested in the broad form of the truth condition, and various more refined analyses share the same broad form.
  Note, in particular, that it being the case that \emph{C} and \emph{M} happening describes an event.
  Given appropriate conditions; salt dissolves, glass breaks, and I mumble when I am tired.
  The key idea is that the property of being disposed to \emph{M} when \emph{C} is analysed in terms of the (possible) event of \emph{M} happening when \emph{C}.

  The parallel to ability is established by noting that ability may also be analysed in terms of a (possible) event, as we have seen.
  In particular, by incorporating volition in the analysans of the Simple Conditional Analysis.
  To illustrate, \citeauthor{Mandelkern:2017aa} trace the Conditional Analysis of ability  to \textcite{Hume:1748tp} and \textcite{Moore:1912te}, among others:
  \begin{quote}
    S can \(\phi\) iff S would \(\phi\) if S tried to \(\phi\)\nolinebreak
    \mbox{}\hfill\mbox{(\citeyear[Cf.][308]{Mandelkern:2017aa})}
  \end{quote}
  Compare to the Simple Conditional Analysis of dispositions:
  The object is some agent \emph{S}, \emph{C} is `S tried to \(\phi\)' and \emph{M} is `S \(\phi\)s' --- it is volition alone which distinguishes the analyses.
  For example, I have the ability to demonstrate that a rectangle with dimensions \(19\text{cm}\) by \(7\text{cm}\) has area \(133\text{cm}^{2}\) only if I would demonstrate that a rectangle with dimensions \(19\text{cm}\) by \(7\text{cm}\) has area \(133\text{cm}^{2}\) if it were the case that I tried that a rectangle with dimensions \(19\text{cm}\) by \(7\text{cm}\) has area \(133\text{cm}^{2}\).
\end{note}

\subsubsection{Doxastic justification}
\label{cha:fcs:sec:dox-just}

\begin{note}
  \citeauthor{Turri:2010aa}

  \begin{quote}
    Necessarily, for all S, \emph{p}, and \emph{t}, if \emph{p} is propositionally justified for S at \emph{t}, then \emph{p} is propositionally justified for S at \emph{t} because S currently possesses at least one means of coming to believe \emph{p} such that, were S to believe \emph{p} in one of those ways, S's belief would thereby be doxastically justified.%
    \mbox{ }\hfill\mbox{(\citeyear[316]{Turri:2010aa})}
  \end{quote}

  Key is that doxastic justification depends on what the agent does.

  \citeauthor{Turri:2010aa}'s focus is on how reasons are used.
  What the agent does.

  Seen with example.

  \begin{quote}
    \begin{enumerate}[label=(P\arabic*)]
      \setcounter{enumi}{4}
    \item
      The Spurs will win if they play the Pistons.
    \item
      The Spurs will play the Pistons.
    \end{enumerate}

    \mbox{}\hfill\(\vdots\)\hfill\mbox{}

    \begin{enumerate}[label=(P\arabic*), resume]
    \item
      Therefore, the Spurs will win.%
    \mbox{ }\hfill\mbox{(\citeyear[317]{Turri:2010aa})}
    \end{enumerate}
  \end{quote}

  Rather than \emph{modus ponens}, `\emph{modus profusus}'.
  Conclude \(r\) from \(p\) and \(q\).
  (\citeyear[317]{Turri:2010aa})

  \begin{quote}
    The way in which the subject performs, the manner in which she makes use of her reasons, fundamentally determines whether her belief is doxastically justified.
    Poor utilization of even the best reasons for believing \emph{p} will prevent you from justifiedly believing or knowing that \emph{p}.%
    \mbox{ }\hfill\mbox{(\citeyear[316]{Turri:2010aa})}
  \end{quote}

  Variant of ~\cite{Prior:1960wh}'s `tonk' connective.
  Though, difference is between connective and rule.
  \(p\) tonk \(q\) would not be propositionally justified.
\end{note}

\begin{note}
  \citeauthor{Turri:2010aa} is similar to \citeauthor{Goldman:1979ui}

  Begin with justification.

  \begin{quote}
    \begin{enumerate}[label=(\arabic*)]
      \setcounter{enumi}{10}
    \item
      Person \emph{S} is \emph{ex ante} justified in believing \emph{p} at \emph{t} if and only if there is a reliable belief-forming operation available to \emph{S} which is such that if \emph{S} applied that operation to this total cognitive state at \emph{t}, \emph{S} would believe \emph{p} at \emph{t}-plus-delta (for a suitably small delta) and that belief would be \emph{ex post} justified.
    \end{enumerate}
  \end{quote}

  Where, sufficient condition for belief would be \emph{ex post} justified:
  \begin{quote}
    \begin{enumerate}[label=(\arabic*)]
      \setcounter{enumi}{4}
    \item
      If S's believing \emph{p} at \emph{t} results from a reliable cognitive belief-forming process (or set of processes), then S's belief in \emph{p} at \emph{t} is justified.%
      \mbox{ }\hfill\mbox{(\citeyear[13]{Goldman:1979ui})}
    \end{enumerate}
  \end{quote}
  Roughly, at least.
  \citeauthor{Goldman:1979ui} refines this a fair bit, but this isn't important.

  Availability of a reliable belief-forming operation!

  Relation here is brittle.
  Account of justification, apply to concluding.
  Well, then all we get is that before concluding, would make sense to conclude only if available.
  Running something like the \citeauthor{Carroll:1895uj} regress, not some state.
  But, this only tells us about suitability to conclude.

  Still, key point is process.

  Another useful thing to highlight is the suitably small delta.
  With \requ{}, this is captured in terms of the option.
\end{note}

\begin{note}
  Significant difference is in the case of justification, we're not interested in the agent's perspective.
  Hence, these accounts are understood in terms of the agent having the ability, roughly.

  With \qzS{}, we're interested in the agent's perspective, and there is no guarantee that the agent really has the ability.
\end{note}

\subsubsection{Ryle}

\begin{note}
  Ideas regarding \citeauthor{Ryle:1946tu}'s distinction between knowing \emph{how} and knowing \emph{that} (Cf.~\citeyear{Ryle:1946tu}).

  Now, I confess my understanding of \citeauthor{Ryle:1946tu}'s distinction is limited --- I have not taken whatever opportunities I have had to read through \citeauthor{Ryle:1946tu}'s work.%
  \footnote{
    Though, I understand enough from passing commentary to note that the idea \emph{I} am perusing here does not, strictly, require that knowledge how and knowledge that are distinct kinds of knowledge.
    (See~\textcite{Pavese:2022up} for more!)
  }

  Following analogy from~\textcite{Ryle:2009us}:

  \begin{quote}
    Knowing `\emph{if p, then q}' is, \dots rather like being in possession of a railway ticket.
    It is having a licence or warrant to make a journey from London to Oxford.
    (Knowing a variable hypothetical or `law' is like having a season ticket.)
    As a person can have a ticket without actually travelling with it and without ever being in London or getting to Oxford, so a person can have an inference warrant without actually making any inferences and even without ever acquiring the premisses from which to make them.%
    \mbox{ }\hfill\mbox{(\citeyear[250]{Ryle:2009us})}
  \end{quote}

  Continuing~\citeauthor{Ryle:2009us}'s analogy, in the case of positive answers to \qzS{}:
  What matters is that the agent is currently in possession of the (season) ticket.

  Even if current possession of the (season) ticket is knowledge that, it is present knowledge.
  And, present without being applied.
\end{note}


\section{The argument for~\autoref{prop:PWEs}}
\label{cha:zSpA:sec:return-simple-arg}

\begin{note}
  Return to simple argument from \autoref{cha:zSpA:sec:simple-argument}.

  Simple argument gets the idea.

  In this section, consider and respond to possible objection.

  Roughly:

  Simple argument, get from agent's perspective.
  However, insufficient to show potential event.
  Only get that the agent has some attitude toward potential event.
  And, attitude may be consistent with indifference to whether there really is a potential event.

  Two sub-sections.

  First, motivate the distinction in this quick (counter-)argument.

  Second, argue that~\autoref{prop:PWEs} follows, granting the distinction.
\end{note}

\begin{note}[Why this is important]
  Why this is important.
  Given proposition, potential event.
  Following chapter.
  Relation of support.
  In other words, potential doesn't matter.
  However, qualify.
  So, even granting arguments to follow, not a relation of support proper.
  Hence, failure of overall goal.
\end{note}

\begin{note}
  Unsure on whether the distinction really makes sense.
  This will show.
  Sometimes convinced, other times, hard to see.
  However, possibility is sufficiently important.
  Hence, do best to motivate.
  And, offer response granting.
\end{note}

\subsection{Limitation}
\label{sec:limitation}
\nocite{Perry:1979vc}
\nocite{Perry:1986aa}

\begin{note}
  Agent's perspective.
  Distinguish.

  \begin{enumerate}
  \item
    \(\phi\) has value \(v\), from the agent's perspective.
  \item
    \(\phi\) is perceived to have value \(v\), from the agent's perspective.%
    \footnote{
      Or, stated a little more carefully, where \vAgent{} stands for the agent:
      \vAgent{} perceives \(\phi\) to have value \(v\), from \vAgent{}' perspective.
    }
  \end{enumerate}

  Second, qualifier.
  \emph{perceived}.
  Terminology is arbitrary.
  Introduction to this section, various attitudes.
  Here, perspective already.
  And, that's really what is at issue.

  Omit qualifier `from agent's perspective' and distinction is straightforward.

  Separate, whether \(\phi\) has value \(v\) and whether \(\phi\) has value \(v\) from agent's perspective.
  This is common.
  Whether agent from their perspective has this option.

  For example\dots

  If so, then answer to \qzS{} only in terms of perception.
  Doesn't follow that potential event, from agent's perspective.

  Issue is whether, perceptive without that perspective involving how things are.

  Perception, visual perception, and factive.

  In part, why chosen.

  Visual illusions.

  So, question is whether the same holds here.

  Not exactly the same.
  After inspecting closely, from perspective, illusion does not hold.

  So, seems the agent has a choice.
  Which perspective to adopt.
  And, distinction disappears with choice.

  Whether principle holds.
\end{note}

\begin{note}
  To move from this to what we want in general, need `\emph{\ptivity{0}}'%
  \footnote{
    The term is a play on factivity.

    Factivity, know \(\pv{\phi}{v}\) only if \(\phi\).
    \ptivity{2}: drop perspective.

    Unlike factivity, converse seems straightforwardly true.
  }

  \begin{principle}[\ptivity{2}]
    \label{def:perspectivity}
    For an agent \vAgent{}, proposition \(\phi\), and value \(v\):
    \begin{enumerate}[noitemsep]
    \item[\emph{If}]
      \begin{enumerate}[label=\alph*., ref=(\alph*)]
      \item
        \(\phi\) is perceived to have value \(v\) from \vAgent{}' perspective.
      \end{enumerate}
    \item[\emph{then}]
      \begin{enumerate}[label=\alph*., ref=(\alph*), resume]
      \item
        \(\phi\) has value \(v\) from \vAgent{}' perspective.
      \end{enumerate}
    \end{enumerate}
    \vspace{-\baselineskip}
  \end{principle}
\end{note}

\begin{note}
  Argument for \ptivity{}.

  Agent's perspective, so of course.
  If perspective, then deletion.%
  \footnote{
    \citeauthor{Collins:1997wn} (\citeyear{Collins:1997wn}) presents an argument along these lines.
    \begin{quote}
    The perspective of the agent, when rightly interpreted, is not a call for introspectible deteterminants of action.
    It is a reminder that it is objective circumstances \emph{as apprehended by the agent} that are relevant.
    The perspective is not the subject matter.
    An agent makes statements about the objective circumstances as he understands them.
    This qualification: `as he understands them' is not a shift to the mental realm.%
    \mbox{ }\hfill\mbox{(\citeyear[120]{Collins:1997wn})}
  \end{quote}

  \citeauthor{Dancy:2000aa} summarises:
  \begin{quote}
    No explanation that obliterated that endorsement would be the correct explanation of the action, since it would fail to give the agent’s perspective on things, and hence fail to capture the light in which the action was done.
    \citeyear[108]{Dancy:2000aa}
  \end{quote}

  And, \citeauthor{Dancy:2000aa} suggests links to ~\citeauthor{Moore:1993wk}'s paradox (though \citeauthor{Collins:1997wn} does not explicitly consider).

  However, delicate.
  \begin{quote}
    We could introduce psychological matters if we mean that they are the things that make his situation and his course of action intelligible to an agent.
    But if objective circumstances are what make his own action intelligible to the agent then we do not depart from the agent’s perspective in putting forward objective circumstances in the context of reason-giving.%
    \mbox{ }\hfill\mbox{(\citeyear[120]{Collins:1997wn})}
  \end{quote}
  In short, \citeauthor{Collins:1997wn}' argument rests on belief.
  \begin{quote}
    Wherever an agent correctly adduces a belief that an objective circumstance obtains in explaining his action, a de-psychologizing restatement that merely makes the objective claim must be ascribable to the agent.%
    \mbox{ }\hfill\mbox{(\citeyear[120]{Collins:1997wn})}
  \end{quote}
  Limited to belief, where belief does require.
  For first passage quoted to hold in full generality `As he understands' must involve belief.

  \citeauthor{Collins:1997wn} does not expand too much on what other psychological matters in second passage quote.
  However, our question is whether something weaker than belief.
  }

  Focused on agent recognises \(\phi\) may not have value \(v\).
  However, this doesn't allow the agent to avoid \(\phi\) having value \(v\).

  If this is correct, then \ptivity{}.
  And, the simple argument goes through.

  However, I don't think this is right.
  Perspective, weak.

  This seems in conflict.
  Focus on belief.
  Look at the abstract.
\end{note}

\begin{note}
  Distinct between representation and what is represented.

  So, clock.

  Look at the clock.
  Time.
  Late.
  Start rushing.

  Is it XX:XX, or perspective?

  Clocks can be wrong.
  Feels the clock may be wrong.
  Looks around apartment for their watch (they lost it).
  Hmpf.

  Doesn't matter.
  Only have the clock to go by.

  Strengthen, doubts about the clock.
  But, do choice.

  From perspective, time in 9:15a.
  But, this can't be quite right.
  Nothing beyond perspective.
  Operative, links to action of rushing around.
  But, perspective is not that the time is 9:15a because I don't do anything other than treat the time as being 9:15a.
\end{note}

\begin{note}
  Different to the horse race case from~\textcite{Hawthorne:2016wv}.

  Here, conflict, but still seems perspective.
\end{note}

\begin{note}[\citeauthor{Descartes:1996vp}]
  Observe, problem for \citeauthor{Descartes:1996vp}.

  Here, two perspectives.

  First, the agent has shifted their perspective.
  There no longer is an external world.

  Second, the agent has weakened their attitude.
  From their perspective, but only their perspective.
\end{note}

\begin{note}
  \begin{quote}
    A fork stabs the cube of meat and we FOLLOW it UP TO the face of Cypher.

    CYPHER

    You know, I know that this steak doesn't exist.
    I know when I put it in my mouth, the Matrix is telling my brain that it is juicy and delicious.
    After nine years, do you know what I've realized?

    He shoves it in, eyes rolling up, savoring the tender beef melting in his mouth.

    CYPHER
    Ignorance is bliss.
  \end{quote}

  Perspective, but just perspective.
\end{note}


\begin{note}[Other examples]
  Why sad?
  They said.
  I think they said.

  Why did the food taste salty.
  Answer, too much salt.
  What answers is, in part, excess salt.
  Well, might be something else.

  {
    \color{red}
    These are hard.
    They don't seem to go either way.
  }
\end{note}

\begin{note}
  Similar, though distinct.
  ~\cite{Donnellan:1966wt}.

  Attributive and referential.

  With attributive, nothing in mind.
  Similar, here the agent gives up what the time actually is.
  However, for~\citeauthor{Donnellan:1966wt}, the way things are still matter.
\end{note}


\begin{note}
  Variation on factivity.
\end{note}


\paragraph{Response in two parts}

\begin{note}
  Two things.

  Whether failure is compatible with \qzS{} as stated.

  Whether \qzS{} should be revised.
\end{note}

\begin{note}
  First, argument similar to second key proposition here.
\end{note}

\begin{note}
  Second, \qzS{} is fine.
\end{note}

\newpage

\begin{note}
  First is given, this is what we're interested in.

  Second, this comes from \citeauthor{Descartes:1996vp}.

  Third, reverse \citeauthor{Descartes:1996vp}.
  For \citeauthor{Descartes:1996vp}, how to go from perspective to how things actually are.
\end{note}

\begin{note}
  So, this just goes back to \citeauthor{Descartes:1996vp}.
  Look, recognise perspective, but question remains as to whether it really is the case.

  The presentation of this is difficult, as we're dealing with an agent.

  Already speaking about agent's perspective, as \requ{1}, etc.\ may hold from agent's perspective.
\end{note}

\begin{note}
  Second, recognise this is my perspective.

  \cite{Descartes:1996vp}

  \begin{quote}
    \dots
    But for all that I am a thing which is real and truly exists.
    But what kind of a thing?
    As I have just said --- a thinking thing.%
    \mbox{ }\hfill\mbox{(\citeyear[18]{Descartes:1996vp})}
  \end{quote}

  Also, \citeauthor{Lichtenberg:1991tf} against Descartes.%
  \footnote{
    \citeauthor{Zoller:1992ud}'s translation follows:
    \begin{quote}
      One should say, \emph{it thinks}, just as one says, \emph{it lightens}.
      It is already saying too much to say \emph{cogito}, as soon as one translates it as \emph{I think}.
      \mbox{ }\hfill\mbox{(\citeyear[418]{Zoller:1992ud})}
    \end{quote}
  }
  \begin{quote}
    \emph{Es denkt}, sollte man sagen, so wie man sagt: \emph{es blitzt}.
    Zu sagen \emph{cogito}, ist schon zu viel, so bald man es durch \emph{Ich denke} \"{U}bersetzt.
    \mbox{ }\hfill\mbox{(\citeyear[412]{Lichtenberg:1991tf}/K76)}
  \end{quote}
\end{note}

\begin{note}
  So, if above is correct, then simple argument is enthymematic.
  Without further statement, relies on \ptivity{0}.
\end{note}

\subsection{Less quick argument}
\label{sec:less-quick-argument}

\subsubsection{Given question}
\label{sec:given-question}

\begin{note}
  Given the question, weakening doesn't work, because anything weaker is a \requ{}.
\end{note}

\begin{note}
  Understanding of `why'.

  Why, from the agent's perspective.

  But, this doesn't entail anything.
  Doesn't follow there is a potential witnessing event.

  Only have something psychological.

  So, whether there is a relation of support is dispensable.

  So, contents is not part of the explanation why from our perspective.
  This is what \citeauthor{Hieronymi:2011aa} terms `Dancy's gap'.
  Worry here is whether we have a clear account of why.
  For, whether the content holds isn't clearly relevant.

  As I understand the gap, issue in terms of our answers to why.
  In some cases, the agent's perspective, and in other cases more than this.

  Or, another way of putting thing, in some cases the agent's perspective is only part.

  Different perspective, even the agent.

  Now, the problem.
  Agent recognises from their perspective.

  So, the idea is, only psychological facts matter.
  Agent recognises this.
  Then, this breaks down the role of the content.
  So, there's no easy movement from content to account of why from agent's perspective.
  For, there's no clear factive presence from the agent's point of view.

  But, this can't work.
  Still matters that the agent thinks.
  And, if non-factive, then no positive answer.

  Push this gap from our perspective, and raise an issue about whether there is anything beyond agent's perspective.
  But, this fails when internalised, at least for positive answers.

  Still, question is irrelevant.
  But, the issue here is that we're talking about mental states.
  It's concluding.

  There's no way to ask this question.
  For, then the question is independent of whether the agent would conclude.
  What the agent would do with other proposition value pairs.
  But, if all that matters is what's going on in the present, this fails.

  Only understand this in terms of our perspective, and then, because reasoning, there's no pressing issue.

  But, I don't think this is relevant.
  For, what we are interested in accounting for is concluding.
\end{note}

\subsubsection{Right question}
\label{sec:right-question}

\begin{note}
  Two points here.
  First, if think weaker question is only question of interest, then there is no distinction.
  Here, relation of support isn't what matters, but judgement that there is a relation of support.

  Second, if distinction, then clearly intelligible.

  Switch these, I think there is a clear distinction.
  However, if you insist, then it doesn't really matter.
\end{note}

\subsubsection{Problem of linking proposition to \qWhy{}}
\label{sec:probl-link-prop}

\begin{note}
  Here, causal deviance.
\end{note}

\newpage

\begin{note}
  Point is potential event in which concludes.

  \autoref{prop:PWEs} may seem straightforward.
  \qzS{} asks whether the agent would conclude.
  So, from agent's perspective, must be the case would conclude.
  And, would conclude only if there is a potential event in which the agent concludes.

  Parallel with more standard case of reason explanation.

  Davidson.
  Pro-attitude.
  Belief.
  So, from agent's perspective, contents of belief.
\end{note}

\begin{note}
  Key move, \(\phi\) has value \(v\) from agent's perspective answers why.
  To, \(\phi\) has value \(v\) from agent's perspective.

  First, Includes a statement about the agent's perspective.
  Second, states how things are from agent's perspective.

  Point is, from perspective, X from perspective.
  Then, X from perspective.

  Doesn't follow that from perspective, X regardless of perspective.

  Don't immediately get any further commitment to X beyond it being the case that X from the agent's perspective.

  Disspaernace of perspective when adopting perspective.

  So, for this arugment I end up using belief, but it's really independent of belief, as you can make the relevant role of agent's perspective as weak as you like.

  Point is, perspective can't limit.

  Problem is dropping the agent's perspective.
\end{note}

\begin{note}
  \citeauthor{Moore:1993wk}
  Here, look, there's something bad about I believe \emph{p} but not-\emph{p}.
  Nothing bad about I believe \emph{p} but I do not know \emph{p}.

  This is the distinction I'm after.
\end{note}

\begin{note}[Distinction]
  It is the case that:
  \begin{itemize}
  \item
    from agent's point of view \emph{p}, and
  \item
    \emph{p} from agent's point of view answers, and
  \item
    \emph{p} does not matter.
  \end{itemize}

  Sollopsistic, to a degree.

  Distinction here between belief and knowledge.
  Note, only that belief we have the option, not claiming this holds for all cases.

  Get the converse in other cases.
  \emph{p}, but don't take my word for it, check yourself.
  Or, that's only my point of view, check yourself.

  Highlights the possibility of mistake.

  However, whether this answers why.
  Point is, possibility of mistake.

  Cite explanation, bridge etc.
  However, there's a difference between belief, and this answering the question.

  Similar observation in \textcite[132ish]{Dancy:2000aa}.
  However, \citeauthor{Dancy:2000aa} is quite different, really.
  From the agent's point of view.
  Case of knowledge.
  From our perspective, fine, mental states, etc.
  However, from agent's point of view, different.

  Well, \citeauthor{Dancy:2000aa} ends up arguing that there's really no difference between factive and non-factive.
  To make this argument, first person is factive.
  But, this is what I'm denying.

  \citeauthor{Collins:1997wn}.
  \textcite[108+]{Dancy:2000aa}
  Moore phenomena.
  As I understand things, it is not the case that the agent has the option of dissenting from \emph{p}.
  However, it is the case that the agent has the option for dissenting from \emph{p} as an answer to \qWhy{}.

  From the agent's perspective, yes, but agent's perspective recognises this is what they consider to be the case.

  Difficult in general.
  No account of general case.
  Here, just answers to \qzS{}.
\end{note}

\begin{note}
  So, some parallel to factivity.

  Why factivity?
  Instead, extract content.

  Main task is getting parallel to factivity, then.
\end{note}

\begin{note}[Argument for~\autoref{prop:PWEs}]
  \begin{argument}
    {
      \color{blue}
      So, this leads to a possible objection.
      The only relevant questions to ask are in terms of proposition-value pairs, or something like this.
      I don't think this works, given all the examples, but it's an option.
    }
  \end{argument}
\end{note}

\subsection{Objections to the argument}
\label{sec:objections-argument}

\begin{note}
  Argument relies two things.
  First, apparent factivity.
  Second, content of \qzS{} relating to event.

  I don't see a way around this.
  However, two possible avenues.
  Deny either of these two things.
  Attempt plausible motivation for both, but show how the motivation fails to move.

  Following, a more interesting chance for failure.
  Does raise problems, but not for this argument.
\end{note}

\subsection{A shortcoming}
\label{sec:shortcoming}

\begin{note}
  \color{red}
  Need to shift.
  This isn't really about positive answers to \qzS{}.
  Rather, this is about how \qzS{} relates to why.
\end{note}

\begin{note}
  Argument relies on tying content to explanation.

  In this respect, there is room for an objection.
  Deviant causal chains.
  Point here is that there are cases where these come apart.

  This isn't only a problem for causal theories of reasoning.
  The point is, some instantiation, and so long as act may be caused by something else, then possibly caused by the instantiation.

  So, possible here.

  Well, hold on.
  What is need is the relevance of the content.
  For this objection to work, need to take a theoretical perspective.
  See, in Davidson's case, the idea is fusing these two things together.
  We answer two different questions with a common thing viewed in two ways.

  Still, I think the objection can be pressed!
  Only \emph{really} an explanation is no deviance.
  To the same extent that potential event matters, it matters to the agent that there is no deviance.

  {
    \color{red}
    Resolution is, if deviance, then no agency.
  }

  I think this makes sense, or at least makes enough sense.
  Answers to `why', on this understanding, are tentative.

  Or, rest on presupposition that agent performed the action.

  So, contingent on showing there is no causal deviance.

  This is different to error.
  With error, thing appealed to isn't the case, but appeal still did work.
  Here, it doesn't matter whether or not the case, no work is done.

  In contrast to more typical instances of the problem, don't need to rule out deviant causal chains.
  Instead, just need one instance to fail to hold.
  One instance of non-deviousness.

  Still a problem for a compatible account which avoids.
  For, here, there can't be any direct link from perspective to reason.

  For example, \citeauthor{Hieronymi:2011aa}

    \begin{quote}
      [W]e explain an event that is an action done for reasons by appealing to the fact that the agent took certain considerations to settle the question of whether to act in some way, therein intended so to act, and successfully executed that intention in action.
    [\emph{T}]\emph{his} complex fact, [\dots] is the reason that rationalizes the action---that explains the action by giving the agent’s reason for acting.%
    \mbox{ }\hfill\mbox{(\citeyear[431]{Hieronymi:2011aa})}
  \end{quote}

  So, here, considerations which settle question, and in so settling question.
  Link between settling the question and acting.

  Following \citeauthor{Hieronymi:2011aa}, no room for deviance.
  Too tight.

  In other words, so long as this fact holds, there is no distinction between settling the question and acting.
  Therefore, no deviance.

  Compatible, I think.
  Question is whether in resolving \qzS{} is sufficiently tied to resolving the question \citeauthor{Hieronymi:2011aa} identifies.
  And, plausibly is.
  This is what the motivation for \qzS{} did.

  Trouble is, for our purposes, need at least sufficient conditions for when this complex fact obtains.
  And, no account of this.

  \citeauthor{Hieronymi:2011aa} notes the gaps.

  Some tension.
  These considerations aren't premises.
\end{note}

\subsection{Issues}
\label{sec:issues}

\begin{note}
  Something wrong.
\end{note}

\subsubsection{Other questions}
\label{sec:other-questions}

\nocite{Smith:1988aa,Smith:1987tz}

\begin{note}
  Similar question:

  By assumption, there is no external world, no food, and no excess salt.
  But, from agent's perspective.

  What happened, what was the case.
\end{note}

\begin{note}
  Put an explicit proposition-value pair in the question.
  Why is it true that the food tasted salty.
  It is true that there was too much salt.

  Still, in part, circumstances.
  True.
  Interpreted.

  No account of true without the circumstances having a role.
\end{note}

\begin{note}
  Any other variation without interpreting, don't get a good answer to the question.
\end{note}

\paragraph{No getting carried away}

\begin{note}[Different values]
  Desire, then something else, likely.
\end{note}

\begin{note}[Mental states]
  Question about why believed, after finding belief is mistaken.
\end{note}

\subsection{Rationalisations}
\label{sec:rationalisations}


\subsubsection{Deviant causal chains}
\label{sec:devi-caus-chains}

\begin{note}
  So, the other option is to embrace deviant causal chains.
  Have the content, but this doesn't work in the way the agent thinks it does.

  Example from Davidson.

  The trouble here is that the content and resulting action match.
  So, things make sense from the agent's point of view.

  Deviant, but maybe not so deviant here.

  Systematic deviance, where content is separated from role of mental state.

  But, I see no motivation for this.

  Solution to causal chains doesn't get round this, because the result is a restricted account.
  So, there's no guaranteed trade-off here.
  Trouble is, it seems hard to see a case where this wouldn't be the case.
\end{note}

\subsubsection{Mistakes and vats}

\begin{note}
  Well, it's possible that the agent is wrong.
  This is fine, from the agent's perspective.
\end{note}


\subsubsection{Answers which are not proposition-value pairs, in part}
\label{sec:answers-which-are}

\begin{note}
  \begin{itemize}
  \item
    Somewhat simple.
  \item
    Is there a flaw?
  \item
    Well, point is there's something that isn't a proposition-value pair.
  \item
    Parallel?
  \item
    Knowledge how and knowledge that.
  \item
    Same argument applies to dispute, in a fairly straightforward way.
    Specifically, regarding knowledge that, regardless of knowledge how.
  \end{itemize}
\end{note}

\begin{note}
  Sketch of the argument.

  \begin{enumerate}
  \item
    Potential witnessing event in which agent concludes.
  \item
    \label{pwe-iff-kh}
    This is the case if and only if knows how to conclude (if not knowledge, then insufficient grasp on witnessing event).
  \item
    \label{kw-is-kt}
    Knowledge how is a species of knowledge that.
  \item
    Knowledge that \(\varphi\)
  \item
    Knowledge that \(\varphi\) does not involve event.
  \item
    Equivalent.
  \item
    So, at least possible to answer with something that does not involve event.
  \end{enumerate}

  So, replacement.

  Limitation is option to replace.

  Still, this is enough to highlight a flaw in \autoref{prop:PWEs}.

  In addition, given that agent is not literally answering the question, additional argument that this is how to understand.
\end{note}

\begin{note}
  \dots In outline, depends on how the details are filled in.

  \autoref{pwe-iff-kh} and \autoref{kw-is-kt} in particular.

  Grant \autoref{pwe-iff-kh}, explore \autoref{kw-is-kt}.
\end{note}

\begin{note}
  Not just strong intellectualism, but\dots

  Knowing that is not,~\cite{Stalnaker:2012tp} `justified true belief, painted over with a Gettier-proof coating of some kind.' (\citeyear[754]{Stalnaker:2012tp})

  Instead, ~\citeauthor{Stanley:2011ut} (and~\citeauthor{Stalnaker:2012tp}'s) views are compatible with knowing that involving, at least in part, a potential witnessing event.
  Sparing the details, characterisation by~\citeauthor{Weatherson:2017tb}:%
  \footnote{
    \textcite{Weatherson:2017tb} investigates kind of dispositions involved.
  }

  \nocite{Stanley:2012wg}
  \begin{quote}
    Knowing that \emph{p} is not just a matter of having \emph{p} written in a knowledge box somewhere in the brain; it can in part be constituted by active dispositions.%
    \mbox{ }\hfill\mbox{(\citeyear[8]{Weatherson:2017tb})}
  \end{quote}

  \citeauthor{Stalnaker:2012tp} highlights irony.

  Active disposition.
  Hence, potential witnessing event.%
  \footnote{
    Question whether views such as those of \cite{Stalnaker:2012tp} and \citeauthor{Stanley:2011ut} help with argument here.
    I have no idea.
    Trying to find ways to argue against proposition.
  }
\end{note}

\begin{note}
  So, finding a theory to make this argument work is not immediate.
  Hence, even if valid, not clear sound.
  Still, this won't deter.
\end{note}


%%% Local Variables:
%%% mode: latex
%%% TeX-master: "master"
%%% End:


\part{\fc{3}}

\chapter{\fc{3}}
\label{cha:foregone-conclusions}

\begin{note}[Foregone-conclusions]
  Basic idea of a foregone-conclusion.

  \begin{restatable}[Foregone-conclusions]{definition}{definitionForegoneC}
    \(\pv{\phi}{v}\) is a foregone-conclusion from some pool of premises \(\Phi\) just in case, given the agent's present epistemic, the agent would not fail to conclude \(\pv{\phi}{v}\) from \(\Phi\) were the agent to reason.
  \end{restatable}

  Whether foregone-conclusion takes agent's present epistemic state as a function.
  However, does not need to be the case that the agent recognises foregone-conclusion.

  At most, witnessing provides information about method.

  For any property \(P\) which would follow from any instance of witnessing reasoning \(\pv{\phi}{v}\) from \(\Phi\), the agent's present epistemic state is sufficient to determine \(P\) without witnessing reasoning from \(\Phi\) to \(\pv{\phi}{v}\).

  Suppose \(P\) follows from concluding.
  Forgegone-conclusion.
  So, agent's present epistemic state, agent would not fail.
  However, it then follows that \(P\).

  Here, restricted \(P\) to follow from any.
  Hence, if there are multiple methods, \(P\) may be restricted.

  However, broaden.
\end{note}

\begin{note}[Intuitive cases]
  Knowing whether and knowing how to.
  More or less interchangeable.

  Know whether \(x + y = z\).
  Know how to calculate \(x + y\).
  Indeed, for any \(z\), know whether \(x + y = z\).

  Sudoku puzzles.
  Know how to figure out.
  So, know whether any solution is valid.

  Of course, in certain cases, there are shortcuts.
  Two even numbers, then know whether by checking whether the last digit is even or odd.
  And, other cases, contingent shortcut, such as two of the same number in a square for Sudoku.

  So, really, knowing how to.
\end{note}

\begin{note}[Weaken]
  \fc{2} is weaker.
  Knowing, factive.
  Though, plausible that these amount to the same thing in various cases.
  Either because \fc{} is determined by knowing how to.
  Or, because knowing is weakened to the agent's perspective.
\end{note}

\begin{note}
  \begin{proposition}
    For any path, present epistemic state determines availability of path.
  \end{proposition}

  Start.
  Then, continue.
  Started from \(\Phi\), so will conclude.
  Hence, no matter choice made, must have taken the possibility of this choice into account.
  So, it must be the case that determined.

  Hence, if witness, then via some path.

  So, witnessing predetermined path.
  Any instances of concluding by witnessing reduces to witnessing predetermined path.

  Witnessing may provide information about path, but witnessing doesn't 


  For any X from W,
  present determines whether or not X from agent's point of view, then forgone conclusion.

  In other words, agent's present epistemic state determines.
  Agent may need to witness to figure out how determined, but witnessing does not influence.
\end{note}

\begin{note}[Non-cases]
  Now, \(p, p \rightarrow q \vdash q\) case.
  Well, determines \(q\), if we ignore possibility of revision.
  However, this doesn't tell us about all X.

  This is much stronger.
  There is nothing witnessing adds which is not already determined.

  2 + 2, peano arithmetic.

  More intuitive.
  Though, still question about deriving 2 + 2 = 4 from peano arithmetic.

  Understanding of arithmetic.
  Then, add two numbers.
  Forgone-conclusion.

  % Sudoku puzzle.
  % Solution is a foregone-conclusion.
\end{note}

\begin{note}
  To \illu{0}, questions and answers.

  Do you know whether \(83\) is prime?

  Not off the top of my head.

  Do you know whether \(28 + 55 = 83\)?

  Sure, but give me a moment.

  Do you know whether \dots

  No.

  Of course, might hold that the agent needs to have figured things out.
  But, then we have a plausible reduction.
  Knowing whether, and witnessing whether.
  Common component.

  Now, idea is a little different, as knowledge implies factivity.
  Interest with concluding is that not necessarily factive.
  From the agent's perspective.

  `Determining whether'.
  Or, rather `\fc{0}'.
\end{note}

\begin{note}
  Similar to Goldman, etc.?
  The idea is justification\dots
\end{note}

\begin{note}[Trimming]
  \begin{proposition}
    Basically, there's no role for anything beyond \(\Phi\) in the case of a foregone-conclusion.

    \(\Phi\), in the context of the agent's present epistemic state is sufficient to secure the conclusion, and the possibility of witnessing reasoning.
  \end{proposition}

  \begin{proposition}
    Foregone-conclusion just in case \(\Phi\) supports \(\pv{\phi}{v}\).
    \begin{argument}
      In short, given agent's present epistemic state, there's a guaranteed path from \(\Phi\) to \(\pv{\phi}{v}\).
    \end{argument}
    In other words, if \(\Phi\) does not support \(\pv{\phi}{v}\), then \(\pvp{\phi}{v}{\Phi}\) is not a foregone-conclusion.
  \end{proposition}

  Now, as the agent has not witnessed reasoning, need information that \(\pvp{\phi}{v}{\Phi}\) is a foregone-conclusion in order to recognise this.
  However, with information that \(\pvp{\phi}{v}{\Phi}\) is a foregone-conclusion, the information has no role in supporting \(\pv{\phi}{v}\).
\end{note}

\begin{note}
  So, that \(\pvp{\phi}{v}{\Phi}\) is a \fc{0} provides information, and explains, in part or whole, \emph{how} the agent concludes \(\pv{\phi}{v}\).
  However, \emph{why} is accounted for by \(\Phi\).
\end{note}

\begin{note}
  Why does it matter which?

  \begin{idea}[Reduction]
    If foregone-conclusion, then witness relation established by being a foregone-conclusion.
    Hence, reduction.
  \end{idea}
  Here, reduction.
  Restriction to `some'.
  It is not the case that every conclusion is a foregone-conclusion.

  However, if foregone-conclusions are of interest, then some motivation.

  Still, why of interest?
  What role do foregone-conclusion have?

  In particular.
  Need to get that some conclusion is a foregone-conclusion.
  From some reasoning.
  So, some pool of premises.
  And, that pool of premises is sufficient for conclusion.

  So, source for 2 + 2 = 4 and general ability.
  Already concluded that 2 + 2 = 4.
  Not quite, still going from general ability to 2 + 2 = 4.

  So, it's not clear that reduction is irrelevant.

  However, it is also not clear that this reduction is general.
  Want to show that foundation for reduction is not limited.
\end{note}

\begin{note}[Two worries]
  Two worries.

  First, that even though \fc{0}, the agent would not conclude.
  Either because \(\Phi\) is unavailable, or because no potential witnessing event.
  So, can't remove \fc{0} from account of why.

  However, then \fc{0} does not support.

  If grant that \fc{0} supports, then this seems to work out.
  Further, if require existence, then things that support get very messy.
  Dopeganger cases.
  Reason is I saw A, but it wasn't A, appealing to something that doesn't exist.
  Various other cases like this.

  Difference.
  In these cases, have premise, thing is that the truth value is distinct.
  Here, possibly no premise.

  Well, this is different.
  However, I don't think this is sufficient to reject the idea.
  Just because this distinction doesn't arise in the case of witnessing doesn't really do much.

  Look, a `bad' premise offers no more support for the agent than no premise.

  Second, need \emph{that} \fc{0}.
  However, the point is that this is about the agent's present epistemic state.
  \emph{Without} \fc{0}, the agent would reason.
  This is just the key point reiterated.
  Know whether, \fc{0} just adds information about which.
\end{note}


\paragraph{Foregone-concluding}

\begin{note}[Foregone-concluding]
  Pair this with a key idea.

  \begin{restatable}[Foregone-concluding]{idea}{ideaForegoneCing}
    \label{idea:reassignment}
    If foregone-conclusion, then may conclude.
    %\vspace{-\baselineskip}
  \end{restatable}

  Cases where concluding by witnessing reduces to witnessing forgone conclusion.
  \emph{Concluding \(\pv{\psi}{v'}\) from \(\Psi\) is just witnessing foregone-conclusion.}
  So, reduction, in certain cases.
  Further, if forgone conclusion, then conclude.
  At least, in certain cases.
\end{note}

\begin{note}[???]
  Only argue for a positive resolution to~{\color{red} issue:Main} given~\autoref{idea:reassignment}.

  And, leave~\autoref{idea:reassignment} as an idea.
  Insight into adopting this idea, or something like this.
\end{note}

\begin{note}
  Positive resolution may read easier if something like `in committing'.
  Commit to location from map, sum from arithmetic.
  Indeed, perhaps intuitive sense is just commitment via witnessing reasoning.

  But, we then have a reduction.
  Question is, what work does commitment do, and what work does witnessing do?

  Still, why think this?
  Why not think that concluding leads to commitments.
  Independent consequence.
  In same way that knowledge as basic entails justification, same for commitments.
  Indeed, in same way that relevant justification may be distinctive in the case of knowledge, same for commitment with concluding.
\end{note}

\begin{note}
  Assume motivate.
  What exactly is concluding?
  This will be beyond the scope of this document.
  I hope motivating a difference in extension motivates further questions about what the relation is, and the importance of witnessing in certain cases.
  As we will see, making this argument is by no means straightforward.

  I think there needs to be some instance of witnessing --- concluding does not arise from nowhere.
  Still, if witnessing leads to additional conclusions, then what do we get from concluding?
  What we will get is a general closure condition.

  To do so we narrow things down a little.
  Focus on particular types of concluding, and when concluding is accompanied by an additional property.
  Still, if in these instances no witnessing, then not more generally.
  Additional property, concluding some proposition from some premises.

  Argument, won't directly rely on intuitions about whether agent has concluded.
\end{note}

%%% Local Variables:
%%% mode: latex
%%% TeX-master: "master"
%%% End:


\chapter{An equivalent statement of \zSN{2}: \zetaS{}}

\begin{note}
  In this chapter, an equivalent statement of \zSN{2}: \zetaS{}.

  \zSN{2} developed in \autoref{cha:zS}.
  And, {\color{link} as noted}, stated in plain terms.
  Whether an agent would conclude.
\end{note}

\section{\zetaS{}}
\label{cha:zS:sec:zetaS}

\begin{note}
  \autoref{cha:zS} introduced a question ---\qzS{}, and defined a property --- \zS{} in terms of positive answers to the question.

  Here, equivalent statement.
\end{note}


\subsection{\zetaS{}}
\label{sec:zs2}

\begin{note}
  With the notion of a \requ{} in hand, we now state when an agent has \zSN{0} for some proposition-value pair:

  \begin{idea}[\izetaS{}]
    \label{idea:zetaS}
    An agent \vAgent{} has \emph{\zetaS{}} for a proposition-value pair \(\pv{\phi}{v}\) when concluding \(\pv{\phi}{v}\) from some pool of premises \(\Phi\) just in case:
    \begin{itemize}
    \item When concluding \(\pv{\phi}{v}\) from \(\Phi\):
      \begin{enumerate}[label=\arabic*., ref=\named{CS:\arabic*}]
      \item
        \label{idea:zetaS::requ}
        For any proposition-value-premises pairing \(\pvp{\psi}{v'}{\Psi}\) which is a \requ{} of concluding \(\pv{\phi}{v}\) from \(\Phi\) either:
        \begin{enumerate}[label=\alph*., ref=\named{CS:1.\alph*}]
        \item
          \label{idea:zetaS::requ-sat:Past}
          \vAgent{} has concluded \(\pv{\psi}{v'}\) from \(\Psi\).
        \item
          \label{idea:zetaS::requ-sat:Pres}
          In concluding \(\pv{\phi}{v}\), \vAgent{} simultaneously concludes \(\pv{\psi}{v'}\) from \(\Psi\).
        \item
          \label{idea:zetaS::requ-sat:Forgone}
          \(\pvp{\psi}{v'}{\Psi}\) is a \fc{0}.
        \end{enumerate}
      \end{enumerate}
    \end{itemize}
    \vspace{-\baselineskip}
  \end{idea}
\end{note}

\begin{note}
  \begin{proposition}[Equivalence between \zS{} and \zetaS{}]
    \label{prop:qzs-tick-equals-iCS}
    For an agent \vAgent{}, the following are equivalent:
    \begin{enumerate}[label=\arabic*., ref=(\arabic*)]
    \item
      \label{prop:qzs-tick-equals-iCS:qzS}
      \vAgent{} has \zS{} for \(\pv{\phi}{v}\) after concluding \(\pv{\phi}{v}\) from \(\Phi\).
      (\qzS{} had a negative resolution when concluding \(\pv{\phi}{v}\) from \(\Phi\))
    \item
      \label{prop:qzs-tick-equals-iCS:ZS}
      \vAgent{} has \zetaS{} for \(\pv{\phi}{v}\) after concluding \(\pv{\phi}{v}\) from \(\Phi\).
    \end{enumerate}
    \vspace{-\baselineskip}
  \end{proposition}
  In other words, we hold that an agent ruling out failure to conclude \(\pv{\phi}{v}\) from \(\Phi\) for any \requ{} \(\pvp{\psi}{v'}{\Psi}\) is \emph{equivalent}%
  \footnote{
    In the context of a negative resolution to \qzS{}.
  }
  to the agent either
  \begin{enumerate*}[label=(\alph*)]
  \item
    having had concluded \(\pv{\psi}{v'}\) from \(\Psi\), or
  \item
    the agent simultaneously concluding \(\pv{\psi}{v'}\) from \(\Psi\) when concluding \(\pv{\phi}{v}\) from \(\Phi\).
  \item
    \(\pvp{\psi}{v'}{\Psi}\) being a \fc{0}.
  \end{enumerate*}
\end{note}

\paragraph*{Observations}

\paragraph*{Contraposition}

\begin{note}[Contraposition]
  An argument for~\autoref{prop:qzs-tick-equals-iCS} is important from the perspective of the overall argument of this document.

  Suppose we have~\autoref{prop:qzs-tick-equals-iCS}.
  Then, view \izetaS{} as a clarification of \qzS{}.

  Specifically, the left-to-right direction.

  Only negative resolution if \izetaS{}.
  I.e.\ only if concluded, for any \requ{}.

  \begin{itemize}
  \item
    If an agent has concluded \(\pv{\phi}{v}\) from \(\Phi\) \emph{without} having concluded \(\pv{\psi}{v'}\) from some \(\Psi\), where \(\pv{\psi}{v'}\) is a \requ{} of concluding \(\pv{\phi}{v}\) from \(\Phi\), then the agent has not \csVed{} for \(\pv{\phi}{v}\) from \(\Phi\).
  \end{itemize}

  Taking the contraposition, if not \izetaS{}, then no negative resolution.
  If intuitions are unclear, then \izetaS{} offers a way to clarify those intuitions.

  Conversely, fix a negative resolution to \qzS{}, then from left-to-right direction, draw out what must also be the case.%
  \footnote{
    Consider, by analogy, knowledge, and the idea that knowledge is closed under known entailment.
    \begin{quote}
      If \vAgent{} knows both
      \begin{enumerate*}[label=(\roman*)]
      \item \(\phi\), and
      \item \(\phi\) entails \(\psi\),
      \end{enumerate*}
      then \vAgent{} knows \(\psi\).
    \end{quote}
    Observe the same dynamics are present.

    If whether an agent knows both \(\phi\) and \(\phi\) entails \(\psi\) is at issue, then establishing the agent does not know \(\psi\) establishes that either the agent does not know \(\phi\) or the agent does not know \(\psi\).

    Conversely, if an agent knows both \(\phi\) and \(\phi\) entails \(\psi\), then by closure of knowledge under known entailment, the agent must also know \(\psi\).

    Of course, whether knowledge \emph{is} closed under known entailment is unclear, but the same dynamics hold for any similar condition.
    In general, these observations amount to little more than both the closure of knowledge under known entailment and \izetaS{} both having the general form of a conditional \(A \Rightarrow B\), such that:
    \begin{itemize}
    \item
      From \(A \Rightarrow B\) and \(A\), one may infer \(B\), and
    \item
      From \(A \Rightarrow B\) and \emph{not}-\(B\), one may infer \emph{not}-\(A\).
    \end{itemize}
  }

  \begin{itemize}
  \item
    If an agent has \zetaS{} for \(\pv{\phi}{v}\) from \(\Phi\), by concluding \(\pv{\phi}{v}\) from \(\Phi\) then, has concluded \(\pv{\psi}{v'}\) from \(\Psi\), for any \(\pvp{\psi}{v'}{\Psi}\) which is a \requ{} of concluding \(\pv{\phi}{v}\) from \(\Phi\).
  \end{itemize}

  In outline, path to tension.
  Cases in which \qzS{} has negative resolution.
  Draw out as a consequence of~\autoref{prop:qzs-tick-equals-iCS} that the agent has concluded.
  So long as cases in which no witnessing, we will have tension.
  On the one hand, negative resolution, and on the other hand, has not witnessed reasoning.

  Now, if some weaker, that does not require concluding, then lack a way to generate tension.

  So,~\autoref{prop:qzs-tick-equals-iCS} should be treated with some caution.

  Of course, this does not guarantee anything interesting.
  Tension will still depend on such cases.
\end{note}

\begin{note}
  With the importance of~\autoref{prop:qzs-tick-equals-iCS} motivated, we now turn to arguing for~\autoref{prop:qzs-tick-equals-iCS}.

  The argument we provide for~\autoref{prop:qzs-tick-equals-iCS} is somewhat involved, and will go via an intermediary lemma.
  We being by outlining the structure of the argument, before turning to the details.
\end{note}

\paragraph*{Same time}

\begin{note}[Importance of at same time]
  Propositional logic.
  These premises allow to conclude two things.
  Then, conclusion that \(\phi \land \psi\) is simultaneously a conclusion that \(\phi\) and that \(\psi\).

  Or, apples in a bag.
  Five.
  Well, could do at least four, three, etc.
  Conclude at the same time.
\end{note}

\begin{note}
  For example, counterexample for some formula of propositional logic.
  Constructed a truth table.
  Identified a line.
  If counterexample, then line makes any tautology of propositional logic true.
  And, do not need to appeal to the line being a counterexample to the relevant formula to do so.
  reason from line to any recognised tautology.
  Conclude, tautology would be true.
\end{note}




\subparagraph*{Almost-transitive}

\begin{note}
  Key idea that may be obvious is that \zetaS{} is almost-transitive.
  If it needs to be the case that I'd conclude \(\pv{\psi}{v'}\) from \(\Psi\) to get \(\pv{\phi}{v}\) from \(\Phi\), and \(\pv{\chi}{v''}\) from \(X\) to get \(\pv{\psi}{v'}\) from \(\Psi\) then, need to get \(\pv{\chi}{v''}\) from \(X\) to get \(\Phi\), and \(\pv{\chi}{v''}\).

  However, this is not quite right.
  For, it may be the case that the agent has the option of getting to \(\pv{\psi}{v'}\) from \(\Psi\) from the different branch.
\end{note}

\begin{note}
  Only about the option of concluding.
  There are various other properties.
  In this respect, \qzS{} is narrow.
\end{note}

\begin{note}
  Most reasoning is short, and composes.
  Further, it needs to be the case that novel proposition-value pair introduces this concern.
  So, some extraneous proposition.
  This is irrelevant.
\end{note}

\begin{note}[Recursion]
   {
    \color{red}
    Here, it is important that we don't go fully recursive.
    For, we're only interested in \requ{1} arising from concluding \(\pv{\phi}{v}\) from \(\Phi\).
    This means that it is no immediate.
    In particular, it may be the case that some of \(\Phi\) remove what would otherwise be a \requ{} of some \requ{}.
    But, then, this is still a \requ{} with respect to the current epistemic state.

    So, I think I actually get the result that this is recursive.
    However, with a slight change.
    For, \ref{idea:zetaS::requ-sat:Past} looks to the past, and instead of \csVImp{} in the past, it only need to be the case that the agent has \csVed{} from present epistemic state.

    The point is, that there may be some pruning without concluding.
    For, something fails to be a \requ{}.
    Likewise, it may be the case that the tree grows, as the agent's epistemic state develops.

    So, we don't get a clear recursive clause.

    This leads to an observation.
    Reasoning about a \requ{} may lead to a revision of the agent's epistemic state.

    This is a somewhat interesting consequence.
    We began with the idea of failing to conclude \(\pv{\phi}{v}\) from \(\Phi\).
    This said nothing about revision.
    However, we see that this raises the possibility of revision.

    From a different perspective, this should come as no surprise.
    If the agent concluded without \csN{}, then this is going to remain a problem for any further conclusions, unless the agent figures out they were mistaken regarding the proposition-value-premises pairing being a \requ{}.
  }
\end{note}

\section{Equivalence between \qzS{} and \izetaS{}}
\label{sec:overview:an-equiv-stat-of-zs}

\begin{note}
  We began this section with the statement of a question ---~\autoref{question:zs}, or \qzS{}  --- concerning whether something is the case when an agent concludes some proposition-value pair \(\pv{\phi}{v}\) from some pool of premises \(\Phi\).
  We then reformulated the question in terms of a property, \zS{}.

  The terminology of \zS{}, or \zSN{0}, is certainly artificial.
  Still, I take the question, and the content of \zS{} to be fairly intuitive.

  Roughly, at issue is whether there is some proposition-value-premises pairing which would prevent an agent from concluding some proposition-value pair from some pool of premises.

  Some technical concepts have been appealed to in order to clarify the question --- in particular with respect to our assumptions regarding concluding, and our focus on a fixed epistemic state --- but the question itself is fairly natural.

  Consider again the \illu{1} regarding a lost pair of keys.
  There does seem to be a difference between an exhaustive consideration of all the places the keys may be before concluding the keys are lost on the one hand, and concluding the keys are lost while expecting some place on hasn't looked will come to mind on the other.
  Or, concluding that something happened based on a friends story by passive acceptance on the one hand, and concluding the thing happened after checking for consistency on the other.

  So, we take \qzS{} to be an intuitive question, and \zS{} to be an intuitive property.
  Non-standard, perhaps, but still intuitive.
\end{note}

\begin{note}
  Our attention now turns to providing an equivalent statement of \zS{}, or in other words necessary and sufficient conditions for negative resolution to \qzS{}.%
  \footnote{
    Indirectly, necessary and sufficient conditions for a positive answer, by negating either side.
  }
  We term this equivalent statement of \zS{}, `\izetaS{}', or `\zetaS{}' for short.

  \zetaS{} will have a key role in our main argument for a negative resolution to~{\color{red} issue:Main}.

  Given the prominent role \zetaS{} will have in our main argument, we will take some additional care in stating \zetaS{}.
  In particular, we will provide an separate characterisation of the relevant \(\pvp{\psi}{v'}{\Psi}\) proposition-value-premises pairings of interest.
  These we will term `\requ{1}' of concluding \(\pv{\phi}{v}\) from \(\Phi\).
  And we will include additional discussion of some subtleties regarding \requ{1} given our assumption regarding concluding from~\autoref{chapter:concluding}.

  The following account of \zetaS{} will then be relatively straightforward.
  We hold that an agent has \zetaS{} for some proposition-value pair \(\pv{\phi}{v}\) with respect to some pool of premises \(\Phi\) just in case the agent has concluded \(\pv{\psi}{v'}\) from \(\Psi\) for any \requ{} \(\pvp{\psi}{v'}{\Psi}\) of concluding \(\pv{\phi}{v}\) from \(\Phi\).

  As noted in the introduction, this leads to a closure condition.
  If an agent has \zetaS{} for some proposition-value pair \(\pv{\phi}{v}\) with respect to some pool of premises \(\Phi\), then it will follow that an agent has concluded \(\pv{\psi}{v'}\) from \(\Psi\) for any \requ{} \(\pvp{\psi}{v'}{\Psi}\) of concluding \(\pv{\phi}{v}\) from \(\Phi\).
  Before arguing that \zetaS{} is equivalent to \zS{}, we will walk through this observation in some detail.
\end{note}

\begin{note}
  Still, it is important to note the (proposed) link between \qzS{}, \zS{}, and \zetaS{}.

  \zetaS{} has the potential to be a strong condition, \emph{if} we show that \zetaS{} applies to some instance of concluding \(\pv{\phi}{v}\) from \(\Phi\) where the agent has not witnessed reasoning from \(\Psi\) to \(\pv{\psi}{v'}\) for some \requ{} \(\pvp{\psi}{v'}{\Psi}\).%
  \footnote{
    I take it this, by itself, is not immediate.
  }
  However, even if \zetaS{} turn out to be a strong condition, it remains motivated by, and --- granting the arguments to follow --- equivalent to a fairly intuitive question.
\end{note}

\begin{note}[Intuition]
  This may seem a trivial point, but I think it is important to keep in mind.
  We have introduced \zS{} via \qzS{}, and it may be easy to grant us authority over what \zS{} is, or what \qzS{} asks.
  However, the interrogative core of \qzS{} is stated in neutral terms.
  So long as you have some intuitive understanding of `concluding', then, granting sufficient information about an agent's epistemic state, you are in position to determine whether \qzS{} has a negative or positive answer, and hence whether and agent has \zS{} for the relevant proposition-value-premises pairing.

  Our argument for the equivalence between \zS{} and \zetaS{} will not further specify the content of \qzS{}.
  Instead, \zetaS{} will provide an alternative characterisation of \zS{} from which we will draw further consequences.
  In short, it will be up to you to evaluate whether the \zetaS{} really is an alternative characterisation of \zS{}.
\end{note}

\paragraph*{The argument for \autoref{prop:qzs-tick-equals-iCS}}

\begin{note}
  We now turn to providing an argument for~\autoref{prop:qzs-tick-equals-iCS}.
  To do so, we argue for an equivalent proposition focusing on \qzS{} and \zetaS{}:

  \begin{proposition}
    \label{prop:qzs-tick-equals-iCS:var}
    For any property \(\chi\):
    \begin{enumerate}[label=\Alph*.]
    \item
      \label{squish:A}
      \squishA{An}{the}
    \end{enumerate}
    \emph{if and only if}:
    \begin{enumerate}[label=\Alph*.,resume]
    \item
      \label{squish:B}
      \squishB{the}.
    \end{enumerate}
    \vspace{-\baselineskip}
  \end{proposition}

  \Autoref{prop:qzs-tick-equals-iCS:var} states that an agent satisfying \izetaS{} is a necessary condition of the agent satisfying an property \(\chi\) which is sufficient to ensure a negative resolution to~\autoref{question:zs}.

  The purpose of arguing for~\autoref{prop:qzs-tick-equals-iCS} via~\autoref{prop:qzs-tick-equals-iCS:var} is option to contrast some property \(\chi\) with \zetaS{}.

  Specifically, if we contrapose the left-to-right direction we have some property \(\chi\) which does not entail satisfaction of \izetaS{}.
  And, we will argue that any such property \(\chi\) is insufficient for a negative resolution to \qzS{}.

  Indeed, the left-to-right direction is the only direction of significant interest with respect to~\autoref{prop:qzs-tick-equals-iCS:var}.
  The right-to-left direction requires little effort.

  This is not to say the argument for the left-to-right direction will be watertight.
  Rather, we will focus on localised tension.
  Still, we are not ready to provide the argument for~\autoref{prop:qzs-tick-equals-iCS:var} just yet.

  First we need to note the equivalence between~\autoref{prop:qzs-tick-equals-iCS} and~\autoref{prop:qzs-tick-equals-iCS:var}.

  Then, with the equivalence in hand, we may turn to arguing for~\autoref{prop:qzs-tick-equals-iCS:var}.
\end{note}

\paragraph*{The equivalence between~\autoref{prop:qzs-tick-equals-iCS} and~\autoref{prop:qzs-tick-equals-iCS:var}}

\begin{note}
  \begin{proposition}
    Equivalence between~\autoref{prop:qzs-tick-equals-iCS} and~\autoref{prop:qzs-tick-equals-iCS:var}.
  \end{proposition}
\end{note}

\begin{note}
  To see the equivalence between~\autoref{prop:qzs-tick-equals-iCS} and~\autoref{prop:qzs-tick-equals-iCS:var} observe that ~\autoref{prop:qzs-tick-equals-iCS} and~\autoref{prop:qzs-tick-equals-iCS:var} entail one another.

  For, assume~\autoref{prop:qzs-tick-equals-iCS} holds.
  Then, an agent satisfying \izetaS{} is both necessary and sufficient for a positive resolution to~\qzS{}.

  Now, take some property \(\chi\) and assume~\ref{squish:A} holds.
  Given~\ref{squish:A} holds, \(\chi\) is sufficient to positively resolve~\qzS{}.
  And, by~\autoref{prop:qzs-tick-equals-iCS}, a positive resolution to \qzS{} entail the agent satisfies \izetaS{}.
  Therefore, whenever the agent satisfies \(\chi\), the agent also satisfies \izetaS{}.
  So,~\ref{squish:B} holds.

  Conversely, take some property \(\chi\) and assume~\ref{squish:B} holds.
  Then, if the agent satisfies \(\chi\), the agent also satisfies \izetaS{}.
  Given~\autoref{prop:qzs-tick-equals-iCS}, satisfying \izetaS{} is sufficient to resolve~\autoref{question:zs}.
  So,~\ref{squish:A} holds.

  Now assume~\autoref{prop:qzs-tick-equals-iCS:var} holds.

  First, suppose the agent has resolved \qzS{}.
  Then, the agent satisfies some property \(\chi\), sufficient to resolve \qzS{}.
  Therefore, by~\autoref{prop:qzs-tick-equals-iCS:var}, we have that the agent also satisfies \izetaS{}.
  Hence, satisfying \izetaS{} is a necessary condition of resolving \qzS{}.

  Second, suppose the agent satisfies \izetaS{}.
  It is immediate that satisfaction of \izetaS{} entails satisfaction of \izetaS{}.
  Therefore, satisfaction of \izetaS{} is sufficient for a positive resolution to~\qzS{}.

  This tells us \izetaS{} is necessary.
  And, as an immediate consequence, \izetaS{} is sufficient.
  For, holds for all sufficient conditions.
  Though, the most straightforward argument for the right-to-left direction involves establishing this directly.
\end{note}

% \begin{note}
%   \footnote{
%     Symbolically, represent this as follows.

%     First, shorthand for \(\pvp{\psi}{v'}{\Psi}\) being a \requ{} of \(\pvp{\phi}{v}{\Phi}\):
%     \begin{itemize}
%     \item \(\pvp{\psi}{v'}{\Psi} \rleadsto \pvp{\phi}{v}{\Phi}\)
%     \end{itemize}
%     Now, quantify:
%     \begin{itemize}
%     \item \(\forall \pvp{\psi}{v'}{\Psi}\colon (\pvp{\psi}{v'}{\Psi} \rleadsto \pvp{\phi}{v}{\Phi})\)
%     \end{itemize}
%     Concludes
%     \begin{itemize}
%     \item \(\mathcal{C}(\pv{\psi}{v'},\psi)\)
%     \end{itemize}
%     Statement that the agent concludes or has concluded \(\pv{\psi}{v'}\) from \(\Psi\) for all \(\pvp{\psi}{v'}{\Psi}\) such that \(\pvp{\psi}{v'}{\Psi}\) is a \requ{} of \(\pvp{\phi}{v}{\Phi}\).
%     \begin{itemize}
%     \item \(\forall \pvp{\psi}{v'}{\Psi}\colon (\pvp{\psi}{v'}{\Psi} \rleadsto \pvp{\phi}{v}{\Phi}) \rightarrow \mathcal{C}(\pv{\psi}{v'},\Psi)\)
%     \end{itemize}
%     Let this be \(\mathcal{c}\).

%     Now, \csN{}.
%     \begin{itemize}
%     \item \(\mathsf{CS}(\pvp{\phi}{v}{\Phi})\)
%     \end{itemize}
%     And, sufficient to resolve question:
%     \begin{itemize}
%     \item \(R_{S}(\chi,?\mathsf{CS}(\pvp{\phi}{v}{\Phi}))\)
%     \end{itemize}
%     So, getting \(\chi\) is sufficient to resolve the question of whether the agent has \csVed{} for \(\pv{\phi}{v}\) from \(\Phi\).

%     First, we get
%     \begin{itemize}
%     \item \(R_{S}(\mathcal{c},?\mathsf{CS}(\pvp{\phi}{v}{\Phi}))\)
%     \end{itemize}
%     Concluding is sufficient.
%     From this, anything that entails \(\mathcal{c}\) is sufficient.

%     \begin{itemize}
%     \item \(\forall \chi((\chi \rightarrow c) \rightarrow R_{S}(\chi,?\mathsf{CS}(\pvp{\phi}{v}{\Phi})))\)
%     \end{itemize}

%     Conversely, if some \(\chi\) is sufficient, \(\chi\) entails \(\mathcal{c}\)

%     \begin{itemize}
%     \item \(\forall \chi(R_{S}(\chi,?\mathsf{CS}(\pvp{\phi}{v}{\Phi})) \rightarrow (\chi \rightarrow c))\)
%     \end{itemize}

%     From the two above:
%     \begin{itemize}
%     \item \(\forall \chi(R(\chi,?\mathsf{CS}(\pvp{\phi}{v}{\Phi})) \leftrightarrow (\chi \rightarrow c))\)
%     \end{itemize}
%   }
%   \end{note}

\paragraph*{The argument for~\autoref{prop:qzs-tick-equals-iCS:var}}

\begin{note}
  Given the equivalence between~\autoref{prop:qzs-tick-equals-iCS} and~\autoref{prop:qzs-tick-equals-iCS:var}, we now turn to arguing for~\autoref{prop:qzs-tick-equals-iCS:var}.

  As suggested above, we split the argument into two parts.
  The right-to-left direction and the left-to-right direction.

  We begin with the right-to-left direction as it is mostly straightforward.
  We then turn to the left-to-right direction, which is significantly more involved.
\end{note}

\paragraph*{Right-to-left}

\begin{note}
  For the right-to-left direction our goal is to establish:

  For any property \(\chi\):
  \begin{quote}
  \begin{enumerate}
    \item[B.]
      \squishB{an}
    \end{enumerate}
    \emph{implies}
    \begin{enumerate}
    \item[A.]
      \squishA{The}{the}.
    \end{enumerate}
  \end{quote}
\end{note}

\begin{note}
  Fix an agent, take some property \(\chi\), and assume \squishB{the}.

  Now, \(\chi\) entails the agent satisfies \izetaS{}, but \(\chi\) may also be \zetaS{}.
  Indeed, \zetaS{} is the minimal property for which satisfaction of \(\chi\) entails \zetaS{}.
  So, though \(\chi\) is arbitrary, we must show that satisfaction of \zetaS{} alone is sufficient to resolve whether the agent has \zS{} for \(\pv{\phi}{v}\) when concluding \(\pv{\phi}{v}\) from \(\Phi\).

  Still, this is relatively straightforward.
  \qzS{} asks whether there is some \requ{} \(\pvp{\psi}{v'}{\Psi}\) such that the agent has not settled whether \(\pv{\psi}{v'}\) follows from \(\Psi\).

  Now, for any such \(\pvp{\psi}{v'}{\Psi}\), satisfying \izetaS{} requires one of the following to conditions holds:
  \begin{itemize}
  \item
    The agent has concluded \(\pv{\psi}{v'}\) from \(\Psi\).
  \item
    When concluding \(\pv{\phi}{v}\) from \(\Phi\) the agent also concludes \(\pv{\psi}{v'}\) from \(\Psi\).
  \end{itemize}
  In both cases, when concluding \(\pv{\psi}{v'}\) from \(\Psi\) for any such \(\pvp{\psi}{v'}{\Psi}\) the agent will have concluded \(\pv{\psi}{v'}\) from \(\Psi\).
  And, given a conclusion of \(\pv{\psi}{v'}\) from \(\Psi\), it seems clear the agent has settled whether \(\pv{\psi}{v'}\) follows from \(\Psi\).
  For, the agent has concluded \(\pv{\psi}{v'}\) from \(\Psi\)!
  Indeed, at question is whether the agent would conclude \(\pv{\psi}{v'}\) from \(\Psi\), and so there is no clearer answer to this question than concluding \(\pv{\psi}{v'}\) from \(\Psi\).

  So, satisfying clauses of \izetaS{} is sufficient for a negative resolve~\qzS{}.
  For, grant that notion of a \requ{0} captures the relevant cases, \izetaS{} requires concluding.
\end{note}

\subsection{Left-to-right}

\begin{note}
  For the left-to-right direction our goal is to establish:

  For any property \(\chi\):
  \begin{quote}
  \begin{enumerate}
    \item[A.]
      \squishA{An}{the}
    \end{enumerate}
    \emph{implies}
    \begin{enumerate}
    \item[B.]
      \squishB{the}.
    \end{enumerate}
  \end{quote}

  To do so, we take some arbitrary property \(\chi\), and then contrapose the conditional.
  In other words, assume \(\chi\) does not entail the agent has satisfied \izetaS{} with the sub-goal of showing that an agent satisfying \(\chi\) is sufficient to resolve whether the agent has \(\zS{}\) for \(\pv{\phi}{v}\) when concluding \(\pv{\phi}{v}\) from \(\Phi\).

  To aid the clarify of the argument we begin by introducing a plausible candidate for \(\chi\).
  We will then use the candidate to develop the core of the argument, and to conclude we will observe how the argument generalises.
\end{note}


\subsubsection{\izetaSm{}, a candidate for \(\chi\)}
\label{overview:sec:iCS-iCSm-limitation-closure}

\begin{note}
  \begin{idea}[\izetaSm{2} --- \izetaSm{}]
    \label{idea:Zsm}
    An agent \vAgent{} has \izetaSm{2} for \(\pv{\phi}{v}\) with respect to some pool of premises \(\Phi\) \emph{only if}:
    \begin{enumerate}[label=\arabic*., ref=\named{\(\zeta^{-}\)S:\arabic*}]
    \item
      \label{idea:Zsm:requ}
      For any proposition-value pair \(\pv{\psi}{v'}\) which is a \requ{} of concluding \(\pv{\phi}{v}\) from \(\Phi\) either:
      \begin{enumerate}[label=\alph*., ref=\named{\(\zeta^{-}\)S:1.\alph*}]
      \item
        \label{idea:Zsm:requ-sat:Past}
        \vAgent{} holds \emph{\vAgent{} would conclude} \(\pv{\psi}{v'}\) from the relevant pool of premises \(\Psi\).
      \item
        \label{idea:Zsm:requ-sat:Pres}
        In concluding \(\pv{\phi}{v}\) \vAgent{} \emph{also} holds \emph{\vAgent{} would conclude} \(\pv{\psi}{v'}\) from the relevant pool of premises \(\Psi\).
      \end{enumerate}
    \end{enumerate}
    \vspace{-\baselineskip}
  \end{idea}
\end{note}

\begin{note}[Difference between \izetaS{} and \izetaSm{}]
  The difference between \izetaS{} and \izetaSm{} is straightforward.

  \izetaS{} requires an agent concludes \(\pv{\psi}{v'}\) from \(\Psi\) for any \requ{} \(\pvp{\psi}{v'}{\Psi}\).
  By contrast, \izetaSm{} requires an agent holds \emph{they would conclude \(\pv{\psi}{v'}\) from \(\Psi\)}.%
  \footnote{
    Expressed differently, the agent concluding \(pv{\psi}{v'}\) from \(\Psi\) is replaced with the agent concluding that they would (conclude \(\pv{\psi}{v'}\) from \(\Psi\)).
    Here, the parentheses indicate that in both~\ref{idea:Zsm:requ-sat:Past} and~\ref{idea:Zsm:requ-sat:Pres} the agent is not required to conclude anything from \(\Psi\) directly.
  }

  The expression of an agent concluding that they would conclude may be somewhat stilted, but expresses a simple idea.
  Rather than concluding \(\pv{\psi}{v'}\) from \(\Psi\), the agent concludes that if they were to reason about whether \(\pv{\psi}{v'}\) follows from \(\Psi\), they would conclude \(\pv{\psi}{v'}\) from \(\Psi\).
  Note, \izetaSm{} does not detail the relevant pool of premises that the agent draw the conclusion from.
\end{note}

\begin{note}[\izetaS{} is (intuitively) stronger than \izetaSm{}]
  In particular, \izetaS{} is intuitively stronger than \izetaSm{}.

  First, observe that~\ref{idea:zetaS::requ-sat:Past} and~\ref{idea:zetaS::requ-sat:Pres} (plausibly) entail ~\ref{idea:Zsm:requ-sat:Past} and~\ref{idea:Zsm:requ-sat:Pres}, respectively.
  For, if an agent has concluded \(\pv{\psi}{v'}\) from \(\Psi\), then \emph{in so doing} the agent has shown that they would conclude \(\pv{\psi}{v'}\) from \(\Psi\).

  Second, observe neither~\ref{idea:Zsm:requ-sat:Past} nor~\ref{idea:Zsm:requ-sat:Pres} (plausibly) entail~\ref{idea:zetaS::requ-sat:Past} nor \ref{idea:zetaS::requ-sat:Pres}, respectively, so long as there are plausible cases in which an agent concludes that they would conclude \(\pv{\phi}{v}\) from \(\Phi\) without concluding \(\pv{\phi}{v}\) from \(\Phi\).

  Indeed, it seems there are plausible cases.

  For, it seems an agent may conclude that they would conclude \(\pv{\phi}{v}\) from \(\Phi\) without concluding \(\pv{\phi}{v}\) from \(\Phi\).
  For example, one may be informed that one would conclude \(\pv{\phi}{v}\) from \(\Phi\) via testimony.
  Hence, the only relevant premises one plausibly requires is that they have been informed they would conclude \(\pv{\phi}{v}\) from \(\Phi\) via testimony, and \(\Phi\) may be arbitrary.
  E.g.\ I tell you that if you looked at the map you would conclude that East Palo Alto is directly north of Palo Alto (shock!) and if you trust the map you may even conclude that East Palo Alto \emph{is} directly north of Palo Alto.
  Still, you do not (obviously) conclude East Palo Alto is directly north of Palo Alto from any premises associated with details of the map.
\end{note}

\begin{note}
  \izetaSm{} as a candidate \(\chi\).
  {
    \color{red}
    Following, arbitrary \(\chi\).
    However, substitute in \izetaSm{} is desired.
  }
\end{note}

\begin{note}
  Now, we have seen how~\ref{idea:Zsm:requ-sat:Past} and~\ref{idea:Zsm:requ-sat:Pres} are (plausibly) \emph{strictly} weaker than~\ref{idea:zetaS::requ-sat:Past} and~\ref{idea:zetaS::requ-sat:Pres}, respectively.
  And, we have noted that both~\ref{idea:Zsm:requ-sat:Past} and~\ref{idea:Zsm:requ-sat:Pres} seem to align with the motivation provided from \csN{}.
  We briefly highlight why the distinction between ~\ref{idea:Zsm:requ-sat:Past} and~\ref{idea:Zsm:requ-sat:Pres} and ~\ref{idea:zetaS::requ-sat:Past} and~\ref{idea:zetaS::requ-sat:Pres}, respectively, will matter for our overall argument.

  Observe, \csN{}%
  \footnote{
    As stated, with~\ref{idea:zetaS::requ-sat:Past} and~\ref{idea:zetaS::requ-sat:Pres} over ~\ref{idea:Zsm:requ-sat:Past} and~\ref{idea:Zsm:requ-sat:Pres}, respectively.
  }
  will lead to tension with a positive resolution to~{\color{red} issue:Main} just in case we manage to find an instance in which an agent \csN{} for \(\pv{\phi}{v}\) from \(\Phi\) without witnessing reasoning from \(\Psi\) to \(\pv{\psi}{v'}\) for some \requ{} \(\pvp{\psi}{v'}{\Psi}\) of concluding \(\pv{\phi}{v}\) from \(\Phi\).
  However, this tension will not follow if the weakened variants of~\ref{idea:zetaS::requ-sat:Past} and~\ref{idea:zetaS::requ-sat:Pres} are adopted.
  For, neither~\ref{idea:Zsm:requ-sat:Past} nor~\ref{idea:Zsm:requ-sat:Pres} would require the agent to conclude \(\pv{\psi}{v'}\) from \(\Psi\).

  Indeed, we will argue for the existence of cases of exactly the kind described.
  Hence, the role of~\ref{idea:zetaS::requ-sat:Past} and~\ref{idea:zetaS::requ-sat:Pres} over~\ref{idea:Zsm:requ-sat:Past} and~\ref{idea:Zsm:requ-sat:Pres} is not merely an issue of motivation, but also crucial to establishing tension.
\end{note}

\paragraph*{Arguing}

\begin{note}
  Fix an agent.
  Take arbitrary \(\chi\), such that \(\chi\) does not imply satisfaction of \izetaS{}.
  (E.g.\ \izetaSm{}, as seen above.)
\end{note}

\begin{note}
  Now, assume the agent is concluding \(\pv{\phi}{v}\) from \(\Phi\).
  There are two cases to consider.
  We state the each case for \(\chi\) generally, and then below state the case with respect to \izetaSm{}.
  \begin{enumerate}[label=\Roman*., ref=\Roman*]
  \item
    \label{iZm:arg:case:I}
    The agent satisfies \(\chi\) when concluding \(\pv{\phi}{v}\) from \(\Phi\).
    \begin{itemize}
    \item
      The agent concludes they would conclude \(\pv{\psi}{v'}\) from \(\Psi\) when concluding \(\pv{\phi}{v}\) from \(\Phi\), for any \(\pvp{\psi}{v'}{\Psi}\) which is a \requ{} of concluding \(\pv{\phi}{v}\) from \(\Phi\).
    \end{itemize}
  \item
    \label{iZm:arg:case:II}
    The agent has already satisfied \(\chi\) prior to concluding \(\pv{\phi}{v}\) from \(\Phi\).
    \begin{itemize}
    \item
      The agent has (already) concluded they would conclude \(\pv{\psi}{v'}\) from \(\Psi\) when concluding \(\pv{\phi}{v}\) from \(\Phi\), for any \(\pvp{\psi}{v'}{\Psi}\) which is a \requ{} of concluding \(\pv{\phi}{v}\) from \(\Phi\).
    \end{itemize}
  \end{enumerate}

  We take each case in turn.
  Further, as \(\chi\) does not imply satisfaction of \izetaS{}, and \izetaS{} is not trivially satisfied, for both cases we will assume the agent does not satisfy \izetaS{}.
\end{note}

\subparagraph*{Case~\ref{iZm:arg:case:I}}

\begin{note}[With \(\chi\)]
  \(\chi\) is sufficient for a negative answer to \qzS{}.

  Key observation.
  In order for \(\chi\) to be sufficient, agent's perspective.

  \begin{proposition}
    \label{prop:chiProp:no-may-fail}
    \(\chi\) must ensure that from the agent's perspective, the agent may not fail to conclude \(\pv{\psi}{v'}\) from \(\Psi\).
    \begin{argument}
      Direct from \qzS{}.
      For, negative answer.
      However, negative answer only if it is not the case that the agent may fail to conclude \(\pv{\psi}{v'}\) from \(\Psi\).
    \end{argument}
  \end{proposition}

  However, from~\ref{question:zs:option}, the agent has the option of concluding \(\pv{\psi}{v'}\) from \(\Psi\).

  By assumption the agent has not yet satisfied \(\chi\), as the agent has not yet concluded \(\pv{\phi}{v}\) from \(\Phi\).
  Hence, the agent has a check on whether they would come to satisfy \(\chi\) when concluding \(\pv{\phi}{v}\) from \(\Phi\).

  For, if the agent reasons about whether \(\pv{\psi}{v'}\) follows from \(\Psi\) and fails to conclude \(\pv{\psi}{v'}\) from \(\Psi\), then it would not be the case that the agent satisfies \(\chi\).

  Rephrase.
  \(\chi\) is sufficient for a negative answer to \qzS{}.
  In order for a negative answer, it may not be the case that the agent may fail to conclude \(\pv{\psi}{v'}\) from \(\Psi\).
  Therefore, the agent not failing, from their perspective, is required to satisfy \(\chi\).
  However, this means that before concluding \(\pv{\phi}{v}\) from \(\Phi\), the agent may establish whether they would come to satisfy \(\chi\).

  So, this means that there is a possibility of branching.
  If the agent reasons about whether \(\pv{\psi}{v'}\) follows from \(\Psi\), then the agent may fail to conclude \(\pv{\phi}{v}\) from \(\Phi\).

  Whether the agent would conclude \(\pv{\phi}{v}\) from \(\Phi\) is at issue.
  And, the agent only gets \(\chi\) when concluding.
  So, whether or not \(\chi\) cannot be sufficient for a negative answer.

  If failing is sufficient for positive answer, then failing is also sufficient to show that the agent does not satisfy \(\chi\).
  Therefore, \(\chi\) cannot be sufficient.
\end{note}

\begin{note}[With \izetaSm{}]
  From \izetaSm{}.
  \(\pvp{\psi}{v'}{\Psi}\) is a \requ{} of concluding concluding.
  Hence, so long as the agent has not already\dots it follows that concluding concluding is not sufficient for a negative answer to \qzS{}.
\end{note}

\begin{note}[Summary]
  Summary.
  \qzS{}, what would happen if the agent first reasoned about whether \(\pv{\psi}{v'}\) follows from \(\Psi\).
  This is the question.
  In order for negative answer, from agent's point of view, would not fail.
  However, question still holds if this is not yet the agent's point of view.
\end{note}

\begin{note}[Observation]
  Observe, as not yet \(\chi\), re-expressed \qzS{} as a question about whether \(\chi\).
  Therefore, the above argument only applies given we are in case~\ref{iZm:arg:case:I}.
  In case~\ref{iZm:arg:case:II}, we assume the agent already satisfies \(\chi\), and hence the argument will be distinct.
\end{note}

\subparagraph*{Case~\ref{iZm:arg:case:II}}

\begin{note}
  {
    \color{red}
    This should be revised, as with the idea of a \fc{0}, I no longer need to worry about the possibility of revision.

    Instead, the basic argument is that if a \requ{0}, then this still applies to whatever \(\chi\) is.
    For anything weaker, check.
  }
\end{note}

\begin{note}
  Now turn to case~\ref{iZm:arg:case:II}.

  With case~\ref{iZm:arg:case:I}, argued that \(\chi\) fails to be sufficient, because \qzS{} applies equally to whether the agent satisfies \(\chi\).
  Possibility of present reasoning branching so the agent does not satisfy \(\chi\).

  Case~\ref{iZm:arg:case:II} requires distinct argument, as by assumption the agent satisfies \(\chi\).
  Again, we have the assumption that the agent has not concluded \(\pv{\psi}{v'}\) from \(\Psi\).
\end{note}

\begin{note}
  Observe, the strategy applied to case~\ref{iZm:arg:case:I} does not extend to case~\ref{iZm:arg:case:II}.
  For, \requ{} of concluding \(\pv{\phi}{v}\) from \(\Phi\).
  Hence, it need not be the case that the agent had the option of concluding \(\pv{\psi}{v'}\) from \(\Psi\) when satisfying \(\chi\).
  In particular, when establishing from their perspective they would not fail to conclude.

  From the perspective of \izetaSm{}, concluded would conclude when the agent did not have the option of concluding.

  For example, consider a case of (apparent) testimony.
  If follow strategy, win game.
  Did not have an understanding of the rules.
  Hence, did not have the option to reason from premises and reach a different conclusion.
  Only after coming to understand rules (or more strictly, adopting the perspective of understanding rules) does the option of evaluating the conditional become available.
\end{note}

\begin{note}
  Our strategy is to split case~\ref{iZm:arg:case:II} into two sub-cases, depending on whether the agent may revise whether or not the agent may revise their satisfaction of \(\chi\).
  Specifically:

  \begin{enumerate}[label=\roman*., ref=\roman*]
  \item
    \label{iZm:arg:case:II:sub:i}
    From the agent's perspective:
    The agent may revise their epistemic state so that the agent does not satisfy \(\chi\), given the agent's current epistemic state.
  \item
    \label{iZm:arg:case:II:sub:ii}
    From the agent's perspective:
    The agent may not revise their epistemic state so that the agent does not satisfy \(\chi\), given the agent's current epistemic state.
  \end{enumerate}

  It may seem only sub-case~\ref{iZm:arg:case:II:sub:ii} is compatible with satisfaction of \(\chi\).

  For, as we have observed in~\autoref{prop:chiProp:no-may-fail}, not the case that the agent may fail.

  Rewriting:

  \begin{enumerate}[label=\roman*\('\)., ref=\roman*\('\)]
  \item
    \label{iZm:arg:case:II:sub:i:var}
    From the agent's perspective:
    The agent may fail to conclude \(\pv{\psi}{v'}\) from \(\Psi\), if the agent were to attempt to conclude \(\pv{\psi}{v'}\) from \(\Psi\).
  \item
    \label{iZm:arg:case:II:sub:ii:var}
    From the agent's perspective:
    The agent may not fail to conclude \(\pv{\psi}{v'}\) from \(\Psi\), if the agent were to attempt to conclude \(\pv{\psi}{v'}\) from \(\Psi\).
  \end{enumerate}

  So, it may appear only sub-case~\ref{iZm:arg:case:II:sub:ii:var} is compatible with the agent satisfying \(\chi\).

  However, it is important to keep in mind the scope of the relevant instance of `may'.
  Given \(\chi\), failure to conclude \(\pv{\psi}{v'}\) from \(\Psi\) may be ruled out from the agent's perspective.
  However, it may also be the case that the agent entertains the possibility of failing to satisfy \(\chi\).
  Hence, as the agent may not satisfy \(\chi\), the agent may fail to conclude \(\pv{\psi}{v'}\) from \(\Psi\).%
  \footnote{
    Of course, if you interpreted \qzS{} in line with sub-case~\ref{iZm:arg:case:II:sub:ii}, you may ignore sub-case~\ref{iZm:arg:case:II:sub:i}.

    Still, I take \qzS{} to be compatible with both sub-cases~\ref{iZm:arg:case:II:sub:i} and~\ref{iZm:arg:case:II:sub:ii}.
  }
\end{note}

\begin{note}[What we will argue]
  Respectively, we will argue:
  \begin{itemize}
  \item
    For sub-case~\ref{iZm:arg:case:II:sub:i}, \(\chi\) is insufficient for a negative answer to \qzS{}.
  \item
    For sub-case~\ref{iZm:arg:case:II:sub:ii}, re-assignment of concluding, hence \(\chi\) (trivially) entails concluding. Hence, conflict with our assumption that \(\chi\) does not entail concluding.
  \end{itemize}

  In other words:
  \begin{itemize}
  \item
    \qzS{} scopes over certain revisions to an agent's epistemic state.
  \item
    If no revision, then the relation between \(\pv{\psi}{v'}\) and \(\Psi\) is sufficient to reduce the relevant instance of concluding to \(R\) and witnessing.
  \end{itemize}

  \(R\) the agent would conclude \(\pv{\psi}{v'}\) from \(\Psi\) if the agent were to reason from \(\Psi\) to \(\pv{\psi}{v'}\), and from the agent's current epistemic state, \(R\) may not fail to hold.

  Here, importance of~\ref{idea:reassignment}.
  Re-assignment.
  The only thing for the agent to do is witness \(R\).
\end{note}

\subparagraph*{\(\pvp{\psi}{v'}{\Psi}\) remains a \requ{0} of concluding \(\pv{\phi}{v}\) from \(\Phi\), given \(\chi\)}

\begin{note}
  We begin with a minor, but important observation.

  \begin{proposition}
    \(\pvp{\psi}{v'}{\Psi}\) remains a \requ{0} of concluding \(\pv{\phi}{v}\) from \(\Phi\), given \(\chi\)
  \end{proposition}
\end{note}

\begin{note}[Still a \requ{}]
  Observe that \(\pvp{\psi}{v'}{\Psi}\) is a \requ{} with respect to concluding \(\pv{\phi}{v}\) from \(\Phi\).

  Even if agent has satisfied \(\chi\), and so from perspective, still holds up.
  Indeed, whether \(\pvp{\psi}{v'}{\Psi}\) is \requ{} is independent of whether the agent satisfies \(\chi\) or has concluded \(\pv{\psi}{v'}\) from \(\Psi\).

  For a \requ{}, what matters is option and failure.
  And, for a negative resolution to \qzS{}, no failure.

  Of course, from agent's perspective they would not fail.
  However, if reason and did fail, problem.

  So, from the present point of view, a \requ{}.
\end{note}

\paragraph{The sub-cases}

\subparagraph*{Sub-case~\ref{iZm:arg:case:II:sub:i}.}

\begin{note}[Distinction between \(\chi\) and concluding]
  There is an important difference between the two cases.
  By assumption, \(\chi\), has not concluded.

  So with \(\chi\), two types of possible epistemic states.
  Concluded, not concluded.
  Further, possibility of either of these epistemic states.

  Point is, with \(\chi\), because no entailment, and assumption, these two types of epistemic state are open.

  From present, the agent only expect to go to one.
  However, this is from the perspective of current epistemic state.

  By contrast, with concluding, the agent already in the relevant epistemic state.
  The agent has concluded \(\pv{\psi}{v'}\) from \(\Psi\).
\end{note}

\begin{note}
  As the agent is not in the relevant type of epistemic state, and has the option to go to the right epistemic state, \qzS{}.
\end{note}

\begin{note}
  At issue is whether the agent may do some reasoning any end up not concluding \(\pv{\phi}{v}\) from \(\Phi\) granted they have not already concluded \(\pv{\psi}{v'}\) from \(\Psi\).
  And, as the agent has not concluded \(\pv{\psi}{v'}\) from \(\Psi\), then regardless of perspective on how reasoning would go, the option is present, and as \(\chi\) does not entail, the agent does not have the result of taking the option.
  Taking the option would give something distinct.
  Option, so figure out whether.
\end{note}

\begin{note}
  Counterpoint.

  Do not need to drop the pen in order to know that it will fall to the ground.

  Or, perhaps, do not need to go and check my car is parked outside in order to know that it is parked outside.

  Difference here is these cases involve acquiring novel information, while \qzS{} involves reasoning with respect to the agent's current epistemic state.
\end{note}

\begin{note}
  Although we have introduced the possibility of revision, it is very narrow.
  Our interest is not with revision in general.
  The idea that an agent would conclude \(\pv{\phi}{v}\) from \(\Phi\) given arbitrary revision is incredibly strong.

  Rather, the revision is quite specific.
  It is because the agent has concluded \emph{that}, and because this is now a \requ{}.
  This does not scope over querying arbitrary premises, nor adopting addition premises.
  Only because this question remains open.
\end{note}


\subparagraph*{Sub-case~\ref{iZm:arg:case:II:sub:ii}.}

\begin{note}
  No revision.

  So, from the agent's perspective, if reason from \(\Psi\), would conclude \(\pv{\psi}{v'}\).
\end{note}

\begin{note}
  Now, by assumption witnessing can't do anything.
  There can be nothing added by witnessing that would matter to concluding \(\pv{\psi}{v'}\) from \(\Psi\).
  If there were, then possibility of revision.

  Okay, so, this means that witnessing doesn't add anything.

  Hence, whether on not X is determined by the agent's present epistemic state.
  This does not mean X, as X may be the result of witnessing.
  However, must have enough.
  So, reduce whether to something independent of witnessing.




  Some X, but then, agent's present epistemic state secures X.







  So, have something in the agent's present epistemic state.
  
\end{note}

\begin{note}
  So, sufficient resources given present epistemic state.

  For, if insufficient, then fail to conclude.

  So, witnessing, putting those resources into action.

  So, take any X.
  If X makes a difference to concluding, then, availability of X + witnessing.

  So, something, X.
  Get this by witnessing.
  However, if this makes a difference, then split into pre-X and witnessing X.
  Given sub-case, no failure.
  So, only pre-X makes a difference.
  Yet, have pre-X.
  Adding in witnessing won't do anything.

  So, then, split.

  Not possible to find some X, such from the agent's point of view, such that whether X is unique to witnessing reasoning from \(\Psi\) to \(\pv{\psi}{v'}\) such that X matters to whether conclude, and X is not composite.

  This is quite strong, but does not generalise easily.
  For, if possibility of branching, revision, etc.\ then there may be some such X.
  Indeed, agent has not concluded, might fail, so conclusion may add something unique.

  But, then, re-assignment.

  Hence, concluded.

  To \illu{1}, deterministic causation.
  Some causal model, a bunch of equations.
  Wouldn't say X has caused Y given application of equations.
  However, nothing more than putting those equations in motion.

  Note, though, that given re-assignment, this is what we get.
  What we're after is this reduction.
  At issue is not the intuitive sense of `concluding', but whether there is some reduction of this kind.

  Observe, this kind of thing does give a reduction.
  However, no clear cases of this happening.
  Too much information required.
  And, deterministic, at least with respect to description at level of premises and conclusions.
  Strong assumptions.

  Further, we have no eliminated role of witnessing.
  As with causation, we have no shown that this relation is specified independently.

  However, we do have a static reduction.
  Nothing of relevance is introduced by the dynamics.
  We might not get a specification without reference to the dynamics, but we don't get anything of relevance from the dynamics themselves.

  This is basically a redescription of the assumption made.
  If there is no possibility of failing, then the dynamics are pre-determined, at least from the agent's perspective.





  Sufficient resources plus witnessing.
  These resources, no possibility of failure.
  So, for anything that does not involve witnessing, we have this.

  In addition, for anything witnessing would get, the agent has a guarantee that they would obtain.

  Not perfect information.
  Maybe exact premises, steps of reasoning, etc.
  Still, all of these are available.
\end{note}



\begin{note}[Edge case]
  \(\chi\), the agent reasons, but does not conclude \(\pv{\psi}{v'}\) from \(\Psi\).
  Nor does the agent fail to conclude \(\pv{\psi}{\overline{v'}}\) from \(\Psi\).

  However, no revision.
  Well, then the agent won't conclude \(\pv{\phi}{v}\) from \(\Phi\).
  If the agent does, then revised epistemic state so that \ref{question:zs:subjunctive} does not hold.
  Hence, \qzS{} would no longer apply.
\end{note}

\begin{note}[Key idea]
  \begin{itemize}
  \item
    \(\chi\) does not entail concluded.
  \item
    Option to conclude.
  \item
    If agent were to reason, to types resulting states.
  \item
    Concluded, failed to conclude.
    Really, various possible resulting states, as reasoning may not terminate either way, and in general there may be various values other than \(v'\) that the agent may conclude.
  \item
    The agent isn't in either type of epistemic state.
  \item
    Now, \(\chi\) states from current perspective, what the result would be.
  \item
    However, it remains the case that these two types of epistemic states are open.
  \item
    Hence, it remains the case that reasoning could lead to either type.
  \item
    So, it remains the case that the agent may fail to conclude \(\pv{\psi}{v'}\) from \(\Psi\).
  \end{itemize}

  The core of the idea is that \qzS{} concerns the dynamics.
  If reason, then what would the result be with respect to concluding \(\pv{\phi}{v}\) from \(\Phi\).
  So, concluding to dynamics.
  Hence, \(\chi\) of the appropriate kind does not settle.
\end{note}

\subsection{No revision to the agent's present epistemic state}
\label{sec:no-revision-agents}

\begin{note}
  Reflect on this constraint.

  This is what stops the argument from going all the way.

  But, there are good reasons for this.

  Difficulty with figuring out what would be the case under revisions to present epistemic state.

  In particular, the relevant conditionals.
\end{note}

\begin{note}
  Look, it's really really hard.
  For, the conditionals, or that the agent has the ability are key.
  Without these, there's no check on the agent's present reasoning.

  Hence, these can't be up for grabs when it comes to checks.

  If still possible to check whether it's really the case that\dots

  WAIT!

  There's a difference between the conditional and the consequent of the conditional.

  This should be part of the argument for getting to it being a \fc{}.

  Okay, this is interesting.
  Though, it gets tricky.
  Because, here, option to appeal to general ability.
  General ability, so conditional.
  This then would lead to the same problem.

  So an example here is testimony.
  Though, this is delicate.
  If testimony comes in just before concluding, then problem.
  However, general ability is testified.
  Then, get this as an instance, where there's really no question whether this is an instance.

  Though, here, this means prior to everything, expanded general to all specific.
  I don't see a problem, generally speaking, here.

  So, a \fc{}.
  From the agent's present epistemic state it's already been settled that the agent would conclude, if they were to reason.
  This is not something that the agent is concluding now.
  If it was, then there would be an issue, for check on whether this conclusion really makes sense.

  So, motivate a shift prior, to when the agent concluded this.
  Now, there's various cases.

  Though, as all I want is the move to general, all I need is that the agent concluded.

  Well, hang on, these are now just a special case.
  That's interesting.
  So, in general, \fc{}.
  But, given this, move back to when not a \fc{}.
  If you allow this move to happen, then there's difficulty.
  So, plausibly this doesn't actually do much.
\end{note}

\section{Notes}

\begin{note}[Witnessing?]
  Intuitively, this suggests a negative resolution to~{\color{red} issue:Main}.
  Agent needs to witness reasoning, else, the agent does not have sufficient information about the relevant dynamics.

  Indeed, if negative resolution to~{\color{red} issue:Main}, then I suspect this brief sketch is intuitive.
  Of course, negative answer to \qzS{} is not required for concluding, as \qzS{} requires that this holds for all \requ{1}.
  Still, given what is intuitively required for negative answer to \qzS{}, witnessing.
\end{note}

\begin{note}
  Now, important to keep in mind is that we have no assumed that concluding involves witnessing.
  By assumption, there is some difference, else we are done.
  But, what exactly this distinction amounts to is unclear.
  An account of concluding, beyond scope.
  Further, still have the issue to resolve.
  However, have a clue.
  Concluding, between \(\pv{\phi}{v}\) and \(\Phi\).
  Concluding that, existence of a relation.
  Whatever the relation of concluding is, it does not yet hold, but the agent has a guarantee that it will hold.
\end{note}

\begin{note}
  The point, in brief, is that \qzS{} is about concluding.
  And, so long as the agent has not concluded \(\pv{\psi}{v'}\) from \(\Psi\) when concluding \(\pv{\phi}{v}\), that failure to conclude would lead to failure.
  So, \(\pvp{\psi}{v'}{\Psi}\) remains unsettled.
  If the agent does the reasoning, then would not conclude.
  So, the agent may fail to conclude, because on reasoning about whether, failure to conclude \(\pv{\psi}{v'}\) from \(\Psi\) would lead to a recognised failure of \(\chi\).

  \qzS{} is about whether the agent would conclude.
  If agent hasn't concluded, then question remains.
  Still, I do not think this is immediate.
\end{note}


\begin{note}
  Important to note, that still all relative to the agent's present epistemic state, and what the agent is interested in concluding.
  For, it may be that the agent revises their epistemic state so that some relation does not hold.
  For example, the agent has concluded that some reduction holds.
  Therefore, if they prove this, then they prove something else, and, also prove the something else by other means.
  However, learn this reduction does not hold.
  Now, no longer a \requ{}.
  Indeed, relation fails because the reduction does not hold, or because the agent does not have the means to prove.
\end{note}

\paragraph*{Closing}

\begin{note}[Closure]
  So, this is our motivation.
  What we have is an intuitive idea which leads to this kind of limitation, and hence conclusre condition.
  So, if there are cases of interest, motivation that an agent concludes.
  But, conversely, the condition is strong.
  So, these cases are harder.
\end{note}

\paragraph{More details on \zetaS{}}

\begin{note}
  We've only focused on failure to conclude.
  However, the agent may also conclude something else.
  Possible that there are premises for the agent \(\Psi\) and \(\overline{\Psi}\) such that from \(\Psi\) get \(\pv{\psi}{v'}\) and from \(\overline{\Psi}\), get \(\pv{\psi}{\overline{v'}}\).

  For sure, but this is a different condition.
  You may also want to impose this given the intuitive motivation for \csN{}.
  Indeed, from the perspective of no branching.
  Distinct condition, and we will not impose this.

  Note, also, that so long as distinction between concluding \emph{that} and concluding, then this is also going to be insufficient in isolation.
  For, though the agent may have exhausted other possibilities, this won't get a conclusion.
  And, if not distinct, then a plausible path to negative resolution.
\end{note}

\begin{note}
  \izetaS{} does not require the conclusion to be any good.
  If you want to build this in, sure.
  However, not for us.
  It is a strong assumption, and would have no function in the arguments to follow.
\end{note}

\paragraph*{Minor clarifications}

\begin{note}[Importance of \csN{}]
  \izetaS{} is key.
  Argument for negative resolution to~{\color{red} issue:Main} largely rests on \izetaS{}.
  As we have seen, closure.

  However, briefly note that a few things.

  First, agent's reasoning.
  At issue is whether the agent may reason to a different conclusion.
  There's nothing that would lead me elsewhere.

  Second, agent's reasoning.
  Independent of whether \(\phi\) has value \(v\), \(\psi\) has value \(v'\), or any of the premises.
  Need not be the case that satisfaction amounts to anything substantial.
  No clause for justification, etc.

  Third, competence, rather than performance.
\end{note}

\paragraph*{\emph{Concluding}}

\begin{note}[Key feature]
  We now turn to the key feature of \izetaS{}:
  The requirement that an agent \emph{concludes} \(\pv{\psi}{v'}\) from \(\Psi\) when \(\pvp{\phi}{v'}{\Psi}\) is a \requ{} of concluding \(\pv{\phi}{v}\) from \(\Phi\).

  We begin by noting why this requirement should be treated with caution from a general perspective.
  We then further motivate caution by first relating the requirement to the motivation we have provided for \csN{}, and second by highlighting how the requirement will have a role in developing tension.
  This discussion will involve considering a weaker requirement, and following our motivation of caution we will argue that no weaker requirement will suffice to capture the motivation we have provided for \csN{}.
\end{note}


\begin{note}[Kettle logic]
  Well, it's true that the person must be acquitted, but at the same time, the person is going to have a hard time explaining how, for example, he brewed coffee for a week.

  Still, highlights what the neighbour needs to conclude.
  Did borrow.
  Was not lent damaged
  And, was returned damaged.

  Here, burden of argument.
  However, we're not interested in whether the neighbour would convince, but whether the neighbour would reach a different conclusion if they were first to reason about one of the three before concluding that the kettle was returned damaged.
\end{note}

\subsection{Literature}
\label{sec:zS:literature}

\paragraph{Circularity}

\begin{note}
  \ideaCS{} is not about circular reasoning in the sense that the term `circularity' suggests that the reasoner has taken the conclusion of the reasoning for granted.

  There's nothing in \ideaCS{} that appeals to getting \(\psi\) having value \(v'\) from \(\phi\) having value \(v\).

  However, does identify a problem in the sense that would prevent the agent from getting \(\psi\) having value \(v'\) from \(\phi\) having value \(v\).
\end{note}

\begin{note}[Testimony 1]
  \begin{illustration}[Testimony 1]
    \label{illu:CS:test:basic}
    \mbox{}
    \begin{enumerate}[label=\arabic*., ref=(\arabic*)]
    \item
      \label{ex:eiS:t:basic:test}
      \nagent{11} stated that they are trustworthy when speaking on matters regarding their personal character.
    \item
      \label{ex:eiS:t:basic:ok}
      \nagent{11} is trustworthy when speaking on matters regarding their personal character.
    \end{enumerate}
  \end{illustration}
  This kind of case is intuitively problematic.
  It seems that already need trustworthy.
  However, in order for \csN{} to apply, need for it to be the case that one has some check on whether \nagent{11} is trustworthy.
  And, by reasoning.

  This need not be the case.
  Of course, this does not mean that an agent need \csN{}.
  May be other necessary conditions.
\end{note}

\paragraph{Sgaravatti}

\begin{note}
  For example, consider what \citeauthor{Sgaravatti:2013wu} terms the `Justification Account' of circularity.\nolinebreak
  \footnote{
    As \citeauthor{Sgaravatti:2013wu} notes, the Justification Account of circularity is a rewriting of the third type of `epistemic dependence' considered by \citeauthor{Pryor:2004ws}~(\citeyear[359]{Pryor:2004ws}).
    Neither \citeauthor{Pryor:2004ws} nor \citeauthor{Sgaravatti:2013wu} endorse the Justification Account, but I take the spirit of the account to sufficient for interest.
    Still, the considerations which follow also apply to distinguish the {\color{red} problem identified} from \citeauthor{Sgaravatti:2013wu}'s favoured account (\Citeyear[\S3]{Sgaravatti:2013wu}) and the fifth type of `epistemic dependence' considered by \citeauthor{Pryor:2004ws}~(\citeyear[359]{Pryor:2004ws}).
  }

  \begin{quote}
    \begin{enumerate}[label=(JA), ref=(JA)]
    \item\label{sg:JA} An argument is circular if and only if for you to have justification to believe the premisses, it is necessary that you have justification to believe the conclusion.\nolinebreak
      \mbox{}\hfill\mbox{(\Citeyear[754]{Sgaravatti:2013wu})}
    \end{enumerate}
  \end{quote}
  Where `justification to believe' is to be read as in terms of having formed the belief in an epistemically appropriate way as opposed to (merely) possessing sufficient resources to form formed the belief in an epistemically appropriate way.\nolinebreak
  \footnote{
    Or, however you prefer to characterise \citeauthor{Firth:1978vi}'s (\Citeyear{Firth:1978vi}) distinction between doxastic and propositional justification (or warrant).
    See also \citeauthor{Silva:2020aa} (\Citeyear{Silva:2020aa}) --- esp.\ fn.\ 1.
  }
  (\citeauthor[Cf.][754--755]{Sgaravatti:2013wu})
\end{note}

\begin{note}
  First, reliance on something like justification.

  With \support{}, we arguably have something distinct.
  Have not placed constraints on reasoning.
  Hence, \ideaCS{} applies even when no justification (or any other epistemic attribute) is found.

  Indeed, to the extent that the value \(v\) need not be truth, \ideaS{} and \ideaCS{} are broader.

  Point extends to relation between the premises and the conclusion of a step of reasoning.
  There's some issue with whether there's a clear reduction to premises.

  Now, both these points may be addressed by linking justification to steps of reasoning.
  However, it still remains that get this kind of circularity by placing a constraint on permissible steps of reasoning.
\end{note}

\begin{note}
  Second, having something.
  Contrasts to reasoning in an interesting way.
\end{note}

\begin{note}[\citeauthor{Sgaravatti:2013wu} on necessity]

  \begin{quote}
    For my present purposes it will suffice to say that a good test of A’s being necessary for B (and thus of B’s being sufficient for A) is the satisfaction of two subjunctive conditionals. First, if A did not hold, B would not hold; secondly, if B were to hold, A would hold.%
    \mbox{}\hfill\mbox{(\citeyear[761]{Sgaravatti:2013wu})}
  \end{quote}
  This is very similar to what is captured by a \requ{}.

  Also, points out only a test due to implications.
  For us, \requ{} is not a test.
  And, the implications are embraced.
  Though, differences limit these somewhat.

  For the moment, point with the implications is that this makes \zS{} fairly strong.
\end{note}

\paragraph{Pryor}

\begin{note}[\citeauthor{Pryor:2004ws}'s Type 4]
  An instance of a limitation arising from assuming that the possibility obtains is the fourth type of dependence between premise and conclusion considered by \citeauthor{Pryor:2004ws}.

  \begin{quote}
    [Type 4] dependence between premise and conclusion is that the conclusion be such that evidence \emph{against it} would (to at least some degree) undermine the kind of justification you purport to have for the premises.\nolinebreak
    \mbox{}\hfill\mbox{(\citeyear[359]{Pryor:2004ws})}
  \end{quote}

  Again, plausible.\nolinebreak
  \footnote{
    A variant of \citeauthor{Pryor:2004ws}'s Type 4 dependence is~\citeauthor{Jackson:1984vk}'s account of circularity.
    \begin{quote}
      [I]t may be that a given argument to a given conclusion is such that anyone --- or anyone sane --- who doubted the conclusion would have background beliefs relative to which the evidence for the premises would be no evidence.\space \dots

      Such an argument could be of no use in convincing doubters, and is most properly said to beg the question.\nolinebreak
      \mbox{}\hfill\mbox{(\Citeyear[111-12]{Jackson:1984vk})}
    \end{quote}
    Still, in contrast to \citeauthor{Pryor:2004ws}'s Type 4, \citeauthor{Jackson:1984vk}'s account of circularity is dialectical.
    Indeed, on \citeauthor{Jackson:1984vk}'s account (without additional constraints on when an agent has justification or evidence) it need not be the case that the agent's own justification would be undermined by someone doubting the conclusion.
    In this respect, \ideaCS{} is further distinguished from a proposal such as \citeauthor{Jackson:1984vk}'s as \ideaCS{} makes mention only of the relevant agent's epistemic state and reasoning.
  }
  Further, weaken from justification to any reasoning.
  In this respect, motivated by \ideaS{}, plausibly.
  However, much stronger.
  \ideaS{} is just about entertaining.
  Subjunctive with stronger is less clear.

  Issue:
  \begin{enumerate}
  \item Evidence undermines the kind of justification the agent purports to have for the premises.
  \end{enumerate}

  And, as \citeauthor{Pryor:2004ws} notes, \emph{kind} is important.
  However, it seems kind is not the only problem.
\end{note}

\begin{note}
  \citeauthor{Pryor:2004ws}'s argument that type 4 over-generates is somewhat interesting.
  Details are in the following footnote.\footnote{
  Compatible with \citeauthor{Pryor:2004ws}'s objection to type 4 dependence.

  % \begin{illustration}
    % \mbox{}
    % \vspace{-\baselineskip}
    \begin{quote}
      Suppose you're watching a cat stalk a mouse. Your visual experiences justify you in believing:

      \begin{enumerate}[label=(\arabic*), ref=(\arabic*)]
        \setcounter{enumi}{10}
      \item
        \label{illu:Pryor:cat:1}
        The cat sees the mouse.
      \end{enumerate}

      You reason:

      \begin{enumerate}[label=(\arabic*), ref=(\arabic*), resume]
      \item
        \label{illu:Pryor:cat:2}
        If the cat sees the mouse, then there are some cases of seeing.
      \item
        \label{illu:Pryor:cat:3}
        So there are some cases of seeing.\nolinebreak
        \mbox{}\hfill\mbox{(\citeyear[361]{Pryor:2004ws})}
      \end{enumerate}
    \end{quote}
  % \end{illustration}

  Setting aside whether this is fine.

  Following \citeauthor{Pryor:2004ws}:

  Bad, given proposal, as if no cases of seeing, then the cat is not seeing. (\citeyear[361]{Pryor:2004ws})

  \citeauthor{Pryor:2004ws}'s position is as follows:

  \begin{quote}
    I don't think you need antecedent justification to believe \ref{illu:Pryor:cat:3}, before your experiences can give you justification to believe \ref{illu:Pryor:cat:1}.
    I also think it's plausible that your perceptual justification to believe \ref{illu:Pryor:cat:1} contributes to the credibility of \ref{illu:Pryor:cat:3}.\nolinebreak
    \mbox{}\hfill\mbox{(\citeyear[361]{Pryor:2004ws})}
  \end{quote}

  This may be compatible with \ideaS{} and \ideaCS{}.
  With \ideaCS{}, somewhat trivial, if \ref{illu:Pryor:cat:3} holds throughout \epVW{1}.

  More generally, weaker proposition.
  Hence, it seems \indicateV{1}.
  So there's no issue with the reasoning.
  However, `contributes to the credibility\dots'.
  }
\end{note}

\begin{note}[Issue]
  Somewhat similar to above.
  Here, however, role of novel information is of interest.
  Hence, dynamic.
  And, \csN{} is, in this respect, static.
\end{note}


\paragraph{Others}

\begin{note}
  This also extends to \citeauthor{Wright:2011wn}.
  For, \citeauthor{Wright:2011wn} relies on the idea of doubt.

  The issue here is what is required in order to doubt.
  One may need to revise one's epistemic state.

  Of course, if idea of claiming support is taken generally, then it should be the case that for any \epPW{}, it is possible for the agent to conclude from reasoning that \(\phi\) having value \(v\) holds for any \epVAd{} \world{}.

  So, if satisfy claiming support, then may satisfy doubt idea.
  However, ideal.
  Pointing out the issue here does not require such a general thing as doubt.
\end{note}

\begin{note}
  Instead, as \(\psi\) not having value \(v'\) is an \ep{}, it is possible that \(\psi\) does not have value \(v'\).
  And, if \(\psi\) does not have value \(v'\), then step \(\delta'\) does not apply to how things are.
  Hence, observing that \(\psi\) having value \(v'\) follows in turn from the conclusion of step \(\delta'\) (together with other premises) is uninformative about how things are.
\end{note}

\begin{note}
  \color{red}
  Some of the \citeauthor{Wright:2011wn} cases are interesting.
  Especially the twin cases.
  In fact, especially this idea that situations are identical.
  For, one way of understanding this is that the agent makes a choice between two disjuncts, and it is possible for the agent to make the other choice, and then come to a different conclusion.
\end{note}

\subsection{Summarising}

%%% Local Variables:
%%% mode: latex
%%% TeX-master: "master"
%%% End:


\chapter{Overview of Tension}
\label{sec:tension}

\subsection{Sketch of tension}
\label{sec:overview-tension}

\begin{note}
  Test
\end{note}

\begin{note}[Ability]
  Here, our interest turns to ability.
  I take it as given that there are various instances of concluding \(\pv{\phi}{v}\) from \(\Phi\) for which the reasoning from \(\Phi\) to \(\pv{\phi}{v}\) is an instance of a general ability.

  For example, I conclude \(31 + 53 = 84\) from some premises, and the reasoning falls under my general ability to perform (simple) arithmetic.
  The specifics may differ, but there is sufficient overlap with concluding \(43 + 81 = 123\) and \(91 + 54 = 145\) to consider the reasoning of the same type.
  Indeed, \(532 - 91 = 441\), \(19 * 32 = 608\), and \(126/36 = 3.5\) may also fall under the same (general) ability.
  In other words, in concluding each equation I witness a specific instance of the general ability.

  Likewise, one may have the (general) ability to solve chess problems, complete \(\{ \text{Sudoku}, \text{KenKen}, \text{Nonogram}, \dots\}\) puzzles, or parse sentences in a given language.

  I would not have concluded \(31 + 53 \ne 84\) and you would not have failed to identify the relevant winning strategy of some chess problem.

  However, in each of the examples of ability noted, every other specific instance of the general ability functions as an independent check on whether one has the relevant ability.
  I should make no mistake about \(85 + 21\) and you should make no mistake with the next chess problem.
  Granting, of course, that we do have the relevant abilities.

  Hence, it seems clear that if an agent \csV{} for \(\pv{\phi}{v}\) when concluding \(\pv{\phi}{v}\) from \(\Phi\) and the agent's reasoning is a specific instance of a general ability, then there at least various \requ{1} associated with concluding \(\pv{\phi}{v}\) from \(\Phi\).

  More generally, concluding \(\pv{\phi}{v}\) from \(\Phi\) from reasoning which is the specific instance of a general ability leads to a \cluster{}.
  And, it is perhaps already intuitive that one does not witness reasoning from the premises of at least some proposition-value-premises pairing in the \cluster{}.

  For example, it seems plausible that the configuration the chess board for any given problem forms of a premise of the agent's reasoning, but so long as one has not seen the relevant problem, it seems implausible that one has witnessed reasoning that includes the relevant configuration.

  Of course, it may seem equally implausible that an agent concludes there is some winning strategy for some configuration of a chess board they have not yet seen.
  However, this intuition should be carefully examined.

  With an unopened chess book by a reputable author before you, I expect you have no problem concluding that each of the solutions are correct.
  {
    \color{red}
    Or, that granting ability, conclude you would (also) conclude winning strategy or not for each chess piece.
  }
  Yet, you have not yet seen any of the solutions.
  So, in general, there seems no issue with an agent concluding there is some winning strategy for some configuration of a chess board they have not yet seen.

  Of course, in the case of the book there is a key premise:
  The book is written by a reputable author.
  However, if you wish to conclude there is a winning strategy via your own reasoning, then the above considerations take effect.%
  \footnote{
    And, indeed, may already hold with respect to the key premises.
    For, so long as you hold you have the general ability to reason about chess problems of the relevant kind, you have an independent check on whether the author really is reputable.
  }
\end{note}

\chapter{Abstract tension}
\label{cha:tension-abstract}

\begin{note}[\requCluster{3}]
  We begin with the definition of a \cluster{}.
  \begin{definition}[A \requCluster{1}]
    \label{def:requCluster}
    Some collection of proposition-value-premises pairings \(\mathcal{C} = \{\pvp{\phi_{i}}{v_{i}}{\Phi_{i}}\}_{i}\) is a \emph{\cluster{}} with respect to some agent \vAgent{}('s epistemic state) just in case:
    \begin{itemize}
    \item
      For any \(\pvp{\phi_{i}}{v_{i}}{\Phi_{i}}\) in \(\mathcal{C}\), each \(\pvp{\phi_{j}}{v_{j}}{\Phi_{j}}\) in \(\mathcal{C}\) (such that \(j \ne i\)) is a \requ{} of \(\pvp{\phi_{i}}{v_{i}}{\Phi_{i}}\).
    \end{itemize}
    \vspace{-\baselineskip}
  \end{definition}

  Intuitively, a \cluster{0} is a collection of proposition-value-premises pairings such that every proposition-value-premises pairing is a \requ{} of every other proposition-value-premises pairing in the collection.
\end{note}

\begin{note}
  With the definition of a \cluster{} in hand, we are ready to state our first proposition.

  \begin{proposition}[\cluster{3} and \zetaS{}]
    \label{prop:cluster:csN}
    Suppose \(\mathcal{C}\) is a \requCluster{0} with respect to an agent \vAgent{}('s epistemic state).
    And, let \(\pvp{\phi}{v}{\Phi}\) be some proposition-value-premises pairing in \(\mathcal{C}\).

    \begin{itemize}
    \item
      \vAgent{} has \zetaS{} for \(\pv{\phi}{v}\) when concluding \(\pv{\phi}{v}\) from \(\Phi\) only if \emph{either}:
      \begin{itemize}
      \item
        For any \(\pvp{\psi}{v'}{\Psi}\) in \(\mathcal{C}\):
        \begin{itemize}
        \item \vAgent{} has at some point in the past concluded that \vAgent{} would conclude \(\pv{\psi}{v'}\) from \(\Psi\).
        \item
          When concluding \(\pv{\phi}{v}\) from \(\Phi\), \vAgent{} simultaneously concludes that \vAgent{} would conclude \(\pv{\psi}{v'}\) from \(\Psi\).
        \item
          \fc{2}.
        \end{itemize}
      \end{itemize}
    \end{itemize}
  \end{proposition}

  \autoref{prop:cluster:csN} follows directly from~\izetaS{} and~\autoref{def:requCluster}.

  \begin{argument}
    Suppose \(\mathcal{C}\) is a \requCluster{0} with respect to an agent \vAgent{}('s epistemic state).
    Let \(\pvp{\phi}{v}{\Phi}\) be some proposition-value-premises pairing in \(\mathcal{C}\).
    And, let \(\pvp{\psi}{v'}{\Psi}\) be some proposition-value-premises pairing in \(\mathcal{C}\).

    By~\autoref{def:requCluster} we have that \vAgent{} concluding that \vAgent{} would conclude \(\pv{\psi}{v'}\) from \(\Psi\) is a \requ{} of concluding \(\pv{\phi}{v}\) from \(\Phi\).
    And, by~\izetaS{}, it must be the case that either:
    \begin{itemize}
    \item
      \vAgent{} has concluded that they would conclude \(\pv{\psi}{v'}\) from \(\Psi\), ref{idea:Zs:overview:requ-sat:Past}.
      Or,
    \item
      In concluding \(\pv{\phi}{v}\) \vAgent{} simultaneously concludes \(\pv{\psi}{v'}\) from \(\Psi\), ref{idea:Zs:overview:requ-sat:Pres}.
    \end{itemize}
    Hence, we have established~\ref{prop:cluster:csN}.
  \end{argument}
\end{note}

\paragraph*{Examples}

\begin{note}[Examples of \requCluster{0}]
  Focused primarily on \scen{1} where we don't have a \requCluster{}.
  Lots of \emph{only if} conditionals.
  In some of these \scen{0}, plausible the conditional goes both ways.
  Hence, get a \requCluster{}.

  More generally, fairly mundane.
  Basic abilities.
\end{note}

\begin{note}[Ability]
  Ability to reason in certain ways leads to \cluster{1}.
  However, not interested in ability in general, but rather relatively simple instances of ability concerning specific problem types.

  Arithmetic.
  Sudoku.
  Chess.

  Indeed, the latter pair for \requCluster{1}.
  For, different starting positions.
\end{note}

\begin{note}
  Finally, ability.

  General ability, specific ability.

  Claim support for having some general ability.

  Now, here, simple cases.
  Basic arithmetic.
  Sudoku puzzles.
  Chess problems with winning strategies.

  Roughly the same.
  More broadly:

  Logic problems.

  Crossword.

  Reading novels up to a certain level.
  Here, if you can't read, then the writing is bad.

  Fluency.

  So, specific instances of the general ability.
\end{note}

\subsection{Observations about \requCluster{}}
\label{sec:observ-about-requcl}


\begin{note}
  \begin{proposition}
    \label{prop:cluster:simul}
    Need to do everything in a cluster at the same time.
  \end{proposition}

  \begin{argument}
    Straightforward.
    For, anything is a \requ{} for any other.
    So, only \csV{} at the same time.
  \end{argument}
\end{note}

\begin{note}[No \(\gamma\)]
  Consequence:

  \begin{corollary}
    \label{prop:cluster:no-general}
    No general \(\pvp{\gamma}{v}{\Gamma}\) within cluster.
  \end{corollary}

  \begin{argument}
    Quickly, because of~\ref{prop:cluster:simul}.
    Only \(\gamma\) at same time as others.

    In more detail.
    For, by assumption, \requ{} means that it's possible for the agent to conclude.
    By the each other \requ{} functions as a check on \(\gamma\).
    If haven't figured out each individual, then the general is in question.
  \end{argument}
\end{note}


\begin{note}
  Indeed, narrow interest to \ragCluster{1}.

  \begin{definition}[\ragCluster{3}]
    For any cluster \(\mathcal{C}\), \(\mathcal{C}\) is a \emph{\ragCluster{}} if and only if:
    \begin{enumerate}
    \item
      There is some \(\pvp{\phi_{i}}{v_{i}}{\Phi_{i}}\) and \(\pvp{\phi_{j}}{v_{j}}{\Phi_{j}}\) such that \(\Phi_{i}\) and \(\Phi_{j}\) do not overlap.
    \end{enumerate}
    \vspace{-\baselineskip}
  \end{definition}

  A \ragCluster{}, then, is just a cluster where at least some distinct premises.
  Hence, avoid issue where same premises allow simultaneous conclusion, and fail to establish tension with \ESU{}.

  \begin{proposition}
    With \ragCluster{} concluded previous or violate \ESU{}.
  \end{proposition}

  Now, some caution.
  It may be the case that reason from some non-minimal collection of premises.
  Hence, some care when establishing \ragged{}.
  This means, argue that \cluster{} \emph{and} argue \cluster{} is \ragged{}.
  Again, without any clear bounds on premises, this argument is non-deductive.
  However, plausible in various cases.
\end{note}

\begin{note}[Relative \jag{1}]
  Given importance of \ragged{} and specific proposition-value-premises pairings of \ragged{}, terminology:

  \begin{definition}[Relative \jag{1} of a \ragCluster{}]
    \(\mathcal{C}\) some \ragCluster{}.
    \(\pvp{\psi}{v'}{\Psi}\) is a \emph{\jag{0}} relative to \(\pvp{\phi}{v}{\Phi}\) if \(\Psi\) differs from \(\Phi\).
  \end{definition}
\end{note}

\begin{note}
  \begin{proposition}
    If \ragCluster{} and no prior conclusion for some \jag{}.
    Either:
    \begin{itemize}
    \item
      \ESU{} does not hold in general.
    \item
      No \csVImp{} for any proposition-value-premises pairing in cluster.
    \end{itemize}
      \begin{argument}
    More-or-less immediate from previous.
  \end{argument}
  \end{proposition}
\end{note}

\begin{note}
  Abstract tension, then, follows if there are instances of \ragCluster{1} with no prior conclusion for some \jag{}.
\end{note}

\begin{note}[Pointed cluster]
    \begin{definition}[Pointed cluster]
    For any cluster \(\mathcal{C}\), \(\mathcal{C}\) is a \emph{pointed cluster} if and only if:
    \begin{enumerate}
    \item
      Some conclusions are the same.
    \end{enumerate}
    \vspace{-\baselineskip}
  \end{definition}
\end{note}

\section{(Aside) Circularity}
\label{sec:aside-circularity}

\paragraph{\citeauthor{Sgaravatti:2013wu} on circularity}

{
  \color{red}
  The main interest of this is simultaneous conclusions.
  Here, pointing out that we don't get circularity.
}

\begin{note}
  \izetaS{} is not about circular reasoning in the sense that the term `circularity' suggests that the reasoner has taken the conclusion of the reasoning for granted.

  Indeed, \iRequ{} is constructed in such a way that rules out this possibility.
  Though, not \emph{to} rule out this possibility --- worry is rather\dots

  Of interest because some similarity.
\end{note}

\begin{note}
  Consider what \citeauthor{Sgaravatti:2013wu} terms the `Justification Account' of circularity.\nolinebreak
  \footnote{
    As \citeauthor{Sgaravatti:2013wu} notes, the Justification Account of circularity is a rewriting of the third type of `epistemic dependence' considered by \citeauthor{Pryor:2004ws}~(\citeyear[359]{Pryor:2004ws}).
    Neither \citeauthor{Pryor:2004ws} nor \citeauthor{Sgaravatti:2013wu} endorse the Justification Account, but I take the spirit of the account to sufficient for interest.
    Still, the considerations which follow also apply to distinguish the {\color{red} problem identified} from \citeauthor{Sgaravatti:2013wu}'s favoured account (\citeyear[\S3]{Sgaravatti:2013wu}) and the fifth type of `epistemic dependence' considered by \citeauthor{Pryor:2004ws}~(\citeyear[359]{Pryor:2004ws}).
  }

  \begin{quote}
    \begin{enumerate}[label=(JA), ref=(JA)]
    \item
      \label{sg:JA}
      An argument is circular if and only if for you to have justification to believe the premisses, it is necessary that you have justification to believe the conclusion.%
      \mbox{}\hfill\mbox{(\citeyear[754]{Sgaravatti:2013wu})}
    \end{enumerate}
  \end{quote}
  Where `justification to believe' is to be read as in terms of having formed the belief in an epistemically appropriate way as opposed to (merely) possessing sufficient resources to form formed the belief in an epistemically appropriate way.\nolinebreak
  \footnote{
    Or, however you prefer to characterise \citeauthor{Firth:1978vi}'s (\citeyear{Firth:1978vi}) distinction between doxastic and propositional justification (or warrant).
    See also \citeauthor{Silva:2020aa} (\citeyear{Silva:2020aa}) --- esp.\ fn.\ 1.
  }
  (\citeyear[754--755]{Sgaravatti:2013wu})
\end{note}

\begin{note}[\citeauthor{Sgaravatti:2013wu} on necessity]

  \begin{quote}
    For my present purposes it will suffice to say that a good test of A's being necessary for B (and thus of B's being sufficient for A) is the satisfaction of two subjunctive conditionals.
    First, if A did not hold, B would not hold; secondly, if B were to hold, A would hold.%
    \mbox{}\hfill\mbox{(\citeyear[761]{Sgaravatti:2013wu})}
  \end{quote}
  This is very similar to what is captured by a \requ{}.

  Important difference is the second subjunctive conditional.
  This does not need to hold.

  However, does in various cases.
  And, as we will see, in important case.

  \emph{Still, in general this is not a premise conclusion relation}
  But, in certain cases it is?
  No, in general it can't be, because then would have a failing of a \requ{}.
  For, conclusion would need to be premise and vice-versa.
\end{note}

\chapter{Resolving tension}
\label{cha:overview:resolving-tension}

\begin{note}
  Note, the tension is not about whether \(\phi\) has value \(v\).
  Instead, the tension is about whether the agent would have a certain property if they were to conclude \(\phi\) has value \(v\).
  Property of having claimed \support{}.
  Expanded, property of holding that any independent check is satisfied.
  Any other reasoning about whether \(\phi\) has value \(v\) would conclude \(\phi\) has value \(v\).
\end{note}

\subsubsection{Resolving tension by additional ideas}
\label{sec:resolv-tens-addit}

\begin{note}[Strong closure]
  So, we have weak constraints on concluding.
  Is there a way to keep {\color{red} witnessing} by strengthening closure?

  The idea is that \zS{} relies on the possibility of an independent check.
  However, strong closure leaves open the option for denying independence.

  In certain cases, this seems viable.
  Arithmetic.
  Perhaps this does give everything.

  However, the other cases are more challenging.
  For, in these cases, specific premises.

  Chess.
  Winning strategy from board.
  So, then need to conclude would conclude winning strategy from board.
  Now, idea is that understanding basics already give you this.
  Therefore, as the reasoning from the board requires understanding rules, it also follows that before concluding from board, have already concluded winning strategy from board.

  Stepping back, relevant instance of reasoning requires certain premises.
  Possible obtaining those premises already involves concluding various things.
  If some of those conclusions are that one would not fail to conclude \dots
  Then, there is no space for failure to witness a \requ{1} of the relevant kind.

  For, there will be no `gap' between premises and conclusions of interest.
  Hence, no independent check on whether one would get conclusion from premises.

  This is really strong closure though.
  Intuitively, there is some gap between introduction and understanding.

  And this, I don't see as genuinely viable.
\end{note}

\subsection{Observations}
\label{sec:overview:observations}

\subsubsection{Ability}

\begin{note}
  \begin{figure}[H]
    \centering
    \saMtxInterpreted{}
    \caption{Distinction matrix with \aben{the}}
    \label{fig:saMtxInterpreted:outline}
  \end{figure}
\end{note}

\begin{note}
  Recap.

  Claiming support.
  Constraint.

  Ability.
  In order to be compatible, satisfy constraint.
  Either of three options.
  Basic, ignore this.
  Property. Incompatible with constraint.
  Witness. Compatible.

  Here, display the matrix.
  I think this is the easiest way to visualise what is going on.
\end{note}



%%% Local Variables:
%%% mode: latex
%%% TeX-master: "master"
%%% End:


\appendix

\part{Appendices}

\chapter{Two ways of concluding}
\label{chap:twoc}

\begin{note}[Goal]
  The goal here is to motivate a distinction between how and why.

  Introduce idea of an \itp{}.

  Suggest the possibility of negative resolution to \issueInclusion{}, though not to \issueConstraint{}.

  In part, motivation for focusing on \issueConstraint{}.

  However, sufficient motivation as negative to \issueConstraint{} entails negative to \issueInclusion{}.
\end{note}

\begin{note}[Two ways of concluding]
  In this section, broad idea.
  \adA{} and \adB{}.
\end{note}

\section{Broad}
\label{chap:twoc:broad}

\subsection{The first type of reasoning: \adA{}}

\begin{note}
  \begin{restatable}[\adA{}]{definition}{defADA}
    \label{AR:adA}
    \label{def:adA}
    \vAgent{} concludes \(\pv{\phi}{v}\) from \(\Phi\) by `\adA{}' if:
    \begin{enumerate}[label=\textsf{S:\arabic*}., ref=(\textsf{S}:\arabic*)]
    \item
      \label{def:adA:psi}
      \vAgent{} concludes \(\pv{\phi}{v}\) by witnessing reasoning from some pool of premises \(\pv{\phi_{1}}{v_{1}},\dots,\pv{\phi_{k}}{v_{k}}\)
    \item
      \(\Phi\) is the collection of all and only the premises \(\pv{\phi_{i}}{v_{i}}\).
    \end{enumerate}
    \vspace{-\baselineskip}
  \end{restatable}

  \adA{} is straightforward.

  As before, concern may be raised about what the relevant premises are, and whether the relevant agent identifies those premises as premises.
  However, granting that an agent always concludes from some collection of premises, the relevant collection exists.

  The restriction to the \emph{exact} collection of premises the agent reasons from is for convenience.
  Nothing in particular hangs on this distinction, but equally nothing much is gained by allowing the inclusion of redundant proposition-value pairs.
\end{note}

\begin{note}[\illu{1}]
  Time from positions of hands on a clock and understanding of how time is represented by such a clock.

  Whether to make a bet from tolerance for risk, distribution of cards in a pack, cards in hand, and rules of the game.
\end{note}

\begin{note}
  \adA{} does not outline a specific way of reasoning.
  Deductive, inductive, etc.\
\end{note}

\begin{note}
  \phantlabel{abstract-adA}
  Basic (abstract) instance of \adA{}:

  {
    \small
    \begin{enumerate}[label=\arabic*., ref=\arabic*, noitemsep]
    \item\label{def:adA:ex:C:Cp} I have concluded \(\phi\) has value \(v\).
    \item\label{def:adA:ex:C:p} So, \(\phi\) has value \(v\). \hfill(From~\ref{def:adA:ex:C:Cp})
    \item\label{def:adA:ex:C:Cps} Likewise, I have concluded \(\psi\) has value \(v'\) when \(\phi\) has value \(v\).
    \item\label{def:adA:ex:C:ps} So, \(\psi\) has value \(v'\) when \(\phi\) has value \(v\). \hfill(From~\ref{def:adA:ex:C:Cps})
    \item\label{def:adA:ex:C:T} If \(\psi\) has value \(v'\) when \(\phi\) has value \(v\) and \(\phi\) has value \(v\), then it must be the case that \(\psi\) has value \(v'\). \hfill (From understanding of `if\dots then\dots')
    \item\label{def:adA:ex:C:s} Hence, \(\psi\) has value \(v'\).\newline
      \mbox{}\hfill (From \ref{def:adA:ex:C:p},~\ref{def:adA:ex:C:ps}~and~\ref{def:adA:ex:C:T})
    \item Therefore, I conclude \(\psi\) has value \(v'\). \hfill (From \ref{def:adA:ex:C:Cp} -- \ref{def:adA:ex:C:s})
    \end{enumerate}
  }
  From this reasoning, two clear premises.
  \(\pv{CS(\pv{\phi}{v})}{\top}\) and \(\pv{CS(\pv{\pv{\phi}{v} \Rightarrow \pv{\psi}{v'}}{\top})}{\top}\).
  Witness reasoning from these premises.
  The reasoning is verbose, premises are that the agent has concluded.
  Here, concluding is not factive.

  % (Consider parallel reasoning with knowledge.%
  % \footnote{The parallel reasoning in full:
  %   \begin{enumerate}[label=\arabic*., ref=\arabic*]
  %   \item\label{def:adA:ex:K:Kp} I know \(\phi\) has value \(v\).
  %   \item\label{def:adA:ex:K:p} So, \(\phi\) has value \(v\). \hfill (From~\ref{def:adA:ex:K:Kp})
  %   \item\label{def:adA:ex:K:Kps} I know \(\psi\) has value \(v'\) when \(\phi\) has value \(v\).
  %   \item\label{def:adA:ex:K:ps} So, \(\psi\) has value \(v'\) when \(\phi\) has value \(v\). \hfill(From~\ref{def:adA:ex:K:Kps})
  %   \item\label{def:adA:ex:K:T} If \(\psi\) has value \(v'\) when \(\phi\) has value \(v\) and \(\phi\) has value \(v\), then it must be the case that \(\psi\) has value \(v'\). \hfill (From understanding of `if\dots then\dots')
  %   \item\label{def:adA:ex:K:s} Hence, \(\psi\) has value \(v'\). \hfill (From \ref{def:adA:ex:C:p},~\ref{def:adA:ex:C:ps}~and~\ref{def:adA:ex:C:T})
  %   \item So, I know \(\psi\) has value \(v'\) as \(\psi\) having value \(v'\) follows from~(\ref{def:adA:ex:K:Kp}) and~(\ref{def:adA:ex:K:Kps}).
  %     \mbox{}\hfill (From \ref{def:adA:ex:K:Kp} -- \ref{def:adA:ex:K:s})
  %   \end{enumerate}
  % }%
  % )
\end{note}


\subsection{The second type of reasoning: \adB{}}

\begin{note}[Turning to \adB{}]
  We now turn to the second type of reasoning: `\adB{}'.

  We begin with a definition of \adB{}.
  However, our attention will quickly turn to a pair of helper definitions which relate some proposition-value pair we identify as an `\itp{}' to some other proposition-value pair and pool of premises.
\end{note}

\begin{note}
  \begin{restatable}[\adB{}]{definition}{defADB}
    \label{def:adB}
    \vAgent{} concludes \(\pv{\phi}{v}\) from \(\Phi\) by `\adB{}' if:
    \begin{enumerate}[label=\textsf{I}:\arabic*., ref=(\textsf{I}:\arabic*)]
    \item
      \label{def:adB:itp}
      \vAgent{} has concluded \(\pv{\mu}{v}\) and  \(\pv{\mu}{v}\) is either:
      \begin{enumerate}
      \item
        \label{def:adB:itp:between}
        An \itp{} \emph{between} \(\pv{\phi}{v}\) and \(\Phi\), or:
      \item
        \label{def:adB:itp:for}
        An \itp{} \emph{for} \(\pv{\phi}{v}\), with \(\Phi\) as the relevant pool of premises.
      \end{enumerate}
    \item
      \label{def:adB:conclude}
      \vAgent{} concludes \(\pv{\phi}{v}\) by appeal to the premises \(\pv{\phi_{1}}{v_{1}},\dots,\pv{\phi_{k}}{v_{k}}\) from the pool of premises \(\Phi\) via the possibility of witnessing the relevant reasoning from \(\pv{\mu}{v}\).
    \end{enumerate}
    \vspace{-\baselineskip}
  \end{restatable}
\end{note}

\begin{note}
  Without definitions of what an \itp{0} between \(\pv{\phi}{v}\) and \(\Phi\) is, or what an \itp{0} for \(\pv{\phi}{v}\) is,~\autoref{def:adB} is incomplete.
  We will shortly turn to the relevant helper definitions.

  Though, working backwards from~\autoref{def:adB} gives a hint.
  An \itp{} should contain information that it is possible for the agent to witness reasoning that conclude \(\pv{\phi}{v}\) from some pool of premises \(\Phi\).
\end{note}

\begin{note}
  Still, before turning to the pair of helper definitions, let me stress a key aspect of~\autoref{def:adB}:
  From~\ref{def:adB:conclude}, the agent concludes \(\pv{\phi}{v}\) from \(\Phi\) and \(\Phi\) alone (without witnessing the relevant reasoning).
  The agent does not conclude \(\pv{\phi}{v}\) from \(\Phi\) and \(\pv{\mu}{v}\).
  From the perspective of defining \adB{}, the latter option may be included, but the possibility of the former will be important when we turn to tension.
\end{note}

\subsection{Contrast}

\begin{note}
  The key difference between \adA{} and \adB{}:
  \begin{itemize}
  \item \adA{} involves the agent appealing to \(\pv{\mu}{v}\) in order to conclude \(\pv{\phi}{v}\), while
  \item \adB{} does not involve the agent appealing to \(\pv{\mu}{v}\) to conclude \(\pv{\phi}{v}\).
    Instead, the role of \(\phi\) is to highlight \(\rho_{1},\dots,\rho_{k}\) and the agent appeals to propositions \(\rho_{1},\dots,\rho_{k}\) to claim support for \(\psi\).
  \end{itemize}

  For the definition to be satisfied, \(\phi\) needs only be involved to the extent that it provides the link.
  Hence, \(\phi\) is not irrelevant.
  Still, the agent does not appeal to \(\phi\).
\end{note}

\paragraph*{\adB{}: Helper definitions}

\begin{note}[]
  We briefly noted, working backwards from~\autoref{def:adB}, that an \itp{} should contain information that it is possible for the agent to witness reasoning that conclude \(\pv{\phi}{v}\) from some pool of premises \(\Phi\).
  We now detail what an \itp{0}.
  Or, rather, what \itp{1} are.

  There are two cases.
  First, an \itp{0} \emph{between} \(\pv{\phi}{v}\) and \(\Phi\).
  Second, an \itp{0} \emph{for} \(\pv{\phi}{v}\).
  The distinction between these cases is whether the relevant \itp{0} identifies a particular pool of premises.

  In practice, we will blur the distinction, but from a definitional perspective the latter is best seen as a generalisation of the former.
\end{note}

\subparagraph*{An \itp{0} between \(\pv{\phi}{v}\) and \(\Phi\)}

\begin{note}[\itp{} between]
  \begin{definition}[An \itp{0} between \(\pv{\phi}{v}\) and \(\Phi\) \hfill \named{I.b}]
    \label{def:itp:b}
    \(\pv{\mu}{v}\) is an \itp{} \emph{between} \(\pv{\phi}{v}\) and \(\Phi\) if and only if:

    \begin{enumerate}[label=\arabic*., ref=\named{\textsf{I.b}:\arabic*}]
    \item
      \label{def:itp:b:pR}
      \(\mu\) having value \(v\) ensures:
      \begin{itemize}
      \item
        It is possible for \vAgent{} to conclude \(\phi\) has value \(v\) by witnessing reasoning from \(\Phi\) to \(\pv{\phi}{v}\), given \vAgent{}'s present epistemic state.
      \end{itemize}
    \item
      \label{def:itp:b:distinct}
      \(\pv{\mu}{v}\) is not equivalent to any \(\pv{\phi_{i}}{v_{i}}\), given \vAgent{}'s present epistemic state.
    \end{enumerate}
    \vspace{-\baselineskip}
  \end{definition}
\end{note}

\begin{note}[Plan]
  The definition of an \itp{} between \(\pv{\phi}{v}\) and \(\Phi\) consists of two components:~\ref{def:itp:b:pR} and~\ref{def:itp:b:distinct}.

  \ref{def:itp:b:pR} is the core of the definition, while~\ref{def:itp:b:distinct} narrows the definition to cases of interest.

  We begin by expanding on~\ref{def:itp:b:pR}, and then motivate the restriction given by~\ref{def:itp:b:distinct}.
\end{note}

\begin{note}[Expanding on~\ref{def:itp:b:pR}]
  Intuitively, think of an \itp{} between \(\pv{\phi}{v}\) and \(\Phi\) as a particular kind of conditional of the (rough) form `if \(\Phi\) then \(\pv{\phi}{v}\)'.

  Indeed, an `\emph{if} \dots \emph{then} \dots' statement between \(\Phi\) and \(\pv{\phi}{v}\) may be constructed from any \itp{} between \(\pv{\phi}{v}\) and \(\Phi\).

  For, if it is possible for an agent to conclude \(\phi\) has value \(v\) by witnessing reasoning from \(\Phi\) to \(\pv{\phi}{v}\), then (from the agent's perspective at least), \(\pv{\phi}{v}\) whenever \(\pv{\phi_{i}}{v_{i}}\) for each \(\pv{\phi_{i}}{v_{i}}\) in \(\Phi\).
  So, if every proposition \(\phi_{i}\) in \(\Phi\) has it's respective value \(v_{i}\), then \(\phi\) also has value \(v\).
  Or, more colloquially, if \(\Phi\) then \(\pv{\phi}{v}\).

  However, an \itp{} between \(\pv{\phi}{v}\) and \(\Phi\) is stronger than `if \(\Phi\) then \(\pv{\phi}{v}\)'.
  For, not only is it the case that \(\pv{\phi}{v}\) whenever \(\pv{\phi_{i}}{v_{i}}\) for each \(\pv{\phi_{i}}{v_{i}}\) in \(\Phi\), but in addition it is possible for the agent to conclude \(\pv{\phi}{v}\) from \(\Phi\) given the agent's present epistemic state.

  Naturally, the possibility for the agent to conclude \(\pv{\phi}{v}\) from \(\Phi\) goes beyond a plain conditional between \(\pv{\phi}{v}\) and \(\Phi\).

  Breaking down \autoref{def:itp:b:pR}, observe we have an `inner' statement:
  \begin{quote}
     It is possible for \vAgent{} to conclude \(\phi\) has value \(v\) by witnessing reasoning from \(\Phi\) to \(\pv{\phi}{v}\).
  \end{quote}
  And, a qualifier:
  \begin{quote}
    [G]iven \vAgent{}'s present epistemic state.
  \end{quote}

  The statement is simple.
  The relevant possibility is just for the agent to conclude \(\pv{\phi}{v}\) from \(\Phi\) by an instance of \adA{}.
  Indeed, the relevant instances of \EAS{} we motivate by developing tension will always involve an \itp{}, and hence will always involve the possibility of witnessing reasoning to the relevant conclusion from some pool of premises.

  Turn now to the qualifier:
  \begin{quote}
    [G]iven \vAgent{}'s present epistemic state.
  \end{quote}
  This is a qualifier on possible witnessing.

  Generally speaking, it may be possible for an agent to conclude \(\pv{\phi}{v}\) from \(\Phi\) by an instance of \adA{} from a distinct epistemic state.
  For example, if the agent were to learn some \(\pv{\phi_{i}}{v_{i}}\) in \(\Phi\) is the case, or if the agent were to improve their reasoning skills.
  However, the innermost qualifier ensures that it possible for an agent to conclude \(\pv{\phi}{v}\) from \(\Phi\) without any revision to the agent's epistemic state.

  An important consequence of this qualifier is that the agent must already hold that for each \(\pv{\phi_{i}}{v_{i}}\) in \(\Phi\), \(\phi_{i}\) has value \(v_{i}\).
  For, if not, then \(\pv{\phi_{i}}{v_{i}}\) would not be available as a premise.
  Of course, the definition of an \itp{} between \(\pv{\phi}{v}\) and \(\Phi\) may be given without this assumption, but we have no use for any more general definition.
\end{note}

\begin{note}[\illu{2}]
  \color{red}
    For example, consider being informed that the first player in a game of tic-tac-toe may always guarantee a draw.
  No premises are specified, but on reflection it is clear to see that one may reason through all the possible games to identify the strategy.
  The relevant \itp{0}, then, is the combination of the novel information and one's understanding of tic-tac-toe, the premises, some general results about tic-tac-toe, and the conclusion the required strategy.

  What guarantees the possibility of concluding --- general properties of tic-tac-toe which follow from the rules --- is intuitively distinct from the relevant pool of premises, which likely be limited to the rules themselves combined with the \dots

\end{note}

\begin{note}
  A quick observation.

  \(\pv{\mu}{v}\) being an \itp{} between \(\pv{\phi}{v}\) and \(\Phi\) depend on whether or not it is actually possible for the agent to conclude \(\pv{\phi}{v}\) from \(\Phi\), given their epistemic state.
  If it is not possible for the agent to witness reasoning from \(\Phi\) to \(\pv{\phi}{v}\), then no such \itp{} will exist.

  However, whether or not an agent \emph{concludes} \(\pv{\mu}{v}\) is an \itp{} between \(\pv{\phi}{v}\) and \(\Phi\) does not depend on whether or not it is actually possible for the agent to conclude \(\pv{\phi}{v}\) from \(\Phi\), given their epistemic state.
  We do not assume that an agent concludes \(\pv{\phi}{v}\) only if \(\phi\) actually has value \(v\).
  And, our interest with \itp{1} will typically be from the perspective of the agent's present epistemic state.
\end{note}

\begin{note}[Expanding on~\ref{def:itp:b:distinct}]
  The above has expanded on~\ref{def:itp:b:pR}.
  Finally, we turn to~\ref{def:itp:b:distinct}.

  In short,~\ref{def:itp:b:distinct} ensures that if \(\pv{\mu}{v}\) is an \itp{0} between \(\pv{\phi}{v}\) and \(\Phi\), then \(\pv{\mu}{v}\) is not a premise that the agent would appeal to when witnessing the reasoning from \(\Phi\) to \(\pv{\phi}{v}\) captured by~\ref{def:itp:b:pR}.

  More strictly, not only is \(\pv{\mu}{v}\) not a premise, but is not equivalent to any \(\pv{\phi_{i}}{v_{i}}\) in \(\Phi\).
  Where, again, equivalence is evaluated from the perspective of the agent.

  From an abstract perspective, if \(\pv{\mu}{v}\) is an \itp{0} between \(\pv{\phi}{v}\) and \(\Phi\), then \(\pv{\mu}{v}\) is purely descriptive of the relationship between \(\pv{\phi}{v}\) and \(\Phi\).

  In other words, \(\pv{\mu}{v}\) is not required to conclude \(\pv{\phi}{v}\) from \(\Phi\).

  Now,~\ref{def:itp:b:distinct} is a somewhat arbitrary restriction.

  In general, it is plausible that some \(\pv{\mu}{v}\) may both inform an agent that they may conclude \(\pv{\phi}{v}\) from \(\Phi\), but is also a member of \(\Phi\).

  Indeed, consider the conditional `if \(\pv{\alpha}{v}\) then \(\pv{\beta}{v'}\)'.
  Granting the conditional allows detachment, then it is surely possible for an agent to reason from the pool of premises \(\{\pv{\alpha}{v}, \text{if} \pv{\alpha}{v} \text{ then } \pv{\beta}{v'}\}\) to \(\pv{\beta}{v'}\).%
  \footnote{
    So long as the agent has already concluded \(\pv{\alpha}{v}\).
  }

  Indeed, without~\ref{def:itp:b:distinct}, \itp{1} would be abundant.
  Hence,~\ref{def:itp:b:distinct} narrows our attention to cases of interest, cases where \(\pv{\mu}{v}\) merely describes --- and does not partake --- in the reasoning of interest.%
  \footnote{
    Our course, ruling out an abundance of potential \itp{1} via~\ref{def:itp:b:distinct} carries risk of ruling out interesting \itp{1}.
    I encourage further investigation.
    Still, as \itp{1} are only of indirect interest, simplicity via arbitrary restrictions is favoured over complexity from general definitions.
  }
\end{note}

\begin{note}[Summary of \itp{0} between]
  \dots
\end{note}

\subparagraph*{An \itp{} for \(\pv{\phi}{v}\)}

\begin{note}[\itp{2} for]
  We now turn to the second helper definition, that of an \itp{0} for some proposition-value pair.
  In short, an \itp{0} \emph{between} \(\pv{\phi}{v}\) and \(\Phi\) ensures the possibility of the agent concluding \(\pv{\phi}{v}\) from \(\Phi\).
  And, by contrast, an \itp{0} \emph{for} \(\pv{\phi}{v}\) ensures there is some \(\Phi\) such that it is possible for the agent to conclude \(\pv{\phi}{v}\) from \(\Phi\).
\end{note}

\begin{note}[def: \itp{2} for]
  \begin{definition}[An \itp{0} for \(\pv{\phi}{v}\) \dots --- \named{I.f}]
    \label{def:itp:f}
    \(\pv{\mu}{v}\) is an \itp{} \emph{for} \(\pv{\phi}{v}\) if and only if:
    \begin{enumerate}[label=\arabic*., ref=\named{I.f:\arabic*}]
    \item
      \label{def:itp:f:pR}
      \(\mu\) having value \(v\) ensures there is some pool of proposition-value pairs \(\Phi\) and proposition-value pair \(\pv{\mu'}{v'}\) such that:
      \begin{itemize}
      \item
        \(\pv{\mu'}{v'}\) is an \itp{} between \(\pv{\phi}{v}\) and \(\Phi\).
      \end{itemize}
    \item
      \label{def:itp:f:distinct}
      \(\pv{\mu}{v}\) is not equivalent to any \(\pv{\phi_{i}}{v_{i}}\), given \vAgent{}'s present epistemic state.
    \end{enumerate}
    \vspace{-\baselineskip}
  \end{definition}
\end{note}

\begin{note}
  As with~\autoref{def:itp:b},~\autoref{def:itp:f} contains two components; a core and a restriction.
\end{note}

\begin{note}
  The core is straightforward.
  \ref{def:itp:f:pR} requires \(\pv{\mu}{v}\) to do ensure two things:
  \begin{enumerate}
  \item The existence of some pool of proposition-value pairs \(\Phi\), and
  \item The existence of an \itp{0} between \(\pv{\phi}{v}\) and \(\Phi\).
  \end{enumerate}
  In other words, then, an \itp{0} \emph{for} \(\pv{\phi}{v}\) just is the guarantee of an \itp{0} \emph{between} \(\pv{\phi}{v}\) and some pool of premises \(\Phi\).

  Simply as it may be, the definition of a \itp{0} for will prove quite useful, as we avoid the need to specify any particular pool of premises.

  Further, any \itp{0} between \(\pv{\phi}{v}\) and \(\Phi\) is always an \itp{0} for \(\pv{\phi}{v}\).
  For, suppose \(\pv{\mu}{v}\) is an \itp{0} between \(\pv{\phi}{v}\) and \(\Phi\).
  Then, \(\Phi\) is the relevant pool of premises and \(\pv{\mu}{v}\) itself is the relevant \(\pv{\mu'}{v'}\), and by definition \(\pv{\mu}{v}\) is not equivalent to any \(\pv{\phi_{i}}{v_{i}}\) in \(\Phi\), given the agent's present epistemic state.

  Note, however, that in general, if \(\pv{\mu}{v}\) is an \itp{0} for \(\pv{\phi}{v}\) then \(\pv{\mu}{v}\) need not be an \itp{0} between \(\pv{\phi}{v}\) and some pool of premises.
  Simply, though any \itp{0} between is also an \itp{0} for, the converse does not hold.
  For, an \itp{0} requires a pool of premises to be specified.
  Therefore, \(\pv{\mu}{v}\) and \(\pv{\mu'}{v'}\) must, in general, be distinct.

  Of course, an \itp{0} for \(\pv{\phi}{v}\) may have the same general statement as an \itp{0} between \(\langle \phi,\Phi \rangle\).

  Consider again tic-tac-toe.
  Above, we considered the \itp{0} between the existence of a strategy for first player in a game to guarantee a draw from the rules of tic-tac-toe.
  Though, with a moments reflection the statement that there existence of a strategy for first player in a game to guarantee a draw will typically lead to an \itp{0} for the existence of the relevant strategy.
  For, some premises must exist, and given the simplicity of tic-tac-toe these are surely within the grasp of an agent who understands the rules of tic-tac-toe.
\end{note}

\begin{note}
  Finally, the restriction~\ref{def:itp:f:distinct} functions in parallel to the restriction \ref{def:itp:b:distinct} of~\autoref{def:itp:b}.
  An \itp{0} for \(\pv{\phi}{v}\) is purely descriptive, and does not participate in concluding \(\pv{\phi}{v}\) from the relevant pool of premises.
  A more general definition may be given without this restriction, but such a definition is beyond present interest.
\end{note}

\paragraph*{\adB{}}

\begin{note}
  With the two helper definitions in hand, let us return to the definition of \adB{}.

  Recall two components:~\ref{def:adB:itp} and~\ref{def:adB:conclude}.

  \ref{def:adB:itp} stated that the agent has concluded \(\pv{\mu}{v}\) where \(\pv{\mu}{v}\) is either:
  \begin{enumerate}[label=(\alph*)]
  \item An \itp{} \emph{between} \(\pv{\phi}{v}\) and \(\Phi\), or
  \item An \itp{} \emph{for} \(\pv{\phi}{v}\), with \(\Phi\) as the relevant pool of premises
  \end{enumerate}
  We have seen the relevant definitions.

  So, given the agent has concluded \(\pv{\mu}{v}\), for some \itp{0} \(\pv{\mu}{v}\) then it is possible for the agent to conclude \(\pv{\phi}{v}\) from some pool of premises \(\Phi\).

  \ref{def:adB:conclude} is key.

  The agent concludes \(\pv{\phi}{v}\) by appeal to the pool of premises \(\Phi\) \emph{via} the possibility of witnessing the relevant reasoning from \(\Phi\) to \(\pv{\phi}{v}\) given by the \itp{0} \(\pv{\mu}{v}\).

  Hence, the agent does \emph{not} conclude \(\pv{\phi}{v}\) from \(\pv{\mu}{v}\) or indeed from some pool of premises for which \(\pv{\mu}{v}\) is a member.
  Indeed, the latter point follows from~\ref{def:itp:b:distinct} and~\ref{def:itp:f:distinct} --- an \itp{0} is always purely descriptive of some possible reasoning.

  Instead, the pool of premises the agent concludes \(\pv{\phi}{v}\) from is the collection of those premises that they agent would appeal to if they were to witness the relevant instance of reasoning given by the \itp{0}.

  Of course, the presence of an \itp{0} is, intuitively, crucial.
  For, without an \itp{0} the agent would lack information that it is possible to conclude \(\pv{\phi}{v}\) from some pool of premises.
\end{note}

\paragraph{A pair of \illu{3}}

\begin{note}
  To \illu{0} \adA{} and \adB{} we work through two \illu{1} in some detail.
  Both \illu{1} share two components:
  \begin{enumerate}[label=\alph*., ref=(\alph*)]
  \item
    \label{adX:illu:struc:mem}
    Memory of creating a syntactic proof for some first order formula.
  \item
    \label{adX:illu:struc:concl}
    Concluding that the relevant formula is a theorem of first-order logic.
  \end{enumerate}

  The key difference between the two \illu{1} is whether the memory~\ref{adX:illu:struc:mem} serves as a premise or an \itp{0} for the conclusion~\ref{adX:illu:struc:concl}.

\end{note}
\begin{note}[Two premises]
  \begin{quote}
    \begin{enumerate}[%
      label={(Mem)},%
      ref={(Mem)}%
      ]
    \item
      \label{ill:Eproof:mem}
      I remember having created a syntactic proof of \formula{\forall x Px \rightarrow \lnot \exists x \lnot P x} (using a sound first-order system).%
      \footnote{
        We use the phrasing `having created' rather than `creating' to imply completion.
      }
    \end{enumerate}
  \end{quote}
  And:
  \begin{quote}
    \begin{enumerate}[%
      label={(\(\exists\mathord{\vdash}{,}\top\))},%
      ref={(\(\exists\mathord{\vdash}{,}\top\))}%
      ]
    \item
      \label{ill:Eproof:def}
      The existence of a syntactic proof of a formula (using a sound first-order system) is sufficient to establish the formula is a (syntactic) theorem of first-order logic.
    \end{enumerate}
  \end{quote}
\end{note}

\paragraph{First \illu{0} (\adA{})}

\begin{note}
  \begin{illustration}[\adA{}]
    \label{ill:ad:proof:mem}
    \mbox{}
    \vspace{-\baselineskip}
    \begin{enumerate}[%
      label=\arabic*.,%
      ref=({I}.{\ref{ill:ad:proof:mem}}:\arabic*)%
      ]
    \item
      \illEproofMem{} \hfill \ref{ill:Eproof:mem}
    \item
      \label{ill:Eproof:exP}
      So, there exists a syntactic proof of \formula{\forall x Px \rightarrow \lnot \exists x \lnot P x} (using a sound first-order system)
    \item
      \label{ill:Eproof:thm}
      Hence, by \ref{ill:Eproof:def}, \formula{\forall x Px \rightarrow \lnot \exists x \lnot P x} is a theorem of first-order logic.
    \end{enumerate}
    \vspace{-\baselineskip}
  \end{illustration}
\end{note}

\begin{note}[Discussion of \autoref{ill:ad:proof:mem}]
  I take \autoref{ill:ad:proof:mem} to be a straightforward case of concluding.
  \ref{ill:Eproof:mem}, memory,


  and \ref{ill:Eproof:def}, to recast the existence of a proof from \ref{ill:Eproof:mem} in terms of the formula being a theorem.

  Neither premise from anything more basic, and without either premise the conclusion would not be obtained.
  For, without \ref{ill:Eproof:mem} no proof, and without \ref{ill:Eproof:def} no recasting.
\end{note}

\begin{note}
  It seems sufficient, generally speaking, to conclude some proposition has value \(v\) by appeal to memory, hence the agent claims support that there was some event which culminated in a syntactic proof of the formula.

  Of course, the agent may have misremembered.
  Still, we do not require that any agent concludes \(\pv{\phi}{v}\) from \(\Phi\) only if \(\phi\) \emph{actually} has value \(v\) and each \(\pv{\phi_{i}}{v_{i}}\) in \(\Phi\), \(\phi_{i}\) \emph{actually} has value \(v_{i}\).

  Following, this allows the agent to conclude a syntactic proof of the formula exists.
  As before, the agent may have failed to \emph{actually} create a syntactic proof of the formula
  Still, from the same perspective this does not prevent the agent from concluding they did (actually) create such a proof.

  Hence, finally, the agent claims support that the formula is a (syntactic) theorem of first-order logic.
\end{note}

\begin{note}
  To concisely summarise, we may say that the agent conclude the is a (syntactic) theorem of first-order logic \emph{because} of their understanding of syntactic theorem-hood and their memory of proving the formula.

  For sure,~\autoref{ill:ad:proof:mem} is designed to be as straightforward as possible.
  Of interest is not whether the agent claims support, but how the role the agent gives to their memory in claiming support.

  The agent appeals to their memory to establish that there exists a syntactic proof of the formula, and then combines the existence of a syntactic proof with~\ref{ill:Eproof:def} to claim support that the formula is a theorem.
  Hence, the agent's memory is directly involved in their claimed support for the formula being a theorem.%
    \footnote{
      \color{red}
      Whether proving is an unsatisfied \requ{}.
      However, recall that allowed a \requ{} to be satisfied by some instance of concluding.
      And, memory of concluding.
      Still question about original proof, but no problem with memory.
      Also, method.
    }
\end{note}

\paragraph{Second \illu{0} (\adB{})}

\begin{note}
  \begin{illustration}[\adB{}]
    \label{ill:ad:proof:eve}
    \mbox{}
    \vspace{-\baselineskip}
    \begin{enumerate}
    \item \illEproofMem{} \hfill \ref{ill:Eproof:mem}
    \item
      \label{ill:ad:proof:eve:app}
      In creating the syntactic proof I appealed to various aspects of some sound first-order system.
    \item
      \label{ill:ad:proof:eve:pos}
      As I created a proof, those various aspects of the sound first-order system are sufficient to ensure there exists a proof.
    \item
      Hence, by \ref{ill:Eproof:def}, \formula{\forall x Px \rightarrow \lnot \exists x \lnot P x} is a theorem of first-order logic.
    \end{enumerate}
    \vspace{-\baselineskip}
  \end{illustration}

  {
    \color{red}
    Premises are rules are part of a sound system, may be combined in this way.%
    \footnote{
      Indeed, relative simplicity is why we chose a syntactic rather than semantic proof.
      With semantic, need an argument which covers all models, and while premises plausibly exist, these have no common codification.
    }
  }

  As with~\autoref{ill:ad:proof:mem}, the agent's memory has a role in~\autoref{ill:ad:proof:eve}, but the role is quite different.
  Above, the agent claimed support for the formula being a theorem primarily \emph{because} they remembered creating a proof.
  By contrast, here the agent claims support for the formula being a theorem primarily because of the properties of some sound first-order system.

  Step~\ref{ill:ad:proof:eve:app} appeals to various aspects of some sound first-order system and, in turn, step~\ref{ill:ad:proof:eve:app} observes that those aspects are sufficient to ensure a proof exists.
  The agent claims support for the existence of a proof by appeal to the various aspects of some first-order system they appealed to when constructing the proof, rather than their memory of constructing the proof.
\end{note}

\begin{note}
  To help clarify, let's fix a particular syntactic proof using the Fitch-style proof system of~\textcite[557--560]{Barwise:1999tu}:

  \begin{figure}[H]
    \centering
    \begin{quote}
      \fitchprf{}{
        \subproof{\pline[1.]{\forall x P x}}{
          \subproof{\pline[2.]{\exists x \lnot Px}}{
            \boxedsubproof[3.]{a}{\lnot Pa}{
              \pline[4.]{Pa}[\lalle{1}] \\
              \pline[5.]{\bot}[\lfalsei{3}{4}]
            }
            \pline[6.]{\bot}[\lexie{2}{3--5}]
          }
          \pline[7.]{\lnot \exists x \lnot Px}[\lnoti{2--6}]
        }
        \pline[8.]{\forall x Px \rightarrow \lnot \exists x \lnot Px}[\lifi{1--7}]
      }
    \end{quote}
    \caption{A syntactic proof}\label{fig:syntx-prf}
  \end{figure}

  The proof consists of single instances of five introduction or elimination rules.
  Each rule is part of the Fitch-style proof system, and the specific application of the rules constitute the proof.
\end{note}


\begin{note}[Before\dots]
  Before returning to~\autoref{ill:ad:proof:eve}, let us observe that with the proof in hand one may claim support that a proof of the formula exists via the contents of~\autoref{fig:syntx-prf}.

  Broadly stated:

  \begin{enumerate}
  \item The proof is constructed from a sound first-order proof system.
  \item And, the particular application of some rules of the system to formulae is such that the proof begins with no assumptions and the last line of the proof is not part of any assumption made during the course of the proof.
  \end{enumerate}
\end{note}

\begin{note}
  Note, appeal to creation of the proof involves appeal to various aspects of the Fitch-style proof system.

  The object itself is mute to whether or not it is a proof.

  For example, adding `\formula{Ba}' as an assumption would void the proof, but you would need to observe that the appeal to existential elimination on line 6 requires that `\formula{a}' does not appear in the proof prior to its introduction on line 3 in order to claim support that the proof is void.

  Indeed, the proof consists of eight steps, each step is permitted by the first-order system, the proof begins with no assumptions, the last line of the proof is not part of any assumption made during the course of the proof and the proof, and so on.

  Sparing the details, claimed support that~\autoref{fig:syntx-prf} is a syntactic proof of \formula{\forall x Px \rightarrow \lnot \exists x \lnot P x} from the creation of~\autoref{fig:syntx-prf} is a matter of claiming support for each step of the creation.

  Indeed, to spare the details in general, let us instead talk of some collection of propositions and steps of reasoning.
  Claiming support that a proof exists from the some creation in the way under discussion is an instance of reasoning from details of the creation to the conclusion that a proof exists.
  Hence, as an instance of reasoning involves certain premises and steps of reasoning.
  And, whatever these turn out to be, the proceed from the creation of the proof rather than from some other source such as memory, testimony, and so on.
\end{note}

\begin{note}
  In other words, one may claim support that a proof of \formula{\forall x Px \rightarrow \lnot \exists x \lnot P x} exists (primarily) \emph{because} of their reasoning from some collection of premises and steps of reasoning concerning the creation to the existence of a proof of \formula{\forall x Px \rightarrow \lnot \exists x \lnot P x}.
\end{note}

\begin{note}[Return to \ref{ill:ad:proof:eve}]
  Now let us return to the reasoning of~\autoref{ill:ad:proof:eve}, and in particular steps~\ref{ill:ad:proof:eve:app} and~\ref{ill:ad:proof:eve:pos}:
  \begin{quote}
    \begin{enumerate}
      \setcounter{enumi}{1}
    \item In creating the syntactic proof I appealed to various aspects of some sound first-order system.
    \item As I created a proof, those various aspects of the sound first-order system are sufficient to ensure there exists a proof.
    \end{enumerate}
  \end{quote}
  Given that the agent remembers having created a syntactic proof, the `various aspects of some sound first-order system' of step~\ref{ill:ad:proof:eve} may be taken as those aspects of the first-order system that were appealed to in the premises and steps of reasoning when the agent created the proof.
  And step \ref{ill:ad:proof:eve}, in turn, appeals to how those various aspects of some sound first-order system were sufficient for the agent to claim support that a proof exists by the reasoning that occurred.

  In short, the agent remembers creating a syntactic proof and claiming support that a proof exists from the creation.
  The instance of claiming support involved reasoning from premises via steps to the relevant conclusion.
  Hence, it is possible to claim support for the conclusion by those premises and steps of reasoning.
  So, in~\ref{ill:ad:proof:eve} the agent observes that those premises and steps of reasoning are sufficient to claim support by way of their memory, and in turn appeals to those premises and steps of reasoning to claim support for the relevant conclusion.
\end{note}

\begin{note}
  {
    \color{red}
    Propositional support.
    (If I talk about this, it should be after the definitions.)
  }
\end{note}

\begin{note}
  Generalising, the way in which the agent claims support in~\autoref{ill:ad:proof:eve} is of interest because the agent appeals to premises and steps of reasoning that are not `part' of their present reasoning.
  The role of memory in the \illu{0} is (merely) a way for the agent to recognise that there are such premises and steps of reasoning.
  And, in the definitions that follow, we will abstract from any particular way in which the recognises that relevant premises and steps of reasoning are available.

  Still, even though memory is contingent, we may briefly observe that the way in which the agent claim support in~\autoref{ill:ad:proof:eve} is compatible with \ESU{}.
  For, \ESU{} requires that an agent may claim support for some conclusion from premises and steps of reasoning only if the agent has witnessed reasoning to the conclusion from those premises via those steps of reasoning.
  So, if the initial instance of claiming support conformed to \ESU{} then the agent will have witnessed reasoning from those steps and premises to the conclusion --- the instance of claiming support in~\autoref{ill:ad:proof:eve} does not involve such witnessing, but the agent's memory would be about how the relevant premises and steps were used to claim support.

  Of course, the way in which the agent claim support in~\autoref{ill:ad:proof:eve} is incompatible with a strengthened variant of \ESU{} which requires the agent to use any premises and steps they appeal to in the \emph{present} instance of reasoning, but the point for the moment is that the way in which the agent claims support in~\autoref{ill:ad:proof:eve} does not already require what we are arguing against: \ESU{}.
\end{note}

\begin{note}
  \color{red}
  In this case, \adB{} is not incompatible with \ESU{}.
  For, the agent has witnessed reasoning (granting memory).
  So, \ESU{} does not lead to an immediate rejection of \adB{}.

  Oh, this is noted.
\end{note}

\subsubsection{Additional illustrations}

\begin{note}

  \begin{illustration}
    \mbox{}
    \vspace{-\baselineskip}
    \begin{itemize}
    \item If bag are overweight then they can't be taken on the flight.
    \item Machine reads\dots
    \item Bag can't be taken on the flight.
    \end{itemize}
  \end{illustration}
  Contents of the bag are overweight.

  Combined weight of the items versus the combination of the individual weights.

  Compare, filling the bag and weighing it, versus summing the weight of the items as you fill the bag.

  Now, seems possible to fill the bag and weight it, then appeal to the sum of the items.

  So, this is a little more subtle.
  The bag has been weighed, and the distinction is between the weight of the contents of the bag, and the combined weight of the items that make up the contents of the bag.

  This is particularly interesting.
  Because, it seems clear that something is strange if someone talks about the weight of the contents of the bag without recognising that this is a function of the combined weight of all the individual elements of the bag.
  However, no idea what the contents of the bag are.

  So, claiming support from what is has been observed, the combined weight, rather than what must be the case in order to have made the observation.
\end{note}

%%% Local Variables:
%%% mode: latex
%%% TeX-master: "master"
%%% End:


\chapter{Concluding}
\label{chapter:concluding}

\begin{note}
  Concluding has a central role in the main body of this document.

  Key assumption made is relation of support.

  This chapter details various other assumptions about concluding that have either an implicit or explicit role in the argument presented.
\end{note}

\section{Overview}
\label{sec:concluding:overview}

\begin{note}[Overview]
  Concluding.
  This is the most basic thing.
  Lay out assumptions as clearly as possible.

  Generally speaking, assumption does two things.
  What concluding is, what concluding is not.
  Assumptions will not provide an exhaustive account of what concluding is and what concluding is not, but sufficient for a main goal.

  Indeed, goal is a substantial result on what some instances of what concluding is.
  And, in turn, a substantial result on what concluding, in general, is not.

  And, as concluding instance of reasoning, the previous two points expand to reasoning in general.
\end{note}

\begin{note}[List of assumptions]
  Ideally, find some way to generate a toc for this section\dots
\end{note}


\paragraph{Values and propositions}

\begin{note}[Value proposition]
  Reasoning and claims to support focus.
  Briefly introduce a pair of propositions to clarify claim to support and reasoning.

  \begin{restatable}[Claimed support is for a proposition having some value]{assumption}{assuCSVP}
    \label{assu:CSVP}
    When an agent concludes \(\phi\) has value \(v\), the agent assigns value \(v\) to the \world{1} described by \(\phi\).
    (Where propositions individuate \world{1} from the perspective of the agent.)
  \end{restatable}

  \autoref{assu:CSVP} fixes terminology.
  To illustrate, when stating the conclusion of the reasoning sketched above we used the proposition that \emph{the area of the rectangle is \(133\text{cm}^{2}\)}.
  The proposition refers to the \world{} in which the area of the rectangle is \(133\text{cm}^{2}\), and speaking a little more precisely, the agent claimed that the proposition has the value `true' --- though it may be the value turns out to be `false'.
  Or, perhaps if the agent was a little unsure about the accuracy of the ruler, that the proposition has the value `likely', `probable', or some quantitative credence.
  And, some other instance of reasoning may have concluded that the proposition has the value `desirable' --- e.g.\ if the agent was searching for a rectangle of some approximate size.
\end{note}

\begin{note}[Notation]
  \begin{notation}[Proposition-value pair]
    Proposition-value pair, abbreviate \(\pv{\phi}{v}\).
  \end{notation}
\end{note}

\begin{note}
  Core idea is that claim of support is that the \world{} is a certain way.
  Proposition, what the thing is.
  Value, the way it is.

  A handful of instances:
  \begin{itemize}
  \item \(p\) is assigned the value `true'. \hfill (\emph{p} is true.)
  \item \(p\) is assigned the value `ought to be'. \hfill (\emph{p} ought to be the case.)
  \item \(p\) is assigned the value `desirable'. \hfill (\emph{p} is desirable.)
  \item \(p\) is assigned the value `improbable'. \hfill (\emph{p} is improbable.)
  \end{itemize}
\end{note}

\begin{note}
In most cases the value will be clear (i.e. that the proposition is true, though sometimes that the proposition is desirable), and so we will talk of claiming support for the proposition.
  A handful of additional examples will be provided when illustrating the next proposition.
\end{note}

\begin{note}
  Nothing hangs on distinction between values.
  Reduce everything to truth and falsity.
  However, we do not assume this, and if you do not think this is the case either, then I would like to not suggest that the assumptions and arguments to follow concern only those propositions which may be evaluated as true or false.
\end{note}

\begin{note}[Quick examples]
  {
    \color{red}
    Relate to some other assumption?
  }

  \begin{itemize}[noitemsep]
  \item \(S\) testified that \(p\), so \(p\) is true.
  \item \(p\) would satisfy every member of the group, so \(p\) ought to be the case.
  \item The song is produced by \(S\), so it is desirable that I listen to it.
  \item The device reads \(p\) and is reliable, so \emph{not}-\(p\) is improbable.
  \end{itemize}
\end{note}

\paragraph{Premises}

\begin{note}[Notation]
  \begin{notation}[Proposition-value-premise pairing]
    Proposition-value-premise pairing, abbreviate \(\pvp{\phi}{v}{\Phi}\).
  \end{notation}
  Have seen \(\Phi\) supports \(\pv{\phi}{v}\).
  Relation of support applies to \(\pvp{\phi}{v}{\Phi}\).

  However, more general use.
\end{note}


\begin{note}[Understanding `having value \(v\)']
  In a deductive case, if the premises are true, then the conclusion is true.
  Means-end reasoning for desire.
  The value is important.
  If it is true that it past 6pm, then it is true the shop is closed.
  Provides value of shop being closed.

  However, if agent desires that it is past 6pm, then it doesn't follow that the agent desires that the shop is closed.
  Question an agent as to why they think their desires conform to truth --- is-ought problem.

  Means-end reasoning.
  It is true that there is cheese at the centre of the maze.
  And, it is desirable that I obtain the cheese at the centre of the maze.
  Further, it is true that I may only obtain the cheese at the centre of the maze by solving the maze.
  Therefore, it is desirable that I solve the maze.
\end{note}

\paragraph{Three distinct options}

\begin{assumption}
  Fix \(\phi\), \(v\), and \(\Phi\).
  Three distinct options:
  \begin{enumerate}
  \item Conclude \(\phi\) has value \(v\) from \(\Phi\).
    \(\pvp{\phi}{v}{\Phi}\).
  \item
    Conclude \(\phi\) has some value other than \(v\) from \(\Phi\).
    \(\pvp{\phi}{\overline{v}}{\Phi}\).
  \item
    Fail to conclude whether \(\phi\) has value \(v\) from \(\Phi\).
    \(\pvp{\phi}{?}{\Phi}\).
  \end{enumerate}
  The second may be further distinguished.
  That \(\phi\) has some other value, and some specific value.
  We have no need for such a finer grained distinction.
\end{assumption}

\paragraph{Conclusing is not (necessarily) factive}
\label{concluding:not-factive}

\begin{note}
  Page~\pageref{mention:concluding-non-factive}, mentioned concluding \(\phi\) has value \(v\) doesn't matter if \(\phi\) has value \(v\).

  Brief \illu{0}.
\end{note}

\begin{note}
  \begin{illustration}
    Agent concludes \(0.999\dots \ne 1\), where the agent holds themselves to have a conventional understanding of real numbers.
  \end{illustration}
  The qualification is important, there are various interpretations under which \(0.999\dots \ne 1\), but it is convention that the Archimedean property holds for real numbers.

  Still, an agent may reason that even if \(0.999\dots = 1\), there must be \emph{some} difference between \(0.999\dots\) and \(1\) --- no matter how small --- and some difference between to things is sufficient to establish that they are not equal.

  Expanding: \(0.9 = (1 - 0.1)\), \(0.99 = (1 - 0.01)\), and so \(0.999\dots = (1 - 0.000\dots 1)\), hence \(1 = (0.999\dots + 0.000\dots 1)\), and because \(0.999\dots\) refers to some quantity, \(0.000\dots 1\) likewise refers to some quantity.

  In other worlds, an agent need not consider it \epPAd{} that the Archimedean property does not hold for real numbers.
  Of course, including either safety or sensitivity as additional constraints on \support{} would rule out this illustration, we do not require such constraints.
\end{note}

\paragraph{Safety and sensitivity}

\begin{note}[Safety and sensitivity]
  {
    \color{red}
    Perhaps good to mention here that \zS{} doesn't have this property either.
  }
  We borrow the following definitions from~\citeauthor{Zalabardo:2017td}:

  \begin{quote}
    S's belief that p is \emph{safe} just in case, if S believed p, p would be true.\newline
    \mbox{}\hfill\mbox{(\citeyear[1]{Zalabardo:2017td})}
  \end{quote}

  \begin{quote}
    S's belief that p is \emph{sensitive} just in case, if p were false, S wouldn't believe p.\nolinebreak
    \mbox{}\hfill\mbox{(\citeyear[2]{Zalabardo:2017td})}
  \end{quote}

  In both definitions, `S's \support{} for p (being true)' may be substituted for `S's belief that p'.

  Further, both definitions consist of a subjunctive conditional which concerns the value that p has.
  In the case of safety, p has the value true, and in the case of sensitivity, p is false.
  And, indeed, it follows from either definition that if S believes that p then p is true.\nolinebreak
  \footnote{
    In the case of safety, this is immediate.
    In the case of sensitivity, if S believes that p and p is false, then it follows by sensitivity that S doesn't believe p.
    Hence, either S believes that p and p is true, or S does not believe that p.

    Note, however, that while both safety and sensitivity ensure that S believes that p only if p is true, the two conditions are distinguished by which possibilities the relevant subjunctive antecedent quantifiers over.
  }
\end{note}

\paragraph{Conclusions are determined}

\begin{note}[Conclusions are determined]
  \begin{assumption}[Conclusions are determined]
    \label{assu:conc:determined}
    For any value type \(v_{\tau}\):

    If an agent has the option of concluding \(\pv{\phi}{v_{\tau}}\), then the agent does not have the option of concluding \(\pv{\phi}{\overline{v_{\tau}}}\).
  \end{assumption}

  In other words, there are no cases where an agent may choose between concluding \(\phi\) has value \(v\) and concluding \(\phi\) has some value other than \(v\) of the same time.

  For example, if an agent has the option of concluding \(\phi\) is true, then the agent does not have the option of concluding \(\phi\) is false.
  Likewise, if the agent has the option of concluding \(\phi\) is undesirable, then the agent does not have the option of concluding \(\phi\) is desirable.
  However, as truth and desirability are distinct value types, an agent may conclude, e.g., \(\phi\) is false but desirable.
\end{note}

\begin{note}
  \Autoref{assu:conc:determined} is delicate.
  If the agent has the option, then it's determined.
  So, if the agent has concluded, then has the option.
  However, this does not guarantee that prior to reasoning the agent has the option.
  May be that in concluding, change in epistemic state, such that either \(\pv{\phi}{v}\) or \(\pv{\phi}{\overline{v}}\) is determined.

  For example, favourite flavour.
  The agent does not have the option of concluding they have a preference.
  However, conclude that X is favourite flavour.
  Now the agent has the option, and the agent does not have the option of concluding Y.
\end{note}

\begin{note}
  Though \autoref{assu:conc:determined} is delicate, will be important.
  However, key present epistemic state in mind.
\end{note}

\begin{note}
  We do not make any explicit assumptions about relations between value types, though we take for granted sensible constraints.
  For example, if an agent has the option of concluding \(\phi\) is true, the agent does not have the option of concluding \(\phi\) is impossible, though whether some proposition is possible or impossible is distinct from whether the proposition is true or false.
\end{note}

\begin{note}
  This is weaker than voluntarism.
  An agent may choose whether or not to conclude.
  And, may choose to whether to retract some previous conclusion.
\end{note}

\paragraph{Concluding is `description-free'}

\begin{note}[Descriptions]
  An important assumption is that an agent need not recognise that the culmination of some instance of reasoning is that some proposition has some value.

  \begin{assumption}[Concluding is `description-free']
    \label{assu:conc:d-free}
    It is not the case that an agent concludes \(\phi\) has value \(v\) only if the agent concludes \(\phi\) has value \(v\) under some description \emph{d}
  \end{assumption}

  In particular,~\autoref{assu:conc:d-free} holds for any description \emph{d} which includes an intensional reading of `\(\phi\) has value \(v\)'.
  More generally, it is possible for an agent to conclude \(\pv{\phi}{v}\) without consciously or otherwise entertaining either `\(\phi\)' or `\(v\)'.%
  \footnote{
    Compare with, for example, \citeauthor{Anscombe:1957aa} on intention action (\citeyear[\S19]{Anscombe:1957aa}) and \citeauthor{Davidson:1963aa} on primary reasons (\citeyear[5]{Davidson:1963aa}).
  }

  For the moment we will focus on intensionality.
  We will briefly explain the distinction between intensional and non-intensional readings, and motivate~\autoref{assu:conc:d-free} with respect to a handful of examples.

  Note, however, we are not providing an analysis of what it is for an agent to conclude \(\pv{\phi}{v}\).
  Rather,~\autoref{assu:conc:d-free} narrows down the particular sense of `concluding' of interest to us.
  There may be, and plausibly is, a sense of `concluding' for which~\autoref{assu:conc:d-free} does not hold (and in particular where the proposition-value pair of the conclusion is always intensional).
  However, our interest is with a sense of `concluding' for which~\autoref{assu:conc:d-free} holds.
\end{note}

\begin{note}[Looking ahead]
  Looking ahead, briefly,~\autoref{assu:conc:d-free} is a key assumption for developing tension.
  Tension will not follow from any reading of `concluding' in which in concluding \(\pv{\phi}{v}\) an agent concludes \(\pv{\phi}{v}\), and \(\pv{\phi}{v}\) alone.
  Rather, the tension we develop will involve an agent concluding \(\pv{\psi}{v'}\) when concluding \(\pv{\phi}{v}\).

  Now, our discussion of intensionality will suggest an agent may conclude \(\pv{\phi}{v}\) when concluding \(\pv{\varphi}{v}\) when there is significant overlap with what \(\phi\) and \(\varphi\) refer to.
  Still,~\autoref{assu:conc:d-free} allows that in concluding \(\pv{\phi}{v}\) an agent may also conclude \(\pv{\psi}{v}\), where there is no overlap between the reference of \(\phi\) and \(\psi\).
  When developing tension we will have interest with sufficient overlap with what \(\phi\) and \(\varphi\) refer to.
  So, we will not require~\autoref{assu:conc:d-free} in full generality, but equally it is only after developing the tension of interest that we will see in how~\autoref{assu:conc:d-free} may be restricted.
\end{note}

\begin{note}[Intensionality]
  Consider the following observation from~\citeauthor{Quine:1943vf}:

  \begin{quote}
    \begin{enumerate}[label=(\arabic*)]
    \item
      Giorgione = Barbarelli,
    \item
      Giorgione was so-called because of his size
    \end{enumerate}
    are true; however, replacement of the name 'Giorgione' by the name 'Barbarelli' turns (2) into the falsehood:

    \begin{center}
      Barbarelli was so-called because of his size.
    \end{center}
    \vspace{-\baselineskip}
    \mbox{ }%
    \mbox{}\hfill\mbox{(\citeyear[113]{Quine:1943vf})}
  \end{quote}

  Both `Giorgione was so-called because of his size' and `Barbarelli was so-called because of his size' are intensional in the sense that the truth value of each expression does not reduce to the reference of the expression's' components.
  Else, as `Giorgionee' and `Barbarelli' are co-referential, the predicate `was so-called because of his size' would apply equally to both `Giorgionee' and `Barbarelli'.

  Now, it is also not the case that in concluding `Giorgione was so-called because of his size', one also concludes `Barbarelli was so-called because of his size'.

  However, this observation is \emph{not} immediate from observing that the agent concluded `Giorgione was so-called because of his size' and did not conclude `Barbarelli was so-called because of his size'.

  The conclusion `Giorgione was so-called because of his size' may be intensional, but being intensional is not a property granted to some proposition-value pair by virtue of the proposition-value pair being a conclusion.

  For example, in concluding `Giorgione is large' the agent may also conclude `Barbarelli is large'.
  Indeed, the expression `Giorgione is large' may be read non-intensionally, and hence is true if and only if `Barbarelli is large', given that `Giorgionee' and `Barbarelli' are co-referential.

  Likewise, in concluding `\(2 + 2 = 4\)', an agent may also conclude `\(4 = 2 + 2\)'.
  Or, in concluding `\nagent{1} is shorter than \nagent{2}', an agent may also conclude `\nagent{2} is taller than \nagent{1}'.
  Though, in concluding `\nagent{1} is shorter than \nagent{2}' an agent may also fail to conclude `\nagent{2} is taller than \nagent{1}'.
  For example, if the relevant agent is not aware of the relationship between `shorter' and `taller'.
\end{note}

\begin{note}
  More broadly, I consider it intuitive that concluding any one of the following includes concluding any other:
  \begin{itemize}
  \item \(\phi\) has value \(v\).
  \item It is true that \(\phi\) has value \(v\).
  \item It is not the case that \(\phi\) does not have value \(v\).
  \item It is true that it is not the case that \(\phi\) does not have value \(v\).
    \begin{center}
      \(\vdots\)
    \end{center}
  \end{itemize}

  There are an infinite number of distinct proposition-value pairs that may be generated along these lines, but these distinct proposition-value pairs do not amount to distinction conclusions.
  A conclusion for one is a conclusion for all.
\end{note}

\begin{note}
  Indeed, we may form an explicit assumption governing certain proposition-value pairs.
  We start with the definition of \indicateN{}.
\end{note}

\begin{note}
  \begin{restatable}[\indicateN{2}]{definition}{defIndicate}
    \label{def:indication}
    \(\phi\) having value \(v\) \emph{\indicateV{1}} \(\psi\) has value \(v'\) if and only if:
    \begin{itemize}
    \item
      It is not \epPAd{}, from \vAgent{}' epistemic state, that \(\psi\) has value \(v'\) while \(\psi\) does not have value \(v'\).
    \end{itemize}
    \vspace{-\baselineskip}
  \end{restatable}
\end{note}

\begin{note}
  The assumption now holds that an agent concludes \(\pv{\psi}{v'}\) when concluding \(\pv{\phi}{v}\) just in case \(\pv{\psi}{v'}\) and \(\pv{\phi}{v}\) co-\indicateV{}.

  \begin{restatable}[\indicateN{2}]{assumption}{assuIndicate}
    \label{assu:indication}
    If \(\pv{\phi}{v}\) \indicateV{1} \(\pv{\psi}{v'}\), and \(\pv{\psi}{v'}\) \indicateV{1} \(\pv{\phi}{v}\), then:

    \begin{itemize}
    \item
      \vAgent{} concludes \(\pv{\phi}{v}\) just in case \vAgent{} concludes \(\pv{\psi}{v'}\).
    \end{itemize}
    \vspace{-\baselineskip}
  \end{restatable}
\end{note}

\begin{note}
  \Autoref{assu:indication} is mild closure condition on claiming support.
  Still, \autoref{assu:indication} is only a closure condition with respect to an agent's epistemic state.
  To illustrate:
  Suppose an agent has concluded that The Scarlet Pimpernel rescued Marquis de Lafayette.
  As `The Scarlet Pimpernel' and `Sir Percy Blakeney' are co-referential, it may be that the agent's conclusion \indicateV{1} that Sir Percy Blakeney' rescued Marquis de Lafayette.
  However, the conclusion will \indicateN{0} \emph{only if} there are no \epPW{1} in which `The Scarlet Pimpernel' and `Sir Percy Blakeney' refer to different individuals.

  Likewise, an agent may conclude that helping The Scarlet Pimpernel is desirable, without the conclusion \indicatePr{} that helping Sir Percy Blakeney is desirable.

  Indeed, an agent's conclusion that it is raining, while standing in Gower Street, may fail to \indicateN{} that it is raining in London if the agent considers it \epPAd{} that they are not in London.
\end{note}

\begin{note}
  Generalising,~\autoref{assu:indication} may be strengthened by weakening the restriction to `If \(\pv{\phi}{v}\) \indicateV{1} \(\pv{\psi}{v'}\)'.%
  \footnote{
    I.e.\ only the first conjunct of the restriction given.
  }

  Hence, in concluding \(\pv{\phi}{v}\) an agent would also conclude any proposition-value pair \(\pv{\psi}{v'}\) \emph{weaker} than \(\pv{\phi}{v}\), from the agent's epistemic state.
  Indeed, this leads to a much stronger closure condition on concluding.%
  \footnote{
    Consider by parallel closure of knowledge under known entailment:
    \begin{itemize}
    \item If an agent knows that \(\phi\) has value \(v\) only when \(\psi\) has value \(v'\), then if the agent knows \(\phi\) has value \(v\), then the agent knows \(\psi\) has value \(v'\).
    \end{itemize}
    This closure condition differs in forms, as it concerns knowledge as a state, by may be reformulated to a closer parallel:
        \begin{itemize}
    \item If an agent knows that \(\phi\) has value \(v\) only when \(\psi\) has value \(v'\), then in coming to know \(\phi\) has value \(v\) the agent comes to know \(\psi\) has value \(v'\).
    \end{itemize}
  }

  We will not assume this stronger variant of~\autoref{assu:indication} holds for the sense of `concluding' we are interested in.
  Rather, we have seen how~\autoref{assu:conc:d-free} allows for the possibility of an agent concluding \(\pv{\psi}{v'}\) when concluding \(\pv{\phi}{v}\), and with the exception of~\autoref{assu:indication} which we take to be sufficiently intuitive, we only advance argument in cases of interest.
\end{note}

\begin{note}
  Indeed, though the stronger variant of~\autoref{assu:indication} may be intuitive, there are certain issues we would like to avoid taking a stance on.
  In particular, whether concluding some statement which quantifiers over various objects includes concluding for each object the quantifier applies to.

  For example, suppose I have conclude that there are infinitely many primes.
  Reflecting a little on the natural numbers, I observe that if there are infinitely many primes, then for every natural number \(n\) there is some prime larger than \(n\) (for, the natural numbers are not dense).
  Hence, for every natural number \(n\) there is some prime larger than \(n\).

  On the stronger variant of~\autoref{assu:indication}, concluding `for every natural number \(n\) there is some prime larger than \(n\)' would also include concluding there is some prime larger than \(n\) for each \(n\).
  E.g.\ there is some prime larger than \(1\), \(16\), \(5^{43}\), \(53!^{793}\), \(54!^{794!}\), and so on\dots

  Specifically, looking ahead to tension, concluding that one has the general ability to witness some kind of reasoning would involve concluding that one has each specific instance of the general ability.
\end{note}

\begin{note}[Witnessing]
    While~\autoref{assu:conc:d-free} focuses on concluding, we take~\autoref{assu:conc:d-free} to apply equally to witnessing reasoning in which an agent concludes.
  Indeed, if a negative resolution to {\color{red} issue:Main} then any instance of concluding is also an instance of witnessing reasoning to the relevant conclusion, and hence~\autoref{assu:conc:d-free} would be in conflict with an assumption which states that an agent only witnesses reasoning which concludes \(\pv{\phi}{v}\) under some description.
  On the other hand, a positive resolution to {\color{red} issue:Main} does not ensure any instance of concluding is also an instance of witnessing reasoning to the relevant conclusion.
  However, we arrive at the same conflict with respect to any instance of concluding which is the result of witnessing reasoning to the relevant conclusion.

  Indeed, if \(\pv{\phi}{v}\) \indicatePr{} \(\pv{\psi}{v'}\), then \(\phi\) has value \(v\) \emph{only if} \(\psi\) has value \(v'\) (from the perspective of the agent's epistemic state).
  Hence, \emph{if} in concluding \(\pv{\phi}{v}\) an agent also concludes some \indicateVed{} \(\pv{\psi}{v'}\), then the relevant premises which allow the agent to also conclude \(\pv{\psi}{v'}\).
\end{note}

\begin{note}[Perspective on issue]
  Looking ahead, perspective on issue.
  In some cases, concluding one would conclude \(\pv{\phi}{v}\) from some premises \(\Phi\) is equivalent to concluding \(\pv{\phi}{v}\) from \(\Phi'\), where \(\Phi\) and \(\Phi'\) are distinct.
\end{note}

\paragraph{Summary}

\begin{note}[Summary]
  Handful of assumptions regarding concluding.

  For the most part, I take these to be straightforward.

  Suspect,~\autoref{assu:conc:d-free} is of interest.
  Again, while there may be a sense of `concluding' for which~\autoref{assu:conc:d-free} does not apply, there is a sense of `concluding' for which~\autoref{assu:conc:d-free} does apply.
  This sense of `concluding'.

  However, I suspect that this does not impact interest.
  For, if recognise conclude in latter sense, then no issue concluding in former sense.
  Still, this is additional argument.
  And, would rather avoid issues regarding phenomenology of concluding.
  Well, do avoid such issues.
\end{note}


%%% Local Variables:
%%% mode: latex
%%% TeX-master: "master"
%%% End:

\chapter{Motivation}
\label{cha:motivation}

\section{Theories}

\subsection{Reasoning}

\begin{note}[Some clarification]
  \begin{quote}
    What is the relation between a reason and an action when the reason explains the action by giving the agent's reason for doing what he did?%
    \mbox{}\hfill\mbox{({\citeyear[685]{Davidson:1963aa}})}
  \end{quote}

  Burgular case.
  There are reasons that an agent has to act which are not necessarily reasons for which the agent acts.

  Davidson distinction between \emph{reasons that one has to act} and \emph{reasons for which one acts}.

  \begin{quote}
    Davidson's article [asserts] a demand for a more ordinary form of explanation: an explanation which shows, not merely what, from another's point of view, \emph{could} count in favour of acting, but why that person did, in fact, act.%
    \mbox{}\hfill\mbox{(\citeyear[417]{Hieronymi:2011aa})}
  \end{quote}

  Following Neta, Anscombe.

  \begin{quote}
    a reason for which you act is always an answer to “a certain sense of the question ‘Why?' \dots that in which the answer, if positive, gives a reason for acting”
  \end{quote}

  Answer question from observer or the agent's perspective.

  With respect to reasoning, a pool of premises, or pools of premises.

  Distinction between \emph{how} and \emph{why} is with respect to \emph{reasons for which one acts}.

  Conclusion from some reasoning.
  Some trace.

  It seems not all components of the trace are relevant to why.
  Possible to go on a detour, or to simplify a line of reasoning.
\end{note}

\begin{note}[Inclusion]
  Note, a positive resolution to \autoref{issue:why-inc-in-how} does not require an answer to why to be equivalent to an answer to how.

  Cases of redundancy.
\end{note}

\paragraph*{Motivation}

\begin{note}
  We being with~\cite{Armstrong:1968vh}.
  \begin{quote}
    We are not concerned here with logicians' questions about inference, but solely with the psychological process of inferring.
    The primary sense of the word is that in which it involves acquiring a belief on the basis of a belief already held.

    \mbox{}\hfill\(\vdots\)\hfill\mbox{}

    \dots to say that A infers \emph{p} from \emph{q} is simply to say that A's believing \emph{q} \emph{causes} him to acquire the belief \emph{p}.
    And the sense of `cause' employed here is the common or billiardball sense of `cause', whatever that sense is.%
    \mbox{}\hfill\mbox{(\citeyear[194]{Armstrong:1968vh})}
  \end{quote}
  \cite{Armstrong:1968vh} goes on to consider some issues, but these reflect the `simplicity' of saying that inference is a matter of causation between beliefs, rather than causation.
  `With these qualifications, it seems that our causal account of inferring can stand.'
  (\citeyear[197]{Armstrong:1968vh})

  Of course, \citeauthor{Armstrong:1968vh} talks about `inference' rather than `reasoning', and in doing so restricts the scope of the proposal from arbitrary ways in which a proposition may be evaluated.
\end{note}

\begin{note}[Wedgwood]
  \begin{quote}
    Reasoning is a causal process, in which one mental event (say, one's accepting the conclusion of a certain argument) is caused by an antecedent mental event (say, one's considering the premises of the argument).%
    \mbox{}\hfill\mbox{(\cite[660]{Wedgwood:2006ui})}
  \end{quote}

  \citeauthor{Wedgwood:2006ui} illustrates how to go about witnessing.
  Of course, \citeauthor{Wedgwood:2006ui} here does not require positive resolution.
  Only talking about instances of reasoning.
  It is not immediate that any account of why is answered by reasoning.

  \begin{quote}
    Reasoning, I shall assume, is the process of \emph{revising ones beliefs or intentions, for a reason}.%
    \mbox{}\hfill\mbox{(\citeyear[660]{Wedgwood:2006ui})}
  \end{quote}

  Various connexions to other elements of epistemic state.
  At least include support.

  So, plausibly goes all the way.

  If why, then concluding is the result of reasoning.
  All reasoning is causal.
  So, positive resolution to \autoref{issue:why-inc-in-how}.
\end{note}

\begin{note}[Broome]
  {
    \color{red}
    In a footnote.
  }
  \citeauthor{Broome:2013aa} also.
  \begin{quote}
    So far as I can see, then, no further conditions need be added.
    I have arrived at necessary and sufficient conditions for a process to be active reasoning.
    Active reasoning is a particular sort of process by which conscious premise-attitudes cause you to acquire a conclusion-attitude.
    The process is that you operate on the contents of your premise-attitudes following a rule, to construct the conclusion, which is the content of a new attitude of yours that you acquire in the process.%
    \mbox{}\hfill\mbox{(\cite[234]{Broome:2013aa})}
  \end{quote}
\end{note}

\begin{note}[Illustration]
  Causal trace from conclusion to calculator.
  But, no relevance to understanding of arithmetic.
\end{note}

\begin{note}[Boghossian]
  Consider \citeauthor{Boghossian:2014aa}'s Taking Condition:
  \begin{quote}
    (Taking Condition): Inferring necessarily involves the thinker \emph{taking} his premises to support his conclusion and drawing his conclusion because of that fact.%
    \mbox{}\hfill\mbox{(\Citeyear[5]{Boghossian:2014aa})}
  \end{quote}
  As with \citeauthor{Armstrong:1968vh}, inference --- and hence reasoning with beliefs --- rather than reasoning more broadly (\Citeyear[cf][2]{Boghossian:2014aa}).

  \begin{quote}
    The intuition behind the Taking Condition is that no causal process counts as inference, unless it consists in an attempt to arrive at a belief by figuring out what, in some suitably broad sense, is supported by other things one believes.

    In the relevant sense, reasoning is something we \emph{do}, not just something that happens to us.
    And it is something \emph{we} do, not just something that is done by sub-personal bits of us.
    And it is something that we do with an \emph{aim}---that of figuring out what follows or is supported by other things one believes.
    It's hard to see how to respect these features of reasoning without something like the Taking Condition.%
    \mbox{}\hfill\mbox{(\Citeyear[5]{Boghossian:2014aa})}
  \end{quote}

  \begin{quote}
    \dots the property of a person's thinking something \emph{for a reason} is not response-dependent.
    To say that R was S's reason for A'ing implies that S took R to support his A'ing at the time that he A'ed, and that his so taking it led to his A'ing.

    I don't see how S's being disposed to \emph{say} that R was his reason for A'ing could \emph{make it the case} that he took R to support his A'ing and that this taking had a certain causal impact.
    His saying it might be very good evidence that R was his reason.
    But saying that R was his reason can't be \emph{constitutive} of R's being his reason.
    Causation is part of the idea of R's being his reason---and causation can't be a response-dependent property.%
    \mbox{}\hfill\mbox{(\citeyear[10--11]{Boghossian:2014aa})}
  \end{quote}

  {
    \color{red}
    See also \textcite[26--28]{Harman:1973ww} for an argument against dispositional accounts.
    In short, disposed to offer reasons targeted to audience.
    And, even if sincerity, possibility of identifying possible reasons rather than actual reasons.
    More basic, identification of reasons and conscious reasons.
  }

  {
    \color{red}
    Here, it's a kind of sufficiency.
    But, this final passage strengthens things.
  }

  {
    \color{green}
    Note, also, that though speak in terms of support, not assuming the taking condition.
  }
\end{note}

\begin{note}[Responding to reasons]
  As final motivation, consider the proposal at the core of \citeauthor{Lord:2018aa}'s (\Citeyear{Lord:2018aa}) thesis that being rational is to correctly respond to reasons.

  \begin{quote}
    \textbf{Correctly Responding:} What it is for A's \(\phi\)-ing to be ex post rational is for A to possess sufficient reason S to \(\phi\) and for A's \(\phi\)-ing to be a manifestation of knowledge about how to use S as sufficient reason to \(\phi\).%
    \mbox{}\hfill\mbox{(\Citeyear[143]{Lord:2018aa})}
  \end{quote}

  An agent's action is rational only if the action is a manifestation of some know-how.
  \citeauthor{Lord:2018aa} summaries:

  \begin{quote}
    \dots when one manifests one's know-how, dispositions that are directly sensitive to normative facts are manifesting. Thus, the competences involved in the relevant know-how make one directly sensitive to the normative facts%
    \mbox{}\hfill\mbox{(\Citeyear[16]{Lord:2018aa})}
  \end{quote}

  For our purposes, following example of manifesting know-how directly relates to reasoning:

  \begin{quote}
    The most salient disposition [when appealing to \emph{p} as a reason]%
    \footnote{Note, \citeauthor{Lord:2018aa} (explicitly) not talking about believing that \emph{p} is a reason, but argues that the cited disposition to present both when appealing to p as a reason and believing that \emph{p} is a reason.}
    is the disposition to (competently) use \emph{p} as a premise in reasoning.%
    \mbox{}\hfill\mbox{(\Citeyear[25]{Lord:2018aa})}
  \end{quote}

  Hence, suppose an agent concludes.
  Then, if the agent does not witness reasoning from pool of premises, it seems the agent does not manifest know-how, which is required for the appeal to meet \citeauthor{Lord:2018aa}'s account of rational action.

  Of course, that the noted disposition is the most salient does not rule out alternative, less noteworthy, dispositions.
  However, it is unclear to me how to \emph{manifest} know-how without use.
\end{note}

\begin{note}[Illustration]
  Clear that there is no manifestation of understanding of arithmetic.
\end{note}

\paragraph*{Less clear cases}

\begin{note}[Valaris]
  Maybe?

  Same observation as \citeauthor{Wedgwood:2006ui}, basically.
  Here, sufficient with additional hypothesis.
\end{note}

\begin{note}
  A note on \citeauthor{Valaris:2014un} here would be useful.
  For, \citeauthor{Valaris:2014un} offers something close, but suitably distinct.
  Reasoning is not to be identified with a causal process, but that's different.
  Constructs a coherent account of reasoning which does not involve causation.

  It's also limited.
  Because, there's a difference between believing p from availability of premises, and believing p from premises.

  It looks as though \citeauthor{Valaris:2014un} falls under witnessing.
\end{note}

\begin{note}[Other cases are less clear]
  For example, \citeauthor{Wright:2014tt}'s `Simple Proposal':
  \begin{quote}
    But consider instead the proposal, not that the status of the transition as inferential depends on the thinker's judgments about his reasons, but that it depends on \emph{what his reasons are}.
    We want his acceptance of the premises to supply his \emph{actual} reasons for accepting the conclusion.

    \mbox{}\hfill\(\vdots\)\hfill\mbox{}

    Call this the Simple Proposal.
    It says that a thinker infers q from p\(_{1}\) \(\cdots\) p\(_{\text{n}}\) when he accepts each of p\(_{1}\) \(\cdots\) p\(_{\text{n}}\), moves to accept q, and does so for the reason that he accepts p\(_{1}\) \(\cdots\) p\(_{\text{n}}\).\newline
      \mbox{}\hfill\mbox{(\Citeyear[33]{Wright:2014tt})}
    \end{quote}

    However, \citeauthor{Wright:2014tt} denies that reasoning must involve a state which connects premises to conclusions and so however \citeauthor{Wright:2014tt}'s Simple Proposal is developed, it will not involve a doxastic state:

    \begin{quote}
      What is needed, then, is an account of, or at least some insight into, what it is for certain intentional states of a thinker to be his actual reasons for his transition to another intentional state.

      [Which avoids] committing to the notion that doing something for certain reasons must involve a state that somehow registers those reasons as reasons for what one does.%
      \mbox{}\hfill\mbox{(\Citeyear[34]{Wright:2014tt})}
    \end{quote}
\end{note}

\subsection{The basing relation}

\begin{note}[Theories of basing]
  Connexion between reasoning and basing.
\end{note}

\begin{note}
  \citeauthor{Pollock:1999tm} introduce the basing relation with the following observation:
  \begin{quote}
    To be justified in believing something it is not sufficient merely to \emph{have} a good reason for believing it.
    One could have a good reason at one's disposal but never make the connection.
    \dots
    Surely, you are not justified in believing [something], despite the fact that you have impeccable reasons for it at your disposal.
    What is lacking is that you do not believe the conclusion on the basis of those reasons.\linebreak
    \mbox{}\hfill\mbox{(\Citeyear[35]{Pollock:1999tm})}
  \end{quote}
  The observation falls short of being an account of the basing relation, but the intuition \citeauthor{Pollock:1999tm} appeal to is instructive.
  It seems that an agent must connect reasons and the content of a belief in order for the belief to be formed on the basis of those reasons, and hence be justified by those reasons.
\end{note}

\begin{note}
  Apply same question to the basing relation.
\end{note}

\begin{note}
  Basing relation is delicate.
  No clear connexion with concluding.

  For example, consider \citeauthor{Evans:2013tw}' dispositional theory:

  \begin{quote}
    S's belief that \emph{p} is based on \emph{m} iff S is disposed to revise her belief that \emph{p} when she loses \emph{m}.%
    \mbox{}\hfill\mbox{(\citeyear[2952]{Evans:2013tw})}
  \end{quote}

  Intuitively, no witnessing.

  However, \citeauthor{Evans:2013tw}' dispositional theory is designed to capture why an agent sustains a belief, rather than why an agent forms a belief.%
  \footnote{
    See also \textcite{Audi:1986to} for a discussion of cases in which `[b]elieving for a reason does not entail having \textbf{come} to believe for that reason, or for any reason.' (\citeyear[32--33]{Audi:1986to})
  }

  \begin{quote}
    [T]he core of the basing relation is a particular sort of dependence:
    for one belief to be based on another is for the one to depend on the other in the right way.
    I think that the sort of dependence in question involves how one would respond were the basis of one's belief lost.
    Intuitively, one would revise one's belief, were its basis lost.
    If one's belief that \emph{p} really is based on one's belief that \emph{q}, one responds to a loss of the belief that \emph{q} by revising one's belief that \emph{p}.
    This is why we say that a belief stands or falls with its basis.%
    \mbox{}\hfill\mbox{(\citeyear[2951]{Evans:2013tw})}
  \end{quote}
\end{note}

\begin{note}
  More generally, what \citeauthor{Leite:2004uv} terms the `Spectatorial Conception' of the basing relation:%
  \footnote{
    See also~\textcite{Bondy:2018tk} for a detailed discussion of how the Spectatorial Conception of the basing relation relates to~\citeauthor{Schaffer:2010vq}'s (\citeyear{Schaffer:2010vq}) debasing demon.
  }
  \begin{quote}
    [T]he facts which determine basing relations are in place independently of the person’s explicit deliberation, reasoning, or declaration of reasons and are not directly determined by any of the person’s explicit deliberative or justificatory activity.%
    \mbox{}\hfill\mbox{(\citeyear[229]{Leite:2004uv})}
  \end{quote}

  Roughly, our approach to concluding is in line with the Spectatorial Conception.

  However, to the extent \citeauthor{Leite:2004uv} motivates rejecting the Spectatorial Conception, a problem.

  {
    \color{red}
    Also,~\cite{Sylvan:2016wq}
    Relation to inference.
  }
\end{note}

\begin{note}
  So, in short, there is no general connexion to the basing relation.

  For, in general, the relevant pre-theoretic characterisation of a basing relation between some basis and proposition-value pair need not bear any (pre-theoretic) relation to the agent concluding the proposition-value pair from the basis.

  Hence, there is no immediate connexion between issues~\ref{issue:why-inc-in-how} and~\ref{issue:has-witnessed} and accounts of the basing relation in general.

  Of course, if one is inclined to reject the Spectatorial Conception of the basing relation, then one may also be inclined to defer attention from issues~\ref{issue:why-inc-in-how} and~\ref{issue:has-witnessed}.
  For, it may be, following~\citeauthor{Evans:2013tw}, that why an agent comes to have some attitude toward some proposition-value pair would only be of interest in the context of why the agent sustains the attitude toward the proposition-value pair.
  And, at the same time, why an agent comes to have some attitude toward some proposition-value pair may have no clear relation to why the agent sustains the attitude toward the proposition-value pair.
\end{note}

\begin{note}
  So, I do not see any clear relation between issues~\ref{issue:why-inc-in-how} and~\ref{issue:has-witnessed} and accounts of the basing relation in general.

  However, specific account of the basing relation may hold some interest.
  In particular, we will briefly discuss two accounts of the basing relation which motivate interest in the broader idea of witnessing over causation.

  These accounts are:
  \begin{itemize}
  \item
    \citeauthor{Swain:1981wd}'s (\citeyear{Swain:1981wd}) causal-counterfactual theory of the basing relation.
  \item
    \citeauthor{Tolliver:1982us}'s (\citeyear{Tolliver:1982us}) doxastic theory of the basing relation.
  \end{itemize}
  We will explore both accounts in some detail, and particular attention will be given to the counterexamples \citeauthor{Tolliver:1982us} presses against \citeauthor{Swain:1981wd}'s theory.
\end{note}

\paragraph*{\citeauthor{Swain:1981wd}}

\begin{note}[\citeauthor{Swain:1981wd}]
  \begin{quote}
    \begin{enumerate}[label=(DB)]
    \item
      S's belief that \(h\) is based upon the set of causal reasons \(r\) at \(t\) \(=_{\text{\emph{df}}}\)
      \begin{enumerate}[label=(\arabic*)]
      \item
        S believes that \(h\) at \(t\); and
      \item
        For every member, \(r_{i}\) of \(R\), there is some time \(t_{n}\) (simultaneous or prior to \(t\)) such that
        \begin{enumerate}[label=(\alph*)]
        \item
          S has (or had) \(r_{j}\) at \(t_{n}\); and
        \item
          Either
          \begin{enumerate}
          \item[(\(i\))]
            S's having \(r_{j}\) at \(t_{n}\) is a cause or genuine overdeterminant of S's believing \(h\) at \(t\) or S's having \(r_{j}\) at \(t_{n}\) is a pseudo-overdeterminant of S's believing that \(h\) at \(t\);
          \item[(\(i\) + 1)]
            for some \(r_{i}\) and \(t_{i}\) that satisfy condition (i), S's having \(r_{j}\) at \(t_{n}\) is either a cause or a pseudo-overdeterminant of S's having \(r_{i}\) at \(t_{i}\).
          \end{enumerate}
        \end{enumerate}
      \end{enumerate}
    \end{enumerate}
  \end{quote}

  Pseudo-overdetermination:

  \begin{quote}
    \begin{enumerate}
    \item[(DPO)]
      Where \(c\) and \(e\) are occurrent events, \(c\) is a pseudo-overdeterminant of \(e\) if:
      \begin{enumerate}[label=(\arabic*)]
      \item
        \(c\) is not a cause of \(e\); and
      \item
        there is some set of occurrent events, \(D = \{d_{1}, d_{2},\dots, d_{n}\}\) (possibly having only one member), such that
        \begin{enumerate}
        \item
          each \(d_{i}\) in \(D\) is a cause of \(e\); and
        \item
          if no member of \(D\) had occurred, but \(c\) and \(e\) had occurred anyway, then there would have been a causal chain from \(c\) to \(e\), and \(c\) would have been causally prior to \(e\).
        \end{enumerate}
      \end{enumerate}
    \end{enumerate}
  \end{quote}

  Key here is occurrent events.

  In other words, only if the agent has witnessed reasoning.
\end{note}

\begin{note}[Swain]
  Here, with peduo overdetermination.

  With Swain, do not have witnessing.
  Well, it seems.
  But, this is not quite right.

  Here, very delicate.
  Present epistemic state.

  Key observation is that although Swain does not offer a causal account of the basing relation, it seems as though witnessed reasoning is required for pseudo-overdetermination.
\end{note}

\paragraph*{\citeauthor{Tolliver:1982us}}

\begin{note}[\citeauthor{Tolliver:1982us}]
  \citeauthor{Tolliver:1982us} is interesting.

  ``The pendulum case (and others like it) concerns problems arising from basing relations between beliefs which imply each other.''

  \begin{quote}
    Suppose A is a physics student who has learned that, from the period of a pendulum, one can calculate its length, and \emph{vice versa}.
    A observes that a particular pendulum \emph{b} has length \emph{l}.
    He calculates that \emph{b} must have period \emph{p}.
    As a result of his calculations, A has a couple of general beliefs:
    ``For all things, call them x, if x is a pendulum of length \emph{l} (call this ``Lx''), then x is a pendulum of period \emph{p} (``Px''), and (x) (if Px, then Lx).
    It seems clear that A's reason for believing Pb is his belief that Lb.
    But is A's belief that \emph{b} has period \emph{p} also one of A's reasons for his believing that \emph{b} has length \emph{l}?
    It would appear not.\newline
    \mbox{ }\hfill\mbox{(\citeyear[152]{Tolliver:1982us})}%
    \footnote{%
      The punctuation of this passage follows \citeauthor{Tolliver:1982us}'s paper, I am not sure why the initial left double quotation mark is not closed, nor am I clear on where it should be closed\dots

      As an aside, \citeauthor{Tolliver:1982us} presents the \scen{0} in this way to ease comparison with \citeauthor{Armstrong:1973vr}'s (\citeyear{Armstrong:1973vr}) account of the basing relation.
      We have, and will continue, to focus on \citeauthor{Swain:1981wd}'s account as \citeauthor{Armstrong:1973vr}'s account explicitly involves causation in a way \citeauthor{Swain:1981wd}'s does not.
    }
  \end{quote}
  \citeauthor{Tolliver:1982us} does not expand on why it appears A's belief that \emph{b} has period \emph{p} is \emph{not} also one of A's reasons for his believing that \emph{b} has length \emph{l}.
\end{note}

\begin{note}[Back to Swain]
  Have:
  \(P = 2\pi\sqrt{\sfrac{L}{g}}\) iff \(L = g \left(\sfrac{P}{2\pi}\right)^{2}\)

  Here, these are numbers rather than variables.

  So, if nothing changes, \(g(\sfrac{P}{2\pi})^{2}\) is \(L\).
  But, then \(P\) is involved.

  This is on 155 of \citeauthor{Tolliver:1982us}.

  So, the objection is delicate.
  The problem is not that the agent would measure the period, and then calculate the length.
  Rather, it's that we have a biconditional.
  And, it is assumed to be immediate that given the biconditional, substitution, roughly.

  Hence, the apparent flaw with Swain's account is the witnessed reasoning from length to period.
  This leads to an strange form of pseudo-overdetermination.
\end{note}

\begin{note}[\citeauthor{Tolliver:1982us}'s theory]
  \citeauthor{Tolliver:1982us} proposes the following doxastic account of the basing relation.
  \begin{quote}
    \begin{enumerate}[label=(B\('\))]
    \item
      A bases his belief that q on p at time t, iff
      \begin{enumerate}[label=(\arabic*)]
      \item
        A believes that q at t and A believes that p at t, and
      \item
        A believes that the truth of p is evidence for the truth of q at t,andd
      \item
        Where A's estimate of the likelihood of q equals h at t \((0 < \text{h} \leq 1)\), \emph{if it were the case that}:
        \begin{enumerate}[label=(\roman*)]
        \item
          A's second·order estimate of the L-proposition ``the likelihood of q is greater than or equal to h'' is less prior to t than it at t, and
        \item
          A did not believe p prior to t, and
        \item
          A came to belive p at t,
        \end{enumerate}
        then, at t, A's second-order estimate of the L-proposition ``the likelihood of q is greater than or equal to h'' would be greater than it was prior to t.%
        \mbox{}\hfill\mbox{(\citeyear[159]{Tolliver:1982us})}
      \end{enumerate}
    \end{enumerate}
  \end{quote}

  Intuitively, combining to believe p makes a difference.

  Now, return to the pendulum case.
  Here, \citeauthor{Tolliver:1982us} argues for no basing.

  \begin{quote}
    [A]ll we need consider is whether the introduction of the belief that Pb into A's doxastic framework is sufficient to increase A's propensity to hold an estimate of the likelihood of Lb higher than or equal to its present value.
    The answer is ``no.''
    The strength of A's propensity for his estimate of the likelihood of Lb not to drop below its present level is not at all increased by his belief that Pb.
    A's belief that Pb will not tend to counteract the influence of any factor tending to reduce A's confidence in the truth of Lb.
  \end{quote}

  So, does belief about the period make any difference with respect to length.

  \citeauthor{Tolliver:1982us}'s point is that this does nothing.

  Footnote, because of no information about period.

  But, this is puzzling.

  Unless witness reasoning from period to length, then it's not clear why getting the period would have any influence.
  This is the agent's estimate.
  Not, a second order estimate given the agent's beliefs.

  So, not causal, and no clear statement of witnessing, but hard to understand without.
\end{note}


%%% Local Variables:
%%% mode: latex
%%% TeX-master: "master"
%%% End:

\printbibliography

\part{Old}

\chapter{Ability}
\label{sec:major-argument}
\label{sec:broad-argum-overv}
\label{sec:all-about-ability}

\section{Scenarios}
\label{sec:cases-interest}

Our goal is to argue for \EAS{} and against \ESU{}.
At the core of the argument is reasoning about ability.
Specifically, a certain type of scenario in which an agent reason to and from information that they have the ability to witness some specific act.
How the agent reasons with such (specific) ability information in the scenarios of interest will provide a type of counterexample to \ESU{} and in turn an argument for \EAS{}.

In this section we outline two key features of the scenarios we are interested in.
Subsection~\ref{sec:type-scenario} will introduce \gsi{0} to characterise how the agent reasons to the (specific) ability information.
Then, subsection~\ref{sec:ability-entailment} will introduce `\aben{the}' to characterise how the agent reason from the (specific) ability information.
Finally, subsection~\ref{sec:scenarios} will combine \gsi{0} and `\aben{the}' to provide an in-depth understanding of the type of scenarios we are interested in.

\subsection{\gsi{2}}
\label{sec:type-scenario}

\begin{note}[Tension, information]
    \begin{restatable}[\gsi{}]{definition}{defGSI}\label{def:gsi}
    \gsi{2} is information that:\nolinebreak
    \footnote{
      Strictly speaking the formulation of \gsi{} as a conditional isn't important.
      What matters is that the agent is required to claim support for the general ability in order to claim support for the specific ability.
      For example, the conditional may be reformulated as:
      \begin{enumerate}[label=(\gsi{}\('\)), ref=(\gsi{}\('\))]
      \item Either \emph{S} does not have the general ability to \(\gamma\), or the agent has a specific ability to \(\varsigma\).
      \end{enumerate}
    }
    \begin{quote}
      If \emph{S} has a general ability to \(\gamma\), then \emph{S} has a specific ability to \(\varsigma\).
    \end{quote}
    Where \emph{S} is some agent, \(\gamma\) is some general ability, \(\varsigma\) is some specific ability, and it is either implicitly or explicitly stated that \(\varsigma\) is instance of \(\gamma\).
  \end{restatable}
  
  The following pair of examples are instances of \gsi{}.
  \begin{enumerate}[label=(\gsi{}:\arabic*), ref=(\gsi{}:\arabic*)]
  \item\label{qe:cond} If you have the ability to reason with the rules of chess, then you have the ability to demonstrate that, given the arrangement of the board, there is a sequences of moves that will ensure a win for one of the players (as an instance of the general ability to reason with the rules of chess).
  \end{enumerate}

  \begin{enumerate}[label=(\gsi{}:\arabic*), ref=(\gsi{}:\arabic*), resume]
  \item\label{qe:cond:french} If you have the ability to read French, then you have the ability to read The Count of Monte Cristo without a translation (as an instance of the general ability to read French).
  \end{enumerate}
  In both examples an agent is informed that they have the ability to perform a specific act --- demonstrating a strategy or reading a book --- so long as they have some general ability --- an understanding of chess or French literacy --- because the witnessing the specific ability act would be an instance of witnessing the agent's general ability.

  \gsi{} does not directly provide the agent with the information that they have the specific ability.\nolinebreak
  \footnote{Nor (looking ahead to section~\ref{sec:ability-entailment}) does \gsi{} directly provide the agent with information that the result of witnessing the specific ability is when \aben{the} holds with respect to the specific ability.}
  The agent is not informed that they have the general ability and that therefore they have a specific ability.
  To illustrate, I am confident I have the ability to reason with the rules of chess, and so given \ref{qe:cond} I may be confident that I am able to demonstrate the existence of such a strategy.
  By contrast, I do not have the ability to read French, and so I do not have the ability to read The Count of Monte Cristo without a translation.

  Still, I may also be mistaken.
  It may be that I am overconfident, that I do not have the ability to reason with the rules of chess, and hence it may be the case that I do not have the ability to demonstrate the existence of the relevant chess strategy.
  Likewise, I may have the ability to read French, and may have the ability to read The Count of Monte Cristo without a translation.
  However unlikely this may be, I haven't tried to read French in quite some time.
\end{note}

\begin{note}[Not direct]
  \gsi{2} contrasts with what we term `\dsi{0}' --- information that the agent has some ability.
    \begin{restatable}[\dsi{}]{definition}{defDSI}\label{def:dsi}
    \dsi{2} is information that:
    \begin{quote}
      \emph{S} has the ability to \(\varsigma\).
    \end{quote}
    Where \emph{S} is some agent and \(\varsigma\) is some specific ability.
  \end{restatable}
  For example, the following is a `direct' recreation of~\ref{qe:cond}:

  \begin{enumerate}[label=(\dsi{}:\arabic*), ref=(\dsi{}:\arabic*), series=dsi_count]
  \item\label{qe:cons} You have the ability to demonstrate that there is a sequences of moves that will ensure a win for one of the players as an instance of your general ability to reason with the rules of chess.
  \end{enumerate}

  If~\ref{qe:cons} is true then the agent has the ability to demonstrate some strategy.
  And, in turn,~\ref{qe:cons} expands on why the agent has the relevant specific ability.
  By contrast,~\ref{qe:cond} may be true even if the agent does not have the ability to demonstrate some strategy.
  Hence, \dsi{} is not in general entailed by \gsi{}.\nolinebreak
  \footnote{
    However, if it is the case that an agent has the general ability mentioned in the antecedent of \gsi{}, then a corresponding instance of \dsi{} will be true.
    Note, this is ensured because the consequent of~\ref{qe:cond} ensures the relevant `instance of' relation obtains.
    % So, if I have the ability to reason with the rules of chess and~\ref{qe:cond} is true with respect to me, then \ref{qe:cons} will also be true with respect to me.
  }
\end{note}

\begin{note}[Important features of \gsi{}]
  \gsi{}, then, has two important features:
  \begin{enumerate}
  \item \gsi{} ensures that the agent is on the hook, so to speak, for claiming support they have the specific ability.
  \item If the agent may claim support for having the relevant general ability, then \gsi{} provides the agent with an account of why they may claim support for having some specific ability.
  \end{enumerate}
  Hence, \gsi{} ensure that an agent must themselves claim support that they have some specific ability while providing the agent with relevant information about why they may claim support for having the specific ability.
\end{note}

\begin{note}[Merging \gsi{} and \dsi{}]
  Finally, though we will focus on \gsi{}, there is a variant that merges \gsi{} and \dsi{} which could be substituted for \gsi{} in further discussion.
  This variant involves informing an agent that they have some general ability, and some specific ability as an instance of that general ability, but requires the agent to identify what the general ability is.

  Here is the variant applied to~\ref{qe:cond}.
  \begin{enumerate}[label=(\gsi{}\(^{'}\):\arabic*), ref=(\gsi{}\(^{'}\):\arabic*)]
  \item
    \begin{enumerate}
    \item You have some general ability \(\gamma\), and a specific ability \(\varsigma\) (as an instance of that general ability). And,
    \item If \(\gamma\) is the ability to reason with the rules of chess, then \(\varsigma\) is the ability to demonstrate that, given the arrangement of the board, there is a sequences of moves that will ensure a win for one of the players (as an instance of the general ability)
    \end{enumerate}
  \end{enumerate}
  The agent remains on the hook, so to speak, for claiming support that they have the relevant specific ability because it is up to the agent to identify the general ability \emph{as} the ability to reason with the rules of chess.
  And, likewise, if the agent may claim support for identifying the general ability in a particular way, then the variant allows the agent to claim support that they have a particular specific ability.

  We favour \gsi{} given it's comparative structural simplicity, but the variant highlights that that the agent claiming support for having some specific ability is not of interest.
  Rather, what is interest is that \gsi{} allows the agent to claim support for the particulars of some specific ability.

  In section~\ref{sec:ability-entailment} we will highlight why the particulars matter.
\end{note}

\subsection{An ability entailment}
\label{sec:ability-entailment}

\begin{note}[\aben{(The)}]
  The second component in scenarios of interest is the availability of an entailment from the specific ability.

  We term an instance of the entailment as an `\aben{}'.

  \begin{restatable}[Ability entailment]{definition}{defAE}\label{def:aben}
    \aben{The} is any entailment of the form:
    \begin{quote}
      \emph{S} has the (specific) ability to \emph{V} that \(\phi\) \emph{therefore} \(\phi\) is the case.
    \end{quote}
    Where \emph{S} is an agent, \emph{V} is some action, and \(\phi\) is some proposition.
  \end{restatable}

  The rough intuition behind instances of \aben{the} is that \(\phi\) being the case does not depend on \emph{S} witnessing the (specific) ability to \emph{V} that \(\phi\).
  So, \aben{the} links ability and something that must be the case in order to have ability and the result of witnessing ability must be the case in order for the agent to have the ability

  For example, \aben{the} holds with respect to the (specific) ability to demonstrate the existence of a chess strategy from \ref{qe:cond} as whether or not a given chess strategy exists depends on the moves permitted by the rules of chess --- a strategy that has not been demonstrated is a strategy.
  Likewise, \emph{S} has the (specific) ability to discover that their keys are in their jacket pocket only if it is the case that their keys are in their jacket pocket --- whether or not \emph{S}'s keys are in their jacket pocket does not depend on \emph{S} discovering that to be the case.

  By contrast, `to read The Count of Monte Cristo without a translation' is an action and so \aben{the} does not apply to the specific ability of~\ref{qe:cond:french}.
  Even so, \aben{the} apply to nearby variants and not others.
  \emph{S} may have the specific ability to read that Dantès was a merchant sailor, and it follows that Dantès was a merchant sailor.
  In contrast, while \emph{S} may have the ability to believe that certain passages cannot be adequately translated, it does not follow that those passages cannot be adequately translated.
  Similarly, \emph{S} may have the ability to hope that they will employ the chess strategy discovered in a competitive game, but it does not follow that \emph{S} will employ the strategy.

  More broadly, \aben{the} holds with respect to factive verbs, such as `see', `know', `understand', and so on.
  Though, I doubt factive verbs are an adequate explanation for \aben{the}.
  Consider `read'.
  I have the ability to read that Elvis Presley was born in 1935, but I also have the ability to read that Elvis is working undercover for the DEA.
  What matters, then, is not the verb used, but how the agent would witness the relevant ability.
  I have the ability to read that Elvis was born in 1935 from a reliable source, and hence \aben{the} applies.
  The same is not true for my ability to read that Elvis is working for the DEA.

  Indeed, \aben{the} merely identifies an entailment.
  It does not provide an account of when or why such entailments hold.
  We identify entailments of this type because our interest is in how (in certain cases) agent's reason with instances of \aben{the}.
\end{note}

\subsection{Details of scenarios}
\label{sec:scenarios}

\begin{note}[Both things are important]
  The scenarios we are interested in combine \gsi{} with \aben{the}.

  The role of \gsi{} is to ensure that the agent is not provided with direct information about specific ability.
  And the role of \aben{the} is to highlight that the agent is in a position to claim support for some further proposition if they claim support for specific ability.
  Hence, scenarios combine claiming support \emph{for} specific ability and claiming support \emph{from} specific ability.

  To illustrate, consider the following pattern of reasoning:
  \begin{enumerate}[label=\arabic*., ref=(\arabic*)]
  \item\label{scen:rp:1} I have the general ability to reason with the rules of chess.
  \item\label{scen:rp:2} I received \gsi{} information that if they have the general ability to reason with the rules of chess then they have the ability to demonstrate the existence of some strategy.
  \item\label{scen:rp:3} So, from~\ref{scen:rp:1} and~\ref{scen:rp:2} it follows that I have the ability to demonstrate the existence of some strategy.
  \item\label{scen:rp:4} And, as \aben{the} hold with respect to~\ref{scen:rp:3}, the relevant strategy exists.
  \end{enumerate}
  I reason to (\ref{scen:rp:1} --- \ref{scen:rp:3}) and from (\ref{scen:rp:3} --- \ref{scen:rp:4}) a specific ability.
  The reasoning pattern seems sound.
  And, at no point do I need to witness their general ability to reason with the rules of chess, or the specific application of the general ability to demonstrate the existence of the strategy.
\end{note}

\begin{note}
  Both components are important.
  Focus on \gsi{} will restrict the interpretation of what the agent claims support for.
  And, in turn, what the agent has claimed support for will determine what the agent appeals to when appealing to \aben{the} entailment.\nolinebreak
  \footnote{
    I suspect it may be possible to focus only on \gsi{}.
    As we will see, this is where the key step of the argument takes place.
    However, this is not trivial.
    Would require more focus on how the agent gets to specific from general.
    By splitting in this way, we avoid details.
    Instead, focus on what it is that the agent gets, and then \aben{the} is forced to work with this.
  }
  \gsi{} and \aben{the} combine to provide a (partial) functional characterisation of reasoning with specific ability.
\end{note}

\begin{note}
  Note, however, that there is a distinction between how an agent reasons about ability, and what ability is.
  We are interested in how agent's reason about (specific) ability, and not what makes it true that an agent has a (specific) ability.
  Our focus will shortly turn to how to interpret (specific) ability when appealed to in the type of scenario described.
  We will outline two general schematic interpretations of ability, argue that these are exhaustive, and note how general constraints such as \ESU{} constrain which interpretation is available.
\end{note}

\begin{note}[Scenario proposition]
  For ease of reference, we wrap scenarios involving the limited information as a proposition.
    \begin{restatable}[\eA{0} --- \eA{}]{proposition}{propScenariosExist}\label{prop:SE}
    There are scenarios in which an agent \emph{S} receives \gsi{} information of the form:
    % \mbox{ }\vspace{5pt}
    \begin{center}
      If \emph{S} has a general ability to \(\gamma\), then \emph{S} has a specific ability to \emph{V} that \(\phi\).
    \end{center}
    % \mbox{ }\vspace{5pt}

    \noindent Such that \aben{the} applies to the specific ability to \emph{V} that \(\phi\).

    In turn:
    \begin{enumerate}
    \item \emph{S} may reason from claimed support that they have the general ability to \(\gamma\) in order to claim support for having the specific ability to \emph{V} that \(\phi\). And,
    \item \emph{S} may reason from their claimed support that they have the ability to \emph{V} that \(\phi\) to claim support that \(\phi\) is the case by appealing to \aben{the}.
    \end{enumerate}
    \vspace{-\baselineskip}
  \end{restatable}
\end{note}

\begin{note}[Possible restrictions]
  First, \eA{} holds only that there are cases in which the agent may appeal to ability to obtain support.
  It is therefore consistent with~\eA{} that there are cases in which the details of the cases outlined are satisfied, but where kind of support is unsuitable for certain purposes.
  For example, some witness of ability may be demanded by a third-party.
  In this respect, the content of \eA{} is similar to an analogous claim with respect to memory.
  If an agent remembers proving that \(\phi\), then \(\phi\) is the case.
  Still, one may still request that an agent provides you with a proof of \(\phi\) in order to for you to be satisfied \(\phi\) is the case --- many exams are like this.
  So, that an agent may not always and in any context claim support for \(\phi\) from claimed support for their ability to \emph{V} that \(\phi\) is not an objection to~\eA{}.
\end{note}

\begin{note}
  Second, \eA{} does not require that an agent reason in the way described given \gsi{} and availability of \aben{the}.

  For example, the following statement is an instance of \gsi{}:
  \begin{enumerate}
  \item Any person who has the (general) ability to reason with the rules of chess has the (specific) ability to identify Alekhine's Defense as a fine opening move.
  \end{enumerate}
  The universal quantifier implies that the statement is true with respect to me, among others.
  Still, I am confident that there is at least one other person who has the ability to reason with the rules of chess, and may therefore infer that Alekhine's Defense as a fine opening move without appealing to my own ability.
  Indeed, if I am inclined to doubt my own (general) ability in contrast to a Grandmaster, then I may be more confident that Alekhine's Defense as a fine opening move if I appeal to the existence of a Grandmaster.

  Again, it is consistent with \eA{} that an agent may reason in such a way.
  Still, in defence of \eA{} it is important to note that \gsi{} information may be limited to the agent in question.
  For example, I may have studied your notes on how to play chess and identified a strategy which follows from those notes.
  I have no doubt that you have the ability to identify the same strategy, so when I provide \gsi{} my emphasis is on whether you have the ability to reason with \emph{chess}, rather than some closely related game.

  There are many ways to build context so that an agents is required to reason with \gsi{} and \aben{the} if the agent is to reason with (specific) ability at all, but I doubt these are required.
  The reasoning described by \eA{} (and illustrated above) seems plain and permissible.
\end{note}

\begin{note}
  Finally, \gsi{} and \aben{the} are constraints which do not hold in all cases of reasoning with specific ability.

  For example, one may be told that a gift of a metal detector grants the ability to discover if there is buried treasure in the garden.
  The former does not entail that there is buried treasure in the garden, and testimony or the metal detector may be claimed as support for the ability.

  So, question about whether this really does anything for general understanding of ability.
  \gsi{} and \aben{the} combine to require a particular interpretation.
  However, interpretation with general applicability is not restricted to instances in which it is forced.
  The role of a counterexample is not (typically) to establish that every instance of a theory is mistaken, but to identify a gap.
  And, even if the original theory may be restricted to non-problematic cases, the alternative theory may compete with the original theory.
  So, given that the particular interpretation is required to hold given additional stipulations, interest is in whether it holds without additional stipulations.
\end{note}

\section{Two (schematic) accounts of (specific) ability}
\label{sec:wr-ar}

\begin{note}
  In the previous section we introduced \gsi{0} and \aben{the}.
  In the present section we motivate two interpretations of (specific) ability in the context of reasoning to (specific) ability from \gsi{} and reasoning from (specific) ability with \aben{the}.

  The two interpretations are termed `\AR{}' and `\WR{}' in turn, and are schematic.
  Roughly:
  \AR{} holds that when appealing to (specific) ability an agent appeals to some property or attribute that they have.
  And, by contrast, \WR{} holds that when appealing to (specific) ability an agent appeals to the action that they would perform by witnessing the relevant ability.
  \AR{} and \WR{} are distinguished, then, by whether an agent reasons with a property (\AR{}) or an event (\WR{}).

  To illustrate by analogy, consider a mechanical clock.
  The clock has the property of displaying the correct time, by it is also involved in the event of changing it's configuration as time passes.
  The property that the clock is displaying the correct time is important for determining whether one is late for a meeting.
  By contrast, the event of changing it's configuration as time passes is important for determining when to remove a brewing teabag.
  A meeting starts at a certain point in time, while tea is brewed over a period of time.
  If the clock does not represent the correct time, then three minutes passing will not, in general, help determine whether one is late to the meeting.
  And, whether or not it is 3pm is not, in general, important with respect to whether or not the tea has finished brewing.
  The qualifier `in general' is important.
  Measuring the passage of is useful if I know the length of time before the meeting is due, and the correct time is useful if I know when I started brewing the tea.

  The distinction between \AR{} and \WR{} is similar.
  Both interpretations may be more or less useful in certain circumstances, and interchangeable in others.
  Still, the combination of \gsi{} and \aben{the} identify a pattern of reasoning in which we may elaborate how the relevant interpretation of (specific) ability is important, and in turn broader principles (\ESU{} and, to be introduced below, \nI{}) will constrain whether the interpretations are permissible.
\end{note}

\section{\AR{} and \WR{}}
\label{sec:ar-wr-1}

\begin{note}[\WR{} and \AR{}]
  We term the two schematic interpretations of \aben{the} `\AR{}' and `\WR{}', respectively.
  Brief descriptions from detached perspective.
  Given that the interpretations are schematic, they fall short of a full account of how an agent claims support by \aben{an}.
  However, the arguments to follow are of interest in part because they concern any way in which the schematic interpretations are filled out.
\end{note}

{
  \color{red}
  I should emphasise that here we're interested in reasoning.

  Also, the distinction is important to ensure that the argument's don't depend on a specific reading of ability.
}

\subsection{\AR{}}
\label{sec:ar-1}

\begin{note}
  \begin{restatable}[\AR{}]{definition}{defAttribution}\label{AR:def}
    An agent's reasoning with an instance of \aben{the} by claiming support for \(\phi\) from \emph{S} having ability to \emph{V} that \(\phi\) is an instance of \emph{\AR{}} when the agent holds that:

    \emph{S} has the ability to \emph{V} that \(\phi\)
    \begin{enumerate*}[label=(\textsf{A}\arabic*), ref=(\textsf{A}\arabic*)]
    \item\label{A:s:1} is or reduces to some (possibly complex) property \emph{P} of \emph{S}, and
    \item\label{A:s:2} \emph{P}, or some part of \emph{P}, entails \(\phi\) is the case.\nolinebreak
      \footnote{Intuitively, because the agent could not have \emph{P} without \(\phi\) already being the case.
      The notion of entailment here does not require \(\phi\) is true \emph{because} of \emph{P}.}
    \end{enumerate*}
  \end{restatable}
  
  {
    \color{red}
    \AR{} identifies instances of reasoning in which an agent applies \aben{the} by holding the ability to \emph{V} that \(\phi\) is a property of an agent.\nolinebreak
    \footnote{
      Note, this does not say anything about what the ability to \(\phi\) is.
      Rather, way in which the agent claims support.
    }
    Note, when appealing to \aben{the} an agent need not be aware of what the (potentially complex) property of \emph{S} is.
    Rather, claimed support that \emph{S} has the ability to \emph{V} that \(\phi\) allows the agent to claim support for the existence of some property of \emph{S} which in turn entails \(\phi\).
  }

  Now, generally speaking properties are things which may be predicated or attributed of other things.
  The coffee is hot, I am thirsty, my mouth is sensitive to heat, I am reckless, I am in pain, and so on\dots
  And, properties come cheap.
  For example, the participation of an agent in some event gives rise to a property that may be attributed to the agent.
  Specifically, the property of participating in the event.
  Moments ago I participated in the event of recklessly drinking hot coffee with a mouth that is sensitive to heat.
  Therefore, I have the property of participating in such an event.

  So,~\ref{A:s:1} is trivially true.
  When we speak of an agent having some ability we are predicating or attributing ability to an agent.
  However,~\ref{A:s:2} requires that the property entails \(\phi\) is the case.
  And, it is not clear that an entailment which follows from an event is always reflected in the property of being a participant in the event.
  For example, it seems that I am in pain because I participated in the event of drinking hot coffee, \emph{not} because I have the property of having participated in the event of drinking hot coffee.
  By contrast, that I have the property of having participated in the event of drinking hot coffee entails that I have the property of having participated in the event of drinking something.

  % From~\ref{A:s:2} it must be the case that the relevant property entails \(\phi\).
  % And, from~\ref{A:s:3} the property must not analysed in terms of there being a potential event in which \emph{S} witnesses the act of \emph{V}ing that \(\phi\).
  % This is, from one perspective, an arbitrary restriction.
  % For example, if there is a potential event in which an agent witnesses the act of \emph{V}ing that \(\phi\), then the agent has the property of being a participant of that potential event.
  % From a different perspective,~\ref{A:s:3}

  Roughly, we may expect the property of interest is akin to having a heart, possessing ¥500, being of a certain age, and so on\dots

  {
    \color{red}
    Key idea with \AR{} is that the agent `directly' claims support for a property when using \aben{the}.
  }

  To illustrate \AR{} we focus on the idea of reducing the ability to \emph{V} that \(\phi\) to some (potentially complex) property of \emph{S}.
  Again, when appealing to \aben{the} an agent need not be aware of what the (potentially complex) property of \emph{S} is.
  Rather, these illustrations suggest that such properties exist.

  \begin{illustration}
    Consider the proposition that \emph{S} has the ability to hear that the birds are signing.
    Again, it seems \aben{the} holds, and one may infer that birds are singing.

    So, by \AR{} there is some (complex) property \emph{P} of \emph{S} such that \emph{P}, or some part of \emph{P}, entails the the birds are signing.

    Consider the complex property of a well-functioning auditory system and sufficient proximity to the birds singing.
    The property of having well-functioning auditory system ensures that \emph{S} has the ability to hear nearby noises.
    And, having well-functioning auditory system together sufficient proximity to the birds singing together ensure that \emph{S} has the ability to hear the nearby noise of the birds singing.

    \aben{the} follows from part of this complex property.
    If the agent has the property of being in sufficient proximity to the birds singing, then it follows that there are birds singing.
  \end{illustration}

  \begin{illustration}
    Consider the proposition that the prosecution has the ability to demonstrate that the defendant is guilty.
    Intuitively, \aben{the} holds, as it is not possible to demonstrate the guilt of an innocent person.\nolinebreak
    \footnote{
      It is a different matter to convince a jury of the guilt of an innocent person.
      And, \aben{the} does not seem to hold with respect to the ability to convince a jury that the defendant is guilty.
    }
    By \AR{} there is some (complex) property \emph{P} of the lawyer such that \emph{P}, or some part of \emph{P}, entails the guilt of the defendant.
    Say, the lawyer is in possession of evidence sufficient to establish guilt of the defendant.
    If so, it is a property of the lawyer that they are in possession of such evidence, and by assumption the evidence entails that the defendant is guilty.

    It seems possession of evidence alone may not be sufficient to establish that the lawyer has the ability to prove that the defendant is guilty.
    For, it is plausible that a lawyer may be in possession of evidence that they do not understand.
    However, as our interest is with \aben{the} it is sufficient to observe that the evidence alone entails the guilt of the defendant.
  \end{illustration}

  Again, these illustrations highlight ways in which \emph{S} having the ability to \emph{V} that \(\phi\) may be reduced to some (complex) property of \emph{S}.
  \AR{} does not hold that an agent identifies such a property when claimed support by an instance of \aben{the}.
  Rather, \AR{} holds that the agent reasons with ability as a property of the agent.
  Indeed, while these suggestions reduce ability to complex properties, \AR{} also admits of the possibility that the ability to \emph{V} that \(\phi\) is a basic property which does not admit of further analysis.
  If so, then it seems that \aben{the} must also be taken as basic.\nolinebreak
  \footnote{
    I lack any suggestion for how to understand \AR{} if the property is indeed basic, but there is no need to rule out this option ---  no part of the following arguments depend on whether or how these schemas may be substantiated.
  }
  So, to summarise.
  The distinguishing feature of \AR{} is that there are instances when an agent claims support for \(\phi\) from claimed support that \emph{S} has the ability to \emph{V} that \(\phi\) because the latter ensures that there is some property \emph{P} holds of \emph{S} and \emph{P} entails \(\phi\).
  If the agent has the ability to \emph{V} that \(\phi\), then there may also be some action, \emph{V}ing, that the agent may witness.
  However, as \AR{} appeals to some property, the witnessing event is irrelevant to the way in which the agent claims support for \(\phi\).
\end{note}

\begin{note}
  {
    \color{red}
    Some additional notes on \AR{} that haven't been merged with the above follow.
  }
\end{note}

\begin{note}[Compatibility]
  \AR{} suggests an alternative way to obtain support for the conclusion of reasoning the agent is able to do.
  Specifically, if order for the agent to \emph{have} the attribute of being able to reason to the conclusion, the conclusion of the reasoning must be true.
  The relevant entailment is in part secured by the verb chosen, and in part by what the verb is applied to.
  Here, `demonstrate' is a factive verb, if an agent demonstrates \(\phi\) has value \(v\), then \(\phi\) has value \(v\).
  And, the existence of a chess strategy does not depend on the agent demonstrating that the relevant strategy exists.

  To take another example, you only have the ability to identify a typo on this page if there is a typo on this page.
  So, if I were to provide you with testimony that you have the ability to identify a typo on this page, you may begin searching for the typo, or you may note that there must be a typo in order for me to be in a position to provide you with testimony that you have the ability.
\end{note}

\begin{note}[Sketch of \AR{}]
  \begin{enumerate}[label=(\textsf{A}\arabic*), ref=(\textsf{A}\arabic*)]
  \item\label{AR:Sketch:1} I have the attribute of being able to \emph{V} that \(\phi\).
  \item\label{AR:Sketch:2} In order to have the attribute of being able to \emph{V} that \(\phi\), \(\phi\) must be the case independent of whether or not I witness the ability.
  \item\label{AR:Sketch:3} \(\phi\) is the case.
  \end{enumerate}

  To keep things simple, we will refer to the principle behind the pattern sketched as \AR{}.
  And agent may bundle~\ref{AR:Sketch:1} and~\ref{AR:Sketch:3} into a conditional, and avoid instantiating the reasoning pattern, but so long as the conditional is (implicitly) held on the basis of the intermediate premise~\ref{AR:Sketch:2}, we take use of such a conditional to be an instance of \AR{}.
\end{note}


\subsection{\WR{}}
\label{sec:wr-1}

\begin{note}[\WR{} def.]
  {
    \color{red}
    Include: observation that the entailment may come from some property of the agent.
    The point of \WR{} is that the agent claims support for details of the event.
  }

  We now turn to \WR{}.
  \begin{restatable}[\WR{}]{definition}{defWitnessing}\label{WR:def}
        An agent's reasoning with an instance of \aben{the} by claiming support for \(\phi\) from \emph{S} having ability to \emph{V} that \(\phi\) is an instance of \emph{\WR{}} when the agent holds that:
    \begin{enumerate}
    \item\label{WR:def:1} \emph{S} has the ability to \emph{V} that \(\phi\) \emph{if and only if} there is a potential event in which \emph{S} witnesses the act of \emph{V}ing that \(\phi\).
    \item\label{WR:def:2} Claim support for event or details of event.
    \item\label{WR:def:3} Details of the event in which \emph{S} witnesses the act of \emph{V}ing that \(\phi\), or part of the event, entails \(\phi\) is the case.\nolinebreak
      \footnote{Again, intuitively, because there could not be a potential event in which \emph{S} witnesses the act of \emph{V}ing that \(\phi\) without \(\phi\) already being the case.
      The notion of entailment here does not require \(\phi\) is true \emph{because} there is some potential event of the relevant kind.}
    \end{enumerate}
  \end{restatable}

  {
    \color{red}
    ~\textcite{Rebuschi:2011ub} talk about \emph{de objecto} attitudes.
    This might be helpful given that the events are potential.
  }

  {
    \color{red}
    Key idea with \WR{} is that the agent appeals to certain things which follow from the event being witnessed.
    Whereas, \AR{} appeals to certain things which must be the case in order for the event to be witnessed.
  }

  {
    \color{red}
    Difference between the existence of an event (~\ref{WR:def:1}) and details of the event (~\ref{WR:def:2}).
    To clarify.
    \(\exists e(V(e) \land \text{agent} = \emph{S} \dots)\).
    \(\phi\) follows.
    However, there are two ways to think about this.
    First, the existential, second the event.
    \emph{De dicto} and \emph{de re}.
    \WR{} is \emph{de re}.

    Consider existential of individuals.
  }

  {
    \color{green}
    Before going into the details, it'll be helpful to highlight the big idea, especially with respect to how things (will) work out with the `master property' from \AR{}.
  }

  \WR{} identifies instances of reasoning in which an agent applies \aben{the} by holding that \emph{S} having the ability to \emph{V} that \(\phi\) ensures there is a possible event in which \emph{S} \emph{V}s that \(\phi\).
  And, in turn, there is a possible event in which \emph{S} \emph{V}s that \(\phi\) entails \(\phi\) is the case.
  In contrast to \AR{}, when an agent claims support as an instance of \WR{} an agent reasons about what must be the case in order for \emph{S} to witness some ability, rather than what must be the case in order for \emph{S} to have the property of possessing some ability.


  We use the term `potential' in place of `possible' when describing the relevant event to highlight that the existence of the event is tied to an ability attribution.
  One may hold that a possible event is any event which is not impossible, and hence it is possible for an arbitrary agent to prove Fermat's Last Theorem.
  Yet, it seems most agent's lack the ability to prove Fermat's Last Theorem, and so `potential' serves to restrict quantifier over events which an agent has the ability to witness --- however the details of that quantification are resolved.

  {
    \color{red}
    \WR{} is more complex than \AR{}.
    There is some action that \emph{S} may witness.
    And, understand what the result of that action is.
    So, we have something akin to a counterfactual.
    However, the counterfactual only relies on witnessing.
    Further, particular status of \(\phi\).
    Hence, as witnessing is the only issue, \(\phi\) is the case.

    Third, regardless.
    \(\phi\) holds regardless, but it does not follow from this that if the agent reasons via \WR{} then support claimed for \(\phi\) would be independent of ability information.
    The agent must recognise \(\phi\) must be the case regardless, but this doesn't require that the agent has any way of reasoning to \(\phi\) other than by witnessing their ability.
    The point is clearer when considering witnessed instances of reasoning.
    \emph{X} testified that \emph{p}.
    Claim support for \emph{p}.
    \emph{p} is not the case because \emph{X} testified that \emph{p}, though my only path to claim support is by appeal to the testimony of \emph{X}.
  }
  To illustrate.

  \begin{illustration}
    I have the ability to calculate \(243 \div 3 = 82\).
    Pen and paper to hand, etc.\
    Result of this will be a calculation that \(243 \div 3 = 82\).
    However, my calculation is irrelevant to whether it is the case that \(243 \div 3 = 82\).
    Hence, it follows that \(243 \div 3 = 82\).
  \end{illustration}

  \begin{illustration}
    Ability to discover that the ball is under the left cup.
    Raise the left cup, and identify the ball.
    Whether or not the ball is under the left cup is independent of this sequence of actions, and therefore it follows that the ball is under the left cup.
  \end{illustration}

  Compare to cases where only gets the counterfactual.
  I have the ability to make it so that the heating is turned out.
  Plausibly, the heating is not on, and depends on witnessing the action of `making it so'.
\end{note}

\begin{note}[`Available resources']
  Delicate.
  Focus is on the witnessing event.
  However, mere possibility isn't sufficient for \aben{the}.
  So, some restriction.
  That is, an account of what makes the witnessing event a \emph{potential} event rather than a \emph{possible} event.
  One way to express this idea is that included in appeal to potential witnessing event is that sufficient resources are available.
  Here, the idea is that nothing further is required for the event to take place.

  This redescription falls short of an analysis as we've shifted the work from `potential' to `available'.
  Still, room for an analogy.
  Consider running a 5K.
  Here, going to require a whole bunch of energy.
  The agent does not `have' the energy.
  However, resources to generate energy.
  Fat reserves, muscle density, and so on.
  In this sense, sufficient resources are available, but not something the agent has.

  \AR{}, whatever it is that generates the sufficient resources.
  \WR{}, the result of having generated the sufficient resources.

  So, the difference between \AR{} and \WR{} isn't necessarily with what the two interpretations reduce down to, but is rather a difference with respect to what the interpretations focus on.
  From \AR{}, the stuff that's true right now, the generator, does the work.
  From \WR{}, it's what will be generated.

  There's still an important difference, though.
  Our interest is in reasoning.
  We are interested in what the agent appeals to.

  Key difference.
  \AR{}, that there is stuff the agent has which will generate.
  \WR{}, that what is generated from the stuff the agent has will do the work.

  The impact of this distinction will be expanded up with respect to \gsi{}.
\end{note}

\subsubsection{\AR{} and \WR{}}

\begin{note}
  Key idea is that \AR{} and \WR{} are different perspectives on the same thing.

  Switching between ability and potential events.
  This is not important, two ways of describing the same thing.

  The ability to \emph{V} that \(\phi\) is equivalent to there being a potential event in which the agent \emph{V}s that \(\phi\).
  For, if there is no such potential event, then the agent does not have the ability to \emph{V} that \(\phi\).
  Conversely, if there is a potential event in which the agent \emph{V}s that \(\phi\), then the agent has the ability to \emph{V} that \(\phi\).
\end{note}

\subsection{\AR{} and \WR{} are exhaustive}
\label{sec:ability-exhaustive}

\begin{note}[Exhaustive]
  \begin{restatable}[]{proposition}{propAbilityExuastive}
    \label{prop:WR-and-AR-exhaustive}
    \label{either-AR-or-WR}
    Any interpretations of an agent's (specific) ability to \emph{V} that \(\phi\) (for which \aben{the} holds) conforms to either:
    \begin{enumerate}
    \item
      \AR{}: It is a property of the agent that they are able to \emph{V} that \(\phi\).
    \item
      \WR{}: There is a potential witnessing event in which the agent \emph{V}s that \(\phi\).
    \end{enumerate}
    \vspace{-\baselineskip}
  \end{restatable}
\end{note}

\begin{note}
  The distinction between \AR{} and \WR{} sets up two (schematic) ways in which agent an agent may claim support given an instance of \aben{the}.
  We now argue that these two (schematic) methods are exhaustive.
\end{note}

\begin{note}[Old arguments]
  Remaining issue is details of the schemas.
  These talk about more than mere reference.
  \AR{}, agent, and \WR{} the result of the witnessing event.
  In turn, these are harmless and the only plausible option.

  \AR{} is simple.
  State of affairs, but as the agent is involved, then it is natural to attribute to the agent.
  Implausible that it's some event.

  \WR{} focuses attention to culmination of event.
  However, need culmination.
  Quirk of English that may `use' relevant verbs in this way.
  Imperfective paradox.
  May consider this a state, but only in the sense that it is a state bought about by some event.
  Focus on event, but given culmination, consider this a state.
  Still, state of culminated event.
  Possible that this is simply a state in which the agent has some appropriate relation.
  Problem is that an ability is the ability to do some thing.
  If abstract away from the act, then it's not clear how to understand conditions as identifying ability.
\end{note}

\section{Recap}
\label{sec:recap-reasoning}

\begin{note}
  Second is something like evidence of evidence is evidence.

  Here, the important difference is that the agent only needs to appeal to general ability.
  And, they've claimed support for this.

  The point is that it's not clear the agent is required to do anything too much with the specific ability.
\end{note}

\begin{note}[Summarising]
  Above we introduced \gsi{}.
  Limited information of the form `If \emph{S} has a (general) ability to \(\gamma\), then \emph{S} has a (specific) ability to \emph{V} that \(\phi\) (as an instance of the general ability).'
  We then noted that certain instances of the (specific) ability to \emph{V} that \(\phi\) entail \(\phi\) is the case.
  Two interpretations of \aben{the}, \AR{} and \WR{}.

  Our focus now turns back to \gsi{}.
  For those instances of \gsi{} when \aben{the} holds, the interpretations \AR{} and \WR{} detail what the agent obtains by reasoning from general to specific ability.
  In other words, \emph{what} the agent is claiming support for.

  As noted, using a conditional such as \gsi{} is not automatic.
  The informer has not provided the agent with any additional way to claim support that the agent has the general ability.
  Rather, outlined something that follows \emph{if} the agent has the general ability.

  So, it is up to the agent to resolve in either way.
  If the agent wants to use the information, then the agent needs to reason from general to specific.
  The issue is that without any additional reasoning, it seems there's no clear way to determine which way the agent should go.
  Here is where the distinction between \AR{} and \WR{} is important.
  Interpretation of specific ability informs how the agent move from general to specific.

  Following two propositions outline combination.
  {
    \color{red}
    The key thing here is about claiming that one has a specific ability.
  }
\end{note}

\begin{note}[\gsi{}++]
  First, \gsi{} applied to \AR{}
  \begin{restatable}[\textsf{|gs-I\space·\space H|}]{idea}{ideaCSbyAR}\label{idea:CS-by-AR}
    % In order for \emph{S} to have the (specific) ability to \emph{V} that \(\phi\) for which \aben{the} holds, claimed support for general and claimed support for \gsi{} are sufficient to claim support that \emph{S} has the property of being able to \emph{V} that \(\phi\).
    Suppose an agent has:
    \begin{enumerate}
    \item Claimed support for some general ability \(\gamma\).
    \item Claimed support that if they have the general ability \(\gamma\) then they have some specific ability to \emph{V} that \(\phi\) (for which \aben{the} holds).
    \end{enumerate}
    Then:
    \begin{enumerate}[resume]
    \item \emph{S} may claim support for having the specific ability \(\sigma\) by reasoning that they have the property of being able to \emph{V} that \(\phi\).
    \end{enumerate}
    \vspace{-\baselineskip}
  \end{restatable}

  Second, \gsi{} applied to \WR{}

  \begin{restatable}[\textsf{|gs-I\space·\space W|}]{idea}{ideaCSbyWR}\label{idea:CS-by-WR}\label{W:s}
    % In order for \emph{S} to have the (specific) ability to \emph{V} that \(\phi\) for which \aben{the} holds, claimed support for general and claimed support for \gsi{} are sufficient to claim support that there is a potential witnessing event in which \emph{S} \emph{V}s that \(\phi\).
    Suppose an agent has claimed support for some general ability \(\gamma\) and has claimed support that if they have the general ability \(\gamma\) then they have some specific ability to \emph{V} that \(\phi\) for which \aben{the} holds.
    Then, an agent may claim support for having the specific ability \(\sigma\) by reasoning that there is a potential witnessing event in which \emph{S} \emph{V}s that \(\phi\).
  \end{restatable}
\end{note}

\begin{note}[Alternatives]
  Appeal to premises and steps is not required by either \AR{} or \WR{}.
  However, most plausible account of what is going on.

  Explored some alternatives for \AR{}, but unclear what is of importance other than reasoning, and hence premises and steps.
  And, in this respect, basic \AR{} seems like a dead end.
  Premises and steps allow the agent to claim support in the same way as they would allow the agent to claim support when used in reasoning.
  It's not at all clear to me that basic \AR{} makes sense from this perspective.
\end{note}

\begin{note}[Limitation of intuition]
  Focused on idea that claiming support in same way as reasoning.

  This is not to imply equivalence of claimed support.

  Said too little about claimed support to make any strong remarks about equivalence.
  Still, intuitive that additional way of being \mom{}.
  For, haven't done the reasoning, so \mom{} about this.
  Not the case if the agent has done the reasoning.
\end{note}



%%% Local Variables:
%%% mode: latex
%%% TeX-master: "master"
%%% End:

\part{Temporary}

\chapter{Notes}
\label{cha:notes}

\paragraph{Reflection}

\begin{note}[Reflection]
  \begin{quote}
    Reflection states that agents should treat their future selves as experts or, roughly, that an agent’s current credence in any proposition A should equal his or her expected future credence in A.%
    \mbox{ }\hfill\mbox{(\citeyear[59]{Briggs:2009up})}
  \end{quote}
\end{note}

\begin{note}[Difference to reflection]
  Key difference is that in these cases, there's no guarantee that the agent will go through with ability.
  So, it's not necessarily a future self of the agent.
  Though, that's only on a quick surface reading of reflection.

  This is somewhat delicate.
  For, reflection has some strong background assumptions.
  Problem with ability is that agent might witness ability.
  With reflection, we don't consider restrictions on the reasoning the agent would do.

  Now, weakening reflection is difficult.

  One the one hand, can consider all evidence that the agent would reason through.
  If so, then it looks as though ability is going to fall within the scope.
  Problem here, however, because the argument for reflection is in terms of coherence.
  And, it's not clear how to apply conditionalisation to boundedness.
  Dutch books are about coherent credence functions.
\end{note}

\paragraph{\zS{}: Probability and norms}

\begin{note}
  The set-up here is whether it makes sense to ask a slightly weaker question than \qzS{}.
  For example, it is likely, or would the agent be violating a norm.
\end{note}

\begin{note}
  Might think that whether the agent would fail to conclude isn't really the issue.
  Instead, from present epistemic state the agent consider it sufficiently likely that the would conclude.

  Here, in particular, we don't run into the \requ{} problem, because it might be the case that the agent would fail.
  So, actually doing the reasoning doesn't work to check this.

  Though, I don't think this is right.
  For, it seems that in the lost keys type of case, it really is the possibility of coming to a distinct conclusion.

  Really, the key idea is that we're dealing with \emph{concluding}.
  This, in turn, places a consistency constraint.
  Now concluding \(\pv{\phi}{v}\) and \(\pv{\phi}{\overline{v}}\).
  This doesn't admit of coming to a different conclusion.

  This then applies no matter whether the value \(v\) concerns the proposition being true, or the proposition being sufficiently likely.

  Basic constraint on concluding, and hence what motivates the worry about reasoning to a different conclusion.
\end{note}

\begin{note}
  Press this objection further.
  As the agent has not done the reasoning, then even if \fc{0}, there's still a chance.

  Well, the key response to this is the way I end up understanding these core cases.
  It's ability.
  However, press the issue.
  Well, look, if the understanding of ability is right, then the idea being pressed comes down to the agent revising their present epistemic state.
  And, \emph{this} is compatible with everything said.
  The agent doesn't have the ability.
  Oh no!
\end{note}

\begin{note}
  A different variation on the same idea is that the key parts of \zS{} just express a norm.
  So, in ideal cases, it will be the case that the agent would not conclude, but in standard cases, it is permissible for the agent to fall short of this norm.

  Parallel, norm of knowledge for assertion.
  Something like this.
  Clearly violated, but this doesn't prevent the norm being in effect.

  Norm of only concluding when would not conclude otherwise, this then is in effect, but doesn't prevent an agent from concluding.

  But this seems strange, and stranger than probability.
  For, in these cases, determine whether failing to live up to the norm.

  I think, if there is motivation for \zS{}, then it suggests that this idea doesn't really work.

  And, at least in the case of ability, plausibly satisfy both this and original, stronger, \qzS{}.
\end{note}



%%% Local Variables:
%%% mode: latex
%%% TeX-master: "master"
%%% End:


\chapter{TempExamples}
\label{cha:tempexamples}

\begin{note}
  \begin{illustration}[Sudoku]
    \label{illu:gist:sudoku}
    % https://tex.stackexchange.com/questions/91422/tikz-sudoku-circle-and-connect-with-lines-some-cells
    \begin{figure}[H]
      \mbox{ }\hfill
      \begin{subfigure}{0.45\linewidth}
        \centering
        \sudokuGrid{}
        \caption{The starting grid \dots}
        \label{fig:sudoku:grid}
      \end{subfigure}
      \hfill
      \begin{subfigure}{0.45\linewidth}
        \centering
        \sudokuGridHints{}
        \caption{\dots with hints}
        \label{fig:sudoku:hint}
      \end{subfigure}
      \hfill\mbox{ }
      \caption{Sudoku}
      \label{fig:sudoku}
    \end{figure}
    An agent presses the `hint' button next to the Sudoku grid shown in~\autoref{fig:sudoku:grid}
    The Sudoku grid is updated as in~\autoref{fig:sudoku:hint}.

    The agent concludes that the current state of the grid rules out \(1\) as a candidate for the bottom left cell of the bottom left square.
  \end{illustration}
\end{note}


% \begin{note}[Illustration of \USE{}]
%   \begin{illustration}
%     \label{ill:rectangle:basic}
%     \mbox{}
%     \vspace{-\baselineskip}
%     \begin{enumerate}
%     \item The length of this rectangle measures \(19\text{cm}\) and the breadth of this rectangle measures \(7\text{cm}\).
%     \item So, the length of this rectangle is \(19\text{cm}\) and the breadth of this rectangle is \(7\text{cm}\).
%     \item
%       It is possible that my measuring device in inaccurate, but I purchased it from a reliable hardware store.
%     \item
%       To calculate the area of a rectangle, one multiples the length of the rectangle by breadth.
%     \item
%       \label{ill:rectangle:basic:reasoning}
%       \(19\text{cm}\) multiplied by \(7\text{cm}\) is \(133\text{cm}^{2}\).
%     \item
%       So, the area of the rectangle is \(133\text{cm}^{2}\).
%     \end{enumerate}
%     \vspace{-\baselineskip}
%   \end{illustration}
% \end{note}

\subparagraph{Theorem \illu{1}}

\begin{note}[A few illustrations]
  Let us now turn to a few illustrations before discussing \nI{} in further detail.

  We'll begin with a somewhat detailed illustration.
  \nI{} identifies a particular way in which an agent may fail to claim support, and the primary goal of the initial demonstration is to highlight why the agent would fail to claim support.
  Hence, the illustration treads a fine line between highlighting a problematic method, but not necessarily a problematic result.
  This is by design.
  And, I will continue to stress that \nI{} concerns a way of claiming support for some proposition, rather than the possibility of claiming support for some proposition.

  Following two illustrations will be variations on the initial.

  Still, it may be helpful to observe how \nI{} relates to an intuitively problematic result.
  Therefore, we will provide an additional, simple, illustration of a failure to claim support.

  The final illustration in the trio will complement the initial par of illustrations by highlighting an instance where~\nI{} does not apply.

  \phantlabel{dogmatism-wrt-nI}
  The reader may note structural similarities between these illustrations and \citeauthor{Kripke:2011wv}'s Dogmatism paradox.
  We will discuss the relation after the illustrations.
\end{note}

\paragraph{First}

\begin{note}[Brief illustration of \nI{}]
  The first illustration considers theories and counterexamples.

  \begin{illustration}
    \label{ill:CE:main}
    Suppose a researcher have constructed a theory of some general phenomenon.

    The theory seems to capture the phenomenon, and the researcher has claimed (inductive) support that the theory is adequate by applying it to various instances of the general phenomenon.
    Even if the theory isn't adequate, the theory has been (seemingly) successful applied to sufficient specific instances of the phenomenon.
    Hence, even if \mom{}.

    However, as the phenomenon is a \emph{general} phenomenon it also makes various predictions about what must happen in all other instances to which the researcher has not (yet, at least) applied the theory to.

    Hence, there is a possible counterexample to the theory associated with each instance the researcher has not (yet) applied the theory to.
    If some particular instance does not conform to the theory, the theory is inadequate.
    Conversely, if the theory is adequate, every particular instance of the phenomenon conforms the theory.
    In other words, if the theory is adequate, then there are no counterexamples to the theory.

    Of course, it may be simple to revise the theory is a counterexample exists, and the fundamental ideas of the theory may remain sound (\cite[Cf.][]{Bonevac:2011tz}).
    And, the theory may have sufficient resources to explain why any apparent counterexample is not a counterexample.
    Yet, it remains the case that the theory would need to be revised in light of a counterexample.

    Now, to summarise, the researcher may claim support for two propositions allow the agent to claim support that there are no counterexamples.

    \begin{enumerate}
    \item The theory is adequate, and
    \item If the theory is adequate, then there are no counterexamples.
    \end{enumerate}

    At issue is whether the researcher may claim support that there are no counterexamples to the theory from the claimed support for the two propositions in the following way:

    \begin{enumerate}
    \item I have claimed support that the theory is adequate.
    \item So, given the claimed support, theory is adequate.
    \item Therefore, as the theory is adequate, given the claimed support, it follows that there are no counterexamples.
    \item Hence, I claim support that there are no counterexamples to the theory.
    \end{enumerate}
    \vspace{-\baselineskip}
  \end{illustration}
\end{note}

\begin{note}[Seems problematic]
  Seems problematic.
  Claimed support that the theory is adequate is qualified by the possibility of counterexamples.
  {
    \color{red}
    Note, agent is, here, only claiming support that there are no counterexamples.
    And, claiming support may be \mom{}.
    So, it does not follow that the agent is ruling out the possibility of counterexamples to the theory.
    Plausible that the agent \emph{may} claim support.
    Problem is the way in which the agent goes about this.
  }

  {
    Even if not convinced about support, this way of claiming support seems problematic.
    Relying on theory being adequate.
    However, if this is the case, then no possible counterexamples.
    Issue is that such counterexamples are possible given the state of your claimed support.
    Hence, claiming support in this way seems to take for granted that there are no counterexamples.
  }

  Problem is that the reasoning only works if there are no counterexamples.
  If there are counterexamples, misled.
  Hence, problem to go from the theory is adequate.
  However, without this step, researcher doesn't get to no counterexamples.
\end{note}

\begin{note}
  So, relation between theory and counterexample \emph{undercuts} using way of using theory to get no counterexample.

  Now, given that the researcher has claimed support that the theory is adequate, the researcher may \emph{expect} that there are no counterexamples to the theory.
  And, it doesn't follow that the researcher may not claim support.
  Plausible that details of the theory provide some way of claiming support.

  Indeed, it seems the researcher is require to take the alternative path --- to show that the proposed counterexample is accounted for by the contents of the theory, regardless of whether the theory is true.

  Fault here is with respect to \ideaCS{}.
  {
    \color{red}
    Here, conditions of~{\color{red} inclusion} are satisfied, but we did not explicitly appeal to them.
    Purpose of~{\color{red} inclusion} is conditions sufficient for this kind of problem to arise.
    So, to do in argument for \nI{} is to develop is why~{\color{red} inclusion} does something similar.
    Upshot is that \nI{} is general.

    In the third illustration, we'll see why the way of claiming support is okay in some cases.
  }
  Difficult part is to account for why~{\color{red} inclusion} sets things up and ensures that things don't go too far.
\end{note}

\paragraph{Second}

\begin{note}[Idea main part of \nI{} works]
  As noted above, it is unclear whether or not there may be some way for the researcher to claim that there are no counterexamples to the theory.

  In other words, one may be wondering whether \ideaCS{} is a plausible constraint on claiming support.
  We gave a general argument for \ideaCS{} in~\autoref{cha:claiming-support}.
  However, it may help to see how the issue highlighted relates to an intuitively problematic instance of reasoning, regardless of how support is claimed.

  \begin{illustration}\label{ill:CE:colleague}
    Suppose a colleague has studied the researcher's theory, and they (the colleague) thinks they have found a counterexample.

    The colleague has informed the researcher that they think they have observed a counterexample.

    However, the colleague has not provided the researcher with any further details about the counterexample.

    Now, the conditional of interest may be made more precise:
    \begin{enumerate}
    \item If the theory is adequate, then the colleague has failed to identify a counterexample to the theory.
    \end{enumerate}

    Now, let's replicate the way of claiming support from before.

    \begin{enumerate}
    \item I have claimed support that the theory is adequate.
    \item So, given the claimed support, theory is adequate.
    \item Therefore, as the theory is adequate, given the claimed support, it follows that the colleague has failed to identify a counterexample to the theory.
    \item Hence, I claim support that the colleague has failed to identify a counterexample to the theory.
    \end{enumerate}
    \vspace{-\baselineskip}
  \end{illustration}

  I take this illustration to be intuitively problematic.
  In short, if claim support, then doesn't need to examine counterexample to claim support that it is not a counterexample.

  Possible response is that researcher does claim support, but information colleague impacts claimed support for theory.
  However, this is also puzzling.
  Researcher has no information.
  Hence, if retain confidence, then equally against counterexample.
  And, if does not retain confidence, then down the theory in a way that seems implausible.

  Seems, instead, that claimed support for theory persists, but that this doesn't extend to counterexample.\nolinebreak
  \footnote{
    Inclined to apply this to previous illustration.

    However, there's a difference between two illustrations.
    Here, someone (the colleague) has reason to think there is a counterexample, and this seems a sufficiently important difference to draw any quick conclusions.
    And, as we don't require a resolution to this issue, I won't explore further.
  }

  Perhaps more detail is needed.
  I have some doubts that claiming support is always bad.
  However, clearer that developed in a way such that problem remains.

  Now, seems that the researcher doesn't get to claim support because if counterexample, then theory is bad.
  Hence, requires that counterexample is not true in order to progress.
  But, then, doesn't make the move regardless of whether or not there is a counterexample.

  So, it seems \ideaCS{} does the work.
\end{note}


\begin{note}
  ``Undercuts using \(\phi\) for \(\psi\).''
  Same problem, failure of \ideaCS{}.

  For, the agent has already `assumed' that they may reason.

  Problem is that the agent doesn't get to claim support for \(\psi\) because fail the \ideaCS{} thing.
  If \(\psi\) isn't really the case, then reasoning collapses.
  Key thing about our understanding of claimed support is that it holds up even if the agent is \mom{} about the value of the proposition.

  {
    \color{red}
    Note:
    There's possible tension here.
    It seems that if the first illustration is okay, then this (second) illustration should also be okay.
    Maybe.
    But, this is too quick.
    Additional information here.

    Now, still some difficulty, as I think \EAS{} might apply to the first.
    So, shouldn't it apply to this?
    Well, no.
    For, \EAS{} only suggests possibility in some cases.
    Fine to think of this additional information as constraint on appeal via ability.
    For, if the colleague thinks they've found a counterexample, then this suggests a problem with the agent's ability.
  }
\end{note}

\paragraph{Third}

\begin{note}[Variation where \nI{} does not apply]
  \begin{illustration}\label{ill:CE:testimony}
    Suppose the researcher has published a paper containing the details of the theory.

    Our attention now turns to a novice who has read far enough into the paper to understand, at least, the general phenomenon that the theory applies to and that the researcher has claimed inductive support for the theory.
    We'll also assume that the novice does not possess the expertise required to apply the theory.\nolinebreak
    \footnote{
      Though I don't think this assumption is important.
    }

    The novice is thinking about instances of the general phenomenon, and identifies one.

    The conditional of interest is:
    \begin{enumerate}
    \item If theory is adequate, it accounts for this instance of the phenomenon.
    \end{enumerate}

    Of course, the novice also recognises that the theory is inadequate if it  does not account for the particular instance of the phenomenon.
    Still, the novice claims support in the familiar manner.

    \begin{enumerate}
    \item I have claimed support that the theory is adequate (this time by reading a published paper).
    \item So, given the claimed support, the theory is adequate.
    \item Therefore, as the theory is adequate, given the claimed support, it follows that the theory accounts for this instance of the phenomenon.
    \item Hence, I claim support that the theory accounts for this instance of the phenomenon.
    \end{enumerate}
    \vspace{-\baselineskip}
  \end{illustration}

  In contrast to the previous illustrations, it seems the novice may claim support in such a way.

  Possibility of being either \mom{} remains.
  Still, not in position to reason through theory and phenomena.
  Hence, claiming support from something like status of peer review --- or testimony.
  And, not accounting would not show peer review is bad.
\end{note}

\paragraph{Summary of illustration and variations}

\begin{note}
  These three illustrations.
  First, kind of scenario that's the main interest.
  Where claiming support in a certain way seems problematic, even if it not clear that the agent may claim support in some other way.

  To stress the problem, considered a cleaner case, where it seems agent may not claim support, and argued that same problem is a plausible account of why.

  Third illustration, way of claiming support is okay.
  As all instances of \nI{}, and hence the previous two illustrations, focus on particular way of claiming support illustrated that it's okay.
\end{note}

\begin{note}[Intuition]
  In short, \nI{} captures a limitation: An agent is not in a position to claim support for some proposition \(\psi\) when circumstances are such that the claimed support requires (from agent's point of view) that the agent is already in position to claim support for \(\psi\).

  No claiming support for\(\psi\) if failure to establish support for \(\psi\) independently of the value of \(\phi\) would reveal problem with the support claim for \(\phi\).

  Hence, \nI{} focuses on when an agent may claim support for some proposition by noting that (from the agent's perspective) that the value of the proposition is determined by further propositions the agent has claimed support for.

  Some other way of claiming support for \(\psi\).
  However, not merely an alternative path, but an alternative path that must be possible given claimed support for \(\phi\).

  Issue is that given {\color{red} background} and~{\color{red} inclusion}, agent expect that they have the resources, and hence expects \(\psi\) is the case.

  So, that \(\phi\) has value \(v\).
  In doing so, resources to claim support for \(\psi\) has value \(v'\).
  Hence, \(\psi\) has value \(v'\).
  So, \(\psi\) having value \(v'\) is a requirement on claimed support for \(\phi\) being any good.
  However, no support claimed for \(\psi\) having value \(v'\).

  In cases of reasoning with a conditional, such as the illustrations given, that value of \(\phi\) constrains value of \(\psi\) is in general helpful information, but in these specific cases it does not help the agent claim support for \(\psi\) having value \(v'\) because if \(\psi\) isn't already so constrained, then no appeal to \(\phi\) having value \(v\).

  Similar to other principles, failure because establishing something that needs to be the case in order to be in a position to establish.
\end{note}

\begin{note}[Illustration, testimony]
  To illustrate, consider expert testimony to a layperson.
  Suppose you, the expert, have testified to me, the layperson, that there are exactly five intermediate logics that have the interpolation property.\nolinebreak
  \footnote{Cf.\ \textcite{Maksimova:1977un}}
  From this it follows that there is an intermediate logics that has the interpolation property.

  However, I am quite confident that I would not be in a position to claim support for the latter proposition without your testimony.
  So, given that I do not have the expertise involved, any failure by me to claim support that there is a intermediate logic with the interpolation property is uninformative.
\end{note}

\begin{note}
  Still, an issue arises if we are both experts.

  To illustrate, suppose you and I are both experts.
  You claim to have developed a sound and complete proof system for an logic and presented me with a paper containing the system and a proof.
  Given that I have the paper and the expertise, I am confident that I would be mistaken or misled by your testimony if I am not in a position to claim support that the system is sound and complete by working through the paper.\nolinebreak
  \footnote{
    Here, complexity of understanding of having resources shows.
    For, it may be that the reader learns something new, a lemma etc.\ which could be considered a new resource.
    Likewise, one may think that it's fine to continue to follow testimony given a problematic proof as one is confident that the prover has the resources to revise the proof.
    If so, not clear whether conditional holds, and will depend having resources.
    If proof synthesises resources, then may still hold.
    If proof introduces new information, then conditional does not hold.

    No clear answer for these cases.
    Intend to be compatible with your understanding of resources.
    Will only take a stance on this when applying.
  }
\end{note}

\begin{note}
  \begin{illustration}
    \label{ideaS:illu:clock}
    Suppose an agent considers it \epPAd{} that the clock is not functioning, and has reasoned as follows:
    \begin{enumerate}
    \item The clock is in the centre of the library, and has been seen by many people.
    \item The clock is functioning.
    \end{enumerate}
    \vspace{-\baselineskip}
  \end{illustration}

  Still, the agent may observe that:
  \begin{enumerate}
    \setcounter{enumi}{-1}
  \item It is exam week so the library is busy and many students will be sensitive to what time it is.
  \end{enumerate}

  {
    \color{red}
    Possibly useful observation here is that it seems there are additional considerations which may `bolster' the agent's conclusion.
    However, this isn't a \requ{}.
  }
\end{note}

\begin{note}[Gifts]
  \begin{illustration}
    \label{illu:S:gifts}
    \begin{enumerate}
    \item S knows me very well.
    \item Whatever S has gifted me satisfies some desire I have.
    \end{enumerate}
  \end{illustration}
  Perhaps wishful thinking, but fine with respect to \zS{}.

  {
    \color{red}
    Maybe useful to include, as this shows how \zS{} is kind of weak.
    The key thing to think about is whether there is something that could lead the agent to conclude that their friend doesn't know them very well.
  }

  The same basic idea underlying~\autoref{illu:S:gifts} extends to various other premise-conclusion instances.
  Consider, for example:
  \begin{enumerate}
  \item The newspaper reported \(S\) said that \(p\).
  \item \(S\) said that \(p\).
  \end{enumerate}
  It seems plausible that the reasoning from premise to conclusion assumes that the paper has a history of accurate reporting.
  And, and if available to the agent are various premises which suggest that the paper does not have such a history, then the agent may refrain from concluding \(S\) said that \(p\) from the report.
  Still, even if the agent were to relax their epistemic state so that the newspaper lacking such a history is \epVAd{}, the agent may still have sufficient premises to conclude that the newspaper does have the history, those premises may resist questioning, and so on.
  Hence, whether or not an agent has \support{} for the conclusion broadly depends on the agent's epistemic state.
  Further, the agent may also have some other premise from which to conclude \(S\) said that \(p\).
  For example, by having been in the room when \(S\) said that \(p\).
\end{note}

\begin{note}
  \begin{illustration}
    Suppose the check engine light of the agent's car has come on.
    As a result the agent has concluded that their is some fault with their car's engine.

    Of course, the agent is aware that it is possible for the check engine light to be on with no fault in the engine.
    Still, even if the agent were to consider the possibility that their is no fault in the engine, the presence of the check engine light would lead the agent to conclude that there is some fault in the engine.
  \end{illustration}

  The agent's reasoning is non-deductive, and the agent recognises that by moving from the check engine light to a problem with the engine that they may conclude that there is a problem with the engine when there is none, but as the agent would reason regardless,~{\color{red} ???} is satisfied.
\end{note}

\begin{note}
  Pair of trivial instances.
  Inspection immediately grants.
  \begin{illustration}
    \begin{figure}[h!]
    \mbox{}\hfill
    \begin{subfigure}{0.45\linewidth}
      \begin{enumerate}
      \item That persons' eyes have been closed for a long time and their breathing is slow.
      \item That person is asleep.
      \end{enumerate}
      \caption{}
    \end{subfigure}
    \hfill
    \begin{subfigure}{0.45\linewidth}
      \begin{enumerate}
      \item The die has rolled even in \(5251\) of \(6000\) samples.
      \item The die is biased.
      \end{enumerate}
      \caption{}
    \end{subfigure}
    \hfill\mbox{}
    \caption{}
    \label{fig:ideaS:basic-examples}
  \end{figure}
\end{illustration}
\end{note}

\begin{note}
  Similar to verifying an algorithm may be implemented.
  Break down all of the steps in the algorithm, and then ensure that it is possible to express each of the steps in the programming language of choice.

  \begin{quote}
    \textsc{factorial}(\(n\)):\newline
    \textbf{if} \(n = 1\)\newline
    \mbox{}\indent \textbf{return} \(1\)\newline
    \textbf{else}\newline
    \mbox{}\indent \textbf{return} \(n \times\) \textsc{factorial}(\(n-1\))
  \end{quote}

  Fortran 77 does not support recursion, a function may not call an instance of itself.\nolinebreak
  \footnote{
    This is not to say that one may not compute factorials using Fortran 77.
    It's a Turing complete language.
    However, would require a different (non-recursive) algorithm.
  }
  By contrast, the recursive factorial algorithm may implemented in languages that support recursion, such as Lisp or Python.
\end{note}

% \begin{note}[Milk]
%   \begin{illustration}
%     Suppose \nagent{13} is interested in concluding that they're set to have coffee before travelling to work from the premises that they have the time and resources to make a cup of coffee.
%     And, the additional premise that they have milk.
%     (For, the milk will cool the coffee quickly enough for \nagent{13} to drink before leaving.)

%     Still, \nagent{13} is unsure about whether the milk is safe to drink.
%     The milk smells okay-ish and \nagent{13} is fairly sure that they bought it fairly recently.

%     \nagent{13} holds that if the milk is past it's expiry date, then it is not safe to drink.
%     But, if the milk is within it's expiry date, then as it smells okay-ish, \nagent{13} would conclude that the milk is safe to drink.
%   \end{illustration}
%   \epVAd{} that the milk is not safe to drink.
%   It is not \epVAd{} for \nagent{13} to conclude conclusion from premises if the milk is not safe to drink.
%   \nagent{13} is holding the milk, so trivial to check the expiry date.

%   It seems intuitive, at least to me, that \nagent{13} should check the expiry date.

%   For present purposes, this intuition is accounted for by there being an antecedent check on whether it makes sense for the agent to conclude.
%   The expiry date.

%   {
%     \color{red}
%     Here, revise to:
%     Bad if the milk is over a week old.
%     So, remember last time went shopping, and reason from there.

%     Then, observe that this also extends more straightforwardly to checking the expiration date.
%   }
% \end{note}

% % % Spot the difference % % %

\begin{note}[Spot the difference]
  \begin{illustration}[Spot the difference]
    \label{illu:CS:spot-the-diff}
    The agent has been working through a spot-the-difference to pass some time.

    Though the time is not completely passed, the agent examined the two images with what seems sufficient care to claim support that they have found all the differences.
    However, the agent did not keep track of the number of differences.

    The agent announces `I have found all the differences' and their companion responds `All fifteen?'.

    \begin{enumerate}[label=\arabic*., ref=(I\ref{illu:CS:spot-the-diff}.\arabic*)]
      \setcounter{enumi}{-1}
    \item
      \label{illu:CS:spot-the-diff:info}
      If I have found all the differences, I have found fifteen differences.
    \end{enumerate}

    The agent then reasons as follows:

    \begin{enumerate}[label=\arabic*., ref=(I\ref{illu:CS:spot-the-diff}.\arabic*), resume]
    \item Exhaustive search.
    \item
      \label{illu:CS:spot-the-diff:all}
      I found all the differences.
    % \item\label{illu:CS:spot-the-diff:info} My companion has testified that there are fifteen differences.
    % \item\label{illu:CS:spot-the-diff:cond} If I have found all the differences, I have found fifteen differences.
    \item
      \label{illu:CS:spot-the-diff:fif}
      So, I have found fifteen differences. \hfill (From \ref{illu:CS:spot-the-diff:info} and \ref{illu:CS:spot-the-diff:all})
    \end{enumerate}
  \end{illustration}

  Before going further, structure of this.

  The agent performed some reasoning, and concluded that they found all the differences.
  However, that reasoning is mentioned but not stated in the \illu{0}.
  Rather, present is distinct instance of reasoning after being provided with information.
  ``If not 15, then problem''.
  Present reasoning appeals to past reasoning, and draws out consequence of this given new information.
  Important: the present reasoning does not consider possibility that the agent did not find all 15 differences.
  Instead, consequence of conclusion of previous instance of reasoning.
  Still, epistemically possible that the agent did not find 15 differences.
\end{note}

\begin{note}
    Providing additional information about what the agent has claimed support for.
  Recall, \autoref{assu:CSVP}, information rather than \world{}.
  \nolinebreak
  \footnote{
    Still slight issue.
    Offering a redescription.
    You met Clark Kent, so you met Superman.
    In this case, rather than claiming support for meeting Superman, provided information is seen as an equivalent formulation.
    It is possible to read \autoref{illu:CS:spot-the-diff} in this way, and this might be the most natural interpretation.
    However, it is not the interpretation under which see the problem.
    Rather, problem is where the conditional is explicit.
    Unlike Superman case, proper conditional.
  }
\end{note}

\begin{note}
  Information leads to \requ{}.

  Possibility of not fifteen.
  And, not merely that the agent performed the reasoning, but that the reasoning identified all.
  If not fifteen, then not all, so would involve appeal to something that is not the case.

  And, present reasoning does not include reasoning about \requ{}.
\end{note}

\begin{note}
  \color{red}
  Though, this is interesting.
  For, the agent may have found fifteen.
  This, then, helps stress the point that it's not just reasoning to the conclusion.
\end{note}

% % % Wally % % %


\begin{note}[Wally]
  \begin{illustration}[Where's Wally]
    \label{illu:CS:wheres-wally}
    \nagent{15} has a book containing numerous drawings of bustling scenes in which various characters are doing a variety of things.
    And, somewhere in each scene is a character called `Wally', identifiable by a collection of individually necessary and jointly sufficient distinguishing features.
    These features include a red and white striped jumper, blue trousers, short brown wavy hair, and so on.

    \nagent{15} has searched through one particular scene, and has identified a character with a variety of the features.
    Before concluding that the character is Wally, \nagent{15} remembers that there is a picture of Wally On the cover of the book, with all the identifying features present.

    Wally is always wearing a pair of round glasses, but this was not a feature \nagent{15} kept in mind when searching for Wally, and it is \epVAd{} for \nagent{15} that the character they identified is not wearing round glasses  --- \nagent{15} only recalls the features they identified.
  \end{illustration}

  Our interest is, generally, in whether \nagent{15} may conclude from the variety of features identified that the character is Wally.

  Descriptively, of course, there seems no barrier.
  An agent may reason to an arbitrary conclusion to arbitrary premises.

  So, specifically, our interest is in whether \nagent{15} would claim support that the character is Wally by concluding that the character is Wally from the variety of features identified.

  The difficulty for \nagent{15} is that so long as they consider it \epVAd{} that the character is not wearing round glasses, then there a clear check on whether \nagent{15} may reason to a different conclusion.
  For, if \nagent{15} were to check whether the character is wearing a pair of round glasses, and the character is not wearing a pair of round glasses, then \nagent{15} would conclude that the character is not Wally.
\end{note}

\begin{note}
  {
    \color{red}
    Here, revise the Wally scenario to involve reasoning about the characteristic features of Wally from memory.

    Part of the interest here, then, is that in some cases there is an `internal parallel' to doing something that doesn't involve reasoning.
  }
\end{note}

%%% Local Variables:
%%% mode: latex
%%% TeX-master: "master"
%%% End:


% \part{Things}

% \chapter{Ideas, etc\dots restated}
\label{cha:restatable}

\subsection{Epistemic states}

\defEState*

\subsection{Claiming support}

\paragraph{Basic assumptions}

\assuCSVP*

\assuCSRR*

\assuIndicate*

Where:

\defIndicate*

\paragraph{Ideas}

\iCSA*

Where:

\defSink*

\ideaEIS*

\paragraph{Core assumption}

\subparagraph{Background for the core assumption: \requ{1}}

\ideaRequisite*

\defResult*

\defRequisite*

\defRequisiteP*

\defRequisiteC*

\defRequisiteCP*

\subparagraph{The core assumption}

\assuCSRReq*


\subsection{Claiming support and `use'}
\label{sec:claiming-support-use}

\paragraph{Basic idea}

\ideaUSE*

\paragraph*{Target}

\targetESU*

\paragraph*{Goal}

\goalEAS*

\thoughtEASw*


\hozline

\paragraph{???}

\ideaCSbyAR*

\ideaCSbyWR*

\subsection*{Definitions}

\defMoM*

\defGSI*

\defDSI*

\defAE*

\defAttribution*

\defWitnessing*

\defADA*

\defADB*

\subsection*{Propositions}

\propRecogniseDefeaters*

% \propCSNai*

\propScenariosExist*

\propAbilityExuastive*

\propNoESUandADB*

\propLCS*

\propFCS*

%%% Local Variables:
%%% mode: latex
%%% TeX-master: "master"
%%% End:


\end{document}

%%% Local Variables:
%%% mode: latex
%%% TeX-master: t
%%% End:
