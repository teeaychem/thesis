\makeatletter
\renewcommand{\PackageInfo}[2]{}% Remove package information
\renewcommand{\@font@info}[1]{}% Remove font information
% \renewcommand{\@latex@info}[1]{}% Remove LaTeX information
\makeatother

\PassOptionsToPackage{unicode}{hyperref}

\documentclass[10pt]{report}
% \usepackage[margin=1in]{geometry}
% \newcommand\hmmax{0}
% \newcommand\bmmax{0}
% % % Fonts% %
% \usepackage{luatexja}

\usepackage[T1]{fontenc}
\usepackage[complete, subscriptcorrection, slantedGreek, mtpfrak, mtpbb, mtpcal]{mtpro2}
\usepackage{bm}% Access to bold math symbols
\usepackage[no-math]{fontspec}
% \usepackage{fontspec}
\defaultfontfeatures{Ligatures=TeX}%,Numbers={Proportional}}
   % \protrudechars=2 % or \pdfprotrudechars=2 and
\adjustspacing=2 %    \pdfadjustspacing=2 with luatex < v0.85
\newfontfeature{Microtype}{protrusion=default;expansion=default;}
\directlua{fonts.protrusions.setups.default.factor=.5}
\setmainfont[Microtype,Ligatures=TeX,BoldFont={*-Semibold}]{Source Serif 4}
\setsansfont[Microtype,Scale=MatchLowercase,Ligatures=TeX,BoldFont={*-Semibold}]{Source Sans 3}
\setmonofont[Scale=0.8]{Source Code Pro}

% \usepackage{selnolig}% For suppressing certain typographic ligatures automaically
% % % % % % %

\usepackage{amsthm}         % (in part) For the defined environments
\usepackage{mathtools}      % Improves  on amsmaths/mtpro2
\usepackage{xfrac}

% \usepackage{silence}
% \WarningsOff[marginnote]% Suppress warnings related to package

\counterwithout{footnote}{chapter} % For continuous footnote numbering
\counterwithout{figure}{chapter}

% % % The bibliography % % %
\usepackage[%
  backend=biber,
  style=authoryear-comp,
  bibstyle=authoryear,
  citestyle=authoryear-comp,
  uniquename=false,
  backref=false,
  hyperref=true,
  url=false,
  isbn=false,
  doi=false,
  useprefix=true,
  ]{biblatex}
\DeclareFieldFormat{postnote}{#1}
\DeclareFieldFormat{multipostnote}{#1}
% \setlength\bibitemsep{1.5\itemsep}
\newcommand{\noopsort}[1]{}
\addbibresource{Ability.bib}
% % % % % % % % % % % % % % %

\usepackage[inline]{enumitem}
\setlist[enumerate]{noitemsep}
\setlist[description]{style=unboxed,leftmargin=\parindent,labelindent=\parindent,font=\normalfont\space}
\setlist[itemize]{noitemsep}

% % % Misc packages % % %
% \usepackage{setspace}
% \usepackage{refcheck} % Can be used for checking references
% \usepackage{lineno}   % For line numbers
% \usepackage{hyphenat} % For \hyp{} hyphenation command, and general hyphenation stuff
\usepackage{nonumonpart} % Disable page numbers on 'Part' title pages
% % % % % % % % % % % % %

% % % Red Math % % %
\usepackage[usenames, dvipsnames]{xcolor}
% \usepackage{everysel}
% \EverySelectfont{\color{black}}
% \everymath{\color{red}}
% \everydisplay{\color{black}}
\definecolor{fuchsia}{HTML}{FE4164}%Neon Fuchsia %{F535AA}%Neon Pink
\definecolor{details}{HTML}{FE4164}%Neon Fuchsia %{F535AA}%Neon Pink
\definecolor{expand}{HTML}{C61AFF}
\definecolor{later}{HTML}{A65978}
\definecolor{return}{HTML}{660066}
\definecolor{reword}{HTML}{F57F17}
% % % % % % % % % %

\usepackage[export]{adjustbox}
\usepackage{subcaption}
\usepackage{float}

% \usepackage{pifont}
% \newcommand{\hand}{\ding{43}}

\usepackage{tabularray} % For fancy table features

\usepackage{multirow}
% \usepackage{adjustbox}

\usepackage{multicol}

\setcounter{secnumdepth}{4}
\setcounter{tocdepth}{4}

% % % % % % % % % % % % TIKZ
\usepackage{tikz}
\usetikzlibrary{bending,arrows,calc,arrows.meta,patterns,fadings}
\usetikzlibrary{trees}
\usetikzlibrary{backgrounds, positioning, fit, backgrounds}
\usepackage{tikz-qtree} %for simple tree syntax
% \usetikzlibrary{positioning,shapes.multipart} %for structured nodes
\usetikzlibrary{tikzmark}
% % % % % % % % % % % %

\usepackage{graphicx} % for images (png/jpeg etc.)
\usepackage{caption} % for \caption* command

% \usepackage{svg}
% \usepackage[off]{svg-extract}
% \svgsetup{clean=true}

\usepackage{extarrows}

% % % % % % % % % % % % MY COMMANDS
\newcommand{\hozlinedash}[0]{%
  \noindent\hdashrule[0.5ex][c]{\textwidth}{.1pt}{2.5pt}
}

\def\nleadsto{\mathrel{%
    \mathchoice{\NLEADSTO}{\NLEADSTO}{\scriptsize\NLEADSTO}{\tiny\NLEADSTO}%
}}
\def\NLEADSTO{{%
    \setbox0\hbox{\leadsto}%
    \rlap{\hbox to \wd0{\hss\hspace{-10pt}\not\hss}}\box0
  }}

\usepackage{dashrule}
\newcommand{\hozline}[0]{%
  \noindent\hdashrule[0.5ex][c]{\textwidth}{.1pt}{}
}
% % % % % % % % % % % %
\usepackage{xskak} % For chess diagram
\usepackage{lplfitch} % For fitch proofs

% % % % % % % % % % % %

% % % My packages % % %
\usepackage{CustomTheoremsThesis}
% \usepackage{FuturePromisedEvents}
\usepackage{ThesisCustom}
\usepackage{Notes}
% % % % % % % % % % % %

\makeatletter
\renewcommand\paragraph{\@startsection{paragraph}{4}{\z@}%
  {-3.25ex\@plus -1ex \@minus -.2ex}%
  {1.5ex \@plus .2ex}%
  {\normalfont\normalsize\bfseries}}
\makeatother

\makeatletter
\renewcommand\subparagraph{\@startsection{subparagraph}{4}{\z@}%
  {-3.25ex\@plus -1ex \@minus -.2ex}%
  {1.5ex \@plus .2ex}%
  {\normalfont\normalsize\bfseries}}
\makeatother

% https://tex.stackexchange.com/questions/94402/creating-a-subsubparagraph
\makeatletter
\newcounter{subsubparagraph}[subparagraph]
\renewcommand\thesubsubparagraph{%
  \thesubparagraph.\@arabic\c@subsubparagraph}
\newcommand\subsubparagraph{%
  \@startsection{subsubparagraph}    % counter
    {6}                              % level
    {\z@}                     % indent
    {3.25ex \@plus 1ex \@minus .2ex} % beforeskip
    {1.5ex \@plus .2ex}%             % afterskip
    {\normalfont\normalsize\bfseries}}
\newcommand\l@subsubparagraph{\@dottedtocline{6}{10em}{5em}}
\newcommand{\subsubparagraphmark}[1]{}
\makeatother

\usepackage[hidelinks,breaklinks,bookmarks=false]{hyperref}

\usepackage[british]{babel}
\addto\extrasbritish{
    \def\chapterautorefname{Chapter}
    \def\sectionautorefname{Section}
    \def\subsectionautorefname{Section}
    \def\subsubsectionautorefname{Section}
}
%Following commands mean autoref writes `section' rather that `subsection', etc.
%https://tex.stackexchange.com/questions/177007/autoref-showing-subsection-and-subsubsection
\let\subsectionautorefname\sectionautorefname
\let\subsubsectionautorefname\sectionautorefname

\title{
  I have the ability to \emph{V} that \emph{p}, so \emph{p}

  \mbox{ }

  {\large
    Or:
  }

  \mbox{ }

  {
    \normalsize
    I have the ability to verify the equation, so the equation is true\\
    I have the ability to demonstrate a strategy, so a strategy exists\\
    I have the ability to prove contraposition, so contraposition is a theorem\\
    I have the ability to establish that you are testifying, so you are testifying\\

    \[\vdots\]
  }
  \mbox{ }

  {\large
    But not:
  }

  \mbox{ }

  {
    \normalsize
    I have the ability to guess that it's raining outside, so it's raining outside\\
    I have the ability to be convinced that I'm lost, so I'm lost\\
    I have the ability to deny that the radio is turned on, so the radio is turned on\\

    \[\vdots\]
  }
}
\author{Ben Sparkes}
% \date{ }

\begin{document}

\pagenumbering{roman}

\maketitle

\tableofcontents

\newpage

\pagenumbering{arabic}

\part*{Introduction}
\label{part:introduction}

\chapter{\qWhy{} and \qHow{}}
\label{cha:intro}

\begin{note}
  Our interest is understanding the way an event in which some agent concludes some proposition has some value happened.

  \begin{scenario}[Multiplication]%
    \label{illu:gist:calc}%
    An agent enters `\(23 \cdot 15\)' into a calculator and presses a button marked `\(=\)'.
    The calculator displays `\(345\)'.

    The agent pauses for a moment.
    They have a good understanding of arithmetic.
    And, given the display on the calculator, it follows that if the calculator is trustworthy, then they would not fail to conclude \(23 \cdot 15 = 345\) via their understanding of arithmetic.

    The agent concludes \(23 \cdot 15 = 345\).
  \end{scenario}

  Intuitively, the agent concluded \(23 \cdot 15 = 345\) from the calculator.
  % Personifying a little, we may say the agent concluded \(23 \cdot 15 = 345\) from the testimony of the calculator.
  More could be said about the way the agent concluded \(23 \cdot 15 = 345\) from the calculator, but these details won't be of too much interest.%
  \footnote{
    For example, classify the agent as regarding the calculator as a source of testimony, adding that testimony are factive, and so concluded \(23 \cdot 15 = 345\) from the testimony of the calculator.
  }

  The agent's understanding of arithmetic, it seems, had no significant role.
  So long as the calculator was correct, the agent had the opportunity to conclude \(23 \cdot 15 = 345\).
  Still, the agent did not perform any arithmetic.
\end{note}

\begin{note}
  Abstracting a little, \(23 \cdot 15 = 345\) corresponds to a state of affairs.
  If the truths of mathematics are necessary, then this state of affairs never fails to be, but \(23 \cdot 15 = 345\) captures the way things are in the same way as `kangaroos have tails' captures the way things are.
  For ease we refer to states of affairs as propositions.

  Continuing the abstraction, when the agent concludes \(23 \cdot 15 = 345\), we may say the agent concludes \(23 \cdot 15 = 345\) is true.
  Even if \(23 \cdot 15 = 345\) never fails to be, the calculator may have been faulty and the agent may have concluded \(23 \cdot 15 = 543\).

  So, say the agent concluded the proposition `\(23 \cdot 15 = 543\)' has value `\(\text{True}\)'.

  The agent also (intuitively) concluded `\(23 \cdot 15 = 543\)' has value `\(\text{True}\)' from the calculator.
  Term this a premise.%
  \footnote{
    Strictly, of interest is that the relevant conclusions follow from the agent evaluating the relevant state of affairs as \(\text{True}\).
    That is, with respect to~\autoref{illu:gist:calc}, of interest is that the agent concludes `\(23 \cdot 15 = 345\)' has value `\(\text{True}\)' \emph{from} the agent evaluating `the calculator testified \(23 \cdot 15 = 345\)' as `\(\text{True}\)'.
    And, with respect to~\autoref{illu:gist:calc} it is likewise the birds `perceiving' an explanation from Snowball exists.
    However, we adopt the shorthand for ease of expression.
  }
  Though, perhaps there are additional premises that may be mentioned.
  So, say the agent concluded `\(23 \cdot 15 = 543\)' has value `\(\text{True}\)' from a \pool{} of premises which includes the calculator.
\end{note}

\begin{note}
  Abstracting from almost all the details of \autoref{illu:gist:calc}, then we may say that an agent concluded some proposition has some value from some \pool{} of premises.
  % We assume an event in which an agent concludes always amounts to a conclusion that some proposition has some value from some \pool{} of premises.

  For ease, we abbreviate `\pool{} of premises' to `\pool{}'.
\end{note}

\begin{note}
  Our interest is understanding the way an event in which some agent concludes some proposition \(\phi\) has some value \(v\) happened.

  In this respect, our interest is with the intuition that the agent concluded `\(23 \cdot 15 = 543\)' has value `\(\text{True}\)' \emph{from} the calculator.
  This `\emph{from}' does work.
  Our understanding of \autoref{illu:gist:calc} would be different if the agent concluded `\(23 \cdot 15 = 543\)' has value `\(\text{True}\)' from the their understanding of arithmetic, or if the agent concluded `\(23 \cdot 15 = 543\)' has value `\(\text{True}\)' from their conviction that the calculator guessed correctly.

  What this relation amounts to is something we will not discuss.
  The core intuition is that the relation of interest captures something beyond the observation that there was a sequence of events in which the agent used the calculator and then concluded `\(23 \cdot 15 = 543\)' has value `\(\text{True}\)'.
  The conclusion was `from', `due to', `because of' \dots etc., the relevant \pool{}.

  % It seems clear the same type of relation does not hold between `\(23 \cdot 15 = 543\)', `\(\text{True}\)', and the agent's understanding of arithmetic.
\end{note}

\begin{note}
  \begin{scenario}[Animalism]
    \label{scen:animalism}
    `Four legs good, two legs bad.'
    This, he said, contained the essential principle of Animalism.
    Whoever had thoroughly grasped it would be safe from human influences.
    The birds at first objected, since it seemed to them that they also had two legs, but Snowball proved to them that this was not so.

    `A bird's wing, comrades,' he said, `\textsc{is} an organ of propulsion and not of manipulation.
    It should therefore be regarded as a leg.
    The distinguishing mark of Man is the \emph{hand}, the instrument with which he does all his mischief.'

    The birds did not understand Snowball's long words, but they accepted his explanation, and all the humbler animals set to work to learn the new maxim by heart.
    \textsc{four legs good, two legs bad}, was inscribed on the end wall of the barn\dots%
    \mbox{ }\hfill\mbox{(\cite[25]{Orwell:1976aa})}%
    \newline
  \end{scenario}

  The agents of interest are the birds, and the conclusion is the essential principle of Animalism:
  `Four legs good, two legs bad'.

  Snowball provided an argument against an objection from the birds, and the birds concluded `Four legs good, two legs bad' from Snowball's explanation.

  Though, as \citeauthor{Orwell:1976aa} highlights, the birds did not conclude `Four legs good, two legs bad' from \emph{the content of} Snowball's explanation.
  Some words were too long.
  Instead, the birds concluded `Four legs good, two legs bad', at least in part, from \emph{Snowball's explanation}.

  In parallel to the agent's understanding of arithmetic from~\autoref{illu:gist:calc} the birds of~\autoref{scen:animalism} did not, in part, conclude `Four legs good, two legs bad' from the content of Snowball's explanation.
\end{note}

\begin{note}
  In line with the abstraction given, we may say:
  The birds conclude the proposition `Four legs good, two legs bad' has value `\(\text{True}\)' from the existence of Snowball's explanation.
  And, a relation holds between `Four legs good, two legs bad', `\(\text{True}\)' and the existence of Snowball's explanation, for each bird.
\end{note}

\section*{\qWhy{}, \qHow{} and \issueInclusion{}}
\label{cha:intro:why-how}

\begin{note}
  Our interest is understanding the way an event in which an agent concludes some proposition has some value happened.

  \phantlabel{how-and-why-first-mention}
  We distinguish two questions; `\qWhy{}' and `\qHow{}':

  \begin{question}{questionWhy}{\qWhy{}}
    \cenLine{
      \begin{VAREnum}
      \item
        Agent: \vAgent{}
      \item
        Proposition: \(\phi\)
      \item
        Value: \(v\)
      \item
        Event: \(e\)
      \item
        \mbox{ }
      \end{VAREnum}
    }
    \medskip

    Given \(e\) is an event in which \vAgent{} concludes \(\phi\) has value \(v\):
    \begin{itemize}
    \item
      Which relations between some proposition, value, and \pool{} explain \emph{why} \(e\) is such that \vAgent{} concluded \(\phi\) has value \(v\)?
    \end{itemize}
    \vspace{-\baselineskip}
  \end{question}

  \begin{question}{questionHow}{\qHow{}}
    \label{q:how}
    \cenLine{
      \begin{VAREnum}
      \item
        Agent: \vAgent{}
      \item
        Proposition: \(\phi\)
      \item
        Value: \(v\)
      \item
        Event: \(e\)
      \item
        \mbox{ }
      \end{VAREnum}
    }
    \medskip

    Given \(e\) is an event in which \vAgent{} concludes \(\phi\) has value \(v\):
    \begin{itemize}
    \item
      Which events explain \emph{how} \(e\) is such that \vAgent{} concluded \(\phi\) has value \(v\)?
    \end{itemize}
    \vspace{-\baselineskip}
  \end{question}
\end{note}

\begin{note}
  \qWhy{} seeks an understanding of the way an agent concluded \(\phi\) has value \(v\) in terms of relations between propositions, values, and \pool{1}.

  Following the analysis of \scen{3}~\ref{illu:gist:calc}~and~\ref{scen:animalism}, the relevant relations, respectively, seemed to be:

  \begin{itemize}[noitemsep]
  \item
    The relation between `\(23 \cdot 15 = 345\)', `\(\text{True}\)', and the calculator.
  \item
    The relation between `Four legs good, two legs bad', `\(\text{True}\)' and the existence of Snowball's explanation.
  \end{itemize}

  As highlighted with respect to \autoref{illu:gist:calc} this relation seems to do work in guiding our understanding of the an event.
  To observe that a relation held indicates something in addition to the observation of what happened.%
  \footnote{
    The \illu{1} of \qWhy{} by \scen{3}~\ref{illu:gist:calc}~and~\ref{scen:animalism} is somewhat limited, as no \scen{0} involves an account of what motivates the respective agents.
    \qWhy{} is not designed to rule out what motivates an agent.
    For example, one may add to \autoref{illu:gist:calc} that the agent wants to calculate \(23 \cdot 15\).
    Functions as a premise, given the understanding of \pool{1}.
    More detail follows in~\autoref{cha:clar}.
  }

  What happened does not entail a corresponding relation.
  For example, in \autoref{illu:gist:calc} the agent presses a button marked `\(=\)', but no relation seems to hold between `\(23 \cdot 15 = 345\)', `\(\text{True}\)', and the pressing the button marked `\(=\)'.

  Further, understanding not only of what happened, but an understanding of what happened as opposed to anything else happening.
  The relation between `Four legs good, two legs bad', `\(\text{True}\)', and existence of Snowball's explanation grants an understanding of why no parallel relation holds between `Four legs good, two legs bad', `\(\text{False}\)', and existence of Snowball's explanation.
  Nothing, in principle, prevents the conclusion `Four legs good, two legs bad' has value `\(\text{False}\)' happening after Snowball's explanation.
  % It seems, respectively, the calculator and Snowball's explanation are sufficient to understand the way the agent concluded the relevant proposition has the specified value.
  %
  % We do not, it seems, need to consider the agent calculating \(23 \cdot 15\) by their understanding of arithmetic, nor the birds understanding Snowball's explanation.
  %
  % In particular why this conclusion as opposed to any other conclusion.
\end{note}

\begin{note}
  \qHow{} seeks an understanding of the way an agent concluded \(\phi\) has value \(v\) in terms of what happened.
  So, answers to \qHow{} when applied to \scen{3}~\ref{illu:gist:calc}~and~\ref{scen:animalism}, at least, are the events captured by the descriptions given.

  No relation seems to hold between `\(23 \cdot 15 = 345\)', `\(\text{True}\)', and the pressing the button marked `\(=\)', but pressing the button marked `\(=\)' answers, in part \qHow{}.
  If the agent's hadn't pressed the button, the calculator wouldn't have displayed `\(345\)'.

  Whether or not a relation that answers \qWhy{} is something which answers \qHow{} is not something we have an opinion on.
  Still, in the case of \scen{3}~\ref{illu:gist:calc}~and~\ref{scen:animalism} it seems clear the events provide `witnesses' for the respective relations.
  Something about \scen{3}~\ref{illu:gist:calc}~and~\ref{scen:animalism} secures the relevant relations.
  % Reasoned from \pool{} to conclusion.
  % Events of \scen{3}~\ref{illu:gist:calc}~and~\ref{scen:animalism} seem to provide an account of the way conclusion was obtained from \pool{}.
\end{note}

\begin{note}
  It does not follow that answers to \qWhy{} are given in terms of answers to \qHow{}.
  To illustrate this point, consider the following \scen{}:

  \begin{scenario}[England AD 932]
    \label{scen:king}
    \vspace{-\baselineskip}
    \begin{screenplay}
    \item[OLD WOMAN:]
      Well, how did you become King, then?
    \item[ARTHUR:]
      The Lady of the Lake, her arm clad in purest shimmering samite, held Excalibur aloft from the bosom of the waters to signify that by Divine Providence\space\dots\space I, Arthur, was to carry Excalibur\dots\space that is why I am your King.
    \item[DENNIS:]
      Look, strange women lying on their backs in ponds handing over swords\space\dots\space that's no basis for a system of government.
      Supreme executive power derives from a mandate from the masses not from some farcical aquatic ceremony.%
      \mbox{ }\hfill\mbox{(\cite[8--9]{Cleese:1974aa})}
    \end{screenplay}
    \vspace{-\baselineskip}
  \end{scenario}

  The old woman asks Arthur \emph{how} the become king.
  Arthur provides an answer in terms of some events which happened, but emphasises that those events are \emph{why} Arthur is king.
  In turn, Dennis accepts the answer provided by Arthur as an account of how Arthur became king but rejects the answer an account of why Arthur is king.
  Instead, an answer is expected in involve the absence of an appropriate system of governance in England.
  So, the answer provided by Arthur is accepted by Dennis as an answer to how, but not as an answer to why.
\end{note}

\begin{note}
  The distinction between `why?' and `how?' present in \autoref{scen:king} parallels \qWhy{} and \qHow{} with respect to conclusions.
  In \autoref{illu:gist:calc}, the agent considered performing mental arithmetic, but the consideration of calculating \(23 \cdot 15 = 345\) seems no basis for the conclusion.
  Instead, it is the use of the calculator which is a basis.

  And, when we consider the relation between `\(23 \cdot 15 = 345\)', `\(\text{True}\)', and the calculator, we understand why (and not merely how) the agent concluded \(23 \cdot 15 = 345\).
\end{note}

\begin{note}
  Our interest is understanding the way an event in which an agent concludes some proposition has some value happened.
  \qWhy{} and \qHow{} are distinct questions, and the following constraint on answers to \qWhy{} in terms of answers to \qHow{} seems intuitive:

  \begin{constraint}{consInclusion}{\issueInclusion{}}
    \mbox{ }
    \vspace{-\baselineskip}
    \begin{itemize}
    \item
      \begin{itenum}
      \item[\emph{If}:]
        Some relation between proposition, value, and \pool{} is, in part, an answer to \qWhy{}.
      \item[\emph{Then}:]
      An event in which the agent concludes the proposition has the value from the \pool{} is, in part, \qHow{}.
    \end{itenum}
  \end{itemize}
  \vspace{-\baselineskip}
  \end{constraint}

  In short, for any relation between a proposition, value, and \pool{} which grants some understanding of the way an agent concluded \(\phi\) has value \(v\), there is an event such that the agent has concluded the proposition has some value from the \pool{}.
\end{note}

\begin{note}
  I consider \issueInclusion{} intuitive.
  In principle, relation between any proposition, value, and \pool{} may answer \qWhy{}.
  However, no other relation.

  The analysis of \scen{3}~\ref{illu:gist:calc}~and~\ref{scen:animalism} may be taken to motivate \issueInclusion{}, but I expect the analysis given is sensible given the adherence to \issueInclusion{}.

  \begin{itemize}[noitemsep]
  \item
    Consider \autoref{illu:gist:calc}.

    The agent concluded `\(23 \cdot 15 = 345\)' and `\(\text{True}\)' from the calculator, and hence if a relation between these items answers \qWhy{} (which it intuitively does), then there is a corresponding answer to \qHow{}.

    And, while a relation may hold between `\(23 \cdot 15 = 345\)', `\(\text{True}\)' and whatever \pool{} would be associated with the agent calculating \(23 \cdot 15 = 345\) from their understanding of Arithmetic.
    There is no event in which the agent concluded `\(23 \cdot 15 = 345\)', has value `\(\text{True}\)' by performing arithmetic.
    So, there is no answer to \qHow{} with respect to this \pool{}, and given \issueInclusion{}, it is not possible for the relation to answer \qWhy{}.
  \item
    Parallel reasoning applies to~\autoref{scen:animalism}.

    The bird's conclusion of `Four legs good, two legs bad' has value `\(\text{True}\)' from the existence of Snowball's explanation answers \qHow{}, and the relation between the proposition, value, and \pool{} answers \qWhy{}, in line with \issueInclusion{}.

    And, while a relation may hold between `Four legs good, two legs bad' `\(\text{True}\)' and the \emph{content} of Snowball's explanation, the birds failed to understand the content.
    Hence, with no answer to \qHow{}, it is not possible for this relation to answer \qWhy{}.
  \end{itemize}
  Hence, \issueInclusion{} seems plausible, and seems to do work.%
  \footnote{
    Note, \qHow{} does not explicitly require the relevant event to be the event in which the agent concludes \(\phi\) has value \(v\) from the \pool{}.
    Hence, a previous event in which the agent concluded `\(23 \cdot 15 = 345\)' has value `\(\text{True}\)' may answer \qHow{}.
    Still, given \(23 \cdot 15 = 345\) in~\autoref{illu:gist:calc} functions as some arbitrary multiplication that the agent may calculation, we may assume the agent has never calculated \(23 \cdot 15 = 345\).
  }\(^{,}\)%
  \footnote{
    Alternatively, \issueInclusion{} may be thought to narrow the range on answers to \qWhy{}.
    That is to say, \issueInclusion{} functions to disambiguate the sense of `why' used in the statement of \qWhy{}.
    If so, then \issueInclusion{} is not a constraint, it is simply part of the way \qWhy{} is understood.
    However, I do not think this is the case.
    I think it is plausible that the sense of `why' present in \qWhy{} may be understood without reference to \issueInclusion{}, and hence that \issueInclusion{} amounts to a substantive constraint.
    The document will assume this is the case, but only the framing depends on this.
  }
\end{note}

\begin{note}
  Still, we have only considered two \scen{1} and \issueInclusion{} is a general constraint.
  Hence, if there is some doubt regarding \issueInclusion{} then further argument is required.
  It is not clear that there are \scen{1} in which \issueInclusion{} fails to hold.
  However, lack of apparent counterexamples is not argument.

  Though I would very much like to provide an argument for \issueInclusion{}, I am not aware of any.%
  \footnote{
    The particular construct of \qWhy{}, \qHow{}, and \issueInclusion{} is somewhat idiosyncratic, and so it is no surprise there are no direct argument for \issueInclusion{}.
    However, I take the idea captured by \issueInclusion{} to be intuitive, and I am not aware of any argument for this idea.
  }
  Perhaps \issueInclusion{} is sufficiently intuitive that it is taken for granted.
  Or, perhaps the failure of \issueInclusion{} is recognised and I have yet to stumble upon the acknowledgement of its failure.

  In any case, no argument will be provided.
  The goal of this document is to provide a recipe for generating counterexamples to \issueInclusion{}.
  By recipe, features of \scen{1} which lead to violations.

  Before closing this introduction, say a little more regarding the recipe for counterexamples.
  First, though no clear arguments for \issueInclusion{}, motivation by relation to reasons.
\end{note}

\section*{\issueInclusion{} and reasons}
\label{sec:reasons}

\begin{note}
  \qWhy{} and \qHow{} are questions about the way an event in which an agent concludes some proposition has some value happened.

  \qWhy{} seeks `explanatory reasons', summarised by \citeauthor{Hieronymi:2011aa} (\citeyear{Hieronymi:2011aa}) as:

  \begin{quote}
    [T]he reasons why things happen, or why things are the way they are.\newline
    \mbox{ }\hfill\mbox{(\citeyear[410]{Hieronymi:2011aa})}
  \end{quote}

  To borrow an an example given by \citeauthor{Hieronymi:2011aa}, the extreme heat or the faulty construction is a reason why the engine failed (\citeyear[409]{Hieronymi:2011aa}).
  Likewise, the engine failing may be a reason why the steamboat is moored, and the steamboat being moored is a reason why the hotel is fully booked, etc.

  With \qWhy{}, the relevant explanatory reason is a relation between a proposition, value, and a \pool{}.
  The relation between `\(23 \cdot 15 = 345\)', `\(\text{True}\)' and the calculator captures a reason why the agent has concluded `\(23 \cdot 15 = 345\)' has value `\(\text{True}\)'.
\end{note}

\begin{note}
  In general, explanatory reasons are distinct from an \emph{agent's} reasons, whatever these may be.
  Though, in some cases, explanatory reasons may be involve an agent's reasons.
  This contrast and involvement forms the opening question of \citeauthor{Davidson:1963aa}'s \citetitle{Davidson:1963aa}:

  \begin{quote}
    What is the relation between a reason and an action when the reason explains the action by giving the agent's reason for doing what he did?
    We may call such explanations \emph{rationalizations}, and say that the reason \emph{rationalizes} the action.%
    \mbox{}\hfill\mbox{(\citeyear[685]{Davidson:1963aa})}
  \end{quote}

  \citeauthor{Davidson:1963aa}'s primary interest is with an explanatory reason which explains an action by giving the agent's reason for performing the action.
  At issue is why the agent did what they did, and the explanatory reason for this involves the agent's reason.

  % A conclusion is also the result of an act and isolate the relevant belief component as the premises.
\end{note}

\begin{note}
  Rationalisation for \citeauthor{Davidson:1963aa} cover all actions.
  A conclusion is a particular specific type of act.%
  % \footnote{
  %   `The agent concluded \(\phi\) has value \(v\)' may be re-expressed as `the agent performed an act in which they concluded \(\phi\) has value \(v\)', in the same way as `the agent buttered the toast' may be re-expressed as `the agent performed an act in which the toast was buttered'.
  % }
  And, with respect to conclusions, a sensible candidate for an agent's reasons are \pool{0}.
  For example, we may say with relative ease that:%
  \footnote{
    These may also be understood, in line with \citeauthor{Smith:1994wo} (\citeyear{Smith:1994wo}), as `motivating reasons':
    \begin{quote}
      The distinctive feature of a motivating reason to \(\phi\) is that, in virtue of having such a reason, an agent is in a state that is \emph{explanatory} of her \(\phi\)-ing, at least other things being equal --- other things must be equal because an agent may have a motivating reason to \(\phi\) without that reason's being overriding.%
      \mbox{}\hfill\mbox{(\citeyear{Smith:1994wo})}
    \end{quote}
    However, our interest is with the \emph{relation} between a proposition, value, and \pool{}.
  }

  \begin{itemize}[noitemsep]
  \item
    The reason for which the agent concluded \(23 \cdot 15 = 345\) was the calculator.
  \item
    The reason for which the birds concluded `Four legs good, two legs bad' was (the existence of) Snowball's explanation.
  \end{itemize}

  So, it seems relations between propositions, values, and \pool{1} may be understood as \citeauthor{Davidson:1963aa}ian rationalisations.
  Or, conversely, one may grant a relation between a proposition, value, and a \pool{1} provides an explanatory reason, and understand the explanatory reason in terms of an agent's reasons by identifying the \pool{1} as the agent's reasons.%
  \footnote{
    For \citeauthor{Davidson:1963aa} an agent's reason is, roughly; \textquote{some feature, consequence, or aspect of the action the agent wanted, desired, prized, held dear, thought dutiful, beneficial, obligatory, or agreeable} (\citeyear[685]{Davidson:1963aa}).
  In this respect, belief-desire pair, and action is something like buttering the toast.
  However, interest with conclusions, and hence ignore the pro-attitude component.
}
\end{note}

\begin{note}
  Now, the key argument of made in \citetitle{Davidson:1963aa} is:
  \begin{quote}
    [R]ationalization is a species of ordinary causal explanation.%
    \mbox{ }\hfill\mbox{(\citeyear[685]{Davidson:1963aa})}
  \end{quote}

  If we grant relations between a proposition, value, and \pool{1} are rationalisations, and rationalisations are causal explanations, and \pool{1} capture an agent's reasons then \issueInclusion{} follows.

  For, suppose some relation between a proposition, value, and \pool{} which answers \qWhy{}.
  By assumption, this relation amounts to a rationalisation.
  And, by assumption rationalisations as causal explanations.
  Hence, there must be some causal relation holding between a \pool{1} and the agent's conclusion that the proposition has the value.
  And, grating there is a causal relation between the agent's conclusion that the proposition has the value and a \pool{1}, then there is some event in which the agent concludes the proposition has the value from the \pool{1}.
  This event is an answer to \qHow{}.
\end{note}

\begin{note}
  \citeauthor{Davidson:1963aa}'s account considers actions in general.
  Still, granting that a conclusion is always the result of reasoning, \citeauthor{Broome:2013aa}'s account of active reasoning explicitly links the same idea to conclusions:
  \begin{quote}
    I have arrived at necessary and sufficient conditions for a process to be active reasoning.

    Active reasoning is a particular sort of process by which conscious premise-attitudes cause you to acquire a conclusion-attitude.
    The process is that you operate on the contents of your premise-attitudes following a rule, to construct the conclusion, which is the content of a new attitude of yours that you acquire in the process.%
    \mbox{ }\hfill\mbox{(\citeyear[234]{Broome:2013aa})}
  \end{quote}

  \citeauthor{Broome:2019aa}'s account of active reasoning parallels \citeauthor{Davidson:1963aa}'d understanding of rationalisations.
  However, \citeauthor{Broome:2019aa}'s account is strictly stronger.
  For, \citeauthor{Davidson:1963aa} assumes rationalisations are such that an action is explained by giving an agent's reason for doing what they did.
  Hence, in certain cases there may be cases in which rationalisations do not exhaust explanatory reasons.
  \citeauthor{Broome:2019aa}, however, holds that causation of a conclusion-attitude by premise-attitudes is sufficient for a process to be active reasoning, and so we do not need to consider anything other than the causal relation to understand the way the event happened.%
  \footnote{
    I claimed above that I am not aware of any argument for \issueInclusion{}.
    However, if the link between \citeauthor{Broome:2013aa}'s account of active reasoning and \issueInclusion{} holds, it seems whatever argument \citeauthor{Broome:2013aa} has given from the account of active reasoning should extend to \issueInclusion{}.

    Still, while \citeauthor{Broome:2013aa} motivate the account of active reasoning, at issue is whether the condition \citeauthor{Broome:2013aa} gives is sufficient.
    \citeauthor{Broome:2013aa} considers and dismisses a number of other conditions to secure sufficiency
    (\citeyear[cf.][\S13.2]{Broome:2013aa}), and while I think this is good motivation, it falls short of an argument.
    For, I see no clear way to extend the specific dismissals to ensure no other condition is required for sufficiency in general.

    Likewise, we noted how \citeauthor{Davidson:1963aa}'s account is compatible with other explanatory reasons.
    %focuses on causal explanation, and does not rule out that there may be cases in a reason explains an action by given the agent's reason for doing what they did in addition to other considerations (which may be causally involved).
  }
\end{note}

\begin{note}
  So, capturing explanatory reasons in terms of causal explanation motivations \issueInclusion{}.
  However, abstracting a little, the role of causation in \citeauthor{Davidson:1963aa} and \citeauthor{Broome:2013aa}'s account is of interest only to the extent that the relevant causal relations ensure the relation between a proposition, value, and \pool{} which holds between the conclusion and the \pool{} is sufficient.
  And, such accounts do not need to appeal to causation to ensure this sufficiency.

  For a final illustration, consider \citeauthor{Hieronymi:2011aa}'s (\citeyear{Hieronymi:2011aa}) account of action explanation:

  \begin{quote}
    [W]henever an agent acts for reasons, the agent, in some sense, takes certain considerations to settle the question of whether so to act, therein intends so to act, and executes that intention in action.

    If this much is uncontroversial (and, under some interpretation, I believe it must be), we can use it as a form for filling out.
    I propose, then, that we explain an event that is an action done for reasons by appealing to the fact that the agent took certain considerations to settle the question of whether to act in some way, therein intended so to act, and successfully executed that intention in action.
    I suggest that \emph{this} complex fact [\dots] explains the action by giving the agent's reason for acting.%
    \mbox{ }\hfill\mbox{(\citeyear[421]{Hieronymi:2011aa})}
  \end{quote}

  The key feature of \citeauthor{Hieronymi:2011aa}'s account is the agent taking certain considerations to settle the question of whether to act in some way.
  \citeauthor{Hieronymi:2011aa} only requires the agent took certain considerations to settle a question, whether this taking amounts to causation or otherwise.
  \citeauthor{Hieronymi:2011aa} understands this taking to be an activity, and hence there is a corresponding event.
  Therefore, if a relation between a proposition, value, and \pool{} answers \qWhy{} and this relation is understood in terms of the agent taking the \pool{} to settle the question of whether to conclude the proposition has the value, we must have, by \citeauthor{Hieronymi:2011aa}'s account, an answer to \qHow{} in line with \issueInclusion{}.
\end{note}

\begin{note}
  Key idea.
  \issueInclusion{} is motivated by how an agent concludes being sufficient to understand why.
  Specifying the \pool{} gets us a reason, and this reason seems to be sufficient.
  Only need to observe cause and effect, or something like this.
\end{note}


\section*{Questioning \issueInclusion{}}

\begin{note}
  Our interest is understanding the way an event in which some agent \vAgent{} concludes some proposition \(\phi\) has some value \(v\) happened.

  We introduced two questions (\qWhy{}, \qHow{}) and a constraint (\issueInclusion{}) on answers to \qWhy{} by answers to \qHow{}.
  And, we motivated the constraint in part by a pair of \scen{1} (\scen{3}~\ref{illu:gist:calc}~and~\ref{scen:animalism}) and then more generally in terms of reasons.
  %(\cite{Davidson:1963aa},\cite{Broome:2019aa},\cite{Hieronymi:2011aa})

  I think \issueInclusion{}, while intuitive, fails to hold in general.
\end{note}

\begin{note}
  I think that answers to \qWhy{} reduce to psychological facts of an agent, specifically psychological facts which hold of the agent when they conclude.%
  \footnote{
    This is incompatible with views of reasons explanation advanced by~(\cite{Dancy:2000aa}) and~(\cite{Alvarez:2013aa}), in which the state of affairs may be an agent's reason (and not just the evaluation of some state of affairs).
    Though, the argument will not depend on assuming that the relations reduce to psychological facts.

    See~(\cite[413--418]{Hieronymi:2011aa}),~(\cite[3--5]{DOro:2013vh}), and~(\cite[\S2]{Alvarez:2017aa}) for more.
  }
  However, I do not think that a relation between the proposition, value, and \pool{} exhaust the relevant psychological facts.
  In particular, I think that in various cases additional relations between distinct propositions, values and \pool{1} answer \qWhy{} without there being an event in which the agent concluded the propositions have values from \pool{1}.
\end{note}

\begin{note}
  As seen above with respect to \autoref{scen:animalism}, it follows from \issueInclusion{} that the content of Snowball's explanation is irrelevant.
  For, the birds do not understand Snowball's explanation, hence there is no event in which the birds reason from the content to `Four legs good, two legs bad' has value `\(\text{True}\)'.
  And, I think this is correct.

  However, parallel reasoning entails the agent's understanding of arithmetic is irrelevant with respect to \autoref{illu:gist:calc}.
  And, I do not think this entailment holds.
  I think it may be the case that a relation between `\(23 \cdot 15 = 543\)', `\(\text{True}\)' and the agent's understanding of arithmetic \emph{may} answer \qWhy{}.
  Whether this is the case will depend on whether some additional details hold of~\autoref{illu:gist:calc}.
  Still, as the details matter, the entailment, and hence \issueInclusion{}, is not right.
\end{note}

\begin{note}
  In other words, I think there are counterexamples to \issueInclusion{}.
  The primary goal of this document is to provide a recipe for generating counterexample to \issueInclusion{}.

  The document is split into four parts:

  \begin{TOCEnum}
  \item
    \autoref{part:prep}: \nameref{part:prep}.

    We begin by clarifying our understanding of conclusions, \qWhy{}, \qHow{}, and \issueInclusion{}.
    In particular, we provide variations to \qWhy{}, \qHow{}, and \issueInclusion{} in order to precisely capture what the counterexamples we generate our counterexamples to.
  \item
    \autoref{part:ing}: \nameref{part:ing}

    We detail three ideas which will be used to generate counterexamples.
    Each idea captures some phenomenon.
    The counterexamples occur when each idea applies to a \scen{0} in which an agent concludes some proposition has some value from some \pool{}.

    The key idea is that of a \fc{}.
    However, the ideas of \tC{}, and \requ{} will tie \fc{1} to instances in which an agent concludes.
  \item
    \autoref{part:dir}: \nameref{part:dir}

    With the preparation and ingredients in hand, we show how to combine the ingredients to generate counterexamples to \issueInclusion{}.

    We also provide a few sample \scen{1} where \issueInclusion{} fails, and consider any leftover issues from the recipe.
    Here we return to \autoref{illu:gist:calc}.
  \end{TOCEnum}

  This covers how things will happen.
  I do not have a brief account of what will happen.
  The details are too important.
  The recipe is not based on an intuitive understanding of any particular \scen{}.
  Instead, the recipe is based on the way a number of ideas come together when we understand the way some agent concludes some proposition \(\phi\) has some value \(v\) happened.
\end{note}



% \begin{note}
%   Now, all this has been said without giving attention to the conditional observed by the agent in \autoref{illu:gist:calc}:

%   \begin{itemize}
%   \item
%     If the calculator is trustworthy, then the agent would not fail to conclude \(23 \cdot 15 = 345\) via their understanding of arithmetic.
%   \end{itemize}

%   Both natural, and somewhat surprising.

%   Consider the contraposition.

%   \begin{itemize}
%   \item
%     If the agent were to fail to conclude \(23 \cdot 15 = 345\) via their understanding of arithmetic, then the calculator is not trustworthy.
%   \end{itemize}

%   \begin{itemize}
%   \item
%     Is it possible for the agent to fail to conclude \(23 \cdot 15 = 345\) via their understanding of arithmetic?
%   \end{itemize}

%   If possible, then difficulty.
%   For, testimony, so must be, but at the same time, possible that it is not.
%   If not possible, then it doesn't seem that observing \(23 \cdot 15 = 345\) via the testimony of the calculator is sufficient.
%   For, by the previous observation, difficulty with the testimony of the calculator.

%   \begin{itemize}
%   \item
%     Testimony of the calculator \emph{only if} \(23 \cdot 15 = 345\) is a \fc{0} given the agent's understanding of arithmetic.
%   \end{itemize}
% \end{note}

% \begin{note}
%   Important:

%   Multiple ways to conclude.
%   So, have a check.

%   Differs to, for example, concluding {\color{red} ???} from a scientific calculator.
%   {\color{red} ???} goes beyond typical understanding of arithmetic.
%   Parallel pair of conditionals does not hold.

%   Or, alternatively, testimony that {\color{red} ???}.
%   Beyond understanding.
% \end{note}

% \begin{note}
%   I am not sure what to make of \ref{illu:gist:calc}.
%   Understanding of arithmetic is a partial check.
%   However, testimony.

%   Unsure because status of a premise.

%   Basic contrary idea only requires some instances.

%   Argument against this intuition.
%   Type of cases, premises are fixed.
%   Check on own reasoning.

%   First, expand on intuition.
%   Then, introduce type of \scen{0} of interest.
% \end{note}

% \begin{note}
%   Follow part, foundations.
%   Following, turn to details.

% {\color{red} Place somewhere?}.%
%   \footnote{
%     Roughly, if it were the agent failed to conclude \(23 \cdot 15 = 345\) in \autoref{illu:gist:calc}, then there would be conflict between the agent's understanding of arithmetic and the testimony of the calculator.
%     Expressed differently, there would be conflict between the agent's failure to conclude \(23 \cdot 15 = 345\) by their understanding of arithmetic, and a premises involved in concluding \(23 \cdot 15 = 345\) via the calculator.
%     I.e. supposing the agent concludes \(23 \cdot 15 \ne 345\), then for the agent the calculator is not a source of testimony.
%     In the \scen{1} of interest, this hypothetical --- or in some cases possible --- conflict will strictly be between the agent's reasoning from \pool{1} to conclusions.
%   }
% \end{note}

% \begin{note}
%   To illustrate, consider \citeauthor{Broome:2013aa}'s (\citeyear{Broome:2013aa}) account of a `motivating reason'%
%   \footnote{
%     \citeauthor{Broome:2013aa} contrasts `motivating reasons' to `normative reasons'.
%     \begin{quote}
%       Whereas motivating reasons explain or help to explain why a person does something, normative reasons explain or help to explain why a person ought to do something, or to believe something, or to hope for something, or to like something, or in general to F, where ‘F' stands for a verb phrase.%
%     \mbox{}\hfill\mbox{(\citeyear[47]{Broome:2013aa})}
%     \end{quote}
%   }
%   \begin{quote}
%     Sometimes the explanation of why a person does something has a particular character:
%     roughly, it involves the person's rationality in a distinctive way that I shall not try to describe.
%     Then we say the person does what she does for a reason.
%     We might say ‘The reason for which Hannibal used elephants was to terrorize the Romans'.
%     The reason for which a person does something is called a ‘motivating reason'.
%     In general, a motivating reason is whatever explains or helps to explain what a person does in the distinctive way that involves her rationality.
%     \mbox{}\hfill\mbox{(\citeyear[46--47]{Broome:2013aa})}
%   \end{quote}
% \end{note}

% \begin{note}
%   \color{red}

%   Further,

%   For \citeauthor{Davidson:1963aa}, primary reason.

%   \begin{quote}
%     \emph{R} is a primary reason why an agent performed the action \emph{A} under the description \emph{d} only if \emph{R} consists of a pro attitude of the agent toward actions with a certain property, and a belief of the agent that \emph{A}, under the description \emph{d}, has that property.\newline
%     \mbox{ }\hfill\mbox{(\citeyear[687]{Davidson:1963aa})}
%   \end{quote}

%   We have distinguished \qWhy{} from pro-attitudes.
%   However, fill in whatever motivation.
%   What matters is the belief.
%   This is the relevant proposition-value pair.

%   If \citeauthor{Davidson:1963aa}, then granting restriction, seems we don't need to look beyond the proposition-value pair.
% \end{note}

% \begin{note}
%   So, answer to \qWhy{} is constrained by answer to \qHow{} by getting to a reason.

%   For, causal relation.
%   Indeed, from `explanatory', these things are identical.
%   However, from `motivating' still distinction.
%   Agent's reason is causal, but the content is not necessarily causal.%
%   \footnote{
%     On my understanding of \citeauthor{Davidson:1963aa}, there's a tight link between then content of some state and the causal relations that arise from the state.

%     So, go from content to state, and then proceed from here.

%     This, I think, is correct.
%     And, the problem of deviant causal chains highlights this.
%     For, \citeauthor{Davidson:1963aa} recognises there's a problem with the link between content and the causal relations which hold between the states.

%     \begin{quote}
%       Beliefs and desires that would rationalize an action if they caused it in the right way—through a course of practical reasoning, as we might try saying---may cause it in other ways.%
%       \mbox{ }\hfill\mbox{(\citeyear[79]{Davidson:1973vd})}
%     \end{quote}
%   }

%   Here, we get a causal trace.
%   No need to look for any relation of support other than premises of reasoning.
% \end{note}

% \begin{note}
%   \begin{scenario}[Sudoku]
%     \label{scen:sudoku:intro}
%     An agent has some free time.
%     They take out copy of \citetitle{Coussement:2007up} and open the book to some Sudoku puzzle.
%     The agent methodically fills in each cell of the puzzle.

%     The agent pauses for a moment.
%     They have a good understanding of the rules of Sudoku and some free time.
%     Hence, if solution is correct, then would not fail to complete any other Sudoku puzzle in the book.

%     If encounter difficulty, then re-examine proposed solution.

%     The agent concludes the proposed solution is the correct solution to the Sudoku puzzle.
%   \end{scenario}

%   Proposition, `The solution is correct', value `\(\text{True}\)'.
%   The \pool{} includes the initial Sudoku grid and the rules of Sudoku.

%   Relation.

%   Possible relation between some other puzzle and the initial Sudoku grid for that other puzzle and the rules of Sudoku.

%   However, as with \scen{3}~\ref{illu:gist:calc}~and~\ref{scen:animalism}, no role.
% \end{note}

% \begin{note}
%   More broadly, I take the basic idea to capture a pre-theoretical constraint on classes of theories.
%   There are theories that agree with the basic idea, such as \citeauthor{Davidson:1963aa}' causal theory of action (when the action is concluding) and there \emph{may be} theories which do not agree with the basic idea --- though I do not know of any specific theories that are explicitly of this kind.
% \end{note}



%%% Local Variables:
%%% mode: latex
%%% TeX-master: "master"
%%% TeX-engine: luatex
%%% End:


\part{Core}

\chapter{Introduction}
\label{cha:introduction}

\begin{note}
  \begin{itemize}
  \item
    Initial \scen{0}, intuitions, questions, relation between answers, motivation for positive answer.
  \end{itemize}
\end{note}

\section{Concluding: Why? and How?}
\label{sec:overview:issue}

\subsection{Initial \scen{0}}

\begin{note}
  \begin{scenario}[Multiplication]
    \label{illu:gist:calc}
    An agent enters `\(23 \times 15\)' into a calculator and presses the button marked `\(=\)'.
    The calculator displays `\(345\)'.

    \mbox{ }

    The agent observes they have the option to calculate \(23 \times 15\) via their understanding of arithmetic.
    And, \emph{if} the calculator is trustworthy, then they would not fail to conclude \(23 \times 15 = 345\) via their understanding of arithmetic.

    \mbox{ }

    The agent concludes \(23 \times 15 = 345\).
  \end{scenario}

  Intuitively, the agent concluded \(23 \times 15 = 345\) from the calculator.
  Personifying a little, we may say the agent has concluded \(23 \times 15 = 345\) from the testimony of the calculator.%
  \footnote{
    Indeed, for our purposes \autoref{illu:gist:calc} may be recast in terms of an agent asking another agent to solve `\(23 \times 15\)', but using a calculator is more natural.
  }

  Still, as noted by the agent, they had the option of concluding \(23 \times 15 = 345\) through their own understanding of arithmetic, and hence without the use of the calculator.

  Still, it is intuitive that an agent concludes some proposition has some value%
  \footnote{In \scen{0} the proposition is `\(23 \times 15 = 345\)' and the value is `true'.
    In isolation, `\(23 \times 15 = 345\)' describes some possible state of affairs, and assigning the value `true' indicates the possible state of affairs is the actual state of affairs.
    Still, the agent may have concluded `\(23 \times 15 = 345\)' is `desired', `impossible', `probable', and so on.
    When speaking generally, we will keep explicit which value a proposition is paired with, though when describing specific \scen{0} or examples, we will leave the associated value implicit.
  }
  from some pool of premises only if the agent reasoned from those premises to the proposition-value pair.

  In other words, the agent may have concluded \(23 \times 15 = 345\) via their understanding of arithmetic, but as the agent did not calculate \(23 \times 15\) themselves, they did not conclude \(23 \times 15 = 345\) from whatever premises would be involved when reasoning via their understanding of arithmetic.

  Indeed, given the agent's understanding of arithmetic, it seems clear that prior to using the calculator the agent knew \emph{whether} \(23 \times 15 = 345\).
  Though, knowing whether \(23 \times 15 = 345\) is not knowing \(23 \times 15 = 345\).
  For example, I expect --- though one of the following equalities does not hold --- you know whether \(345 \times 11 = 3,795\), whether \(3,795 \div 5 = 760\), and whether \(760 \div 8 = 95\).

  Rephrasing things a little, and keeping track of truth, we may say \(23 \times 15 = 345\) was a \fc{} for the agent in \autoref{illu:gist:calc}.
  And, for you, \(345 \times 11 = 3,795\), \(3,795 \div 5 \ne 760\), and \(760 \div 8 = 95\) are \fc{1}.

  Instead, it seems the agent concluded \(23 \times 15 = 345\) by use of the calculator, regardless of whether \(23 \times 15 = 345\) was a \fc{1}.
  And, likewise, if you have concluded \(3,795 \div 5 \ne 760\) from the paragraph above, it was due to my testimony, and not your understanding of arithmetic.
\end{note}

\begin{note}[Summary of basic intuitions]
  So, intuitively, in \autoref{illu:gist:calc} the agent concluded \(23 \times 15 = 345\) from the testimony of the calculator.
  And, intuitively, \(23 \times 15 = 345\) being a \fc{0} had no significant role in the agent concluding \(23 \times 15 = 345\) from the testimony of the calculator.
\end{note}

\subsection{Two questions: Why? and How?}

\begin{note}[Not just concluding]
  So far we have spoken about intuitions of the basic form:

  \begin{itemize}
  \item
    Agent \(A\) concluded some proposition \(\phi\) has some value \(v\) from some pool of premises \(\Phi\).
  \end{itemize}

  I take intuitions of this basic form to be readily available in a variety of \scen{1}, and I also take those intuitions expressed in regards to \autoref{illu:gist:calc} to be fairly clear.

  \phantlabel{how-and-why-first-mention}
  Still, concluding is something an agent does, and in this respect there are (at least) two distinct questions intuitions of the above form may answer:%

  \begin{restatable}[\qWhy{}]{question}{questionWhyBasic}
    \label{q:why}
    Which proposition-value-premises pairings an involved in explaining \emph{why} the agent concluded \(\phi\) has value \(v\)?
  \end{restatable}

  \begin{restatable}[\qHow{}]{question}{questionHowBasic}
    \label{q:how}
    Which proposition-value-premises pairings an involved in explaining \emph{how} the agent concluded \(\phi\) has value \(v\)?
  \end{restatable}

  In basic form, focus is on what the agent did, or alternatively whether some action way performed.
  Did the agent conclude \(\phi\) has value \(v\)?
  Or, did the agent conclude \(\phi\) has value \(v\) from the pool of premises \(\Phi\)?
  Or, perhaps, did the agent fail to conclude \(\phi\) has value \(v\)?

  `How?' and `why?' by contrast, take for granted the agent concluded \(\phi\) has value \(v\) and consider, respectively, how and why this action was performed.

  How do intuitions regarding whether or not an agent performed an action relation to how and why the agent performed the action?
\end{note}

\begin{note}
  \color{red}
  Designed to allow intermediate conclusions.
\end{note}

\begin{note}
  \qWhy{} and \qHow{} are distinct questions.
  Particular instances of broader `why?' and `how?'.
  And, generally speaking, `why?' and `how?' may have distinct answers.

  For example, consider asking why and how some agent arrived at some location.
  `By bicycle' answers how the agent arrived at the post office, but does not answer why the agent arrived at the post office.
  Likewise, `to post a letter' intuitively answers why the agent arrived at the post office, but does not answer how the agent arrived at the post office.%
  \footnote{
    Of course, things get complex.
    Action, motivating reason, belief-desire pair.
    Hence, desire to post a letter is part of how.

    Preface with `intuitively'.
    The point is that may refine intuition regarding the agent concluding \(23 \times 15 = 345\).
  }

  So, the intuitions expressed with respect to \autoref{illu:gist:calc} may answer `how?' but not `why?', or `why?' but not `how?'.

  Still, I suspect this is not the case.
  In the case of \autoref{illu:gist:calc} both have the same rough answer:

  \begin{itemize}
  \item
    The pairing of testimony of the calculator with \(23 \times 15 = 345\), is, in part, \emph{both} how \emph{and} why the agent concluded \(23 \times 15 = 345\).
  \end{itemize}
  That premises associated with the agent's understanding of arithmetic do not answer either `how?' or `why?' is implicit by omission.

  Again, the agent used the testimony of the calculator to conclude \(23 \times 15 = 345\), and the agent appealed to the testimony of the calculator to conclude \(23 \times 15 = 345\).
  The agent did not use their understanding of arithmetic to conclude \(23 \times 15 = 345\), and the agent did not appeal to their understanding of arithmetic to conclude \(23 \times 15 = 345\).
\end{note}

\subsection{Relationship between \qWhy{} and \qHow{}}

\begin{note}
  Our observation that the testimony of the calculator seems to answer both `why?' and `how?' the agent concluded \(23 \times 15 = 345\) suggests, even if --- as a single \scen{0} --- only slightly, the following basic idea:%
  \footnote{
    The observation also suggests the converse holds --- an answer, in part, to `how?' is also, in part, an answer to `why?' --- though I think the converse faces some immediate difficulties.
    For, it seems answers to `how?' may include details that are irrelevant to `why?'.
    For example, typing digits and operators into the calculator answers, in part, how the agent concluded \(23 \times 15 = 345\) but these actions seems irrelevant to why the agent concluded \(23 \times 15 = 345\).
    Rather, an answer to `why?' seems to be limited to the calculator providing testimony that \(23 \times 15 = 345\), regardless of whether it was the agent who used the calculator, or whether the agent observed someone else using the calculator.
  }

  \phantlabel{how-and-why-relation-first-mention}
  \begin{restatable}[\issueInclusion{}]{issue}{issueInclusionFirst}
    \label{issue:why-inc-in-how}
    Some proposition-value-premises pairing is, in part, an answer to \qWhy{} only if that proposition-value-premises pairing is (also), in part, an answer to \qHow{}.
  \end{restatable}

  In other words, in order for premise to, in part, answer \emph{why} an agent concluded \(\phi\) has value \(v\), that premise must also, in part, answer \emph{how} the agent concluded \(\phi\) has value \(v\).

  With respect to \autoref{illu:gist:calc}, the testimony of the calculator satisfies the constraint imposed by the idea, while the agent's understanding of arithmetic does not.
  Specifically, the testimony of the calculator was part of how the agent concluded \(23 \times 15 = 345\), and so the testimony of the calculator may be, in part, an answer to why the agent concluded \(23 \times 15 = 345\).
  However, the agent's understanding of arithmetic was \emph{not} part of how the agent concluded \(23 \times 15 = 345\), and so the agent's understanding of arithmetic \emph{may not}, in part, an answer to why the agent concluded \(23 \times 15 = 345\).

  In addition, the basic idea may be taken to capture some explanatory significance and we may even say:
  The agent's understanding of arithmetic is not, in part, an answer to why the agent concluded \(23 \times 15 = 345\) \emph{because} the agent's understanding of arithmetic was \emph{not} part of how the agent concluded \(23 \times 15 = 345\).
\end{note}

\subsection{Motivation from \citeauthor{Davidson:1963aa}}

\begin{note}
  The basic idea, rather than intuitions regarding specific \scen{1} is our interest.
  Roughly, at least.%
  \footnote{
    We will shortly refine our understanding of `why?' and `how?' to focus on support between premises and conclusions, and will motivate a slightly weaker idea with respect to support.
  }

  Additional \scen{1} may provide additional motivation for the basic idea.
  Though, I think the basic idea is sufficiently intuitive independently of individual \scen{1}.
  Instead, observe the basic idea may be motivated not only by \scen{1}, but also by theories.
  Perhaps the most prominent is \citeauthor{Davidson:1963aa}' causal theory of action.

  \citeauthor{Davidson:1963aa} opens \textcite{Davidson:1963aa} with the following question:

  \begin{quote}
    What is the relation between a reason and an action when the reason explains the action by giving the agent's reason for doing what he did?
    We may call such explanations \emph{rationalizations}, and say that the reason \emph{rationalizes} the action.%
    \mbox{}\hfill\mbox{(\citeyear[685]{Davidson:1963aa})}
  \end{quote}

  As noted, concluding is an action, and hence \qWhy{} is a particular instance of \citeauthor{Davidson:1963aa}' question.
  And, answers to \qWhy{} will be reasons that rationalise the agent concluding \(\phi\) has value \(v\).

  \citeauthor{Davidson:1963aa} argues, in short, for the following answer to the relation between a reason and the rationalisation of an action:

  \begin{quote}
    \begin{enumerate}[label=\arabic*]
      [R]ationalization is a species of ordinary causal explanation.\newline
      \mbox{ }\hfill\mbox{(\citeyear[685]{Davidson:1963aa})}
    \end{enumerate}
  \end{quote}

  Following \citeauthor{Davidson:1963aa}, an answer to \qWhy{} is a rationalisation, rationalisation is an instance of ordinary causal explanation.
  So, the answer to why an agent concluded \(\phi\) has value \(v\) will, in part, by a cause of the agent concluding \(\phi\) has value \(v\).
  Therefore, an answer to why an agent concluded \(\phi\) has value \(v\) is, in part, an answer to how the agent concluded \(\phi\) has value \(v\).

  Implicit in this quick argument is the idea that a causal explanation answers `how?'.
  Note, however, that we did not appeal to the converse.
  Causal theories of action seem to motivate the basic idea, though I do not think the basic idea (directly, at least) motivates causal theories of action (or, specifically, concluding).
  In other words, for our purposes, answers to `how?' need not be causal explanations, though they may be.

  More broadly, I take the basic idea to capture a pre-theoretical constraint on classes of theories.
  There are theories that agree with the basic idea, such as \citeauthor{Davidson:1963aa}' causal theory of action (when the action is concluding) and there \emph{may be} theories which do not agree with the basic idea --- though I do not know of any specific theories that are explicitly of this kind.
\end{note}

\subsection{Questioning the intuitive relationship}

\begin{note}
  The basic idea is more-or-less the basic issue of this document.

  Both intuitions, such as those regarding \autoref{illu:gist:calc}, and theories, such as \citeauthor{Davidson:1963aa} causal theory of action, provide motivation for the basic issue.

  Our goal is to motivate the following, basic, contrary idea:

  \begin{itemize}
  \item
    There are cases in which something is, in part, an answer to `why?' and that something is \emph{not} (also), in part, an answer to `how?'.
  \end{itemize}

  The basic contrary idea is the negation of the basic idea.
  For, the basic idea states, roughly, answers to `why?' are always included in answers to `how?' while the basic contrary idea states that there are cases in which something that answers `why?' does not also answer `how?'.

  The basic contrary idea, then, has the form of an existential.
  We will not motivate the idea that there is always something which answers `why?' but does not also answer `how?'.
  And, in particular, it may be the case that the intuitions observed with respect to \autoref{illu:gist:calc} are correct.
\end{note}

\begin{note}
  Now, all this has been said without giving attention to the conditional observed by the agent in \autoref{illu:gist:calc}:

  \begin{itemize}
  \item
    If the calculator is trustworthy, then the agent would not fail to conclude \(23 \times 15 = 345\) via their understanding of arithmetic.
  \end{itemize}

  Both natural, and somewhat surprising.

  Consider the contraposition.

  \begin{itemize}
  \item
    If the agent were to fail to conclude \(23 \times 15 = 345\) via their understanding of arithmetic, then the calculator is not trustworthy.
  \end{itemize}

  \begin{itemize}
  \item
    Is it possible, from the agent's perspective, fail to conclude \(23 \times 15 = 345\) via their understanding of arithmetic?
  \end{itemize}

  If possible, then difficulty.
  For, testimony, so must be, but at the same time, possible that it is not.
  If not possible, then it doesn't seem that observing \(23 \times 15 = 345\) via the testimony of the calculator is sufficient.
  For, by the previous observation, difficulty with the testimony of the calculator.

  \begin{itemize}
  \item
    Testimony of the calculator \emph{only if} \(23 \times 15 = 345\) is a \fc{0} given the agent's understanding of arithmetic.
  \end{itemize}
\end{note}

\begin{note}
  The only if, interesting.

  Calculator provides information about what is a \fc{}.
  Agent's understanding of arithmetic is why it is a \fc{}.

  So, if being a \fc{0} is involved in concluding, then it may be that understanding of arithmetic is, in part, an answer to `why?'.
  And, as the agent has not concluded, not, in part, an answer to `how?'.
\end{note}

\begin{note}
  Important:

  Multiple ways to conclude.
  So, have a check.

  Differs to, for example, concluding {\color{red} ???} from a scientific calculator.
  {\color{red} ???} goes beyond typical understanding of arithmetic.
  Parallel pair of conditionals does not hold.

  Or, alternatively, testimony that {\color{red} ???}.
  Beyond understanding.
\end{note}

\begin{note}
  I am not sure what to make of \ref{illu:gist:calc}.
  Understanding of arithmetic is a partial check.
  However, testimony.

  Unsure because status of a premise.

  Basic contrary idea only requires some instances.

  Argument against this intuition.
  Type of cases, premises are fixed.
  Check on own reasoning.

  First, expand on intuition.
  Then, introduce type of \scen{0} of interest.
\end{note}

\subsection{Key things going forward}

\begin{note}
  Three things of interest.

  \begin{enumerate}
  \item
    \scen{1} like \autoref{illu:gist:calc} in which an agent concludes some proposition has some value.
  \item
    Relation between Why and how and agent concludes.
    Basic idea, and basic contrary idea.
  \item
    Idea of a \fc{} and how \fc{1} may be involved in concluding.
  \end{enumerate}


  Chapter will be split into three parts
  In \autoref{cha:clarification}, we will clarify what is of interest.
  In particular, \autoref{cha:clarification} will be split into two subsections.

  In~\autoref{sec:clarification:support} we will clarify why we are interested in intuitions regarding \scen{1} such as \autoref{illu:gist:calc}.


  First, clarify what is of interest.
  Two things in particular.
  1. What it is about concluding.
  2. Scenarios of interest.
  \autoref{illu:gist:calc} is similar to the type of \scen{0} that will be the focus on this document.
  However, the argument we provide will not directly apply to \scen{1} like \autoref{illu:gist:calc}.

  In \autoref{sec:clar:type-of-scen} we present the type of \scen{0} we will present an instance of the type of \scen{0} interested in, provide a general description of the \scen{0} type.
  Further, we will provide a detailed contrast between the type of \scen{0} we are interested in and \autoref{illu:gist:calc}.%
  \footnote{
    Roughly, if it were the agent failed to conclude \(23 \times 15 = 345\) in \autoref{illu:gist:calc}, then there would be conflict between the agent's understanding of arithmetic and the testimony of the calculator.
    Expressed differently, there would be conflict between the agent's failure to conclude \(23 \times 15 = 345\) by their understanding of arithmetic, and a premises involved in concluding \(23 \times 15 = 345\) via the calculator.
    I.e. supposing the agent concludes \(23 \times 15 \ne 345\), then from the agent's perspective the calculator is not a source of testimony.
    In the \scen{1} of interest, this hypothetical --- or in some cases possible --- conflict will strictly be between the agent's reasoning from pools of premises to conclusions.
  }

  Generally speaking, I am unsure about the intuitions expressed with respect to~\autoref{illu:gist:calc}.
  \Autoref{illu:gist:calc} shares an important feature with the type of \scen{0} that we will explore in detail.
  Yet, \dots

  I am inclined to think intuitions are fine.
  variation of scenario, what the difference is, and why focus.
  Second, sketch general argument of the paper, and role of \fc{1}.
\end{note}



%%% Local Variables:
%%% mode: latex
%%% TeX-master: "master"
%%% End:

\part{Claiming support}
\label{part:claimingSupportI}

\chapter{\support{2} and claiming support}
\label{cha:claiming-support}

\begin{note}
  From \autoref{assu:CSVP}, claiming support is an instance of reasoning that concludes that some proposition \(\phi\) has some value \(v\).
  Again, the conclusion of an instance of claiming support is \emph{unqualified}.
  If an agent concludes that the grass is wet, then the agent has claimed support that the grass is wet.
  The agent has not (merely) concluded that the grass is wet so long as it rained last night, or that the sun is not to warm, etc.\

  Still, \autoref{assu:CSVP} only places a restriction on how an instance of reasoning concludes.
  To illustrate, it is compatible with \autoref{assu:CSVP} that an agent concludes that the grass is wet arbitrarily.
  Consider:
  \begin{enumerate}
  \item\label{ex:assu:CSVP:lim:1} Birds are singing, so the grass is wet.
  \item\label{ex:assu:CSVP:lim:2} I dreamt of rain, so the grass is wet.
  \end{enumerate}
  It is not clear how birds signing or dreams of rain relate to the grass being wet, but \autoref{assu:CSVP} is satisfied given that the conclusion of both~\ref{ex:assu:CSVP:lim:1} and~\ref{ex:assu:CSVP:lim:2} is that the grass is wet.
\end{note}

%%% Local Variables:
%%% mode: latex
%%% TeX-master: "master"
%%% End:

\part{Reasoning}
\label{part:reasoning}

\chapter{Introduction}
\label{cha:reasoning-introduction}

\begin{note}
  The purpose of this chapter is to introduce two ideas concerning reasoning --- specifically reasoning which concludes that some proposition \(\phi\) has some value \(v\).
  In contrast to {\color{red} chapter ???}, we will begin with a pair of definitions, and the two ideas will be motivated with reference to the definitions.
\end{note}

\begin{note}
  {
    \color{red}
    The two definitions, and the corresponding ideas.
  }
\end{note}

\begin{note}
  In relation to the broad argument, \ESU{} is what we're arguing against, and \EAS{}, what we're arguing for.
  More specifically, tension between \ideaCS{}, \ESU{}, and ability.
\end{note}

\begin{note}
  Tension, well, only \ESU{} really matters.
  And, \EAS{} is just the negation.
  So, \adB{}, this is something of a detour.

  However, the detour serves two purposes.
  First, clarify \ESU{}.
  Second, \EAS{}, our goal is not only to establish tension, but also to offer a resolution to the tension by rejecting \ESU{}.
  In this respect, \adB{} is important to motivate.

  Further, stated prior, and independently.
\end{note}

\section{Ideas \& definitions}
\label{sec:ideas}

\begin{note}
  The two key ideas of this chapter are \ESU{} and \EAS{}.
\end{note}

\begin{note}
  \targetESU*
\end{note}

\begin{note}
  \goalEAS*
\end{note}

\begin{note}
  Argue that \ESU{} when combined with \ideaCS{} leads to tension in certain cases.
  \EAS{} is, roughly, the negation of \ideaCS{}.
  Our interest with \EAS{} is from the perspective of motivating a rejection of \ESU{} as a resolution to the mentioned tension.

  For the primary argument of this document, then, may focus only on \ESU{}.
  Consideration of \EAS{} may be postponed for a second pass after determining whether the tension developed is of interest.

  Still, given that \ESU{} and \EAS{} are paired, deal with them simultaneously.
  Indeed, begin by providing a abstract characterisation of reasoning which satisfies \ESU{} and \EAS{}.
  Term these two types of reasoning `\adA{}' and `\adB{}'.
\end{note}

\begin{note}
  \defADA*
\end{note}

\begin{note}
  \defADB*
\end{note}

\begin{note}
  So, \ESU{}, this is requiring that we always go with \adA{}.
  And, \EAS{}, some instances of \adB{}.
\end{note}

%%% Local Variables:
%%% mode: latex
%%% TeX-master: "master"
%%% End:

\chapter{Two ways of concluding \(\phi\) has value \(v\)}
\label{cha:reasoning-two-ways}

\section{Introduction}
\label{sec:reasoning-two-ways:intro}



\begin{note}
  In the introduction, outlined two ways of claiming support, and how they relate to two key ideas.
  Here, develop in more detail.
  In the following chapter, then consider ideas in more detail.
\end{note}


\subsection{Initial \illu{1}}

\begin{note}[What we're going to look at]

  {\color{red} footnote}\nolinebreak
  \footnote{
    Here, as with other examples, focus on existential, as this is relevant.
    However, question about semantic counterpart.
    Every model, or there does not exist a model.
    In contrast to existential, pointing to some specific thing won't do.
  }

  \begin{enumerate}[label=\named{\(\exists\mathord{\vdash}{,}\top\)}, ref=\named{\(\exists\mathord{\vdash}{,}\top\)}]
  \item\label{ill:Eproof:def} The existence of a syntactic proof of a formula (using a sound first-order system) is sufficient to establish the formula is a theorem of first-order logic.
  \end{enumerate}
\end{note}

\begin{note}[Memory]
  \begin{illustration}\label{ill:ad:proof:mem}
    \mbox{}
    \vspace{-\baselineskip}
    \begin{enumerate}
    \item\label{ill:Eproof:mem} I remember having created a syntactic proof of \formula{\forall x Px \rightarrow \lnot \exists x \lnot P x} (using a sound first-order system).
    \item\label{ill:Eproof:exP} So, there exists a syntactic proof of \formula{\forall x Px \rightarrow \lnot \exists x \lnot P x} (using a sound first-order system)
    \item\label{ill:Eproof:thm} Hence, by \ref{ill:Eproof:def}, \formula{\forall x Px \rightarrow \lnot \exists x \lnot P x} is a theorem of first-order logic.
    \end{enumerate}
    \vspace{-\baselineskip}
  \end{illustration}

  \autoref{ill:ad:proof:mem} seems a straightforward case of claiming support.\nolinebreak
    \footnote{
      \color{red}
      Whether proving is an unsatisfied \requ{}.
      However, recall that allowed a \requ{} to be satisfied by some instance of concluding.
      And, memory of concluding.
      Still question about original proof, but no problem with memory.
    }

  First, the agent has claimed support for \ref{ill:Eproof:def} by their familiarity with systems of first order logic.

  Second, the agent remembers having created a syntactic proof of the relevant formula.
  And, it seems sufficient, generally speaking, to claim support for some proposition by appealing to memory, hence the agent claims support that there was some event which culminated in a syntactic proof of the formula.\nolinebreak
  \footnote{
    It may be more natural to say `I remember creating\dots' or `I remember proving\dots'.
    The particular phrasing is chosen to remove any ambiguity about whether the agent \emph{finished} the activity creating or proving.
  }
  Of course, the agent may have misremembered, but there seems no issue with the agent expecting that appeal to their memory is \nmom{}.

  Following, this allows the agent to claim support that a syntactic proof of the formula exists.
  As before, the agent may have been \mom{} about whether what they created really was a syntactic proof of the formula
  And, as before it seems the agent may expect that they were not \mom{}.

  Hence, finally, the agent claims support that the formula is a (syntactic) theorem of first-order logic.

  To concisely summarise, we may say that the agent claimed support for the formula being a (syntactic) theorem of first-order logic \emph{because} of their understanding of syntactic theorem-hood and their memory of proving the formula.

  For sure,~\autoref{ill:ad:proof:mem} is designed to be as straightforward as possible.
  Of interest is not whether the agent claims support, but how the role the agent gives to their memory in claiming support.

  The agent appeals to their memory to establish that there exists a syntactic proof of the formula, and then combines the existence of a syntactic proof with~\ref{ill:Eproof:def} to claim support that the formula is a theorem.
  Hence, the agent's memory is directly involved in their claimed support for the formula being a theorem.
\end{note}


\begin{note}
  \begin{illustration}\label{ill:ad:proof:eve}
    \mbox{}
    \vspace{-\baselineskip}
    \begin{enumerate}
    \item I remember having created a syntactic proof of \formula{\forall x Px \rightarrow \lnot \exists x \lnot P x} (using a sound first-order system).
    \item\label{ill:ad:proof:eve:app} In creating the syntactic proof I appealed to various aspects of some sound first-order system.
    \item\label{ill:ad:proof:eve:pos} As I created a proof, those various aspects of the sound first-order system are sufficient to ensure there exists a proof.
    \item Hence, by \ref{ill:Eproof:def}, \formula{\forall x Px \rightarrow \lnot \exists x \lnot P x} is a theorem of first-order logic.
    \end{enumerate}
    \vspace{-\baselineskip}
  \end{illustration}

  As with~\autoref{ill:ad:proof:mem}, the agent's memory has a role in~\autoref{ill:ad:proof:eve}, but the role is quite different.
  Above, the agent claimed support for the formula being a theorem primarily \emph{because} they remembered creating a proof.
  By contrast, here the agent claims support for the formula being a theorem primarily because of the properties of some sound first-order system.

  Step~\ref{ill:ad:proof:eve:app} appeals to various aspects of some sound first-order system and, in turn, step~\ref{ill:ad:proof:eve:app} observes that those aspects are sufficient to ensure a proof exists.
  The agent claims support for the existence of a proof by appeal to the various aspects of some first-order system they appealed to when constructing the proof, rather than their memory of constructing the proof.
\end{note}

\begin{note}
  To help clarify, let's fix a particular syntactic proof using the Fitch-style proof system of~\textcite[557--560]{Barwise:1999tu}:

  \begin{figure}[H]
    \centering
    \begin{quote}
      \fitchprf{}{
        \subproof{\pline[1.]{\forall x P x}}{
          \subproof{\pline[2.]{\exists x \lnot Px}}{
            \boxedsubproof[3.]{a}{\lnot Pa}{
              \pline[4.]{Pa}[\lalle{1}] \\
              \pline[5.]{\bot}[\lfalsei{3}{4}]
            }
            \pline[6.]{\bot}[\lexie{2}{3--5}]
          }
          \pline[7.]{\lnot \exists x \lnot Px}[\lnoti{2--6}]
        }
        \pline[8.]{\forall x Px \rightarrow \lnot \exists x \lnot Px}[\lifi{1--7}]
      }
    \end{quote}
    \caption{A syntactic proof}\label{fig:syntx-prf}
  \end{figure}

  The proof consists of single instances of five introduction or elimination rules.
  Each rule is part of the Fitch-style proof system, and the specific application of the rules constitute the proof.
\end{note}


\begin{note}[Before\dots]
  Before returning to~\autoref{ill:ad:proof:eve}, let us observe that with the proof in hand one may claim support that a proof of the formula exists via the contents of~\autoref{fig:syntx-prf}.

  Broadly stated:

  \begin{enumerate}
  \item The proof is constructed from a sound first-order proof system.
  \item And, the particular application of some rules of the system to formulae is such that the proof begins with no assumptions and the last line of the proof is not part of any assumption made during the course of the proof.
  \end{enumerate}
\end{note}

\begin{note}
  Note, appeal to creation of the proof involves appeal to various aspects of the Fitch-style proof system.

  The object itself is mute to whether or not it is a proof.

  For example, adding `\formula{Ba}' as an assumption would void the proof, but you would need to observe that the appeal to existential elimination on line 6 requires that `\formula{a}' does not appear in the proof prior to its introduction on line 3 in order to claim support that the proof is void.

  Indeed, the proof consists of eight steps, each step is permitted by the first-order system, the proof begins with no assumptions, the last line of the proof is not part of any assumption made during the course of the proof and the proof, and so on.

  Sparing the details, claimed support that~\autoref{fig:syntx-prf} is a syntactic proof of \formula{\forall x Px \rightarrow \lnot \exists x \lnot P x} from the creation of~\autoref{fig:syntx-prf} is a matter of claiming support for each step of the creation.

  Indeed, to spare the details in general, let us instead talk of some collection of propositions and steps of reasoning.
  Claiming support that a proof exists from the some creation in the way under discussion is an instance of reasoning from details of the creation to the conclusion that a proof exists.
  Hence, as an instance of reasoning involves certain premises and steps of reasoning.
  And, whatever these turn out to be, the proceed from the creation of the proof rather than from some other source such as memory, testimony, and so on.
\end{note}

\begin{note}
  In other words, one may claim support that a proof of \formula{\forall x Px \rightarrow \lnot \exists x \lnot P x} exists (primarily) \emph{because} of their reasoning from some collection of premises and steps of reasoning concerning the creation to the existence of a proof of \formula{\forall x Px \rightarrow \lnot \exists x \lnot P x}.
\end{note}

\begin{note}[Return to \ref{ill:ad:proof:eve}]
  Now let us return to the reasoning of~\autoref{ill:ad:proof:eve}, and in particular steps~\ref{ill:ad:proof:eve:app} and~\ref{ill:ad:proof:eve:pos}:
  \begin{quote}
    \begin{enumerate}
      \setcounter{enumi}{1}
    \item In creating the syntactic proof I appealed to various aspects of some sound first-order system.
    \item As I created a proof, those various aspects of the sound first-order system are sufficient to ensure there exists a proof.
    \end{enumerate}
  \end{quote}
  Given that the agent remembers having created a syntactic proof, the `various aspects of some sound first-order system' of step~\ref{ill:ad:proof:eve} may be taken as those aspects of the first-order system that were appealed to in the premises and steps of reasoning when the agent created the proof.
  And step \ref{ill:ad:proof:eve}, in turn, appeals to how those various aspects of some sound first-order system were sufficient for the agent to claim support that a proof exists by the reasoning that occurred.

  In short, the agent remembers creating a syntactic proof and claiming support that a proof exists from the creation.
  The instance of claiming support involved reasoning from premises via steps to the relevant conclusion.
  Hence, it is possible to claim support for the conclusion by those premises and steps of reasoning.
  So, in~\ref{ill:ad:proof:eve} the agent observes that those premises and steps of reasoning are sufficient to claim support by way of their memory, and in turn appeals to those premises and steps of reasoning to claim support for the relevant conclusion.
\end{note}

\begin{note}
  {
    \color{red}
    Propositional support.
    (If I talk about this, it should be after the definitions.)
  }
\end{note}

\begin{note}
  Generalising, the way in which the agent claims support in~\autoref{ill:ad:proof:eve} is of interest because the agent appeals to premises and steps of reasoning that are not `part' of their present reasoning.
  The role of memory in the \illu{0} is (merely) a way for the agent to recognise that there are such premises and steps of reasoning.
  And, in the definitions that follow, we will abstract from any particular way in which the recognises that relevant premises and steps of reasoning are available.

  Still, even though memory is contingent, we may briefly observe that the way in which the agent claim support in~\autoref{ill:ad:proof:eve} is compatible with \ESU{}.
  For, \ESU{} requires that an agent may claim support for some conclusion from premises and steps of reasoning only if the agent has witnessed reasoning to the conclusion from those premises via those steps of reasoning.
  So, if the initial instance of claiming support conformed to \ESU{} then the agent will have witnessed reasoning from those steps and premises to the conclusion --- the instance of claiming support in~\autoref{ill:ad:proof:eve} does not involve such witnessing, but the agent's memory would be about how the relevant premises and steps were used to claim support.

  Of course, the way in which the agent claim support in~\autoref{ill:ad:proof:eve} is incompatible with a strengthened variant of \ESU{} which requires the agent to use any premises and steps they appeal to in the \emph{present} instance of reasoning, but the point for the moment is that the way in which the agent claims support in~\autoref{ill:ad:proof:eve} does not already require what we are arguing against: \ESU{}.
\end{note}

\subsection{Definitions}

\begin{note}
  With a somewhat detailed pair of contrasting \illu{1} in hand, we now turn to fixing a pair of definitions which capture the general way in which the agent claims support in the respective illustrations.

  The two ways will be termed `\adA{}' and `\adB{}', respectively.
\end{note}

\begin{note}
  \defADA*
\end{note}

\begin{note}
  \adA{} does not outline a specific way of reasoning.
  Rather, captures the role of claimed support for \(\phi\) having value \(v\) in some instance of reasoning when the agent claims support for \(\psi\) having value \(v'\).

  Intuitive idea is that claimed support for \(\phi\) having value \(v\)~\ref{def:adA:phi} provides agent with resource to claim support for \(\psi\) having value \(v'\) to~\ref{def:adA:psi}.
\end{note}

\begin{note}
  Applied to the two sketches seen, claim support by existence of proof, or by specific ability to \emph{V} that \(\phi\).
  Key thing is that claimed support for existence of proof or the specific ability to \emph{V} that \(\phi\) rather than something else.

  \phantlabel{abstract-adA}
  Indeed, noting and abstracting from the role of conditionals in these two \illu{1}, basic (abstract) instance of \adA{}:

  {
    \small
    \begin{enumerate}[label=\arabic*., ref=\arabic*]
    \item\label{def:adA:ex:C:Cp} I have concluded \(\phi\) has value \(v\).
    \item\label{def:adA:ex:C:p} So, \(\phi\) has value \(v\). \hfill(From~\ref{def:adA:ex:C:Cp})
    \item\label{def:adA:ex:C:Cps} Likewise, I have concluded \(\psi\) has value \(v'\) when \(\phi\) has value \(v\).
    \item\label{def:adA:ex:C:ps} So, \(\psi\) has value \(v'\) when \(\phi\) has value \(v\). \hfill(From~\ref{def:adA:ex:C:Cps})
    \item\label{def:adA:ex:C:T} If \(\psi\) has value \(v'\) when \(\phi\) has value \(v\) and \(\phi\) has value \(v\), then it must be the case that \(\psi\) has value \(v'\). \hfill (From understanding of `if\dots then\dots')
    \item\label{def:adA:ex:C:s} Hence, \(\psi\) has value \(v'\).\newline
      \mbox{}\hfill (From \ref{def:adA:ex:C:p},~\ref{def:adA:ex:C:ps}~and~\ref{def:adA:ex:C:T})
    \item Therefore, I conclude \(\psi\) has value \(v'\). \hfill (From \ref{def:adA:ex:C:Cp} -- \ref{def:adA:ex:C:s})
    \end{enumerate}
  }
  The reasoning is a verbose because claimed support is not necessarily factive
  \nolinebreak
  \footnote{
    It may be that an agent has claimed support for \(\phi\) having value \(v\) while \(\phi\) has value \(v'\).
  }
  yet the agent has claimed support about \(\phi\) having value \(v\) and how that relates to \(\psi\) having value \(v'\), rather than how claimed support for \(\phi\) having \(v\) relates to claimed support for \(\psi\) having value \(v'\),
  (Consider parallel reasoning with knowledge, rather than (mere) claimed support.\nolinebreak
  \footnote{The parallel reasoning in full:
    \begin{enumerate}[label=\arabic*., ref=\arabic*]
    \item\label{def:adA:ex:K:Kp} I know \(\phi\) has value \(v\).
    \item\label{def:adA:ex:K:p} So, \(\phi\) has value \(v\). \hfill (From~\ref{def:adA:ex:K:Kp})
    \item\label{def:adA:ex:K:Kps} I know \(\psi\) has value \(v'\) when \(\phi\) has value \(v\).
    \item\label{def:adA:ex:K:ps} So, \(\psi\) has value \(v'\) when \(\phi\) has value \(v\). \hfill(From~\ref{def:adA:ex:K:Kps})
    \item\label{def:adA:ex:K:T} If \(\psi\) has value \(v'\) when \(\phi\) has value \(v\) and \(\phi\) has value \(v\), then it must be the case that \(\psi\) has value \(v'\). \hfill (From understanding of `if\dots then\dots')
    \item\label{def:adA:ex:K:s} Hence, \(\psi\) has value \(v'\). \hfill (From \ref{def:adA:ex:C:p},~\ref{def:adA:ex:C:ps}~and~\ref{def:adA:ex:C:T})
    \item So, I know \(\psi\) has value \(v'\) as \(\psi\) having value \(v'\) follows from~(\ref{def:adA:ex:K:Kp}) and~(\ref{def:adA:ex:K:Kps}).
      \mbox{}\hfill (From \ref{def:adA:ex:K:Kp} -- \ref{def:adA:ex:K:s})
    \end{enumerate}
  }%
  )
  Still, the reasoning is a clear instance of claiming support for \(\psi\) having value \(v'\) by \adA{} from \(\phi\) having value \(v\) as the agent claims support for \(\psi\) having value \(v'\) by appealing to their claimed support for \(\phi\) having value \(v\) to satisfy the antecedent of a conditional.

  Still, \adA{} need not involve a conditional.
  Consider, for example, claiming support that they have claimed support for a contradiction from claimed support that \(\phi\) and not-\(\phi\) are both true.
  It seems plausible that so claiming need only involve a reflection on what \(\phi\) and not-\(\phi\) amounts to.
\end{note}

\begin{note}
  \defADB*
\end{note}

\begin{note}
  With \adA{} \(\phi\) having value \(v\).
  \adB{} does not involve the agent claiming support for \(\phi\) having value \(v'\) by \(\phi\) having value \(v\).
  Instead, some (distinct) collection of premises \(\rho_{1},\dots,\rho_{k}\) with respective values and steps \(\delta_{1},\dots,\delta_{m}\).

  The key difference between \adA{} and \adB{}:
  \begin{itemize}
  \item \adA{} involves the agent appealing to \(\phi\) in order to claim support for \(\psi\), while
  \item \adB{} does not involve the agent appealing to \(\psi\) to claim support for \(\psi\).
    Instead, the role of \(\phi\) is to highlight \(\rho_{1},\dots,\rho_{k}\) and the agent appeals to propositions \(\rho_{1},\dots,\rho_{k}\) to claim support for \(\psi\).
  \end{itemize}

  For the definition to be satisfied, \(\phi\) needs only be involved to the extent that it provides the link.
  Hence, \(\phi\) is not irrelevant.
  Still, the agent does not appeal to \(\phi\).
\end{note}

\begin{note}[How appeal?]
  \autoref{def:adB} does not state relation the agent has to premises.
  Issue here is that we only need a way in which \autoref{def:adB}.
  So, no reason to narrow definition.
  And, certainly no point in motivating restriction.

  Assume:
  \begin{itemize}
  \item Agent has already claiming support for \(\rho_{1},\dots,\rho_{n}\).
  \end{itemize}

  This is in the background of the \illu{1}.

  Stronger, not explicitly ruled out:

  \begin{enumerate}
  \item Possibility \(\rho_{1},\dots,\rho_{n}\) secured by \(\phi\), and this is sufficient.
  \end{enumerate}

  This seems unintuitive.
  However, there is no point to ruling this out.
\end{note}

\begin{note}
  Only seen \adB{} with respect to proof \illu{0}.
  Remembered proving \(\phi\), that secures the possibility, but claiming support from the details of the proof itself.

  Intuitively, applies to ability in the same way.
  Premises and steps of reasoning work in the same way as components of a proof.
  However, we will delay details until we've seen a few more \illu{1}.
\end{note}

\begin{note}
  Broad distinction, agent may claim support by appeal to some thing, but it is also possible to break that thing down in to parts or elements such that the agent may claim support by appeal to those parts or elements (and how they compose).

  `Break down' is metaphorical.

  In some cases, the thing itself, in other cases, more basic stuff that must be the case in order for the thing to be the case.

  Break down does the work.
  Agent will typically recognise.
  Break down is not required.

  In this sense, break down is more fundamental.

  `Because\dots'

  Unifying feature is that \adA{} allows claim support for \adB{}, so not clear that need to go via \adB{}.
  Indeed, unclear, given \ESU{}, that may claim support by \adB{}.
  We will only push this question with respect to ability, though.
\end{note}

\begin{note}[\ESU{}]
  We noted above that the reasoning of~\ref{ill:ad:proof:eve} was compatible with \ESU{}.
  The reasoning of~\ref{ill:ad:proof:eve} is an instance of \adB{}.
  Hence, there are instances of \adB{} which are compatible with \ESU{}.

  Argue that there are instances which are not compatible.
\end{note}


\subsection{Additional illustrations}

\begin{note}

  \begin{illustration}
    \mbox{}
    \vspace{-\baselineskip}
    \begin{itemize}
    \item If bag are overweight then they can't be taken on the flight.
    \item Machine reads\dots
    \item Bag can't be taken on the flight.
    \end{itemize}
  \end{illustration}
  Contents of the bag are overweight.

  Combined weight of the items versus the combination of the individual weights.

  Compare, filling the bag and weighing it, versus summing the weight of the items as you fill the bag.

  Now, seems possible to fill the bag and weight it, then appeal to the sum of the items.

  So, this is a little more subtle.
  The bag has been weighed, and the distinction is between the weight of the contents of the bag, and the combined weight of the items that make up the contents of the bag.

  This is particularly interesting.
  Because, it seems clear that something is strange if someone talks about the weight of the contents of the bag without recognising that this is a function of the combined weight of all the individual elements of the bag.
  However, no idea what the contents of the bag are.

  So, claiming support from what is has been observed, the combined weight, rather than what must be the case in order to have made the observation.
\end{note}

\begin{note}[Fire alarm]
  \begin{illustration}
    \mbox{}
    \vspace{-\baselineskip}
    \begin{itemize}
    \item Fire alarm is ringing.
    \item Fire in the building.
    \item Should leave by the nearest exit.
    \end{itemize}
  \end{illustration}
  So, claiming for getting out of the building.
  Fire alarm.
  Or, fire, fire alarm has picked this up.

  So, difference between that there is a fire in the building, and \emph{the} fire in the building.

  The point here is that, okay, you need to go from alarm to fire, that's all fine, but fire itself is sufficient to claim support.
\end{note}

\begin{note}

  \begin{illustration}\label{ill:ad:factorial}
    \mbox{}
    \vspace{-\baselineskip}
    \begin{itemize}
    \item It is possible to write recursive functions in C.
    \item It is possible to write a recursive implementation of the factorial function in C.
    \end{itemize}
  \end{illustration}
  With proofs, abstract objects.

  Consider programming.

  Recursive implementation of factorial in C (chosen to make the implication clear).

  So, \adA{} is just the fact, so to speak.
  But, \adB{} points to the key step of calling function.
  Of course, this is just recursion, but appeal here is to the concept, so to speak, rather than the truth of the statement.

  Don't need to understand details.
  Go by form, so to speak.

  Claiming support by logical relation, rather than the states of affairs that ensure those logical relations hold up.

  Or, the definition is such that\dots
\end{note}

\begin{note}[Existentials]
  In a sense, the point here is that \adA{} cases of interest mean that there's something more.
  This is viewed in terms of some complex of more basic things existing.
  And, \adB{} follows the reference.
\end{note}

\subsection{The distinctions are (sufficiently) exhaustive}
\label{sec:ar-wr-are}

\begin{note}[Style of argument]
  Well, with respect to claiming support for \(\psi\) such that \(\phi\) is involved.

  Either \adA{} or \adB{}.
  \adA{} seems sufficiently clear, so:
  Transform this to: If not \adA{} then \adB{}.
  Equivalent.
\end{note}

\begin{note}[Idea]
  So, \(\phi\) is involved, but isn't \adA{}.
  Hence, agent claims support by something else.
  If \(\phi\) isn't related to that stuff in any way, then completely redundant.
  Note, from agent's perspective, rather than possibility of revising without.
  So, seems it can only be about how those other things relate to the conclusion.

  Okay, so idea is that if no \adA{} then \(\phi\) isn't part of claiming support.
  If other stuff without \(\phi\) then redundant.
  So, if \(\phi\) is involved, about how the other stuff relates.
\end{note}


%%% Local Variables:
%%% mode: latex
%%% TeX-master: "master"
%%% End:

\chapter{Two ideas}
\label{cha:reasoning-two-ideas}

\paragraph{Beyond causation}
\label{sec:motivating-ESU:beyond-causation}

\begin{note}
  \begin{quote}
    ``Plenty of blank leaves, I see!'' the Tortoise cheerily remarked.
    ``We shall need them \emph{all}!''
    (Achilles shuddered.)
    ``Now write as I dictate:---

    \begin{enumerate}[label=(\emph{\Alph*})]
    \item Things that arc equal to the same are equal to each other.
    \item The two sides of this Triangle are things that are equal to the same.
    \item If \emph{A} and \emph{B} are true, \emph{Z} must be true.
      \setcounter{enumi}{25}
    \item The two sides of this Triangle are equal to each other.''
    \end{enumerate}

    ``You should call it \emph{D}, not \emph{Z},'' said Achilles.
    ``It comes \emph{next} to the other three.
    If you accept \emph{A} and \emph{B} and \emph{C}, you \emph{must} accept Z.''

    ``And why \emph{must} I?''

    ``Because it follows \emph{logically} from them.
    If A and B and C are true, Z \emph{must} be true.
    You don't dispute \emph{that}, I imagine?''

    ``If \emph{A} and \emph{B} and \emph{C} are true, \emph{Z} \emph{must} be true,'' the Tortoise thoughtfully repeated.
    ``That's \emph{another} Hypothetical, isn't it?
    And, if I failed to see its truth, I might accept \emph{A} and \emph{B} and \emph{C}, and \emph{still} not accept \emph{Z}, mightn't I ?''

    \mbox{}\hfill\(\vdots\)\hfill\mbox{}

    ``Then Logic would take you by the throat, and force you to do it!''
    Achilles triumphantly replied. ``Logic would tell you 'You ca'n't help yourself.
    \dots''\nolinebreak
    \mbox{}\hfill\mbox{(\Citeyear[279--280]{Carroll:1895uj})}
  \end{quote}
  Achilles seems to be arguing that accepting \emph{A} and \emph{B} and \emph{C} would be sufficient.
  Tortoise denying that \emph{A} and \emph{B} and \emph{C} are sufficient.
  In particular, possible to accept \emph{A} and \emph{B} and \emph{C} without accepting \emph{Z}.

  Achilles is not citing a causal relation between [\emph{A} and \emph{B} and \emph{C}] and \emph{Z} (or, rather, \emph{D}).
  However, Achilles does seem to be citing a causal relation between accepting [\emph{A} and \emph{B} and \emph{C}] and accepting \emph{Z}.

  The Tortoise would violate some causal law.
  However, despair sets in as Achilles observes that the Tortoise is violating the proposed causal law, and hence it is no law at all.
\end{note}

\begin{note}
  Though the Tortoise has accepted \emph{A} and \emph{B} and \emph{C}, the Tortoise has not \emph{taken} \emph{A} and \emph{B} and \emph{C} to support \emph{Z}.

  For \citeauthor{Boghossian:2014aa}, The Taking Condition expands on a causal process.
  
\end{note}

\subsection{\EAS{0} --- \EAS{}}
\label{sec:eas}

\begin{note}
  Briefly stated,
  \AR{} understands ability in terms of some (complex) property.
  \WR{} understands ability in terms of possible witnessing events.

  For example, \AR{} may involve the property (attribution) of understanding geometry, perhaps broken down into the understanding or availability of various definitions, propositions, lemmas, theorems, and steps of reasoning.
  While, \WR{} would involve reasoning with particular definitions, propositions, lemmas, theorems, and steps of reasoning.

  So, agent appeals to property, or the reasoning itself.

  The purpose of this distinction is to ensure that our argument against \ESU{} does not rest on a particular way of understanding ability that may not extend to other ways of understanding ability.

  Conjecture that these are fundamentally connected.
  Witnessing event only if understanding.
  And, understanding only if possible to witness reasoning.

  Still, difference.
  Relevant properties are properties of the agent as they are.
  The witnessing event, by contrast, is a possible event.\nolinebreak
  \footnote{
    Property of there being a possible event involving the agent.
    In this case, still distinct from \WR{} as that the agent is part of possible event is still distinct from the reasoning that the agent would witness in the relevant event.
  }
\end{note}

\begin{note}
  Now turn to the kind of reasoning involved.

  Motivated \AR{} in terms of understanding of premises and steps of reasoning, and \WR{} in terms of a possible event in which agent reasons with particular premises and steps.

  However, a further distinction in terms of what appeal to the relevant premises and steps or instance of reasoning amounts to.

  First, there is the \emph{existence} of premises and steps, or the \emph{possibility} of the witnessing event.
  Second, there is the premises and steps themselves, or the witnessing event.

  Difference from perspective of step of reasoning.
\end{note}

\subsubsection{Ability and dispositions}

\begin{note}[Parallel]
  To further clarify the motivation for \EAS{} we introduce a parallel between abilities and dispositions.
  The primary function of the parallel will be to highlight the importance of reasoning about an event.
  In the case of dispositions the event is the manifestation of the disposition, and in the case of ability the event is the agent witnessing the ability.

  The parallel is of interest because \EAS{} concerns the premises and steps of reasoning that the agent would use to witness the relevant ability.
  We will suggest that claiming support that some object has some disposition and that some agent has some ability may both be understood in terms of claiming support that the relevant event is a possible event.

  In turn, if reasoning \emph{to} a specific ability is understood in terms of claiming support that it is possible for the agent to witness the event, then reasoning \emph{from} a specific ability may be understood in terms of claiming support from what would happen in the possible event.
  \end{note}

\begin{note}[Parallel between dispositions and ability]
  Consider \citeauthor{Choi:2021wg}'s characterisation of the Simple Conditional Analysis of dispositions:
  \begin{quote}
    An object is disposed to \emph{M} when \emph{C} iff it would \emph{M} if it were the case that \emph{C}.\nolinebreak
    \mbox{}\hfill\mbox{(\Citeyear{Choi:2021wg})}
  \end{quote}
  For example, an object is disposed to dissolve when it is placed in water iff the object would dissolve if it were the case that it is placed in water.

  The Simple Conditional Analysis may be challenged, but for our purposes it is adequate.
  We are interested in the broad form of the truth condition, and various more refined analyses share the same broad form.
  Note, in particular, that it being the case that \emph{C} and \emph{M} happening describes an event.
  Given appropriate conditions; salt dissolves, glass breaks, and I mumble when I am tired.
  The key idea is that the property of being disposed to \emph{M} when \emph{C} is analysed in terms of the (possible) event of \emph{M} happening when \emph{C}.

  The parallel to ability is established by noting that ability may also be analysed in terms of a (possible) event, as we have seen.
  In particular, by incorporating volition in the analysans of the Simple Conditional Analysis.
  To illustrate, \citeauthor{Mandelkern:2017aa} trace the Conditional Analysis of ability  to \textcite{Hume:1748tp} and \textcite{Moore:1912te}, among others:
  \begin{quote}
    S can \(\phi\) iff S would \(\phi\) if S tried to \(\phi\)\nolinebreak
    \mbox{}\hfill\mbox{(\Citeyear[Cf.][308]{Mandelkern:2017aa})}
  \end{quote}
  Compare to the Simple Conditional Analysis of dispositions:
  The object is some agent \emph{S}, \emph{C} is `S tried to \(\phi\)' and \emph{M} is `S \(\phi\)s' --- it is volition alone which distinguishes the analyses.
  For example, I have the ability to demonstrate that a rectangle with dimensions \(19\text{cm}\) by \(7\text{cm}\) has area \(133\text{cm}^{2}\) only if I would demonstrate that a rectangle with dimensions \(19\text{cm}\) by \(7\text{cm}\) has area \(133\text{cm}^{2}\) if it were the case that I tried that a rectangle with dimensions \(19\text{cm}\) by \(7\text{cm}\) has area \(133\text{cm}^{2}\).
\end{note}

\begin{note}[Claiming support]
  Parallel analyses in hand, we now turn to claiming support.
  We start with dispositions.

  As with ability, there are various ways in which an agent may claim support that some object is disposed to \emph{M} when \emph{C}.
  For example, I may claim support that my shoes are disposed to squeak when wet because I have had sufficient occasion to observe the phenomenon.
  Likewise, I may claim support that any shoe of the same model is disposed to squeak when wet because I have traced the source of the squeak to a manufacturing choice.
  In short, support may be claimed by past event and shared properties.

  Still, take a novel act and a object pair.
  Personally, I have a empty fountain pen that I haven't placed in water.
  I claim that the fountain pen is disposed to float when placed in water.
  My reasoning is fairly simple.
  The fountain pen is quite light, especially so while empty of ink.
  And, the cap and loading mechanism seem to be quite well sealed, so the weight of the fountain pen will not increase by taking on water.
  So, given that the weight of the fountain pen will be unchanged, and given how light the pen is, it seems that the upward force exerted by the water against the fountain pen will be sufficient to keep the pen afloat.

  In short, I've noted a few properties of the pen, claimed support for a handful of others, and then considered what would happen.
  Our interest is with the last step.
  I appeal to, and use, the possible event.\nolinebreak
  \footnote{
    I may be wrong about the event, but that isn't at issue.
    It remains the case that I appeal to it.
  }
  The noted properties are relevant because they suggest that the event of floating would happen if it were the case that the fountain pen were placed in water.
\end{note}

\begin{note}
  The fountain pen is not the only object on my desk.
  Beside the fountain pen is a collection of instruments that I may use to investigate the fountain pen.
  And, stored in my mind is a basic understanding of fluid dynamics.

  If I were to measure the fountain pen, ensure that it is airtight, and appeal to some known facts, then an application of Archimedes' principle would allow me to demonstrate that the fountain pen is disposed to float when placed in water (of some specified density).
  Indeed, such a demonstration would be a straightforward refinement of the way in which I claimed support for the proposition that the pen is disposed to float when placed in water.

  Now, by similar reasoning I have claimed support for the proposition that I have the ability to demonstrate that the proposition that the pen is disposed to float when placed in water is true.
  Here, in addition to appealing to properties of the fountain pen, I also appealed to various mental properties.
  There is an important difference, however, regarding the relevant event.
  When reasoning about the disposition, the event is the fountain pen floating in water, but when reasoning about my ability to demonstrate the event is the demonstration --- a series of measurements and calculations.
\end{note}

\begin{note}[Diverge]
  Now to turn to \EAS{}.

  If I have the ability, then it follows that the fountain floats in water.
  As noted above, it is not possible for me to demonstrate something that is not the case.

  Claim support for the proposition that the fountain floats in water.

  Still, disposition, fountain pen is not floating in water.
  Likewise with respect to ability, I have not demonstrated that the fountain pen floats in water.
  I noted various things, but did not piece these together into a demonstration.

  Yet, in claiming support, there's the event of demonstrating.
  And, so I appeal to those premises and steps I would use in the event.
  This is \EAS{}.

  Appeal to what happens in the event.
  And, reasoning to claim possible event is viewed in terms of ensuring that the resources are available.
  I have not used the relevant premises and steps of reasoning, nor am I clear on the specific form they will take.
  Still, they are available.

  Final point of interest, then.
  In both cases, there's an appeal to an event.
  If \EAS{} holds with respect to ability, does something similar hold with respect to dispositions?

  First, important clarification.
  The reasoning outlined for disposition was claiming support for event.
  Here, no clear issue with \ESU{}.
  Similarly, no clear issue with \ESU{} with respect to claiming support for having an ability.
  Tension with \ESU{} arises when using ability as a premise in further reasoning.

  Second, key divergence.
  Conclusion obtained is something that is true independent of ability.
  Unclear to me whether similar reasoning with dispositions.
  For, ability is about an event involving the agent.

  In addition, there is no issue with supposing that the agent reasons with (and hence uses) to all the relevant features of the event.\nolinebreak
  \footnote{There may me details of reasoning that one is not easily able to express, but it doesn't follow that those details are not used.}
  Ability is in part interesting because it is clear that an agent does not witness the relevant event.
  This is not to say that a variant of \EAS{} does not hold with respect to dispositions.
  Rather, I am expressing
  \begin{enumerate*}
  \item hesitancy that there are comparable entailments, and
  \item concern that there is no clear argumentative path.
  \end{enumerate*}
\end{note}

\begin{note}[Concluding parallel]
  To summarise.
  \begin{itemize}
  \item Parallel between analysis of dispositions and abilities.
  \item Event in analysis of both.
  \item Reason about event.
  \item Motivation for \EAS{} by considering reason to and from event.
  \item This doesn't provide anything close to a clear theoretical account of the reasoning performed if \EAS{} is true, but it does hint at such at how developing such an account may be approached.
  \item Now turn to related conclusion.
  \item In turn, fill in some details on the account.
  \end{itemize}
\end{note}

%%% Local Variables:
%%% mode: latex
%%% TeX-master: "master"
%%% End:


\part{Ability}
\label{part:ability}

\chapter{Ability}
\label{sec:major-argument}
\label{sec:broad-argum-overv}
\label{sec:all-about-ability}

\section{Scenarios}
\label{sec:cases-interest}

Our goal is to argue for \EAS{} and against \ESU{}.
At the core of the argument is reasoning about ability.
Specifically, a certain type of scenario in which an agent reason to and from information that they have the ability to witness some specific act.
How the agent reasons with such (specific) ability information in the scenarios of interest will provide a type of counterexample to \ESU{} and in turn an argument for \EAS{}.

In this section we outline two key features of the scenarios we are interested in.
Subsection~\ref{sec:type-scenario} will introduce \gsi{0} to characterise how the agent reasons to the (specific) ability information.
Then, subsection~\ref{sec:ability-entailment} will introduce `\aben{the}' to characterise how the agent reason from the (specific) ability information.
Finally, subsection~\ref{sec:scenarios} will combine \gsi{0} and `\aben{the}' to provide an in-depth understanding of the type of scenarios we are interested in.

\subsection{\gsi{2}}
\label{sec:type-scenario}

\begin{note}[Tension, information]
    \begin{restatable}[\gsi{}]{definition}{defGSI}\label{def:gsi}
    \gsi{2} is information that:\nolinebreak
    \footnote{
      Strictly speaking the formulation of \gsi{} as a conditional isn't important.
      What matters is that the agent is required to claim support for the general ability in order to claim support for the specific ability.
      For example, the conditional may be reformulated as:
      \begin{enumerate}[label=(\gsi{}\('\)), ref=(\gsi{}\('\))]
      \item Either \emph{S} does not have the general ability to \(\gamma\), or the agent has a specific ability to \(\varsigma\).
      \end{enumerate}
    }
    \begin{quote}
      If \emph{S} has a general ability to \(\gamma\), then \emph{S} has a specific ability to \(\varsigma\).
    \end{quote}
    Where \emph{S} is some agent, \(\gamma\) is some general ability, \(\varsigma\) is some specific ability, and it is either implicitly or explicitly stated that \(\varsigma\) is instance of \(\gamma\).
  \end{restatable}
  
  The following pair of examples are instances of \gsi{}.
  \begin{enumerate}[label=(\gsi{}:\arabic*), ref=(\gsi{}:\arabic*)]
  \item\label{qe:cond} If you have the ability to reason with the rules of chess, then you have the ability to demonstrate that, given the arrangement of the board, there is a sequences of moves that will ensure a win for one of the players (as an instance of the general ability to reason with the rules of chess).
  \end{enumerate}

  \begin{enumerate}[label=(\gsi{}:\arabic*), ref=(\gsi{}:\arabic*), resume]
  \item\label{qe:cond:french} If you have the ability to read French, then you have the ability to read The Count of Monte Cristo without a translation (as an instance of the general ability to read French).
  \end{enumerate}
  In both examples an agent is informed that they have the ability to perform a specific act --- demonstrating a strategy or reading a book --- so long as they have some general ability --- an understanding of chess or French literacy --- because the witnessing the specific ability act would be an instance of witnessing the agent's general ability.

  \gsi{} does not directly provide the agent with the information that they have the specific ability.\nolinebreak
  \footnote{Nor (looking ahead to section~\ref{sec:ability-entailment}) does \gsi{} directly provide the agent with information that the result of witnessing the specific ability is when \aben{the} holds with respect to the specific ability.}
  The agent is not informed that they have the general ability and that therefore they have a specific ability.
  To illustrate, I am confident I have the ability to reason with the rules of chess, and so given \ref{qe:cond} I may be confident that I am able to demonstrate the existence of such a strategy.
  By contrast, I do not have the ability to read French, and so I do not have the ability to read The Count of Monte Cristo without a translation.

  Still, I may also be mistaken.
  It may be that I am overconfident, that I do not have the ability to reason with the rules of chess, and hence it may be the case that I do not have the ability to demonstrate the existence of the relevant chess strategy.
  Likewise, I may have the ability to read French, and may have the ability to read The Count of Monte Cristo without a translation.
  However unlikely this may be, I haven't tried to read French in quite some time.
\end{note}

\begin{note}[Not direct]
  \gsi{2} contrasts with what we term `\dsi{0}' --- information that the agent has some ability.
    \begin{restatable}[\dsi{}]{definition}{defDSI}\label{def:dsi}
    \dsi{2} is information that:
    \begin{quote}
      \emph{S} has the ability to \(\varsigma\).
    \end{quote}
    Where \emph{S} is some agent and \(\varsigma\) is some specific ability.
  \end{restatable}
  For example, the following is a `direct' recreation of~\ref{qe:cond}:

  \begin{enumerate}[label=(\dsi{}:\arabic*), ref=(\dsi{}:\arabic*), series=dsi_count]
  \item\label{qe:cons} You have the ability to demonstrate that there is a sequences of moves that will ensure a win for one of the players as an instance of your general ability to reason with the rules of chess.
  \end{enumerate}

  If~\ref{qe:cons} is true then the agent has the ability to demonstrate some strategy.
  And, in turn,~\ref{qe:cons} expands on why the agent has the relevant specific ability.
  By contrast,~\ref{qe:cond} may be true even if the agent does not have the ability to demonstrate some strategy.
  Hence, \dsi{} is not in general entailed by \gsi{}.\nolinebreak
  \footnote{
    However, if it is the case that an agent has the general ability mentioned in the antecedent of \gsi{}, then a corresponding instance of \dsi{} will be true.
    Note, this is ensured because the consequent of~\ref{qe:cond} ensures the relevant `instance of' relation obtains.
    % So, if I have the ability to reason with the rules of chess and~\ref{qe:cond} is true with respect to me, then \ref{qe:cons} will also be true with respect to me.
  }
\end{note}

\begin{note}[Important features of \gsi{}]
  \gsi{}, then, has two important features:
  \begin{enumerate}
  \item \gsi{} ensures that the agent is on the hook, so to speak, for claiming support they have the specific ability.
  \item If the agent may claim support for having the relevant general ability, then \gsi{} provides the agent with an account of why they may claim support for having some specific ability.
  \end{enumerate}
  Hence, \gsi{} ensure that an agent must themselves claim support that they have some specific ability while providing the agent with relevant information about why they may claim support for having the specific ability.
\end{note}

\begin{note}[Merging \gsi{} and \dsi{}]
  Finally, though we will focus on \gsi{}, there is a variant that merges \gsi{} and \dsi{} which could be substituted for \gsi{} in further discussion.
  This variant involves informing an agent that they have some general ability, and some specific ability as an instance of that general ability, but requires the agent to identify what the general ability is.

  Here is the variant applied to~\ref{qe:cond}.
  \begin{enumerate}[label=(\gsi{}\(^{'}\):\arabic*), ref=(\gsi{}\(^{'}\):\arabic*)]
  \item
    \begin{enumerate}
    \item You have some general ability \(\gamma\), and a specific ability \(\varsigma\) (as an instance of that general ability). And,
    \item If \(\gamma\) is the ability to reason with the rules of chess, then \(\varsigma\) is the ability to demonstrate that, given the arrangement of the board, there is a sequences of moves that will ensure a win for one of the players (as an instance of the general ability)
    \end{enumerate}
  \end{enumerate}
  The agent remains on the hook, so to speak, for claiming support that they have the relevant specific ability because it is up to the agent to identify the general ability \emph{as} the ability to reason with the rules of chess.
  And, likewise, if the agent may claim support for identifying the general ability in a particular way, then the variant allows the agent to claim support that they have a particular specific ability.

  We favour \gsi{} given it's comparative structural simplicity, but the variant highlights that that the agent claiming support for having some specific ability is not of interest.
  Rather, what is interest is that \gsi{} allows the agent to claim support for the particulars of some specific ability.

  In section~\ref{sec:ability-entailment} we will highlight why the particulars matter.
\end{note}

\subsection{An ability entailment}
\label{sec:ability-entailment}

\begin{note}[\aben{(The)}]
  The second component in scenarios of interest is the availability of an entailment from the specific ability.

  We term an instance of the entailment as an `\aben{}'.

  \begin{restatable}[Ability entailment]{definition}{defAE}\label{def:aben}
    \aben{The} is any entailment of the form:
    \begin{quote}
      \emph{S} has the (specific) ability to \emph{V} that \(\phi\) \emph{therefore} \(\phi\) is the case.
    \end{quote}
    Where \emph{S} is an agent, \emph{V} is some action, and \(\phi\) is some proposition.
  \end{restatable}

  The rough intuition behind instances of \aben{the} is that \(\phi\) being the case does not depend on \emph{S} witnessing the (specific) ability to \emph{V} that \(\phi\).
  So, \aben{the} links ability and something that must be the case in order to have ability and the result of witnessing ability must be the case in order for the agent to have the ability

  For example, \aben{the} holds with respect to the (specific) ability to demonstrate the existence of a chess strategy from \ref{qe:cond} as whether or not a given chess strategy exists depends on the moves permitted by the rules of chess --- a strategy that has not been demonstrated is a strategy.
  Likewise, \emph{S} has the (specific) ability to discover that their keys are in their jacket pocket only if it is the case that their keys are in their jacket pocket --- whether or not \emph{S}'s keys are in their jacket pocket does not depend on \emph{S} discovering that to be the case.

  By contrast, `to read The Count of Monte Cristo without a translation' is an action and so \aben{the} does not apply to the specific ability of~\ref{qe:cond:french}.
  Even so, \aben{the} apply to nearby variants and not others.
  \emph{S} may have the specific ability to read that Dantès was a merchant sailor, and it follows that Dantès was a merchant sailor.
  In contrast, while \emph{S} may have the ability to believe that certain passages cannot be adequately translated, it does not follow that those passages cannot be adequately translated.
  Similarly, \emph{S} may have the ability to hope that they will employ the chess strategy discovered in a competitive game, but it does not follow that \emph{S} will employ the strategy.

  More broadly, \aben{the} holds with respect to factive verbs, such as `see', `know', `understand', and so on.
  Though, I doubt factive verbs are an adequate explanation for \aben{the}.
  Consider `read'.
  I have the ability to read that Elvis Presley was born in 1935, but I also have the ability to read that Elvis is working undercover for the DEA.
  What matters, then, is not the verb used, but how the agent would witness the relevant ability.
  I have the ability to read that Elvis was born in 1935 from a reliable source, and hence \aben{the} applies.
  The same is not true for my ability to read that Elvis is working for the DEA.

  Indeed, \aben{the} merely identifies an entailment.
  It does not provide an account of when or why such entailments hold.
  We identify entailments of this type because our interest is in how (in certain cases) agent's reason with instances of \aben{the}.
\end{note}

\subsection{Details of scenarios}
\label{sec:scenarios}

\begin{note}[Both things are important]
  The scenarios we are interested in combine \gsi{} with \aben{the}.

  The role of \gsi{} is to ensure that the agent is not provided with direct information about specific ability.
  And the role of \aben{the} is to highlight that the agent is in a position to claim support for some further proposition if they claim support for specific ability.
  Hence, scenarios combine claiming support \emph{for} specific ability and claiming support \emph{from} specific ability.

  To illustrate, consider the following pattern of reasoning:
  \begin{enumerate}[label=\arabic*., ref=(\arabic*)]
  \item\label{scen:rp:1} I have the general ability to reason with the rules of chess.
  \item\label{scen:rp:2} I received \gsi{} information that if they have the general ability to reason with the rules of chess then they have the ability to demonstrate the existence of some strategy.
  \item\label{scen:rp:3} So, from~\ref{scen:rp:1} and~\ref{scen:rp:2} it follows that I have the ability to demonstrate the existence of some strategy.
  \item\label{scen:rp:4} And, as \aben{the} hold with respect to~\ref{scen:rp:3}, the relevant strategy exists.
  \end{enumerate}
  I reason to (\ref{scen:rp:1} --- \ref{scen:rp:3}) and from (\ref{scen:rp:3} --- \ref{scen:rp:4}) a specific ability.
  The reasoning pattern seems sound.
  And, at no point do I need to witness their general ability to reason with the rules of chess, or the specific application of the general ability to demonstrate the existence of the strategy.
\end{note}

\begin{note}
  Both components are important.
  Focus on \gsi{} will restrict the interpretation of what the agent claims support for.
  And, in turn, what the agent has claimed support for will determine what the agent appeals to when appealing to \aben{the} entailment.\nolinebreak
  \footnote{
    I suspect it may be possible to focus only on \gsi{}.
    As we will see, this is where the key step of the argument takes place.
    However, this is not trivial.
    Would require more focus on how the agent gets to specific from general.
    By splitting in this way, we avoid details.
    Instead, focus on what it is that the agent gets, and then \aben{the} is forced to work with this.
  }
  \gsi{} and \aben{the} combine to provide a (partial) functional characterisation of reasoning with specific ability.
\end{note}

\begin{note}
  Note, however, that there is a distinction between how an agent reasons about ability, and what ability is.
  We are interested in how agent's reason about (specific) ability, and not what makes it true that an agent has a (specific) ability.
  Our focus will shortly turn to how to interpret (specific) ability when appealed to in the type of scenario described.
  We will outline two general schematic interpretations of ability, argue that these are exhaustive, and note how general constraints such as \ESU{} constrain which interpretation is available.
\end{note}

\begin{note}[Scenario proposition]
  For ease of reference, we wrap scenarios involving the limited information as a proposition.
    \begin{restatable}[\eA{0} --- \eA{}]{proposition}{propScenariosExist}\label{prop:SE}
    There are scenarios in which an agent \emph{S} receives \gsi{} information of the form:
    % \mbox{ }\vspace{5pt}
    \begin{center}
      If \emph{S} has a general ability to \(\gamma\), then \emph{S} has a specific ability to \emph{V} that \(\phi\).
    \end{center}
    % \mbox{ }\vspace{5pt}

    \noindent Such that \aben{the} applies to the specific ability to \emph{V} that \(\phi\).

    In turn:
    \begin{enumerate}
    \item \emph{S} may reason from claimed support that they have the general ability to \(\gamma\) in order to claim support for having the specific ability to \emph{V} that \(\phi\). And,
    \item \emph{S} may reason from their claimed support that they have the ability to \emph{V} that \(\phi\) to claim support that \(\phi\) is the case by appealing to \aben{the}.
    \end{enumerate}
    \vspace{-\baselineskip}
  \end{restatable}
\end{note}

\begin{note}[Possible restrictions]
  First, \eA{} holds only that there are cases in which the agent may appeal to ability to obtain support.
  It is therefore consistent with~\eA{} that there are cases in which the details of the cases outlined are satisfied, but where kind of support is unsuitable for certain purposes.
  For example, some witness of ability may be demanded by a third-party.
  In this respect, the content of \eA{} is similar to an analogous claim with respect to memory.
  If an agent remembers proving that \(\phi\), then \(\phi\) is the case.
  Still, one may still request that an agent provides you with a proof of \(\phi\) in order to for you to be satisfied \(\phi\) is the case --- many exams are like this.
  So, that an agent may not always and in any context claim support for \(\phi\) from claimed support for their ability to \emph{V} that \(\phi\) is not an objection to~\eA{}.
\end{note}

\begin{note}
  Second, \eA{} does not require that an agent reason in the way described given \gsi{} and availability of \aben{the}.

  For example, the following statement is an instance of \gsi{}:
  \begin{enumerate}
  \item Any person who has the (general) ability to reason with the rules of chess has the (specific) ability to identify Alekhine's Defense as a fine opening move.
  \end{enumerate}
  The universal quantifier implies that the statement is true with respect to me, among others.
  Still, I am confident that there is at least one other person who has the ability to reason with the rules of chess, and may therefore infer that Alekhine's Defense as a fine opening move without appealing to my own ability.
  Indeed, if I am inclined to doubt my own (general) ability in contrast to a Grandmaster, then I may be more confident that Alekhine's Defense as a fine opening move if I appeal to the existence of a Grandmaster.

  Again, it is consistent with \eA{} that an agent may reason in such a way.
  Still, in defence of \eA{} it is important to note that \gsi{} information may be limited to the agent in question.
  For example, I may have studied your notes on how to play chess and identified a strategy which follows from those notes.
  I have no doubt that you have the ability to identify the same strategy, so when I provide \gsi{} my emphasis is on whether you have the ability to reason with \emph{chess}, rather than some closely related game.

  There are many ways to build context so that an agents is required to reason with \gsi{} and \aben{the} if the agent is to reason with (specific) ability at all, but I doubt these are required.
  The reasoning described by \eA{} (and illustrated above) seems plain and permissible.
\end{note}

\begin{note}
  Finally, \gsi{} and \aben{the} are constraints which do not hold in all cases of reasoning with specific ability.

  For example, one may be told that a gift of a metal detector grants the ability to discover if there is buried treasure in the garden.
  The former does not entail that there is buried treasure in the garden, and testimony or the metal detector may be claimed as support for the ability.

  So, question about whether this really does anything for general understanding of ability.
  \gsi{} and \aben{the} combine to require a particular interpretation.
  However, interpretation with general applicability is not restricted to instances in which it is forced.
  The role of a counterexample is not (typically) to establish that every instance of a theory is mistaken, but to identify a gap.
  And, even if the original theory may be restricted to non-problematic cases, the alternative theory may compete with the original theory.
  So, given that the particular interpretation is required to hold given additional stipulations, interest is in whether it holds without additional stipulations.
\end{note}

\section{Two (schematic) accounts of (specific) ability}
\label{sec:wr-ar}

\begin{note}
  In the previous section we introduced \gsi{0} and \aben{the}.
  In the present section we motivate two interpretations of (specific) ability in the context of reasoning to (specific) ability from \gsi{} and reasoning from (specific) ability with \aben{the}.

  The two interpretations are termed `\AR{}' and `\WR{}' in turn, and are schematic.
  Roughly:
  \AR{} holds that when appealing to (specific) ability an agent appeals to some property or attribute that they have.
  And, by contrast, \WR{} holds that when appealing to (specific) ability an agent appeals to the action that they would perform by witnessing the relevant ability.
  \AR{} and \WR{} are distinguished, then, by whether an agent reasons with a property (\AR{}) or an event (\WR{}).

  To illustrate by analogy, consider a mechanical clock.
  The clock has the property of displaying the correct time, by it is also involved in the event of changing it's configuration as time passes.
  The property that the clock is displaying the correct time is important for determining whether one is late for a meeting.
  By contrast, the event of changing it's configuration as time passes is important for determining when to remove a brewing teabag.
  A meeting starts at a certain point in time, while tea is brewed over a period of time.
  If the clock does not represent the correct time, then three minutes passing will not, in general, help determine whether one is late to the meeting.
  And, whether or not it is 3pm is not, in general, important with respect to whether or not the tea has finished brewing.
  The qualifier `in general' is important.
  Measuring the passage of is useful if I know the length of time before the meeting is due, and the correct time is useful if I know when I started brewing the tea.

  The distinction between \AR{} and \WR{} is similar.
  Both interpretations may be more or less useful in certain circumstances, and interchangeable in others.
  Still, the combination of \gsi{} and \aben{the} identify a pattern of reasoning in which we may elaborate how the relevant interpretation of (specific) ability is important, and in turn broader principles (\ESU{} and, to be introduced below, \nI{}) will constrain whether the interpretations are permissible.
\end{note}

\section{\AR{} and \WR{}}
\label{sec:ar-wr-1}

\begin{note}[\WR{} and \AR{}]
  We term the two schematic interpretations of \aben{the} `\AR{}' and `\WR{}', respectively.
  Brief descriptions from detached perspective.
  Given that the interpretations are schematic, they fall short of a full account of how an agent claims support by \aben{an}.
  However, the arguments to follow are of interest in part because they concern any way in which the schematic interpretations are filled out.
\end{note}

{
  \color{red}
  I should emphasise that here we're interested in reasoning.

  Also, the distinction is important to ensure that the argument's don't depend on a specific reading of ability.
}

\subsection{\AR{}}
\label{sec:ar-1}

\begin{note}
  \begin{restatable}[\AR{}]{definition}{defAttribution}\label{AR:def}
    An agent's reasoning with an instance of \aben{the} by claiming support for \(\phi\) from \emph{S} having ability to \emph{V} that \(\phi\) is an instance of \emph{\AR{}} when the agent holds that:

    \emph{S} has the ability to \emph{V} that \(\phi\)
    \begin{enumerate*}[label=(\textsf{A}\arabic*), ref=(\textsf{A}\arabic*)]
    \item\label{A:s:1} is or reduces to some (possibly complex) property \emph{P} of \emph{S}, and
    \item\label{A:s:2} \emph{P}, or some part of \emph{P}, entails \(\phi\) is the case.\nolinebreak
      \footnote{Intuitively, because the agent could not have \emph{P} without \(\phi\) already being the case.
      The notion of entailment here does not require \(\phi\) is true \emph{because} of \emph{P}.}
    \end{enumerate*}
  \end{restatable}
  
  {
    \color{red}
    \AR{} identifies instances of reasoning in which an agent applies \aben{the} by holding the ability to \emph{V} that \(\phi\) is a property of an agent.\nolinebreak
    \footnote{
      Note, this does not say anything about what the ability to \(\phi\) is.
      Rather, way in which the agent claims support.
    }
    Note, when appealing to \aben{the} an agent need not be aware of what the (potentially complex) property of \emph{S} is.
    Rather, claimed support that \emph{S} has the ability to \emph{V} that \(\phi\) allows the agent to claim support for the existence of some property of \emph{S} which in turn entails \(\phi\).
  }

  Now, generally speaking properties are things which may be predicated or attributed of other things.
  The coffee is hot, I am thirsty, my mouth is sensitive to heat, I am reckless, I am in pain, and so on\dots
  And, properties come cheap.
  For example, the participation of an agent in some event gives rise to a property that may be attributed to the agent.
  Specifically, the property of participating in the event.
  Moments ago I participated in the event of recklessly drinking hot coffee with a mouth that is sensitive to heat.
  Therefore, I have the property of participating in such an event.

  So,~\ref{A:s:1} is trivially true.
  When we speak of an agent having some ability we are predicating or attributing ability to an agent.
  However,~\ref{A:s:2} requires that the property entails \(\phi\) is the case.
  And, it is not clear that an entailment which follows from an event is always reflected in the property of being a participant in the event.
  For example, it seems that I am in pain because I participated in the event of drinking hot coffee, \emph{not} because I have the property of having participated in the event of drinking hot coffee.
  By contrast, that I have the property of having participated in the event of drinking hot coffee entails that I have the property of having participated in the event of drinking something.

  % From~\ref{A:s:2} it must be the case that the relevant property entails \(\phi\).
  % And, from~\ref{A:s:3} the property must not analysed in terms of there being a potential event in which \emph{S} witnesses the act of \emph{V}ing that \(\phi\).
  % This is, from one perspective, an arbitrary restriction.
  % For example, if there is a potential event in which an agent witnesses the act of \emph{V}ing that \(\phi\), then the agent has the property of being a participant of that potential event.
  % From a different perspective,~\ref{A:s:3}

  Roughly, we may expect the property of interest is akin to having a heart, possessing ¥500, being of a certain age, and so on\dots

  {
    \color{red}
    Key idea with \AR{} is that the agent `directly' claims support for a property when using \aben{the}.
  }

  To illustrate \AR{} we focus on the idea of reducing the ability to \emph{V} that \(\phi\) to some (potentially complex) property of \emph{S}.
  Again, when appealing to \aben{the} an agent need not be aware of what the (potentially complex) property of \emph{S} is.
  Rather, these illustrations suggest that such properties exist.

  \begin{illustration}
    Consider the proposition that \emph{S} has the ability to hear that the birds are signing.
    Again, it seems \aben{the} holds, and one may infer that birds are singing.

    So, by \AR{} there is some (complex) property \emph{P} of \emph{S} such that \emph{P}, or some part of \emph{P}, entails the the birds are signing.

    Consider the complex property of a well-functioning auditory system and sufficient proximity to the birds singing.
    The property of having well-functioning auditory system ensures that \emph{S} has the ability to hear nearby noises.
    And, having well-functioning auditory system together sufficient proximity to the birds singing together ensure that \emph{S} has the ability to hear the nearby noise of the birds singing.

    \aben{the} follows from part of this complex property.
    If the agent has the property of being in sufficient proximity to the birds singing, then it follows that there are birds singing.
  \end{illustration}

  \begin{illustration}
    Consider the proposition that the prosecution has the ability to demonstrate that the defendant is guilty.
    Intuitively, \aben{the} holds, as it is not possible to demonstrate the guilt of an innocent person.\nolinebreak
    \footnote{
      It is a different matter to convince a jury of the guilt of an innocent person.
      And, \aben{the} does not seem to hold with respect to the ability to convince a jury that the defendant is guilty.
    }
    By \AR{} there is some (complex) property \emph{P} of the lawyer such that \emph{P}, or some part of \emph{P}, entails the guilt of the defendant.
    Say, the lawyer is in possession of evidence sufficient to establish guilt of the defendant.
    If so, it is a property of the lawyer that they are in possession of such evidence, and by assumption the evidence entails that the defendant is guilty.

    It seems possession of evidence alone may not be sufficient to establish that the lawyer has the ability to prove that the defendant is guilty.
    For, it is plausible that a lawyer may be in possession of evidence that they do not understand.
    However, as our interest is with \aben{the} it is sufficient to observe that the evidence alone entails the guilt of the defendant.
  \end{illustration}

  Again, these illustrations highlight ways in which \emph{S} having the ability to \emph{V} that \(\phi\) may be reduced to some (complex) property of \emph{S}.
  \AR{} does not hold that an agent identifies such a property when claimed support by an instance of \aben{the}.
  Rather, \AR{} holds that the agent reasons with ability as a property of the agent.
  Indeed, while these suggestions reduce ability to complex properties, \AR{} also admits of the possibility that the ability to \emph{V} that \(\phi\) is a basic property which does not admit of further analysis.
  If so, then it seems that \aben{the} must also be taken as basic.\nolinebreak
  \footnote{
    I lack any suggestion for how to understand \AR{} if the property is indeed basic, but there is no need to rule out this option ---  no part of the following arguments depend on whether or how these schemas may be substantiated.
  }
  So, to summarise.
  The distinguishing feature of \AR{} is that there are instances when an agent claims support for \(\phi\) from claimed support that \emph{S} has the ability to \emph{V} that \(\phi\) because the latter ensures that there is some property \emph{P} holds of \emph{S} and \emph{P} entails \(\phi\).
  If the agent has the ability to \emph{V} that \(\phi\), then there may also be some action, \emph{V}ing, that the agent may witness.
  However, as \AR{} appeals to some property, the witnessing event is irrelevant to the way in which the agent claims support for \(\phi\).
\end{note}

\begin{note}
  {
    \color{red}
    Some additional notes on \AR{} that haven't been merged with the above follow.
  }
\end{note}

\begin{note}[Compatibility]
  \AR{} suggests an alternative way to obtain support for the conclusion of reasoning the agent is able to do.
  Specifically, if order for the agent to \emph{have} the attribute of being able to reason to the conclusion, the conclusion of the reasoning must be true.
  The relevant entailment is in part secured by the verb chosen, and in part by what the verb is applied to.
  Here, `demonstrate' is a factive verb, if an agent demonstrates \(\phi\) has value \(v\), then \(\phi\) has value \(v\).
  And, the existence of a chess strategy does not depend on the agent demonstrating that the relevant strategy exists.

  To take another example, you only have the ability to identify a typo on this page if there is a typo on this page.
  So, if I were to provide you with testimony that you have the ability to identify a typo on this page, you may begin searching for the typo, or you may note that there must be a typo in order for me to be in a position to provide you with testimony that you have the ability.
\end{note}

\begin{note}[Sketch of \AR{}]
  \begin{enumerate}[label=(\textsf{A}\arabic*), ref=(\textsf{A}\arabic*)]
  \item\label{AR:Sketch:1} I have the attribute of being able to \emph{V} that \(\phi\).
  \item\label{AR:Sketch:2} In order to have the attribute of being able to \emph{V} that \(\phi\), \(\phi\) must be the case independent of whether or not I witness the ability.
  \item\label{AR:Sketch:3} \(\phi\) is the case.
  \end{enumerate}

  To keep things simple, we will refer to the principle behind the pattern sketched as \AR{}.
  And agent may bundle~\ref{AR:Sketch:1} and~\ref{AR:Sketch:3} into a conditional, and avoid instantiating the reasoning pattern, but so long as the conditional is (implicitly) held on the basis of the intermediate premise~\ref{AR:Sketch:2}, we take use of such a conditional to be an instance of \AR{}.
\end{note}


\subsection{\WR{}}
\label{sec:wr-1}

\begin{note}[\WR{} def.]
  {
    \color{red}
    Include: observation that the entailment may come from some property of the agent.
    The point of \WR{} is that the agent claims support for details of the event.
  }

  We now turn to \WR{}.
  \begin{restatable}[\WR{}]{definition}{defWitnessing}\label{WR:def}
        An agent's reasoning with an instance of \aben{the} by claiming support for \(\phi\) from \emph{S} having ability to \emph{V} that \(\phi\) is an instance of \emph{\WR{}} when the agent holds that:
    \begin{enumerate}
    \item\label{WR:def:1} \emph{S} has the ability to \emph{V} that \(\phi\) \emph{if and only if} there is a potential event in which \emph{S} witnesses the act of \emph{V}ing that \(\phi\).
    \item\label{WR:def:2} Claim support for event or details of event.
    \item\label{WR:def:3} Details of the event in which \emph{S} witnesses the act of \emph{V}ing that \(\phi\), or part of the event, entails \(\phi\) is the case.\nolinebreak
      \footnote{Again, intuitively, because there could not be a potential event in which \emph{S} witnesses the act of \emph{V}ing that \(\phi\) without \(\phi\) already being the case.
      The notion of entailment here does not require \(\phi\) is true \emph{because} there is some potential event of the relevant kind.}
    \end{enumerate}
  \end{restatable}

  {
    \color{red}
    ~\textcite{Rebuschi:2011ub} talk about \emph{de objecto} attitudes.
    This might be helpful given that the events are potential.
  }

  {
    \color{red}
    Key idea with \WR{} is that the agent appeals to certain things which follow from the event being witnessed.
    Whereas, \AR{} appeals to certain things which must be the case in order for the event to be witnessed.
  }

  {
    \color{red}
    Difference between the existence of an event (~\ref{WR:def:1}) and details of the event (~\ref{WR:def:2}).
    To clarify.
    \(\exists e(V(e) \land \text{agent} = \emph{S} \dots)\).
    \(\phi\) follows.
    However, there are two ways to think about this.
    First, the existential, second the event.
    \emph{De dicto} and \emph{de re}.
    \WR{} is \emph{de re}.

    Consider existential of individuals.
  }

  {
    \color{green}
    Before going into the details, it'll be helpful to highlight the big idea, especially with respect to how things (will) work out with the `master property' from \AR{}.
  }

  \WR{} identifies instances of reasoning in which an agent applies \aben{the} by holding that \emph{S} having the ability to \emph{V} that \(\phi\) ensures there is a possible event in which \emph{S} \emph{V}s that \(\phi\).
  And, in turn, there is a possible event in which \emph{S} \emph{V}s that \(\phi\) entails \(\phi\) is the case.
  In contrast to \AR{}, when an agent claims support as an instance of \WR{} an agent reasons about what must be the case in order for \emph{S} to witness some ability, rather than what must be the case in order for \emph{S} to have the property of possessing some ability.


  We use the term `potential' in place of `possible' when describing the relevant event to highlight that the existence of the event is tied to an ability attribution.
  One may hold that a possible event is any event which is not impossible, and hence it is possible for an arbitrary agent to prove Fermat's Last Theorem.
  Yet, it seems most agent's lack the ability to prove Fermat's Last Theorem, and so `potential' serves to restrict quantifier over events which an agent has the ability to witness --- however the details of that quantification are resolved.

  {
    \color{red}
    \WR{} is more complex than \AR{}.
    There is some action that \emph{S} may witness.
    And, understand what the result of that action is.
    So, we have something akin to a counterfactual.
    However, the counterfactual only relies on witnessing.
    Further, particular status of \(\phi\).
    Hence, as witnessing is the only issue, \(\phi\) is the case.

    Third, regardless.
    \(\phi\) holds regardless, but it does not follow from this that if the agent reasons via \WR{} then support claimed for \(\phi\) would be independent of ability information.
    The agent must recognise \(\phi\) must be the case regardless, but this doesn't require that the agent has any way of reasoning to \(\phi\) other than by witnessing their ability.
    The point is clearer when considering witnessed instances of reasoning.
    \emph{X} testified that \emph{p}.
    Claim support for \emph{p}.
    \emph{p} is not the case because \emph{X} testified that \emph{p}, though my only path to claim support is by appeal to the testimony of \emph{X}.
  }
  To illustrate.

  \begin{illustration}
    I have the ability to calculate \(243 \div 3 = 82\).
    Pen and paper to hand, etc.\
    Result of this will be a calculation that \(243 \div 3 = 82\).
    However, my calculation is irrelevant to whether it is the case that \(243 \div 3 = 82\).
    Hence, it follows that \(243 \div 3 = 82\).
  \end{illustration}

  \begin{illustration}
    Ability to discover that the ball is under the left cup.
    Raise the left cup, and identify the ball.
    Whether or not the ball is under the left cup is independent of this sequence of actions, and therefore it follows that the ball is under the left cup.
  \end{illustration}

  Compare to cases where only gets the counterfactual.
  I have the ability to make it so that the heating is turned out.
  Plausibly, the heating is not on, and depends on witnessing the action of `making it so'.
\end{note}

\begin{note}[`Available resources']
  Delicate.
  Focus is on the witnessing event.
  However, mere possibility isn't sufficient for \aben{the}.
  So, some restriction.
  That is, an account of what makes the witnessing event a \emph{potential} event rather than a \emph{possible} event.
  One way to express this idea is that included in appeal to potential witnessing event is that sufficient resources are available.
  Here, the idea is that nothing further is required for the event to take place.

  This redescription falls short of an analysis as we've shifted the work from `potential' to `available'.
  Still, room for an analogy.
  Consider running a 5K.
  Here, going to require a whole bunch of energy.
  The agent does not `have' the energy.
  However, resources to generate energy.
  Fat reserves, muscle density, and so on.
  In this sense, sufficient resources are available, but not something the agent has.

  \AR{}, whatever it is that generates the sufficient resources.
  \WR{}, the result of having generated the sufficient resources.

  So, the difference between \AR{} and \WR{} isn't necessarily with what the two interpretations reduce down to, but is rather a difference with respect to what the interpretations focus on.
  From \AR{}, the stuff that's true right now, the generator, does the work.
  From \WR{}, it's what will be generated.

  There's still an important difference, though.
  Our interest is in reasoning.
  We are interested in what the agent appeals to.

  Key difference.
  \AR{}, that there is stuff the agent has which will generate.
  \WR{}, that what is generated from the stuff the agent has will do the work.

  The impact of this distinction will be expanded up with respect to \gsi{}.
\end{note}

\subsubsection{\AR{} and \WR{}}

\begin{note}
  Key idea is that \AR{} and \WR{} are different perspectives on the same thing.

  Switching between ability and potential events.
  This is not important, two ways of describing the same thing.

  The ability to \emph{V} that \(\phi\) is equivalent to there being a potential event in which the agent \emph{V}s that \(\phi\).
  For, if there is no such potential event, then the agent does not have the ability to \emph{V} that \(\phi\).
  Conversely, if there is a potential event in which the agent \emph{V}s that \(\phi\), then the agent has the ability to \emph{V} that \(\phi\).
\end{note}

\subsection{\AR{} and \WR{} are exhaustive}
\label{sec:ability-exhaustive}

\begin{note}[Exhaustive]
  \begin{restatable}[]{proposition}{propAbilityExuastive}
    \label{prop:WR-and-AR-exhaustive}
    \label{either-AR-or-WR}
    Any interpretations of an agent's (specific) ability to \emph{V} that \(\phi\) (for which \aben{the} holds) conforms to either:
    \begin{enumerate}
    \item
      \AR{}: It is a property of the agent that they are able to \emph{V} that \(\phi\).
    \item
      \WR{}: There is a potential witnessing event in which the agent \emph{V}s that \(\phi\).
    \end{enumerate}
    \vspace{-\baselineskip}
  \end{restatable}
\end{note}

\begin{note}
  The distinction between \AR{} and \WR{} sets up two (schematic) ways in which agent an agent may claim support given an instance of \aben{the}.
  We now argue that these two (schematic) methods are exhaustive.
\end{note}

\begin{note}[Old arguments]
  Remaining issue is details of the schemas.
  These talk about more than mere reference.
  \AR{}, agent, and \WR{} the result of the witnessing event.
  In turn, these are harmless and the only plausible option.

  \AR{} is simple.
  State of affairs, but as the agent is involved, then it is natural to attribute to the agent.
  Implausible that it's some event.

  \WR{} focuses attention to culmination of event.
  However, need culmination.
  Quirk of English that may `use' relevant verbs in this way.
  Imperfective paradox.
  May consider this a state, but only in the sense that it is a state bought about by some event.
  Focus on event, but given culmination, consider this a state.
  Still, state of culminated event.
  Possible that this is simply a state in which the agent has some appropriate relation.
  Problem is that an ability is the ability to do some thing.
  If abstract away from the act, then it's not clear how to understand conditions as identifying ability.
\end{note}

\section{Recap}
\label{sec:recap-reasoning}

\begin{note}
  Second is something like evidence of evidence is evidence.

  Here, the important difference is that the agent only needs to appeal to general ability.
  And, they've claimed support for this.

  The point is that it's not clear the agent is required to do anything too much with the specific ability.
\end{note}

\begin{note}[Summarising]
  Above we introduced \gsi{}.
  Limited information of the form `If \emph{S} has a (general) ability to \(\gamma\), then \emph{S} has a (specific) ability to \emph{V} that \(\phi\) (as an instance of the general ability).'
  We then noted that certain instances of the (specific) ability to \emph{V} that \(\phi\) entail \(\phi\) is the case.
  Two interpretations of \aben{the}, \AR{} and \WR{}.

  Our focus now turns back to \gsi{}.
  For those instances of \gsi{} when \aben{the} holds, the interpretations \AR{} and \WR{} detail what the agent obtains by reasoning from general to specific ability.
  In other words, \emph{what} the agent is claiming support for.

  As noted, using a conditional such as \gsi{} is not automatic.
  The informer has not provided the agent with any additional way to claim support that the agent has the general ability.
  Rather, outlined something that follows \emph{if} the agent has the general ability.

  So, it is up to the agent to resolve in either way.
  If the agent wants to use the information, then the agent needs to reason from general to specific.
  The issue is that without any additional reasoning, it seems there's no clear way to determine which way the agent should go.
  Here is where the distinction between \AR{} and \WR{} is important.
  Interpretation of specific ability informs how the agent move from general to specific.

  Following two propositions outline combination.
  {
    \color{red}
    The key thing here is about claiming that one has a specific ability.
  }
\end{note}

\begin{note}[\gsi{}++]
  First, \gsi{} applied to \AR{}
  \begin{restatable}[\textsf{|gs-I\space·\space H|}]{idea}{ideaCSbyAR}\label{idea:CS-by-AR}
    % In order for \emph{S} to have the (specific) ability to \emph{V} that \(\phi\) for which \aben{the} holds, claimed support for general and claimed support for \gsi{} are sufficient to claim support that \emph{S} has the property of being able to \emph{V} that \(\phi\).
    Suppose an agent has:
    \begin{enumerate}
    \item Claimed support for some general ability \(\gamma\).
    \item Claimed support that if they have the general ability \(\gamma\) then they have some specific ability to \emph{V} that \(\phi\) (for which \aben{the} holds).
    \end{enumerate}
    Then:
    \begin{enumerate}[resume]
    \item \emph{S} may claim support for having the specific ability \(\sigma\) by reasoning that they have the property of being able to \emph{V} that \(\phi\).
    \end{enumerate}
    \vspace{-\baselineskip}
  \end{restatable}

  Second, \gsi{} applied to \WR{}

  \begin{restatable}[\textsf{|gs-I\space·\space W|}]{idea}{ideaCSbyWR}\label{idea:CS-by-WR}\label{W:s}
    % In order for \emph{S} to have the (specific) ability to \emph{V} that \(\phi\) for which \aben{the} holds, claimed support for general and claimed support for \gsi{} are sufficient to claim support that there is a potential witnessing event in which \emph{S} \emph{V}s that \(\phi\).
    Suppose an agent has claimed support for some general ability \(\gamma\) and has claimed support that if they have the general ability \(\gamma\) then they have some specific ability to \emph{V} that \(\phi\) for which \aben{the} holds.
    Then, an agent may claim support for having the specific ability \(\sigma\) by reasoning that there is a potential witnessing event in which \emph{S} \emph{V}s that \(\phi\).
  \end{restatable}
\end{note}

\begin{note}[Alternatives]
  Appeal to premises and steps is not required by either \AR{} or \WR{}.
  However, most plausible account of what is going on.

  Explored some alternatives for \AR{}, but unclear what is of importance other than reasoning, and hence premises and steps.
  And, in this respect, basic \AR{} seems like a dead end.
  Premises and steps allow the agent to claim support in the same way as they would allow the agent to claim support when used in reasoning.
  It's not at all clear to me that basic \AR{} makes sense from this perspective.
\end{note}

\begin{note}[Limitation of intuition]
  Focused on idea that claiming support in same way as reasoning.

  This is not to imply equivalence of claimed support.

  Said too little about claimed support to make any strong remarks about equivalence.
  Still, intuitive that additional way of being \mom{}.
  For, haven't done the reasoning, so \mom{} about this.
  Not the case if the agent has done the reasoning.
\end{note}



%%% Local Variables:
%%% mode: latex
%%% TeX-master: "master"
%%% End:


\part{Tension}
\label{part:tension}

\chapter{\ESU{} and ability}
\label{sec:first-conditional}

\begin{note}[Summary]
  In this section we argue that \ESU{} constrains what an agent may claim support for when reasoning from general to specific ability.

  In short, \ESU{} rules out claiming support by \adB{}.
\end{note}

\begin{note}
  Two ways corresponding to two sides of (specific) ability
  First, with respect to \aben{the}: appealing to (specific) ability.
  Second, with respect to \gsi{}: establishing (specific) ability from (general) ability.
\end{note}

\begin{note}[Expand: \gsi{}]
  Start with \gsi{}.

  Agent is claiming support for specific ability.
  Hence, claiming support that there is a potential event in which they \emph{V} that \(\phi\).
  Expanding potential event, claiming support that sufficient resources are available.
  Note, the agent may not (merely) \emph{expect} that sufficient resources are available, as availability of resources is part of claim for potential event.
  Rather, the agent may qexpect that there are no defeaters to claim that resources are available.

  To illustrate.
  Suppose I claim support that I know the train will be late.
  It's not (merely) that I expect that the train will be late.
  In order to claim support, some considerations sufficient to establish that the train will be late, and that there are no defeaters for these considerations.
  Expect would be absence of materia that train is on time.
  But absence alone doesn't push either way.\nolinebreak
  \footnote{
    Absence may be materia, though.
    For example, at least five minutes before train will arrive there is a message broadcast at the station.
    We are at the station, and it is a three minutes before the train is scheduled to arrive.
  }

  Task is to account for why an agent may claim support for availability of sufficient resources.
  In rough outline, answer is simple.
  Claimed support for general ability.
  Specific ability to \emph{V} that \(\phi\) is an `instance' of the general ability.
  So, given context, general ability supplements sufficient additional premises and steps of reasoning.

  However, without witnessing specific ability, agent is not aware of which additional premises and steps of reasoning are used.
\end{note}

\begin{note}[Moving to incompatibility]
  Incompatibly with \ESU{} will be from common point of appeal to sufficient resources.
  To this we now turn.
\end{note}

\section{Constraints on reasoning with \gsi{} given \ESU{}}
\label{sec:incomp-wr-ura}

\begin{note}[Argument outline]
  \color{red}
  There are two issues.
  \begin{itemize}
  \item \ESU{} means that the agent may not `directly' establish the existence of a witnessing event.
    For, in order to do so, the agent would need to claim support that the conclusion follows from some collection of premises and steps of reasoning.
    However, as the agent does not witness, then this isn't possible given \ESU{}.
    \begin{itemize}
    \item The objection here is that the agent doesn't necessarily need to go directly.
      It's possible that the agent claim support for some property, hence gets specific ability, and then reasons that this means that there's a witnessing event.
      So, to the extent that \WR{} needs first the existence of a witnessing event, \ESU{} might be okay.
    \end{itemize}
  \item Second, the agent can't reason with the details of the witnessing event.
    This then blocks \WR{}, and does so conclusively.
    For, the agent is not permitted to appeal to a relation of support between premises and conclusion.
    The agent is only permitted to appeal to the existence of an event that would establish such a relation.
    So this is the main objection.
  \end{itemize}
\end{note}

Key proposition of this section.

\begin{note}[Proposition]
    \begin{restatable}[\ESU{} and \adB{}]{proposition}{propNoESUandADB}\label{mcA:WR-and-denied-claim}
    \emph{If} \ESU{} is true \emph{then} no claiming support by \adB{} with respect to \ARD{} or \WR{}.
  \end{restatable}
\end{note}

\begin{note}
  Argument is straightforward.
  \adB{} then claiming support by property of witnessing event.
  But, agent has not used those things in reasoning.
\end{note}

\begin{note}
  So, the key thing with this proposition is that in cases where an agent reasons with \gsi{0}, the agent claims support for a property.
  Hence, if an agent reasons from specific ability via \aben{the}, then must be an instance of \AR{}.

  Now, \autoref{mcA:WR-and-denied-claim} doesn't say that \ESU{} and \WR{} are incompatible in general.
  We'll see this in the argument for~\autoref{mcA:WR-and-denied-claim}.

  The argument that \WR{} is an incorrect interpretation of (specific) abilities of the form \emph{S} has the ability to \emph{V} that \(\phi\) (for which \aben{the} entailment holds) has two components.
  First, difficulty establishing by \gsi{}.
  Second, rules out \aben{the}.

  Difficulties with \gsi{}.
  Result of claiming support by \gsi{} is that agent claims support for specific.
  And, given \WR{}, this involves claiming support that some sufficient collection of premises and steps of reasoning are available to the agent.
  Given \ESU{}, the agent is required to use these steps and premises in order to appeal.
  Therefore, \ESU{} requires a partial witnessing event.
  Partial only, as the depending on how premises and steps are understood, certain premises or steps may be reused, and a single use may be sufficient for \ESU{}.

  The issue is strengthened when turning to \aben{the}.
  For, the conclusion is that \(\phi\) is the case.
  And, a partial witnessing event does not establish \(\phi\) is the case.
\end{note}

\section{Summary}
\label{sec:uRa-and-wr-summary}

\begin{note}[Table]
  \begin{figure}[H]
    \centering
    \saMtxRuledOutESU{}
    \repeatCaptionPrime{fig:saMtxRuledOut}{Distinction matrix}
  \end{figure}
\end{note}

%%% Local Variables:
%%% mode: latex
%%% TeX-master: "master"
%%% End:

\chapter{\ideaCS{} and ability}
\label{sec:second-conditional}
\label{sec:LCS-applied}

\begin{note}
  So, two main tasks.
  \begin{enumerate}
  \item Why tension with \adA{}.
  \item Why tension does not arise for \adB{}.
  \end{enumerate}
\end{note}

\begin{note}[Finding tension, still]
  \color{red}
  We have outlined a type of scenario built primarily on an agent receiving information that the agent has some specific ability so long as the agent has some general ability.
  The agent has support for having the general ability, but there are two ways in which the agent's support for having the general ability may be used to establish support for {\color{red} the result of having the specific ability} --- \AR{} and \WR{}.

  The previous section argued that~\ESU{} constrains how an agent may use the received information.
  If an agent is required to traces support from premises to conclusion through reasoning, then an agent may not appeal to the support for the premises and steps of reasoning that the agent would use to witness the specific ability.

  The (initial) plausibility of~\ESU{}, then, suggests that the agent may only establish support for having the {\color{red} result of the specific ability} from the support they have for the general ability by \AR{}:
  The support the agent has for the general ability is support that it is true that the agent has the general ability.
  In turn, given the information received it is true that the agent has the specific ability, and it is only possible for the agent to have the specific ability if the result of witnessing the specific ability is true.

  The argument of this section is that the sketch of \AR{} given conflicts with a different, but equally plausible, premise.
  The premise concerns the way in which the agent obtains support for having the specific ability from the support for the general ability.
  We state conditional, the proceed to the premise.
  The initial statement of the premise is abstract and after providing a handful of clarifications we then link the premise to the type of scenario of interest.
\end{note}

\section{\LCS{} and ability}
\label{sec:ni-ability}

\begin{note}[Notes]
  The key here is \gsi{} and \adA{}.
  This is going to lead to a problem, given \LCS{}.

\end{note}

\begin{note}
  So, the general plan is to observe that if \ESU{} holds, then need to appeal to having the ability.
  And, this involves appeal to moving to having the ability from prior reasoning.
  Without the second step here, the argument breaks down.
  However, it seems unavoidable.
  There's no other reasoning possible for the agent.
  Ability seems required.
\end{note}

\subsection{\LCS{}}
\label{sec:ability-and-lcs}

\begin{note}
  \color{red}
  \begin{itemize}
  \item So, get \requ{}.
  \item This much is fine.
  \item Question is whether it is possible for the agent to do anything about the \requ{}.
  \item Arguably no.
  \item One big idea is that the agent has claimed support for general ability on some `core' such that this provides strong indication that general ability extends to all cases.
  \item This much is fine, then problem, however, is that we've still got specific information about what is outside of the core.
  \item So, the probability of any possible defeater is super low.
  \item And, this is enough to hold onto claimed support regardless.
  \item Because, I considered the possibility of those unknown defeaters, and still gathered enough to claim support regardless.
  \item So, this seems to allow claiming support for specific from general.
  \item But at the same time this seems bad.
  \item It has the same feel as the problems with \requ{1}.
  \item Because, all the stuff gathered was without recognition of this possibility.
  \end{itemize}

  \begin{itemize}
  \item I mean, problem is before, got the probability low as unrecognised.
  \item Question is whether this remains the case now recognised.
  \item Well, nothing really follows from probability being low.
  \item In a sense, this should already be the case.
  \item The issue isn't that these possible defeaters a \emph{likely}.
  \item The issue is that the agent should think that support holds regardless of whether they hold.
  \item ``Category mistake''
  \end{itemize}

  Okay, so this kind of works against low probability.
  Hence, argument here is that there's no way to get rid of this \requ{} if agent only relies on claimed support for general ability.
  Of course think it's unlikely, but the worry is not that the defeater is there, rather than it's not clear how the evidence goes against the defeater.

  The redux, then, is that this idea of a `core' doesn't really rely on the probability idea.
  But then this just goes against the initial assumption.
  Of course, this is kind of what \citeauthor{Pryor:2000tl} does.
  However, this seems to conflict with the idea of claimed support.
  If we've got some kind of dogmatist position, then it doesn't seem that the possibility of being \mom{} is such an issue.
  Indeed, the problem here is how to make something like this consistent with that assumption.
\end{note}

\subsection{\adA{}}
\label{sec:lcs-and-adA}

\begin{note}
  Observed \LCS{}.
  The important thing here is that \FCS{}.

  So, goal here is to argue that the additional constraints of \FCS{} also hold with respect to ability.
\end{note}

\subsubsection{\gsi{}, \adA{}, and \FCS{}}
\label{sec:ni-ability:adA}

\begin{note}
  The idea here is simple.
  \begin{itemize}
  \item The thing with \adA{} is that the agent appeals to ability `as a whole', so to speak.
  \item This gives us the relevant \(\phi\) instance for \nI{}.
  \item And, \aben{the} is such that the agent needs to appeal to ability, rather than mere claimed support.
  \item Then, the key focus is {\color{red} inclusion}.
  \end{itemize}
\end{note}

\begin{note}
  Now, the basic observation is that with \adA{} one moves from general to specific, and from ability to proposition.

  Here, only really interested in \aben{the}.
  However, as we've observed, goes from either general or specific.

  I mean, the basic observation is that the agent doesn't reason about general or specific ability.
  So, reasoning follows from it being the case that agent has attribute, or that there is a witnessing event.

  Ohhhh, the point is that the agent is relying on these conditionals.
  First, to move from general to specific.
  Second, to move from ability to proposition.

  With respect to these conditionals, it's \adA{}, so there's no way to move between these things without using the value of one thing to constrain the value of the other.

  So, instance of \adA{}, generally.
  And, because of the construction of the scenarios, the case of \adA{} we're interested involves appeal to the value of the proposition.
\end{note}

\subsubsection{\nI{} and \adB{} (excluding \ARB{})}
\label{sec:ni-ability:adB}

\begin{note}
  Key observation is that \adB{} doesn't go by value.

  However, there is a problem.

  For, it may seems as though the agent \emph{does} go by value because they require the premises, etc.

  This is clearest with the idea that:
  \begin{itemize}
  \item If \(\phi\) isn't the case, then some premise or step isn't part of ability.
  \end{itemize}
  Question about whether this gets a violation of \ideaCS{}.

  But, point is that agent at present is okay with claiming support that the reference resolves.

  So, this really isn't that problematic.

  Obviously it could break down.

  The point is that the agent at present outlines claim to support even if \mom{}.
\end{note}

\subsubsection{\requ{3}}
\label{sec:requ1}

\begin{note}
  Objection here is that we've identified failure due to \requ{1}, roughly.
  And, ability to claim support, and this is going to involve some \requ{1}.
  However, no use so no reasoning about.

  Yes, this is somewhat difficult.

  Unsatisfactory response would be to observe that \requ{} is only defined with respect to reasoning performed.
  Unsatisfactory because only necessary conditions, and plausible that there are additional necessary conditions.

  Possible line of response is appeal to ability.
  However, this will lead to an instance of \nI{} all over again.

  Instead, witnessing ability is itself sufficient for claiming support.
  And, as would amount to claiming support, this will involve reasoning about \requ{}.
  Important, \EAS{} does not necessarily hold for anything weaker.
  \nI{} has been stated for reasoning quite broadly, so problem if the agent appeals to claiming support, or any other reasoning with interdependence.
  However, witnessing is not appealing in this way.
\end{note}

\subsubsection{Appeal to premises and steps requires appeal to ability}

\begin{note}
  It's true that the combination implies the ability, and so the combination seems to lead to the same problem.
  We get \(\psi\) as an \requ{} of combining all of the premises and steps.

  However, what we're relying on is appeal to the individual components.
  The thing here is that it seems fine for the agent to witness.
  This doesn't block claiming support.

  Hence, if this is the case then it can't be that the problem is simply what follows from the combination.
  Rather, it must be something about not witnessing.
  However, this returns us to \ESU{}.
  This is the very intuition that we're arguing against.
  Hence, the question is whether this really is something that is the case.
\end{note}

%%% Local Variables:
%%% mode: latex
%%% TeX-master: "master"
%%% End:

\include{TensionEstablished}

\appendix

\chapter{Tension sensed}

\begin{note}
  Three things
  \begin{enumerate}
  \item \ideaCS{}
  \item \ESU{}
  \item It is possible to claim support for general ability without witnessing each specific instance of the general ability.
  \end{enumerate}

  Develop tension with these three things, roughly.

  \begin{itemize}
  \item So, \ideaCS{} really does the work.
    This sets up two instance of a \requ{}.
    In the simple case, clearly need to deal with specifics before getting general.
    In the more difficult case, where not appealing to general, then always something to check.
  \end{itemize}
\end{note}

\begin{note}
  Now, developing the tension is one thing.
  We don't really require much more of an understanding of claiming support, reasoning, and ability than what has been given.

  However, our goal is not only to establish tension.
  Rather, motivate a way out by rejecting \ESU{}.
  To do this, some more things about claiming support and ability need to be said.
  In particular, how reasoning may avoid \ESU{}, and specifically with respect to ability.

  For this reason, we will limit tension to a brief sketch for now.
  The full argument will be given in {\color{red} ???}.
  First, however, we will go to ability and reasoning.

  Though not needed for tension, needed for way out.
  Hence, rather than tension then complexities, complexities then tension.
  For, then, don't need to worry about re-verifying tension in light of complexities.
\end{note}

\begin{note}
  So, these three ideas are going to be in tension.
  Motivated \ideaCS{} and \ESU{}.
  Ability, have not yet considered.
  Indeed, part of what's interesting here is that we don't need a detailed account of ability.

  Still, interest is in giving up \ESU{}, hence do need some account of ability to retain the possibility.
  That is, it can't be that reasoning with ability is an instance of \ESU{}.
  
\end{note}

\begin{note}
  \color{red}
  New plan.

  The problem, really, is even getting to general ability.
  For, there are always going to be checks, in the form of specific abilities.
  If you want to use general to conclude specific, then the problem is getting to general first.

  Here, this is no doubt a non-deductive step.
  However, it is a completely reasonable non-deductive step.

  There are plenty of examples of this.
  Logic, solving problems in basic textbooks.
  Reading, going through YA books.
  Chess, solving basic problems.
  And so on.

  The problem is making this leap, so to speak.

  Now, there are two perspectives on general ability.
  First, is this property.
  Second, is witnessing.

  With property, there is trouble.
  For, with property, specific is going to always be a \requ{}.

  However, with witnessing, the idea here is that you get confident enough with premises and steps that you apply these to the relevant specific cases.

  What matters, then, is concluding that the basics are good enough.
  Then, this extends to all the specifics.

  So, tension is arrived at by focusing on claiming support for having the general ability.
  The two pressures are
  \begin{enumerate}
  \item Via \ideaCS{}, no \requ{1}.
  \item Via \ESU{}, that the relevant premises are `used' --- for any premise, the agent reasons from that premise.
  \end{enumerate}
  Here's how they work.

  First, going to general.
  So, it's going to be the case that we get something stronger.
  Hence, there's all these specific instances of the general ability.
  And, as these are weaker, these function as \requ{1} (specifically, \crequ{1}).
  Hence, by \ideaCS{}, the agent may not claim support for having the general ability given presence of \requ{1}.

  Hence, the general ability won't work to get the specific instances.

  In other words, in order for the agent to conclude that they have the general ability, the agent must have already concluded that they have the specific instance of the general ability.

  There is always some antecedent check.

  Further, there are \emph{lots} of antecedent checks.

  Now, by \ESU{}, if the agent appeals to some premise, then the agent must reason from that premise.

  So, in order to conclude for all specific instances of the general ability, the agent must have some premise(s) that they witness reasoning from.

  Now, what's tempting here is a uniqueness assumption.
  There's a single premise that works for all cases.
  But I don't think I need that.
  
\end{note}

\begin{note}
  So, set up is lots of experience.
  Very good at propositional logic.

  Now, novel problem (of which there are countably many).
  Here, this is a \requ{}.
  Why is it the case that I wouldn't arrive at some other conclusion, or fail.
  Well, now, this seems quite obvious.

  However, \ESU{} got to witness reasoning from the relevant premise.
  Yet, the general ability as a premise won't do.
  Because, the specific ability is a \requ{}.
\end{note}

\begin{note}
  Here, to include how the argument is going to work.
  That is, the distinction matrix, and what this will amount to.

  So, matrix.

  For this we need three two things.
  Cases.
  Ability.
  Different types of reasoning.

  So, develop these, and then work through the problems.
\end{note}

\begin{note}
  \begin{figure}[H]
    \centering
    \saMtxInterpreted{}
    \caption{Distinction matrix with \aben{the}}
    \label{fig:saMtxInterpreted}
  \end{figure}
\end{note}

\begin{note}[Matrix, ruled out]
  \begin{figure}[H]
    \centering
    \saMtxRuledOut{}
    \caption{Distinction matrix}
    \label{fig:saMtxRuledOut}
  \end{figure}
\end{note}

%%% Local Variables:
%%% mode: latex
%%% TeX-master: "master"
%%% End:

% \part{Supplements}
% \label{part:supplements}

% \include{Supplement}

\part{Temporary}

\chapter{Notes}
\label{cha:notes}

\paragraph{Reflection}

\begin{note}[Reflection]
  \begin{quote}
    Reflection states that agents should treat their future selves as experts or, roughly, that an agent’s current credence in any proposition A should equal his or her expected future credence in A.%
    \mbox{ }\hfill\mbox{(\citeyear[59]{Briggs:2009up})}
  \end{quote}
\end{note}

\begin{note}[Difference to reflection]
  Key difference is that in these cases, there's no guarantee that the agent will go through with ability.
  So, it's not necessarily a future self of the agent.
  Though, that's only on a quick surface reading of reflection.

  This is somewhat delicate.
  For, reflection has some strong background assumptions.
  Problem with ability is that agent might witness ability.
  With reflection, we don't consider restrictions on the reasoning the agent would do.

  Now, weakening reflection is difficult.

  One the one hand, can consider all evidence that the agent would reason through.
  If so, then it looks as though ability is going to fall within the scope.
  Problem here, however, because the argument for reflection is in terms of coherence.
  And, it's not clear how to apply conditionalisation to boundedness.
  Dutch books are about coherent credence functions.
\end{note}

\paragraph{\zS{}: Probability and norms}

\begin{note}
  The set-up here is whether it makes sense to ask a slightly weaker question than \qzS{}.
  For example, it is likely, or would the agent be violating a norm.
\end{note}

\begin{note}
  Might think that whether the agent would fail to conclude isn't really the issue.
  Instead, from present epistemic state the agent consider it sufficiently likely that the would conclude.

  Here, in particular, we don't run into the \requ{} problem, because it might be the case that the agent would fail.
  So, actually doing the reasoning doesn't work to check this.

  Though, I don't think this is right.
  For, it seems that in the lost keys type of case, it really is the possibility of coming to a distinct conclusion.

  Really, the key idea is that we're dealing with \emph{concluding}.
  This, in turn, places a consistency constraint.
  Now concluding \(\pv{\phi}{v}\) and \(\pv{\phi}{\overline{v}}\).
  This doesn't admit of coming to a different conclusion.

  This then applies no matter whether the value \(v\) concerns the proposition being true, or the proposition being sufficiently likely.

  Basic constraint on concluding, and hence what motivates the worry about reasoning to a different conclusion.
\end{note}

\begin{note}
  Press this objection further.
  As the agent has not done the reasoning, then even if \fc{0}, there's still a chance.

  Well, the key response to this is the way I end up understanding these core cases.
  It's ability.
  However, press the issue.
  Well, look, if the understanding of ability is right, then the idea being pressed comes down to the agent revising their present epistemic state.
  And, \emph{this} is compatible with everything said.
  The agent doesn't have the ability.
  Oh no!
\end{note}

\begin{note}
  A different variation on the same idea is that the key parts of \zS{} just express a norm.
  So, in ideal cases, it will be the case that the agent would not conclude, but in standard cases, it is permissible for the agent to fall short of this norm.

  Parallel, norm of knowledge for assertion.
  Something like this.
  Clearly violated, but this doesn't prevent the norm being in effect.

  Norm of only concluding when would not conclude otherwise, this then is in effect, but doesn't prevent an agent from concluding.

  But this seems strange, and stranger than probability.
  For, in these cases, determine whether failing to live up to the norm.

  I think, if there is motivation for \zS{}, then it suggests that this idea doesn't really work.

  And, at least in the case of ability, plausibly satisfy both this and original, stronger, \qzS{}.
\end{note}



%%% Local Variables:
%%% mode: latex
%%% TeX-master: "master"
%%% End:


\printbibliography

% \part{Things}

% \chapter{Ideas, etc\dots restated}
\label{cha:restatable}

\subsection{Epistemic states}

\defEState*

\subsection{Claiming support}

\paragraph{Basic assumptions}

\assuCSVP*

\assuCSRR*

\assuIndicate*

Where:

\defIndicate*

\paragraph{Ideas}

\iCSA*

Where:

\defSink*

\ideaEIS*

\paragraph{Core assumption}

\subparagraph{Background for the core assumption: \requ{1}}

\ideaRequisite*

\defResult*

\defRequisite*

\defRequisiteP*

\defRequisiteC*

\defRequisiteCP*

\subparagraph{The core assumption}

\assuCSRReq*


\subsection{Claiming support and `use'}
\label{sec:claiming-support-use}

\paragraph{Basic idea}

\ideaUSE*

\paragraph*{Target}

\targetESU*

\paragraph*{Goal}

\goalEAS*

\thoughtEASw*


\hozline

\paragraph{???}

\ideaCSbyAR*

\ideaCSbyWR*

\subsection*{Definitions}

\defMoM*

\defGSI*

\defDSI*

\defAE*

\defAttribution*

\defWitnessing*

\defADA*

\defADB*

\subsection*{Propositions}

\propRecogniseDefeaters*

% \propCSNai*

\propScenariosExist*

\propAbilityExuastive*

\propNoESUandADB*

\propLCS*

\propFCS*

%%% Local Variables:
%%% mode: latex
%%% TeX-master: "master"
%%% End:


\end{document}

%%% Local Variables:
%%% mode: latex
%%% TeX-master: t
%%% End:
