\makeatletter
\renewcommand{\PackageInfo}[2]{}% Remove package information
\renewcommand{\@font@info}[1]{}% Remove font information
% \renewcommand{\@latex@info}[1]{}% Remove LaTeX information
\makeatother

\PassOptionsToPackage{unicode}{hyperref}

\documentclass[10pt]{report}
\usepackage[british]{babel}

\usepackage[online]{Suthesis-2e}

\usepackage{amsthm}         % (in part) For the defined environments
\usepackage{mathtools}      % Improves  on amsmaths/mtpro2
\usepackage{amssymb}

\usepackage{titlesec}
\titleformat{\chapter}[display]
{\normalfont\huge\bfseries}{\chaptertitlename\ \thechapter}{5pt}{\huge}
\titlespacing*{\chapter}{0pt}{0pt}{40pt}

\counterwithout{footnote}{chapter} % For continuous footnote numbering
\counterwithout{figure}{chapter}

\RequirePackage[usenames, dvipsnames]{xcolor}

\definecolor{fuchsia}{HTML}{FE4164}%Neon Fuchsia %{F535AA}%Neon Pink
\definecolor{details}{HTML}{FE4164}
\definecolor{later}{HTML}{A65978}
\definecolor{return}{HTML}{660066}
\definecolor{link}{HTML}{66FFFF}
\definecolor{comment}{HTML}{616161}
\definecolor{offblack}{HTML}{434241}
\definecolor{highlight}{HTML}{FE4164}

% % % My packages % % %
\usepackage{myNotation}
\usepackage{ThesisCustom}
\usepackage{CustomEnvs}
\usepackage{ThesisFig}
% % % % % % % % % % % %

% \usepackage{selnolig}% For suppressing certain typographic ligatures automatically
% % % % % % %
\usepackage{xfrac}
\usepackage{array}
\usepackage{arydshln}
\usepackage{multirow}
% https://tex.stackexchange.com/questions/12703/how-to-create-fixed-width-table-columns-with-text-raggedright-centered-raggedlef
\newcolumntype{L}[1]{>{\raggedright\let\newline\\\arraybackslash\hspace{0pt}}m{#1}}
\newcolumntype{C}[1]{>{\centering\let\newline\\\arraybackslash\hspace{0pt}}m{#1}}
\newcolumntype{R}[1]{>{\raggedleft\let\newline\\\arraybackslash\hspace{0pt}}m{#1}}

% % % The bibliography % % %
\usepackage[%
  backend=biber,
  style=authoryear-comp,
  bibstyle=authoryear,
  citestyle=authoryear-comp,
  uniquename=false,
  backref=false,
  hyperref=true,
  url=false,
  isbn=false,
  doi=false,
  useprefix=true,
  maxbibnames=99,
  ]{biblatex}
\DeclareFieldFormat{postnote}{#1}
\DeclareFieldFormat{multipostnote}{#1}
\newcommand{\noopsort}[1]{}
\addbibresource{Ability.bib}
\DefineBibliographyExtras{british}{\def\finalandcomma{\addcomma}} % Enable Oxford Comma
% % % % % % % % % % % % % % %

\usepackage[inline]{enumitem}
\SetEnumitemValue{labelindent}{standard}{.25\parindent}
\setlist[itemize]{labelindent=standard}
\setlist[enumerate]{labelindent=standard}

\newlist{TOCEnum}{itemize}{1}
\setlist[TOCEnum]{
  style=standard,
  leftmargin=*,
  label=,%
}

\newlist{VAREnum}{itemize*}{1}
\setlist[VAREnum]{
  style=standard,
  font=\normalfont,
  label=\(\circ\),
  noitemsep,
}

\newlist{itenum}{enumerate}{1}
\setlist[itenum]{
  style=standard,
  font=\normalfont,
  labelwidth = \widthof{\emph{Then}:},
}

% % % Misc packages % % %
% \usepackage{setspace}
% \usepackage{refcheck} % Can be used for checking references
% \usepackage{lineno}   % For line numbers
% \usepackage{hyphenat} % For \hyp{} hyphenation command, and general hyphenation stuff

% % % % % % % % % % % % %

\usepackage[export]{adjustbox}
\usepackage{subcaption}
\usepackage{float}

% % % % % % % % % % % % TIKZ
\usepackage{tikz}
\usetikzlibrary{bending,arrows,calc,arrows.meta,patterns,fadings}
\usetikzlibrary{trees}
\usetikzlibrary{backgrounds, positioning, fit, backgrounds}
\usetikzlibrary{math}
\usetikzlibrary{tikzmark}
% % % % % % % % % % % %

\usepackage{graphicx} % for images (png/jpeg etc.)
\usepackage{caption} % for \caption* command

% % % % % % % % % % % % MY COMMANDS
\newcommand{\hozlinedash}[0]{%
  \noindent\hdashrule[0.5ex][c]{\textwidth}{.1pt}{2.5pt}
}

% % % % % % % % % % % %
\usepackage{xskak} % For chess diagram
\usepackage{fitch} % For better fitch proofs?
\usepackage{bussproofs} % For Gentzen-style?
%  \usepackage[ruled, linesnumbered]{algorithm2e} % for algorithms
% \SetKwInOut{Input}{Input}
% \SetKwInOut{Output}{Output}

% % % % % % % % % % % %

\makeatletter
\renewcommand\paragraph{\@startsection{paragraph}{4}{\z@}%
  {-3.25ex\@plus -1ex \@minus -.2ex}%
  {1.5ex \@plus .2ex}%
  {\normalfont\normalsize\bfseries}}
\makeatother

\makeatletter
\renewcommand\subparagraph{\@startsection{subparagraph}{4}{\z@}%
  {-3.25ex\@plus -1ex \@minus -.2ex}%
  {1.5ex \@plus .2ex}%
  {\normalfont\normalsize\bfseries}}
\makeatother


% % % Fonts% %
\usepackage[no-math]{fontspec}
% \defaultfontfeatures{Ligatures=TeX}%,Numbers={Proportional}}
% \defaultfontfeatures{Ligatures={NoCommon, NoDiscretionary, NoHistoric, NoRequired, NoContextual}}
\defaultfontfeatures{Ligatures={NoDiscretionary, NoHistoric, NoRequired, NoContextual}}
% \protrudechars=2 %
\adjustspacing=2 %
\newfontfeature{Microtype}{protrusion=default;expansion=default;}
\setmainfont[Microtype]{Libertinus Serif}
\setsansfont[Microtype,Scale=MatchLowercase]{Libertinus Sans}
\setmonofont{Libertinus Mono}
% \setmainfont[Microtype]{Times New Roman}
% \setsansfont[Microtype,Scale=MatchLowercase]{Helvetica}

\usepackage[match]{luatexja-fontspec}
% \setmainjfont{SourceHanSerif-Regular}
% \setsansjfont{SourceHanSans-Regular}
% \setmonojfont{SourceHanMono}

\usepackage[math-style=ISO]{unicode-math}
\setmathfont{Libertinus Math}

\usepackage[
breaklinks,
bookmarks=false,
hidelinks,
linkcolor=highlight,
citecolor=highlight,
% colorlinks=true,
colorlinks=false,
]{hyperref}

\usepackage{csquotes}

% \hyperlink{cite.MyKey}{link}
% https://tex.stackexchange.com/questions/285710/hyperref-link-to-bibliography-entry

% https://tex.stackexchange.com/questions/83872/getting-hyperref-to-work-with-citeyear-and-citeyearpar-in-biblatex

\DeclareCiteCommand{\citeyear}
{\usebibmacro{prenote}}
{\bibhyperref{\printfield{year}}\bibhyperref{\printfield{extrayear}}}
{\multicitedelim}
{\usebibmacro{postnote}}

\DeclareCiteCommand{\citeyearpar}[\mkbibparens]
{\usebibmacro{prenote}}
{\bibhyperref{\printfield{year}}\bibhyperref{\printfield{extrayear}}}
{\multicitedelim}
{\usebibmacro{postnote}}

% https://tex.stackexchange.com/questions/75902/hyperlinking-author-names-in-biblatex-when-using-citeauthor

\DeclareCiteCommand{\citeauthor}
{\boolfalse{citetracker}%
  \boolfalse{pagetracker}%
  \usebibmacro{prenote}%
}%
{\ifciteindex%
{\indexnames{labelname}}%
{}%
\printtext[bibhyperref]{\printnames{labelname}}}%
{\multicitedelim}%
{\usebibmacro{postnote}}

\addto\extrasbritish{
  \def\chapterautorefname{Chapter}
  \def\sectionautorefname{Section}
  \def\subsectionautorefname{Section}
  \def\subsubsectionautorefname{Section}
  \def\paragraphautorefname{Section}
  \def\subparagraphautorefname{Section}
  \def\AlgoLineautorefname{Line}
  \def\pageautorefname{Page}
  \def\footnoteautorefname{Footnote}
  \def\scenarioCounterautorefname{Scenario}
  \def\observationCounterautorefname{Observation}
  \def\specificationCounterautorefname{Specification}
  \def\illustrationCounterautorefname{Illustration}
  \def\sketchCounterautorefname{Sketch}
  \def\linkCounterautorefname{Link}
  \def\constraintCounterautorefname{Constraint}
  \def\questionCounterautorefname{Question}
  \def\assumptionCounterautorefname{Assumption}
  \def\defiCounterautorefname{Definition}
  \def\propCounterautorefname{Proposition}
  \def\ideaCounterautorefname{Idea}
  \def\condCounterautorefname{Condition}
  \def\intuitionCounterautorefname{Intuition}
  \def\notationCounterautorefname{Notation}
  \def\applicationCounterautorefname{Application}
}

\title{Foregone-conclusions}
\author{Ben Sparkes}
\dept{Philosophy}
% \date{ }

\begin{document}

% % %
\nocite{Lewis:1973aa}
% % %


% \maketitle

\thispagestyle{empty}
\mbox{ }
\vfill
\begin{center}
  {\Large
  FOREGONE-CONCLUSIONS
  \vspace{40pt}

  A DISSERTATION \\
  SUBMITTED TO THE DEPARTMENT OF PHILOSOPHY \\
  AND THE COMMITTEE ON GRADUATE STUDIES \\
  OF STANFORD UNIVERSITY \\
  IN PARTIAL FULFILLMENT OF THE REQUIREMENTS \\
  FOR THE DEGREE OF \\
  DOCTOR OF PHILOSOPHY
  \vspace{40pt}

  Ben Sparkes

  July 2023
  }
\end{center}
\vfill
\mbox{ }

\newpage

\pagenumbering{roman}
\setcounter{page}{4}


\chapter*{Abstract}
\label{cha:abstract}

\begin{note}
  This document is about partial explanations for why an \eiw{} an agent concludes happened.
\end{note}

\begin{note}
  The kind of partial explanations for why an \eiw{} an agent concludes happened of interest are routine.

  For example, while playing a game of chess an agent concludes moving one of their rooks to e4 secures checkmate.
  At least in part, this conclusion is explained by observing the agent has a good understanding of chess and applied their understanding to identify the checkmate.

  In a more detail, the agent may have performed an exhaustive search over the remaining pieces for a checkmate.
  Or, perhaps the agent had cornered their opponent with a particular strategy.

  In contrast, some things are not partial explanations for why the \eiw{} the agent concludes to move one of their rooks to e4 happened.

  For example, consider the agent's conclusion to eat chocolate rather than vanilla ice cream while playing, or whatever the agent concludes to do in the opening of the next game they play.
  Neither seems irrelevant.
\end{note}

\begin{note}
  The argument of this document is oriented around a general constraint on partial explanations for why an \eiw{} an agent concludes happened.

  Very roughly, the constraint holds that partial explanation are limited by what happens when the agent concludes, or what happened prior to the agent concluding.

  I argue the constraint does not hold.
  
\end{note}

\begin{note}
  The contribution of this document may be split into four connected areas:

  First, a a fairly theory-neutral framework to talk about the way \eiw{1} an agent concludes happen.

  Second, a statement of the relevant constraint within the framework.

  Third, a deductive argument which shows the constraint fails when certain conditions obtain.

  Fourth, supporting argument to establish the relevant conditions often obtain, and so failure of the constraint is fairly common.
\end{note}


\begin{note}
  This document is orientated around the constraint.
  Still, the focus of the document is the details of the argument against the constraint.

  The argument highlights the way a handful of observations combine.
  As such, I have worked to ensure the observations lead to general ideas which may be applied or adapted to other arguments.
\end{note}





\newpage

\begin{quote}
  \textsc{Othello} O monstrous, monstrous!

  \textsc{Iago}\phantom{O monstrous, monstrous! Nay,} Nay, this was but his dream.

  \textsc{Othello} But this denoted a foregone conclusion.

  \textsc{Iago} 'Tis a shrewd doubt, though it be but a dream;\newline
  \phantom{\textsc{Iago} 'Tis}And this may help to thicken other proofs\newline
  \phantom{\textsc{Iago} 'Tis}That do demonstrate thinly.\newline
  \mbox{ }\hfill\mbox{(\citetitle{Shakespeare:2003vc}, 3.3.428--432)}
\end{quote}

\vfill

\begin{quote}
  ``In any case, how can we ever know? Essentially a man is what he hides\dots''

  Walter shrugged his shoulders and brought his hands together like a child making a mud pie.

  ``A miserable little pile of secrets\dots''

  ``A man is what he does!'' my father answered sharply.\newline
  \mbox{ }\hfill\mbox{(\cite[20]{Malraux:1968aa})}
\end{quote}

\nocite{Tichy:1976tp}
\nocite{Chisholm:1955aa}
\nocite{Kratzer:1989aa}
\nocite{Davidson:1973vd}
\nocite{Hackl:1998tt,Austin:1961vz}

\tableofcontents

\newpage

\pagenumbering{arabic}
\setcounter{page}{1}

\chapter{Introduction}
\label{cha:introduction}


\begin{note}
  This document shows a constraint on (partial) explanations of an \eiw{0} an agent draws a conclusion from some \pool{} of premises does not hold.
  This introduction introduces the constraint in a relaxed manner.

  The main document is quite precise.
  Here, hopefully, enough is said for you to decide whether you would like to work through some ideas very carefully.
\end{note}


\begin{note}
  When an \eiw{0} an agent draws a conclusion from some \pool{} of premises some relation holds between conclusion and the \pool{} of premises from the \agpe{}.

  We talk about `\fingfr{1}' and two key assumptions are made:
  %
  \begin{enumerate}[label=\Alph*., ref=(\Alph*)]
  \item
    \label{start:A}
    If an agent is in the process of concluding some conclusion \(C\) from some \pool{} \(P\), then \(C\) \fof{} \(P\), from the \agpe{}.
  \item
    \label{start:B}
    It \emph{may} be \(C'\) \fof{} \(P'\), from the \agpe{}, though the agent has not been in the process of concluding \(C'\) from \(P'\).%
    \footnote{
      \ref{start:A} already entails it may be \(C'\) \fof{} \(P'\), from the \agpe{}, though the agent has not concluded \(C'\) from \(P'\).
    }
  \end{enumerate}
\end{note}

\begin{note}
  In support of \ref{start:A} and \ref{start:B}, when an agent is in the process of concluding \(C\) from \(P\) something ensures it is the case the agent is concluding \(C\) from \(P\) rather than concluding \(C'\) from \(P\) or concluding \(C\) from \(P'\).

  So, for example, when an agent is concluding \(43 \times 12 = 516\) from their understanding of arithmetic, \(43 \times 12 = 516\) \fof{} some \pool{} of premises associated with the agent's understanding of arithmetic, from the \agpe{}.

  Likewise, nothing about the agent's understanding of arithmetic changed when the agent began concluding, so it seems \(43 \times 12  = 516\) followed from the \pool{} of premises associated with the agent's understanding of arithmetic, from the \agpe{}, before the agent began concluding.%
  \footnote{
    By contrast, if an agent is choosing between different flavours of ice cream, it need not be the case that there is a \fingfr{} between choice of chocolate and some \pool{}.
    For, it may not be the case the agent is concluding to choose chocolate  --- the eventual choice may be spontaneous.

    If you're unsure about this idea in general, then it is fine to narrow attention to cases like arithmetic where an agent has some well-understood method of obtaining a conclusion from a \pool{} of premises, and needs only apply the method to obtain the conclusion.

    All the important cases in the document have this form.
  }
\end{note}


\begin{note}
  Given an \eiw{} an agent concludes \(43 \times 12 = 516\) we make ask:
  Why was the event an \eiw{} the agent concluded \(43 \times 12 = 516\)?%
  \footnote{
    Stated a little awkwardly so as not to assume agency had any role.
  }

  And, in response one may observe the agent has a fair understanding of arithmetic, and they're on point today.
  So, when the agent started to think about \(43 \times 12\) they started to apply their understanding of arithmetic.
  And, they kept going until they figured out \(43 \times 12 = 516\).

  In short, the agent was concluding.
  And, what happened while an event was in progress (partially) explains why an event happened.
\end{note}


% \begin{note}
%   Why did the agent capture the rook?

%   Well, as it's the endgame there are very few moves available to the agent, so they started an exhaustive search for checkmate and found a case by capturing the rook.
%   And, before this action, the agent concluded to take the rook.

%   Alternatively, it may be there is no answer of this kind to why the agent concluded to capture the rook.
%   For, it may be the case the agent was bored and only looked for a piece they could move --- so, if the agent's line of sight had wandered a little differently they would have moved the bishop.
% \end{note}


\begin{note}
  Given \ref{start:A}, when an agent is concluding \(C\) from \(P\) it is the case \(C\) \fof{} \(P\).
  So, \(C\) \fingfr{1} \(P\) may (partially) explain why an agent concluded \(C\) from \(P\).

  Further, given \ref{start:B} there may be other \fingfr{} present when the agent is concluding \(C\) from \(P\).
  So, perhaps other \fingfr{1} may (partially) explain why an agent concluded \(C\) from \(P\)?
  Or, perhaps the following constraint holds:

  \begin{constraint}{consFirst}{A first pass}
    \(C'\) \fingfr{} \(P'\) (partially) explains why an \eiw{0} an agent concluded \(C\) from \(P\) happened \emph{only if} \(C'\) is \(C\) and \(P'\) is \(P\).
  \end{constraint}

  \noindent%
  In other words, no \fingfr{} \emph{other than} \(C\) \fingf{} \(P\) may (partially) explain why an \eiw{0} an agent concluded \(C\) from \(P\) happened.
\end{note}


\begin{note}
  I think \autoref{consFirst} makes some sense.
  % For, in what connexion could \(C'\) \fingfr{} \(P'\) have to an \eiw{0} an agent concluded \(C\) from \(P\), given \(C\) is different from \(C'\) or \(P\) is different from \(P'\)?

  For example, consider \(3 \times 3 = 9\) \fingf{} some \pool{} associated with the agent's understanding of arithmetic.
  The agent concludes \(43 \times 12 = 516\) rather than \(3 \times 3 = 9\), and in principle it seems the agent could conclude \(43 \times 12 = 516\) without ever figuring out \(3 \times 3 = 9\).
  Nothing about what happened seems to rest on \(3 \times 3 = 9\) \fingf{} any \pool{} associated with the agent's understanding of arithmetic.
  So, \(3 \times 3 = 9\) \fingf{} some \pool{} associated with the agent's understanding of arithmetic seems explanatorily irrelevant to the \eiw{} the agent concludes \(43 \times 12 = 516\).
\end{note}


\begin{note}
  Still, I don't think \autoref{consFirst} makes complete sense.

  For example, an \eiw{0} an agent concluded \(C\) from \(P\) may include a sub-event in which the agent concludes \(C'\) from \(P'\).
  And, in this respect \(C'\) \fingfr{} \(P'\) may (partially) explain why the \eiw{0} an agent concluded \(C\) from \(P\) happened.
  So, I'd like to consider a weaker constraint:

  \begin{constraint}{consSecond}{A second pass}
    \(C'\) \fingfr{} \(P'\) (partially) explains why an \eiw{0} an agent concluded \(C\) from \(P\) happened \emph{only if} the agent has concluded or is concluding \(C'\) from \(P'\).
  \end{constraint}

  \noindent%
  Given \autoref{consSecond}, a \fingfr{} requires some extant from relation to be (partially) explanatory.
\end{note}


\begin{note}
  \autoref{consFirst} entails \autoref{consSecond}, and \autoref{consSecond} covers sub-conclusions and more.

  \autoref{consSecond} may even be too lax.
  Still, the scope of \autoref{consSecond} has an upshot.
  
  For, if \autoref{consSecond} fails to hold, then any weaker constraint (such as \autoref{consFirst}) also fails to hold.
\end{note}

\begin{note}
  This document shows \autoref{consSecond} fails to hold.

  Part of this document amounts to making the idea of \fofr{1} precise, and further motivating the idea that \fofr{1} may explain why.
  And, part of this document amounts to showing the precise variant of \autoref{consSecond} fails to hold --- specifically, a deductive argument establishes \emph{if} certain conditions hold \emph{then} \autoref{consSecond} fails to hold, and supporting argumentation suggests the relevant conditions are common.

  For those still tentative, \autoref{cha:intro} contains more details and a sketch of the overall argument.
\end{note}

% \begin{note}
%   Document is centred around failure of the constraint.
%   Still, focus of the document is on the argument.

%   What I mean by this is that I don't take the failure of the constraint to be a significant contribution.
%   Instead, the argument.
%   Even if you think something is clear, still need to show it is the case.
%   And, in this case I think the details are worth working through.
%   Indeed, though tailored to a particular result, I have attempted to keep things quite general.
%   Rather than failure reducing to a single idea, failure is due to the way a collection of ideas interact.
%   These ideas may be separated and sent separate ways.
% \end{note}

% \begin{note}
%   \autoref{consFirst} is a first pass.

%   A more careful statement of \autoref{consFirst} both specifies the senses of `why' and `how' present in \autoref{consFirst} and clarifies (at least) sufficient conditions for a conclusion to happen (as used to show the failure of \autoref{consFirst}).
%   In reverse order we touch on a few key details:
%   \begin{itemize}
%   \item
%     Conclusions are more-or-less those identified by natural language.

%     For example, one concludes \(43 \times 12 = 516\) from their understanding of arithmetic, a winning move in a game of chess from their understanding of chess and the state of a game of chess, and what the time is from their impression of a clock and their understanding of the way clocks display the time.%
%     \footnote{
%       Further examples are present throughout the document.
%       Some detailed examples are given in \autoref{cha:intro}.
%       And, for the interested reader \autoref{cha:clar} reduces \eiw{0} an agent concludes to a collection of definitions and assumptions.
%     }
%   \item
%     How \eiw{0} an agent concludes happen is fairly unconstrained.
%     Somehow an agent moves from their understanding of chess and the state of a game of chess to a winning move.
%     And, however this happens answers how.
%     Of importance is only that whatever happened happened.
%   \item
%     (Partial) explanations of why an \eiw[\(e\)]{0} an agent concludes happened (partially) establish why \(e\) was an \eiw{0} the agent concludes, rather than \(e\) being \eiw{0} some other thing happened.

%     For example, why was \(e\) an \eiw{0} the agent concluded \(43 \times 12 = 516\) from their understanding of arithmetic as opposed to an \eiw{0} the agent concluded \(43 \times 12 = 516\) by using a calculator, or an \eiw{} the agent concluded \(43 \times 12 = 506\).%
%     \footnote{
%       Chapters \ref{cha:intro} and \ref{cha:events-progress} significantly expand on the (partial) explanations of why at issue.
%     }
%   \end{itemize}

%   \noindent
%   On some days I find the failure of \autoref{consFirst} surprising, and on other days the failure of \autoref{consFirst} seems obvious.
%   Either way, what follows is a careful argument for the failure of \autoref{consFirst}.
% \end{note}

% \begin{note}
%   Specifically, I establish that so long as various conditions obtain, then \autoref{consFirst} fails.
%   Further, I motivate each of the conditions as either pre-theoretically useful or an instance of a commonly accepted idea.

%   That is, I provide a deductive argument against \autoref{consFirst} with strong supporting (though non-deductive) argumentation for the soundness of the relevant premises.
%   As such, significant work has been put into obtaining an suitable balance between premises which are stated with sufficient precision and contain sufficient information to entail further results while holding sufficiently close to an intuitive understanding of the relevant phenomenon in order to be seen as sound.
% \end{note}







% \begin{note}
%   In form, \autoref{consFirst} is a constraint on (partial) explanations of why some thing came to be in terms of how that thing came to be.
  

  
%   Constraints of this kind  \citeauthor{Davidson:1963aa}.
%   Rationalisations are a relation between a reason and an action such that the reason explains the action by giving the agent's reason for doing the action.
%   And, \citeauthor{Davidson:1963aa} argues rationalisations are causal explanations.

%   So, an action is explained by giving the agent's reason for doing the action only if there is some causal trace from the agent's reason for the action to the action.
% \end{note}


% \begin{note}
%   Failure of constraint is compatible with maintaining the \citeauthor{Davidson:1963aa}ian position of rationalisations being causal explanations.

%   Though I think there are two ways to purse this contrast with \citeauthor{Davidson:1963aa}.

%   The first is to consider differences in the target of the constraint.
%   Rationalisations target an agent's reason for doing some action.
%   By contrast, the constraint we consider targets a premise-conclusion relation.

%   Second is to re-examine rationalisations and similar general constraints.
%   Perhaps way failures are identified extends to or approximates other failures.%
%   \footnote{
%     In the case of \citeauthor{Davidson:1963aa} the reason included in the rationalisation is the \emph{agent's} reason.
%     And, I think the way this attribution is understood may make a difference.
%     Still, this line of inquiry is for an other time.
%   }
% \end{note}





%%% Local Variables:
%%% mode: latex
%%% TeX-master: "master"
%%% TeX-engine: luatex
%%% End:


\include{introduction}

\chapter{Overview}
% \addcontentsline{toc}{chapter}{Overview}

\begin{note}
  This chapter provides a brief overview of the argument which follows.
\end{note}

\section{The road ahead}
\label{sec:road-ahead}

\begin{note}
  \autoref{cha:intro} introduced the way we understand \eiw{1} an agent concludes, questions \qWhy{} and \qHow{}, and the constraint \issueInclusion{}.
  We sketched some motivation for \issueInclusion{} and then the general form counterexamples to \issueInclusion{} take.
\end{note}

\begin{note}
  The remainder of this document amounts to working through a handful of ideas fairly carefully.
  Important ideas are applied to \autoref{illu:gist:roots:F} as worked through.
  So, we identify a counterexample to \issueInclusion{} throughout the document.
\end{note}


\begin{note}
  \autoref{cha:events-progress} expands on \autoref{idea:why} to state sufficient conditions for answers to \qWhy{}.

  Specifically, we split \autoref{idea:why} into two ideas:
  An \se{} event and \progEx{}.

  To help keep things simple we have talked about \eiw{1} an agent concludes.
  However, when turning to the details, it is useful to distinguish between events and (true) descriptions of events.

  With a distinction between events and descriptions in hand, \(\edn{\flat}\) under description \(\edo{\flat}\) is an \emph{\se{}} of \(\edn{}\) under description \(\edo{}\) just in case:

  \begin{enumerate}[label=\alph*., ref=\alph*]
  \item
    \label{sketch:se:r}
    \(\edn{}\) under \(\edo{}\) is a result of \(\edn{\flat}\) under \(\edo{\flat}\).
  \item
    \label{sketch:se:p}
    \(\edn{\flat}\) under \(\edo{\flat}\) is such that \(\edn{}\) under \(\edo{}\) is in progress.
  \end{enumerate}
  %
  \ref{sketch:se:r} follows Clause \ref{idea:why:result} of \autoref{idea:why}, and (I argue) \ref{sketch:se:p} entails clause \ref{idea:why:favour}.%
  \footnote{
    I.e., \(\edn{\flat}\) under \(\edo{\flat}\) favours \(\edn{}\) under \(\edo{}\) due to it being the case \(\edn{}\) is in progress.
  }

  In turn, \progEx{} expands on Clause \ref{idea:why:feat} of \autoref{idea:why}.
  Roughly, \progEx{} states that whatever is captured of \(\edn{\flat}\) by description \(\edo{\flat}\) is a (partial) explanation of `why' an agent concluded \(\pv{\phi}{v}\) from \(\Phi\).
\end{note}

\begin{note}
  The remainder of the document amounts to arguing certain descriptions capture \fingfr{1}, and some such \fingfr{1} are incompatible with \issueInclusion{}.
\end{note}

\begin{note}
  Chapters~\ref{cha:fcs}, \ref{cha:ros}, and \autoref{cha:requs} establish the way a description of an event captures a \fingfr{}.
  Following, \autoref{cha:requs} outlines a way to obtain \fingfr{1} which may amount to counterexamples to \issueInclusion{}, and \autoref{cha:ces} provides a pair of detailed counterexamples.
\end{note}

\begin{note}
  In a little more detail, chapters~\ref{cha:fcs} and \ref{cha:ros} focus on identifying \fingfr{}.

  Specifically, \autoref{cha:fcs} introduces the idea of \(\pv{\psi}{v'}\) being a \fc{} from \(\Psi\) for an agent.
  Where, roughly, \(\pv{\psi}{v'}\) is a \fc{0} from \(\Psi\) just in case the agent may do some action and be concluding \(\pv{\psi}{v'}\) from \(\Psi\).
  For example, \pv{\rootsCon{}}{\valI{True}} was plausibly a \fc{} from a \pool{} which included the \agents{} understanding of factorisation \emph{prior to} \autoref{illu:gist:roots:F}, as the agent plausibly was concluding \pv{\rootsCon{}}{\valI{True}} from the \pool{} when the agent begins their reasoning.

  In turn, \autoref{cha:ros} states:
  %
  \begin{itemize}
  \item
    A conclusion of \(\pv{\phi}{v}\) from \(\Phi\) is sufficient for \(\pv{\phi}{v}\) to \fof{} \(\Phi\) from the \agpe{} when the agent concludes \(\pv{\phi}{v}\) from \(\Phi\).
  \item
    \(\pv{\psi}{v'}\) being a \fc{} from \(\Psi\) is sufficient for \(\pv{\phi}{v}\) to \fof{} \(\Phi\) from the \agpe{}.
  \end{itemize}
  %
  So, if a description \(\edo{\flat}\) of an event \(\edn{\flat}\) entails a \fc{0}, \(\edo{\flat}\) also entails a corresponding \fingfr{}.
  And, if \(\edo{\flat}\) captures a \se{} of an \eiw{0} an agent concludes, the \fingfr{} answers \qWhy{}.

  A pair of quick notes may be helpful here:

  \begin{itemize}
  \item
    I expect you to have a somewhat intuitive grasp on the `\fof{}' relation and \fc{1} hopefully identify some instances compatible with your intuitive grasp.
  \item
    When we speak of \fc{1} our interest is with whatever it is that makes it the case \(\pv{\psi}{v'}\) is a \fc{} from \(\Psi\), rather than the (foregone-)conclusion of \(\pv{\phi}{v}\) from \(\Phi\).
    That is, whatever it is that makes it the case \(\pv{\psi}{v'}\) is a \fc{} from \(\Psi\) amounts to something which makes it the case that \(\pv{\psi}{v'}\) `follows from' \(\Psi\), from the \agpe{}.
  \end{itemize}
\end{note}


\begin{note}
  \autoref{cha:requs} details the way \fc{1}, \fingfr{1}, and \progEx{} interact to provide sufficient conditions for an answer to \qWhy{}.

  Here, we identify a few answers to \qWhy{} which are compatible with \issueInclusion{}.

  So, taken as a unit, chapters \autoref{cha:intro} to \autoref{cha:requs} detail a way to find answers to \qWhy{}, and the answers we find are compatible with \issueInclusion{}.
\end{note}


\begin{note}
  \autoref{cha:typical} introduces the idea an agent \tCV{} to help identify answers to \qWhy{} which are not compatible with \issueInclusion{}.

  The basic idea of an agent \tCV{} \(\pv{\phi}{v}\) from \(\Phi\) is that there is some generality to the \agents{} reasoning.
  As noted above, with respect to \autoref{illu:gist:roots:F}, \pv{\propM{\rootsConEqExV{12}{4}{3}}}{\valI{True}} plausibly follows from a sufficient understanding of factorisation.
  In turn, the agent may do some action any be concluding \pv{\propM{\rootsConEqExV{12}{4}{3}}}{\valI{True}} from a \pool{} which captures their understanding of factorisation.
  By the ideas of chapters~\ref{cha:fcs} and \ref{cha:ros} this identifies a \fingfr{}, and so long as the agent has not concluded \pv{\propM{\rootsConEqExV{12}{4}{3}}}{\valI{True}} from the relevant \pool{}, \issueInclusion{} fails to hold.
\end{note}


\begin{note}
  Finally, \autoref{cha:ces} details counterexamples to \issueInclusion{}.
  In particular, we highlight in detail the way ideas as applied to \autoref{illu:gist:roots:F} show \issueInclusion{} fails to hold.
  And, with some luck you, yourself, create a counterexample to \issueInclusion{}.
\end{note}

\begin{note}
  The argument is more about the way a handful of ideas interact and apply to various \scen{1} rather than any single idea.

  To help separate the main line of argument about the interaction between various ideas from detailed arguments and applications of ideas the argument is built from `definitions', `ideas', `assumptions', and `propositions'.

  Definitions, ideas, and assumptions are stated and motivation follows in an unstructured way, while propositions are always accompanied by a corresponding argument.

  In addition, `applications' and `observations' are included.
  Applications are used to highlight the way a particular definition or idea applies to a \scen{} and a handful of applications are used to highlight a failure of \issueInclusion{} (with respect to \autoref{illu:gist:roots:F}) while observations amount to notes which may be helpful but can be safely ignored.
  Like propositions, applications and observations come with marked motivation.

  You are encouraged to read the statement of a proposition, application, or observation and then move on if you'd prefer not to work through the details.

  Also, sometimes sections are marked as `optional'.
  Nothing in a section marked as optional is involved in the main line of argument.
\end{note}

\begin{note}
  Though we only provide a few explicit counterexamples to \issueInclusion{}, these counterexamples are obtained by applying general ideas developed to specific \scen{1}.
  So, by the close of this document I hope to have given you the resources to find other failures of \issueInclusion{}.
\end{note}



\paragraph*{A note on reading this document}

\begin{note}
  The argument of this document is fairly tightly connected.
  Ideas introduced and definitions made in earlier chapters are often used to introduce further ideas or make additional definitions in later chapters.
  And, propositions argued for often build on prior propositions.

  If you are reading this as a PDF, references to ideas stated in this document are hyperlinked.
  For example, clicking on --- \autoref{illu:gist:roots:F} --- takes you to \autoref{illu:gist:roots:F}, and clicking on --- \qWhy{} --- takes you to the statement of \qWhy{}.

  I have also tried to include some page references where I think they may be helpful.

  In addition, \autoref{cha:re} collects together a few recurring definitions, ideas, and propositions.
  So, if you have a print-out I recommend setting \autoref{cha:re} to one side for easy reference.
  Or, if you have a PDF, opening a second copy of this document in a separate window to \autoref{cha:re} may be helpful.
\end{note}



%%% Local Variables:
%%% mode: latex
%%% TeX-master: "master"
%%% TeX-engine: luatex
%%% End:


\chapter{Events, in progress}
\label{cha:events-progress}


\begin{note}
  Our interest is understanding the way an \eiw{0} an agent concludes happens.

  This chapter outlines our understanding of both events and events (such that some other event is) in progress.

  This understanding is then applied to an \eiw{1} an agent concludes and \eiw{0} an agent concludes is in progress --- or colloquially an \eiw{0} an agent is concluding --- throughout the rest of the document.

  A key takeaway of this chapter is `\progEx{}' which expands on \autoref{idea:why} (\autoref{cha:intro}, \autopageref{idea:why}) and characterises the sense of `why' present in \qWhy{} in terms of features of an event in progress.
\end{note}


\section{Events}
\label{sec:events}

\begin{note}
  We understand events in a broadly (Neo-)\citeauthor{Davidson:1967aa}ian framework.
  In short:
  Events are things we refer to by way of descriptions which are true of the event.%
  \footnote{
    Keeping track of both events and descriptions gives me a headache, and perhaps it gives you a headache too.
    I tried to avoid the complication, but the headache was almost unbearable.

    The notation introduced below is designed so that a description is easy to ignore when it is of little importance.
    And, when a description is important attention will be drawn to its importance.
    Descriptions are important throughout this chapter, though less so in later chapters.
  }

  For example, a natural language sentence such as:
  % 
  \begin{enumerate}[label=\arabic*., ref=(\arabic*), series=ESERIES]
  \item
    \label{ESERIES:toast}
    Sam buttered some toast in the kitchen.
  \end{enumerate}
  % 
  Is understood as stating there is some event \(\edn{}\) such that \(\edn{}\) is a butter event, the agent of \(\edn{}\) is Sam, the theme of \(\edn{}\) is some toast, and the location of \(\edn{}\) is the kitchen.%
  \footnote{
    Alternatively:
    \(\exists e [\textsc{butter}(e)\text{ \& }\textsc{agent}(e, \text{Sam})\text{ \& }\exists x(\textsc{theme}(e, \text{toast}(x)))\text{ \& }\textsc{in}(e, \text{the kitchen})]\)
  }

  Likewise:
  %
  \begin{enumerate}[label=\arabic*., ref=(\arabic*), resume*=ESERIES]
  \item
    \label{ESERIES:gistCalcEq}
    Max concludes \gistCalcEq{} has value \valI{True}.
  \end{enumerate}
  %
  Is understood as sating there is some event \(\edn{}\) such that \(\edn{}\) is a conclude event, the agent of \(\edn{}\) is Max, the \prop{0} of \(\edn{}\) is \gistCalcEq{} and the \val{} is \valI{True}.

  As events are referred to by descriptions true of the event, a description captures a specific event only if there is a unique event which satisfies the description.

  So, neither \ref{ESERIES:toast} nor \ref{ESERIES:gistCalcEq} explicitly refer to a unique event.
  For example, the description of \ref{ESERIES:toast} is compatible with Sam buttering three or five pieces of toast and doing so yesterday or a few months ago.
  Likewise, \ref{ESERIES:gistCalcEq} is compatible with Max using a calculator or their understanding of arithmetic to conclude \gistCalcEq{} has value \valI{True}.

  Still, when discussing events we will almost always have a specific event.
  For our purposes, the role of descriptions is to highlight particular features of events.%
  \footnote{
    In particular, both \qWhy{} and \qHow{} are asked with respect to a specific event.
  }
  So, to ease the way we talk about events, we make use of the following notation:

  \begin{notation}[Events and descriptions]%
    \label{assu:HaUniqueD}%
    \vspace{-\baselineskip}
    \begin{itemize}
    \item
      \(\edn{}\) picks out a specific event (i.e.\ an event under some unique description).
    \item
      \(\ed{}\) captures (the specific event) \(\edn{}\) under the particular description \(\edo{}\).
    \end{itemize}
    \vspace{-\baselineskip}
  \end{notation}

  \noindent%
  For example, given \autoref{assu:HaUniqueD} we may re-express \ref{ESERIES:gistCalcEq} to capture a specific event by writing:
  % 
  \begin{enumerate}[label=\arabic*\('\)., ref=(\arabic*\('\))]
    \setcounter{enumi}{1}
  \item
    The event \(\edn{}\) under description \(\edo{}\) `Max concludes \gistCalcEq{} has \val{0} \valI{True}'.
  \end{enumerate}
  % 
  Where \(\edn{}\) captures all features of the event, such as when the event took place, the method by which Max concluded \gistCalcEq{} has value \valI{True}, how many times Max blinked while concluding, and so on.
\end{note}

\begin{note}
  Some technicalities (such as our use of \prop{1}, \val{1}, and \pool{1}) influence the descriptions we use, but something like `the agent concluded \gistCalcEq{} is true by using a calculator' is about the level of detail I have in mind, though what follows is compatible with additional detail.
\end{note}



\section{Events in progress}
\label{sec:events-progress}

\begin{note}
  Our interest is understanding the way an \eiw{0} an agent concludes happens.
  And, the idea of an event is progress is a key idea with respect to our understanding of the way an \eiw{0} an agent concludes happens.

  This section briefly characterises the way we understand events in progress and states two important assumptions we make about events in progress.

  Stated a little more carefully, our interest is with events such that the event is described as being an \eiw{0} some other event is in progress.
  Still, `events in progress' is a little easier to parse.
\end{note}



\subsection{The progressive}
\label{sec:progressive}


\begin{note}
  Events in progress are intuitively understood via the progressive aspect.

  For example:
  % 
  \begin{enumerate}
  \item
    The agent is making soup.\newline
    \mbox{} \hfill \(\leadsto\) An \eiw{0} the agent makes soup is in progress.
  \item
    The agent is reading Henley's `Invictus'.\newline
    \mbox{} \hfill \(\leadsto\) An \eiw{0} the agent reads Henley's `Invictus' is in progress.
  \item
    The agent is riding the slope.\newline
    \mbox{} \hfill \(\leadsto\) An \eiw{0} the agent rides the slope is in progress.
  \end{enumerate}
  % 
  Specifically:

  \begin{intuition}[Events in progress and the progressive]
    \label{def:es-in-prog}
    \vspace{-\baselineskip}
    \begin{itemize}
    \item
      \(\ed{\flat}\) is an \eiw{0} \(\ed{}\) is in progress.
    \end{itemize}
    % 
    \emph{If and only if}:
    % 
    \begin{itemize}
    \item
      \(\edn{}\) is a (maybe non-actual) development of \(\edn{\flat}\).
    \item
      \(\edo{\flat}\) entails \(\edo{}\) is happening.\newline
      \mbox{ }\hfill (Where \(\edo{}\) is happening is understood in terms of the progressive aspect.)
    \end{itemize}
    \vspace{-.5\baselineskip}
  \end{intuition}

  \noindent%
  For example, going by \autoref{def:es-in-prog}, the following are equivalent:
  \begin{itemize}
  \item
    \(\ed{\flat}\) is an \eiw{0} Max concludes \pv{\propM{\gistCalcEq{}}}{\valI{True}} is in progress.
  \item
    \(\edo{\flat}\) is true of \(\edn{\flat}\) and entails Max concludes \pv{\propM{\gistCalcEq{}}}{\valI{True}} is happening.
  \item
    \(\edo{\flat}\) is true of \(\edn{\flat}\) and entails Max is concluding \propM{\gistCalcEq{}} is \valI{True}.
  \end{itemize}

  \noindent%
  We assume an implicit understanding of the progressive aspect,%
  \footnote{
    \nocite{Portner:1998um}
    \nocite{Engelberg:1999vi}
    Note, though, that English does not have a quick, unambiguous, way of expressing events in progress.
    For, consider the sentence:
    \begin{enumerate}[label=\arabic*., ref=(\arabic*)]
    \item
      \label{prog:abmig}
      \textquote{John is studying for an exam}.
    \end{enumerate}
    \ref{prog:abmig} may be understood to express either the continuous or progressive aspect.

    Under the continuous aspect, \ref{prog:abmig} captures something about John, rather than something about an event happening.
    Hence, it need not be the case that John is engaged in an event of studying when \ref{prog:abmig} is said.
    For example, we may expand:
    \textquote{Sam is studying for an exam, but is having a short nap.}

    By contrast, \ref{prog:abmig} under the progressive captures an event where John studying is in progress.
    For example, we may expand:
    \textquote{Sam is studying for an exam, so they aren't having a nap.}

    See,~\textcite{Richards:1981wo},~\textcite{Portner:2011vi}, etc.\ for a general overview of the progressive.
    In particular, I suggest \textcite{Landman:1992wh} as a nice introduction.
    \citeauthor{Szabo:2004ul} (\citeyear[34]{Szabo:2004ul}) provides a concise summary:
    \begin{quote}
      [A] progressive sentence is true at some time just in case some event occurs at that time, and if we follow the development of the event (within our world as long as it goes, then jumping into a nearby world, and iterating the process within the limits of reasonability) we will reach a possible world where the perfective correlate is true of the continuation.
    \end{quote}
    For a more in depth summary see (\cite[764--766]{Portner:1998um}) and for some issues with \citeauthor{Landman:1992wh}'s account, see
    (\cite{Bonomi:1997uq}),
    (\cite[49--50]{Engelberg:1999vi}),
    (\cite[35]{Szabo:2004ul}),
    (\cite[767]{Portner:1998um}),
    and (\cite[1256]{Portner:2011vi}).
  }
  and explicitly assume a common feature of analyses of the progressive holds of events in progress:%
  \footnote{
    See, e.g.:
    (\cite{Bennett:1972uw}),
    (\cite{Dowty:1979vq}),
    (\cite{Parsons:1990aa}),
    (\cite{Landman:1992wh}), and
    (\cite{Portner:1998um}).

    \assuPP{2} is sometimes motivated by the imperfective paradox (\cite[Ch.3.1]{Dowty:1979vq}, \cite[12]{Bach:1986tb}).
  }

  \begin{assumption}[\assuPP{2}]%
    \label{assu:PP}%
    \vspace{-\baselineskip}
    \begin{itenum}
    \item[\emph{If}:]
      \(\ed{\flat}\) is an \eiw{0} \(\ed{}\) is in progress.
    \item[\emph{Then}:]
      There is some \progAdj{0} event \(\edn{\sharp}\) such that~\ref{assu:PP:pe:dev} and~\ref{assu:PP:pe:verb} are both true:
      \begin{enumerate}[label=\roman*., ref=(\roman*)]
      \item
        \label{assu:PP:pe:dev}
        \(\edn{\sharp}\) is a development of \(\edn{\flat}\).
      \item
        \label{assu:PP:pe:verb}
        \(\edo{\sharp}\) is true of \(\edn{\sharp}\), where \(\edo{\sharp}\) is the perfective correlate of \(\edo{}\).
      \end{enumerate}
    \end{itenum}
    \vspace{-\baselineskip}
  \end{assumption}
\end{note}

\begin{note}[Interest with the progressive]
  The immediate role of \assuPP{} is to help fix intuitions about events in progress by reflecting on the sense of possibility at issue.%
  \footnote{
    \assuPP{2} is denied by some.
    For example, \citeauthor{Szabo:2004ul} argues:
    \textquote{Sometimes we are \emph{doing} things even though there is no real chance that we could get them \emph{done}, and this is true even if we abstract away from the possibility of miraculous intervention.}
    (\citeyear[40]{Szabo:2004ul})
    E.g., \citeauthor{Szabo:2004ul} denies~\ref{Szabo:Arch} is necessarily false:
    \begin{quote}
      \begin{enumerate}[label=(\arabic*), ref=(\arabic*)]
        \setcounter{enumi}{12}
      \item
        \label{Szabo:Arch}
        As the architect was building the cathedral he knew that, although he would be building it for another year or so, he couldn't possibly complete it.%
        \mbox{ }\hfill\mbox{(\citeyear[38]{Szabo:2004ul})}
      \end{enumerate}
    \end{quote}
    Though,~\ref{Szabo:Arch} seems always false to me.
    The only sense with which I read~\ref{Szabo:Arch} as true under the progressive requires factivity of knowledge to fail, thus allowing the cathedral to be built.
    I'd rephrase \autoref{Szabo:Arch} to state the architect was building \emph{part of} the cathedral and knew they couldn't possibly complete the entire thing.

    See (\cite[1245]{Portner:2011vi}) for additional, distinct, discussion of (\cite{Szabo:2004ul}).
  }
  The deferred role of \assuPP{} is to help motivate a couple of ideas introduced in later chapters and ease a handful of arguments.%
  \footnote{
    Specifically, \fc{1} in \autoref{cha:fcs} and \fingfr{1} in \autoref{cha:ros}.
  }

  We make three observations to help firm intuition.

  \begin{observation}[\assuPP{2} and existential modality]%
    \label{obs:prog-not-reg-poss}%
    The sense of possibility in \assuPP{} does not reduce to existential \{logical, metaphysical, nomic, \dots\} possibility.
  \end{observation}
  \begin{motivation}{obs:prog-not-reg-poss}
    Suppose an agent is sitting a multiple-choice exam.
    To pass the exam the agent only needs to make a sufficient number of correct choices.
    It is certainly logically, metaphysically, and nomically possible that the agent chooses a sufficient number of correct choices.
    However, it does not follow that the agent is passing the exam, as there need be no guarantee the agent makes a sufficient number of correct choices.%
    \footnote{
      See also Igal Kvart's example of Mary wiping out the Roman army (\cite[18]{Landman:1992wh}).
    }
  \end{motivation}

  \begin{observation}[\assuPP{2} and counterfactuals]%
    \label{obs:prog-not-cfs}%
    There is no simple relation between the sense of possibility in \assuPP{} and `close-world' analyses of counterfactuals.%
    \footnote{
      E.g.\ as found in (\cite{Todd:1964aa}), (\cite{Stalnaker:1968vt}), and (\cite{Lewis:1973th}).
      See (\cite[\S2]{Starr:2022aa}) for an overview.
    }
  \end{observation}
  \begin{motivation}{obs:prog-not-cfs}
    Suppose an agent is passing an exam without external help.
    Then, a classmate passes answers to the agent, which the agent uses.
    The agent is no longer passing the exam without external help.
    And, any close possible world where the classmate does not pass answers, it may be some other classmate does.
  \end{motivation}

  \begin{observation}[\assuPP{2} and uniqueness]%
    \label{obs:prog-no-unique}%
    An event may be in progress without the event being sufficiently developed to `indicate' a unique outcome.\newline
  \end{observation}
  \begin{motivation}{obs:prog-no-unique}
    Suppose an agent has drawn a straight line on a piece of paper.
    It may be true that the agent is drawing a triangle.
    However, the straight line is compatible with the agent drawing an \(n\)-sided polygon, for any \(n\) within some reasonable bound.%
    \footnote{
      This observation is inspired by \citeauthor{Dowty:1979vq}'s example involving a circle and a triangle (\citeyear[133]{Dowty:1979vq}).
    }
  \end{motivation}

  \noindent%
  Loosely paraphrased, if an event is in progress, then intuitively there is \emph{something} about the way things are which ensures the existence of a possible completion event (\autoref{obs:prog-no-unique}) which is robust against external influence (\autoref{obs:prog-not-cfs}) and does not require luck (\autoref{obs:prog-not-reg-poss}).
\end{note}


\begin{note}
  Finally, a brief observation highlights a quirk \autoref{assu:HaUniqueD}:

  \begin{observation}[Events in progress and common satisfaction]%
    \label{obs:eip-partial}%
    Sometimes, when \(\ed{}\) is an \eiw{0} \(\ed{\ast}\) is in progress, \(\edo{\ast}\) may be satisfied by multiple events.
  \end{observation}

  \begin{motivation}{obs:eip-partial}
    Consider an agent flipping a coin until it lands heads or lands tails ten times in a row.
    Given sufficient determination from the agent, it is true that an \eiw{0} the agent flips a coin until it lands heads or lands tails ten times in a row is in progress.

    Now, a unique \eiw{0} the agent flips a coin until it lands heads or lands tails ten times in a row involves an \(i\)th throw the coin lands heads on, or ten throws where the coin lands tails.
    However, by \autoref{cons:no-f-ref}, it must be true of \(\ed{}\) that \(\ed{\ast}\) is in progress.
    And, as coin flips are quite random (\cite{Gelman:2002ww}), \(\ed{}\) may fail to entail that the coin lands heads on the \(i\)th, or that the coin lands tails on all ten throws.
    Hence, \(\ed{\ast}\) may be satisfied by such an event.
  \end{motivation}
\end{note}

\begin{note}
  Note, given \autoref{assu:HaUniqueD}, \(\ed{\ast}\) being an \eiw{0} \(\ed{}\) is in progress is distinct from \(\ed{\ast}\) being an \eiw{0} \(\edn{}\) is in progress.
  For, \(\ed{}\) being in progress only requires the features of \(\edn{}\) as captured by description \(\edo{}\) are in progress.
  And, by contrast, \(\edn{}\) being in progress requires all the features of \(\edn{}\) are in progress.

  For example, suppose \(\edn{}\) is an \eiw{0} an agent completes a crossword without using a dictionary and an earlier event \eiw[\(e^{\ast}\)]{0} the agent is completing the crossword.
  Given the description \(\edo{\ast}\) `the agent is completing the crossword' it follows \(\edn{}\) under the description `the agent completes the crossword' is in progress.
  However, it does not follow that the agent completing the crossword without using a dictionary is in progress.
  For, without expanding \(\edo{\ast}\) we have no guarantee the agent may solve the crossword without using a dictionary.

  In this respect, it may seem a little odd to mention \(\edn{}\).
  For, to say \(\edn{}\) is in progress is just to say there is \emph{some} event in progress such that \(\edo{}\) is true of the event.
  However, the option of referencing features of \(\edn{}\) not captured by \(\edo{}\) is used later.
\end{note}



\subsection{A constraint and an assumption}
\label{sec:assumptions-1}


\paragraph{The constraint}


\begin{note}
  We place one important constraint on descriptions:

  \begin{constraint}{Fog}{Fog}%
    \label{cons:no-f-ref}%
    A description \(\edo{}\) of an event \(\edn{}\) is limited to what is true of \(\edn{}\) when \(\edn{}\) happens.
  \end{constraint}

  \noindent%
  \autoref{cons:no-f-ref} rules out describing an event by using information about what happens after the relevant event happens.
  And, \autoref{cons:no-f-ref} is important given the way hindsight interacts with descriptions of events.
  To illustrate, consider the following passage from \citeauthor{Bhatt:2008aa} (\citeyear{Bhatt:2008aa}) with respect to ability:%
  \footnote{
    Similarly, \citeauthor{Austin:1961vz} remarks `it follows merely from the premiss that he does it, that he has the ability to do it, according to ordinary English' (\citeyear[175]{Austin:1961vz}).
    See also \citeauthor{Boylan:2020aa}'s discussion of `Past Success' (\citeyear[\S1.1]{Boylan:2020aa}).
  }
  % 
  \begin{quote}
    \begin{enumerate}[label=(\arabic*)]
      \setcounter{enumi}{314}
    \item
      (from~\cite{Thalberg:1969ta})
      \begin{enumerate}[label=\alph*., ref=(315\alph*)]
      \item
        \label{Bhatt:Thal-a}
        Yesterday, Brown hit three bulls-eyes in a row.
        Before he hit three bulls-eyes, he fired 600 rounds, without coming close to the bullseye; and his subsequent tries were equally wild.
      \item
        \label{Bhatt:Thal-b}
        Brown was able to hit three bulls-eyes in a row.
      \item
        \label{Bhatt:Thal-c}
        Brown had the ability to hit three bulls-eyes in a row.
      \end{enumerate}
    \end{enumerate}
    % 
    From~\ref{Bhatt:Thal-a}, we can conclude~\ref{Bhatt:Thal-b} but not~\ref{Bhatt:Thal-c}.
    Brown could have hit the target three times in a row by pure chance and he does not need to have had any ability for~\ref{Bhatt:Thal-b} to be true.\newline
    \mbox{ }\hfill\mbox{(\citeyear[167]{Bhatt:2008aa})}
  \end{quote}
  % 
  I read \ref{Bhatt:Thal-b} and \ref{Bhatt:Thal-c} as synonyms.
  Still, the distinction \citeauthor{Bhatt:2008aa} highlights is clear.
  Expressed in terms of events in progress:
  When Brown was throwing darts it was not the Brown was throwing three bulls-eyes in a row, as Brown may have missed the dartboard on any throw.
  However, in hindsight one may say Brown was hitting three bulls-eyes in a row, as Brown hit three bulls-eyes in a row.

  Given \autoref{cons:no-f-ref} Brown was hitting three bulls-eyes only if something about the \eiw{0} Brown was throwing darts made it the case that Brown was hitting three bulls-eyes in a row regardless of what may be said with hindsight.

  The same applies to conclusions.
  For example, suppose an agent has completed a crossword.
  Intuitively, with hindsight an \eiw{0} the agent completes the crossword puzzle was in progress while the agent worked on the puzzle.
  However, it may not have been the case that an \eiw{0} the agent completes the crossword puzzle was in progress in the sense of interest.
  For, the agent may only have figured out a word by randomly stringing together vowels and consonants, and the agent may have failed to hit upon the right string to jog their memory.
\end{note}


\begin{note}
  In broader context, our interest in this document is understanding the way an \eiw{0} an agent concludes happens.
  The role of events in progress is to help understand the way an \eiw{0} an agent concludes happens, and \autoref{cons:no-f-ref} ensures any explanation given is not informed by hindsight.

  In short, the explanations of interest do not do so with the benefit of hindsight.
\end{note}



% \begin{note}
%   \nocite{Maier:2018uo}
%   First, in the terminology of \citeauthor{Whittle:2010wr} of interest are specific (or local) ability which concern `what the agent is able to do now, in some particular circumstances', rather than general (or global) abilities which concern `what an agent is able to do in a large range of circumstances'. (\citeyear[2]{Whittle:2010wr}).%
%   \footnote{
%   See also \citeauthor{Austin:1961vz}'s (\citeyear{Austin:1961vz}) distinction between `categorical' abilities and opportunities.

%   In particular, this sets aside accounts of ability such as~\citeauthor{Carter:2021wd}'s~(\citeyear{Carter:2021wd}) `fallibilist',~\citeauthor{Kikkert:2022wp}'s~(\citeyear{Kikkert:2022wp}) `robust', and \citeauthor{Maier:2013vk}'s (\citeyear{Maier:2013vk}) `general' account, among others.
% }

%   Second, specific ability is often accompanied by a `control' intuition.
%   For example,
%   `[A]n agent has the ability to \(\phi\) iff there are accessible worlds at which she \(\phi\)s simply by deciding to \(\phi\).' (\citeyear[19]{Schwarz:2020aa})

%   This is far stronger than an event in progress.

%   At issue is that a conclusion may be in progress even though there is no action available to the agent which results in the conclusion.
%   In particular, to adhere to some strategy is not to perform a single action.
%   For example, consider an agent working on a crossword.
%   So long as the agent is quite good at crosswords and the crossword is not too difficult, it is plausible an \eiw{0} the agent concludes all the words is in progress.
%   However, conclusion is not the result of a single act.
%   For, as the agent works through the crossword they develop additional hints, and reasoning builds on those hints.

%   Of course, \assuPP{} gets a world, but not necessarily that this is a result of deciding or any other action that may be specified in advance.%
%   \footnote{
%   Similar accounts of the control intuition are found in (\cite{Brown:1988tl}), (\cite{Horty:1995wu}), (\cite{Jaster:2020wv}), and (\cite{Kikkert:2022wp}).
%   Further, the control intuition is a key feature of both \citeauthor{Mandelkern:2017aa}'s (\citeyear{Mandelkern:2017aa}) and \citeauthor{Boylan:2020aa}'s (\citeyear{Boylan:2020aa}) `act conditional' analysis of ability.
%   Abstracting from a few minor details, the act conditional analysis of ability states:
%     %   
%   \begin{quote}
%     `\(S\) is able to \(\varphi\)' is true just in case there is some action \(A\) available to \(S\) such that if \(S\) tried to \(A\) then S would \(\varphi\).
%   \end{quote}
%     %   
%   Formally:
%     %   
%   \[%
%     \sem[c,w]{\text{S is able to }\varphi} = 1\text{ iff }\exists A \in \mathcal{A}_{S,c,w,t}\colon \forall v \in f_{c}(\text{S does }A,w),  \sem[c,v]{\varphi(S)} = 1%
%   \]
%     %   
%   Where:
%     %   
%   \begin{itemize}
%   \item
%     \(f_{c}\) is a selection function from proposition-world pairings to set of worlds.
%   \item
%     \(\mathcal{A}_{S,c,w}\) is the set of actions that are available to \(S\) in context \(c\) and world \(w\).
%   \end{itemize}
%     %   
%   In other words, \(S\text{ is able to }\varphi\) is true at some world \(w\) in context \(c\), just in case there is some action available to the agent, such that for every world for which it is true that \(S\text{ tries to}A\) determined by the selection function \(f_{c}\), it is the case that \(S \varphi\text{s}\).

%   The primary difference between the analyses of \citeauthor{Mandelkern:2017aa} and \citeauthor{Boylan:2020aa} is the specification of \(f_{c}\).
%   For \citeauthor{Mandelkern:2017aa},
%   \(f_{c}\) is~\citeauthor{Stalnaker:1968vt}'s selection function.
%   I.e.\ \(f_{c}(\psi, w) = \{\text{\emph{the} `closest' world to }w\text{ where }\psi\text{ is true.}\}\).
%   (\citeyear[Cf.][314]{Mandelkern:2017aa}).
%   (Though, a `\citeauthor{Lewis:1973th}ian' approach that allows multiple close worlds seems viable.)

%   In contrast, \citeauthor{Boylan:2020aa}'s \(f_{c}\) selects all (possibly indeterminate) worlds which are identical to \(w\) up until time \(t\) (in which \(S\) does \(A\)).
%   (\citeyear[\S3.3]{Boylan:2020aa})

%   For our purposes, the key part of the act conditional analysis is not the details of the relevant selection function, but the observation that \(S \varphi\)s \emph{follows from} \(S\text{ does }A\) in all worlds captured by the selection function.
%   This is the control intuition --- the agent doing some action results in \(\phi\) being the case.
% }
% \end{note}



\paragraph{The assumption}


\begin{note}
  Given \autoref{def:es-in-prog} the progressive helps fix intuitions about what it is for an event (under some description) to be such that some other event (under some other description) is in progress.
  However, we may the following key assumption which may conflict with intuitions about the progressive:

  \begin{assumption}[Exclusive progress]
    \label{assu:p:ex}
    \vspace{-\baselineskip}
    \begin{itenum}
    \item[\emph{If}:]
      It is not possible for \(\edo{\alpha}\) and \(\edo{\omega}\) to both be true of an event.
    \item[\emph{Then}:]
      For any events \(e\) and \(\ed{\flat}\):
      \begin{itemize}
      \item
        It is not possible for \(\ed{\flat}\) to be such that \(\ed[\alpha]{}\) \emph{and} \(\ed[\omega]{}\) are in progress.
      \end{itemize}
    \end{itenum}
    \vspace{-\baselineskip}
  \end{assumption}

  \noindent%
  Paraphrased, \autoref{assu:p:ex} assumes that if an event is in progress, then no incompatible event is (also) in progress.

  For example:
  It is not possible for both Team A wins the game of hockey and for Team B to wins the game of hockey to be true of an event.
  So, by \autoref{assu:p:ex}, it is not possible for Team A to be winning the game of hockey and for Team B to be winning the game of hockey.
  Hence, if Team A is winning the game of hockey, Team B is not winning the game of hockey.
\end{note}


\begin{note}
  I take \autoref{assu:p:ex} to be intuitive.
  Still, \autoref{assu:p:ex} does not clearly hold of the progressive.
  For example, \citeauthor{Landman:1992wh} writes:%
  \footnote{
    See \textcite{Bonomi:1997uq} for additional discussion.
  }

  \begin{quote}
    Suppose I was on a plane to Boston which got hijacked and landed in Bismarck, North Dakota.
    What was going on before the plane was hijacked?
    One thing I can say is:
    `I was flying to Boston when the plane was hijacked'
    This is reasonable.
    But another thing I could say is:
    `I was flying to Boston.
    Well, in fact, I wasn't, I was flying to Bismarck, but I didn't know that at the time'
    And this is also reasonable.%
    \mbox{ }\hfill\mbox{(\citeyear[30--31]{Landman:1992wh})}
  \end{quote}
  %
  \citeauthor{Landman:1992wh}'s observation is about the progressive, and is only incompatible with \autoref{assu:p:ex} to the extent if it is reasonable to say `I was flying to Boston' \emph{and} `I was flying to Bismarck', then it was the case an event in which I was flying to Boston \emph{and} I was flying to Bismarck was in progress.

  However, I don't think this entailment makes sense.
  It is not possible to have flown to both Boston and Bismarck at the same time in the same way it is not possible for team A and team B to have won a game of hockey at the same time --- having flown to Boston entails one is not in Bismarck and a win for Team A entails a loss for Team B.

  Indeed, \citeauthor{Landman:1992wh}'s observation as stated may be compatible with this.
  For, \citeauthor{Landman:1992wh}'s first statement is explicitly clarified as given with respect to what they knew while \citeauthor{Landman:1992wh}'s second statement takes into account details about what happened after the event (cf.\ our \autoref{cons:no-f-ref}).
  \citeauthor{Landman:1992wh} does not make both statements from the same perspective, and only the second statement involves knowledge.

  Still, I see no real benefit in raising \autoref{def:es-in-prog} to a definition by digressing into an account of the progressive.
  Even if \autoref{assu:p:ex} does hold of the progressive,%
  \footnote{
    An account of the progressive which ties the progressive to causation, such as \citeauthor{Szabo:2004ul}'s (\citeyear{Szabo:2004ul}) account, arguably entails \autoref{assu:p:ex}.
  }
  an argument would amount to a significant digression.
  So, \autoref{assu:p:ex} captures an important way our understanding of events in progress \emph{may} differ from the progressive.
\end{note}



\section{\se{3} and \progEx{}}
\label{sec:se3-progex}


\begin{note}
  The role of events in progress is to help understand the way an \eiw{0} an agent concludes happens.
  This section outlines the way events in progress help understand the way an \eiw{0} an agent concludes happens.

  This section is split into two sub-sections.

  The first sub-section introduces the idea of a `\se{}'.
  And, with the idea of a \se{} in hand, the second section links \se{} to explanations which involve `why' with the sense of `why' present in \qWhy{} by a pair of propositions we collectively term `\progEx{}'.
\end{note}



\subsection{\se{3}}

\begin{note}
  An \se{} is an event \(\edn{\flat}\) under some description \(\edo{\flat}\) such that some other event \(\edn{}\) under description \(\edo{}\) is in progress and \(\ed{}\) (in part) as a result \(\ed{\flat}\).%
  \footnote{
    The `p' in `\se{}' stands for `progress' and the `r' stands for `result'.
  }

  We define \se{1}, make a few observations, and then apply the definition of a \se{} to \autoref{illu:gist:roots:F}.
\end{note}


\begin{note}
  \begin{rdefinition}{def:se}{\se{3}}
    \vspace{-\baselineskip}
    \begin{itemize}
    \item
      \(\ed{\flat}\) is a \emph{\se{0}} of \(\ed{}\).
    \end{itemize}
    \emph{If and only if}:
    \begin{itemize}
    \item
      Clauses~\ref{assu:p:se:prog} and \ref{assu:p:se:hCon} hold:
      \begin{enumerate}[label=\Alph*., ref=\Alph*]
      \item
        \label{assu:p:se:prog}
        \(\ed{\flat}\) is such that \(\ed{}\) is in progress.
      \item
        \label{assu:p:se:hCon}
        \(\ed{}\) partly happens as a result of \(\ed{\flat}\).
      \end{enumerate}
    \end{itemize}
    \vspace{-\baselineskip}
  \end{rdefinition}

  \noindent%
  \(\ed{\flat}\) being a \emph{\se{0}} of \(\ed{}\) concerns specific events \(\edn{\flat}\) and \(\edn{}\).
  And, of interest is whether specific descriptions \(\edo{}\) and \(\edo{\flat}\) capture a particular connexion between \(\edn{\flat}\) and \(\edn{}\).

  Still, our interest is only with parts of \(\edn{\flat}\) and \(\edn{}\), respectively.
  Intuitively:
  %
  \begin{itemize}
  \item
    Clause~\ref{assu:p:se:prog} `looks forward':

    The description \(\edo{\flat}\) of \(\edn{\flat}\) captures that an event described by \(\edo{}\) is in progress.
  \item
    Clause~\ref{assu:p:se:hCon} `looks backward':

    \(\edn{}\) as described by \(\edo{}\) happened in part as a result of \(\edn{\flat}\) as described by \(\edo{\flat}\).
  \end{itemize}

  The role of Clause~\ref{assu:p:se:prog} is to ensure the event \(\edn{}\) is favoured over some other event.
  For, by \autoref{assu:p:ex} as \(\ed{}\) is in progress, it is not the case that \(\ed[x]{}\) is in progress, grating it is not possible for \(\edo{}\) and \(\edo{x}\) to both be true of an event.

  And, the role of Clause~\ref{assu:p:se:hCon} is to ensure \(\edn{}\) happens as a result of being favoured.
\end{note}


\begin{note}
  The role of clauses~\ref{assu:p:se:prog} and \ref{assu:p:se:hCon} combined to give answers to `why' (in the sense of \qWhy{}) \(\ed{}\) happened.
  Still, for the moment our interest is with what clauses~\ref{assu:p:se:prog} and \ref{assu:p:se:hCon} say rather than their role.

  As Clause~\ref{assu:p:se:prog} simply places a constraint on the relevant description \(\edo{\flat}\) our discussion focuses on Clause~\ref{assu:p:se:hCon}.
\end{note}


\begin{note}
  I have little to say in abstract about what it is for an event to `partly' happen `as a result of' some other event.
  Roughly, it was not possible for \(\ed{}\) to happen without \(\ed{\flat}\) happening.
  In this respect, \(\ed{}\) only happened --- in part --- as a result \(\ed{\flat}\).%
  \footnote{
    I suspect `causally depends on' may be substituted in place of `partly happens as a result of'.

    For example, if \(\ed{}\) causally depends on \(\ed{\flat}\) \emph{just in case} \(\ed{}\) would not occur were \(\ed{\flat}\) not occur (\cite[cf.][1.1]{Menzies:2020aa}), then the agent buttering toast causally depends on the bread being toasted.

    However, in contrast to an event being partly being the result of some other event, I doubt there is much pre-theoretical intuition for what causal dependence amounts to.

    Some of Aesop's fables, in translation at least, talk about events being results of other events and I have not found a (translation of a) fable that talks about causal dependence.
    E.g.:
    % 
    \begin{quote}
      A deer had fallen ill and was resting on the grassy plain.
      When the other animals came to see her, they ate up all the grass in her pasture.
      As a result, when the deer recovered from her illness, she ended up dying since her pasture had come to an end.%
      \mbox{ }\hfill\mbox{(\cite[124]{Aesop:2002aa})}
    \end{quote}
    % 
    Still, Clause~\ref{assu:p:se:prog} requires \(\ed{}\) is in progress, and if the progressive is understood via causation as suggested, e.g. \textcite{Szabo:2004ul}, then both clauses may then be expressed via causation.
  }

  For example, it is not possible for an agent to be paddling in the ocean unless the agent steps foot into the ocean.
  Hence, the agent paddling in the ocean partly happened as the result of the agent stepping foot into the ocean.

  Likewise, it is not possible for an agent to butter some toast without some bread being toasted.
  Hence, the agent buttering toast partly happened as a result of the bread being toasted.
\end{note}


\begin{note}
  The following observations and proposition highlight important features of Clause~\ref{assu:p:se:hCon}.
\end{note}


\begin{note}
  First, we observe \(\ed{}\) partly happening as a result of \(\ed{\flat}\) does not follow from \(\ed{\flat}\) being such that an event \(\ed{}\) is in progress:

  \begin{observation}[Irrelevant progress]%
    \label{obs:se-need-hCon}%
    It may be the case that:
    \begin{itemize}
    \item
      \(\ed{\flat}\) is such that an event \(\ed{}\) is in progress and \(\ed{}\) happens.
    \item
      \(\ed{}\) does not partly happen as a result of \(\ed{\flat}\).
    \end{itemize}
    \vspace{-\baselineskip}
  \end{observation}

  \noindent%
  I.e., Clause~\ref{assu:p:se:hCon} of \autoref{def:se} does not follow from Clause~\ref{assu:p:se:prog}.

  \begin{motivation}{obs:se-need-hCon}
    Suppose an agent is passing an exam without external help.
    Then, a classmate passes the agent some answers, which the agent uses.
    The agent passes the exam, and was passing an exam without external help.
    Still, the agent did not pass the exam as a result passing an exam without external help.
  \end{motivation}
\end{note}


\begin{note}
  Second, observe \(\ed{}\) partly happening as a result of \(\ed{\flat}\) concerns \(\edn{}\) under description \(\edo{}\) and \(\edn{\flat}\) under description \(\edo{\flat}\).
  These descriptions capture features of the relevant events, and so of interest is whether the features of \(\edn{}\) as captured by description \(\edo{}\) partly happened as a result of the feature of \(\edn{\flat}\) as captured by description \(\edo{\flat}\).
  This brief observation has an important consequence:

  \begin{rproposition}{prop:se-d-lim}{Limited descriptions}
    \vspace{-\baselineskip}
    \begin{itenum}
    \item[\emph{If}:]
      \(\ed{}\) partly happens as a result of \(\ed{\flat}\).
    \item[\emph{Then}:]
      \(\edo{\flat}\) does not include features of \(\edn{\flat}\) that \(\ed{}\) does not partly happen as a result of.
    \end{itenum}
    \vspace{-\baselineskip}
  \end{rproposition}

  \begin{argument}{prop:se-d-lim}
    As highlighted above, our interest is in whether \(\edn{}\) as described by \(\edo{}\) partly happens as a result of \(\edn{\flat}\) as described by \(\edo{\flat}\).
    In this respect, \(\ed{}\) must be a result of everything captured by \(\edo{\flat}\).%
    \footnote{
      Our interest is \emph{not} with whether \(\edn{}\) as described by \(\edo{}\) partly happens as a result of \(\edn{\flat}\) as described by \emph{some part of} \(\edo{\flat}\).
    }
  \end{argument}

  \noindent%
  For example, suppose \(\edo{}\) is the description `an agent shuffles a deck so that \mainCard{} is \mainCardPos{} by sleight of hand', and \(\edo{}\) is true of \(\edn{}\).
  Further, suppose \(\edo{\flat}\) is the description `the agent marks the \mainCard{} so they do not lose the card in a shuffle' and \(\edo{\flat}\) is true of \(\edn{\flat}\).

  Now, intuitively \(\ed{}\) partly happens as a result of \(\ed{\flat}\).
  For, \(\ed{\flat}\) captures part of the sleight of hand.
  It is not possible for the agent to shuffle a deck so that \mainCard{} is \mainCardPos{} by sleight of hand without ensuring they do not lose \mainCard{} in the shuffle.%
  \footnote{
    Of course, the agent may randomly shuffle a deck so that \mainCard{} is \mainCardPos{}, but then the agent has not shuffled a deck \emph{so that} \mainCard{} is \mainCardPos{}.
  }

  However, \(\edo{\flat}\) may be expanded to include further details.
  For example, let \(\edo{\flat +}\) be the description `the agent marks the \mainCard{} so they do not lose the card in a shuffle while finding a hole in their sock'.
  Intuitively, it is not the case that \(\ed{}\) partly happens as a result of \(\ed{\flat +}\).
  For, the agent finding a hole in their sock had no influence on whether the agent marked \mainCard{}.

  Though, if \(\edn{}\) is described by \(\edo{+}\) as `shuffles the deck by performing sleight of hand while coming to regret not buying a new pair of socks in the January sale', then \(\ed{+}\) plausibly happens as a result of \(\ed{\flat +}\).
\end{note}


\begin{note}
  Finally, our interest is not with whether \emph{any} event captured by description \(\edo{}\) is a result of any event captured by description \(\edo{\flat}\).
  In particular, features of \(\edn{}\) which are not captured by \(\edo{}\) and features of \(\edn{\flat}\) which are not captured by \(\edo{\flat}\) may be relevant to \(\ed{}\) partly happening as a result of \(\ed{\flat}\).
  This is highlighted by the following observation:

  \begin{observation}[Events, described]%
    \label{obs:theEventsHuh}%
    It may be the case \(\ed{}\) partly happens as a result of \(\ed{\flat}\) only given details of \(\edn{}\) and \(\edn{\flat}\) which are not captured by \(\edo{}\) and \(\edo{\flat}\).
  \end{observation}

  \begin{motivation}{obs:theEventsHuh}
    Consider the example following \autoref{prop:se-d-lim}.

    \(\edo{}\) is the description `an agent shuffles a deck so that \mainCard{} is \mainCardPos{} by sleight of hand', and \(\edo{}\) is true of \(\edn{}\).
    And, \(\edo{\flat}\) is the description `the agent marks the \mainCard{} so they do not lose the card in a shuffle' and \(\edo{\flat}\) is true of \(\edn{\flat}\).

    \(\ed{}\) intuitively partly happens as a result of \(\ed{\flat}\).
    However, \(\edo{\flat}\) does not guarantee that the agent successfully completes the shuffle.
    For example, it is consistent with \(\edo{\flat}\) that the agent loses their mark.
    Still, the implicit understanding of \(\edn{}\) and \(\edn{\flat}\) is that the two shuffles are connected.
    And, so long as the two shuffles are connected \(\ed{}\) plausibly does partly happen as a result of \(\ed{\flat}\) for the motivation given above.
  \end{motivation}
\end{note}


\begin{note}
  \autoref{obs:theEventsHuh} may be seen to pair with \autoref{prop:se-d-lim}.
  For, our interest is in whether \(\edn{}\) as described by \(\edo{}\) partly happens as a result of \(\edn{\flat}\) as described by \(\edo{\flat}\).
  This means that everything in the description \(\edo{\flat}\) must be relevant to whether \(\edn{}\) as described by \(\edo{}\) happens.
  However, this does not mean everything relevant to whether \(\edn{}\) as described by \(\edo{}\) happens is captured by \(\edo{\flat}\).
\end{note}


\begin{note}
  In short, I take the idea \(\ed{}\) partly happening as a result of \(\ed{\flat}\) to be intuitive.

  Our interest is in the feature of \(\edn{}\) captured by description \(\edo{}\) partly being a result of the feature of \(\edn{\flat}\) captured by description \(\edo{\flat}\).
  Hence, \(\edo{\flat}\) is limited to what is relevant to \(\edn{}\) as captured by description \(\edo{}\) happening (\autoref{prop:se-d-lim}) though \(\edo{\flat}\) need not exhaust what is relevant to \(\edn{}\) as captured by description \(\edo{}\) happening (\autoref{obs:theEventsHuh}).
\end{note}


\begin{note}
  To close this section we provide an argument for \se{} with respect to \autoref{illu:gist:roots:F}.

  \begin{application}[A \se{} of \autoref{illu:gist:roots:F}]
    \label{obs:se-inst}%
    Given:
    % 
    \begin{itemize}
    \item
      \(\edn{}\) is the event described by \autoref{illu:gist:roots:F}.
    \item
      \(\Phi\) includes the \agents{} understanding of factorisation prior to \(\edn{}\).
    \item
      \(\edo{}\) is the description:
      `The agent concludes \pv{\propM{\rootsCon{}}}{\valI{True}} from \(\Phi\)'.
    \item
      \(\edn{\flat}\) covers Step~\ref{illu:gist:roots:F:factor} of the \agents{} reasoning in \autoref{illu:gist:roots:F}
    \item
      \(\edo{\flat}\) is the description:
      `The agent figures out \rootsConEqExV{6}{3}{2} with the aim to identify the factors of \rootsConEq{}'.
    \end{itemize}
    % 
    It is the case that:
    % 
    \begin{itemize}
    \item
      \(\ed{\flat}\) is a \se{} of \(\ed{}\).
    \end{itemize}
    \vspace{-\baselineskip}
  \end{application}

  \noindent%
  Note, Step~\ref{illu:gist:roots:F:factor} of \autoref{illu:gist:roots:F} amounts to the agent figuring out \rootsConEqExV{6}{3}{2}.
  However, \autoref{illu:gist:roots:F} did not explicitly characterise the agent attempting to identify the factors of \rootsConEq{}.
  Still, we take it to be implicit the agent was attempting to identify the factors of \rootsConEq{}.
  Alternatively, consider this a retcon.
  Either way, \(\edn{}\) is true of \(\edn{}\) and \(\edo{\flat}\) is true of \(\edn{\flat}\).
\end{note}


\begin{note}
  The detail for \autoref{obs:se-inst} are somewhat complex.
  I sort of recommend working the detail for \autoref{obs:se-inst} as it highlights the way events and event descriptions interact.
  Still, if you would like to skip ahead the very brief argument is:

  \begin{itemize}
  \item
    Clause~\ref{assu:p:se:prog} of \autoref{def:se} is satisfied as an agent figuring out \rootsConEqExV{6}{3}{2} with the aim to identify the factors of \rootsConEq{} is clearly working towards an \eiw{0} the agent concludes \pv{\propM{\rootsCon{}}}{\valI{True}} from \(\Phi\).
  \item
    Clause~\ref{assu:p:se:hCon} of \autoref{def:se} is satisfied as it makes no sense for an agent to conclude \pv{\propM{\rootsCon{}}}{\valI{True}} from a \pool{} which includes the \agents{} understanding of factorisation if the conclusion does not partly happen as a result of factorisation, which is exactly what Step~\ref{illu:gist:roots:F:factor} of \autoref{illu:gist:roots:F} captures.

    Further, it is plausibly the case that the \agents{} conclusion is partly a result of the agent aiming to figure out the factors of \rootsConEq{}.

    So, the \agents{} conclusion is partly a result of factoring with the aim to figure out the factors of \rootsConEq{}.
  \end{itemize}

  \begin{dets}{obs:se-inst}
    Our goal is to show clauses~\ref{assu:p:se:prog} and \ref{assu:p:se:hCon} of \autoref{def:se} hold.
    \medskip

    \noindent%
    In order for Clause~\ref{assu:p:se:prog} to hold it must be the case \(\ed{\flat}\) is such that \(\ed{}\) is in progress.
    In other words, form \(\edo{\flat}\) being true of \(\edn{\flat}\) it must be the case that \(\edn{}\) as described by \(\edo{}\) is in progress.

    Now, \(\edn{\flat}\) as described by \(\edo{\flat}\) is an \eiw{0} the agent figures out \rootsConEqExV{6}{3}{2} with the aim of figuring out the relevant factors.
    And, by assumption, the agent has a sufficient understanding of factorisation.

    So, as the agent figures out \rootsConEqExV{6}{3}{2}, the only reasoning which remains is captured by steps \ref{illu:gist:roots:F:zero} to \ref{illu:gist:roots:F:factor:done} of \autoref{illu:gist:roots:F}.
    These steps are more-or-less straightforward.
    Hence, I take it to be clear an \eiw{0} the agent concludes \pv{\propM{\rootsCon{}}}{\valI{True}} from \(\Phi\) \emph{may} be in progress.

    Further, as by \(\edo{\flat}\) the agent has the aim to identify the factors of \rootsConEq{} the previous observation may be strengthened to state an \eiw{0} the agent concludes \pv{\propM{\rootsCon{}}}{\valI{True}} from \(\Phi\) \emph{is} in progress.

    In short, the reasoning which remains to conclude \pv{\propM{\rootsCon{}}}{\valI{True}} from \(\Phi\) is more-or-less straightforward and the agent plans to complete the relevant reasoning.
    So, \(\ed{\flat}\) is building to an \eiw{0} the agent concludes \pv{\propM{\rootsCon{}}}{\valI{True}} from \(\Phi\).

    This is what we wanted to show, Clause~\ref{assu:p:se:prog} of \autoref{def:se} holds.
    \medskip

    \noindent
    Note, in particular, our goal was only to show \(\edn{}\) under the description `the agent concludes \pv{\propM{\rootsCon{}}}{\valI{True}} from \(\Phi\)' is in progress.
    Our goal was not to show \(\edn{}\) under any other description is in progress.

    And, note the importance of the \agents{} aim to identify the factors of \rootsConEq{}.
    If \(\edn{\flat}\) is only described as an \eiw{0} the agent figures out \rootsConEqExV{6}{3}{2} then there is nothing to guarantee the agent puts the factorisation of \rootsConEq{} to any further use.
    \bigskip

    \noindent%
    Clause~\ref{assu:p:se:hCon} is a little more involved.
    We start by arguing for a slightly distinct claim:
    % 
    \begin{enumerate}[label=X., ref=(X)]
    \item
      \label{obs:se-inst:phew}
      For any event \(\edn{+}\) such that \(\edo{}\) is true of \(\edn{+}\), clauses \label{obs:se-inst:phew:d} and \label{obs:se-inst:phew:e} hold:
      \begin{enumerate}[label=\alph*., ref=\alph*]
      \item
        \label{obs:se-inst:phew:d}
        There is some description \(\edo{\sharp}\) such that:
        \begin{itemize}
        \item
          \(\edo{\sharp}\) is the description: `The agent figures out \rootsConEqExV{6}{3}{2}'
        \end{itemize}
      \item
        \label{obs:se-inst:phew:e}
        There is some event \(\edn{\times}\) such that clauses \ref{obs:se-inst:phew:e:t} and \ref{obs:se-inst:phew:e:h} hold:
        \begin{enumerate}[label=\roman*., ref=\roman*]
        \item
          \label{obs:se-inst:phew:e:t}
          \(\edo{\sharp}\) is true of \(\edn{\times}\)
        \item
          \label{obs:se-inst:phew:e:h}
          \(\ed[]{+}\) partly happens as a result of \(\ed[\sharp]{\times}\).
        \end{enumerate}
      \end{enumerate}
    \end{enumerate}
    %
    In short, \ref{obs:se-inst:phew} states an \agents{} conclusion of \pv{\propM{\rootsCon{}}}{\valI{True}} from a \pool{} which includes the \agents{} understanding of factorisation prior to \(\edn{}\) partly happens as the result of the agent figuring out \rootsConEqExV{6}{3}{2}.
    \medskip

    \noindent%
    The argument for \ref{obs:se-inst:phew} is by contradiction.

    Suppose \(\edn{+}\) such that \(\edo{}\) is true of \(\edn{+}\) and either Clause~\ref{obs:se-inst:phew:d} or Clause~\ref{obs:se-inst:phew:e} fails to hold.

    Clause~\ref{obs:se-inst:phew:d} only concerns the existence of a description, and it is clear \(\edo{\sharp}\) is a description.
    Hence, given our assumption Clause~\ref{obs:se-inst:phew:e} must fail to hold.
    In particular, there is no event \(\edn{\times}\) such that Clause~\ref{obs:se-inst:phew:e:t} or Clause~\ref{obs:se-inst:phew:e:h} fails to hold.

    Now, consider the collection of events that \(\ed[]{+}\) partly happens as a result of.
    As Clause~\ref{obs:se-inst:phew:e:h} is true of every such event, Clause~\ref{obs:se-inst:phew:e:t} must fail to be true of every such event.

    In short, \(\ed{}\) does not partly happen as a result of an \eiw{0} the agent figures out \rootsConEqExV{6}{3}{2}.

    This is a contradiction.
    For, \(\edo{}\) describes \(\edn{+}\) as an \eiw{0} the agent concludes \pv{\propM{\rootsCon{}}}{\valI{True}} from \(\Phi\).
    And, \(\Phi\) includes the \agents{} understanding of factorisation.
    Yet, given the assumption made for a contradiction, \(\edo{}\) does not partly happen as a result of the agent factoring.
    And, if \(\edo{}\) does not partly happen as a result of the agent factoring then it is not the case that \(\edn{}\) is an \eiw{0} the agent concludes anything by their understanding of factorisation.
    For, the \agents{} understanding of factorisation is clearly irrelevant to whatever happened in \(\edn{+}\), and so \(\edo{}\) is not true of \(\edn{+}\).

    In short, it is built into describing an event as a conclusion by factorisation that factoring took place.%
    \footnote{
      Indeed, it is built into \(\edn{+}\) being described as an \eiw{0} the agent concludes \pv{\propM{\rootsCon{}}}{\valI{True}} from \(\Phi\) that \(\edn{+}\) partly happens as a result of reasoning which amounts to steps~\ref{illu:gist:roots:F:eq}~to~\ref{illu:gist:roots:F:factor:done} of \autoref{illu:gist:roots:F}.
    }
    \medskip

    \noindent%
    With \ref{obs:se-inst:phew} in hand we return to \(\ed{\flat}\) and \(\ed{}\).
    Note, \(\edo{}\) is true of \(\edn{}\) and so clauses~\ref{obs:se-inst:phew:d}~and~\ref{obs:se-inst:phew:e} hold with respect to \(\edn{}\).

    Intuitively, \(\edn{\flat}\) is the relevant instance of \(\edn{\times}\).
    I do not argue for this point.%
    \footnote{
      However, note that if \(\edn{\flat}\) is \emph{not} the relevant instance of \(\edn{\times}\) then by \ref{obs:se-inst:phew} there must be some other \eiw{0} the agent figures out \rootsConEqExV{6}{3}{2} that \(\ed{}\) partly happens as a result of.
      And, \autoref{illu:gist:roots:F} does not involve the agent (re-)figuring out \rootsConEqExV{6}{3}{2} after \(\edn{\flat}\) so any such event must happen prior to \(\edn{\flat}\).
      So, the relevant event must happen prior to \(\edn{\flat}\).

      To rule out this possibility, we may add to \autoref{illu:gist:roots:F} that the agent performs the reasoning of Step~\ref{illu:gist:roots:F:factor} for the first time.
    }
    \medskip

    \noindent%
    Now, as this point we have established \(\ed{}\) partly happens as a result of \(\ed[\sharp]{\flat}\).
    However, \(\edo{\sharp}\) and \(\edo{\flat}\) are not equivalent.
    For, \(\edo{\flat}\) adds to \(\edo{\sharp}\) that the agent figures out \rootsConEqExV{6}{3}{2} with the aim to identify the factors of \rootsConEq{}.
    And, by \autoref{prop:se-d-lim}, \(\ed{}\) does not partly happen as a result of \(\ed{\flat}\) if \(\edo{\flat}\) includes features of \(\edn{\flat}\) that \(\ed{}\) does not happen as a result of.

    So, we need to show \(\ed{}\) partly happens as a result of the agent having the aim to identify the factors of \rootsConEq{}.
    This, I take to be sufficiently plausible, and so may either be read into \autoref{illu:gist:roots:F} or explicitly added to \autoref{illu:gist:roots:F}.
    For, it seems if the agent does not have the aim to identify the factors of \rootsConEq{}, the agent does not identify the factors of \rootsConEq{}.

    So, Clause~\ref{assu:p:se:hCon} of \autoref{def:se} is satisfied.
  \end{dets}

  \noindent%

  % The key idea of \autoref{obs:se-inst} is somewhat straightforward:
  % Observe it makes no sense for \(\edo{}\) to be true of \(\edn{}\) if it is not the case that \(\ed{}\) partly happens as a result of \(\edo{\flat}\) being true of \(\edn{\flat}\).

  % In particular, most of the detail of \autoref{obs:se-inst} amounted to arguing it makes no sense for an agent to conclude \pv{\propM{\rootsCon{}}}{\valI{True}} from \(\Phi\) without the conclusion partly happening as a result of the agent figuring out \rootsConEqExV{6}{3}{2}.
\end{note}



\subsection{\progEx{2}}
\label{sec:ProgEx}


\begin{note}
  We now argue \se{1} (\autoref{def:se}, \autopageref{def:se}) to provide explanations about why an event happened --- given the sense of `why' present in \qWhy{}.

  The argument is split into two propositions, titled `\progExI{}' and `\progExII{}', respectively.
  We refer to both propositions by the term `\progEx{}'.
\end{note}


\begin{note}
  Before starting, recall we understand the sense of `why' present in \qWhy{} in terms of why did the \eiw[\(\ed{}\)]{0} an agent concludes some \prop{0} \(\phi\) has \val{0} \(v\) from some \pool{0} \(\Phi\) happen, rather than any incompatible event --- e.g., an \eiw{0} the agent failed to conclude \(\pv{\phi}{v}\) from \(\Phi\).
\end{note}



\subsubsection{\progExI{}}


\begin{note}
  \begin{rproposition}{prop:PEbasic}{\progExI{}}%
    Given \(\ed{}\) is an \eiw{0} \vAgent{} does \(a\), and the sense of `why' present in \qWhy{}:

    \begin{itenum}
    \item[\emph{If}:]
      \(\ed{\flat}\) is a \se{} of \(\ed{}\).
    \item[\emph{Then:}]
      \(\edo{\flat}\) being true of \(\edn{\flat}\) explains `why' \(\ed{}\) happened.
    \end{itenum}
    \vspace{-\baselineskip}
  \end{rproposition}

  \begin{motivation}{prop:PEbasic}
    Assume \(\ed{\flat}\) as a \se{} of \(\ed{}\).
    \medskip

    \noindent%
    By Clause~\ref{assu:p:se:prog} of \autoref{def:se} (\autopageref{def:se}), \(\ed{\flat}\) is such that \(\ed{}\) is in progress.
    So, by \autoref{assu:p:ex}:
    For any description \(\ed{\ast}\) such that it is not possible for \(\edo{}\) and \(\edo{\ast}\) to both be true of an event, it is not the case that \(\ed[\ast]{}\) is in progress.
    So, \(\ed{\flat}\) being such that \(\ed{}\) is in progress favours \(\ed{}\) happening over any other incompatible event (\(\ed[\ast]{}\)) happening.
    \medskip

    \noindent%
    Further, by Clause~\ref{assu:p:se:hCon} of \autoref{def:se}, \(\ed{}\) partly happens as a result of \(\ed{\flat}\).
    So, \(\ed{}\) partly happened as a result of something with favoured \(\ed{}\) happening over any other incompatible event happening.
    \medskip

    \noindent%
    In this respect, \(\edo{\flat}\) being true of \(\ed{}\) explains `why' \(\ed{}\) happened, as opposed any other event.
    For, \(\edo{\flat}\) being true of \(\edn{\flat}\) gets that \(\ed{}\) was favoured over any other (incompatible) event and \(\ed{}\) partly happened as a result of \(\ed{\flat}\).\newline
  \end{motivation}

  \noindent%
  In other words, given \(\ed{\flat}\) is a \se{} of \(\ed{}\), \(\edo{\flat}\) being true of \(\ed{}\) explains `why' \(\ed{}\) happened in a basic sense, as by Clause~\ref{assu:p:se:hCon} \(\ed{}\) partly happened as a result of \(\ed{\flat}\).
  And, Clause~\ref{assu:p:se:prog} allows us to expand this observation to see \(\ed{}\) partly happened as a result of something which favoured \(\ed{}\) happening over any (incompatible) event.
\end{note}


\begin{note}
  For example, \autoref{obs:se-inst} established the description \(\edo{\flat}\) such that the event \(\edn{\flat}\) which covers Step~\ref{illu:gist:roots:F:factor} of the \agents{} reasoning is a \se{} of \(\ed{}\).
  Specifically, \(\edo{\flat}\) was `the agent is concluding \pv{\propM{\rootsCon{}}}{\valI{True}} from \(\Phi\) in \(\edn{\flat}\) and figures out \rootsConEqExV{6}{3}{2}'
  Hence, by \autoref{prop:PEbasic}, \(\edo{\flat}\) being true of \(\edn{\flat}\) explains `why' the agent concluded \pv{\propM{\rootsCon{}}}{\valI{True}} from \(\Phi\).

  So, given the argument for \autoref{obs:se-inst}, \(\edo{\flat}\) being true of \(\edn{\flat}\) explains `why' the agent concluded \pv{\propM{\rootsCon{}}}{\valI{True}} from \(\Phi\) as the \agents{} conclusion happened as a result of the agent figuring out \rootsConEqExV{6}{3}{2} and as the agent was concluding \pv{\propM{\rootsCon{}}}{\valI{True}} from \(\Phi\) while figuring out \rootsConEqExV{6}{3}{2}.
\end{note}


\begin{note}%
  \nocite{Bromberger:1966aa}%
  Setting the particular way \progEx{} is motivated aside, I take the basic idea to be intuitive.
  For example, consider \citeauthor{Hempel:1965aa}'s Deductive-Nomological account of scientific explanation:
  \phantlabel{mention:Hempel:1}
  % 
  \begin{quote}
    [A Deductive-Nomological] explanation answers the question
    `\emph{Why} did the explanandum-phenomenon occur?'
    by showing that the phenomenon resulted from certain particular circumstances, specified in \(C_{1}, C_{2}, \dots C_{k}\), in accordance with the laws \(L_{1}, L_{2}, \dots L_{\gamma}\).
    By pointing this out, the argument shows that, given the particular circumstances and the laws in question, the occurrence of the phenomenon \emph{was to be expected}; and it is in this sense that the explanation enables us to \emph{understand why} the phenomenon occurred.%
    \mbox{ }\hfill\mbox{(\citeyear[337]{Hempel:1965aa})}
  \end{quote}
  % 
  Observe, \citeauthor{Hempel:1965aa} stress a connexion with observing an event \textquote{was to be expected} and an explanation allowing us to \textquote{understand why} the event occurred (the italics are \citeauthor{Hempel:1965aa}'s).
  For \citeauthor{Hempel:1965aa} the relevant expectation follows from circumstances and laws while for us the relevant expectation follows from identifying some event such that the event of interest is in progress (and partly happens as a result of).

  Indeed, circumstances and laws capture events in progress, so long as the laws are true.
  For, if some event is to happen given certain circumstances and laws, then the circumstances make it the case the event is in progress.
  Further, if the relevant laws are causal, the event happens partly as a result of the circumstances, and a \se{} may be identified.

  Still, our account of events in progress and \se{} does not rest on the existence of laws, and in this respect \progExI{} is distinct from \citeauthor{Hempel:1965aa}'s account of Deductive-Nomological explanation --- not to mention the emphasis we place of descriptions of the relevant events.%
  \footnote{
    Looking ahead, our full method for constructing counterexamples to \issueInclusion{} will appeal to a law-like idea (specifically the idea of an agent \tC{} in \autoref{cha:typical}).
    However, the law-like idea we appeal to is not used to provide additional motivation with respect to what explains.
    Rather, that law-like idea is used to extract additional features of particular descriptions.
  }
\end{note}



\subsubsection{\progExII{}}


\begin{note}
  \progExI{} observes the key way a description of a \se{} of some event explains `why' the event happens.
  However, \qWhy{} asks about \fingfr{1}, and we have not connected \fingfr{1} to \se{1}.
  There are (at least) two possible ways to argue for a connexion:
  % 
  \begin{itemize}
  \item
    Argue \fingfr{1} are part of some descriptions which explains `why' some event happened.
  \item
    Argue that in some cases what is entailed by a description which explains `why' some event happened \emph{also} explains `why' the event happened, and that \fingfr{1} are entailed by certain descriptions which explain `why'.
  \end{itemize}
  % 
  We opt for the second option.
  For, the second option allows us to easily make general observations about when a \fingfr{} answers \qWhy{} based on descriptions which do not (directly) appeal to \fingfr{1}.

  \progExII{} expands on \progExI{} and argues what is entailed by a description which explains `why' some event happened \emph{also} explains `why' the event happened.
  And later, in \autoref{cha:ros}, we establish the way \fingfr{1} are entailed by certain descriptions.
  Still, we explicitly highlight the relevant connexion between \progExII{} and \fingfr{} after the statement of, and argument for, \progExII{}.
\end{note}


\begin{note}
  \begin{rproposition}{prop:PEentail}{\progExII{}}%
    Given \(\ed{}\) is an \eiw{0} \vAgent{} does \(a\), and the sense of `why' present in \qWhy{}:

    \begin{itenum}
    \item[\emph{If}:]
      There is some \se{} \(\ed{\flat}\) of \(\ed{}\) such that:
      \begin{itemize}
      \item
        \(\edo{\flat}\) being true of \(\edn{\flat}\) (non-vacuously) entails feature \(f\) is true of \(\edn{\flat}\).
      \end{itemize}
    \item[\emph{Then:}]
      Feature \(f\) being true of \(\edn{\flat}\) explains `why' \(\ed{}\) happened.
    \end{itenum}
    \vspace{-\baselineskip}
  \end{rproposition}


  \begin{argument}{prop:PEentail}
    Assume \(\ed{\flat}\) is a \se{} of \(\ed{}\), and feature \(f\) such that \(f\) being true of \(\edn{\flat}\) is (non-vacuously) entailed by \(\edo{\flat}\) being true of \(\edn{\flat}\).

    Now, consider the description which combines of \(\edo{\flat}\) and \(f\): \(\edo{\flat + f}\).
    We claim \(\edo{\flat + f}\) is a \se{} of \(\ed{}\), and then argue \(f\) being true of \(\edn{\flat}\) alone explains `why' \(\ed{}\) happened, in the sense of `why' present in \qWhy{}.
    \medskip

    To see that \(\ed[\flat + f]{\flat}\) is a \se{} of \(\ed{}\), we check both clauses of \autoref{def:se} are satisfied:

    \begin{itemize}
    \item
      Clause~\ref{assu:p:se:prog} is satisfied.

      For, by assumption \(\ed{\flat}\) is such that \(\ed{}\) is in progress.
      So, as \(\edo{\flat + f}\) combines \(\edo{\flat}\) and \(f\), \(\ed[\flat + f]{\flat}\) must be such that \(\ed{}\) is in progress
    \item
      Clause~\ref{assu:p:se:hCon} is satisfied.

      For, \(f\) is (non-vacuously) entailed by \(\edo{\flat}\).
      So, \(\edo{\flat + f}\) captures no more of \(\edn{\flat}\) than \(\edo{\flat}\) implicitly captures.
      Hence, as \(\ed{}\) partly happens as a result of \(\ed{\flat}\), \(\ed{}\) partly happens as a result of \(\ed[\flat + f]{\flat}\).
    \end{itemize}
    %
    So, \(\ed[\flat + f]{\flat}\) is a \se{} of \(\ed{}\).
    \medskip

    Now, as \(\ed[\flat + f]{\flat}\) is a \se{} of \(\ed{}\) it follows by \autoref{prop:PEbasic} that \(\edo{\flat + f}\) being true of \(\edn{\flat}\) explains `why' \(\ed{}\) happened, in the sense of `why' present in \qWhy{}.

    Finally, observe for some thing to explain `why' \(\ed{}\) happened, given the sense of `why' present in \qWhy{}, the thing need not amount to a complete explanation of `why' \(\ed{}\) happened.
    Hence, and particular part of \(\edo{\flat + f}\) being true of \(\edn{\flat}\) explains `why' \(\ed{}\) happened, and specifically \(f\) being true of \(\edn{\flat}\) explains `why' \(\ed{}\) happened.
  \end{argument}

  Given \autoref{prop:PEentail}, if \(\edo{\flat}\) is a \se{} of \(\ed{}\), then any part of \(\edo{\flat}\) explains `why' \(\ed{}\) happened.
  Hence, given \autoref{obs:se-inst} the agent figuring out \rootsConEqExV{6}{3}{2} explains why the agent concluded \pv{\propM{\rootsCon{}}}{\valI{True}} from \(\Phi\) in \autoref{illu:gist:roots:F}.
\end{note}



\subsubsection{\progEx{2} and \qWhy{}}


\begin{note}
  An simple consequence of \autoref{prop:PEentail} clarifies the way we obtain answers to \qWhy{}:

  \begin{rproposition}{sketch:PE:cROS}{\progEx{2}, conclusions, and \fingfr{1}}%
    Given \(\ed{}\) is an \eiw{0} \vAgent{} concludes \(\phi\) has value \(v\) from \(\Phi\):

    \begin{itenum}
    \item[\emph{If}:]
      Conditions \ref{sketch:PE:cROS:a} and \ref{sketch:PE:cROS:b} both hold:
      %
      \begin{enumerate}[label=\arabic*., ref=\arabic*]
      \item
        \label{sketch:PE:cROS:a}
        \(\ed{}\) is an \eiw{0} agent concludes \(\pv{\phi}{v}\) from \(\Phi\).
      \item
        \label{sketch:PE:cROS:b}
        There is some \se{} \(\ed{\flat}\) of \(\ed{}\) such that:
        \begin{itemize}
        \item
          \(\edo{\flat}\) entails it is true of \(\edn{\flat}\) that:
          \fofb{\(\pv{\psi}{v'}\)}{\(\Psi\)} through \(\ed{\flat}\).
        \end{itemize}
      \end{enumerate}
    \item[\emph{Then:}]
      \fingfb{\(\pv{\psi}{v'}\)}{\(\Psi\)} answers \qWhy{}.
    \end{itenum}
    \vspace{-\baselineskip}
  \end{rproposition}

  \begin{argument}{sketch:PE:cROS}
    Assume conditions \ref{sketch:PE:cROS:a} and \ref{sketch:PE:cROS:b} hold.

    By Condition~\ref{sketch:PE:cROS:b}, \fingfb{\(\pv{\psi}{v'}\)}{\(\Psi\)} through \(\ed{\flat}\) explains `why' \(\ed{}\) happened, in the sense of `why' present in \qWhy{}.
    And, by Condition~\ref{sketch:PE:cROS:a}, \(\ed{}\) just is an \eiw{0} agent concludes \(\pv{\phi}{v}\) from \(\Phi\).
    Hence, \fingfb{\(\pv{\psi}{v'}\)}{\(\Psi\)} answers \qWhy{}.
  \end{argument}
\end{note}


\begin{note}
  Given \autoref{sketch:PE:cROS}, the answers to \qWhy{} of interest may be recast as answers to a slightly different, but perhaps a little more precise question:

  \begin{question}{questionWhyV}{\qWhyV{}}%
    Given \(\ed{}\) is an \eiw{0} \vAgent{} concludes \(\pv{\phi}{v}\) from \(\Phi\):

    \begin{itemize}
    \item
      Which \fingfr{1} are such that:
      \begin{itemize}
      \item
        There is some \se{} \(\ed{\flat}\) of \(\ed{}\) such that:
        \begin{itemize}
        \item
          \(\edo{\flat}\) entails is true of \(\edn{\flat}\) that:
          \fofb{\(\pv{\psi}{v'}\)}{\(\Psi\)} through \(\ed{\flat}\).
        \end{itemize}
      \end{itemize}
    \end{itemize}
    \vspace{-1\baselineskip}
  \end{question}

  \noindent%
  For, answers to \qWhyV{} satisfies conditions \ref{sketch:PE:cROS:a} and \ref{sketch:PE:cROS:b} of \autoref{sketch:PE:cROS}.
  Hence, by \autoref{sketch:PE:cROS} the answer to \qWhyV{} is also an answer to \qWhy{}.

  Still, to reduce the overhead of keeping track of variant questions and constraints we stick with \qWhy{}.
\end{note}


\section*{Summary}
\label{sec:summary}

\begin{note}
  This chapter introduced a key idea: \progEx{}.

  The role of \progEx{} is to help identify answers to \qWhy{}.
  Broadly, \qWhy{} asks for (partial) explanations of why some event is an event where an agent concluded \(\pv{\phi}{v}\) from \(\Phi\), rather than an event in which some other thing happened.
  And, the idea of \progEx{} is to identify answers to why a conclusion happened rather than anything else by identifying features of the conclusion when in progress.

  As background for \progEx{}, this chapter clarified the way we understand events, descriptions, and events in progress.

  In short, events are things we refer to by way of descriptions which are true of the event, and events in progress mostly correspond with the kind of thing the (English) progressive is used to talk about.
  E.g., an event of making soup is an \eiw{} soup is made is in progress.
\end{note}


\begin{note}
  \autoref{sketch:PE:cROS} structures the argument to follow.
  Broadly stated, three tasks remain:

  \begin{enumerate}[label=\arabic*., ref=\arabic*]
  \item
    \label{eip:task:1}
    Demonstrate a description of some event entails a \fingfr{} holds.
  \item
    \label{eip:task:2}
    Show some \se{} of an \eiw{0} an agent concludes some \prop{0} \(\phi\) has \val{0} from some \pool{0} \(\Phi\) entails a \fingfr{} holds.
  \item
    \label{eip:task:3}
    Show some \se{} entails a \fofb{\(\pv{\psi}{v'}\)}{\(\Psi\)} even when an agent has not concluded \(\pv{\psi}{v'}\) from \(\Psi\) when the agent concludes \(\pv{\phi}{v}\) from \(\Phi\).
  \end{enumerate}

  Task~\ref{eip:task:1} establishes a \fingfr{0} may explain `why' an event happened, given \progExII{}/\autoref{sketch:PE:cROS}.

  Task~\ref{eip:task:2} establishes \fingfr{1} do explain `why' an event happened, given \progExII{}/\autoref{sketch:PE:cROS} and hence establishes certain answers \qWhy{}.

  And, Task~\ref{eip:task:3} establishes \issueInclusion{} fails to hold.
  For, by \autoref{sketch:PE:cROS} the relevant \fingfr{} between \(\pv{\psi}{v'}\) and \(\Psi\) answers \qWhy{}, and \issueInclusion{} holds only if the agent has also concluded \(\pv{\psi}{v'}\) from \(\Psi\) when the agent concludes \(\pv{\phi}{v}\) from \(\Phi\).
\end{note}


% \subsection{Ex}

% {
% \color{red}
% Here, this example is to be reworked to highlight interesting features of events in progress and \progEx{}.

% Basically, when scoring darts, event in progress through a variety of different things happen.
% Then, as play goes on, this constrains what the agent may do (as the need to score 501).
% E.g. if it's the last throw and the agent has a score of 481, then they need to hit a 20.

% In turn, one the game is over, understand why agent won, as, e.g. they were hitting a 20 on the last throw.
% }

%   \begin{note}
%     \begin{illustration}[Darts]
%       Agent wins at darts just in case there is some action available to the agent, such that if the agent were to perform the action they would be winning at darts.

%       Winning is a complex action.
%       An agent has three dart throws to lower their score from 501 to 0 before play switches to the other player, and play continues until neither player may lower their score further on their next turn (without going past 0).
%       Playing a game is a complex action, as the region of a dartboard an agent wishes to hit changes according to previous throws.
%       For example, if the agent's score is 51 with three throws remaining, the agent will not wish to hit bullseye, as there is no way to reduce their score by a single point using two darts.
%       If the agent goes on to hit 20, then the score of the remaining to darts should equal 31, and so on.
%     \end{illustration}

%     Note, the initial sequence of actions may be more or less arbitrary.
%     It is not possible to score 501 in three or six dart throws, so an agent \emph{could} start by throwing a few darts blindly, so long as they have sufficient skill to recover on subsequent throws.

%     Of course, throwing darts is quite different from concluding, but this note extends.
%     An agent may be concluding a theorem is true even though their first line of enquiry turns out to be a dead end, etc.

%     This is the appeal of the progressive.
%   \end{note}



%%%   Local Variables:
%%%   mode: latex
%%%   TeX-master: "master"
%%%   TeX-engine: luatex
%%%   End:


\chapter{\fc{3}}
\label{cha:fcs}

\nocite{Ryle:1946tu}

\section{Introduction}
\label{cha:fcs:sec:introduction}

\begin{note}
  \autoref{sec:clar:type-of-scen} introduced the general type of \scen{} we are interested in.

  Here, \fc{1}.
  Two key things.
  \autoref{cha:sec:fcs-def}, account of \fc{1}.
  \autoref{cha:fcs:sec:fcs-support}, \fc{1} and support.

  If \fc{}, then relation of support.

  Important idea.
  However, limited.
  Support without witnessing doesn't raise a problem for \issueConstraint{}.
  Rather, support needs to be \emph{why}.

  \autoref{cha:zS} will develop, \check{1} on concluding.

  Main focus is \autoref{cha:sec:fcs-def}.
  What it is for some proposition-value pairing \(\pv{\phi}{v}\) to be a \fc{} from some pool of premises \(\Phi\).
  With the right set-up, \support{} --- focus on \autoref{cha:fcs:sec:fcs-support} --- will be simple.

  Progressive, action which is concluding \(\pv{\phi}{v}\) from \(\Phi\).
  Contrast with ability, `ability to conclude \(\pv{\phi}{v}\) from \(\Phi\)'.
  Argue that there is no clear path.
  Though, implicit upshot is reducing ability to progressive (given assumption).
\end{note}

\begin{note}
  Breakdown
  \begin{itemize}
  \item
    \autoref{cha:sec:fcs-def} definition of \fc{0}.
  \item
    \autoref{cha:fcs:sec:fcs-support}, support.
  \end{itemize}
\end{note}

\begin{note}
  Support.

  Idea here is something distinguished about concluding, but is independent of whether or not the agent has witnessed concluding.

  \autoref{idea:support} and \autoref{idea:support:possible}.

  Noted, independence does not entail relation of support from agent's perspective without concluding.

  Parallel to propositional and doxastic justification.
\end{note}

\section{Intuition}
\label{sec:intuition}

\begin{note}
  General term `foregone conclusion' is ambiguous.
  \begin{itemize}
    \item
    Inevitable results of reasoning.
  \item
    Conclusion which has been settled in advance of reasoning.
  \end{itemize}

  Interest is with the second meaning.
  Two examples of general use.

  % \begin{quote}
  %   [どうぜ]\dots Expresses an attitude of resignation or carelessness on the part of the speaker, in the sense that regardless of what s/he does, the conclusion or outcome is foregone and cannot be changed by the will or effort of an individual.%
  %   \mbox{ }\hfill\mbox{(\citeyear[332--333]{kurufushamashii:2015un})}
  % \end{quote}
  A clear example of the first meaning is found in~\citeauthor{Machover:1996vu}'s~\citetitle{Machover:1996vu}:
w
  \begin{quote}
    I have omitted its proof, but added a detailed analysis of the meaning of the lemma and the reason why its proof works. When this is understood, the proof itself becomes a mere technicality, almost a foregone conclusion.%
    \mbox{ }\hfill\mbox{(\citeyear[viii]{Machover:1996vu})}
  \end{quote}

  \citeauthor{Machover:1996vu} is discussing a proof, and whether or not it is inevitable that one would complete the proof (conclude that the relevant theorem is true) after understanding the lemma and why it works.
  There is no relevant sense in which the truth of the theorem has been settled in advance of reasoning.
  Though, as the proof is somewhat difficult,~\citeauthor{Machover:1996vu} only states the proof is `almost' a foregone conclusion.%
  \footnote{
    The proof is question is of the G\"{o}del-Rosser First Incompleteness Theorem.
    (\citeyear[Cf.][226]{Machover:1996vu})
  }%
  \(^{,}\)
  \footnote{
    For a similar example without qualification, consider the following from~\textcite{Jacquette:2002up}:
    \begin{quote}
    It is nevertheless important to recognize that Russell's evaluation of such sentences as false is predetermined by his existence presuppositional semantics for the ‘existential' quantifier, and by the fact that his logic permits no alternative means of considering the semantic status of sentences ostensibly containing proper names for nonexistent objects.
    This makes it an altogether philosophically foregone conclusion that sentences like ‘Pegasus is winged,' which many logicians would otherwise consider to be true propositions of mythology, are false.%
    \mbox{ }\hfill\mbox{(\citeyear[6]{Jacquette:2002up})}
  \end{quote}
  }

  For an additional example, consider the following from~\citeauthor{Grice:1957vg}'s~\citetitle{Grice:1957vg}:%
  \footnote{
    Same is found in \textcite[219]{Grice:1989uf}.
  }
  \begin{quote}
    He intends the audience's recognition of his intention to produce that response to be effective in producing that response.
    He does not regard it as a foregone conclusion that his action will produce the intended response, whether or not his intention is recognized.\newline
    \mbox{ }\hfill\mbox{(\citeyear[385]{Grice:1957vg})}
  \end{quote}

  In this case, the term `foregone conclusion' is embedded under negation, to highlight that the agent in question entertains the possibility that the agent's action will not produce the intended response.

  By contrast, the following passage from \textcite{Kadane:1996vu} is an example of the second meaning:

  \begin{quote}
    When can a Bayesian select an hypothesis \emph{H} and design an experiment (or a sequence of experiments) to make certain that, given the experimental outcome(s), the posterior probability of \emph{H} will be greater than its prior probably?
    We discuss an elementary result that establishes sufficient conditions under which this reasoning to a foregone conclusion cannot occur.%
    \mbox{ }\hfill\mbox{(\citeyear[1228]{Kadane:1996vu})}
  \end{quote}

  At issue is whether a Bayesian may chose some hypothesis \emph{H} and then guarantee some increase in probability for \emph{H} by running some experiments.
  So, are there cases in which the Bayesian first chooses a hypothesis \emph{H} and then ensures they reason to an increase in the probability of \emph{H}?
\end{note}

\begin{note}
  Our interest is with the first meaning, though narrow.
  To avoid ambiguity with first meaning, write `\fc{0}' rather than `foregone conclusion'.
  That is, the hyphen signifies when we are speaking about the technical term.
\end{note}

\section{\fc{3}}
\label{cha:sec:fcs-def}

\begin{note}[\fc{2} definition]
  We define \(\pv{\phi}{v}\) being a \emph{\fc{0}} as follows:

  \begin{restatable}[\fc{3}]{definition}{definitionForegoneC}
    \label{def:fc}
    For an agent \vAgent{}, and some proposition-value-premises pairing \(\pvp{\phi}{v}{\Phi}\):

    \begin{itemize}
    \item
      \(\pv{\phi}{v}\) is a \emph{\fc{0}} from \(\Phi\), for \vAgent{}.\newline
      \mbox{ }\hfill(equiv.\ \(\pvp{\phi}{v}{\Phi}\) is a \fc{0} for \vAgent{})
    \end{itemize}
    \emph{If and only if}
    \begin{enumerate}[label=]
    \item
      Both~\ref{def:fc:is-pe-good} and~\ref{def:fc:no-pe-bad} are true:
      \begin{enumerate}[label=\alph*., ref=(\alph*)]
      \item
        \label{def:fc:is-pe-good}
        There is \emph{some} \pevent{} \(p\) in which \vAgent{} concludes \(\pv{\phi}{v}\) from \(\Phi\).
      \item
        \label{def:fc:no-pe-bad}
        There is \emph{no} \pevent{} \(p\) in which \vAgent{} concludes some proposition-value-premises pairing which is incompatible with concluding \(\pv{\phi}{v}\) from \(\Phi\).%
        \footnote{
          Incompatible:
          \(\pv{\chi}{v''}\) from \(X\) where:
        If conclude \(\pv{\chi}{v''}\) from \(X\), then does not conclude \(\pv{\phi}{v}\) from \(\Phi\).
        }
      \end{enumerate}
    \end{enumerate}
    \vspace{-\baselineskip}
  \end{restatable}
\end{note}

\begin{note}[Intuition]
  Significant attention will be given to what is means for there to be a potential event in which an agent performs some action in \autoref{cha:sec:fcs-def:potential-events}, below.
  However, the basic features of \autoref{def:fc} follow from substituting `possible' for `\pevent{}'.
  Given this substitution:

  \begin{itemize}[noitemsep]
  \item
    Clause~\ref{def:fc:is-pe-good} ensures that there is some possibility in which the agent to conclude \(\pv{\phi}{v}\) from \(\Phi\).
  \item
    Clause~\ref{def:fc:no-pe-bad} ensures that there is no possibility in which the agent concludes something incompatible with concluding \(\pv{\phi}{v}\) from \(\Phi\).
    Hence, Clause~\ref{def:fc:no-pe-bad} rules out the agent failing to conclude \(\pv{\phi}{v}\) from \(\Phi\) because some incompatible proposition-value-premises pairing is (also) a \fc{0}.
  \end{itemize}

  Intuitively, and inevitable result of possible reasoning.
  For, there is some possibility in which the agent concludes \(\pv{\phi}{v}\) from \(\Phi\) via Clause~\ref{def:fc:is-pe-good}.
  And, at no point prior to concluding could the agent have concluded some other proposition-value-premises pairing which would prevent the agent from concluding \(\pv{\phi}{v}\) from \(\Phi\) via~\ref{def:fc:no-pe-bad}.

  Further, not merely that agent would not prevent on success, but that there is no way to block success.
\end{note}

\begin{note}
  Still, mere possibility is too general.
  There is a possibility in which I immediately obtain a comprehensive understanding of theoretical computer science and settle whether P is equal to NP (or show that the P versus NP problem is undecidable).
  Hence, \fc{1} are defined with respect to \pevent{1}.

  For the moment we will set \pevent{1} aside, and consider a handful of \illu{1} in \autoref{cha:fcs:sec:illu}.
  In \autoref{cha:sec:fcs-def:potential-events} we will provide a fairly detailed account of \pevent{1}.
  (If you would prefer to skip the \illu{1}, please turn to~\autopageref{cha:sec:fcs-def:potential-events}.)
\end{note}

\begin{note}[Neutral perspective]
  \phantlabel{fcs-neutral-perspective}
  Still, before turning to the \illu{1} and \pevent{1}, a final observation:

  Observe that whether \(\pvp{\phi}{v}{\Phi}\) is a \fc{0} is stated from a neutral perspective --- at issue is whether there is a \pevent{} in which the agent concludes.
  Our interest with \fc{1} will be from an agent's perspective, rather than whether \(\pvp{\phi}{v}{\Phi}\) \emph{is} a \fc{0}.%
  \footnote{
    Specifically, whether \(\pv{\phi}{v}\) is a \fc{0} from \(\Phi\) with respect to \vAgent{}, from \vAgent{}' perspective.
  }
  But, by defining \fc{1} from a neutral perspective, we straightforwardly understand whether \(\pvp{\phi}{v}{\Phi}\) is a \fc{0} from the agent's perspective by shifting our perspective to match the agent's perspective.
\end{note}

\section{Illustrations}
\label{cha:fcs:sec:illu}

\begin{note}
  Intuition.
  In particular, proofs.

  Before turning to detailed account of \fc{1} in \autoref{cha:sec:fcs-def}, handful of \illu{1}.

  This section consists of two parts:
  \begin{itemize}
  \item
    \autoref{cha:fcs:sec:illu:yes}, \illu{} where \(\pv{\phi}{v}\) from \(\Phi\) (plausibly) \emph{is} a \fc{}.
  \item
    \autoref{cha:fcs:sec:illu:no}, \illu{} where \(\pv{\phi}{v}\) from \(\Phi\) (plausibly) is \emph{not} a \fc{}.
  \end{itemize}
\end{note}

\subsection{\fc{3}}
\label{cha:fcs:sec:illu:yes}

\begin{note}[Chess I]
  \begin{illustration}[Chess I]
    \label{illu:fc:chess:I}
    Consider the following game state:

    \mbox{ }\hfill%
    \begin{adjustbox}{minipage=\linewidth,scale=.9}
      \centering
      \newchessgame[
      setwhite={pa2,pb2,pc2,pd3,pf2,pg3,ra1,re1,bd4,kg1,qe5},
      addblack={ra8,pa7,ba6,pb5,rc8,pd5,pf2,kg8,qg4,ph7,ph4},
      ]%
      \setchessboard{showmover=false}%
      \chessboard
    \end{adjustbox}%
    \label{fig:chess:easy}%
    \hfill\mbox{ }

    Is possible for White to checkmate in a single move?%
    \footnote{
      \citeauthor{Emms:2000aa}' Puzzle 113 (\citeyear[33]{Emms:2000aa}).
    }
  \end{illustration}
\end{note}

\begin{note}
  \fc{2} of interest:%
  \footnote{
    May also consider whether is possible for White to checkmate in a single move to be a \fc{}, in the sense that it is possible to decide.
    In this case, reduces to solution.
    However, in general decidable may be \fc{} without either answer being a \fc{}.
    In particular, consider an arbitrary first-order formula.
    First-order logic is decidable, however determining which is a different matter.
  }
  \begin{itemize}
  \item
    It is possible for White to checkmate in a single move.
  \end{itemize}
  Clearest way is to do the reasoning, and then observe would not have made a different conclusion.
\end{note}

\begin{note}
  Consider different pieces.
  If like me, first move (\wmove{Qe8}) does not result in checkmate.
  However, do not conclude that it is not possible.
  Other moves to check (e.g\ \wmove{Qf6}, \wmove{Re4}, \wmove{h4}, etc.).

  At some point, consider moving the queen from e5 to h8, which results in checkmate.

  Simple.
  Key observation is that although not immediate, conclude it is possible for White to checkmate in a single move.
  And, would not have concluded otherwise, or indeed concluded something incompatible which would have prevented (for example, that it is possible for Black's king to move to b8 between the two rooks.)

  Perhaps get bored or distracted, and didn't conclude.
  Remains the case that \fc{}.
\end{note}

\begin{note}
  Following from chess, similar structure.
  \begin{illustration}[Sudoku]
    \label{illu:gist:sudoku}
    % https://tex.stackexchange.com/questions/91422/tikz-sudoku-circle-and-connect-with-lines-some-cells
    Consider the following Sudoku puzzle:%
    \footnote{
      From~\textcite[84]{Coussement:2007up}.
    }
    \vspace{\baselineskip}

    \mbox{ }\hfill%
    \begin{adjustbox}{minipage=0.45\linewidth,scale=1}
      \centering
        \begin{tikzpicture}[scale=.5]
          \begin{scope}
            \draw (0, 0) grid (9, 9);
            \draw[very thick, scale=3] (0, 0) grid (3, 3);
            \setcounter{row}{1}
            % Single entries
            \setrow { }{ }{ }  { }{ }{ }  {1}{ }{ }
            \setrow { }{ }{ }  { }{ }{ }  { }{5}{ }
            \setrow {9}{ }{ }  { }{ }{ }  { }{ }{2}
            \setrow { }{ }{3}  { }{2}{ }  { }{ }{ }
            \setrow { }{ }{ }  {8}{ }{ }  {4}{6}{5}
            \setrow { }{4}{ }  { }{5}{9}  { }{ }{8}
            \setrow { }{8}{7}  {2}{3}{1}  { }{4}{6}
            \setrow {2}{1}{ }  {5}{ }{ }  { }{ }{3}
            \setrow {3}{ }{6}  {4}{ }{8}  { }{ }{ }
        \end{scope}
      \end{tikzpicture}
    \end{adjustbox}%
    \hfill\mbox{ }
    \vspace{\baselineskip}
  \end{illustration}

  In contrast to~\autoref{illu:fc:chess:I}, \autoref{illu:gist:sudoku} involves a number of salient \fc{}.
  Specifically, for every empty cell in the Sudoku grid consider the disjunction:
  \[
    \bigvee\{ \phi \text{ is a \fc{}} \mid \phi \in \text{Choices} \}
  \]
  where \(\text{Choices} = \{ i\text{ is the correct number to place in the cell} \mid 1 \leq i \leq 9 \}\).

  For a particular instance:
  \begin{itemize}
  \item
    1 is the correct number to place in the centre cell of the centre sub-grid is a \fc{}.
  \end{itemize}

  In other words, the solution to the Sudoku puzzle is a \fc{}, and each part of the solution to the Sudoku puzzle is a \fc{}.

  Indeed, the relevant \fc{1} follow from a basic understanding the rules of Sudoku, in same way that the possibility for White to checkmate follows from understanding of the rules of chess in \autoref{illu:fc:chess:I}.

  Still, in contrast to \autoref{illu:fc:chess:I}, there is a interesting chance of error.
  For example, accidentally placing 9 above the 8 in the bottom centre sub-grid before observing the 9 in the same column.
  Or, 4 in the top-right of the top-left sub-grid before realising already placed 4 in top-left of the top-left sub-grid.

  However, errors do not (necessarily) amount to conclusions.
  One may make an error while completing the Sudoku puzzle, but refrain from concluding that the mistaken number is the correct number to place in the cell.
  Indeed, given the possibility of error one may only conclude \(i\) is the correct number to place in cell \(c\) only when all cells have been filled and they have ensured there are no errors.

  Though, if an agent is less cautious and is inclined to immediately conclude that \(i\) is the correct number to place in cell \(c\) then each instance of the disjunction may be false.
  For, prior to attempting the puzzle there is a possibility that the agent conclude \(i\) is the correct number to place in cell \(c\) and there is a possibility that the agent may conclude \(j\) is the correct number to place in cell \(c\), where \(i \ne j\).
  Hence, there is the possibility that the agent may conclude either of two (or more) conflicting proposition-value pairs.
\end{note}

\begin{note}
  Games.
  Clear set of rules, and understanding of rules leads to answers to certain questions being determined.
  Mathematics and logic.
\end{note}

\begin{note}
  \begin{illustration}[Fraction]
    \label{illu:fc:surds}
    \[\frac{(3 + \sqrt{3})^{2} + \sqrt{6}^{2} - (2\sqrt{3})^{2}}{2(3 + \sqrt{3})\sqrt{6}} = \frac{1}{\sqrt{2}}\]
  \end{illustration}

  Granting knowledge of a handful of equalities, beyond basic addition and subtraction,%
  \footnote{
    \(\sfrac{ab}{ac} = \sfrac{b}{c}\),
    \(\sqrt{a b} = \sqrt{a}\sqrt{b}\), and
    \((a + b)^{2} = (a^{2} + 2ab + b^{2})\).
  }
  whether or not the equation is true a \fc{}, and further the truth of the equation is a \fc{}.%
  \footnote{
    \label{illu:fc:surds:fn}
    First, consider the numerator.
    Each element of the numerator may be rewritten as follows:
    \((3 + \sqrt{3})^{2} = 12 + 6\sqrt{3}\), \(\sqrt{6}^{2} = 6\), \((2\sqrt{3})^{2} = 12\).
    By summing the elements we obtain \(6\sqrt{3} + 6\).
    Hence, by rewriting, the numerator may be replaced with, \(2(3\sqrt{3} + 3)\).

    Now consider the denominator.
    Observe we may cancel multiplication by \(2\) from both the numerator and denominator.
    Further, observe \(\sqrt{6} = \sqrt{2}\sqrt{3}\).
    Hence, by distributing we  obtain, \((3\sqrt{3} + \sqrt{3}\sqrt{3})\sqrt{2}\).
    Likewise, observe \(\sqrt{3}\sqrt{3} = \sqrt{9} = 3\).
    Hence, by rewriting the denominator reads \((3\sqrt{3} + 3)\sqrt{2}\).
    As both the numerator and denominator contain \((3\sqrt{3} + 3)\), we may cancel to obtain the desired equality.
  }
  Though, if like me a few dead-ends before stumbling across the path to the solution.
  However, don't conclude any intermediary miscalculations.
  And, keep going until solution is clear.
\end{note}

\begin{note}
  \begin{illustration}[Modal logic I]
    \label{illu:fc:logic:CR}
    The modal system obtained from adding \(\Diamond\Box p \rightarrow \Box\Diamond p\) as an axiom to \(\mathbf{K}\) is canonical for the Church-Rosser property.

    I.e. the canonical model \(W,R,V\) for \(\mathbf{K} + \Diamond\Box p \rightarrow \Box\Diamond p\) is such that \(\forall s,t,u((Rst \land Rsu) \rightarrow \exists v(Rtv \land Ruv))\).
  \end{illustration}

  \autoref{illu:fc:logic:CR} is a \fc{} for me.
  Though, in contrast to the previous \illu{1}, I think there is a reasonable change that \autoref{illu:fc:logic:CR} is not a \fc{} for you.

  Fairly routine, but two important things.
  First, grasp on the relevant concepts.
  If you are unaware of how to construct canonical models for normal modal logics, then unlikely that you will complete the relevant proof.
  Second, sufficient familiarity with the relevant concepts.
  The proof is mostly straightforward, though some care needs to be taken in showing that the canonical model for \(\mathbf{K} + \Diamond\Box p \rightarrow \Box\Diamond p\) has the Church-Rosser property.
  Proof by contradiction is my preferred way of obtaining the result, but this requires keeping certain facts about the canonical model in mind.%
  \footnote{
    A slightly more interesting variation is showing that \(\mathbf{K} + \Diamond\Box p \rightarrow \Box\Diamond p\) is (strongly) complete with respect to the class of frame which have the Church-Rosser property without detour via a canonical model.
  }

  Similar features as \illu{1} given above.

  In particular, perhaps clearer than \autoref{illu:gist:sudoku} and \autoref{illu:fc:surds} in terms of mistakes.
  For, go down some wrong path, still will not conclude until counterexample.
  And, this is very hard to get.
\end{note}

\begin{note}
  Four \illu{}.
  Share common characteristic.
  Result of deductive reasoning with more-or-less explicit collection of rules.

  These characteristics are not (nor any other shared characteristic) required.
  Clear account of why \(\pv{\phi}{v}\) from \(\Phi\) is a \fc{}.

  What is required is available information, conclusion, no divergence.
\end{note}

\begin{note}[Non-deductive \illu{1}]
  The following is a simple \illu{} involving non-deductive conclusion:
  \begin{illustration}[Sunny days]
    It's mid summer in the Bay Area.
  \end{illustration}
  For me, it is a \fc{} that it will not rain tomorrow.

  Of course, I recognise there is a possibility that it \emph{may} rain tomorrow.
  However, I haven't checked the weather forecast, and with no information to the contrary I see no way of \emph{failing} to conclude that tomorrow will be sunny.
  You may object, and perhaps I am too quick to conclude that it will not rain tomorrow.

  Still, no matter the gravitas with which I consider the possibility of rain, I am sufficiently committed to some uniformity principle that the principle, combined with past experience, lead me to conclude that it will be sunny tomorrow.
  Hence, prior to reasoning, the truth of the proposition is a \fc{}.%
  \footnote{
    Same extends to various skeptical hypotheses.
    Entertain the possibility that there is no external world, but nothing that prevents me from concluding that there is an external world.
    Though, your perspective on such issues may differ.
  }

  Note, whether or not it rains tomorrow has no bearing on whether or not it is a \fc{} (for me) that it will not rain tomorrow.
  What happens in the future has no direct bearing on what I may (or may not) conclude in the present.%
  \footnote{
    Consider~\citeauthor{Russell:1912th}'s chicken\dots (Cf.~\citeyear[63]{Russell:1912th})
  }
\end{note}

\begin{note}[Poppies]
  We conclude the \illu{1} with a slightly more speculative \illu{0}:
  \begin{illustration}[Poppies]
    \mbox{ }
    \vspace{-\baselineskip}
    \begin{quote}
      Was Tarquinius Superbus in seinem Garten mit den Mohnköpfen sprach, verstand der Sohn, aber nicht der Bote.

      [What Tarquinius Superbus said in the garden by means of the poppies, the son understood but the messenger did not].\newline
    \mbox{ }\hfill\mbox{(Cf.~\cite[3]{Kierkegaard:1983ta}, and~\cite[190]{Hamann:1822vp})}
  \end{quote}
  \vspace{-\baselineskip}
  \end{illustration}
  The above quite is from the epigraph to~\citeauthor{Kierkegaard:1983ta}'s \hyperlink{cite.Kierkegaard:1983ta}{Fear and Trembling}.
  \hyperlink{cite.Kierkegaard:1983ta}{H.\ Hong and E.\ Hong} detail the relevant background:

  \begin{quote}
    When the son of Tarquinius Superbus had craftily gotten Gabii in his power, he sent a messenger to his father asking what he should do with the city.
    Tarquinius, not trusting the messenger, gave no reply but took him into the garden, where with his cane he cut off the flowers of the tallest poppies.
    The son understood from this that he should eliminate the leading men of the city.%
    \mbox{ }\hfill\mbox{(\citeyear[339]{Kierkegaard:1983ta})}
  \end{quote}
  That he should eliminate the leading men of the city was a \fc{0} for Superbus' son, but not for the messenger.
  Or, at the very least Superbus \emph{expected} the command to eliminate the leading men of the city to be a \fc{} for his son.
\end{note}

\subsection{Not clearly \fc{1}}
\label{cha:fcs:sec:illu:no}

\begin{note}
  Provided a handful of (plausible) instances of knowing whether which (plausibly) involve \fc{1}.
  A pair of (plausible) instances whether which (plausibly) do not involve \fc{1}.
\end{note}

\begin{note}[ML II]
  \begin{illustration}[Modal logic II]
    \label{illu:fc:ML2}
    The modal system \(\mathbf{GL} = \mathbf{K} + \Box(\Box p \rightarrow p) \rightarrow \Box p\) is weakly complete with respect to the class of finite strict partial orders (that is, the class of finite irreflexive transitive frames).
  \end{illustration}

  \autoref{illu:fc:ML2} is similar in structure to \autoref{illu:fc:logic:CR}.
  Indeed, both proofs involve constructing a canonical model.
  The key distinguishing feature of \autoref{illu:fc:ML2}, however, is the difficulty of establishing the canonical model has the desired properties.
  In particular, the general method I keep in mind for proving the relevant result requires a syntactic proof that \(\vdash_{\mathbf{GL}} \Box p \rightarrow \Box \Box p\).
  And, as I have failed to recall the relevant syntactic on sufficient occasion, I do not consider the result a \fc{0} from my understanding of modal logic.

  Hence, the result (plausibly) fails to be a \fc{} from my understanding of modal logic because there is no guarantee that I would provide a proof if I set out to do so.

  On the other hand, I have completed the relevant proof a sufficient number of times.
  So, the result is a \fc{0} from whatever premises are associated with my memory.
\end{note}

\begin{note}[Chess II]
  Observation that absence of \fc{} due to failure to conclude extends to other cases.
  What follows is a more difficult chess problem.
  \begin{illustration}[Chess II]
    \label{illu:fc:chess:II}
    Consider the following game state:

    \mbox{ }\hfill%
    \begin{adjustbox}{minipage=\linewidth,scale=0.9}
      \centering
      \newchessgame[
      setwhite={ka5,pa3,pb4,pc4,pe5,pf6,bg5,bh5},
      addblack={pa6,pb7,pc6,pe6,pf7,kc7,nd7,nd4},
      ]%
      \setchessboard{showmover=false}%
      \chessboard
    \end{adjustbox}%
    \label{fig:chess:intro}%
    \hfill\mbox{ }

    It is possible for Black to checkmate in four moves?%
    \footnote{
      \citeauthor{Emms:2000aa}' Puzzle 150 (\citeyear[33]{Emms:2000aa}).
    }
  \end{illustration}
  As with \autoref{illu:fc:ML2} it is plausible that I would not conclude that it is possible for Black to checkmate in four moves or conversely.

  Though perhaps the bound is too low.
  If I gave it my all and attempted to work my way though all the possibilities present it may be the case that I conclude either way.
  Still, there are a lot of moves to consider, and I lack any intuition about which is correct.%
  \footnote{
    \citeauthor{Emms:2000aa} provides the following solution:
    \begin{quote}
      \variation{1... Nb6!}
      (threatening \variation{2... Nb3\#})
      \variation{2. b5}
      (or \variation{2. Bd1 Nxc4+} \variation{3. Ka4 b5\#})
      \variation{2... c5!}
      \variation{3. bxa6 Nxc4+}
      \variation{4. Ka4 b5\#}
      \textbf{(0-1)}%
      \mbox{}
      \hfill
      (\citeyear[46]{Emms:2000aa})
    \end{quote}
    My statement above remains true---I don't have sufficient background to parse this solution.
  }
  And, if you think I am doing myself a disservice, then a variant of \autoref{illu:fc:chess:II} may be restated with and increase number of moves.
\end{note}

\begin{note}
  Conflicting conclusions.

  \illu{3} may be obtained by taking a proposition-value pairing which conflicts with \fc{}.
  For example, consider again \autoref{illu:fc:surds}.
  The following is \emph{not} a \fc{}
  \[\frac{(3 + \sqrt{3})^{2} + \sqrt{6}^{2} - (2\sqrt{3})^{2}}{2(3 + \sqrt{3})\sqrt{6}} = \frac{1}{\sqrt{3}}\]
  For, by applying the reasoning outlined in \autoref{illu:fc:surds:fn}, I would conclude the left hand side of the equation is equal to \(\sfrac{1}{\sqrt{2}}\).
\end{note}

\begin{note}
  \begin{illustration}[Knowing whether and belief]
    \citeauthor{Barker:1975un} suggests the following two principles hold with respect to knowing whether:%
  \footnote{
    \citeauthor{Barker:1975un} also, as far as I can tell, endorses the principles.
  }
    \begin{enumerate}[label=(\Alph*), ref=(\Alph*), noitemsep]
    \item
      \label{Barker:1975un:A}
      If \emph{S} knows whether \emph{p} and \emph{S} believes that \emph{p}, then \emph{p}.
    \item
      \label{Barker:1975un:B}
      If \emph{S} knows whether \emph{p} and \emph{S} believes that not-\emph{p}, then not-\emph{p}.\newline
      \mbox{ }\hfill\mbox{(\citeyear[281]{Barker:1975un})}
    \end{enumerate}
  \end{illustration}
  I suggest neither principle is \fc{}, as you may conclude counterexamples exist to both.%
  \footnote{
    For example, consider two agents, \emph{A} and \emph{B} playing chess where each move is timed.
  It's the end game, and \emph{A} believes that \emph{B} has a winning strategy.
  Further, \emph{A} (plausibly) knows whether \emph{B} has a winning strategy.
  For, an observer has determined whether or not \emph{B} has a winning strategy, and \emph{A} is capable of tracing the reasoning of the observer.
  So, if \ref{Barker:1975un:A} holds then \emph{B} has a winning strategy.
  But, the observer knows that \emph{B} \emph{does not} have a winning strategy, and \emph{A}'s belief is mistaken.
  }
\end{note}


\section{\pevent{3}}
\label{cha:sec:fcs-def:potential-events}

\begin{note}
  \autoref{def:fc} appeals to~\ref{def:fc:is-pe-good} the existence of some \pevent{} and~\ref{def:fc:no-pe-bad} the non-existence of some \pevent{}.

  However, \autoref{def:fc} does not rely on anything more than existential quantification.
  The choice is deliberate.
  We given necessary and sufficient conditions for the \emph{existence} of some \pevent{} in terms of
  \begin{enumerate*}[label=(\roman*)]
  \item
    actions available to the agent, and
  \item
    truth conditions for the progressive.
  \end{enumerate*}
\end{note}

\begin{note}[\pevent{2} definition]
  We define a \pevent{} as follows:
  \begin{restatable}[\pevent{3}]{definition}{definitionPEvent}
    \label{def:potenital-event}
    For an agent \vAgent{} and action description \(\alpha\):
    \begin{itemize}
    \item
      There is a \pevent{} \(p\) in which \vAgent{} \(\alpha\)s
    \end{itemize}
    \emph{if and only if}
    \begin{enumerate}[label=]
    \item
      Both~\ref{def:PE:action} and~\ref{def:PE:prog} are true:
      \begin{enumerate}[label=\alph*., ref=(\alph*)]
      \item
        \label{def:PE:action}
        There is some action \(a\) that \vAgent{} may immediately perform.
      \item
        \label{def:PE:prog}
        \(\text{Prog}(e, \alpha)\) would be true in the event \(e\) of \vAgent{} doing \(a\).
      \end{enumerate}
    \end{enumerate}
    Where \(\text{Prog}(e, \alpha)\) stands for the progressive from of \(\alpha\) when evaluated with respect to \(e\).%
    \footnote{
      I.e.\ \(\text{Prog}(e, \alpha)\) is true \emph{iff} event \(e\) is an event of \(\alpha\)ing.
      See,~\textcite{Richards:1981wo},~\textcite{Portner:2011vi}, etc.
    }
  \end{restatable}

  In short,~\autoref{def:potenital-event} states that there is a \pevent{} in which an agent performs some action \(\alpha\) just in case there is some action the agent may (immediately) perform which would result in the agent \(\alpha\)ing.
\end{note}

\begin{note}[Division of labour between the clauses]
  The division of labour between clauses~\ref{def:PE:action} and~\ref{def:PE:prog} is, in reverse order:
  \begin{itemize}[noitemsep]
  \item
    Clause~\ref{def:PE:prog} captures a sense of possibility, via the progressive, such that that the agent \(\alpha\)s (but in such a way that how the agent \(\alpha\)s is not necessarily settled by the action).
\item
  Clause~\ref{def:PE:action} distinguishes the existence of a \pevent{} from the agent \(\alpha\)ing by existential quantification over actions, but binds the existence of a \pevent{} to circumstances be restriction to immediate actions.
  \end{itemize}
\end{note}

\begin{note}
  Following section \autoref{cha:sec:fcs-def:ability} will get problems with ability.
  Then, \autoref{cha:sec:fcs-def:progressive}, develop understanding of the progressive by examining and modifying \citeauthor{Landman:1992wh}'s (\citeyear{Landman:1992wh}) account of the progressive.
\end{note}

\subsection{The progressive}

\begin{note}[Interest with the progressive]
  Our interest with the progressive is due to the delicate sense of possibility required for a sentence stating an event in the progressive to be true.

  \phantlabel{imperfective-paradox:intro}
  Perhaps the clearest example is the `imperfective paradox' (\citeyear[cf.][Ch.3.1]{Dowty:1979vq}).

  \citeauthor{Bach:1986tb} summarises:
  \begin{quote}
    [H]ow can we characterize the meaning of a progressive sentences like \ref{Bach:impP:17} on the basis of the meaning of a simple sentence like \ref{Bach:impP:18} when \ref{Bach:impP:17} can be true of a history without \ref{Bach:impP:18} ever being true?
    \begin{enumerate}[label=(\arabic*), ref=(\arabic*)]
      \setcounter{enumi}{16}
    \item
      \label{Bach:impP:17}
      John was crossing the street.
    \item
      \label{Bach:impP:18}
      John crossed the street.%
      \mbox{ }\hfill\mbox{(\citeyear[12]{Bach:1986tb})}
    \end{enumerate}
  \end{quote}

  No completion is required, and often some surprise.
  Something unexpected happened while John was crossing the street.
  Sense of inertia associated with the agent \(\alpha\)ing.

  Expectation that that John reaching the other side of the street does not reduce to \(\{\text{logical}, \text{metaphysical}, \text{nomic}, \dots\}\) possibility.

  For, suppose John is sitting a multiple choice exam.
  To pass the exam John only needs to chose some number of correct choices.
  It is certainly logically, metaphysically, and nomically possible that John chooses a sufficient number of correct choices.
  However, it does not follow that John is passing the exam.%
  \footnote{
    See also Igal Kvart's example of Mary wiping out the Roman army (\cite[18]{Landman:1992wh}).
  }

  Likewise, there is no simple relation to counterfactuals.
  Consider a scenario in which John is passing the exam without external help.
  Then, a classmate slips John some answers, which John then uses.
  It is no longer true that John is passing the exam without external help.
  And, in the closest possible world where the classmate does not slip John answers, it need not be true that John passes the exam without external help.
  For, if John is surrounded by students of a similar mindset then it is plausible that the in closest possible world a different classmate slips John the same answers.
\end{note}

\begin{note}
  Way the modality functions is tied to the event.

  \citeauthor{Dowty:1979vq} adds:
  \begin{quote}
    Notice, furthermore, that to Say that John was drawing a circle is not the same as saying that John was drawing a triangle, the difference between the two activities obviously having to do with the difference between a circle and a triangle.
    Yet if neither activity necessarily involves the existence of such a figure, just how are the two to be distinguished?%
    \mbox{ }\hfill\mbox{(\citeyear[133]{Dowty:1979vq})}
  \end{quote}

  As \citeauthor{Dowty:1979vq} highlights, event is sufficiently specific to determine some outcome over some other.%
  \footnote{
    Though, the force of \citeauthor{Dowty:1979vq}'s observation is perhaps clearer by substituting `square' for `circle'.
    For, straight line\dots
  }
  So, the truth of the progressive doesn't require completion and doesn't require significant progress toward completion.
\end{note}

\begin{note}
  \autoref{def:potenital-event} relies on important (but common)%
  \footnote{
    See, for example:
    \textcite{Bennett:1972uw},
    \textcite{Dowty:1979vq},
    \textcite{Parsons:1990aa},
    \textcite{Landman:1992wh}, and
    \textcite{Portner:1998um}.

    However,~\autoref{assu:prog-modal-shift} is denied by~\textcite{Szabo:2004ul}.
    \citeauthor{Szabo:2004ul} writes:
    \begin{quote}
      Sometimes we are \emph{doing} things even though there is no real chance that we could get them \emph{done}, and this is true even if we abstract away from the possibility of miraculous intervention.%
      \mbox{ }\hfill\mbox{(\citeyear[40]{Szabo:2004ul})}
    \end{quote}
    To illustrate, \citeauthor{Szabo:2004ul} denies~\ref{Szabo:Arch} is necessarily false:
    \begin{quote}
      \begin{enumerate}[label=(\arabic*), ref=(\arabic*)]
        \setcounter{enumi}{12}
      \item
        \label{Szabo:Arch}
        As the architect was building the cathedral he knew that, although he would be building it for another year or so, he couldn't possibly complete it.%
        \mbox{ }\hfill\mbox{(\citeyear[38]{Szabo:2004ul})}
      \end{enumerate}
    \end{quote}
    Though,~\ref{Szabo:Arch} seems false to me, without some priming.
    And, the only priming on which~\ref{Szabo:Arch} reads true involves interpreting the architect's knowledge from the architect's perspective, allowing a failure of factivity, thus allowing the cathedral to be built.

    Still, \autoref{assu:prog-modal-shift} is an assumption.
    The goal is not to tie potential to progressive, but to evaluation associated with the progressive granting assumption.
  }
  assumption regarding the progressive.

  \begin{assumption}[Progressive perfection]
    \label{assu:prog-modal-shift}
    For any event \(e\) and action description \(\alpha\):
    \begin{enumerate}
    \item[\emph{If}:]
      \begin{enumerate}[label=\alph*., ref=(\alph*)]
      \item
        \(\text{Prog}(e, \alpha)\) is true.
      \end{enumerate}
    \item[\emph{Then}:]
      \begin{enumerate}[label=\alph*., ref=(\alph*), resume]
      \item
        There is some possible event \(e'\) such that \(e'\) is a continuation of \(e\) and \(\alpha\) is true of \(e'\).
      \end{enumerate}
    \end{enumerate}
    \vspace{-\baselineskip}
  \end{assumption}

  \autoref{assu:prog-modal-shift}, shift evaluation to some possible event in which something related is true.

  Possible here is arbitrary.
  Important is continuation.

  Though, in same way it is common to restrict attention to some sense of possibility via an adjective, we may speak instead of( event-)continuative-possibility.

  So, task of an account of the progressive is to narrow the relevant sense of continuative-possibility.
  \autoref{assu:prog-modal-shift} holds that success is a necessary condition on continuative-possibility.
  Applied, in particular, to concluding, \autoref{assu:prog-modal-shift} holds that a agent is concluding \(\pv{\phi}{v}\) from \(\Phi\) only if there is some continuative-possibility in which the agent concludes \(\pv{\phi}{v}\) from \(\Phi\).

  Here, \fc{}.
  concluding \(\pv{\phi}{v}\) from \(\Phi\).
  Concludes \(\pv{\phi}{v}\) from \(\Phi\).
  \fc{2}.
\end{note}

\begin{note}
  Paired with choice, allows complex `incomplete' actions.
  Again, progressive develops.

  \begin{illustration}[Darts]
    There is a \pevent{} in which agent scores 180 at darts just in case there is some action available to the agent, such that if the agent were to perform the action they would be scoring 180 at darts.
  \end{illustration}

  Slightly more interesting.
  Determine the available actions.
  Though, similar, no guarantee.
  Hand is knocked at point of release, still scoring.

  Scoring 180 is a complex action.
  Though, interesting.
  First throws don't matter.

  Again, key idea is that sufficient understanding of progressive.

  And, case of interest:

  \begin{illustration}[Concluding]
    There is a \pevent{} in which agent concludes \(\pv{\phi}{v}\) from \(\Phi\) just in case there is some action available to the agent, such that if the agent were to perform the action they would be concluding \(\pv{\phi}{v}\) from \(\Phi\).
  \end{illustration}

  What is it to be concluding something.
  Like crossing the road, fail to complete.
  Like darts, recover from a bad opening.
\end{note}

\begin{note}
  Intuitive distinction between which actions may and may not perform.

  However,~\ref{def:PE:action} without.
  Allow arbitrary division of actions, what matters is immediate.

  Then, agent doing \(a\) is in progressive, so make sure that doing \(a\) is also instance of \(\alpha\).
\end{note}


\begin{note}
  Still, no full account of the progressive.
  Quite difficult.
  Progressive is familiar, intuitive understanding.
  Work through in sufficient detail to be useful.

  A little on choice.
  Then, highlight issue with ability.
  Then, present and modify \citeauthor{Landman:1992wh}'s (\citeyear{Landman:1992wh}) account of the progressive.
\end{note}

\paragraph*{Summary}

\begin{note}[Summarising]
  To summarise the preceding:
  We began with the definition of a \fc{} (\autoref{def:fc}).
  Definition of a \fc{} relies of the idea of a \pevent{}.
  And, defined \pevent{} in terms of the truth of the progressive aspect applied to a minimal event.

  The exact details of \pevent{} depends on progressive.
  \autoref{cha:sec:fcs-def:progressive-landman}, \citeauthor{Landman:1992wh}'s (\citeyear{Landman:1992wh}) account of the progressive, reconstructed with (selected) observations from \textcite{Szabo:2004ul}.

  However, gnarly.
  Before turning to the progressive, consider ability.
  The following section --- \autoref{cha:sec:fcs-def:ability} --- will raise difficulties with this suggestion.%
  \footnote{
    And implicitly suggest that any sense of ability sufficient for purpose may be analysed in terms of progressive.
  }
\end{note}

\subsection{\fc{3} and ability}
\label{cha:sec:fcs-def:ability}

\begin{note}
  Alternative suggestion is to say \pevent{} just in case agent `can \(\alpha\)' or `has the ability to \(\alpha\)'.
  Preferable, ability.

  \begin{quote}
    \(\pv{\phi}{v}\) from \(\Phi\) is a \fc{} just in case agent has the ability to conclude \(\pv{\phi}{v}\) from \(\Phi\) (and has the ability to avoid concluding something incompatible).
  \end{quote}
\end{note}


\begin{note}
  \begin{itemize}[noitemsep]
  \item
    \autoref{cha:sec:fcs-def:ability:abil-gener-spec}, distinguish general and specific abilities.
  \item
    \autoref{cha:sec:fcs-def:ability:past}, specific ability and the past.
  \item
    \autoref{cha:sec:fcs-def:ability:control-intuition} `Control'.
  \end{itemize}
\end{note}

\subsubsection{Ability, general and specific}
\label{cha:sec:fcs-def:ability:abil-gener-spec}

\begin{note}
  Particular sense of ability.

  Recall \autoref{illu:fc:ML2}.

  In general, not a \fc{} that \(\mathbf{GL}\) is weakly complete with respect to the class of finite strict partial orders.
  For, method relies on a syntactic proof \(\vdash_{\mathbf{GL}} \Box p \rightarrow \Box \Box p\).

  In this respect, it seems I do not have the ability to prove \(\mathbf{GL}\) is weakly complete with respect to the class of finite strict partial orders.

  However, if I have just (by some luck) completed or (by some studying) rehearsed a syntactic proof \(\vdash_{\mathbf{GL}} \Box p \rightarrow \Box \Box p\), then the relevant theorem is a \fc{}.

  In short, it may be true that \(\pvp{\psi}{v'}{\Psi}\) is a \fc{} for an agent while it is false that the agent has the ability to conclude \(\pv{\psi}{v'}\) from \(\Psi\).
\end{note}

\begin{note}
  Still, while there may not be an \emph{immediate} link, whether or not \(\pvp{\psi}{v'}{\Psi}\) is a \fc{} may still reduce to ability, when ability is appropriately understood.

  \phantlabel{ability-g-s-dist}%
  \nocite{Maier:2018uo}
  For, we may distinguish between `general', `categorical' or `global' abilities and `specific' or `local' abilities.

  Following \textcite[2]{Whittle:2010wr} the distinction is roughly as follows:%
  \footnote{
    Though, see~\textcite[esp.\ \S4]{Kittle:2015tb} and~\textcite[1--2]{Kikkert:2022wp} for additional discussion.%
  }
  \begin{itemize}[noitemsep]
  \item
    General (or global) abilities concern `what an agent is able to do in a large range of circumstances', while
  \item
    Specific (or local) ability concern `what the agent is able to do now, in some particular circumstances'.
  \end{itemize}

  General is just given in terms of specific.
  Not conversely, where specific is general and circumstances permit.%
  \footnote{
    For an example of this approach, see \citeauthor{Austin:1961vz}'s (\citeyear{Austin:1961vz}) discussion of `categorical' abilities and opportunities:

    \begin{quote}
      Consider the case where what we wish to assert is that somebody had the opportunity to do something but lacked the ability---`He could have smashed that lob, if he had been any good at the smash':
      here the \emph{if}-clause, which may of course be suppressed and understood, relates not to opportunity but to ability.
      \dots
      `He could have read \emph{Emma}, if he had had a copy', does seem to assert `categorically' that he had a certain ability, although he lacked the opportunity to exercise it.%
      \mbox{ }\hfill\mbox{(\citeyear[177]{Austin:1961vz})}
    \end{quote}
  }

  Example of what \textcite{Hackl:1998tt} terms `opportunity-can' (\citeyear[14]{Hackl:1998tt}):

  \begin{quote}
    \begin{enumerate}
    \item[(92)]
      \begin{enumerate}[label=\alph*., ref=(\alph*)]
      \item
        \label{Hackl:OC:a}
        A star gazer can see the solar eclipse of this year from the Cayman islands.\newline
        So if you were a star gazer and if you were on the Cayman islands at the right time you would see this year's solar eclipse.
      \item
        \label{Hackl:OC:b}
        John can see Mary from where he is standing.\newline
        So if you were standing in his place, you would see Mary.
      \end{enumerate}
    \end{enumerate}

    [\ref{Hackl:OC:b}] says that whoever is in this situation located at John's position and has normal eyesight and directs his/her gaze towards Mary will succeed in seeing Mary.%
    \mbox{ }\hfill\mbox{(\citeyear[39]{Hackl:1998tt})}
  \end{quote}
  \citeauthor{Hackl:1998tt}'s analysis straightforwardly extends to \ref{Hackl:OC:a}:
  A star gazer who is in the Cayman islands at the right time this year and looks for the solar eclipse will succeed in seeing the solar eclipse.

  So, a tentative proposal is to understand whether or not \(\pvp{\psi}{v'}{\Psi}\) is a \fc{} for an agent in terms of whether or not the agent has the \emph{specific} ability to conclude \(\pv{\psi}{v'}\) from \(\Psi\).

  Hence, we set aside \citeauthor{Austin:1961vz}'s `categorical' ability.
  Likewise we set aside `general' accounts of ability such as~\citeauthor{Carter:2021wd}'s~(\citeyear{Carter:2021wd}) `fallibilist',~\citeauthor{Kikkert:2022wp}'s~(\citeyear{Kikkert:2022wp}) `robust', and \citeauthor{Maier:2013vk}'s (\citeyear{Maier:2013vk}) `general' account, among others.

  Two issues.
  Specific ability and the past.
  `Control'

  Discussion will centre around \textcite{Boylan:2020aa}.
  Clear that specific ability (\citeyear[23, fn.3]{Boylan:2020aa})
\end{note}

\subsubsection{(Specific) ability and the past}
\label{cha:sec:fcs-def:ability:past}

\begin{note}
  First is specific ability and what actually happens.
  Two entailments.
  First, \BoyPS{} following \textcite{Boylan:2020aa}.
  Second, \BoyPSC{} the converse of \BoyPS{}.

  Combined, had the ability to if and only if did.

  Embedded in the past.
  So, doesn't say too much.
  However, difficulty with this is \fc{} depends on something not happening.
\end{note}

\begin{note}
  The first entailment is termed `\BoyPS{}'.

  \begin{enumerate}[label=]
  \item
    \label{Boylan:Past-Success}
    \BoyPS{}: \(\text{Past}(S\text{ does }\phi) \Rightarrow \text{Past}(S\text{ is able to }\phi)\)%
    \mbox{ }\hfill\mbox{(\citeyear[\S1.1]{Boylan:2020aa})}
  \end{enumerate}

  \citeauthor{Boylan:2020aa} motivates \BoyPS{} in the following way:
  \begin{quote}
    \begin{quote}
      \textbf{Fluky Dartboard}.
      I am a terrible dartplayer.
      I struggle to even hit the board whenever I take a shot.
      However, I take my shot and I flukily hit the bullseye.
    \end{quote}

    Once I have taken the shot and hit the bullseye, I can compellingly argue:

    \begin{enumerate}
      \setcounter{enumi}{2}
    \item
      I hit the bullseye on that throw.\newline
      So, I was able to hit the bullseye on that throw.
    \end{enumerate}

    If you know that I have been successful, you must concede I was able to.%
    \mbox{ }\hfill\mbox{(\citeyear[2]{Boylan:2020aa})}
  \end{quote}

  Intuitions regarding \citeauthor{Boylan:2020aa}'s case may be unclear.
  However, recall we are interested in \emph{specific} ability.
  Therefore, the argument provided is consistent with \citeauthor{Boylan:2020aa} failing to have the \emph{general} ability to hit the bullseye.%
  \footnote{
    \textcite{Bhatt:2008aa} observes:
    \begin{quote}
      \begin{enumerate}[label=(\arabic*)]
        \setcounter{enumi}{314}
      \item
        (from~\cite{Thalberg:1969ta})
        \begin{enumerate}[label=\alph*., ref=(315\alph*)]
        \item
          \label{Bhatt:Thal-a}
          Yesterday, Brown hit three bulls-eyes in a row.
          Before he hit three bulls-eyes, he fired 600 rounds, without coming close to the bullseye; and his subsequent tries were equally wild.
        \item
          \label{Bhatt:Thal-b}
          Brown was able to hit three bulls-eyes in a row.
        \item
          \label{Bhatt:Thal-c}
          Brown had the ability to hit three bulls-eyes in a row.
        \end{enumerate}
      \end{enumerate}
      From~\ref{Bhatt:Thal-a}, we can conclude~\ref{Bhatt:Thal-b} but not~\ref{Bhatt:Thal-c}.
      Brown could have hit the target three times in a row by pure chance and he does not need to have had any ability for~\ref{Bhatt:Thal-b} to be true.%
      \mbox{ }\hfill\mbox{(\citeyear[167]{Bhatt:2008aa})}
    \end{quote}
    Distinction between `was able' and `had the ability'.
    \citeauthor{Boylan:2020aa} only `was able', and so agrees with \citeauthor{Bhatt:2008aa}.

    Still, the distinction between~\ref{Bhatt:Thal-b} but not~\ref{Bhatt:Thal-c} is due to the `specific'/`general' divide.
    And, indeed, \citeauthor{Bhatt:2008aa}'s proposal, \emph{to my understanding}, identifies `was able' with the specific reading of ability and `had the ability' with the general reading of ability.

    Indeed, \citeauthor{Boylan:2020aa} makes a similar observation with respect to \citeauthor{Maier:2018uo}'s (\citeyear{Maier:2018uo}) hybrid (modal/generic) account of ability. (\citeyear[23, fn.3]{Boylan:2020aa})
  }%
  \(^{,}\)%
  \footnote{
    Finding additional instances of \BoyPS{} has difficult.

    \textcite[1]{Boylan:2020aa} mentions \citeauthor{Austin:1961vz}'s  remark that `it follows merely from the premiss that he does it, that he has the ability to do it, according to ordinary English' (\citeyear[175]{Austin:1961vz}).
    However, reasoning patterns are often not made explicit.
    Most instances of `was able to' seem to correspond to the converse entailment, \BoyPSC{}, discussed below.

    Still, a clear instance comes from \textcite{Taylor:2011uh}:
    \begin{quote}
      Consider, then, the R-statement (S):

      \begin{quote}
        Stilpo walks through the Diomean Gate at t\textsubscript{2}
      \end{quote}

      and assume that statement, tenselessly expressed so as to avoid ambiguity in what follows, to be true.

      \hbox to \hsize{\hfil{\vdots}\hfil}

      if S is true, then it follows that Stilpo was able to be walking through the gate at t\textsubscript{2}, that being, in fact, precisely what he was doing.%
      \mbox{ }\hfill\mbox{(\cite[139--143]{Taylor:2011uh})}
    \end{quote}

    More generally, one may consider the following to be instance of \BoyPS{}, in which the cited proof explains(?) why the ability attribution is true:

    \begin{quote}
      Cantor was able to show (by a proof we will not reproduce here) that \([0, 1]\) is equivalent to the power set of the integers, and thus its cardinal number is \(2^{\aleph_{0}}\).\newline
      \mbox{ }\hfill\mbox{(\cite[65]{Partee:1990tu})}
    \end{quote}

    \begin{quote}
      Blok [\hyperlink{cite.Blok:1980th}{16}] was able to give a detailed analysis of frame incompleteness by drawing on algebraic methods.
      In particular, he did so by investigating splittings (a concept from lattice theory) of the lattice of normal modal logics\dots\newline
      \mbox{ }\hfill\mbox{(\cite[74]{Blackburn:2007wa})}
    \end{quote}
  }

  Second is the converse of \BoyPS{}
  \phantlabel{BoyPSC:Start}

  \begin{enumerate}[label=]
  \item
    \label{Boylan:Past-Success:C}
    \BoyPSC{}: \(\text{Past}(S\text{ is able to }\phi) \Rightarrow \text{Past}(S\text{ does }\phi)\)
  \end{enumerate}

  \citeauthor{Bhatt:2008aa}

  \begin{quote}
    The two readings associated with \emph{be able to} allow different interpretive possibilities for indefinite/bare plural subjects.

    \begin{enumerate}[label=(\arabic*), ref=(\arabic*)]
      \setcounter{enumi}{300}
    \item
      A fireman was/Firemen were able to eat five apples.
      \begin{enumerate}[label=\alph*., ref=(301\alph*)]
      \item
        \label{Bhatt:apples:ae}
        Yesterday at the apple eating contest, a fireman was/firemen were able to eat five apples.
        (Past episodic, actuality implication, existentially interpreted subject)
      \item
        In those days, a fireman were/firemen were able to eat five apples in an hour (Generic, no actuality implication, generically interpreted subject)%
        \mbox{}\hfill\mbox{(\citeyear[160]{Bhatt:2008aa})}
      \end{enumerate}
    \end{enumerate}
  \end{quote}

  \ref{Bhatt:apples:ae} `actuality implication'.%
  \footnote{
    Following \textcite{Alxatib:2019wf}:
\begin{quote}
      Actuality Entailments (AEs) are inferences from premises that appear to be modal, like~\ref{Alxatib:a}, but their content is that the modality is effectuated in the evaluation world~---~\ref{Alxatib:b}.

      \begin{enumerate}[label=(\arabic*)]
      \item
        \begin{enumerate}[label=\alph*., ref=(1\alph*)]
        \item
          \label{Alxatib:a}
          Pierre a dû \phantom{to.pfv} prendre le \phantom{e} train \newline
          Pierre had.to.\textsc{pfv} take \phantom{dre} the train\newline
          `Pierre had to take the train'
        \item
          \label{Alxatib:b}
          \emph{Inference}: Pierre took the train.%
          \mbox{}\hfill\mbox{(\citeyear[701]{Alxatib:2019wf})}
        \end{enumerate}
      \end{enumerate}
    \end{quote}

    \citeauthor{Alxatib:2019wf} stresses the reading of `had' in (1a) is `unambiguously deontic' (\citeyear[703]{Alxatib:2019wf}).

    See \textcite{Asher:2012vr}, \textcite{Bhatt:2008aa}, \textcite{Hacquard:2006to,Hacquard:2009ta}, \textcite{Palmer:1977wb}, \textcite{Pinon:2003te}, and~\textcite{Werner:2011tp} for examples and additional discussion of actuality entailments.
  }
  Follows that a fireman/firemen ate five apples.%
  \footnote{
    In contrast, to \BoyPS{}, examples of \BoyPSC{} are plentiful.
    Two examples involving reasoning follow:

    \begin{quote}
      One can then, because of the special ``linear'' nature of the electrical process, calculate the distortion of a very complicated signal, such as Uncle Fred's voice, simply by treating it as a series of gradual ``turnings on'' and ``turnings off'' of the unit step response and adding up their combined causal influence.
      Using his operational calculus, Heaviside was able to calculate the unit step response in very quick order and then solve more complicated cases in the manner suggested.%
      \mbox{ }\hfill\mbox{(\cite[316]{Wilson:1988wx})}
    \end{quote}

     \begin{quote}
       One senses from a reading of Russell how he was able to overlook this point:
       the trouble was his failure to focus upon the distinction between ``propositional functions'' as attributes, or relations-in intension, and ``propositional functions'' as expressions\dots%
      \mbox{ }\hfill\mbox{(\cite[152]{Quine:1967tv})}
    \end{quote}

  }

  % Put these together, and specific ability, embedded under past tense reduces to what happened.

  % \begin{enumerate}[label=]
  % \item
  %   \label{Boylan:Past-Success:IFF}
  %   \BoyPSIFF{}: \(\text{Past}(S\text{ is able to }\phi) \Longleftrightarrow \text{Past}(S\text{ does }\phi)\)
  % \end{enumerate}

  The difficulty with respect to \fc{1} is that concluding \(\pv{\phi}{v}\) from \(\Phi\) doesn't entail that \(\pvp{\phi}{v}{\Phi}\) was a \fc{}.
  For, Clause~\ref{def:fc:no-pe-bad} of \autoref{def:fc}.

  In short, \(\pvp{\phi}{v}{\Phi}\) is a \fc{} only if there is no potential event in which the agent concludes something incompatible with concluding \(\pv{\phi}{v}\) from \(\Phi\).

  For, given \BoyPS{} it is not possible to express Clause~\ref{def:fc:no-pe-bad} as:
  \begin{enumerate}[label=\emph{n}., ref=(\emph{n})]
  \item
    \label{Ability:past:narrow}
    The agent has the ability to \emph{not} [conclude something incompatible with concluding \(\pv{\phi}{v}\) from \(\Phi\)].
  \end{enumerate}
  As, so long as the agent concludes \(\pv{\phi}{v}\) from \(\Phi\), then the agent (plausibly) won't have concluded something incompatible, and hence the ability attribution will be true.

  Hence, negation must scope over the ability attribution, e.g.:

  \begin{enumerate}[label=\emph{w}., ref=(\emph{w})]
  \item
    \label{Ability:past:wide}
    The agent does \emph{not} [have the ability to conclude something incompatible with concluding \(\pv{\phi}{v}\) from \(\Phi\)].
  \end{enumerate}

  Still, \label{Ability:past:wide} will differ in truth value from \label{Ability:past:narrow} only if the negated variant of \BoyPS{} does not hold:
  \begin{enumerate}[label=]
  \item
    \label{Boylan:Past-Success:CQC}
    \BoyPSCQC{}: \(\text{Past}(S\text{ does \emph{not} }\phi) \Rightarrow \text{Past}(S\text{ is \emph{not} able to }\phi)\)
  \end{enumerate}
  \BoyPSCQC{} may seems false on first glace.
  However, the entailment may be motivated in parallel to \BoyPS{}.
  For, suppose \citeauthor{Boylan:2020aa} takes the shot and does not hit the bullseye.
  It seems we may then argue that:

  \begin{enumerate}[label=\arabic*\('\).]
    \setcounter{enumi}{2}
  \item
    You didn't hit the bullseye on that throw.\newline
    So, you were not able to to hit the bullseye on that throw.
  \end{enumerate}
  So, it is by no means clear that \BoyPSCQC{} fails for the relevant sense of ability.%
  \footnote{
    Further, observe the same substitution applied to \BoyPSC{} intuitively holds.
    In particular, consider \citeauthor{Bhatt:2008aa}'s \ref{Bhatt:apples:ae} where the firemen \emph{weren't} able to eat five apples.
  }
\end{note}

\begin{note}
  To summarise, immediate issue is with entailments.
  Seems these don't coincide with \fc{}.

  However, \BoyPS{}, \BoyPSC{}, nor \BoyPSCQC{} only concern past.

  Possible to give an indirect account of ability, tying specifically to present.

  Assuming unique sense of specific ability, explore.

  For, it is not immediate that \BoyPS{}, \BoyPSC{}, and \BoyPSCQC{} to hold where `Future' or `Present' is substituted for `Past'.

  Additional motivation required.

  If do, then significant problem.
  If do not, though, remains a general question about relationship.

  And, if distinct senses of specific ability, issue is identifying relevant sense for reduction.
  In particular, though \BoyPS{} and \BoyPSCQC{} are difficult, \BoyPSC{} is robust, and raises issue.

  Following section will focus on \BoyPS{} and idea of control.
\end{note}

\subsubsection{(Specific) ability and control}
\label{cha:sec:fcs-def:ability:control-intuition}

\begin{note}[Segue]
  \autoref{cha:sec:fcs-def:ability:past} raised concerns about specific ability and what does (or does not) happen.

  In this section we present and focus on idea of `control' common to a various analyses of specific ability.
  And, we will argue that idea of control as captured by the act conditional analysis of ability is incompatible with an account of \(\pvp{\phi}{v}{\Phi}\) being a \fc{1} for an agent in terms of the agent have the ability to conclude \(\pv{\phi}{v}\) from \(\Phi\).%
  \footnote{
    Part of the interest of \textcite{Boylan:2020aa} is combining the validity of \BoyPS{} with the failure of `Present Success'.
    However, the combination isn't of interest to us.
    Though, ensures specific ability.
  }
\end{note}

\begin{note}[\AbControl{}]
  \phantlabel{ability:control}
  \textcite{Mandelkern:2017aa} express the idea of control as follows:%
  \footnote{
    \label{fn:control-accounts}
    A similar account of the control intuition is found in \textcite{Jaster:2020wv}:

  \begin{quote}
    \dots think of the ability to sing a song, to build a shag, to play tennis --- all have an action as their manifestation: the agent controls what is going on and she also controls whether to exercise the ability at all.%
    \mbox{ }\hfill\mbox{(\cite[34]{Jaster:2020wv})}
  \end{quote}

  And, \citeauthor{Boylan:2020aa}'s (\citeyear{Boylan:2020aa}) statement of the control intuition is limited to a specific example:
      \begin{quote}
        Imagine a great wave is rising and I have dashed into the sea with my surfboard.
        You know nothing about me: perhaps I am one of the world’s great surfers; perhaps I am a fool. [\dots]

    When said before the fact, the claim that I can surf that wave is strong it says that surfing that wave is within my control.
    This intuition, call it the \emph{control intuition}\dots\newline
    \mbox{ }\hfill\mbox{(\citeyear[1]{Boylan:2020aa})}
  \end{quote}

  Similar accounts may be also be found in~\textcite{Brown:1988tl},~\textcite{Kikkert:2022wp}, and~\textcite{Horty:1995wu}.
  }
  {
    \newbox\qqBoxA
    \newdimen\qqCornerHgt
    \setbox\qqBoxA=\hbox{$\ulcorner$}
    \global\qqCornerHgt=\ht\qqBoxA
    \newdimen\qqArgHgt
    \def\Quinequote #1{%
      \setbox\qqBoxA=\hbox{$#1$}%
      \qqArgHgt=\ht\qqBoxA%
      \ifnum     \qqArgHgt<\qqCornerHgt \qqArgHgt=0pt%
      \else \advance \qqArgHgt by -\qqCornerHgt%
      \fi \raise\qqArgHgt\hbox{$\ulcorner$} \box\qqBoxA %
      \raise\qqArgHgt\hbox{$\urcorner$}}

    \begin{quote}
      When someone says \(\Quinequote{\text{I [am able to] }\varphi}\), she is assuring her interlocutors that \(\sem[c]{\varphi}\) is within her control in a certain way.\newline
      \mbox{ }\hfill\mbox{(\citeyear[326]{Mandelkern:2017aa})}
    \end{quote}
  }
  For ease of reference we will refer to the idea expressed via `\AbControl{}'.
\end{note}

\begin{note}[Control via \citeauthor{Schwarz:2020aa}]
  Similar to \citeauthor{Mandelkern:2017aa}, \textcite{Schwarz:2020aa} motivates \AbControl{} as follows:

    \begin{quote}
    Suppose Cyril does not know the first 10 digits of \(\pi\).
    Intuitively,~\ref{Schwarz:pi} is then false.

    \begin{enumerate}[label=(\arabic*), ref=(\arabic*)]
      \setcounter{enumi}{2}
    \item
      \label{Schwarz:pi}
      Cyril can recite the first 10 digits of \(\pi\).
    \end{enumerate}

    \dots when we say that someone can recite the first 10 digits of \(\pi\), we don't just mean that no relevant facts preclude them from uttering `three, one, four,' etc.
    Rather, the agent must have a certain kind of intentional control over performing the act under the description of `reciting digits of \(\pi\)'.%
    \mbox{ }\hfill\mbox{(\citeyear[2]{Schwarz:2020aa})}
  \end{quote}
\end{note}

\begin{note}[Control via \citeauthor{Boylan:2020aa}]
  Likewise, \citeauthor{Boylan:2020aa} (\citeyear{Boylan:2020aa}), inspired by~\textcite{Kenny:1976vh} motivates the idea of control with the following scenario:
  \begin{quote}
    \begin{quote}
      \textbf{Unreliable Dartboard}.
      I am a fairly bad dartplayer.
      I regularly hit the bottom half when I aim for the top; and vice versa.
      But I never miss the board entirely.
    \end{quote}

    I am about to take a shot.
    I am skilled enough to know I will hit the board; so I know the following:

    \begin{enumerate}[label=(\arabic*)]
      \setcounter{enumi}{6}
    \item
      I will hit the top half of the board on this throw or I will hit the bottom half of the board on this throw.
    \end{enumerate}

    But it does not seem that I should ascribe myself either of the following abilities here:

    \begin{enumerate}[label=(\arabic*), ref=(\arabic*), resume]
    \item
      I can hit the top on the throw.
    \item
      I can hit the bottom on this throw.
    \end{enumerate}

    Even the disjunction does not seem true:

    \begin{enumerate}[label=(\arabic*), ref=(\arabic*), resume]
    \item
      \label{Boylan:10}
      I can hit the top of the board on this throw or I can hit the bottom of the board on this throw.%
      \mbox{ }\hfill\mbox{(\citeyear[3]{Boylan:2020aa})}
    \end{enumerate}
  \end{quote}

  Intuitively, \citeauthor{Boylan:2020aa} lacks control over where the dart lands on the board, the exercises control over whether the dart lands on the dartboard.
  (\citeyear[\S2,19--20]{Boylan:2020aa})
\end{note}

\begin{note}[\BoyVS{}]
    As \citeauthor{Boylan:2020aa} observes,~\ref{Boylan:10} may be further expanded into a more complex disjunction of regions on the dartboard.
  (\citeyear[4]{Boylan:2020aa})
  For example, intuitively it is not the case that:
  \begin{enumerate}[label=(\arabic*'), resume]
    \setcounter{enumi}{10}
  \item
    I can hit outside the bullseye this throw or I can hit the upper-left-quadrant of the bullseye on this throw or I can hit the lower-right-quadrant of the bullseye on this throw or \dots
  \end{enumerate}

  Indeed, \AbControl{} leads to the \emph{invalidity} of \BoyVS{}:

  \begin{enumerate}[label=]
  \item
    \label{Boylan:Or-Success}
    \BoyVS{}: \(S\text{ will }\phi \lor S\text{ will }\psi \Rightarrow S\text{ is able to }\phi \lor S\text{ is able to }\psi\)\newline
    \mbox{ }\hfill\mbox{(\citeyear[\S1.2]{Boylan:2020aa})}
  \end{enumerate}
\end{note}

\begin{note}[Need to get precise]
  Now, \AbControl{} is an idea, but is under-specified by the motivation provided.
  \citeauthor{Mandelkern:2017aa} hedge with `in a certain way', \citeauthor{Schwarz:2020aa} hedges with `certain kind', and \citeauthor{Boylan:2020aa} does not provide an explicit statement of the idea (see Footnote~\ref{fn:control-accounts}).
  Hence, \(\pvp{\phi}{v}{\Phi}\) being a \fc{1} for an agent may be equivalent to the agent having the (controlled) ability to conclude \(\pv{\phi}{v}\) from \(\Phi\).
  Or, the proposed equivalence may fail.

  So, interest turns to details of the accounts of `is able to' advanced by \textcite{Mandelkern:2017aa} and \textcite{Boylan:2020aa} in order to obtain sufficient clarity on what \AbControl{} amounts to on their understanding.
\end{note}

\begin{note}[ACA]
  We present a generalised account of the `act conditional' analysis of ability, common to \textcite{Boylan:2020aa}, \textcite{Mandelkern:2017aa}, and \textcite{Schwarz:2020aa}.%
  \footnote{
    Though \citeauthor{Schwarz:2020aa} is non-committal with respect to a formal account of ability (\citeyear[cf.][13]{Schwarz:2020aa}), the spirit of \citeauthor{Schwarz:2020aa}'s analysis is sufficiently close to \citeauthor{Boylan:2020aa}'s for the issue to arise:
    `[A]n agent has the ability to \(\phi\) iff there are accessible worlds at which she \(\phi\)s simply by deciding to \(\phi\).' (\citeyear[19]{Schwarz:2020aa})
    Decision to action, but then the decision itself must sufficiently determine the action.
  }

  \[%
    \sem[c,w]{\text{S is able to }\varphi} = 1\text{ iff }\exists A \in \mathcal{A}_{S,c,w,t}\colon \forall v \in f_{c}(\text{S does }A,w),  \sem[c,v]{\varphi(S)} = 1%
  \]

  Where:
  \begin{itemize}
  \item
    \(f_{c}\) is a selection function from proposition-world pairings to set of worlds.
  \item
    \(\mathcal{A}_{S,c,w}\) is the set of actions that are available to \(S\) in context \(c\) and world \(w\).
  \end{itemize}

  So, \(S\text{is able to }\varphi\) is true at some world \(w\) in context \(c\), just in case there is some action available to the agent, such that for every world in which it is true that \(S\text{ tries to}A\) determined by the selection function \(f_{c}\), it is the case that \(S \varphi\text{s}\).%
  \footnote{
    Strictly, both \citeauthor{Mandelkern:2017aa} and \citeauthor{Boylan:2020aa} omit universal quantification over worlds returned by the selection function.
    For \citeauthor{Mandelkern:2017aa}, as discussed below, the selection function returns a unique world, though, as discussed below, this assumption is problematic.
    For \citeauthor{Boylan:2020aa}, universal quantification is implicit by embedding `\(\varphi(S)\)' under a modal `\(\mathcal{W}\)' corresponding to `will'.
    % Issue is `if performs act, then \dots'
    % Restrictor semantics for conditional.
    % \(\sem[c,w,f]{\text{if }\phi,\psi} = 1 \text{ iff } \sem[c,w,f^{\sem[c,w,f]{\phi}}]{\psi} = 1\).
    % With universal, effectively inserting a modal.
    % Complexity of \citeauthor{Boylan:2020aa}'s account is getting the right modal.
    % Simplicity of \citeauthor{Mandelkern:2017aa}'s account is avoiding modal by assuming unique world.
  }

  Paraphrased, the act conditional analysis of ability holds:
  `\(S\) is able to \(\varphi\)' is true just in case there is some action \(A\) available to \(S\) such that if \(S\) tried to \(A\) then S would \(\varphi\).
\end{note}

\begin{note}[Selection functions]
  The primary difference between the analyses of \citeauthor{Mandelkern:2017aa} and \citeauthor{Boylan:2020aa} is the specification of \(f_{c}\), though in practice the difference seems minor:
  \begin{itemize}
  \item
    For \citeauthor{Mandelkern:2017aa},
    \(f_{c}\) is~\citeauthor{Stalnaker:1968vt}'s selection function.
    I.e.\ \(f_{c}(\psi, w) = \{v\}\) where \(v\) is the `closest' world to \(w\) where \(\psi\) is true.
    (\citeyear[Cf.][314]{Mandelkern:2017aa})

    However, the assumption of a unique `closest' world is clearly problematic given \BoyVS{}.
    For, an agent has the ability to throw a dart at the dartboard.
    Hence, in the closest possible world where the agent attempts to throw a dart, the agent succeeds.
    Further, the dart lands at some exact region of the dartboard.
    Hence, as there is only one closest world to consider, `\(S\text{ it able to throw a dart at the dartboard}\)' is strengthened to `\(S\text{ it able to throw a dart at the \emph{exact region of the} dartboard}\)'.%
    \footnote{
      In particular, \citeauthor{Mandelkern:2017aa} do not require that what the agent tries to do and what the agent does satisfy the same description (\citeyear[310,314]{Mandelkern:2017aa}).

      The same problem applies to the `orthodox approach' of~\textcite{Hilpinen:1969vw}, \textcite{Kratzer:1977aa,Kratzer:1981vn}~and~\textcite{Lewis:1976us}.
      See \textcite[\S1.3]{Boylan:2020aa} and \textcite[\S2]{Mandelkern:2017aa} for more on the orthodox approach.
    }

    Still, a \citeauthor{Lewis:1973th}ian approach where the selection function return a set of `closest' worlds resolves this issue.
    For, we may assume that the closest possible worlds determine some inexact region of the dartboard.
  \item
    For \citeauthor{Boylan:2020aa}, rather than selecting `close' worlds, \(f_{c}\) selects all worlds which are identical to \(w\) up until time \(t\) (in which \(S\) does \(A\)).%
    \footnote{
      Strictly, there is more.
      For, Non-classical Strong Kleene account of disjunction.
      (\citeyear[\S5]{Boylan:2020aa})
      Though, I really don't get it.
      Just think of \(\mathcal{W}\) as \(G\).
      `Indeterminate' is just \(F \phi \land F \lnot \phi\).
    }
  \end{itemize}

  For present purposes, the key part of the act conditional analysis for capturing \AbControl{} is that \(S \varphi\)s \emph{follows from} \(S\text{ does }A\) in all worlds captured by the selection function.

  In this respect, that the agent \(\varphi\)s is a \emph{consequence} of performing \(A\).
  In particular, it is not possible for \(A\) to set `\(\varphi\)ing in motion'.
  For, we have seen with progressive and the imperfective paradox, \(\varphi\)ing does not entail an agent \(\varphi\)s.
\end{note}

\begin{note}[Availability]
  What counts as an available action is a more complex issue.
  Thankfully, the details of \citeauthor{Mandelkern:2017aa} and \citeauthor{Boylan:2020aa} may be avoided.%
  \footnote{
    Specifically, \citeauthor{Boylan:2020aa} says little on what makes it the case that an action is available to an agent:
    \begin{quote}
      I think of an agent's available actions as their options.
      And, for simplicity at least, we can typically think of options as a set of tryings.%
      \mbox{ }\hfill\mbox{(\citeyear[14]{Boylan:2020aa})}
    \end{quote}
    In contrast, \citeauthor{Mandelkern:2017aa} consider the issue in detail.
    In short:
    \begin{quote}
      [A]n action counts as practically available only if the agent knows that it is a way of bringing about the prejacent \emph{relative to a given description of her practical situation}.\newline
      \mbox{ }\hfill\mbox{(\citeyear[321]{Mandelkern:2017aa})}
    \end{quote}
    On my understanding, there is still some gap between knowing an action is a way of bringing something about and performing the action.

    Hence, the account allows for the possibility that an agent has the ability and fails.
    For, may fail to perform the relevant action.
      See \citeauthor{Maier:2013vk} the importance of allowing for failure.
  }

  For present purposes, a sufficient understanding of when an action is available to an agent by considering \citeauthor{Boylan:2020aa}'s scenario illustrating the invalidity of \BoyVS{}.
  For, if the agent throws a dart and it hits a certain region of the dartboard, then the agent performed the act of throwing the dart at that region.
  However, throwing the dart at that region could not have been an action available to the agent on pain of \BoyVS{} having a true premise and true conclusion.
  Hence, if an agent \emph{lacks} the ability to \(\varphi\), then it cannot be the case that there is an action \(A\) available in which the agent \(\varphi\)s by doing \(A\).
\end{note}

\begin{note}[Summary of \AbControl{}]
  To summarise, it seems that on an act conditional analysis of ability, \AbControl{} amounts to the availability of some action \(A\) such that the agent \(\varphi\)ing is a consequence of performing \(A\).
\end{note}

\begin{note}[Difficulty with \fc{1}]
  With understanding of \AbControl{} in hand, we now turn to \fc{1}.

  We make the simple observation that \(\pvp{\phi}{v}{\Phi}\) may be a \fc{} without the agent having appropriate control (in the sense of \AbControl{}) over concluding \(\pv{\phi}{v}\) from \(\Phi\).

  Specifically, cases where \(\pvp{\phi}{v}{\Phi}\) is a \fc{} but for any action \(A\), it is either the case that \(A\) is inconsequential or unavailable.

  The particular \fc{} is of little importance, so we take the abstract \(\pvp{\phi}{v}{\Phi}\)-pairing.
  The key to failure of \AbControl{} is the assumption that the agent does \emph{not} have the ability to avoid distraction.%
  \footnote{
    E.g.\ the agent may be interrupted at any time, become bored, or think of something else they would prefer to do.
  }
  In particular, consider relatively simple tasks such as long but simple calculations, simple sudoku puzzles, basic chess problems, or routine proofs.

  Now, suppose \(\pvp{\phi}{v}{\Phi}\) being a \fc{0} for an agent is equivalent to the agent having the ability to conclude \(\pv{\phi}{v}\) from \(\Phi\) (where \AbControl{} holds for the relevant sense of ability).

  Given \AbControl{} it must be the case that concluding \(\pv{\phi}{v}\) from \(\Phi\) is a result of performing some action \(A\).
  However, by assumption, the agent does not have the ability of avoid distraction, and so \(A\) is not an action available to the agent.
  For, if \(A\) were available to the agent, the agent would \emph{have} the ability to avoid distraction.

  Conversely, suppose any available action allows for the possibility of distraction.
  Then, it straightforwardly follows that concluding \(\pv{\phi}{v}\) from \(\Phi\) is \emph{not} a result of performing that action.
  For, if the agent gets distracted, then the do not conclude \(\pv{\phi}{v}\) from \(\Phi\).

  In short, \AbControl{} requires an available action such concluding \(\pv{\phi}{v}\) from \(\Phi\) is a result of performing some action.
  However, \(\pvp{\phi}{v}{\Phi}\) being a \fc{} tolerates the absence of any such action.

  We may express the difference in either of two ways.
  \begin{enumerate}[label=\arabic*.]
  \item
    In order for \(\pvp{\phi}{v}{\Phi}\) to be a \fc{} it need only be the case that there is some action which results in the agent concluding \(\pv{\phi}{v}\) from \(\Phi\) (and no action where the agent does not conclude anything incompatible), though this action does not need to be \emph{available} to the agent.
  \item
    In order for \(\pvp{\phi}{v}{\Phi}\) to be a \fc{} there must be some action available to the agent, but it need not be the case that concluding \(\pv{\phi}{v}\) from \(\Phi\) (and no action where the agent does not conclude anything incompatible), is a \emph{consequence} of performing the action.
  \end{enumerate}
  As indicated by interest in the progressive, I think the second expression is correct, but either is sufficient observe that \(\pvp{\phi}{v}{\Phi}\) being a \fc{} does not require \AbControl{} over concluding \(\pv{\phi}{v}\) from \(\Phi\).
\end{note}

\subsubsection{Summary}

\begin{note}
  Entertained reducing \fc{1} to ability.
  Specific, rather than general ability.
  However, questions about specific ability.
  Specifically, with what happens, and the past.
  And, \AbControl{}.
\end{note}

\begin{note}
  Naturally, I have not argued that there is no sense of `ability' such that \(\pvp{\phi}{v}{\Phi}\) being a \fc{0} for an agent is equivalent to the agent having the ability to conclude \(\pv{\phi}{v}\) from \(\Phi\).
  However, identifying the (or a) sense of ability suitable for the equivalence is difficult.

  Hence, we pursue an account of \(\pvp{\phi}{v}{\Phi}\) being a \fc{0} in terms of \pevent{1} and the progressive.
\end{note}

\paragraph[Independent difficulty]{Independent difficulty \hfill (Optional)}

\begin{note}
  This is the `plan' account of ability.
  It's kind of insane.
  Whether or not ability reduces to action such that choice and secure outcome.
\end{note}

\begin{note}
  This is kind of wild.
  For, actions are kind of huge.
  Similar to that paper with minimalism about intentions.
\end{note}

\begin{note}
  Our direct interest with account finishes with universal.
  However, clear additional problem.
  Co-operation.
\end{note}

\begin{note}
  Uh, think.
  Has the ability to X with my help.
  There's no action in advance.
  For, whatever is chosen, I intervene prior, changing the course.
  Well, the point is, I only help if the agent gives up on whatever they had been planning to do.

  This isn't odd, cooperative activity.
  So, actually, refine example a little.
  For, point is that there's the cooperation condition.
\end{note}

\subsection{The progressive}
\label{cha:sec:fcs-def:progressive}

\begin{note}
  \autoref{cha:sec:fcs-def:ability} raised difficulty with linking \fc{1} to ability.
  Specific.
  Ability and the past.
  Understanding of \AbControl{}.
\end{note}

\begin{note}
  Understand \fc{} in terms of \pevent{3}:
  \begin{quote}
    \definitionForegoneC*
  \end{quote}
  And, \pevent{3} in terms of the progressive.
  \begin{quote}
    \definitionPEvent*
  \end{quote}
  Choice of action, similar to act conditional analysis.

  Together with progressive captures specific.
  Just need to be concluding after performing action.
  This doesn't rely on general performance.

  Key difference is with respect to \AbControl{}.
  Consequence of the action is that \(S\) is \(\varphi\)\emph{ing}.
  That \(S\) \(\varphi\)s is not a consequence of the action, only that there is some possible continuation (\autoref{assu:prog-modal-shift}).

  Task is clarifying sense of possibility.

  To develop, focus on \citeauthor{Landman:1992wh}'s (\citeyear{Landman:1992wh}) account of the progressive.

  Focus on \citeauthor{Landman:1992wh} as goes through the reasoning.
  Interest with \citeauthor{Landman:1992wh}'s account is somewhat arbitrary.
  Minimal background, events and possible worlds, and counterfactuals.
  A number of important areas where \citeauthor{Landman:1992wh}'s analysis needs to be adjusted.
  Careful distinction between event and world event takes place in.
\end{note}


\subsubsection[\citeauthor{Landman:1992wh}'s account of the progressive (modified)]{\citeauthor{Landman:1992wh}'s (\citeyear{Landman:1992wh}) account of the progressive (modified)}
\label{cha:sec:fcs-def:progressive-landman}
\nocite{Portner:1998um}
\nocite{Engelberg:1999vi}


\begin{note}
  Borrow summary from \textcite{Szabo:2004ul}:
  \begin{quote}
    [A] progressive sentence is true at some time just in case some event occurs at that time, and if we follow the development of the event (within our world as long as it goes, then jumping into a nearby world, and iterating the process within the limits of reasonability) we will reach a possible world where the perfective correlate is true of the continuation.%
    \mbox{ }\hfill\mbox{(\citeyear[34]{Szabo:2004ul})}
  \end{quote}
  The perfective correlate, links to \autoref{assu:prog-modal-shift}.
\end{note}

\begin{note}
  \begin{enumerate}
  \item
    \label{prog:max:bad}
    Max is crossing the street.
  \end{enumerate}
  True just in case there is some continuation of the actual world such that in that world, Max crossed the street.

  In the actual world, Max doesn't cross the street because Max is hit by a bus cruising at thirty miles per hour.
  (\citeyear[764]{Portner:1998um})
  Intuitively, however, \ref{prog:max:bad} remains true.
  Max was hit by the bus, but Max was not destined to be hit by the bus.
  And, if Max had not been hit by the bus, Max would have continued to cross the street.
  In other words, there is some possible world \(v\) that branched from the actual world before Max was hit by the bus.
  And, in \(v\) Max was not by the bus, and continued a little further across the road.

  Still, behind bus \#1 a second bus, bus \#2, was ready to hit max.
  And, in \(v\) Max was hit by bus \#2.
  (\citeyear[766]{Portner:1998um})
  However, like bus \#1 in \(w\), Max was not destined to be hit by bus \#2 in \(v\).
  Hence, there is some world \(u\) which branched from \(v\) in which Max continued a little further across the road.

  So long as there are a finite number of busses and no bus is destined to hit Max, then prior to being hit by a given bus, Max makes it a little further across the road.
  And, so long as the road is finite, it follows that eventually Max will have crossed the road.
\end{note}

\begin{note}
  \autoref{fig:max-bus} is a recreation of \citeauthor{Portner:1998um}'s figure 1. (\citeyear[767]{Portner:1998um})
  \begin{figure}[!h]
    \centering
    \begin{tikzpicture}
      \tikzmath{
        % x positions
        \x1 = 11;
        \xb1 = 2/9*\x1; \xb2 = 4/9*\x1; \xb3 = 6/9*\x1;
        % y positions
        \y1 = 2/5*\x1; \ymid = 1/2*\y1;
        \yw1 = \y1; \yw2 = 1/2*\y1; \yw3 = 0*\y1; \yb2 = 1/5*\y1;
        % event e
        \xe = 1/2*\xb1; \yediff = \yw2 - \yb2;
        \ye = \yw2 - 1/2*\yediff;
        \enudge = .1;
        \xel = 0; \xer = \xb1; \yen = \yw2 - \enudge;
        % bus 1 description location
        \xbx = 1.5/9*\x1; \xby = 4/5*\y1;
        % bus 2 description location
        \xb9 = 2.5/9*\x1;
      }
      % Paths
      \draw[line width=0.25mm, line cap=round] (\xb1,\ymid) -- (\xb3,\yw1); % world 1
      \draw[line width=1mm, line cap=round] (0,\ymid) -- (\xb1,\ymid) -- (\xb2,\yb2) -- (\xb3,\yw2); % world 2
      \draw[line width=0.25mm, line cap=round] (\xb2,\yb2) -- (\xb3,\yw3); % world 3
      % World descriptions
      \filldraw[black] (\xb3,\yw1) circle (0pt) node[anchor=west, align=left]{world 1: Max hit by \\ bus \# 1};
      \filldraw[black] (\xb3,\yw2) circle (0pt) node[anchor=west, align=left]{world 2: Max \\ crosses street};
      \filldraw[black] (\xb3,\yw3) circle (0pt) node[anchor=west, align=left]{world 3: Max hit by \\ bus \# 2};
      % Event
      \draw[] (\xe,\ye) -- (\xel,\yen); % event l
      \draw[] (\xe,\ye) -- (\xer,\yen); % event r
      % Event description
      \filldraw[black] (\xe,\ye) circle (0pt) node[anchor=north, align=left]{event e};
      % Splits
      \filldraw[black, dashed] (\xbx,\xby) circle (0pt) node[anchor=south, align=left]{bus \#1 hits Max};
      \filldraw[black, dashed] (\xb9,\yw3) circle (0pt) node[anchor=north, align=left]{bus \#2 hits Max};
      % Split descriptions
      \draw[-Stealth, dashed] (\xbx,\xby) -- (\xb1,\ymid + \enudge); % bus 2 arrow
      \draw[-Stealth, dashed] (\xb9,\yw3) -- (\xb2 - \enudge,\yb2 - \enudge); % bus 2 arrow
    \end{tikzpicture}
    \caption{
      Continuation path of `Max was crossing the street'. \\
    }
    \label{fig:max-bus}
  \end{figure}
\end{note}

% \begin{note}
% {
%   \color{red}
%   It's not obvious that this is the case on \citeauthor{Landman:1992wh}'s account.
% }
%   Max was rather unfortunate to be hit by a bus, but fortunes may be reversed.
%   The same reasoning applies to

%   \begin{enumerate}
%   \item
%     \label{prog:max:good}
%     Max was failing at the exam.
%   \end{enumerate}

%   Multiple choice.
%   Max goes for broke.
%   For each question, Max puts down the right answer.
%   However, Max was not destined to write down the correct answer there is a branch from how things actually happened in which Max chose the incorrect answer.
%   And, after a sequence of incorrect choices, Max failed the exam.
% \end{note}

\paragraph[\citeauthor{Landman:1992wh}~(\citeyear{Landman:1992wh})]{\citeauthor{Landman:1992wh}'s (\citeyear{Landman:1992wh}) account of the progressive}

\begin{note}
  \citeauthor{Landman:1992wh}'s account of the progressive:

  \begin{quote}
    \(\sem{\text{PROG}(e, P)}_{w,g} = 1\) iff \(\exists f \exists v\colon \langle f,v \rangle \in \text{CON}(g(e), w)\)\newline
    \phantom{an} and \(\sem{P}_{v,g}(f) = 1\)\par

    where \(\text{CON}(g(e), w)\) is the continuation branch of \(g(e)\) in \(w\).\newline
    \mbox{ }\hfill\mbox{(\citeyear[27]{Landman:1992wh})}
  \end{quote}

  Immediate goal is to present \citeauthor{Landman:1992wh}'s account of a continuation branch.
  In turn, this will require expansion on three further points.
  Stages of an event.
  Continuations and stops with respect to events.
  Reasonable options.

  With understanding in hand algorithmic reconstruction of continuation branch.
  Expand in two ways.
  First, tree, to allow for forks.
  Second, a different account of how to identify branches.
\end{note}

\subparagraph{Continuation branch}

\begin{note}
  \citeauthor{Landman:1992wh}'s account of a continuation branch is as follows:
  \begin{quote}
    The \emph{continuation branch} for \(e\) in \(w\) is the smallest set of pairs of events and worlds such that
    \begin{enumerate}
    \item
      \label{Landman:CB:continues}
      for every event \(f\) in \(w\) such that \(e\) is a stage of \(\langle f,w \rangle \in C(e,w)\);
      the continuation stretch of \(e\) in \(w\);
    \item
      \label{Landman:CB:stops}
      if the continuation stretch of \(e\) in \(w\) stops in \(w\), it has a maximal element \(f\) and \(f\) stops in \(w\).
      Consider the closest world \(v\) where \(f\) does not stop:
      \begin{enumerate}[label=--]
      \item
        if \(v\) is not in \(R(e, w)\), the continuation branch stops.
      \item
        if \(v\) is in \(R(e, w)\), then \(\langle f,v \rangle \in C(e,w)\).
        In this case, we repeat the construction:
      \end{enumerate}
    \item
      \label{Landman:CB:continues:again}
      for every \(g\) in \(v\) such that \(f\) is a stage of \(g\), \(\langle g,v \rangle \in C(e,w)\), the continuation stretch of \(e\) in \(v\);
    \item
      \label{Landman:CB:stops:again}
      if the continuation stretch of \(e\) in \(v\) stops, we look at the closest world \(z\) where its maximal element \(g\) does not stop:
      \begin{enumerate}[label=--]
      \item
        if \(z\) is not in \(R(e, w)\), the continuation branch stops.
      \item
        if \(z\) is in \(R(e, w)\), then \(\langle g,z \rangle \in C(e,w)\) and we continue as above, etc.%
        \mbox{ }\hfill\mbox{(\citeyear[26--27]{Landman:1992wh})}
      \end{enumerate}
    \end{enumerate}
  \end{quote}

  Describes the process of following an event \(e\) and a world \(w\) and jumping to nearby reasonable worlds when \(e\) stops in \(w\) (or \(w'\), etc.).

  Observe, the construction of a continuation branch is iterative.
  Clauses~\ref{Landman:CB:continues:again} and~\ref{Landman:CB:stops:again} and duplicates of Clauses~\ref{Landman:CB:continues} and~\ref{Landman:CB:stops} shifted to \(g\) --- some development of \(e\) --- in some possible world \(v\).

  Reconstruction via a recursive algorithm.
  For the moment we leave \citeauthor{Landman:1992wh}'s account of a continuation branch without commentary.
\end{note}

\subparagraph{Stages}

\begin{note}
  An event being a stage of some other event.
  Clause~\ref{Landman:CB:continues} (and~\ref{Landman:CB:continues:again}).

  \citeauthor{Landman:1992wh}'s definition is light:
  \begin{quote}
    An event is a stage of another event if the second can be regarded as a more developed version of the first, that is, if we can point at it and say, ``It's the same event in a further stage of development.''\newline
    \mbox{ }\hfill\mbox{(\citeyear[23]{Landman:1992wh})}
  \end{quote}
  \citeauthor{Landman:1992wh}'s definition is from an agent's perspective.
  Even if the elaboration is ignored, the initial expansion is qualified by the term `can be regarded as'.
  However, we may provide a definition of a stage independent of an agent's perspective:
  \begin{definition}[Stage]
    For events \(e\) and \(f\):

    \begin{itemize*}
    \item
      \(e\) is a stage of \(f\).
    \item
      \emph{If and only if}
    \item
      \(f\) is a development of \(e\).
    \item
      \mbox{ }
    \end{itemize*}
  \end{definition}
  At issue is what it is for \(f\) to be a \emph{development} of \(e\).

  Stage is distinguished from part-of.
  \nocite{Davidson:1967aa}

  Buttering the toast.
  Bread was toasted.
  But, toasting the bread was not a stage of buttering the toast.
  Sufficiently distinct events.
  Marked by `toast'.
  Toasting and buttering the bread, then plausible stage.
  Likewise, scooping butter onto the knife, stage of buttering the toast.

  Stage does not entail anything of significance happens.
  Waiting for the post to arrive.
  Stage, nothing is happening, event has developed.

  So, definition by intuitive distinction.

  Importance is for shifting-to and tracing-events-through possible worlds.
\end{note}

\subparagraph{Continuations and Stops}

\begin{note}[Continuations and Stops]
  The definition of a stage is important for defining both the continuation and when an event stops.
  We present \citeauthor{Landman:1992wh}'s definition of both terms, and then provide restated definitions.
\end{note}

\begin{note}
  \citeauthor{Landman:1992wh} combines the definition of a continuation and when an event stops:
  \begin{quote}
    This is where stages come in: we cannot say that when an event stops in a world, there is no bigger event of which it is part in that world, but we can say that when it stops, there is no bigger event in the world of which it is a \emph{stage}:
    \begin{enumerate}[label=, noitemsep]
    \item
      Let \(e\) be an event that goes on at \(i\) in \(w\).
      Let \(f\) be an event that goes on at \(j\) in where \(i\) is a subinterval of \(j\).
    \item
      \(j\) is a continuation of \(e\) iff \(e\) is a stage of \(f\).
    \item
      Let \(j\) be a non-final interval.
    \item
      \(f\) stops at \(j\) in \(w\) iff no event of which \(f\) is a stage goes on beyond \(i\) in \(w\) (i.e., at a later ending interval).\newline
      \mbox{ }\hfill\mbox{(\citeyear[23--24]{Landman:1992wh})}
    \end{enumerate}
  \end{quote}

  To clarify the definitions, we borrow the relevant definitions regarding intervals from \textcite{Dowty:1979vq}:

  \begin{quote}
    \(I\) is a subinterval of \(J\) iff \(I \subseteq J\), where \(I\) and \(J\) are intervals.
    \(I\) is a proper subinterval of \(J\) iff \(I \subset J\).
    \(I\) is an \emph{initial subinterval} of \(J\) iff \(I\) is a subinterval of \(J\) and there is no \(t \in (J - I)\) for which there is \(t' \in I\) such that \(t \leq t'\).
    \emph{Final subinterval} is defined similarly\dots\newline
    \mbox{ }\hfill\mbox{(\citeyear[140]{Dowty:1979vq})}
  \end{quote}
  So, following \citeauthor{Dowty:1979vq}:
  \begin{quote}
    \(I\) is a \emph{final subinterval} of \(J\) iff \(I\) is a subinterval of \(J\) and there is no \(t \in (J - I)\) for which there is \(t' \in I\) such that \(t' \leq t\).
  \end{quote}
  And, from the definition of a non-final subinterval follows similarly:
  \begin{quote}
    \(I\) is a \emph{non-final subinterval} of \(J\) iff \(I\) is a subinterval of \(J\) and there is \emph{some} \(t \in (J - I)\) for which there is \(t' \in I\) such that \(t' \leq t\).
  \end{quote}
  Intuitively, then, \(I\) is a \emph{non}-final interval of \(J\), just in case \(I \subseteq J\) and \(J\) progresses further in time than \(I\).

  Let us now turn to restating the definitions of a continuation and a stop.
  We take each in turn.
\end{note}

\begin{note}
  The definition of \(f\) being a continuation of \(e\) requires two things:
  First, it must be the case the event at which \(e\) takes place is a subinterval of the event at which \(f\) takes place, and it must be the case that \(e\) is a stage of \(f\).%
  \footnote{
    Though, it seems to me the latter/\ref{def:Landman:conts:stage} implies the former/\ref{def:Landman:conts:interval}.
    I.e.\ if \(e\) is a stage of \(f\) then the interval at which \(e\) takes place must be a subinterval of the interval at which \(f\) takes place.
  }
  In full:

  \begin{definition}[Continuations]
    \label{def:Landman:conts}
    For events \(e\) and \(f\):
    \begin{itemize}
    \item \(f\) is a \emph{continuation} of \(e\)
    \end{itemize}
    \emph{if and only if}
    \begin{itemize}
    \item
      The following jointly hold:
      \begin{enumerate}[label=\alph*., ref=(\alph*)]
      \item
        \label{def:Landman:conts:interval}
        \begin{enumerate}
        \item[\emph{If}:]
          \begin{enumerate}[label=\roman*.]
          \item
            \(i\) is the interval at which \(e\) takes place in \(w\).
          \end{enumerate}
        \item[\emph{And}:]
          \begin{enumerate}[label=\roman*., resume]
          \item
            \(j\) is the interval at which \(f\) takes place.
          \end{enumerate}
        \item[\emph{Then}:]
          \begin{enumerate}[label=\roman*., resume]
          \item
            \(i\) is a subinterval of \(j\).
          \end{enumerate}
        \end{enumerate}
      \item
        \label{def:Landman:conts:stage}
        \(e\) is a stage of \(f\).
      \end{enumerate}
    \end{itemize}
    \vspace{-\baselineskip}
  \end{definition}
\end{note}

\begin{note}
  The definition of \(f\) stopping at \(j\) in \(w\) takes a little more work.

  There are two immediate issues with \citeauthor{Landman:1992wh}'s definition.

  First, the definition requires that \(j\) is a non-final interval.
  But, of what?
  By assumption \(i\) is a subinterval of \(j\), so \(j\) cannot be a non-final interval of \(i\).%
  \footnote{
    The term `non-final interval' only appears in the above quote from \citeauthor{Landman:1992wh}.
  }

  Second, there must be no event of which \(f\) is a stage that goes beyond \(i\) in \(w\).
  However, by assumption \(f\) is an event that goes on at \(j\), and \(i\) is a subinterval of \(j\).
  But why is \(f\) bound by some arbitrary interval \(i\)?

  I propose to resolve both issues with the following definition:
  \begin{definition}[Stops]
    \label{def:Landman:Stops}
    For an event \(e\), interval \(i\), and world \(w\):
    \begin{itemize}
    \item
      \(e\) \emph{stops} at \(i\) in \(w\)
    \end{itemize}
    \emph{If and only if:}
    \begin{itemize}
    \item
      There no is interval \(j\) nor event \(f\) such that the following jointly hold:
      \begin{enumerate}[label=\alph*., noitemsep]
      \item
        \(i\) is non-final subinterval of \(j\).
      \item
        \(f\) goes on at \(j\).
      \item
        \(f\) is a stage of \(e\).
      \end{enumerate}
    \end{itemize}
    \vspace{-\baselineskip}
  \end{definition}
  In short, there is no expansion from \(i\) \emph{forward} in time to obtain an interval \(j\) such that an event which is a stage of \(e\) got on at \(j\).

  Hence, \autoref{def:Landman:Stops} captures \citeauthor{Landman:1992wh}'s initial gloss: `[W]hen [\(e\)] stops, there is no bigger event in the world of which [\(e\)] is a \emph{stage}' (\citeyear[23]{Landman:1992wh})
\end{note}

\begin{note}[Example of \autoref{def:Landman:Stops}]
  Interesting case to consider.
  Time limits.
  Running the race in record time.
  But, falls short of the record.

  Now, it may be false.
  Or, it may be true.
  Stops, no longer possible to complete race in record time.
  Shift to possible world.
  However, stops when event is still in progress.
  So, shift two possible world at time when stop will be completing race in record time.
\end{note}

\subparagraph{Reasonable options}

\begin{note}
  Reasonable options
  \begin{quote}
    \(v \in R(e, w)\) iff there is a reasonable chance on the basis of what is internal to \(e\) in \(e\) that \(e\) continues in \(w\) as far as it does in \(v\).%
    \mbox{ }\hfill\mbox{(\citeyear[26]{Landman:1992wh})}
  \end{quote}

  Look at possible worlds.
  Consider event \(e\) in possible world \(v\).
  Consider \(e\) in actual world.
  If there is a reasonable chance that \(e\) continues in \(w\) as in \(v\), then \(v\) is reasonable, on the basis of what is `internal' to \(e\).

  Not at all clear.
  However, significant insight from role in \citeauthor{Landman:1992wh}'s account of the progressive.
  \autoref{cha:sec:fcs-def:progressive-landman:alg:branches} considers in detail.

  Briefly, however, `reasonable' is intuitively epistemic, but need not be.
\end{note}

\paragraph{An algorithmic (re)construction}
\label{cha:sec:fcs-def:progressive-landman:alg}

\begin{note}
  Breaking this down into a recursive algorithm.
  Goal is to create a tree, which will be a set of ordered event-world-pairing pairs indexed according to depth.
  For example:
  \[\text{Tree} = \{\langle \langle e,w \rangle_{1}, \langle f,w \rangle_{1} \rangle, \langle \langle f,w \rangle_{1}, \langle g,v \rangle_{2} \rangle, \langle \langle f,w \rangle_{1}, \langle g,v' \rangle_{2} \rangle, \dots \}\]

  Root, inital event world pair, and then branch from event world pair to some distinct event world pair just in case ordered pair.
  %
  \footnote{
    The indexing is not required to construct the tree.
    However, important for keeping track of the stage of construction.
  }

  {\color{red} Pseudocode}
  \nocite{Cormen:2009uw}

  We start by identifying three basic functions, and following a (re)construction of these functions with explanation added, we (re)construct a recursive function to build a continuation branch of some event-world pairing.

  The four basic algorithms will be termed:
  `\AlgAC{}', `\AlgGetStops{}', `\AlgGetPStops{}', and `\AlgFindBranches{}'.
  The recursive algorithm will be termed:
  `\AlgDevelopTree{}'.

  Interest will be in difference between \AlgGetStops{} and \AlgGetPStops{}.
\end{note}

\subparagraph{Continuations}
\label{cha:sec:fcs-def:progressive-landman:alg:conts}

\begin{note}[\AlgAC{}]
  Clause~\ref{Landman:CB:continues} (and~\ref{Landman:CB:continues:again}) of \citeauthor{Landman:1992wh}'s definition of a continuation branch, which characterises the idea of a `continuation stretch' of some event \(g\) in world \(u\).
  We term the algorithm `\AlgAC{}':

  \begin{algorithm}[H]
    \label{PrAl:g-a-c}
    \caption{\AlgAC{}}
    \SetAlgoLined
    \DontPrintSemicolon
    \Input{\(\langle g,u \rangle_{i}\) \hfill An (indexed) event-world pairing}
    \KwResult{Continuation \hfill The continuation of \(g\) in \(u\)}
    \Begin{
      \(\text{Continuation} \longleftarrow \emptyset\)\;
      \label{PrAl:g-a-c:CSetInt}
      \(t_{s} \longleftarrow \text{start time of }g\text{ in }u\)\;
      \label{PrAl:g-a-c:ts}
      \(t_{e} \longleftarrow \text{end time of }g\text{ in }u\)\;
      \label{PrAl:g-a-c:te}
      \(g_{x} \longleftarrow g\)\;
      \label{PrAl:g-a-c:gx}
      \While{\(g_{x}\) is an event in \(u\)}
      {
        \label{PrAl:g-a-c:while:s}
        \(g_{x} \longleftarrow \emptyset\)\;
        \label{PrAl:g-a-c:gx:discard}
        \(t_{e} \longleftarrow t_{e} + 1\)\;
        \label{PrAl:g-a-c:te:plus}
        \(I \longleftarrow [t_{s},t_{e}]\)\;
        \label{PrAl:g-a-c:te:I}
        \For{\(g_{y} \in \{g_{y} \mid g_{y} \text{ is an event in } u\}\)}{
          \label{PrAl:g-a-c:for:s}
          \If{\(g_{y}\) occurs in \(I\) \emph{and} \(g_{y}\) is a stage of \(g\)}
          {
            \label{PrAl:g-a-c:for:test}
            \(\text{Continuation} \longleftarrow \text{Continuation} \cup \langle \langle g_{x},u \rangle_{i}, \langle g_{y},u \rangle_{i} \rangle\)\;
            \label{PrAl:g-a-c:C:new}
            \(g_{x} \longleftarrow g_{y}\)\;
            \label{PrAl:g-a-c:gx:new}
          }
        }
      }
      \label{PrAl:g-a-c:while:e}
      \Return{\(\text{Continuation}\)}
      \label{PrAl:g-a-c:return}
    }
  \end{algorithm}

  \AlgAC{} takes an event-world pairing \(\langle g,u \rangle_{i}\) and returns a set containing a (non-branching tree) which captures the continuation stretch of \(g\) in \(u\).

  Intuitively, \AlgAC{} starts with \(\langle g,u \rangle_{i}\).
  Then, \AlgAC{} finds the smallest event \(g^{+}\) such that \(g^{+}\) is a stage of \(g\) in \(u\), and adds \(\langle g,u \rangle_{i}, \langle g^{+},u \rangle_{i} \rangle\) as a continuation.
  Now, \(g\) may develop further in \(u\).
  So, \AlgAC{} continues to the smallest \(g^{++}\) such that \(g^{++}\) is a stage of \(g\) in \(u\) at a later time than \(g^{+}\), and adds \(\langle g^{+},u \rangle_{i}, \langle g^{++},u \rangle_{i} \rangle\) as a continuation.
  The hypothetical set \(\text{Continuation}\) is now \(\{\langle g,u \rangle_{i}, \langle g^{+},u \rangle_{i} \rangle, \langle g^{+},u \rangle_{i}, \langle g^{++},u \rangle_{i} \rangle\}\).

  This process repeats until there are no further stages of \(g\) in \(u\).
\end{note}

\begin{note}[Motivation for \AlgAC{}]
  There are two pieces of motivation for the construction of \(\text{Continuation}\).

  The first piece of motivation is \citeauthor{Landman:1992wh}'s account of the progressive.
  For, \(\sem{\text{PROG}(e, P)}_{w,g}\) checks to see if there exists \emph{some} event-world pairing \(\langle f,v \rangle\) in the continuation branch of \(e\) (in \(w\)) such that \(\sem{P}\) is true of \(f\) in \(v\).
  Hence, it is not possible, in general, to only consider the `maximal' continuation of \(g\) in \(u\).
  Rather, we must have the option of identifying the particular stage \(f'\) of \(f\) such that \(P\) is true of \(f'\).

  The second piece of motivation is the critiques of \textcite{Bonomi:1997uq} and \textcite{Szabo:2004ul} which suggest modifying \citeauthor{Landman:1992wh}'s account of the progressive to include universal quantification over branches of a continuation tree.
  Therefore, it is important to ensure that the continuation of \(g\) in \(u\) does not itself involve branching.%
  \footnote{
    I.e., this constraint rules out including both \(\langle \langle g,u \rangle_{i}, \langle g^{+},u \rangle_{i} \rangle\) and \(\langle \langle g,u \rangle_{i}, \langle g^{++},u \rangle_{i} \rangle\), as this would indicate a branch.
  }
\end{note}

\begin{note}[Construction of \AlgAC{}]
  Finally, then, we have the way in which \(\text{Continuation}\) is constructed.

  We start by initialising \(\text{Continuation}\) as an empty set (\autoref{PrAl:g-a-c:CSetInt}).
  Then, we identify the start and end times of the interval in which \(g\) takes place (Lines~\ref{PrAl:g-a-c:ts} and~\ref{PrAl:g-a-c:te}).

  The task is then to continue to expand \(\text{Continuation}\) so long as there are further stages of \(g\) in \(u\).
  We achieve this a while loop %
  (Lines~\ref{PrAl:g-a-c:while:s}--\ref{PrAl:g-a-c:while:e}) %
  that will fail when we are no longer considering an event in \(u\).
  The variable event \(g_{x}\) is initially set to \(g\), to guaranteed at least one pass through the loop (\autoref{PrAl:g-a-c:gx}).

  In a instance of the while loop we first discard \(g_{x}\) to ensure the loop will fail if there are no further stages of \(g\) (\autoref{PrAl:g-a-c:gx:discard}).
  Then, we construct an interval \(I\) by shifting forward one step in time (Lines~\ref{PrAl:g-a-c:te:plus}~and~\ref{PrAl:g-a-c:te:I}).

  At this point, we have a new interval \(I\) to consider, but no immediate event.
  So, we consider every event \(g_{y}\) which occurs in \(u\) and test to see if \(g_{y}\) happens in \(I\) \emph{and} is a stage of \(g\) (Lines~\ref{PrAl:g-a-c:for:s}~and~\ref{PrAl:g-a-c:for:test}).
  If successful, we update \(\text{Continuation}\) and \(g_{x}\) (Lines~\ref{PrAl:g-a-c:C:new}~and~\ref{PrAl:g-a-c:gx:new}).

  Here we make two assumptions:
  \begin{itemize}[noitemsep]
  \item
    First, if there is some stage of \(g\), then there is stage of \(g\) at an interval obtained by stepping one tick forward in time.%
  \footnote{
    This may be avoided by separating the test on~\autoref{PrAl:g-a-c:for:test} into two separate tests, and shifting the reassignment on~\autoref{PrAl:g-a-c:gx:new} outside the scope of the if-clause.
    Though the while-loop will then only terminate when there are no further events in \(u\).
  }
  \item
    Second, for any interval, if there is an event which is a stage of \(g\), then the event is unique.%
    \footnote{
      This assumption is key to ensure \(\text{Continuation}\) does not introduce branching.
      Though, the assumption may be avoided by choosing a different representation for the tree.
    }
  \end{itemize}

  After the stages of \(g\) in \(u\) have been exhausted, \AlgAC{} returns \(\text{Continuation}\) (\autoref{PrAl:g-a-c:return}) and terminates.
\end{note}

\subparagraph{Stops}
\label{cha:sec:fcs-def:progressive-landman:alg:stops}

\begin{note}[\AlgGetStops{}]
  To the \emph{antecedent} of Clause~\ref{Landman:CB:stops} (and~\ref{Landman:CB:stops:again}) of \citeauthor{Landman:1992wh}'s definition of a continuation branch asks whether the continuation stretch of \(g\) stops in \(u\), and considers the `maximal' event \(g'\) in \(u\) (if \(g\) stops).

  Intuitively, the algorithm `\AlgGetStops{}' searches through some continuation stretch of \(g\) in \(i\) (i.e.\ the result of \AlgAC{\((\langle g,u \rangle_{i})\)}) to identify the stopping point of \(g\) in \(u\).
  Hence, if \(g\) stops in \(u\) \AlgGetStops{} returns the relevant `maximal' event.
  Otherwise, \AlgGetStops{} does not return an event-world pairing.
  So, by inspecting the result of \AlgGetStops{} we may determine whether the antecedent of Clause~\ref{Landman:CB:stops} (and~\ref{Landman:CB:stops:again}) is fulfilled.

  Strictly, the construction of \AlgGetStops{} is generalised to cover finding the stopping points of multiple continuation stretches, but the same intuition extends to the more general case.

  After identifying the relevant stopping points of an event \(g\) in world \(u\), the following task will be to find continuations of \(g\) in other worlds.

  \AlgGetStops{} is as follows:

  \begin{algorithm}[H]
    \label{PrAl:g-s}
    \caption{\AlgGetStops{}}
    \SetAlgoLined
    \DontPrintSemicolon
    \Input{\(\text{Continuations}\) \hfill I.e.\ results of \AlgAC{} --- in general, a tree}
    \KwResult{Stops \hfill Stopping points for each event in \(\text{Continuations}\)}
    \Begin{
      \(\text{Stops} \longleftarrow \emptyset\)\;
      \label{PrAl:g-s:mk-st-Stops}
      \For{\(\langle \langle f,v \rangle_{i-1}, \langle g,u \rangle_{i} \rangle \in \text{Continuation}\)}
      {
        \label{PrAl:g-s:for:start}
        \If{\(g\text{ stops in }u\)}
        {
          \label{PrAl:g-s:if:start}
          % \(g' \longleftarrow\text{ the stopping point of }g\text{ in }u\)\;
          % \label{PrAl:g-s:max}
          \(\text{Stops} \longleftarrow \text{Stops} \cup \{\langle \langle f,v \rangle_{i-1}, \langle g,u \rangle_{i} \rangle\}\)\;
          \label{PrAl:g-s:make}
        }
      }
      \Return{\(\text{Stops}\)}
      \label{PrAl:g-s:return}
    }
  \end{algorithm}
\end{note}

\begin{note}[Construction of \AlgGetStops{}]
  The construction of \AlgGetStops{} is simple:%
  \footnote{
    \AlgGetStops{} differs in presentation from \citeauthor{Landman:1992wh}.
    Recall:
    \begin{quote}
      \begin{enumerate}
        \setcounter{enumi}{1}
      \item
        if the continuation stretch of \(e\) in \(w\) stops in \(w\), it has a maximal element \(f\) and \(f\) stops in \(w\).
        Consider the closest world \(v\) where \(f\) does not stop:\dots
      \end{enumerate}
    \end{quote}
    Clause~\ref{Landman:CB:continues} reads as an imperative to find the relevant maximal event.
    Still, so long as \AlgGetStops{} is applied to results \AlgAC{}, the result is equivalent, for the set of continuation given by \AlgAC{} will include the relevant maximal event.
  }

  \AlgGetStops{} initialises \(\text{Stops}\) as an empty set (\autoref{PrAl:g-s:mk-st-Stops}), and for every element of \(\text{Continuations}\), \AlgGetStops{} takes the right-event-world pairing and queries whether the event stops in the world (Lines~\ref{PrAl:g-s:for:start}--\ref{PrAl:g-s:if:start}).
  If the event stops, \AlgGetStops{} takes the point at which the event stops and adds the element of \(\text{Continuations}\)%
  % with the maximal continuation substituted for the event
  to \(\text{Stops}\) (\autoref{PrAl:g-s:make}).
  \AlgGetStops{} returns \(\text{Stops}\) (\autoref{PrAl:g-s:return}) and terminates.
\end{note}

\begin{note}
  Applied to a single instance of \AlgAC{}, \AlgGetStops{} will either return an empty set or a singleton set containing a pairing in which the event of the right element stops.
  And, applied to a set containing multiple results of \AlgAC{}, \AlgGetStops{} will either return an empty set or a set of size bounded by the instances of \AlgAC{}.
\end{note}

\begin{note}
  Role of \AlgGetStops{} is somewhat artificial.
  It would be far more effective to identify a stopping event in the construction of \AlgAC{}.

  However, after introducing \AlgFindBranches{} to find the continuations of a stopping event in some possible world (\autoref{cha:sec:fcs-def:progressive-landman:alg:branches}) we will introduce variant of \AlgGetStops{} (\autoref{cha:sec:fcs-def:progressive-landman:alg:R-stops}).
\end{note}

\subparagraph{Branches}
\label{cha:sec:fcs-def:progressive-landman:alg:branches}

\begin{note}
  The \emph{consequent} of Clause~\ref{Landman:CB:stops} (and~\ref{Landman:CB:stops:again}) of \citeauthor{Landman:1992wh}'s definition of a continuation branch requires identifying the continuation of an event \(g\) which stops in \(u\) in some possible world \(u'\), if the continuation exists.

  The algorithm `\AlgFindBranches{}' is designed to return such continuations of event-world pairings in general.%
  \footnote{
    As with \AlgGetStops{}, the existence of a continuation is given by contrasting the inputs to the output of \AlgFindBranches{}.
  For, if \(\langle \langle f,v \rangle_{i-1}, \langle g,u \rangle_{i}\rangle\) is an element of the input to \AlgGetStops{}, then there is a continuation of \(g\) in some other world \(u'\) just in case \(\langle \langle f,v \rangle_{i-1}, \langle g,u' \rangle_{i}\rangle\) is an element of the output.
  }
  Though, the primary use-case is applying \AlgFindBranches{} to the result of \AlgGetStops{}.

  \AlgFindBranches{} is as follows:

  \begin{algorithm}[H]
    \label{PrAl:find-branches}
    \caption{\AlgFindBranches{}}
    \SetAlgoLined
    \DontPrintSemicolon
    \Input{\(\langle \langle f,v \rangle_{i-1}, \langle g,u \rangle_{i}\rangle\) \hfill An element of a tree\\
      \(e\) \hfill An event \\
      \(w\) \hfill A world
    }
    \KwResult{Branches \hfill A set containing event-world pairs \\
      \hfill such that \(g\) continues in \(u'\)}
    \Begin{
      \(\text{CloseWorlds} \longleftarrow \{u' \mid u' \text{ is among the closest world to } u\}\)\;
      \label{PrAl:find-branches:close}
      \(\text{ReasonableWorlds} \longleftarrow \text{CloseWorlds} \cap R(e,w)\)\;
      \label{PrAl:find-branches:loop:R}
      \(\text{Branches} \longleftarrow \emptyset\)\;
      \For{\(u' \in \text{ReasonableWorlds}\)}{
        \label{PrAl:find-branches:loop:start}
        \If{\(g\) is an event in \(u'\)}{
          \(\text{Branches} \longleftarrow \text{Branches} \cup \{ \langle \langle f,v \rangle_{i-1}, \langle g,u' \rangle_{i}\rangle \}\)\;
        }
      }
      \label{PrAl:find-branches:loop:end}
      \Return{\(\text{Branches}\)}
    }
  \end{algorithm}
\end{note}

\begin{note}
  From a broad perspective, \AlgFindBranches{} considers some event-world pairing \(\langle g,u \rangle\) and returns a collection of event-world pairings \(\langle g,u' \rangle\) where \(u'\) is some possible world.

  From this broad perspective, the core of \AlgFindBranches{} is a for-loop (Lines~\ref{PrAl:find-branches:loop:start}--\ref{PrAl:find-branches:loop:end}) which scans through some collection of possible worlds and identifies those worlds in which \(g\) is present in the possible world.

  From a more nuanced perspective, the core of \AlgFindBranches{} is with how the possible worlds are specified (i.e.\ Lines~\ref{PrAl:find-branches:close} and~      \ref{PrAl:find-branches:loop:R}).
\end{note}

\begin{note}
  Let us begin by considering \AlgFindBranches{} applied to a particular event-world pair \(\langle g,u \rangle\).

  First, observe that \(g\) is an event in \(u\).
  Hence, \AlgFindBranches{} return some \(\langle g,u' \rangle\)-pairing such that \(g\) happened in \(u'\) in parallel to \(g\) happening in \(u'\).
  Stressed, \AlgFindBranches{} \emph{does not} return \(\langle g,u' \rangle\)-pairing such that \(g\) happened in \(u'\) even though \(g\) didn't happen in \(u'\).
  This follows \citeauthor{Landman:1992wh}'s definition.
  For, while \(g\) stopping in \(u\) means only that \(g\) does not continue in \(u\) (cf.~\autoref{def:Landman:Stops} on \autopageref{def:Landman:Stops}).

  In turn, this means \(\langle g,u' \rangle\) may be passed as an argument to \AlgFindBranches{}.
  So, second, observe Line~\autoref{PrAl:find-branches:close} considers possible worlds close to the argument event-world pairing.
  Hence, repeated application of \AlgFindBranches{} may lead to `drift'.
  \(u'\) may be close to \(u\), and \(u''\) may be close to \(u'\), but \(u''\) need not be close to \(u\).
  This is an important feature of \citeauthor{Landman:1992wh}'s account of the progressive.
  Recall the instance of the progressive from \autopageref{prog:max:bad}:
  \begin{enumerate}
  \item
    Max is crossing the street.
  \end{enumerate}
  Also recall \ref{prog:max:bad} is true even though Max is hit by a bus.
  Now, in close worlds to the actual world, Max may be hit by a bus in various different ways, and hence there is no close world \emph{to the actual world} in which Max crosses the street.
  However, by the process of drifting through possible worlds, we expect to find a possible world in which Max is not hit by a bus (and crosses the street).
  {
    \color{red} not to counterfactual reduction?
  }

  However, in principle it seems possible to drift though possible worlds so that (almost) any arbitrary eventuality results from some event.
  The point is especially clear if we combine \AlgFindBranches{} with \AlgAC{} to allow events to develop.

  Now, consider:
  \begin{enumerate}[label=\arabic*., ref=(\arabic*)]
    \setcounter{enumi}{1}
  \item
    Cyril is stating the first ten digits of (the decimal representation of) \(\pi\).
  \end{enumerate}
  Cyril does not know the first ten digits of \(\pi\) but has started with `\(3\)' and then `\(1\)'.
  In the actual world, Cyril it interrupted, but in a close world Cyril utters `\(4\)', and we may continue drifting through possible worlds until Cyril utters the first ten digits of \(\pi\).%
  \footnote{
    And, indeed any finite sequence of digits from (the decimal representation of) \(\pi\).
    See \citeauthor{Landman:1992wh} (\citeyear[\S2.3]{Landman:1992wh}) for additional examples and discussion.
  }

  Considerations such as the above motivate Line~\ref{PrAl:find-branches:loop:R}.
  Recall:
  \begin{quote}
    \(v \in R(e, w)\) iff there is a reasonable chance on the basis of what is internal to \(e\) in \(e\) that \(e\) continues in \(w\) as far as it does in \(v\).%
    \mbox{ }\hfill\mbox{(\citeyear[26]{Landman:1992wh})}
  \end{quote}
  Consider a possible world where \(e\) continues, and query whether \(e\) may have continued in \(v\).

  Phrased otherwise, \(\text{CloseWorlds}\) captures closeness between worlds, while \(R(e,w)\) captures closeness between events.

  Perhaps uttering `\(4\)' in place of `\(5\)' is reasonable, but as we progress through `\(1\)', `\(5\)', `\(9\)', `\(2\)', \dots it is, intuitively, quite unreasonable.
  For that Cyril does not know the first ten digits of \(\pi\).
  Hence, Cyril uttering the correct sequence of digits would require considerable luck.%
  \footnote{
    If ten digits seems borderline, extend the sequence.
    Given a decimal representation and random choice by Cyril, the probability of the Cyril uttering the correct sequence is \(.1^{l}\) where \(l\) is the length of the sequence.
  }
  In addition, note the set of reasonable worlds is determined by two arguments given to \AlgFindBranches{} that are independent of the event-world pairing given as an argument.
  Intuitively, and in practice, these arguments correspond to the initial event and world from which the continuation tree is constructed.
  Therefore, the extent to which \AlgFindBranches{} may lead to drift through possible worlds is limited by the initial event in the actual world.
\end{note}

\begin{note}
  Still, \citeauthor{Landman:1992wh}'s account is difficult.

  Unclear how to think about \(e\) continuing in \(w\).
  Can't be closest worlds in which \(e\) does not stop.
  For, by assumption, drifted from \(w\).

  Problem to solve is clear.
  Allow event to drift, but not indefinitely.
  And, restriction given by initial event in actual world.
  Events in progress.
  Failure not because there is no possibility, but because things required to secure possibility stray too far from the event as it has developed in the actual world.

  So, \(g\) is a reasonable stage of \(e\).

  Following \citeauthor{Landman:1992wh}, no way to reduce to similarity between worlds.
  For, \citeauthor{Landman:1992wh} holds that unreasonable even though what happened.
  Suppose Cyril hadn't been interrupted.
  Still unreasonable.

  Best I can do, if \(g\) then \(e\) from \(w\), where restrict interest in \(w\) up until \(e\) happening and interest in worlds up until \(g\) happens.%
  \footnote{
    \color{red}
    May captured by attention to fact.

  \citeauthor{Veltman:2005tj}'s (\citeyear{Veltman:2005tj}) revision to~\citeauthor{Tichy:1976tp}'s (\citeyear{Tichy:1976tp}) counterexample to \citeauthor{Lewis:1979vm}'s (\citeyear{Lewis:1973th,Lewis:1979vm}) theory of counterfactuals.

  Some `facts' are fixed, and it is not possible to alter these facts under a counterfactual assumption.
  \(e\) has been set in motion.
  Now, keeping \(e\) fixed, if not X then e would continue.

  Fix what has been set in motion by \(e\), keep all laws the same.
  Then, buses at different times, but Mary doesn't get superpowers.
  }
  So, getting \(g\) `because' continuation of \(e\).

  These are different.
  For, we look to see whether \(g\) is the case in \(u'\) but also that \(g\) is the result.
  Though, this clause is indirect.
  Really part of \AlgAC{}.
  Inefficient, but better placed here.
  As, we won't include in tree if not returned by this algorithm.
\end{note}

\begin{note}
  Motivated restriction on world.
  What about converse?
  Is it possible to consider only the `reasonable' worlds?
  No, because requires that search for \(g\).
  Even after drifting, don't get a continuation of \(g\) in close worlds.
\end{note}

\begin{note}
  See if the event at some world continues in some other world.
  \citeauthor{Landman:1992wh} assumes closest world, but following \textcite[37]{Szabo:2004ul} and~\textcite{Bonomi:1997uq}, allow for multiple close worlds.%
  \footnote{
    See in particular \citeauthor{Bonomi:1997uq}'s discussion of the `multiple-choice paradox' (\citeyear[\S4]{Bonomi:1997uq}).
  }
  Note, importance of \(e\) and \(w\)!
  No shift in index, as the function identifies the same event in some other possible world.
\end{note}

\begin{note}
  Two general notes:

  \citeauthor{Landman:1992wh}, existential, so risk of finding some continuation in `unreasonable' worlds.
  Now suggested universal, then failure in an `unreasonable' world does not entail failure of the progressive.
\end{note}

\begin{note}
  Here, shift the index.
  When run recursive algorithm, each call of the algorithm will explore branches.
  Increase index for next call.

  Also, note, \AlgGetPStops{}.
  Pass the result of \AlgGetPStops{}, and this will get 
\end{note}

\subparagraph{Reasonable stops}
\label{cha:sec:fcs-def:progressive-landman:alg:R-stops}

\begin{note}
  In \autoref{cha:sec:fcs-def:progressive-landman:alg:R-stops} we gave a reconstruction of the antecedent of Clause~\ref{Landman:CB:stops} (and~\ref{Landman:CB:stops:again}) of \citeauthor{Landman:1992wh}'s definition of a continuation branch.

  In this section, we motivate and detail an alternative algorithm.

  The basic observations are
  \begin{enumerate}[label=\arabic*., ref=(\arabic*), noitemsep]
  \item
    \label{landman:alg:R-stops:ob:1}
    If an agent performs some action \(\alpha\), then so long as \(\alpha\) is not instantaneous, there is some prior event such that the agent is \(\alpha\)ing.
  \item
    \label{landman:alg:R-stops:ob:2}
    It is not the case that Observation~\ref{landman:alg:R-stops:ob:1} holds, in general.
  \end{enumerate}
  To resolve this tension, we consider a modification of \AlgGetStops{} which returns `reasonable' rather than actual stops of an event.
  We term the modification `\AlgGetPStops{}'.

  First, we substantiate Observations~\ref{landman:alg:R-stops:ob:1} and~\ref{landman:alg:R-stops:ob:2}.
\end{note}

\begin{note}
  Observation~\ref{landman:alg:R-stops:ob:1} follows from  Clause~\ref{Landman:CB:stops} (and~\ref{Landman:CB:stops:again}) of \citeauthor{Landman:1992wh}'s definition of a continuation branch.
  For, Clause~\ref{Landman:CB:stops} (and~\ref{Landman:CB:stops:again}) is a conditional:
  \begin{quote}
    \begin{enumerate}
      \setcounter{enumi}{1}
    \item
      if the continuation stretch of \(e\) in \(w\) stops in \(w\), it has a maximal element \(f\) and \(f\) stops in \(w\).
      Consider the closest world \(v\) where \(f\) does not stop: \dots
    \end{enumerate}
  \end{quote}
  Hence, for an event \(e\) and world \(w\), Clause~\ref{Landman:CB:stops} only includes the closest world where \(e\) does not stop \emph{if} \(e\) stops in \(w\).
  So, if \(e\) does not stop in the actual world, and \(e\) develops into an event \(e'\) such that the agent \(\alpha\)s in \(e'\), then it will be true that the agent is \(\alpha\)ing with respect to \(e\).
\end{note}

\begin{note}
  We now turn to Observation~\ref{landman:alg:R-stops:ob:2}.

  Observation~\ref{landman:alg:R-stops:ob:2} is the converse of the above%
  \footnote{
    See the discussion of the imperfective paradox on~\autopageref{imperfective-paradox:intro}.
  }
  observation that an event in which an agent \(\alpha\)s does not need to happen for it to be true that the agent is \(\alpha\)ing.
  For, restated, Observation~\ref{landman:alg:R-stops:ob:2} reads:%
  \footnote{
    \citeauthor{Landman:1992wh} considers a similar observation under the term `the problem of non-interruptions' (\citeyear[14--17,30--31]{Landman:1992wh}).
    \citeauthor{Landman:1992wh}'s suggestion is to distinguish events by the perspective of the agent. (\citeyear[31]{Landman:1992wh})
    The basic idea, as I understand \citeauthor{Landman:1992wh} is to deny that the event \(e'\) in which the agent \(\alpha\)s is sufficiently distinct from any prior event \(e\).
    Hence, it seems that strictly speaking \citeauthor{Landman:1992wh} does not endorse Observation~\ref{landman:alg:R-stops:ob:2}.
  }
  \begin{enumerate}[label=\arabic*\('\)., ref=(\arabic*\('\))]
    \setcounter{enumi}{1}
  \item
    There are cases in which \(e\) is not an instance of an agent \(\alpha\)ing, though \(e\) develops into \(e'\) and \(e'\) is an event in which the agent \(\alpha\)s.
  \end{enumerate}

    We present a pair of \illu{1} with both provide instances of Observation~\ref{landman:alg:R-stops:ob:2}.
    Following, we turn to theoretical motivation.
\end{note}

\begin{note}
  \begin{illustration}[Flipping coins]
    \label{illu:fc:coins}
    Agent is flipping a coin in the air and recording how it lands.%
    \footnote{
      Cf.~\textcite{Gelman:2002ww} and~\textcite{Keller:1986tz}.
    }
    The coin lands heads ten times in a row.

    The following seem true:
    \begin{itemize}
    \item
      The coin landed heads ten in a row.
    \item
      The agent landed the coin heads ten times in a row.
    \end{itemize}
    However, the following seem false:
    \begin{itemize}
    \item
      The coin was landing heads ten times in a row.
    \item
      The agent was landing a coin heads ten times in a row.
    \end{itemize}
    \vspace{-\baselineskip}
  \end{illustration}
  Intuition is clear.
  No time prior to the coin landing heads on the tenth flip was the coin landing heads ten times in a row.
  Consider prior.
  \begin{itemize}
  \item
    I am landing this coin heads ten times in a row.
  \end{itemize}
  Only reasonable interpretation is if the agent will continue to keep flipping the coin until the coin lands heads ten times in a row.
  However, if the agent is limited to ten flips, absurd.

  Though, the tenth flip is a clear continuation of the event containing the previous nine coin flips.

  Possible objection.
  Event was not the result of the agent.
  And, if agent then works out because prior to agent completing action, progressive.
  However, the action is flipping the coin so that it lands heads.
  No different from encrypting a message so that it will not be understood by the messenger.
  Once sent, the agent has no involvement in the messenger's understanding.
  Though, given a sufficiently strong algorithm, it seems true that the agent is encrypting the message when working through the algorithm.
\end{note}

\begin{note}[Second]
  \begin{illustration}[Chess III]
    \label{illu:fc:chess:III}
    Consider as the game state on the left transitions to the game state on the right.

    \noindent\mbox{ }\hspace{2em}%
    \begin{adjustbox}{minipage=.4\linewidth,scale=.75}%
      \centering%
      \newchessgame[%
      setwhite={ka6,pa5,pb6,pc7,rg2,bh1},%
      addblack={ka8,rg8,na3},%
      ]%
      \setchessboard{showmover=false}%
      \chessboard%
    \end{adjustbox}%
    \hfil%
    \begin{adjustbox}{minipage=.4\linewidth,scale=.75}%
      \centering%
      \newchessgame[%
      setwhite={ka6,pa5,pb6,pc7,rg2,bh1},%
      addblack={ka8,rc8,na3},%
      ]%
      \setchessboard{showmover=false}%
      \chessboard%
    \end{adjustbox}%
    \hspace{2em}\mbox{ }

    On the left, white has just moved their pawn from c6 to c7.
    Though, white is quite bad at chess.
    And, the strength of their pieces on the board is the result of black intentionally playing even worse.
    So, it is not true that:
    \begin{itemize}
    \item
      White is winning the game of chess.
    \end{itemize}
    For example, if black moves their rook to h8 then on the following turn white may move their pawn to c8 and exchange it for a rook which black may then capture, etc.\dots

    Still, on the following turn, black moves their rook from g8 to c8, as shown on the right.
    As a result:
    \begin{itemize}
    \item
      White wins the game of chess.
    \end{itemize}
  \end{illustration}

  May argue about the interpretations of `wins'.
  For, white still needs to make a move.
  However, any move made by white results in checkmate for black.
  Hence, true.

  In contrast to \autoref{illu:fc:coins}, \autoref{illu:fc:chess:III} does not involve luck.
  Instead, \autoref{illu:fc:chess:III} involves the action by some other agent which leads to desired outcome.
  %
  \footnote{
    More generally, one consider the structure of \autoref{illu:fc:chess:III} applied to any competitive activity in which a mistake by one participant hands victory to the other participant.
    It seems clear to me that a win does not entail that the participant was winning at any point throughout the game.
    E.g.\ a racing car driver may be handed a win due to their competitor having a bad pit stop, or a football team may be handed a win due to an unexpected injury on the other team, etc.
  }
\end{note}

\begin{note}[Theoretical motivation]
  On the one hand, intuition.
  On the other hand, motivated by \citeauthor{Landman:1992wh}'s analysis.
  Distinction between event and world.
  Interested in the event getting to completion regardless of what happens in the world.
  Hence, allows us to shift to possible worlds.
  Progressive is not false because of contingencies of the actual world.
  By parallel, progressive is not true because of contingencies of actual world.
\end{note}

\begin{note}
  Given this, understand both \illu{1}.
  Deviance.

  Once noted, plausible to distinguish where in event progressive comes true.
  Final \illu{}.
  \begin{illustration}[Choosing cards]
    An agent is presented with a shuffled deck of playing cards and is asked to choose a number \(1 < n \leq 52\).
    After the agent announces the number, the presenter of the deck of cards will remove \(n - 1\) cards from the top of the deck and reveal the \(n\)th card.

    The agent considers their options with gravitas.
    \begin{itemize}
    \item
      First, the agent decides they will choose and odd number
    \item
      Second, the agent decides they will choose a card from the top half of the deck.
    \item
      Third, the agent decides they will choose a prime number.
    \item
      Finally, the agent chooses \(17\).
    \end{itemize}
    The agent is unaware that the pack is shuffled so that every card corresponding to a prime number is a red card, though the \(9\)th card isn't a red card.
    Hence, after the third choice, but not before, it is true that:
    \begin{itemize}
    \item
      The agent is choosing a red card.
    \end{itemize}
    \vspace{-\baselineskip}
  \end{illustration}
  So, before third decision, agent may have chosen a non-prime number.
  In particular, \(9\) is not a prime number and the \(9\)th card is not a red card.
  However, after the decision, no turning back.
\end{note}

\begin{note}
  So, Observations~\ref{landman:alg:R-stops:ob:1} and~\ref{landman:alg:R-stops:ob:2}.

  The suggestion is simple.
  Rather than search the continuation of an event \(f\) in world \(v\) 

  Slight difficulty is closest world.
  Instead, closest worlds.
  Intuitively, obtained from counterfactuals against what happened in the actual world.
  However, reasonable, as these are worlds in which the event continues.
\end{note}

\begin{note}
  Modification builds on principles for \AlgFindBranches{}.
  For, \AlgFindBranches{} finds continuation of event in nearby world.


  \AlgGetPStops{}:

  \begin{algorithm}[H]
    \label{PrAl:g-s}
    \caption{\AlgGetPStops{}}
    \SetAlgoLined
    \DontPrintSemicolon
    \Input{%
      \(\text{Continuation}\) \hfill I.e.\ the result of \AlgAC{} --- in general, a tree \\
      \(e\) \hfill An event \\
      \(w\) \hfill A world
    }
    \KwResult{ReasonableStops \hfill \emph{Reasonable} stopping points \\
      \mbox{ } \hfill for each event in \(\text{Continuation}\)}
    \Begin{
      \(\text{R-Stops} \longleftarrow \emptyset\)\;
      \For{\(\langle \langle f,v \rangle_{i-1}, \langle g,u \rangle_{i} \rangle \in \text{Continuation}\)}
      {
        \(\text{Branches} \longleftarrow \AlgFindBranches{(\langle \langle f,v \rangle_{i-1}, \langle g,u \rangle_{i} \rangle, e,w)}\)\;
        \For{\(\langle \langle f,v \rangle_{i-1}, \langle g,u' \rangle_{i} \rangle \in \text{Branches}\)}
        {
          \If{\(g\text{ stops in }u'\)}
          {
            \(\text{R-Stops} \longleftarrow \text{R-Stops} \cup \{\langle \langle f,v \rangle_{i-1}, \langle g,u' \rangle_{i} \rangle\}\)\;
          }
        }
      }
      \Return{\(\text{R-Stops}\)}
    }
  \end{algorithm}

  Test%
  \footnote{
    Expanding and simplifying a little, the body of \AlgGetPStops{} reads:

    \begin{algorithm}[H]
      % \SetAlgoLined
      \DontPrintSemicolon
      \Begin{
        \(\text{R-Stops} \longleftarrow \emptyset\)\;
        \For{\(\langle \langle f,v \rangle_{i-1}, \langle g,u \rangle_{i} \rangle \in \text{Continuation}\)}
        {
          \(\text{CloseWorlds} \longleftarrow \{u' \mid u' \text{ is among the close world to } u\}\)\;
          \(\text{ReasonableWorlds} \longleftarrow \text{CloseWorlds} \cap R(e,w)\)\;
          \For{\(u' \in \text{ReasonableWorlds}\)}
          {
            {
              \If{\(g\text{ is an event in }u'\text{ and }g\text{ stops in }u'\)}
              {
                \(\text{R-Stops} \longleftarrow \text{R-Stops} \cup \{\langle \langle f,v \rangle_{i-1}, \langle g,u' \rangle_{i} \rangle\}\)\;
              }
            }
          }
        }
        \Return{\(\text{R-Stops}\)}
      }
    \end{algorithm}
  }

  \AlgGetPStops{} is a variation on \AlgGetStops{}.

  The for-loop from \AlgGetStops{}, repeated, and check from \AlgGetStops{} is repeated.
  However, check within a for-loop for whether there is a `reasonable world' in which the event of some event-world pairing stops.
\end{note}

\begin{note}
  Note, branch.
  However, only if result of \AlgGetPStops{} is part of tree.
  And, antecedent of conditional.
  Branches will be included, but only when calling \AlgFindBranches{}.
  This we now turn to.
\end{note}


\subparagraph{Build tree}
\label{cha:sec:fcs-def:progressive-landman:alg:tree}

\begin{note}
  With \AlgAC{}, \AlgGetStops{}/\AlgGetPStops{}, and \AlgFindBranches{} in hand, we now turn to \AlgDevelopTree{}, a recursive algorithm which completes any tree.

  Focus here is on linking \AlgAC{}, \AlgGetStops{}/\AlgGetPStops{}, and \AlgFindBranches{} in the same way as \citeauthor{Landman:1992wh}, and clarifying the `\dots and we continue as above, etc.' from \ref{Landman:CB:stops:again}.

  \begin{algorithm}[H]
    \label{PrAl:dev-tree}
    \caption{\AlgDevelopTree{}}
    \SetAlgoLined
    \DontPrintSemicolon
    \Input{%
      \(\text{Tree}\) \hfill A partially completed tree\\
      \(e\) \hfill The initial event of the tree\\
      \(w\) \hfill The initial world of the tree\\
      \(n\) \hfill An index to keep track recently added branches\\
    }
    \KwResult{Tree expanded so that there are no further stopping events}
    \Begin{
      \label{PrAl:dev-tree:start}
      \(\text{Stems} \longleftarrow \{ \langle g,u \rangle_{n} \mid \exists f,v \colon \langle \langle f,v \rangle_{n - 1}, \langle g,u \rangle_{n}\rangle \in \text{Tree}\}\)\;
      \label{PrAl:dev-tree:Extend:start}
      \label{PrAl:dev-tree:Extend:Stems}
      \(\text{GrownStems} \longleftarrow \emptyset\)\;
      \label{PrAl:dev-tree:Extend:FreshContsVar}
      \For{\(\langle g,u \rangle_{n} \in \text{Stems}\)}
      {
        \label{PrAl:dev-tree:Extend:Loop:start}
        \(\text{Growth} \longleftarrow \AlgAC{}(\langle g,u \rangle_{n})\)\;
        \(\text{GrownStems} \longleftarrow \text{GrownStems} \cup \text{Growth}\)\;
      }
      \label{PrAl:dev-tree:Extend:end}
      \(\text{Tree} \longleftarrow \text{Tree} \cup \text{GrownStems}\)\;
      \label{PrAl:dev-tree:Extend:merge}
      \textcolor{comment}{\texttt{//} \(\text{Stops} \longleftarrow \AlgGetStops{}(\text{GrownStems})\)}\;
      \label{PrAl:dev-tree:Stops:Land}%
      \(\text{Stops} \longleftarrow \AlgGetPStops{}(\text{GrownStems})\)\;
      \label{PrAl:dev-tree:Stops:Me}
      \eIf{\(\text{Stops} == \emptyset\)}
      {
        \label{PrAl:dev-tree:Stops:cond:start}
        \textbf{return} \(\text{Tree}\)\;
        \label{PrAl:dev-tree:Stops:cond:no-stops-finish}
      }
      {
        \label{PrAl:dev-tree:Stops:cond:else:start}
        \(\text{Tree} \longleftarrow \text{Tree} \cup \text{Stops}\)\;
        \label{PrAl:dev-tree:Stops:cond:tree-fix}
        \(\text{Branches} \longleftarrow \emptyset\)\;
        \label{PrAl:dev-tree:Stops:cond:else:futureB:start}
        \For{\(\langle \langle g,u \rangle_{n}, \langle h,u \rangle_{n+1}\rangle \in \text{Stops}\)}
        {
          \label{PrAl:dev-tree:Stops:cond:else:futureB:loop:start}
          \(\text{Temp} \longleftarrow \AlgFindBranches{}(\langle \langle g,u \rangle_{n}, \langle h,u \rangle_{n+1}\rangle, e, w)\)\;
          \label{PrAl:dev-tree:Stops:cond:else:futureB:loop:getBranches}
          \(\text{Branches} \longleftarrow \text{Branches} \cup \text{Temp}\)\;
          \label{PrAl:dev-tree:Stops:cond:else:futureB:loop:gather}
        }
        \label{PrAl:dev-tree:Stops:cond:else:futureB:end}
        \eIf{\(\text{Branches} == \emptyset\)}{
          \label{PrAl:dev-tree:Stops:cond:else:futureB:process:start}
          \textbf{return} Tree\;
          \label{PrAl:dev-tree:Stops:cond:else:futureB:process:cancel}
        }
        {
          \(\text{Tree} \longleftarrow \text{Tree} \cup \text{Branches}\)\;
          \label{PrAl:dev-tree:Stops:cond:else:futureB:process:expand}
          \AlgDevelopTree{}(\(\text{Tree}, e,w, n+1\))\;
          \label{PrAl:dev-tree:Stops:cond:else:futureB:process:end}
        }
        \label{PrAl:dev-tree:Stops:cond:else:end}
      }
      \label{PrAl:dev-tree:Stops:cond:end}
    }
  \end{algorithm}

  \AlgDevelopTree{} processes single instance of branching on each run.
  So, given a tree, where terminal nodes are event world pairing \(\langle f,v \rangle_{i}\) such that \(f\) \emph{may} continue in \(v\).
  Term these terminal nodes `stems'.

  Three tasks.
  \begin{enumerate}
  \item
    Extend stems.%
    \hfill%
    Lines~\ref{PrAl:dev-tree:Extend:start}--\ref{PrAl:dev-tree:Extend:merge}.
  \item
    Identify any (immediate) branches of the stems.%
    \hfill%
    \autoref{PrAl:dev-tree:Stops:Land} or \autoref{PrAl:dev-tree:Stops:Me}.
  \item
    Process the (immediate) branches of the stems.%
    \hfill%
    Lines~\ref{PrAl:dev-tree:Stops:cond:start}--\ref{PrAl:dev-tree:Stops:cond:end}.
  \end{enumerate}

  \begin{itemize}
  \item
    Extend
    \begin{itemize}
    \item
      Lines~\ref{PrAl:dev-tree:Extend:start}--\ref{PrAl:dev-tree:Extend:end} extend tree given as input with continuations of all terminal branches.

      \autoref{PrAl:dev-tree:Extend:Stems}, collect all the fresh branches.
      \autoref{PrAl:dev-tree:Extend:FreshContsVar} create a set.
      Lines \ref{PrAl:dev-tree:Extend:Loop:start}--\ref{PrAl:dev-tree:Extend:end} loop over fresh branches, adding extensions to set.
    \item
      \autoref{PrAl:dev-tree:Extend:merge}, merge continuations.
      Now have extension of tree.
      However, only for fresh branches.
      Interest is whether there is further branching.
      Remainder of algorithm checks for branching, and for whether to continue.
    \end{itemize}
  \item Choices
    \begin{itemize}
    \item
      \autoref{PrAl:dev-tree:Stops:Land} gets \citeauthor{Landman:1992wh}.
    \item
      \autoref{PrAl:dev-tree:Stops:Me} gets revised.
    \end{itemize}
  \item Branches
    \begin{itemize}
    \item
      \autoref{PrAl:dev-tree:Stops:cond:start} to \autoref{PrAl:dev-tree:Stops:cond:end}, determining whether to pursue completion.
      Else 
      \begin{itemize}
      \item
        Have branches from \autoref{PrAl:dev-tree:Stops:Me}.
        \autoref{PrAl:dev-tree:Stops:cond:start} is simple check.
        If no stops, then \autoref{PrAl:dev-tree:Stops:cond:no-stops-finish} return Tree.
        Done.
      \item
        Else, some stops.
        Lines~\ref{PrAl:dev-tree:Stops:cond:else:start}--\ref{PrAl:dev-tree:Stops:cond:else:end}.
      \item
        Given stops, does the event continue in some other world?
        \autoref{PrAl:dev-tree:Stops:cond:else:futureB:start}, variable to collect branches.
        \autoref{PrAl:dev-tree:Stops:cond:else:futureB:loop:start} starts loop over stops.
        \autoref{PrAl:dev-tree:Stops:cond:else:futureB:loop:getBranches}, get branches via function.
        \autoref{PrAl:dev-tree:Stops:cond:else:futureB:loop:gather}, add these to future branches.
      \item
        \autoref{PrAl:dev-tree:Stops:cond:else:futureB:process:start} to \autoref{PrAl:dev-tree:Stops:cond:else:futureB:process:end} process future branches.
        \begin{itemize}
        \item
          \label{PrAl:dev-tree:Stops:cond:else:futureB:process:start} check to see if there are.
        \item
          \autoref{PrAl:dev-tree:Stops:cond:else:futureB:process:cancel} if none, then nothing more to do.
        \item
          Else, branched, so \autoref{PrAl:dev-tree:Stops:cond:else:futureB:process:expand}, expand tree and \autoref{PrAl:dev-tree:Stops:cond:else:futureB:process:end} recursive call, with index shifted.
        \end{itemize}
      \end{itemize}
    \end{itemize}
  \end{itemize}
  Now, start with \(\langle e,w \rangle\).
  Slight thing, ordered pairs.
  So, pass a vacant event.
  Run, \AlgDevelopTree{}(\(\langle \langle -,- \rangle_{0}, \langle e,w \rangle_{1} \rangle, e, w, 1\))
\end{note}

\begin{note}
  \autoref{PrAl:dev-tree:Stops:cond:tree-fix} is important given \AlgGetPStops{}.
  For, reasonable stop and may be stopping point which shows the progressive is false.%
  \footnote{
    Careful consider shows that we don't include all branches.
    This is an interesting issue.
    For, truth of the progressive looks to see whether description is true of event.
    May think that, given stops in reasonable worlds, should also consider continuations.

    Relevant possibility is in all cases where event stops, continuation where thing happens.
    But, some continuation in possible world where didn't stop and thing didn't happen.

    This doesn't strike me as plausible.

    Note, only relevant on first pass through the tree, and \AlgAC{} applies to all stops.

    Still, if inclined, cover the case by modifying \AlgGetPStops{} where conditional is switched to false, and adding the results to \(\text{Tree}\) prior to \autoref{PrAl:dev-tree:Stops:cond:start}.
    }
\end{note}

\begin{note}
  This gives us continuation tree.
  Following \citeauthor{Szabo:2004ul}, revise account of the progressive.

  \begin{quote}
    \begin{enumerate}[label=(\Roman*), ref=(\Roman*)]
      \setcounter{enumi}{5}
    \item
      \emph{Prog}[\(\varphi\)] is true at \(t\) in \(w\) iff there is an \(e\) at \(t\) in \(w\) and for every \(\langle e^{\ast}, w^{\ast} \rangle\) on the continuation tree for \(e\) in \(w\) if \(\varphi\) is not true of \(e^{\ast}\) at \(w^{\ast}\) then there is an \(\langle e', w' \rangle\) on the continuation tree for \(e\) in \(w\) such that \(e'\) is a continuation of \(e^{\ast}\) in \(w'\) and \(\varphi\) is true of \(e'\) at \(w'\).%
      \mbox{ }\hfill\mbox{(\citeyear[37]{Szabo:2004ul})}
    \end{enumerate}
  \end{quote}

  Note, here, for every there exists.
  Existential is important, because there are so many events.

  As an aside, this plausibly also gets failure of \BoyVS{}.
  Surprising, this seems to capture all the entailments that \citeauthor{Boylan:2020aa} is interested in.
  Though, I think we've just reduced ability to choice of action.
\end{note}

\begin{note}
  \pevent{} just in case there is some action avaiable to the agent, and were the agent to perform the action, the agent would be \(\alpha\)-ing.
\end{note}

\begin{note}
  With respect to concluding, reasonable constraint is neat.
  How the agent is in the actual world.
  And, shifts to closest worlds avoid blunders.
\end{note}

\begin{note}
  False negatives.
  \(\varphi\) to \(prog \varphi\).
  But, this entailment isn't really of interest.
  At least get this in the case of concluding.
\end{note}

\paragraph{Summary}

\begin{note}
  \citeauthor{Landman:1992wh}'s account with two key changes.

  First, allow for multiple branching.
  Motivated by concerns from \citeauthor{Bonomi:1997uq} and \citeauthor{Szabo:2004ul}.
  Important for \fc{1}, as explore all paths in reasoning.

  Second, possible stops.
  Other ways things could have gone.
  Motivated by intuitive instances of the progressive.
  Important for \fc{1}, again, all parts in reasoning.

  So, basically, progressive is true just in case, some event in progress, and no matter how the event develops, there is always some possibility in which complete.
  Possibility is existential as shift to possible world.
\end{note}

\paragraph{The progressive and ability}

\begin{note}
  Now, \BoyVS{}.
  Holds on \citeauthor{Landman:1992wh}'s account of the progressive.
  Holds on revised account, \emph{only} if disjoin all possible results.

  Two aspects.
  First, \AlgGetPStops{}, different from how things actually happen.
  Second, universal quantification over branches.

  So, if some point not considered, then disjunction without point fails to hold.

  This seems good.

  Then can't works, because \AlgGetPStops{}.

  Finally, \BoyPS{}.
  Complex.

  Holds, given the same method of evaluation.
  Which worlds are nearby worlds.

  Doesn't hold for the progressive in general.
  Though, I don't think it holds for ability.

  However, will hold when fully determined.
  Had the ability to win the race, after got into first.
  Had the ability to hit the bullseye, just before releasing the dart.

  Deals with wind.

  And, then this also gets joint abilities.
\end{note}

\subsection{`Foregone-concluding'}
\label{sec:fc-progressive}

\begin{note}
  So, action such that would be concluding.
  On understanding of the progressive, no matter what happens, still conclude.
  Might need to get `lucky' with action.
  However, start then path to the conclusion.

  This `luck' is limited.
  Nothing conflicting.
\end{note}

\begin{note}
  Worry, jumping to conclusions.
  But, in this case, \pevent{} in which agent concludes that step of reasoning is bad.
  Hence, incompatible.
\end{note}

\section{\fc{3} and \support{0}}
\label{cha:fcs:sec:fcs-support}

\begin{note}
  \autoref{cha:sec:fcs-def}, account of \fc{1}.
  Now tie to \support{0}.
\end{note}

\begin{note}
  \begin{proposition}
    \label{prop:fcs-only-if-support}
    For an agent \vAgent{} and proposition-value-premises pairing \(\pvp{\phi}{v}{\Phi}\):
    \begin{enumerate}
    \item[\emph{If}:]
      \begin{enumerate}[label=\alph*., ref=(\alph*.)]
      \item
        \(\pvp{\phi}{v}{\Phi}\) is a \fc{0}, from \vAgent{}' perspective.
      \end{enumerate}
    \item[\emph{then}:]
      \begin{enumerate}[label=\alph*., ref=(\alph*.), resume]
      \item
        \support{2} holds between \(\pv{\phi}{v}\) and \(\Phi\), from \vAgent{}' perspective.
      \end{enumerate}
    \end{enumerate}
    \vspace{-\baselineskip}
  \end{proposition}

  {
    \color{red}
    Why this is somewhat interesting.
  }
  However, before turning to the argument for \autoref{prop:fcs-only-if-support}, it is important to note the limitations of \autoref{prop:fcs-only-if-support} with respect to \issueConstraint{}.
  For, in order to argue against \issueConstraint{}, need some \(\pvp{\psi}{v'}{\Psi}\) such that answers \qWhyV{}.
  Answer \qWhyV{} only with dependence.
  Does not follow from \fc{0} that we get dependence.

  Note, also, qualifications.
  From the agent's perspective.
  Mirrored in both cases.
  Plausible that variant of \autoref{prop:fcs-only-if-support} holds unqualified.
  However, we have said nothing of \support{} independent of agent's perspective.
\end{note}

\begin{note}
  The argument for \autoref{prop:fcs-only-if-support} is {\color{red} mostly immediate for ideas regarding support}.

  \begin{goal}
    If conclude only if \fc{}, then support, in part, answers \qWhyV{}.
  \end{goal}

  So, to get answer to \qWhyV{}, need dependency.
  Here, if not \support{} then not \fc{}.
  If not \fc{} then not conclude.

  This is fine, just need to be careful with the counterfactual.
  Relation between \support{} and \fc{} is plain conditional.
  So, it survives any counterfactual changes.
\end{note}

\begin{note}[Argument]
  Argument is straightforward.
  Possible support, by assumption.
  Contraposition.
  If not support, then no \fc{}.
\end{note}

\paragraph{Potential relations of support}

\begin{note}
  Start with the following proposition.
  \begin{proposition}
    \label{prop:fcs-only-if-pot-support}
    For an agent \vAgent{} and proposition-value-premises pairing \(\pvp{\phi}{v}{\Phi}\):
    \begin{enumerate}
    \item[\emph{If}:]
      \begin{enumerate}[label=\alph*., ref=(\alph*.)]
      \item
        \(\pvp{\phi}{v}{\Phi}\) is a \fc{0}, from \vAgent{}' perspective.
      \end{enumerate}
    \item[\emph{then}:]
      \begin{enumerate}[label=\alph*., ref=(\alph*.), resume]
      \item
        (A) potential (relation of) \support{} holds between \(\pv{\phi}{v}\) and \(\Phi\), from \vAgent{}' perspective.
      \end{enumerate}
    \end{enumerate}
    \vspace{-\baselineskip}
  \end{proposition}

  Argument is fairly straightforward:
  \begin{argument}
    Suppose \(\pvp{\phi}{v}{\Phi}\) is a \fc{0}.
    Then, from agent's perspective, \pevent{} in which concludes.
    Now, consider the \pevent{}.
    The culmination of the event, agent concludes.

    So, from~\autoref{idea:support}, a relation of support holds, from the agent's perspective.

    Therefore, in whatever sense event is potential, \support{} between \(\pv{\phi}{v}\) and \(\Phi\) is likewise potential.
  \end{argument}
  From the agent's perspective, there is no difference between witnessed relation of support and potential relation of support.
\end{note}

\begin{note}
  \emph{Potential} relation of support, but it does not follow that there is a relation of support, from the agent's perspective.
\end{note}

\begin{note}
  \begin{proposition}
    \label{prop:pot-support-onlyIf-support}
    For an agent \vAgent{} and proposition-value-premises pairing \(\pvp{\phi}{v}{\Phi}\):
    \begin{enumerate}
    \item[\emph{If}:]
      \begin{enumerate}[label=\alph*., ref=(\alph*.)]
      \item
        (A) potential (relation of) \support{} holds between \(\pv{\phi}{v}\) and \(\Phi\), from \vAgent{}' perspective.
      \end{enumerate}
    \item[\emph{then}:]
      \begin{enumerate}[label=\alph*., ref=(\alph*.), resume]
      \item
        \support{2} holds between \(\pv{\phi}{v}\) and \(\Phi\), from \vAgent{}' perspective.
      \end{enumerate}
    \end{enumerate}
    \vspace{-\baselineskip}
  \end{proposition}

  \begin{argument}
    \autoref{idea:support:possible}.
    It is possible for there to be.
    So, we have everything needed.
    Both necessary and sufficient.
    Hence, form agent's perspective, relation of support.

    So, every necessary property that does not involve witnessing.
    But, then, every necessary property.
    Therefore, sufficient.
    For, if not sufficient, then missing a necessary property.
    Contradiction.

    Slight issue, disjunction of properties.
    But, this doesn't change the argument.
    Disjunction.
  \end{argument}
\end{note}

\begin{note}
  \color{red}
  Worry.
  \support{2} doesn't rely on witnessing.
  Now, if this goes through, then seems \support{} for any conclusion before making the conclusion.
  However, possible for the agent to reason to different conclusions.
  For, some faulty reasoning.
  Toggle the fault.
  Therefore, \support{} for contradictory conclusions.
\end{note}

\section{Conclusions, foregone}
\label{sec:fc3-1}

\paragraph{Premises and past conclusions}

\begin{note}[Premises]
  So, as we have seen with testimony, status of a premises blocks a \requ{}.

  Whether the same may hold for this problem.

  It's the case that, part of agent's present epistemic state that they would conclude.

  Problem is, if attempt and fail, then this premise does nothing.
  Their present epistemic state develops into a dead-end.
\end{note}

\begin{note}[Note!]
  This doesn't hold in general, for all premises.

  In particular, premise is past conclusion.

  Consider cases of being somewhat impaired, e.g., via exhaustion.
  Indeed, exhaustion is interesting.
  Basic consistency checks.
  Should be the case that conclude A, but just concluded \emph{not}-A, or something like this\dots

  Denying that past continues to secure in all instances.
  So, just need the potential to revise perspective on previous conclusion.
\end{note}

\section{\fc{3} and support}
\label{cha:fcs:sec:fc3-support}

\begin{note}
  \begin{proposition}
    For any path, present epistemic state determines availability of path.
  \end{proposition}

  Start.
  Then, continue.
  Started from \(\Phi\), so will conclude.
  Hence, no matter choice made, must have taken the possibility of this choice into account.
  So, it must be the case that determined.

  Hence, if witness, then via some path.

  So, witnessing predetermined path.
  Any instances of concluding by witnessing reduces to witnessing predetermined path.

  Witnessing may provide information about path, but witnessing doesn't contribute given a \requ{}.

  For any X from W,
  present determines whether or not X from agent's point of view, then \fc{}.

  In other words, agent's present epistemic state determines.
  Agent may need to witness to figure out how determined, but witnessing does not influence.
\end{note}

\begin{note}[Two worries]
  Two worries.

  First, that even though \fc{0}, the agent would not conclude.
  Either because \(\Phi\) is unavailable, or because no \pevent{}.
  So, can't remove \fc{0} from account of why.

  However, then \fc{0} does not support.

  If grant that \fc{0} supports, then this seems to work out.
  Further, if require existence, then things that support get very messy.
  Dopeganger cases.
  Reason is I saw A, but it wasn't A, appealing to something that doesn't exist.
  Various other cases like this.

  Difference.
  In these cases, have premise, thing is that the truth value is distinct.
  Here, possibly no premise.

  Well, this is different.
  However, I don't think this is sufficient to reject the idea.
  Just because this distinction doesn't arise in the case of witnessing doesn't really do much.

  Look, a `bad' premise offers no more support for the agent than no premise.

  Second, need \emph{that} \fc{0}.
  However, the point is that this is about the agent's present epistemic state.
  \emph{Without} \fc{0}, the agent would reason.
  This is just the key point reiterated.
  Know whether, \fc{0} just adds information about which.
\end{note}

\subsection{Interpretation}
\label{cha:zSpA:sec:interpretation}

\begin{note}
  Doxastic justification.
\end{note}

% \begin{note}
%   Important to observe here that with dispositions and ability, the subjunctive analysis is an analysis.
%   So, in principle possible to provide a distinct analysis.
%   This is surely the case, and I can probably find some example.

%   By contrast, in the case of positive answers to \qzS{}, the subjunctive is `built in' to the question.
% \end{note}

% \subsubsection{Dispositions and ability}
% \label{sec:dispositions}

% \begin{note}[Parallel between dispositions and ability]
%   Consider \citeauthor{Choi:2021wg}'s characterisation of the Simple Conditional Analysis of dispositions:
%   \begin{quote}
%     An object is disposed to \emph{M} when \emph{C} iff it would \emph{M} if it were the case that \emph{C}.\nolinebreak
%     \mbox{}\hfill\mbox{(\citeyear{Choi:2021wg})}
%   \end{quote}
%   For example, an object is disposed to dissolve when it is placed in water iff the object would dissolve if it were the case that it is placed in water.

%   The Simple Conditional Analysis may be challenged, but for our purposes it is adequate.
%   We are interested in the broad form of the truth condition, and various more refined analyses share the same broad form.
%   Note, in particular, that it being the case that \emph{C} and \emph{M} happening describes an event.
%   Given appropriate conditions; salt dissolves, glass breaks, and I mumble when I am tired.
%   The key idea is that the property of being disposed to \emph{M} when \emph{C} is analysed in terms of the (possible) event of \emph{M} happening when \emph{C}.

%   The parallel to ability is established by noting that ability may also be analysed in terms of a (possible) event, as we have seen.
%   In particular, by incorporating volition in the analysans of the Simple Conditional Analysis.
%   To illustrate, \citeauthor{Mandelkern:2017aa} trace the Conditional Analysis of ability  to \textcite{Hume:1748tp} and \textcite{Moore:1912te}, among others:
%   \begin{quote}
%     S can \(\phi\) iff S would \(\phi\) if S tried to \(\phi\)\nolinebreak
%     \mbox{}\hfill\mbox{(\citeyear[Cf.][308]{Mandelkern:2017aa})}
%   \end{quote}
%   Compare to the Simple Conditional Analysis of dispositions:
%   The object is some agent \emph{S}, \emph{C} is `S tried to \(\phi\)' and \emph{M} is `S \(\phi\)s' --- it is volition alone which distinguishes the analyses.
% \end{note}


\subsubsection{Doxastic justification}
\label{cha:fcs:sec:dox-just}

\begin{note}
  \citeauthor{Turri:2010aa}

  \begin{quote}
    Necessarily, for all S, \emph{p}, and \emph{t}, if \emph{p} is propositionally justified for S at \emph{t}, then \emph{p} is propositionally justified for S at \emph{t} because S currently possesses at least one means of coming to believe \emph{p} such that, were S to believe \emph{p} in one of those ways, S's belief would thereby be doxastically justified.%
    \mbox{ }\hfill\mbox{(\citeyear[316]{Turri:2010aa})}
  \end{quote}

  Key is that doxastic justification depends on what the agent does.

  \citeauthor{Turri:2010aa}'s focus is on how reasons are used.
  What the agent does.

  Seen with example.

  \begin{quote}
    \begin{enumerate}[label=(P\arabic*)]
      \setcounter{enumi}{4}
    \item
      The Spurs will win if they play the Pistons.
    \item
      The Spurs will play the Pistons.
    \end{enumerate}

    \hbox to \hsize{\hfil{\vdots}\hfil}

    \begin{enumerate}[label=(P\arabic*), resume]
    \item
      Therefore, the Spurs will win.%
    \mbox{ }\hfill\mbox{(\citeyear[317]{Turri:2010aa})}
    \end{enumerate}
  \end{quote}

  Rather than \emph{modus ponens}, `\emph{modus profusus}'.
  Conclude \(r\) from \(p\) and \(q\).
  (\citeyear[317]{Turri:2010aa})

  \begin{quote}
    The way in which the subject performs, the manner in which she makes use of her reasons, fundamentally determines whether her belief is doxastically justified.
    Poor utilization of even the best reasons for believing \emph{p} will prevent you from justifiedly believing or knowing that \emph{p}.%
    \mbox{ }\hfill\mbox{(\citeyear[316]{Turri:2010aa})}
  \end{quote}

  Variant of ~\cite{Prior:1960wh}'s `tonk' connective.
  Though, difference is between connective and rule.
  \(p\) tonk \(q\) would not be propositionally justified.
\end{note}

\begin{note}
  \citeauthor{Turri:2010aa} is similar to \citeauthor{Goldman:1979ui}

  Begin with justification.

  \begin{quote}
    \begin{enumerate}[label=(\arabic*)]
      \setcounter{enumi}{10}
    \item
      Person \emph{S} is \emph{ex ante} justified in believing \emph{p} at \emph{t} if and only if there is a reliable belief-forming operation available to \emph{S} which is such that if \emph{S} applied that operation to this total cognitive state at \emph{t}, \emph{S} would believe \emph{p} at \emph{t}-plus-delta (for a suitably small delta) and that belief would be \emph{ex post} justified.
    \end{enumerate}
  \end{quote}

  Where, sufficient condition for belief would be \emph{ex post} justified:
  \begin{quote}
    \begin{enumerate}[label=(\arabic*)]
      \setcounter{enumi}{4}
    \item
      If S's believing \emph{p} at \emph{t} results from a reliable cognitive belief-forming process (or set of processes), then S's belief in \emph{p} at \emph{t} is justified.%
      \mbox{ }\hfill\mbox{(\citeyear[13]{Goldman:1979ui})}
    \end{enumerate}
  \end{quote}
  Roughly, at least.
  \citeauthor{Goldman:1979ui} refines this a fair bit, but this isn't important.

  Availability of a reliable belief-forming operation!

  Relation here is brittle.
  Account of justification, apply to concluding.
  Well, then all we get is that before concluding, would make sense to conclude only if available.
  Running something like the \citeauthor{Carroll:1895uj} regress, not some state.
  But, this only tells us about suitability to conclude.

  Still, key point is process.

  Another useful thing to highlight is the suitably small delta.
  With \requ{}, this is captured in terms of the option.
\end{note}

\begin{note}
  Significant difference is in the case of justification, we're not interested in the agent's perspective.
  Hence, these accounts are understood in terms of the agent having the ability, roughly.
\end{note}

%%% Local Variables:
%%% mode: latex
%%% TeX-master: "master"
%%% End:


\chapter{\ros{3}}
\label{cha:ros}


\begin{note}
  \ros{3} capture the way in which a \prop{0}-\val{0} pair \(\pv{\phi}{v}\) `follows from' some \pool{} \(\Phi\), from an \agpe{}.
  Where the `following from' relation contrasts with `from' in the sense that  an agent concludes \(\pv{\phi}{v}\) `from' \(\Phi\).
\end{note}


\begin{note}
  Broad idea.

  In turn, \ros{3} are answers to \qWhy{}, and if \issueConstraint{} holds, for any \ros{} which answers \qWhy{} there is some event where agent concludes.
\end{note}


\begin{note}
  Specifying \ros{} in any significant detail is beyond the scope of this document.
  \issueInclusion{} is understood as a (plausible) constraint on theories about the why and how of reasoning rather than the details of any given theory.
  Instead, we characterise \ros{} by two ideas, and anything which satisfies these two ideas in the details of a given theory may be understood as a \ros{}.
\end{note}


\section{\supportI{}}
\label{cha:ros:I}


\begin{note}
  The role of \supportI{} is to characterise when a \ros{} holds between some \prop{0}-\val{0} pair and \pool{} \emph{given} an event in which the agent concludes the \prop{0}-\val{0} pair from the \pool{}.

  \begin{idea}[\supportI{}]
    \label{idea:support}
    \vspace{-\baselineskip}
    \begin{itenum}
    \item[\emph{If}:]
      \(e_{d}\) is an event in which \vAgent{} concludes \(\pv{\phi}{v}\) from \(\Phi\).
    \item[\emph{Then}:]
      When \vAgent{} \eval{1} \(\phi\) as having value \(v\): %as a \se{0} of \(e_{d}\):
      \begin{itemize}
      \item
        A \emph{\ros{}} between \(\pv{\phi}{v}\) and \(\Phi\) holds, from \agpe{\vAgent{}'}.
      \end{itemize}
    \end{itenum}
    \vspace{-\baselineskip}
  \end{idea}

  \noindent%
  In short, \supportI{} states that a \ros{} between \(\pv{\phi}{v}\) and \(\Phi\) holds form an \agpe{} when the agent concludes \(\pv{\phi}{v}\) from \(\Phi\).%
  \footnote{
    The existence of a \ros{} is restricted to the sub-event when the agent \evals{} \(\phi\) as having \val{} \(v\) as a concludes event may span the \agents{} reasoning to \(\pv{\phi}{v}\) from \(\Phi\) (\autoref{assu:ConRea}).
  }

  The motivation for \supportI{} is straightforward:
  The role of \ros{0} a \ros{} between \(\pv{\phi}{v}\) and \(\Phi\) is to abstractly capture the way in which \(\pv{\phi}{v}\) `follows from' \(\Phi\) from the \agpe{}.
  And, if an agent concludes \(\pv{\phi}{v}\) from \(\Phi\), then \(\pv{\phi}{v}\) `follows from' \(\Phi\) from the \agpe{}.

  In other words, the \ros{} between \(\pv{\phi}{v}\) and \(\Phi\) just captures whatever it is, for the agent, that led to the agent concluding \(\pv{\phi}{v}\) from \(\Phi\).
\end{note}


\begin{note}
  \supportI{} ensures a \ros{} between \pv{\propM{\rootsCon{}}}{\valI{True}} and some \pool{} which captures the \agents{} understanding of factorisation or the quadratic formula holds when the agent makes the relevant conclusion in \scen{1}~\ref{illu:gist:roots:F}~and~\ref{illu:gist:roots:QF}.
  Likewise, \supportI{} ensures a \ros{} between \pv{\propI{What Pritcher said was a sign}}{\valI{True}} and some \pool{} \(\Psi\) holds from \agpe{Fox's} in \autoref{scen:countS} when Fox responds to `I come from Miran' with `Miran is early this year'.
\end{note}


\begin{note}
  Our characterisation of \ros{} by \supportI{} rests on the way in which we understand conclusions.
  In particular, \supportI{} allows for \ros{} to hold between any \prop{0}-\val{0}-\pool{0} pair, so long as the agent concludes the relevant \prop{0} has the specified \val{0} from the given \pool{}, or the \prop{0}-\val{0} pair is a \fc{} from the \pool{}.
  Hence, a \ros{} may hold between, e.g., \pv{\propI{Fish are mammals}}{\valI{True}} and some \pool{} from the \agpe{}.

  If the way in which conclusions are understood is restricted, the \ros{} as characterised are likewise restricted.
  And, that \ros{1} hold between certain \prop{0}-\val{0} pairs and \pool{1} is important for the argument to follow, not that \ros{1} may hold between arbitrary \prop{0}-\val{0} pairs and \pool{1}.
  The benefit of the characterisation given is simplicity.
  For example, I expect the argument to follow is compatible with restricting \ros{} to justified conclusions.
\end{note}



\section{\wit{3} for \ros{1}}
\label{cha:ros:W}


\begin{note}
  With \supportI{}, and hence some characterisation of \ros{}, in hand we define a `\wit{0}' for \ros{} between \(\pv{\phi}{v}\) and \(\Phi\).
  In full:

  \begin{definition}[A \wit{0} for a \ros{1}]%
    \label{def:witnessing}%
    \vspace{-\baselineskip}
    \begin{itemize}
    \item
      An event \(e^{-}_{d^{-}}\) is \emph{\wit{0}} for a \ros{} between \(\pv{\phi}{v}\) and \(\Phi\), for \vAgent{} through event \(e_{d}\)
    \end{itemize}

    \emph{If and only if:}

    \begin{itemize}
    \item
      \(e^{-}_{d^{-}}\) is an event in which \vAgent{} concludes \(\pv{\phi}{v}\) from \(\Phi\).
    \item
      \(e^{-}_{d^{-}}\) occurs prior to or at the same time as \(e_{d}\).
    \end{itemize}
    \vspace{-\baselineskip}
  \end{definition}

  \noindent%
  In short, a \wit{} for a \ros{} between \(\pv{\phi}{v}\) and \(\Phi\) is some (extant) event in which the agent concludes \(\pv{\phi}{v}\) from \(\Phi\).

  And, if an agent has concluded \(\pv{\phi}{v}\) from \(\Phi\), we say the agent `has a \wit{}' for a \ros{} between \(\pv{\phi}{v}\) and \(\Phi\).
\end{note}


\begin{note}
  An important, but trivial, case of \autoref{def:witnessing} is when an agent concludes \(\pv{\phi}{v}\) from \(\Phi\).
  For, if an agent concludes \(\pv{\phi}{v}\) from \(\Phi\) then it is immediate that there is some event in which the agent concludes \(\pv{\phi}{v}\) from \(\Phi\) --- the very same event --- and hence the agent has a \wit{} for the \ros{} between \(\pv{\phi}{v}\) and \(\Phi\).
\end{note}



\section{\supportII{}}
\label{cha:ros:II}


\begin{note}
  The role of \supportII{} is to characterise a sufficient condition for a \ros{} to hold between some \prop{0}-\val{0} pair and \pool{} from an \agpe{} \emph{regardless of} an event in which the agent concludes the \prop{0}-\val{0} pair from the \pool{}.

  In \autoref{cha:intro} we motivated the idea of such \ros{} via some action may immediately do such that the agent is concluding \(\phi\) has value \(v\) from \(\Phi\) when the agent does the action (\autopageref{rosFirst}).
  In full, we use the idea of a \fc{0} to identify such \ros{1}:

  \begin{idea}[\supportII{}]%
    \label{idea:support:possible}%
    \vspace{-\baselineskip}
    \begin{itenum}
    \item[\emph{If}:]
      \(\pv{\phi}{v}\) is a \fc{0} from \(\Phi\) for \vAgent{} throughout \(e_{d}\).
    \item[\emph{Then}:]
      A \ros{} between \(\pv{\phi}{v}\) and \(\Phi\) holds from \agpe{\vAgent{}'} throughout \(e_{d}\).
    \end{itenum}
    \vspace{-\baselineskip}
  \end{idea}

  \noindent%
  A \ros{} is designed to abstractly capture the way in which some \prop{0}-\val{0} pair `follows from' some \pool{} from an \agpe{}.
  \supportII{}, states that \(\pv{\phi}{v}\) being a \fc{} from \(\Phi\) is sufficient for a \ros{} between \(\pv{\phi}{v}\) and \(\Phi\) to hold from the \agpe{}.
\end{note}




\begin{note}
  \noindent%
  Expanded with the definition of a \fc{}, \supportII{} reads:
  \begin{itenum}
  \item[\emph{If}:]
    \begin{itemize}
    \item
      Throughout \(e_{d}\) there is some action \(a\) \vAgent{} may immediately perform such that both \ref{def:fc:act} and \ref{def:fc:result} are true:
      \begin{enumerate}[label=\alph*., ref=(\alph*)]
      \item
        For each \prop{0}-\val{0} pair \(\pv{\phi'}{v'}\) in \(\Phi\), \vAgent{} \evals{} \(\phi'\) as having value \(v'\) prior to doing \(a\).
      \item
        The event \(e^{\sharp}_{d^{\sharp}}\) in which \vAgent{} does \(a\) is an event in which \vAgent{} is concluding \(\pv{\phi}{v}\) from \(\Phi\).
      \end{enumerate}
    \end{itemize}
  \item[\emph{Then}:]
    A \ros{} between \(\pv{\phi}{v}\) and \(\Phi\) holds, from \agpe{\vAgent{}'} throughout \(e_{d}\).
  \end{itenum}

  \noindent%
  In short, something about the \agpe{\agents{} present} already secures anything that is required to hold from the \agpe{} when the agent concludes \(\pv{\phi}{v}\) from \(\Phi\).
\end{note}



\begin{note}
  \supportII{} captures the idea that a \ros{} between \pv{\propM{\rootsCon{}}}{\valI{True}} and some \pool{} which captures their understanding of factorisation or the quadratic formula holds when the agent is concluding \pv{\propM{\rootsCon{}}}{\valI{True}} from the relevant \pool{} in \scen{1}~\ref{illu:gist:roots:F}~and~\ref{illu:gist:roots:QF}.

  Likewise, \supportII{} states a \ros{} holds between the relevant \prop{0}-\val{0} pair and \pool{} for each of the \fc{1} in \autoref{cha:fcs}.%
  \footnote{
    Perhaps one may say a good puzzle of the kind given in \autoref{scen:fc:chick} is a puzzle in which a \ros{} already holds between the intended solution and some \pool{} for the agent (after reading the puzzle) and obtaining a \wit{} for the \ros{} is enjoyable.
  }

  In \autoref{scen:countS}, by contrast, \supportII{} does state a \ros{} between \pv{\propI{What Pritcher said was a sign}}{\valI{True}} and some \pool{} \(\Psi\) holds \emph{prior to} Fox's conclusion of \pv{\propI{What Pritcher said was a sign}}{\valI{True}} from \(\Psi\).
  For, for example, Pritcher may have said `I come from Wilau' and failed to sign and hence \pv{\propI{What Pritcher said was a sign}}{\valI{True}} was not a \fc{} from \(\Psi\).

  And, \supportII{} does not states a \ros{} holds between the relevant \prop{0}-\val{0} pair and \pool{} for any absent \fc{1} in \autoref{cha:fcs}.
\end{note}


\begin{note}
  Additional example.

  \begin{scenario}[Fibonacci numbers]%
    \label{scen:fc:fib}%
    The Fibonacci numbers are recursively defined as follows:

    \[
      F_{n} = \left\{
        \begin{array}{ll}
          0 & \text{if } n = 0 \\
          1 & \text{if } n = 1 \\
          F_{n-1} + F_{n-2} & \text{if } n > 1 \\
        \end{array}
      \right.
    \]
  \end{scenario}

  \begin{enumerate}[label=C\thescenarioCounter., ref=(C\thescenarioCounter)]
  \item
    \label{scen:fc:fib:c}
    \pv{\propI{The sixth number in the Fibonacci sequence is 5}}{\valI{True}}
  \end{enumerate}
  %
  Given an understanding of \(f\) and sufficient motivation, an agent may do some action and be concluding \pv{\propI{The six number in the Fibonacci sequence is 5}}{\valI{True}}.
  Hence, \pv{\propI{The six number in the Fibonacci sequence is 5}}{\valI{True}} is a \fc{} from some \pool{}.
  Hence, so long as the action is available a \ros{} holds between \pv{\propI{The six number in the Fibonacci sequence is 5}}{\valI{True}} and some \pool{} which captures the \agents{} understanding of \(f\).

  Though, \pv{\propI{The sixtieth number in the Fibonacci seq.\ is 1548008755920}}{\valI{True}} need not be \fc{0}.
  For, even if one uses memos, the \emph{sixtieth} number requires a lot of work, and the agent may not have sufficient resources.
  So, it need not be the case that a \ros{} holds by \supportII{}.
\end{note}



\section{Optional notes}


\begin{note}
  This section contains three optional notes.

  The first note contrasts our understanding of the connexion between \ros{1} and an event in which an agent concludes to \citeauthor{Boghossian:2014aa}'s understanding of the connexion between `support' an event in which an agent infers.

  The second note draws a parallel between our \supportII{} and \citeauthor{Goldman:1979ui}'s account of \emph{ex ante} justification.

  The third note considers the possibility of \ros{} occurring within \ros{} and states we ignore the possibility.
\end{note}



\subsection{\supportI{} and \citeauthor{Boghossian:2014aa} Taking Condition}


\begin{note}
  \supportI{} is similar to, but distinct from,~\citeauthor{Boghossian:2014aa}'s Taking Condition:%
  \footnote{
    Strictly,~\citeauthor{Boghossian:2014aa} states the Taking Condition in terms of inferring.
    I.e., the Taking Condition reads: `Inferring necessarily involves the thinker \emph{taking} \dots'
    However, as \citeauthor{Boghossian:2008vf} is interested in conclusions, an event in which an agent infers (in~\citeauthor{Boghossian:2014aa} sense) is an event in which an agent concludes (in our sense).
  }

  \begin{quote}
    (Taking Condition):
    [An event in which an agent concludes] necessarily involves the thinker \emph{taking} his premises to support his conclusion and drawing his conclusion because of that fact.%
    \mbox{}\hfill\mbox{(\citeyear[5]{Boghossian:2014aa})}
  \end{quote}

  \noindent%
  For, `taking' is understood by \citeauthor{Boghossian:2014aa} to be component of the agent's reasoning.
  \citeauthor{Boghossian:2014aa} illustrates the Taking Condition as follows:
  %
  \begin{quote}
    On waking up one morning I recall that:

    \begin{enumerate}[label=(\arabic*), ref=(\arabic*), series=BogEx]
    \item
      \label{BogEx:1}
      It rained last night.
    \end{enumerate}

    I combine this with my knowledge that

    \begin{enumerate}[label=(\arabic*), ref=(\arabic*), resume*=BogEx]
    \item
      \label{BogEx:2}
      If it rained last night, then the streets are wet.
    \end{enumerate}

    to conclude:

    So,

    \begin{enumerate}[label=(\arabic*), ref=(\arabic*), resume*=BogEx]
    \item
      \label{BogEx:3}
      The streets are wet.
    \end{enumerate}
    This belief then affects my choice of footwear.%

    [\dots M]y inferring from~\ref{BogEx:1} and~\ref{BogEx:2} to~\ref{BogEx:3} must involve my arriving at the judgment that~\ref{BogEx:3} in part \emph{because} I \emph{take} the presumed truth of~\ref{BogEx:1} and~\ref{BogEx:2} to provide support for~\ref{BogEx:3}.%
    \mbox{ }\hfill\mbox{(\citeyear[2,4]{Boghossian:2014aa})}
  \end{quote}
  %
  Hence, for \citeauthor{Boghossian:2014aa}, the Taking Condition captures something \emph{in addition} to~\ref{BogEx:3} being a conclusion from a \pool{} which includes~\ref{BogEx:1} and~\ref{BogEx:2}.

  In contrast, we do not require that a \ros{} has any particular role \emph{for the agent} in event in which an agent concludes \(\pv{\phi}{v}\) from \(\Phi\).
  If an agent concludes~\ref{BogEx:3} from~\ref{BogEx:1} and~\ref{BogEx:2}, then a \ros{} holds between~\ref{BogEx:3} and \(\{\ref{BogEx:1}, \ref{BogEx:2}\}\).
  However, the \ros{} between~\ref{BogEx:3} and \(\{\ref{BogEx:1}, \ref{BogEx:2}\}\) need not itself have a role in the \agents{} conclusion of~\ref{BogEx:3} from \(\{\ref{BogEx:1}, \ref{BogEx:2}\}\).%
  \footnote{
    Also, about the type of reasoning by which the agent concludes.
    This comes from \textcite{Boghossian:2008vf,Boghossian:2012vb}.
    Rule following, taking gets account of rule.
  }

  To illustrate, consider \citeauthor{Wright:2014tt}'s (\citeyear{Wright:2014tt}) `Simple Proposal':
  %
  \begin{quote}
    % [C]onsider instead the proposal, not that the status of the transition as inferential depends on the thinker's judgments about his reasons, but that it depends on \emph{what his reasons are}.
    % We want his acceptance of the premises to supply his \emph{actual} reasons for accepting the conclusion.
    % [\dots]
    %
    % Call this the Simple Proposal.
    % It says that a thinker infers q from p\(_{1}\) \(\cdots\) p\(_{\text{n}}\) when he accepts each of p\(_{1}\) \(\cdots\) p\(_{\text{n}}\), moves to accept q, and does so for the reason that he accepts p\(_{1}\) \(\cdots\) p\(_{\text{n}}\).%
    [The Simple Proposal] says that a thinker infers q from p\(_{1}\) \(\cdots\) p\(_{\text{n}}\) when he accepts each of p\(_{1}\) \(\cdots\) p\(_{\text{n}}\), moves to accept q, and does so for the reason that he accepts p\(_{1}\) \(\cdots\) p\(_{\text{n}}\).%
    \mbox{}\hfill\mbox{(\citeyear[33]{Wright:2014tt})}
  \end{quote}
  %
  \citeauthor{Wright:2014tt}'s proposal is that the relation between a conclusion and some \pool{} need not be part of what moves the agent to conclude the conclusion from the \pool{}.
  Hence, \citeauthor{Wright:2014tt} denies that reasoning must involve a state which connects premises to conclusions.
  So, \citeauthor{Wright:2014tt} denies \citeauthor{Boghossian:2008vf}'s Taking Condition on inference  (\citeyear[Cf.][33-34]{Wright:2014tt}).

  \supportI{} is compatible with \citeauthor{Wright:2014tt}'s Simple Proposal as \supportI{} only entails a \ros{} holds between \(\pv{\propM{q}}{\valI{True}}\) and a \pool{} containing \(\pv{\propM{p_{1}}}{\valI{True}}\), \(\cdots\) \(\pv{\propM{p_{n}}}{\valI{True}}\).%
  \footnote{
    Still, there is an important between~\supportI{} and \citeauthor{Wright:2014tt}'s Simple Proposal.
    For,~\supportI{} is an entailment, while \citeauthor{Wright:2014tt}'s Simple Proposal is an identity statement.
    Inferring, on the Simple Proposal, is an agent accepting some conclusion for the reason that they accept premises from some \pool{}.
    \supportI{} does not entail that concluding is nothing more than moving to accept \(\pv{\phi}{v}\) as a result of accepting each element of \(\Phi\).
  }\(^{,}\)%
  \footnote{
    There are various other objections to~\citeauthor{Boghossian:2014aa}'s Taking Condition.

    For example,~\citeauthor{Hlobil:2014tq} argues against the Taking Condition as it distracts from what accounts of reasoning out to explain, rather than arguing against the Taking Condition directly.
    Likewise, \citeauthor{McHugh:2016vp} present and summarise various objections to \emph{interest} with the Taking Condition.

    In particular,~\supportI{} is closer to what \citeauthor{McHugh:2016vp} term the `Consequence Condition': \textquote{Inferring q from p entails taking p to support q}.
    (\citeyear[316]{McHugh:2016vp})
    And, as \citeauthor{McHugh:2016vp} observe, the condition is \textquote{consistent with the idea that in inference we take our premises to support our conclusion just in virtue of reasoning from the former to the latter}.
    (\citeyear[316]{McHugh:2016vp})

    \citeauthor{McHugh:2016vp} suggest the arguments they consider against the Taking Condition `put pressure' on the Consequence Condition (\citeyear[327]{McHugh:2016vp}).
    However, these arguments concern interest, rather than whether condition is true.
    And, we, uh, have interest in \ros{1}\dots
  }
\end{note}

% \begin{note}
%   \color{red}
%   Humpty Dumpty arbitrary chooses for `glory' to mean `a nice knock-down argument' (\cite[190]{Carroll:2009aa}).
% \end{note}


\subsection{\supportII{} and \citeauthor{Goldman:1979ui}'s account of \emph{ex ante} justification}


\begin{note}
  From a structural perspective, our approach to characterising \ros{1} without a given an event in which the agent concludes is similar to \citeauthor{Goldman:1979ui}'s account of \emph{ex ante} justification in terms of \emph{ex post} justification.%
  \footnote{
    \citeauthor{Goldman:1979ui}'s notions of \emph{ex ante} and \emph{ex post} justification is similar to the distinction between doxastic and propositional justification (see \cite{Firth:1978vi} and \cite[esp.\ fn.1]{Silva:2020aa}).

    Given the parallels between \ros{1} and \emph{ex ante} justification, may think of \ros{1} in line with propositional justification.
  }%
  \(^{,}\)%
  \footnote{
    \citeauthor{Turri:2010aa} (\citeyear{Turri:2010aa}) provides similar account.
    However, \citeauthor{Turri:2010aa} does not hold that this is sufficient.
  }

  Here's a paraphrase of \citeauthor{Goldman:1979ui}'s account of \emph{ex post} justification:
  \begin{quote}
    Person \emph{S} is \emph{ex post} justified in believing \emph{p} when \emph{S} believes \emph{p}, and we say \emph{S}' believing \emph{p} is~justified.%
    \mbox{ }\hfill\mbox{(\citeyear[Cf.][21]{Goldman:1979ui})}
  \end{quote}
  % 
  And here's \citeauthor{Goldman:1979ui}'s account of \emph{ex ante} justification:
  % 
  \begin{quote}
    Person \emph{S} is \emph{ex ante} justified in believing \emph{p} at \emph{t} if and only if there is a reliable belief-forming operation available to \emph{S} which is such that if \emph{S} applied that operation to his total cognitive state at \emph{t}, \emph{S} would believe pat \emph{t}-plus-delta (for a suitably small delta) and that belief would be \emph{ex post} justified.%
    \mbox{ }\hfill\mbox{(\citeyear[21]{Goldman:1979ui})}
  \end{quote}
  %
  In a broad stroke, someone is \emph{ex ante} justified in believing \emph{p} at \emph{t} just in case there is some action the person may immediately do and as a result of doing the action, the person is \emph{ex post} justified in \emph{p}.

  Likewise, if \support{} characterises an \emph{ex post} \ros{0} and \supportII{} characterises an \emph{ex ante} \ros{0} we may say, in a broad stroke, an \emph{ex ante} \ros{} holds from an \agpe{} just in case there is some action the person may immediately do and as a result of doing the action a \emph{ex post} \ros{} holds from the \agpe{}.

  There is, however, a small difference:
  \citeauthor{Goldman:1979ui} defines \emph{ex ante} justification in terms of \emph{ex post} justification, whereas we only provide a sufficient condition for an \emph{ex ante} \ros{} in terms of an \emph{ex post} \ros{}.
\end{note}


\subsection{\ros{3} which contain \ros{1}}


\begin{note}
  A \ros{} holding between \(\pv{\phi}{v}\) and \(\Phi\) is a way things are.
  Therefore, it is possible for an agent to \eval{} the \prop{} \propI{A \ros{} between \(\pv{\phi}{v}\) and \(\Phi\)} as \valI{True}, \valI{Possible}, \valI{Desired}, and so on\dots.

  Hence, it is possible for both \ref{Embed:no} and \ref{Embed:yes} to occur:

  \begin{enumerate}[label=\arabic*., ref=(\arabic*)]
  \item
    \label{Embed:no}
    A \ros{0} between \(\pv{\phi}{v}\) and \(\Phi\) holds from \agpe{an \agents{}}.
  \item
    \label{Embed:yes}
    A \ros{0} between \(\pv{\psi}{v'}\) and \(\Psi\) holds from \agpe{an \agents{}}, where \(\Psi\) contains:

    \pv{\propI{A \ros{} between \(\pv{\phi}{v}\) and \(\Phi\) holds from \agpe{my}}}{\valI{True}}
  \end{enumerate}

  Now, in the case of \ref{Embed:yes}, it is not necessarily the case that a A \ros{0} between \(\pv{\phi}{v}\) and \(\Phi\) holds from \agpe{the}.
  For example, if an agent may think the answer to the puzzle of \autoref{illu:fc:chess:II} is a \fc{}, though in reality the agent isn't particularly good at thinking through chess problems.

  When we speak of \ros{} we do not consider \ros{} which are embedded within \ros{}.
  Hence, we do not consider \ros{} which are embedded within \ros{} as answers to \qWhy{}.%
  \footnote{
    Though, \ref{Embed:no} and \ref{Embed:yes} may occur simultaneously, and \ref{Embed:no} may be the case, in part, due to \ref{Embed:yes} being the case.
    However, as this takes some effort to think about, we will not (directly, at least) consider cases where \ref{Embed:no} is the case due to \ref{Embed:yes} being the case.
  }

  Relevant definitions could be refined to explicitly exclude embedded \ros{1}.
  However, in this case I think ignoring is preferable to precision.
\end{note}



\section*{Summary}




% \begin{note}
%   It need not be the case that an agent has a \wit{0} for a \ros{0} in order for \ros{} to be involved in answering \qWhyV{}.

%   For, suppose an agent does not have a \wit{0} for the \ros{} between \(\pv{\psi}{v'}\) and \(\Psi\).
%   The upshot of the distinction between~\ref{Embed:no} and~\ref{Embed:yes} is as follows:

%   \begin{itemize}
%   \item
%     If the \ros{0} of \ref{Embed:no} is, in part, an answer to \qWhyV{} then the \ros{0} is a counterexample to \issueConstraint{}.
%   \item
%     If the \ros{0} of \ref{Embed:yes} is, in part, an answer to \qWhyV{} then the \ros{0} is \emph{not} a counterexample to \issueConstraint{}.
%   \end{itemize}

%   The difference is \emph{the way in which} the \ros{} functions with respect to the agent pairing \(\phi\) with \(v\).
%   Whether the \ros{} functions as a premise when the agent concludes \(\pv{\phi}{v}\), or whether the \ros{} functions in a way that is different to a premise.
% \end{note}

% \begin{note}
%   \begin{itemize}
%   \item
%     \(\pv{\phi}{v}\) is supported by \(\Phi\), from the \agpe{}.
%   \item
%     \(\pv{\psi}{v'}\) is supported, in part, by [the way in which \(\pv{\phi}{v}\) is supported by \(\Phi\), from the \agpe{}], from the \agpe{}.
%   \item
%     The way in which \(\pv{\psi}{v'}\) is supported, in part, by the \agpe{} on [the way in which \(\pv{\phi}{v}\) is supported by \(\Phi\), from the \agpe{}].
%   \end{itemize}

%   \ros{} of \ref{Embed:no} is a \ros{} which holds from the \agpe{}
%   \ros{} of \ref{Embed:yes} is a \ros{} which holds from the \agpe{}, from the \agpe{}.
%   Expanded a little more carefully, the primary \ros{1} of \ref{Embed:no} and \ref{Embed:yes} are paraphrased as capturing:
%   %   
%   \begin{itemize}
%   \item
%     The way in which \(\pv{\phi}{v}\) is supported by \(\Phi\), from the \agpe{}.
%   \item
%     The way in which \(\pv{\psi}{v'}\) is supported, in part, by [the way in which \(\pv{\phi}{v}\) is supported by \(\Phi\), from the \agpe{}], from the \agpe{}.
%   \end{itemize}

%   When we refer to a \ros{} reference is to a state of affairs from \agpe{our}, as in \ref{Embed:no}.
%   Shorthand, an `\rosNE{}' \ros{}.
%   And, a \ros{} from \agpe{an \agents{}} such as \ref{Embed:no} a `\rosE{}' \ros{}.

% \end{note}

% {
% \color{red}
% To refer to a \ros{} as an \rosE{}, need a second \ros{}.
% Hence, this clears things up.
% }

%   \begin{note}
%     The present section concerns, for some arbitrary proposition-value pair \(\pv{\phi}{v}\) and \pool{} \(\Psi\), the distinction between:
%   %     

%   %     
%     \ref{Embed:no} is a \ros{} between \(\pv{\phi}{v}\) and \(\Phi\).
%     By contrast, \ref{Embed:yes} is a \ros{0} between \(\pv{\psi}{v'}\) and \(\Psi\) which \emph{involves} a \ros{} between \(\pv{\phi}{v}\) and \(\Phi\).

%   %     
%     Throughout this document our interest is with \ros{} that do not occur within some other (relevant) \ros{}.
%     In particular, we have implicitly assumed there are no \ros{} which answer \qWhy{} due to the \ros{} occurring within some other \ros{}.
%     And, the variant of \qWhyV{} introduced in \autoref{cha:var} (on \autopageref{questionWhyV}) explicitly requires this.

%     If a \ros{} between \(\pv{\phi}{v}\) and \(\Phi\) referenced due the \ros{} occurring within some other \ros{} (as in \ref{Embed:yes}), we say the reference is to an `\rosE{0}' \ros{}.
%     Otherwise, we say the reference is to an `\rosNE{0}' \ros{} (as in \ref{Embed:no}).

%     Throughout this document reference to a \ros{} is almost always clearly reference to an \rosNE{0} \ros{}.
%     Hence, we only distinguish between \rosE{0} and \rosNE{0} when some ambiguity may be present.
%   \end{note}

%   \begin{note}
%     Idea is somewhat familiar from distinction between object- and meta-language with respect to propositional logic.
%     Certain kind of equivalence between proof and conditional.
%     It is possible to find a corresponding conditional to any proof with a finite number of premises, proof captures derivation of conclusion from premises.

%     Corresponding conditional is not a premise, nor any part, of the proof.

%     For example, consider a proof from \(P\) and \(P \rightarrow Q\) to \(Q\) by conditional detachment.
%     Corresponding conditional is \((P \land (P \rightarrow Q)) \rightarrow Q\).
%     However, not part of the proof.

%     Intuitive distinction between what a proof and a conditional refer to.
%     However, informally there is no difficulty in treating a proof as a premise.
%     \(P\), and I have a proof of \(P \rightarrow Q\), therefore \(Q\).
%   \end{note}

%   \subsubsection[Definitions]{Definitions \hfill (Optional)}
%   \label{cha:var:ros:Emb:defs}

%   \begin{note}
%     Distinction between an \rosE{0} and \rosNE{0} is intuitive, but imprecise.

%     This section provide definition.

%     To keep things simple the following definition assumes:
%     \begin{itenum}
%     \item[\emph{If}:]
%       \(\phi\) having value \(v\) entails \(\phi'\) has value \(v'\), from the \agpe{}.
%     \item[\emph{Then}:]
%       For any \pool{} \(\Phi\), \pv{\phi'}{v'} is in \(\Phi\) whenever  \(\pv{\phi}{v}\) is in \(\Phi\).
%     \end{itenum}
%     E.g., if \(\pv{\phi'\text{ and }\phi''}{\valI{True}}\) is in \(\Phi\) then both \(\pv{\phi'}{\valI{True}}\) and \(\pv{\phi''}{\valI{True}}\) are in \(\Phi\).

%     \begin{definition}[Degree of a \prop{0}-\val{0} pair within a \ros{}]%
%       \label{def:embedding:degree}%
%       For a proposition-value pairs \(\pv{\psi}{v'}\), \(\pv{\phi}{v}\), \pool{} \(\Phi\), and \(i \in \mathbb{N}\):

%       \begin{itemize}
%       \item
%         \(\pv{\psi}{v'}\) has a \emph{degree \(1\)} with respect to a \ros{} between \(\pv{\phi}{v}\) and \(\Phi\) if and only if \(\pv{\psi}{v'} \in \Phi\).
%       \item
%         \(\pv{\psi}{v'}\) is has a \emph{degree \(i\)} with respect to a \ros{} between \(\pv{\phi}{v}\) and \(\Phi\) if and only if:
%         \begin{itemize}
%         \item
%           There exists some \(\pv{\theta}{v''}\) and \(\Theta\) such that:
%           \begin{itemize}
%           \item
%             \(\pv{\psi}{v'} \in \Theta\)
%           \item
%             \(\pv{\propI{A \ros{} between }\pv{\theta}{v''}\propI{ and }\Theta}{\valI{True}}\) has degree \(i - 1\) with respect to the \ros{} between \(\pv{\phi}{v}\) and \(\Phi\).
%           \end{itemize}
%         \end{itemize}
%       \end{itemize}
%       \vspace{-\baselineskip}
%     \end{definition}

%     The cases of interest to us are where \pv{\propI{A \ros{} between \(\pv{\psi}{v'}\) and \(\Psi\)}}{\valI{True}} has degree \(n\) within the \ros{} between \(\pv{\phi}{v}\) and \(\Phi\).

%     Reference to \ros{} between A \ros{} between \(\pv{\psi}{v'}\) and \(\Psi\) due to the \ros{} having degree \(n\) within the \ros{} between \(\pv{\phi}{v}\) and \(\Phi\).
%     That's reference to a \rosE{}.

%     is \rosE{0} within in a \ros{} between \(\pv{\phi}{v}\) and \(\Phi\), no matter the degree of embedding:

%     \begin{definition}[Embedding within a \ros{}]%
%       \label{def:embedding}%
%       For a proposition-value pairs \(\pv{\psi}{v'}\), \(\pv{\phi}{v}\), and a \pool{} \(\Phi\):

%       \begin{itemize}
%       \item
%         \(\pv{\psi}{v'}\) is \emph{\rosE{0}} within in a \ros{} between \(\pv{\phi}{v}\) and \(\Phi\)
%       \end{itemize}

%       \emph{If and only if:}

%       \begin{itemize}
%       \item
%         \(\pv{\psi}{v'}\) is has a degree of embedding \(i\) with respect to the \ros{} between \(\pv{\phi}{v}\) and \(\Phi\), for some \(i \in \mathbb{N}\).
%       \end{itemize}
%       \vspace{-\baselineskip}
%     \end{definition}

%     The definition of an embedding covers arbitrary proposition-value pairs.
%     However, the cases of embedding of interest to us are where \ros{1} are \rosE{0} within a \ros{}.
%     A final definition captures when this is the case:

%     \begin{definition}[A \prop{0}-\val{0} pair \rosE{0} in a \ros{1}]
%       For a proposition-value pairs \(\pv{\psi}{v'}\), \(\pv{\phi}{v}\), and \pool{1} \(\Phi\), \(\Psi\):

%       \begin{itemize}
%       \item
%         A \ros{} between \(\pv{\psi}{v'}\) and \(\Psi\) is \rosE{0} within the \ros{} between \(\pv{\phi}{v}\) and \(\Phi\).
%       \end{itemize}

%       \emph{If and only if}

%       \begin{itemize}
%       \item
%         For some proposition-value pair \(\pv{\chi}{v''}\) in \(\Phi\):
%         \begin{itemize}[noitemsep]
%         \item
%           \(\chi\) is the proposition: \propI{A \ros{} between \(\pv{\psi}{v'}\) and \(\Psi\)}.
%         \item
%           \(v''\) is the value: \valI{True}
%         \end{itemize}
%       \end{itemize}
%       \vspace{-\baselineskip}
%     \end{definition}
%   \end{note}



% \paragraph*{Denying \supportII{} \hfill (Optional)}
% \label{sec:denying-supportii}


% \begin{note}
%   \supportII{} positive condition for \ros{1} without a \wit{}.
%   This doesn't help me.
%   For, I need \ros{} for other work.
%   It's not merely the case that what I'm getting is a \ros{} without a \wit{}.
% \end{note}

% \begin{note}
%   In addition to \wit{} for a \ros{}, we also define a \pwit{} for a \ros{}:

%   \begin{definition}[A \pwit{} for a \ros{}]
%     \label{def:Pwit}%
%     \vspace{-\baselineskip}
%     \begin{itemize}
%     \item
%       An event \(e^{-}_{d^{-}}\) is \emph{\pwit{0}} for a \ros{} between \(\pv{\phi}{v}\) and \(\Phi\), for \vAgent{} through event \(e_{d}\)
%     \end{itemize}

%     \emph{If and only if:}

%     \begin{itemize}
%     \item
%       \(e^{-}_{d^{-}}\) is an event in which \vAgent{} is concluding \(\pv{\phi}{v}\) from \(\Phi\).
%     \item
%       \(e^{-}_{d^{-}}\) occurs prior to or at the same time as \(e_{d}\).
%     \end{itemize}
%     \vspace{-\baselineskip}
%   \end{definition}

%   \noindent%
%   The definition of a \pwit{} is motivated by \assuPP{}.
%   For, if \(e^{-}_{d^{-}}\) is an event in which an agent is concluding \(\pv{\phi}{v}\) from \(\Phi\) then by \assuPP{}, \(e^{-}_{d^{-}}\) develops into an event \(e^{+}_{d^{+}}\) in which the agent concludes \(\pv{\phi}{v}\) from \(\Phi\).
% \end{note}


% \begin{note}
%   Given \autoref{def:Pwit}, an observation:

%   \begin{observation}%
%     \label{obs:supportIIplus}%
%     If deny \supportII{} then \ros{} either \wit{} or something compatible with absence of \wit{}.
%   \end{observation}

%   \begin{argument}{obs:supportIIplus}
%     Suppose all compatible with absence of a \wit{}.
%     Now,

%     Suppose \fc{} and no \ros{}.
%     Then, it must be the case that \wit{} for \ros{}.
%     For, there is some action the agent does, the agent may do the action, and the completion of the action concludes, and at this point, a \ros{} by \supportII{}.

%     Hence, it is always the case that \wit{}.
%   \end{argument}

%   Our interest is with the initial disjunct.
%   It must be the case that \wit{}.
%   We tie \ros{1} to conclusions and to avoid commitment.
%   However, free to strengthen if you like.

%   Key insight here is that you are free to strengthen, so long as compatible.
% \end{note}



%%% Local Variables:
%%% mode: latex
%%% TeX-master: "master"
%%% TeX-engine: luatex
%%% End:


\chapter{\requ{3}}
\label{cha:requs}

\begin{note}
  A \requ{} is a relation between \(\pv{\psi}{v'}\) from \(\Psi\) being a \fc{} and an event in which an agent concludes \(\pv{\phi}{v}\) from \(\Phi\).
\end{note}

\begin{note}
  Important to generate counterexamples to \issueConstraint{}.
  Still, only certain \requ{1} generate counterexamples to \issueConstraint{}.%
  \footnote{
    I.e.\ one may jointly hold:
    \begin{enumerate*}[label=(\alph*)]
    \item
      various \requ{1} exist, and
    \item
      answers to \qWhyV{} are constrained by answers to \qHowV{} via \issueConstraint{}
    \end{enumerate*}%
    .
    For more, see \autoref{prop:requ-not-n-ce} on \autopageref{prop:requ-not-n-ce}, below.
  }
  Hence, we introduce and motivate the idea of a \requ{0} with \requ{1} which are compatible with \issueConstraint{}.
\end{note}

\section{\requ{3}}
\label{cha:requs:requs}

\begin{note}
  \begin{definition}[A \requ{0}]%
    \label{def:requ}%
    For any event \(e_{d}\) such that \(e_{d}\) is an event in which \vAgent{} concludes \(\pv{\phi}{v}\) from \(\Phi\):
    %
    \begin{itemize}
    \item
      \(\pv{\psi}{v'}\) being a \fc{} from \(\Psi\) for \vAgent{} throughout \(e^{\flat}_{d^{\flat}}\) is a \emph{\requ{}} of \(e_{d}\).
    \end{itemize}

    \emph{If and only if}

    \begin{itemize}
    \item
      \begin{itenum}
      \item[\emph{If}:]
        \(e^{\flat}_{d^{\flat}}\) develops into an event \(e_{d}\).
      \item[\emph{Then}:]
        \(\pv{\psi}{v'}\) is a \fc{} from \(\Psi\) for \vAgent{} throughout \(e^{\flat}_{d^{\flat}}\).
      \end{itenum}
    \end{itemize}
    \vspace{-\baselineskip}
  \end{definition}

  \noindent%
  Broad idea of a \requ{} is that we fix an event, and then consider what must be the case with respect to some sub-event.
  Specifically, applied to an event in which an agent concludes, and holds just in case what must be the case is a \fc{}.%
  \footnote{
    \itc{2}, so either does not develop or \fc{}.

    Something about the event secures requirement.
  }%
  \footnote{
    Note, it is not always the case that \(\pvp{\phi}{v}{\Phi}\) is a \requ{} of an agent concluding \(\pv{\phi}{v}\) from \(\Phi\).
    For example, consider the partnered case from \autoref{obs:cds-arb}
  }

  In this respect, the definition of a \requ{} is similar to \qWhy{}.
  However, in contrast to \qWhy{}, a \requ{} concerns \fc{1} rather than \ros{1} and is a statement.

  Still, there is a close connexion between \fc{1} and \ros{} given \supportII{} and we soon observe \ros{} answers \qWhy{}.

  We define \requ{} in terms of \fc{1} as a \ros{} is an abstraction for which we only have sufficient conditions.
  A \fc{} on the other hand, is something we have defined.
\end{note}

\subsection{Illustrations}
\label{sec:illustrations}

\begin{note}
  Work through a \scen{} to illustrate the definition of a \requ{}.
\end{note}

\paragraph*{Lucas numbers}

\begin{note}
  \begin{scenario}[Lucas numbers]%
    \label{scen:LucasNums}%
    The Lucas numbers are recursively defined as follows:%
    \footnote{
      Starting at \(2\), see \hyperlink{cite.OEIS.:aa}{OEIS sequence A000032} for more details.
    }
    %
    \[
      L_{n} = \left\{
        \begin{array}{ll}
          2 & \text{if } n = 0 \\
          1 & \text{if } n = 1 \\
          L_{n-1} + L_{n-2} & \text{if } n > 1 \\
        \end{array}
      \right.
    \]
    %
    An agent is alone and quite bored.
    The agent sets out to calculate the first ten Lucas numbers by applying the recursive definition.
    %
    The agent begins:
    %
    \[
      \begin{array}{cccccc}
        L_{0} & L_{1} & L_{2} & L_{3} & L_{4} & \cdots \\
        \hline
        2 & 1 & 3 & 4 & 7 & \cdots \\
      \end{array}
    \]
    %
    And, eventually concludes:
    %
    \[
      \pv{\propI{The first ten Lucas numbers are 2, 1, 3, 4, 7, 11, 18, 29, 47, and 76}}{\valI{True}}
    \]
    %
    From some \pool{} \(\Phi\).
  \end{scenario}

  Event of interest is \(e\) such that \propI{L\textsubscript{[:9]} = [2, 1, 3, 4, 7, 11, 18, 29, 47, 76]}{\valI{True}} from \(\Phi\).%
  \footnote{
    \propI{L\textsubscript{[:9]} = [2, 1, 3, 4, 7, 11, 18, 29, 47, 76]} shortens \propI{The first ten Lucas numbers are 2, 1, 3, 4, 7, 11, 18, 29, 47, and 76}.
  }

  \begin{observation}[\requ{3} of \autoref{scen:LucasNums}]%
    \label{obs:LucasRequ}%
    There is some sub-event \(e^{\flat}\) of \(e\) such that:
    %
    \begin{itemize}
    \item
      \(\pv{\propI{L\textsubscript{8} = 47}}{\valI{True}}\) from some \pool{} \(\Phi_{8}\) is \requ{1} of the agent concluding .
    \end{itemize}
  \end{observation}

  \begin{motivation}{obs:LucasRequ}
    The agent concludes \pv{\propI{L\textsubscript{[:9]} = [2, 1, 3, 4, 7, 11, 18, 29, 47 76]}}{\valI{True}} from \(\Phi\).

    The agent reasons by the recursive definition.
    Hence, the agent obtains \(L_{9}\) by adding \(L_{8}\) and \(L_{7}\).
    So, agent concludes \(\pv{\propI{L\textsubscript{8} = 47}}{\valI{True}}\) from \(\Phi_{8}\).
    And, \pv{\propI{L\textsubscript{7} = 29}}{\valI{True}} from \(\Phi_{7}\)
    For, without \(L\textsubscript{8}\) and \(L\textsubscript{7}\), the agent does not get \(L\textsubscript{9}\).

    Sub-event \(e^{8}\)

    Now, \(e^{8}\) agent concludes.
    So, by parallel reasoning, \(L\textsubscript{7}\) and \(L\textsubscript{6}\).
    But, then \(L\textsubscript{7}\), \(L\textsubscript{6}\) and definition.
    Hence, action, and when the agent does this, \(L\textsubscript{8}\).

    So, must be \(L\textsubscript{8}\) and clear just prior to this, \(L\textsubscript{8}\) is a \fc{}.
  \end{motivation}

  The same applies to \(L\textsubscript{7}\), \(L\textsubscript{6}\), and so on until we reach the base cases of \(L\textsubscript{1}\) and \(L\textsubscript{0}\), as these are given.
\end{note}

\begin{note}
  A \requ{} is nothing special.
  It's an observation about what is.

  In this case, given an event, and works backwards given additional details to observe which \fc{1} held.
  Note, however, that when the agent concludes \pv{\propI{L\textsubscript{[:9]} = [2, 1, 3, 4, 7, 11, 18, 29, 47 76]}}{\valI{True}} from \(\Phi\) the agent has a \wit{} for all \ros{} mentioned.
\end{note}

\paragraph*{Lost keys}

\begin{note}
  \requ{} does not require an event has happened.
  In particular, consider development of an event.
  \requ{} helps observe an event is not such that concluding.

  \begin{scenario}[Lost keys]%
    \label{illu:lost-key}%
    An agent thinks they may have lost their keys.
    They usually leave place my keys on the right side of their desk, near a copy of~\citeauthor{Vickers:1989tr}'s~\citetitle{Vickers:1989tr} they've been saving for a rainy day.
    Their keys aren't there.

    They've searched over, under, and beside the desk.
    They haven't found their keys.

    Still, the agent holds the following principle:
    \begin{quote}
      If they thinks of a place to look for an object, they do not conclude the object is lost without searching the place.
    \end{quote}
  \end{scenario}

  \begin{observation}
    \label{obv:lK:requ}
    If some action such that agent consider may be in  \(l\), then agent is not concluding.
  \end{observation}

  {
    \color{red}
  \begin{itemize}
  \item
    \pv{\propI{The agent has no further idea of where to look}}{\valI{True}} being a \fc{} from \(\Psi\) is a \requ{} of the agent concluding \pv{\propI{The agent has lost their keys}}{\valI{True}} from \(\Phi\).
  \end{itemize}
  }

  \begin{motivation}{obv:lK:requ}
    Suppose agent is concluding.
    Then, by \assuPP{} event in which agent concludes.
    Therefore, \requ{} of any such event.
    For, else the agent goes and searches.

    Antecedent assumes there is some action.
    Therefore, problem.
  \end{motivation}

  Note, this doesn't rule out \(l\) and no as a \fc{}.
  The problem is, not in a position to determine which conclusion.
\end{note}

\section{\requ{3}, \qWhyV{}, and \issueConstraint{}}
\label{sec:comining-ingredients}

\begin{note}
  Section links \requ{0} to \qWhyV{} and \issueConstraint{}.

  Start by defining a collection of conditions.
  Then, \qWhy{}.
  From this, additional condition to get counterexample to \issueConstraint{}.
\end{note}

\subsection{\requ{3} and \qWhyV{}}

\begin{note}
  \begin{proposition}[\requ{3} and \qWhyV{}]
    \label{prop:requ-WhyV}
    \vspace{-\baselineskip}
    \begin{itenum}
    \item[\emph{If}:]
      Conditions~%
      \ref{def:rCs:C}~and~\ref{def:rCs:Cing}~jointly hold:
      \begin{enumerate}[label=\arabic*., ref=\arabic*]
      \item
        \label{def:rCs:C}
        \(e_{d}\) is an event in which \vAgent{} concludes \(\pv{\phi}{v}\) from \(\Phi\).
      \item
        \label{def:rCs:Cing}
        There is some sub-event \(e^{\flat}_{d^{\flat}}\) of \(e_{d}\) such that:
        \begin{itemize}
        \item
          \(\pv{\psi}{v'}\) being a \fc{} from \(\Psi\) for \vAgent{} throughout \(e^{\flat}_{d^{\flat}}\) is a \requ{} of \(e_{d}\).
        \end{itemize}
      \end{enumerate}
    \item[\emph{Then}:]
      A \ros{0} between \(\pv{\psi}{v'}\) and \(\Psi\) answers \qWhyV{}.
    \end{itenum}
    \vspace{-\baselineskip}
  \end{proposition}

  \begin{argument}{prop:requ-WhyV}
    Assume conditions~\ref{def:rCs:C}~and~\ref{def:rCs:Cing}~jointly hold.

    From Condition~\ref{def:rCs:C}, \(e_{d}\) is an event in which \vAgent{} concludes \(\pv{\phi}{v}\) from \(\Phi\).
    So, \qWhyV{} applies to \(e_{d}\).

    Our task is to show the conditional of \qWhyV{} is true.

    Let \(e^{\flat}_{d^{\flat}}\) by an event which satisfies the existential of Condition~\ref{def:rCs:Cing}.

    Further, suppose a \ros{} between \(\pv{\psi}{v'}\) and \(\Psi\) fails to hold for \vAgent{} through \(e^{\flat}\).

    Now, as an instance of \supportII{}:
    %
    \begin{itenum}
    \item[\emph{If}:]
      \(\pv{\psi}{v'}\) is a \fc{0} from \(\Psi\) for \vAgent{} throughout \(e^{\flat}_{d^{\flat}}\).
    \item[\emph{Then}:]
      A \ros{0} between \(\pv{\psi}{v'}\) and \(\Psi\) holds for \vAgent{} throughout \(e^{\flat}_{d^{\flat}}\).
    \end{itenum}
    %
    \noindent%
    Hence, as a \ros{0} between \(\pv{\psi}{v'}\) and \(\Psi\) \emph{fails} to hold for \vAgent{} throughout \(e^{\flat}_{d^{\flat}}\), it follows that \(\pv{\psi}{v'}\) is \emph{not} a \fc{0} from \(\Psi\) for \vAgent{} throughout \(e^{\flat}_{d^{\flat}}\).

    And, as \(e^{\flat}_{d^{\flat}}\) satisfies the existential of \ref{def:rCs:Cing} and is \(\pv{\psi}{v'}\) is \emph{not} a \fc{0} from \(\Psi\) for \vAgent{} throughout \(e^{\flat}_{d^{\flat}}\), it follows that \(e^{\flat}_{d^{\flat}}\) is not, or does not develop into, \(e_{d}\).
    This contradicts \(e^{\flat}_{d^{\flat}}\) being a \se{0} of \(e_{d}\), and hence contradicts Condition~\ref{def:rCs:Cing}.
  \end{argument}
\end{note}



\subsection{\requ{3} and \issueConstraint{}}
\label{cha:binding:sec:requ-iC}

\begin{note}
  \autoref{prop:requ-WhyV} is the basis for counterexamples to \issueConstraint{}.
  For, the following proposition immediately follows from \autoref{prop:requ-WhyV} and \issueConstraint{}.

  \begin{proposition}[\requ{3} and \issueConstraint{}]
    \label{prop:requ-WhyV-ces}
    \vspace{-\baselineskip}
    \begin{itenum}
    \item[\emph{If}:]
      Conditions~\ref{prop:requ-WhyV-ces:C}~and~\ref{prop:requ-WhyV-ces:Cing}~jointly hold:
      \begin{enumerate}[label=\arabic*., ref=(\arabic*)]
      \item
        \label{prop:requ-WhyV-ces:C}
        \(e\) is an event in which \vAgent{} concludes \(\pv{\phi}{v}\) from \(\Phi\).
      \item
        \label{prop:requ-WhyV-ces:Cing}
        There is some sub-event \(e^{\flat}\) of \(e\) such that:
        \begin{enumerate}[label=\alph*., ref=(\arabic{enumi}\alph*)]
        \item
          \label{prop:requ-WhyV-ces:Cing:requ}
          \(\pv{\psi}{v'}\) being a \fc{} from \(\Psi\) for \vAgent{} throughout \(e^{\flat}\) is a \requ{} of \(e\).
        \end{enumerate}
      \end{enumerate}
    \item[\emph{And}:]
      \label{prop:requ-WhyVCes:noW}
      \vAgent{} does not have a \wit{} for a \ros{} between \(\pv{\psi}{v'}\) and \(\Psi\) when \vAgent{} concludes \(\pv{\phi}{v}\) from \(\Phi\).
    \item[\emph{Then}:]
      \issueConstraint{} does not hold.
    \end{itenum}
    \vspace{-\baselineskip}
  \end{proposition}

  \begin{argument}{prop:requ-WhyV-ces}
    Suppose both antecedents holds.

    Observe conditions~\ref{prop:requ-WhyV-ces:C}~and~\ref{prop:requ-WhyV-ces:Cing} from the first antecedent entail a \ros{0} between \(\pv{\psi}{v'}\) and \(\Psi\) answers \qWhyV{} via \autoref{prop:requ-WhyV}.
    Therefore, by \issueConstraint{} it must be the case that \vAgent{} has a \wit{} for the \ros{} between \(\pv{\psi}{v'}\) and \(\Psi\).
    However, by the second antecedent, \vAgent{} does not have a \wit{} for a \ros{} between \(\pv{\psi}{v'}\) and \(\Psi\) when \vAgent{} concludes \(\pv{\phi}{v}\) from \(\Phi\).
  \end{argument}
\end{note}

\begin{note}
  The existence of \requ{1} are compatible with \issueConstraint{}.

  \begin{observation}%
    \label{prop:requ-not-n-ce}%
    \autoref{prop:requ-WhyV} is compatible with \issueConstraint{}.
  \end{observation}

  \begin{motivation}{prop:requ-not-n-ce}
    \requ{3} are about an agent concluding.
    \fc{} when the agent concludes.
    An agent may fail to have a \wit{} for \fc{}.

    However, \issueConstraint{}, \wit{} when the agent concludes.
    Therefore, so long as \wit{} when the agent concludes, fine.

    Two possibilities.
    \begin{enumerate}
    \item
      The agent may already have a \wit{} for a \ros{} which follows from a \fc{}.

      In other words, the agent may have already concludes \(\pv{\psi}{v'}\) from \(\Psi\).
    \item
      Agent may obtain a \wit{} prior to conclusion.

      For example, first \scen{0}.
      If concluding sequence, \fc{}.
      Prior to getting \(L_{8}\), \fc{}.

      However, expectation of \wit{} when sequence.
      Indeed, \(L_{9}\) from \(L_{8} + L_{7}\).
    \end{enumerate}
    \vspace{-\baselineskip}
  \end{motivation}

  \noindent%
  Hence, \autoref{prop:requ-WhyV-ces} does not guarantee the existence of counterexamples to \issueConstraint{}.
\end{note}


%%% Local Variables:
%%% mode: latex
%%% TeX-master: "master"
%%% TeX-engine: luatex
%%% End:



\chapter{\tC{2}}
\label{cha:typical}
\nocite{Wilson:1994aa}
\nocite{Goodman:1983aa}

\begin{note}
  This chapter introduces and develops the idea of an agent \emph{\typeAdj{}} concluding some \prop{0}-\val{0} pair from some \pool{}.
  Intuitively, an agent is \typeAdj{} concluding just in case there is some generality to the agent's reasoning which concluding.
  For example, the agent is reasoning by modus ponens, arithmetic, or the categorical imperative, and so on.

  The role of an agent \tCV{} in the overall argument is motivation.
  Counterexamples to \issueConstraint{} without \tCV{}.
  However, depend on way in which \scen{0} is understood.
  If agent is \tCV{}, then leverage.

  \autoref{cha:requs} introduces the key idea of a \requ{}, and instances of \requ{1} are motivated by the issue of whether or not an agent is \tCV{}.
\end{note}

\begin{note}
  The chapter has two main sections:
  \begin{TOCEnum}
  \item
    \TOCLine{cha:typical:int}

    General idea, \illu{1}
  \item
    \TOCLine{cha:typical:tCDef}

    Link to \fc{1}.
    The link to \fc{1} is important for motivating instances of \requ{} in \autoref{cha:requs}.
  \end{TOCEnum}
\end{note}



\section{\tC{2}}
\label{cha:typical:int}

\begin{note}
  Our interest is characterising an event in which an agent is concluding.
  By~\autoref{assu:ConRea} (\autopageref{assu:ConRea}), whenever an concludes, and agent reasons.
  Hence, whenever an agent is concluding, and agent is reasoning.
  And, some instances of reasoning are, intuitively, of a type:
  There are sufficiently similar characteristics between two or more events in which an agent reasons for the agent to be reasoning by `type of reasoning \(T\)', where `type of reasoning \(T\)' may be `modus ponens', `means-end reasoning',%
  % \footnote{
  %   To illustrate, consider the following passage:
  %   \begin{quote}
  %     \indent ``I'm giving this to Eeyore,'' he explained, ``as a present.
  %     What are you going to give?''
  %
  %     ``Couldn't I give it too?'' said Piglet.
  %     ``From both of us?''
  %     
  %     ``No,'' said Pooh.
  %     ``That would not be a good plan.''
  %
  %     ``All right, then, I'll give him a balloon.
  %     I've got one left from my party.
  %     I'll go and get it now, shall I?''
  %     
  %     ``That, Piglet, is a very good idea.
  %     It is just what Eeyore wants to cheer him up.
  %     Nobody can be uncheered with a balloon.''%
  %     \mbox{ }\hfill\mbox{(\cite[78--79]{Milne:2009aa})}\newline
  %     \mbox{ }
  %   \end{quote}
  %   
  %   Two instances of means-end reasoning by Piglet.
  %   Common end of cheering up Eeyore.
  %   First, jointly giving a gift with Pooh.
  %   Second, giving a balloon as a present.
  % }
  `arithmetic', `consequentialism', and so on.

%   For example, \assumptionName{3}~\ref{assu:concluding:pools},~\ref{assu:ConRea},~\ref{assu:PP}~and \definitionName{3}~\ref{def:witnessing},~\ref{def:fc},~and~\ref{def:NScon} all involve a material conditional.%
%   \footnote{
%     Pages: \pageref{assu:concluding:pools},~\pageref{assu:ConRea},~\pageref{assu:PP} and~\pageref{def:witnessing},~\pageref{def:fc},~and~\pageref{def:NScon}, respectively.
%   }%
%   \(^{,}\)%
%   \footnote{
%     The indicative conditional is more complex.
%     See, for example,~(\cite{McGee:1985tz}),~(\cite{Kolodny:2010aa}), and the discussion of selection tasks starting on \autopageref{par:selection-tasks}, below.
%   }

%   Two key pieces of reasoning:

%   \begin{itemize}
%   \item
%     \emph{From} \pv{\propI{If }\phi\propI{ then }\psi}{\valI{True}} \emph{and} \pv{\propI{\phi}}{\valI{True}}, \emph{get} \pv{\propI{\psi}}{\valI{True}}.
%   \item
%     \emph{From} \pv{\propI{\phi}}{\valI{True}} \emph{and} \pv{\propI{\psi}}{\valI{False}}, \emph{get} \pv{\propI{If }\phi\propI{ then }\psi}{\valI{False}}
%   \end{itemize}

%   First is useful for applying.
%   Second is useful for rejecting.

%   For example:
%   First when considering arguments for propositions.
%   Second when determining whether there is a problem with one or more.

%   Definitions etc. modus ponens with assumption this pattern is recognised.
%   Additional work to fill in the content, but nothing about what connects the content.
\end{note}

\begin{note}
  I expect this observation is intuitive, and specific examples will follow.
\end{note}

\begin{note}
  Our interest is stating a necessary condition for whether or not an event in which an agent is concluding is an event in which an agent's reasoning is of some type.

  Provide idea.
  \illu{3} of idea.
  Expand, dispositions.
\end{note}


\subsection{Idea}
\label{sec:idea}

\begin{note}
  Abstractly, consider `\vAgent{} is \emph{\tCV{}} \(\pv{\phi}{v}\) from \(\Phi\) by type of reasoning \(T\)' as a predicate of an event.
  Necessary condition for predicate.

  \begin{idea}[\tCN{2}]%
    \label{idea:tC}%
    \vspace{-\baselineskip}
    \begin{itemize}
    \item
      \(\ed{}\) is an event in which \vAgent{} is \emph{\tCV{}} \(\pv{\phi}{v}\) from \(\Phi\).\newline
      \hfill(By some type of reasoning \(T\).)
    \end{itemize}

    \emph{Only if}:

    \begin{itemize}
    \item
      For some collections; \(\mathcal{E}\) of events, \(\mathcal{X}\) of \prop{0}-\val{0}-\pool{0}~pairings:
      \begin{itemize}
      \item
        For every event \(\ed{\prime}\), there is some \prop{0}-\val{} pair \(\pv{\psi}{v'}\) and \pool{0} \(\Psi\) in \(\mathcal{X}\) such that:
        \begin{itenum}
        \item[\emph{If}:]
          \(\ed{\prime}\) is in the collection of events \(\mathcal{E}\).
        \item[\emph{Then}:]
          \(\ed{\prime}\) is an event in which \vAgent{} is concluding \(\pv{\psi}{v'}\) from \(\Psi\).
        \end{itenum}
      \end{itemize}
    \end{itemize}
    \vspace{-\baselineskip}
  \end{idea}

  \noindent%
  Intuitively, \autoref{idea:tC} expresses the idea that the predicate `\vAgent{} is \tCV{} \(\pv{\phi}{v}\) from \(\Phi\) by type of reasoning \(T\)' applies to an event only if some `law' holds.%
  \footnote{
    Note, this is distinct from the position that concluding/reasoning is rule governed.
    \cite{Boghossian:2008vf,Boghossian:2012vb}, \cite{Broome:2002aa}.

    When reasoning, following a rule.
    We are talking about \emph{type} of concluding, rather than concluding.
    But, the object is not following the predicate.
  }
  Where, `law' is understood in the colloquial sense of a universally quantified material conditional.%
  \footnote{
    For example, consider:

    \citeauthor{Helmholtz:1977aa}'s characterisation of laws of nature:%
    \begin{quote}
      \nocite{Wilson:2006aa}
      Every law of nature asserts that upon preconditions alike in a certain respect, there always follow consequences that are alike in a certain other respect.%
      \mbox{ }\hfill\mbox{(\citeyear[122]{Helmholtz:1977aa})}
    \end{quote}
    The law of large numbers:
    \begin{quote}
      Things of every kind of nature are subject to a universal law which one may well call \emph{the Law of Large Numbers}.
      It consists in that if one observes large numbers of events of the same nature depending on causes which are constant and causes which vary irregularly, \dots, one finds that the proportions of occurrence are almost constant \dots\newline
      \mbox{ }\hfill\mbox{(\citeauthor{Seneta:2013aa}'s (\citeyear[9--10]{Seneta:2013aa}) translation of (\cite[7]{Poisson:1837aa}))}
    \end{quote}
    The law of truly large numbers:
    \begin{quote}
      [W]hen enormous numbers of events and people and their interactions cumulate over time, almost any outrageous event is bound to occur.%
      \mbox{ }\hfill\mbox{(\cite[853]{Diaconis:1989aa})}
    \end{quote}
    \citeauthor{Hempel:1965aa}'s Deductive-Nomological account of scientific explanation, \citeauthor{Boole:1854aa}'s laws of thought, etc.
  }

  \autoref{idea:tC} does not specify the content of the relevant law.
  However, intuitively, the law is to capture other cases where the agent is expected to be concluding by the same type of reasoning, and requires the agent \emph{is} concluding by the same type of reasoning.
\end{note}

\begin{note}
  Before turning to \illu{1}, note:
  \autoref{idea:tC} is not an analysis.
    Rather, \autoref{idea:tC} connects two truth functional statements by a conditional.
\end{note}



\subsection{\illu{3}}
\label{sec:illu3-1}

\begin{note}
  A few \illu{1}.
\end{note}


\paragraph*{Selection tasks}
\nocite{Wason:1968aa}
\nocite{Wason:1971aa}
\label{par:selection-tasks}

\begin{note}
  \citeauthor{Wason:1966aa} details their initial section task as follows:

  \begin{quote}
    The subjects (students) were presented with an array of cards and told that every card had a letter on one side and a number on the other side, and that either would be face upwards.
    They were then instructed to decide which cards they would need to turn over in order to determine whether the experimenter was lying in uttering the following statement:
    \emph{if a card has a vowel on one side then it has an even number on the other side}.%
    \mbox{ }\hfill\mbox{(\citeyear[145--146]{Wason:1966aa})}
  \end{quote}

  An example task is given in \autoref{fig:sectionTask}.

  \begin{figure}[H]
    \centering
    \begin{tikzpicture}[
      cardnode/.style={
        rectangle,
        minimum width=10mm,
        minimum height=14mm,
        align=center,
        rounded corners,
        font = {\Large\sffamily},
        very thick,
      },
      node distance=5mm,
      ]

      \node[cardnode, draw] (1) {2};
      \node[cardnode, draw, right = of 1] (2) {N};
      \node[cardnode, draw, right = of 2] (3) {E};
      \node[cardnode, draw, right = of 3] (4) {1};
    \end{tikzpicture}
    \caption{A selection task}
    \label{fig:sectionTask}
  \end{figure}

  \citeauthor{Wason:1966aa} observes the results are consistent with the following hypothesis:%
  \footnote{
    \citeauthor{Wason:1966aa} does not provide a detailed summary of the results.
    However, \citeauthor{Johnson-Laird:1969aa} detail results of \emph{twenty four} University College London students!
    Specifically, 19 of the 24 responded as excepted given \citeauthor{Wason:1966aa}'s hypothesis.
    (\citeyear[369--370]{Johnson-Laird:1969aa}).
  }
  \begin{quote}
    Subjects assume implicitly that a conditional statement has, not two truth values, but three: true, false and `irrelevant'.
    Vowels with even numbers verify, vowels with odd numbers falsify and consonants with any number are irrelevant.%
    \mbox{ }\hfill\mbox{(\citeyear[146]{Wason:1966aa})}
  \end{quote}
\end{note}

\begin{note}
  With respect to each subject, four distinct instances of \autoref{idea:tC}, corresponding to the four possible features of a card.
  We specify these with a variable `\(C\)' to represent the relevant card:

  \begin{center}
    \begin{tabular}{R{.45\textwidth} L{.45\textwidth}}
      \prop{2}-\val{0} pair & \pool{2} \\
      \hline
      \pv{C\propI{ needs to be turned over}}{\valI{True}} & \pv{C\propI{ has a vowel}}{\valI{True}} \\
      \pv{C\propI{ needs to be turned over}}{\valI{True}} & \pv{C\propI{ has an odd number}}{\valI{True}} \\
      \pv{C\propI{ needs to be turned over}}{\valI{False}} & \pv{C\propI{ has consonant}}{\valI{True}} \\
      \pv{C\propI{ needs to be turned over}}{\valI{False}} & \pv{C\propI{ has an even number}}{\valI{True}} \\
    \end{tabular}
  \end{center}

  \noindent%
  Short argument.
  Begin with two premises.

  \begin{enumerate}[label=\Alph*., ref=(\Alph*), noitemsep]
  \item
    \label{WasArg:results}
    Fail to express conclusions.
  \item
    \label{WasArg:IdeaI}
    \autoref{idea:tC} as described.
  \end{enumerate}

  Link these together:

  \begin{enumerate}[label=\arabic*., ref=(\arabic*), noitemsep]
  \item
    \label{WasArg:nC}
    Subjects did not conclude.

    From \ref{WasArg:results}.
  \item
    \label{WasArg:nCing}
    Subjects were not concluding.

    From \ref{WasArg:nC}, nothing to prevent the agent's from concluding.
  \item
    \label{WasArg:Done}
    The agent's were not \tCV{} by truth functional reasoning.

    From \ref{WasArg:nCing} and \ref{WasArg:IdeaI}
  \end{enumerate}

  Strength of the result depends on which instances of reasoning.

  {
    \color{red}

    Even if there are cases of truth-functional reasoning, something separates this from the reasoning found in selection tasks --- does not extend to all circumstances.
  }
\end{note}

\begin{note}
  Various other arguments may be seen to parallel the broad argument from \ref{WasArg:results} and \ref{WasArg:IdeaI} to \ref{WasArg:Done}.
  For example, consider the \citeauthor{Allais:1979aa} paradox (\cite{Allais:1979aa}),
  the Ellsberg paradox (\cite{Ellsberg:1961aa}), \citeauthor{Makinson:1965aa}'s Paradox of the Preface (\citeyear{Makinson:1965aa}), \citeauthor{Kyburg:1997aa}'s Lottery Paradox (\citeyear{Kyburg:1997aa}), \citeauthor{Quinn:1990aa}'s  puzzle of the self-torturer (\citeyear{Quinn:1990aa}), \citeauthor{Bratman:1981aa}'s arguments against the desire-belief model of practical reasoning (\citeyear{Bratman:1981aa,Bratman:1987aa}), and so on.%
  \footnote{
    Consider also \citeauthor{Harman:1984aa}'s (\citeyear{Harman:1984aa,Harman:1986ux}) arguments against a strong connexion between logical principles and principles of belief revision.

    \begin{quote}
      Logical principles are not directly rules of belief revision.
      [\dots]
      Logical principles hold universally, without exception, whereas the corresponding principles of belief revision would be at best prima facie principles, which do not always hold.%
      \mbox{ }\hfill\mbox{(\citeyear[107--108]{Harman:1984aa})}
    \end{quote}
  }
  Common to each mentioned is the idea that an agent failing to conclude something shows that other instances of the agent's reasoning does not have some general characteristic.
\end{note}


\paragraph*{Rules}

\begin{note}
  Selection tasks are events which happen, and \citeauthor{Wason:1966aa}'s hypothesis is that people \emph{in general} do not do not reason about conditionals using only \valI{True} and \valI{False}.
  However, the events the law of \autoref{idea:tC} quantifies over need not happen, and the event at issue may be a single event.
\end{note}

\begin{note}
  \begin{scenario}[Addition]%
    \label{illu:quus}%
    An agent is given pairs of numbers \(x\) and \(y\) and asked to respond with \(x + y\).
    The table below represents the event as the agent responds to the pairs.

    \medskip
    \hspace{2.8em}%
    \(
      \begin{array}{ccccccc}
      x & 3 & 54 & 21 & 3 & 17 & 0 \\
      y & 7 & 32 & 64 & 2 & 25 & 6 \\
      \hline
      \text{Response} & 10 & 86 & 85 & 5 & 42 & 6 \\
    \end{array}
    \)
    \medskip

    \noindent%
    The agent is distracted.
    However, if the agent had not been distracted, they would have continued as follows:

    \medskip
    \hfill%
    \(
    \begin{array}{ccccc}
      \cdots & 8 & 68 & 21 & 58 \\
      \cdots & 92 & 57 & 23 & 92 \\
      \hline
      \cdots & 100 & 5 & 44 & 5 \\
    \end{array}
    \)%
    \hspace{2.8em}%
    \mbox{ }%
    \newline%
  \end{scenario}

  \noindent%
  It seems the agent was not reasoning by addition.%
  \footnote{
    Rather, it seems the agent was reasoning by quaddition.
    Consider \citeauthor{Kripke:1982aa}'s (\citeyear{Kripke:1982aa}) def.\ of `quss':
    \begin{align*}
      x \text{ quss y} &= x \text{ plus } y, \text{ if } x,y < 57 \\
                       &= 5 \phantom{pl if x,,,} \text{ otherwise }
    \end{align*}
    \vspace{-\baselineskip}
  }

  For, consider the counterfactual event.
  The agent concluded \pv{\propI{x + y is 5}}{\valI{True}} from some \pool{} containing \pv{\propI{x is 68}}{\valI{True}} and \pv{\propI{y is 57}}{\valI{True}}.
  Hence, it seems the agent was not concluding \pv{\propI{x + y is 125}}{\valI{True}}.

  Further, though the counterfactual event suggests the agent was not reasoning by addition, it is less clear that the agent never reasons by addition.
  The agent's interpretation of `\(+\)' as something other than `plus' may be no different from interpreting `\(A \land B\)' as `\(A\) and \(B\)' rather than `the meet of sets \(A\) and \(B\)'.
\end{note}


\paragraph*{Powers}

\begin{note}
  Selection tasks and addition.
  Final \illu{0}, not all failures show an agent reasoning fails to be of some type.
\end{note}

\begin{note}
  \begin{scenario}[Powerful multiplication]%
    \label{illu:tR:powers}%
    A student has been studying algebra and has just been introduced to the rule of multiplication for powers (\(a^{n} \cdot a^{m} = a^{n + m}\)).

    At hand are a handful of exercises (from \cite[32]{Gelfand:1993aa}):
    %
    \begin{quote}
      \begin{enumerate}[label=(\alph*), ref=(\alph*)]
      \item
        \label{mfp:a}
        You know that \(2^{1001} \cdot 2^{n} = 2^{2000}\).
        What is \(n\)?
      \item
        \label{mfp:b}
        You know that \(2^{1001} \cdot 2^{n} = \sfrac{1}{4}\).
        What is \(n\)?
      \item
        \label{mfp:c}
        Which is bigger: \(10^{-3}\) or \(2^{-10}\)?
      \item
        \label{mfp:d}
        You know that \(\sfrac{2^{1000}}{2^{n}} = 2^{501}\).
        What is \(n\)?
      \item
        \label{mfp:e}
        You know that \(\sfrac{2^{1000}}{2^{n}} = \sfrac{1}{16}\).
        What is \(n\)?
      \item
        \label{mfp:f}
        You know that \(4^{100} = 2^{n}\).
        What is \(n\)?
      \item
        \label{mfp:g}
        You know that \(2^{100} \cdot 3^{100} = a^{100}\).
        What is \(a\)?
      \item
        \label{mfp:h}
        You know that \((2^{10})^{15} = 2^{n}\).
        What is \(n\)?
      \end{enumerate}
    \end{quote}
    %
    The student starts work on exercise~\ref{mfp:f}, and is concluding \(\pv{n\propI{ is }200}{\valI{True}}\).
  \end{scenario}

  \noindent%
  Is it the case that the student is reasoning with an understanding of multiplication for powers?

  Consider the event in which the agent starts work on~\ref{mfp:g}.
  Does it need to be the case that the agent is concluding \pv{\propI{a is 6}}{\valI{True}} from \pv{\propI{\(2^{100} \cdot 3^{100} = a^{100}\)}}{\valI{True}} in order for the agent to be reasoning by the rule of multiplication for powers when working on~\ref{mfp:f}?

  I think arguments may be made either way.
  On the one hand, the solution to \ref{mfp:f} may be obtained by straightforward pattern matching, while \ref{mfp:g} requires some insight.
  Hence, reasoning to the solution to \ref{mfp:g} may be required for the agent to be reasoning `with' the rule, rather than `by' the rule.
  On the other hand, the rule of multiplication for powers immediately follows from a basic understanding of exponentiation, and hence reasoning `with' the rule is simply reasoning by exponentiation --- there is no need for the agent to follow a rule.
\end{note}

\begin{note}
  \begin{observation}%
    \label{obs:typeLim}%
    It is plausible conditions \ref{obs:typeLim:E} and \ref{obs:typeLim:nX} may be true at the same time:
    \begin{enumerate}[label=\arabic*., ref=\arabic*]
    \item
      \label{obs:typeLim:E}
      \(\ed{}\) is an event in which \vAgent{} is \tCV{} \(\pv{\phi}{v}\) from \(\Phi\).
    \item
      \label{obs:typeLim:nX}
      For some \(\pv{\psi}{v'}\), \(\Psi\) in \(\mathcal{X}\):
      \begin{itemize}
      \item
        There is no event \(\ed{\prime}\) in which \vAgent{} is concluding \(\pv{\psi}{v'}\) from \(\Psi\).
      \end{itemize}
    \end{enumerate}
    \vspace{-\baselineskip}
  \end{observation}

  \begin{motivation}{obs:typeLim}
    Consider \autoref{illu:tR:powers}.
    Exercises~\ref{mfp:b},~\ref{mfp:d}, and~\ref{mfp:e} all involve fractions.
    However, agent is shaky on fractions.
    Rule of multiplication is good, but not with fractions.%
  \footnote{
    Consider also \citeauthor{Chomsky:2015aa}'s distinction between competence and performance.
    %
    \begin{quote}
      Arithmetical competence yields the correct number z for every pair~(x,~y) under addition or multiplication.
      But only a small finite subpart of arithmetical competence can be exhibited without external aids (by calculating in one's head).
      Obviously, that fact does not imply that arithmetical competence is correspondingly limited.%
      \mbox{ }\hfill\mbox{(\citeyear[xii]{Chomsky:2015aa})}
    \end{quote}
    %
    Though, \citeauthor{Chomsky:2015aa} also motivates the distinction by errors (\citeyear[2]{Chomsky:2015aa}).

    For us, errors and mistakes are dealt with via the progressive and the relevant \torN{}.
    An agent may make errors or mistakes, while concluding --- so long as the agent is concluding.
    Hence, if an agent is permitted to fail to be concluding some \(\pv{\psi}{v'}\) from \(\Psi\) while \tCV{} \(\pv{\phi}{v}\) from \(\Phi\), then \(\pvp{\psi}{v'}{\Psi}\) is not a \tI{} of the relevant \torNa{}.
    (See the discussion of \autoref{illu:tR:powers} on \autopageref{illu:tR:powers}.)
}
  \end{motivation}
\end{note}



\section{\tC{2} and \fc{1}}
\label{cha:typical:tCDef}


\begin{note}
  We now link \tCV{} to \fc{1}.
\end{note}



\subsection{\torN{3}}
\label{cha:typical:tCDef:ToRdef}

\begin{note}
  \torN{3} are defined in terms of \prop{1}, \val{1}, and \pool{1}:

  \begin{definition}[A \torN{0}]
    \label{def:tor}
    \mbox{ }
    \vspace{-\baselineskip}
    \begin{itemize}
    \item
      \(T\) is a \torN{}.
    \end{itemize}

    \emph{If and only if}:

    \begin{itemize}
    \item
      \(T\) is a collection of \prop{0}-\val{0}-\pool{0} pairings.
    \end{itemize}
    \vspace{-\baselineskip}
  \end{definition}

  \noindent%
  Intuitively, a \torN{} (as defined) as the \emph{extension} of a \torN{}.
  The motivation for \autoref{def:tor} has two parts:

  First, by \autoref{assu:concluding:pvp} (\autopageref{assu:concluding:pvp}) and agent conclusion is a \prop{0}-\val{0} pairing, and by \autoref{assu:concluding:pools} (\autopageref{assu:concluding:pools}) an agent always concludes from some \pool{}.
  Hence, whenever an agent is concluding there is always some relevant \prop{0}-\val{0}-\pool{0} pairing.

  Second, by \autoref{assu:ConRea} (\autopageref{assu:ConRea}), if an agent concludes, the agent reasons to the \prop{0}-\val{0} pair from the \pool{0}.
  However, we place no additional constraints on reasoning.
  Hence, the framework with which we work does not allow finer grain.
\end{note}

\begin{note}
  To \illu{0}, consider the type `multiplication by powers' with respect to \autoref{illu:tR:powers}:
  \begin{center}
    \begin{tabular}{R{.45\textwidth} L{.45\textwidth}}
      \multicolumn{2}{c}{\prop{2}-\val{}-\pool{} pairings in type `multiplication by powers'} \\
      \hline\hline
      \prop{2}-\val{0} pair & \pool{2} \\
      \hline
      \pv{\propI{n is 999}}{\valI{True}} & \pv{\propI{\(2^{1001} \cdot 2^n = 2^{2000}\)}}{\valI{True}}, \dots \\
      \pv{\propI{n is -1003}}{\valI{True}} & \pv{\propI{\(2^{1001} \cdot 2^{n} = \sfrac{1}{4}\)}}{\valI{True}}, \dots \\
      \pv{\propI{\(10^{-3}\) is bigger than \(2^{-10}\)}}{\valI{True}} & \dots \\
      \pv{\propI{n is 500}}{\valI{True}} & \pv{\propI{\(\sfrac{2^{1000}}{2^{n}} = 2^{501}\)}}{\valI{True}}, \dots \\
    \end{tabular}
  \end{center}

  \begin{center}
    \begin{tabular}{R{.45\textwidth} L{.45\textwidth}}
      \multicolumn{2}{c}{\prop{2}-\val{}-\pool{} pairings not in type `multiplication by powers'} \\
      \hline\hline
      \prop{2}-\val{0} pair & \pool{2} \\
      \hline
      \hline
      \pv{\propI{n is 996}}{\valI{False}} & \pv{\propI{\(\sfrac{2^{1000}}{2^{n}} = \sfrac{1}{16}\)}}{\valI{True}}, \dots \\
      \pv{\propI{n is 2000}}{\valI{True}} & \pv{\propI{\(4^{100} = 2^{n}\)}}{\valI{True}}, \dots \\
      \pv{\propI{a is 5}}{\valI{True}} & \pv{\propI{\(2^{100} \cdot 3^{100} = a^{100}\)}}{\valI{True}}, \dots \\
      \pv{\propI{n is \(150\)}}{\valI{Want}} & \pv{\propI{\((2^{10})^{15} = 2^{n}\)}}{\valI{True}}, \dots \\
    \end{tabular}
  \end{center}

  The collection may be understood representative of the broad type `multiplication by powers'.
  In this respect, many other \prop{0}-\val{0}-\pool{0} pairs, such as \pv{\propI{n is 10}}{\valI{True}} paired with \pv{\propI{\(2^{10} \cdot 2^n = 2^{10}\)}}{\valI{True}}, \dots, and \pv{\propI{n is -7}}{\valI{True}} paired with \pv{\propI{\(2^{10} \cdot 2^n = 2^{3}\)}}{\valI{True}}, \dots.

  Still, the collection may also be understood \tI{1} of the narrow type `multiplication by powers with respect to problems \ref{mfp:a}--\ref{mfp:h}' --- here additional pairings which cover problems \ref{mfp:e} to \ref{mfp:h} may be included.
  Or, understood as \tI{1} of the very narrow type `multiplication by powers with respect to problems \ref{mfp:a}--\ref{mfp:d}' --- here the collection seems complete.

  \torN{3} are, for our interests, collections of \prop{0}-\val{0}-\pool{0} pairings.
  There need be no `natural' description of the type.
\end{note}

\subsection{\ptC{2}}
\label{sec:ptr0}

\begin{note}
  With a \torN{} in hand, we define what it is for an agent to be `\ptypeAdv{0}' concluding:

  \begin{definition}[\ptC{2}]
    \label{def:ptC}
    \vspace{-\baselineskip}
    \begin{itemize}
    \item
      \(\ed{}\) is an event in which \vAgent{} is \emph{\ptCV{0}} \(\pv{\phi}{v}\) from \(\Phi\) by type \(T\).
    \end{itemize}

    \emph{If and only if}:

    \begin{itemize}
    \item
      For every \tI{} \(\pvp{\psi}{v'}{\Psi}\) of \(T\):
      \begin{itemize}
    \item
      For any sub-event \(e^{\flat}\) of \(e_{d}\) there is some description \(d^{\flat}\) such that \(\ed{\flat}\) is an event in which \vAgent{} may immediately do action \(a\) and conditions \ref{def:ptC:act} and \ref{def:ptC:result} are true:
      %
      \begin{enumerate}[label=\Alph*., ref=\Alph*, series=fcCounter]
        %
      \item
        \label{def:ptC:act}
        The event \(e^{\sharp}_{d^{\sharp}}\) in which \vAgent{} does \(a\) is an event in which \vAgent{} is concluding \(\pv{\psi}{v'}\) from \(\Psi\).
        %
      \item
        \label{def:ptC:result}
        For each \prop{0}-\val{0} pair \(\pv{\phi'}{v'}\) in \(\Phi\), \vAgent{} \evals{} \(\phi'\) as having value \(v'\) prior to doing \(a\).
        %
      \end{enumerate}
    \end{itemize}
    %   \begin{itemize}
    %   \item
    %     There is some action \(a\) \vAgent{} may immediately do such that:
    %     \begin{itemize}
    %     \item
    %       \vAgent{} is concluding \(\pv{\psi}{v'}\) from \(\Psi\) when \vAgent{} does \(a\).
    %     \end{itemize}
    %   \end{itemize}
    \end{itemize}
    \vspace{-\baselineskip}
  \end{definition}

  \noindent%
  An agent \ptCV{0} \(\pv{\phi}{v}\) from \(\Phi\) by type \(T\) is a strong condition.
  In short, it must be the case that for every \tI{} of the type, there is a some action the agent may do such that the agent is concluding the \tI{} when they do the action.

  {
    \color{blue}
    \fc{2}\dots
  }
\end{note}

\begin{note}
  To illustrate, consider the agent of \autoref{illu:tR:powers} when concluding \pv{\propI{a is 6}}{\valI{True}} from \pv{\propI{\(2^{100} \cdot 3^{100} = a^{100}\)}}{\valI{True}}.

  With respect to the type `multiplication by powers', it is plausible the agent is not \ptCV{}.
  For, it is plausible there is no action such that the agent is concluding, say, \pv{\propI{n is 5}}{\valI{True}} from \pv{\propI{\(3^{10} \cdot 3^{n} = 14,348,907\)}}{\valI{True}}.%
  \footnote{
    Even if the agent infers there is some natural number \(m\) such that \(3^{m} = 14,348,907\), the agent will plausible choose to do something else other than calculate the number.
  }

  However, with respect to the narrow type `multiplication by powers with respect to problems \ref{mfp:a}--\ref{mfp:h}' it seems plausible the agent is \ptCV{}.
  For, if the agent were to begin working on a different problem, the agent would be concluding the appropriate answer.
\end{note}

% \begin{note}
%   So, an agent \ptCV{0} is a strong condition.
%   Still, two important features:

%   First, there is a close connexion between an agent \ptCV{} \(\pv{\phi}{v}\) from \(\Phi\) and \(\pv{\phi}{v}\) from \(\Phi\) being a \fc{}.

%   Second, there is a straightforward relation between an agent \tCV{} and the agent \ptCV{0}:

%   \begin{proposition}[\typeAdv{2} \ptypeAdj{}]%
%     \label{prop:tCV-ptCV}%
%     \vspace{-\baselineskip}
%     \begin{itenum}
%     \item[\emph{If}:]
%       \(\ed{}\) is an event in which \vAgent{} is \tCV{} \(\pv{\phi}{v}\) from \(\Phi\) by \torNa{} \(T\).
%     \item[\emph{Then}:]
%       There exists some \torN{} \(T'\) such that:
%       \begin{itemize}
%       \item
%         \(\ed{}\) is an event in which \vAgent{} is \ptCV{} \(\pv{\phi}{v}\) from \(\Phi\) by \torNa{} \(T'\).
%       \end{itemize}
%     \end{itenum}
%     \vspace{-\baselineskip}
%   \end{proposition}

%   \begin{argument}{prop:tCV-ptCV}
%     Suppose  \(\ed{}\) is an event in which \vAgent{} is \tCV{} \(\pv{\phi}{v}\) from \(\Phi\) by \torN{} \(T\).
%     Then, by \autoref{idea:tC} there is some collections \(\mathcal{E}\) of events and \(\mathcal{X}\) of \prop{0}-\val{0}-\pool{0}~pairings such that:

%     \begin{itemize}[noitemsep]
%     \item
%       For every event \(\ed{\prime}\), there is some \prop{0}-\val{} pair \(\pv{\psi}{v'}\) and \pool{0} \(\Psi\) in \(\mathcal{X}\) such that:
%       \begin{itenum}[noitemsep]
%       \item[\emph{If}:]
%         \(\ed{\prime}\) is in the collection of events \(\mathcal{E}\).
%       \item[\emph{Then}:]
%         \(\ed{\prime}\) is an event in which \vAgent{} is concluding \(\pv{\psi}{v'}\) from \(\Psi\).
%       \end{itenum}
%     \end{itemize}

%     \noindent%
%     Now, consider the collection of events \(\mathcal{E}'\) such that:
%     \begin{itemize}[noitemsep]
%     \item
%       For some \(\pvp{\psi}{v'}{\Psi}\) in \(\mathcal{X}\):
%       \begin{itemize}[noitemsep]
%       \item
%         \(\ed{\prime}\) is an event in which \vAgent{} is concluding \(\pv{\psi}{v'}\) from \(\Psi\) after doing some action \(a\) available to \vAgent{}.
%       \end{itemize}
%     \end{itemize}

%     \noindent%
%     Finally, consider the type \(T'\) given by pairings \(\pvp{\psi}{v'}{\Psi}\) such that:
%     \begin{itemize}[noitemsep]
%     \item
%       For some event \(\ed{t}\) in \(\mathcal{E}\):
%       \begin{itemize}[noitemsep]
%       \item
%         \(\ed{t}\) is an event in which \vAgent{} is concluding \(\pv{\psi}{v'}\) from \(\Psi\) after doing some action \(a'\) available to \vAgent{}.
%       \end{itemize}
%     \end{itemize}

%     \noindent%
%     It is immediate by construction that:
%     \begin{itemize}[noitemsep]
%     \item
%       For every \tI{} \(\pvp{\psi}{v'}{\Psi}\) of \(T'\).
%       \begin{itemize}[noitemsep]
%       \item
%         There is some action \(a\) available to \vAgent{} such that:
%         \begin{itemize}
%         \item
%           \vAgent{} is concluding \(\pv{\psi}{v'}\) from \(\Psi\) when \vAgent{} does \(a\).
%         \end{itemize}
%       \end{itemize}
%     \end{itemize}
%     \vspace{-1.5\baselineskip}
%   \end{argument}
% \end{note}



\subsection{\rotoc{2}}
\label{sec:rotoc}


\begin{note}
  The final definition :

  \begin{definition}[A \rotoc{}]%
    \label{def:rotoc}%
    \vspace{-\baselineskip}
    \begin{itemize}
    \item
      \(T'\) is a \emph{\tRep{}} of \vAgent{} \tCV{} \(\pv{\phi}{v}\) from \(\Phi\) by type \(T\) in \(\ed{}\).
    \end{itemize}

    \emph{If and only if:}

    \begin{itemize}
    \item
      \begin{itenum}
      \item[\emph{If}:]
        \(\ed{}\) is an event in which \vAgent{} is \tCV{0} \(\pv{\phi}{v}\) from \(\Phi\)~by~type~\(T\).
      \item[\emph{Then}:]
        \(\ed{}\) is an event in which \vAgent{} is \ptCV{0} \(\pv{\phi}{v}\) from \(\Phi\)~by~type~\(T'\).
      \end{itenum}
    \end{itemize}
    \vspace{-.5\baselineskip}
  \end{definition}

  \noindent%
  A \rotoc{} links an agent \ptCV{} to an agent \tCV{}.


  The role of \autoref{def:rotoc} is to easily talk about any if-then connexion between an agent \tCV{} \(\pv{\phi}{v}\) from \(\Phi\) by type \(T\) and \ptCV{} \(\pv{\phi}{v}\) from \(\Phi\) by type \(T'\).
\end{note}


\begin{note}
  Observe, \rotoc{} only states what must be the case if the agent is \tCV{}.
  \autoref{def:rotoc} does not provide sufficient resources to infer an agent is \tCV{} by some \torN{} when an agent is \ptCV{} by some type.%
  \footnote{
    Indeed, not even when the relevant types are identical.
  }
\end{note}


% \begin{note}
%   Simple proposition:

%   \begin{proposition}
%     \label{prop:tRepptCVtCV}
%     \vspace{-\baselineskip}
%     \begin{itenum}
%     \item[\emph{If}:]
%       \(T'\) is a \tRep{} of \vAgent{} \tCV{} \(\pv{\phi}{v}\) from \(\Phi\) by type \(T\) in \(\ed{}\).
%     \item[\emph{And}:]
%       \(\ed{}\) is an event in which \vAgent{} is \tCV{0} \(\pv{\phi}{v}\) from \(\Phi\) by~type~\(T\).
%     \item[\emph{Then}:]
%       \(\ed{}\) is an event in which \vAgent{} is \ptCV{0} \(\pv{\phi}{v}\) from \(\Phi\) by~type~\(T'\).
%     \end{itenum}
%     \vspace{-\baselineskip}
%   \end{proposition}

%   \noindent%
%   Note, if both antecedents of \autoref{prop:tRepptCVtCV} hold, then this entails the relevant \tRep{} is non-empty.
% \end{note}


\begin{note}
  In the case of selection tasks, close to the idea of a \tRep{}.
  However, slightly different.
\end{note}

\begin{note}
  \begin{illustration}[ジョジョリオン]%
    \nocite{huangmufeiluyan:2011aa}%
    Book.
    List of chapters.
    Is the agent reading the chapter titles?

    \begin{center}
      \bgroup
      \def\arraystretch{1.125}
      \begin{tabular}{R{.45\textwidth} L{.45\textwidth}}
        \multicolumn{2}{c}{Translations and chapter titles in `reading'} \\
        \hline\hline
        Translation & Title \\
        \hline
        Soft and wet & ソフト&ウェット \\
        \hdashline
        Safety above everything else & \multirow{3}*{無事が何より} \\
        Safety first & \\
        Gotta be safe & \\
        \hdashline
        Every day is summer vacation & 毎日が夏休み \\
        \hdashline
        A hair clip from ??? period & 清の時代の髪留め \\
      \end{tabular}
      \egroup
    \end{center}

    \noindent%
    First and third examples are straightforward.
    Difference is katakana and common kanji with a little grammar.
    Second, overly literal and slightly different translations due to lack of information about context.
    Fourth, do not need complete translation.

    \noindent%
    However, agent may fail to translate certain chapter titles such as:

    \begin{center}
      \bgroup
      \def\arraystretch{1.125}
      \begin{tabular}{R{.45\textwidth} L{.45\textwidth}}
        \multicolumn{2}{c}{Translations and chapter titles not in `reading'} \\
        \hline\hline
        Translation & Title \\
        \hline
        Software and wet & ソフト&ウェット \\
        \hdashline
        The Qing Dynasty Hair Clip & 清の時代の髪留め \\
        \hdashline
        ??? & 母と子
      \end{tabular}
      \egroup
    \end{center}

    \noindent%
    First, translation ruled out by context, though possible.
    Second and third, require knowledge that may be expected for fluency, but not reading.
    Second, did not require complete translation.
    Third, failure to translate \textquote{Mother and child}.
  \end{illustration}
\end{note}

% \begin{note}
%   \begin{observation}[Trivial \tRep{1}]%
%     \label{obs:tR:trivialRep}%
%     It may be the case that the only \rotoc{} is type which contains \(\pv{\phi}{v}\), etc.
%   \end{observation}

%   \begin{motivation}{obs:tR:trivialRep}
%     Exam.
%     Consider any other question, then immediately return to present question.%
%     \footnote{
%       To go further, plausibly get to counterfactuals.

%       For example, rather than action to the agent, go via prompting.
%       I give question, you get answer.
%     }
%     Here, \tRep{} is just the question.
%   \end{motivation}
% \end{note}


\subsection{\tCV{3} and \fc{1}}
\label{cha:typical:tC-fc}

\begin{note}
  We now link an agent \tCV{} \(\pv{\phi}{v}\) from \(\Phi\) to \fc{1}.

  \begin{proposition}[\tCV{2} and \fc{1}]
    \label{prop:tC-and-fc}

    \vspace{-\baselineskip}
    \begin{itenum}
    \item[\emph{If}:]
      Conditions~\ref{prop:tC-and-fc:A:rep},~\ref{prop:tC-and-fc:A:tC},~\ref{prop:tC-and-fc:A:tI},%
      ~and~\ref{prop:tC-and-fc:A:novel} jointly hold:
      \begin{enumerate}[label=\arabic*., ref=\arabic*]
        %
      \item
        \label{prop:tC-and-fc:A:rep}
        \(T'\) is a \tRep{} of \vAgent{} \tCV{} \(\pv{\phi}{v}\) from \(\Phi\) by type \(T\) in \(\ed{}\).
        % 
      \item
        \label{prop:tC-and-fc:A:tC}
        \(\ed{}\) is an event in which \vAgent{} is \tCV{0} \(\pv{\phi}{v}\) from \(\Phi\) by~type~\(T\).
        % 
      \item
        \label{prop:tC-and-fc:A:tI}
        \(\pvp{\psi}{v'}{\Psi}\) is a \tI{} of \(T'\)
        % 
      \item
        \label{prop:tC-and-fc:A:novel}
        For each \prop{0}-\val{0} pair \(\pv{\phi'}{v'}\) in \(\Phi\), \vAgent{} \evals{} \(\phi'\) as having value \(v'\) prior to and throughout \(\ed{}\).
        % 
      \end{enumerate}
    \item[\emph{Then}:]
      \(\pv{\psi}{v'}\) is a \fc{} from \(\Psi\) for \vAgent{} through \(\ed{}\).
    \end{itenum}
    \vspace{-\baselineskip}
  \end{proposition}


  \begin{argument}{prop:tC-and-fc}%
    Assume conditions \ref{prop:tC-and-fc:A:tC}, \ref{prop:tC-and-fc:A:rep}, \ref{prop:tC-and-fc:A:tI}, and~\ref{prop:tC-and-fc:A:novel} hold.

    \noindent%
    By Condition~\ref{prop:tC-and-fc:A:rep} \(T'\) is a \tRep{} of \vAgent{} \tCV{} \(\pv{\phi}{v}\) from \(\Phi\) by type \(T\) in \(\ed{}\).
    So, by \autoref{def:rotoc}:

    \begin{itenum}
    \item[\emph{If}:]
      \(\ed{}\) is an event in which \vAgent{} is \tCV{0} \(\pv{\phi}{v}\) from \(\Phi\)~by~type~\(T\).
    \item[\emph{Then}:]
      \(\ed{}\) is an event in which \vAgent{} is \ptCV{0} \(\pv{\phi}{v}\) from \(\Phi\)~by~type~\(T'\).
    \end{itenum}

    \noindent%
    And, by Condition~\ref{prop:tC-and-fc:A:tC} \(\ed{}\) is an event in which \vAgent{} is \tCV{0} \(\pv{\phi}{v}\) from \(\Phi\) by~type~\(T\).
    So:

    \begin{itemize}
    \item
      \(\ed{}\) is an event in which \vAgent{} is \ptCV{0} \(\pv{\phi}{v}\) from \(\Phi\)~by~type~\(T'\).
    \end{itemize}

    \noindent%
    Hence, by \autoref{def:ptC}:

    \begin{itemize}
    \item
      For every \tI{} \(\pvp{\chi}{v''}{X}\) of \(T'\).
      \begin{itemize}[noitemsep]
      \item
        There is some action \(a\) available to \vAgent{} such that:
        \begin{itemize}
        \item
          \vAgent{} is concluding \(\pv{\psi}{v'}\) from \(\Psi\) when \vAgent{} does \(a\).
        \end{itemize}
      \end{itemize}
    \end{itemize}

    \noindent%
    By Condition~\ref{prop:tC-and-fc:A:tI}, \(\pvp{\psi}{v'}{\Psi}\) is a \tI{} of \(T'\).
    Hence:

    \begin{itemize}[noitemsep]
    \item
      There is some action \(a\) available to \vAgent{} such that:
      \begin{itemize}[noitemsep]
      \item
        \vAgent{} is concluding \(\pv{\psi}{v'}\) from \(\Psi\) when \vAgent{} does \(a\).
      \end{itemize}
    \end{itemize}

    \noindent%
    Now, we have some action \(a\) such that:%

    \begin{itemize}[noitemsep]
    \item
      \(a\) is available to \vAgent{}.
    \item
      \vAgent{} is concluding \(\pv{\psi}{v'}\) from \(\Psi\), when \vAgent{} does \(a\).
    \end{itemize}

    \noindent%
    And, by Condition~\ref{prop:tC-and-fc:A:novel}:

    \begin{itemize}[noitemsep]
    \item
      For each \prop{0}-\val{0} pair \(\pv{\phi'}{v'}\) in \(\Phi\), \vAgent{} \evals{} \(\phi'\) as having value \(v'\).
    \end{itemize}

    \noindent%
    So, by \autoref{def:fc} (\autopageref{def:fc}), \(\pv{\psi}{v'}\) is a \fc{0} from \(\Psi\), for \vAgent{}.
  \end{argument}
\end{note}



\section*{Summary}

\begin{note}
  \tCN{2}.

  Extension account of \torN{}.
  Due to abstracting over theories.

  Then, necessary condition on \tC{}.
\end{note}

\begin{note}
  Only motivated \tC{} by intuition.
  Have not argued that this intuition is correct.
  \rotoc{2}.
  Key piece, and intuitive.
\end{note}




% \section[\citeauthor{Carroll:1895uj}]{\citeauthor{Carroll:1895uj}\hfill(Optional)}

% \nocite{Black:1951aa}

% \begin{note}
%   The point here is that with Carroll, generality that goes beyond any single instance.
%   Must apply to all instances, to be valid.
%   But, cannot hope to cover all instances in a single move.
% \end{note}

% \begin{note}
%   A difficulty found on a reading of \citeauthor{Carroll:1895uj}'s \citetitle{Carroll:1895uj}.
% \end{note}

% \begin{note}
%   \begin{quote}
%     ``Plenty of blank leaves, I see!'' the Tortoise cheerily remarked.
%     ``We shall need them \emph{all}!''
%     (Achilles shuddered.)
%     ``Now write as I dictate:---

%     \begin{enumerate}[label=(\emph{\Alph*}), ref=\emph{\Alph*}]
%     \item
%       \label{AatT:a}
%       Things that are equal to the same are equal to each other.
%     \item
%       \label{AatT:b}
%       The two sides of this Triangle are things that are equal to the same.
%     \item
%       \label{AatT:c}
%       If~\ref{AatT:a} and~\ref{AatT:b} are true,~\ref{AatT:z} must be true.
%       \setcounter{enumi}{25}
%     \item
%       \label{AatT:z}
%       The two sides of this Triangle are equal to each other.''
%     \end{enumerate}

%     ``You should call it~\ref{AatT:d}, not~\ref{AatT:z},'' said Achilles.
%     ``It comes \emph{next} to the other three.
%     If you accept~\ref{AatT:a} and~\ref{AatT:b} and~\ref{AatT:c}, you \emph{must} accept~\ref{AatT:z}.''

%     ``And why \emph{must} I?''

%     ``Because it follows \emph{logically} from them.
%     If~\ref{AatT:a} and~\ref{AatT:b} and~\ref{AatT:c} are true,~\ref{AatT:z} \emph{must} be true.
%     You don't dispute \emph{that}, I imagine?''

%     ``If~\ref{AatT:a} and~\ref{AatT:b} and~\ref{AatT:c} are true,~\ref{AatT:z} \emph{must} be true,'' the Tortoise thoughtfully repeated.
%     ``That's \emph{another} Hypothetical, isn't it?
%     And, if I failed to see its truth, I might accept~\ref{AatT:a} and~\ref{AatT:b} and~\ref{AatT:c}, and \emph{still} not accept~\ref{AatT:z}, mightn't I ?''

%     \mbox{}\hfill\(\vdots\)\hfill\mbox{}

%     ``Then Logic would take you by the throat, and force you to do it!''
%     Achilles triumphantly replied.
%     ``Logic would tell you 'You ca'n't help yourself.''%
%     \mbox{ }\hfill\mbox{(\citeyear[279--280]{Carroll:1895uj})}
%   \end{quote}

%   The Tortoise has written down three premises,~\ref{AatT:a},~\ref{AatT:b}, and~\ref{AatT:c}.
%   Achilles holds that~\ref{AatT:z} follows from~\ref{AatT:a},~\ref{AatT:b}, and~\ref{AatT:c}.
%   The Tortoise observes they have the possibility of refraining to accept~\ref{AatT:z} follows from~\ref{AatT:a},~\ref{AatT:b}, and~\ref{AatT:c}.
%   And (initially), the Tortoise does not accept~\ref{AatT:z} follows from~\ref{AatT:a},~\ref{AatT:b}, and~\ref{AatT:c}.
%   Achilles requests the Tortoise accepts that~\ref{AatT:z} follows from~\ref{AatT:a},~\ref{AatT:b}, and~\ref{AatT:c}, and the Tortoise complies.
%   Specifically, the Tortoise grants:

%   \begin{quote}
%     \begin{enumerate}[label=(\emph{\Alph*}), ref=\emph{\Alph*}]
%       \setcounter{enumi}{3}
%     \item
%       \label{AatT:d}
%       If~\ref{AatT:a} and~\ref{AatT:b} and~\ref{AatT:c} are true,~\ref{AatT:z} must be true.%
%       \mbox{ }\hfill\mbox{(\citeyear[279]{Carroll:1895uj})}
%     \end{enumerate}
%   \end{quote}

%   But, does not accept~\ref{AatT:z} follows from~\ref{AatT:a},~\ref{AatT:b},~\ref{AatT:c}, and~\ref{AatT:d}.
% \end{note}

% \begin{note}
%   Modus ponens.

%   \begin{quote}
%     From \(\phi\) and \emph{if} \(\phi\) then \(\psi\), infer \(\psi\).
%   \end{quote}

%   Modus ponens is general.
%   For \emph{any} \(\phi\), \(\psi\).

%   Now, there is a difference between \emph{modus ponens} and conditional.

%   However, take any instance.
%   Then, if \(P\), \(P \rightarrow Q\), \(Q\) must be true.
%   But, then this means that the conditional is true.

%   Consequence of the deduction theorem.

%   Likewise, deduction theorem goes the other way.

%   However, going from \(P\), \(P \rightarrow Q\) to \(Q\) need not be an instance of \emph{modus ponens}.
% \end{note}

% \begin{note}
%   Well, this is a headache.
%   \citeauthor{Carroll:1895uj} is talking about a specific A, B, and Z.
%   There is no clear generality.
% \end{note}

% \begin{note}
%   So, consider at issue is modus ponens.
%   For any specific instance accept, there is a further instance.
%   For, \(A, (A \rightarrow B) \vDash B\).
%   Then, \(\vDash (A \land (A \rightarrow B) \rightarrow B)\).
%   However, now, \(A \land (A \rightarrow B), (A \land (A \rightarrow B) \rightarrow B) \vDash B\).
%   And, so on.

%   The general pattern, get conditional, but then this gives a new instance of modus ponens, which must be true in order for modus ponens to be valid rule of inference.

%   \citeauthor{Carroll:1895uj}, by contrast, starts with \(A \vDash B\).
%   This is different.
%   However, rather than focusing on a single rule of inference, the puzzle turns on what validity amounts to.

%   Validity is a general thing, with specific instances.
%   However, grant any particular instance of validity without employing validity in general.
% \end{note}

% \begin{note}
%   \begin{quote}
%     My paradox \dots turns on the fact that, in a Hypothetical, the \emph{truth} of the Protasis, the \emph{truth} of the Apodosis, and the \emph{validity of the sequence}, are 3 distinct Propositions.

%     \mbox{}\hfill\(\vdots\)\hfill\mbox{}

%     Suppose I say ``I grant~\ref{AatT:a} and~\ref{AatT:b} and~\ref{AatT:c}, but I do \emph{not} grant that I am thereby \emph{obliged} to grant~\ref{AatT:z}.''
%     Surely, my granting~\ref{AatT:z} must \emph{wait} until I have been made to see the validity of this sequence: i.e.\ in order to grant~\ref{AatT:z}, I must grant~\ref{AatT:a},~\ref{AatT:b},~\ref{AatT:c}, and~\ref{AatT:d}! And so on.%
%     \mbox{ }\hfill\mbox{(\citeyear[472]{Carroll:1977wl})}
%   \end{quote}

%   My interpretation of the point \citeauthor{Carroll:1895uj} makes in this passage is that the truth of A B and the truth of C is distinct from the validity of A B C.
%   Granting is substantial, not merely moving.
%   But, in order to grant, this means granting all other cases.

%   So, the paradox is that, on the one hand, don't need validity for any specific true things.
%   But, on the other hand, only of interest if via validity.

%   The Tortoise is slowly working through each instance, but this has no hope of getting the Tortoise to general validity.
%   So, how does the Tortoise ever make it there?
% \end{note}

% \begin{note}
%   This point differs from received interpretation.

%   \citeauthor{Wieland:2013vf} (\citeyear{Wieland:2013vf}) characterises the general understanding of \textcite{Carroll:1895uj} in terms of two lessons:
%   \begin{quote}
%     [T]he negative lesson is that if you add ever more premises to an argument \dots, then you will never demonstrate that its conclusion follows logically.\newline
%     \mbox{ }\hfill\mbox{(\citeyear[984]{Wieland:2013vf})}
%   \end{quote}

%   Parallel, static answers, still option for concluding otherwise.

%   \begin{quote}
%     [T]he positive lesson is that rules of inference, rather than premises of the form `if premises such and such are true, then the conclusion is true', will do the job.\newline
%     \mbox{ }\hfill\mbox{(\citeyear[984]{Wieland:2013vf})}
%   \end{quote}

%   \begin{quote}
%     [\citeauthor{Carroll:1895uj}] simply lacked any distinct conception of a deduction as opposed to the assertion (``granting'' of) a hypothetical proposition.
%     \dots
%     Any attempt by Carroll to tackle the question of inference was bound to begin in confusion and end in constipation-all those premises piling up, but no motion.
%   \end{quote}
% \end{note}

% \paragraph{The Dichotomy}

% \begin{note}
%   Achilles and the Tortoise, Zeno's argument.

%   Surely, right?

%   Two ways to understand.
%   Does the Tortoise move at all, or does the Tortoise arrive at the end?
%   I mean, as formulated by Zeno, it's about catching up, no matter how much one moves.

%   It is different from Zeno's Dichotomy paradox.


%   If so, then we should expect the Tortoise to be making some movement.
%   Adding rules of inference is of no help, because the problem is not movement, it's about how to move so much in a single step.
% \end{note}

% \begin{note}
%   \color{red}
%   Something about logic forcing.
%   The Tortoise hasn't arrived.

%   Nothing hangs on validity.
%   Same issue with testimony.
%   `A'.
%   Why?
%   Testified A, so A.
%   Okay, but another instance of testimony.
%   Testified(Testified A, so A), so Testified A, so A.
% \end{note}

% \begin{note}
%   \begin{quote}
%     But if we who wish to represent his belief in Q as based on P are to write in our notebook everything his having that belief on that basis consists in then when we have written only P and Q we will not have written enough.
%     Someone can believe P and believe Q and still not believe Q on the basis of P whatever the relations between the propositions P and Q happen to be.
%     He might believe Q for some reason completely unconnected with P, or perhaps for no reason at all (if that is possible).%
%     \mbox{ }\hfill\mbox{(\citeyear[185]{Stroud:1979aa})}
%   \end{quote}
%   However, the moral drawn is narrow
%   \begin{quote}
%     The moral is that for every proposition or set of propositions the belief or acceptance of which is involved in someone's believing one proposition on the basis of another there must be something else, not simply a further proposition accepted, that is responsible for the one belief's being based on the other.%
%     \mbox{ }\hfill\mbox{(\citeyear[187]{Stroud:1979aa})}
%   \end{quote}

%   Even if we grant each individual is \ros{}, rather than an instance of the material conditional, \emph{logic} hasn't done anything.
% \end{note}

% \paragraph{General and specific: Contrast}

% \begin{note}
%   Use \citeauthor{Carroll:1895uj} to illustrate this point.

%   However, given the worry, various other things may be understood this way.

%   Hume, constant conjunction.
%   Part of the problem is identifying cause.
%   We get the famous line about observing.
%   However, Hume goes on.
%   It's not only no observation, but no generality.

%   Right, so more narrow than Hume.
%   Because, with Hume, at issue is whether we have grounds for this general thing.
%   With Carroll, it's whether we even really get to the general thing.
% \end{note}

% \begin{note}
%   \begin{quote}
%     Let me ask this: what has the expression of a rule—say a sign-post—got to do with my actions?
%     What sort of connexion is there here?%
%     ---%
%     Well, perhaps this one:
%     I have been trained to react to this sign in a particular way, and now I do so react to it.

%     But that is only to give a causal connexion; to tell how it has come about that we now go by the sign-post; not what this going-by-the sign really consists in.
%     On the contrary; I have further indicated that a person goes by a sign-post only in so far as there exists a regular use of sign-posts, a custom.%
%     \mbox{ }\hfill\mbox{(\citeyear[\S198]{Wittgenstein:1958aa})}
%   \end{quote}

%   Regular use of sign-posts, custom.

%   Ugh, this is ambiguous.
% \end{note}


% %
%   \(^{,}\)
%   \footnote{
%     \citeauthor{Hlobil:2014tq}'s ``Inferential Moorean Phenomenon'':
%   \begin{quote}
%     \begin{enumerate}
%     \item[(IMP)]
%       It is either impossible or seriously irrational to infer \emph{P} from \emph{Q} and to judge, at the same time, that the inference from \emph{Q} to \emph{P} is not a good inference.
%     \end{enumerate}
%     \dots
%     By the ``goodness'' of an inference I mean the feature that makes the relevant inference permissible. Thus, if the inference under consideration is an inductive inference, the relevant kind of goodness is not deductive validity.%
%     \mbox{ }\hfill\mbox{(\citeyear[\S1]{Hlobil:2014tq})}
%   \end{quote}
%   Though, this really isn't more basic given the interest in \tR{}.
%   For, the puzzle is what it is to `infer'.

%   Rationality isn't part of the picture.
%   And, this is a significant drawback of \citeauthor{Hlobil:2014tq}'s approach.
% }


% \subsection{Types and explanation}
% \label{cha:typical:sec:tor:bkgd}

% \begin{note}
%   There is a related, stronger claim, that generality derives from rule following.

%   For this, \citeauthor{Boghossian:2008vf}:

%   \begin{quote}
%     [O]ur internalization of general epistemic rules---like Modus Ponens and Induction---explain and rationalize why we form the beliefs that we form.
%     And that seems intuitively correct.

%     As in the case of our linguistic and conceptual abilities, our ability to form rational beliefs is \emph{productive}: on the basis of finite learning, we are able to form rational beliefs under a potential infinity of novel circumstances.
%     The only plausible explanation for this is that we have, somehow, internalized a rule that tells us, in some general way, what it would be rational to believe under varying epistemic circumstances.%
%     \mbox{ }\hfill\mbox{(\citeyear[483]{Boghossian:2008vf})}
%   \end{quote}

%   Strictly, \citeauthor{Boghossian:2008vf}, rules \textquote{represent our conception of how it would be most rational for a thinker to form beliefs under different epistemic circumstances} (\citeyear[473]{Boghossian:2008vf}).

%   The difference in approach is clearest with \citeauthor{Boghossian:2008vf}'s account of modus ponens:%
%   \footnote{
%     \citeauthor{Boghossian:2008vf} notes the rule is distinct from modus pones as found in textbooks.
%     Remarks: \textquote{It is actually quite mysterious what the logic textbook rule is supposed to be} (\citeyear[472,fn.2]{Boghossian:2008vf})
%     I don't think there is any mystery about the rule in most logic textbooks.
%     Instead, the mystery is the way in which logic relates to reasoning.
%     (Cf.~\cite{Harman:1986ux,MacFarlane:2004aa,Steinberger:2022aa}, etc.)
%     % Issue for the presentation.
%     % Literature is full of issues.
%     % The most well known, Gricean pragmatics.
%     % Though, also McGee, McFarlane, sweet conditonals, the miners paradox, etc.
%   }

%   \begin{quote}
%     (Modus Ponens):
%     If you are rationally permitted to believe both that \emph{p} and that `If \emph{p}, then \emph{q}', then, you are prima facie rationally permitted to believe that \emph{q}.%
%     \mbox{ }\hfill\mbox{(\citeyear[472]{Boghossian:2008vf})}
%   \end{quote}

%   Here, we have permissions.
%   What the agent is allowed to do.
%   However, this is distinct from what the agent does.
% \end{note}

% \begin{note}
%   \tR{} is distinct.
%   Whether came to \emph{q} from \emph{p} , if \emph{p} then \emph{q}.

%   Rationality is not part of our understanding.
%   Rather, generality.%
%   \footnote{
%     Observe, ~\cite{Kolodny:2005aa} is of no interest here.
%     Why be rational is distinct from whether there is some generality.
%   }
% \end{note}

% \begin{note}
%   Likewise, means-end reasoning is distinct from \citeauthor{Broome:2013aa}'s

%   \begin{quote}
%     \emph{End to Means Transmission}.
%     ((\emph{S} requires of \emph{N} that \emph{p}) \& necessarily \newline (\emph{p} \(\supset\) \emph{q}) \& \emph{q} is a means to \emph{p}) \(\supset\) (\emph{S} requires of \emph{N} that \emph{q}).%
%     \mbox{ }\hfill\mbox{(\citeyear[126]{Broome:2013aa})}
%   \end{quote}

%   \emph{S} is some source, such as morality.
%   \emph{N} is a person. (\citeyear[117]{Broome:2013aa})

%   Instead, the significantly weaker idea that the agent has reasoned from some end to a means to that end.
% \end{note}

% \begin{note}
%   On my understanding, this is, in part, the role of \citeauthor{Boghossian:2014aa}'s Taking Condition.

%   Way in which \dots

%   Indeed, \citeauthor{Boghossian:2014aa} highlights how condition allow to draw distinction between deductive and inductive.
%   With taking, get generality.

%   Indeed, \textcite{Boghossian:2014aa} is structured so that Taking is a generalisation of rule.
% \end{note}

% \begin{note}
%   However, \tor{} does not need to amount to a rule.
%   Rather, \tR{} only requires the rough phenomenon that \citeauthor{Boghossian:2008vf} argues rule following is the only plausible explanation of.%
%   \footnote{
%     Our interest with \tor{1} is independent of the worries about rule following raised by~\textcite{Kripke:1982aa}, to the extent that the worries raised by~\citeauthor{Kripke:1982aa} concern \emph{which} rule an agent is following, rather than \emph{whether} the agent is following a rule.
%     At interest is not whether the \tor{} corresponds to plus or quus, but whether the agent's reasoning is of some type.
%   }
% \end{note}

% \begin{note}
%   Same for modus ponens.

%   \citeauthor{Davies:2004aa} discussing~\textcite{Wright:2004aa} with respect to~\citeauthor{Moore:1959aa}'s proof of an external world (\citeyear{Moore:1959aa}):

%   \begin{quote}
%     Moore's argument can be set out as follows:
%     \begin{quote}
%       \begin{enumerate}[label=MOORE (\Roman*), ref=MOORE (\Roman*)]
%       \item
%         \label{MoorePoEW:1}
%         I am having an experience as of one hand [here] and another [here].
%       \item
%         \label{MoorePoEW:2}
%         I have hands.

%         If I have hands then an external world exists.
%       \end{enumerate}

%       Therefore:

%       \begin{enumerate}[label=MOORE (\Roman*), ref=MOORE (\Roman*), resume]
%       \item
%         \label{MoorePoEW:3}
%         An external world exists.
%       \end{enumerate}
%     \end{quote}

%     [\dots] the key question at this point in Wright's account is whether the support for~\ref{MoorePoEW:2} is transmitted to~\ref{MoorePoEW:3} across the modus ponens inference in which the conditional premise is supported by an elementary piece of philosophical theorising.\newline
%     \mbox{ }\hfill\mbox{(\citeyear[215]{Davies:2004aa})}
%   \end{quote}
% \end{note}



\begin{note}
  Or, whether properly based.%
  \footnote{
    \citeauthor{Schaffer:2010vq}'s (\citeyear{Schaffer:2010vq}) Debasing demon.

    The debasing demon \textquote{throws her victims into the belief state on an improper basis, while leaving them with the impression as if they had proceeded properly}. (\citeyear[231]{Schaffer:2010vq})

    (However, see \textcite{Bondy:2018tk} for ways in which the \citeauthor{Schaffer:2010vq}'s demon fails.)
  }
\end{note}

% \begin{note}
%   To \illu{1} with the basic case of modus ponens:

%   \begin{center}
%     \begin{tabular}{R{.45\textwidth} L{.45\textwidth}}
%       \multicolumn{2}{c}{\prop{2}-\val{}-\pool{} pairings in type `by modus ponens'} \\
%       \hline\hline
%       \prop{2}-\val{0} pair & \pool{2} \\
%       \hline
%       \pv{Q}{\valI{True}} & \pv{P}{\valI{True}}, \pv{\propI{If }P\propI{ then }Q}{\valI{True}}, \dots \\
%     \end{tabular}
%   \end{center}

%   \noindent%
%   Where, \(P\), \(Q\) are arbitrary \prop{1}.
%   Likewise, in the following \(R\) is an arbitrary \prop{0}.

%   \begin{center}
%     \begin{tabular}{R{.45\textwidth} L{.45\textwidth}}
%       \multicolumn{2}{c}{\prop{2}-\val{}-\pool{} pairings not in type `by modus ponens'} \\
%       \hline\hline
%       \prop{2}-\val{0} pair & \pool{2} \\
%       \hline
%       \pv{R}{\valI{True}} & \pv{P}{\valI{True}}, \pv{\propI{If }P\propI{ or }Q\propI{ then }R}{\valI{True}}, \dots \\
%       \pv{Q}{\valI{False}} & \pv{P}{\valI{True}}, \pv{\propI{If }P\propI{ then }Q}{\valI{True}}, \dots \\
%     \end{tabular}
%   \end{center}
% \end{note}


%%% Local Variables:
%%% mode: latex
%%% TeX-master: "master"
%%% TeX-engine: luatex
%%% End:


\chapter{Counter-samples}
\label{cha:ces}

\begin{note}
  \autoref{cha:lit}, in addition to intuition, constraint seems to often be a theoretical assumption.

  Purpose of variants is to motivate counterexamples to constraint.
  Specifically in terms of answers to \qWhyV{} which are not answers to \qHowV{}.
  In other words, \ros{} such that \ros{} explains, in part, why agent concludes but is such that the agent does not have a \wit{} for the \ros{}.

  In this section we outline in rough form how we will (attempt) to provide counterexamples.

  In short, need:
  An agent, event in which agent concludes \(\pv{\phi}{v}\) from \(\Phi\), and \ros{} between \(\pv{\psi}{v'}\) and \(\Psi\) such that:

  \begin{itemize}
  \item
    The agent does not have a \wit{} for the \ros{} between \(\pv{\psi}{v'}\) and \(\Psi\).
  \item
    The \ros{} between \(\pv{\psi}{v'}\) and \(\Psi\), in part, answers \qWhyV{}.
  \end{itemize}

  Our goal is motivate a general method for generating examples in which some \ros{} for which an agent does not have a \wit{} such that the \ros{} answers \qWhyV{}.
\end{note}

\section{Nonogram puzzles}
\label{sec:nonogram-puzzles}

\url{https://blog-imgs-144.fc2.com/o/t/o/otokojima/logic_228.htm}

\section{Sudoku puzzles}

\begin{note}
  \begin{illustration}[Sudoku]
    \label{illu:gist:sudoku}
    % https://tex.stackexchange.com/questions/91422/tikz-sudoku-circle-and-connect-with-lines-some-cells
    Consider the following Sudoku puzzle:%
    \footnote{
      From~\textcite[84]{Coussement:2007up}.
    }
    \vspace{\baselineskip}

    \mbox{ }\hfill%
    \begin{adjustbox}{minipage=0.45\linewidth,scale=1}
      \centering
      \begin{tikzpicture}[scale=.5]
        \begin{scope}
          \draw (0, 0) grid (9, 9);
          \draw[very thick, scale=3] (0, 0) grid (3, 3);
          \setcounter{row}{1}
          % Single entries
          \setrow { }{ }{ }  { }{ }{ }  {1}{ }{ }
          \setrow { }{ }{ }  { }{ }{ }  { }{5}{ }
          \setrow {9}{ }{ }  { }{ }{ }  { }{ }{2}
          \setrow { }{ }{3}  { }{2}{ }  { }{ }{ }
          \setrow { }{ }{ }  {8}{ }{ }  {4}{6}{5}
          \setrow { }{4}{ }  { }{5}{9}  { }{ }{8}
          \setrow { }{8}{7}  {2}{3}{1}  { }{4}{6}
          \setrow {2}{1}{ }  {5}{ }{ }  { }{ }{3}
          \setrow {3}{ }{6}  {4}{ }{8}  { }{ }{ }
        \end{scope}
      \end{tikzpicture}
    \end{adjustbox}%
    \hfill\mbox{ }
  \end{illustration}

  Interactive.
  Fill in the grid.
  Difference between filling in the grid and concluding that solution to the puzzle.
  So, before conclude for any particular square, or for the grid as a whole.
  Is it the case that you would fill in the grid the same way?

  Key intuition, stop before committing to any mistake.

  Two aspects.

  First, may give up completely.
  Second, catch any mistakes and fix before moving on.
\end{note}

\begin{note}
  \autoref{illu:gist:sudoku} parallels \autoref{scen:squish}.

  In both \illu{1}, \(\pv{\phi}{v}\) follows from \(\Phi\) via a rules.

  However, rules are not of direct interest.
  \autoref{scen:squish} is a syntactic proof, but variant \scen{0} in which semantic proof.
  Relevant reasoning may be rule governed, but semantic proofs are not constrained.

  Rather, familiarity.

  The type of reasoning is general.
  Syntactic and semantic proofs, Sudoku puzzles, simple instances of chess problems, all seem to involve general reasoning.
  Likewise, counting, adding, subtracting, and so on.
  Competence established through various proofs, puzzles, problems, and practice.

  In this respect, there is no reasonable doubt of \tR{}.
  At issue is whether specific performance.
\end{note}

\begin{note}
  Same as \autoref{prop:requ-not-n-ce} on \autopageref{prop:requ-not-n-ce}.
  Attention only wrt.\ to reasoning.
  Hence, \wit{}.
\end{note}

\begin{note}
  Intuition for each of the points.

  \begin{itemize}
  \item
    \fc{1}, understand how to solve Sudoku puzzles, repeat any instances.
  \item
    Catch mistakes.
  \end{itemize}
\end{note}

\section{\scen{3}}
\label{sec:cscen1}


\begin{note}
  \begin{illustration}
    Semantics for squish.
  \end{illustration}
\end{note}



\begin{note}[Prior to concluding\dots]
  Not particularly marked.
  Allow agent to have built up a bunch of stuff while reasoning.

  Example.

  \begin{illustration}[Velocity]
    \label{ill:velocity}
    Agent is provided with information about how far a car has travelled north as a function of time travelled.

    From this, take the derivative of the function to obtain the (instantaneous) velocity of the car at a handful of points in time.

    And, from the (instantaneous) velocity of the car, the agent calculates the (instantaneous) acceleration of the car at each of the points in time.

    The agent also has information about the speed of the car as a function of time travelled, and the agent may calculate speed by the taking magnitude of the (instantaneous) velocity of the car.
  \end{illustration}

  \autoref{ill:velocity}, two step calculation.
  Velocity, acceleration.
  After the first step, check by taking the magnitude.
  Calculation of velocity is correct only if taking the magnitude matches speed.

  So, two events for which the agent is concluding.
  Distinct \requ{1} associated with each event.
\end{note}

\section{\tR{2} and \wit{1}}
\label{sec:sr2}

\begin{note}
  Plausible that \wit{} for each component of the type of reasoning!

  But, what to make of this?
\end{note}

\section{Point}
\label{sec:point}

\begin{note}
  Now, the upshot here is perhaps quite minor.

  \issueConstraint{} still does work to see which \ros{} the agent has a \wit{} for.

  The point is more-or-less this.
  \issueConstraint{} does this work, but it does not do any additional work.
\end{note}

%%% Local Variables:
%%% mode: latex
%%% TeX-master: "master"
%%% End:


\appendix

\part*{Appendices}


\chapter{Conclusions, concludes, concluding}
\label{cha:clar}

\begin{note}
  This chapter details the way we understand: 
  Conclusions, an \eiw{0} an agent concludes, and an \eiw{0} an agent is concluding.
\end{note}


\begin{note}
  There are three sections to this chapter:

  The first section beings with definitions of \prop{1}, \val{1}, \eval{1}, and \pool{1} and then defines \eiw{1} an agent concludes, conclusions, and \eiw{1} an agent is concluding.

  The second section details a few assumptions about the defined terms.

  The final section applies the terms defined to some theories from the literature.
\end{note}



\section{Definitions}
\label{cha:clar:sec:Cons}


\subsection{\prop{3}, \val{1}, \evalN{1}, and \pool{1}}
\label{cha:clar:sec:Cons:pvp}

\begin{note}
  We being by defining \prop{1} and \val{1}.
  With definitions of \prop{1} and \val{1} in hand we define \evalN{1}.
  The section closes by defining conclusions.
\end{note}

\paragraph*{\prop{3} and \val{1}}

\begin{note}
  \begin{definition}[\prop{3} and \val{1}]
    \label{def:prop-val}
    \mbox{ }%
    \vspace{-\baselineskip}
    \begin{enumerate}[noitemsep, label=]
    \item
      \begin{itemize}
      \item
        A \emph{\prop{0}} is some way things may be.
      \item
        A \emph{\val{0}} is an \agpe{} on a state of affairs.
      \end{itemize}
    \end{enumerate}
    \vspace{-\baselineskip}
  \end{definition}

  \noindent%
  \prop{3} are ways things may be, whether actual, possible, true ideal, regrettable, etc.
  \val{3} capture an \agpe{} on the state of affairs, whether \emph{being} actual, possible, true, ideal, or regrettable, etc.
\end{note}

\paragraph*{\evalN{3}}

\begin{note}
  \evalN{3} capture way things are for an agent in terms of \prop{0}-\val{0} pairs.

    \begin{definition}[\evalN{3}]
      \label{def:evals}
      \mbox{ }
      \vspace{-\baselineskip}
      \begin{itemize}
      \item
        An \emph{\evalN{}} is a \prop{0}-\val{0} pair \(\pv{\phi}{v}\) such that:
        \begin{itemize}
        \item
          \(\phi\) having \val{0} \(v\) captures a way things are from an \agpe{}.
        \end{itemize}
      \end{itemize}
      \vspace{-\baselineskip}
    \end{definition}

    \noindent%
    Here are a few examples of a conclusion expressed in natural language on the left with a corresponding \prop{0}-\val{0} pair on the right:%
    \footnote{
      We do not introduce notation to distinguish an arbitrary \prop{0}-\val{0} pair \(\pv{\phi}{v}\) from an \eval{} that \(\phi\) has \val{0} \(v\) by an agent.
      Context is sufficient to resolve any ambiguity.
    }

    \begin{enumerate}[label=\arabic*., ref=(\arabic*), noitemsep, series=propValExC]
    \item
      The bag is heavy.%
      \hfill%
      \pv{\propI{The bag is heavy}}{\valI{True}}
    \item
      The room isn't full.%
      \hfill%
      \pv{\propI{The room is full}}{\valI{False}}
    \item
      \label{pvEx:bC}
      I like bitter coffee.%
      \hfill%
      \pv{\propI{Bitter coffee}}{\valI{Desirable}}
    \item
      I should go to work.%
      \hfill%
      \pv{\propI{I go to work}}{\valI{Ought}}
    \item
      There's no chance they find the needle.%
      \hfill%
      \pv{\propI{The needle is found}}{\valI{Unlikely}}
    \item
      It's too bad the balloon deflated.%
      \hfill%
      \pv{\propI{The balloon is deflated}}{\valI{Unfortunate}}
    \end{enumerate}

    \noindent%
    Nothing of importance rests on the \val{1} chosen.
    % Indeed, the relevant \val{1} may be restricted to \valI{True} and \valI{False}.
    For example, \ref{pvEx:bC} may be recast as \pv{\propI{I like bitter coffee}}{\valI{True}}.
    Still, distinct \val{1} allows us to capture different \evals{0} of the same \prop{0}.
    E.g.: \pv{\propI{The coffee is bitter}}{\valI{True}} \emph{and} \pv{\propI{The coffee is bitter}}{\valI{Desirable}}.
    % 
    % \begin{enumerate}[label=\arabic*., ref=(\arabic*), noitemsep, resume*=propValExC]
    % \item I like how bitter the coffee is.%
    %   \hfill%
    %   \pv{\propI{The coffee is bitter}}{\valI{True}}\newline
    %   \hfill%
    %   \pv{\propI{The coffee is bitter}}{\valI{Desirable}}
    % \end{enumerate}
    % % 
    In this respect, \evalN{} are similar to propositional attitudes.
    Expressed in folk psychological terms, I both \emph{believe} \propI{The coffee is bitter} and \emph{desire} \propI{The coffee is bitter}.

    Still, some \evalN{1} are more flexible than propositional attitudes.
    For example, \pv{\propI{The coffee is bitter}}{\valI{True}} does not distinguish between whether the agent \emph{believes} or \emph{knows} \propI{The coffee is bitter}.
    If you like, you may substitute coarse \val{1} from finer-grained values such as \valI{believed} and \valI{known}.
    For our purposes, this is an upshot.
    Our interest is with an \agpe{} quite generally, rather than specific aspects of an \agpe{}.
  \end{note}

  \begin{note}
    \nocite{Scriven:1962vq}%
    \nocite{Woodward:2021ue}%
    \nocite{Perry:1979vc}%
    \nocite{Perry:1986aa}%
    \nocite{Collins:1997wn}%
    To clarify what is meant by `an \agpe{}' in a little more detail, consider \autoref{fig:zollner-illusion}, an instance of the \citeauthor{Zollner:1860vx} illusion.

    \begin{figure}[!h]
      \centering
      \def\svgwidth{\columnwidth}
      \input{ZIOh.pdf_tex}
      \caption{An instance of the \citeauthor{Zollner:1860vx} illusion --- crop of ~\textcite{Fibonacci:2007vj}}
      \label{fig:zollner-illusion}
    \end{figure}

    \noindent%
    The \citeauthor{Zollner:1860vx} illusion contains eight long black lines crossed with short black lines.
    If the illusion has been recreated, the long lines are not \emph{seen} as parallel.
    However, on closer inspection (perhaps with the aid of the ruler), you may establish the long lines are parallel.

    So, on first pass, or without further investigation, \propI{The long lines are parallel} is \evaled{} \valI{False} from \agpe{your}.
    However, if you become convinced the lines are parallel, then from \agpe{your}, \propI{The long lines are parallel} is re-\evaled{} \valI{True}.

    In short, an \agpe{} is about the way things are, rather than the way things visually appear, etc.%
    \footnote{
      Consider also Cypher's thoughts while eating steak. (\cite[330--331]{Wachowski:2000uh})
    }
    \phantlabel{mention:concluding-non-factive}
    In this respect, some \prop{0} is \evaled{0} with some \val{0}, though things are otherwise.
    \evalN{3} of the \citeauthor{Zollner:1860vx} illusion may shift, and \propI{The stuff is gin} may be \evaled{} \valI{True} though the stuff is petrol  (\cite[cf.][18]{Williams:1979wi}).
    % Hence, \evalN{1} capture a way the things are from an \agpe{}, \evalN{1} are distinct from \prop{0}-\val{0} pairs \emph{given various assumptions}.
  \end{note}

  % \begin{note}
  %   When we speak of an agent's \evalN{} of \propI{The long lines are not parallel} as \valI{True}, then the longs lines are not parallel, from the \agpe{}.
  %   This is different from concluding \propI{The long lines are not parallel} as \valI{True} under the assumption that one's vision is factive.
  % \end{note}

  % \begin{note}
    %
    % \footnote{
    % For a slightly more involved case, an agent may \eval{} \propI{\(0.999\dots \ne 1\)} as \valI{True} while holding themselves to have a conventional understanding of real numbers.
    % (There are various interpretations under which \(0.999\dots \ne 1\), but it's convention the property holds for the reals.)

    % For, the agent may have failed to grasp the Archimedean property does not hold for real numbers, and may think that, though \(0.999\dots\) approaches \(1\), there must be \emph{some} difference between \(0.999\dots\) and \(1\) --- no matter how small --- and so they are not equal.
    % }
  % \end{note}



\paragraph*{\pool{3}}


\begin{note}
  With a definition of \evalN{1} in hand, we define \pool{1} as follows:

  \begin{definition}[\pool{3}]
    \label{def:pools}
    \mbox{ }
    \vspace{-\baselineskip}
    \begin{itemize}
    \item
      A \emph{\pool{}} is a (maybe empty) collection of \evalN{1}.
    \end{itemize}
    \vspace{-\baselineskip}
  \end{definition}

  \noindent%
  The role of a \pool{} is to keep track of which \evalN{1} an agent concludes from.

  In many cases, a \pool{}, or a collection of \evalN{} which form the basis of a \pool{} are clearly associated with a conclusion.
\end{note}



\subsection{Concludes, conclusion, concluding}
\label{sec:concl-events-which}


\begin{note}
  An \eiw{0} an agent concludes is an \eiw{0} the agent \evals{} some \prop{0} to have some \val{0}:

  \begin{definition}[\eiw{3} an agent concludes]
    \label{def:conclusionE}
    \mbox{ }
    \vspace{-\baselineskip}
    \begin{itemize}
    \item
      \(\ed{}\) is an \eiw{0} \vAgent{} \emph{concludes} \(\pv{\phi}{v}\).
    \end{itemize}
    %
    \emph{If and only if}:
    %
    \begin{itemize}
    \item
      \(\ed{}\) includes an \eiw{0} \vAgent{} \evals{} \(\phi\) to have \val{} \(v\).
    \end{itemize}
    \vspace{-\baselineskip}
  \end{definition}
\end{note}


\begin{note}
  In turn, conclusions are understood in terms of \eiw{1} an agent concludes:

  \begin{definition}[Conclusions]
    \label{assu:concluding:pvp}
    \mbox{ }
    \vspace{-\baselineskip}
    \begin{itemize}
    \item
      A \emph{conclusion} that \(\phi\) has \val{} \(v\) is an
      % {\color{green} (fresh)}
      \evalN{} of \(\phi\) having \val{} \(v\) from an \eiw{0} an agent concludes \(\pv{\phi}{v}\).%
    \end{itemize}%
    \vspace{-\baselineskip}%
  \end{definition}
  % \noindent%
  % As \evalN{1}, \prop{1}, and \val{1} are abstractions, we do not identify conclusions with \evals{} of \prop{1} having \val{1}.
  % Rather, we identify conclusions with the applicability of these abstractions.
  %
  % We have no interest in what conclusions beyond \evals{} of \prop{0}-\val{0} pairs from an \agpe{}.
  %
  % For simplicity we limit our interest to `\emph{fresh}' \evals{}.
  % Where `fresh' identifies an \eval{} that did not hold of the agent prior to the conclusion.
  %
  % For example, if an agent steps outside without looking at the weather forecast and notices it is raining, then \propI{It is raining} is freshly \evaled{} \valI{True} from the \agpe{}.
  % However, if the agent had looked at the weather forecast prior to stepping outside (and trusted the forecast), that \propI{It is raining} is \evaled{} \valI{True} may come to mind, but is not a fresh \evalN{}.
  %
  % This limitation has no direct role in the arguments which follow,
\end{note}


\begin{note}
  Our interest is with understanding the way an \eiw{0} some agent concludes some \prop{0} has some \val{0} happens.
  The way we develop this understanding involves events such that an \eiw{0} the agent concludes is \emph{in progress} --- the agent is `concluding'.

  \begin{definition}[\eiw{3} an agent is concluding]
    \label{def:conclusionE}
    \mbox{ }
    \vspace{-\baselineskip}
    \begin{itemize}
    \item
      \(\ed{}\) is an \eiw{0} an agent is \emph{concluding} \(\pv{\phi}{v}\) if \(\ed{}\) is an event such that an \eiw[\(\ed{\prime}\)]{0} the agent concludes \(\pv{\phi}{v}\) is in progress.
    \end{itemize}
    \vspace{-\baselineskip}
  \end{definition}

  \noindent%
  In short, our use of the term \textquote{concluding} always captures an event in progress.

  Note, if an agent is concluding and then concludes, we do not assume the conclusion partly happened as a result of concluding.
  The role of \se{} (\autoref{def:se}, \autopageref{def:se}) is to identify such cases.
  (See, in particular, \autoref{obs:se-need-hCon} on \autopageref{obs:se-need-hCon}.)
\end{note}


% \begin{note}
%   \assuPP{2} entails there is some possible \eiw{0} the agent concludes:

%   \begin{proposition}[Possible \eiw{1} an agent concludes]%
%     \label{prp:peventC}%
%     \vspace{-\baselineskip}
%     \begin{itenum}
%     \item[\emph{If}:]
%       \(\ed{}\) is an \eiw{0} \vAgent{} is concluding \(\pv{\phi}{v}\) from \(\Phi\).
%     \item[\emph{Then}:]
%       There is a possible \eiw[\(\ed{\prime}\)]{0} \vAgent{} concludes \(\pv{\phi}{v}\) from \(\Phi\).
%     \end{itenum}
%     \vspace{-\baselineskip}
%   \end{proposition}

%   \begin{argument}{prp:peventC}
%     Immediate by \assuPP{}~(\autoref{assu:PP}).
%   \end{argument}

%   \noindent%
%   Though, the quasi-converse of \assuPP{} does not necessarily hold:

%   \begin{observation}[Conclusions without concluding]%
%     \label{obs:cds-arb}%
%     The following conditional is not necessarily true:
%     \begin{itenum}
%     \item[\emph{If}:]
%       \(\ed{}\) is an \eiw{0} \vAgent{} concludes \(\pv{\phi}{v}\) from \(\Phi\).
%     \item[\emph{Then}:]
%       There is some \se{0} \(\ed{\flat}\) of \(\ed{}\) where \vAgent{} is concluding \(\pv{\phi}{v}\) from \(\Phi\).
%     \end{itenum}
%     \vspace{-\baselineskip}
%   \end{observation}

%   \begin{motivation}{obs:cds-arb}
%     {\color{green} motivated already by def of \se{0}}

%     There are (at least) two additional ways the conditional may fail:

%     \begin{itemize}
%     \item
%       There may have been no `inertia' with respect to the events prior to the \eiw{0} the agent concludes.

%       For example, there are exactly five intermediate logics that have the interpolation property (\cite[cf.][]{Maksimova:1977un}).
%       However, \citeauthor{Maksimova:1977un} may have been ready to give up on the proof at any point.%
%       \footnote{%
%         Though, admittedly, it is unlikely \citeauthor{Maksimova:1977un} would have given up.
%       }
%     \item
%       It may fail to be the case that the \emph{agent} is concluding.

%       For example, consider being guided through a complex argument.
%       (E.g. \citeauthor{Maksimova:1977un}'s proof.)
%       At each step, the guide signposts the way to the conclusion.
%       However, without the guide the agent would fail to reach the conclusion.
%     \end{itemize}
%     \vspace{-\baselineskip}
%   \end{motivation}
% \end{note}



\section{Assumptions}
\label{sec:assumptions}

\begin{note}
  Basic definitions in hand, assumptions.
\end{note}

\subsection{Conclusions from \pool{1}}
\label{sec:pools-premises}

\begin{note}
  \begin{assumption}[Conclusions from \pool{3}]
    \label{assu:concluding:pools}
    \vspace{-\baselineskip}
    \begin{itenum}
    \item[\emph{If}:]
      \(\pv{\phi}{v}\) is a conclusion by \vAgent{}.
    \item[\emph{Then}:]
      \(\pv{\phi}{v}\) is a conclusion by \vAgent{} from some \pool{} \(\Phi\).
    \end{itenum}
    \vspace{-\baselineskip}
  \end{assumption}

  \noindent%
  Here are a few quick examples:
  %
  \begin{enumerate}[label=\arabic*., ref=(\arabic*), noitemsep]
  \item
    \emph{A} testified that \(\phi\) is true, so \(\phi\) is true.
  \item
    \(\phi\) would be nice, so \(\phi\) ought to be the case.
  \item
    The song is produced by \emph{B}, so I want to listen to it.
  \item
    The device reads `\(\phi\)' and is reliable, so \emph{not}-\(\phi\) is unlikely.
  \end{enumerate}
  %
  Each sentence highlights a conclusion, marked by `so', and the conclusion \fof{0} from some collection of \evalN{1} --- some \pool{}.

  I doubt each sentence captures the full extent of the premises involved.
  However, we have no interest in specifying exactly which \evalN{} belong to any given \pool{}.
\end{note}

\begin{note}
  Note, we allow \pool{1} to be empty, and so \autoref{assu:concluding:pools} does not rule out cases where an agent makes a conclusion independent of any prior \evalN{1}.%
  \footnote{
    To motivate, consider the parallel between the~\citeauthor{Ramsey:1929tf} test for conditionals and a Fitch-style rule for conditional introduction in propositional logic.
    \textcite{Read:1995wf} describe the test as follows:
    %
    \begin{quote}
      One should believe a conditional. `if \emph{A} then \emph{B}' if one would come to believe \emph{B} if one were to add A to one's stock of beliefs.%
      \mbox{ }\hfill\mbox{(\citeyear[47]{Read:1995wf})}
    \end{quote}
    %
    A Fitch-style rule for conditional introduction in propositional logic is as follows on the left, which an instance on the right (cf. \cite[206]{Barwise:1999tu}, \cite{Pelletier:2021vp}):

    \begin{center}
      \begin{fitch}
        \ftag{\text{\scriptsize \emph{i}}}{\fa \fh P} & \\
        \ftag{\text{\scriptsize }}{\fa \fa \vdots} & \\
        \ftag{\text{\scriptsize \emph{j}}}{\fa \fa Q} & \\
        \ftag{\text{\scriptsize \emph{j+1}}}{\fa P \rightarrow Q} & \(\rightarrow\)\textbf{Intro:}\emph{i}--\emph{j} \\
      \end{fitch}%
      \hfil%
      \begin{fitch}
        \ftag{\text{\scriptsize 1}}{\fa \fh P \land Q} & \\
        \ftag{\text{\scriptsize 2}}{\fa \fa Q} & \(\land\)\textbf{Elim:}\emph{1} \\
        \ftag{\text{\scriptsize 3}}{\fa (P \land Q) \rightarrow Q} & \(\rightarrow\)\textbf{Intro:}\emph{1}--\emph{2} \\
      \end{fitch}
    \end{center}

    \noindent%
    The rule states that at any point in a proof, one may assume \(P\), then, after deriving \(Q\) from the assumption of \(P\), one may discharge the assumption of \(P\) and introduce \(P \rightarrow Q\).
    Note, the assumption of \(P\) on line \emph{i} corresponds to adding \(P\) to a collection of \prop{1} proved or assumed up to line \emph{i}, and hence to one's stock of beliefs.

    Now, both the \citeauthor{Ramsey:1929tf} test and the Fitch-style rule for conditional introduction describe a clear \emph{processes} for which a conditional is a conclusion, but there may be no \emph{premises} associated with the conclusion.
    Observe, the instance derivation does not involve any premises.
  }
\end{note}

\subsection{Reasoning}
\label{cha:clar:sec:Concls:reasoning}

\begin{note}
  We assume conclusions are the result of reasoning:

  \begin{assumption}[Conclusions by reasoning]
    \label{assu:ConRea}
    \vspace{-\baselineskip}
    \begin{itenum}
    \item[\emph{If}:]
      Clauses \ref{assu:ConRea:event} and \ref{assu:ConRea:prog} hold:
      \begin{enumerate}[label=\Alph*., ref=\Alph*]
      \item
        \label{assu:ConRea:event}
        \(\ed{}\) is an \eiw{0} \vAgent{} concludes \(\pv{\phi}{v}\) from \(\Phi\).
      \item
        \label{assu:ConRea:prog}
        \(\ed{}\) may be partitioned into a collection of \se{} \(\ed{1}, \dots, \ed{k}\) such that for each \(\ed{i}\), \vAgent{} is concluding \(\pv{\phi}{v}\) from \(\Phi\).
      \end{enumerate}
    \item[\emph{Then}:]
      \(\ed{}\) is an \eiw{0} \vAgent{} \emph{reasons} to \(\pv{\phi}{v}\) from \(\Phi\).
    \end{itenum}
    \vspace{-\baselineskip}
  \end{assumption}

  \noindent%
  In short, an \eiw{0} an agent concludes is particular case of an \eiw{0} an agent reasons, so long as the conclusion was no immediate.

  For example, an agent may conclude \(\pv{\psi}{\valI{True}}\) from a \pool{} containing \pv{\(\phi\)}{\valI{True}} and \pv{\propI{If }\(\phi\)\propI{ then }\(\psi\)}{\valI{True}} by reasoning via \emph{modus ponens}.
  Given~\autoref{assu:ConRea}, the \eiw{0} the agent concludes includes the agent's use of \emph{modus ponens}.

  However, an agent may conclude \pv{\propI{The room is red}}{\valI{True}} due to visual stimuli.
  In this case, it is plausible no reasoning takes place, and Clause~\ref{assu:ConRea:prog} means we do not assume reasoning took place --- though see (\cite{Siegel:2017aa}) for counterpoint.
\end{note}


\begin{note}
  In short, \autoref{assu:ConRea} fixes the \emph{scope} of an \eiw{0} an agent concludes to span the \eiw{0} the agent reasons to the conclusion \(\phi\) has \val{0} \(v\) from the \pool{} \(\Phi\).
  For, the event includes the reasoning leading up to some conclusion and does not focus only on a more-or-less instantaneous \eiw{0} the agent makes the relevant conclusion.

    Hence, when discussing an \eiw{0} an agent concludes, we may talk in terms of the agent's reasoning, where natural.
\end{note}


\begin{note}
  To illustrate different perspectives on the scope of reasoning we contrast to instances of the term `concludes'.

  The first instance is from \citeauthor{Gardner:1986wp}'s discussion of Newcomb's problem:
  %
  \begin{quote}
    A large number of those who recommended taking only the second box performed the expected-value calculation and concluded that, provided the probability that the Being was correct was at least .5005, they would take only the second box.%
    \mbox{ }\hfill\mbox{(\citeyear[166]{Gardner:1986wp})}
  \end{quote}
  %
  The second instance is from \citeauthor{Bratman:1979aa}'s discussion of temptation:
  %
  \begin{quote}
    Sam thinks both that in certain respects his drinking would be, \emph{prima facie}, best, and that in certain other respects his abstaining would be, \emph{prima facie}, best.
    Weighing these conflicting considerations he concludes that it would be best to abstain, rather than drink.%
    \mbox{ }\hfill\mbox{(\citeyear[156]{Bratman:1979aa})}
  \end{quote}
  %
  Both passages capture a similar process.
  From some \agpe{}, various states of affairs are evaluated in different ways, and after some thought the agent chooses to do something.

  The contrast between the two passages in the event captured by `concludes'.
  \citeauthor{Gardner:1986wp} presents the scenario as a sequence;
  Two verbs (`performs', `concludes') are linked by `and'.
  \citeauthor{Bratman:1979aa}, modifies a verb (`concludes') by a different verb (`weighs').
  On a natural reading, the \eiw{0} the agent concludes as described by \citeauthor{Gardner:1986wp} is distinct from the \eiw{0} the agent performs the expected-value calculation.
  And, by contrast, \citeauthor{Bratman:1979aa} describes a single \eiw{0} the agent concludes by weighing conflicting considerations.

  \autoref{assu:ConRea} is compatible with our reading of \citeauthor{Bratman:1979aa}'s description and incompatible with our reading of \citeauthor{Gardner:1986wp}'s description.
  % However, if the agent concluded they would take only the second box from whatever premises are associated with the expected-value calculation given \citeauthor{Gardner:1986wp}'s description, then the event \citeauthor{Gardner:1986wp} identifies with `concludes' is too narrow.
  % Following \citeauthor{Bratman:1979aa}, we may recast \citeauthor{Gardner:1986wp}'s description as \textquote{Performing these expected-value calculations, they concluded  they would take only the second box}.%
  % \footnote{
  %   In addition, it seems to me that whether or not one defaults to thinking of a more-or-less instantaneous event or an inclusive event comes apart in multi-agent cases.

  % For example, consider Sam and Max working on some problem \(p\) together.
  % Sam and Max trade ideas back and forth, and settle on an answer \(a\).

  % Consider the events described by the following sentences:
  % \begin{enumerate}[label=\arabic*., ref=(\arabic*), noitemsep]
  % \item
  %   \label{Cing:SandM:j}
  %   Sam and Max (jointly) conclude \(a\) is an answer to problem \(p\).
  % \item
  %   \label{Cing:SandM:s}
  %   Sam concludes \(a\) is an answer to problem \(p\).
  % \item
  %   \label{Cing:SandM:m}
  %   Max concludes \(a\) is an answer to problem \(p\).
  % \end{enumerate}

  % My default is to consider the events captured by \ref{Cing:SandM:s} and \ref{Cing:SandM:m} in line with \citeauthor{Gardner:1986wp}.
  % When considering Sam or Max in isolation, the \eiw{0} Sam (or Max) concludes more-or-less instantaneous event.
  % Though, the relevant events may be expanded to include the trading of ideas by appending `by trading ideas with Max/Sam'.

  % By contrast, I read \ref{Cing:SandM:j} in line with \citeauthor{Bratman:1979aa}.
  % When considering Sam and Max as a pair, in which a non-distributive reading of `conclude' is forced by the parenthetical `jointly', the event includes the trading of ideas back and forth.
  % Though, the relevant event includes \se{0} where Sam and Max separately evaluate that \(a\) is an answer to problem \(p\).
  % }
\end{note}

\begin{note}
  We do not place any constraints on what reasoning is, nor does any argument depend on the absence of constraints on what reasoning is.
  Hence, the arguments to follow should be compatible with whatever you think reasoning is.%
  \footnote{
    Consider in particular \citeauthor{Broome:2013aa}'s discussion of reasoning from \pv{\propI{It is raining}}{\valI{True}} and \pv{\propI{If it is raining the snow will melt}}{\valI{True}} to \pv{\propI{I hear trumpets}}{\valI{True}}.
    The conclusion does not follow by something like \emph{modus pones}.
    However, \citeauthor{Broome:2013aa} grants the possibility of a distinct rule which allows the agent to move from the premises to the conclusion, and holds that \emph{if} the agent is following the rule, the agent is reasoning.
    (\citeyear[233]{Broome:2013aa})
    The way we understand reasoning is compatible with this.
  }%
  % \(^{,}\)%
  % \footnote{
  %   For contrast, \citeauthor{Wedgwood:2006ui} assumes reasoning \textquote{is the process of \emph{revising ones beliefs or intentions, for a reason}} (\citeyear[600]{Wedgwood:2006ui}).

  %  \begin{quote}
  %   If a set of antecedent mental states makes it rational for one to form a new belief or intention, then those antecedent mental states are surely of a suitable type and content so that it is \emph{intelligible} that they could represent one's reason for forming that belief or intention.\newline
  %   \mbox{ }\hfill\mbox{(\citeyear[662]{Wedgwood:2006ui})}
  % \end{quote}

  %  \citeauthor{Wedgwood:2006ui} provides the following contrasts to clarify their understanding of intelligibility:

  % \begin{quote}
  %   [T]he belief that the \emph{Oxford Dictionary of National Biography} says that Hume died in 1776 seems a mental state of a suitable type and content so that it could intelligibly represent one's reason for believing that Hume died in 1776.
  %   On the other hand, it is not (except in the presence of some rather extraordinary background beliefs) a mental state of a suitable type and content so that it could intelligibly represent one's reason for believing that every even number is the sum of two primes.%
  %   \mbox{ }\hfill\mbox{(\citeyear[662]{Wedgwood:2006ui})}
  % \end{quote}

  % Strictly, \citeauthor{Wedgwood:2006ui} understanding reasoning in terms of dispositions that respond to what `rationalizes' (\citeyear[672]{Wedgwood:2006ui}) and \citeauthor{Wedgwood:2006ui} considers `rationalizes' and  `makes intelligible' as equivalent.
  %   Therefore, the following conditional, states a sufficient rather than necessary condition which leads to \citeauthor{Wedgwood:2006ui}'s assumption, rather than being an expansion of what \citeauthor{Wedgwood:2006ui}'s assumption amounts to.
  %   I find this particularly confusing.
  % }
\end{note}

% \begin{note}
%   \begin{observation}[Bounds]%
%     \label{obs:newPVp-newE}%
%     Suppose an agent concludes \(\pv{\phi}{v}\).
%     The agent beings reasoning from a \pool{} of premises \(\Phi\).
%     However, part way through the agent comes to \eval{} \(\psi\) to have \val{0} \(v'\) independent from \(\Phi\) and appeals to \(\pv{\psi}{v'}\) to conclude \(\pv{\phi}{v}\).

%     The \eiw{0} the agent concludes \emph{includes} but is \emph{not limited to} starting with the agent \evaling{} \(\psi\) with \(v'\).
%   \end{observation}

%   \begin{motivation}{obs:newPVp-newE}%
%     By stipulation, the agent concludes \(\pv{\phi}{v}\) from \(\Phi\) together with \(\pv{\psi}{v'}\).
%     So, from \autoref{assu:ConRea} it follows the \eiw{0} the agent concludes includes the \evalion{} \(\psi\) has \val{0} \(v'\).
%     However, \autoref{assu:ConRea} does not limit the \eiw{0} the agent concludes to the \eiw{0} the agent reasons.
%   \end{motivation}

%   We tacitly assume the \eiw{0} an agent concludes \(\pv{\phi}{v}\) from \(\Phi\) is the minimal \eiw{0} the agent reasons to \(\pv{\phi}{v}\) from \(\Phi\).
%   However, nothing in particular hangs on this.
% \end{note}



\section{Applications}
\label{sec:compatibility}


\begin{note}
  The previous sections defined \prop{1}, \val{1}, \eval{1}, \pool{1}, \eiw{1} an agent concludes, conclusions, and \eiw{1} an agent is concluding.
  And, a handful of assumptions were made.

  Our interest is with \issueInclusion{}, understood as a pre-theoretical constraint on understanding the way an \eiw{} an agent concludes happens.
  So, the definitions and assumptions are intended to be fairly neutral.
  To highlight the intended neutrality, this section adapts the definitions to a handful of accounts of the phenomenon the definitions are designed to capture from the literature.
\end{note}


\begin{note}
  First up is \citeauthor{Armstrong:1968vh}'s (\citeyear{Armstrong:1968vh}) causal account of inferring:%
  \footnote{
    \citeauthor{Armstrong:1968vh} traces the causal account of inference to ~\citeauthor{Moore:1962up} and Hume.
    In contrast, \citeauthor{Frege:1979aa} focuses on justification: \textquote{To make a judgment because we are cognisant of other truths as providing a justification for it is known as \emph{inferring}} (\citeyear{Frege:1979aa}).
  }
  %
  \begin{quote}
    [\dots] to say that A infers \emph{p} from \emph{q} is simply to say that A's believing \emph{q} \emph{causes} him to acquire the belief \emph{p}.
    And the sense of `cause' employed here is the common or billiardball sense of `cause', whatever that sense is.\newline
    \mbox{}\hfill\mbox{(\citeyear[194]{Armstrong:1968vh})}
  \end{quote}
  %
  What \citeauthor{Armstrong:1968vh} identifies as an inference falls under our understanding of a conclusion.%
  \footnote{
    \citeauthor{Armstrong:1968vh} tightens the account of inference a little to avoid including any belief \emph{r} in the causal chain leading to an agent acquiring the belief \emph{p} as a premise, but we set these details aside.
    (\citeyear[195--197]{Armstrong:1968vh})
  }
  If an agent believes \emph{p} then the agent to \eval{} whatever \prop{0} is captured by \emph{p} as \valI{True}.
  Hence, expressed in our terms, the core of \citeauthor{Armstrong:1968vh}'s account reads:
  %
  \begin{quote}
    An agent infers \emph{p} from \emph{q} just in case the agent concludes \pv{\propI{p}}{\valI{True}} from a \pool{} which contains \pv{\propI{q}}{\valI{True}}, and the conclusion is a result of the \prop{0}-\val{0} pairs in the \pool{} causing the agent to conclude \pv{\propI{p}}{\valI{True}}.
  \end{quote}
  %
  The basic observation here is that the definitions provided allow us to capture when an \agpe{} on some way things are changes as a result of the \agpe{} on some (other) way things are --- e.g.\ believing \emph{p} as the result of believing \emph{q}.
  This is the phenomenon the definitions given are designed to capture, and hence the definitions are compatible with any further restrictions placed on the phenomenon, such as causation.
  In short, we do not assume that a conclusion is caused by whatever is captured by the relevant \pool{0}.
  Still, the definitions given are compatible with this idea.
\end{note}


\begin{note}
  In parallel to our treatment of \citeauthor{Armstrong:1968vh}'s causal account of inference, \citeauthor{Broome:2002aa}'s (\citeyear{Broome:2013aa}) rule following account of (active) reasoning:
  %
  \begin{quote}
    Active reasoning is a particular sort of process by which conscious premise-attitudes cause you to acquire a conclusion-attitude.
    The process is that you operate on the contents of your premise-attitudes following a rule, to construct the conclusion[\dots]
    \mbox{ }\hfill\mbox{(\citeyear[234]{Broome:2002aa})}
  \end{quote}
  %
  May be expressed in our terms as follows:%
  \footnote{
    As an aside note how both \citeauthor{Armstrong:1968vh} and \citeauthor{Broome:2002aa} include agency into their account of reasoning.
    The premise attitudes cause \emph{them}, or \emph{you}, to acquire a conclusion --- the premise attitudes do not directly, or via a path that captures adherence to a rule, cause the conclusion.
  }
  %
  \begin{quote}
    An agent concludes \(\pv{\phi}{v}\) from some \pool{} \(\Phi\) by active reasoning just in case the \agents{} conclusion of \(\pv{\phi}{v}\) from \(\Phi\) is a result of the \prop{0}-\val{0} pairs in \(\Phi\) causing the agent to operate on the \prop{0}-\val{0} pairs in \(\Phi\) by following some rule to construct \(\pv{\phi}{v}\).
  \end{quote}
  %
  Conclusions, for us, are about changes in an \agpe{}, and any additional insight into the way these changes occur is compatible with the definitions as given.

  Similar re-expressions hold for \citeauthor{Boghossian:2014aa}'s (\citeyear{Boghossian:2008vf,Boghossian:2014aa}) rule based account of reasoning, \citeauthor{Wedgwood:2006ui}'s (\citeyear{Wedgwood:2006ui}) causal account of reasoning which appeals to causally efficacious normative facts, and \citeauthor{Longino:1978wv}'s (\citeyear{Longino:1978wv}) account of inferring which includes appeal to epistemic evidential relations.
\end{note}


\begin{note}
  Further, as our interest is with changes in an \agpe{}, `minimal' accounts of inference --- such as \citeauthor{Wright:2014tt}'s (\citeyear{Wright:2014tt}) `Simple Proposal' (discussed in a different context on \autopageref{wrightSimp}) --- and \cite{Valaris:2014un}'s (\citeyear{Valaris:2014un}) `constitutive' account of reasoning may be expressed in terms of the definitions given.

  Likewise, \citeauthor{Thomson:1965vv}'s (\citeyear{Thomson:1965vv}) non-causal account of reasoning which holds an agent reasons from \(\phi\) to \(\psi\) just in case the agent believes that \(\phi\) is a reason for \(\psi\) may be captured by placing additional requirements on the relevant \pool{} which an agent concludes from.
\end{note}



% \section*{Summary}

% \begin{note}
%   This chapter detailed the way we understand conclusions, \eiw{1} an agent concludes, and \eiw{1} an agent is concluding.
% \end{note}


% \begin{note}
%   A brief summary:
%   \begin{itemize}
%   \item
%     Conclusions are \evalN{1} (\autoref{assu:concluding:pvp}).

%     Specifically, an \eval{} of a \prop{0}/state of affairs by some \val{0} (\autoref{def:prop-val}), where \prop{0}-\val{0} pair captures the way things are from the relevant \agpe{} (\autoref{def:evals}).
%   \item
%     A conclusion is always from a \pool{1} of \evalN{1} (\autoref{assu:concluding:pools}).
%   \item
%     An \eiw{0} an agent concludes \(\pv{\phi}{v}\) from \(\Phi\) is an \eiw{0} the agent reasons to \(\pv{\phi}{v}\) from \(\Phi\) (\autoref{assu:ConRea}).
%   \end{itemize}
% \end{note}


% \subsection[\citeauthor{Landman:1992wh}'s (\citeyear{Landman:1992wh}) account of the progressive]{\citeauthor{Landman:1992wh}'s (\citeyear{Landman:1992wh}) account of the progressive \hfill (Optional)}
% \label{cha:sec:fcs-def:progressive-landman}
% \nocite{Portner:1998um}
% \nocite{Engelberg:1999vi}

% \begin{note}
%   Progressive has an important role.
%   Natural language.
%   However, useful to refine intuitions.

%   In this section we more-or-less follow~\citeauthor{Portner:1998um}'s (\citeyear{Portner:1998um}) summary of \citeauthor{Landman:1992wh}'s (\citeyear{Landman:1992wh}) of the progressive.

%   \citeauthor{Landman:1992wh}'s account of the progressive has a number of drawbacks.
%   However, for illustrative purposes, the way \citeauthor{Landman:1992wh} develops an algorithmic understanding hopefully helps clarify the kind of thing the progressive is.
% \end{note}

% \begin{note}
%   In broad summary:
%   \citeauthor{Landman:1992wh} holds that an action in the progressive holds of some event just in case the event, if `allowed' to develop, would develop into an \eiw{0} the action is performed.

%   As we have seen with the perfective paradox, some action in the progressive need not continue in the actual world, and hence the core of \citeauthor{Landman:1992wh}'s account of the progressive is an account of allowing an event to continue.

%   In slightly more detail, the way an event is `allowed' to develop is captured by idea of a continuation tree, which we present via a recursive algorithm.
%   \footnote{
%     See \citeauthor{Landman:1992wh} (\citeyear[26--27]{Landman:1992wh}) for \citeauthor{Landman:1992wh}'s account of algorithm.
%     \citeauthor{Landman:1992wh}'s construction is iterative, though I find the recursive easier.

%     \textcite{Szabo:2004ul} provides a concise summary:
%     \begin{quote}
%       [A] progressive sentence is true at some time just in case some event occurs at that time, and if we follow the development of the event (within our world as long as it goes, then jumping into a nearby world, and iterating the process within the limits of reasonability) we will reach a possible world where the perfective correlate is true of the continuation.%
%       \mbox{ }\hfill\mbox{(\citeyear[34]{Szabo:2004ul})}
%     \end{quote}
%   }

%   \begin{algorithm}[H]
%     \caption{Build continuation branch}
%     \label{PrAl:basic}
%     \SetAlgoLined
%     \DontPrintSemicolon
%     \Input{
%       \(e,w,v\)
%     }
%     \KwResult{The continuation of \(e\)}
%     \Begin{
%       Continue the development of \(e\) in \(v\)\;
%       \If{\(e\) reaches a point where \(e\) does not develop further in \(v\)}{
%         Let \(u\) be the closest world to \(v\).\;
%         \If{\(u\) exists and is plausible with respect to \(w\)}{
%           Extend the continuation of \(e\) via \(C(e,w,u)\)\;
%         }
%       }
%       \Return{\(e\)}
%     }
%   \end{algorithm}

%   Progressive is true just in case some event on continuation path.
% \end{note}

% \begin{note}
%   \autoref{PrAl:basic} takes three arguments:
%   An event \(e\), an initial world \(w\), and a possible world \(v\).
%   \autoref{PrAl:basic} starts by developing \(e\) in \(v\).
%   \(e\) may develop in \(v\) without a problem.
%   If so, \autoref{PrAl:basic} simply returns the way \(e\) developed.

%   For example, \(e\) may be the event of an agent may be walking across the road.
%   If the agent crosses the road in \(v\), then \(e\) developed without a problem, and \autoref{PrAl:basic} returns \(e\).

%   However, \(e\) may be interrupted.
%   For example, the agent may be hit by a bus.
%   If so, there is a point just before the agent is hit by a bus where the agent is crossing the road.
%   We are then instructed to consider the closest world \(u\) to \(v\).
%   If \(u\) exists and is `plausible' with respect to the initial world \(w\), then we switch to \(e\) in \(u\) and repeat the construction.

%   There are three clear issues.

%   First, assumption of a closest world.
%   Taken up elsewhere.
%   \autoref{PrAl:basic} follows \citeauthor{Landman:1992wh} in assuming that if a close world exists, then a (unique) closest world exists.

%   The other two problems are more significant.

%   Illustrated the idea of \(e\) developing in terms of the progressive.
%   It is not clear how to specify \(e\) without appeal to the progressive.

%   Plausible.%
%   \footnote{
%     \citeauthor{Landman:1992wh} uses the term `reasonable'.
%     We substitute `plausible' for `reasonable' in order to avoid epistemic connotations.
%     See \textcite[17--19,24--26]{Landman:1992wh} for \citeauthor{Landman:1992wh}'s discussion of what a world being `reasonable' amounts to.
%   }
%   However, no clear understanding of what plausibility amounts to.
%   It is not closeness.

%   Goal is to allow event to develop, but make sure the way the event develops makes sense.
%   By shifting to close worlds, consider how things may have otherwise been.
%   And, by doing this relative, allow the continuation to be somewhat distant from initial world.
%   However, as drift to distant worlds, make sure that the incremental changes do not mean the possible world is `too distant' from initial world.

%   Still, this idea of mixing two modalities captures interest of the progressive.

%   Setting aside difficulties.
%   To construct continuation branch for event \(e\) in initial world, run \((C(e,w,v)\).
%   So, possible world is initial world.
% \end{note}

% \begin{note}
%   To see how the above sketch functions in practice, we follow~\citeauthor{Portner:1998um}'s (\citeyear[764--766]{Portner:1998um}) illustration of \citeauthor{Landman:1992wh}'s account.
% \end{note}

% \begin{note}
%   Our interest is with the following sentence:
%   \begin{enumerate}
%   \item
%     \label{prog:max:bad}
%     Max is crossing the street.
%   \end{enumerate}
%   Evaluated with respect to an event \(e\) in world \(w\).
%   (I.e. \(\text{Prog}(e,\text{Max crossed the street})\) is true in \(w\).)

%   Following \citeauthor{Landman:1992wh} (and in line with \assuPP{}), Max is crossing the street is true of \(e\) just in case there is a continuation branch such that \(e\) develops into an \eiw{0} Max crosses the street.

%   Now, suppose that in \(w\) \(e\) develops into an event \(e'\) where Max is hit by a bus cruising at thirty miles per hour (before Max crosses the street).
%   Somewhat ominously, let us identify this bus as `bus \#1'.

%   Event \(e'\) includes Max being hit bus \#1, but \(e\) does not, and had things been a little different, it is reasonable Max continued a little further across the street.
%   For example, if the bus had been travelling at twenty five miles per hour, Max may have been just ahead of the bus.
%   Hence, we may consider some world \(v\) which is close to \(w\) where \(e\) develop a little further.

%   So far so good, but in \(w\) Max was hit by bus \#1 in \(w\).
%   Hence, as \(v\) is \emph{close} to \(w\), the way \(v\) is close to \(w\) may require that Max is hit by a bus in \(w\).%
%   \footnote{
%     \citeauthor{Fine:1975tj}'s notice on \citeauthor{Lewis:1973th}'s (\citeyear{Lewis:1973th}) initial account of counterfactuals or \citeauthor{Veltman:2005tj}'s (\citeyear{Veltman:2005tj}) use of an example given by \textcite{Tichy:1976tp} to raise problems for \citeauthor{Lewis:1979vm}'s (\citeyear{Lewis:1979vm}) revision.
%   }
%   So, although Max makes it a little further across the road in \(v\), Max is still hit by a bus.
%   We identify the bus in \(v\) as `bus \#2'.
%   (Perhaps the bus swerves a little to the left.)

%   However, as with bus \#1 in \(w\), it seems Max may have walked a little further across the street in some possible world close to \(v\).
%   Hence, by the same reasoning we may consider some possible world \(u\) close to \(v\) and so on\dots

%   Still, does \(e\) develop into an \eiw{0} Max crosses the road?
%   By shifting through close worlds, avoid bus.
%   So long as result of shifting is plausible, then find an \eiw{0} Max crosses the road.

%   % The key property of \citeauthor{Landman:1992wh}'s account is that closeness is understood relative to the development of \(e\), rather than \(e\) itself as \(e\) happened in \(w\).
%   % Prior to bus \#2 hitting Max in \(v\), we shifted to \(u\), a world which is close to \(v\) rather than \(w\).
%   % So, as Max progresses a little further each time a possible world where Max crosses the street gets a little closer until, eventually, Max crosses the street.
%   To borrow a piece of terminology from \textcite{Dowty:1979vq}, \(e\) has sufficient \emph{inertia} to develop in some possible world \(v\), and as \(e\) develops in \(v\), inertia continues to build until Max crosses the road.

%   % However, an important restriction is placed on shifts to possible worlds.
%   % Intuitively, it is not the case that Max avoids being hit by bus \#\(j\) because Max has the strength to stop as moving bus.
%   % Yet, a possible world where Max has the strength to stop a moving bus may be close to the world where Max is not hit by bus \#\(j - 1\).
%   % In \citeauthor{Landman:1992wh}'s terminology, the relevant possible worlds where \(e\) develops must be `reasonable'.
% \end{note}

% \begin{note}
%   \autoref{fig:max-bus} is a modification of \citeauthor{Portner:1998um}'s figure 1. (\citeyear[767]{Portner:1998um})
%   \begin{figure}[!h]
%     \centering
%     \begin{tikzpicture}
%       \tikzmath{
%         % x positions
%         \x1 = 10;
%         \xb1 = 2/9*\x1; \xb2 = 4/9*\x1; \xb3 = 6/9*\x1;
%         % y positions
%         \y1 = 1.25/5*\x1; \ymid = 1/2*\y1;
%         \yw1 = \y1; \yw2 = 1/2*\y1; \yw3 = 0*\y1; \yb2 = 1/5*\y1;
%         % event e
%         \xe = 1/2*\xb1; \yediff = \yw2 - \yb2;
%         \ye = \yw2 - 1/2*\yediff;
%         \enudge = .1;
%         \xel = 0; \xer = \xb1; \yen = \yw2 - \enudge;
%         % bus 1 description location
%         \xbx = 1.5/9*\x1; \xby = 4/5*\y1;
%         % bus 2 description location
%         \xb9 = 2.5/9*\x1;
%       }
%       % Paths
%       \draw[line width=0.25mm, line cap=round] (\xb1,\ymid) -- (\xb3,\yw1); % world 1
%       \draw[line width=1mm, line cap=round, dash pattern=on 175pt off 5pt on 5pt off 5pt on 5pt off 5pt on 5pt off 5pt on 5pt off 5pt on 5pt off 5pt on 5pt off 5pt on 5pt off 5pt on 5pt off 5pt on 5pt off 5pt on 5pt off 5pt on 5pt off 5pt on 5pt off 5pt on 5pt off 5pt on 5pt off 5pt on 5pt off 5pt on 5pt off 5pt on 5pt off 5pt on 5pt off 5pt on 5pt off 5pt] (0,\ymid) -- (\xb1,\ymid) -- (\xb2,\yb2) -- (\xb3,\yw2); % world 2
%       \draw[line width=0.25mm, line cap=round] (\xb2,\yb2) -- (\xb3,\yw3); % world 3
%       % World descriptions
%       \filldraw[black] (\xb3,\yw1) circle (0pt) node[anchor=west, align=left]{world 1: Max hit by \\ bus \# 1};
%       \filldraw[black] (\xb3,\yw2) circle (0pt) node[anchor=west, align=left]{world \(i\): Max \\ crosses street};
%       \filldraw[black] (\xb3,\yw3) circle (0pt) node[anchor=west, align=left]{world 2: Max hit by \\ bus \# 2};
%       % Event
%       \draw[] (\xe,\ye) -- (\xel,\yen); % event l
%       \draw[] (\xe,\ye) -- (\xer,\yen); % event r
%       % Event description
%       \filldraw[black] (\xe,\ye) circle (0pt) node[anchor=north, align=left]{event e};
%       % Splits
%       \filldraw[black, dashed] (\xbx,\xby) circle (0pt) node[anchor=south, align=left]{bus \#1 hits Max};
%       \filldraw[black, dashed] (\xb9,\yw3) circle (0pt) node[anchor=north, align=left]{bus \#2 hits Max};
%       % Split descriptions
%       \draw[-Stealth, dashed] (\xbx,\xby) -- (\xb1,\ymid + \enudge); % bus 2 arrow
%       \draw[-Stealth, dashed] (\xb9,\yw3) -- (\xb2 - \enudge,\yb2 - \enudge); % bus 2 arrow
%     \end{tikzpicture}
%     \caption{
%       Continuation path of `Max was crossing the street'. \\
%     }
%     \label{fig:max-bus}
%   \end{figure}

%   The thick black line captures \(e\) as \(e\) is allowed to develop, and the changes in angle reflect shifts to alternative possible worlds.

%   The dashed line indicates that Max may need to avoids being hit by additional busses in order for \(e\) to develop into an \eiw{0} Max crosses the road.
% \end{note}

  % \citeauthor{Broome:2013aa}'s rule following account of reasoning is unconstrained in terms of the rules an agent may follow.
  % This aspect  of \citeauthor{Broome:2013aa}'s account is highlighted in the following passage:%
  % \footnote{
  %   \citeauthor{Broome:2013aa} doesn't require an agent has beliefs about rules followed (\citeyear[cf.][\S13.2]{Broome:2013aa}).
  % }

  % \begin{quote}
  %   [S]hould we exclude this bizarre rule: from the \prop{0} that it is raining and the \prop{0} that if it is raining the snow will melt, to derive the \prop{0} that you hear trumpets.
  %   Following this rule would lead you to believe you hear trumpets when you believe it is raining and believe that if it is raining the snow will melt.
  %   If you did this, should we count you as reasoning?

  %   I think we should.
  %   If you derive this conclusion by operating on the premises, following the rule, we should count you as reasoning.
  %   \dots
  %   I think we should not impose a limit on rules.%
  %   \mbox{}\hfill\mbox{(\citeyear[233]{Broome:2013aa})}
  % \end{quote}


%%% Local Variables:
%%% mode: latex
%%% TeX-master: "master"
%%% TeX-engine: luatex
%%% End:


\chapter{\tpro{3} and dispositions}
\label{cha:tc2-dispositions}


\begin{note}
  This chapter briefly draws parallels between the way in which we understand \tpro{1} with respect to an agent \tC{} and (what we term) the \dBCA{0} of dispositions.
  In addition, we highlight a type of objection to the \dSCA{0} does not apply to the \dBCA{0}.
\end{note}



\section{\tCV{2} and the \dBCA{0}}
\label{sec:dispositions}


\begin{note}
  Consider the `\dBCA{0}' of dispositions:%

  \begin{sketch}[\dBCA{2} --- \dBCAa{0}]
    \label{sketch:dBCA}
    \vspace{-\baselineskip}
    \begin{itemize}
    \item
      Object \(o\) has disposition \(d\)
    \end{itemize}
    \emph{If and only if}:
    \begin{itemize}
    \item
      There are descriptors \(C\)(ondition) and \(M\)(anifestation) such that:
      \begin{itemize}
      \item
        \emph{If} it were the case that \(C\), \emph{then} \(o\) would \(M\).
      \end{itemize}
    \end{itemize}
    \vspace{-\baselineskip}
  \end{sketch}

  \noindent%
  The \dBCA{0} is common.
  For example:
  %
  \begin{quote}
    To say that an object \(a\) is (water-) \emph{soluble} at time \(t\) is to say that if \(a\) were in water at \(t\), \(a\) would dissolve at \(t\).%
    \mbox{ }\hfill\mbox{(\cite[203]{Quine:2013aa})}
  \end{quote}
  %
  \begin{quote}
    Dispositional words like `know', `believe', `aspire', `clever' and `humorous' are determinable dispositional words.
    They signify abilities, tendencies or pronenesses to do, not things of one unique kind, but things of lots of different kinds.%
    \mbox{ }\hfill\mbox{(\cite[118]{Ryle:1949aa})}
  \end{quote}
  %
  \begin{quote}
    [A] statement like

    \(w\) is inflammable

    amounts to [\dots] some such fainthearted counterfactual as

    If all conditions had been propitious and \(w\) had been heated enough, it would have burned.%
    \mbox{ }\hfill\mbox{(\cite[39]{Goodman:1983aa})}
  \end{quote}
\end{note}

\begin{note}
  There are clear parallels between the \dBCA{0} (\autoref{sketch:dBCA}) and our definition of a \tpro{} (\autoref{def:tpro}, \autopageref{def:tpro}).
  % Re-stated, the definition reads:

  % \reDefinition{def:tpro}

  \noindent%
  For, contrast the conditional by which \tpro{1} are defined with the \emph{if \dots then \dots} direction of \autoref{sketch:dBCA}.
  The conditional by which \tpro{1} are defined concerns a collection of \prop{0}-\val{0}-\pool{0} and event pairings, however, it is easy to rephrase the consequent of the conditional in terms of a condition \(C_{\mathbb{P}}\) the events of \(\mathbb{P}\) satisfy, and in turn the relevant manifestation condition \(M_{\mathbb{P}}\) is the agent concluding \(\pv{\psi}{v'}\) from \(\Psi\).
\end{note}


\begin{note}
  Indeed, a straightforward consequence of the \dBCA{0} mirrors \autoref{prop:tpro-switch} (\autopageref{prop:tpro-switch}):

  \begin{proposition}[Basic proposition]%
    \label{obs:disp:basic}%
    The \dBCA{0} entails:

    \begin{itenum}
    \item[\emph{If}:]
      Object \(o\) has disposition \(d\)
    \item[\emph{Then}:]
      There are descriptors \(C'\) and \(M'\) such that:
      \begin{itemize}
      \item
        For every \scen{0}:
        \emph{If} \(C'\) is the case, \emph{then} \(o\) manifests \(M'\).
      \end{itemize}
    \end{itenum}
    \vspace{-\baselineskip}
  \end{proposition}

  \begin{argument}{obs:disp:basic}
    Suppose the \dBCA{0} holds and condition some object \(o\) with disposition \(d\).
    Given the \dBCAa{0}, there are descriptors \(C\) and \(M\) such that:
    \emph{If} it were the case that \(C\), \emph{then} \(o\) would \(M\).

    Consider the \scen{1} where \(C\) are the case, and what \(M\) captures.
    Now, re-express \(C\) as \(C'\) and \(M\) as \(M'\) such that \(C'\) and \(M'\) do not depend on context (e.g.\ world) of evaluation.%
    \footnote{
      We understand the \dBCAa{} as specifying \(C\) and \(M\) relative to a context, and hence \(C'\) and \(M'\) are given with respect to some context.
      However, it does not follow that \(C\) and \(M\) need to take the relevant context as an argument.
      (\autoref{idea:tC} is also understood this way.)
    }%
    \(^{,}\)%
    \footnote{
      Though, in general, \(C'\) need not capture all conditions.
      For, \autoref{obs:disp:basic} is a \emph{only if} statement.
    }
  \end{argument}
\end{note}


\section{The \dSCA{}}
\label{sec:dsca2}


\begin{note}
  With the parallel between the \dBCAa{} and \autoref{def:tpro} in hand, briefly observe a number of issues for the \emph{simple} conditional analysis of dispositions do not apply to the \dBCAa{}.
  And, hence, similar issues do not extend to \autoref{def:tpro}.

  The \dSCA{} is as follows:%
  \footnote{
    Compare to, e.g. \citeauthor{Lewis:1997wg}'s account:
    \textquote{%
      Something \(x\) is disposed at time \(t\) to give response \(r\) to stimulus \(s\) iff, if \(x\) were to undergo stimulus \(s\) at time \(t\), \(x\) would give response \(r\).%
    }
    (\citeyear[143]{Lewis:1997wg})
  }

  \begin{sketch}[The \dSCA{} --- \dSCAa{}, cf.\ \cite[\S1.2]{Choi:2021wg}]%
    \label{sketch:dSCA}
    \vspace{-\baselineskip}
    \begin{itemize}
    \item
      An object \(o\) disposed to \(M\) when \(C\)
    \end{itemize}
    \emph{If and only if}:
    \begin{itemize}
    \item
      \(o\) would \(M\) if it were the case that \(C\).
    \end{itemize}
    \vspace{-\baselineskip}
  \end{sketch}

  \noindent%
  Observe, \(C\) and \(M\) are used to characterise the disposition, and therefore the choice of \(C\) and \(M\) in the counterfactual is fixed.
  By contrast, the \dBCAa{} analysed a disposition \(d\) which did not contain \(C\) and \(M\), and hence allowed free choice of \(C\) and \(M\) in the corresponding counterfactual.%
    \footnote{
    Examples of \dBCA{1} are those cited by \citeauthor{Choi:2021wg} as endorsements of \dSCA{1}.
    This is clearly not the case.
    The same is true of \citeauthor{Manley:2008aa} (\citeyear[60]{Manley:2008aa}).
    Though, \citeauthor{Manley:2008aa}'s discussion is almost identical to that of \citeauthor{Fara:2006aa}~(\citeyear[\S2.1]{Fara:2006aa})\dots

    Anyway, \citeauthor{Choi:2021wg} distinguish the \dSCAa{} from `Entailment':
    %
    \begin{quote}
      \emph{F} is a disposition iff there are an associated stimulus condition and manifestation such that, necessarily, \emph{x} has \emph{F} only if \emph{x} would produce the manifestation if it were in the stimulus condition.%
      \mbox{ }\hfill\mbox{(\citeyear[\S2.1]{Choi:2021wg})}
    \end{quote}
    %
    And, `Entailment' is equivalent to the \dBCAa{0}.
    However, \citeauthor{Choi:2021wg} add:
    %
    \begin{quote}
      If disposition ascriptions do not entail corresponding counterfactual conditionals, then Entailment is hopeless.
      Note that the apparent counterexamples to [the \dSCAa{}] may seem to show just that.
      But let's leave this claim aside for the sake of argument.
    \end{quote}
    %
    There is nothing to set aside for our interest.
    For, the \dSCAa{} and the \dBCAa{} are substantially different.
  }

  Further, well-know counterexamples to the \dSCAa{} require the choice of \(C\) and \(M\) being fixed.
  For example, consider~\citeauthor{Clarke:2010aa} idea of `masks' as \textquote{something that prevents a disposition from manifesting despite the occurrence of that disposition's characteristic stimulus} (\citeyear[153]{Clarke:2010aa}).%
  \footnote{
    \cite{Johnston:1992aa} provides counterexamples to a specific instance of the \dBCAa{}.
    Parallel observations apply to (reverse) finks.
    (\cite{Martin:1994aa})
  }.
  A sweet is digested when ingested, but a sweet wrapped in plastic is not.
  Hence, being wrapped in plastic is a mask of `an object is disposed to be digested when ingested' given the \dSCAa{}.
  For, there is little preventing someone from ingesting a sweet wrapped in plastic, and it is not the case that a sweet wrapped in plastic is digested when ingested.

  Still, masks are not a problem for the \dBCAa{}.
  For, \dBCAa{} does not place constraints on the relevant \(C\) and \(M\) descriptors.
  Indeed, the relevant \(C\) and \(M\) descriptors are not assumed to be part of the disposition being analysed under a \dBCAa{}, in contrast to the \dSCAa{}.

  For sure, a masks may be counterexamples to \emph{instances} of the \dBCAa{}, but in contrast to the \dSCAa{}, masks are not counterexamples to the (basic conditional) \emph{analysis}.%
  \footnote{
    As \citeauthor{Bonevac:2011tz} stress:
    \begin{quote}
      Counterexamples must be deployed as counterexamples to specific proposals.
      The example of a glass packed in styrofoam can perhaps show that fragile cannot be analysed as would break if struck, but it shows nothing about a proposed analysis of fragile as would break if struck when unwrapped, and certainly shows nothing about any proposed analysis of a different dispositional term, such as irascible.%
      \mbox{ }\hfill\mbox{(\citeyear[1144]{Bonevac:2011tz})}
    \end{quote}
    %
    \nocite{Manley:2007aa}
    In response,~\citeauthor{Manley:2011aa} (\citeyear{Manley:2011aa}) argue that what is of interest is whether it is possible for the \dBCAa{0} to function as an analysis of disposition \emph{ascriptions} --- not whether the \dBCAa{0} is true (\citeyear[cf.][\S1.3]{Manley:2011aa}).
  }%
  \(^{,}\)%
  \footnote{
    This observation and predates masks.
    Consider, e.g., the following passage from \citeauthor{Goodman:1983aa}:

    \begin{quote}
      [W]e can define ``flexible'' if we find an auxiliary manifest predicate that is suitably related to ``flexes'' through `causal' principles or laws.
      The problem of dispositions is to define the nature of the connection involved here:
      the problem of characterizing a relation such that if the initial manifest predicate ``Q'' stands in this relation to another manifest predicate or conjunction of manifest predicates ``A'', then ``A'' may be equated with the dispositional counterpart---``Q-able'' or ``Q\textsc{d}''---of the predicate ``Q''.\nolinebreak
      \mbox{ }\hfill\mbox{(\citeyear[45]{Goodman:1983aa} --- first published in 1955)}
    \end{quote}
  }
\end{note}


%%% Local Variables:
%%% mode: latex
%%% TeX-master: "master"
%%% TeX-engine: luatex
%%% End:


\chapter{Restatements}
\label{cha:re}

\reversemarginpar

\begin{note}
  This appendix restates a handful of definitions, ideas, and propositions which are referenced multiple times throughout the main argument.

  To the left of each restatement is a page reference for the original statement.
\end{note}

\section{Definitions}
\label{sec:definitions}

\reDefinition{def:se}

\reDefinition{def:fc}

\reDefinition{def:witnessing}

\reDefinition{def:requ}

\section{Ideas}
\label{sec:ideas}

\reIdea{idea:support}

\reIdea{idea:support:possible}


\section{Propositions}
\label{sec:propositions}

\reProposition{prop:PEbasic}

\reProposition{prop:PEentail}

\reProposition{sketch:PE:cROS}

\reProposition{prop:requ-WhyV-ces}

\reProposition{prop:tpro-switch}

%%% Local Variables:
%%% mode: latex
%%% TeX-master: "master"
%%% TeX-engine: luatex
%%% End:


\printbibliography

\end{document}

%%% Local Variables:
%%% mode: latex
%%% TeX-master: t
%%% TeX-engine: luatex
%%% End:

