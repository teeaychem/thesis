\documentclass[10pt]{article}
% \usepackage[margin=1in]{geometry}
% \newcommand\hmmax{0}
% \newcommand\bmmax{0}

% % % Fonts% %
\usepackage[T1]{fontenc}
   % \usepackage{textcomp}
   % \usepackage{newtxtext}
   % \renewcommand\rmdefault{Pym} %\usepackage{mathptmx} %\usepackage{times}
\usepackage[complete, subscriptcorrection, slantedGreek, mtpfrak, mtpbb, mtpcal]{mtpro2}
   \usepackage{bm}% Access to bold math symbols
   % \usepackage[onlytext]{MinionPro}
   \usepackage[no-math]{fontspec}
   \defaultfontfeatures{Ligatures=TeX,Numbers={Proportional}}
   \newfontfeature{Microtype}{protrusion=default;expansion=default;}
   \setmainfont[Ligatures=TeX]{Minion 3}
   \setsansfont[Microtype,Scale=MatchLowercase,Ligatures=TeX,BoldFont={* Semibold}]{Myriad Pro}
   \setmonofont[Scale=0.8]{Atlas Typewriter}
   % \usepackage{selnolig}% For suppressing certain typographic ligatures automatically
   \usepackage{microtype}
% % % % % % %
\usepackage{amsthm}         % (in part) For the defined environments
\usepackage{mathtools}      % Improves  on amsmaths/mtpro2
\usepackage{amsthm}         % (in part) For the defined environments
\usepackage{mathtools}      % Improves on amsmaths/mtpro2

% % % The bibliography % % %
\usepackage[backend=biber,
  style=authoryear-comp,
  bibstyle=authoryear,
  citestyle=authoryear-comp,
  uniquename=false,%allinit,
  % giveninits=true,
  backref=false,
  hyperref=true,
  url=false,
  isbn=false,
]{biblatex}
\DeclareFieldFormat{postnote}{#1}
\DeclareFieldFormat{multipostnote}{#1}
% \setlength\bibitemsep{1.5\itemsep}
\addbibresource{Thesis.bib}

% % % % % % % % % % % % % % %

\usepackage[inline]{enumitem}
\setlist[itemize]{noitemsep}
\setlist[description]{style=unboxed,leftmargin=\parindent,labelindent=\parindent,font=\normalfont\space}
\setlist[enumerate]{noitemsep}

% % % The following section relates to theorems, etc. % % %
\usepackage{thmtools}

\declaretheoremstyle[
spaceabove=6pt, spacebelow=6pt,
headfont=\normalfont\bfseries,
notefont=\mdseries, notebraces={(}{)},
bodyfont=\normalfont,
% postheadspace=1em,
% qed=\qedsymbol
]{defstyle}

\declaretheoremstyle[
spaceabove=6pt, spacebelow=6pt,
headfont=\normalfont\bfseries,
notefont=\normalfont\bfseries, notebraces={}{},
bodyfont=\normalfont,
% postheadspace=1em,
% qed=\qedsymbol
]{defsstyle}


\declaretheoremstyle[
spaceabove=6pt, spacebelow=6pt,
headfont=\normalfont\bfseries,
notefont=\normalfont\bfseries, notebraces={}{},
bodyfont=\normalfont\color{red},
% postheadspace=1em,
qed=\qedsymbol
]{notestyle}

\declaretheorem[name=Theorem,numberwithin=section]{theorem}
\declaretheorem[sibling=theorem,style=remark]{remark}
\declaretheorem[sibling=theorem,name=Corollary]{corollary}
\declaretheorem[sibling=theorem,name=Lemma]{lemma}
\declaretheorem[sibling=theorem,name=Fact]{fact}
\declaretheorem[sibling=theorem,name=Proposition]{proposition}
\declaretheorem[sibling=theorem,name=Definition,style=defstyle]{definition}
\declaretheorem[sibling=theorem,name=Assumption,style=defstyle]{assumption}
\declaretheorem[name=Definitions,numbered=no,style=defsstyle]{definitions}
\declaretheorem[sibling=theorem,name=Example,style=defstyle]{example}
\declaretheorem[name=Note,style=notestyle]{note}
\declaretheorem[name=Ramble,style=notestyle]{ramble}
\declaretheorem[name=Scenario,style=defstyle]{scenario}
% % % % % % % % % % % % % % % % % % % % % % % % % % % % % %

% % % Misc packages % % %
\usepackage{setspace}
% \usepackage{refcheck} % Can be used for checking references
% \usepackage{lineno}   % For line numbers
% \usepackage{hyphenat} % For \hyp{} hyphenation command, and general hyphenation stuff

% % % % % % % % % % % % %

% % % Red Math % % %
    \usepackage[usenames, dvipsnames]{xcolor}
    % \usepackage{everysel}
    % \EverySelectfont{\color{black}}
    % \everymath{\color{red}}
    % \everydisplay{\color{black}}
% % % % % % % % % %

\usepackage{pifont}
\newcommand{\hand}{\ding{43}}
\usepackage{array}
\usepackage{epigraph}


\usepackage{multirow}
\usepackage{adjustbox}
\usepackage{verse}


\usepackage{titlesec}



\makeatletter
\newcommand{\clabel}[2]{%
   \protected@write \@auxout {}{\string \newlabel {#1}{{#2}{\thepage}{#2}{#1}{}} }%
   \hypertarget{#1}{#2}
}
\makeatother

\newcommand{\boxarrow}{%
  \mathrel{\mathop\Box}\mathrel{\mkern-2.5mu}\rightarrow
}
\newcommand{\diamondarrow}{%
  \mathrel{\mathop\Diamond}\mathrel{\mkern-2.8mu}\rightarrow
}


\titleclass{\subsubsubsection}{straight}[\subsection]

\newcounter{subsubsubsection}[subsubsection]
\renewcommand\thesubsubsubsection{\thesubsubsection.\arabic{subsubsubsection}}
\renewcommand\theparagraph{\thesubsubsubsection.\arabic{paragraph}} % optional; useful if paragraphs are to be numbered

\titleformat{\subsubsubsection}
  {\normalfont\normalsize\bfseries}{\thesubsubsubsection}{1em}{}
\titlespacing*{\subsubsubsection}
{0pt}{3.25ex plus 1ex minus .2ex}{1.5ex plus .2ex}

\makeatletter
\renewcommand\paragraph{\@startsection{paragraph}{5}{\z@}%
  {3.25ex \@plus1ex \@minus.2ex}%
  {-1em}%
  {\normalfont\normalsize\bfseries}}
\renewcommand\subparagraph{\@startsection{subparagraph}{6}{\parindent}%
  {3.25ex \@plus1ex \@minus .2ex}%
  {-1em}%
  {\normalfont\normalsize\bfseries}}
\def\toclevel@subsubsubsection{4}
\def\toclevel@paragraph{5}
\def\toclevel@paragraph{6}
\def\l@subsubsubsection{\@dottedtocline{4}{7em}{4em}}
\def\l@paragraph{\@dottedtocline{5}{10em}{5em}}
\def\l@subparagraph{\@dottedtocline{6}{14em}{6em}}
\makeatother

\newcommand{\sem}[1]{\ensuremath{[\kern-.5mm[{#1}]\kern-.5mm]}}

\setcounter{secnumdepth}{4}
\setcounter{tocdepth}{4}

% \titleclass{\todopar}{straight}[\section]
% \newcounter{todopar}
% \renewcommand{\thetodopar}{\Alph{todopar}.}
% \titleformat{\todopar}[runin]{\normalfont\normalsize\bfseries\color{WildStrawberry}}{\thesection.\thetodopar}{\wordsep}{}
% \titlespacing*{\todopar} {\parindent}{3.25ex plus 1ex minus .2ex}{1em}

\usepackage{tikz}
\usetikzlibrary{arrows,positioning}

\usepackage[hidelinks,breaklinks]{hyperref}

\title{Satisfactory reasoning} %Aberrant desire}
\author{Ben Sparkes}
% \date{ }

\begin{document}

\maketitle


\section{Introduction}
\label{sec:introduction}

{\color{red}
  Desires really aren't important for what I have in mind.
  One can take anything they want which would serve the function of desire, so reasons for action in a broad sense.
  The question is about the status of these.
  And, the basic idea is that there's a problem with taking these as basic, and further understanding success conditions.
  Taking them as basic does, for sure, allow one to state the success conditions.
  But, this is not the way we reason.
  It's the success conditions themselves which keep us ticking.
  It's trying to get a handle on this.
}

\section{Two Scenarios}
\label{sec:scenarios}

\begin{quote}
%   {\color{red} Here I can make references to desires as dispositional states to help motivate the worry.
%   Though, it's somewhat clear that this is a problem for dispositional theories too.}
% \end{quote}

% \begin{quote}
%   Suppose I have a piece of paper according to which, inter alia, Collingwood is east of Fitzroy.
%   Can I tear the paper up so that I get one snippet that has exactly the content that Collingwood is east of Fitzroy, nothing more and nothing less?
%   If the paper is covered with writing, maybe I can; for maybe 'Collingwood is east of Fitzroy' is one of the sentences written there.
%   But if the paper is a map, any snippet according to which Collingwood is east of Fitzroy will be a snippet according to which more is true besides.\nolinebreak
%   \mbox{ }(\citeyear[310]{Lewis:1999ab})
\end{quote}

\begin{scenario}\label{scn:gift}
  You've travelled up to London for a meeting, but with a little time to spare you decide to detour through Selfridges to see if you can find a gift for your friend.
  A little searching proves successful, but the item is on a high shelf and as realise that the item would be suitable your eyes focus on the clock on the far wall and you see that unless you leave now you'll be late.
  Still, the meeting won't last too long, so you make a mental note to return and buy the item.
  The meeting was strenuous, but when leaving you remember that Selfridges had a item you wanted to buy and you return.
  You walk up the stairs and through the aisles, but you can't recall what exactly it was which seemed so suitable as a gift.
  You think you've found the spot, but it doesn't seem quite right, and so you search some more.
  You remember that you saw a clock, and that helps narrow your search, but it's a big store and clocks can be seen from a variety of places.
  Perhaps, even, you're on the wrong floor.
\end{scenario}

There's something puzzling about this scenario.
You're searching for something, but you can't recall what it is.
If you could recall the item it would be straightforward to say that you desire to buy it, but that you cannot remember where it is, and as such your beliefs are failing you.
Still, here you cannot recall the item, and so if you do have a desire to buy the item it is a desire without representational content.
But if this is the case, it seems you have a desire that you can't explain, because it is a desire that you cannot represent.

For sure, when you first saw the item you desired to buy it, and one may argue that this desire is what leads you to search the store on your return.
However, you also desire to buy a gift for your friend and the particular item you saw was a means to this end.
Given the desire to buy a gift for your friend we can find fault in your beliefs.
You believed that the item was a suitable gift, and you formed an auxiliary desire for that item given this belief, but this desire had no independent motivational force, and you can explain that you returned to the store because of this desire to buy a gift and your belief that an item would satisfy that desire.

% \begin{scenario}\label{scn:meal}
%   It's early afternoon and it's already been a stressful day in the office.
%   You find yourself day-dreaming in a meeting about a meal you had last year, and while the meeting is adjourned you phone the restaurant to confirm that it's still on the menu and with this confirmed you book a table for you and your partner in the evening.
%   At the end of the day you see that you have a table booked and you make your way over to the restaurant.
%   After being seated you start to look at the menu, and while it's all very appetising you can't recall why you're there.
%   You ask your partner if they booked the table, but you did.
%   You remember calling the restaurant to ask about a meal, and you recall your desire for a particular meal.
%   You look through the menu, but after multiple passes you can't recall the exact meal you had been day-dreaming about.
%   % It has been a stressful day, and that explains why you're eating out, but not why you're at this particular restaurant.
%   % You booked the table, but you booked the table for a reason, and there are handful of other restaurants you'd default to if you simply wanted to eat out.
%   % You realise that you probably wanted a particular item from the menu, but you can't recall which.
%   % Nor are you sure that you're desire to come wasn't premised on the atmosphere of the place.
%   % Still, you recognise that you desire to be here.
% \end{scenario}

\begin{scenario}\label{scn:song}
  It's early afternoon and you're talking with a colleague in a coffee shop.
  A song comes on the radio.
  The coffee shop is busy and you're trying to focus on the current topic of conversation, but you'd like to listen to the song properly.
  You briefly ask your colleague if they know what song is playing and they tell write down the title and the artist on a napkin for you.
  At the end of the day you walk to a record store while humming to see if you can purchase a copy of the song, but the store is closed.
  So, you set yourself a reminder to return at the weekend while running errands.

  As you enter the store you remember the napkin and take it from your pocket, but as you pull out the napkin you realise the ink has run and you only have the name of the artist.
  You recall that you asked your colleague to write down a song, but you can't bring to mind anything other than thanking your colleague as you put the napkin in your pocket.
  Still, the store offers you a booth so you can search through the artists discography for the song you're looking for.
  And, as you think that upon hearing the song again you'll recognise it and remember why you asked for its information you sit down and being to listen.
\end{scenario}

Scenario~\ref{scn:song} shares many similarities to scenario~\ref{scn:gift}, but while in scenario~\ref{scn:gift} you could not recall a means to an end, in scenario~\ref{scn:song} you cannot recall the end to which you have performed the means.
As such, this appears to be a puzzle about desire.
What explains why you are searching through a discography is your desire to listen to a particular song.
However, your desire lacks representational content; you can't recall anything about the song itself, nor why your colleague wrote it on a napkin.
So, sitting in the booth you have both a desire and a means to satisfy it, but you can't explain what the means are for.
Perhaps you wanted to listen to the song, but maybe you wanted to give a copy to a friend, or it may have been a recommendation from your colleague.

All that can be said is that you desire to do something with the song that was meant to be written on the napkin.
In other words, you desire to satisfy the end to which you are taking means.

For sure, you are disposed to take actions that will likely bring about listening to the song again, but you do not believe that you are taking these actions in order to bring about listening to the song.
Nor is your attention is directed toward the song prior to hearing it and thoughts of the song do not occur to you in a favourable light, though on hearing the song again it would appear `good' or pleasurable.
The difficulty is that you will be satisfied when hearing the song in the booth.
Equally, you would be satisfied if you heard the song on the radio when at home, or if a friend played it to you.

These considerations suggest that your desire persists, even though you cannot attribute content to it.
It seems right to think that if you could recall what is was that you desire you would be motivated to pursue that thing, perhaps by a more direct application of the means you're already engaged, or perhaps by some other means.
It seems more difficult to think that if you could recalled what is was that you \emph{desired}, you would be motivated to pursue that thing.
Perhaps this is because the latter formulation permits the possibility that your desires have changed, which should be ruled out.
The suggestion, however, is that to rule out the possibility that your desires have changed is to simply posit that the desires persist.
Of course, as the desire does not have representational content it cannot persist as an `ordinary' desire.
If it did you would be exploring rather than searching through the discography.

Of course, it is possible that you no longer have the desire to hear the song again.
You may tire of searching through the discography, or you may hear some other song which draws your attention.
It may be that you only have the desire to recognise the song that you desired to hear again, perhaps you mistake the song for another, or when played again you're sure you heard something else.
So, you desire to hear the song again may morph into a different desire, and in this respect perhaps your desire morphed into something else when the thought of the song slipped from your mind.
However, on the assumption that you are satisfied on hearing the song again it is hard to see why we should be committed to change in desire.
For sure, there is a change in your relation to what you desire, but what you desire does not change.\nolinebreak
\footnote{\color{red} Note connexions to Rip van Winkle in \textcite{Perry:1997aa}.}

\paragraph{ }
To put cards on the table, I adhere to the received wisdom about desire, which \citeauthor{McDaniel:2008aa} characterise by the following three theses:\nolinebreak
\footnote{\color{red} Also happy to endorse conditional desire, but it's not that relevant here.}
\begin{quote}
  \begin{enumerate}[label=RW\arabic*., ref=(RW\arabic*)]
  \item Desire is a relation between a person and a proposition.
  \item \emph{S}'s desire that \(P\) is satisfied iff \(P\) is true.
  \item \emph{S}'s desire that \(P\) is frustrated iff \(P\) is not true.\nolinebreak
    \mbox{ }\hfill(\citeyear[269]{McDaniel:2008aa})
  \end{enumerate}
\end{quote}

Desire is a propositional attitude which consist of a person (at a time) desiring that some proposition is true.
This tells us \emph{what} is desired (that a proposition be true) but not \emph{how} it is desired.
And, it is the \emph{how} not the \emph{what} of your desire that is at issue in scenario~\ref{scn:song}.
I follow \citeauthor{Kaplan:1989ab} and \citeauthor{Perry:1993aa} in distinguishing the cognitive significance of the proposition an agent has an attitude toward from the proposition thereby related.\nolinebreak
\footnote{\color{red} Pointing out that a bunch of authors highlight that these puzzles arise for desire as well as belief (and other attitudes) and that this is pretty much an exploration of what's going on in such cases, with a focus on dynamics.}

There are a number of ways `how' can be understood.
Following \citeauthor{Kaplan:1989ab} we may talk of the \emph{character} of a desire, say the desire to alight at the last northbound stop of the Caltrain, which may relate you to the proposition that you alight at 4th \& King or the proposition that you alight at 22nd St., depending on whether a full service is running and with other contextual factors similarly fixed.
Still, while character affords us representational content which can relate the same desire to different propositions depending on content, it requires some content in order to specify which proposition an agent is related to given a context and the problem we have is that your desire to listen to the song again has no representational content.
So, following \citeauthor{Perry:1997aa} we may talk of the \emph{causal role} of desires; the various combinations of factors that bring the desire about and the various combinations of factors that the desire brings about in turn. (\citeyear[360--361]{Perry:1997aa})
Causal roles generalise characters they require only a functional specification of the relevant state, and as such need not be characterised by some utterance or thought.

Of course, at a base level we are (part of) causal networks and so talk of the causal role of desires can be seen to occupy an uneasy middle ground between folk-psychology and full blown psychology in which the conception of desire as a relation between an agent and a proposition may serve no theoretical role.
So, \citeauthor{Perry:1997aa}'s talk of causal roles should be understood, following \textcite{Lewis:1979aa}, with respect to how we (as folk) systematise the minds of ourselves and others.
Taking desires to be a relation between an agents and a propositions and investigating how these relations are secured facilitate systematic common-sense psychology; an account of how you and I reason about ourselves and others.
(\citeyear[{\color{red} ???}]{Lewis:1999ab})
It is analytic that we employ folk-theory when reasoning about ourselves and others and that this systematises causal relations, but it doesn't follow from this that causal relations have a substantive role in our systematisation.
For this reason I wish to eschew talk of the causal role of desires and will instead talk of how desires \emph{participate} in the practical reasoning of agents.
Desires participate an agent's practical reasoning by being brought about by various combinations of factors both internal and external to the agent and by bringing about various combinations of factors both internal and external to the agent.
Nothing changes from \citeauthor{Perry:1997aa}'s characterisation of causal roles, but we have unburdened ourselves from some unease and have allowed ourselves some distance from \citeauthor{Perry:1997aa}'s understanding of propositional attitudes, should we need it.\nolinebreak
\footnote{\color{red} As noted above (or maybe just here) the authors all suggest that while the focus is on belief, same issues hold for desire.}
We also now have a slogan for our puzzle: participation without representation.
Still, participation without representation is not a puzzle for folk-theory.
The scenario given above follows ordinary reasoning and so its coherence with folk-theory is secure, instead the puzzle concerns our high-level philosophical characterisations of folk-theory.



It may be argued that participation without representation is insufficiently commonplace to require attention in our high-level characterisations of folk-theory.
We doubt that participation without representation is so rare.
Scenario~\ref{scn:song} is contrived to make the phenomenon clear, and more standard cases can be found.
Further examples include driving to a supermarket to fetch an important ingredient only to traverse the aisles, beginning a sentence only to pad while trying to recall the motivating thought, or hurrying to finish a task only to pace the room while trying recall what you planned to do at leisure.
For sure, the quick presentation of these examples does not require us to posit the participation of a desire which you cannot grasp, as for this we must assume that when you find the ingredient, finish the thought, or engage in the leisure activity that you satisfy your prior desire.
It is possible that you return empty-handed, let the sentence unravel, or decide to engage in some other activity, but often your desire is satisfied.
And, while these examples rely on the same means-end schema used to generate scenario~\ref{scn:song}, the broad idea may find use in exploring instances of the phenomenon noted by \citeauthor{Anscombe:1957aa} in which the question `Why?' has application as an appropriate question by has no answer (\citeyear[\SS17--18]{Anscombe:1957aa}).
The point is that by considering participation without representation we enrich our characterisation of folk-theory, and we hope to show that this enrichment is instructive.

For sure, participation without representation is a particular instance of the way in which participation and representation come apart.
Generalising \citeauthor{Kaplan:1989aa}'s notion of character grants us representation with varying participation and participation with varying representation.
Likewise, representation without participation may be explored.
We focus on the former due to the prevalence of appeal to representation in the characterisation of desires, pro-attitudes, and practical reasoning, and as the lessons of \citeauthor{Kaplan:1989ab}, \citeauthor{Lewis:1979aa}, and \citeauthor{Perry:1993aa} with respect to the latter cases straightforwardly generalise from beliefs to desires.

\section{Two perspectives on satisfaction}
\label{sec:two-persp-satisf}

Resolving the puzzle of participation without representation requires a nuanced understanding of the role of desires in an agent's practical reasoning.
This much is clear.
In this section we argue that this puzzle is resolved by characterising the role of desires in practical reasoning as derivative of an agent's reasoning about satisfaction.
In short: for an agent to desire a proposition is for that proposition to capture the conditions under which the agent expects to be satisfied.
This perspective takes `desire' to be a placeholder for whatever mental states bear the appropriation to a substantive understanding of satisfaction.\nolinebreak
\footnote{\color{red} This is not hedonism about desire.}
The contrast perspective takes `desire' to be substantive and identifies satisfaction with the conditions imposed by the mental state of desiring.

\begin{quote}
  \begin{description}[style=unboxed, leftmargin=\parindent,labelindent=\parindent,font=\normalfont\bfseries\space]
  \item[Substantive-satisfaction] For an agent to desire that p is for the agent's reasoning to be oriented such that p is (expected to be) satisfactory.
  \item[Substantive-desire] For an agent to desire that p is for p to meet the conditions of satisfaction determined by the agent's reasoning.
  \end{description}
\end{quote}



To help distinguish these two perspectives, consider a simple Buridan case in which an agent desires two objects which are in all aspects relevant to the desire.
To take an example from \citeauthor{Rescher:1960aa}'s (\citeyear{Rescher:1960aa}) survey of the issue, consider \citeauthor{Al-Ghazali:1963aa}'s scenario in which a person has two dates in front of them for which they have a strong desire for, but is unable to take both with all distinguishing qualities, such as beauty, nearness, or facility in taking are assumed absent.
(\citeyear[26--27]{Al-Ghazali:1963aa};\citeyear[147--148]{Rescher:1960aa})

Adopting the perspective that satisfaction is a placeholder we must attribute two distinct desires to the individual, one for each date.
Of course, it is possible for the individual to have a single desire for either date, but this is not the scenario they find themselves in.
They desire both dates and are seemingly unable to choose one over the other given the absence of distinguishing qualities.

In contrast, adopting the perspective that satisfaction is substantive, we may say that the agent expects both dates to satisfy them to the same degree and that  given the absence of distinguishing features, both dates will satisfy the agent in the same way.
The agent still has two distinct desires as there are two distinct objection which may satisfy their desires.
However, the lack of distinguishing features serves as evidence that the individual has a desire for either date, and in contrast to the former case the issue is not that the agent fails to have a desire for either date, but has failed to recognise that either date would satisfy the same locus of desire.
This is idea is captured by \citeauthor{Averroes:1954aa} in his commentary of \citeauthor{Al-Ghazali:1963aa}, noting that [w]hichever of the two dates [the person] may take, [their] aim will be attained and [their] desire satisfied (\citeyear[23]{Averroes:1954aa}).

The two readings of the scenario differ in their account of what the individuals desire amounts to, but the key distinction is in their understanding of what it would take for the individual to desire either date.
The desire-as-placeholder account requires some third desire to be formed, while the contrasting account allows the two existing desires to be collapsed by some inferential process.\nolinebreak
\footnote{\color{red} Advertising.}
So, in a certain sense the individual may have two desires as there are tow objects (the dates) that would satisfy them.
However, in another sense this may be a distinction without a difference, as the two objects would satisfy the individual in the same way and here again we may have an example of a desire participating in an agent's practical reasoning without representation.

This reading of a simple Buridan case does not extend to more complex Buridan cases in which an agent desires two suitably distinct objects, such as \citeauthor{Montaigne:1965aa}'s individual who is placed between a bottle of wine and a gammon of bacon with equal appetite to eat and drink (\cite[cf.][156]{Rescher:1960aa}) as intuitively these desires relate to different aspects of satisfaction.
Still, in the complex case we can understand the individual's dilemma in terms of satisfaction.
Both thirst and hunger (or any other conflicting states) serve to indicate to the individual how they may be satisfied, and hence explain their desire for particular means to their satisfaction.

Satisfaction, as we are using the term, is a technical notion and has the theoretical role of allowing us to identify aspects of an agent's reasoning as practical its orientation.
Broadly stated, satisfaction is to practical reasoning as truth is to theoretical reasoning.
Desires, in turn, capture information about what would satisfy an agent, just as beliefs capture information about what is true of the world.
This establishes a parity between desires and beliefs as both attitudes capture information about something independent from an agent's reasoning, and the distinction between desires and beliefs arises from the role of this information in the agent's reasoning.
The independence of a (substantive) account of satisfaction from an agent's reasoning is the primary factor which distinguishes our account from accounts of practical reasoning which arise from more orthodox conceptions of desire.

To illustrate, consider \citeauthor{Pettit:1990aa}'s (\citeyear{Pettit:1990aa}) strict background view of desire.
\citeauthor{Pettit:1990aa}'s view  combines two conceptions of practical reasoning.
On the one hand, an \emph{intentional} conception every action is causally explained by the beliefs and desires of an agent.
On the other, a \emph{deliberative} conception in which at some point in practical reasoning there is (typically) a belief that a chosen option has some property which provides justification for choosing it.
(\citeyear[565--566]{Pettit:1990aa})
\citeauthor{Pettit:1990aa} observe that these two conceptions of practical reasoning are compatible, and argue that desires are always in the background of practical reasoning as these are always part of the motivating reason for an agent's choice of an option.
(\citeyear[573]{Pettit:1990aa})
Desires may also appear in the foreground of practical reasoning, as an agent may choose an option due to the belief that they desire that option, but for our purposes it is the backgrounding of desires which is key.
\citeauthor{Pettit:1990aa}'s argument for the background view is straightforward.
\begin{quote}
  \begin{enumerate}[label=\arabic*., ref=(\arabic*)]
  \item\label{ps:1} Having a reason to \(\Phi\), specifically a motivating reason to \(\Phi\), is having a goal: say, the goal that p.
  \item\label{ps:2} Having such a goal is being disposed, given appropriate beliefs, to act so that p.
  \item\label{ps:3} And being so disposed is desiring that p.
  \end{enumerate}
  From \ref{ps:1}, \ref{ps:2} and \ref{ps:3} it follows that having a reason to \(\Phi\)-necessarily involves the presence of an appropriate desire.\nolinebreak
  \mbox{ }\hfill(\citeyear[573]{Pettit:1990aa})
\end{quote}
The argument is straightforward and is largely neutral on whether desires are cognitive or non-cognitive states.
The problem is that for \citeauthor{Pettit:1990aa} to have a (motivating) reason is to have a goal, which \citeauthor{Pettit:1990aa} give a dispositional analysis of.
However, if the mental is to be dispositionally analysed, we can move directly from the identification of a (motivating) reason to a disposition.
\citeauthor{Pettit:1990aa}'s initial premise would then be:
\begin{enumerate}[label=\arabic*\('\)., ref=(\arabic*\('\))]
\item\label{ps:1R} Having a reason to \(\Phi\), specifically a motivating reason to \(\Phi\), is being disposed, given appropriate beliefs, to act in a certain way: say, so that p.
\end{enumerate}
Given~\ref{ps:3} it follows that to have a (motivating) reason to \(\Phi\) is to have a desire to act in a certain way, but to infer that an agent has a goal we need the converse of~\ref{ps:2}.
\begin{enumerate}[label=\arabic*\('\)., ref=(\arabic*\('\))]\setcounter{enumi}{1}
\item\label{ps:2R} Being disposed, given appropriate beliefs, to act so that p is to have a goal that p.
\end{enumerate}
Although \citeauthor{Pettit:1990aa} do not state what having a goal amounts to, but it is natural to read the talk of goals as specifications of conditions under which the agent would be satisfied.\nolinebreak
\footnote{
  \citeauthor{Smith:1987aa} argues that the goals (motivating) reasons embody have a role in assigning to an agent's motivating reasons `the minimal justificatory role possible: the role of justifying from the perspective of the value that that very reason embodies.'  (\citeyear[38--39]{Smith:1987aa}), that `reasons must themselves be constituted by goals' (\citeyear[45]{Smith:1987aa}) and that having a goal is being in a `\emph{state with which the world must fit}, rather than \emph{vice versa}' (\citeyear[54]{Smith:1987aa}).
}
Without an argument for~\ref{ps:2R}, then, it does not follow from a dispositional analysis of (motivating) reasons that having a desire requires having a specification of the conditions under which one would be satisfied.
Instead, we have a analysis of desires in terms of the dispositions of an agent to act in a certain way, given appropriate beliefs.

On the substantive-satisfaction view~\ref{ps:2R} would be replaced with~\ref{ps:S2}, leading to a slight reformulation of~\ref{ps:3}.
\begin{enumerate}[label=\arabic*\(''\)., ref=(\arabic*\(''\))]\setcounter{enumi}{1}
\item\label{ps:S2} Being disposed, given appropriate beliefs, to act so that p is for the agent's reasoning to be oriented such that p is (expected to be) satisfactory.
\item\label{ps:S3} For an agent's reasoning to be so oriented is for an agent to desire that p.
\end{enumerate}
Premise~\ref{ps:S2} is the core of the substantive-satisfaction view of desire.
What an agent desires is not determined by the content of their mental states, but by the dynamics of these states.
Desires with representational content aid an agent in satisfying their desires, but so long as the agent's reasoning is oriented so that they are disposed to act in a certain way.
Talk of the `orientation' of an agent's reasoning is imprecise, but by positing satisfaction we can state in broad terms what the agent's reasoning is oriented toward, and as observers we can give this representational content through our understanding of the dynamics of the agent's reasoning without assuming such content is present to the agent.
In scenario~\ref{scn:song} your reasoning is oriented such that listening to the song again is expected to be satisfactory, because you expect to be satisfied by the end to which you are undertaking means.
When walking the through the store to find the missing ingredient (which you cannot recall) you expect satisfaction when you find what you're looking for.
And, when stumbling to recover the thought you began to give expression to, you expect the expression of whatever the thought is to be satisfying.
If you did not have the expectation of satisfaction your actions would be different, you would discard the napkin, leave the store, or move to another thought.
It is likely to be the case that without representational content the expectation of satisfaction will fade.
Still, lack of representational content does not entail a failure of satisfaction, as you are satisfied when you hear the song, find the ingredient, or finish the thought, and so likewise lack of representational content does not mean that you cannot expect to be satisfied by acting in a certain way.






\newpage


\subsection{Weak and strong theses}
\label{sec:weak-strong-theses}


The weak is that the have access to success conditions at a certain point in our reasoning, and additional reasoning can lead us away from these.
The strong is that sometimes we don't even have access to the success conditions.

This comes out in the case of belief.
It's the fixing the stopwatch via the pendulum and via the church.

Reasoning, then, is structuring information around all of this.
What I come to form regarding pro-attitudes based on this kind of stuff doesn't amount to an intention, it's not necessarily in the slightest robust, and that's part of the point (and this is how the concerns link into the earlier paper).
However, it does suggest that desire-belief based reasoning has a lot to say for it.


\subsection{Propositions as tests}
\label{sec:prop-as-tests}

\newpage



{\color{red}
  I want the following when I've got the talk of structure in place.
  It's that this `objective' nature of satisfaction allows us to talk about information in this way.
}
A difficulty in our talk about participation without representation is that the information available to an agent may be distinct from the information available to us when describing the agent.
Still, this parallels belief.
You may believe that it is raining in Z-land by believing that \emph{it is raining} without having any representation of Z-land given the isolated group to which you belong because your situation suitably fixes the participation of Z-land in your belief and we have the resources to make this participation explicit.
(\cite[cf.][]{Perry:1986aa})


{\color{red}
  The key difference, I think, between what I'm proposing and the ideas of \citeauthor{Pettit:1990aa} and \citeauthor{Schroeder:2007aa} is that for these authors desire is part of the agent's reasoning, it's something that's there.
  However, on the satisfaction view this isn't part of practical reasoning, it's something independent, and it's the orientation of reasoning that distinguishes it as practical.
}


On the view we endorse, practical reasoning can lead to the formation of desires through an agent reasoning that they would be satisfied were certain circumstances to obtain.
And, the puzzle of participation without representation reduces to a straightforward situation in which an agent is unable to infer the conditions under which they would be satisfied but has appropriate premises to expect that some aspect of a situation they are able to bring about will lead to their being satisfied.
{\color{red}
  Ah, so it's important that there's partial information.
  This, perhaps, links to cases of failures of introspection.
  Of course, failure of introspection here is something different.
  Though, I don't think I want to explore the issue in quite so much detail at this point in time.
  Right here the issue should be on some of the more Humean aspects.
}



{\color{red}
  Here I want to say something about having a desire to figure out what you desired.
  This can happen, but it doesn't need to, and the way I've set the case up is quite unreflective.
  But even if it does, this doesn't really explain anything, you have no way of representing that desire, and that's the problem.
  It's also that this doesn't get the satisfaction conditions correct.
  Either you then are satisfied when you find out, and you reform the desire or it was there all along.
}



{\color{green}
  Oh, and in cases of akrasia you're reasoning about something which you expect to satisfy you but for which you don't have the means.
  This, is really nice.
  (Though also rather sad.)
}

{\color{blue}
  Well, then, given the above relations to propositions are still the core of desire, and the idea is that having access to representational content is what allows agents to actually do things.
  Well, it allows us to do thing in the way we do them.
  The motivational component comes first, and we use practical reasoning to guide this.
  (This is really what deals with the problem of akrasia, and the proposal only makes sense in this context.)
  And, the way the argument is going is that there's a way in which we can understand motivation without resorting to a given proposition.
  For better or worse we can't really reason without some representation content, and the interesting thing about us is that we are able to do this.
  
}

{\color{blue} Well, the general idea is that it seems as though this is a problem with introspection, and when we grant that introspection can fail, there's no problem with specifying various desires.
  So, it's the case that the puzzling desires really aren't the same desires without representational content, but that they're different desires.
  In short, the continuity is somewhat misplaced.}



\section{Intro}
\label{sec:intro}

Getting the TV channel wrong and enjoying a film.
This is a case where the extrinsic desire is fine, but there's a problem.
You think you're satisfying a desire, but really you're finding out that something else satisfies the same desire.
Or, two YouTube links or whatever, one copied and you fail to copy the second.
Either way, you now have more information about what satisfies you.

So, that's the proposal.
Desire something because one reasons that it may satisfy them.

In these examples you're reasoning about what would satisfy you, in the same way that you may reason about what is true when you try to recall some fact that you've forgotten.
However, if this is the case then desires can't be `givens'.

The key is that in these scenarios one is trying to recreate the content.
One deosn't use the other desires one has.
There's cognitive significance without content.
Though, this term isn't quite right.


Then, washing machine example.
You lose some of the relevant propositional information.

Conversely, some other example where you refine the desire into something concrete.
Of course, here we can speak of desires being formed, but there's an appealing symmetry.
Indeed, in the relevant example it's simply going to be reading through the spec sheet, desiring something with a something or other spin.


So, what's going on?
Well, either you recover the content, or you start to construct a new desire, arguably.



\section{The Idea}
\label{sec:idea}

There's also the case where someone forms a desire, and then the beliefs on which they form this desire no longer hold.
Ray had faculty with boots, I had adviser and quoting Shakespeare.
The puzzling thing about this is that the person is able to cite the certain determinate conditions, but none of these capture the desire.
Any condition stated doesn't work, unless they say that they want red boots.
Perhaps this is the way to go, but then an instrumental desire has turned into a self-standing desire.
Ah, in this sense I like Shakespeare case, as what's different is that the desire looks in some sense irrational.
Right, and with the Shakespeare they've started quoting Milton outside the office of the new chair.


\subsection{Objection}
\label{sec:objection}

The key objection, at least to my understanding, is that it seems possible to have some instance of reasoning which is \emph{about} or \emph{concerns} satisfaction without this having some motivational component.
This, then, pushes the inclusion of some desire proper, which stands behind the reasoning, and explains its motivational force.

So, there's some quality of the mental which makes certain states motivationally effective.
And, while the above distinction does some work, it doesn't capture this distinction.

The analogy I then want is with cases in which certain recognised beliefs aren't motivationally effective.
That is, an argument that there's something missing in our talk of belief.
So, in a sense, beliefs which don't enter into the causal nexus of things, but are likewise indistinguishable from other beliefs.
Examples involving phobias seem to be a good candidate for this.
One has a very specific motivational desire.
But, despite believing that this thing isn't an instance of what they have a phobia toward, they still act as though it was.
It's quite plausible, I want to argue, that the belief isn't correctly hooked up to the causal mechanisms.
And, it doesn't make sense to explain this via some other desire.
Say, they've done this before.
Actually, right, the agent is there and avows all of the `right things'.
Further, could enrich so that these types of cases have happened before, and that eventually the belief does make it's way.



\section{Tests}
\label{sec:tests}

Typically we think of propositions as sets of worlds.
E.g.\ \(\phi = \{w_{i}, \dots, w_{k}, \dots\}\) where the it is the commonalities between \(w_{i}, \dots, w_{k}, \dots\) which determine the meaning of \(\phi\).
This, however, isn't the full story, as we can likewise think of propositions as \emph{tests}, functions from (sets of worlds) to truth values.
From a certain perspective, this is what's captured by the use of the lambda calculus.
One can take \(\lambda w.\phi(w) = \top\), etc.\

But, this isn't quite the full story.


\begin{figure}[h]
  \centering
  \begin{tikzpicture}
    \node[draw] (sat) {\(\mathcal{S}\)};

    \node[draw, above right=of sat, rectangle] (inf) {\(\sigma\)};

    \node[draw, left=of sat] (world) {\(\omega\)};

    \node[draw, above left=of inf] (exp) {\(\mathbb{E}\)};

    \node[above right=of inf] (ext) {};

    \node[above=of ext] (ext2) {};

    \draw[->, to path={-| (\tikztotarget)}] (sat) edge (inf);
    \draw[->, to path={|- (\tikztotarget)}] (world) edge (sat);
    \draw[->, to path={-- (\tikztotarget)}] (exp) edge (sat);
    \draw[->, to path={|- (\tikztotarget)}] (ext) edge (inf);
    \draw[->, to path={|- (\tikztotarget)}] (inf) edge (exp);
    \draw[to path={-- (\tikztotarget)}, dashed] (ext) edge (ext2);

  \end{tikzpicture}
  \caption{Cards on the table}
  \label{fig:cards}
\end{figure}

\newpage

\begin{itemize}
\item Fulfilment
\item Desire
\item Recognised desire (or pro-attitude)
\end{itemize}

Fulfilment is something objective, etc., it's about what would in fact make one's life go well, and what would satisfy an agent's desires doesn't necessarily coincide with what would make the agent's life go well.
However, the puzzle doesn't really depend on whether we take fulfilment or desire-satisfaction to be key, which is a useful observation.
I'll frame everything in terms of desire-satisfaction, but this isn't essential.

The basic problem is something like failure of introspection.
However, I'm not sure this is quite right, especially as this doesn't hold for the fulfilment case.
What's common between desire-satisfaction and fulfilment is that conditions are specified.
The problem comes with a failure to recognise these conditions.
The core argument is that this happens in the case of desire, as it's somewhat intuitive that this fails in the case of fulfilment.

In the record scenario, what you have is that there exists some end \(e\) such that \(m\) is a means to \(e\) (\(\exists e (m \text{ is a means to } e)\)).
So, some end, some conditions, and some representation of these.


\newpage
\printbibliography

\end{document}