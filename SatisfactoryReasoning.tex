\documentclass[10pt]{article}
\usepackage[margin=1in]{geometry}
% \newcommand\hmmax{0}
% \newcommand\bmmax{0}

\usepackage{luatexbase} % While TeXLive is broken.

% % % Fonts% %
\usepackage[T1]{fontenc}
   % \usepackage{textcomp}
   % \usepackage{newtxtext}
   % \renewcommand\rmdefault{Pym} %\usepackage{mathptmx} %\usepackage{times}
\usepackage[complete, subscriptcorrection, slantedGreek, mtpfrak, mtpbbi, mtpcal]{mtpro2}
   \usepackage{bm}% Access to bold math symbols
   % \usepackage[onlytext]{MinionPro}
   \usepackage[no-math]{fontspec}
   \defaultfontfeatures{Ligatures=TeX,Numbers={Proportional}}
   \newfontfeature{Microtype}{protrusion=default;expansion=default;}
   \setmainfont[Ligatures=TeX]{Financier Text} %{Minion 3}
   \setsansfont[Microtype,Scale=MatchLowercase,Ligatures=TeX,BoldFont={* Semibold}]{Myriad Pro}
   \setmonofont[Scale=0.8]{Atlas Typewriter}
   % \usepackage{selnolig}% For suppressing certain typographic ligatures automatically
   \usepackage{microtype}
% % % % % % %
\usepackage{amsthm}         % (in part) For the defined environments
\usepackage{mathtools}      % Improves  on amsmaths/mtpro2
\usepackage{amsthm}         % (in part) For the defined environments
\usepackage{mathtools}      % Improves on amsmaths/mtpro2

% % % The bibliography % % %
\usepackage[backend=biber,
  style=authoryear-comp,
  bibstyle=authoryear,
  citestyle=authoryear-comp,
  uniquename=allinit,
  % giveninits=true,
  backref=false,
  hyperref=true,
  url=false,
  isbn=false,
]{biblatex}
\DeclareFieldFormat{postnote}{#1}
\DeclareFieldFormat{multipostnote}{#1}
% \setlength\bibitemsep{1.5\itemsep}
\addbibresource{Thesis.bib}

% % % % % % % % % % % % % % %

\usepackage[inline]{enumitem}
\setlist[itemize]{noitemsep}
\setlist[description]{style=unboxed,leftmargin=\parindent,labelindent=\parindent,font=\normalfont\space}
\setlist[enumerate]{noitemsep}

% % % The following section relates to theorems, etc. % % %
\usepackage{thmtools}

\declaretheoremstyle[
spaceabove=6pt, spacebelow=6pt,
headfont=\normalfont\bfseries,
notefont=\mdseries, notebraces={(}{)},
bodyfont=\normalfont,
% postheadspace=1em,
% qed=\qedsymbol
]{defstyle}

\declaretheoremstyle[
spaceabove=6pt, spacebelow=6pt,
headfont=\normalfont\bfseries,
notefont=\normalfont\bfseries, notebraces={}{},
bodyfont=\normalfont,
% postheadspace=1em,
% qed=\qedsymbol
]{defsstyle}


\declaretheoremstyle[
spaceabove=6pt, spacebelow=6pt,
headfont=\normalfont\bfseries,
notefont=\normalfont\bfseries, notebraces={}{},
bodyfont=\normalfont\color{red},
% postheadspace=1em,
qed=\qedsymbol
]{notestyle}

\declaretheorem[name=Theorem,numberwithin=section]{theorem}
\declaretheorem[sibling=theorem,style=remark]{remark}
\declaretheorem[sibling=theorem,name=Corollary]{corollary}
\declaretheorem[sibling=theorem,name=Lemma]{lemma}
\declaretheorem[sibling=theorem,name=Fact]{fact}
\declaretheorem[sibling=theorem,name=Proposition]{proposition}
\declaretheorem[sibling=theorem,name=Definition,style=defstyle]{definition}
\declaretheorem[sibling=theorem,name=Assumption,style=defstyle]{assumption}
\declaretheorem[name=Definitions,numbered=no,style=defsstyle]{definitions}
\declaretheorem[sibling=theorem,name=Example,style=defstyle]{example}
\declaretheorem[name=Note,style=notestyle]{note}
\declaretheorem[name=Ramble,style=notestyle]{ramble}
\declaretheorem[name=Scenario,style=defstyle]{scenario}
% % % % % % % % % % % % % % % % % % % % % % % % % % % % % %

% % % Misc packages % % %
\usepackage{setspace}
% \usepackage{refcheck} % Can be used for checking references
% \usepackage{lineno}   % For line numbers
% \usepackage{hyphenat} % For \hyp{} hyphenation command, and general hyphenation stuff

% % % % % % % % % % % % %

% % % Red Math % % %
    \usepackage[usenames, dvipsnames]{xcolor}
    % \usepackage{everysel}
    % \EverySelectfont{\color{black}}
    % \everymath{\color{red}}
    % \everydisplay{\color{black}}
% % % % % % % % % %

\usepackage{pifont}
\newcommand{\hand}{\ding{43}}
\usepackage{array}
\usepackage{epigraph}


\usepackage{multirow}
\usepackage{adjustbox}
\usepackage{verse}


\usepackage{titlesec}
\usepackage[hidelinks,breaklinks]{hyperref}


\makeatletter
\newcommand{\clabel}[2]{%
   \protected@write \@auxout {}{\string \newlabel {#1}{{#2}{\thepage}{#2}{#1}{}} }%
   \hypertarget{#1}{#2}
}
\makeatother



\newcommand{\boxarrow}{%
  \mathrel{\mathop\Box}\mathrel{\mkern-2.5mu}\rightarrow
}
\newcommand{\diamondarrow}{%
  \mathrel{\mathop\Diamond}\mathrel{\mkern-2.8mu}\rightarrow
}


\titleclass{\subsubsubsection}{straight}[\subsection]

\newcounter{subsubsubsection}[subsubsection]
\renewcommand\thesubsubsubsection{\thesubsubsection.\arabic{subsubsubsection}}
\renewcommand\theparagraph{\thesubsubsubsection.\arabic{paragraph}} % optional; useful if paragraphs are to be numbered

\titleformat{\subsubsubsection}
  {\normalfont\normalsize\bfseries}{\thesubsubsubsection}{1em}{}
\titlespacing*{\subsubsubsection}
{0pt}{3.25ex plus 1ex minus .2ex}{1.5ex plus .2ex}

\makeatletter
\renewcommand\paragraph{\@startsection{paragraph}{5}{\z@}%
  {3.25ex \@plus1ex \@minus.2ex}%
  {-1em}%
  {\normalfont\normalsize\bfseries}}
\renewcommand\subparagraph{\@startsection{subparagraph}{6}{\parindent}%
  {3.25ex \@plus1ex \@minus .2ex}%
  {-1em}%
  {\normalfont\normalsize\bfseries}}
\def\toclevel@subsubsubsection{4}
\def\toclevel@paragraph{5}
\def\toclevel@paragraph{6}
\def\l@subsubsubsection{\@dottedtocline{4}{7em}{4em}}
\def\l@paragraph{\@dottedtocline{5}{10em}{5em}}
\def\l@subparagraph{\@dottedtocline{6}{14em}{6em}}
\makeatother

\newcommand{\sem}[1]{\ensuremath{[\kern-.5mm[{#1}]\kern-.5mm]}}

\setcounter{secnumdepth}{4}
\setcounter{tocdepth}{4}

% \titleclass{\todopar}{straight}[\section]
% \newcounter{todopar}
% \renewcommand{\thetodopar}{\Alph{todopar}.}
% \titleformat{\todopar}[runin]{\normalfont\normalsize\bfseries\color{WildStrawberry}}{\thesection.\thetodopar}{\wordsep}{}
% \titlespacing*{\todopar} {\parindent}{3.25ex plus 1ex minus .2ex}{1em}

\title{Propositions as Tests and Desire}
\author{Benji Sparkes}
% \date{ }

\begin{document}

\maketitle

\section{The Idea}
\label{sec:idea}

The basic idea is that there's what goes on in the mind of an agent.
Here, it's not just beliefs and desires.
It's something to do with the flow of information.
But there's information from two distinct sources.
One is the world, the other is the self, so to speak.
Trouble is, this information is noisy.
But this is only part of the problem, the other part is that this information is sometimes sparse.
There's a difference between sailing a small boat on a lake where one can see the shore, and sailing a boat on the ocean.
This is like walking down a path in comparison to walking through some woods.
Something like this, what's interesting about these cases, in some respect, is that they're to do one's location, but I don't think this is an important feature.


The core of the idea is that there's transition between information states, and that there doesn't need to be a causal connexion between the information state and the world.


\subsection{Weak thesis}
\label{sec:weak-thesis}

Assuming that one forms desires and beliefs in a way which ensures that appropriate states are proper, so to speak.
That one can have transitions between `desire states' which can diverge from the underlying (or perhaps better put motivating) desire.
This, arguably, parallels cases of belief.

\subsection{Strong thesis}
\label{sec:strong-thesis}


The weak thesis, but also denies that there's this requirement for appropriate states.
In the case of belief, this is forming the attitude independently of any evidence, so the parallel is thinking I have a fix on the pendulum by looking at the church.




\section{Timepieces}
\label{sec:timepieces}

There's something about clocks.

Clocks are about time, they represent time, and clocks participate in time.
Still, while the tick of a second hand relies on the passage of time, time in which causal forces drive the internal mechanism of the clock to move the hand from resting in one position to another, the passing second does not itself cause the hand to move.
Clocks are about time, but they do not interact with it.

Clocks are about time, but they do not interact with it.
Of course, hours, minutes, and seconds are somewhat arbitrary, and it may be claimed that there is---properly speaking---no second to have caused the hand to move.
Be this as it may, seconds pass and we individuate on second from another by the minute in which it occured, and the hour in which that minute occured.
Or, perhaps we count the seconds which have passed and individuate minutes and hours based on this count.
03:14:07 on Tuesday the 19th of January 2038 may be a large number, but there are an infinity of those.
Either way, even if there's no such thing as seconds to cause the ticking of clocks, seconds exists as measurements of time and by indexing some point in time and measuring some number of seconds (\dots minutes, or hours) we can travel from different parts of the Cotswolds to meet at The Exchange.

The trasition between states of the clock are not caused by the passage of time, as noted by~\textcite{Smith:1988aa}.

\section{Sailing and Hunger}
\label{sec:sailing-hunger}

There's some nice parallels here.
Navigation at sea and hunger.
Both cases there are observations and rational transitions between these.
I mean, with the hunger it's not necessary, and so perhaps there's still a better case to consider.
What I need is a transition between mental states with marked observations to calibrate these.
Reading a book, perhaps.
One doesn't consider their enjoyment at every stage, but from time to time.
Running a race is like this too, and most sort of sport-like activities can be understood in these terms.
Ah, the trouble is, unlike hunger, these sorts of things can be explained by other pro-attitudes.
Well, running is different, as there are cases where one doesn't really have a goal in mind, and is just going for a run, and you don't check whether you `want' to continue, instead you do this somewhat randomly.

\section{Some basic ideas}
\label{sec:some-basic-ideas}

Beliefs represent the world, but there's no representation in desire.
This is one way to get at the idea of different directions of fit---the world should come to represent desires.
This, at a certain level of abstraction is a useful characterisation of the difference between belief and desire.

However, it suggests that there's little more to belief and desire than their connexion to the world.
This, I think, is a mistake.

Take the case of belief.
We do quote a lot of thinking, or reasoning, and this often goes beyond what we have immediate access to.
The world is there, for sure, but here in the office I have no connexion to what's happening in London, yet I form certain beliefs about it, and then I go and telephone my Sister to find out whether the beliefs I reasoned to were correct.
If they were, my reasoning stands, but if they were not, I'll withold from making the same sorts of inference next time I think (or reason) about London.
The point being that while belief has a mind-to-world direction of fit, there's no constant causal connexion between mind and world.

Perhaps the same is true of desire, in that I'll attempt to make the world a certain way, allow it to continue, and then check back at a later time.
But this is not where I want to draw the parallel.
For, in the case of belief it is the transition between beliefs that is of interest, and corresponding to this, in the case of desire it is the transition between desires that's of interest.

So, the better way to motivate this would be via sources of information.
Doing a direction of fit type thing doesn't seem right.
Pointing out that one can do all sorts of things with this.
And that reasoning and the result of this can really get in the way.
So, the party and taking a drink.
This isn't a case where I reason against the deisre.
It's a case where I make the desire go away.
(And here I really shold make a distinction between desire and pro-attitude to help the exposition.)
Then, the question is why this can't be internalised.
But this isn't quite enough, as it's just `blocking' deires, so to speak, it doesn't get to the idea of reasoning going beyond desires.





\section{Folk-theory}
\label{sec:folk-theory}

Folk-theory isn't all that profound, it's simply the assumption that people aren't all that complex at a fundamental level.
Without good reason we should not treat the complex web of attitudes that we have as anything more than variations of a theme.
Beliefs and desires enter the picture only in so far as there are distinct directions of fit, so to speak.
And, in this respect both beliefs and desires are placeholders.
A number of things may fill these places, and a number of things might occupy both places simulatenously.

In this respect, the Humean thinks that there are two places.
Standard anti-Humeans allow for the possibility that there's a single place.
I, on the other hand, think that am not so interested in the places, but with reasoning.


\section{Tests}
\label{sec:tests}

Typically we think of propositions as sets of worlds.
E.g.\ \(\phi = \{w_{i}, \dots, w_{k}, \dots\}\) where the it is the commonalities between \(w_{i}, \dots, w_{k}, \dots\) which determine the meaning of \(\phi\).
This, however, isn't the full story, as we can likewise think of propositions as \emph{tests}, functions from (sets of worlds) to truth values.
From a certain perspective, this is what's captured by the use of the lambda calculus.
One can take \(\lambda w.\phi(w) = \top\), etc.\

But, this isn't quite the full story.


\end{document}